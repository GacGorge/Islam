\chapter{l'homme}

\mn{27/9}

\paragraph{Pierre Lory}
\href{https://fr.wikipedia.org/wiki/Pierre_Lory}{Pierre Lory}

Après un travail sur la magie en Islam africain, travaille sur l'homme dans le sunnisme 
\begin{quote}
    « Pour un musulman croyant, la religion ne consiste pas seulement à croire dans l’intime de soi au Dieu unique et à son prophète, mais également à accomplir des actes cultuels (prière, jeûne) et suivre certaines prescriptions (alimentaires par exemple). La religion, c’est aussi se conformer à ces prescriptions. De ce fait, la laïcité doit être repensée, réévaluée si l’on veut tenir compte de la réalité musulmane en France. »
\end{quote}
\paragraph{Dans le coran, homme faible et Dieu Allah tout puissant}. Une énigme face aux autres créatures, l'être humain est l'accomplissement de la création, les autres créatures sont subordonnées et mises à son service.

\paragraph{Une autorité à Adam}

\begin{quote}
    30. Lorsque Ton Seigneur confia aux Anges : "Je vais établir sur la terre un vicaire "Khalifa ". Ils dirent : "Vas-Tu y désigner un qui y mettra le désordre et répandra le sang, quand nous sommes là à Te sanctifier et à Te glorifier? " - Il dit : "En vérité, Je sais ce que vous ne savez pas! ".

31. Et Il apprit à Adam tous les noms (de toutes choses), puis Il les présenta aux Anges et dit : "Informez-Moi des noms de ceux-là, si vous êtes véridiques! " (dans votre prétention que vous êtes plus méritants qu'Adam).

32. - Ils dirent : "Gloire à Toi! Nous n'avons de savoir que ce que Tu nous a appris. Certes c'est Toi l'Omniscient, le Sage".

33. - Il dit : "Ô Adam, informe-les de ces noms; " Puis quand celui-ci les eut informés de ces noms, Allah dit : "Ne vous ai-Je pas dit que Je connais les mystères des cieux et de la terre, et que Je sais ce que vous divulguez et ce que vous cachez? "

34. Et lorsque Nous demandâmes aux Anges de se prosterner devant Adam, ils se prosternèrent à l'exception d'Iblis\sn{Satan} qui refusa, s'enfla d'orgueil et fut parmi les infidèles.

\end{quote}

\paragraph{Les noms}
L'homme est supérieur aux anges car il connait les noms. de quels noms exactement ? une interprétation est qu'il s'agit des noms de Dieu. Et la chute d'Iblis est qu'il ne veut pas se prosterner devant les hommes mais uniquement devant Dieu.

\paragraph{Pas de raison de la supériorité de l'homme sur le reste de la création}


\paragraph{D'une certaine façon, les anges ont raison} car l'homme se rebelle. 
\begin{quote}
    35. Et Nous dîmes : "Ô Adam, habite le Paradis toi et ton épouse, et nourrissez-vous-en de partout à votre guise; mais n'approchez pas de l'arbre que voici : sinon vous seriez du nombre des injustes".

36. Peu de temps après, Satan les fit glisser de là et les fit sortir du lieu où ils étaient. Et Nous dîmes : "Descendez (du Paradis); ennemis les uns des autres. Et pour vous il y aura une demeure sur la terre, et un usufruit pour un temps.

37. Puis Adam reçut de son Seigneur des paroles , et Allah agréa son repentir car c'est Lui certes, le Repentant, le Miséricordieux.
\end{quote}

A noter, Eve n'est pas tentatrice.
Il y a de nombreux descendants d'Adam, dans le genre humain, ira en enfer.

\paragraph{Ambiguité} Dieu sait déjà que dans cette dernière créature, une partie ira en enfer. Les anges et les animaux peuvent paraitre plus parfaite car ils ne vivent que pour louer Dieu et ils n'iront pas en enfer. Et pourtant, c'est l'homme qui est au sommet de la hierarchie de la création.


\paragraph{les juristes : Ce qui sauve l'homme, c'est le respect de la sharia} Pour les juristes, comme l'homme est faible, il ne sait comment atteindre le paradais. Et donc il lui faut suivre la loi d'en haut, la sharia, qui vient compenser la faiblesse de l'homme. L'être humain n'est pas humain en lui même, il devient humain quand il conforme sa volonté à la volonté divine. 

\paragraph{Une autre vision, les philosophes soufis} soulignent l'âme divine, l'intelligence dans l'homme.

\paragraph{Mais les anges et les animaux font preuve d'intelligence et sont aussi soumis} Selon le coran et les Hadith, ange, homme et animaux sont formés de la même façon corps et esprit (\emph{rûh}).  Le corps de l''homme est fait d'argile et d'eau. Il est décrit comme la plus belle des créations.
S 7 (de l'alliance), 172 : 
\begin{quote}
172. Et quand ton Seigneur tira une descendance des reins des fils d'Adam et les fit témoigner sur eux-mêmes : "Ne suis-Je pas votre Seigneur? " Ils répondirent : "Mais si, nous en témoignons..." - afin que vous ne disiez point, au Jour de la Résurrection : "Vraiment, nous n'y avons pas fait attention",
\end{quote}
Les hommes sont des particules ici, ils n'ont pas leur forme. Avant de naître dans des corps matériels, rassemblé par Dieu. L'homme préexiste sous forme de particule.

\paragraph{L'homme particule intelligente de lumière} A partir du IXè, ce que Dieu crée en premier, c'est la \textit{lumière de Mohamed} dont sont issus les hommes. A partir de cette lumière, les hommes vont être créés. Un Hadith :
\begin{quote}
Dieu a créé une descendance... frotta le dos d'Adam... Paradis... enfer. 
\end{quote}

\subparagraph{les djinns} corps de feu, sexué et une âme, le \textit{rûh}. Comme l'ombre des humains, des bons et des mauvais djinns. Ils ont une destinée eschatologique. Ils expliquent ce que l'on ne comprend pas. Les djinns sont sur la terre. Ils peuvent prendre possession des hommes et des femmes. Il y a des exorcismes. Les djinns rentrent par l'occiput ou la bouche ouverte (donc on ne baille pas).

\paragraph{le rûh} envoyé dans l'embryon le 127è jour de gestation. 

\paragraph{Les anges} intelligents, sages et pures. Alors que les djinns sont comme les humains, certains croyants et vertueux, d'autres mécréants et pervers

\paragraph{les animaux} dotés de conscience, d'une forme d'intelligence, mais ils ne sont pas responsables devant la sharia. \mn{Une écologie musulmane}

\paragraph{le nafs} une autre âme, permettant de définir l'homme, \textit{l'âme vitale}, elle anime les animaux et l'homme (et les djinns). Par contre, les humains et les anges ont un \textit{rûh}.  \mn{An-nafs (l'âme, la psyché) et ar-rûh (l'esprit)}

\paragraph{rûh} Ce rûh confère un \textit{je}, une capacité de raison, sa capacité d'agir moralement et de comprendre le sens de la vie et sa capacité à répondre à l'appel de Dieu (l'alliance entre Dieu et les hommes avant la Création). Esprit froid, \sn{ à la différence du nafs, esprit chaud,} qui part du coeur. On est vraiment dans une vision différente de la vision occidentale.


Pour les juristes, c'est la distinction vis à vis des anges et des philosophes qui importe. Pour les philosophes, le nafs animal et le rûh divin qui le rapproche vers Dieu.

\paragraph{Pas de nature humaine en Islam } (\textit{phusis} mais une \textit{disposition} (fitra). On pourrait le traduire par \textit{nature} humaine mais c'est plutôt un statut accordé par Dieu à la naissance. Ce statut est le même pour animaux, homme, Djinn et ange. Il n'y a rien de \textit{Naturel}, tout vient de Dieu, pas d'autonomie.

\paragraph{des frontières imprécises} et fluctuantes. 
\begin{quote}
Devenez des singes abjects (Hadith sur les juifs).
\end{quote}
Cette idée de transfomation des humains.
On est transformé en un corps beau et jeune pour aller dans le paradis.

\mn{pratique de vider le sang d'un animal : permet de manger la viande tout de suite. Car sinon, il faut attendre car la viande est trop dur, mais risque qu'elle soit avariée}


\paragraph{Corps n'est pas l'apanage de l'homme}
Son corps humain n'est qu'un aspect (\textit{accident}) de l'individu, plutot caractérisé comme espèce adamique.
A contrario, plein d'êtres peuvent ressembler à l'homme sans être du genre humain : les djinns, Jibril (Gabriel), les houri (paradis) et les échansons (paradis), des corps humains vraiment beau mais ne sont pas des descendants d'Adam.

\paragraph{l'intelligence} partagé avec les animaux, qui peuvent reconnaitre leur créateur. Un Hadith qui dit qu'un chameau se prosterna devant le Prophète, et un palmier qui crie pour être béni par le prophète.


\paragraph{pas de surnature} pas de nature stable, tout tient dans la main de Dieu, qui peut changer à chaque instant ce statut. Pas de supériorité ontologique en faveur des humains. 

Ce qui fait l'excellence de Mohammad, c'est le statut que Dieu lui donne.

\paragraph{l'homme n'est pas achevé} il est une potentialité, à la différence des animaux et des anges. Plus que leur action liturgique, l'homme doit accomplir une trajectoire morale dans laquelle il s'engage pour une transformation profonde, réduisant le \textit{nafs} et développant le \textit{rûh} en lui.
L'homme n'est pas supérieur à l'animal à l'origine, c'est le \textit{cheminement} qui le fait devenir meilleur.


Seulement l'homme est en face d'un destin qui est individualisé. 
Dieu tient toute la création dans sa main, à l'exception de la conscience des hommes. La place exceptionnelle de l'homme réside dans sa capacité à dire \textit{je}, à \textit{transgresser}, ce qui n'est pas permis pour les animaux. Cela amène soit à la sainteté, soit au péché. 
\begin{Synthesis}
Destin individualisé, ressembler à Mohammed, placé dans un statut d'excellence. 
\end{Synthesis}












