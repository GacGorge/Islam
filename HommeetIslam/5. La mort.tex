\chapter{Eschatologie et mort dans l'Islam}

\mn{le 15/11/ 22 la référence du cours en France : Gardet}

\paragraph{Plan du cours}
\begin{itemize}
    \item Mort des humains
    \item rituels funéraires
    \item eschatologie : jugement dernier pour que le mort connaisse son jugement
\end{itemize}


\section{Rituels mortuaires}

\paragraph{Structures légères destinées à s'effacer}

\begin{quote}
    L'aplanissement des tombes, en vue de la disparition dans le paysage (Hadith)
\end{quote}
seule est tolérée une élévation de brique crue pour signaler l'emplacement de la tombe.

\paragraph{Pas de matériaux durables}

\paragraph{pas de monuments}, tombes, décorations. Tout élément distinctif relatif aux morts. \mn{Taaj Mahaal}

Il est formellement interdit des pratiques de l'âge d'ignorance - en pratique chrétienne, de rites auprès du mort.

\paragraph{Enterrer quelqu'un} qabara  : enterrer mais à aussi le sens d'\textit{effacer le sens}, oublier. 
Daraha : creuser une tombe; veut dire, \textit{éloigner, rejeter loin de toi}. 

\subsection{Dans la pratique}

\paragraph{Tout le monde a le droit aux mêmes funerails} Consistent en trois actions : 
\begin{itemize}
    \item La toilette du cadavre
    \item inhumation dans une tombe
    \item prière au dessus de la tombe
\end{itemize}

\paragraph{le martyr} seule distinction : va être enterrée à l'endroit même de son martyr. Il n'attend pas le jugement dernier, du coup le corps est moins important ? Le martyr (shaaid), le témoin, ne meurt pas. 

\paragraph{La tombe} La taille va dépendre de la taille de la personne. On est enterré dans un linceul en coton. La tête au Nord, les pieds au sud, le visage tourné vers la Mecque. On met de la terre jusqu'au niveau du sol. On met en pratique une protection pour éviter que des animaux ne déterrent les corps. \mn{En France, peut on enterrer sans cercueil ?}.
En dessus de la structure en brique, on met souvent des plantes.


\paragraph{pas de stèle} normalement, uniquement une plaque pour dire enfant ou adulte. 
L'idée est d'éviter un culte des saints. \mn{On ne fait pas d'embaumement, en contradiction avec la foi musulmane. Mais il peut y avoir des reliques, marque du sabot du cheval du prophète, barbe du prophète}


\paragraph{pas d'incinération}

\paragraph{on enterre le jour même}
on annonce à la mosquée. Repas pour tous. si vous n'êtes pas un proche, aller à un tel repas vous apporte aussi à vous des bonnes actions. ils vont aller enterrer on sert des repas aux gens qui sont venus on prie. 40 jours pendant lesquels les femmes elles organisent des sessions de prières.
\paragraph{En attendant le jugement dernier, un état provisoire} L'âme, le \textit{nafs,} a quitté le corps avec la dernière inspiration. Ce qui reste, c'est le \textit{ruh}, l'esprit, reste dans le monde et est lié au cadavre. Ce \textit{ruh} n'est pas vraiment dans la tombe, il est dans un espace, \textit{barzakh}, inter-monde ou \textit{monde provisoire}, entre ici bas et le ciel, entre aujourd'hui et le jugement dernier. 

\paragraph{Tourments de la tombe} Le \textit{ruh} connaît le \textit{tourment ou supplice de la tombe}, qui commence dès la fin des rituels funéraire : dire qui il est , professé les prières musulmanes, confesser ses péchés. Cela échauffe les morts, en Asie du Sud Est, on arrose les tombes.


\paragraph{Jugement dernier : loin et dans longtemps} les morts vont être oubliés : pas de trace sur la tombe donc c'est uniquement dans la mémoire des vivants que son nom est rappelé. 

\paragraph{Des actions autour des morts} On va nourrir pendant 40 jours \sn{cycle de Deuil, temps de la patience} une famille. Au bout de 40 jours, on va aller prier puis après 1 an, on retourne au cimetière pour prier. puis, c'est fini. 
Les femmes peuvent aller au cimetière mais pas l'enterrement. 
Après 1 an, on cesse d'entretenir la tombe.


\section{Fin des temps}

qayama : relever.

70 fois le terme de resurrection.
3 mouvements : 
\begin{itemize}
    \item fin du monde, anéantissement des créatures
    \item résurrection des morts
    \item puis, jour du jugement
\end{itemize}
On trouve le terme de \textit{retour} pour parler de résurrection. Une seule fois dans le Coran : \textit{lieu où on revient}. C'est un retour à Dieu. 


\paragraph{Signes précurseurs} annonçant la disparition des créatures. Entre 4 et 10 signes : 
\begin{itemize}
    \item seismes
    \item planètes se disperseront
    \item \ldots
\end{itemize}

\paragraph{Sourates}
Très présente dans les sourates mecquoises 
\begin{quote}
   Retentira le retentissement  Sou 55, 28 (?)
\end{quote}
Son de la trombe dans lequel l'ange de la mort \textit{Idrafil} alors toutes les âmes gouteront la mort. Seul Dieu subsistera à la fin du monde.
\begin{quote}
    Tout est en train de périr sur terre sauf sa face (Sour. 28, 88)
\end{quote}
Ces sourates mecquoises sont les premières révélations coraniques.
Au fur et à mesure de la pensée musulmanes, ces hypothèses vont être 

\paragraph{résurrection des morts} au second coup de trompe, les hommes vont résusciter.
\begin{quote}

\TArabe{
وَٱسْتَمِعْ يَوْمَ يُنَادِ ٱلْمُنَادِ مِن مَّكَانٍۢ قَرِيبٍۢ} 

Et sois à l'écoute, le jour où le Crieur criera d'un endroit proche,

\TArabe{
يَوْمَ يَسْمَعُونَ ٱلصَّيْحَةَ بِٱلْحَقِّ ۚ ذَٰلِكَ يَوْمُ ٱلْخُرُوجِ}

le jour où ils entendront en toute vérité le Cri. Voilà le Jour de la Résurrection.
    Sour 50, 41-42
\end{quote}
Surgissement, passage de la non vie à la vie. Soudaineté de ce retour.

\paragraph{Rassemblement du monde} Rassemblement des hommes, des anges, des djinns et des démons. Aussi les martyrs, les saints, les Imams. Rassemblement universel.
Le premier à se lever : le Prophète. 
Pour certains auteurs, les bêtes participeront aussi à ce rassemblement.

\paragraph{Passion dans l'attente du jugement} Les prophètes, les anges, et les \textit{justes} sont épargnés de cette attente.

\paragraph{Analogie du jugement}
\begin{itemize}
    \item Jour qui suit la nuit
    \item 
\end{itemize}


\section{Questions philo-théologique d'une telle séquence}

\paragraph{Châtiment du tombeau} S'il y a survie de la \textit{ruh}, ce n'est pas très clair dans le Coran mais c'est la compréhension des gens. 
Ce que dit le Coran : 
\begin{itemize}
    \item La mort et l'interrogation des deux anges.
    \item entre les deux coups de trompe et non après les funerails. Pour les gens, c'est tout de suite après les funérails.
    \item âme : on ne peut pas parler d'une âme à priori immortelle. Elle demeure, reste vive, revivifié au moment de la résurrection. Il n'y a pas besoin que le corps tout entier revive, seule les parties essentielles : le coeur, les reins. Et si le corps a disparu, Dieu est capable de redonner vie. 
\end{itemize}

\subsection{Paradis et Enfer}
\paragraph{Paradis} \textit{Janna} : \textit{racine de ce qui est couvert, jardin}. 40 fois dans le Coran. surtout dans sa forme plurielle. Parfois pour désigner des jardins terrestres. On voit aussi 11 fois le terme \textit{Eden}. \textit{sirdaws} : niveau le plus élevé du paradis.

\paragraph{Jardin des délices} Idée que le paradis, c'est un jardin des délices. La notion de Paradis : paradis originel (Adam) que le paradis des croyants.
Adam est le premier homme et a une dimension prophétique.

\paragraph{Paradis originel } Péché originel n'existe pas dans le Coran mais idée de tentation et de chute présente. 
\begin{quote}
    Sourate 2
\end{quote}

Trois courants pour comprendre la chute: 
\begin{itemize}
    \item Adam est un prophète donc il est inpeccable et sa faute est minime. 
    \item Sa faute est aussi rejeté sur Satan
    \item La vie prophétique d'Adam ne commence qu'après sa chute.
\end{itemize}

\paragraph{Paradis des croyants} Une des choses qui fait l'originalité de l'eschatologie musulmane, c'est la description précise de la géographie du paradis : situé en Paradis, des murs l'entourent, des portes. \textit{Dieu fortifié}.
\begin{Ex}
    L'eau coule,... 4 fleuves, eau, lait, vin et miel.
\end{Ex} 
Cette description topographique n'est pas dans le coran mais dans une littérature spécifique. Elle part du Coran et des Hadiths, et s'est ensuite développée. Description précise et matérielles du paradis : les portes, le mobilier. On décrit les belles personnes. La vaisselle. 

\begin{quote}
    Allumer le paradis et éteindre l'enfer (Rabiah)
\end{quote}

\paragraph{Face à cette description, une opposition muzahalite}

\paragraph{Une opposition plus contemporaine des wahhabites}

\paragraph{Habitants du paradis}
Compagnon de la droite, ceux qui sont devant, et les proches. Mais pas de définition. On peut penser que les proches, ce sont les saints (même racine) : \textit{ami, proche de Dieu}
\begin{quote}

    Sourate 56
\end{quote}


\paragraph{Et l'enfer} \textit{Nar} : le feu. Il y a d'autres termes : fournaise, feu ardent, aussi utilisés. Jahannan : provient de l'hébreu, et designe la vallée des enfants de  sinnoun (?). 7 niveaux des enfers. En dessous du Paradis, avec une muraille; Lieu obscur, clos, même s'il a 7 portes. Il y a un gardien, aidé de 19 anges.

Des rivières remplies de fluides dégoutants, des montagnes, des vallées, des vents pestilentiels. Un seul type d'arbre dont les fruits ressemblent à des têtes : 
\begin{quote}
    En atteignant les tréfonds de l’enfer, Muhammad, porté par al-Burâq et précédé de l’ange Gabriel, arrive devant le Zaqqoum, un arbre à trois grosses branches dotés de très longues épines, où se dressent en guise de fruits des têtes d’animaux sauvages ou fabuleux dont certains sont reconnaissables : dragon, ours, serpent, lion, éléphant, panthère, dromadaire, renard. \href{https://krapooarboricole.wordpress.com/2014/04/03/une-curiosite-de-lenfer-musulman-larbre-zaqqoum/}{Zaqqoum}
\end{quote}

\paragraph{qui va en enfer} les incrédules (ceux qui ont nié la puissance divine, ceux qui se sont moqués du Coran, ceux qui parmi les croyants n'ont pas combattu dans les chemins de Dieu). idée : Dieu a donné tous les moyens à l'homme d'aller au Paradis, ceux qui n'y vont pas, le font par choix.

\paragraph{Musulman sans le savoir} qu'on trouve aussi dans le Coran. Voir où ?


Malek Chebel : l'enfer est décrit comme particulièrement effrayant mais les musulmans même bons, vont passer en enfer, un court instant : traversée de purgatoire. Mêmes les enfants, les morts nés, les déficients, sont reçus en enfer. 

\begin{Synthesis}
    la mort a des aspects effrayants et plus aimables.
    \begin{itemize}
        \item Effrayant : par le jugement. 
        \item Aimable : elle porte l'amant dans la présence du bien aimé. Comble l'âme du désir du croyant de trouver la paix éternelle auprès de son Dieu.
    \end{itemize}
\end{Synthesis}

A la fin du monde, certains penseurs pensent qu'il y aura un anéantissement du paradis et de l'enfer.

\subsection{Et chez les chiites} Lié au retour du XII eme imam, Al Maadi, dont le travail est de vaincre de façon définitive les forces de l'ignorance. 
Ce premier retour accomplit par le Maadi. Le signe précurseur de ce retour. Signe de sa venue : perte du lien à dieu, de la valeur de l'amour. Les chiites voient aussi des persécutions de la part des sunnites.

\paragraph{5 signes du retour } Le Maadi va être aidé par le retour de Husseyn, Jésus et Ismael : tous les trois sont réunis par l'impiété de leur peuple.
les anges. 
\textit{La frayeur}
Les 33 croyants
vont aider le Maadi. 