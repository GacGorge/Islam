\chapter{Figure d'autorité dans le Sunnisme : Le Prophète}


\section{Mohammad}
\paragraph{autorité : bien comprendre le lien avec Mohammad} Dans l'enseignement du coran et la Sunna, l'idée d'\textit{imiter le prophète} et de lui accorder \textit{amour et vénération}

\paragraph{débat sur la figure du prophète} 

\begin{itemize}
    \item simple messager. le législateur, dont la loi sha'ria doit être suivie
    \item pour certains, sa réalité est supratemporelle, intemporelle, composée de lumière. Sa présence est toujours actuelle dans le monde et on peut donc espérer de lui une \textit{intercession dans l'au delà}
\end{itemize}

\paragraph{juste un prophète serviteur ou un prophète Roi ?} qui gouverne. 

\paragraph{distinction entre prophète et envoyé} \mn{\href{https://fr.wikipedia.org/wiki/Messagers_de_l\%27islam}{wikipedia}}
\paragraph{\Rasul}  Dieu envoie un envoyé de Dieu, un apôtre (\Rasul) \TArabe{رسول}. \mn{le message est toujours le même mais ce sont les hommes qui comprennent mal, sur l'unicité de Dieu et le jugement dernier}. Il est le représentant d'un peuple
On trouve {\Rasul} dans trois sourates du Coran.

\paragraph{\Nabi} connaissance de la divinité. Idée que les prophètes sont la source de toute connaissance supérieure. Il n'est pas le représentant d'un peuple. Sont \Nabi Jésus, ... c'est un statut plus faible que celui de \Rasul.

\paragraph{Les caractères spécifiques des prophètes}
qui permettent de distinguer un prophète :
\begin{itemize}
    \item Quand ils sont sous le coup de leur révélation, ils peuvent s'abstraire de leur nature humaine.
    \item Naturellement portés vers le bien.
    \item Ils agissent en faveur de la religion.
    \item Ils ont une noble ascendance.
    \item Ils font des prodiges en parole et action. Le plus grand prodige, c'est le \textit{Coran}
    \item clarté, lucidité
    \item Intelligence / créativité.
    \item sait se faire obéir
    
\end{itemize}




\paragraph{littérature populaire} va parler des miracles du prophète, pas forcément orthodoxe. Au début de l’Islam, peu d’intérêt pour Mohammed.

\paragraph{historique}  peu d’information sur lui. On ne sait pas quand il est né. On fête sa naissance, le 12 \textit{Rabi sul Awwal}\mn{mois lunaire}.*

\paragraph{Illétrisme}
Il a été présenté comme \textit{illettré}. Il était commerçant. On peut douter de cet illettrisme car ce n'est pas compatible avec son métier de commerçant.
Son illettrisme désigne le \textit{prophète du peuple} qui n'appartenait pas aux \textit{gens du livre}
\mn{Les musulmans se pensent dans la continuité du Christianisme, dans la continuité du judaïsme : Juifs, chrétiens et les {sabéens} les hindous appartiennent donc aux gens du libre via les sabéens, puisqu'ils ont les \textit{véda} }. 

\begin{Def}[Contribule]
Appartenir à une tribu
\end{Def}
Il passe du statut de Contribule au statut d'envoyé de Dieu, quand il est à la Mecque. Puis chef de guerre, chef d'état dans la ville de Médine. 
A la Mecque, il annonce que le Règne de Dieu est proche. Réformateur religieux qui prône un strict monothéisme. 

On sait qu'il est persécuté par les polythéistes Mecquois.  
\paragraph{Ascension} Son ascension (Miraj) : 27 rajab. Il accomplit cette ascension, avec comme véhicule \textit{burq}, un âne avec une tête de femme. Il va à Jérusalem en rêve. Il prend son élan pour monter dans les cieux. Le ciel est divisé en 7. il monte les 7 strates et voit la \textit{lumière divine}.


\paragraph{Représentation de l'homme} L'interdiction est récente, avec les mouvements de réforme du XIX et dans le sunnisme. L'interdiction porte sur l'image de Dieu.

\paragraph{L'exil à Yatrib}, la ville \emph{medina} du prophète. Hégire : 622-632 après JC. Très bon chef et stratège. Instaure une \textit{théocratie} à Médine et exige l'adhésion à l'islam des tribus arabes mas pas des juifs ou des chrétiens. il instaure la \href{https://fr.wikipedia.org/wiki/Djiz%C3%AEa}{djizia} 
\begin{Def}[djizia]
 La jizîa, djizîa  est dans le monde musulman un impôt annuel de capitation évoqué dans le Coran et collecté sur les hommes pubères non musulmans (dhimmis) en âge d'effectuer le service militaire  contre leur protection - en principe . Certains dhimmis en sont théoriquement exemptés : les femmes, les enfants, les personnes âgées, les infirmes, les esclaves, les moines, les anachorètes, et les déments . En sont également exemptés ceux des dhimmis qui sont autorisés à porter les armes pour effectuer un service militaire.
   
\end{Def}

En 630 il rentre à la Mecque et la plupart des habitants adoptent l'Islam, sans combat. Premier pélerinage le 8 juin 632.
Il est enterré à Médine.

\section{Vénération du Prophète}
Peu à peu se met en place un bénération (ce qui n'existait pas  au début) avec une doubkle dimension : 
\begin{itemize}
    \item Homme ordinaire
    \item Et être élu pour une mission universelle
\end{itemize}

Denil Gril\sn{Islamologue}, il existe 2 sourates (la Victoires, les Appartements) qui font allusion au caractère sacré du prophète.
Vénération du prophète et glorification de Dieu vont de pair. 

\paragraph{modèle pour les musulmans} de par la façon dont Dieu a considéré le prophète pendant sa vie. Parfois très favorisé, parfois malmené, Dieu ne lui accorde pas tout ce qu'il peut attendre.

\paragraph{Attitude de patience} du Prophète qui doit inspirer le croyant. 

\paragraph{Connaissances incommensurables} qui font de lui un modèle (en contrdiction avec l'idée d'un prophète illétré). Mais cette connaissance est celle de Dieu (les 99 noms) et des autres envoyés. La \textit{relation des croyants à Dieu} passe par le prophète.

\paragraph{Obeissance} passe par la vénération et l'amour

\begin{quote}
    Aucun d'entre vous ne croira tant qu'il ne m'aimera pas plus que son père, son enfant et tous les hommes (attribué au Prophète). 
\end{quote}

\paragraph{Sunna et Sira} contribuent à la vénération du Prophète.

\paragraph{2 formes de vénération} 
Double aspect divin dans le Prophète, et double aspect de vénération
\begin{itemize}
    \item Prophète, être de rigueur et de miséricorde
    \item être de majesté et de beauté
\end{itemize}

Crainte référentielle et amour (qui englobe celui de Dieu). Désir de contact avec des objets qui ont appartenus au Prophète : présence du sacré
Compagnonnage avec le corps du Prophète comme gage du salut : aujourd'hui culte de reliques (poil de Mohammed).
Cet amour ne vient pas pour le prophète en tant qu'individu mais par sa dimension sacrée, dont on espère qu'elle nous accompagne dans l'autre monde.

\subsection{Nature et fonctions du prophète}

\paragraph{Débat à sa mort} sur la question du califat; qui doit succéder au prophète. Le prophète est il roi ou serviteur ? Plusieurs Hadiths à ce sujet. 
\begin{itemize}
    \item Le prophète-roi dont l'archétype est le roi david. 
\item ou simple serviteur, renonçant, ascète.



\end{itemize}
Le prophète doit faire un choix : quand il fait face à son clan,il doit choisir. 

\paragraph{Tension qui reste aujourd'hui} Le hadith du choix, on peut interpreter qu'il ne detient aucun pouvoir ni terrestre, Sa mission est d'être un annonciateur qui s'il est chef de sa communauté, n'est là que pour appliquer la loi divine.  A Médine, il se comporte bien en chef de guerre mais la loi qu'il applique est la loi de Dieu et pas la sienne.


\section{Le prophète dans la mystique - soufisme }

Tustari(Xs),  sourate 53.
Notion de lumière muhammadienne, matière première.
\paragraph{Création}
Quand Dieu crée Adam, il se sert de cette lumière dans laquelle il y a de l'argile. L'argile qui va créer Adamn est celle qui va servir pour Mohammed historique. \textit{Il existe un Mohammed préexistant}. Après qu'il a créé Adam, Dieu crée les prophètes et les saints ensuite en frontant les reins d'Adam. Dieu demande aux prophètes de conclure le pacte prééternel appelé le \textit{Convenant}. 
Puis il crée le reste de l'humanité enfin, tout le monde intègre les reins d'Adam.
Tout est issu de la lumière Muhammadienne.

\paragraph{Wasiti} Pour lui, Mohammed a d'abord été créé comme un esprit. Dieu crée en premier l'esprit de Mohammed,germe de l'arbre des existants, ie que tous les esprits vont être créés à partir de l'esprit \mn{et non de la lumière} du prophète. 

\paragraph{Ibn Arabi} Réalité Muhammadienne : premier intellect, esprit universel, à comprendre comme l'homme parfait par sa vertu, ses sentiments et ses actes. Mohammed réalise par ses actes tous le degrés de l'être humain. Accomplissement à l'état potentiel chez la plupart des humains mais il n'y a que quelques grands saints qui peuvent la réaliser. Mohmmad représente l'origine et la finalité de toute chose.

La fête du \textit{Mawlid, } commémoration de la naissance de Mohammed. Mais débat à nouveau, seul Dieu doit recevoir un culte et des dévotions.

