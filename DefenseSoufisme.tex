\begin{quote}
Cheikh Ahmad al-Alawî
\end{quote}

Lettre ouverte à celui qui critique le soufisme

\begin{quote}
Comme son nom l'indique, ce texte est une épître qui vise à réfuter les
arguments fallacieux des adversaires du Soufisme. Se référant aux textes
les plus authentiques et les plus autorisés, le Cheikh montre également
le fondement coranique et traditionnel (c'est-à- dire fondé aussi sur la
Sunna et les hadiths du Prophète) de la doctrine Soufie. L'argumentation
développée par le Cheikh al- Alawî se révèle être décisive par sa
rigueur et la force de son évidence. C'est dans le cadre des polémiques
opposant soufis et milieux réformistes que le Cheikh Ahmed al-Alawî eut
l'occasion d'écrire en 1921 l'épître dont nous présentons un extrait de
traduction. Servi par une rhétorique efficace et un style incisif, il y
réfute une par une les critiques des adversaires du soufisme, et cite la
multitude de source scripturaires (Coran et Hadîth) sur lesquelles
s'appuie le Tassawwuf.

La traduction, jusqu'à ce jour inédite, de cette œuvre du Cheikh al-
Alawî représente une contribution très intéressante, compte tenu de
l'autorité du Cheikh, à l'étude détaillée des critiques et préjugés les
plus fréquents que nourrissent, de nos jours, les milieux religieux de
l'Islam à l'égard de ce que représente, à l'intérieur de ce dernier, le
soufisme D'autre part, ce livre servira également à ceux qui souhaitent
comprendre, de manière objective et " au plus près ", le véritable
caractère de certaines polémiques qui, de fait, se produisent dans
toutes les religions au cours de leur développement historique, d'une
manière ou d'une autre, bien que ce soit en climat musulman qu'elles
apparaissent aujourd'hui de la façon la plus paradigmatique.

En effet, cette œuvre n'est pas autre chose que la réponse la plus
directe et précise possible à certaines attaques que le soufisme eut à
subir de la part de l'un des représentants des courants " réformistes "
du début du XXe siècle, milieux qui correspondent aujourd'hui grosso
modo à ce que l'on appelle " fondamentalisme ", et qui tentaient à cette
époque de détruire l'énorme influence que le Tassawwuf a depuis toujours
exercé sur l'ensemble de la communauté musulmane.

Certains s'étonneront peut-être du caractère assez polémique de ce
texte, sans réussir à comprendre comment un personnage dont le rôle est
d'un ordre largement plus élevé peut en être à l'origine, et pourtant,
l'histoire nous montre que d'autres, tout aussi éminents, ont agi dans
le même sens.

Tout d'abord, notons qu'il s'agit d'une polémique très ancienne, dont on
pourrait retrouver les traces aux origines de toute révélation
prophétique et universelle, dans la mesure où celle-ci se trouve
confrontée à l'ignorance de ceux dont l'horizon, dans la façon qu'ils
ont d'appréhender la réalité, est borné soit par leur limites propres,
soit par leurs intérêts matériels Ainsi, du fait même de cet "
affrontement ", il se produit, dans le cadre de certaines étapes
historiques, une détérioration progressive du caractère universel et
spirituel, non pas de la Révélation en elle- même bien sûr, mais de ses
formes d'expressions religieuses Cette opposition, qu'elle soit
consciente ou non, a par définition un caractère ténébreux, puisqu'elle
tente de réduire l'influence de la

force lumineuse dont le Message Divin est porteur, pour le conditionner
en fonction des exigences d'une logique précaire Il s'agit là sans nul
doute d'une bid'a, d'une innovation, des plus pernicieuses qui soient,
crime dont, paradoxalement, on accuse souvent ceux qui ne se sont pas
écartés d'un iota de la Volonté Divine et du comportement prophétique.

C'est dans cette perspective qu'il faut situer l'intervention du Cheikh
al-Alawi A travers sa réponse à une telle attitude se manifeste le zèle
qui l'anime, quand il s'agit de préserver le bénéfice qu'il y a à rester
fidèle aux authentiques interprètes de l'esprit de la Parole Divine et
de l'enseignement prophétique qui l'accompagne, face à l'intransigeance
et à l'audace de ceux qui n'" interprètent " en réalité qu'eux-mêmes,
assumant ainsi la responsabilité de la décadence spirituelle et morale
de toute une communauté de croyants.

D'autre part, il convient de souligner le caractère exceptionnel d'une
intervention de cette nature, qui ne se justifie qu'à titre de respect
d'une obligation collective - l'accomplissement par une personne du fard
kifâya en dispense les autres - et concerne, avant tout, celui qui
assume cette responsabilité En effet, en tout autre cas : "Les
serviteurs du Miséricordieux sont ceux qui marchent humblement sur la
terre ; et lorsque les ignorants leur adressent la parole, ils
répondent: " Paix! ", Sourate 25 : Le discernement (Al Furqan) verset
63.

Comme le Cheikh le signale lui-même, la polémique est, autant pour les
Prophètes que pour les saints, l'ultime et le plus pénible des recours,
acceptable pour autant qu'elle soit faite de la meilleure manière ".

Par ailleurs, l'argumentation, le style et les moyens qu'emploie le
Cheikh dans cette épître, indépendamment de certaines références
ponctuelles et " techniques " parfois étrangères au contexte qui est le
nôtre, permettent de mieux comprendre comment doivent être appliqués les
principes qui régissent la transmission d'un authentique enseignement
religieux, relativement à la " lettre " et à " l'esprit ".

Quant à la première, il est nécessaire de connaître la Révélation et les
sources traditionnelles et de s'y conformer, afin d'éviter les
interprétations individuelles ou tendancieuses, aussi raisonnables
qu'elles puissent nous paraître ; de plus, par principe, il faut
toujours essayer d'avoir une bonne opinion a priori (husn al- dhann)
quand il s'agit d'autrui Ces deux principes sous-tendent l'argumentation
du Cheikh tout au long de cette lettre Concernant " l'esprit ", il faut
admettre qu'il ne saurait y avoir de réelle tolérance en l'absence d'une
connaissance véritable La vision du Cheikh est une vision sans limites,
avec une perspective à très long terme, qui suscite confiance et amour
pour ceux qu'Allah a choisis comme intermédiaires et sujets de la
contemplation de cette Grandeur qui dépasse toute capacité humaine.

En ce sens, la Vérité ne peut être manipulée, tout simplement parce
qu'elle englobe tout, y compris sa propre manipulation Quoi qu'il en
soit, l'action du Cheikh consiste à essayer de sauver ce qui peut l'être
et empêcher la destruction de ce qui reste, en dénonçant ceux qui sèment
le doute et la méfiance parmi les plus faibles, de façon à les séparer
des dépositaires de la Foi.

Puisse cette lettre, par la Grâce Divine, nous être à tous profitable,
et remercions celui dont l'effort d'interprétation fidèle et le savoir
utile nous ont permis d'y accéder.

\textbf{Introduction}

\emph{Au Nom de Dieu, le Tout-Miséricordieux, le Très-Miséricordieux}

Louange à Dieu, qui nous a épargné ces épreuves auxquelles Il a soumis
bon nombre de Ses créatures ! Que la grâce et la paix soient sur le
Prophète et sa famille Cette lettre émane d'un esclave de son Seigneur
ayant beaucoup de méfaits à se reprocher : Ahmad Ibn Mustafa al-Alawî -
que Dieu lui accorde Sa grâce et lui inspire, ainsi qu'aux croyants, de
suivre la voie la plus droite !

Le destinataire en est le juriste réputé, le Cheikh Sidi Uthman Ibn
al-Makki, professeur à la grande mosquée de Tunis - que Dieu le fasse
prospérer et le purifie de tout démon révolté !

Que la paix de Dieu soit sur vous, aussi longtemps que vous montrerez de
la déférence à l'égard des membres des confréries : "Celui qui vénère ce
que Dieu a déclaré sacré en tirera bénéfice auprès de son Seigneur",
Sourate 22 : Le pèlerinage (Al-Hajj) verset 30.

J'ai découvert l'épître issue de votre plume intitulée " Le Miroir
manifestant les égarements " La prenant en considération, je l'ai
feuilletée avec attention, en rendant grâce à Dieu qu'il reste encore de
nos jours des personnes fermes en matière de religion, des gens qui ne
craignent le blâme d'aucun censeur dès lors qu'il s'agit de Dieu.

Certes, son titre me gênait quelque peu en raison du terme " égarements
", mais ce que j'ignorais à ce stade, c'est que le texte ainsi intitulé
était encore plus gênant.

Le peu que j'en lus suffit à me désappointer : mon intérêt s'émoussa
aussitôt, et je ressentis une peine à la mesure de ma réjouissance
initiale J'en fus à tel point affligé que je faillis m'écrier

: " Il est absolument illicite de poser son regard sur un quelconque
miroir, que ce soit pour y contempler des égarements ou quelque autre
forme ! "et cela en raison des attaques et atteintes à l'honneur que
contient votre" Miroir " Peu s'en faut qu'elle ne déborde de colère : en
direction des gens du Souvenir ( Dhakiroun ), elle lance

des étincelles de la taille d'une forteresse, et son discours fiévreux
détruit les croyants Je cherchai bien à distinguer l'écrivain de son
œuvre ; mais à chaque fois, l'idée me reprenait qu'un discours est
toujours le reflet de son auteur et que la caque sent toujours le
hareng.

Les mensonges que contient votre "Miroir" et le caractère immoral de son
contenu constituent une atteinte à l'honneur des gens du rattachement à
Dieu : vous les avez proprement calomniés Aussi, la Jalousie divine et
la ferveur {[}que je porte{]} à l'Islam m'ont poussé à vous écrire, par
vénération pour ces membres des confréries que vous avez caricaturés
Venant au secours des gens du Souvenir que vous avez trahis, je ne fais
que mettre en pratique la parole suivante du Prophète - sur lui la
prière et la paix - : "Celui qui assiste à l'humiliation d'un croyant
sans venir à son secours alors qu'il en a les moyens, Dieu l'humiliera
devant témoins au jour de la Résurrection".

Dans le sahih, il est rapporté d'après Abû Umâma que le Prophète

- sur lui la prière et la paix - a également dit : " Quiconque aura
défendu l'honneur de son frère verra son visage écarté du Feu au jour de
la Résurrection " et, d'après Abû Dardâ : " Quiconque aura défendu
l'honneur de son frère sera protégé du Feu par un voile, au jour de la
Résurrection ".

Ces propos ont une portée générale : l'honneur de tout croyant, quel
qu'il soit, doit être défendu ; quant à l'honneur des gens du Souvenir
(Dhakirouna), c'est Dieu Lui-même qui S'en charge particulièrement Le
plus véridique dans Ses Paroles n'a-t-Il pas dit: "C'est Lui qui protège
les Justes ? Quiconque leur cherche querelle s'attaque en réalité à Dieu
; et quiconque leur porte secours vient en aide à Dieu".

Les gens de la Grâce n'ont cessé et ne cesseront d'assurer la sauvegarde
de la Voie de Dieu (la Relation avec Dieu) en tout temps ; en effet, le
Peuple (al-Qawm, terme qui signifie "peuple, tribut, gens, groupe" et
qui désigne en général la communauté soufie) - que l'agrément divin soit
sur lui - suscitera toujours des partisans et des opposants Telle est la
coutume de Dieu à l'égard de ceux qui vécurent autrefois "Et tu ne
trouveras point de

changement dans Sa coutume", Sourate 33 : Les coalisés (Al- Ahzab)
verset 62.

Il y aura donc toujours des gens bienveillants pour faire son éloge et
des envieux pour le critiquer Cela dit, ces attaques et ces critiques
peuvent aussi bien viser des personnes vraiment religieuses que d'autres
plus faibles sur ce plan ; le censeur, lorsqu'il crie à la déviation,
peut fort bien se baser sûr des apparences qui s'avèrent tout à fait
trompeuses.

Quant à toi, en critiquant sans distinction aucune les membres des
confréries, et en réfutant publiquement leurs convictions - c'est ce que
tu fais, ô Cheikh, lorsque tu argues qu'il n'y a là qu'erreur, ignorance
et égarement ! -, tu manifestes une attitude sans précédent chez les
savants religieux (exceptés ceux de différentes sectes déviées qui
contestent le principe même d'une élection divine, simplement parce
qu'ils n'en sont pas les bénéficiaires).

Les gens de la Tradition (Ahl-a-Sunna), pour leur part, n'ont jamais
émis de critiques, si ce n'est à propos de personnes dont la sainteté ne
faisait pas l'unanimité Leur point de vue sur le soufisme a toujours
consisté à le respecter et à en magnifier le degré ; leurs paroles à ce
sujet sont les témoins les plus équitables dont on puisse enregistrer la
déposition.

De façon générale, les gens de la Tradition éprouvent naturellement de
l'amour pour le soufisme et ses adeptes On constate d'ailleurs que celui
qui s'aventure à dénigrer leur doctrine (Madhab - école de pensée)
baisse rapidement dans l'estime du savant comme dans celle du croyant de
base : en réalité, cela montre qu'il a baissé dans l'estime de Dieu -
qu'Il nous préserve d'une telle déchéance C'est pourquoi on a dit :

\emph{Quiconque s'oppose aux gens du Souvenir (Dhakirouna) En
s'acharnant contre eux injustement}

\emph{Par la haine des créatures, Dieu l'éprouvera rapidement}

Je viens donc te donner un conseil sincère, en espérant que cela mettra
un frein à tes attaques - s'il plaît à Dieu, "Et Dieu vous met en garde
contre Lui-même", Sourate 3 : La famille d'Imran (Al-

Imran) verset 28. Dieu a dit dans une tradition sanctissime (hadith
Qoudoussi) : " Quiconque nuit à l'un de Mes saints, Je lui déclare la
guerre ", Rapporté par Al Boukhari. Or, qui s'expose à la guerre divine
n'est pas en sécurité, assurément ! Le Prophète a dit - sur lui la
prière et la paix - : " Les gens de ma famille (Ahl-Albayt) et les
saints de ma communauté sont deux bosquets empoisonnés : qui s'y frotte
s'y pique ! ".

Quant aux paroles des savants à ce sujet, elles sont innombrables Abû
l-Mawâhib al-Tunusi raconte notamment que son Maître Abû Uthman - que
Dieu soit satisfait des deux - disait publiquement dans ses cours : "
Que la malédiction divine frappe celui qui réprouve cette communauté
{[}des soufis{]} ! Et quiconque croit en Dieu et au jour dernier se doit
de faire la même imprécation ".

Laqqânî - que Dieu soit satisfait de lui - disait quant à lui : "
Quiconque polémique au sujet des soufis risque de mal finir ; un
traitement sévère et un emprisonnement prolongé seront son lot ". " Dieu
vous exhorte à ne plus jamais recommencer, si vous êtes croyants ! "
Sourate 24 : La lumière (An-Nûr) verset 17, (passage du Qoran où il est
question justement de calomnies).

Tu constateras ainsi qu'un imam scrupuleux éprouve toujours beaucoup de
réticence à parler en mal du commun des croyants, pour ne rien dire des
membres des confréries ! Mais enfin, si leur islam est la seule chose
qui te paraisse acceptable en eux, leur reconnaître la qualité de
musulman t'oblige alors à les respecter et à t'abstenir de porter
atteinte à leur honneur, en évitant par conséquent de te mêler de leurs
affaires privées, conformément aux mises en garde du Législateur.

Le fils d'Omar - que Dieu soit satisfait de son père et de lui -
rapporte ainsi la parole suivante du Prophète - sur lui la grâce et la
paix - : " Quiconque divulgue les secrets ( `awra) d'un musulman et le
déshonore de ce fait injustement, Dieu l'avilira dans le feu au jour de
la Résurrection ".

Tel est le châtiment réservé à celui qui divulgue les secrets d'un seul
musulman : que peut donc bien espérer celui qui se mêle des affaires
privées de la masse comme de l'élite des musulmans pour

les déshonorer au sein de la communauté, voire même auprès des
non-musulmans si la chose parvenait à leurs oreilles ?

Or, c'est bien ce que tu as fait, ô Cheikh ! Tu t'es répandu en
réprobations, passant au peigne fin des choses sans intérêt ; tu t'es
cru le seul et unique représentant de l'orthodoxie sunnite, le reste de
l'univers étant peuplé d'ignorants, d'innovateurs ou de transgresseurs
égarés Oui, c'est bien ainsi que tu juges les fils de ta religion !

Quant à nous, nous ignorons le jugement de Dieu à ton égard ; mais
{[}nous sommes certains que{]} si tu t'occupais de tes propres affaires,
tu aurais suffisamment de quoi faire, et cela te dispenserait de
t'intéresser à celles des autres Tu es l'exemple même de ces personnes à
propos desquelles le Prophète - sur lui la prière et la paix - a dit : "
On distingue le brin de paille dans l'œil de son frère alors qu'on
oublie la poutre qui obstrue le sien ".

Et de fait, tu en oublies pour ta part de fort nombreux, comme tu vas
bientôt t'en rendre compte En te faisant prendre conscience de ces "
poutres ", je t'amènerai peut-être à t'en débarrasser, à supposer que tu
en sois capable Pour ce faire, tu n'as pas d'autre solution que de
reconnaître purement et simplement {[}tes erreurs{]}, et cela dépend de
ta capacité à être objectif : si tu as cette qualité, cette épître
travaillera en ta faveur ; dans le cas contraire, elle constituera une
preuve à ta charge De toute façon, lorsque tu la liras, montre une vue
perçante, une raison saine, et place ton cœur à l'abri du sectarisme.

Si j'écris ces lignes, c'est avec l'espoir que par elles, Dieu te
délivre de ce mal qui te frappe ; ou qu'Il délivre tes semblables, ou
toute personne qui trouve plaisir à lire ton triste " Miroir " ou se
réjouit d'assister à tes affligeants discours Je m'en vais donc te
signaler ces " poutres " dont tu aurais pu oublier qu'elles obstruaient
ta vue, si Dieu ne les avait pas assez mises en évidence au moyen de ton
" Miroir " !

En premier lieu, tu introduis ton ramassis d'atteintes à l'honneur des
musulmans par la citation suivante : " Louange à Dieu qui

nous a guidés vers cela ; nous n'aurions pu suivre la bonne direction
s'il ne nous avait guidés ", Sourate 7 : Al-A'araf verset 43.

Je ne sais quelle était ici ton intention : voulais-tu simplement
bénéficier de la bénédiction attachée à ce noble verset, ou bien
s'agissait-il d'insinuer que ces atteintes à l'honneur des gens du
Souvenir et de leurs semblables, auxquelles Dieu t'a conduit, relèvent
de la guidance divine ? Dans le premier cas, c'est très bien

! Mais sinon, sache que la guidance ne peut prendre la forme d'une
critique calomnieuse des gens de Dieu, sauf lorsque " guidance " prend
le sens qu'il a dans cette Parole de Dieu - exalté soit-II - :
"Guidez-les alors sur le chemin de l'Enfer ", Sourate 37 : Les rangés
(As-Saffât) verset 23, ou dans d'autres passages semblables.

Tu as bien raison d'appeler ton ouvrage " Le Miroir manifestant les
égarements " : ce titre correspond admirablement à son contenu ! Ton "
Miroir " met effectivement en évidence ce qui t'habite, et sans lui, qui
pourrait constater ton égarement ? L'écrit est à l'image de
l'intelligence, et l'intérieur transparaît dans le discours.

Un peu plus loin, tu entames une rubrique intitulée " Introduction au
sujet du commandement du bien et de l'interdiction du mal ", dans
laquelle, sous prétexte d'appliquer ce précepte coranique, tu réunis ces
quelques références scripturaires qui te servent de subterfuge pour
porter atteinte à l'honneur des croyants Mais devant Dieu, cela ne te
servira à riens : de quelque façon qu'on l'habille, la médisance reste
la médisance Même en admettant que tu n'aies souhaité qu'arranger les
choses, ta prose témoigne de ton incapacité à distinguer entre le bien
et le mal : cela est excusable, mais pas de la part de quelqu'un qui
entreprend de commander et d'interdire !

Quelle que soit la façon d'envisager ton cas, tu es loin d'être au
dessus de tout soupçon Si tu ne savais pas, c'est un mal en soi que
d'être ignorant, mais si tu savais, le mal n'en est que plus grand Si tu
n'as pas une intuition claire de ce qui distingue le bien du mal,
comment peux-tu ordonner ceci et rejeter cela ? Avant de te prononcer
sur un sujet quelconque, tu dois t'en faire une juste conception, le
jugement particulier n'étant que l'application de celle-ci Et lorsque tu
tranches, tu ne dois le faire que selon le

jugement de Dieu, ordonnant ou interdisant suivant les ordres et
interdits divins Scrupuleux à l'extrême, tu dois t'abstenir de parler de
la religion selon ton opinion ou de prononcer des interdits en fonction
de tes préférences Dieu - exalté soit-Il - n'a-t-Il pas dit : " Ceux qui
ne jugent pas d'après ce que Dieu a révélé, ceux-là sont les injustes
!", Sourate 5 : La table servie (Al-Maidah) verset 45.

As-tu bien appliqué cela, toi qui viens interdire ceci, blâmer cela,
déclarer tel groupe dans l'égarement et traiter tel autre d'innovateur ?
Ton attitude avec Ses créatures ne témoigne pas d'une grande crainte de
Dieu, pas plus que ton respect pour Muhammad ne transparaît dans ton
comportement envers sa communauté !

Tu crois pouvoir ordonner le bien et interdire le mal, mais en es-tu
bien digne ? Le Prophète - sur lui la prière et la paix - a dit : " Seul
peut commander le bien ou interdire le mal celui qui fait preuve de
douceur lorsqu'il ordonne ou interdit ; celui qui est patient et
intelligent lorsqu'il ordonne ou interdit ; celui qui connaît et
comprend {[}véritablement{]} les règles religieuses lorsqu'il ordonne ou
interdit ".

La première partie du hadîth signifie - mais Dieu est plus savant -
qu'il ne formule ordres et interdits qu'avec douceur : c'est exactement
le contraire de ce que tu as fait dans ton " Miroir ", ô Cheikh ! Tu
aurais mieux fait de t'abstenir de toute initiative tant que tu ne
connaissais pas les conditions d'exercice de cette fonction, telles que
Dieu les a fixées : cela t'aurait permis d'entrer dans la maison {[}du
commandement du bien et de l'interdiction du mal{]} par sa porte.

N'as-tu jamais entendu l'histoire de ce jeune homme qui vint trouver le
Prophète - sur lui la prière et la paix -, lui demandant d'une voix
forte : " O Envoyé de Dieu, me permets-tu d'avoir des relations
sexuelles en dehors du mariage ? " {[}Scandalisés,{]} les gens
poussaient des exclamations, mais le Prophète ordonna soudain : "
Laissez-le, laissez-le ! " Puis il lui demanda d'approcher et lui dit
avec douceur : " Aimerais-tu qu'on fasse une chose pareille avec les
femmes de ta famille ? ", et il se mit à énumérer ses proches parentes :
sa mère, sa sœur et son épouse ; à chaque fois, le jeune

homme répondait : " Non, ça ne me plairait pas ! " Le Prophète conclut
alors : " Eh bien, les gens sont comme toi ; ils n'aiment pas que l'on
fasse cela avec les femmes de leur famille " Puis il mit sa noble main
sur sa poitrine et fit cette invocation : " Mon Dieu, purifie son cœur,
pardonne lui sa faute, et préserve sa chasteté " Par la suite, nulle
chose ne parut plus répugnante à ce jeune homme que la fornication.

Les récits de ce genre sont nombreux dans l'histoire de la vie du
Prophète et de ses compagnons Il y a notamment l'anecdote bien connue du
bédouin qui urina dans un coin de la mosquée D'un seul bond, les
Compagnons se levèrent pour l'expulser sans ménagement, mais le Prophète
- sur lui la prière et la paix - les en empêcha et couvrit l'homme de
son manteau, lui disant {[}même de ne pas se presser Lorsqu'il en eut
terminé, le bédouin s'écria : " Mon Dieu, accorde-nous Ta miséricorde, à
Muhammad et à moi- même, mais ne l'accorde à personne d'autre ! " Le
Prophète dit alors : " Tu limites là quelque chose d'immense, ô bédouin
! ".

Mais toi et moi, avons-nous d'aussi nobles manières ? La douceur ne fait
qu'embellir les choses tandis que la brutalité ne fait que les enlaidir
Voilà une partie de ce que l'on pouvait dire à propos du fait d'ordonner
et d'interdire avec douceur Quant aux qualités de patience et
d'intelligence que doit avoir celui qui ordonne ou interdit, elles ont
généralement un effet bénéfique sur la personne à laquelle il s'adresse,
car elles supposent une réelle sollicitude pour cette dernière La
Révélation y fait ainsi allusion : " Plein de sollicitude envers vous,
bon et miséricordieux à l'égard des croyants", Sourate 9 : Le repentir
(At-Tawbah) verset 128.

Ne pas chercher à avoir le dessus lorsqu'on refuse de vous écouter ou
qu'on vous fait subir des revers en raison de ce que vous ordonnez et
interdisez : voilà un signe de patience et d'intelligence

! Sais-tu qu'au moment où l'une de ses dents fut brisée {[}au cours de
la bataille d'Uhud{]}, le Prophète - sur lui la prière et la paix - se
contenta de dire : " Mon Dieu, pardonne à mon peuple car ils ne savent
pas ".

Mais peut-être n'es-tu pas d'un naturel clément ? Dans ce cas, ton
devoir est d'acquérir cette qualité autant que faire se peut, en vertu
de cette parole du Prophète - sur lui la prière et la paix - : " La
science s'acquiert par l'étude, et c'est en s'efforçant d'être clément
(tahallum) qu'on réalise cette vertu ".

N'as-tu jamais entendu cette parole de Jésus - sur lui la paix - à
propos des destinées de son peuple après lui, telle que nous la rapporte
le Coran : " Si Tu les châties Ils ne sont que Tes serviteurs Et si Tu
leur pardonnes Tu es, en vérité, le Tout Puissant, le Sage ".

Considère l'excellence de cette parole et la bienveillance dont elle
témoigne ! Pourtant, en dépit de l'associationnisme dont son peuple se
rendit coupable par la suite, il n'a pas été jusqu'à dire ce que, toi,
tu as affirmé des gens de la communauté d'Ahmad : qu'ils sont les pires
créatures ; et ceci, simplement parce que d'après toi, c'est pécher que
de vénérer les saints Ton cœur est dur, et tu es sans pitié pour les
croyants : voilà la véritable raison de tes allégations ! Jabir lbn
Abdallâh rapporte du Prophète - sur lui la grâce et la paix - la parole
suivante : " Qui n'est pas miséricordieux envers les hommes, Dieu ne le
sera pas à son égard " C'est donc une qualité particulière que doit
avoir celui qui ordonne ou interdit.

Quand à la compréhension de la religion dont doit faire preuve celui qui
ordonne ou interdit, c'est là le fond du problème, le point central de
toute cette question du commandement du bien et de l'interdiction du
mal, parce que l'incompréhension de la religion d'Allah amène
généralement à statuer au rebours de Son jugement, en ordonnant le mal
ou en interdisant le bien. Quelle abominable façon d'exercer l'autorité
religieuse, en prétendant prescrire ce qui convient !

Pour ta part, ô Cheikh, tu as blâmé dans ton épître le bien le plus
élevé, créant ainsi un trouble immense et vraiment néfaste pour les
musulmans. La personne qui referme ton "Miroir", à supposer que cette
lecture ne lui cause pas un dommage irrémédiable, se mettra dans le
meilleur des cas à douter de sa religion et de son devenir puisque les
actes qu'elle pensait être des offrandes à Allah, lui permettant de se
rapprocher de Lui, lui apparaîtront alors comme

une transgression méritant châtiment. Quel désastre pourrait-il causer
plus de tort à la religion ? "Nous sommes à Allah et nous retournons à
Lui !" (Qoran)

C'est une idée de bon sens, largement partagée, que de penser qu'une
seule réunion en vue du Souvenir efface bon nombre de mauvaises réunions
; sur ce point, la conviction de l'élite et celle du commun des croyants
s'accordent parfaitement. Mais toi, ô Cheikh, tu prétends prouver que
ces réunions en vue du dhikr, quelle que soit la manière de le
pratiquer, ne sont que des innovations blâmables, contraires aux
pratiques des anciens, sans nous préciser ce que sont ces assemblées du
Souvenir que la Loi recommande {[}indubitablement{]}. Vraiment, tu dois
rendre tes lecteurs bien perplexes ! Tout cela résulte probablement de
ton manque de compréhension de la religion divine. Voilà la raison pour
laquelle le Prophète - sur lui la grâce et la paix - subordonnait la
mission d'ordonner le bien et d'interdire le mal à une compréhension
réelle de la religion, pour éviter qu'on n'en arrive à commander
l'inverse de ce qu'il convient comme nous l'avons dit.

Avant d'occuper cette fonction, il faut au préalable avoir bien compris
les notions de bien et de mal, au moyen de définitions claires et
explicitées par la Loi, pour ne pas s'égarer dans la direction inverse
de celle-ci. C'est pourquoi, les plus grands savants sont extrêmement
prudents lorsqu'ils abordent une question religieuse dont aucun texte
explicite ou quasi explicite ne traite. Quant aux questions où nulle
source explicite ne permet de trancher, les décisions prises à leur
égard n'obligent que leur auteur, lequel ne fait qu'émettre une opinion
personnelle, et c'est pourquoi les applications juridiques sont si
variées ; pourtant, l'unité des principes qui les sous-tendent n'en
demeure pas moins sauve : louange à Allah ! Ceci résulte de la facilité
qui caractérise la religion divine, ainsi que l'a dit le Prophète - sur
lui la grâce et la paix - : "Le meilleur culte, c'est le plus facile ;
et la meilleure œuvre, c'est de comprendre la religion (al-Fiqh)".

En conséquence, qui ne la comprend pas devrait s'abstenir d'en parler.
Selon Ibn Abd al-Barr `Atâ' disait ceci : "Celui qui n'est pas au fait
des différences {[}note : Il s'agit, au-delà des différences d'école
juridique, de l'intégration par le Fiqh des spécificités de

chaque lieu, de chaque époque et de chaque groupe humain.{]} Qui
existent entre les gens doit s'abstenir de leur donner des avis
juridiques ; car en ce cas, la science qui lui échappe est largement
plus importante que celle qu'il détient".

Ce que nous disons ici de la nécessité d'approfondir n'intervient
cependant qu'en cas d'ambiguïté. Lorsque le caractère illicite ou
obligatoire d'une chose est établi sans le moindre doute par la
religion, tout musulman au fait de ce statut se doit d'ordonner le bien
et d'interdire le mal à ce sujet - quand bien même il n'en tiendrait pas
compte concernant sa propre personne. Mais ce dont nous devons nous
défier, c'est de cette voie que tu as choisie, ô Cheikh ! Tu interdis ou
autorises en fonction de ton opinion personnelle et de la jalousie que
tu nourris envers les autres. Te laissant entraîner par ta nature et tes
penchants, tu assimiles le bien à ce que tu approuves et décrètes
blâmable ce que tu réprouves !

Mais quelle autorité avez-vous donc en la matière, toi et tes semblables
? Ce sont bien plutôt Allah, Son Prophète et les gens enracinés dans la
science qui en ont la charge ! Pour ta part, contente-toi de blâmer ce
que la religion a clairement déclaré blâmable, et d'ordonner ce dont
elle a indubitablement établi le caractère louable, en l'appliquant avec
résolution en ce qui te concerne ; quant au reste, tu n'as qu'à t'en
remettre à Allah. Et surtout, respecte les différents efforts
d'interprétation des autorités compétentes, qu'elles soient d'entre les
soufis ou non. Ne sais-tu pas qu'il y a des choses ambiguës que telle
école juridique a décidé d'interdire et telle autre d'autoriser, tandis
qu'une troisième incline à leur trouver un caractère recommandable et
qu'une autre encore se contente de les déconseiller ?

Cette question n'exige pas de longues explications ; mais qu'en pense
mon contradicteur ? Lui faut-il qu'un mujtahid {[}le mujtahid est le
savant autorisé à faire un effort d'interprétation, de par sa
science.{]} se plie à l'opinion d'un autre ? Cela n'est pas nécessaire,
à moins d'être aveuglé par une intolérance sectaire telle que celle qui
t'affecte ! Tu voudrais qu'un courant largement majoritaire, qui
rassemble une multitude de gens sur la terre entière, se soumette à ton
faible point de vue, t'imaginant que le soufisme ne s'appuie sur

aucun fondement solide ? Non, par Allah, et tu juges fort mal les gens
du soufisme, ô Cheikh ! Voici la seule réponse que tu mérites (et c'est
aussi valable pour tous ceux qui te ressemblent) : le moindre soufi
montre assurément plus de scrupule que toi dans sa pratique religieuse !
{[}Pour asseoir ton autorité,{]} tu prétends t'appuyer sur Sa Parole -
exalté soit-Il - : "Vous êtes la meilleure communauté suscitée pour les
hommes ; vous ordonnez le bien et interdisez le mal". (Qoran)

A quoi je répondrai que personne ne conteste le sens de ce verset ou des
autres citations que tu fais : ordonner le bien et interdire le mal sont
effectivement des obligations qui incombent à toute personne qui croit
en Allah, au Prophète et au Jour dernier. Ce que je conteste en
revanche, c'est ta façon de donner à ce "mal" auquel il convient de
s'opposer un sens qu'il n'a pas dans ce verset, en y incluant les
réunions du Souvenir et l'ensemble des pratiques du soufisme. Et à mon
avis, ce sont bien plutôt les idées que tu soutiens dans ton "Miroir"
qui mériteraient d'être corrigées.

Sa Parole - exalté soit-Il - : "Vous êtes la meilleure communauté", peut
s'adresser aux croyants d'une façon générale ou à l'élite de ceux-ci.
Pris dans son sens général, ce verset signifie que les croyants sont
chargés, entre toutes les communautés, de commander le bien et
d'interdire le mal ; cette fonction est celle des Prophètes, des Envoyés
et des Véridiques (Siddiqûna), et ils l'exercent à l'égard de l'ensemble
des autres communautés ; dans ce cas, le "mal" est une expression
désignant toute forme d'associationnisme, tandis que le "bien" réfère à
l'attestation de l'Unicité divine et à tout ce qui en découle. Pris dans
son sens particulier, ce verset traite des ordres et interdictions que
les gens de l'élite s'adressent mutuellement ; le "mal" et le "bien"
désignent alors respectivement les mœurs blâmables et louables. Mais
dans ce dernier cas, le pronom "vous" ne s'adresse au fond, à proprement
parler, qu'à ceux qui guident les créatures et les appellent à Allah par
Allah. C'est à leur sujet que le Prophète - sur lui la grâce et la paix
- a dit : "Il y aura toujours sur terre quarante hommes semblables à
{[}Abraham,{]} l'Ami du Miséricordieux. Par eux vous recevrez la pluie,
et par eux vous serez nourris. Chaque fois que l'un d'entre eux mourra,
Allah le remplacera par un autre". (kanz al-'Ummal d'Al Hindi n° 34603
et 34602).

C'est ainsi qu'à chaque Prophète est spirituellement associée une
catégorie de personnes de la communauté de Muhammad - sur lui la grâce
et la paix - ; et ces cohortes qui existent à chaque époque sont au fond
les interlocuteurs les plus directs de cette apostrophe divine. Ils sont
en effet les plus qualifiés pour accomplir cette mission d'ordonner le
bien et d'interdire le mal. Façonnés pour cela de toute éternité, ils
détiennent naturellement les qualités que cette fonction exige. Si
d'autres l'assurent, ce n'est qu'à titre occasionnel et en fonction de
circonstances passagères. Pour ma part, je pense qu'en général, ces
personnes dont il est question n'existent que parmi les gens du
Souvenir, eux qui, selon les termes d'un hadîth qui sera cité plus loin,
"s'abandonnent totalement à l'invocation de Dieu".

Ce n'est que parmi les adhérents du soufisme, ceux-là mêmes que tu
traites d'innovateurs, que l'on rencontre des gens "s'abandonnant
totalement à l'invocation de Dieu" ou "étant follement épris de son
Souvenir", pour reprendre les expressions que l'on trouve dans plusieurs
traditions. Les autres, quels qu'ils soient, n'atteignent pas leur degré
dans l'invocation d'Allah ; les seuls à être du même niveau sont ceux
qui les aiment, leurs ancêtres spirituels et les gens de leur chaîne
initiatique. Bien évidemment, je mets à part les trois premières
générations {[}de musulmans{]} en faveur desquelles le Prophète a
témoigné ; mais tout cela est évident lorsque l'on a vraiment compris ce
que sont le soufisme et les soufis.

Quant à celui pour qui cette expression ne désigne qu'une foule de gens
appartenant à la lie du peuple, il ne risque pas de se faire une idée
exacte du soufisme, identifiant le soufisme, qu'il ne connaît pas, aux
pratiques de ces gens qu'il connaît et appelle lui-même soufis. Mais
quelle différence entre ce dont tu as connaissance et ce soufisme dont
tu ne sais rien ! Par Allah !, mon frère, si la nature du soufisme, son
commencement et son terme t'étaient dévoilés, tu te contenterais de
n'être qu'un enfant en présence des gens d'Allah !

Tu invoques en faveur de ta thèse Sa parole - exalté soit-Il - : "Les
croyants et les croyantes se protègent les uns les autres, ordonnant le
bien et interdisant le mal". (Qorân)

Mais ici, tu ne t'intéresses qu'à la dernière partie du verset et en
négliges le début ! Or celui-ci conditionne pourtant celle-là,
établissant le principe de cette protection mutuelle que doivent
s'accorder les croyants, avec le caractère sacré de leurs biens, de leur
honneur et de leur sang qui en découle. II convient donc de bien définir
la nature de cette foi qui nous oblige à la fraternité, à la
responsabilité et à l'entraide les uns envers les autres.

Qu'est-ce que la foi ? La réponse est simple - mais Allah est plus
savant - puisque le législateur nous l'a Lui-même fournie. Il s'agit de
croire en Allah, à Ses Anges, Ses Livres, Ses Envoyés et au Jour
dernier. Il est obligatoire de protéger celui qui professe cette foi et
interdit de l'agresser. Or c'est bien une telle foi qui caractérise -
mais Allah est plus savant - chaque individu de la communauté, et ce,
malgré la multitude des courants et en dépit des divergences en matière
d'application des principes : tant que ces derniers sont saufs, les
différences restent bénignes. Ainsi, celui qui est autorisé par Allah à
s'exprimer doit s'assurer que, ce faisant, il préserve les liens de
l'Islam et favorise la fraternité religieuse. Il ne doit pas s'attaquer
aux convictions des membres de la communauté ni dénigrer leurs doctrines
ni décréter qu'elles sont fausses, car cela conduirait à des schismes et
des rejets mutuels, supprimant alors toute possibilité d'entente
harmonieuse entre les musulmans.

N'es-tu pas conscient, ô Cheikh, du désarroi de la communauté, fruit des
erreurs du passé ? Voilà à quoi nous a conduit le sectarisme exagéré de
ceux qui n'admettent que leur propre école ! Chacun déshonore l'autre et
le juge en fonction de ses propres convictions. Tous sont pourtant bien
croyants, même si' l'exclusivisme de certains les a conduits à dissoudre
les liens de fraternité religieuse ; ils ont fini par rompre l'unité née
des deux témoignages de foi, de la pratique de la prière, de l'aumône,
du pèlerinage, du jeûne de Ramadan, de la récitation du Qoran et de tous
les principaux rites musulmans.

II était pourtant bien inutile de s'occuper des erreurs du passé ! Par
Allah, qu'as-tu fait, ô Cheikh ! Pourquoi t'es-tu empressé de raviver
les troubles du passé en tentant de saper un des piliers les plus
essentiels de l'Islam, un principe fondamental sur lequel s'appuient les
musulmans et dans le respect duquel ils ont été élevés ? C'est de
l'amour des membres des confréries dont je veux parler. Aujourd'hui, les
musulmans ont des égards pour eux et les vénèrent naturellement ; ils
ont une haute estime du soufisme et de ses adeptes. Mais toi, au
contraire, tu clames qu'il n'est qu'erreur, ignorance et égarement,
entre autres accusations dont tu l'accables ! Tu as ainsi brisé des
cœurs de manière irréparable, à moins de te repentir sincèrement et de
t'excuser.

Tu n'aurais pas dû entreprendre de critiquer cette école avant de savoir
qui l'a instaurée et quels en sont les dix principes : n'exiges- tu pas
toi-même une connaissance préalable de ces éléments pour chaque
discipline ? Ce minimum acquis, tu aurais pu alors en parler à ta guise.
Mais j'ai bien l'impression que tes connaissances sont légères ; ou bien
alors ce sont tes capacités de compréhension qui sont faibles ; ou ce
peut être l'un et l'autre à la fois. Cela expliquerait que rien dans les
textes dont tu disposes, ceux de Zanjânî ou d'Ibn Ajrum par exemple,
n'ait pu te renseigner sur l'art du soufisme.

Si tu t'étais borné ne serait-ce qu'à des abrégés, deux textes au moins
ne t'auraient pas échappé : le Murshid al-Mu'în concernant les œuvres
religieuses et le Jowhar al-Maknûn à propos de la rhétorique. Ces deux
ouvrages intéressent au soufisme : dans le premier, une section
indépendante lui est consacrée {[}en fin d'ouvrage{]} ; le second aborde
le sujet dans le cadre de digressions destinées à attirer l'attention du
lecteur - qu'Allah récompense son auteur. Les as-tu écartés parce que tu
rejettes le soufisme par principe ? Te paraissent-ils négligeables ? Je
n'en sais rien, mais de toute façon, ta critique du soufisme va beaucoup
trop loin ; quoi qu'il en soit, sa renommée nous dispense d'appeler les
témoins à la barre. Enfin, si Allah te prête vie et que tu veux
t'occuper de questions religieuses, voire conseiller les autres dans
leur pratique, fais en sorte que tes propos favorisent l'unité de la
communauté musulmane ; il faut renforcer les liens religieux et la
fraternité musulmane, et laisser de côté les différences de point de vue
dans

l'application des principes. Dis : "O gens du Livre ! Acceptez une
parole qui nous soit commune : nous n'adorons que Dieu et nous ne Lui
associons rien ; que certains d'entre nous n'en prennent pas d'autres
comme seigneurs en dehors de Dieu".

Par Allah, as-tu bien réfléchi au pourquoi de ce verset et à qui il
s'adressait ? Quelle excellente manière de réunir les cœurs ! Mais
quelle différence avec ta manière de procéder ! Peut-être me diras- tu
que ce verset concerne explicitement les gens du Livre ? Eh bien, je
dirai que tu dois au minimum accorder aux soufis le même rang : tu ne
confirmes pas leurs dires, mais ne les traites pas non plus de menteurs.
C'est le minimum de l'équité ; mais qui donc aujourd'hui se préoccupe
d'équité ?

Tu prétends mettre à contribution Ghazali - qu'Allah soit satisfait de
lui -. Mais tes convictions excluent totalement que tu puisses te parer
de son autorité ! Lui, c'est un soufi, alors que toi tu rejettes le
soufisme.

Tu as également recours aux propos du Prophète - sur lui la grâce et la
paix - rapportés par Ibn `Abbâs - qu'Allah soit satisfait de lui -

: "Quiconque délaisse le commandement du bien et l'interdiction du mal
ne croit pas au Qorân...", Mais penses-tu qu'il lui dénie absolument
toute foi ? Non, sinon c'en serait terminé de la communauté ! C'est la
foi parfaite qu'il lui dénie, celle qui résulte de l'acceptation totale
et sincère du message ; cette foi particulière, de nombreux hadîth nous
la décrivent, comme celui-ci par exemple : "Nul d'entre vous n'a la foi
tant qu'il ne désire pas pour son frère ce qu'il souhaite pour
lui-même."

Quant à la foi commune, elle est d'une simplicité totale comme on l'a vu
précédemment. II existe même un célèbre hadîth qui nous la rend encore
plus accessible. On raconte qu'un des Compagnons se devait d'affranchir
un esclave croyant. Il vint donc accompagné d'une servante noire chez le
Prophète - sur lui la grâce et la paix -, voulant que ce dernier juge de
sa qualité de croyante. Le Prophète

- sur lui la grâce et la paix - lui ayant demandé : "Où est ton Seigneur
?" Elle répondit en désignant le ciel de son index. Le Prophète témoigna
alors de sa foi et le Compagnon affranchit cette femme. En citant Ibn
`Arafa, tu confirmes toi-même que ce n'est

pas la foi au sens général qui est visée {[}dans le hadîth cité{]} ; car
pour cet auteur, l'obligation de commander le bien et d'interdire le mal
incombe à la communauté dans son ensemble et non à chaque individu en
particulier. Voilà ! Tu commences par édifier une forteresse au moyen du
hadîth, puis c'est une ville entière que tu démolis avec cette citation
d'Ibn `Arafa ! On se demande vraiment pourquoi tu enchaînes ces hadîth,
dont la formulation semble montrer que chacun des musulmans est
concerné, si c'est pour conclure finalement que l'obligation en question
incombe à la communauté d'une façon collective ! Mais dis-moi au fait :
pourquoi donc en serais-tu responsable, toi, plutôt qu'un autre ?
Puisque tu manifestes des velléités d'écrire, sache qu'une simple
accumulation de citations est inutile ; les références scripturaires
doivent être citées à propos et conformément à leur sens, et c'est même
là une forme de cette sagesse dont Il a dit - exalté soit-Il - : "Celui
auquel est donnée la sagesse bénéficie d'un grand bien".

Quant au hadîth que tu cites : "N'est pas des nôtres celui qui n'est pas
miséricordieux avec nos enfants et n'honore pas nos vieillards", il va
dans le même sens que tout ce qui vient d'être signalé concernant {[}la
manière{]} de commander le bien et d'interdire le mal. Mais au vu des
références que tu as sélectionnées, j'ajouterai que, en un certain sens,
les "enfants" symbolisent le commun des croyants de la communauté - car
ils sont {[}humbles et donc{]} "petits", quand bien même ils seraient
très âgés -, tandis que les "vieillards" en représentent l'élite,
indépendamment de l'âge. On juge en effet l'homme à sa réalité
intérieure et non à ses caractéristiques physiques. Tu comprends mieux
maintenant en quoi ce hadîth te concerne, car toi, tu n'as pas fait
preuve de miséricorde à l'égard des "enfants", c'est-à-dire des
musulmans en général ; au lieu de t'adresser à eux avec gentillesse et
douceur, comme un père âgé parle à son jeune fils, tu les as rudoyés et
accablés de tes reproches. Tu n'as pas plus honoré les "vieillards",
c'est-à-dire ceux qui sont les sources de la sagesse et les soutiens de
la religion de cette communauté ; dénonçant leurs prétendues erreur et
ignorance, tu t'es plu à les considérer comme des ennemis, osant faire
référence au hadîth rapporté par Ibn

`Abbâs dans lequel le Prophète dit - sur lui la grâce et la paix - :
"Recherchez la faveur d'Allah grâce à certains transgresseurs..." Les
assimiler à des transgresseurs ! Par Allah, quelle impudence !

Comment peux-tu appliquer aussi facilement ce hadîth à des gens qui se
réunissent pour invoquer Allah et pratiquer d'autres œuvres du même
ordre ?

En résumé, toutes ces preuves amassées pour montrer qu'il est
obligatoire de commander le bien et d'interdire le mal ne prêtent pas à
discussion. C'est le sens que tu donnes à l'expression "mal" qui est
hautement contestable, car tu finis par déclarer tel ce qui est
intrinsèquement un bien ou, en tous cas, une réalité plus proche de la
vérité que de l'erreur.

Sois certain qu'il est préférable pour toi d'avoir tort lorsque tu
cherches à réformer les pratiques religieuses de tes frères, plutôt que
de voir tes critiques s'avérer en fin de compte justifiées.
Ignorerais-tu que l'honneur des musulmans doit être préservé, tout comme
leurs biens et leur vie ? Et cela, du simple fait qu'ils ont prononcé
les deux témoignages de foi. Tu cites la Risâla d'Ibn Abî Zayd
al-Qayrawânî - qu'Allah soit satisfait de lui - : "Commander le bien et
interdire le mal sont des obligations qui incombent à tous ceux qui
exercent le pouvoir temporel ou disposent d'une autorité quelconque.
S'il est impossible d'agir, on le fera par la parole, et si cela s'avère
également impossible, on le pensera en son for intérieur".

L'auteur se réfère ici à un hadîth que je me permets de citer, au cas où
tu n'en aies pas connaissance : "Celui d'entre vous qui est témoin d'un
mal doit s'y opposer en actes, en paroles s'il ne le peut, et en son
cœur sinon : c'est le degré le plus faible de la foi".

Voilà une excellente méthode pour commander le bien et interdire le mal
! Mais rapporter d'Ibn `Arafa que le commandement du bien et
l'interdiction du mal ne sont qu'une obligation collective ne milite
vraiment pas en faveur de cette épître que tu as entrepris de rédiger !
Pauvre de toi ! Si seulement tu t'étais borné à citer les quelques
hadîth qui précèdent ! Ils montrent en effet que commander le bien et
interdire le mal s'impose à toute personne distinguant le bien du mal ;
que le licite et l'illicite sont clairement identifiés ; qu'il faut
s'abstenir de trancher dans les cas ambigus ; et que la manière de
réagir face au mal est nécessairement fonction des individus, puissants
ou faibles selon les cas, et des situations :

quiconque a la possibilité de modifier le cours des événements, le
détenteur du pouvoir politique par exemple, doit agir et ne peut se
soustraire à cette obligation (à supposer que son pouvoir soit réel) ;
les savants musulmans, qui n'ont pas cette fonction, doivent s'y opposer
en paroles ; enfin, celui que les circonstances rendent impuissant doit
s'y opposer par le cœur, ce qui est le degré le plus faible de la foi
comme le dit le hadîth.

Tu énonces ensuite quelques phrases sans consistance, affirmant qu' "il
est obligatoire de se conformer à la Vérité, à la Tradition de Muhammad,
et de suivre les traces des pieux anciens- qu'Allah soit satisfait
d'eux. Ils avaient en effet pour habitude d'aimer les partisans de la
Tradition, de les estimer hautement et de les vénérer, tandis qu'ils
délaissaient au contraire ceux qui s'en détournaient, ne leur
accordaient aucune importance et les détestaient. Cette nature était
tellement ancrée en eux que, pour atteindre un rang élevé à leurs yeux,
il fallait manifester son orthodoxie : même un personnage peu
recommandable n'avait d'autre solution que d'être considéré comme un
partisan de la Tradition."

Concernant l'obligation "de se conformer à la Vérité", je dirai que
c'est effectivement de la plus impérieuse nécessité, mais seulement
lorsqu'on la connaît de façon très claire. Celui qui est dans le doute
et que Satan a violemment frappé, comment pourrait-il connaître la
Vérité ? A supposer qu'il en vienne à La connaître, cela ne pourrait de
toute façon se produire que par l'intermédiaire des humains ; il lui est
donc impossible de se conformer {[}directement{]} à la Vérité, à moins
bien sûr qu'Allah - qu'Il soit exalté - n'ouvre sa vision intérieure et
purifie ses pensées intimes de toute basse supputation à l'égard des
Justes. L'Imam `Ali - qu'Allah soit satisfait de lui - a dit : "Ne sois
pas de ceux qui connaissent la Vérité par le truchement des hommes, mais
connais la Vérité directement ; tu connaîtras alors Ses gens".

Tu décris les pieux anciens comme aimant les partisans de la Tradition.
Mais qui, parmi ceux qui ont foi en Allah et en Son Prophète, n'aime pas
les gens de la Tradition ? Le Prophète - qu'Allah lui accorde la grâce
et la paix - n'a-t-il pas dit : "Quiconque n'éprouve pas d'amour n'a pas
de foi" ?

Ignores-tu que les soufis, ceux-là mêmes que tu accuses d'erreur,
d'ignorance et d'égarement, ont instauré l'amour comme base de leur voie
? A moins - mais Allah est plus savant - que tu entendes par "gens de la
Tradition" les personnes dans ton genre et non les musulmans d'une façon
générale ! Selon les termes de ta piètre prose, les anciens
"délaissaient les gens se détournant de la Tradition, ne leur
accordaient aucune importance et les détestaient".

Jusque-là, rien ne permettait d'identifier ces adversaires de la
Tradition, mais tu as alors spécifié clairement : "comme les soufis de
notre époque ". En lisant cela, je me suis dit : "Ca y est ! Le bébé
dont le Cheikh vient d'accoucher se met à crier !" Ce mal auquel tu
faisais allusion, objet de toute cette épître, est maintenant bien
identifié : il s'agit du soufisme, calamité des plus graves selon toi !
Et toutes ces turpitudes que tu détailles par la suite ne sont que des
digressions, puisque l'essentiel d'un essai figure en introduction, à
moins bien sûr de supposer que tu aies voulu introduire ton épître par
une mention des soufis à titre de bénédiction : cela m'étonnerait
vraiment !

Finalement, tout ce mal et toutes ces innovations blâmables auxquels tu
fais allusion sont circonscrits par cette précision : "comme les soufis
de notre époque" ; en dehors d'eux, il n'y a donc rien de nuisible dont
il faille se préserver. Cela dit, puisque tu limitais ta critique aux
soufis de notre temps, tu n'aurais pas soulevé notre colère si tu t'en
étais tenu là, mais voilà ! Il a fallu que tu cites les propos de
Turtûshî, pour qui le courant du soufisme en général n'est qu'erreur,
ignorance et égarement. Pauvre de toi ! Si seulement ses paroles
n'étaient pas parvenues à tes oreilles ! Ton cœur aurait pu en effet
rester vierge de toute critique à l'égard des guides spirituels du
passé, et Allah - qu'Il soit exalté - n'aurait eu alors à trancher
qu'entre tes contemporains et toi-même.

Tu continues : "La plupart de nos contemporains se sont empêtrés dans
les mensonges qu'ont forgés les innovateurs, ces gens qui se détournent
lorsqu'on s'oppose à leurs innovations et coutumes répréhensibles non
autorisées, même en dehors des écoles juridiques habituelles".

En parlant de ceux qui "se sont empêtrés dans les mensonges qu'ont
forgés les innovateurs", ne ferais-tu pas allusion aux groupes de
disciples ? Si c'est le cas, alors quel audacieux juriste tu fais et de
quelle belle sagacité tu fais preuve ! L'inconscient s'imagine que son
absence de retenue est une preuve de bravoure, sans se rendre compte que
"la retenue fait partie de la foi".

Plus retorses et plus fielleuses encore sont tes allégations selon
lesquelles personne n'autoriserait leurs prétendues innovations, "même
en dehors des écoles juridiques habituelles". Mais bien sûr, tu as tout
exploré et résumé pour nous - Allah te bénisse ! Mais par Allah, quelles
sont-elles ces innovations non autorisées ? S'agit-il des réunions de
disciples où l'on invoque Allah - qu'Il soit exalté - et l'on rappelle
les gens à Lui ? Vises-tu l'invocation en groupe et à voix haute ?
Veux-tu parler des invocations rythmées par le mouvement du corps ou de
leurs efforts pour provoquer l'illumination spirituelle ? Ces trois
choses sont-elles ce dont tu t'es éreinté à rechercher les traces dans
les recueils des écoles juridiques sans y trouver de permission ? J'ai
l'impression que tu n'en as pas trouvé mention, pas même dans la
catégorie des choses déconseillées ; et d'ailleurs, même si cela avait
été le cas, les actes déconseillés n'en sont pas moins légalement permis
: voilà ce qui aurait dû modérer ton ardeur !

La raison que tu avances pour prouver qu'ils sont des innovateurs est
assez comique : "car soient ils prétendent que le savant entreprenant
(c'est peut-être de toi qu'il s'agit !) Entrave leur liberté, soit ils
affirment que c'est l'instigateur de leurs innovations qui a raison."
C'est donc pour cela que tu les accuses de s'adonner à de blâmables
innovations pour lesquelles on ne trouve aucune autorisation ? Quelle
étrange rhétorique ! Quelle singulière méthode !

Tu ajoutes ensuite : "Parfois, ils l'injurient et se moquent de lui".
Peut-être semblable mésaventure t'est-elle arrivée ? De telles
expériences, aussi pénibles soient-elles, n'ont rien d'étonnant dans ton
cas : c'est la réponse du berger à la bergère. La manière dont tu t'y
prends pour commander le bien, interdire le mal et appeler à Allah -
qu'Il soit exalté -, ne témoigne pas d'une grande science : voilà la
raison d'une telle mésaventure ; tu n'as pas respecté les

consignes qu'Allah - qu'Il soit exalté - a transmises à Son Prophète

- qu'Allah lui accorde la grâce et la paix - quant à la façon d'appeler
les gens à Allah : "Appelle les hommes à la voie de ton Seigneur par la
sagesse et une belle exhortation ; et ne discute avec eux que de la
meilleure manière" . Sourate 16 : Les abeilles (An- Nahl) verset
125.....

*****

\emph{(Note : Un peu plus loin dans cet ouvrage, le Cheikh al-Alawî
répondait au Cheikh Uthman Ibn al-Makki qui faisait des Soufis l'un des
groupes qui iront en enfer selon la parole du Prophète - sur lui la
prière et la paix -}

\emph{: " Ma communauté se divisera en soixante-dix et quelques groupes
Tous sont voués à l'enfer sauf un : c'est le groupe de ceux qui auront
suivi cette voie qui est la mienne et celle de mes Compagnons ", le
Cheikh al- Alawî eut la réponse qui va suivre. Fin de note).}

Mais pourquoi donc ne cites tu pas le hadith qu'a rapporté l'Imam
Ghazali dans son Fasl al-tafriqât ? Le Prophète a dit : " Ma communauté
se séparera en soixante-dix et quelques groupes Ils iront tous au
paradis, excepté le groupe des hérétiques " Bien sûr, ton regard n'est
pas tombé sur ce hadith ! Il s'est arrêté à ce qui t'arrangeait pour
promettre le feu au reste des Musulmans et vous réserver exclusivement
le paradis, à tes semblables et à toi même Dis : " Si la demeure
dernière auprès de Dieu vous est réservée, à l'exclusion de tout autre,
souhaitez donc la mort si vous êtes sincères ! " Mais ils ne la
désireront jamais à cause des œuvres qu'ils ont accomplies Dieu connaît
bien les injustes ", (Sourate 2, Verset 94).

J'imagine que tu dois te demander comment l'on peut concilier ces deux
paroles du Prophète Tu ne trouveras qu'un soufi pour résoudre cette
difficulté ou d'autres du même ordre. Malheureusement tu ne pourras
t'abaisser à le questionner, car la jalousie a clos en toi la porte de
l'objectivité et t'empêche de reconnaître tes carences Quoi qu'il en
soit, je dirai ce que Dieu a révélé {[}à ce soufi{]} ; à supposer que
n'en aies pas besoin, cela pourra toujours servir aux autres.

Ces deux paroles sont aisément conciliables Il suffit pour cela de
considérer que le terme " communauté " désigne l'ensemble de ceux
auxquels le message est prêché dans le premier hadith, et l'ensemble de
ceux auxquels qui répondent à cet appel dans le second.

Le sens s'éclaircit dès lors que l'on exploite la forme complète du
hadith, qui est le suivante Le Prophète a dit : " Les Juifs se sont
séparés en 71 groupes et les Chrétiens en 72. Quant à ma communauté,
elle se séparera en 73 groupes ; tous sont voués à l'enfer sauf un :
c'est le groupe de ceux qui suivent cette voie qui est la mienne et
celle de mes Compagnons ".

La succession mentionnée met en évidence qu'il existait 70 religions
(croyances) avant la venue de Moïse - sur lui la paix -, la sienne
constituant la 71 ème. Ces groupes sont voués à l'enfer, en dehors de
ceux qui ont suivi cette voie qui était la sienne - sur lui la paix - et
celle de ses compagnons. L'ensemble des 71 groupes peut être appelé sa "
communauté " parce qu'il était l'Envoyé de Dieu pour cette époque, et
que sa prédication d'adressait donc à eux Après la venue de Jésus - sur
lui la paix -, qui complète le chiffre de 72, tous les groupes autre que
ceux qui suivaient sa voie et celle de ses disciples sont destinés au
feu.

Ahmad - sur lui la prière et la paix - fut par la suite envoyé avec la
religion Ahmadienne simple.

\emph{(Note : Allusion au hadith : " J'ai été envoyé avec la Hanîfiyya
as- Samha", la Hanîfiyya désigne le monothéisme abrahamique pur, de ce
fait l'Islam est une religion facile "Samha", c'est à dire simple,
conformément au verset coranique (22, 78) : " Il ne vous a imposé aucune
gène dans la religion ; la religion de votre père Abraham ". Ahmad est
le nom "céleste" du Prophète Muhammad. Fin de note.)}

La Hanîfiyya as-Samha qui correspond au 73ème des groupes mentionnés ;
tous sont voués à l'enfer sauf un : c'est le groupe de ceux qui suivent
cette voie qui est la sienne et celle de ses Compagnons. Et là encore,
le mot "communauté" désigne l'ensemble des gens auxquels sa prédication
s'adresse ; il disait en effet - sur lui la prière et la paix - : "Je
suis l'Envoyé de Dieu pour

tout homme vivant à mon époque ou né après moi". (Note : On pourrait
s'étonner de trouver sous la plume du Cheikh al-Alawi un développement
aussi exclusiviste à l'égard des non-musulmans et aussi tolérant pour la
généralité des musulmans. En réalité, ce passage correspond surtout à ce
qui pouvait être dit, compte tenu du contexte de l'Algérie de ce temps,
de l'époque et, par dessus tout, des limitations des personnes
auxquelles s'adressait cette épitre : lorsque l'on a déjà bien du mal à
convaincre que les soufis n'iront pas nécessairement en enfer, on
n'entreprend pas d'aller explicitement à contre courant des idées ayant
cours parmi bon nombre de musulmans au sujet des chrétiens et des juifs,
entre autres.

Il faut donc souligner que l'interprétation du hadith comporte toujours
plusieurs niveaux Ici, le Cheikh opère une première transposition du
sens du terme "communauté", celle qui convient à son interlocuteur et à
ses lecteurs, c'est à dire un public exclusivement musulman Cependant
d'autres interprétations plus universalistes des notions de "communauté"
et de "voie" prophétique sont possibles Signalons d'ailleurs que, selon
M. Chodkiewicz, "{[}pour Ibn Arabi,{]} le statut ultime et totalisateur
de la Shari'a dont le Prophète est porteur a pour effet de valider les
législation précédentes, lorsque les communautés qui y restent attachées
paient la jiziyya, la capitation : par là même, en effet, elles sont
incluses dans la communauté Muhammadienne" Mais d'un certain point de
vue - lorsque le Prophète est envisagé dans sa réalité spirituelle de
Principe Prophétique, celle qui correspond au hadith rapporté par
Tirmidhî : "J'étais Prophète alors qu'Adam se trouvait entre le corps et
l'esprit" -, c'est l'humanité toute entière qui constitue sa
"communauté", et chaque révélation historique exprime alors un aspect de
sa "voie". (kanz al-'Ummal d'Al Hindi n°31917)

Rappelons enfin que l'Islam est explicitement universaliste, l'un des
fondements scripturaires de cette ouverture étant le verset coranique
(2,62) : En vérité les croyants, les juifs, les chrétiens, les sabéens,
ceux qui croient en Dieu et au Jour dernier et agissent justement, voila
ceux qui trouveront leur récompense auprès de leur Seigneur Ils
n'éprouveront alors plus aucune crainte et ne seront pas affligés Le
Cheikh al-Alawi en donne le commentaire

suivant dans son Bahr al-Masjûr : "Le fait de citer côte à côte ces
différents groupes, et de ne pas distinguer les croyants {[}musulmans{]}
des autres, doit nous conduire à ne considérer personne, musulman ou
infidèle, pieux ou transgresseur, comme nous étant inférieur, et ce
toute notre vie durant : en effet, notre destin nous est inconnu, et
c'est l'état de notre foie au moment de la mort qui compte Les hommes,
du point de vue de la prédestination, sont tous à égalité {[}{]}

Après lui, la religion Ahmadienne s'est divisée, selon le deuxième
hadith, en soixante-dix et quelques groupes ; ils représentent les
différentes écoles et les approches divergentes, dont les partisans
iront tous au paradis, à l'exclusion des hérétiques.

Voila ce qu'exigent la bonté Muhammadienne et la miséricorde divine !
S'il n'en était ainsi, c'est la presque totalité de la communauté qui
serait perdue, puisque seule une partie sur soixante-dix et quelques
serait sauvée ; d'ailleurs, en l'occurrence, rien ne permet d'identifier
clairement cette partie, et ce qui le prouve, c'est que chaque groupe
prétend être l'heureux élu (Note : Référence au hadith suivant rapporté
par Bukhâri et Muslim, (kanz al-'Ummal d'Al Hindi n°1135 et 1136).

Le Prophète a dit : "Dieu - exalté soit-il - a dit : "Je suis conforme à
l'opinion que Mon Serviteur se fait de Moi". Dans d'autres variantes de
ce hadith, le discours divin continue ainsi : "Alors qu'il pense de Moi
ce qu'il veut", ou encore : "Alors qu'il ait une bonne opinion de Moi ".

Quant à moi, j'affirme que Dieu - gloire à Lui - est conforme à la
{[}bonne{]} opinion qu'ont de Sa Personne ceux qui croient en Lui, à Son
Prophète et au Jour dernier, lorsqu'ils font un effort pour se
rapprocher de Lui S'ils tombent juste, deux récompenses leur échoient
(Note : C'est à dire l'une pour la sincérité de l'intention et l'autre
pour le bon résultat : ce sont les termes d'un hadith rapporté par
Muslim (n°4261) à propos de la fonction de juge. Fin de note), dans
l'hypothèse inverse, ils en obtiennent au moins une. Il sont donc
récompensés quoi qu'il arrive, que tu le veuilles ou non, car les
créatures ne sont pas dans l'obligation d'être infaillibles ; elles sont
simplement tenues d'essayer d'être dans le

vrai, et cela s'explique par la "largesse" de la voie Ahmadienne, à
laquelle fait allusion ce verset : Il ne vous a imposé aucune gène dans
la religion (Qoran 22,78) En témoigne également le hadith (marfû')
rapporté par Tabarânî, selon lequel le Prophète a dit : "300 chemins
(tarîqa) différentes mènent à ma loi (Shari'a) Il suffit de suivre l'un
d'entre eux pour être sauvé" Mais ce qui corrobore plus encore cette
idée, c'est le hadith rapporté par Suyûtî dans son Jâmi' al-Saghîr,
selon lequel le Prophète a dit : "Dans toute communauté, une partie des
gens va au paradis tandis qu'une autre se retrouve dans le feu, sauf
dans le cas de ma communauté qui, toute entière, ira au paradis" (kanz
al-'Ummal d'Al Hindi n°34484 ) , et - s'il plait à Dieu - il en sera
bien ainsi !

\emph{Extrait de : "Lettre ouverte à celui qui critique le soufisme" du
Cheikh Ahmed al-Alawî. Traduit par M. Chabry. Edition : La Caravane}.
\end{quote}
