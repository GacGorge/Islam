%\chapter{Théologie des religions et \textit{Laudato Si’}}

\begin{comment}
  \paragraph{Instructions - } A rendre avant le 15 juin
30 pages;
bibliographie critique de chaque livre, probalématique et méthode.
Se concentrer sur un auteur : Laudato Si’
  
\end{comment}



Nous utiliserons les raccourcis suivants : 
\begin{itemize}
    \item LS : Laudato Si', encyclique du pape François \cite{francois_laudato_2015}, mai 2015
    \item EG : Evanglii Gaudium, exhortation apostolique du Pape François, novembre 2013
    \item GS : Gaudium \& Spes, constitution pastorale de Vatican II, décembre 1965
\end{itemize}
%-------------------------------------------------------------------------------------------------------
\section{Problématique proposée} 
%-------------------------------------------------------------------------------------------------------

 

Le pape François mentionne 5 fois le mot \textit{religions} dans son encyclique LS, plus que EG (3) et GS (1). Pourquoi une telle importance dans un texte sur la \textit{sauvegarde de la maison commune} ? Notre hypothèse est que ce regard sur les religions, important et positif, est la conséquence du diagnostic théologique de la crise écologique,  \textit{un \emph{style} de vie hégémonique lié à un mode de production} (LS 145). Reprenant ce terme de \textit{style} (19 occurrences dans LS), notre approche s'appuiera sur le \textit{style} chrétien proposé par LS comme hypothèse de la théologie doctrinale sous-tendant le texte : il nous faudra d'abord montrer la proximité de l'approche de style pensée par Merleau-Ponty et repris par Theobald \cite{theobald_christianisme_2007} avec celle du pape. Ensuite, nous étudierons comment ce style s'enracine dans la Bible en particulier autour des figures bibliques de l'idolâtrie, du prophète et de l'Apocalypse (certaines étant traditionnellement attribuées aux autres religions). Enfin, nous nous proposons d'étudier comment les ressources  de ce \textit{style chrétien} sont organisées dans l'encyclique et permettent de répondre au contexte mondiale de la crise écologique. Pour cela, il nous semble nécessaire de faire un détour par l'enjeu d'une éthique universelle pour le christianisme et plus largement du rapport entre LS et la doctrine sociale de l'Eglise. En effet, il convient de comprendre le rapport de l'Eglise à la société et à la vérité et pourquoi le recours des autres religions est pertinent. 


Dans une deuxième partie, nous étudierons les conséquences pour la théologie chrétienne des religions : comment penser le christianisme comme \textit{style} de dialogue avec les autres religions (pour reprendre le terme de l'encyclique) ?  Nous positionnerons cette approche parmi les diverses théologies chrétiennes pensant le pluralisme religieux (exclusiviste, inclusiviste, pluraliste, post-libéral et celle de Theobald), en montrant les points de convergences et les différences. Nous étudierons aussi le déplacement théologique par rapport notamment au texte de la CTI \cite{commission_theologique_internationale_christianisme_1997}.
Enfin, nous appliquerons cette approche en essayant de proposer ce que LS peut dire pour l'Islam, dans l'esprit du travail de Theobald sur le judaisme et l'Islam \cite{theobald_christianisme_2007}. 
 


\section{Proposition de plan}


\subsection{Le Style de Laudato Si'}

\paragraph{Travail sur EG et LS} Travail personnel sur l'utilisation des mots \textit{Religions} et \textit{idolâtrie} dans EG et LS. 

\paragraph{Importance du Style de Laudato Si’} \textit{La première réforme, le style chrétien,} François ; Antonio Spadaro, sj \cite{francois_premiere_2013}; \textit{Le Christianisme comme Style - Laudato Si'} - Christoph Theobald \cite{theobald_courage_2021}, \textit{Laudato Si’ : un changement dans ce que signifie la conversion ?} - Patrick Goujon \cite{goujon_laudato_2022}

\subsection{Les ressources chrétiennes face à la crise écologique}



\paragraph{Une éthique universelle pour répondre à l'enjeu universel de la crise écologique ?} \textit{De quel genre de pensée a-t-on besoin pour aborder la crise environnementale contemporaine ?} Howles, Timothy ; Kremer, Robert \cite{howles_quel_2022};  \cite{thomasset_recherche_2019}, \textit{La Charte de la terre} citée par LS, \textit{Manifeste pour une éthique planétaire}, Kuschel, Karl-Josef ; Küng, Hans ; \cite{kuschel_manifeste_1995} : \textit{L’éthique planétaire, d’un point de vue philosophique}, Küng, Hans \cite{kung_lethique_2009} \textit{La recherche d’une éthique universelle dans la tradition catholique. La méthode de Laudato Si’ }- A. Thomasset 

\paragraph{Ressources bibliques pour une réponse chrétienne}
\textit{Quelques mots avant l’Apocalypse - lire l’Evangile en temps de crise} - A. Candiard \cite{candiard_quelques_2022}
\textit{Between Exile and the New Jerusalem : Prophetic Mourning, Lament and the Ecological Crisis} - Daniel Castillo \cite{cavanaugh_between_2018}; \textit{Création à l’âge de l’anthropocène} - Christoph Theobald \cite{theobald_repenser_2019},  \textit{Parler de la création après Laudato si’}, Lasida, Elena \cite{lasida_parler_2020}

\paragraph{LS et la doctrine Sociale de l’Eglise} \textit{La doctrine sociale de l’Eglise selon François} - Ch. Theobald \cite{theobald_lenseignement_2016}, \textit{Economie, idolâtrie et sécularisation depuis Gaudium et Spes} - W. Cavanaugh \cite{cavanaugh_fragile_2018}


\subsection{Pour une théologie du dialogue comme théologie chrétienne des religions} 

\paragraph{Positionner LS dans les différences approches de la théologie des religions} au delà des références au cours \textit{Christologies au défi de la culture pluraliste} et \textit{Théologie Chrétienne des Religions} : \textit{Le christianisme et les religions}, Commission Théologique Internationale\cite{commission_theologique_internationale_christianisme_1997}, cité par EG, \textit{L’ unique christ : la symphonie différée}, Christian Duquoc \cite{duquoc_unique_2002}; \textit{Dieu au pluriel : penser les religions ,} Rémi Cheno \cite{cheno_dieu_2017}; \textit{La nature des doctrines. Religion et théologie à l’âge du postlibéralisme. }, George Lindbeck \cite{lindbeck_nature_2002}

\paragraph{Quelles ressources pour les autres religions } \textit{l’Unique et ses témoins - Jalons pour une
théologie de la rencontre entre juifs, chrétiens et musulmans}, Christoph Theobald \cite{theobald_christianisme_2007},  

\paragraph{Un exemple pratique : ce que les chrétiens peuvent dire à l'Islam sur ses ressources à mobiliser face à la crise écologique} \textit{Laudato Si’ : Engaging Islamic Tradition and Implications for Legal Thought}, Powell, Russell \cite{powell_laudato_2017}, : \textit{Ecologie et Religions -
Colloque IDEO 2022,} Pisani, Emmanuel ; Candiard, Adrien ; Hilal, Aziz ; Revol, Fabien \cite{pisani_ecologie_2022}, \textit{Écologie en islam et dialogue interreligieux,} Pisani, Emmanuel \cite{pisani_ecologie_2016}, \textit{The religious vision of nature in the light of Laudato
Si’ : An interreligious reading between Islam and Christianity}. Puglisi, Antonino ; Buitendag, Johan \cite{puglisi_religious_2020}, \textit{Anthologie du soufisme}, Hamès Constant. \cite{hames_vitray-meyerovitch_1988}, cité par LS (Alî al-Khawwâç)

\newpage

%-------------------------------------------------------------------------------------------------------
\subsection{L'Ecologie est un enjeu majeur pour les religions}

%-------------------------------------------------------------------------------------------------------
\paragraph{Question de la pertinence des religions}
\begin{singlequote}
        Qu’une religion soit raisonnable [donc universelle] dépend largement de ses
pouvoirs d’assimilation, de sa capacité à fournir dans ses propres termes une
interprétation intelligible des diverses situations et réalités que rencontrent
ses adhérents. Les religions que nous qualifions de primitives échouent régulièrement
à ce test quand elles sont confrontées à des changements importants,
tandis que les religions mondiales développent de plus grandes ressources pour
faire face aux vicissitudes \cite[ p. 175]{lindbeck_nature_2002}.
\end{singlequote}
%-------------------------------------------------------------------------------------------------------
\paragraph{Enjeu majeur pour les religions : pertinence par rapport au changement climatique}

 Un enjeu de pertinence pour les religions : \textit{mondiale}, \textit{existentielle}, \textit{ne se joue pas à l'échelle individuelle mais d'une transformation collective}

\paragraph{penser un salut collectif mais à travers une démarche qui entraîne tout le monde } d'une certaine façon nous oblige à définir ce qu'est le \textit{salut écologique} 
\label{Comment:MemoireISTR1}






%-------------------------------------------------------------------------------------------------------
\subsection{Le Style de Laudato Si’}  

Style; approche de Theobald; donc regarder construction; derrière le style "pastorale", une vraie théologie

%-------------------------------------------------------------------------------------------------------
\paragraph{La doctrine sociale de l'Eglise selon François} \cite{theobald_lenseignement_2016}

\cite{theobald_repenser_2019}

%-------------------------------------------------------------------------------------------------------
\paragraph{Première lecture de Laudato Si’ : six occurrences du mot Religions} et en particulier la section V du Chapitre V : Les religions dans le dialogue avec les sciences. \cite{francois_laudato_2015}
\begin{singlequote}
     nous ne pouvons pas ignorer qu’outre l’Église catholique, d’autres Églises et communautés chrétiennes – comme aussi d’autres religions – ont nourri une grande préoccupation et une précieuse réflexion sur ces thèmes qui nous préoccupent tous » [LS 7)
        Dans le sillage du concile Vatican II, l’encyclique insiste sur la contribution des religions en tant que vecteur d’une vision et d’une relation à la nature qui permet de répondre aux défis environnementaux et de proposer une alternative ancrée dans une sagesse séculaire pour éviter « l’indifférence, la résignation facile ou la confiance aveugle dans les solutions techniques » [LS 14]. Elles constituent une richesse « pour une écologie intégrale et pour un développement plénier de l’humanité » [LS 62].  « Tous, nous pouvons collaborer comme instruments de Dieu pour la sauvegarde de la création, chacun selon sa culture, son expérience, ses initiatives et ses capacités » [LS 15].  \cite{francois_laudato_2015}
\end{singlequote}
\begin{singlequote}
    201. La majorité des habitants de la planète se déclare croyante, et cela devrait inciter les religions à entrer dans un dialogue en vue de la sauvegarde de
la nature, de la défense des pauvres, de la construction de réseaux de respect
et de fraternité. 
\end{singlequote}
 
 


%-------------------------------------------------------------------------------------------------------
\paragraph{Comment penser Dialogue inter religieux et changement climatique} Notre hypothèse sera que Laudato Si’ n'est pas un simple texte de circonstance, qui doit \textit{cocher des cases} et en particulier le \textit{dialogue inter-religieux}, avec une articulation "Et" : dialogue interreligieux \textit{et} conversion écologique. Mais au contraire, reconnaître le style profondément théologique et construit de l'encyclique. 
 
Dès lors, trouver la pointe du \textit{dialogue inter-religieux} ne peut faire l'économie d'un travail théologique. 

%-------------------------------------------------------------------------------------------------------
\paragraph{Approche retenue : lecture de Laudato Si’ à la lumière de Theobald} Style. Proximité jésuite : dogmatique dans la pastorale. "Tout se tient". 


\paragraph{Apocalytique} p. \pageref{theob:apocalytique}
Discussions sur la place de la dimension apocalytique ? Espérance.


\begin{comment}
    

%-------------------------------------------------------------------------------------------------------
\subsection{LS et la doctrine sociale de l'Eglise selon François}  

%-------------------------------------------------------------------------------------------------------
\paragraph{Idée : changement de Style qu'il faudra qualifier et que Style c'est dogmatique}


%-------------------------------------------------------------------------------------------------------
\subsection{Laudato Si’, entre Prophétisme et Dialogue}  

%-------------------------------------------------------------------------------------------------------
\paragraph{La réponse traditionnelle de l'Eglise}  lien avec la doctrine sociale de l'Église, articulation de la justice nécessaire et de l'action individuelle et collective.  Alors que GS, marquée par la sécularisation, en restait aux principes et laissait l'autonomie à l'économie \cite{cavanaugh_idolatrie_2022}.   François : pas d'autonomie de l'économie (thomiste/sécularisation) qui en s'autonomisant, a pris comme religion l'argent. Permet de voir la nouveauté. 

%-------------------------------------------------------------------------------------------------------
\paragraph{En dialogue} Le titre de l'encyclique \textit{Maison Commune - Oikos} en est une première piste. une logique sapientielle "dialogue". Un nouveau rapport à la vérité. ouverture aux autres religions qui sont appelées à relever ensemble ce défi.

%-------------------------------------------------------------------------------------------------------
\paragraph{Conversion, Prophétisme et dénonçant l'idôlatrie} "Conversion", du registre prophétique
{Laudato Si’} prend un \textit{kairos} et en ce sens, devient prophétique, car parle en situation au sein d'une crise. Mal qui perturbe le peuple et Dieu. \textit{epistrophe}, retour vers Dieu, en ayant horreur de son comportement (metanoia). 


%-------------------------------------------------------------------------------------------------------
\paragraph{Comment articuler les deux}   A la différence Dt, qui critiquait fortement les religions extérieures et toutes les compromissions, ici, il semble que nous ayons un paradoxe : positivité des religions non chrétienne et négativité des "compromissions" mais par rapport à une "religion non nommée".


 %-------------------------------------------------------------------------------------------------------
\subsection{Théologie des religions}
 
\paragraph{Limites des approches pluralistes}

\paragraph{Approche retenue : la théologie de Theobald}

 %-------------------------------------------------------------------------------------------------------
\subsection{Ce que les autres religions peuvent dire de la crise écologique ? }
 
\end{comment}


%-------------------------------------------------------------------------------------------------------
\section{Le Style de Laudato Si’ - Bibliographie}
%-------------------------------------------------------------------------------------------------------

%-------------------------------------------------------------------------------------------------------
\subsection{Lecture de Laudato Si’ - mentions de \textit{Religions}}
%-------------------------------------------------------------------------------------------------------

\begin{singlequote}
        207. La Charte de la Terre nous invitait tous à tourner le dos à une étape d’autodestruction et à prendre un nouveau départ, mais nous n’avons pas encore développé une conscience universelle qui le rende possible. Voilà pourquoi j’ose proposer de nouveau ce beau défi : “Comme jamais auparavant dans l’histoire, notre destin commun nous invite à chercher un nouveau commencement [...] Faisons en sorte que notre époque soit reconnue dans l’histoire comme celle de l’éveil d’une nouvelle forme d’hommage à la vie, d’une ferme résolution d’atteindre la durabilité, de l’accélération de la lutte pour la justice et la paix et de l’heureuse célébration de la vie”.[148]
\end{singlequote}
       
\paragraph{Miser sur un autre style de vie}
\begin{singlequote}
        203. Étant donné que le marché tend à créer un mécanisme consumériste compulsif pour placer ses produits, les personnes finissent par être submergées, dans une spirale d’achats et de dépenses inutiles. Le consumérisme obsessif est le reflet subjectif du paradigme techno-économique. Il arrive ce que Romano Guardini signalait déjà : l’être humain « accepte les choses usuelles et les formes de la vie telles qu’elles lui sont imposées par les plans rationnels et les produits normalisés de la machine et, dans l’ensemble, il le fait avec l’impression que tout cela est raisonnable et juste ».[144] Ce paradigme fait croire à tous qu’ils sont libres, tant qu’ils ont une soi-disant liberté pour consommer, alors que ceux qui ont en réalité la liberté, ce sont ceux qui constituent la minorité en possession du pouvoir économique et financier. Dans cette équivoque, l’humanité postmoderne n’a pas trouvé une nouvelle conception d’elle-même qui puisse l’orienter, et ce manque d’identité est vécu avec angoisse. Nous possédons trop de moyens pour des fins limitées et rachitiques.

\end{singlequote}
\paragraph{les religions dans LS}    
 \begin{singlequote}
        « nous ne pouvons pas ignorer qu’outre l’Église catholique, d’autres Églises et communautés chrétiennes – comme aussi d’autres religions – ont nourri une grande préoccupation et une précieuse réflexion sur ces thèmes qui nous préoccupent tous » [LS 7]
        Dans le sillage du concile Vatican II, l’encyclique insiste sur la contribution des religions en tant que vecteur d’une vision et d’une relation à la nature qui permet de répondre aux défis environnementaux et de proposer une alternative ancrée dans une sagesse séculaire pour éviter « l’indifférence, la résignation facile ou la confiance aveugle dans les solutions techniques » [LS 14]. Elles constituent une richesse « pour une écologie intégrale et pour un développement plénier de l’humanité » [LS 62]. Il s’agit donc pour toutes les religions de puiser dans « leur propre héritage éthique et spirituel », de revenir « à leurs sources » pour « mieux répondre aux nécessités actuelles » [LS 200]. « Tous, nous pouvons collaborer comme instruments de Dieu pour la sauvegarde de la création, chacun selon sa culture, son expérience, ses initiatives et ses capacités » [LS 15]. Cette crise, source de migrations violentes et contenant en elle la possibilité prochaine des guerres, peut aussi être un lieu de rencontre, de dialogue et d’action [LS 15] entre tous les hommes. Dans une perspective dont on a souligné les accents blondéliens.
    \end{singlequote}       

 
        %Juan Carlos Scannone, « La filosofia dell’azione di Blondel   le pape y voit la possibilité de susciter une communion d’action afin d’ouvrir à une « nouvelle solidarité universelle » [LS 14).
 

    
\paragraph{les religions dans le dialogue avec les sciences}    
\begin{singlequote}
        199. On ne peut pas soutenir que les sciences empiriques expliquent complètement la vie, la structure de toutes les créatures et la réalité dans son ensemble. Cela serait outrepasser de façon indue leurs frontières méthodologiques limitées. Si on réfléchit dans ce cadre fermé, la sensibilité esthétique, la poésie, et même la capacité de la raison à percevoir le sens et la finalité des choses disparaissent.[141] Je veux rappeler que « les textes religieux classiques peuvent offrir une signification pour toutes les époques, et ont une force de motivation qui ouvre toujours de nouveaux horizons [...] Est-il raisonnable et intelligent de les reléguer dans l’obscurité, seulement du fait qu’ils proviennent d’un contexte de croyance religieuse ? ».[142] En réalité, il est naïf de penser que les principes éthiques puissent se présenter de manière purement abstraite, détachés de tout contexte, et le fait qu’ils apparaissent dans un langage religieux ne les prive pas de toute valeur dans le débat public. Les principes éthiques que la raison est capable de percevoir peuvent réapparaître toujours de manière différente et être exprimés dans des langages divers, y compris religieux.

        200. D’autre part, toute solution technique que les sciences prétendent apporter sera incapable de résoudre les graves problèmes du monde si l’humanité perd le cap, si l’on oublie les grandes motivations qui rendent possibles la cohabitation, le sacrifice, la bonté. De toute façon, il faudra inviter les croyants à être cohérents avec leur propre foi et à ne pas la contredire par leurs actions ; il faudra leur demander de s’ouvrir de nouveau à la grâce de Dieu et de puiser au plus profond de leurs propres convictions sur l’amour, la justice et la paix. Si une mauvaise compréhension de nos propres principes nous a parfois conduits à justifier le mauvais traitement de la nature, la domination despotique de l’être humain sur la création, ou les guerres, l’injustice et la violence, nous, les croyants, nous pouvons reconnaître que nous avons alors été infidèles au trésor de sagesse que nous devions garder. Souvent les limites culturelles des diverses époques ont conditionné cette conscience de leur propre héritage éthique et spirituel, mais c’est précisément le retour à leurs sources qui permet aux religions de mieux répondre aux nécessités actuelles.

        201. La majorité des habitants de la planète se déclare croyante, et cela devrait inciter les religions à entrer dans un dialogue en vue de la sauvegarde de la nature, de la défense des pauvres, de la construction de réseaux de respect et de fraternité. Un dialogue entre les sciences elles-mêmes est aussi nécessaire parce que chacune a l’habitude de s’enfermer dans les limites de son propre langage, et la spécialisation a tendance à devenir isolement et absolutisation du savoir de chacun. Cela empêche d’affronter convenablement les problèmes de l’environnement. Un dialogue ouvert et respectueux devient aussi nécessaire entre les différents mouvements écologistes, où les luttes idéologiques ne manquent pas. La gravité de la crise écologique exige que tous nous pensions au bien commun et avancions sur un chemin de dialogue qui demande patience, ascèse et générosité, nous souvenant toujours que « la réalité est supérieure à l’idée ».[143]
\end{singlequote}

\paragraph{champ lexical biblique}
\begin{itemize}
    \item  7 citations du livre de la Sagesse, \textit{poumon d'Israël pour respirer l'air commun} (Paul Beauchamp) \cite{goujon_laudato_2022} 
    \item 18 occurrences du vocabulaire stylistique \cite{theobald_courage_2021}. 
    \item 8 occurrences du mot \textit{religion}
\end{itemize}
\begin{comment}
    https://rpubs.com/datadataguy13/1011253
    https://pypi.org/project/lexicalrichness/#example-use-cases
    https://newscatcherapi.com/blog/ultimate-guide-to-text-similarity-with-python
\end{comment}
%-------------------------------------------------------------------------------------------------------
\subsection{Importance du Style de Laudato Si’ - Bibliographie}
%-------------------------------------------------------------------------------------------------------

%-------------------------------------------------------------------------------------------------------
\paragraph{Le Christianisme comme Style - Christoph Théobald} -  \textit{in }\textit{le Courage de penser l'avenir} \cite[p 169-196]{theobald_courage_2021}. Le pape François aime le vocabulaire stylistique. Ainsi, le terme \textit{style} est-il utilisé vingt-deux fois dans EG et dix-huit dans LS. Malheureusement, le terme n'est pas défini précisément. S'agit-t-il de la même définition que celle qu'en donne l'A. dans son livre éponyme \cite{theobald_christianisme_2007} \textit{Le Christianisme comme style}  ? 
\begin{comment}
    Nous nous proposons de lire ce chapitre soulignant la notion de \textit{recours} des religions et la distinction   entre \textit{ressources et sources} déjà entrevu dans \cite{theobald_lenseignement_2016}. 
    notion de \textit{recours}. Distinction entre \textit{ressources et sources}.  Style prophétique et contemplatif de \textit{Laudato Si’}. Nous n
\end{comment}
Reprenant l'analyse que LS fait du monde (présentée p. \pageref{theo:diagnosticLS}), la réponse ne peut être de remédier aux symptômes :
\begin{singlequote}
     il faut adopter résolument un «regard différent» et différencié sur le réel: « une pensée, une politique, un programme éducatif, un style de vie et une spiritualité qui constitueraient une résistance face à l'avancée du paradigme technocratique » [LS, 111]. Ce nouveau style de vie est donc le noyau de ce que l'Encyclique appelle une « \textit{écologie intégrale} »
\end{singlequote}

Ce style comprend vertus solides, apprentissage de "petites actions quotidiennes" ancrées par l'exercice et la répétition [LS 211], style de vie prophétique et contemplatif, qui n'est pas obsédé par la consommation [LS 222], avec un horizon global (\textit{la maison commune}) et local.
L'A. postule donc que le concept de \textit{style} sous-jacent à l'encyclique est celui de la phénoménologie de Merleau-Ponty :

\begin{singlequote}
« Tout style est la mise en forme des éléments du monde qui permettent d'orienter celui-ci vers une de ses parts essentielles ». Il y a signification lorsque les données du monde sont par nous soumises à une « déformation cohérente » \cite[p. 55, cité par l'A.]{merleau-ponty_signes_1960}

\end{singlequote}
L'opération stylistique consiste à envisager la métamorphose du monde. En ce sens, notre action est, dans le vocabulaire judeo-chrétien, \textit{messianique} avec une visée eschatologique et critique du monde actuel, très présente dans LS et EG. L'A. souligne l'articulation très précise entre une conscience aigue de la complexité du monde de la vie et la critique simultanée de ce monde dans ses ressorts d'aliénation de l'autre, comme le montre l'exemple de la conception des villes et des espaces publiques [LS 150 ss] ou la description très précise et scientifiquement fondée des enjeux écologiques dans le premier chapitre de LS.

Dans un premier temps, le style même de l'encyclique LS est performatif : il réalise ce qu'il annonce. L'introduction par la citation du Cantique de François d'Assise s'adresse à l'affectivité du lecteur et l'introduit dans le  mystère joyeux que nous  contemplons dans la joie [LS 12].

Il s'agit ensuite de penser l'articulation entre doctrine et pastorale : 

\begin{singlequote}
    François, quand il introduit le vocabulaire stylistique et parle du «style de vie de l'Évangile» et d'un «style évangélisateur» se sert-il d'une terminologie secondaire, en quelque sorte conventionnelle et conforme à une mentalité post-moderne, mais qui n'ajoute rien de neuf [\ldots] ? Ou bien ce vocabulaire véhicule-t-il une spécificité théologique, voire doctrinale qui ne peut être exprimée autrement, une spécificité dont les textes eux-mêmes seraient conscients?  \cite[p. 180]{theobald_courage_2021}
\end{singlequote}
L'A. fait l'hypothèse de cette spécificité doctrinale, en s'appuyant sur les nombreuses références aux \textit{spirituels} et \textit{mystiques}. Face au "rêve d'une doctrine monolithique, défendue par tous sans nuances" (EG 40), il nous faut apprendre l'art apostolique de circuler entre la cohérence du mystère chrétien (EC 39) et la simplicité du cœur de l'Évangile (EG, 34-45) d'une part et la diversité culturelle et personnelle de ses destinataire d'autre par, ultimement assurée par l'Esprit Saint et incarnée dans des \textit{charismes ou styles religieux}. L'A. reprend ici la présentation des 4 principes vus p. \pageref{theo:principesLS} pour renforcer l'hypothèse d'un choix conscient du pape de placer le \textit{doctrinal} au service de la \textit{pastorale}.
\begin{singlequote}
    «le christianisme n'a pas un modèle culturel unique, mais tout en restant pleinement lui-même, dans l'absolue fidélité à l'annonce évangélique et à la tradition ecclésiale, il revêtira aussi le visage des innombrables cultures et des innombrables peuples où il est accueilli et enraciné ».  [EG ]
\end{singlequote}
L'A. étudie ensuite la nouveauté des styles proposés par EG 18 (\textit{"Style évangélisateur déterminé"}) et LS 222 (\textit{"style de vie prophétique et contemplatif" }), en montrant que la seconde expression peut se comprendre comme une détermination de la première.
Dans LS, la foi chrétienne est présentée de manière \textit{étonnamment neuve}, non pas sous l'angle de la vérité et du registre de l'apologétique, mais comme une \textit{ressource} dont dispose l'humanité pour répondre à la crise écologique. François propose une démarche de conversion, à oser transformer en souffrance personnelle ce qui se passe dans le monde et ainsi à reconnaître la contribution que chacun peut apporter [LS 19]. \label{theo:mourning}

Face à la complexité de la crise, à l'insuffisance des réponses actuelles [LS 2], il est nécessaire d'avoir recours aux diverses richesses culturelles des peuples, à l'art et à la poésie, à la vie intérieure et à la spiritualité \cite[p. 188]{theobald_courage_2021}.
La spécificité chrétienne se traduit par une double argumentation :
\begin{singlequote}
La première, d'ordre épistémologique, consiste à articuler les différents niveaux du réel et à faire dialoguer les différentes disciplines: les sciences et la technique, l'économie et la politique, toujours traités - sur le fond - de manière positive, avant de critiquer ce qui risque de fausser leur apport: l'instinct de puissance et de domination et, surtout, leur spécialisation, apparemment innocente (n° 110), qui occulte cependant de plus en plus les grandes questions et, en particulier,
    l'horizon éthique de l'agir humain avec les principes fondamentaux du « bien commun » et de la « justice entre générations » (traités à la fin du chap. IV). Nous retrouvons ici le combat de François contre le paradigme homogène et unidimensionnel de la technocratie et pour une approche différenciée du réel. Ainsi est ménagée une ouverture où peuvent et doivent intervenir des «ressources» d'énergie autres, nécessaires pour affronter les mutations des mentalités et les conversions collectives qui nous attendent: «ressources » d'énergie intérieure, proprement spirituelle, qui ne relèvent ni de la science et des techniques, ni de la diplomatie et de la gestion économique de nos biens, mais de la culture et de la sagesse religieuse \cite[p. 189]{theobald_courage_2021}.
\end{singlequote}
L'autre argument relève de la théologie de la création, la sagesse faisant le lien entre les deux : 
\begin{singlequote}

    Peu fréquent dans l'Encyclique mais suffisamment précis et significatif, le vocabulaire de la «sagesse» s'inscrit, d'un côté, dans celui de la «culture » prise dans toute sa complexité; il comprend, de l'autre côté, l'ensemble des traditions religieuses, tout en spécifiant le récit biblique (selon le titre qui introduit les n° 65 à 75).Dès le premier chapitre, le texte insiste sur les conditions d'accès à la « vraie sagesse », « fruit de la réflexion, du dialogue et de la rencontre généreuse entre les personnes » (LS, 47; voir aussi 63), honorant ainsi les exigences d'une approche différenciée et non homogène du réel. La sagesse désigne donc une « manière de vivre » ou un « style» spécifique qui forme un « héritage » ou un «trésor», mais sans cesse à redécouvrir au sein de nos cultures, conditionnées par leurs limites et leurs préjugés (LS, 200). Dans le chap. II, cette manière de vivre est rapportée à Dieu et à sa propre sagesse, au travail au sein même de la création et de l'histoire (LS, 69) où elle fonde les traits caractéristiques du style biblique et évangélique, déjà évoqués au début de ce parcours, en particulier le rapport intrinsèque entre nos relations fraternelles avec les autres et avec la nature (LS, 70).  \cite[p. 189]{theobald_courage_2021}.
\end{singlequote}
La nouveauté de LS est donc de présenter la tradition chrétienne comme \textit{ressource} dont la spécificité est un style \textit{prophétique et contemplatif}, les deux dimensions étant liés. \textit{Prophétique} d'abord par la critique d'un anthropocentrisme despotique (LS 68,69, 119,122) et contre une compréhension erronée d'une théologie de la création comme mandat donné à l'homme de dominer la terre.  Il s'agit aussi de réinvestir la présence de l'Esprit Saint (LS 80) et du mystère du Christ dans l'ensemble de la réalité naturelle, sans pour autant en affecter l'autonomie (LS 99). La dénonciation prophétique de LS n'édulcore par l'insoutenable mais ne revêt aucune dimension catastrophique, équilibrée par l'acte d'espérance, marque du style contemplatif \cite[p. 192]{theobald_courage_2021}.  \label{theob:apocalytique} Etre contemplatif, c'est \textit{capable d'apprécier profondément les choses sans être obsédé par la consommation} (LS 222). François suit la théologie de Saint Bonaventure (LS 66 et 233) 
\begin{singlequote}
    « La contemplation est d'autant plus éminente que l'homme sent en lui-même l'effet de la grâce divine et qu'il sait trouver Dieu dans les créatures extérieures »
\end{singlequote}
Cette contemplation permet d'entrevoir le lien mystérieux entre la fraternité avec les plus fragiles et la relation aux mondes et aux créatures [LS 84-92, 221-232]. C'est éclairé par l'approche stylistique [LS 121] qu'il faut comprendre le terme d'\textit{écologie intégrale} utilisé par l'encyclique.
L'A. termine en s'interrogeant sur les ressources de l'approche stylistique de la tradition chrétienne par rapport à la question du mal ou de la violence, tout en soulignant que l'"incomplétude" de la réponse est un élément essentiel de la pensée théologique de François \cite{francois_premiere_2013}.







%-------------------------------------------------------------------------------------------------------
\paragraph{Laudato Si’ : un changement dans ce que signifie la conversion ? - Patrick Goujon} \cite{goujon_laudato_2022} L'A. est jésuite, directeurs des RSR, et intervient ici dans le cadre d'une conférence tenue en 2022 au Centre Sèvres sur la conversion écologique. En quoi consiste l'appel à la conversion écologique de LS ? que devient Dieu dans cette conversion ? Cependant, cette conversion est moins en un sens prophétique que selon une démarche de sagesse, dans le sens où elle dessine un avenir commun dans le \textit{dialogue}. 
Reprenant l'article conversion \cite{lacoste_conversion_2007} de A. Wénin, 
\begin{singlequote}
    Dans un temps où menace la guerre, pèse sur le peuple le risque de l’idolâtrie, de son péché, « d’un mal qui perturbe la relation entre Israël et Dieu ». L’appel presse à revenir vers Dieu (conversio, epistrophè) en abandonnant les idoles et modifiant son action, en prenant en horreur son comportement passé (aversio, metanoia). Centrée sur Dieu, dont le prophète rappelle la miséricorde (Osée) et le jugement (Isaïe), la prédication prophétique insiste sur la responsabilité personnelle et collective, tout en révélant que la conversion est un don de Dieu (Ez 26,25-32) : « Je vous donnerai un cœur nouveau, je mettrai en vous un esprit nouveau ». Aversion, retour et alliance avec Dieu sont les termes clés du mouvement que veut provoquer le prophète par sa parole. \cite[p. 392]{goujon_laudato_2022}
\end{singlequote}

On retrouve ces éléments prophétiques de l'aversion, de la conversion et de l'alliance avec Dieu dans LS : 
\begin{singlequote}
     Invitant à une transformation du rapport aux biens et aux autres, en liant « clameur de la terre et clameur des pauvres », François reste dans la lignée morale qui unit appel à la sainteté, personnelle, et à la justice sociale, selon des traits propres aux invocations des prophètes. Une forme d’ascèse s’impose, une limitation volontaire de notre consommation en réponse à l’avidité et au gaspillage.\cite[p. 393]{goujon_laudato_2022}
\end{singlequote}

Cependant, une autre démarche sous-tend LS, la sagesse, marquée par la \textit{rencontre} de l'autre : 
\begin{singlequote}
     La vraie sagesse, fruit de la réflexion, du dialogue et de la rencontre généreuse entre les personnes, ne s’obtient pas par une pure accumulation de données qui finissent par saturer et obnubiler.  [LS 47]
\end{singlequote}




\begin{singlequote}
 [l'encyclique] tend à l’extrême [les opinions], entre la réponse néolibérale à la crise par la confiance au progrès technique qui trouvera de nouvelles solutions et à l’autre extrémité, l’affirmation de la nécessaire réduction de la présence humaine comme seule possibilité de la sauvegarde de la planète. Et de conclure : « Entre ces deux extrêmes, la réflexion devrait identifier des scénarios futurs, parce qu’il n’y a pas qu’une seule issue » [ES 60]. \cite[par. 5]{goujon_laudato_2022}
 \end{singlequote}
 En affirmant l’existence d’issues multiples, la théologie fondamentale est mobilisée en invitant au dialogue entre science et religion, idée forte de LS :
 \begin{singlequote}
« il est nécessaire d’avoir aussi recours aux diverses richesses culturelles des peuples, à l’art, à la poésie, à la vie intérieure, à la spiritualité » [ES 63]
\end{singlequote}


Pour analyser la situation, “\textit{la réalité est supérieure à l’idée}” : partir de la réalité est par ailleurs un impératif pour les chrétiens car il est lié à l'incarnation de la Parole  [LS 233]. Pour l’analyse et pour l’action, on peut rappeler un quatrième principe avancé par le pape François : “le tout est supérieur à la partie”. Dans EG 235, il précise qu’il faut toujours élargir le regard […] sans pour autant se déraciner”.  
Puis l'A. souligne la spécificité du christianisme dans cette démarche de dialogue [LS 121]. Le discours spirituel doit reconnaître le caractère unique de ce monde, autrement dit qu’il ne suggère en rien un autre monde, qui serait « spirituel », double du monde matériel, mais bien un même monde que l’on peut habiter spirituellement. B. Latour \cite{latour_jubiler_2002} s'interrogeait sur la possibilité d'un discours spirituel à une époque où la communication se réduit à l'information instantanée. Une parole spirituelle demeure pourtant possible.

\begin{singlequote}
Il y a peut-être une façon spirituelle de parler dans ce monde, qui diffère en effet radicalement du transport d’information double clic, mais il n’y a pas de “monde spirituel” en supplément de l’autre »\cite[p.40]{latour_jubiler_2002}
\end{singlequote}
La parole spirituelle doit alors être transformative, à travers la conversation
\begin{singlequote}
transformant quelqu’un qui était lointain en proche – la conversion \cite[p.45]{latour_jubiler_2002}
\end{singlequote}
C'est précisément l'approche de LS d'un langage performatif, langage dans toutes ses formes y compris poésie.
 
La pratique du dialogue se déduit d’une théologie de la création, ce que Laudato Si’ appelle l’harmonie de toute la création. 
La sagesse, la capacité de « saisir la variété des choses dans leurs relations multiples » [LS 86] selon la définition de Saint Thomas d'Aquin, est \textit{pédagogie} de la diversité du monde commun, tout en nous faisant pressentir l’unité de ce qui anime nos existences.
\begin{singlequote}
    le fruit de ce dialogue est de faire naître de l’intérieur même des interlocuteurs ce souci commun de la terre, de l’ensemble de la création, et des êtres, souci de la terre (et non pas « sauvegarde ») et souci d’autrui. \cite[par. 7]{goujon_laudato_2022}
    \end{singlequote}

LS s'adresse à tous et pas uniquement les chrétiens. 
Cependant, cela n'empêche pas de proposer ce que peut être une spiritualité écologique chrétienne [ES 216] non par prosélytisme mais pour dire aux chrétiens comme leur relation au Christ  affermit leur conversion écologique [LS 217]. 
Le pape présente alors la figure de François d'Assise [LS 218], modèle de la joie  [LS 10] qui signe la prophétie chrétienne illuminée de l'expérience pascale : 



\begin{singlequote}
La figure de François d’Assise fait transition entre le propos sapientiel et prophétique de l’appel à la conversion et la conversion à laquelle appelle Jésus-Christ. Le lecteur a déjà été initié à la force de transformation que recèle la figure de François d’Assise, comme étant celui qui par excellence prend \textit{soin} (\textit{cura}, en italien, \textit{cuidado} en espagnol, \textit{care} en anglais) dans la joie et la paix (LS 10, puis 222 et suivants). François d’Assise occupe une position de sage. Il est comme un point de passage entre chrétiens et non-chrétiens. Il joue maintenant comme figure de sainteté, conduisant au Christ et à ce qu’il peut inspirer dans la vie des croyants, attitudes que l’on retrouve chez l’un comme chez l’autre. Ces attitudes de «\textit{fraternité sublime avec la création}» [LS 221], de joie, de paix, d’amour civil et politique (comme le détaillent les n° 222 à 232) vont permettre de franchir le seuil de la foi chrétienne. \cite[par. 10]{goujon_laudato_2022}\end{singlequote}

La rencontre de l'autre crée une tension : si on va vers l'autre, la rencontre, la vraie sagesse, comme revenir, se convertir à Dieu seul ? 
A côté de la sobriété heureuse et de la paix, vertus écologiques permettant le vivre ensemble, la conversion implique Dieu : 
\begin{singlequote}
    Le critère théologal montre comment la conversion écologique commune se joue pour le chrétien comme retour à Dieu, en tant que donateur du monde, comme reconnaissance de sa présence cachée dans le monde unique, faisant naître ainsi du sein même de cette reconnaissance l’attitude nécessaire du soin à prendre d’autrui et de la terre, mais aussi une attitude religieuse de reconnaissance, la louange de Dieu, à même la contemplation du monde. \cite[par. 11]{goujon_laudato_2022}
    \end{singlequote}
Ainsi, nous répondons à notre vocation commune à être rassemblé \textit{tout en tous} (1 Co 15,28 selon LS 20)
    \begin{singlequote}
    
L’itinéraire du Christ présente aux chrétiens engagés dans la militance écologique la figure de la Croix, comme mise en échec, renoncement à la violence et espérance qui a de quoi dérouter nos impatiences à vouloir convertir les chrétiens, ou tout homme, à l’écologie. Le second critère, théologal, qui reconduit à la reconnaissance de l’œuvre de Dieu, ne s’ajoute pas de l’extérieur à l’expérience que nous faisons de ce monde : il surgit comme un chant d’émerveillement et une supplication, termes sur lesquels se conclut Laudato Si’. \cite[par. 12]{goujon_laudato_2022}
    \end{singlequote}
La forme sapientielle n'affaiblit pas le caractère prophétique ni ne nie le combat et le cri de la terre et des pauvres. 
       \begin{singlequote}
Cette figure nouvelle de l’humanité transformée apparaît à la toute fin de l’encyclique, dans un passage auquel j’ai failli ne pas prêter attention parce qu’il m’avait semblé d’abord annexe, comme une finale pieuse, reprenant la figure imposée d’une invocation mariale. « Marie, la Mère qui a pris soin de Jésus, lit-on dans les derniers numéros (241), prend soin désormais de ce monde blessé, avec affection et douleur maternelles ». Marie est convoquée comme la figure de l’humanité convertie, pour avoir médité dans son cœur toute la vie de Jésus, poursuit l’encyclique. Mais ce qui apparaît dans cette conversion, ce n’est pas la disparition de la souffrance du monde, mais la compassion. Marie « compatit à la souffrance des pauvres crucifiés et des créatures de ce monde saccagées par le pouvoir humain ». La conversion ne nous fait pas entrer dans un monde spirituel, délivré du mal, mais nous donne d’entendre et de prendre soin de celles et ceux, humains et non-humains, qui souffrent du même sort que Jésus le crucifié, et d’œuvrer dans le temps de ce monde, avec compassion, temps défini comme un « entre-temps » (244), entre le temps présent et celui de la fin où l’on entendra d’une voix forte « Voici, je fais l’univers nouveau » (Ap 21,5, cité n° 243). Cette espérance, qui vient de l’avenir vers nous, voix prophétique, soutient nos luttes (244) et le soin que nous prenons avec compassion pour la planète et ses pauvres. \cite[par. 12]{goujon_laudato_2022}
\end{singlequote}


 


%-------------------------------------------------------------------------------------------------------
\section{Les ressources chrétiennes face à la crise écologique - Bibliographie}
%-------------------------------------------------------------------------------------------------------

%-------------------------------------------------------------------------------------------------------
\subsection{Une éthique universelle pour répondre à l'enjeu universel de la crise écologique ?}
%-------------------------------------------------------------------------------------------------------
\paragraph{Une éthique universelle pour répondre à l'enjeu universel de la crise écologique ?} \textit{De quel genre de pensée a-t-on besoin pour aborder la crise environnementale contemporaine ?} Howles, Timothy ; Kremer, Robert \cite{howles_quel_2022};  \cite{thomasset_recherche_2019}, \textit{La Charte de la terre} citée par LS, \textit{Manifeste pour une éthique planétaire}, Kuschel, Karl-Josef ; Küng, Hans ; \cite{kuschel_manifeste_1995} : \textit{L’éthique planétaire, d’un point de vue philosophique}, Küng, Hans \cite{kung_lethique_2009} \textit{La recherche d’une éthique universelle dans la tradition catholique. La méthode de Laudato Si’ }- A. Thomasset 



%-------------------------------------------------------------------------------------------------------
\paragraph{La Charte de la terre citée par le pape}


\begin{singlequote}
  La Charte de la Terre nous invitait tous à tourner le dos à une étape d’autodestruction et à prendre un nouveau départ, mais nous n’avons pas encore développé une conscience universelle qui le rende possible. [LS 207]
\end{singlequote}


%    vérifier que charte de la terre et Manifeste =

%-------------------------------------------------------------------------------------------------------
\paragraph{Manifeste pour une éthique planétaire} \cite{kuschel_manifeste_1995} 
 
\begin{singlequote}
    une approche pluraliste :
éthique (ou ethos) planétaire, c’est à dire un accord fondamental en matière d’axiologie, de critères indiscutables et de choix essentiels. A défaut d’un consensus éthique fondamental, toute communauté court tôt ou tard le risque du chaos ou de la dictature.  Un ordre mondial meilleur ne peut se concevoir sans éthos planétaire  \cite{kuschel_manifeste_1995}
(préface. p6) 
{...}
“ethique planétaire ne signifie ni idéologie planétaire, ni religion mondiale unitaire à côté des religions existantes, ni quelque forme syncrétique de toutes les autres religions. Notre humanité est lasse des idéologies unitaires et les diverses religions du monde sont de toute manière si différentes dans l’expression de leurs croyance et dans leurs dogmes, dans leur symbolique et leurs rites, que tout effort d’”unification” est dénué de sens. P 6
\end{singlequote}

\begin{singlequote}
    Un principe se retrouve depuis des milliers d’années dans beaucoup de traditions religieuses et éthiques de l’humanité qui l’ont conservé, c’est la “règle d’or”; ce que tu ne veux pas qu’on fasse à ton endroit, ne le fais pas à l’endroit d’aucun autre. \cite{kuschel_manifeste_1995} P 23
\end{singlequote}


\label{Comment:MemoireISTR2}




%-------------------------------------------------------------------------------------------------------
 \paragraph{Kung : éthique planétaire}
 \cite{kung_lethique_2009}

\begin{singlequote}
Le projet d’éthique planétaire se situe dans la foulée de l’éthique de la responsabilité de Max Weber. Il propose une fondation rationnelle de l’éthique (voir K.-O. Apel et J. Habermas). L’être humain jouit d’une autonomie intramondaine mais ne peut fonder seul l’universalité de l’obligation éthique. Onze thèses fondatrices sont alors énoncées comme, par exemple : le jeu a besoin de règles; le fair-play suppose l’observation de normes; éthique n’équivaut pas à doctrine sociale mais à conscience, conviction et attitudes morales ; les règles éthiques peuvent être fondées à partir de la raison sans référence transcendante, etc.
\end{singlequote}

\label{Comment:MemoireISTR3}



\begin{singlequote}
            Il faut plutôt chercher à atténuer, par une solution pragmatique de problèmes urgents, les oppositions entre visions du monde, sans tenir compte des différences idéologiques : cela pourrait à long terme établir des points communs, y compris justement un \textit{éthos} commun. Le conflit des visions du monde ou des idéologies devrait être apaisé de cette façon. \cite{kung_lethique_2009}
\end{singlequote}

%-------------------------------------------------------------------------------------------------------
\paragraph{Critique de l'approche} 

\begin{singlequote}
  [les théologiens contemporains] ont, pour la plupart, sous des formes diverses, essayé de découvrir un horizon commun, impliquant à la fois le mouvement globale de l'histoire et le devenir des religions; Pour ma part, je ne pense pas que l'on puisse lui donner un contenu défini ou concret, on peut à la rigueur le viser avec des formulations formelles empruntées à ce que H. Küng appelle une "éthique planétaire". Cette perspective, il est vrai, demeure floue et ne suscite pas l’enthousiasme soulevé par les grandes utopies du XX siècle. Elles ont démontré par leur réalisation politique leur caractère illusoire et trop souvent dérisoire et cruel.Il me paraît donc inutile de désigner un horizon commun qui serait supposé favoriser le dialogue en proposant une base minimale d'accord.   \cite[p 243]{duquoc_unique_2002}
\end{singlequote}
\begin{singlequote}
 la principale critique adressée aux théologies pluralistes, c’est leur prétention à disposer d’un lieu tiers, d’un arrière-plan qui se situerait au delà des religions particulières et à partir duquel on pourrait les embrasser toutes : le plan nouménal de la Réalité ultime pour John Hick, une même expérience mystique pour Raimon Pannikar \textit{ou encore un même projet éthique pour la justice et la gestion écologique des ressources de notre planète.}   p. 111  \cite{cheno_dieu_2017}
\end{singlequote}

%-------------------------------------------------------------------------------------------------------
\paragraph{La recherche d'une éthique universelle dans la tradition catholique. La méthode de Laudato Si’ - A. Thomasset} \cite{thomasset_recherche_2020}
\begin{singlequote}
    La tradition catholique s’est toujours souciée de garder une dimension universelle à son discours éthique. Historiquement, cette réflexion théologique qui souhaite s’adresser à tous s’est d’abord fondée sur la notion de loi naturelle, un concept susceptible de deux interprétations. L’encyclique du pape François\textit{ Laudato Si’} se présente comme une nouvelle manière de faire et un laboratoire de la recherche visée : elle allie une perception adéquate de la crise, une vision de ce qui est recherché, un dialogue à tous les niveaux et la proposition de ressources spirituelles. Un tel processus dynamique insiste sur la double nécessité de convertir nos attitudes et de rechercher un consensus sur nos manières d’être, nos styles de vie. Il souligne le rôle des religions et des sagesses pour fournir une « mystique qui nous anime ».\cite{thomasset_recherche_2020}
\end{singlequote}

La théologie morale a d'abord cherché une éthique universelle sous la forme d'une loi naturelle, d'inspiration stoïcienne à partir de normes tiées de l'observation ou d'une réflexion sur l'ordre naturel créé par Dieu. 

\begin{singlequote}
    Elle le fait en particulier à partir des inclinations naturelles que l’homme découvre comme partie intégrante de sa nature, inclinations qui sont considérées comme menant à leur accomplissement les personnes et la société tout entière. La loi naturelle est donc une notion théologique (elle est fondée sur une théologie de la création) et une notion métaphysique, puisqu’elle est enracinée dans une capacité rationnelle commune à tous les hommes et une vision de l’être (l’être des personnes humaines, l’être de l’ordre social et politique). \cite{thomasset_recherche_2020}
\end{singlequote} 

Elle a été remise en avant par Jean-Paul II après avoir été délaissée par Vatican II. L’objectif de ce retour était de lutter contre les tendances au relativisme et au subjectivisme pour fonder une éthique sur des bases objectives solides. Il demande à la Commission Théologique Internationale de travailler ce sujet. 
 Le document qu'elle produit interroge quelques-unes des traditions de sagesse et des religions du monde afin d’y mettre en lumière l’existence d’un patrimoine moral commun, que certaines sagesses font même découler des exigences inscrites dans la nature en général et la nature humaine en particulier (n° 12). Ainsi sont envisagés l’hindouisme (n° 13), le bouddhisme (n° 14), les sagesses de la Chine (n° 15), les traditions africaines (n° 16) et enfin l’islam (n° 17).\cite{bonino_questions_2011}
Par ailleurs, cette éthique universelle est seule « susceptible de fonder un ordre juste et pacifique dans les relations entre les personnes et les communautés » \cite[Commission Théologique Internationale]{bonino_questions_2011}. 
\begin{singlequote}
     la dimension désormais planétaire de la responsabilité écologique (le réchauffement climatique), le phénomène économique de la mondialisation, dont la récente crise financière a rappelé la face obscure, ou encore l’explosion des biotechnologies, qui touchent aux sources mêmes de l’humain \cite{bonino_questions_2011}
\end{singlequote}

L'A. propose d'explorer la démarche de LS dans  une telle recherche éthique universelle à l’heure de la crise écologique. 
Tout d'abord, l'A. note la différence de traitement par l'application dans l'articulation entre la loi naturelle et le discernement, entre d'une part la morale sexuelle (peu d'espace au discernement individuel) et d'autre part la morale sociale (insistance sur l'ordre de la raison). LS se situe dans cette lignée en insistant sur les ressources de la raison humaine. Pour une recherche d’une éthique universelle en vue de la transition écologique, LS est un exemple et un laboratoire de la mise en œuvre du processus que nous tentons de dévoiler et qui met en interaction des convictions, des attitudes, des manières de procéder et des expérimentations.
Il propose le plan suivant : 
\begin{singlequote}
    Le dialogue nécessaire à cette recherche d’une éthique universelle suppose en effet un diagnostic (les premiers chapitres) et une visée façonnée par des convictions (c’est ce qui apparaît dans le chapitre 4) ; elle exige une aptitude et une ouverture au dialogue avec d’autres (chapitre 5) ; elle nécessite des attitudes et des ressources spirituelles pour être mises en œuvre et expérimentées (chap. 6). \cite{thomasset_recherche_2019}
\end{singlequote}
LS commence cette recherche d'une éthique universelle par un diagnostic (les premiers chapitres) : Ce qui met en mouvement l'éthique, c'est la perception de ce qui est intolérable -dans LS, la fragilité de la maison commune et face à cela la faiblesse des réactions. François insiste sur le caractère inédit de cette crise (LS 17) mais aussi sur sa dimension systémique et planétaire qui exige un effort commun. Mais dès cette phase de diagnostic, il reconnaît l'existence de divergence [LS 60].
\begin{singlequote}
    La manière de voir la réalité est en jeu dans ce processus et suppose dès l’origine une conversion des manières de penser et d’agir [\ldots]. Ce point indique bien la dimension circulaire de la démarche d’une éthique universelle. Ce sont les expériences \cite{spohn_jesus_2010} , les actions concrètes ou les réflexions qui ouvrent le regard amenant les personnes à changer leur mode de penser et de vivre. D’une façon ou d’une autre, il faut oser plonger dans le cercle. \cite{thomasset_recherche_2019}
\end{singlequote} 
Pour le pape, les racines du mal sont humaines et l’enjeu éthique : un anthropocentrisme déviant, qui met au-dessus de tout la raison technique et la domination sur la nature, allié à un relativisme pratique, centré sur le bien-être personnel immédiat, ont fait perdre de vue le bien de l’humanité, notamment des plus fragiles, et la relation vitale avec la terre et les autres vivants. 

Le chapitre 4 de LS synthétise les convictions de l'encyclique et propose une visée  pour la recherche et l’action, la notion d’« écologie intégrale ». Elle manifeste le souci d’une prise en compte de toutes les relations qui conditionnent notre manière d’habiter le monde : relation à la nature et aux autres créatures, relation aux autres humains, relation à nous- mêmes et à Dieu (LS 10, 237) : « tout est lié ». Cette vision conditionne la manière de rechercher une éthique universelle. 

La visée éthique ne sera pas seulement de l’ordre d’une recherche de normes communes (recherche procédurale à la Rawls) mais suppose des convictions sur ce qu’il serait bon de vivre \cite[p.335]{ricoeur_soi-meme_1990} : pour LS, la solution implique une conscience de mutuelle appartenance qui implique de « nouvelles convictions, attitudes et formes de vie » (LS 202).

Pour ces convictions soient partagées au niveau universel, il faut développer un \textit{style de dialogue} à tous les niveaux, international (LS 164-175), national et local (LS 176-175) tout en visant la visée bonne: c'est l'objet du chapitre 5.  
\begin{singlequote}
    Deux raisons justifient cette corrélation entre les divers niveaux de décision : le souci d’une égalité entre les partenaires (compte tenu du fait de l’inégalité actuelle entre nations ou au sein des nations) et le souci de transparence et de contrôle dans les processus de décision (compte tenu du fait que les conséquences néfastes des modes actuels de production et de consommation affectent tout le monde, en particulier les plus fragiles). \cite{thomasset_recherche_2019}
\end{singlequote}
Le dialogue nécessaire doit aussi concerner l’économie et la politique. Il doit encore concerner les religions avec les sciences (LS 199-200), ainsi que les religions entre elles (LS 201). 
\begin{singlequote}
    « En réalité, il est naïf de penser que les principes éthiques puissent se présenter de manière purement abstraite, détachés de tout contexte, et le fait qu’ils apparaissent dans un langage religieux ne les prive pas de toute valeur dans le débat public » (LS 199).
\end{singlequote}
 Les principes du pape François (« le temps est supérieur à l’espace » (LS 178), « l’unité est supérieure au conflit » (LS 198), « la réalité est supérieure à l’idée » (LS 201)) sont convoquées dans ce chapitre comme méthode. Ce qui demande patience, ascèse et générosité.

Enfin, l'Encyclique étudie dans son chapitre 6 les attitudes et les ressources spirituelles qui peuvent être mises en oeuvre et expérimentées. Les religions fournissent des ressources spirituelles nécessaires à cette visée universelle. 

\begin{comment}
    religion universelle ou non ? 
\end{comment}

\begin{singlequote}
Il peut sembler paradoxal de faire appel à la particularité des religions et sagesses (qui peuvent être en rivalité) pour bâtir une vision et un engagement communs pour sauvegarder la maison commune. \cite{thomasset_recherche_2019}
\end{singlequote}
Certains philosophes sont en effet sceptiques sur la valeur des religions dans le débat éthique. L'A. cite néanmoins Rawls et Habermas
\begin{singlequote}
    « Il serait ainsi déraisonnable d’écarter a priori d’un revers de main l’idée selon laquelle les religions universelles (…) conservent cependant leur place dans l’espace différencié de la modernité, parce que leur contenu cognitif n’est toujours pas tari. On ne peut en tout cas pas exclure qu’elles soient porteuses de potentiels sémantiques qui, si on en libère les contenus profanes de vérité, pourraient dégager une force d’inspiration valant pour la société dans son entier. » (	\cite[ p. 204.]{habermas_entre_2008})
\end{singlequote} 

LS 199 rappelle la valeur des traditions religieuses et leur capacité à apporter du sens dans des situations et des temps différents. Puis il développe des éléments pour « \textsc{une mystique de l’action} » (LS 216), selon le voir-juger-agir propre à la doctrine sociale de l’Église  : 
\begin{itemize}
    \item \textit{voir et discerner } : prise  de conscience écologique (LS 7),
    \item \textit{cultiver les vertus} : C’est seulement en cultivant de solides vertus écologiques et sociales que le don de soi dans un engagement écologique est possible (LS 211), permettant une prise de conscience collective. Nous sommes dans le domaine de l'éthique et non des normes.
    \item \textit{Nouvelles habitudes} :  éducation pour une « citoyenneté écologique » qui passe par l’apprentissage d’attitudes, des gestes simples de la vie quotidienne (LS 230), par l’ouverture des cœurs à la beauté (LS 226),par une autre manière de consommer. L'A. souligne l'insistance du pape sur le point face aux addictions (consumérisme, individualisme, foi en la technique) .
\end{itemize}
C'est un processus dynamique, les différents éléments sont en lien les uns les autres.
 Pour le pape, « l’attitude fondamentale de se transcender, en rompant avec l’isolement de la conscience et l’autoréférentialité, est la racine qui permet toute attention aux autres et à l’environnement, et qui fait naître la réaction morale de prendre en compte l’impact que chaque action et chaque décision personnelle provoquent hors de soi-même » (LS 208). 
\begin{comment}
    est ce le christianisme ou de toute religion hétéronome par essence ? 
\end{comment}
 

Une éthique mondiale nécessite d'alimenter les règles internationales de l'intérieur par certaines manières d'être : les religions peuvent apporter humilité et sobriété, la gratitude, le sens d'être connecté aux autres créatures (LS 220).
Les religions peuvent aussi permettre de faire naitre des acteurs de la transition écologiques avec de solides conviction et engagés pour le bien commun et nourris par des pratiques et rites.
Pour les chrétiens, Le pape insiste sur l'eucharistie  (LS 236) et l'exemple des Saints comme témoins de ces vertus (saint François ou Thérèse de Lisieux) 

\begin{singlequote}
    La manière de faire dans la recherche éthique – qui inclut le dialogue, la transparence, l’inclusion des sagesses et religions du monde, la participation de tous à divers niveaux –, suppose l’acquisition d’attitudes de respect, d’ouverture, de bienveillance, de dépassement de soi, de renoncement à imposer sa propre opinion. À leur tour, ces attitudes ne seront pas acquises sans une éducation qui les promeut et sans des ressources spirituelles qui les renforcent. Par ailleurs, il s’agit aussi de penser un cercle vertueux entre les petits gestes quotidiens, les expérimentations collectives de toute sorte, les réflexions éthiques, et les engagements politiques.\cite{thomasset_recherche_2019}
\end{singlequote}

  \begin{singlequote}
      Ceci nous amène à la troisième leçon de ce parcours : les religions et les sagesses du monde ont un grand rôle à jouer dans ce processus, comme l’ont montré les discussions de la COP 21 et la mobilisation des Églises et des religions pour le climat. Leur vision du monde, qui est à la fois universelle, sociale et cosmique, est à la hauteur des enjeux contemporains. Elles sont également en mesure de fournir les motivations d’actions nécessaires pour bousculer les habitudes présentes. La conversion « des mentalités et des structures », dont parle Vatican II dans Gaudium et Spes17, suppose une « mystique qui nous anime » (LS 216). Elles sont enfin capables d’alimenter l’espérance indispensable pour entamer un tel chemin.\cite{thomasset_recherche_2019}
  \end{singlequote}


%-------------------------------------------------------------------------------------------------------
\subsection{Ressources bibliques pour une réponse chrétienne}
%-------------------------------------------------------------------------------------------------------

\paragraph{Ressources bibliques pour une réponse chrétienne}
\textit{Quelques mots avant l’Apocalypse - lire l’Evangile en temps de crise} - A. Candiard \cite{candiard_quelques_2022}
\textit{Between Exile and the New Jerusalem : Prophetic Mourning, Lament and the Ecological Crisis} - Daniel Castillo \cite{cavanaugh_between_2018}; \textit{Création à l’âge de l’anthropocène} - Christoph Theobald \cite{theobald_repenser_2019},  \textit{Parler de la création après Laudato si’}, Lasida, Elena \cite{lasida_parler_2020}

%-------------------------------------------------------------------------------------------------------
\paragraph{Quelques mots avant l'Apocalypse -  lire l’Evangile en temps de crise - A. Candiard}\cite{candiard_quelques_2022} L'Auteur, dominicain, islamologue à l'IDEO - Caire, réfléchit aux ressources de l'Evangile face à la multiplicité des crises et en particulier la crise écologie. Il commence par une critique du mythe du progrès, ancré dans notre vision du monde : 



\begin{singlequote}
        Le progrès scientifique et technique exceptionnel que nous avons connu ces derniers siècles, en particulier les réussites incontestables de la médecine, confirmait cette vision des choses, comme la remarquable expansion économique qui l’a accompagné, dont nous commençons tout juste à comprendre qu’elle comporte aussi des effets délétères. nous savions naturellement que tout n’allait pas bien , mais nous pouvions croire cependant que les choses s’amélioraient.

        Ces schémas de pensée sont si naturellement ancrés, ils informent si puissamment notre appréhension du réel que nous avons du mal à y renoncer tout à fait devant les démentis flagrants que nous offre l’actualité. Nous restons, volontairement ou non, consciemment ou non, orphelins des mythes du progrès, et nous serions bien contents de leur trouver un substitut chrétien, une garantie divine que, malgré quelques péripéties, tout ira pour le mieux.

        Ne serait-ce pas un juste retour des choses puisque de l’avis général, ces philosophies de l’histoire auraient simplement transposé sur terre une espérance chrétienne, laicisé la foi au salut et au paradis ? \cite[pp 89-91]{candiard_quelques_2022}
\end{singlequote}



Les ressources que propose la foi chrétienne ne sont pas uniquement de l'ordre de la prière même s'il s'agir pour le chrétien de reconnaître que ce ne sont pas ses efforts qui sauvent : 

 
\begin{singlequote}
    Il serait naïf, évidemment, de prétendre lutter contre les désastres climatiques en s'en remettant à la seule prière, mais il ne serait pas moins naïf d'imaginer vaincre le mal sans s'attaquer à ses causes, et d'oublier que le premier lieu où je peux envisager de le déraciner, c'est dans ma propre vie.
En moi, le combat eschatologique est déjà engagé, avec sa violence et ses incertitudes :
c'est lui qui est à l'œuvre dans mes crampes d’égoïsme et dans mes envies de bien faire, dans mes fidélités et mes impatiences.Et dans ce combat, l'emporter, c'est d'abord accepter que la victoire a déjà été acquise, non pas par mes efforts, mais par l'amour infini qui se donne à voir dans la croix de Jésus et qu'il me faut, peu à peu, laisser entrer dans ma propre vie. \cite[p.91]{candiard_quelques_2022} 
\end{singlequote}

L'A. se propose de réhabiliter la vision apocalytique de la foi chrétienne : 
\begin{singlequote}
Dans notre cas à nous, les conséquences apocalyptiques du péché ne sont pas justes, car elles ne frappent pas spécialement les pécheurs et, moins encore, à proportion du péché. [\ldots] Impossible de dire à un paysan philippin qui a dû quitter sa terre à cause d'inondations dramatiques que c'est après tout de sa faute, et qu'il aurait dû moins polluer, car il fait face, en réalité, à une véritable injustice immanente: il assume les conséquences terribles d'actions dont il n'est nullement responsable.

[\ldots] Si Jésus tient un discours d'apocalypse, de révélation, ce n'est pas pour nous terrifier plus ou moins utilement, mais bien pour nous faire comprendre ce qui se joue sous nos yeux: non la punition divine des fautes de l'homme, mais le déploiement du mal et de ses conséquences destructrices; autrement dit, la fin des temps à l'œuvre, non comme événement inquiétant dont on attendrait la proximité, mais comme cette réalité présente dans l'histoire depuis son début, véritable trame sous-jacente aux événements du monde. Nous avons besoin de ce dévoilement car, tant que la nature du mal restera inconnue, on pourra croire béatement à l'efficacité de solutions purement techniques aux mena ces qui pèsent sur nos existences. Il est sans doute nécessaire, dans bien des domaines, d'améliorer la législation, de modifier nos modes d'organisation, de négocier la réduction des arsenaux nucléaires ou celle des émissions de gaz polluants, de faire évoluer les opinions publiques; l'engagement politique ou l'action associative peuvent être  inexorable, loin de toute volonté consciente initiale: nous ne maîtrisons pas nos propres catastrophes. \cite[p.xx]{candiard_quelques_2022} 
\end{singlequote}


%-------------------------------------------------------------------------------------------------------
\paragraph{Between Exile and the New Jerusalem : Prophetic Mourning, Lament and
the Ecological Crisis - Daniel Castillo} Daniel Castillo est un théologien catholique américain travaillant sur l'interaction entre la théologie de la libération et la théologie "écologique". 

Dans ce chapitre du livre collectif \textit{Fragile World : ecology and the Church} édité par W. Cavanaugh, l'A. explore l'importance du deuil et de la lamentation des Prophètes (\textit{prophetic mourning}) comme réponse de l'Eglise à la réalité de la crise écologique.  Il les lit aux prières de plainte (\textit{lament and protest to God}).  Pour cela,il étudie le livre de l'Apocalypse pour souligner le rôle de ces prières dans la vie chrétienne, non seulement sur les effets de la crise écologiques mais aussi sur la manière dont le péché est à l'origine de ce contexte \cite[p. 152]{cavanaugh_between_2018}
Il commence par une critique du capitalisme global qui identifie l'homme comme consommateur (\textit{person-as-consumer} selon l'expression de Leslie Sklair). Il s'inscrit dans la filiation de Jean-Baptiste Metz et de sa critique du progrès qui a longtemps irrigué la métaphysique occidentale.

Puis, il étudie le rôle du Prophète, en particulier lors de l'Exil. Dans une société marquée par la figure  royale, établie par ordre divin, le prophète doit lutter contre l'engourdissement face à la mort (\textit{numbness about death}) qui tétanise le peuple face à la crise que connaît Israël. La question, de l'A. est la suivante : pourquoi le chagrin ? Quel est le but d'adopter ou même de cultiver une pratique du deuil ? Brueggemann suggère que l'engourdissement, le déni et le désespoir nourris par la figure royale ne peuvent être brisés que "\textit{par l'acceptation de la négativité et par l'expression publique de notre peur et de notre honte pour l'avenir que nous avons choisi}". La tâche du deuil, par conséquent, est le travail qui permet à la personne ou à la communauté de voir la réalité dans toute sa sévérité et de ne pas être submergée par les ténèbres. Le deuil est la pratique qui permet de briser le cycle du déni et du désespoir, permettant ainsi aux possibilités de conversion, de consolation et d'espoir d'émerger. A partir des travaux de la psychanalyste Joanna Macy, il constate que le déni et le désespoir  empêche la communauté d'agir face à une menace extérieure. Nous avons tendance à sombrer davantage dans le déni ou à être encore plus accablés par un sentiment d'inutilité. En d'autres termes, les informations seules sont souvent inefficaces ou contre-productives. Cependant, dans sa pratique, Macy a constaté que grâce au deuil honnête et douloureux - ce qu'elle décrit comme un "travail de désespoir" - la personne ou la communauté peut non seulement trouver le pouvoir de confronter les réalités de la dévastation écologique et sociale, mais aussi imaginer un nouvel avenir.
La tâche prophétique du deuil consiste alors \cite[p. 159, cité par l'A. ]{brueggemann_prophetic_1978} à : \begin{enumerate}
    \item  Offrir des symboles à la hauteur de l'horreur et de l'expérience qui nous submerge et  qui suscite l'engourdissement et  déni.
    \item Exprimer publiquement ces peurs et terreurs même qui ont été niées si longtemps et si profondément refoulées que nous ne savons pas qu'elles sont là.
    \item Parler métaphoriquement mais concrètement de la réalité mortifère qui plane sur nous et qui nous ronge de l'intérieur, et en parler non pas avec une grâce bon marché mais avec une franchise née de l'angoisse et de la passion.
\end{enumerate}

Mais il ne s'agit pas uniquement des effets bienfaisants de la catharsis émotionnelle, car face à une menace qui reste présente, son effet serait de courte durée. L'A. propose comme réponse de cultiver l'espoir en quelque chose au-delà de soi-même et de son propre pouvoir.
L'A. cite le roman \textit{The Good Apprentice}, d'Iris Murdoch : un jeune homme nommé Stuart tente de réconforter son demi-frère Edward,  accablé de désespoir face à sa complicité dans la mort d'un ami. Trouvant Edward désespéré, Stuart résiste à la tentation du déni et implore :
\begin{singlequote}
    "Essaie de prier d'une certaine manière, dis 'délivre-moi du mal', dis que tu es désolé, demande de l'aide, trouve une lumière, quelque chose que les ténèbres ne peuvent obscurcir. Il doit y avoir des choses que tu as, des choses auxquelles tu peux accéder, de la poésie, quelque chose de la Bible, du Christ, s'il signifie encore quelque chose pour toi. Laisse la douleur continuer, mais laisse quelque chose d'autre la toucher comme un rayon venant de l'extérieur, de cet endroit à l'extérieur" (Murdoch, The Good Apprentice, 47).
\end{singlequote}
 

Tout espoir chrétien est fondé ultimement sur la fidélité de Dieu. L'A. s'appuie sur le livre de l'Apocalypse : le Royaume de Dieu est "ce lieu extérieur que les ténèbres ne peuvent obscurcir" où toutes les larmes seront essuyées (Ap 21:4), où la justice et la paix s'embrasseront (Ps 85:10). Du point de vue théologal, le dépassement du désespoir est réalisé de manière appropriée en se tournant - voire en criant - vers Dieu. L'acte de lamentation est une prière qui rassemble la colère, la souffrance, le deuil et l'espoir. 
Les lamentations bibliques sont variées : souffrance innocente (Job),  persécution subie par les saints pour leur fidélité au Christ (Apocalypse),  et l'exil, (expérience collective de culpabilité). Les Églises du Sud - ayant contribué relativement peu au changement climatique - se trouvent par exemple dans une position proche de celle de Job. L'exil reflète l'expérience des églises occidentales. 

Castillo souligne que dans l'Apocalypse, il n'y a pas d'extase (\textit{rapture}) : 
\begin{singlequote}
     "Au contraire, Dieu est 'emporté' vers la terre pour y prendre résidence et 'demeurer (skene, skenoo) avec nous \cite{rossing_rapture_2005}" Il est important de noter que cette "extase" descendante ne se produit qu'après la chute de la "grande ville de Babylone", symbole dans le texte de l'Empire romain. L'espoir apocalyptique raconté dans le livre de l'Apocalypse ne désire donc pas tant la fin du monde que la fin du système mondial du pouvoir impérial romain. Comme le dit l'Apocalypse à ses lecteurs, ce sont les forces qui détruisent la terre qui seront détruites (Ap 11,18). \cite[p. 161] {cavanaugh_between_2018}
\end{singlequote}

Les prières de lamentations sont donc adaptées à la crise écologique : la lamentation \textit{engendre la fureur} face aux injustices et aux abus du monde et pousse à \textit{résister}. C'est l'esprit de la religion pour Jean-Baptiste Metz, pour qui la religion est simplement l'\textit{arrêt}. La lamentation s'apparente au cri de "stop !". Mais Castillo ne s'arrête pas là et souligne l'espérance du règne de Dieu, de \textit{la nouvelle Jérusalem} qui anime le chrétien. 



 


%-------------------------------------------------------------------------------------------------------
\subsection{LS et la doctrine Sociale de l'Eglise - Bibliographie}
%-------------------------------------------------------------------------------------------------------


%-------------------------------------------------------------------------------------------------------
\paragraph{La doctrine sociale de l'Eglise selon François - Ch. Theobald} \cite{theobald_lenseignement_2016} Les deux textes de François qui parlent de la doctrine sociale, \textit{Evangelii gaudium} (EG) et \textit{Laudato Si’} [LS]  se complètent mutuellement et présentent un style différent et signifiant des textes traditionnels du magistère sur la doctrine sociale de l'Eglise, qui nous implique dans un parcours de conversion \cite[par. 4]{theobald_lenseignement_2016}. Cela exige tout d'abord une \textit{unification} théologique de l'enseignement sociale de l'Eglise,  opérée grâce à son inscription dans la globalité de nos Écritures qui lie foi en Christ et Règne de Dieu (Lc 4,43 repris par EG 180) . Cet enseignement doit être reçu de façon concrète, dans un monde pluriel, nouveauté par rapport à Gaudium et Spes, texte référent de l'enseignement social de l'Eglise.  C'est pour l'A. ici qu'intervient le vocabulaire stylistique.  Dans EG, 

\begin{singlequote}
 L’analyse s’affine ici, en particulier dans le chapitre 3 de \textit{Laudato Si’} qui examine « la racine humaine de la crise écologique ». S’inspirant largement du philosophe et théologien allemand Romano Guardini et sans doute (sans les nommer) d’un Ivan Illich et de l’école de Francfort, le pape démonte « la manière dont l’humanité a, de fait, assumé la technologie et son développement  \textit{avec un paradigme homogène et unidimensionnel} » [LS 106]. Le réductionnisme qu’il dénonce s’enracine dans le rapport que nous entretenons avec nos objets : « Il faut reconnaître que les objets produits par la technique ne sont pas neutres, parce qu’ils créent un cadre qui finit par conditionner les \textit{styles de vie}, et orientent les possibilités sociales dans la ligne des intérêts de groupes de pouvoir déterminés » (\textit{LS} 107). Il en résulte « l’imposition (par toute une culture) d’un \textit{style de vie} hégémonique lié à un mode de production » (\textit{LS} 145), la domination technocratique globale, y compris sur l’économie et la politique, par quelques-uns (\textit{LS} 109), faisant que « c’est devenu une contre-culture de choisir un \textit{style de vie }avec des objectifs qui peuvent être, au moins en partie, indépendants de la technique, de ses coûts, comme de son pouvoir de globalisation et de massification » (\textit{LS} 108). \cite[par. 12]{theobald_lenseignement_2016} \label{theo:diagnosticLS}
\end{singlequote}
L'encyclique présente donc un cadre conflictuel et le pape présente ensuite le style de vie alternatif \textit{spécifiquement chrétien}, illustré par le Cantique de François d'Assise au début de \textit{LS}, qui  introduit à l'expérience d'écoute [LS 1-2], des pauvres et de la terre.
La réflexion part de l'écriture et le Règne de Dieu est sout-tendu par une lecture de la Génèse, avec tout le deuxième chapitre de LS consacré à l' \textit{"Evangile de la Création"} et une réflexion sur les récits de Cain et Abel et de Noé :
\begin{singlequote}
    Quand toutes ces relations sont négligées, quand la justice n'habite plus la terre, la Bible nous dit que toute la vie est en danger [LS 70].
\end{singlequote}
Quelle guérison pour cette rupture de relation, ce "péché" ? Il s'agit de quitter les styles de vie unidimensionnelles qui sont aujourd'hui hégémoniques pour s'inscrire dans le \textit{style de vie de l'Evangile}. L'A. identifie 4 principes :
\begin{enumerate}
    \item le temps est supérieur à l'espace
    \item l'unité prévaut sur le conflit
    \item la réalité est plus importante que l'idée
    \item le tout est supérieur à la partie \label{theo:principesLS}

\end{enumerate}
Ce dernier élément est développé par l'A. car il présente une réelle nouveauté par rapport à  Vatican II (GS) , où
 \begin{singlequote}
     Le \textit{singulier} (\textit{tel} individu, \textit{telle} culture ou langue, \textit{tel} peuple), n’y a pas de place ou, disons plutôt, n’y est pas [\ldots] objet d’intérêt. Nous sommes plutôt dans un univers homogène et unidimensionnel, selon le vocabulaire de Laudato Si’ [\ldots]. 
     
     Par contre, la vision du monde d’Evangelii gaudium se comprend selon le modèle du polyèdre (EG 234-237). Le discours doctrinal qui insiste sur les principes n’y perd pas sa nécessaire fonction régulatrice, mais il ne parviendra jamais à rejoindre \textit{« chaque chrétien}, en quelque lieu ou situation où il se trouve » (EG 3), voire « \textit{chaque personne} qui habite cette planète » [LS 3] selon leur singularité en relation, intégrée dans des ensembles sociaux et environnementaux toujours plus larges, mais maintenant leur « \textit{originalité} », selon l’expression du texte. Seule une\textit{ approche stylistique} le permet, car elle est sensible à la confluence de tous les éléments partiels dans une donnée singulière où ces éléments conservent leur originalité tout en étant habités par le tout qu’est la « plénitude de la richesse de l’Évangile ».\cite[par. 29-30]{theobald_lenseignement_2016}
\end{singlequote}
Mais il s'agit aussi de quitter un \textit{anthropocentrisme despotique} [LS 68, 69, 118, 119 et 122], et de quitter le mythe du progrès, perspective absente de GS. LS ne propose pas un scénario mais recommande d'identifier les \textit{possibles scenarios futurs} [LS 60] en refusant les positions extrêmes d'un anthropocentrisme dévié et d'un biocentrisme qui refuse l'intervention de l'homme. Il s'agit d'assurer une recevabilité universelle de l'enseignement social de l'Eglise : 
\begin{singlequote}
    les deux textes du pape François ne se contentent nullement d’une argumentation biblique mais développent un vrai enseignement social qui est particulièrement attentif à sa recevabilité universelle. [\ldots] La différence [avec GS] porte sur la manière de donner droit de cité à l’altérité et à ce qui est divers et pluriel – signifié par la métaphore du polyèdre – et donc au dialogue social qui, s’il est mené en vérité, ne peut qu’introduire la foi chrétienne comme « ressource » vitale ou comme style de vie, fondé sur le principe de « gratuité ». \cite[par. 37]{theobald_lenseignement_2016}
\end{singlequote}
Particulièrement par rapport au thème qui nous intéresse, 
\begin{singlequote}
    Le pape reconnaît parfaitement que « \textit{certains relèguent la richesse que les religions peuvent offrir dans le domaine de l’irrationnel} » [LS 62] ; mais il montre également que la complexité de la crise exige une pluralité d’interprétations et d’apports : « \textit{Il est nécessaire, écrit-il, d’avoir aussi recours aux diverses richesses culturelles des peuples, à l’art et à la poésie, à la vie intérieure et à la spiritualité} » [LS 63]. Et il ajoute que, pour ce qui est des chrétiens et d’autres croyants, « {les convictions de la foi [leur] offrent de \textit{grandes motivations pour la protection de la nature et des frères et sœurs les plus fragiles }}» [LS 64]. \cite[par. 33]{theobald_lenseignement_2016}
\end{singlequote}
Le terme même de "ressource" indique un statut décentré des religions, dans une \textit{vision multidimensionnelle de l'homme au sein de la création}. C'est le rôle de la \textit{"sagesse"} de faire
\begin{singlequote}
    [\ldots] le lien entre ce qui anime chacun et les « ressources » apportées par les religions et la tradition chrétienne ; car la « sagesse » s’inscrit, d’un côté, dans la « culture » prise dans toute sa complexité et comprend, de l’autre côté, l’\textit{ensemble} des traditions religieuses, tout en spécifiant le récit biblique.\cite[par. 34]{theobald_lenseignement_2016}
\end{singlequote}
Cette approche permet de reconnaître théologiquement la valeur du mouvement écologique comme \textit{sagesse}.
\begin{singlequote}
     Le « spirituel » n’est donc nullement réservé aux chrétiens mais s’avère déjà être le fruit du travail de la sagesse au sein de l’humanité. Elle s’exprime aussi à travers les textes cités par l’Encyclique, non seulement ceux de différentes conférences épiscopales nationales et continentales, mais aussi et surtout la Déclaration de Rio sur l’environnement et le développement, reconnue comme « prophétique » par Laudato Si’ [LS 167 et 186], et la Charte de la terre de La Haye [LS 207).\cite[par. 35]{theobald_lenseignement_2016}
\end{singlequote}

La posture de l'Eglise ainsi proposée est donc humble et exigeante, non en aplomb, avec le monopole de l'interprétation du fait sociale [EG 184] mais en écoutant pour \textit{promouvoir le débat} [LS 46 et 188).  Il s'agit d'une vraie mutation du concept même d’enseignement social de l’Église par sa capacité déjà avérée d’initier des « processus» de conversion sociale, au lieu de proposer une synthèse achevée (cf.LS 121), privilégiant le travail à long terme que la possession d'espace. Pratiquement, l'A. repère trois traits de ce style :
\begin{enumerate}
    \item chemin de conversion vers la \textit{sortie de soi}, dépassant nos individualismes [LS 208]. Pour réussir cette conversion, l'Eglise ne doit pas s'adresser simplement à la raison mais adopter un langage direct pour rejoindre le coeur de nos interlocuteurs. 
    \begin{singlequote}
        [\ldots] en étant conscient qu’ils sont divers, situés dans une diversité de cultures et de situations concrètes ; il lui faut donc adopter une forme ou un style « polyédrique » pour s’adresser à eux. \cite[par. 44]{theobald_lenseignement_2016}
    \end{singlequote}
    \item un style \textit{prophétique et contemplatif}, jamais l'un sans l'autre dans l'Evangile. 
    \begin{singlequote}
        L’Évangile est d’abord une « ressource » de bonté radicale déjà à l’œuvre dans les sagesses humaines. C’est pour cela que le prophétisme mis en œuvre par Laudato Si’ ne revêt aucun accent catastrophiste ou « apocalyptique ». Certes l’insoutenable n’est jamais nié ou édulcoré ; mais la dénonciation est d’emblée mise au service d’une espérance inaliénable qui repose sur la création comme « don » [LS 76] : « capables de se dégrader à l’extrême, les êtres humains peuvent aussi se surmonter, opter de nouveau pour le bien et se régénérer, au-delà de tous les conditionnements mentaux et sociaux qu’on leur impose » [LS 205 et 61].\cite[par. 45]{theobald_lenseignement_2016}
    \end{singlequote}
    \item un regard contemplatif "capable d'apprécier profondément les choses sans être obsédé par la consommation" [LS 222]
\end{enumerate}




%-------------------------------------------------------------------------------------------------------
\paragraph{Conversion dans la doctrine sociale de l'Eglise}
 

%-------------------------------------------------------------------------------------------------------
\subsection{Eco-théologie de la libération - Bibliographie}
%-------------------------------------------------------------------------------------------------------
%-------------------------------------------------------------------------------------------------------   
\paragraph{De quel genre de pensée a-t-on besoin pour aborder la crise environnementale contemporaine ? }
        \cite{howles_quel_2022}


\label{Comment:MemoireISTR4}        
   	
    
        Doctrine sociale de l’Eglise : positif sur le role des acteurs

        ecomodernisme : progrès
\begin{singlequote}

        Cependant, pour la nouvelle écologie politique, cette sorte d’écomodernisme est une simple réaffirmation du dualisme de l’être humain face à un monde de la nature passif, non animé et inerte, dans l’attente que grâce à son ingéniosité il soit en mesure de dominer et de maîtriser ce monde. Malgré de bonnes intentions individuelles dans des situations particulières, cela mène invariablement à la perpétuation du paradigme technocratique moderniste et ne réussit pas à prendre en compte ce que le pape François appelle « les racines humaines de la crise écologique » [11]. C’est pourquoi la nouvelle écologie politique rejette entièrement ce paradigme.
\end{singlequote}

        {une critique du judeochristianisme, responsable du modernisme.}
\begin{singlequote}

        Fredric Jameson fait remarquer avec humour que « de nos jours il semble plus aisé d’imaginer la fin du monde que celle du capitalisme »[14]. Bruno Latour considère que cette attitude s’appuie sur des idées religieuses de providence et d’achèvement eschatologique, où le but et point final de l’histoire est décrété d’avance et où les croyants sont invités à structurer en conséquence leurs choix personnels et leurs décisions. Cela a, selon lui, un effet démobilisateur sur les énergies politiques qu’il considère nécessaires pour une action environnementale radicale, révolutionnaire et efficace aujourd’hui.
\end{singlequote}

\label{Comment:MemoireISTR5}  
%-------------------------------------------------------------------------------------------------------
\paragraph{Théologie de la libération} Léonard Boff
{Cri de la terre et clameur des pauvres, quels chantier théologique et quelle pratique ecclesiale ?} \cite{thomasset_recherche_2020}
 


%-------------------------------------------------------------------------------------------------------
\paragraph{Une eco-théologie de la libération} 
\href{https://www.loyola.edu/academics/theology/faculty/castillo}{Daniel Castillo} expose comment une éco-théologie  de la libération : rapport à Dieu, aux hommes à la terre (et de tout ce qui provient de la terre - homme animaux).   il s'appuie sur le livre de la parole de Dieu. 


Il s'appuie sur Gustavo Gutierrez, une libération intégrale : socioéconomie, économique. Des structures, des valeurs et des imaginations et un combat entre le péché et la grâce. Une Influence sur LS 70.
 
\begin{singlequote}
    Dans ces récits si anciens, emprunts de profond symbolisme, une conviction actuelle était déjà présente : tout est lié, et la protection authentique de notre propre vie comme de nos relations avec la nature est inséparable de la fraternité, de la justice ainsi que de la fidélité aux autres. LS 70
\end{singlequote}
 

\textbf{L'écologie intégrale comme libération intégrale - Castillo} 
 Dieu est lui même le jardinier. Joseph est l'anti-adam et préfiguration du Christ.  L'exode, c'est le chemin d'apprentissage où le peuple est amené à retrouver sa vocation. 

\textbf{Lecture politico-écologique de la parole de Dieu} ne pas accumuler la manne. Glaneur donc faible. confiance en Dieu, limitation de l'accaparation.
Sabbat. Repos de la terre et du travail aussi des servantes \textit{et des animaux}. Lv : année sabbatique et année jubilaire. Pdt l’année sabbatique, on retrouve sa position de glaneur. Jubilaire : Quadruple restauration (terre, esclave,...) Jésus, Lc 4 : une année de bienfaits. Option préférentielle
pour les pauvres. cf LS 237 (Dimanche et sabbat).



\paragraph{Theobald - Création à l'âge de l'anthropocène}
\cite{theobald_repenser_2019}



%-------------------------------------------------------------------------------------------------------
\section{Rapport aux autres religions - Bibliographie}
%------------------------------------------------------------------------------------------------------- 


%-------------------------------------------------------------------------------------------------------
\subsection{Visions traditionnelles du Rapport aux religions - Bibliographie}
%-------------------------------------------------------------------------------------------------------

%-------------------------------------------------------------------------------------------------------
\section{Bibliographie - Théologie des religions}
%-------------------------------------------------------------------------------------------------------
\paragraph{Sous-Problématique : Rôle des religions et de l'Eglise dans la conversion écologique } Revenir à la source théologique



%-------------------------------------------------------------------------------------------------------
\paragraph{Duquoc}    
\begin{singlequote}
    L'asymétrie pousse les Églises à la conscience ferme d'une ignorance parce que les desseins de Dieu sur ce monde intermédiaire ne lui sont révélés que sous des espérances aux contenus indicibles ou aux contours flous et des impératifs éthiques non étrangers à la marche plus ou moins chaotique de chaque fragment. En ce sens, l'Église est sacrement du salut dans la mesure où elle renonce à être le tout, c'est-à-dire à étre unique lieu de l'Esprit et la déléguée de son Seigneur. Une analogie se dessine ici entre le parcours historique de Jésus et le chemin terrestre de l'Église. \cite[p 241]{duquoc_unique_2002}
\end{singlequote}


\begin{singlequote}
En ce sens, l'Église est sacrement du salut dans la mesure où elle renonce à être le tout, c'est-à-dire à étre unique lieu de l'Esprit et la déléguée de son Seigneur. Une analogie se dessine ici entre le parcours historique de Jésus et le chemin terrestre de l'Église.    \cite[p. 241]{duquoc_unique_2002}
\end{singlequote}

 %   On pourra voir comment François dessine le rôle de l'Eglise



%-------------------------------------------------------------------------------------------------------
\paragraph{Dieu au Pluriel -  l'approche culturo-linguistique}    
        \cite{cheno_dieu_2017}

\begin{singlequote}
        
        la principale critique adressée aux théologies pluralistes, c’est leur prétention à disposer d’un lieu tiers, d’un arrière-plan qui se situerait au delà des religions particulières et à partir duquel on pourrait les embrasser toutes : le plan nouménal de la Réalité ultime pour John Hick, une même expérience mystique pour Raimon Pannikar ou encore un même projet éthique pour la justice et la gestion écologique des ressources de notre planète. [critique du Manifeste pour une éthique planétaire de Kung]  \cite[p. 111]{cheno_dieu_2017}
\end{singlequote}

        
\begin{singlequote}
       [impossible car] Nous sommes des humains, insérés dans une culture et des pratiques qui nous façonnent. \cite[p. x]{cheno_dieu_2017}

\end{singlequote}


\begin{singlequote}
    Chéno montre que contrairement à l’approche libérale centrée sur l’expérience du sujet et qui serait le lieu où les croyants se retrouvent, le post-libéralisme appelle à ne pas minimiser les différences, bien au contraire ; dans une approche utilitariste, elles sont mêmes précieuses car ce sont ces différences qui donnent à chaque religion une valeur singulière et vitale.  \cite{pisani_cheno_2018}
\end{singlequote}


  


%-------------------------------------------------------------------------------------------------------
\paragraph{The Nature of Doctrine} \cite{lindbeck_nature_2002}
\begin{singlequote}
    Une religion contribuera probablement davantage au futur de l’humanité si elle préserve ses propres caractéristiques et son intégrité que si elle cède aux tendances homogénéisantes qui vont avec l’expressivisme expérientiel libéral \cite[ p. 115]{lindbeck_nature_2002}
\end{singlequote}

 


%-------------------------------------------------------------------------------------------------------
\subsection{Prophétisme et Idolatrie - Bibliographie}
%-------------------------------------------------------------------------------------------------------
\paragraph{Problématique}
est ce que Idolâtrie nous permet de creuser comment le dialogue inter religieux doit être pensé. 
Une critique de l'idolâtrie, \textit{nouveauté} du pape François

Le penser dans le cadre plus large du rapport au monde et à la sécularisation. 




%-------------------------------------------------------------------------------------------------------

\paragraph{Economie, idolâtrie et sécularisation depuis \textit{Gaudium et Spes} - W. Cavanaugh} 
W. Cavanaugh est théologien américain et explore les relations entre l'Eglise et le monde :
\begin{singlequote}
    L'Eglise est un corps social d'un genre particulier et unique en ce monde (\textit{sui generis}) qui porte la "politique" de Dieu afin de transformer ce monde  en vue du royaume des cieux.\cite[p. 10]{cavanaugh_idolatrie_2022}
\end{singlequote}
Dans son article \textit{Economie, idolâtrie et sécularisation depuis \textit{Gaudium et Spes}} paru en anglais en décembre 2015, de façon concommitante à l'encyclique LS, l'A. explore la continuité et les différences entre Vatican II (GS) et la pensée sociale du pape François. 

 
François ne parle pratiquement jamais de l’économie contemporaine sans adresser une accusation d’idolâtrie, accusation absente dans GS, et presque entièrement absente de Vatican II dans son ensemble.

Comment expliquer cette différence de traitement des questions économiques dans GS et chez le pape François ?

 
\begin{singlequote}
    Ma thèse est que François représente une opportunité pour changer le discours catholique sur la sécularisation, une opportunité qui a des implications dans la manière de considérer non seulement l'économie mais aussi d'autres phénomènes séculiers. 
    La pensée catholique progressiste dans la période du Concile Vatican II avait tendance à considérer le monde séculier comme désenchanté. François suggère au contraire, que nous ne sommes pas tant confronté à une perte de foi qu’à une nouvelle religion et un foi idolâtre. 
    \cite[p. 126]{cavanaugh_idolatrie_2022}
\end{singlequote}
Pour cela, l'A. souligne que la sécularisation de certains phénomènes comme l'économie considérée comme acquises par Vatican II, est aujourd'hui remise en cause. 
\begin{singlequote}
    Je soutiendrai que les attaques de François envers l'idolâtrie de l'argent peuvent être comprises dans l'esprit de Vatican II, non comme un jugement négatif sur le monde, mais comme une reconnaissance des aspirations profondes du monde. \cite[p. 127]{cavanaugh_idolatrie_2022}
\end{singlequote}

Cavanaugh prend tout d'abord l'exemple de ce que dit GS de l'athéisme : considéré comme un manque de foi et non une croyance, et souvent motivé par l'hypocrisie des croyants. GS semble s'adresser à un public européen instruit.
\begin{singlequote}
    Lors du débat sur le schéma XIII (qui deviendra GS) au cours de la troisième session du Concile, les évêques du tiers-monde se plaignirent que le document était trop axé sur le contexte des pay industrialisés, trop centré sur la sécualirsation et sur le communisme - cette plainte a également été formulée lors de la version finale. Des passages, comme celui du n° 61 dans le document final qui recommande d'utiliser son temps libre pour le tourisme et les activités sportives, ne font que renforcer l'impression que le texte était écrit par et pour des bourgeois européens. \cite[p. 130]{cavanaugh_idolatrie_2022}
\end{singlequote}
De même, GS 63 montre un certain optimisme sur la possibilité de la raison de corriger les déséquilibres économiques. La liste des principes économiques nécessaires que liste GS dessine \textit{in fine} ce qui ressemble à une économie saine. Mais, en restant au principe, le texte acte l'autonomie de la science économique, sans exclure néanmoins Dieu car "elle correspond à la volonté du Créateur. C'est en vertu de la création même que toutes choses sont établies selon leur ordonnance et leurs lois et leurs valeurs propres, que l'homme doit peu à peu apprendre à connaître, à utiliser et à organiser  [GS 36]. Le domaine de l'économie reste celui des hommes sobres et rationnels qui tentent d'aménager correctement le monde matériel.

Or, malgré son  importance,  premier des Dix commandements, l'idolâtrie n'est mentionné que trois fois dans les documents du Concile : l'Eglise doit arracher les gens \textit{à l'esclavage de l'erreur (\textit{slavery of error and of idols}} (LG 17),"ceux qui se fiant plus que de raison aux progrès de la science et de la technique, sont enclins à une sorte d'idolâtrie des choses temporelles : ils en deviennent les esclaves plutôt que les maîtres" (\textit{Apostolicam actuositatem} 7), et dans GS 41, nous sommes assurés que grâce à notre foi en l'incarnation, la croix et la résurrection du Christ, \begin{singlequote}
    "l'Eglise peut soustraire la dignité de la nature humaine à toutes les fluctuations des opinions qui , par exemple, rabaissent exagérément le corps humain, ou au contraire l'exaltent (\textit{idolize) }sans mesure".[GS 41]
\end{singlequote}

Des son premier texte (\textit{lumen fidei}), François cite quatorze fois le mot idole. 

A la différence de la vision de l'athéisme de GS, quand on cesse de croire en Dieu, on ne cesse pas de croire mais on croit à toute autre sorte de choses d'où la personification de l'argent en Mammon (Mt6,24).
\begin{singlequote}
    Pour cela, l'idolâtrie est toujours un polythéisme, un mouvement sans but qui va d'un seigneur à l'autre. [\ldots] Celui qui ne veut pas faire confiance à Dieu doit écouter les voix des nombreuses idoles qui lui crient : "Fais-moi confiance !" (\textit{Lumen Fidei, 2013, 13]}
\end{singlequote}

Dans \textit{EG}, la critique de l'économie mondiale contemporaine se fait encore plus vive :  
\begin{singlequote}
    Une telle économie tue. Il n'est pas possible que le fait qu'une personne âgée réduite à vivre dans rue, meure de froid ne soit pas une nouvelle, tandis que la baisse de deu points en bourse en soit une (EG 53)
\end{singlequote}

A la différence de la vision neutre de GS, le marché mondialisé s'est absolutisé, de sorte que le vrai Dieu ne peut plus apparaître que comme une menace "incontrôlable" (EG 57).
l'idolâtrie est associée à l'oubli, par opposition à la fidélité de Dieu : 
\begin{singlequote}
    Libère-nous de l'idolâtrie du présent à laquelle se condamne celui qui oublie. [François, 23 mai 2013]
\end{singlequote}
Face à ces idoles, il nous résister et non nous résigner et user de ruse \cite[p. 138]{cavanaugh_idolatrie_2022}. 
En soi, la richesse n'est pas le problème, c'est le sentiment d'auto-suffisance, le refus de reconnaître sa dépendance par rapport à Dieu qui marque l'idolâtrie.


La différence entre GS et la pensée du pape François ne peut être comprise avec les dichotomies qui ont marqué le Concile et sa réception (progressiste vs conservateur, thomiste vs Augustinien, herméneutique de la rupture ou de la continuité). 
La pensée du pape est à la fois réaliste et joyeuse : 
\begin{singlequote}
    Le chrétien est joyeux, il n'est jamais triste. Dieu nous accompagne. [\ldots] Le péché et la mort ont été vaincus. Le chrétien ne peut pas être pessimiste. ! [François, 24 juillet 2013, JMJ de Rio]
\end{singlequote}

La thèse de l'A., c'est que la différence de François avec GS vient moins de sa perspective sud-américaine que de la profonde évolution du regard sur la sécularisation depuis GS. Lors de Vatican II, la \textit{thèse de la sécularisation} est à son apogée, présentant la sécularisation comme un désenchantement du monde. 
\begin{singlequote}
    Mais plutôt que de considérer la sphère séculière comme un domaine neutre et dépourvu de transcendance, je pense, au contraire, qu'il y a une une migration du sacré de l'Eglise vers le monde, de sorte que le capitalisme, par exemple, est mieux compris non comme dépourvu de dieux, mais comme un nouveau type de religion, souvent idolâtre. \cite[p.143]{cavanaugh_idolatrie_2022}
\end{singlequote}
Cette thèse a entraîné un large consensus : en utilisant les termes de Max Weber, il y a \textit{désenchantement / dé-magication} du monde (\textit{Entzauberung}). Ce consensus a limité la possibilité même d'un discours chrétien sur le monde.
La sécularisation est considéré par certains théologiens progressistes comme l'aboutissement de la tradition judéo-chrétienne elle-même, luttant contre les forces animistes. 

L'A. montre comment le regard sur la sécularisation évolue depuis Chenu, Geffré et Schillebeekx, qui pense la sécularisation dans un contexte marqué par l'existentialisme. Depuis la chute de Berlin, la thèse de la sécularisation a connu des temps difficiles.
Peter Berger qui écrivait en 1968 qu'au XIIè siècle, on ne trouvera les croyants religieux probablement que dans de petites sectes, regroupées pour résister à une culture séculière mondiale" a depuis reconnu :
\begin{singlequote}
    Le monde d'aujourd'hui est aussi furieusement religieux qu'il ne l'a jamais été, et dans certains endroits, plus que jamais. Cela signifie que toute une littérature scientifique produite par les historiens et les spécialistes des sciences sociales, vaguement appelée "théorie de la sécularisation" est fondamentalement erronée". \cite{berger_desecularization_1999}
\end{singlequote}

L'A. reprend la thèse de Talal Asad (1993) que le clivage religieux / séculier est une invention de l'Occident moderne. Il ne s'agit pas de dire que la sécularisation n'existe pas mais que la distinction contemporaine entre religieux et séculier ne correspond pas à la distinction entre sacré et profane : deux processus sont à l'oeuvre dan la société contemporaine : la sécularisation de la religion et la sacralisation du séculier. \cite[p. 155]{cavanaugh_idolatrie_2022} En particulier, dans le domaine de l'économie, l'hypothèse de Max Weber qu'une plus grande rationalisation entraîne un désenchantement du monde est fausse, symbolique en sont les cathédrales de la consommation (Ritzer, 2010) ou l'influence des marques comme substitut à la religion traditionnelle (Shachar, Erdem et all, 2011), avec ces milliers de personnes qui communient à l'achat du nouvel iphone en attendant l'ouverture des magasins à minuit. La nouveauté depuis les années 1960 et Vatican II est donc l'historicisation du clivage religieux / séculier et le brouillage qui en découle. 
\begin{singlequote}
    La religion est notoirement difficile à définir. Cependant, si nous adoptons un point de vue fonctionaliste [à la Durkheim], et que nous comprenons la religion comme ce qui nous construit, en nous les enseignant, ce qu'\textit{est le } monde et quel est notre \textit{rôle} dans le monde, il devient évident que les religions traditionnelles remplissent de moins en moins ce rôle parce que cette fonction est supplantée - ou écrasée- par d'autres systèmes de croyances et de valeur. Aujourd'hui, pour expliquer le monde, l'alternative la plus puissante est la science, et le système de valeurs le plus attrayant est devenu le consumérisme. Leur progéniture académique est l'économie, probablement la plus influente des "sciences sociales".En réponse, cet article soutient que notre système économique actuel devrait également être compris comme notre religion, car il en est venu à remplir une fonction religieuse pour nous. David R. 
 \textsc{Loy}, \textit{ The religion of the Market, \textit{Journal of the American Academy of Religion}, 65, 1997; p. 275}
\end{singlequote}
Le domaine de l'économie n'est pas autonome et vide de sacré. L'Eglise elle-même est affectée par l'"attrait de l'argent", la critique de l'idolâtrie par François étant avant tout une auto-critique \cite[p.160]{cavanaugh_idolatrie_2022}. Par ailleurs, face au dérèglement économique, François prône l'espérance et non l'optimisme.  Enfin, il est possible de voir dans l'idolâtrie de l'argent une aspiration plus profonde de l'homme à rechercher Dieu, certes de façon bien mal ordonnée. En Ac 17, 16, Saint Paul est affligé par la ville d'Athènes pleine d'idoles et il appelle les Athéniens à se repentir de leur idolâtrie, mais il le fait avec une forme de sympathie, car il voit dans leur idolâtrie un tâtonnement inchoatif vis-à-vis du vrai Dieu. L'A. conclut par ce double regard sur l'idolâtrie comme celui de l'esprit de Vatican II.

 

\begin{comment}
\begin{singlequote}
    les fonctionnalistes préfèrent définir la "religion", non pas en termes de \textit{ce que} croient les hommes religieux, mais en termes de la \textit{manière dont} ils croient (c'est-à-dire en fonction du rôle que la croyance joue dans la vie des gens". Clrke / byrne 1993
\end{singlequote}
    à la suite de Durkheim.

    voir règne de Dieu
 

  si sécularisation : nouvel idolatrie, on comprend l'ouverture aux autre religions. 

 
    lire Joseph Komochak : Chenu  inclinaison aristotelicienne et thomiste à l'autonomie vs Ratzinger Augustin séparation de nature et Grace. La valuatazioni sulla Gaudium et spes : Cheni, Dossetti and Ratzinger". in Joseph Doré et Alberto Melloni : volti di fine concilio. 2000

    Alors que le père Chenu voyait une chance dans le processus de sécularisation et dans l'autonomisation d'un nombre grandissant d'actions de l'homme, Claude Geffré était déjà plus critique  tout en reconnaissant le caractère positif de la désacralisation, inscrit au coeur même de la religion judéo-chrétienne. Annoncer la Parole de Dieu à un homme qui a conquis son autonomie et qui a démystifié un certain nombre d'aliénations, permet pour Geffré de lui donner le sens qui est absent désormais du monde sécularisé.

\end{comment}
 



%-------------------------------------------------------------------------------------------------------
\paragraph{Conclusion Intermédiaire : un paradoxe} 
A la différence Dt, qui critiquait fortement les religions extérieures et toutes les compromissions, ici, il semble que nous ayons un paradoxe : positivité des religions non chrétienne et négativité des "compromissions" mais par rapport à une "religion non nommée".

%-------------------------------------------------------------------------------------------------------
\subsection{théologie chrétienne du pluralisme religieux - Bibliographie}
%-------------------------------------------------------------------------------------------------------
\paragraph{Introduction à cette partie}
\begin{singlequote}
    Il s’agit donc pour toutes les religions de puiser dans « leur propre héritage éthique et spirituel », de revenir « à leurs sources » pour « mieux répondre aux nécessités actuelles » [LS 200].
\end{singlequote}

%-------------------------------------------------------------------------------------------------------
\paragraph{Dieu au pluriel, Rémi Cheno} \cite{cheno_dieu_2017}   
 

{dominicain de la province de France} Remi Cheno est né en 1959. Après l'école polytechnique, l'ENSG, il entre au noviciat dominicain en 1994 et continue ainsi ses études, passant de bac +6 à bac +13 après sa thèse de document en théologie dogmatique. Spécialisé en ecclésiologie, il s’est intéressé par la suite plus spécialement à la pneumatologie et à l’eschatologie. 

 


 

{Dieu au pluriel, penser les religions}
{Problématique du livre} Dans ce nouvel environnement que le premier chapitre essaye de qualifier, la question du livre est : \textit{Le dialogue interreligieux a-t-il encore un sens et sous quelles formes ? À quelles conditions ?} 

Rémi Chéno le fait de façon structurée mais la plus accessible possible. 

{Les différentes approches chrétiennes} Le premier chapitre s'applique à définir le monde dans lequel nous sommes, marqué par la condition postmoderne. Puis Rémi Chéno, dans son deuxième chapitre reprend et complète la typologie d'Alan Race des différentes théologies chrétiennes du pluralisme religieux, tout en ne retenant pas la notion de \textit{progrès} qui était présente chez Alan Race. Pour cela, il prend un grand théologien pour chacun des courants : l'exclusivisme avec Karl Barth, l'inclusivisme avec Karl Rahner, abandonnant l'ecclesiocentrisme, les théologies pluralistes, abandonnant le christocentrisme avec John Hick et Paul Knitter (regnocentrisme), une approche post-libérale dont il annonce dès l'introduction qu'il la développera particulièrement. Il suit particulièrement deux théologiens anglo-saxons,  George Lindbeck \cite{lindbeck_nature_2002}   et DiNoia. Cette approche ne cherche pas à trouver un plus petit dénominateur commun mais ce qui \textit{cristallise dans chaque religion} et la rend pertinente au futur de l'humanité, avec une analogie linguistique (religion comme une langue). 

Il termine par l'hypothèse qu'au sein d'une personne, on puisse faire l'expérience d'être \textit{bilingue} en différentes religions, avec une religion maternelle mais la possibilité de comprendre le champ culturel de l'autre.  Il ne s’agit pas de justifier de la double
appartenance : le théologien avertit de l’absolue nécessité de la non confusion ;
mais il s’agit de pouvoir goûter l’autre tradition, au point de se sentir comme
l’autre croyant. 

L'A. commence par présenter la condition post-moderne : l'hypothèse d'une \textit{culture techno-scientifique commune et universelle}, qui pouvait sous tendre la modernité (avec sa foi en la raison et dans le progrès) et de récits englobant, n'est plus tenable. La mondialisation fait cohabiter, parfois paisiblement, souvent de façon chaotique, les religions , les systèmes ou les modèles dans un même voisinage \cite[p. 10]{cheno_dieu_2017}. Les récits de la "modernité" croient en l'émancipation du sujet rationnel et celui de l'histoire de l'esprit universel. Face à la disparition de récits unifiants, On se fait sa petite sauce entre différentes religions et pensées philosophiques en une sorte de   « bricolage religieux », juxtaposition des croyances et de pratiques multiples et contrastées, quête d’identités fortes balisant le quotidien, Le religieux connaît un redéploiement de ses manifestations dans un contexte sociétal caractérisé par la généralisation de « l’individualisme narcissique » (Lasch, Sennet, Lipovetsky).  « Bricolage religieux », juxtaposition des croyances et de pratiques multiples et contrastées, quête d’identités fortes balisant le quotidien, conversions fragmentées, perte des références aux grands récits mythologiques, métissage… le religieux connaît un redéploiement de ses manifestations dans un contexte sociétal caractérisé par la généralisation de « l’individualisme narcissique ».  


Quelles réponses individuelles possibles ? Tout d'abord, R. Chéno présente la réponse intégriste/intégrale, qui essaye de reconstituer un récit unifiant de sa vie à partir d'anciens grands récits à mettre à jour \cite[p. 16]{cheno_dieu_2017}. Elle ne doit pas être confondue avec la réponse identitaire, dont le but n'est plus de donner du sens mais de définir non ce que nous sommes mais en prenant comme définition le groupe lui-même ("ceux qui sont pieux" par opposition "qu'est ce que la piété"). "ils sont chrétiens" "français".
L'A. souligne au contraire la chance de \textit{vivre aux éclats} :
\begin{singlequote}
  J'aime ce monde. La condition post-moderne n'est pas un fardeau, [\ldots]  
    Elle est une invitation à la rencontre, à l'échange et, peut être, au dialogue. \cite[p. 20]{cheno_dieu_2017}
\end{singlequote}

Comment penser théologiquement l'existence des autres religions en ce monde post-moderne ? Où est Dieu et qu’est-Il ? Le dialogue interreligieux a-t-il encore un sens et sous quelles formes ? À quelles conditions ?
Si la modernité n'a pas été "tendre" avec les religions, l'A. indique bien que les dangers pour la religion sont maintenant différents. Relégué par la modernité, le problème n'est plus le grand récit de l'athéisme moderne qui réfuterait les religions, mais la juxtaposition de  rationalités. L'A. convoque tout d'abord K. Barth pour la vision exclusiviste ("hors de l'Eglise, point de salut") et K. Rahner pour la vision inclusiviste ("hors du Christ, point de salut"). Il s'appuie à démontrer la pertinence des approches qui ne sont pas rendues obsolètes par les suivantes.
Pour les visions pluralistes, théologies libérales, il convoque  tour à tour Hick, Knitter, Panikkar, Amaladoss et Pieris. Il met en lumière le paradoxe que ces théologies qui invitent à la reconnaissance de la faillibilité de chaque religion – à commencer par la sienne – en vue de promouvoir une approche mutualiste, mais qui glissent « vers un impérialisme de la pensée ». Car nos théologiens pluralistes, 
\begin{singlequote}
    « à vouloir entrer dans un dialogue mutuel, tendent à réduire la diversité, voire à la rejeter »  \cite[p. 104]{cheno_dieu_2017}
\end{singlequote}

Quant aux conditions du dialogue, ils imposent la aussi une vision impérialiste puisqu’il faudrait abandonner tout ce qui n’est pas négociable. Le cadre dans lequel est pensé le pluralisme devrait s’imposer à toutes les religions. Or, ce cadre ne reflète-t-il pas des conceptions propres aux traditions de leurs auteurs ? Par ailleurs, la prétention à s’extraire de sa propre tradition religieuse pour contempler les convergences ne revient-elle pas à adopter le point de vue de Dieu lui-même ? \cite{pisani_cheno_2018}

La dernière approche présentée est celle du théologien luthérien George Lindbeck \cite{lindbeck_nature_2002}et l'approche post-libérale. il s'agit de ne pas minimiser les différences entre religions  car ce sont elles  qui donnent à chaque religion une valeur singulière et vitale : 
\begin{singlequote}
     Une religion contribuera probablement davantage au futur de l’humanité si elle préserve ses propres caractéristiques et son intégrité que si elle cède aux tendances homogénéisantes qui vont avec l’expressivisme expérientiel libéral \cite[p. 115]{cheno_dieu_2017} 
\end{singlequote}
Lindbeck propose une analogie avec l'approche culturo-linguistique : chaque religion a son langage, son territoire et produit une vision du monde propre. 
Mais il s’ensuit une question : la vérité d’une doctrine est-elle liée exclusivement à la communauté qui la produit, à sa cohérence interne ? Plus encore, le caractère incommensurable des religions peut-il être postulé indépendamment de la culture qui la produit ? Peut on même encore parlé de religions universelles ? 

Et qu’en est-il du dialogue interreligieux ? Dans la perspective post-libérale, le dialogue interreligieux se fonde sur la reconnaissance de la cohérence interne à chaque religion et à sa prétention à l’exclusivité (insurpassabilité) \cite[p. 129]{cheno_dieu_2017}. Si l'approche post-libérale permet de défendre le caractère providentielle de la diversité des religions, le dialogue interreligieux connaît une nouvelle acuité avec la pluralité des territoires dans lesquels habite l’individu.


\paragraph{Quelques aspects critiques}
 R. Cheno soutient la possibilité « d’habiter plusieurs religions à la fois » sans confusion\cite[p. 147]{cheno_dieu_2017}. Il s’agit de pouvoir goûter l’autre tradition, au point de se sentir comme l’autre croyant. On peut se demander si Remi Cheno, par sa capacité à évoluer dans ce nouveau monde comme un \textit{anywhere}, selon la définition qu'en donne David Goodhart, ne décrit pas une vision de la religion que ne serait accessible qu'aux \textit{happy fews} de la mondialisation : 
\begin{singlequote}
    La mentalité des \textit{Anywhere} [\ldots]lui semble révélatrice d'un « individualisme progressiste ». «Elle accorde beaucoup de valeur à l'autonomie, à la mobilité et à l'innovation, et nettement moins à l'identité de groupe, à la tradition et aux pactes nationaux (Église, patrie, famille). La plupart des Anywhere voient d'un bon oeil l'immigration, l'intégration européenne et la diffusion des droits humains, autant d'éléments qui ont tendance à diluer les revendications nationales » (p. 19). ... Un groupe social représentant 20 à 25 \% de la population de nos démocraties, qui «  prédomine parmi les décideurs et les faiseurs d'opinion » (p. 48), et comporte un sous-groupe plus radical de 5 \% qu'il appelle les « villageois planétaires » et qui, lui, se recrute principalement « dans l'enseignement supérieur et dans les milieux de la création» (p. 61). \cite{christophe_boutin_anywhere_2022}
\end{singlequote}

\newpage



 
 
 
%-------------------------------------------------------------------------------------------------------
\paragraph{l'Unique et ses témoins - Ch. Theobald} \cite{theobald_christianisme_2007}   

Dans son livre \textit{Le Christianisme comme style, une manière de faire de la théologie en post-modernité} paru en 2007 \cite{theobald_christianisme_2007}, Christoph Theobald prolonge sa théologie vers le dialogue avec les religions monothéistes dans un chapitre que nous nous proposons d'étudier, intitulé : \textit{l'Unique et ses témoins, Jalons pour une théologie de la rencontre entre juifs, chrétiens et musulmans}.  
 Le chapitre que nous étudions se trouve dans la partie IV ainsi introduite : 

 \begin{singlequote}
     Après avoir "ausculté" notre présent et désigné le \textit{kairos} qu'il représente (I) et après avoir réfléchi longuement à l'enracinement spirituel (II) et scripturaire (III) de la théologie chrétienne, le moment est venu d'aborder directement ce que celle-ci doit donner à penser aujourd'hui : le christianisme comme style qui ouvre à une intelligence de lui-même, libre et accessible à tous, petits et grands. 
     \cite[p 699]{theobald_christianisme_2007}
 \end{singlequote}
 
l'A. organise cette partie à partir du \textit{Credo}, en appliquant une \textit{herméneutique dogmatique} \cite[p. 700]{theobald_christianisme_2007}  qui fait le va et vient entre le dogme, versant normatif du mystère chrétien, les textes canoniques, l'histoire et la pratique actuelle de l'Eglise. La première partie présente la foi en Dieu et de la Trinité en post-modernité, en lien avec l'\textit{éthos} chrétien  : 
 

\begin{singlequote}
    On pourrait certes envisager une relation "binaire" entre l'homme et le tout Autre, relation d'alliance avec le "Dieu unique" dont Abraham reste le prototype, mais notre entrée - à égalité\;- dans une relation de familiarité avec Dieu, telle que Jésus de Nazareth l'a risquée avec son "Père", resterait inconcevable; or, c'est cet accès, surprenant et apparemment excessif pour nos possibilités humaines, que nous attribuons à l'"Esprit de Sainteté". \cite[p. 705]{theobald_christianisme_2007}
 \end{singlequote}
 
 Puis Theobald explore le lien entre foi trinitaire des chrétiens et lien social, pour enfin couvrir la question de la multiplicité des témoins et religions monothéistes, le chapitre qui nous intéresse : \textit{ l'Unique et ses témoins, Jalons pour une théologie de la rencontre entre juifs, chrétiens et musulmans.}

 
La problématique se déplace sensiblement entre l'article du colloque et le chapitre éponyme : le colloque tente d'éclairer la question de l'unicité de Dieu à travers plusieurs religions ou \textit{témoins} et pour cela, fait \textit{jouer} le système \textit{Dieu unique, témoin} (prophète,...) et \textit{tiers} (Egypte ou Nations pour Israel, juifs et grecs pour les Chrétiens,...). 
Le chapitre  du livre questionne quant à lui  \textit{l'énigme de la violence entre les trois témoins} et de la difficulté à communiquer entre religions \cite[p.780]{theobald_christianisme_2007}. Cette violence discrédite les religions monothéistes pour nos contemporains. Comme il n'est pas possible de répondre à la place de l'autre, la réponse viendra ici d'une réflexion proprement chrétienne sur la rencontre entre religions.

 

 
 Méditant la figure de Melchisédech, son hypothèse est que le mystère de l'Incarnation et de la Trinité - différence fondamentale du christianisme par rapport au judaïsme et à l'islam - ,  est en même temps le lieu  où se définit une\textsc{ théologie de la rencontre} \cite[p. 793]{theobald_christianisme_2007}. 
 
 

\newline
 
L'argumentation générale se structure autour de l'idée de la rencontre comme apprentissage.  Tout d'abord, l'auteur présente une hypothèse de classement du judaïsme, du christianisme et de l'islam, appelant :
\begin{itemize}
    \item le judaïsme, \textsc{monothéisme éthique}, Israël se définissant par l'injonction éthique d\textit{'aimer l'étranger car en Egypte, vous fûtes des étrangers} (Dt 10,17-1).  
    \item le christianisme, \textsc{monotheisme méta-éthique} au sens où il insiste sur la communication de l'amour excessif de Dieu à tout être humain.
    \item l'islam, \textsc{monothéisme pre-éthique} car sa lutte pour l'unicité de Dieu surpasse toute préoccupation éthique. 
\end{itemize}
 A l'issue de cette première comparaison, une première difficulté de la communication entre religions apparaît car elles ne se placent pas aux mêmes plans. 


{POur éviter la violence entre religions, la comparaison entre religions doit se situer au niveau de leur \textit{style}.} Nous l'avons vu, la modernité impose ses règles du jeu dans la communication entre religions, et en particulier elle impose le \textit{comparatisme}, ce qui peut se traduire par une réelle violence. Chaque religion est donc invitée à relire son propre patrimoine et à entrer dans le \textit{jeu difficile d'une communication qui consiste désormais à conjuguer le regard interne à sa foi sur les deux autres traditions et la perspective externe des deux autres sur lui} \cite[p. 788]{theobald_christianisme_2007}.

 Les sociétés modernes imposent que toute rencontre véritable soit un processus d'apprentissage. Mais un tel processus n'est pas extrinsèque aux religions monothéistes, à travers différentes tournures  comme la figure du prophète pour le judaïsme. Pour les chrétiens, Jésus est le grand "apprenant" par sa souffrance et son obéissance (He 5,8), et  aussi par ses rencontres rapportées par les synoptiques où Jésus \textit{apprend} des autres qui il est.  

{Pratiquement cette rencontre est un processus d'apprentissage.} Comme pour tout style, cet apprentissage passe un processus; d'abord une purification de nos préjugés et le refus de la substitution de l'un ou l'autre témoin, de son exclusion ou de l'inclusion de l'autre dans sa propre mission. La \textit{Règle d'or} est alors l'étalon de la justesse de notre attitude dans la communication avec l'autre témoin. Puis, il s'agit de penser positivement nos liens, soit par la mystique, les courants spirituels traversant aisément les frontières entre les religions,  soit en pensant le jugement d'excellence que je porte sur ma propre religion sans qu'il ne produise de la violence. Pour cela, il faut  accepter que la raison de la multiplicité des \textit{témoins} fasse partie du dessein de Dieu et en rendre compte, chacun avec les ressources propres de sa religion. 

 
l'A. propose alors une application pratique proprement chrétienne : il s'agit de penser la multiplicité des témoins, en ne s'arrêtant pas au nombre de trois mais en lien avec le mystère de l'Incarnation et de la Trinité, ce que Theobald appelle son \textit{hypothèse d'une théologie de la rencontre}. 
 


 L'A. explicite alors ce qu'est le style chrétien, à partir de nombreuses références bibliques (Ps 110, 3; Mt 5, 20.44; Mt 7,12, Lc 3, 22) : la sainteté de Dieu à laquelle nous sommes appelés est l'amour démesuré qui n'attend aucune réciprocité. Elle dépasse donc la Règle d'or. Etre\textit{ engendré}, devenir fils de notre Père, c'est finalement reconnaître  cet appel démesuré à être \textit{comme} Dieu, toujours dans telle ou telle situation pratique. L'A. s'appuie sur la lettre aux Hébreux, où Jésus est à la fois \textit{associé} à l'unicité de Dieu (He 7, 2s) mais, par son sacerdoce selon l'ordre du roi Melchisédech, prince de la \textit{paix}, il ouvre à la multitude les chemins de la Sainteté de Dieu et leur permet d'être engendrés fils. 

 
Ayant ainsi défini ce qu'est le style de vie chrétien, on peut alors revenir au processus d'apprentissage et définir le {style proprement chrétien de la rencontre.} . La purification nécessaire sera pour les chrétiens, celle des peurs qui marquent notre \textit{mesure humaine}. Et de façon positive, le chrétien qui s'est affronté dans sa propre vie à la question de la sainteté et de la démesure divine, peut admettre que juifs et musulmans sont eux aussi, aux prises aux mêmes combats. Theobald définit finalement ce \textit{style chrétien de la rencontre}, comme le \textit{style} qui se caractérise par une singulière manière d'espérer la paix en affrontant la violence. Il conclut avec Philon d'Alexandrie (\textit{le mode de la victoire n'est pas le même pour tous mais que tous sont dignes d'estime}) en assignant aux chrétiens la mission de montrer l'unicité de chaque témoin. 



%-------------------------------------------------------------------------------------------------------
\paragraph{Ecologie - Approche de Théobald } Dynamique spirituelle à l’œuvre dans le projet, selon une lecture chrétienne inspirée par les analyses de Christoph Theobald sur la foi élémentaire qui s’exprime chez certains de ceux qui contribuent à de tels projets au service de la justice sociale et écologique, sans référence explicite à Dieu.      
\cite{de_benaze_conversion_2020}








%-------------------------------------------------------------------------------------------------------
\subsection{Théologie des religions et Ecologie - Une lecture Theobaldienne - Bibliographie}
%-------------------------------------------------------------------------------------------------------

%-------------------------------------------------------------------------------------------------------
\paragraph{\textit{conversion} et \textit{dialogue} : comment l'articulation de ces deux pôles éclaire la théologie des religions ?}
 \cite{lasida_parler_2020} \cite{campos_laudato_2017} \cite{puglisi_religious_2020}


 \paragraph{décentrement}
 \begin{singlequote}
   Le changement de style de vie qui vise à s'arracher au consumérisme est fondé sur l'invitation à sortir de soi, à quitter une attitude autoréférentielle, pour faire attention à l'impact de chacune de nos actions sur les autres et sur l'environnement (208).   Le repos et l'eucharistie sont également présentés comme une manière d'inscrire notre agir dans une dimension réceptive et gratuite (237). Un seul et même mouvement rassemble ces différentes dimensions de la conversion écologique : un mouvement de décentrement.   Don, interdépendance et espérance sont également au cœur des deux prières qui clôturent le chapitre et l'encyclique, et qui sont encore un appel à construire un avenir partagé.   \cite[p. 191]{francois_loue_2020}  
 \end{singlequote}


%-------------------------------------------------------------------------------------------------------
\paragraph{Laudato Si’ et religion} \cite{powell_laudato_2017}    \cite{pisani_ecologie_2016}
 

   
  
  
%-------------------------------------------------------------------------------------------------------
\section{Les réponses des différentes religions au défi écologique - Bibliographie}
%-------------------------------------------------------------------------------------------------------
\paragraph{Problématique} ouverture aux autres religions qui sont appelées à relever ensemble ce défi. Le rôle de l'Eglise est de révéler, accueillir les autres religions dans la façon propre. Theobald. Retrouver les points durs des autres religions pour nous aider à ne pas proposer une solution de type "Dieu et l'écologie" mais bien penser comment s'articule l'un et l'autre.  


%-------------------------------------------------------------------------------------------------------  
\paragraph{13 novembre 2022 - en marge de la COP 27}
L’archevêque anglican Rowan Williams conduit des chefs religieux sur la colline du Parlement pour une cérémonie de repentance climatique le 13 novembre à Londres.  
 
%------------------------------------------------------------------------------------------------------- 
\subsection{Une réponse Chrétienne}

%------------------------------------------------------------------------------------------------------- 
\subsection{Une réponse de l'Islam pour la théologie Chrétienne}

%------------------------------------------------------------------------------------------------------- 
\paragraph{Ecologie en Islam et Dialogue Interreligieux} \cite{pisani_ecologie_2016} 
    Toynbee montrait aussi que c’est au contact les unes des autres, dans l’interaction de leurs mythes et de leurs théologies que se créent les conditions de l’avenir.
    Laudato Si’ : l'enjeu nécessite le concours de tous.
   
    importance de la situation de ‘Alī al-Ḫawwāṣ dans LS (reconnaissance de l'héritage écologique de l'Islam).     
    \textit{habitus écologique} : définies par des rites ou s'affirment des attitudes singulières ou s'entremêlent monde spirituel et monde matériel. 
    Une réponse : l’islam est la solution (si crise, c'est qu'on n'est pas assez islam; pas d'ouverture aux autres religions).
    
    fitra : revenir à un état originaire. la nature vrai musulman, ne se rebelle pas.
    


 
%-------------------------------------------------------------------------------------------------------     
\paragraph{L'analogie avec l'injustice du taux à intérêt} Les religions monothéistes ont toujours porté des interdictions fortes car source d'inégalité. 
  
%-------------------------------------------------------------------------------------------------------   
\paragraph{Ecologie et Religions - Colloque IDEO 2022 } \cite{pisani_ecologie_2022}

 
\begin{quote}

S2 : 30 - Lorsque Ton Seigneur confia aux Anges: "Je vais établir sur la terre un vicaire "Khalifa". Ils dirent: "Vas-Tu y désigner un qui y mettra le désordre et répandra le sang, quand nous sommes là à Te sanctifier et à Te glorifier?" - Il dit: "En vérité, Je sais ce que vous ne savez pas!".

\end{quote}
\label{Comment:MemoireISTR6}   





\paragraph{Emmanuel Pisani - Al-Ghazali et Ecologie} \cite{pisani_ecologie_2022} \cite{bouguignat_denys_revue_2023}

\label{theol:AlGazali25}
\label{Comment:MemoireISTR7}    




\paragraph{Ibn-Taymiyya et Ecologie} \cite{pisani_ecologie_2022}
 \label{Comment:MemoireISTR8}    


\paragraph{Ecologie et religions - Fabien Révol} \cite{pisani_ecologie_2022}
 \label{Comment:MemoireISTR9}   



%------------------------------------------------------------------------------------------------------- 
\subsection{quelle regard chrétien d'une religion écologique}
 \label{Comment:MemoireISTR10} 


\newpage