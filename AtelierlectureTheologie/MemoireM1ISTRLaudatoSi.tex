\chapter{Le Dialogue des Religions et \textit{Laudato Si}}

\textit{La présentation du travail lors du Forum des masters est plus longue que les deux pages de consigne mais il s'agit principalement de mon plan de travail ainsi que des principales citations pertinentes lues à ce jour et la bibliographie indicative. 
}
% --------------------------------------------
\section{Problématique} 
\paragraph{Question de la pertinence des religions}
\begin{singlequote}
        Qu’une religion soit raisonnable [donc universelle] dépend largement de ses
pouvoirs d’assimilation, de sa capacité à fournir dans ses propres termes une
interprétation intelligible des diverses situations et réalités que rencontrent
ses adhérents. Les religions que nous qualifions de primitives échouent régulièrement
à ce test quand elles sont confrontées à des changements importants,
tandis que les religions mondiales développent de plus grandes ressources pour
faire face aux vicissitudes \cite[ p. 175]{lindbeck_nature_2002}.
\end{singlequote}

\paragraph{Enjeu majeur pour les religions : pertinence par rapport au changement climatique}

\paragraph{Laudato Si}  \cite{francois_laudato_2015}

\subsection{Première lecture de Laudato Si }

\paragraph{six occurrences du mot Religions} et en particulier la section V du Chapitre V : Les religions dans le dialogue avec les sciences.
\begin{singlequote}
     nous ne pouvons pas ignorer qu’outre l’Église catholique, d’autres Églises et communautés chrétiennes – comme aussi d’autres religions – ont nourri une grande préoccupation et une précieuse réflexion sur ces thèmes qui nous préoccupent tous » (LS 7)
        Dans le sillage du concile Vatican II, l’encyclique insiste sur la contribution des religions en tant que vecteur d’une vision et d’une relation à la nature qui permet de répondre aux défis environnementaux et de proposer une alternative ancrée dans une sagesse séculaire pour éviter « l’indifférence, la résignation facile ou la confiance aveugle dans les solutions techniques » (LS 14). Elles constituent une richesse « pour une écologie intégrale et pour un développement plénier de l’humanité » (LS 62).  « Tous, nous pouvons collaborer comme instruments de Dieu pour la sauvegarde de la création, chacun selon sa culture, son expérience, ses initiatives et ses capacités » (LS 15).  \cite{francois_laudato_2015}
\end{singlequote}
\begin{singlequote}
    201. La majorité des habitants de la planète se déclare croyante, et cela de-
vrait inciter les religions à entrer dans un dialogue en vue de la sauvegarde de
la nature, de la défense des pauvres, de la construction de réseaux de respect
et de fraternité. 
\end{singlequote}
 
 



\paragraph{Comment penser Dialogue inter religieux et changement climatique} Notre hypothèse sera que Laudato Si n'est pas un simple texte de circonstance, qui doit \textit{cocher des cases} et en particulier le \textit{dialogue inter-religieux}, avec une articulation "Et" : dialogue interreligieux \textit{et} conversion écologique. Mais au contraire, reconnaître le style profondément théologique et construit de l'encyclique. 

\paragraph{On ne peut faire l'économie du travail théologique}
Dès lors, trouver la pointe du \textit{dialogue inter-religieux} ne peut faire l'économie d'un travail théologique. Le titre de l'encyclique \textit{Maison Commune} en est une première piste.

% --------------------------------------------
\section{Les ressources théologiques qui précèdent Laudato Si}


% --------------------------------------------
\subsection{La doctrine traditionnelle Sociale de l'Eglise}

\paragraph{dimension universelle de l'Eglise à son discours éthique}
\begin{singlequote}
    La tradition catholique s’est toujours souciée de garder une dimension universelle à son discours éthique. Historiquement, cette réflexion théologique qui souhaite s’adresser à tous s’est d’abord fondée sur la notion de loi naturelle, un concept susceptible de deux interprétations. L’encyclique du pape François\textit{ Laudato si} se présente comme une nouvelle manière de faire et un laboratoire de la recherche visée : elle allie une perception adéquate de la crise, une vision de ce qui est recherché, un dialogue à tous les niveaux et la proposition de ressources spirituelles. Un tel processus dynamique insiste sur la double nécessité de convertir nos attitudes et de rechercher un consensus sur nos manières d’être, nos styles de vie. Il souligne le rôle des religions et des sagesses pour fournir une « mystique qui nous anime ».\cite{thomasset_recherche_2020}
\end{singlequote}

\paragraph{Conversion dans la doctrine sociale de l'Eglise}

\paragraph{Doctrine occidentale sociale}

\paragraph{Loi Naturelle} \cite{thomasset_recherche_2020}


% --------------------------------------------
\subsection{Hans Kung}


\paragraph{Hans Kung, Ethos et dialogue} Approche \textit{pragmatique}. 
\begin{singlequote}
            Il faut plutôt chercher à atténuer, par une solution pragmatique de problèmes urgents, les oppositions entre visions du monde, sans tenir compte des différences idéologiques : cela pourrait à long terme établir des points communs, y compris justement un \textit{éthos} commun. Le conflit des visions du monde ou des idéologies devrait être apaisé de cette façon. \cite{kung_lethique_2009}
\end{singlequote}


\paragraph{La charte de la terre} 
\begin{singlequote}
    une approche pluraliste :
éthique (ou ethos) planétaire, c’est à dire un accord fondamental en matière d’axiologie, de critères indiscutables et de choix essentiels. A défaut d’un consensus éthique fondamental, toute communauté court tôt ou tard le risque du chaos ou de la dictature.  Un ordre mondial meilleur ne peut se concevoir sans éthos planétaire  \cite{kuschel_manifeste_1995}
(préface. p6) 
{...}
“ethique planétaire ne signifie ni idéologie planétaire, ni religion mondiale unitaire à côté des religions existantes, ni quelque forme syncrétique de toutes les autres religions. Notre humanité est lasse des idéologies unitaires et les diverses religions du monde sont de toute manière si différentes dans l’expression de leurs croyance et dans leurs dogmes, dans leur symbolique et leurs rites, que tout effort d’”unification” est dénué de sens. P 6

\end{singlequote}

\paragraph{Des résultats décevants mais la charte citée  par le Pape}  
\begin{singlequote}
    Un principe se retrouve depuis des milliers d’années dans beaucoup de traditions religieuses et éthiques de l’humanité qui l’ont conservé, c’est la “règle d’or”; ce que tu ne veux pas qu’on fasse à ton endroit, ne le fais pas à l’endroit d’aucun autre. \cite{kuschel_manifeste_1995} P 23
\end{singlequote}
\begin{singlequote}
     \cite{noauthor_charte_2022} LS 207 La Charte de la Terre nous invitait tous à tourner le dos à une étape d’autodestruction et à prendre un nouveau départ, mais nous n’avons pas encore développé une conscience universelle qui le rende possible.
\end{singlequote}

\begin{singlequote}
  [les théologiens contemporains] ont, pour la plupart, sous des formes diverses, essayé de découvrir un horizon commun, impliquant à la fois le mouvement globale de l'histoire et le devenir des religions; Pour ma part, je ne pense pas que l'on puisse lui donner un contenu défini ou concret, on peut à la rigueur le viser avec des formulations formelles empruntées à ce que H. Küng appelle une "éthique planétaire". Cette perspective, il est vrai, demeure floue et ne suscite pas l’enthousiasme soulevé par les grandes utopies du XX siècle. Elles ont démontré par leur réalisation politique leur caractère illusoire et trop souvent dérisoire et cruel.Il me paraît donc inutile de désigner un horizon commun qui serait supposé favoriser le dialogue en proposant une base minimale d'accord.   \cite[p 243]{duquoc_unique_2002}
 
\end{singlequote}
\begin{singlequote}
    

 la principale critique adressée aux théologies pluralistes, c’est leur prétention à disposer d’un lieu tiers, d’un arrière-plan qui se situerait au delà des religions particulières et à partir duquel on pourrait les embrasser toutes : le plan nouménal de la Rélité ultime pour John Hick, une même expérience mystique pour Raimon Pannikar \textit{ou encore un même projet éthique pour la justice et la gestion écologique des ressources de notre planète.}   p. 111  \cite{cheno_dieu_2017}

\end{singlequote}
\begin{singlequote}
    Cette crise, source de migrations violentes et contenant en elle la possibilité prochaine des guerres, peut aussi être un lieu de rencontre, de dialogue et d’action (LS 15) entre tous les hommes.
\end{singlequote}


\paragraph{Rôle des religions et de l'Eglise} Revenir à la source 

\begin{singlequote}
    L'asymétrie pousse les Églises à la conscience ferme d'une ignorance parce que les desseins de Dieu sur ce monde intermédiaire ne lui sont révélés que sous des espérances aux contenus indicibles ou aux contours flous et des impératifs éthiques non étrangers à la marche plus ou moins chaotique de chaque fragment. En ce sens, l'Église est sacrement du salut dans la mesure où elle renonce à être le tout, c'est-à-dire à étre unique lieu de l'Esprit et la déléguée de son Seigneur. Une analogie se dessine ici entre le parcours historique de Jésus et le chemin terrestre de l'Église. \cite[p 241]{duquoc_unique_2002}

\end{singlequote}
\begin{singlequote}
    Chéno montre que contrairement à l’approche libérale centrée sur l’expérience du sujet et qui serait le lieu où les croyants se retrouvent, le post-libéralisme appelle à ne pas minimiser les différences, bien au contraire ; dans une approche utilitariste, elles sont mêmes précieuses car ce sont ces différences qui donnent à chaque religion une valeur singulière et vitale.  \cite{pisani_cheno_2018}
\end{singlequote}

\begin{singlequote}
    Il s’agit donc pour toutes les religions de puiser dans « leur propre héritage éthique et spirituel », de revenir « à leurs sources » pour « mieux répondre aux nécessités actuelles » (LS 200).
\end{singlequote}

\paragraph{Rôle de l'Eglise} On pourra voir comment François dessine le rôle de l'Eglise
\begin{singlequote}
En ce sens, l'Église est sacrement du salut dans la mesure où elle renonce à être le tout, c'est-à-dire à étre unique lieu de l'Esprit et la déléguée de son Seigneur. Une analogie se dessine ici entre le parcours historique de Jésus et le chemin terrestre de l'Église.    \cite{duquoc_unique_2002}
p241
\end{singlequote}
% --------------------------------------------
\subsection{eco-théologie de la libération}

\paragraph{Théologie de la libération} Léonard Boff
{Cri de la terre et clameur des pauvres, quels chantier théologique et quelle pratique ecclesiale ?} \cite{thomasset_recherche_2020}
 

\paragraph{Une eco-théologie de la libération} 
\href{https://www.loyola.edu/academics/theology/faculty/castillo}{Daniel Castillo} expose comment une éco-théologie  de la libération : rapport à Dieu, aux hommes à la terre (et de tout ce qui provient de la terre - homme animaux). Contrairement à , il s'appuie sur le livre de la parole de Dieu. 
\begin{singlequote}
    Une religion contribuera probablement davantage au futur de l’humanité si elle préserve ses propres caractéristiques et son intégrité que si elle cède aux tendances homogénéisantes qui vont avec l’expressivisme expérientiel libéral \cite[ p. 115]{lindbeck_nature_2002}
\end{singlequote}

Il s'appuie sur Gustavo Gutierrez, une libération intégrale : socioéconomie, économique. Des structures, des valeurs et des imaginations et un combat entre le péché et la grâce. Une Influence sur LS 70.
 
\begin{singlequote}
    Dans ces récits si anciens, emprunts de profond symbolisme, une conviction actuelle était déjà présente : tout est lié, et la protection authentique de notre propre vie comme de nos relations avec la nature est inséparable de la fraternité, de la justice ainsi que de la fidélité aux autres. LS 70
\end{singlequote}
 

\subparagraph{L'écologie intégrale comme libération intégrale - Castillo} 
 Dieu est lui même le jardinier. Joseph est l'anti-adam et préfiguration du Christ.  L'exode, c'est le chemin d'apprentissage où le peuple est amené à retrouver sa vocation. 

\subparagraph{Lecture politico-écologique de la parole de Dieu} ne pas accumuler la manne. Glaneur donc faible. confiance en Dieu, limitation de l'accaparation.
Sabbat. Repos de la terre et du travail aussi des servantes \textit{et des animaux}. Lv : année sabbatique et année jubilaire. Pdt l’année sabbatique, on retrouve sa position de glaneur. Jubilaire : Quadruple restauration (terre, esclave,...) Jésus, Lc 4 : une année de bienfaits. Option préférentielle
pour les pauvres. cf LS 237 (Dimanche et sabbat).




\paragraph{l'argent comme idolâtrie }et pas seulement un désenchantement du monde

\begin{singlequote}
   François ne parle pratiquement jamais de l’économie contemporaine sans adresser une accusation d’idôlatrie, acccusation absente dans GS, et presque entièrement absente de Vatican II dans son ensemble.
Comment expliquer cette différence de traitement des questions économiques dans GS et chez le pape François ?
Ma thèse est que François représente une opportunité pour changer le discours catholique sur la sécularisation; une opportunité qui a des implications dans la manière de considérer non seulement l’économie, mais aussi d’autres phénomènes séculiers.
La pensée catholique progressiste dans la période du Concile Vatican II avait tendance à considérer le monde séculier comme désenchanté. François suggère au contraire, que nous ne sommes pas tant confronté à une perte de foi qu’à une nouvelle religion et une foi idolâtre. \cite{cavanaugh_idolatrie_2022} p. 126
  
\end{singlequote}



 



% --------------------------------------------
\section{Laudato Si, un texte théologique}

% --------------------------------------------
\subsection{Analyse de Laudato Si : un texte théologiquement structuré}

Peu de lectures encore sur la théologie de Laudato Si à part :  \cite{goujon_laudato_2022}.

 

\paragraph{Conversion et Dialogue : deux mots qui structurent LS}  "Conversion", du registre prophétique  à une logique sapientielle "dialogue". 

\paragraph{Laudato Si} prend un \textit{kairos} et en ce sens, devient prophétique, car parle en situation au sein d'une crise. Mal qui perturbe le peuple et Dieu. \textit{epistrophe}, retour vers Dieu, en ayant horreur de son comportement (metanoia). 

\paragraph{pilier prophétique et apocalyptique} Responsabilité personnelle et collective . Le début de Laudato Si
souligne l'urgence climatique et le changement radical du comportement. 
\begin{singlequote}
    Il serait naïf, évidemment, de prétendre lutter contre les désastres climatiques en s'en remettant à la seule prière, mais il ne serait pas moins naïf d'imaginer vaincre le mal sans s'attaquer à ses causes, et d'oublier que le premier lieu où je peux envisager de le déraciner, c'est dans ma propre vie.
En moi, le combat eschatologique est déjà engagé, avec sa violence et ses incertitudes :
c'est lui qui est à l'œuvre dans mes crampes d’égoïsme et dans mes envies de bien faire, dans mes fidélités et mes impatiences.Et dans ce combat, l'emporter, c'est d'abord accepter que la victoire a déjà été acquise, non pas par mes efforts, mais par l'amour infini qui se donne à voir dans la croix de Jésus et qu'il me faut, peu à peu, laisser entrer dans ma propre vie. \cite{candiard_quelques_2022} p.91
\end{singlequote}

\subparagraph{affliction prophétique} importance du \textit{mourning} - deuil affliction- chez les prophètes repris chez François - rôle des religions
\begin{singlequote}
26. As Brueggemann writes, the prophetic task of mourning "has three parts: (1) To offer symbols that are adequate to confront the horror and massiveness of the experience that evokes numbness and requires denial. ... (2) To bring to public expression those very fears and terrors that have been denied so long and suppressed so deeply that we do not know that they are there.... (3) To speak metaphorically but concretely about the real deathliness that hovers over us and gnaws within us and to speak neither in rage not with cheap grace, but with the candor born of anguish and passion?" Brueggeman, Prophetic Imagination, 45. 
From Mourning to Lament (p. 159) cité par \cite{cavanaugh_between_2018}
\end{singlequote}
Contre le déni - acceptation de la réalité
\paragraph{Pilier du dialogue} dès le début : 
\begin{singlequote}
    14. J’adresse une invitation urgente à un nouveau dialogue sur la façon dont nous construisons l’avenir de la planète. 
\end{singlequote}
 7 citations du livre de la Sagesse, \textit{poumon d'Israel pour respirer l'air commun} (Paul Beauchamp). 

\paragraph{Une définition de la Sagesse qui implique la vraie rencontre de l'autre} Ls 47 (un défi pour mon mémoire!) Voir  \cite{howles_quel_2022}
insistance sur la vraie rencontre de l'autre et réalité : 
 reprise de trois idées forces développées dans l’exhortation La joie de l’Evangile, qui méritent réflexion : “le temps est supérieur à l’espace” (178), ‘l’unité est supérieure au conflit” (198), “la réalité est supérieure à l’idée” (201).
\begin{singlequote}
C’est le troisième principe de lecture proposé dans la joie de l’Evangile : “la réalité est, tout simplement; l’idée s’élabore”.
“l’idée déconnectée de la réalité est à l’origine des idéalismes […] inefficaces, qui […] n’impliquent pas. Ce qui implique, c’est la réalité éclairée par le raisonnement”.
Pour un chrétien, “ce critère est lié à l’incarnation de la Parole et à sa mise en pratique” (233). Pour l’analyse et pour l’action, on peut rappeler un quatrième pricinpe avancé par le pape François : “le tout est supérieur à la partie”. Dans EG 235, il précise qu’il faut toujours élargir le regard […] sans pour autant se déraciner”. \cite{francois_loue_2020}
\end{singlequote}

\paragraph{Une tension} si on va vers l'autre, la rencontre, la vraie sagesse, comme revenir, se convertir à Dieu seul ? 

\paragraph{Un changement de paradigme pour la théologie}

\begin{singlequote}
   Entre ces deux extrêmes, la réflexion devrait identifier de possibles scénarios futurs, \textit{parce qu’il n’y a pas une seule issue}. Cela donnerait lieu à divers apports qui pourraient entrer dans un dialogue en vue de réponses intégrales. LS 60 
\end{singlequote}

François ne se situe pas ici dans une logique théologique mais néanmoins fait de la théologie.


 \paragraph{Une parole pas seulement informative mais transformative} Laudato Si est étonnant pas l'utilisation de poesies, des hymnes, des prières.  Certes des vérités dogmatiques sont présentes mais la manière de le dire.  Le texte est encapsulé entre Saint François d'Assise et de Marie (prière finale) : figure de sagesse prophétique, avec au centre le \textit{cri des pauvres et de la Terre}
 
% --------------------------------------------
\subsection{Lecture de la partie sur le dialogue inter-religieux}

\paragraph{\textit{conversion} et \textit{dialogue} : comment l'articulation de ces deux pôles éclaire la théologie des religions ?}
 \cite{lasida_parler_2020} \cite{campos_laudato_2017} \cite{puglisi_religious_2020}


 \paragraph{décentrement}
 \begin{singlequote}
   Le changement de style de vie qui vise à s'arracher au consumérisme est fondé sur l'invitation à sortir de soi, à quitter une attitude autoréférentielle, pour faire attention à l'impact de chacune de nos actions sur les autres et sur l'environnement (208).   Le repos et l'eucharistie sont également présentés comme une manière d'inscrire notre agir dans une dimension réceptive et gratuite (237). Un seul et même mouvement rassemble ces différentes dimensions de la conversion écologique : un mouvement de décentrement.   Don, interdépendance et espérance sont également au cœur des deux prières qui clôturent le chapitre et l'encyclique, et qui sont encore un appel à construire un avenir partagé.   \cite{francois_loue_2020}  P°191
 \end{singlequote}
 
\paragraph{Laudato Si et religion} \cite{powell_laudato_2017}    \cite{pisani_ecologie_2016}
 

       
 

 

  

 



 

  
