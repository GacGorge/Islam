\chapter{Santedi - Les défis de l'évangélisation
dans
l'Afrique contemporaine }

\section{Santedi}

\paragraph{Léonard Santedi Kinkupu }  Né à Kinshasa le 1er octobre 1960, il  est ordonné Prêtre de l'Archidiocèse de. Kinshasa le 1er août 1985.

\paragraph{ICP } Maintenant université du Congo. 




\paragraph{Repérer l'univers du théologie} Il cite Geffré (Herméneutique : interpréter la foi dans le contexte). Il ne s'agit pas de répéter la même chose. Schillebeeckx : il s'agit de partir de l'expérience.  
Théologie de la libération (Boff). Le magistère est peu cité, y compris Vatican II (en tout cas dans cette partie). 

\paragraph{Thème} En théologie pratique de la mission et dialogue des religions, peu de références aux pères de l'Eglise. Il y a des pans de l'activité de l'Eglise qui n'ont pas été \textit{pensée}. 

\subsection{L'ouvrage}

2005.

3 parties : 
\begin{itemize}
    \item Défi des sectes (mouvements religieux), 
    \item éléments de l'évangélisation

    \item cas concrets de la nouvelle évangélisation
\end{itemize}

\paragraph{La joie de l'Evangile} écrit par François suite au synode sur la \textit{nouvelle Evangélisation}. François revient à la mission. 
% -----------------
\section{Texte}

\paragraph{Critique de Santedi}
Santedi est assez critique. Il argumente par contraste. Il dénonce non seulement ce qui se passe en Afrique et à l'intérieur de l'Eglise. \textit{Renoncer} de son lien avec les puissants. 

\paragraph{modélisation par contraste} voila ce qu'on pensait. Cela ne marche plus. et voilà une nouvelle approche. On peut parfois caricaturer l'approche ancienne. La rupture est parfois trop dialectique. 
\begin{Ex}
    Missionnaires en Afrique. Ne pas faire d'anachronisme. Il ne s'agit pas de critiquer le passé mais plus que cela s'est fait avant \textit{dans certain contexte} et que cela ne veut pas dire qu'il faut continuer à faire de même. 
    
\end{Ex}


\subsection{Éléments d'une nouvelle
évangélisation en Afrique}

\begin{quote}
    Pour relever tous ces défis, n'éprouvons-nous pas le besoin
de revoir nos grilles de lecture dans un moment où, trop souvent,
la tentation est grande de nous enfermer dans les schémas
de la pensée unique ? Dans la résignation ? Dans notre petit univers
quotidien ? Au-delà des « scénarios de crises » et des catastrophes,
comment regarder l'Afrique autrement, sans vouloir
redécouvrir l'inventivité des sociétés dont les « réveils » et les
dynamismes bouleversent les certitudes des discours institués ?1
Au regard de l'immense tableau des défis auxquels les
Églises d'Afrique font face, il appert qu'elles sont aujourd'hui
devant une tâche délicate et périlleuse, et certainement urgente
et vitale. Cela excède manifestement les minuscules ressources
d'une évangélisation« triomphante» et de surface. Cette tâche
est de penser, d'inventer une nouvelle manière d'annoncer
l'Évangile, une nouvelle évangélisation (militante!) qui affronte
sans détour les innombrables maux assaillant l'Afrique; elle
révèle, par ce fait, la créativité des Églises d'Afrique et conduit à
l'émergence des expressions marquant les nouveaux visages
d'une Afrique engagée dans un vaste mouvement de libération.
\end{quote}

\section{Texte}
\subsection{Une évangélisation dialoguale} Foi et Culture. 
\textbf{convivialité}. 
\begin{itemize}
    \item Respecter son interlocuteur
    \item se reconnaitre d'une culture
    \item égalité
\end{itemize}

\paragraph{pratiquement}
\begin{quote}
    Au demeurant, une \textbf{évangélisation qui ne parvient pas à
apporter sa contribution à l'immense processus de gestation
d'une humanité solidaire, structurée dans la participation, la
diversité et la communion, ne sert pas le Dieu-Trinité.} La nouvelle
évangélisation doit être éminemment dialogale. Et puisqu'elle
ne doit viser la « christianisation » des cultures qu'en
raison du contexte pluraliste, elle ne peut qu'animer chacune
d'elles comme une force prophétique. P. 98
\end{quote}

\subsection{Une évangélisation prophétique}
\paragraph{Enraciné dans l'Evangile}  Mouvement. Le prophète doit unifier, exhorter, consoler (1 Co 14, 3)

\subparagraph{Défi politique, injustice} que le prophète africain doit dénoncer. Une église


\paragraph{Une Église qui dénonce} 



 
\paragraph{Une Église qui annonce}

\paragraph{Une Église qui renonce}
\begin{quote}
    Le Christ ne proclame pas simplement la parole prophétique.
Lui-même est cette annonce prophétique qui est en même
temps Parole effective.
\end{quote}

Recherche du salut d'ici bas : la santé, la réussite. Le Marabout. Contre des mouvements pentecotistes, qui proposent cela et partent des besoins. Et ces mouvements sont ils 

\subparagraph{Tout le monde est frère}

\subsection{Une évangélisation inventive}
\paragraph{Respecter la culture du pays} Chez moi. 

\paragraph{Action de l'Esprit Saint}


\paragraph{homme au présent}
\begin{quote}
    Comme le souligne P. Poucouta, le
prophète chrétien n'est donc ni un devin, ni un magicien, ni un
féticheur. Il n'est pas celui qui prédit l'avenir. Il est avant tout
l'homme du présent. Eclairé par la parole de Dieu, il essaie avec
ses frères de discerner la volonté de Dieu dans l' aujourd'hui.
C'est en ce sens qu'il peut ouvrir des perspectives
\end{quote}
% ---------------------------------------
\section{Critique}

\paragraph{Y a t'il une culture africaine ?}

\paragraph{Pourquoi le terme de nouvelle Evangélisation} laquelle est la première ?

\paragraph{ne pas comparer avec l'Eglise d'occident} 
\begin{Ex}[rite Congolais]
\begin{quote}
    Ce dont l'Afrique a besoin aujourd'hui, c'est d'autre chose
qu'une liturgie rubriciste et ritualiste1 il faudrait lui apporter du
fondamental là où l'homme rencontre son Dieu en vérité et en
Esprit et se laisse toucher et transformer par Lui. Et alors sa vie
devient amour, solidarité partage et non pas seulement rites et
culte. L1Église doit donc affermir chez tous les Africains l' espérance
en une vraie libération. La confiance de l'Afrique est fondée1
en dernière instance, sur la conscience de la promesse divine
nous assurant que notre histoire présente ne reste pas fermée
sur elle-même, mais qu'elle est ouverte au Règne de Dieu. C'est
pourquoi ni le désespoir, ni le pessimisme ne peuvent être justifiés
quant à l'avenir de l'Afrique et de toutes les autres régions
du monde.
\end{quote}
    
\end{Ex}


\paragraph{Accueil de cette théologie en Afrique} Programmatique

\paragraph{Incarnation} On demande au missionnaire \textit{d'oublier sa culture} . Pas tout à fait la mpeme chose, \textit{incarnation} et \textit{inculturation}

\begin{quote}
    Et c'est au nom de ce dialogue que nous congédions ce
discours chrétien sur l'inculturation évoquant l'image de
l'Évangile à la conquête des cultures, alors que l'Évangile
n'arrive que médiatisé déjà dans une culture (une triple culture
· même : celle de la Bible, celle de l'Église primitive et celle du
missionnaire). C'est une image de domination entretenue par la
prétention à la possession exclusive de la Parole. « La seule
option pour nous qui sommes dans cette situation, soutient
M. Amalados, n'est pas l'incarnation mais le dialogue qui nous
prépare à recevoir aussi bien qu'à donner. Ce qui signifie que
l'incarnation n'est donc peut-être pas le paradigme qu'il faut
utiliser pour comprendre le processus de la rencontre entre
Évangile et culture » p. 93
\end{quote}

\paragraph{Lexique employé} 
\begin{itemize}
    \item Vocabulaire personaliste : famille, frère, interpersonnel...
    \item on n'est pas dans la métaphysique grecque
    \item vocabulaire de la théologie de la libération : oppression, ...
\end{itemize}
Son texte, on le mettrait en Afrique mais on pourrait le mettre en Amérique latine. 

\paragraph{Elements Africains} La famille, la magie. 


\section{Annexe}

\subsection{Qu’est-ce que l’Église famille de Dieu en Afrique ?}

\mn{La croix}

10-13 novembre 2004 : ouverture du premier symposium d'évêques africains et européens, organisé par le Conseil de Conférences épiscopales européennes (CCEE) et les Conférences Épiscopales de l'Afrique et du Madagascar (SECAM), à Rome. Mgr Amédée Grab, évêque de Coire en Suisse, et Président du CCEE, Rome, Vatican.

Du 20 au 29 juillet se tient en Ouganda, le cinquantenaire du Symposium des conférences épiscopales d’Afrique et de Madagascar (Sceam) sur le thème « Église famille de Dieu en Afrique ».
Cette image, expliquent les théologiens, permet de « saisir en un même nœud, les valeurs de solidarité collective avec celles de fraternité familiale ».

\paragraph{D’où vient le concept d’Église « famille de Dieu » ?} 
Le concept d’Église comme « famille de Dieu » est absent de la Bible mais il puise ses fondements dans les Écritures et dans la Tradition. Il a été employé, pour la première fois dans les schémas préparatoires de la Constitution dogmatique Lumen Gentium du Concile Vatican II où il a été, après maintes explications bibliques et anthropologiques, « ajouté au nombre des images classiques de l’Église », indique le père Modeste Some, théologien originaire du Burkina Faso (1).

C’est à Mgr Simon Hoa Nguyen van Hien, évêque de Dalat au Vietnam, que l’on doit l’introduction de l’idée de l’Église comme Famille de Dieu, dans le document du Concile. Mais avant lui, les évêques de Haute-Volta (actuel Burkina Faso), dans un message pastoral adressé aux chrétiens avant de partir pour le Concile Vatican II, avaient déjà comparé l’Église à la famille humaine.

Pour le père Léonard Santédi, théologien et recteur de l’Université catholique du Congo (UCC) qui explicite ce concept, « dans l’histoire du salut, Dieu se présente comme une famille et il veut que tous les hommes vivent en sa famille ». À ses yeux, « l’expression “famille de Dieu” correspond parfaitement à la vision fondamentale de l’ecclésiologie de communion du Concile Vatican II, en référence au thème de “frères” ainsi qu’à la grande prière de Jésus : “Que tous soient un” ».

A lire : En Ouganda, le président Museveni offre les vêtements liturgiques de la célébration du jubilé d’or du Sceam

\paragraph{Pourquoi parle-t-on surtout d’Église famille de Dieu en Afrique ?} 
Pour l’Afrique, l’image de l’Église comme famille de Dieu constitue « une expression particulièrement appropriée de la nature de l’Église », faisait remarquer le pape Jean-Paul II dans son exhortation apostolique Ecclesia in Africa (1995). « Cette image se distingue par son esprit communautaire et met en évidence la solidarité, la sollicitude, la générosité, l’accueil, l’hospitalité, souligne le père Léonard Santédi. De nombreux théologiens africains ont vu, dans ce concept, une image permettant de saisir en un même nœud, les valeurs de solidarité collective avec celles de fraternité familiale. »

Mais pour le père Augustin Ramazani Bishwende, théologien originaire lui aussi de RD-Congo, il peut y avoir un danger à établir une analogie entre Église et famille africaine. « La conception patriarcale, gérontocratique et la solidarité organique, très clanique, tribale et ethnique, auraient des conséquences graves en ecclésiologie », fait-il remarquer (2). Il ne s’agit donc pas de calquer notre conception de l’Église sur la réalité de la famille humaine africaine. « L’Église famille de Dieu en Afrique implique à la fois la communion avec Dieu et la communion avec des frères et des sœurs, chrétiens, appelés à “une communion de vie, d’amour et de vérité” », précise le père Santédi.

\paragraph{ Dans les faits, l’Église est-elle famille de Dieu en Afrique ?}
Vivre l’Église comme famille constitue tout un programme pour les chrétiens d’Afrique. Ils sont appelés à bâtir une nouvelle fraternité, au-delà des ethnies et des tribus. « Hélas, dans une Afrique qui donne encore le spectacle d’un continent où des frères sont pris dans l’engrenage des affrontements ethniques et des violences tribales, construire dans le quotidien l’Église famille de Dieu demeure un grand défi qui interpelle les Églises, les pasteurs, les fidèles et les pasteurs africains », fait remarquer le père Santédi.

L’Église famille de Dieu en Afrique
Pour ce théologien, à l’heure de la célébration du jubilé d’or du Sceam, c’est une mission urgente pour les Églises d’Afrique de s’engager à construire une Église dans laquelle « la première identité des disciples du Christ soit la fraternité de tous en Jésus-Christ afin que personne ne se prévale d’une autre identité : ethnique, raciale, régionale, clanique pour semer la haine, la division et la mort dans l’Église ou la société ».

Lucie Sarr
 