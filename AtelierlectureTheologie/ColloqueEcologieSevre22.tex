\chapter{Colloque Ecologie RSR}

\section{Introduction}

\mn{Patrick Goujon}
\paragraph{penser la violence, celle faite à la terre}


\paragraph{Paradoxe du titre Laudato Si} Louer.

\paragraph{Une question } En quoi la question de conversion écologique est elle pertinente ? Autant théologiquement que pratiquement

\paragraph{La conversion s'impose à tous} Mais aussi pour les chrétiens à la Foi : 
\begin{itemize}
    \item les obstacles mis à la conversion radicale. 
    Conversion sous mode de l'urgence et impossible à contourner.
    Au nom d'une théologie de l'alliance que les chrétiens, au nom de leur foi, doivent participer à l'effort mondiale.
    Moins par devoir que par tâche morale devant l'emerveillement de la création.
\end{itemize}


 \subsection{« Adam, où en es-tu ? » : In memoriam Bruno Latour}
 \mn{Jean Greisch}

Adversaire farouche de tout réductionnisme, Bruno Latour était un penseur atypique. 
\paragraph{Laudato Si}
Réflexions stimulantes sur Laudato Si : contraste entre \textit{jubilez} et \textit{Laudato Si}. 
\begin{quote}
    Clameur des pauvres et de la terre à ne pas opposer. Dignité des exclus. Préserver la nature.
\end{quote}

\paragraph{Personnage terrassé } conversion de Saint Paul du Caravage. Nombreux contrastes de Latour. 

\paragraph{Dès 2010} Latour disait que les nouveaux mouvements ascétiques, conversion ne sont pas forcément des oppositions au capitalisme mais peuvent être liés à la théologie. 

\paragraph{3 caractèristiques du philosophe} de Deleuze. 
\begin{itemize}
    \item \textsc{Plan d'Immanence} : s'orienter dans la pensée \sn{Kant}. Carte des positions\sn{Où atterrir en politique, Latour}. les affects et enjeux en politique. Ce même plan d'immanence se retrouve dans l'utilisation de l'adjectif \textit{terrestre}.  \textit{Place centrale du climat et à sa dénégation}, clé de lecture de la politique depuis 50 ans. \textit{L'homme qui pensait la terre}
    \begin{quote}
        Mes frères, soyez fidèle à la terre .. vertu. 
        Mais
        Ainsi parlait Zarathoustra.
    \end{quote}
    critique insistance de la pervertion des notions religieuses d'\textit{ici bas}. 
    \item \textsc{concepts opératoires}
    \item \textsc{personnages} Gaia, hypothèse Gaia \mn{Liber figurarum de Joachim de Fiore. Risque d'une contre religion gnostique. D'après Latour, il faut maintenir l'horizon de la parousie. Sans la terre, quel pourrait être le sens de l'Esprit ? Lecture anti-gnostique de Latour}
\end{itemize}

\paragraph{Ici bas} depuis Galilée, . Nous perdions notre condition adamique de terrien ou terreux. Dénégation de la terre, à contre courant de l'entropie : \textit{c'est toujours la surprise}. Il n'y a plus de terre accessible. forme religieuse d'un monde 

\begin{quote}
    Il faut imaginer Sysiphe Heureux (Camus)
\end{quote}
Latour nous invite à imaginer le scarabée de Kafka (métamorphose) comme un homme heureux.

\paragraph{Quelle métamorphose la crise actuelle nous oblige à consentir ?} Critique acerbe de Latour de ceux qui ne pensent que par l'économie. 
\begin{quote}
    l'économie a ceci d'étrange, car elle s'occupe des choses quotidiennes mais elle le fait de façon le plus éloignée de nous. \textit{scientifique}
\end{quote}

\paragraph{conclusion}
Grâce à une situation donnée, abandonner les faux espoirs pour une nouvelle histoire : se mettre au boulot. Mon boulot, c'est de mettre des mots pour aider à la prise de conscience, mobiliser contre la catastrophe. 

\mn{Ou suis je ? Latour}
 \subsection{Table ronde sur le numéro préparatoire, avec les auteurs}
\mn{Didier Luciani, professeur émerite de Louvain. Loi véterotestamentaire; \textit{le pouvoir}

Arnaud Montoux : 1er cycle du Théologicum.  Poétique. 

François Euvé : Agrégé en physique, Etudes, D. Theo}

\paragraph{Parcours historique de F. Euvé} Une série de tournants (conversion ?). 1962 : livre de Rachel Carson, début d'une sensibilité écologique. GS : activité humaine. Dans l'Eglise de cette époque, manifestation teihardienne d'un regard positif sur la modernité et de l'action de l'homme.

Double mouvement : écologie critiquant l'anthropocentrisme du monde au profit d'une circulation : il n'y a plus de centre. Et critique annexe du Christianisme comme origine de la modernité \mn{avec un aspect paradoxal que la modernité a été en opposition contre le christianisme}.

\subparagraph{Collapsologie : vers un cosmocentrisme}
1970 - Paul VI - un texte sur la collapsologie.

\paragraph{Quelle thèse ? } Quel est la place de l'humain si on prend au sérieux ce courant de vie commençant à la création, tout en donnant une place particulière à l'acteur humain. 
Drame de la création : 


\paragraph{Mauvaise lecture} pour la réduction à quelques textes \sn{Didier Luciani}
\begin{Ex}
Ps 36,7 "homme et bêtes, tu les sauves".
    
\end{Ex}
Et pourtant, juste après, on parle du gras : on n'arrête pas les sacrifices.

Lv 25
Lv 23, 9-22 : le calendrier 

\begin{quote}
09 Le Seigneur parla à Moïse et dit :

10 « Parle aux fils d’Israël. Tu leur diras : Quand vous serez entrés dans le pays que je vous donne, et que vous y ferez la moisson, vous apporterez au prêtre la première gerbe de votre moisson.

11 Il la présentera au Seigneur en faisant le geste d’élévation pour que vous soyez agréés. C’est le lendemain du sabbat que le prêtre fera cette présentation.

12 Le jour où vous ferez le geste d’élévation de la gerbe, vous offrirez au Seigneur l’holocauste d’un agneau de l’année, sans défaut.

13 L’offrande sera de deux dixièmes de fleur de farine pétrie à l’huile, nourriture offerte, en agréable odeur pour le Seigneur ; et la libation de vin sera d’un quart de hine.

14 Vous ne mangerez pas de pain, ni d’épis grillés ni de grain frais moulu, avant ce même jour, avant d’avoir apporté le présent réservé à votre Dieu. C’est un décret perpétuel pour toutes vos générations, partout où vous habitez.

15 À partir du lendemain du sabbat, jour où vous aurez apporté votre gerbe avec le geste d’élévation, vous compterez sept semaines entières.

16 Le lendemain du septième sabbat, ce qui fera cinquante jours, vous présenterez au Seigneur une nouvelle offrande.

17 Vous apporterez de chez vous deux pains à offrir avec le geste d’élévation, chacun de deux dixièmes de fleur de farine cuits au levain, en prémices pour le Seigneur.

18 Avec le pain, vous apporterez en holocauste pour le Seigneur sept agneaux de l’année, sans défaut, un taureau et deux béliers, accompagnés d’une offrande et d’une libation, nourriture offerte, en agréable odeur pour le Seigneur.

19 Vous ferez aussi un sacrifice pour la faute avec un bouc, et un sacrifice de paix avec deux agneaux de l’année.

20 Le prêtre offrira ces deux agneaux, ainsi que le pain des prémices, avec le geste d’élévation devant le Seigneur. Ce sont des choses saintes pour le Seigneur, qui reviendront au prêtre.

21 En ce même jour, vous convoquerez les fils d’Israël ; ce sera pour vous une assemblée sainte, vous ne ferez aucun travail, aucun ouvrage. C’est un décret perpétuel pour toutes vos générations, partout où vous habitez.

22 Lorsque vous moissonnerez vos terres, tu ne moissonneras pas jusqu’à la lisière du champ. Tu ne ramasseras pas les glanures de ta moisson : tu les laisseras au pauvre et à l’immigré. Je suis le Seigneur votre Dieu. »
\end{quote}

"\textit{loi de sacristie}" et tout d'un coup, on parle du droit du glanage, déjà cité en Lv 19. 
Mais si on prend au serieux, ce texte. 
Donne une clé herméneutique au texte : une ligne sociale permet de créer une ressource pour penser que la Loi et leur Oeuvres, au mpmee niveau. 

                                                         
\paragraph{Rapppel que dire le monde influe sur notre monde} \mn{Arnaud Montoux} On est avant la grande période scholastique au XII. Il y a un autre langage, celui de la visio. A certains moments, on a estimé que le concept n'était pas adapté et du coup on utilise la visio : l'\textit{Ekphrasis} : epuise par excès le sujet en montrant à celui qui ne voit pas ce qu'il pourrait voir.

Occasion de repenser d'autres manières de dire Dieu : une manière chaotique de dire Dieu : dans l'humain blessé, souffrant, il peut y avoir une façon de dire Dieu.

Et du coup, dans la poesie, une façon de dire le monde.

\paragraph{Discussions} Tim Hawls. Penser notre univers interconnecté, vient interroger le statut de l'homme comme acteur responsable (imputation). Bruno Latour.

\subparagraph{Scheme de la restauration} Retour à un passé. Mais par le christianisme, voir ce qui est dans le présent et dans la promesse. Difficulté de court circuiter la tension eschatologique : vouloir faire aujourd'hui l'eschatologie aujourd'hui, avec le risque de violence. 

\paragraph{Comment en théologie penser un problème nouveau} La notion de nature n'est pas préoccupant dans les échanges. FEuv : nous aspirons à l'harmonie. l'empreinte humaine minimale pour laisser la nature se réparer elle-même. 
\begin{quote}
    Dieu crée le monde en courant. 
\end{quote}
Qu'est ce qu'on entend par restauration ? fantasme de stabilité ? Or, la nature n'existe que dans une procession, mouvement.

\paragraph{Necessité de créer des concepts adéquats pour penser la question nouvelle} avec le risque de rapatrier les concepts passés. Du coup, l'article de Rebecca sur la restauration permet de voir comment les outils de théologie peuvent critiquer un concept nouveau de restauration, mais à l'inverse comment la \textit{restauration} interroge les concepts théologiques classiques.

\section{Transformer la théologie}

 \subsection{Une anthropologie des relations}
\mn{Elena Lasida Economiste ICP \textit{parler de la création après Laudato Si}}


\paragraph{comment l'économie / écologie interroge la théologie} le travail à faire dans nos disciplines ne peut se faire que de façon interdisciplinaire. 

\paragraph{Notion de relation} tellement centrale qu'elle est devenue banale. Clé de voûte de \textit{laudato Si}. De même Ecologie, science de la relation entre les êtres vivants. Economie : \textit{médiation sociale} plutot que production. Et pourtant la dimension relationnelle est devenue une banalité, pas de question. Tellement certaine et claire. Grande frustration. 
\paragraph{Instrumentalisation de la relation} perd le caractère existentiel. Il faut desinstrumentaliser la relation. Comment rendre à la relation son caractère existentielle ? Ce n'est pas ce qui aide à vivre mais la \textbf{relation est la vie elle-même}. 

\paragraph{Conversion écologique de Laudato Si} Première ressource : 
\begin{quote}
    220. Cette conversion suppose diverses attitudes qui se conjuguent pour promouvoir une protection généreuse et pleine de tendresse. En premier lieu, elle implique gratitude et gratuité, c’est-à-dire une reconnaissance du monde comme don reçu de l’amour du Père, ce qui a pour conséquence des attitudes gratuites de renoncement et des attitudes généreuses même si personne ne les voit ou ne les reconnaît : « Que ta main gauche ignore ce que fait ta main droite [...] et ton Père qui voit dans le secret, te le rendra » (Mt 6, 3-4). Cette conversion implique aussi la conscience amoureuse de ne pas être déconnecté des autres créatures, de former avec les autres êtres de l’univers une belle communion universelle. Pour le croyant, le monde ne se contemple pas de l’extérieur mais de l’intérieur, en reconnaissant les liens par lesquels le Père nous a unis à tous les êtres. En outre, en faisant croître les capacités spécifiques que Dieu lui a données, la conversion écologique conduit le croyant à développer sa créativité et son enthousiasme, pour affronter les drames du monde en s’offrant à Dieu « comme un sacrifice vivant, saint et agréable » (Bm 12, 1). Il ne comprend pas sa supériorité comme motif de gloire personnelle ou de domination irresponsable, mais comme une capacité différente, lui imposant à son tour une grave responsabilité qui naît de sa foi.
\end{quote}
Elle implique communion, gratitude et gratuité, créativité (pas un mode pre établi). On peut les traduire par : 
\begin{itemize}
    \item donner
    \item se donner
    \item tout est fragile (pas une invitation à réparer mais à faire du nouveau)
\end{itemize}

\subparagraph{chgt de paradigme}
\begin{itemize}
    \item communion \textit{vient interroger} : La communion invite à penser l'autonomie comme interdependance. Ce qui rend autonome, ce qui me relie à autrui et non ce qui me sépare.
    \item la gratuité : vient interroger la prosperité. Plutot que de posséder, appartenir à une \textit{maison commune} qui nous singularise et nous universalise. Une reciprocité autre qu'instrumentale. Alliance plutot contrat.
    \item creativité : vient interroger \textit{la sécurité}, obsessionnelle. Accueillir l'inattendu. La fragilité : non pas reproduction mais émergence d'une nouveauté. 
\end{itemize}

\textit{{Instrumentalisation de la relation au frère au service de la relation à Dieu ?} Première question. Comment articuler pour la théologie transcendance divine et transcendance humaine ?
}
\paragraph{Economie} On peut retrouver une approche instrumentale de la relation, en particulier dans la \textit{valeur travail} et \textit{valeur utilité}. Or dans les deux cas, la valeur réside dans le bien lui même soit dans son coût de production ou d'utilité : \textit{approche substantive} pour les deux approches. André Orlean\sn{\href{https://journals.openedition.org/ress/2307}{L'empire de la valeur} Le problème principal de la science économique d’inspiration néoclassique est qu’elle raisonne en aval du processus de construction de la valeur : elle « suppose la question de la valeur résolue avant même que ne débutent les interactions » (p. 371). Or, selon lui, c’est précisément « l’étude des forces sociales » produisant les valeurs qui devrait constituer le cœur de l’analyse. Pour y parvenir, il est impératif de s’intéresser à la monnaie et, de façon plus générale, aux institutions et aux conventions sociales. Orléan plaide ainsi pour la construction d’un « cadre unidisciplinaire pour penser la valeur » (p. 210) qui emprunterait à l’ensemble des disciplines de sciences humaines et sociales. } propose une \textit{valeur relationnelle}. 

\paragraph{mésologie}

\begin{Def}[mésologie]
    Science du milieu (Augustin Berque)
\end{Def}
\begin{quote}
    L'être se crée en créant son milieu \sn{\href{https://usbeketrica.com/fr/article/penser-le-vivant-avec-augustin-berque-l-etre-se-cree-en-creant-son-milieu}{Berque}}
\end{quote}
Lien intrinsèque entre l'être et le lieu. Pensée japonaise : "langue de l'être là" vs français "universaliste", du je indépendamment du lieu où le \textit{je } se trouve. Logique logocique fondée sur l'identité avec le tiers exclu. A ou non A mais on ne peut pas être A et non A. 
\begin{Ex}[neige]
    la neige existe avec différents prédicats : ressource (hôtelier), risque (automobiliste).
    La réalité n'est ni seulement objective ni seulement subjective.
\end{Ex}

\paragraph{penser la vie bonne de façon différente}

\subsection{Écologie intégrale, comment la crise écologique conduit à des transformations de la pratique de la théologie}

\mn{Fabien Revol, président de la société théologique écologique de Lyon 
\textit{penser l'écologie dans la tradition Catholique.} Cours croire et comprendre Laudato Si}

\paragraph{Réponse 1 : un nouveau chapitre de la doctrine sociale de l'Eglise} sujet clos. Mais \textit{laudato Si} change l'approche.

\paragraph{Reponse 2 : créant une nouvelle discipline ecothéologie} Elle comprend d'emblée que toute la théologie est impactée.

1962 : thèse de théologie protestante en France. 

Des réponses à Llyn WHite sur la responsabilité chrétienne pour la crise écologique.

Le théologien se découvre une responsabilité éthique. Reexplorer le patrimoine théologique pour y trouver les traces de la protection de la maison commune (ex : Franciscain).

\paragraph{Ecothéologie}
\begin{Def}[Eco-Théologie]
 2008 l'écothéologie, différentes facettes de la théologie ... concernant l'environnement. Théologie contextuelle mais globale (concerne tout le monde)
\end{Def}
Et donc on pourra faire une Eco Theologie \textit{en contexte} (africain,....). \mn{question sur les autres religions}. On va passer sur le cosmique, dimension cosmique de la Foi chrétienne pour être audible.

\paragraph{Resistance catholique} Jean Bastaire avait néanmoins bien vu les enjeux.

\paragraph{contexte} série de colloques sur l'écologie et religion à la fin des années 90 à Harvard avec 600 théologiens. \mn{les actes en 10 volumes}
\begin{quote}
    pas uniquement la théologie mais le phénomène religieux (rite,...) est transformé.
    Les religions... réexaminées. Les religions donnent des récits, une vision du monde sur la nature. 
    Les religions... vision du monde et une éthique qui structure notre façon d'agir.
\end{quote}

\paragraph{revenir à White} 
\begin{quote}
    Ecologie humaine... structurée par 
\end{quote}
Clé épistémologique en 4 volets : 
\begin{itemize}
    \item des mentalités
    \item civilisation ; des images de la nature
    \item ces images conditionnent l'action des personnes
    \item historiquement ces images dérivent des discours religieux
\end{itemize}
Fonction critique de la théologie par ses représentations et images de la nature. 

Ainsi pour Tucker et Grimm\sn{Préface aux actes théologie et religions}, proposer des imaginaires et des récits permettant la survie humaine, des imaginaires désirables. 
Retour au cosmos. 
Dans le volume \textit{christanity and ecology} \sn{1998 Harvard}, \textit{respecter la terre comprise comme creation de Dieu...}

\paragraph{Critères}

\begin{itemize}
    \item 
    \item prise en compte des cosmologies des terriens. Passer de l'anthropolgie à la cosmologie chrétienne
    \item justice sociale (20 ans avant Laudato Si). LS49 
    \item reconsidérer les disciplines dans leur interdependance. Unité de crise dont toutes les autres sont des manifestations contextualisées. 
\end{itemize}

\paragraph{LS chapitre 3} Role de la théologie pour comprendre ce qui ne va pas dans les représentations de la nature (avec les accointances entre christianisme et modernité). 
\paragraph{risque} considérer que l'écologie est le critère de la théologie. La conscience écologique instrumentalisant la \textit{théologie}.

Paul Knitter : oecuménisme profond : trouver un terreau commun sur une terre commune. On voit ici une forme de dérive, où l'écologie devient un critère externe de validité théologique.

\paragraph{un petit pari à faire} ce n'est pas l'écologie qui doit être le critère mais comment penser un critère interne à la Création : finalité éthique interne d'être \textit{gardien de la création}. Une thèse : tout énoncé théologique doit être conforme à être \textit{gardien de la maison commune}. cf LS 2 : 
\begin{quote}
    2. Cette sœur crie en raison des dégâts que nous lui causons par l’utilisation irresponsable et par l’abus des biens que Dieu a déposés en elle. Nous avons grandi en pensant que nous étions ses propriétaires et ses dominateurs, autorisés à l’exploiter. La violence qu’il y a dans le cœur humain blessé par le péché se manifeste aussi à travers les symptômes de maladie que nous observons dans le sol, dans l’eau, dans l’air et dans les êtres vivants. C’est pourquoi, parmi les pauvres les plus abandonnés et maltraités, se trouve notre terre opprimée et dévastée, qui « gémit en travail d’enfantement » (Bm 8, 22). Nous oublions que nous-mêmes, nous sommes poussière (cf. Gn 2, 7). Notre propre corps est constitué d’éléments de la planète, son air nous donne le souffle et son eau nous vivifie comme elle nous restaure.
\end{quote}

Dans EG Evangelii Gaudium 21 
\begin{quote}
    fidèle à une formulation mais pas à la substance.
\end{quote}
C'est ce que fait le pape François en LS 3 : 
\begin{quote}
    3. Il y a plus de cinquante ans, quand le monde vacillait au bord d’une crise nucléaire, le Pape saint Jean XXIII a écrit une Encyclique dans laquelle il ne se contentait pas de rejeter une guerre, mais a voulu transmettre une proposition de paix. Il a adressé son message Pacem in terris « aux fidèles de l’univers » tout entier, mais il ajoutait « ainsi qu’à tous les hommes de bonne volonté ». À présent, face à la détérioration globale de l’environnement, je voudrais m’adresser à chaque personne qui habite cette planète. Dans mon Exhortation Evangelii gaudium, j’ai écrit aux membres de l’Église en vue d'engager un processus de réforme missionnaire encore en cours. Dans la présente Encyclique, je me propose spécialement d’entrer en dialogue avec tous au sujet de notre maison commune.
\end{quote}

Dans le chapitre \textit{la bonne nouvelle de la révélation}, ?


\paragraph{Théologie en dialogue écologie} Ecologie, pas éthique mais scientifique. Du coup, le théologien doit s'inclure dans un travail interdisciplinaire, avec l'écologie comme \textit{science} (+ philosophie). 
Articuler les trois domaines pour permettre le respect des discours et ne pas sombrer dans une forme de concordisme. Jean Ladrillère sur la question du sens.

\paragraph{Doctrine sociale de l'Eglise ?} L'écologie intégrale, interdisciplaire exemplaire, paradigme de l'être relié. Clé de lecture sur lequel doit s'articuler toute la théologie et la doctrine sociale de l'Eglise. 

\paragraph{Pour conclure : peut on mieux habiter la maison commune} Si en revanche la théologie détruit la maison commune, elle n'est pas chrétienne.  Ecclésiale  et génésique (gardien de la Création).

Timothy Howles : 
\begin{quote}
    Société : transformation sociétale authentique... 
\end{quote}

Faire de l'Espérance, comme le moteur.






\section{Soirée à l’ICP - Renouveler la théologie}

 
\subsection{“Our Theological Traditions Revisited to Face the Ecological Challenge”}
\mn{John Behr}

Cosmologie : oeuvre de Saint Irénée.situer la question de la mort et de la disparition de la figure du monde. Que cherchons nous à faire en théologie ? Non pas une éco-théologie. 

\mn{traduire sauvegarde par soin dans LS, plus proche de l'italien}

\paragraph{Contre les hérésies} 
\begin{quote}
  la figure du monde entier doit passer, le temps de son passage une fois venu, pour que le froment soit rassemblé dans le grenier, et la paille abandonnée et jetée au feu (CH 4.4.3)  
\end{quote}

\paragraph{Croissance de l'être humain}

Ce qui est crée ne peut pas être incréé. 

\paragraph{Le don de la mort}

\subsection{Discussion avec Elena Lasida, Fabien Revol et John Behr}

\mn{20h45-21h30	La conversion écologique de la théologie}

\paragraph{figure du monde liée au monde} Grégoire de Nysse : ils prenaient au sérieux la fin de l'épitre au corinthiens. 
\begin{quote}
     et nous faisons figure de faux témoins de Dieu, pour avoir affirmé, en témoignant au sujet de Dieu, qu’il a ressuscité le Christ, alors qu’il ne l’a pas ressuscité si vraiment les morts ne ressuscitent pas. 1 Co 15
     Car si les morts ne ressuscitent pas, le Christ non plus n’est pas ressuscité.

17 Et si le Christ n’est pas ressuscité, votre foi est sans valeur, vous êtes encore sous l’emprise de vos péchés ;
 \ldots

26 Et le dernier ennemi qui sera anéanti, c’est la mort,
\ldots

36 – Réfléchis donc ! Ce que tu sèmes ne peut reprendre vie sans mourir d’abord ;

37 et ce que tu sèmes, ce n’est pas le corps de la plante qui va pousser, mais c’est une simple graine : du blé, par exemple, ou autre chose.
\ldots
 

50\textbf{ Je le déclare, frères} : la chair et le sang sont incapables de recevoir en héritage le royaume de Dieu, et ce qui est périssable ne reçoit pas en héritage ce qui est impérissable.

51 C’est un mystère que je vous annonce : nous ne mourrons pas tous, mais tous nous serons \textbf{transformés},

52 et cela en un instant, en un clin d’œil, quand, à la fin, la trompette retentira. Car elle retentira, et les morts ressusciteront, impérissables, et nous, nous serons transformés.

53 Il faut en effet que cet être périssable que nous sommes revête ce qui est impérissable ; il faut que cet être mortel revête l’immortalité.

54 Et quand cet être périssable aura revêtu ce qui est impérissable, quand cet être mortel aura revêtu l’immortalité, alors se réalisera la parole de l’Écriture : La mort a été engloutie dans la victoire.
     
\end{quote}
Nous entendons mort et ressurection
mais 1 Co 15, il change de propos : "behold, voici un grand mystère". c'est la transformation  : nous serons tous changé (la mort et la resurrection est une sous section de cette transformation).
Mais la question est comment nous comprenons cette transformation, à travers image et analogie. Paul va utiliser l'image de la grain qui doit mourir pour devenir une plante : continu mais totalement différent.
Comme la chenille, chrysalide et le papillon. 
La figure du monde va passer mais le monde va rester.

\paragraph{Emphase sur la mort et la resurrection} Notre monde ne connait plus la mort. Or, Jésus vient nous annoncer la vie \textit{car nous sommes morts}. Dans la conversion écologique, il ne s'agit pas de garder \textit{la vie en abondance} (vue comme une accumulation) mais de recevoir \textit{la vie}. 

\paragraph{la question est : qu'est ce que vous pensez faire quand vous faites cela} cela : moins prendre l'avion. Vie chrétienne : \textit{être activement passif}, \textit{être activement en attente. } La vivification par l'Esprit qui va fonder notre action en écologie (et toute action éthique), cette vivification se fonde dans le rapport à la mort à la vie. 

\paragraph{Il reste une ambiguité}
Crise écologique comme \textit{refus de la mort}. Hypothèse chrétienne que la vie passe à travers la mort.
Le cancer, c'est fondamentalement une cellule qui refuse de mourir. \textit{nous sommes devenus une société de cancer.}


\paragraph{Irénée} pas un grand succès à l'époque, en particulier son livre V. Et les pères du désert. Ils continuent la voix des martyrs du IIe siècle jusqu'au desert. Mais en fait vision de l'extérieur, de l'intérieur, ils décrivent la tention. 

\begin{Def}[être humain]
L'être humain est de la terre qui souffre.lettre à Barnabé
\end{Def}
Les mains de Dieu travaillent en permanence autour de Dieu, toutes nos joies et peines sont parce que nous sommes modelés par Dieu. L'argile, c'est ce que nous serons à la fin. Or, dans la Gn, Dieu prend de l'argile. C'est la mort qui vient avant la vie. Quelle sorte d'argile allons nous devenir ? Nous pouvons venir comme une argile malléable ou dure, dans les mains de Dieu. 
La question n'est pas mort paisible ou dure, mais notre attachement à la vie : "entre tes mains, je remets mon esprit" ou bien "ce n'est pas juste". Quoiqu'il en soit nous mourrons.

\paragraph{Hans Jonas} faire que les conditions de la planète soit possible pour l'humanité. Respecter la personne humaine, c'est de protéger la planète car c'est bien beau les questions de la personne humaine, mais si on ne peut plus vivre, la question est non pertinente.



\subsection{Ateliers : où atterir}

\subsection{Des pratiques qui renouvellent la pensée (partie 1)}
\mn{Président de séance : François Euvé}


\subsection{Quelle rationalité pour quelle conversion écologique ?}
\mn{Les six portes du Manuel de la Grande Transition
Cécile Renouard}

\paragraph{Antonio Guterres} Allons nous coopéré ou suicide collectif ? pessimisme de la pensée et optimisme de l'action.

\paragraph{Campus de la transition} On est hors système. "Ne vous occupez pas de comment cela sera traduit en enseignement ?" lien entre les savoirs et les savoirs être. Ecocampus : accueille des enseignants chercheurs

\paragraph{retour à la finitude} \textit{pas de planete B}.  Analogie avec la bombe atomique et le changement climatique. Mais en fait, le changement climatique nous concerne tous. Largement complice. 

\paragraph{Justice sociale} Combat spirituel pratique et intellectuelle.

\paragraph{Sobriété}  : comment avoir une empreinte carbone. Les 10\% américain : 70 tonnes, alors qu'il faut 2 tonnes. En france, 30 tonnes et 10 tonnes en moyenne.

\paragraph{Recherche de cohérence entre ce qu'on dit et ce qu'on fait}

\paragraph{personnel et structurel} on pense svt des critères d'un point de vue personnel mais moins collectif pour que cela débouche sur des transformations institutionnelles, nous en sommes très loin. \textit{péché structurel}

\paragraph{équilibre instable} expression de Simone Weil. Comment cultiver une dimension d'espérance 
Pour les chrétiens, la conversion chrétienne est une conversion écologique mais pour beaucoup d'hommes dans le monde, la conversion écologique n'est pas une conversion chrétienne. Comment le penser ? 

Foi élémentaire de Théobald

\paragraph{transformations} on chauffe moins, moins carné. Mobilité douce. Jusqu'où on peut aller ? 

\paragraph{dimension spirituelle} méditant militant.

\paragraph{quelles institutions} Seine et Marne, gilet jaune

\paragraph{les six portes du manuel}
\begin{itemize}
    \item \textit{Oikos} : maison commune, diagnostic, science du climat, de l'ingénieur, vision des communs
    \item \textit{ethos} : questionnement éthique basé sur Ricoeur de la visée bonne.
    \item \textit{nomos} : quel modèle pilote nos décisions publiques ? métriques ? gouvernance
    \item \textit{logos} : kmer vert. Nous devons travailler sur les mots. ex : guardian, Urgence climatique. regarder monde désirable et monde probable.
    \item \textit{praxis} : question de l'action et différents types d'action pour quels acteurs
    \item \textit{violence} : non violence active suffisante. désobéissance civile. 
    \item \textit{dunamis} : écopsychologie et écospiritualité. réfléchir à partir des émotions qui nous traversent.  \textit{tout est foutu} :  
\end{itemize}

épistémologique : quelles sont les rationalités à l'oeuvre de ces 6 portes

\paragraph{rationalité techno scientifique} traditionnellement positive sur la traversée de la crise écologique

\paragraph{rationalité symbolique} symbole, récits imaginaires qui façonnent nos sociétés, pas neutres.

\paragraph{Exercice} rapport à la mort et la finitude. Simon Weil (1941) : notion de valeur, détachement de se situer sporitel et pratique :
\begin{quote}
    le détachement est un renoncement à toutes les fins possibles.
    \ldots
    hierarchie vraie entre les valeurs, toutes les valeurs. 
    
\end{quote}
 cet acte philosophique doit passer par un détachement, une mort. L'idée d'ordre de S. Weil\sn{enracinement}. 
 pas un appel à l'inaction. Meilleure vie pour aujourd'hui et pour demain pour d'autres.

 \begin{quote}
    la philo ne consiste pas à une accumulation de connaissance mais un changement de toute l'âme.  
 \end{quote}
\subsection{Les Ateliers Où Atterrir ?}
\mn{Anne-Sophie Breitwiller, ICP, féminisme, éco finisme et Verónica Calvo Valenzuela, science po}

 \paragraph{Atelier atterrir} Bernardins. de Bruno Latour

 \paragraph{Gaia} Latour avait pris de plein fouet le retour de Gaia. Une terre qui s'emeut (Michel Serre, 1990) qui réagit devant nos actions, disproportionnées.

 \paragraph{livre où atterrir ?} livre de combat de Latour : livre d'activiste. Bruno Latour a pris au sérieux son texte. Il est devenu cet activiste. Il a fondé le consortium \textit{ou atterrir}. Collège des Bernardins. 
 Il croit sur la territorialité de l'Eglise. 

 \paragraph{Atelier atterrir} parcours en trois étapes : observer (exercice d'attention). S'orienter (discernement). Oeuvrer. 
 Ressemble à voir / discerner / agir de l'action catholique. 
 Parcours d\textit{exercices}.
 \paragraph{Exercices } filiation des exercices spirituels. Mais il s'adresse à des individus dans des structures.
 La capacité de l'exercice de mettre à l'ouvrage. 
 Quels sont nos attachements au territoire qui nous permet de vivre ? 

 \paragraph{Territoire} formation dans les diocèses, sur plusieurs mois.



\begin{Ex}[récit de la relation à un non-humain]
    5ème exercice. Fin du cycle voir.
    disposer en rond. feuille ou carnet.
    l'animateur écrit
    \begin{quote}
        à partir de mon carnet ethnographique, je décris un non-humain rencontré hier et je décris ma relation en un page : jardin, virus, bactérie, animal,... 
        Quel est son terrain de vie ? Est ce qu'il dépend de moi ? et inversement ?
        15 mn. 
        Puis chacun présente son récit, sans commentaire.  Sorte de disposition à l'écoute.

 choix varié de non humain.

 Puis \textit{qu'est ce que je ressens ? Qu'est ce qui a eu du gout ? de la joie ? ou pas du tout ? Ce que j'ai éprouvé ? }
    \end{quote}

Au début, le peu d'indication de l'animateur amène un certain inconfort. Et l'utilisation du terme \textit{relation} peut mettre mal à l'aise car pas tjs le terme utilisé naturellement.

Injonction d'un récit : décrire ses propres conditions d'existence. Les avis sont suspendus laissant une place à ce qui peut advenir. 
Proche d'un \textit{dialogue contemplatif}.

Caractère bienfaisant et thérapeutique. 

Un prochain ici est un autre qui est non-humain. 

Espace individuel et collectif 

3 aspects de l'expérience spirituelle : 
\begin{itemize}
\item action
\item mise en relation
\item thérapeutique
\end{itemize}

\paragraph{Choix entre François et Bruno Latour} Temps supérieur à l'espace (François) : initier des processus. La formule du pape cohérence avec la démarche. 
bruno Latour insiste sur l'espace (eschaton n'est pas que dans le temps mais aussi l'espace). 


\end{Ex}

\begin{Def}[exercice]
processus  d'action
    
\end{Def}
Et garder le terme rituel (sacré). Réflechir quand le processus (pragmatique) devient sacré ?

\paragraph{déterritorialisation de l'Eglise} La Paroisse n'est plus pertinente pour cette action. Plutot le diocèse. 

\subsection{Entre religio et religare, l'engagement politique écologiste}
\mn{Fabrice Flipo}


\paragraph{repenser l'émancipation dans notre période} socialisme,... 
\href{https://fr.wikipedia.org/wiki/Alfred_North_Whitehead}{Alfred Whitehead}. Les \textbf{cosmologies} concernent non slt les peuples non modernes mais tous les peuples.

Kant : émancipation, rechercher l'harmonie : on ne peut l'atteindre.

Principe Espérance chez Ernst Bloch : tout etre humain aspire à l'état d'harmonie s'il est sincère. \mn{division chez Duquoc un peu fort}

\begin{Def}[Cosmologie]
système d' idée qui rend compte de l'expérience, cohérence \textit{betweeness}
\end{Def}

\begin{itemize}
    \item tous les savoirs, y compris \textit{soft} (agriculteurs)
    \item le système, c'est la fin de l'enquête. 
    \item nature, de nature processielle, donc la vérité change. Qu'est ce qu'on apprend : \textit{qu'est ce que j'ai appris en plus ? }
    
\end{itemize}

\paragraph{3 dimensions de la vérité} \begin{itemize}
    \item vérité comme adéquation. \textit{les faits}. description qui coincide avec l'expérience. A noter le symbolique est un fait. 
    \item la vérité comme ce qui est important. \textit{normatif}. Bien peser nos convictions. 
    \item la vérité comme cohérence logique. Jeu de règle (ni bien ni mal), ni ne décrit les faits. \textit{la grammaire}.
\end{itemize}
On retrouve la tripartition de Hegel. 

\paragraph{l'harmonie} lier les trois discours de vérité. 

\paragraph{cosmologie chez Whitehead} la science  : savoirs spécialisés, des régions de savoir qui essayent de s'ajuster. Image de l'archipel. 

\paragraph{On pense par délégation} si je sais que la table est solide, c'est qu'on me l'a dit.
Mais il peut y avoir des savoirs imparfaits (?) : ex : la croissance est infinie.
Le dogmatisme, c'est d'attacher une importance essentielle à un jeu de règles qui fait que jamais on ne changera. 

\paragraph{les faits qui sont éloignés } dans le temps et l'espace. \mn{ex : L'urbain est en contact avec le béton et lui cache un certain nombre de dépendances.} Un certain nombre de croyance et de discours mal placés.

\paragraph{Importance} ce qui est important, \textit{belief}, à traduire par conviction et non croyance : une conviction, c'est le résultat d'un travail. \mn{ex : covid, on créé un conseil scientifique car on pense que les gens ont des croyance alors qu'ils ont des convictions}

\paragraph{Effervessence}  Quand les règles, on n'y croit plus,  effervessence, on a une grande créativité (ex : Covid) , avec Taiwan et Corée du Sud très créatif.

\paragraph{Story telling} ex : Tesla. Stéphane Courtois\sn{le livre noir du xxx} parle de dichotomie logique. Ex sur l'écologie : Auto mal et vélo bien. Mais sortir du déchotomie, triporteur peut être plus adapté pour le rural.








\subsection{La vie monastique, ressource pour une existence écologique Dialogue}

 
\mn{Président de séance : Dominique Iogna-Prat(animé par Patrick C. Goujon)
entre Fr. Jean-Michel Grimaud (Landévennec), Sr. Dominique Racinet (Monastère de Taulignan - Drôme),
Danièle Hervieu-Léger et Bertrand Hervieu, questions sur les neo-ruraux \textit{le temps des moines}, \textit{vers l'implosion}. Bertrand, sociologue du monde agricole, ancien directeur INRA, \textit{une agriculture sans agriculteur}. }

\paragraph{lien avec la matinée D. Hervieu léger} on est pris entre deux postures. Insister sur l'urgence d'initier des processus pour changer de façon radicale notre vision du monde. Et de l'autre, de changer par le bas, en ne chauffant pas.
Et entre ces deux urgences, qu'est ce qu'on peut imaginer comme modèle d'être ensemble en société ? 
Campus de la transition : besoin d'ancrage (repli) et la nécessité de s'ouvrir au monde tel qu'il est. Cette dialectique, dedans / dehors, m'a fait pensé à la dialectique cloture / hospitalité des monastères.
Faire sauter l'évidence du caractère écologique du monastère.


\paragraph{Certains monastères et question agricole au XX} l'abbaye de Maredsou en Belgique. \mn{Bertrand Hervieu}
Après la guerre, Dominique de Bruwne, il envisage une usine pour mettre le monastères dans le temps du monde tel qu'il se construit. La pierre qui vire acquiert une ferme en 1938. La JAC (Jeunesse Agricole Catholique) pousse la Pierre qui Vire à moderniser la ferme : selection génétique de l'élevage. Développement. Les 2/3 des syndicats agricoles viennent du JAC : la terre n'est pas un patrimoine mais un outil de travail. Jusqu'au jour en 1980 : montagne de beurre. La Pierre qui Vire décide alors de passer au bio. Une vraie remise en cause de la communauté. Et troisième étape, cette expérience bio va épuiser la communauté et mise en bail.

\mn{Daniele Hervieu Léger}

\paragraph{communauté néo rurale qui ont fleuri dans les années 70-80 et les moines} retour à la nature et un autre livre sur l'apocalypse. Toute à la fin de l'enquete, ils ont exprimé leur intérêt pour le monachisme. Une ressource pour ces mouvements ? fascinés par l'allégeance d'humains à un lieu, frugalité, par la maitrise du temps.
\begin{itemize}
    \item Comment faites vous pour vivre ensemble ? parlez nous de votre règle ?
    \item Demande d'une mémoire utopique. 
\end{itemize}
La question aux moines : \textit{quel type d'écoles pouvez vous offrir à la conversion écologique aujourd'hui}

\mn{Jean Michel Grimaud}

\paragraph{Tel Arbed, commencement du monde plutôt que finistère} 860, abbé de Landevenec. Lieu ... qui nous attendait depuis toujours. 
 
\begin{itemize}
    \item Ecoute (St Benoit), sens de l'intériorité, attention à autrui. \textit{discretio}, sens de la mesure. Dans l'esprit de Benoit, être mesuré, ce ne pas être tiede mais ajusté. Culture de la sobriété, qui donne du gout aux choses sobres. Fruit de joie et de paix.
    \item stabilité dans la communauté. Le prophès promet stabilité, dans la commaunauté des frères et soeurs que je n'ai pas choisi mais Dieu.
    Stabilité d'un lieu : lien de communauté avec les personnes vivants autour. Rejeter la fuite devant les difficultés.
    \item Ch 31 (Cellerier) : prendre soin des malades et des pauvres mais aussi des vases sacrés... 
    \item ne rien préférer qu'à l'oeuvre de Dieu : quand la cloche sonne, on se dirige vers la messe.

\mn{Sr dominique, en dessous de Valence, Drôme}

\paragraph{contexte de la restauration de Taulignan} \href{https://www.dominicaines-taulignan.fr/demarche-ecologique}{Taulignan} dans les années 1956, arrivée dans la Drome.  6 hectares mis en fermage. 

\paragraph{2007 : le paysan résilie son bail} les soeurs voulaient quelque chose de plus simple, où chacun puisse travailler.  Herbes médicinales
\paragraph{changer son regard} voir que dans la forêt, il y a du Romarin, du thym....
\paragraph{Pouvoir en rendre compte} Fabien Revol. permet de voir pourquoi cela unifie. 
\end{itemize}

\paragraph{Discussions - hospitalité} les hôtes sont aussi des forces d'interpellation. une entrée en dialogue.

\paragraph{Un faible écho de cette démarche écologique}\mn{DHL} Un style chrétien\sn{théobald} qui devrait être plus repris par l'Eglise. L'Eglise est obsedée par son magistère morale, discours de la nouvelle présence et repousse à la marge les inventions d'un style chrétien aujourd'hui. 

\paragraph{Importance du lieu} un lieu monastique puisse faire écho à la beauté du lieu. préserver l'espace de la Création, qui témoigne de celui qui dépasse.

\paragraph{Providence} confiance dans le seigneur. 

\subsection{Cri de la terre et clameur des pauvres, quels chantier théologique et quelle pratique ecclesiale ?}
\mn{Alain Thomasset, ingénieur agronome, théologie morale}

\paragraph{Une transformation des imaginaires par l'histoire du Salut} 
\href{https://www.loyola.edu/academics/theology/faculty/castillo}{Daniel Castillo} expose comment une éco théologie  de la libération : rapport à Dieu, aux hommes à la terre (et de tout ce qui provient de la terre - homme animaux). Contrairement à Léonard Boff, il s'appuie sur le livre de la parole de Dieu. Il s'appuie sur Gustavo Gutierrez, une libération intégrale : socioéconomie, économique. Des structures, des valeurs et des imaginations et un combat entre le péché et la grâce.
Ici uniquement lecture des imaginaires.
LS 70
\begin{quote}
    70. Dans le récit concernant Caïn et Abel, nous voyons que la jalousie a conduit Caïn à commettre l’injustice extrême contre son frère. Ce qui a provoqué à son tour une rupture de la relation entre Caïn et Dieu, et entre Caïn et la terre dont il a été exilé. Ce passage est résumé dans la conversation dramatique entre Dieu et Caïn. Dieu demande : « Où est ton frère Abel ? ». Caïn répond qu’il ne sait pas et Dieu insiste : « Qu’as-tu fait ? Écoute le sang de ton frère crier vers moi du sol ! Maintenant, sois maudit et chassé du sol fertile » (Gn 4, 9-11). La négligence dans la charge de cultiver et de garder une relation adéquate avec le voisin, envers lequel j’ai le devoir d’attention et de protection, détruit ma relation intérieure avec moi-même, avec les autres, avec Dieu et avec la terre. Quand toutes ces relations sont négligées, quand la justice n’habite plus la terre, la Bible nous dit que toute la vie est en danger. C’est ce que nous enseigne le récit sur Noé, quand Dieu menace d’exterminer l’humanité en raison de son incapacité constante à vivre à la hauteur des exigences de justice et de paix : « La fin de toute chair est arrivée, je l’ai décidé, car la terre est pleine de violence à cause des hommes » (Gn 6, 13). Dans ces récits si anciens, emprunts de profond symbolisme, une conviction actuelle était déjà présente : tout est lié, et la protection authentique de notre propre vie comme de nos relations avec la nature est inséparable de la fraternité, de la justice ainsi que de la fidélité aux autres.
\end{quote}

\paragraph{L'écologie intégrale comme libération intégrale - Castillo}
Pour Castillo, il faut élargir la lecture de la Genèse. Gn 2, 15 (jardin gardien). clé, humain jardinier, prendre soin.  Dieu est lui même le jardinier. Joseph est l'anti-adam et préfiguration du Christ.  L'exode, c'est le chemin d'apprentissage où le peuple est amené à retrouver sa vocation. Triple vocation.  

\paragraph{Lecture politico-écologique de la parole de Dieu} ne pas accumuler la manne. Glaneur donc faible. confiance en Dieu, limitation de l'accaparation.
Sabbat. Repos de la terre et du travail aussi des servantes \textit{et des animaux}. 
Lv : année sabbatique et année jubilaire. Pdt l'année sabbatique, on retrouve sa position de glaneur. Jubilaire : Quadruple restauration (terre, esclave,...)
Jésus, Lc 4 : une année de bienfaits. Option préférentielle pour les pauvres.

\textit{Praxis de la libération}

Humilité, gratitude. On passe de l'homme consommateur à l'homme jardinier.
K. Barth : 
5 points du sabbat : 
repos, compassion, discernement, convocation, celebration.

\paragraph{Dimanche} LS

\begin{quote}
    237. Le dimanche, la participation à l’Eucharistie a une importance spéciale. Ce jour, comme le sabbat juif, est offert comme le jour de la purification des relations de l’être humain avec Dieu, avec lui-même, avec les autres et avec le monde. Le dimanche est le jour de la résurrection, le “premier jour” de la nouvelle création, dont les prémices sont l’humanité ressuscitée du Seigneur, gage de la transfiguration finale de toute la réalité créée. En outre, ce jour annonce « le repos éternel de l’homme en Dieu »[168]. De cette façon, la spiritualité chrétienne intègre la valeur du loisir et de la fête. L’être humain tend à réduire le repos contemplatif au domaine de l’improductif ou de l’inutile, en oubliant qu’ainsi il retire à l’œuvre qu’il réalise le plus important : son sens. Nous sommes appelés à inclure dans notre agir une dimension réceptive et gratuite, qui est différente d’une simple inactivité. Il s’agit d’une autre manière d’agir qui fait partie de notre essence. Ainsi, l’action humaine est préservée non seulement de l’activisme vide, mais aussi de la passion vorace et de l’isolement de la conscience qui amène à poursuivre uniquement le bénéfice personnel. La loi du repos hebdomadaire imposait de chômer le septième jour « afin que se reposent ton bœuf et ton âne et que reprennent souffle le fils de ta servante ainsi que l’étranger » (Ex 23, 12). En effet, le repos est un élargissement du regard qui permet de reconnaître à nouveau les droits des autres. Ainsi, le jour du repos, dont l’Eucharistie est le centre, répand sa lumière sur la semaine tout entière et il nous pousse à intérioriser la protection de la nature et des pauvres.
\end{quote}
Dimanche, pour le pape François, une écologie intégrale. 

\paragraph{Les trois dimension de l'historicité à relier entre elles}

\paragraph{les trois histoires selon Gaston Fessard}
\begin{itemize}
    \item histoire naturelle, cosmologique
    \item  humaine, langage, signifie la liberté (performative ?)
    \item et surnaturelle, en donne le sens. Pas uniquement chrétien mais toute histoire qui donne le sens et le but. Souvent les premiers documents écrits. Pas de déterminisme de l'histoire surnaturelle : elle éclaire les choix.
\end{itemize}
Trois conditions de l'historicité viennent de la dialectique des \textit{Exercices Spirituels}. Une piste qui mériterait une recherche dans l'analyse de l'histoire. 

\textit{une dimension transcendentale } le Sabbat, c'est l'émergence du surnaturel. 

\paragraph{desyncrhonisation des temps}
Accéleration sans fin. 

\paragraph{Développer de nouvelles attitudes interieures}
Addiction par rapport aux structures du péché consummeristes

\paragraph{l'importance des vertus} les lois ne suffisent pas à changer les comportements. Vertus s'apprennent et se perfectionnent par des exercices adaptés.

\paragraph{la pratique du sabbat et de l'eucharistie} il faut des vertus sociales, des structures de grâce par rapport aux structures consummeristes. L'eucharistie, le jour du repos, répand sa lumière sur la semaine. 

\paragraph{La vertu d'espérance à l'école des pauvres} Importance. Face au défi de la conversion écologique, espérance. Les contours nous échappent. Naissance, perseverance dans l'épreuve, une attente. Ricoeur : des figuratifs de notre libération. En ce temps d'angoisse, espérance : voir l'essentiel. 

\begin{quote}
ce n'est pas vrai parce que je me force à croire... ils sont vrais dans la vie des autres. ... en particulier dans les chemins éprouvés
    Diakonia 2013 parole des pauvres
\end{quote}
L'espérance s'incarne à travers ceux qui ont traversé l'épreuve. 

K. Rahner : humain, être de désir; 

\paragraph{Conclusion}
transformer notre imaginaire, notre rapport au temps et notre attitude intérieur. Repos; temps de Dieu et temps des pauvres.


\subsection{Comment localiser le global}
\mn{Raphaël et Catherine Larrère}


\href{https://www.cairn.info/revue-projet-2015-5-page-95a.htm}{PENSER ET AGIR AVEC LA NATURE, Une enquête philosophique}
\subsection{Discussion}


\section{La conversion, une problématique chrétienne ?}

\mn{Président de séance : Jean-Louis Souletie}

\subsection{Conversion politique ? En quoi parler de « conversion écologique » est-il pertinent dans le champ politique ?
}
\mn{9h-9h30	
Fabrice Boissier, ENA, Mines, Ancien Directeur Général de l'ADEME, Rouen, Baccaleauréat Canonique}
ADEME : mobiliser tous les acteurs.
Mais Conversion écologique, une problématique chrétienne. Comment en parler dans une France sécularisée ?

\paragraph{Conversion politique en Jonas} Raisonner par analogie. Israel, identité politique car peuple. Livre de Jonas : un conte qui parle. interdiction de paitre et de boire de Dieu. 

\begin{itemize}
    \item "peut être Dieu... " : un espace pour l'agir de l'homme
    \item cela commence par les citoyens, puis décret
    \item une mise sans dessus dessous : sens négatif mais positif. A la fin de l'histoire, la société est bouleversée.
\end{itemize}

\paragraph{Conversion politique en Amos} corruption, une société pervertie, oracle de malheur mais toujours cette ouverture ("peut être le Dieu de l'univers...") Même terme entre Jonas et Amos. Description précise des maux de la société.

\paragraph{Conversion politique en Jérémie} Politique extérieure, faut il se tourner vers Egypte ou Babylone. Quelle valeur morale à l'un ou l'autre ? pas hyper clair d'un point de vue moral. \textit{Si tu rejoins l'état major de Babylone...}. Conversion politique, \textit{historiquement située}.

\paragraph{Pour faire jouer les analogies, s'assurer de la correspondance} Existence même de la société, ... Il faut justifier que les ressorts en jeu (Alliance avec Israel), sont toujours pertinents aujourd'hui dans une société sécularisée ?

\paragraph{Limites planétaires} 9 grands équilibres \sn{2009 Roestroem}. 3 limites sur 9 déjà dépassés. 4, en 2015, 5 en janvier 2022, 6 en juillet 2022. On parle en mois. 
Il reste une discussion politique sur la sobriété vs technologie. A l'ADEME, il y a un scénario qui existe par la sobriété mais pas de scénario via changement technologique. 

\paragraph{L'alliance avec Dieu : boussole du peuple Israel} On peut parler de conversion quand il s'agit de conformer l'agir à l'Alliance

\paragraph{Quel Nord dans notre société sécularisée ?} Quel Nord ? La doctrine sociale de l'Eglise nous propose un Nord, via le bien commun ? De même, quand on parlait de conversion d'Israel, de même aujourd'hui s'orienter vers le bien commun.
\paragraph{Pourquoi on ne suit pas le Nord}
\begin{itemize}
    \item Hypothèse 1  : Un dévoiement de l'agir politique ? Structure de péché du paradigme technocratique ? 
    \item Hypothèse 2 : Le bien commun n'est pas immuable mais dynamique (le Nord bouge), si l'agir politique ne suit pas le Nord parce qu'il bouge. Gaston Fessart. visée supérieure du Bien Commun. Visée originaire homme nature. 
\end{itemize}

\begin{Ex}[phytosanitaire à l'exportation]
\begin{itemize}
    \item  En 2017, interdiction d'exportation des produits sanitaires interdits en France .
     \item   En 2018, intense lobbying des producteurs (2000 emplois)
   \item     En 2019, loi pacte qui réautorise l'exportation mais cassé.
   \item     En juillet 2019, QPC, interdiction 
     \item   Conseil constitutionnel : Protection de l'environnement > liberté d'entreprendre (pourtant principe constitutionnel)
    
\end{itemize}
   on a ici une évolution de l'évaluation du bien commun.
\end{Ex}

\paragraph{Inventivité} C'est quand le Nord évolue, qu'il est plus facile de faire bouger les gens. Conversion politque plus complexe. 
\subsection{Conversion morale ?
La « conversion écologique » : un concept pas si évident en éthique}
\mn{9h30-10h	
Walter Lesch

SSH/TECO  --  Faculté de théologie et d'étude des religions (TECO)

SSH/RSCS  --  Institut de recherche Religions, spiritualités, cultures, sociétés

SSH/ISP  --  Institut supérieur de philosophie}

\paragraph{Encore cette conversion écologique ! première réaction}

\paragraph{Changer de mode de vie} impatience devant le changement lent. vers un monde moins injuste.

\paragraph{Conversion dans le champ éthique} Façon chrétienne de parler \textit{ad Intra}. mais audible \textit{ad extra}, sans présupposé confessionnel. 

MAIS
\paragraph{Pas un paradigme unique} comment l'Eglise peut elle intervenir sur l'éthique alors qu'elle fait \textit{humanae vitae}. Laudato Si, premier document sur l'écologie.

\paragraph{Contreverse sur le role du christianisme de Lynn White} Relecture des textes bibliques dans le sens d'une responsabilité \textit{particulière} de la nature. Litterature un peu pathétique avec une premiere partie sur la Bible ("il suffit de lire le texte bien"). Mais cela ne change rien à l'histoire de la lecture de ces textes. A coté du travail de la \textit{bonne lecture} du texte fondateur, conversion.

\paragraph{Greening of religion} pratiquement toutes les religions voient la question d'articuler religion et écologie.

Une anthropologie responsable est compatible avec l'anthropocène. Le succès de Laudato Si vient de sa capacité de synthèse qui facilite les liens entre différents courants de pensée.

\paragraph{Retard ecclésial} l'écologie politique partage bcp plus avec le féminisme, le tiers-mondisme. Lien avec les croyants ? La réception de l'écologie : développement durable, un des piliers
Cathos. Convertis tardivement à l'écologie. 

\paragraph{Herméneutique du soupçon} Ecologie Integrale : vise un public mondial, résonnance spécifique en France. \textit{intégrale} peut rassurer les sceptiques du fait de l'intégration spirituelle. Mais lutte idéologique par rapport à la place des écologies. Eviter le jargon qui complique inutilement.

\paragraph{Revue Limites}

Ecologie humaine, limites, 2015 : montre le problème des lignes de fracture au sein du catholicisme. Flirt avec droite et extreme droite. 
Voir Hervieux léger et la revue Limite.

\paragraph{Catégorie de \textit{salut}} a perdu son intérêt dans l'immanence du monde contemporain. Des jeunes très sensibles à l'urgence écologique mais 
Critique
Luc Ferry 1992 : controverse idéologique, \href{https://fr.wikipedia.org/wiki/Le_Nouvel_Ordre_%C3%A9cologique}{le nouvel ordre écologique}
Eric Fassin 2010, configuration de discours qui n'est pas sans problème.

\paragraph{Une analyse fine des argumentations} Problématisation. Un certain modèle d'éthique\sn{Ethique Autonome de Tubingen} doit s'ouvrir à l'universalité. L'Eglise catholique doit développer une \textbf{nouvelle éthique}, qui a perdu son pouvoir sociale et cherche à se repositionner sur la question des mauvaises consciences. Cela ne marche pas. 

\paragraph{Insupportable de donner des leçons de conversion écologique} à des gens qui ne sont pas bien.

\paragraph{quel message libre qui permette une conversion ? } Responsabilité énorme des Eglises de théologie de la Création. 

\paragraph{conclusion} invitation à sortir de sa bulle confessionnelle. Contribution originale qui serait de l'ordre de la subversion, du christianisme \sn{K. Rahner, \textit{la foi qui aime la Terre}, tout un programme des précurseurs de l'éco théologie. Théobald : "oui"}

\paragraph{Ecologie intégrale} en milieu catholique (Revol) cherche à sortir du dualisme entre écologie environnementale et écologie humaine. \mn{2013 Limite ?
\href{https://www.cairn.info/revue-la-pensee-ecologique-2019-1-page-86.htm}{Pensée écologique}}
Bémol : pourquoi aimer ce mot \textit{intégral}, comme humanité \textit{intégrale}. Est ce que cela donne une sécurité ? Tendance idéologique en France (cf Limites). 



\subsection{Laudato si’: un changement dans ce que signifie la conversion ?}
\mn{
10h-10h30	
Patrick C. Goujon
Histoire des discours spirituels et son lien avec la société au XVII }

Frappé par la convergence des discussions. 

Adresse oecuménique au Président Macron Octobre 2022, avec des 

\paragraph{Laudato Si} Conversion et Dialogue. "Conversion", du registre prophétique (pas trop dialogue) à une logique sapientielle. Permet ad extra (sapientielle) et ad intra (théologale).

\paragraph{Laudato SI} prend un \textit{kairos} et en ce sens, devient prophétique, car parle en situation au sein d'une crise. Mal qui perturbe le peuple et Dieu. \textit{epistrophe}, retour vers Dieu, en ayant horreur de son comportement (metanoia).

Responsabiltié personnelle et collective, mais en rappelant que c'est un don de Dieu (Jr 31, "je mettrai...").
Mais la parole prophétique est souvent dramatique, temps dramatique mais aussi le prophète souffre.
\paragraph{Une seule figure récapitulative : Jean Baptiste } La prophétie prend une autre couleur, eschatologie, couleur du matin de Pâques. 

\paragraph{début de Laudato Si} souligne l'urgence climatique, chgt radical du comportement. Mais dès le début, pas uniquement un soucis prophétique : 
LS 14 : dialogale. 

deux caractéristiques :
\begin{itemize}
    \item marqué par la joie (LS 10). Cet éthos de Saint François
    \item conversion : normalement retour à Dieu mais ici, promesse d'un avenir commun à recevoir de Dieu. 
\end{itemize}

\paragraph{conversion par la sagesse ?} Place d'autres visions. 
LS 47 : définition de la sagesse. ... pas que des données (Barrere). 
Importance de la rencontre effective, de sa joie.
La rencontre s'oppose au retour au Dieu seul.
Conversion vers une \textit{relativisation}. 

\begin{quote}
LS 60 : entre ces deux extrêmes, possibles scenarios futurs... \textit{parce qu'il n'y a pas une seule issue.}
    
\end{quote}

François ne se situe pas ici dans une logique théologique mais néanmoins fait de la théologie, relativisme.

\paragraph{LS discours spirituel} \textit{en passant}\sn{très classique de François}, il présente son discours théologique : 

\begin{quote}
    dans le dialogue dans les nouvelles situations historiques
\end{quote}

Karl Rahner 
Assimilation créatrice

\paragraph{crédibilité du discours} Unicité de ce monde. Car sinon pas audible. "une seule terre".

Bruno Latour 2002 : il regrette la situation linguistique vient obturer les champs de la langue, uniquement informative 
la parole ne joue pas que dans celui de l'information.

Passer à une parole \textit{transformative}. 

\paragraph{LS langage transformatif} Poesie, des hymnes, des prières.  Certes des vérités dogmatiques présentes mais aussi la \textit{manière de le dire}. Le dialogue, une modalité cohérente d'une théologie de la Création. Il devient possible de l'habiter ensemble par le dialogue.
Non pas comme un échange d'information mais un regard de sagesse.
THomas D'aquin, Sagesse : ce qui permet de saisir les choses \textit{dans leur relation}. 

Paul Beauchamp : la sagesse, poumon d'israel pour respirer l'air commun.

\paragraph{Explicitation du seuil de la foi} non pas proposé aux non croyants mais \textit{aux croyants}, une spiritualité écologique. François indique aux chrétiens comment leur conversion écologique (LS 217) les lie à Dieu.
François : figure de la sagesse, le \textit{soin des plus pauvres et de la Création}. Mais qui est Saint : renvoie aussi au Christ, secret qui habite la Création et reconnu par les Chrétiens.

\paragraph{comment être sage en tant d'urgence : Marie} François termine par une prière à Marie. figure de sagesse prophétique, figure de la conversion qui a prie soin de Jésus et prend soin du monde, figure de l'humanité convertie pour avoir méditée la vie de Jésus. Ce n'est pas la suppression de la souffrance du monde, des \textit{pauvres crucifiés}, mais compassion, nous donne d'entendre cette souffrance.
Entre-temps : Cri des pauvres et de la Terre.

\paragraph{Polysémie du sens conversion} dans laudato Si. Pape François est loin de ne pas être un théologien, en introduisant le contexte sapientiel.

\paragraph{Discernement sur les fins et les moyens} \sn{Cécile Renouard}Prophète : urgence. et sur les moyens, entre des rationalités, économie, il serait de l'ordre dialogale et sapientiale. Sapientiale : roi. Pretre, Prophète et Roi. 
On manque de cadre (Etat-Marché ne marche plus). 

\paragraph{Crise} pose la question du lien entre la parole et l'action. La parole du Christ joue sur l'interiorité de l'interlocuteur. Parole qui invite à l'action (en particulier il n'y a pas de pardon quand on utilise son pouvoir Pharisien). Et pourtant, Jésus ne renonce à l'action mais jamais il refuse l'action ("range ton arme" à Pierre). 

\paragraph{Revolution méthodologique de LS} Notion de bien commun, mais dont l'Eglise n'est pas propriétaire. 