\begin{comment}
\section{Genizah}
\paragraph{Universalité}

\paragraph{Inculturation ignatienne} Chine mais alors Islam ? et c'est quoi la culture ?

\paragraph{Conversion ou dialogue} Intérêt de l'autre doit nous motiver à le convertir. mais principe d'indifférence. 
\begin{itemize}
    \item \href{https://mission-ismerie.com/}{Mission Ismérie}  : 
\end{itemize}



\paragraph{Spiritualité des exercices} soufisme

\paragraph{Théobald} Hospitalité Chrétienne

\paragraph{Disciple Missionnaire} Travailler le thème de disciple
missionnaire en Cvx ? influence
Jésuite ? Matteo Ricci ? Question

\begin{quote}
    10. [Cheminer avec une Eglise missionnaire] Le Kairos dans notre Eglise nous appelle à être des \textbf{disciples missionnaires} pour le monde, au travers d’une rencontre avec Jésus qui nous ouvre à l’amour du Père.5 Austen Ivereigh, l’un des biographes du Pape François, nous a partagé ce que signifie entrer dans l’esprit missionnaire : être Christ dans notre monde blessé, en aidant les personnes à se reconnecter avec la création et le monde en tant que créatures de Dieu ; faire l’expérience de la famille et de la communauté, qui sont les liens de confiance et d’amour inconditionnel qui construisent la résilience, la personnalité et l’estime de soi ; et aussi aider les personnes à trouver un sanctuaire. Ce cheminement nous invite à nous laisser guider par la réalité et le Saint-Esprit dans notre mission. \sn{\href{http://www.cvx-clc.net/filesNewsReports/AM2018_FinalDocument\%20(FRENCH).pdf}{17è Assemblée Mondiale de la Communauté de Vie Chrétienne Buenos Aires}, Argentine 2018 CVX, Un Don pour l’Eglise et le Monde ‘Combien de pains avez-vous?... Allez voir’ (Mc. 6, 38) }
\end{quote}


\paragraph{test} \cite{Gardet:IntroductionTheoMusulmane}

\begin{quote}
    « En face de la bienveillance universelle du bouddhisme, du désir chrétien du dialogue, l'intolérance musulmane adopte une forme inconsciente chez ceux qui s'en rendent coupables ; car s'ils ne cherchent pas toujours, de façon brutale, à amener autrui à partager leur vérité, ils sont pourtant (et c'est plus grave) incapables de supporter l'existence d'autrui comme autrui. Le seul moyen pour eux de se mettre à l'abri du doute et de l'humiliation consiste dans une "néantisation" d'autrui, considéré comme témoin d'une autre foi et d'une autre conduite. La fraternité islamique est la converse d'une exclusive contre les infidèles qui ne peut pas s'avouer, puisqu'en se reconnaissant comme telle, elle équivaudrait à les reconnaître eux-mêmes comme existants. » \href{https://fr.wikipedia.org/wiki/Tristes_Tropiques}{Tristes Tropiques}

— Fin du chapitre XXXIX « Taxila », p. 466 (édition Plon de 1955, impression février 1972)



\end{quote}
\end{comment}
