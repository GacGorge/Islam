\begin{comment}
\section{Genizah}
\paragraph{Universalité}

\paragraph{Inculturation ignatienne} Chine mais alors Islam ? et c'est quoi la culture ?

\paragraph{Conversion ou dialogue} Intérêt de l'autre doit nous motiver à le convertir. mais principe d'indifférence. 
\begin{itemize}
    \item \href{https://mission-ismerie.com/}{Mission Ismérie}  : 
\end{itemize}



\paragraph{Spiritualité des exercices} soufisme

\paragraph{Théobald} Hospitalité Chrétienne

\paragraph{Disciple Missionnaire} Travailler le thème de disciple
missionnaire en Cvx ? influence
Jésuite ? Matteo Ricci ? Question

\begin{quote}
    10. [Cheminer avec une Eglise missionnaire] Le Kairos dans notre Eglise nous appelle à être des \textbf{disciples missionnaires} pour le monde, au travers d’une rencontre avec Jésus qui nous ouvre à l’amour du Père.5 Austen Ivereigh, l’un des biographes du Pape François, nous a partagé ce que signifie entrer dans l’esprit missionnaire : être Christ dans notre monde blessé, en aidant les personnes à se reconnecter avec la création et le monde en tant que créatures de Dieu ; faire l’expérience de la famille et de la communauté, qui sont les liens de confiance et d’amour inconditionnel qui construisent la résilience, la personnalité et l’estime de soi ; et aussi aider les personnes à trouver un sanctuaire. Ce cheminement nous invite à nous laisser guider par la réalité et le Saint-Esprit dans notre mission. \sn{\href{http://www.cvx-clc.net/filesNewsReports/AM2018_FinalDocument\%20(FRENCH).pdf}{17è Assemblée Mondiale de la Communauté de Vie Chrétienne Buenos Aires}, Argentine 2018 CVX, Un Don pour l’Eglise et le Monde ‘Combien de pains avez-vous?... Allez voir’ (Mc. 6, 38) }
\end{quote}


\paragraph{test} \cite{Gardet:IntroductionTheoMusulmane}

\begin{quote}
    « En face de la bienveillance universelle du bouddhisme, du désir chrétien du dialogue, l'intolérance musulmane adopte une forme inconsciente chez ceux qui s'en rendent coupables ; car s'ils ne cherchent pas toujours, de façon brutale, à amener autrui à partager leur vérité, ils sont pourtant (et c'est plus grave) incapables de supporter l'existence d'autrui comme autrui. Le seul moyen pour eux de se mettre à l'abri du doute et de l'humiliation consiste dans une "néantisation" d'autrui, considéré comme témoin d'une autre foi et d'une autre conduite. La fraternité islamique est la converse d'une exclusive contre les infidèles qui ne peut pas s'avouer, puisqu'en se reconnaissant comme telle, elle équivaudrait à les reconnaître eux-mêmes comme existants. » \href{https://fr.wikipedia.org/wiki/Tristes_Tropiques}{Tristes Tropiques}

— Fin du chapitre XXXIX « Taxila », p. 466 (édition Plon de 1955, impression février 1972)



\end{quote}
\end{comment}


\begin{comment}
\ref{Comment:MemoireISTR1}
   \paragraph{Ecologie \textit{ou Changement climatique}} \begin{itemize}
    \item Changement climatique : question \item \textit{scientifique} \textit{Ecologie} : rapport au monde, plus large, et porteur d'une vision du monde. Une certaine perméabilité entre les deux terme.
\end{itemize} 
\end{comment}


\begin{comment}

\ref{Comment:MemoireISTR2}
         Notes :
        explication de Hans Kung
        sur la difficulté du genre, pas casuistique (mais rappeler l’importance e la vie), pas un énoncé des droits universels (si on n’a rien à dire,…), pas une dissertation philosophique. Autocritique
begin{singlequote}
En Allemand, la déclaration portera le titre de déclaration pour un éthos planétaire, non pas pour une éthique planétaire. Ethos désigne la disposition morale fondamentale de l’homme, tandis qu’éthique nome la doctrine (philosophique ou théologique) concernant les dispositions, valeurs et normes morales.P60
end{singlequote}
        Un texte “moderne” : non européocentré (cf discussion sur le nom de “Dieu”)
        mais le risque de tomber dans un PGCD est réel. Est ce que les religions font tout en passant par cette déclaration ?
        Parti pris que les religions ont chacune à travailler directement le sujet.
        une approche plraliste
begin{singlequote}
éthique (ou ethos) planétaire, c’est à dire un accord fondamental en matière d’axiologie, de critères indiscutables et de choix essentiels. A défaut d’un consensus éthique fondamental, toute communauté court tôt ou tard le risque du chaos ou de la dictature.  Un ordre mondial meilleur ne peut se concevoir sans éthos planétaire
        (préface. p6)
end{singlequote}
        MAIS AUSSITOT
begin{singlequote}
“ethique planétaire ne signifie ni idélogie planétaire, ni religion mondiale unitaire à côté des religions existantes, ni quelque forme syncrétique de toutes les autres religions. Notre humanité est lasse des idéologies unitaires et les diverses religions du monde sont de toute manière si différentes dans l’expression de leurs croyance et dans leurs dogmes, dans leur symbolique et leurs rites, que tout effort d’”unification” est dénué de sens. P 6
end{singlequote}
“ce manifeste … point de départ”P 7
begin{singlequote}
Il nous faut tendre à l’instaration d’un ordre social et économique juste, au sein duquel chacun jouise de chances égales, au bénéfice de toutes ses possibilités humaines. P14
        Il est illusoire de vouloir rendre cette planète meilleure, sans changer d’abord la conscience des individus. Nous nous engageons dès lors à élargir notre capacité de perception, en acceptant pour nos esprits la discipline de la méditation, de la prière et de la réflexion. Refuser aucun risque ou aucun sacrifice revient à empêcher toute mutation sensible de notre situation présente. C’est pourquoi nous nus engageons à vivre selon cette éthique planétaire, dans l’intelligence réciproque, et à respecter entre nus des modes d’existence propres à promouvoir la tolérance mutuelle, la paix sociale et internationale et le respect de la nature. P 14
        Un pricnipe se retrouve depuis des milliers d’années dans beaucoup de traditions religieues et éthiques de l’humanité qui l’ont conservé, c’est la “règle d’or”; ce que tu ne veux pas qu’on fasse à ton endroit, ne le fais pas à l’endroit d’aucun autre. P 23
end{singlequote}
\end{comment}



\begin{comment}
\label{Comment:MemoireISTR3}
\begin{singlequote}
9 – Une argumentation rationnelle abstraite ne parvient que difficilement à convaincre des gens de différentes cultures et de différents milieux
John Rawls déduit des règles éthiques de principes généraux de justice, entendue comme fair-play, abstraite consciemment de tout contexte concret et de toute situation. Mais ce n’est qu’une idée étendue de justice qui lui permet après coup de développer une conception du droit et de la justice qui puisse aussi s’appliquer aux principes et aux normes du droit international et des relations internationales.
L’éthique de la discussion de Karl-Otto Apel et de Jürgen Habermas insiste à juste titre sur l’importance du consensus rationnel et de la discussion. Ils prétendent en cela pouvoir formuler des normes, dans une moindre dépendance à l’égard du contexte, qui vaillent inconditionnellement, et ce à partir de la communauté humaine de communication et d’argumentation. Les principes religieux et les interprétations religieuses de la morale, dévalorisés face à l’espace public, doivent être remplacés par une discussion rationnelle, par un jeu de langage régulé, par la « contrainte de l’argument sans contrainte ». Compte tenu de la réalité concrète de la vie, il est discutable qu’on puisse atteindre un éthos global (pour ainsi dire jusqu’au dernier village indien ou africain), qui soit réellement obligatoire et contraignant, à l’aide d’une discussion rationnelle abstraite.
\end{singlequote}
\begin{singlequote}
        11 – Les traditions religieuses ne doivent pas être objet de mépris, mais de réflexion critique
        L’anthropologie culturelle nous l’enseigne : les normes éthiques concrètes, les valeurs et les intuitions éthiques se sont développées graduellement, selon un processus socio-dynamique extrêmement complexe.
        Selon que des besoins vitaux apparaissaient, que des urgences et des nécessités entre humains se manifestaient, dès le début il a fallu des orientations et des régulations de l’action : des conventions déterminées, des sagesses, des mœurs, bref des critères éthiques, des règles, des normes, qui au cours des siècles et des millénaires ont été éprouvés. En cela, il est frappant que certaines normes éthiques se ressemblent partout dans le monde. Mais c’est un fait historique : à travers des millénaires, ce sont les religions qui fournirent des systèmes d’orientation, qui formèrent les bases d’une certaine morale, qui les légitimèrent, et qui souvent sanctionnèrent les déviations par des peines.
        À l’origine, la philosophie et la religion, la philosophie et la théologie ont plutôt vécu en symbiose ; celle-ci ne peut plus être rétablie. Mais une coopération plus intensive est recommandable, en vue d’une même vision d’espérance : “To make the world a better place”. Pour réaliser cette espérance, il faut au principe un changement de conscience vers un éthos humain, au service d’une culture de la non-violence et d’un respect de toute vie, de la solidarité et d’un ordre économique juste, de la tolérance et de la vie en vérité, d’une égalité des droits et d’un partenariat entre hommes et femmes.
\end{singlequote}
\subparagraph{approche philosophique}        
        eviter le recours au concept de loi naturelle
\begin{singlequote}   
        Il faut plutôt chercher à atténuer, par une solution pragmatique de problèmes urgents, les oppositions entre visions du monde, sans tenir compte des différences idéologiques 
            Il était clair dès le départ que le projet « Éthique planétaire » se situait dans la ligne de l’éthique de responsabilité de Max Weber, qui ne met pas entre parenthèses l’orientation juste, mais veut concentrer l’attention sur les conséquences raisonnables. Jean-Paul Sartre voyait déjà que le domaine de validité de la responsabilité s’étendait à tout le monde des humains. Hans Jonas l’a étendue à toute la biosphère, appelant à réfléchir aux conséquences dangereuses de l’agir, y compris pour les générations à venir. En cela, le « principe responsabilité » de Jonas et le « principe espérance » d’Ernst Bloch ne s’excluent pas. Emmanuel Levinas a fait ressortir que la situation entre les humains est au centre de cette responsabilité, et qu’il faut prêter attention à l’altérité de l’autre, comme à son caractère étranger, et qu’il faut avoir des égards même pour les ennemis. À la recherche de compétences pour une communication responsable, Hannah Arendt a encouragé en particulier une façon de penser plus large, un imaginaire et un sens commun, et – ultimement – elle a mis avant tout en relief la vertu de vérité, qui s’efforce d’atteindre la vérité des faits, car naturellement, sans elle, il n’y a aucune communication ouverte possible entre les hommes.
            Il faut plutôt chercher à atténuer, par une solution pragmatique de problèmes urgents, les oppositions entre visions du monde, sans tenir compte des différences idéologiques : cela pourrait à long terme établir des points communs, y compris justement un éthos commun. Le conflit des visions du monde ou des idéologies devrait être apaisé de cette façon.
\end{singlequote}
\end{comment}    


\begin{comment}
    \ref{Comment:MemoireISTR4}  
   	    \begin{singlequote}
L'écologie politique contemporaine donne un nouvel infléchissement aux débats environnementaux qui risquent, sinon, de rester bloqués dans un paradigme réducteur et moderniste. Il est intéressant de noter que cette nouvelle écologie politique s'inspire de plus en plus du langage et de concepts théologiques, en particulier dans l'œuvre de Bruno Latour. Le présent article explore les raisons pour lesquelles il en est ainsi et quelle contribution cette approche peut apporter. L'écologie politique assigne un rôle à la religion en ce que celle-ci génère le genre de conversion aux valeurs humaines qui s'avèrent nécessaires pour une véritable transformation sociétale. En procédant ainsi, l'écologie politique pourrait même être considérée comme un partenaire de dialogue (surprenant) pour la théologie catholique et pour des approches de la crise environnementale qui s'appuient plus largement sur la tradition de l'enseignement social catholique.
\end{singlequote}
\begin{singlequote}
        À première vue, la popularité et la diffusion de ces idées est peut-être surprenante, car dans leur engagement radical à faire exploser la dichotomie moderniste supposée des humains et de la nature, ces théoriciens compliquent la compréhension du rôle de l’action humaine pour faire face à la crise planétaire à laquelle nous sommes confrontés. Ils sont notamment pessimistes quant au potentiel des réponses managériales ou technologiques d’origine humaine et visent fréquemment des partisans de l’« écomodernisme » et ceux qui proposent des stratégies grâce auxquelles les sociétés humaines pourraient envisager d’atteindre un « bon Anthropocène » [9]. 
\end{singlequote}
   	\end{comment}


    \begin{comment}
\ref{Comment:MemoireISTR5}   
            Latour : sous jacent du film dystopique sur l’asteroide : quelques personnes font le salut des autres (geo ingénierie” : apocalypse pour les autres
\begin{singlequote}
        Il existe de très nombreux travaux relatifs à la théorie de Gaïa, des points de vue scientifiques comme non scientifiques, à la fois positifs et critiques vis-à-vis de sa méthodologie et de son potentiel explicatif [20]. Il est certain que nombre des tenants de l’écologie politique en question lisent Lovelock en non-spécialistes et l’abordent « d’une manière enthousiaste mais créative », comme le formule un commentateur[21]. Néanmoins, le concept de Gaïa leur sert d’outil pour décrire un système vraiment immanent qui n’est pas dirigé ou piloté par une force ou un acteur externe.
        La valence écologique de l’idée de Lovelock réside dans la manière dont elle rappelle aux êtres humains leur statut d’acteurs opérant au sein des systèmes écologiques de la planète Terre. Cette approche implique au moins deux avantages pour l’écologie politique. D’abord, elle réfute la logique écomoderniste qui situe l’acteur humain pour ainsi dire à l’extérieur de ce système, avec la fausse assurance qu’il est équipé pour le gouverner lui-même de manière quasi divine. Car la pensée Gaïa stipule qu’un système stable de la Terre est la fonction de processus multiples, intimement imbriqués, mais non régis de l’extérieur. Bruno Latour le formule de la manière suivante : « Je pense que c’est cela que Lovelock laisse sous-entendre en quelque sorte, à savoir que la Terre est connectée. Chaque élément des entités qui la constituent construit son propre environnement, mais il n’y a pas d’“organisateur” ». Il n’y a pas de Dieu, en fin de compte » [23]. Il n’est donc pas étonnant que Latour considère Gaïa comme un outil « séculier » pour réorganiser la politique, loin de l’impasse théologique dans laquelle elle a été entraînée jusqu’ici [24], ce qui fait progresser, en ce sens, « l’intuition […] entièrement séculière de Lovelock » [25]. 
        Le second atout de l’idée de Lovelock, aux yeux des penseurs écologiques, consiste en ce qu’elle invite à un mode d’action adéquatement responsable. Isabelle Stengers souligne que comprendre Gaïa c’est se rendre compte que la Terre elle-même est devenue « chatouilleuse », « susceptible » d’une certaine manière, dans le sens où ses mécanismes homéostatiques finement accordés ne sont pas indépendants de nos actions, et que nous ne pouvons pas non plus présumer qu’ils rebondiront en réponse au stress que nous leur imposons [26]. Nous avons beau être une partie mineure d’un système plus large, il n’en reste pas moins que notre empreinte environnementale particulière importe vraiment, qu’elle soit grande ou petite.
    \end{singlequote}
\end{comment}


\begin{comment}
\ref{Comment:MemoireISTR6}   
    Alors cette histoire de Dieu a établi sur terre un vicaire, un sous régent, ça veut dire lui a confié un dépôt sacré pour s'en occuper en quelque sorte.
L'homme, dans un autre verset, était un petit peu naïf. Il a accepté ce dépôt sacré parce qu'il a été au départ confié à toute la nature. Nous avons exposé le dépôt sacré Amana à à tous, sur tous les cieux et les terres. Ils ont refusé, mais l'homme a accepté.
Alors l'homme a accepté d'être le vainqueur, d'être le sous régent pour s'occuper de la terre. Même si les Anges ont un peu mis en garde le Seigneur. Attention, peut être que vous allez mettre quelqu'un qui va répandre le sang et qui va mettre le désordre, ça veut dire qu'il va mettre du désordre dans la nature.
{Paradis}et ce modèle, bien sûr, c'est le paradis avec les rivières limpides, avec le miel des rivières de miel, avec les rivières de lait, avec les quatre voies sont connues pour les Arabes.
Vous savez que c'est un milieu austère. Et quand on parle de miel et d'eau extraordinaire, ça, c'est l'image de la nature parfaite. Cette nature parfaite. Finalement, vu qu'on vivait dans un désert, qu'on vivait dans un milieu austère, c'était presque le modèle et le rêve. Et tout ce qui pourrait exister sur le sur terre et éphémère. Et c'est ce paradis, cette nature idéale, cette nature extraordinaire.
C'est cette nature qu'on a perdu, qu'on pleure quand tous les poètes ont vraiment pleuré.
\end{comment}

\begin{comment}
\ref{Comment:MemoireISTR7}   
Alors Al-Ghazali ne connaît pas la question l'anthropocène, capacité qu'a l'homme à modifier son environnement, L'extermination des dodo ? Il ne connaît pas.
Je dirais même que pour lui, la création, au contraire, est un réservoir infini de biens disponibles pour l'homme. Et c'est le signe même de la manifestation, de la bienveillance, de la bonté, de la bienfaisance de Dieu à l'égard de l'homme que de lui donner.
Tout est là pour l'homme à son service. Et si Al-Ghazali envisage l'activité désordonnée de l'homme, elle ne saurait de toute manière avoir des répercussions décisives sur la création.
{Théologie de l'action}
Pourtant, et c'est mon deuxième point, il y a chez lui une théologie d'action qui n'est pas sans incidence pour notre sujet. En effet, et dans sa réflexion, c'est un théologien.
Il part donc de Dieu. Il constate que méditer sur l'essence de Dieu qui est Dieu, ce n'est pas possible pour l'homme . Une méditation sur l'essence divine et pour mieux se faire comprendre. Peut être que je vous ai déjà perdus en disant cela.
Si on lui disait à la mouche que son créateur ne possédait ni des ailes, ni même ni pied, ni capacité à voler, elle le trouverait plus imparfait qu'elle même qu'elle n'est.
Or, malheureusement, déplore Al-Ghazali, ce raisonnement, le raisonnement de la mouche et celui de la majorité des hommes. Et comme il n'écrit pas seulement pour des savants. Il est donc préférable, pour parler de Dieu, non pas de partir d'une méditation de son essence, mais de ce qu'il crée et donc de la création.
{Nous sommes insensible à la beauté de la Création}
Aussi, il l'invite à une méditation sur la création. Et dans cette optique, la création, vous le voyez, n'est pas seulement uniquement la disposition de l'homme pour satisfaire ses besoins, mais elle est là pour \textsc{permettre à l'homme de retrouver le chemin du Créateur et admirer la grandeur de Dieu}, grandeur qui suscite la gratitude et l'adoration. Alors, bizarrement, déplore Al-Ghazali, les hommes trouvent beau devant un tableau peint par des hommes par des mains d'hommes, Mais ils restent insensibles à l'artisan des merveilles de la création.
{C'est Dieu qui écrit dans la Création} la création acquiert un véritable statut théologique. La méditation sur les choses créées vient ainsi tempérer l'image d'une pensée où la création ne serait là que comme un réservoir disponible à l'homme pour sa consommation et son confort. La création, au contraire, acquiert une qualité qui implique une attention, parce qu'au delà de la chaîne des causalités, il s'agit de découvrir qui est Dieu écrit.
Tous les êtres vivants sont des effets de la puissance de Dieu et une des lumières de son essence. Il n'y a rien de plus obscur que le néant et de plus lumineux que l'existence. L'existence de toute chose est une des lumières de l'essence de Dieu.
{la méditation transforme}La \textit{méditation} est donc pour le Ghazali, le chemin de l'illumination. Elle permet d'accéder à la connaissance divine, ce qui n'est pas sans incidence profonde pour l'homme.
Les passions de son cœur sont alors transformées. Les facultés de la vision intime intérieure développée par ces lumières divines auxquelles il accède. La perspective est clairement celle du soufisme. La méditation sur le créé permet voir ce pas visible du premier coup d'œil et l'ouvre à la possibilité un spirituel plus élevé. Mais quid ? Quid de son comportement à l'égard du Christ ?
Al-Ghazali s'appuie sur une tradition prophétique qui dit que la méditation sur les choses créées est la moitié de l'adoration. Mais le peu de nourriture, l'adoration entière. Ainsi, pour lui, la méditation de la création met sur la voie de l'adoration, adoration qui n'a que plénitude par la saisie.
\paragraph{l'ascèse régule nos appétits}
L'ascèse, en effet, est au cœur du dispositif du chemin spirituel qui conduit vers Dieu. Chez Al Ghazali, l'ascèse régule nos appétits et plus encore dans une perspective fonctionnaliste. Alors Ghazali va justifier la nécessité d'éprouver la fin. Et mesdames et messieurs, il va distinguer dix bonnes raisons d'éprouver la fin. 
\begin{itemize}
\item Premièrement, elle purifie le cœur et oui, elle anime le tempérament et à la fuite, le regard intérieur. Alors que la satiété provoque l'apathie, elle aveugle le cœur et donc brûle le cerveau. Ainsi, l'ascèse ouvre à la connaissance des réalités divines.
\item Deuxièmement, la purification du cœur prépare à l'émotion de l'énoncé de l'énoncé de Dieu. Souvent, on prononce le Dieu, mais sans s'en délecter.
Mais celui qui a faim, quand il convoque le nom de Dieu, il le fait avec délectation.    \item Troisièmement, la fin suscite la crainte de Dieu. Elle brise lames dans le sens de son impudence, de son orgueil, de sa rébellion à l'égard de Dieu ou de son oublie de Dieu. Mais la foi permet à l'homme de voir son impuissance, sa faiblesse et elle redonne toute sa place à Dieu.
  \item Quatrièmement, la faim permet de se souvenir des épreuves qui attendent ceux qui sont éprouvés par Dieu. Elle rappelle les tourments de l'au delà et elle crée une solidarité avec ceux qui ont faim ou. Pourquoi tiens tu autant à la faim alors que tu possèdes les trésors de la terre ? Le Prophète répondait J'ai peur de manger à ma faim et d'oublier ceux qui souffrent.
Ce qui en souffre.     \item Cinquièmement, elle permet d'affaiblir les passions liées aux actes de désobéissance. Au fond, la faim de l'homme, la faim permet de contrôler tous ces désirs.     \item Sixième sens permet de repousser le sommeil, de s'habituer à la vie.     \item Septième. Nous, elle facilite l'assiduité à l'adoration. On gagne du temps à ne pas se nourrir et se tend. Pour quoi faire ?
Enfin, pour jouer sur son téléphone portable. Nous sommes au XIIᵉ siècle, mais ce temps pour adorer, pour se recueillir, pour se souvenir du nom de Dieu. On gagne du temps à ne pas faire les courses et à préparer les repas, à se laver les mains et la bouche, à aller plusieurs fois aux toilettes parce que l'on a trop bu et pour manger ?
Oui, avec Al-Ghazali, on est toujours au cœur du quotidien et de tous ces moments.    \item Huitième Non, cela donne la santé car trop de nourriture alourdit le corps et le rend malade. L'ascèse épargne des maladies au corps, épargne les cures de maladies comme l'impulsivité ou d'autres vices.     \item Neuvième un Manger peu. On fait des économies, les prix augmentent. Mangeons moins    
\item et 10ᵉ enfin Manger peu ou pas.
\end{itemize}
{donner, c'est conserver dans les coffres de la grâce divine}
L'altruisme nous rend bons envers les orphelins et les pauvres. Toujours avec un esprit très prosaïque, À la Al-Ghazali écrit que la nourriture dont il se nourrit finit dans les latrines, alors que la nourriture dont il fait l'aumône est conservée dans les coffres de la grâce divine. Alors, pour conclure, nous étions partis de la vision consumériste de la création chez Al-Ghazali et visiblement de la difficulté à élaborer une théologie islamique de l'écologie.
{La Création, chemin vers Dieu}
Nous avons vu cependant, à partir de ces livres spirituels qui invitent à voir toute chose comme comportant une sagesse, qu'il s'agisse des étoiles, des mers, des plantes ou des animaux. Pour lui, la création n'est pas d'abord à la disposition des estomacs et des passions des hommes, mais elle est un chemin pour parvenir adieu à la connaissance de Dieu.
Certes, certains abusent, gaspillent, ont un comportement ingrat. Ainsi écrivent critiques. Cassé la branche d'un arbre sans but précis est une ingratitude. Pour autant, il poursuit en disant que l'usage des arbres pour les hommes est conforme à la volonté divine, selon le principe que si l'homme et l'arbre sont tous deux périssables, l'homme est plus noble que l'âme, et il est sage que ce qui est moins noble contribue à la pérennité de ce qui est plus noble.
Si on devait s'arrêter à ce niveau de réflexion, on retrouverait la vision d'une création mise simplement au service de l'homme le plus noble des créatures.
{Aller plus loin en ajoutant l'idée de justice}
Mais il faut aller plus loin et elle va plus loin en y adjoignant l'idée de justice. Ainsi, utiliser l'arbre du voisin, qui donc ne m'appartient pas, est une injustice. Or, rappelle t il, le propriétaire ultime de toute chose créé, c'est qui ?
C'est Dieu ? Ainsi, indépendamment de la main divine, de la bonté de Dieu, celui qui prend au delà de ce dont il a besoin se comporte comme un homme injuste. Le questionnement de Al-Ghazali sur l'écologie n'aura donc pas été vain, et il me semble que l'on peut voir dans ces concepts, dans ces exemples, dans cette réflexion spirituelle, les braises encore chaudes pour penser à partir de cette éthique islamique de la responsabilité, une écologie islamique.
Et commencer à voir un peu mieux notre quotidien des lectures que nous faisons du Coran, de la poésie andalouse, des philosophes, des textes mystiques.
\end{comment}


 
\begin{comment}
\label{Comment:MemoireISTR8}    

    L'omnipotence de Dieu est affirmée de façon absolument massive, totale, inlassablement répétitive. Dieu est puissant sur toute chose. La formule revient comme un refrain. Je crois que je vais trouver une petite quarantaine d'occurrences de cette formule, précise, et bien des versets vont en préciser le sens. Dieu fait ce qu'il veut, il crée ce qu'il veut, il fait le bien.
Il crée le bien, mais aussi le mal. Il châtie et récompense à son gré. Rien n'arrive qu'il ne l'ait voulu.  Il envoie aux hommes bénédiction, mais aussi fléaux et calamités.
{Comment réconcilier liberté de l'homme ?}Et cette omnipotence divine, affirmée dans sa version la plus extensive, va poser au premier théologien de l'islam un certain nombre de difficultés, et en particulier parce qu'elle entre en concurrence avec la liberté et la responsabilité de l'homme.
Si Dieu, en effet, crée toute chose et en particulier chacune de nos actions, nous les bonnes comme les mauvaises. Alors, au nom de quoi va t il nous juger ? Au dernier jour, il enverra en enfer les méchants, mais pour des actions qu'il a lui même créées. Bizarre, c'est quand même pas très juste cette tension entre la responsabilité humaine et la toute puissance de Dieu.
{Conséquence pour la réflexion écologique, chaine naturelle de causes et d'effets}
Cette question théologique et morale est passionnante et je pourrais en parler littéralement des heures, mais je comprend que c'est assez mobiliser l'énergie de théologiens médiévaux. Mais si je vous en parle ce soir, c'est parce que cette envahissante toute puissance de Dieu n'a pas seulement des effets sur la morale humaine. Elle risque en effet de rendre également impossible toute réflexion écologique.
La réflexion écologique, en effet, suppose comme premier préalable l'existence d'une chaîne naturelle de causes et d'effets. Si on réunit la COP 27 , c'est bien parce qu'on constate les effets de l'action de l'homme sur le climat, sur la biodiversité et qu'on espère que d'autres actions ont d'autres effets, ont des effets correcteurs, et cetera Cela nous paraît peut être assez évident cette idée qu'il y a des causes et des effets, mais cette évidence n'est pas partagée.
En effet, la fascination pour la toute puissance divine a pu pousser des penseurs de l'islam, parfois des intellectuels immenses, à minimiser, voire à nier cette chaîne des causes qui régit notre monde. Au profit de quoi ? De l'action immédiate de Dieu, créateur de toute chose à tout instant. Et donc, si ce verre tombe, si je le lâche, ça ne veut pas dire qu'il tombe parce que je le lâche dans cette optique là, parce que ça ferait de moi le créateur d'une réalité.
{Combattre ou accepter le changement climatique ?}
Peut être même doit on le combattre. S'il est en revanche le résultat de la volonté active de Dieu, alors ce réchauffement, il faut l'accueillir dans la foi, peut être comme une épreuve imposée à l'humanité. Mais on ne peut pas vraiment lutter contre lui, car ce serait alors lutter contre la volonté de Dieu. La seule attitude raisonnable est alors ou pas précisément, comme on le dit parfois, le fatalisme, parce qu'il ne s'agit pas d'une fatalité, mais l'abandon, la contemplation de Dieu dans la réalité qui est devant moi, sans cesse, dans ses beautés comme dans ces calamités qui expriment l'insondable et l'adorable volonté de Dieu.
Mais la théologie islamique classique, la théologie du kalam, a, dans certaines de ses écoles,  essayé de produire des des éléments conceptuels qui rendent pensable cette envahissante toute puissance de Dieu contre la chaîne naturelle des causes et des effets. Mais le même Kalam a aussi vu naître en son sein les plus farouches opposants à cette conception la toute puissance divine, à savoir les représentants de l'école mu'tazilite.
\end{comment}


\begin{comment}
 \label{Comment:MemoireISTR9}   
    En ce sens, la thèse du frère Emmanuel est importante dans sa lecture de Ghazali. Je cite cet appel à retrouver les merveilleux mondes de ce qui est donné à l'homme dans la création et sans doute le fondement pour aujourd'hui d'une attitude écologique. Et c'est une thèse pertinente sur cette question de l'émerveillement qui implique la question du regard que l'on porte sur la nature.
{Importance du regard sur la nature }
La manière dont on regarde la nature n'est pas innocente. Ensuite, dans la manière dont nous allons nous comporter par rapport à elle, que l'on considère une nature purement de manière profane ou une nature en relation avec Dieu comme création. Mais alors, aujourd'hui, les résultats du GIEC et de l'EEP sur la biodiversité ou les différents travaux des différentes COP, ce n'est pas l'émerveillement qui préside.
C'est plutôt l'imaginaire de la catastrophe qui est censée faire fonctionner les leviers de l'engagement écologique. L'engagement, l'émerveillement est il suffisant ? Est il suffisamment puissant comme levier pour supplanter cette heuristique de la peur à court terme ? Certainement pas. Alors, qu'est ce qu'on fait là ? Si ce n'est pas le cas. Mais justement, ce qui est intéressant ce soir, c'est que finalement, nous ne travaillons pas sur les ressources à court terme pour lutter contre la crise écologique, mais sur les ressources de Sens qui ont été évoquées par mes collègues.
Efficientes dans le long terme et dont il est urgent de bien commencer à travailler leur réception dès aujourd'hui. En effet, les représentations, ça peut être dangereux dans les mentalités humaines. Si je prends un exemple, ce qui a été donné par le frère Emmanuel, il a dit Pour lui, la création est au contraire comprise comme un réservoir infini de biens disponibles pour l'homme.
Mais si on prend ça au pied de la lettre, c'est dramatique en termes de relation à la nature, parce que ça veut dire qu'on peut épuisé toutes les ressources avec l'idée qu'il y en aura toujours et que, parce que Dieu va les fournir sans arrêt. Et donc là, on prend quelques risques pour l'avenir de l'humanité sur la terre.
Donc le site, cette représentation, elle est aussi contextualisée. C'est celle d'un médiéval qui vit dans un monde qui lui semble immense, avec des ressources qui lui apparaissent comme infinies. Et donc c'est cette représentation, Eh bien, elle conditionne sa façon de penser le rapport de Dieu à la création, mais aussi le rapport de l'humain à la création. Donc, je vous dis ça pour dire que les représentations de la nature ne sont pas anodines, tant sur le plan religieux que sur le plan strictement écologique.
Pour penser le rapport humain à la création. Alors, un deuxième propos préalable, c'est peut être pour parler d'un premier, d'un grand absent des représentations dans le discours de mes collègues de ce soir. Je m'excuse de commencer par ça, mais au moins ce sera fait. Justement, il manque la question de l'interdépendance dans les critères qu'ils ont soulevé. Mes collègues vont parler d'écologie.
Parce que le mot écologie est basé sur le concept de maison. Wiko logo en grec, le discours sur la maison, sur l'habitat. Donc si c'est une bonne nouvelle finalement, alors sous cet angle de représentation de la nature, la maison est une issue du discours religieux. J'aimerais relever trop brièvement si lieu de résonance, quelques minutes chacun. D'abord le thème de l'intendant de la création.
Deuxièmement, celui du livre de la nature. Troisièmement, l'autonomie des réalités terrestres. Quatrièmement, la présence de Dieu dans la création. Cinquièmement, la question de l'ascèse en rapport avec la sobriété. Et finalement, la destination est scatologique. Des créatures. Une autre façon de parler de la fin du monde. Alors, d'abord, la question de la représentation de Dieu dans la création, l'Intendant de Dieu dans la Création.
Le Coran est effectivement à au moins deux reprises la sourate deux et la sourate six parle de cette instauration du califat humain sur la création, la lieutenance, la vice régence, l'intendance, où, le successeur même de Dieu dans la création, et cela a des implications éthiques très importantes.
C'est une idée reprise dans les hadiths aussi, et de ce fait, au niveau éthique, les condamnations du gaspillage dans le Coran sont compris. Le gaspillage est compris comme un méfait contre la création. Et Dieu n'aime pas que l'homme maltraite la terre. À la sourate deux ou à la source de. Cette idée de méfaits contre la création est présent, alors ça renvoie à la responsabilité humaine.
Dans le premier chapitre de la Genèse, dans la Bible qui est créé à l'image de Dieu, puisque être créé à l'image de Dieu dans cette tradition du Proche-Orient antique, c'est recevoir aussi un mandat de service du peuple qui est ainsi confié au nom, au nom de la divinité, au nom de Dieu, et ce service est confié à toute l'humanité.
Toute l'humanité est faite et créée à l'image de Dieu, et c'est cette, cette création à l'image de Dieu. Un pour implication le service politique de l'intendance de toute la création. Donc je pourrais développer là aussi des œuvres sur sur ce sujet, comme le frère pourrait parler des heures de l'occasion dans la tradition musulmane. Mais moi je le sais, c'est un thème vraiment que je trouve important.
Pour un peu, déminer le terrain de la domination biblique de la création qui est une interprétation purement cartésienne de Genèse un. Et c'est un thème très présent dans la recherche en théologie de l'écologie aujourd'hui. Deuxième point le livre de la nature. C'est ce que j'appelle la dimension théophanie de la création. Le livre de la nature, c'est un vieux et un vieux concept qui remonte à la fin.
Le mot remonte au Moyen Âge, mais l'idée qui recouvre est très ancienne chez les Pères de l'Église. On est dans cette idée que la création nous dit quelque chose de Dieu. Ainsi, il le dit, je le cite la nature, aussi loin qu'elle se trouve dans l'ordre de l'être, n'en reçoit pas moins la lumière de l'être, ce qui permet à chaque chose en particulier l'homme de se tourner vers son principe premier.
Mais ça, c'est tout à fait convergent et congruence avec la tradition chrétienne. Qui est qui ? Qui nous dit depuis même les textes bibliques, que la création est témoin du Créateur, celui qui veut avoir un autre aspect de la connaissance, de la création, en complément où une lumière projetée par les Écritures sur la création pour la connaissance de Dieu.
Il a là cette capacité par son intelligence, de reconnaître les signes de l'existence de Dieu, de sa grandeur, de sa splendeur, de sa beauté. Donc la matière n'est pas indigne d'être le support et la médiation de la lumière divine qui se laisse voir par les yeux humains. Et ça, cela participe en fait de la bonté. Un thème qui va revenir assez souvent maintenant, de la création montée qui va contribuer à construire ce que le pape François appelle dans son encyclique Laudato si sur la sauvegarde de la maison commune, la valeur propre et la valeur intrinsèque des créatures ou de la création.
Quant à lui, le frère Emmanuel a dit je cite ainsi la médiation des actes de Dieu et de nécessité. Et si la création était à la disposition de l'homme ? Ce n'est pas seulement pour satisfaire ses appétits et ses besoins de toutes sortes, mais c'est aussi pour lui permettre de retrouver le chemin du Créateur et d'admirer sa grandeur qui doit engendrer de sa part gratitude et adoration.
Ici, on a plus spécifiquement des résonances bibliques à travers le livre de la nature, des résonances qui sont elles mêmes évoquées par le pape François dans \textit{Laudato si}. Dans son chapitre sur la bonne nouvelle de la création, le chapitre deux de l'encyclique au numéro 84 de l'encyclique, je cite par exemple Tout l'univers matériel est un langage de l'amour de Dieu, de sa tendresse démesurée envers nous.
Le sol, l'eau, les montagnes, tout est caresse de Dieu au numéro 85. Un peu plus loin, il dit encore Cette contemplation de la création nous permet de découvrir à travers chaque chose un enseignement que Dieu va nous transmettre. Parce que pour le croyant, contempler la création, c'est aussi écouter un message, entendre une voix paradoxale et silencieuse. Nous pouvons affirmer qu'à côté de la révélation proprement dite qui est contenue dans les Saintes Écritures, il y a donc une manifestation divine dans le soleil qui resplendit comme dans la nuit qui tombe.
Ce qui fait écho aux propos d'Assise tout à l'heure, le frère Emmanuel rajoute toutes choses par sa nature, dit dans sa langue la majesté de Dieu. Si ce n'est pas un écho du psaume 18, les cieux proclament la gloire de Dieu. Le firmament raconte l'ouvrage de ses mains. La formulation d'Assise est néanmoins proche de celle que l'on trouve dans la tradition théologique chrétienne, occidentale et orientale.
Dieu se fait connaître donc par ses œuvres, et notamment la tradition franciscaine qui est à convoquer ici. Pour saint François d'Assise, toute créature est une occasion de rencontre du Créateur, car d'après saint Bonaventure, son disciple, elle est porteuse des vestiges du Dieu trinitaire. On pourrait développer sur le rapport entre relations écologiques et vestiges de trinité dans la création, comme le pape François fait dans Laudato si, Mais je n'ai pas le temps.
Je voudrais passer maintenant sur le thème sur le thème de l'autonomie des réalités terrestres. Le frère Adrien propose comme enjeu de réflexion écologique les enjeux de la causalité. Alors on agit en chrétien. Je vous assure que c'est une grosse question, déjà, en particulier dans le registre du dialogue entre science et religion. Mais si je résume votre thèse Frère Adrien, vous nous dites que vous nous proposez de penser que si la causalité est réelle dans le monde, alors oui, la responsabilité humaine, par l'engagement de sa liberté est possible.
Et donc, dire que l'on peut lutter contre la crise écologique a un sens, surtout dans un contexte qui suppose que l'être humain reçoive une vocation à l'aide, à l'intendance ou à la sauvegarde de la création. Alors je suis heureux d'apprendre que le courant des élites semble apte à penser une telle piste de réflexion écologique en régime islamique. Et c'est tout à fait en résonance avec ce que propose déjà depuis quelques décennies l'Église catholique avec le concile de Vatican II.
Au paragraphe 36 de Gaudium et spes, qui s'intitule justement Autonomie des réalités terrestres, je cite le paragraphe 36 sous paragraphe deux ci Par autonomie des réalités terrestres, on veut dire que les choses créées et les sociétés elles mêmes ont leurs droits, ont leurs lois et leurs valeurs propres que l'homme doit peu à peu apprendre à connaître, à utiliser et à organiser une telle exigence d'autonomie pleinement et pleinement légitime.
Non seulement elle est revendiquée par les hommes de notre temps, mais elle correspond à la volonté du Créateur. C'est en vertu de la création même que toutes les choses sont établies selon leurs ordonnances et leurs lois et leurs valeurs propres, que l'homme doit peu à peu apprendre à connaître, à utiliser et à organiser. Donc, on voit bien ce rapport entre causalité de la nature dans son ordre et la capacité de l'être humain à agir dessus en toute responsabilité.
Mais l'exercice de la toute puissance divine en régime chrétien se comprend donc différemment qu'en régime islamique, en tout cas dans le régime de la branche principale du Kalam. Mais c'est toute puissance de bien effective en régime chrétien. Le concile continue, même si, par autonomie du temporel, on veut dire que les choses créées ne dépendent pas de Dieu et que l'homme peut en disposer sans référence au Créateur.
La fausseté de tels propos ne peut échapper à quiconque reconnaît Dieu. En effet, la créature, son créateur, s'évanouit. Du reste, tous les croyants, à quelque religion qu'ils appartiennent, ont toujours entendu la voix de Dieu et sa manifestation dans le langage des créatures, et même l'oubli de Dieu, rend opaque la créature elle même. Donc la toute puissance divine. Elle est à comprendre en une référence à la foi directe.
Si Dieu cesse son activité de soutient de la création, la création retourne dans le néant. Mais c'est une activité qui permet, tout en tenant la création dans sa main, de laisser la création fonctionner selon ses lois propres. Des lois créées par Dieu. Le frère Adrien continue en citant la branche secondaire du Kalam que l'on semble retrouver aujourd'hui. Dieu a créé un véritable monde où entrent en jeu les lois de la physique et la volonté des vivants qui ne sont pas de simples illusions masquant la seule force agissante et la toute puissance divine.
 Et pour le Ghazali, apparemment, cela passe par l'exercice de l'ascèse.
 c'est le même mot exercice et ascèse en grec. Alors c'est cette vision ascétique du rapport aux créatures. Elle est aussi très présente dans la tradition chrétienne et plus, mais plutôt dans la tradition orientale et orthodoxe, et qui est une ressource spirituelle fortement mobilisée. D'ailleurs, actuellement depuis les années 80 et 90, dont la lutte contre la crise écologique avec ses ressources spirituelles proposées par le patriarche Bartholomée premier et le sens de de cette ascèse en régime chrétien.
Si je reprends les propos du patriarche Bartholomée, c'est s'entraîner à vivre en ressuscités,  c'est mettre en œuvre une forme d'ascèse. Mais qu'est ce que c'est que cette ascèse ? Eh bien, c'est la mise en œuvre pratique d'une réflexion et d'une intégration du sens des limites, de la création, du sens des limites de l'être humain dans la création et de son rapport à Dieu.
L'hubris de l'humain, l'absence de limites de l'humain. C'est quand il se prend pour Dieu qui est celui qui est absolu. Eh bien, retrouver le sens de la limite, c'est retrouver le sens du statut de créature et retrouver les limites des autres créatures afin de pouvoir avoir une relation juste et ajustée avec elle. Et c'est ça le sens de la sobriété dont parle le pape François dans Laudato si.
Retrouver ce sens d'une ascétique au sens joyeux du terme, à finalité écologique. Alors, l'originalité, effectivement, ici, c'est éprouver le manque ou éprouver la faim. Et ça ? Peut être est ce quelque chose qui peut rentrer dans un dialogue un peu dialectique avec la perspective, la sobriété heureuse, parce que la sobriété, elle va dire on peut arriver à trouver satisfaction en ayant une meilleure connaissance de ses besoins réels.
Et dans cette articulation entre besoin et satisfaction, on peut vraiment développer un sens de la joie de vivre. Et donc peut être que c'est un lieu à creuser, cette place de la faim qui n'est pas nécessairement présent aujourd'hui dans les courants de sobriété heureuse. Alors, pour conclure, merci aux organisateurs d'avoir permis cette rencontre. Car si l'écologie a une vertu, c'est d'accomplir sa propre nature.
La mise en relation pour former des écosystèmes vivants. Ce soir, ça marche aussi en théologie. En d'autres termes, je suis convaincu que l'écologie est un lieu privilégié pour faire avancer le dialogue interreligieux aux différents niveaux de ces exercices tant pratiques que spéculatifs. Je le vis déjà de manière intense en contexte œcuménique. Merci de me le faire goûter. Dans le cadre de ce dialogue avec Islam et merci de votre attention.
\end{comment}

\begin{comment}
 \label{Comment:MemoireISTR10} 
 ("gaia"),... Peut on être chrétien et Gaia ? ou autre idolâtrie potentielle mais non dénoncée ( 
\end{comment}


      



 
 
 \begin{comment}
   \paragraph{Reenchantement du monde}
 Djinn, esprits, Apocalypse
\paragraph{Récits}
 : religion : mythe explicite; vision du monde : toucher les religions sur le mythe du progrès.   
 \end{comment}


  



 

% -----------------------------------------------

\begin{comment}
\paragraph{En quoi cela concerne le dialogue inter religieux} et pas uniquement Christianisme et écologie / ...
D'abord, Règne de Dieu. ensuite, intuition comme pour le dialogue inter-religieux que le Christianisme porte dans sa matrice une universalité et une hospitalité aux questions du temps, qui l'ouvre peut être de façon privilégiée à ces questions. 
\paragraph{Dialogue inter religieux et écologie} hypothèse que les religions peuvent aider à un effort maintenant pour un gain plus long, donner du sens, Règne de Dieu. Voir comment le Christianisme peut être pertinent sur le sujet.
Dialogue inter-religieux dans ce contexte. Sociologie des religions : permettant de valider cette hypothèse.
Règne de Dieu et Ecologie
\end{comment}