\chapter{Scheuer : Théologie Comparative}

\mn{\cite{scheuer_vingt_2011}}

\paragraph{Jacques Scheuer} jésuite, 1942. D. Paris III sorbonne, Louvain la Neuve. Inde : études de théologie. Ils discutent avec les pionniers du dialogue inter-religieux
\begin{itemize}
    \item Sagesse et prière du monde
    \item un chrétien dans les pas de Bouddha
\end{itemize}

\paragraph{6 parties} 

\paragraph{Mouvement qui a 20 ans : théologie comparative} Une étude comparative précise. \textit{revenir, back \& forth}. 

\paragraph{Contexte : pluralité} \textit{mais il y a aussi la stérilité du discours englobant de théologie des religions}

\paragraph{Quelle problématique} {définir la théologie comparative}. faire connaitre une approche théologique dans le monde francophone. \textit{et ses fruits}

\paragraph{soucis confessionnel}




\paragraph{Critique d'Alberto} faire connaissance d'autres religions. intellectuel. 
\paragraph{mot retour} commence par une mobilité. \textbf{double retour à soi (220)} et à sa tradition.

\begin{quote}
    Paul Ricœur (1913-2005) analyse dans nombre de ses textes le grand détour que doit entreprendre le sujet pour revenir à soi. Avec cette démarche se trouve mis au jour un sujet qui représente davantage un point d'arrivée de l'effort philosophique que son point de départ. Mais qui est le sujet s'il n'est pas une vérité purement formelle ? Qui est celui qui se trouve en se comprenant et en s'interprétant ? Qui suis-je, moi qui dis « je » ?

À cette question, nous répondons spontanément en mettant en avant nos traits de caractère, nos façons d'être, bref, ce qui demeure fixe et nous identifie comme étant la même personne malgré les changements. Or cette forme d'identité a ses limites. Car, à strictement parler, la permanence dans le temps de ce que je suis ne permet pas de répondre à la question « qui suis-je ? », mais plutôt « que suis-je ? ». Ricœur, pour parer à ce glissement, propose de distinguer deux types d'identité : la première au sens de l'idem ou « mêmeté » (idem signifie « le même » en latin), et l'autre au sens de l'ipse ou du soi-même (on parlera alors d'« ipséité »). L'identité-mêmeté vaut pour tout objet qui subsiste dans le temps. Mais si tant est que le sujet n'existe pas simplement à la façon d'une chaise ou d'une pierre, son identité ne saurait se réduire à celle de l'idem. Elle renvoie plutôt à la dimension de l'ipséité qui se manifeste concrètement par le \textbf{maintien volontaire de soi devant autrui}, par la manière qu'a une personne de se comporter telle qu'« autrui peut compter sur elle ». Ce qu'illustre pour Ricœur la figure emblématique de la promesse dans laquelle j'engage d'abord qui je suis et non ce que je suis (c'est précisément au-delà de ce que je suis aujourd'hui que je m'engage à tenir parole).
\end{quote}
\paragraph{personalisme}; de Hegel, tout homme advient par le regard / dialogue de l'autre.




\paragraph{Religions traditionnelles} liquidité. est ce qu'on peut appliquer cela ? 

\paragraph{uniquement chrétien}


\section{Quelques représentants de la « théologie comparative »}

\begin{quote}
        Robert Neville
\end{quote}
\begin{quote}
        théologien anglican d’Oxford, Keith Ward
 
        révélation, la création, la nature humaine et la communauté [5] [5] Keith Ward, Religion and Revelation ; Religion and Creation ;…
\end{quote}
 \begin{quote}
        Francis Clooney. Jeune jésuite,
\end{quote}
 
 
 
 
 \section{Objectifs et méthode}
 
 \begin{quote}
        Dans notre monde globalisé et pluriel, les frontières entre les univers religieux se font poreuses et les identités, quand elles ne se durcissent pas dans une réaction de défense, deviennent fluides.
\end{quote}
Refus de la synthèse surplombante et rationalité "sur un petit espace".

 \begin{quote}
        Ce travail collectif s’arrête cependant en deçà de véritables reprises philosophiques ou théologiques fondées sur la comparaison.
\end{quote}
 
 \begin{quote}
        « La théologie comparative requiert des lecteurs, pas des consommateurs \sn{Fr. Clooney, Comparative Theology, p. 60..} » Il n’y a pas de raccourci magique, rien qui autorise à court-circuiter le labeur patient de cet apprentissage de la « lecture », de l’« acte de lire » dans un mouvement de va-et-vient (« back and forth »).
\end{quote}

\begin{quote}
        Francis Clooney recommande sans cesse de procéder « à petits pas »,
\end{quote}

\paragraph{Une définition de la religion} textes; \textit{transmission}
\begin{Def}[Religion - sens théologie comparative]
        Sans entrer dans d’interminables débats, il peut être éclairant de remarquer l’importance, pour la « théologie comparative » telle qu’elle s’est le plus souvent exercée jusqu’ici, de deux facteurs : \begin{itemize}
            \item 1) l’autorité d’un corpus de textes fondateurs (« Écritures », mais aussi « traditions » orales), auquel s’adossent le plus souvent des chaînes de commentaires et dont se réclament la pratique et la réflexion ;
            \item 2) une communauté qui transmet cette tradition et qui s’en nourrit.
        \end{itemize} 
\end{Def}


\paragraph{Le monde du Texte} pour Paul Ricoeur. Le risque, en ce concentrant sur le texte, c'est d'oublier le \textit{monde du texte}, comment ce texte se vit concrètement dans le monde. La tradition est vivante : prière, culte, institution,...

 \section{Théologie comparative et histoire des religions}
  \begin{quote}
 La théologie comparative hérite avec reconnaissance des méthodes comparatives développées dans le cadre des sciences humaines. Aux procédures mises au point au xixe siècle par l’histoire des religions puis, dans la première moitié du xxe, par l’école phénoménologique, elle associe les recherches et les méthodes plus récentes des sciences du langage, de la critique littéraire, de l’herméneutique, de la sociologie de la connaissance… Le recours rigoureux et inventif à ces disciplines garantit le caractère scientifique de la démarche théologique et l’originalité de sa production.
\end{quote}
 \begin{quote}
        La théologie comparative se distingue cependant de l’étude des religions comparées, par sa visée constructive et par son caractère confessionnel.
\end{quote}


 \begin{quote}
        La démarche, de soi, n’est certes pas inédite. Conduite avec rigueur et menée à son terme (un terme toujours provisoire), elle est cependant neuve. Elle n’est comparative et vraiment théologique, c’est-à-dire procédant de la foi et visant une intelligence plus plénière et cohérente de la foi, qu’à deux conditions et en deux temps. Premier temps et première condition : l’exploration, l’étude et la compréhension approfondie d’une manifestation religieuse relevant d’une autre tradition. Second temps et seconde condition : une comparaison fine entre ce qu’il découvre et ce qu’il comprenait de sa propre foi, non par curiosité superficielle, ni pour aboutir à une juxtaposition statique, mais en acceptant que la compréhension qui est la sienne de sa foi et de sa tradition se laisse interroger et peut-être vivifier par ce qu’il aura découvert. La démarche est théologale si elle est inspirée de bout en bout par la foi. Elle est théologique dans la mesure où elle est conduite avec la rigueur propre à cette science.
\end{quote}




 \section{Théologie comparative et théologie des religions}
 

 \begin{quote}
        Il s’agit donc d’un travail comparatif et d’une opération théologique. Cette double qualification répond, dans l’esprit des promoteurs, à la situation culturelle et spirituelle de pluralité, voire de pluralisme, que nous connaissons.
\end{quote}

\begin{quote}
        Les théologies chrétiennes donnent l’impression de formuler, sur la pluralité des religions, des appréciations massives (plus ou moins positives et optimistes) dont on n’aperçoit guère sur quelles données elles se fondent ni ce qui, dans les religions historiques concrètes, les justifie. La théologie comparative propose d’aborder ces mêmes religions d’une tout autre manière. Elle suggère de procéder par petits pas, par études « locales » (tel texte, telle image, tel rite…). Elle recommande d’être attentif au détail, au contexte précis, à l’éclairage particulier qu’un enseignement, un symbole, un sentiment reçoivent à telle étape d’une histoire religieuse ou spirituelle. Si la confrontation et la comparaison conduisent à formuler des conclusions ou des appréciations, ce sera toujours avec prudence, de manière révisable et en se défiant de toute généralisation.
\end{quote}
\begin{quote}
        Le véritable travail de théologie et de théologie comparative commence lorsque, après avoir parcouru une portion du territoire d’une tradition différente, nous amorçons le mouvement de « retour » — retour à soi et retour à notre propre tradition.
\end{quote}

\paragraph{dépassement inclusivisme et pluralisme}
\begin{quote}
        Parmi les praticiens chrétiens d’une théologie comparative, certains affichent des sympathies plus ou moins prononcées pour une position pluraliste tandis que d’autres se réclament clairement d’une forme d’inclusivisme [22] [22] C’est le cas de J. Fredericks dans The New Comparative…. Ils estiment cependant que l’heure n’est pas venue de trancher ce débat.
\end{quote}



\paragraph{critique de la théologie des religions} Alors que la théologie des religions peut dépasser l'inclusivisme et le pluralisme en regardant comment la théologie Chrétienne se renouvelle par la prise en compte des pluralités des religions irréductibles. 


 \section{Fruits de la théologie comparative}
 \begin{quote}
        Un premier fruit de la démarche consiste évidemment dans la découverte ou la connaissance plus fine et approfondie de telle facette d’une tradition religieuse autre. La sélection opérée s’explique pour une large part par des affinités culturelles et spirituelles
\end{quote}
 
 
 \begin{quote}
        Ce parcours et cette expérience nous placent en position d’obligé et nous font un devoir de reconnaissance, au double sens du mot : aveu de ce que nous n’avons pu que recevoir et gratitude à l’égard de ceux — proches et lointains, vivants et défunts — qui ont rendu possible cette rencontre
\end{quote}


\paragraph{Responsabilité du Théologien} \textit{citation de la Bible}
 \begin{quote}
        Francis Clooney, par exemple, rappelle régulièrement que son travail est celui d’un théologien chrétien et plus précisément catholique ; c’est dans cette perspective qu’il convient de lire ses considérations sur la responsabilité (responsibility, answerability, accountability) du théologien à l’égard de sa communauté
\end{quote}
 \begin{quote}
        Le croyant — et en particulier le théologien, dont la foi cherche à comprendre, à se construire dans la cohérence et à rendre raison — devra vérifier cette cohérence et la mettre à l’épreuve : l’apport venu de l’extérieur, mais qui rejoint probablement en lui une certaine affinité, peut-il participer à la cohérence toujours en train de se construire autour de l’axe du Christ ? La question n’est pas ici de partir du principe que le chrétien peut identifier et reconnaître, dans telle religion, des éléments de vérité et de bonté dont il devrait penser que, si et dans la mesure où ils sont vrais et bons, ils sont déjà présents dans le christianisme.
\end{quote}

\begin{quote}
        Le croyant chrétien qui pratique la théologie comparative fait en quelque sorte le pari (ou vit dans l’espérance) que son enracinement dans le Christ, loin de le fermer à des apports provenant d’autres traditions, lui permet de se laisser surprendre par ce qui est nouveau (pour lui) et de l’accueillir avec gratitude.
\end{quote}

\paragraph{Wittgenstein}
\begin{quote}
        Signalons du moins que les protagonistes font fréquemment appel à divers apports de la philosophie de Wittgenstein : réalité et langage, jeux de langage, différence entre les fonctions cognitive ou propositionnelle et performative ou praxéologique des expressions doctrinales… La prise en compte de ces apports imposerait de travailler sur des documents délimités et situés dans leur contexte : elle impose de renoncer aux vues totalisantes et aux jugements globaux dont la théologie des religions est coutumière
\end{quote}

\paragraph{Defamiliarisation}
\begin{quote}
        Une telle ouverture demeure bien limitée et soumise à une forme d’autocensure ; elle confirmerait le chrétien et sa communauté dans la persuasion qu’ils n’ont rien à découvrir, moins encore à recevoir. La question est plutôt de discerner des éléments susceptibles de compléter ce que nous connaissions déjà, ou d’identifier des déplacements dans cette cohérence toujours inachevée qui est la nôtre, ou encore de relire dans une autre lumière ce qui était familier, trop familier.
\end{quote}
\begin{quote}
        Vulnérable, donc. Mais jusqu’où ? Et comment ? 36Une première manière d’être affecté par la découverte d’une tradition autre, c’est de prendre conscience de facettes de notre propre tradition qui n’avaient guère retenu notre attention — peut-être pour la simple raison qu’elles faisaient partie d’un paysage trop familier. Reprenant une catégorie d’un théoricien de l’art, Hugh Nicholson parle de « défamiliarisation »
\end{quote}

 \section{Pour faire le point : quelques observations}
 
 \begin{quote}
        Une première constatation : cette théologie comparative a jusqu’ici été le fait surtout d’auteurs chrétiens.
\end{quote}
 \begin{quote}
        Une deuxième constatation : toutes les religions n’ont pas également retenu l’attention des chrétiens pratiquant la théologie comparative. L’hindouisme et le bouddhisme semblent jusqu’ici les plus fréquemment mis à contribution. L’islam ou les religions dites « traditionnelles » interviennent moins. D’une part, cela reflète probablement le degré de popularité de ces diverses traditions dans l’Occident d’aujourd’hui. D’autre part, par-delà les intérêts personnels et les affinités, la faveur que rencontrent (certaines formes choisies dans) l’hindouisme et le bouddhisme tient probablement à la combinaison de deux facteurs : leur éloignement relatif en termes doctrinaux et la disponibilité de vastes ensembles d’Écritures et de commentaires. Entre islam et christianisme, les écarts peuvent paraître moindres, mais avec des positions plus figées de part et d’autre.
\end{quote}
 

\section{méthode}

Ici, il s'agit vraiment de rester dans la définition. 

\paragraph{Plan intéressant}

\paragraph{Présenter les figures les plus importantes} Quand on approche un nouveau sujet, il peut être intéressant de commencer par présenter les protagonistes.

\paragraph{Présentation de façon positive} puis ensuite de façon négative : \textit{ce n'est ni du syncrétisme\sn{On va chez l'autre et on considère le PGDC}}. Dit l'essentiel dans l'introduction, il présente les protagonistes, il distingue. 

\paragraph{Approche par distinction} ce n'est pas de l'histoire de la religion, ni de la théologie de la religion.





 
















