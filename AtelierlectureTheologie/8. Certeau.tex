\chapter{Michel de Certeau}

\mn{24/11/22}


\paragraph{Biographie}

\paragraph{Vie } Michel de Certeau, né le 17 mai 1925 à Chambéry (France) et mort le 9 janvier 1986 à Paris, est un prêtre jésuite français, philosophe, théologien et historien. Il est l'auteur d'études d'histoire religieuse (surtout la mystique des xvie et xviie siècles), notamment avec son ouvrage La Fable mystique, édité en 1982, et d'ouvrages de réflexion plus générale sur l'histoire et son épistémologie, la psychanalyse, et le statut de la religion dans le monde moderne.

\paragraph{L'analyse du « braconnage culturel »}

L'un des apports principaux des travaux de Michel de Certeau se situe au niveau des pratiques culturelles qu'il relève dans la société contemporaine. Renversant le postulat alors mis de l'avant par Foucault dans Surveiller et punir, Michel de Certeau récuse la thèse selon laquelle les individus sont des êtres passifs et dépossédés et ne peut se résoudre à considérer les masses comme un tout homogène. À leur supposée inactivité, il met plutôt en l'avant leur fonction créative, laquelle serait cachée dans un ensemble de pratiques quotidiennes, qu'il appelle ruses, et qui s'opposeraient aux stratégies des gens au pouvoir ou aspirant à y accéder. Ces ruses subtiles, qui ne peuvent être détectées par les autorités, prendraient place dans des lieux communs. Par ce qu'il appelle une « pratique de l'espace », l'individu est en mesure de se composer un espace propre à partir de fragments de sens « braconnés » de part et d'autre.

\paragraph{Christus} revue qui s'intéresse au passé et au présent.

\paragraph{Pierre Favre et Surin}
\subsection{le livre}

\paragraph{Christus, le devoir missionnaire}

\paragraph{Octobre 1963   Concile}



\%%%%%%%%%% 
\section{la conversion du missionnaire}


\paragraph{Le titre} ce n'est pas "théologie de la mission" mais conversion du missionnaire.


\paragraph{préambule   tout laisser} épreuve, échec. 
\begin{quote}
    Partir, quitter les étroites frontières du pays qu'habite
déjà visiblement le Seigneur, tout laisser pour aller
annoncer à ceux qui l'ignorent la Parole que Dieu leur
 adresse et qui doit ouvrir leur existence ~ le missionnaire
s'en va ainsi, envoyé par l'Eglise, désireux de
n'avoir et de ne donner que cet Evangile auquel il voudrait
seulement ajouter le commentaire de sa vie.
En réalité, il emporte un lourd bagage. Il profite d'un
travail plusieurs fois centenaire; l'intelligence qu'il a
de la foi s'inscrit dans la tradition où s'est longuement
élaboré le langage qu'il reprend à son compte; sa sensibilité
même a trouvé sa forme et son épanouissement
dans un climat familial et culturel. Il transmet l'universelle
vérité, mais à travers l'expérience particulière qu'il
en a. et qui fait de lui, dans le pays où il se rend, un
étranger.
\end{quote}

cite Mt 25. 
\subsection{Les différentes approches du missionnaire}

\paragraph{Intransigeante} Transmet la culture

\paragraph{Accueillir les moeurs du pays} pas suffisant. le langage restera une barrière



\paragraph{incompréhension des oeuvres du misionnaires} la pauvreté peut ne pas être comprise. Et le service comme le \textit{rebouteau}


\%%%%%%%%%% 
\section{critique}

\paragraph{Michel de Certeau comme missionnaire type ?} pas un paradoxe à souligner les études longues et le besoin en ethnographie. 

\subsection{Quelle parole}

\paragraph{un témoignage muet ne sert à rien}
\paragraph{Une école de respect, sans préconçu} un nouveau monde disparate, où les pièces sont reliées 

\paragraph{Pour une empathie de l'autre y compris dans sa culture} La rencontre est justifiée car on ne peut rencontrer Dieu que par l'autre et l'étranger.  Il donne le sens. On n'est plus dans une mission parce que le Règne vienne. 



\paragraph{passer par le détour de l'analyse socio éthnographique} accepter que la culture ne nous est pas directement accessible


\paragraph{Un paradoxe} 

\begin{quote}
    S'il est ainsi éprouvé, ce qu'à la différence de l'éthongraphe, le missionnaire n'est pas un observateur, mais le témoin de l'\textit{unique} vérité. Il est donc lui-même  intéressé et blessé par la division. P 521
\end{quote}
Division liée à la distance culturelle alors qu'il annonce une unique vérité.

Il voyage entre deux fidélités, On est voyageur. On est étranger, on a besoin d'être accueilli. Jeu de dialectique. Remet en cause la défin

\paragraph{Rencontre du prochain, dans l'attente} 

\paragraph{discerner l'esprit} Dieu n'est pas partout, il faut discerner où il est. 

\%%%%%%%%%% 
\section{fécondité du texte}

\paragraph{La mission est une conversion à Dieu} Trente parle de la conversion  

\begin{quote}
    CHAPITRE 5
\textbf{Nécessité pour les adultes d 'une préparation à la justification. Son origine. }
1525 Le concile déclare, en outre, que la justification elle-même chez les adultes a son origine dans la grâce prévenante de Dieu par Jésus Christ 1553 , c'est-à-dire dans un appel de Dieu par lequel ils sont appelés sans aucun mérite en eux. De la sorte, ceux qui s'étaient détournés de Dieu par leurs péchés, poussés et aidés par la grâce, se disposent à se tourner vers la justification que Dieu leur accorde, en acquiesçant et coopérant librement à cette même grâce 1554-1555 . De cette manière, Dieu touchant le cœur de l'homme par l'illumination de l'Esprit Saint, d'une part l'homme lui-même n'est pas totalement sans rien faire, lui qui accueille cette inspiration qu'il lui est possible de rejeter, d'autre part, pourtant, sans la grâce de Dieu, il ne lui est pas possible, par sa propre volonté, d'aller vers la justice en présence de Dieu 1553 . Aussi, lorsqu'il est dit dans la sainte Écriture   “Tournez-vous vers moi et moi je me tournerai vers vous” (Za 1,3), notre liberté nous est rappelée ; lorsque nous répondons “Tourne-nous vers toi, Seigneur, et nous nous convertirons” Lm 5,21, nous reconnaissons que la grâce de Dieu nous prévient.
\end{quote}

\paragraph{Christ} deux citations mais paradoxales (Mt 24, 16)   ce sont les missionnaires à qui il s'adresse, il parle du Christ et non jésus.  Approche inclusiviste. 

\paragraph{problématique} Européenne. Idéal missionnaire et il y a une réalité. un choc, comment le penser ? pas slt un échec mais quelque chose de positive. partir de la réalité., 


\paragraph{rire}
\begin{quote}
    La conversion du missionnaire ou le rire partagé C'est par des rires que je souhaitais introduire mon voyage dans l'oeuvre de Michel de Certeau, à l'invitation de Luce Giard que je remercie, de tout coeur, pour ce signe d'amitié qui est aussi, pour moi, un honneur 1. Ce voyage dans les écrits du jésuite « braconneur » se fera à partir de mes terrains de recherche qui sont la culture missionnaire des XVIe et XVIIe siècles ainsi que les terres du Brésil. Le rire partagé De 1611 à 1615, les Français tentent d'implanter une France équinoxiale au nord du Brésil. Si l'expérience tourne court-les Français sont facilement délogés par une expédition portugaise-, il reste néanmoins deux magnifiques témoignages de cette expérience, écrits par les capucins français qui faisaient partie de l'expédition. Le livre d'Yves d'Evreux, écrit après un séjour de deux ans parmi les Indiens, est particulièrement riche sur l'interaction entre le missionnaire et les Indiens. Il contient notamment la transcription de longues « conférences » que le missionnaire aurait eues avec différents Indiens, principaux et sorciers. Ce nom de « conférence » n'est pas choisi au hasard par le capucin   il renvoie aux disputes théologiques auxquels se livrent, en France, protestants et catholiques depuis qu'ils ont posé les armes. Le livre s'achève sur une de ces conversations, entre La Vague, un des chefs principaux de la tribu de Comma, et Yves d'Evreux. Le chef indien présente son fils âgé de vingt ans, afin que le capucin lui administre le baptême car, selon son père, le jeune sauvage sait déjà parler le français. Le jeune homme se met alors à réciter quelques phrases dans un français tout déformé. Pour préserver toute la saveur de la scène, écoutons Yves d'Evreux la raconter   
    \begin{quote}
        Ayant dit ces paroles, il fit signe à son fils qu'il s'approchast   puis il luy commanda de raconter tout ce qu'il savait de François. J'avois bien de la peine à me contenir de rire, \& ne pouvois iouyr de mon Truchement, tant il estoit transporté de la passion de rire sur la simplicité de ce personnage   neantmoins ie le retins luy faisant faire son excuse sur les singeries d'un petit Perroquet que i'avois, à fin que ce bon homme ne pensast que ce fust de luy qu'il rioit. Ce ieune homme son fils me recita la Doctrine qu'il avoit propre, disoit son père, & suffisante à recevoir le Baptesme en cette sorte   Bonioure, monseïeur, come re vo reporteré vou. Ben monseïeur, à vostre service, volè vou mangeare, Oy   du pain, peïsson, char, may teste, men chapeyau, pourpuin, Chausse, Chamise. Ie ne peus en entendre davantage, si ie n'eusse voulu debonder   Ie luy fis donc dire, que c'estoit assez, que ie voioy bien par là, qu'il n'avoit point perdu son temps. […] 1
    \end{quote}
     Mes remerciements vont également à Denis Pelletier pour ses remarques sur la version orale de mon texte.
\end{quote}


 \begin{quote}
     Michel de Certeau\mn{Charlotte de Castelnau-L’Estoile. La conversion du missionnaire ou le rire partagé. Luce Giard. Michel de Certeau Le voyage de l’œuvre„ pp.181-193, 2017. ￿halshs-02042413} et la conversion du missionnaire Ce morceau savoureux d’écriture missionnaire nous plonge dans de multiples thèmes liés à l’œuvre de Michel de Certeau   la littérature de voyage et en particulier celle qui a trait au Brésil dont il était un spécialiste comme les notes de son article sur Léry le montrent 6,  l’Ethnographie, écrite avec un trait d’union, et sa réflexion sur l’oralité qui est au cœur de tout processus de connaissance sur les mondes sans écriture. Certeau invitait à une archéologie de l’ethnologie qu’il n’aura finalement pas menée à bien. Enfin, ces extraits renvoient à la richesse des sources religieuses de la France du XVIIe siècle dont Certeau était un interprète lumineux. Il a su les faire résonner comme des témoignages d’un monde qui commence à douter du « système religieux » ancien 7. La question de la mission auprès des « sauvages » ne constitue cependant pas un objet central de l’œuvre de Michel de Certeau, au même titre que la littérature mystique ou la question de la sorcellerie. Ses écrits sur la mission ne sont pas très nombreux, même s’ils sont souvent inspirants. Ainsi son article « La réforme de l’intérieur au temps d’Aquaviva 1581-1615 » 8 est un texte fondamental sur les difficultés internes à la Compagnie de Jésus, au tournant du XVIIe siècle. Certeau y analyse les inquiétudes que ressentent les jésuites, face à la croissance fulgurante de l’ordre et à la multiplicité des tâches auxquelles ils font face. Les jésuites dénoncent « le danger de l’expansion « au dehors » », par opposition au monde du dedans, au monde des collèges, plus rassurant. Une enquête portant sur les remèdes à apporter à ces difficultés structurelles est ainsi lancée par le Supérieur général Aquaviva. En choisissant de donner pour titre à ma thèse l’expression employée par le général lui même pour parler des jésuites du Brésil, « Les ouvriers d’une vigne stérile » 9, j’avais cherché à souligner ces difficultés, longtemps camouflées par une historiographie missionnaire, triomphante ou dénonciatrice, et à souligner l’apport heuristique d’une reprise distanciée et critique des « catégories indigènes », étant entendu que les indigènes étaient ici les jésuites. En fait, Certeau s’est fortement intéressé à la question missionnaire mais dans une partie de son œuvre qui est moins connue des historiens. Il a consacré à l’activité missionnaire le chapitre 4 « La vie commune » dans L’Étranger ou l’union dans la différence, publié en 1969, ouvrage 
 \end{quote}

 \paragraph{Exculturation} de Daniele Hervieu Léger. Dans ce cadre, la réflexion sur \textit{la communauté catholique} à créer pour s'acculturer est une piste riche pour nous  
 \begin{quote}
     Or c'est tout
un peuple qui doit devenir une église; c'est une communauté
catholique qui doit naitre.  n faut donc que
le témoignage apostolique aide le converti ou le sympathisant
à se situer par  rapport à lui-même d~s son
peuple et lui fournisse non un langage tout fait ou la
carte -chrétienne d'une autre région, mais le moyen de
trouver lui-même dans sa foi comment la révélation
donne leur sens et leur orientation aux chemins de son
pays. Le Christ n'est-il pas le  révélateur des originalités
humaines? Il secoue une collectivité dès qu'il y est
reconnu par quelques-uns, mais c'est que, brusquement;
sa lumière les tire de l'ombre; peu à peu, elle touchera
les coins les plus  reculés du paysage, éveillant à leurs
couleurs ses collines et ses bourgs. Ce que la foi  recrée,
c'est l'homme lui-même. Aussi chaque élément d'une
culture doit-il recevoir un éclat qui corresponde à celui
du lieu où s'est posé le premier rayon. En  réalité, ce
paysage n'est pas un espace, mais une histoire. La révolution
chrétienne se traduit donc, ici comme tant d'autres
fois, par une valorisation nouvelle de tout ce qui, dans
le présent, se  réfère au passé - telles les coutumes, les
pensées et les mots qui gardent au converti l'immense
et lointaine présence de ses pères. Le nouveau chrétien
doit pénétrer avec Jésus dans ce passé encore vécu,
éclairer de sa foi les régions de sa mémoire et comprendre
ainsi la tradition où jusque là il n'avait pas vu
qui lui parlait. Ressuscité, le Christ «' descendit aux
enfers » pour convertir à sa nouveauté l'histoire dont
 il était issu; aux chrétiens qu'il fait renaître, il demande
de continuer avec lui l'invention de leur passé et de
travailler à cette récapitulation que le Credo leur présente
comme un « article >> et un acte présent de leur
foi 1.
Comment le missionnaire peut-il les y aider? Non pas
en faisant ce travail à leur place. Ce n'est pas lui qui suscite la vie des siens, mais l'Esprit en eux. n n'a donc
pas à définir de l'extérieur un langage et des institutions
qui. combineraient des éléments extraits de deux traditions,
la sienne et la leur. Seuls ceux qui l'expérimentent
peuvent savoir comment l'Esprit refait du dedans leur
univers mental, sentir comment se développe une notion
restée jusque là inerte, ou déceler l'incompatibilité
d'une autre qu'on aurait pu croire, intellectuellement,
plus proche de l'énoncé dogmatique. C'est Dieu qui
convertit, lentement et de l'intérieur; c'est lui qui les
attire, à partir de ce qu'ils sont. Mais grâce à l'attention
qu'il porte à cette naissance, le missionnaire acquiert
l'intelligence spirituelle de l'histoire à laquelle il appartient
lui-même par son passé. ll comprend mieux comment
la foi, qui n'existe jamais à l'état pur, a inspiré
du dedans l'immense élaboration qu'il identifiait d'abord
à la révélation et dont il était porté ensuite à n'envisager
que l'aspect culturel. Ce qui se passe autour de lui,
dans ce peuple récemment touché par la foi, est, quoique
différemment, ce qui s'est fait et continue de se produire
dans le peuple qu'il avait cru quitter. Et lorsqu'il
« descend » lui aussi dans sa propre tradition, lorsqu'il
y perçoit mieux comment l'Eglise s'est frayé une voie
difficile entre l'adhésion à l'Evangile et la fidélité à l'histoire
paléotestamentaire, comment elle a soulevé dans
son pays la vieille pâte d'un ferment inattendu et tiré
du fonds national les richesses nouvelles désormais
indissociables de la culture locale, il trouve dans cette
longue et prodigieuse expérience de quoi éclairer les
nouveaux chrétiens sur leur tâche et discerner chez eux
la même OEuvre ou des tentations semblables.
C'est bien là, de plus en plus, son rôle essentiel. Il doit
méditer constamment l'histoire de l'Eglise à la lumière
de celle qui se déroule autour de lui. A propos des mots
à utiliser dans le catéchisme, des formes à trouver dans
la liturgie, des modifications à introduire dans -les coutumes,
il se  retourne vers l'expérience passée, non pour
la répéter, mais pour en  retrouver le sens, pour y  reconnaître
l'intuition créatrice et pour éclairer ainsi le besoin
ou l'invention d'aujourd'hui. Il guide l'élaboration nouvelle,
mais comme le « directeur spirituel » agit avec ses
« fils spirituels», au nom d'une expérience qui n'est pas a confirmée. Il peut les mettre en garde lorsqu'ils veulen!
briser avec leur pa~sé pour s'en tenir au lan~a~e assure
qu'ils recoivent du dehors, ou lorsque entraines par la
force de; éléments traditionnels, ils tendent à ramener
la vérité catholique à des conceptions ethniques et particulières;
mais il en est capable parce que leurs recherches
.l'aident à découvrir ce que la vie .de l'Eglise lui enseigne
de !'Esprit et, dans cette mesure même, saisissant la
portée de l'exclusion ou de l'adaptatio~ dont ils sentent
l'urgence, il soutient le mouvement sp i.ntuel qm se dessine
en eux et le distingue de ses contrefaçons 1
• Ils
sollicitent précisément cela de lui, afin qu'au cours d'une
incessante confrontation ils arrivent à trouver l'i,mmuable
dans la forme qui leur est propre et à discerner, daps 1~
prodigalité de ses oeuvres, l'action d'un même Espnt. SI,
contrairement à une illusion trop répandue, « tout progrès
culturel est fonctio 11. d'une ,coalition ent~e les
cultures » et si « cette coahhon est d autant plus feconde
qu'elle s'établit entre des cultures plus différenc~ées 2 »,
l'intervention du missionnaire doit promouvoir une
« Renaissance » chrétienne beaucoup plus profonde, provoquée,
chez les chréti~ns, par la  ré°?~ion de leur p~~s~
en fonction de leur foi et, de son cote, par une fidehte
plus exigeante à son propre christianisme.
La venue de son Maître, que rapôtre, émerveillé, peut
contempler dans sa petite communauté, il ,la dis?~rne av~.c
un pareil émerveillement dans le passe chrehen qu il
croyait connaître, et cette découverte simultanée lui ".aut
d'assister le Roi qui naît dans fa nuit d'un peuple JUSqu'alors
fermé sur lui-même. n doit donc aussi, dans ce
but, se rattacher plus étroitement à son évêque et à
l'enseignement traditionnel,  retrouver dans les d~~isions
de la hiérarchie et dans le corpus doct rmal ce qu 11 pensait
pouvoir néalige r - tant de choses qui reprennent
sens, font question ou sont renouvelées en fonction de l'oeuvre présente. En collaborant à l'édification d'une
nouvelle église, il se convertit à l'Eglise. 
L'évolution qu'il constate autour de lui ne lui conseille
pas une autre attitude. La diffusion des techniques
estompe en effet les différences culturelles et  répand le
langage universel d'un savoir plus abstrait et plus efficace.
L'industrialisation est un fait qui s'étend, gagnant
peu à peu les villages et déracinant les hommes. Mais si
le progrès procure aux petites sociétés isolées d'immenses
moyens de production et de transformation économique
s'il les met dans une situation nouvelle à l'égard du milie~
naturel et s'il détériore par là même les structures sociales
et religieuses répondant à un ancien mode de vie il ne
fournit pas, avec la perspective de la. prospérité, 'l'indication
du sens à lui donner. Il aiguise au contraire, chez
les plus lucides de ses bénéficiaires, le sentiment d'un
déséquilibre et le besoin de retrouver ou de préserver leur
âme en prenant le blen de l'ouvrier ou la blouse de l'ingénieur.
A titre d'exemple, on peut citer ces romanciers
noirs dont Ies personnages achèvent le cycle de leurs expériences
occidentales par un retour au village natal 1 • Le
retour à la « brousse », paradis symbolique, p rincipe et
terme de l'aventure humaine, traduit moins un refus
des nouvelles acquisitions que le sentiment d'une absence
intérieure; sans renoncer à la puissance qu'il a gagnée,
l'évolué  êve et cherche désormais ce que lui a fait perdre
son ennchlssement. Et sa quete le reconduit mentalement
dans les lieux où il avait appris à vivre,
Peut-être plus accentué chez l'évolué, ce pèlerinage aux
sources ne lui est pas propre. Lui aussi, comme maître et
prisonnier du même pouvoir, l'occidental se crée des îlots
de vie « naturelle » - des loisirs rustiques et des distractions
artisanales - et il entreprend un nouvel inventaire
de sa culture, afin de compenser par là, au moins partiellement,
ce qu~ ne lui donne plus son activité professionnelle.
Retour a la nature, retour au passé   l'occidental et
l'évolué juxtaposent ainsi aux instruments qui leur
deviennent communs le symbole des origines qui leur sont
propres. Sous la dualité qui prend, de part d d'autre, des
formes différentes, un même problème est posé   l'universalisation des moyens techniques laisse à chacun le soin
de chercher ailleurs, ou en soi, l'originalité et le sens
de l'usage qu'il en doit faire.
Le missionnaire reconnait- là, mais plus grave, plus
complexe, une tension analogue à celle qu'éprouvaient
les convertis partagés entre la  fidélité à leur tradition
et l'adhésion à la prédication d'un étranger. Il
voit mieux ainsi que son apostolat le rattache étroitement
aux efforts des prêtres restés dans sa première patrie.
Dans le lieu où il est envoyé, il travaille pour sa part à
une tâche commune. Car l'Eglise opère partout la même
oeuvre réconciliatrice   signe entre. les nations, elle
témoigne de la vérité qui est dans la créature son Origine
vivante, appelant les hommes à découvrir en eux la p~ésence
qui les rassemble du dedans pour un nouvel avenir.
Cette mystérieuse naissance humaine de Dieu, le missionnaire
doit encore l'annoncer aux évolués et la respecter
en eux. n ne décide donc pas à l'avance si le style ancien
ou le style moderne est plus apte à exprimer la vie de
l'Esprit; si Dieu est mieux chez lui dans le village ancestral
ou dans les faubourgs de la ville. Qu'en sait-il? Il
 risque toujours d'adopter comme définitives des formes
déjà périmées ou de considérer comme nécessaire u?langage
qui, pour être plus moderne, ne correspondrait
pas à l'expérience des fidèles. U est toujours tenté d'identifier
à ce qu'il connaît, d'un côté ou de l'autre, la vie
originale qui se donne peu à peu son visage propre. Il
n'a donc pas à fixer leur langage chrétien aux membres
de la société qui s'industrialise. Mais il reste chargé de
leur faire entendre de quelle nature est l'oeuvre de l'Esprit,
et d'y être suffisamment attentif pou r la guider
selon les critères qu'elle l'oblige à mieux discerner dans
la tradition de l'Eglise.
 \end{quote}