\chapter{Bilan}

\mn{le 8/12}

\section{nos réactions - Bilan sur le parcours}
ce que nous avons appris sur la méthodologie


\subsection{Méthode}

\paragraph{Dialectique} Partir de la réalité / fond de notre foi.

\paragraph{Déterminer le genre littéraire} et le contexte. Comment l'écriture nourrit le texte.

\paragraph{comment l'auteur nous séduit} voir les références, les images, les arguments. 

\paragraph{Ne pas chercher La solution} elle nous dépasse toujours mais on peut enlever les mauvaises solutions qui nous empechent d'avancer. La théologie systématique est utile car elle permet de voir la \textit{cohérence} mais elle ne les résout pas. Une certaine symphonie de la théologie systématique mais \textit{inachevée}. 

\paragraph{créativité} non dans une pensée nouvelle mais dans une lecture nouvelle d'auteurs importants.

\paragraph{Accueillir l'Ecriture dans la grande tradition} On écoute d'abord. Le devoir M1 insiste sur cette partie.

\paragraph{Intelligence de la foi } Le devoir M1 insiste sur cette partie.
\paragraph{fécondité de la Foi} il faut qu'il y ait du fruit. Elle doit nous aider à vivre de la parole de Dieu. il faut penser les trois pôles : accueil, intelligence et fécondité.

\paragraph{méthode quand on lit un texte non théologique} comme une lettre de Saint François. Comme le dit Ricoeur, ce genre de documents nous ouvre sur les questions théologiques. A travers les images, quelle théologie sous jacente ? Ex : saint François Xavier : lire les textes des théologiens qu'il cite. Mais c'est un gros travail qui nous dépasse la plupart du temps. 

\paragraph{Herméneutique} Plus modestement, regarder le vocabulaire ("croissance de la mission", "Enfer"...). Le travailler à la façon de Paul Ricoeur, l'hermeneutique, \textit{quel est le monde du Texte}. \mn{Temps et Récit}


\subsection{fond}

\paragraph{}
\paragraph{Le risque de la singularité} c'est de prendre un fragment de l'autre religion et réduire l'autre à cela.

\paragraph{Division féconde} de Duquoc. Provocateur. On a tendance à faire l'unité en gommant les différences.



\section{Où en est on sur notre licence}

