\chapter{Introduction}
\mn{Atelier de lecture
Théologie des religions et mission
Licence canonique et Diplôme Supérieur
 
P. Xavier Gué }
\section{Programme}

double démarche : 
\begin{itemize}
    \item \textbf{forme} : une démarche scientifique, méthodologique à présenter un sujet. En scientifique, la forme présente une certaine rigueur qui montre qu'on suit une méthode. Un certain protocole.
Il s'agit de savoir rédiger. Construire une argumentation en bon français. 
Savoir rédiger les notes de bas de page. fonder sur les sources.
Apprendre à faire une bibliographie. LA géographie d'une question : les grands textes, savoir se repérer : dictionnaire de théologie.
    \item \textbf{fond} : ré-appropriation des éléments théologiques de base sur la théologie des religions. On peut croiser les démarches (histoire de l'art, oecuménisme,...) mais il doit y avoir de théologie.
\end{itemize}

\paragraph{Epistémologie de la théologie} Référence à l'écriture fait par exemple partie de la théologie \mn{Question sur l'actuariat de l'épistémologie de l'actuariat}. 


\paragraph{1ère séance : la démarche scientifique}    15 septembre 2022
\paragraph{2ème Séance : les ressources de la bibliothèque}    22 septembre 2022 
\paragraph{3ème séance : Pourquoi le dialogue ?}    29 septembre 2022 


Jacques DUPUIS, « Le dialogue interreligieux, praxis et théologie », dans \textit{Vers une théologie chrétienne du pluralisme religieux}, Paris, Cerf, 1997, p. 543-582.
Tous les étudiants doivent lire le texte de Dupuis et repérer à la lumière de la fiche « Pour lire et exposer un article » les éléments attendus de lecture et d’analyse. 
L’exposé sera réalisé par le directeur de l’atelier. 
Il s’en suivra une discussion entre étudiants. 

\paragraph{4ème séance : La théologie des religions }   6 octobre 2022

Guillaume
Lecture : Tous les étudiants doivent avoir lu les chapitres 1 à 2 de l’ouvrage de Rémi CHÉNO, \textit{Dieu au pluriel}, Paris, Cerf, 2016. 
Exposé : chapitres 1 à 2,  exposé de 15 minutes. 
Travail en séance : Après l’exposé, chaque étudiant est amené à : 
\begin{itemize}
    \item 	identifier deux aspects positifs de l’exposé : sur la forme, sur le fond (restitution d’une théologie, argumentation, contextualisation). 
\item	Identifier les difficultés ou les limites de l’exposé (ce qui n’a pas été vu, ce qui manque de clarté dans la formulation, dans les précisions conceptuelles, dans la critique de l’auteur…)
\item	demander des clarifications sur les concepts ou expressions théologiques des chapitres présentés par l’étudiant. L’exposant doit être en mesure de répondre aux questions des autres sur le texte qu’il a présenté. 
\end{itemize}


\paragraph{5ème séance : La théologie des religions (suite) }   13 octobre 2022

Genevieve
Lecture : Tous les étudiants doivent avoir lu les chapitres 3 à 4 de l’ouvrage de Rémi CHÉNO, Dieu au pluriel, Paris, Cerf, 2016. 


\paragraph{6ème séance : Théologie comparée versus théologie comparative}    20 octobre 2022 

Alberto
Lecture : Tous les étudiants doivent avoir lu l’article de Jacques SCHEUER, « Vingt ans de ‘‘Théologie comparative’’. Visée, méthode et enjeux d’une jeune discipline », Nouvelle revue théologique 133, 2011, p. 207-227.

\paragraph{7ème séance : État des lieux de projets de mémoire – Considérations méthodologiques sur une problématique de master}    27 octobre 2022 


Mémoire de M1 : 1er état des lieux : quelles pistes envisagées ? thème ? auteur ?  
 
\paragraph{8ème séance : Unicité du Christ et pluralisme religieux : la symphonie différée de Christian Duquoc}    10 novembre 2022

Guillaume
 À Abū Dhabi, le pape François a signé avec Ahmad al-Tayeb un Document sur la fraternité humaine. Il ouvre à une manière renouvelée pour penser les différences religieuses comme providentielles. Comment articuler l’unique médiation du Christ et le pluralisme religieux ? 
Texte à lire : Christian DUQUOC, « La division féconde », dans L’unique Christ. La symphonie différée, Paris, Cerf, 2002, p. 201-255. 
\paragraph{9ème séance : Théologie de la mission : modèles et défi (1)}    17 novembre 2022

Genevieve
Texte à lire : Michel DE CERTEAU, « La conversion du missionnaire », Christus n°40, octobre 1963, p. 514-533.

\paragraph{10ème séance : Théologie de la mission :  (2)}    24 novembre 2022

Guillaume
Texte à lire : Claude GEFFRE, « La rencontre du christianisme et des cultures », Revue d’éthique et de théologie morale. Le Supplément, n°192/1, 1995, p. 69-91. 


\paragraph{11ème séance : La théologie des religions du Père Dupuis}   1er décembre 2022

Alberto
Texte à lire : Léonard SANTEDI KINKUPU, « Eléments d’une nouvelle évangélisation en Afrique », dans Les Défis de l’évangélisation dans l’Afrique contemporaine, Paris, Karthala, 2005, p. 87-120.



\paragraph{12ème séance : Bilan de l’atelier – état des lieux des sujets de M1 et formulation d’une problématique}   8 décembre 2022




\section{Les trois éléments en théologie}

\paragraph{Auditus fidei} à l'écoute de la parole de Dieu. On ne parle pas de 0 mais de la Révélation : Ecritures Saintes, Patristiques. 

\paragraph{Intellectus fidei} l'intelligence de la foi. Tout ce que l'on reçoit doit être cohérent. Si ce n'est pas cohérent, si cela heurte la raison, on ne pourra pas en vivre ni la transmettre. Certes, le mystère nous dépasse, mais il doit y avoir une cohérence, cohérence adaptée à la \textit{culture}. 

\paragraph{feconditas fidei} la fécondité de la foi, en sorte que notre réflexion théologique (ce que nous produisons) n'est pas un savoir d'érudit mais permettre au peuple de Dieu de bénéficier la parole de Dieu. la Théologie n'est pas là pour elle même mais de mieux accueillir la parole de  Dieu. 


 \chapter{la démarche scientifique en théologie}


\section{Déterminer un sujet}

\paragraph{Idée d'un thème} Pourquoi le sujet m'intéresse ?  Une question enfouie. Intérêt de justifier le choix. Essayer de lire les texte sur le thème pour délimiter un thème.

\paragraph{Thème délimité} Dans la contrainte de 30 pages ou même d'un doctorat, sous-thème, 


\paragraph{Délimiter un corpus} 

\paragraph{Sujet Validé} quelle est la problématique ? Voir dans ce cadre la fécondité du sujet. C'est bien le sexe des anges mais en quoi cela peut être fécond.


\paragraph{Une bonne problématique suscite le plan } Il faut faire des choix et pas tout dit. "je me limiterais à ".

\section{les Normes d'un mémoire de théologie}

\paragraph{des repères} des signaux qui permettent de nous repérer dans le texte. "une charité pour le lecteur" (!)

\paragraph{Manuscrit} un grammage de 80 g.  Format A4. Marges : 2.5 en haut/bas/gauche/droite. Interligne : 1.5. Times New Roman. 12. Recto Verso. Pagination en bas, soit au centre, soit sur les cotés. Biographique : en Italique pour un ouvrage, pour les expressions dans une autres langues. 

\paragraph{Dissertation} Page de garde / Engagement de non plagiat / liste des abbréviations utilisées. 

\paragraph{Introduction} Annexes / Bibliographie / Table des matières. 

\paragraph{Couverture a une norme}. Couverture répétée dans la première page. 

\paragraph{Le plan}. 2/ 4 parties sur 30 pages. Table des matières : sous partie et pages.

\paragraph{Introduction} mettre en valeur la problématique. Situer dans un contexte, intérêt de la problématique. Annoncer le plan. les oeuvres sur laquelle on a travaillé. Donner envie de lire.

\paragraph{Corpus} Dans les parties doivent être introduites, une conclusion. guider le lecteur. 

\paragraph{les citations} pour bien montrer qu'on a fait des recherches. les citations plus de 4 lignes, taille 11, retrait de 0.5 pas de guillemets dans ce cas. 

patronyme : majuscules. 

\paragraph{le plagiat} Il y a la logue du vol mais aussi se situer dans une logique d'interprétation : on cite quelqu'un mais on le situe dans une problématique et on l'interprète.  Thomas d'Aquin, Albert le Grand sont de grands interprétateurs. 


\section{Plan de l'atelier}


\section{Plan du Cours}


Atelier de lecture
Théologie des religions et mission
Licence canonique et Diplôme Supérieur
 
P. Xavier Gué 
  
\paragraph{1ère séance} : la démarche scientifique // 15 septembre 2022

\paragraph{2ème Séance }: les ressources de la bibliothèque // 22 septembre 2022

\paragraph{3ème séance : Pourquoi le dialogue ?} // 29 septembre 2022 
Jacques DUPUIS, « Le dialogue interreligieux, praxis et théologie », dans Vers une théologie chrétienne du pluralisme religieux, Paris, Cerf, 1997, p. 543-582.
Tous les étudiants doivent lire le texte de Dupuis et repérer à la lumière de la fiche « Pour lire et exposer un article » les éléments attendus de lecture et d’analyse. 
L’exposé sera réalisé par le directeur de l’atelier. 
Il s’en suivra une discussion entre étudiants. 

\paragraph{4ème séance : La théologie des religions} // 6 octobre 2022
Lecture : Tous les étudiants doivent avoir lu les chapitres 1 à 2 de l’ouvrage de Rémi CHÉNO, Dieu au pluriel, Paris, Cerf, 2016. 
Exposé : chapitres 1 à 2,  exposé de 15 minutes : Guillaume
Travail en séance : Après l’exposé, chaque étudiant est amené à : 
•	identifier deux aspects positifs de l’exposé : sur la forme, sur le fond (restitution d’une théologie, argumentation, contextualisation). 
•	Identifier les difficultés ou les limites de l’exposé (ce qui n’a pas été vu, ce qui manque de clarté dans la formulation, dans les précisions conceptuelles, dans la critique de l’auteur…)
•	demander des clarifications sur les concepts ou expressions théologiques des chapitres présentés par l’étudiant. L’exposant doit être en mesure de répondre aux questions des autres sur le texte qu’il a présenté. 
 
\paragraph{5ème séance : La théologie des religions (suite)} // 13 octobre 2022
Lecture : Tous les étudiants doivent avoir lu les chapitres 3 à 4 de l’ouvrage de Rémi CHÉNO, Dieu au pluriel, Paris, Cerf, 2016. 
 Exposé : Geneviève
6ème séance : Théologie comparée versus théologie comparative // 20 octobre 2022 
Lecture : Tous les étudiants doivent avoir lu l’article de Jacques SCHEUER, « Vingt ans de ‘‘Théologie comparative’’. Visée, méthode et enjeux d’une jeune discipline », Nouvelle revue théologique 133, 2011, p. 207-227.
Exposé : Alberto

\paragraph{7ème séance : État des lieux de projets de mémoire} – Considérations méthodologiques sur une problématique de master // 27 octobre 2022 
Mémoire de M1 : 1er état des lieux : quelles pistes envisagées ? thème ? auteur ?  
 
\paragraph{8ème séance : Unicité du Christ et pluralisme religieux} : la symphonie différée de Christian Duquoc // 10 novembre 2022
 À Abū Dhabi, le pape François a signé avec Ahmad al-Tayeb un Document sur la fraternité humaine. Il ouvre à une manière renouvelée pour penser les différences religieuses comme providentielles. Comment articuler l’unique médiation du Christ et le pluralisme religieux ? 
Texte à lire : Christian DUQUOC, « La division féconde », dans L’unique Christ. La symphonie différée, Paris, Cerf, 2002, p. 201-255. 
Exposé : Guillaume

\paragraph{9ème séance : Théologie de la mission} : modèles et défi (1) // 17 novembre 2022
Texte à lire : Michel DE CERTEAU, « La conversion du missionnaire », Christus n°40, octobre 1963, p. 514-533.
Exposé : Geneviève

\paragraph{10ème séance : Théologie de la mission} :  (2) // 24 novembre 2022
Texte à lire : Claude GEFFRE, « La rencontre du christianisme et des cultures », Revue d’éthique et de théologie morale. Le Supplément, n°192/1, 1995, p. 69-91.
Exposé : Mohammed

\paragraph{11ème séance : La théologie des religions du Père Dupuis} // 1er décembre 2022
Texte à lire : Léonard SANTEDI KINKUPU, « Eléments d’une nouvelle évangélisation en Afrique », dans Les Défis de l’évangélisation dans l’Afrique contemporaine, Paris, Karthala, 2005, p. 87-120.
Exposé : Alberto

\paragraph{12ème séance : Bilan de l’atelier – état des lieux des sujets de M1 et formulation d’une problématique} // 8 décembre 2022



\section{Bibliothèque}
 
 
 \textbf{IndexReligiosus} : très fort index.  Mais pas de document écrit.
 
 \textbf{Atla} Base religieuse
 
 \textbf{JStor} : propose gratuitement les articles \mn{voir livre Risques et accidentologie}

 \textbf{Sources Chrétienne} :
 
 \textbf{Europresse} toutes les revues.
 
 
 \begin{Synthesis}
 voir la section \textit{se former}
 
 \end{Synthesis}

Voir Zotero, pour gestion des références.


\section{Forum des Masters}

\mn{le 15/12/22}

\paragraph{Decontextualisation et reconstextualisation}
On a toujours en tête qu'il faudrait faire des introductions sur le sujet (Gnose,...) alors qu'on n'a pas le temps en Master. il faut directement aller sur le sujet. 
\begin{Ex}
    Pourquoi Irénée s'intéresse à la Gnose ? Et non, la Gnose, super long.
\end{Ex}

\paragraph{Caractériser le problème contemporain} Et l'argument biblique / Patristique ne vient que dans un second temps. 


\paragraph{Quel est l'angle d'attaque ? }IL ne s'agit pas d'ajouter un livre d'érudition en plus.

\paragraph{La question de départ} doit permet d'éliminer les questions à ne pas traiter.
\begin{Ex}
    Si ma question est "Individualisme et Eglise", il ne faut pas traité tous les thèmes ecclesiologiques d'Irénée.
    Faire des choix.
\end{Ex}

 

\begin{Ex}[théologie narrative]
Joseph Moingt  \mn{\textit{L'homme qui venait de Dieu} Joseph Moingt} raconte son expérience et une mise à distance qui raconte une expérience de théologie narrative : comment il a jeté son cours pour partir sur autre chose parce que ce n'était pas audible.

    
\end{Ex}

\paragraph{Initiation à la vie Chrétienne} alors qu'on a réduit à l'initiation à la Foi Chrétienne. 
\begin{Ex}
    Scoutisme
\end{Ex}

\begin{Ex}[Vocation universelle]
    LG 31
    Chez Thérèse, il y a l'idée que toute vocation est universelle au sens qu'elle touche l'humanité.
    Pas tout à fait la même chose avec le fait que tout homme est appelé dans sa singularité.
    
\end{Ex}

\paragraph{difficulté de partir d'un philosophe} Ici Jean Louis Chrétien et sa phénoménologie chrétienne, appel et réponse. Mais il faut faire le lien théologique. Ex André Wenin : anthropologie biblique.

\paragraph{Mettre clairement l'auteur source} Ex : si c'est Thérèse, mettre d'abord Thérèse puis interrogée par Jean Louis Chrétien.

\paragraph{Le plan de la recherche n'est pas le plan de la présentation} La présentation ne doit pas tromper le lecture. 