\chapter{Sujets de mémoires ISTR}

\paragraph{Partir d'un texte} 
Se concentrer sur un auteur : laudato Si

\paragraph{Reenchantement du monde}
 Djinn, esprits, Apocalypse

\paragraph{Récits}
 : religion : mythe explicite; vision du monde : toucher les relions sur le mythe du progrès. 


\section{Dialogue inter religieux et changement climatique}

\paragraph{Mise en problématique autour de l'idolâtrie}
\begin{itemize}
    \item Un enjeu de pertinence pour les religions : \textit{mondiale}, \textit{existentielle}, \textit{ne se joue pas à l'échelle individuelle mais d'une transformation collective}
    \item la réponse de l'Église catholique : lien avec la doctrine sociale de l'Église, articulation de la justice nécessaire et de l'action individuelle et collective. 
    \item Une \textit{nouveauté} du discours du pape François, une critique de l'idolâtrie dans des \textit{modes de vie} individualiste
    \item Et ouverture aux autres religions qui sont appelées à relever ensemble ce défi
    \item A la différence Dt, qui critiquait fortement les religions extérieures et toutes les compromissions, ici, il semble que nous ayons un paradoxe : positivité des religions non chrétienne et négativité des "compromissions" mais par rapport à une "religion non nommée".
  
\end{itemize}


\paragraph{Quelques pistes à explorer}
\begin{itemize}
    \item François : pas d'autonomie de l'économie (thomiste/sécularisation) qui en s'autonomisant, a pris comme religion l'argent. Alors qur GS, marquée par la sécularisation, en restait aux principes et laissait l'autonomie à l'économie \cite{cavanaugh_idolatrie_2022}. 
    \item  quelle regard chrétien d'une religion écologique ("gaia"),... Peut on être chrétien et Gaia ? ou autre idolâtrie potentielle mais non dénoncée (
\end{itemize}



\section{Le changement climatique, \textit{Enjeu de pertinence} contemporaine pour les religions} 


\paragraph{Quel mode de vie  pour une sobriété heureuse ?}  


 Un enjeu de pertinence pour les religions : \textit{mondiale}, \textit{existentielle}, \textit{ne se joue pas à l'échelle individuelle mais d'une transformation collective}

\paragraph{penser un salut collectif mais à travers une démarche qui entraine tout le monde } d'une certaine façon nous oblige à définir ce qu'est le \textit{salut écologique} 

\paragraph{Ecologie \textit{ou Changement climatique}} \begin{itemize}
    \item Changement climatique : question \item \textit{scientifique} \textit{Ecologie} : rapport au monde, plus large, et porteur d'une vision du monde. Une certaine perméabilité entre les deux terme.

\end{itemize}


% -----------------------------------------------
\section{la réponse de l'Église catholique dans la lignée de la Doctrine Sociale}    

\paragraph{critique de la modernité}
\begin{singlequote}
Ces schémas de pensée [du progrès scientifique et technique] sont si naturellement ancrés, ils informent si puissamment notre appréhension du réel que nous avons du mal à y renoncer tout à fait devant les démentis flagrants que nous offre l’actualité. Nous restons, volontairement ou non, consciemment ou non, orphelins des mythes du progrès, et nous serions bien contents de leur trouver un substitut chrétien, une garantie divine que, malgré quelques péripéties, tout ira pour le mieux. \cite[p.89]{candiard_quelques_2022}

\end{singlequote}

\paragraph{L'analogie avec l'injustice du taux à intérêt} Les religions monothéistes ont toujours porté des interdictions fortes car source d'inégalité. 

 \begin{singlequote}
        « nous ne pouvons pas ignorer qu’outre l’Église catholique, d’autres Églises et communautés chrétiennes – comme aussi d’autres religions – ont nourri une grande préoccupation et une précieuse réflexion sur ces thèmes qui nous préoccupent tous » (LS 7)
        Dans le sillage du concile Vatican II, l’encyclique insiste sur la contribution des religions en tant que vecteur d’une vision et d’une relation à la nature qui permet de répondre aux défis environnementaux et de proposer une alternative ancrée dans une sagesse séculaire pour éviter « l’indifférence, la résignation facile ou la confiance aveugle dans les solutions techniques » (LS 14). Elles constituent une richesse « pour une écologie intégrale et pour un développement plénier de l’humanité » (LS 62). Il s’agit donc pour toutes les religions de puiser dans « leur propre héritage éthique et spirituel », de revenir « à leurs sources » pour « mieux répondre aux nécessités actuelles » (LS 200). « Tous, nous pouvons collaborer comme instruments de Dieu pour la sauvegarde de la création, chacun selon sa culture, son expérience, ses initiatives et ses capacités » (LS 15). Cette crise, source de migrations violentes et contenant en elle la possibilité prochaine des guerres, peut aussi être un lieu de rencontre, de dialogue et d’action (LS 15) entre tous les hommes. Dans une perspective dont on a souligné les accents blondéliens \sn{Juan Carlos Scannone, « La filosofia dell’azione di Blondel e…, le pape y voit la possibilité de susciter une communion d’action afin d’ouvrir à une « nouvelle solidarité universelle » (LS 14).}

    \end{singlequote}


\paragraph{}
% -----------------------------------------------
\section{Une critique de l'idolâtrie, \textit{nouveauté} du pape François}


\paragraph{Idolâtrie, une opportunité invitant à considérer de façon différente l'économie et es autres phénomènes séculiers.}

\begin{singlequote}
François ne parle pratiquement jamais de l’économie contemporaine sans adresser une accusation d’idolâtrie, acccusation absente dans GS, et presque entièrement absente de Vatican II dans son ensemble.

Comment expliquer cette différence de traitement des questions économiques dans GS et chez le pape François ?

Ma thèse est que François représente une opportunité pour changer le discours catholique sur la sécularisation; une opportunité qui a des implications dans la manière de considérer non seulement l’économie, mais aussi d’autres phénomènes séculiers.

La pensée catholique progressiste dans la période du Concile Vatican II avait tendance à considérer le monde séculier comme désenchanté. François suggère au contraire, que nous ne sommes pas tant confronté à une perte de foi qu’à une nouvelle religion et un foi idolâtre. p. 126
\end{singlequote}


% -----------------------------------------------
\section{ouverture aux autres religions qui sont appelées à relever ensemble ce défi}



% -----------------------------------------------
\section{Paradoxe ?}
A la différence Dt, qui critiquait fortement les religions extérieures et toutes les compromissions, ici, il semble que nous ayons un paradoxe : positivité des religions non chrétienne et négativité des "compromissions" mais par rapport à une "religion non nommée".


\paragraph{est ce que Idolâtrie nous permet de creuser comment le dialogue inter religieux doit être pensé}

\paragraph{Dépasser une approche purement éthique} Kung

[critique du Manifeste pour une éthique planétaire de Kung]
\begin{singlequote}
la principale critique adressée aux théologies pluralistes, c’est leur prétention à disposer d’un lieu tiers, d’un arrière-plan qui se situerait au delà des religions particulières et à partir duquel on pourrait les embrasser toutes : le plan nouménal de la Rélité ultime pour John Hick, une même expérience mystique pour Raimon Pannikar ou encore un même projet éthique pour la justice et la gestion écologique des ressources de notre planete. \cite[p. 111]{cheno_dieu_2017} 
\end{singlequote}


\paragraph{Retrouver les points durs des autres religions pour nous aider à ne pas proposer une solution de type "Dieu et l'écologie"} mais bien penser comment s'articule l'un et l'autre. 


\paragraph{Par le thème d'idolâtrie, vocabulaire religieux}


\section{Points non intégrés à ce stade}
\paragraph{Et l'écologie comme religion scientifico ?} corpus de textes + transmission ?

% -----------------------------------------------

\begin{comment}


\paragraph{En quoi cela concerne le dialogue inter religieux} et pas uniquement Christianisme et écologie / ...
D'abord, Règne de Dieu. ensuite, intuition comme pour le dialogue inter-religieux que le Christianisme porte dans sa matrice une universalité et une hospitalité aux questions du temps, qui l'ouvre peut être de façon privilégiée à ces questions. 

\paragraph{Dialogue inter religieux et écologie} hypothèse que les religions peuvent aider à un effort maintenant pour un gain plus long, donner du sens, Règne de Dieu. Voir comment le Christianisme peut être pertinent sur le sujet.
Dialogue inter-religieux dans ce contexte. Sociologie des religions : permettant de valider cette hypothèse.
Règne de Dieu et Ecologie

\end{comment}



%------------------------------------
\section{Bibliographie consultée}

\subsection{Colloque RSR Ecologie}


Jeudi 17 novembre  Après-midi au Centre Sèvres  Introduction  15h-15h30       
Ouverture du colloque et présentation de la problématique                           Patrick C. Goujon 

15h30-17h30 Table ronde sur le numéro préparatoire, avec les auteurs    Transformer la théologie  

18h-18h30  Une anthropologie des relations   Elena Lasida 

18h30-19h Écologie intégrale, comment la crise écologique conduit à des transformations de la pratique de la théologie  Fabien Revol 

Soirée à l’Institut Catholique de Paris  Renouveler la théologie  
\textbf{20h-20h45 “Our Theological Traditions Revisited to Face the Ecological Challenge”   John Behr }

20h45-21h30 La conversion écologique de la théologie Discussion avec Elena Lasida, Fabien Revol et John Behr  Vendredi 18 novembre  Centre Sèvres   Matin  Des pratiques qui renouvellent la pensée (partie 1)  


9h-9h30 Le Campus de la transition et le Manuel de la transition   Cécile Renouard 

9h30-10h Les ateliers de la sociologie de la description    Anne-Sophie Breitwiller 

10h-10h30 Discussion  

\textbf{11h-11h45 Entre religio et religare, l'engagement politique écologiste   Fabrice Flipo  }

11h45- 12h30 Travaux en groupes  Après-midi Des pratiques qui renouvellent la pensée (partie 2)  

14h30-15h30 La vie monastique, ressource pour une existence écologique Dialogue entre Fr. Jean-Michel (Landévennec), une Sœur du Monastère de Taulignan, Danièle Hervieu-Léger et Bertrand Hervieu  

15h30-16h15 Cri de la terre et clameur des pauvres Alain Thomasset  

16h45-17h30 Comment localiser le global   Raphaël et Catherine Larrère 17h30-18h15  Discussion      

Samedi 19 novembre  Centre Sèvres  \textbf{La conversion, une problématique chrétienne ?}  9h-9h30 Conversion politique ?  En quoi parler de « conversion » est-il pertinent dans le champ politique ? Fabrice Boissier 

\textbf{9h30-10h Conversion morale ?  La conversion écologique relève-t-elle de la conversion morale et si oui, en quoi ?     Walter Lesch } 

10h-10h30 « Porter un fruit qui demeure ». Itinéraires bibliques et spirituels de conversion.  Patrick C. Goujon  
11h-12h30 Débats


\subsection{Laudato Si}
\cite{francois_laudato_2015}

\paragraph{Résumé}  
\begin{singlequote}
       La 4e de couverture indique : "Avec cette encyclique, le pape François s'adresse à tous les hommes de bonne volonté. Il les invite tous à un dialogue amical sur la crise écologique et sociale qui menace notre maison commune et il demande de suivre une voie conjointe pour répondre à ce défi mondial. Il ne s'agit pas de considérations théoriques avec quelques objectifs pratiques. Le pape ne veut pas seulement une amélioration dans des détails, mais une conversion fondamentale au vu de l'aggravation critique de la situation générale, qui ne permet plus d'esquive. Il s'agit de prendre conscience que nous habitons la même maison, donnée par Dieu, et que nous sommes les enfants de l'unique Créateur et Père des Cieux."
\end{singlequote}

\begin{singlequote}
        207. La Charte de la Terre nous invitait tous à tourner le dos à une étape d’autodestruction et à prendre un nouveau départ, mais nous n’avons pas encore développé une conscience universelle qui le rende possible. Voilà pourquoi j’ose proposer de nouveau ce beau défi : “Comme jamais auparavant dans l’histoire, notre destin commun nous invite à chercher un nouveau commencement [...] Faisons en sorte que notre époque soit reconnue dans l’histoire comme celle de l’éveil d’une nouvelle forme d’hommage à la vie, d’une ferme résolution d’atteindre la durabilité, de l’accélération de la lutte pour la justice et la paix et de l’heureuse célébration de la vie”.[148]
\end{singlequote}
       
\paragraph{Miser sur un autre style de vie}
\begin{singlequote}
        203. Étant donné que le marché tend à créer un mécanisme consumériste compulsif pour placer ses produits, les personnes finissent par être submergées, dans une spirale d’achats et de dépenses inutiles. Le consumérisme obsessif est le reflet subjectif du paradigme techno-économique. Il arrive ce que Romano Guardini signalait déjà : l’être humain « accepte les choses usuelles et les formes de la vie telles qu’elles lui sont imposées par les plans rationnels et les produits normalisés de la machine et, dans l’ensemble, il le fait avec l’impression que tout cela est raisonnable et juste ».[144] Ce paradigme fait croire à tous qu’ils sont libres, tant qu’ils ont une soi-disant liberté pour consommer, alors que ceux qui ont en réalité la liberté, ce sont ceux qui constituent la minorité en possession du pouvoir économique et financier. Dans cette équivoque, l’humanité postmoderne n’a pas trouvé une nouvelle conception d’elle-même qui puisse l’orienter, et ce manque d’identité est vécu avec angoisse. Nous possédons trop de moyens pour des fins limitées et rachitiques.

\end{singlequote}

        
\paragraph{les religions dans le dialogue avec les sciences}    
\begin{singlequote}
        199. On ne peut pas soutenir que les sciences empiriques expliquent complètement la vie, la structure de toutes les créatures et la réalité dans son ensemble. Cela serait outrepasser de façon indue leurs frontières méthodologiques limitées. Si on réfléchit dans ce cadre fermé, la sensibilité esthétique, la poésie, et même la capacité de la raison à percevoir le sens et la finalité des choses disparaissent.[141] Je veux rappeler que « les textes religieux classiques peuvent offrir une signification pour toutes les époques, et ont une force de motivation qui ouvre toujours de nouveaux horizons [...] Est-il raisonnable et intelligent de les reléguer dans l’obscurité, seulement du fait qu’ils proviennent d’un contexte de croyance religieuse ? ».[142] En réalité, il est naïf de penser que les principes éthiques puissent se présenter de manière purement abstraite, détachés de tout contexte, et le fait qu’ils apparaissent dans un langage religieux ne les prive pas de toute valeur dans le débat public. Les principes éthiques que la raison est capable de percevoir peuvent réapparaître toujours de manière différente et être exprimés dans des langages divers, y compris religieux.

        200. D’autre part, toute solution technique que les sciences prétendent apporter sera incapable de résoudre les graves problèmes du monde si l’humanité perd le cap, si l’on oublie les grandes motivations qui rendent possibles la cohabitation, le sacrifice, la bonté. De toute façon, il faudra inviter les croyants à être cohérents avec leur propre foi et à ne pas la contredire par leurs actions ; il faudra leur demander de s’ouvrir de nouveau à la grâce de Dieu et de puiser au plus profond de leurs propres convictions sur l’amour, la justice et la paix. Si une mauvaise compréhension de nos propres principes nous a parfois conduits à justifier le mauvais traitement de la nature, la domination despotique de l’être humain sur la création, ou les guerres, l’injustice et la violence, nous, les croyants, nous pouvons reconnaître que nous avons alors été infidèles au trésor de sagesse que nous devions garder. Souvent les limites culturelles des diverses époques ont conditionné cette conscience de leur propre héritage éthique et spirituel, mais c’est précisément le retour à leurs sources qui permet aux religions de mieux répondre aux nécessités actuelles.

        201. La majorité des habitants de la planète se déclare croyante, et cela devrait inciter les religions à entrer dans un dialogue en vue de la sauvegarde de la nature, de la défense des pauvres, de la construction de réseaux de respect et de fraternité. Un dialogue entre les sciences elles-mêmes est aussi nécessaire parce que chacune a l’habitude de s’enfermer dans les limites de son propre langage, et la spécialisation a tendance à devenir isolement et absolutisation du savoir de chacun. Cela empêche d’affronter convenablement les problèmes de l’environnement. Un dialogue ouvert et respectueux devient aussi nécessaire entre les différents mouvements écologistes, où les luttes idéologiques ne manquent pas. La gravité de la crise écologique exige que tous nous pensions au bien commun et avancions sur un chemin de dialogue qui demande patience, ascèse et générosité, nous souvenant toujours que « la réalité est supérieure à l’idée ».[143]
\end{singlequote}
        

\subsection{Vision Catholique de l'Ecologie}

\paragraph{Candiard}

Risque de l'écologie philosophique qui pense la décroissance mais ne sait plus pour quel but ? 

\cite{candiard_quelques_2022}
\begin{singlequote}
        Le progrès scientifique et technique exceptionnel que nous avons connu ces derniers siècles, en particulier les réussites incontestables de la médecine, confirmait cette vision des choses, comme la remarquable expansion économique qui l’a accompagné, dont nous commençons tout juste à comprendre qu’elle comporte aussi des effets délétères. nous savions naturellement que tout n’allait pas bien , mais nous pouvions croire cependant que les choses s’amélioraient.

        Ces schémas de pensée sont si naturellement ancrés, ils informent si puissamment notre appréhension du réel que nous avons du mal à y renoncer tout à fait devant les démentis flagrants que nous offre l’actualité. Nous restons, volontairement ou non, consciemment ou non, orphelins des mythes du progrès, et nous serions bien contents de leur trouver un substitut chrétien, une garantie divine que, malgré quelques péripéties, tout ira pour le mieux.

        Ne serait-ce pas un juste retour des choses puisque de l’avis général, ces philosophies de l’histoire auraient simplement transposé sur terre une espérance chrétienne, laicisé la foi au salut et au paradis ?

        p 89 à 91

\end{singlequote}

\begin{singlequote}
du mal qui est censé en supporter les inconvénients, sans quoi il n'y a plus de justice. Si l'accident de voiture que cause mon imprudence blesse ou tue un innocent, impossible de parler de justice. Dans notre cas à nous, les conséquences apocalyptiques du péché ne sont pas justes, car elles ne frappent pas spécialement les pécheurs et, moins encore, à proportion du péché. Un pacifiste n'est pas moins menacé par la destruction nucléaire qu'un dictateur. Impossible de dire à un paysan philippin qui a dû quitter sa terre à cause d'inondations dramatiques que c'est après tout de sa faute, et qu'il aurait dû moins polluer, car il fait face, en réalité, à une véritable injustice immanente: il assume les conséquences terribles d'actions dont il n'est nullement responsable.

Cette responsabilité est d'autant moins personnelle, et la conséquence d'autant moins juste, qu'une conséquence en entraîne une autre, en chaîne, de manière souvent imprévisible et surtout Si Jésus tient un discours d'apocalypse, de révélation, ce n'est pas pour nous terrifier plus ou moins utilement, mais bien pour nous faire comprendre ce qui se joue sous nos yeux: non la punition divine des fautes de l'homme, mais le déploiement du mal et de ses conséquences destructrices; autrement dit, la fin des temps à l'œuvre, non comme événement inquiétant dont on attendrait la proximité, mais comme cette réalité présente dans l'histoire depuis son début, véritable trame sous-jacente aux événements du monde. Nous avons besoin de ce dévoilement car, tant que la nature du mal restera inconnue, on pourra croire béatement à l'efficacité de solutions purement techniques aux mena ces qui pèsent sur nos existences. Il est sans doute nécessaire, dans bien des domaines, d'améliorer la législation, de modifier nos modes d'organisation, de négocier la réduction des arsenaux nucléaires ou celle des émissions de gaz polluants, de faire évoluer les opinions publiques; l'engagement politique ou l'action associative peuvent être  inexorable, loin de toute volonté consciente initiale: nous ne maîtri sons pas nos propres catastrophes.
\end{singlequote}

\subparagraph{Quelques mots avant l'Apocalypse}
\begin{singlequote}
        
des voies nobles et utiles pour rendre meilleure la vie de tous. Il serait naïf, évidemment, de prétendre lutter contre les désastres climatiques en s'en remettant à la seule prière, mais il ne serait pas moins naïf d'imaginer vaincre le mal sans s'attaquer à ses causes, et d'oublier que le premier lieu où je peux envisager de le déraciner, c'est dans ma propre vie.

En moi, le combat eschatologique est déjà engagé, avec sa violence et ses incertitudes :

c'est lui qui est à l'œuvre dans mes crampes d’égoïsme et dans mes envies de bien faire, dans mes fidélités et mes impatiences. Et dans ce combat, l'emporter, c'est d'abord accepter que la victoire a déjà été acquise, non pas par mes efforts, mais par l'amour infini qui se donne à voir dans la croix de Jésus et qu'il me faut, peu à peu, laisser entrer dans ma propre vie.
\end{singlequote}


\subsection{Ecologie et Dialogue}
\paragraph{Ecologie en Islam et Dialogue Interreligieux} \cite{pisani_ecologie_2016} 
    Toynbee montrait aussi que c’est au contact les unes des autres, dans l’interaction de leurs mythes et de leurs théologies que se créent les conditions de l’avenir.
    Laudato Si : l'enjeu nécessite le concours de tous.
   
    importance de la situation de ‘Alī al-Ḫawwāṣ dans LS (reconnaissance de l'héritage écologique de l'Islam).     
    \textit{habitus écologique} : définies par des rites ou s'affirment des attitudes singulières ou s'entremelent monde spirituel et monde matérieul. 
    Une réponse : l’islam est la solution (si crise, c'est qu'on n'est pas assez islam; pas d'ouverture aux autres religions).
    
    fitra : revenir à un état originaire. la nature vrai musulman, ne se rebelle pas.
    
 
    

  

    
    \paragraph{
   Economie, idolâtrie et sécularisation depuis Gaudium \& Spes}
    \cite{cavanaugh_idolatrie_2022}
    Résumé 	La sacralisation de l'argent, du pouvoir et par là même de l'individu est au fondement des multiples crises que nous traversons aujourd'hui. L'Église a un rôle déterminant à jouer pour parer à toute tentation idolâtrique, y compris en son propre sein. En prenant appui sur l'Évangile, la Tradition et les témoins de la foi, le peuple des baptisés a le devoir de débusquer, dans nos sociétés sécularisées, tout ce qui peut blesser l'image divine inscrite au coeur de chaque être humain. Par une réflexion théologique originale, William Cavanaugh nous aide à retrouver l'adoration véritable qui permet d'accéder à la liberté authentique des enfants de Dieu
    
    Notes :
\begin{singlequote}
François ne parle pratiquement jamais de l’économie contemporaine sans adresser une accusation d’idolâtrie, accusation absente dans GS, et presque entièrement absente de Vatican II dans son ensemble.

Comment expliquer cette différence de traitement des questions économiques dans GS et chez le pape François ?

Ma thèse est que François représente une opportunité pour changer le discours catholique sur la sécularisation; une opportunité qui a des implications dans la manière de considérer non seulement l’économie, mais aussi d’autres phénomènes séculiers.

La pensée catholique progressiste dans la période du Concile Vatican II avait tendance à considérer le monde séculier comme désenchanté. François suggère au contraire, que nous ne sommes pas tant confronté à une perte de foi qu’à une nouvelle religion et un foi idolâtre. p. 126
\end{singlequote}

        pour un nouvel Dt, choisis la vie.

        voir p 127 l’économie dans Gaudium \& Spes

GS ce qui doit changer dans la société p 131        
\begin{singlequote}
Le passage sur l'économie [de GS] a également été critiqué pour son moralisme et son incapacité à requérir l'avis de professionnels dans le domaine de l'économie. Mais les Pères concilaires n'avaient pas l'intention de faire davantage qeu de fournir des "principes de justice et d'équité, demandés par la droite raison" 132


L'autonomie et la rationalité de la sphère économique sont impliquéees, plus en amont dans la constitution, par l'affirmation de l'"autonomie des réalités terrestres" et de l'autonomie des sciences" (n°36), ce qui inclut vraisemblablement la science économique. p132

\end{singlequote}        

\subparagraph{Pape François et l'économie}
lumen fidei : quatorze fois idole. le contraire de la foi n'est pas un manque de foi mais bien l'idolâtrie. Quand on cesse de croire en Dieu, on ne cesse pas simplement de croire, mais on croit à autre chose.  l'idolâtrie et toujours un polytheisme; un mouvement sans but qui va d'un seigneur à l'autre. \textbf{"confiance".}
\begin{singlequote}

\end{singlequote}



\paragraph{Dieu au Pluriel.  Chapitre sur l'approche culturo-linguistique}    
        \cite{cheno_dieu_2017}

    Notes :
\begin{singlequote}
        
        la principale critique adressée aux théologies pluralistes, c’est leur prétention à disposer d’un lieu tiers, d’un arrière-plan qui se situerait au delà des religions particulières et à partir duquel on pourrait les embrasser toutes : le plan nouménal de la Rélité ultime pour John Hick, une même expérience mystique pour Raimon Pannikar ou encore un même projet éthique pour la justice et la gestion écologique des ressources de notre planete. [critique du Manifeste pour une éthique planétaire de Kung] p. 111
\end{singlequote}

        impossible car
\begin{singlequote}
        Nous sommes des humains, insérés dans une culture et des pratiques qui nous façonnent. p

\end{singlequote}

        Le modèle culturo-linguistique
\paragraph{The Nature of Doctrine} \cite{lindbeck_nature_2002}

    
\paragraph{De quel genre de pensée a-t-on besoin pour aborder la crise environnementale contemporaine ? }
        \cite{howles_quel_2022}
Résumé   	
\begin{singlequote}

L'écologie politique contemporaine donne un nouvel infléchissement aux débats environnementaux qui risquent, sinon, de rester bloqués dans un paradigme réducteur et moderniste. Il est intéressant de noter que cette nouvelle écologie politique s'inspire de plus en plus du langage et de concepts théologiques, en particulier dans l'œuvre de Bruno Latour. Le présent article explore les raisons pour lesquelles il en est ainsi et quelle contribution cette approche peut apporter. L'écologie politique assigne un rôle à la religion en ce que celle-ci génère le genre de conversion aux valeurs humaines qui s'avèrent nécessaires pour une véritable transformation sociétale. En procédant ainsi, l'écologie politique pourrait même être considérée comme un partenaire de dialogue (surprenant) pour la théologie catholique et pour des approches de la crise environnementale qui s'appuient plus largement sur la tradition de l'enseignement social catholique.
\end{singlequote}
 
Notes :

\begin{singlequote}

        À première vue, la popularité et la diffusion de ces idées est peut-être surprenante, car dans leur engagement radical à faire exploser la dichotomie moderniste supposée des humains et de la nature, ces théoriciens compliquent la compréhension du rôle de l’action humaine pour faire face à la crise planétaire à laquelle nous sommes confrontés. Ils sont notamment pessimistes quant au potentiel des réponses managériales ou technologiques d’origine humaine et visent fréquemment des partisans de l’« écomodernisme » et ceux qui proposent des stratégies grâce auxquelles les sociétés humaines pourraient envisager d’atteindre un « bon Anthropocène » [9]. 
\end{singlequote}

        Doctrine sociale de l’Eglise : positif sur le role des acteurs

        ecomodernisme : progrès
\begin{singlequote}

        Cependant, pour la nouvelle écologie politique, cette sorte d’écomodernisme est une simple réaffirmation du dualisme de l’être humain face à un monde de la nature passif, non animé et inerte, dans l’attente que grâce à son ingéniosité il soit en mesure de dominer et de maîtriser ce monde. Malgré de bonnes intentions individuelles dans des situations particulières, cela mène invariablement à la perpétuation du paradigme technocratique moderniste et ne réussit pas à prendre en compte ce que le pape François appelle « les racines humaines de la crise écologique » [11]. C’est pourquoi la nouvelle écologie politique rejette entièrement ce paradigme.
\end{singlequote}

        {une critique du judeochristianisme, responsable du modernisme.}
\begin{singlequote}

        Fredric Jameson fait remarquer avec humour que « de nos jours il semble plus aisé d’imaginer la fin du monde que celle du capitalisme » [14]. Bruno Latour considère que cette attitude s’appuie sur des idées religieuses de providence et d’achèvement eschatologique, où le but et point final de l’histoire est décrété d’avance et où les croyants sont invités à structurer en conséquence leurs choix personnels et leurs décisions. Cela a, selon lui, un effet démobilisateur sur les énergies politiques qu’il considère nécessaires pour une action environnementale radicale, révolutionnaire et efficace aujourd’hui.
\end{singlequote}

        Latour : sous jacent du film dystopique sur l’asteroide : quelques personnes font le salut des autres (geo ingénierie” : alpocalypse pour les autres
\begin{singlequote}

        Il existe de très nombreux travaux relatifs à la théorie de Gaïa, des points de vue scientifiques comme non scientifiques, à la fois positifs et critiques vis-à-vis de sa méthodologie et de son potentiel explicatif [20]. Il est certain que nombre des tenants de l’écologie politique en question lisent Lovelock en non-spécialistes et l’abordent « d’une manière enthousiaste mais créative », comme le formule un commentateur [21]. Néanmoins, le concept de Gaïa leur sert d’outil pour décrire un système vraiment immanent qui n’est pas dirigé ou piloté par une force ou un acteur externe.

        La valence écologique de l’idée de Lovelock réside dans la manière dont elle rappelle aux êtres humains leur statut d’acteurs opérant au sein des systèmes écologiques de la planète Terre. Cette approche implique au moins deux avantages pour l’écologie politique. D’abord, elle réfute la logique écomoderniste qui situe l’acteur humain pour ainsi dire à l’extérieur de ce système, avec la fausse assurance qu’il est équipé pour le gouverner lui-même de manière quasi divine. Car la pensée Gaïa stipule qu’un système stable de la Terre est la fonction de processus multiples, intimement imbriqués, mais non régis de l’extérieur. Bruno Latour le formule de la manière suivante : « Je pense que c’est cela que Lovelock laisse sous-entendre en quelque sorte, à savoir que la Terre est connectée. Chaque élément des entités qui la constituent construit son propre environnement, mais il n’y a pas d’“organisateur” ». Il n’y a pas de Dieu, en fin de compte » [23]. Il n’est donc pas étonnant que Latour considère Gaïa comme un outil « séculier » pour réorganiser la politique, loin de l’impasse théologique dans laquelle elle a été entraînée jusqu’ici [24], ce qui fait progresser, en ce sens, « l’intuition […] entièrement séculière de Lovelock » [25]. 

        Le second atout de l’idée de Lovelock, aux yeux des penseurs écologiques, consiste en ce qu’elle invite à un mode d’action adéquatement responsable. Isabelle Stengers souligne que comprendre Gaïa c’est se rendre compte que la Terre elle-même est devenue « chatouilleuse », « susceptible » d’une certaine manière, dans le sens où ses mécanismes homéostatiques finement accordés ne sont pas indépendants de nos actions, et que nous ne pouvons pas non plus présumer qu’ils rebondiront en réponse au stress que nous leur imposons [26]. Nous avons beau être une partie mineure d’un système plus large, il n’en reste pas moins que notre empreinte environnementale particulière importe vraiment, qu’elle soit grande ou petite.
    \end{singlequote}

 \paragraph{Kung : éthique planétaire}
 \cite{kung_lethique_2009}
{Résumé}
\begin{singlequote}
Le projet d’éthique planétaire se situe dans la foulée de l’éthique de la responsabilité de Max Weber. Il propose une fondation rationnelle de l’éthique (voir K.-O. Apel et J. Habermas). L’être humain jouit d’une autonomie intramondaine mais ne peut fonder seul l’universalité de l’obligation éthique. Onze thèses fondatrices sont alors énoncées comme, par exemple : le jeu a besoin de règles; le fair-play suppose l’observation de normes; éthique n’équivaut pas à doctrine sociale mais à conscience, conviction et attitudes morales ; les règles éthiques peuvent être fondées à partir de la raison sans référence transcendante, etc.
        
\end{singlequote}
    
Notes :
\begin{singlequote}

9 – Une argumentation rationnelle abstraite ne parvient que difficilement à convaincre des gens de différentes cultures et de différents milieux

John Rawls déduit des règles éthiques de principes généraux de justice, entendue comme fair-play, abstraite consciemment de tout contexte concret et de toute situation. Mais ce n’est qu’une idée étendue de justice qui lui permet après coup de développer une conception du droit et de la justice qui puisse aussi s’appliquer aux principes et aux normes du droit international et des relations internationales.

L’éthique de la discussion de Karl-Otto Apel et de Jürgen Habermas insiste à juste titre sur l’importance du consensus rationnel et de la discussion. Ils prétendent en cela pouvoir formuler des normes, dans une moindre dépendance à l’égard du contexte, qui vaillent inconditionnellement, et ce à partir de la communauté humaine de communication et d’argumentation. Les principes religieux et les interprétations religieuses de la morale, dévalorisés face à l’espace public, doivent être remplacés par une discussion rationnelle, par un jeu de langage régulé, par la « contrainte de l’argument sans contrainte ». Compte tenu de la réalité concrète de la vie, il est discutable qu’on puisse atteindre un éthos global (pour ainsi dire jusqu’au dernier village indien ou africain), qui soit réellement obligatoire et contraignant, à l’aide d’une discussion rationnelle abstraite.
\end{singlequote}

\begin{singlequote}
        11 – Les traditions religieuses ne doivent pas être objet de mépris, mais de réflexion critique

        L’anthropologie culturelle nous l’enseigne : les normes éthiques concrètes, les valeurs et les intuitions éthiques se sont développées graduellement, selon un processus socio-dynamique extrêmement complexe.

        Selon que des besoins vitaux apparaissaient, que des urgences et des nécessités entre humains se manifestaient, dès le début il a fallu des orientations et des régulations de l’action : des conventions déterminées, des sagesses, des mœurs, bref des critères éthiques, des règles, des normes, qui au cours des siècles et des millénaires ont été éprouvés. En cela, il est frappant que certaines normes éthiques se ressemblent partout dans le monde. Mais c’est un fait historique : à travers des millénaires, ce sont les religions qui fournirent des systèmes d’orientation, qui formèrent les bases d’une certaine morale, qui les légitimèrent, et qui souvent sanctionnèrent les déviations par des peines.

        À l’origine, la philosophie et la religion, la philosophie et la théologie ont plutôt vécu en symbiose ; celle-ci ne peut plus être rétablie. Mais une coopération plus intensive est recommandable, en vue d’une même vision d’espérance : “To make the world a better place”. Pour réaliser cette espérance, il faut au principe un changement de conscience vers un éthos humain, au service d’une culture de la non-violence et d’un respect de toute vie, de la solidarité et d’un ordre économique juste, de la tolérance et de la vie en vérité, d’une égalité des droits et d’un partenariat entre hommes et femmes.
\end{singlequote}

\subparagraph{approche philosophique}        

        eviter le recours au concept de loi naturelle
\begin{singlequote}
        
        Il faut plutôt chercher à atténuer, par une solution pragmatique de problèmes urgents, les oppositions entre visions du monde, sans tenir compte des différences idéologiques :

            Il était clair dès le départ que le projet « Éthique planétaire » se situait dans la ligne de l’éthique de responsabilité de Max Weber, qui ne met pas entre parenthèses l’orientation juste, mais veut concentrer l’attention sur les conséquences raisonnables. Jean-Paul Sartre voyait déjà que le domaine de validité de la responsabilité s’étendait à tout le monde des humains. Hans Jonas l’a étendue à toute la biosphère, appelant à réfléchir aux conséquences dangereuses de l’agir, y compris pour les générations à venir. En cela, le « principe responsabilité » de Jonas et le « principe espérance » d’Ernst Bloch ne s’excluent pas. Emmanuel Levinas a fait ressortir que la situation entre les humains est au centre de cette responsabilité, et qu’il faut prêter attention à l’altérité de l’autre, comme à son caractère étranger, et qu’il faut avoir des égards même pour les ennemis. À la recherche de compétences pour une communication responsable, Hannah Arendt a encouragé en particulier une façon de penser plus large, un imaginaire et un sens commun, et – ultimement – elle a mis avant tout en relief la vertu de vérité, qui s’efforce d’atteindre la vérité des faits, car naturellement, sans elle, il n’y a aucune communication ouverte possible entre les hommes.

            Il faut plutôt chercher à atténuer, par une solution pragmatique de problèmes urgents, les oppositions entre visions du monde, sans tenir compte des différences idéologiques : cela pourrait à long terme établir des points communs, y compris justement un éthos commun. Le conflit des visions du monde ou des idéologies devrait être apaisé de cette façon.
\end{singlequote}

\paragraph{Manifeste pour une éthique planétaire}
    \cite{kuschel_manifeste_1995}
    Notes :

        explication de Hans Kung

        sur la difficulté du genre, pas casuistique (mais rappeler l’importance e la vie), pas un énoncé des droits universels (si on n’a rien à dire,…), pas une dissertation philosophique. Autocritique
\begin{singlequote}
En Allemand, la déclaration portera le titre de déclaration pour un éthos planétaire, non pas pour une éthique planétaire. Ethos désigne la disposition morale fondamentale de l’homme, tandis qu’éthique nome la doctrine (philosophique ou théologique) concernant les dispositions, valeurs et normes morales.P60
\end{singlequote}

        Un texte “moderne” : non européocentré (cf discussion sur le nom de “Dieu”)

        mais le risque de tomber dans un PGCD est réel. Est ce que les religions font tout en passant par cette déclaration ?

        Parti pris que les religions ont chacune à travailler directement le sujet.

        une approche plraliste
\begin{singlequote}
éthique (ou ethos) planétaire, c’est à dire un accord fondamental en matière d’axiologie, de critères indiscutables et de choix essentiels. A défaut d’un consensus éthique fondamental, toute communauté court tôt ou tard le risque du chaos ou de la dictature.  Un ordre mondial meilleur ne peut se concevoir sans éthos planétaire

        (préface. p6)

\end{singlequote}

        MAIS AUSSITOT
\begin{singlequote}
“ethique planétaire ne signifie ni idélogie planétaire, ni religion mondiale unitaire à côté des religions existantes, ni quelque forme syncrétique de toutes les autres religions. Notre humanité est lasse des idéologies unitaires et les diverses religions du monde sont de toute manière si différentes dans l’expression de leurs croyance et dans leurs dogmes, dans leur symbolique et leurs rites, que tout effort d’”unification” est dénué de sens. P 6

\end{singlequote}


“ce manifeste … point de départ”P 7

\begin{singlequote}

Il nous faut tendre à l’instaration d’un ordre social et économique juste, au sein duquel chacun jouise de chances égales, au bénéfice de toutes ses possibilités humaines. P14

        Il est illusoire de vouloir rendre cette planète meilleure, sans changer d’abord la conscience des individus. Nous nous engageons dès lors à élargir notre capacité de perception, en acceptant pour nos esprits la discipline de la méditation, de la prière et de la réflexion. Refuser aucun risque ou aucun sacrifice revient à empêcher toute mutation sensible de notre situation présente. C’est pourquoi nous nus engageons à vivre selon cette éthique planétaire, dans l’intelligence réciproque, et à respecter entre nus des modes d’existence propres à promouvoir la tolérance mutuelle, la paix sociale et internationale et le respect de la nature. P 14

        Un pricnipe se retrouve depuis des millers d’années dans beaucoup de traditions religieues et éthiques de l’humanité qui l’ont conservé, c’est la “règle d’or”; ce que tu ne veux pas qu’on fasse à ton endroit, ne le fais pas à l’endroit d’aucun autre. P 23
\end{singlequote}
        
        
\subsection{MOOC Bernardin}

 	SEMAINE 6
Habiter la maison commune
 
par Fabien Revol

	Les actualités du MOOC :
o	La séance 5 a montré comment l'écologie de la vie quotidienne correspond à une écologie humaine.
o	Poursuivez avec la séance 6 sur les grands concepts de l'écologie intégrale !

\href{https://www.lecampusdesbernardins.fr/resource/26/?utm_source=sendinblue&utm_campaign=20221107_NL_MOOC_sans_dons&utm_medium=email}{lien avec le mooc}

\section{Actualité}
\paragraph{13 novembre 2022 - en marge de la COP 27}
L’archevêque anglican Rowan Williams conduit des chefs religieux sur la colline du Parlement pour une cérémonie de repentance climatique le 13 novembre à Londres. -
 
\begin{singlequote}
    « La majorité des habitants de la planète se déclare croyante, et cela devrait inciter les religions à entrer dans un dialogue en vue de la sauvegarde de la nature. » (Pape François)
\end{singlequote}
