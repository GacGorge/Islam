\chapter{Sujets de mémoires ISTR}

\paragraph{Universalité}

\paragraph{Inculturation ignatienne} Chine mais alors Islam ? et c'est quoi la culture ?

\paragraph{Conversion ou dialogue} Intérêt de l'autre doit nous motiver à le convertir. mais principe d'indifférence. 
\begin{itemize}
    \item \href{https://mission-ismerie.com/}{Mission Ismérie}  : 
\end{itemize}

\paragraph{Dialogue inter religieux et écologie} hypothèse que les religions peuvent aider à un effort maintenant pour un gain plus long, donner du sens, Règne de Dieu. Voir comment le Christianisme peut être pertinent sur le sujet.
Dialogue inter-religieux dans ce contexte. Sociologie des religions : permettant de valider cette hypothèse.
Règne de Dieu et Ecologie

\begin{itemize}
    \item PISANI Emmanuel, « Écologie en islam et dialogue interreligieux », \textit{Transversalités}, 2016/4 (n° 139), p. 53-64. DOI : 10.3917/trans.139.0053. URL : \href{https://www-cairn-info.icp.idm.oclc.org/revue-transversalites-2016-4-page-53.htm }{lien}
    Toynbee montrait aussi que c’est au contact les unes des autres, dans l’interaction de leurs mythes et de leurs théologies que se créent les conditions de l’avenir.
    Laudato Si : l'enjeu nécessite le concours de tous.
    \begin{quote}
        « nous ne pouvons pas ignorer qu’outre l’Église catholique, d’autres Églises et communautés chrétiennes – comme aussi d’autres religions – ont nourri une grande préoccupation et une précieuse réflexion sur ces thèmes qui nous préoccupent tous » (LS 7)
        Dans le sillage du concile Vatican II, l’encyclique insiste sur la contribution des religions en tant que vecteur d’une vision et d’une relation à la nature qui permet de répondre aux défis environnementaux et de proposer une alternative ancrée dans une sagesse séculaire pour éviter « l’indifférence, la résignation facile ou la confiance aveugle dans les solutions techniques » (LS 14). Elles constituent une richesse « pour une écologie intégrale et pour un développement plénier de l’humanité » (LS 62). Il s’agit donc pour toutes les religions de puiser dans « leur propre héritage éthique et spirituel », de revenir « à leurs sources » pour « mieux répondre aux nécessités actuelles » (LS 200). « Tous, nous pouvons collaborer comme instruments de Dieu pour la sauvegarde de la création, chacun selon sa culture, son expérience, ses initiatives et ses capacités » (LS 15). Cette crise, source de migrations violentes et contenant en elle la possibilité prochaine des guerres, peut aussi être un lieu de rencontre, de dialogue et d’action (LS 15) entre tous les hommes. Dans une perspective dont on a souligné les accents blondéliens \sn{Juan Carlos Scannone, « La filosofia dell’azione di Blondel e…, le pape y voit la possibilité de susciter une communion d’action afin d’ouvrir à une « nouvelle solidarité universelle » (LS 14).}

    \end{quote}
    importance de la situation de ‘Alī al-Ḫawwāṣ dans LS (reconnaissance de l'héritage écologique de l'Islam).     
    \textit{habitus écologique} : définies par des rites ou s'affirment des attitudes singulières ou s'entremelent monde spirituel et monde matérieul. 
    Une réponse : l’islam est la solution (si crise, c'est qu'on n'est pas assez islam; pas d'ouverture aux autres religions).
    
    fitra : revenir à un état originaire. la nature vrai musulman, ne se rebelle pas.
    \item CANDIARD, Adrien \textit{Quelques mots avant l’Apocalypse, Lire l’Évangile en temps de crise}, Cerf, 120 p.
    \begin{quote}
        Il réhabilite les « trois désirs conduisant au mal » de la théologie classique. «Libido dominandi, qui est la passion de l’emporter et de soumettre ; libido sentiendi, qui est celle de jouir et de posséder ; libido sciendi, la science sans conscience ». On en mesure la dévastatrice contagiosité pour les humains et pour leur environnement. Quelle est la bonne nouvelle ? L’amour du Christ, qui nous donne de « regarder le danger en face », sans craindre la mort. C’est dans nos « failles » et « fragilités » qu’opère sa grâce. à côté des catastrophes, la « miraculeuse gestation » du royaume de Dieu est à l’œuvre, nous acheminant vers une « Création tout entière renouvelée ». \sn{critique de la croix}
    \end{quote}
    
\end{itemize}
Risque de l'écologie philosophique qui pense la décroissance mais ne sait plus pour quel but ? 


\paragraph{Spiritualité des exercices} soufisme

\paragraph{Théobald} Hospitalité Chrétienne

\paragraph{Disciple Missionnaire} Travailler le thème de disciple
missionnaire en Cvx ? influence
Jésuite ? Matteo Ricci ? Question

\begin{quote}
    10. [Cheminer avec une Eglise missionnaire] Le Kairos dans notre Eglise nous appelle à être des \textbf{disciples missionnaires} pour le monde, au travers d’une rencontre avec Jésus qui nous ouvre à l’amour du Père.5 Austen Ivereigh, l’un des biographes du Pape François, nous a partagé ce que signifie entrer dans l’esprit missionnaire : être Christ dans notre monde blessé, en aidant les personnes à se reconnecter avec la création et le monde en tant que créatures de Dieu ; faire l’expérience de la famille et de la communauté, qui sont les liens de confiance et d’amour inconditionnel qui construisent la résilience, la personnalité et l’estime de soi ; et aussi aider les personnes à trouver un sanctuaire. Ce cheminement nous invite à nous laisser guider par la réalité et le Saint-Esprit dans notre mission. \sn{\href{http://www.cvx-clc.net/filesNewsReports/AM2018_FinalDocument\%20(FRENCH).pdf}{17è Assemblée Mondiale de la Communauté de Vie Chrétienne Buenos Aires}, Argentine 2018 CVX, Un Don pour l’Eglise et le Monde ‘Combien de pains avez-vous?... Allez voir’ (Mc. 6, 38) }
\end{quote}


\paragraph{test} \cite{Gardet:IntroductionTheoMusulmane}