\chapter{L'unique Christ. Christian Duquoc}
\mn{Guillaume Gorge le 3/1122}
\section{Christian Duquoc} Christian Duquoc, dominicain, a été longtemps professeur à la faculté de théologie de l'université catholique de Lyon et directeur de Lumière et Vie, revue dominicaine qui s'est arrêté en 2013. Il a notamment publié Dieu différent, Je crois en l'Église (1999), Christianisme, mémoire pour l'avenir (2000) et L'Unique Christ (2002). 

Il est connu pour sa prise en compte de la réalité et du présent \sn{méfiance du passé ou des spéculations sur le futur, qui nous est caché} et sa pensée systématique autour de Jésus historique, avec la conviction  que l’auteur a fréquemment affirmée dans ses ouvrages christologiques : la portée universelle du message de sa Résurrection ne masque pas la réalité historique de l’existence de Jésus, elle s’y enracine.

\mn{L'unique Christ, la symphonie différée (2002)
Christianisme (2000)
La Femme, le clerc et le laïc, oecuménisme et ministère (1989)
Libération et progressisme (1987)
Dieu différent (1977)
Jésus, homme libre (1973)
Ambiguïté des théologies de la sécularisation (1972)
Vivre l'Eucharistie (1969)
Le Mariage (1966)
L'Église et le progrès (1964)
L'Argent (1958)
Le Mariage (1944)}


 

\section{Dieu au pluriel, penser les religions}




\subsection{Contexte}
\paragraph{importance d'Assise} de Jean-Paul II.


\paragraph{Dominus Jesus 2000} Cité deux fois dans notre partie - 

En septembre 2000, l'affirmation contenue dans la déclaration "Dominus Iesus"\sn{déclaration Dominus Iesus
sur l'unicité et l'universalité savifique de Jésus Christ et de l'Eglise} sur l'unicité et l'universalité salvifique de Jésus-Christ et de son Eglise : 

\begin{quote}
    "On doit donc tenir fermement la distinction entre la foi théologale et la croyance dans les autres religions" (DI7) 
\end{quote} pose, au-delà de l'usage du vocabulaire, une question théologique centrale pour les relations entre la foi chrétienne et la "foi" des Autres : comment la spécificité de la foi chrétienne, définie par l'articulation entre l'acte de croire et son objet et centrée autour de la vérité révélée en Jésus-Christ, permet-elle d'envisager la "foi" des autres, tant dans sa démarche que dans son objet? 
\begin{quote}
    La déclaration \textit{Dominus Jesus est } une défense et un commentaire de cette affirmation scripturaire : Jésus le Christ est le Sauveur de Tous
\end{quote}

  
\paragraph{11 septembre 2001} la recherche d'un consensus entre religion parait vain.


 
\mn{REY Bernard, « La médiation de Jésus-Christ. Les propositions de Christian Duquoc dans L'unique Christ. La symphonie différée », Revue des sciences philosophiques et théologiques, 2003/2 (Tome 87), p. 303-312.  }

\subsection{Problématique}
\paragraph{Problématique du livre}   comment rendre compte aujourd’hui de la médiation unique du Christ, clairement affirmée par la confession de foi chrétienne en tenant compte de la signification théologique des autres religions ?   La médiation du Christ ne serait-elle donc plus unique ? Un chrétien ne peut renoncer à cette unicité, mais comment la penser ? Faut-il parler, comme certains le font, d’une médiation unique du Verbe, en la détachant du destin historiquement et géographiquement particulier de Jésus de Nazareth, confessé comme Fils de Dieu par les chrétiens ?  \textbf{En répondant à cette question en  rigoureusement dans sa christologie la signification universelle du salut proposé par Jésus Ressuscité à son existence historique particulière.}  

\paragraph{s’interroger sur la signification théologique des autres religions. } La médiation du Christ ne serait-elle donc plus unique ? \textbf{Un chrétien ne peut renoncer à cette unicité, mais comment la penser ?} 

\paragraph{médiation unique du Verbe } Faut-il parler, comme certains le font, d’une médiation unique du Verbe, en la détachant du destin historiquement et géographiquement particulier de Jésus de Nazareth, confessé comme Fils de Dieu par les chrétiens ? 

\paragraph{universalité du Christ}
 Comme son titre le suggère, ce livre propose une réflexion sur un point central de la foi chrétienne : l’universalité du Christ \sn{ (« Tout est par lui et pour lui »,  Col 1, 16)}. L’ouvrage ne concerne pas seulement l’interprétation de la « médiation unique » du Christ, envisagée dans le cadre du pluralisme des religions, aujourd’hui reconnu par l’Église catholique mais son rapport à l'histoire et au cosmos. 
 
\subsection{Les premières parties}

\paragraph{Plan}
 quatre parties, ayant respectivement pour titre : la déchirure, les religions en fragments, l’espérance voilée, la division féconde.
\paragraph{Paradoxe du règne universel et pour cela coupure avec le judaisme}
 \textit{Le paradoxe, objet de la première partie, est que le Christ annonce le Règne (universel) de Dieu} tout en provoquant une Déchirure, c\textit{elle qui sépare le christianisme du judaïsme.}
 
\paragraph{4 points ancrage d' du Christianisme}
 
\begin{enumerate}

    \item  L’annonce du Règne concerne une victoire de la vie sur la mort, et la résurrection de Jésus est une justification par Dieu du combat de son envoyé. 
    \item   La croix atteste l’échec de son projet mais elle est réinterprétée par ses disciples comme l’expression d’une existence solidaire; \sn{elle signifie également que le Dieu annoncé par Jésus ne se manifeste pas par une toute-puissance s’exerçant par le jugement et le châtiment. L’analyse des motifs de la mort de Jésus ne la fait pas apparaître comme un accident ou un événement aléatoire : elle est la conséquence du refus qu’on opposa à l’action de Jésus, à son message et à sa présentation du Salut. Les premiers chrétiens interpréteront donc cette mort comme une expression de la solidarité de Dieu avec tous les rejetés du monde : l’Alliance, qui disait le souci de Dieu pour le prochain, ce qu’exprimaient  les dix commandements n’est donc pas contredite. }
    \item Cette visée fait perdre à la Loi le rôle qui, du temps de Jésus, lui était aussi assignée et selon lequel elle confirmait au peuple son identité venue d’une élection, mais elle n’abolit pas son rôle pratique, celui d’ouvrir au souci du prochain, ce prochain étant tout homme, donc aussi le païen.  
    \item   L’ensemble de cette lecture de la vie et de la mort de Jésus propose de Dieu l’image d’un Dieu familier, un Dieu Père des hommes, tous sont appelés à avoir part aux biens de Dieu (« Tout ce qui est à moi est à toi » ).  

\end{enumerate}

\paragraph{Les religions en fragments}
 La deuxième partie de l’ouvrage concerne la dispersion des religions. Comment, en effet, comprendre l’universalité de la mission du Christ dans le cadre d’une telle fragmentation ?
 
\subparagraph{revenir à la singularité du Nazaréen}
 Il convient de considérer à nouveau l’action de Jésus pour dégager le lien existant entre la singularité du Nazaréen et le caractère universel du Salut, révélé par sa Résurrection d’entre les morts.  Nous nous trouvons là devant une ferme conviction : « Jésus ne se replie pas sur l’instant, il l’ouvre à sa profondeur », car le présent est « l’habitat de Dieu » (p. 113). 

 \subparagraph{pas d'espace de consensus entre les religions}
{L’auteur est ainsi conduit à refuser au dialogue des religions la possibilité de définir un espace de consensus en tentant, par exemple, de se mettre d’accord sur des critères éthiques mettant en valeur un plus petit dénominateur commun 
 de l'homme. }  Pour Duquoc, Il n’est pas possible de déterminer, par conséquent, des convergences visant une vérité à venir précise. valeur  du présent, où Dieu règne et exprime son souci de l’homme, L’affirmation de la valeur du présent est  conjointe à celle de sa profondeur ignorée ou, du moins, se dérobant sans cesse.

\paragraph{3eme partie Fragments d'unité : uniquement ce qui nous est donné }
Quel peut être le rôle particulier du christianisme ? L’espérance des chrétiens est fondée sur leur foi en la maîtrise du Christ sur l’histoire et le cosmos : en raison de l’événement pascal, il est le Seigneur. 
\subparagraph{seigneurie à partir du Jésus historique ne voulant pas transformer la société} 
 Jésus n’a pas voulu transformer le monde, ni la société, ni la nature. Il a refusé d’être un thaumaturge, de s’arroger le rôle d’un chef politique. Le Règne présent avec lui reste donc obscur.  
\paragraph{sainteté et non justice : les béatitudes}
L’enseignement des béatitudes proclamées par Jésus, et qui rejoint celui de sa vie, fait valoir que la médiation du Christ ressuscité s’exerce aujourd’hui grâce aux Justes. L’espérance chrétienne se fonde donc sur le fait que l’injustice ne triomphera pas du monde 
 


\section{partie IV : qui nous interesse}

 
\paragraph{Problématique de la quatrième partie - la division féconde} Constat des divisions des religions. on juge négatif la division. tension entre la communion en un seul corps - Jésus rassemble / unique Règne de Dieu (1 Co 12) et la réalité objective des religions .  Comment penser le paradoxe de façon féconde ?
 

 Prenant acte de cette situation, Christian Duquoc, au lieu de déplorer ces divisions, les perçoit\textbf{ comme fécondes.  }
\paragraph{Trois chapitres} 
\begin{itemize}
    \item le premier\textit{ L'Esprit et le dévoilement} sur l'Esprit saint, pour justifier de la positivité de la division
    \item le deuxième "de Jésus le Juif au Christ universel" : répondre à la question de n'est pas exclu que les divisions puissent être positives et fécondes ? Ne s'opposent elles pas au travail unificateur du Christ ? Quelle tache pour le Christ ? Pour préserver l'unité prématurée, faut il relativiser l'universalité de \textbf{Jésus le Christ} ?
    \item le troisième, la symphonie différée. 
\end{itemize}


  





\paragraph{Réflexion sur l'ES comme point de départ}  
 Le développement prend son point de départ dans une réflexion sur l’Esprit Saint à partir du NT : c’est lui qui dévoile l’articulation entre \textit{le caractère \textbf{particulier} de l’existence de Jésus et l’\textbf{universalité} attribuée au Christ ressuscité}. Dynamique vitale, il est source de liberté et d’amour et porte la Parole. Son action n’est pas répétitive, mais créatrice. Il ouvre un avenir, fonde une espérance, mais dans la patience.



\subparagraph{sur la patience : Simone Weil et l'hypomonê} p 215 \begin{quote}
    humilité : acceptation de l'attente. socialement, la marque des inférieurs est qu'on les fait attendre. 
\end{quote}

\subparagraph{L'Eprit dévoile , il ne révèle pas sans retrait} Il indique que le présent est habité par Dieu, mais il garde de mettre Dieu à notre disposition comme s'il était un avoir sn{\textit{buisson ardent}}



\paragraph{Babel : Esprit Division}
Ces rappels concernant l’action de l’Esprit permettent de mieux comprendre le \textbf{rôle positif que peuvent jouer les divisions} \mn{voir la Création de Luther}. Souvent perçues comme un mal à éviter, elles peuvent s’avérer nécessaires et fécondes. C'est une approche assez protestante de revalorisation de la division. 

  La division peut en effet devenir une nécessité lorsqu’il s’agit de faire émerger la multiplicité des richesses de l’humanité. En revanche, précipiter l’unité voire la forcer, institutionnellement ou même religieusement, peut conduire à des tyrannies comme l’histoire ne cesse de l’enseigner.  « Les divisions représentent donc des formes fragmentaires d’accès au vrai ». Il exclue toute mainmise sur Dieu.  \sn{RELIRE}

\paragraph{mais tension avec que tous soit un ?}
S’il en est ainsi, comment comprendre la prière de Jésus, rapportée en Jn 17 : « Que tous soient un » ?  le recours à l’attitude du Jésus historique est éclairante et Duquoc reproche à certains auteurs d’en amoindrir la portée. Jésus est ouvert à tous, y compris aux païens mais, dans le même temps, il annonce une « stratégie de séparation », comme en témoigne Mt 10, 34 \sn{ N’allez pas croire que je sois venu apporter la paix sur la terre; je ne suis pas venu apporter la paix mais l’épée. »}; Ce point avait été développée dans la première partie avec la division du judaisme.

\paragraph{Laisser le seigneur qui patiente faire son oeuvre d'unité}
Si l’Église tend vers l’universalité, c’est donc comme Jésus : à l’intérieur de sa propre singularité : « Le Seigneur ne tarde pas, il patiente. Il n’impose pas ce qui serait prématuré. Les divisions sont les garanties d’une universalité qui refuse de s’immiscer dans l’histoire par la violence institutionnelle et le mépris des créations et aventures humaines. Le Ressuscité laisse le monde à sa maturation multiforme et les religions à leurs expériences diversifiées » (p. 233).

\paragraph{Role spécifique de l'Eglise dans cette polyphonie ? - le chapitre de la Symphonie différée}
Le dernier chapitre du livre donne donc aussi la clé du titre. Cette métaphore, qui évoque une composition musicale, permet à l’auteur d’exposer ce qu’il appelle lui-même « la théologie du fragment » (p. 235). Selon lui, les religions ressemblent à de multiples compositions musicales dont nous échappe l’unité qui se trouvera réalisée et manifestée dans la symphonie finale. Il est nécessaire que ces multiples fragments de la symphonie en train d’être composée puissent exprimer leur singularité propre. L’obsession de l’unité risque de briser la création qu’effectue chaque religion. L’Église n’a pas une situation privilégiée dans l’organisation du concert, qui relève de Dieu seul, même si elle estime connaître le chef d’orchestre. 

\paragraph{Quelle mission pour l'Eglise ?}
\begin{quote}
    « \textit{la mission n’a pas pour fin d’intégrer à une institution unique, mais de donner forme à l’interdépendance latente, en incitant à débattre librement des questions issues du monde environnant et qu’aucune religion ne maîtrise} » (p. 248)
\end{quote} 
pour pouvoir dire sa singularité propre : 
\begin{quote}
    « Occulter ce qui habite notre propre identité chrétienne reviendrait à inviter les autres religions à se replier sur elles-mêmes en taisant ce qui les fait vivre. L’ignorance mutuelle conduit au triomphe de la rumeur, de la peur et finalement de la violence. La mission dans une ouverture sans impérialisme peut être garante d’un échange qui exorcise la peur et faire reculer l’hostilité presque instinctive » (ibid.).
\end{quote}




\section{Quelques aspects critiques} 


 \subsection{sur la forme}
 
 \paragraph{Pertinence d'une approche dialectique entre la réalité du monde avec sa part d'ombre telle qu'elle est et la fidélité à l'Evangile et la Tradition}
Accepter le paradoxe fécond. 
D'une certaine façon, approche sur la forme fidèle au Christ (dans les paraboles et métaphores).


\paragraph{Des arguments bibliques, des temps mais patristique et Tradition?}
Dans son développement et ses argumentations, l’auteur met surtout en rapport les données bibliques, notamment celles du Nouveau Testament et les interpellations venues de l’état du monde (mondialisation, violences, pluralisme religieux, y compris dans nos sociétés occidentales, divisions des Églises chrétiennes). Il cite aussi Dominus Iesus et Vatican II. ,
 


\subsection{sur le fond}

\paragraph{paradoxe des paradoxes} Le théologien recommande à chaque église particulière de cultiver sa singularité propre. Or l'une des caractéristiques de l'Eglise catholiques, c'est de valoriser l'unité visible des Chrétien. 
Il peut paraître surprenant que dans un monde de plus en plus livré à la violence, alors que se développe un discours sur la mondialisation, un théologien fasse l’éloge de la différence et même de la division. 

\begin{quote}
Il n’est pas impossible que ce type de discours soit interprété par certains comme une excessive concession aux réalités de ce monde, voire comme l’expression d’un christianisme qui baisse les bras devant la montée des autres religions. Mais face aux oppositions et malgré le réel état du monde, la tentation peut être grande de réaffirmer les discours du passé, centrés sur une Église hors de laquelle il n’est point de salut. La tendance est alors au repli et à la nostalgie d’une chrétienté idéale qui, selon certains, aurait jadis existé… (Rey)
    
\end{quote}



\paragraph{A force de dialectique, risque de lisser le scandale de la division} La première référence de l'Esprit Saint dans les Actes, le \textit{choix de Matthias} fait référence à la Genese, Abel et Cain et Joseph et peut être lue comme une refonte de l'unité.
Même si elle n'est pas réalisée, l'unité est lue comme signe de l'Esprit (cf les dernieres paroles de Jésus en Jn 21).


\paragraph{Jusqu'où la dialectique peut elle allée} Peut on par la dialectique dépasser tout ? Tout scandale, y compris celui de la division ? 


\paragraph{Vision d'une société bâtie sur l'amour entre trop individualisme et trop idéaliste}
  l'esprit source d'amour p 212. 
\begin{quote}
    la conversion que l'esprit opère ne s'identifie pas à la conformité à la loi  comme la parole, la liberté, investie par l'amour, est dangereuse : elle se réfère à un contenu qui la démarque de l'opinion des institutions et des organisations sociales. p213
\end{quote} Moins convaincu. Justice important\sn{cf "Une vie bonne, avec et pour autrui, dans des institutions justes"} ? étonnant que Duquoc marqué par le réalisme et soulignant que le Christ n'annonce pas une société par les béatitudes ?
le christianisme comme anti-système. 


\section{Discussions}

\paragraph{Contre le totalitarisme} d'un Christ qui récapitule les fragments de religions

\paragraph{Division + Christ roi} hypothèse : porter féconde de la division. 

Hegel est parti de la croix. 

\paragraph{manque d'appui sur les théologiens et la Tradition}

Abu dhabi : les différences (sexe, religion) sont voulues par Dieu. 
Nostra Aetate : mouvement de l'Eglise qui est en train de reinterpréter la religion. 

\paragraph{Lumen Gentium } L'Eglise estg le signe de l'Esprit qui travaille dans le monde. Il critique un peu Vatican II, on pense encore que . GS : le Christ récapitule tout (théologie de l'histoire). Vision inclusive et globale. Duquoc se détache de cette vision par l'action de l'Esprit Saint. Une théologie de l'histoire qu'on ne maîtrise pas. On a des fragments mais on ne peut pas reconstruire.

\paragraph{Témoignage de l'Eglise} comme le Christ qui n'annonce pas une société concrète, l'Eglise est au service de l'action de l'ES. 
p. 240 
\begin{quote}
    {l'Esprit travaille à la maturation de chaque fragment. ... Le retrait du Christ}
\end{quote}
L'Eglise ne remplace pas le Christ, elle est bcp plus le nazaréen qui marche vers sa croix.

\begin{quote}

Le chrétien ne sait rien de la forme que revêtira l'exécution réalisée, il ne peut en induire le contenu concret a partir de ce qu'il croit être le « sacrement » de ce don ultime de Dieu, il ignore quelle place y tiendra son Eglise et quelle forme de partition elle exécutera, il sait seulement que l'espérance de cet accord dernier n'est pas un rêve enfantin puisqu'il confesse que le chef d'orchestre présentement absent ne faillira pas à l'exercice bénéfique pour tous de sa souveraineté. Il existe bien une asymétrie entre les Églises et les autres fragments : \textit{elle tient à la confession chrétienne consciente du chef d'orchestre et de son activité actuelle par le don de l'Esprit,} elle ne relève en aucun cas d'une situation privilégiée dans l'organisation du concert puisque l'Esprit travaille à ce qu'aucune exécution prématurée puisse prétendre être la symphonie dernière.
L'asymétrie pousse les Églises à la conscience ferme d'une ignorance parce que les desseins de Dieu sur ce monde inter- médiaire ne lui sont révélés que sous des espérances aux contenus indicibles ou aux contours flous et des impératifs éthiques non étrangers à la marche plus ou moins chaotique de chaque fragment. En ce sens, l'Église est sacrement du salut dans la mesure où elle renonce à être le tout, c'est-à-dire à étre unique lieu de l'Esprit et la déléguée de son Seigneur. Une analogie se dessine ici entre le parcours historique de Jésus et le chemin terrestre de l'Église.
p241

\end{quote}
\paragraph{La métaphore} fait passer une vision. 


\paragraph{préférer le mot différence au mot division}

\paragraph{éthique planétaire floue}

\begin{quote}
    [les théologiens contemporains] ont, pour la plupart, sous des formes diverses, essayé de découvrir un horizon commun, impliquant à la fois le mouvement globale de l'histoire et le devenir des religions; Pour ma part, je ne pense pas que l'on puisse lui donner un contenu défini ou concret, on peut à la rigueur le viser avec des formulations formelles empruntées à ce que H. Küng appelle une "éthique planétaire". Cette perspective, il est vrai, demeure floue et ne suscite pas l'enthousiasme soulevé par les grandes utopies du XX siècle. Elles ont démontré par leur réalisation politique leur caractère illusoire et trop souvent dérisoire et cruel.Il me paraît donc inutile de désigner un horizon commun qui serait supposé favoriser le dialogue en proposant une base minimale d'accord. P. 243
\end{quote}