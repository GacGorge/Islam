\chapter{Claude Geffré - La rencontre du Christianisme et des cultures}


\section{Biographie}
\paragraph{Claude Geffré, théologien de la rencontre des religions} \mn{Céline Hoyeau, le 10/02/2017 à 00:00}
Ce brillant théologien a formé des générations d’étudiants et participé au renouveau de la théologie après le concile Vatican II.Il laisse une œuvre aux accents prophétiques à l’heure du pluralisme religieux.



Il fut l’un des théologiens français de l’après-Concile les plus en vue. Claude Geffré s’est éteint hier matin à l’âge de 90 ans. Sa santé s’était fortement dégradée ces derniers mois.

Le nom de ce dominicain né à Niort (Deux-Sèvres), entré dans l’ordre des prêcheurs en 1948 et formé à l’école de saint Thomas d’Aquin, s’est très vite identifié à deux axes majeurs de la théologie contemporaine : le « tournant herméneutique » et le pluralisme religieux.

C’est en arrivant en 1965 à l’Institut catholique de Paris, chargé de la théologie fondamentale, qu’il repense sa « méthode » : après Vatican II, le théologien ne peut plus se concentrer sur la défense des dogmes, mais doit passer à « l’âge herméneutique », cette science de l’interprétation des textes. Autrement dit, recueillir la signification de l’Écriture et la traduire dans le contexte contemporain. Il conçoit alors sa tâche comme « celle d’une sorte de médiateur entre la foi de l’Église et l’expérience historique des hommes et des femmes », écrit-il.

Sans transiger sur ce qui fait la spécificité du christianisme, Claude Geffré, en pionnier, cherche à concilier l’unique médiation du Christ avec le pluralisme des religions. Il va jusqu’à reconnaître dans les religions non chrétiennes, dont il prend très au sérieux « l’altérité » et « leur différence irréductible », d’authentiques « voies de salut ».
\begin{quote}
    « Si beaucoup d’hommes et de femmes sont sauvés en Jésus-Christ, ce n’est pas en dépit de leur appartenance à telle tradition religieuse, mais en elle et à travers elle »
\end{quote}
, avance celui qui par ailleurs fut directeur de l’École biblique et archéologique de Jérusalem (1996-1999), membre de la Conférence mondiale des religions et du Groupe de recherche islamo-chrétien.



« Il a poussé aussi loin que possible les avancées de Vatican II sur le dialogue interreligieux, dans des accents d’une grande actualité aujourd’hui », évoque le F. Laurent Lemoine, sous-prieur du couvent Saint-Jacques à Paris où il vivait.



Directeur de la prestigieuse collection « Cogitatio Fidei » aux Éditions du Cerf, Claude Geffré écrit lui-même beaucoup : Le Christianisme au risque de l’interprétation (1983), Passion de l’homme, passion de Dieu (1991), De Babel à Pentecôte (2006)… C’est sans doute cet ouvrage qui lui vaut la méfiance de Rome. En 2006, la Congrégation pour l’éducation catholique s’oppose au souhait de la Faculté de théologie de Kinshasa en RD-Congo de lui remettre un doctorat honoris causa pour l’ensemble de son œuvre. Sans toutefois donner d’explication. Mais le dicastère aurait exprimé oralement des réserves sur le regard critique qu’il porte sur certains aspects de la déclaration Dominus Iesus, publiée par le cardinal Ratzinger en 2000 et condamnant « des théories relativistes qui justifient le pluralisme religieux ».



Claude Geffré en fut très blessé, n’ayant jamais bien compris, ou à demi-mot. Mais, ses dernières années, ce brillant professeur qui forma des générations d’étudiants fut surtout affecté de ne plus enseigner. Très présent à sa communauté et attentif jusqu’au bout à ce qui se passait dans l’Église, doté de solides amitiés, il était aussi, selon le F. Lemoine, « un prédicateur de talent et un grand priant ».
\mn{La croix}

\subsection{}
\paragraph{Herméneutique}
\begin{Def}[herméneutique]
    
\end{Def}



Les précurseurs de l'herméneutique contemporaine

 
\textbf{Schleiermacher}
C'est Friedrich Schleiermacher (1768 – 1834) qui posa les bases de l'herméneutique contemporaine. Schleiermacher mit également en évidence le cercle herméneutique (l'expression est de Dilthey). Pour comprendre un texte, il faut avoir compris l'œuvre, mais pour comprendre l'œuvre, il faut avoir compris les textes.

\textbf{Dilthey}
Wilhelm Dilthey (1833 – 1911) voit dans l'herméneutique la possibilité d'une fondation pour les sciences humaines. Les sciences de la nature ne cherchent qu'à « expliquer » (Erklären) leur objet, tandis que les sciences de l'homme, et l'histoire en particulier, demandent également à « comprendre » (Verstehen) de l'intérieur et donc à prendre en considération le vécu.

 \paragraph{Naissance de l'herméneutique philosophique}

L'herméneutique philosophique contemporaine se conçoit comme une théorie de l'interprétation, et de la réception de l'œuvre (littéraire ou artistique). Elle questionne la textualité en elle-même, et son rapport à l'auteur (processus d'explication) et au lecteur (processus de compréhension).

L'herméneutique philosophique cherche à analyser ce qui se manifeste, ce qui se présente de soi dans l'œuvre d'art (perspective phénoménologique). Elle pose donc de manière originale le problème de la représentation et de la phénoménalisation, s'inspirant en cela des travaux novateurs de Husserl (lequel avait livré une théorie très élaborée de l'imagination, notamment dans les Ideen I, à défaut d'esthétique à proprement parler).

Le langage de l'art représente pour les herméneutes le lieu où la vérité de l'Être se déploie, au-delà de la description scientifique des étants particuliers. L'herméneutique se fonde ainsi sur une nouvelle interrogation du verbe « être », à la fois grammaticale, ontologique et esthétique, à partir des importants travaux de Martin Heidegger dans Être et Temps (et dans ses œuvres ultérieures, dont la tentation hermétiste sera critiquée)28.

L'herméneutique philosophique utilise comme paradigme majeur la poésie, notamment la poésie romantique, symboliste, surréaliste ou d'inspiration hermétiste, c'est-à-dire la poésie qui ne se comprend pas à la première lecture, mais qui nécessite un effort pour être décryptée. Les philosophes herméneutes analysent par exemple les textes et l'esprit de Hölderlin, Mallarmé, Valéry, Rilke, Artaud ou encore Ponge.

Le deuxième grand paradigme de l'herméneutique est le roman, notamment les œuvres subversives qui remettent en cause les normes traditionnelles d'écriture. Ainsi, on croisera sous la plume des grands herméneutes Rabelais, le Marquis de Sade, Joyce, Kafka, Bataille, ou encore d'autres grands écrivains comme Goethe ou Borges.

\textbf{Heidegger}
 
Martin Heidegger étend la conception de Dilthey et conçoit à un certain moment l'herméneutique comme la tâche même de la philosophie si l'existence – objet de la philosophie – demande à être interprétée et si elle n'est autre qu'un processus d'interprétation, une compréhension de soi. L'herméneutique est en ce sens un dépassement de la phénoménologie car elle s'applique à ce qui ne se montre pas, à détruire plutôt un rapport de conscience qui dissimule un rapport authentique à l'être. L'herméneutique constitue ainsi l'ontologie.

\textbf{Gadamer}
L'élève de Heidegger, Hans-Georg Gadamer publia en 1960 l'ouvrage qui passe encore pour son livre le plus important : Vérité et Méthode.

Cette œuvre affirme, en contestation de la fausse objectivité souvent présente dans les sciences humaines, que « la méthode ne suffit pas ». Une œuvre ne peut pas être expliquée uniquement selon notre propre horizon d'attente. La lecture est faite dans la tension existant entre le texte du passé et l'horizon d'attente actuel.

De plus, Gadamer affirme que « tout texte est réponse à une question. » Si le texte parle encore aux lecteurs présents, c'est qu'il répond encore à une question. Le travail de l'historien est de trouver à quelle question le texte répondait dans le passé et à laquelle il répond aujourd'hui.

\textbf{Ricœur}
Paul Ricœur entreprend une herméneutique du soi, herméneutique dans la mesure où le moi ne se connaît pas par simple introspection, mais par un ensemble de symboles. Il s'agit de déchiffrer le sens caché dans le sens apparent.

Pour Ricoeur, la psychanalyse est une forme d'herméneutique (interprétation des symptômes du malade)29.

\textbf{Jauss}
Hans Robert Jauss, appartenant à l'École de Constance, dans Pour une esthétique de la réception (1972), reprenant les enseignements de Gadamer, affinera la théorie herméneutique. Il proposera l'usage d'une « triade » herméneutique pour l'étude des œuvres.

La triade herméneutique de Jauss :

L'interprétation du texte où il faut réfléchir, rétrospectivement et trouver les significations.
La reconstruction historique, où l'on cherche à comprendre l'altérité portée par le texte.
La compréhension immédiate du texte, de sa valeur esthétique et de l'effet que sa lecture produit sur soi-même.
L'herméneute qui utilise ce modèle s'implique donc énormément dans l'étude et tente de comprendre la valeur novatrice de l'œuvre.

\textbf{Foucault}
En 1982, Michel Foucault intitule son cours au Collège de France : « herméneutique du sujet ». Il est question en réalité d’une « herméneutique de soi » au sens d’une forme de connaissance de soi. La notion fondamentale est la pensée grecque de l'epimeleia heautou (le souci de soi). Cette question est en même temps esthétique : une « esthétique de l’existence » entendue comme une éthique, soit la production de normes qui ne soient pas cryptées, mais que le sujet fonde ou découvre, et par lesquelles il se découvre également.

Foucault considère que la « généalogie » nietzschéenne, qui interprète les jugements de valeur (vrai/faux, bien/mal, beau/laid) à partir de l'histoire et de la physiologie (état de santé du corps), est une herméneutique30.


\section{L'interprétation de la Bible dans l'Église}
\mn{COMMISSION BIBLIQUE PONTIFICALE}

L'interprétation des textes bibliques continue de susciter un vif intérêt de nos jours, provoquant de vives discussions qui, ces dernières années, ont également pris de nouvelles dimensions. Etant donné l'importance fondamentale de la Bible pour la foi chrétienne, pour la vie de l'Église et pour les relations des chrétiens avec les fidèles des autres religions, la Commission biblique pontificale a été priée de s'exprimer sur cette question.  
 
  \begin{enumerate}
      \item donner une brève description des différentes méthodes et approches, [1] en indiquant leurs possibilités et leurs limites ;  (Historico critique, contextuel -  libération, femme,...)

      \item examinera quelques questions d'herméneutique ; 

      \item proposera une réflexion sur les dimensions caractéristiques de l'interprétation catholique de la Bible et sur ses relations avec les autres disciplines théologiques ; 

      \item examinera enfin la place qu'occupe l'interprétation de la Bible dans la vie de l'Église.    
  \end{enumerate}
 
 \subsection{QUESTIONS D'HERMÉNEUTIQUE  }


  

\textbf{A. Herméneutique philosophique  }

La voie de l'exégèse est appelée à être repensée en tenant compte de l'herméneutique philosophique contemporaine, qui a mis en évidence l'implication de la subjectivité dans le savoir, notamment historique. La réflexion herméneutique prend un nouvel élan avec la publication des travaux de Friedrich Schleiermacher, Wilhelm Dilthey et surtout Martin Heidegger. Dans le sillage de ces philosophies, mais aussi en s'en éloignant, plusieurs auteurs se sont penchés sur la théorie herméneutique contemporaine et ses applications à l'Écriture. Parmi eux on citera notamment Rudolf Bultmann, Hans Georg Gadamer et Paul Ricœur. Il n'est pas possible de résumer ici leur pensée ; il suffira d'indiquer quelques idées centrales de leur philosophie qui ont un impact sur l'interprétation des textes bibliques.[3]   

\textbf{1. Perspectives modernes  }

Constatant la distance culturelle entre le monde du Ier siècle et celui du XXe, et soucieux de faire parler à l'homme contemporain la réalité dont parle l'Écriture, Bultmann insiste sur la précompréhension nécessaire à toute compréhension et élabore la théorie de l'existence existentielle. Interprétation des écrits du Nouveau Testament. S'appuyant sur la pensée de Heidegger, il affirme que l'exégèse d'un texte biblique n'est pas possible sans quelques postulats qui guident sa compréhension. La précompréhension ( Vorverstandnis ) est fondée sur une relation vitale ( Leben verhaltnis) de l'interprète avec la chose dont parle le texte. Pour éviter le subjectivisme, il faut cependant que la précompréhension se laisse approfondir et enrichir, voire modifier et corriger, par ce dont parle le texte.   

S'interrogeant sur la conceptualité correcte qui définirait les questions sur la base desquelles les textes de l'Écriture pourraient être compris par l'homme d'aujourd'hui, Bultmann prétend trouver la réponse dans l'analytique existentielle de Heidegger. Les existentiels heideggériens auraient une portée universelle et offriraient les structures et les concepts les plus appropriés pour la compréhension de l'existence humaine révélée dans le message du Nouveau Testament.   

Gadamer souligne également la distance historique entre le texte et son interprète. Il reprend et développe la théorie du cercle herméneutique. Les anticipations et les idées préconçues qui marquent notre compréhension proviennent de la tradition qui nous soutient. Celle-ci consiste en un ensemble de données historiques et culturelles, qui constituent notre contexte vital, notre horizon de compréhension. L'interprète a le devoir d'entrer en dialogue avec la réalité dont parle le texte. La compréhension a lieu dans la fusion des différents horizons du texte et de son lecteur ( Horizontverschmelzung) et n'est possible que s'il existe un sentiment d'appartenance ( Zugehörigkeit), c'est-à-dire une affinité fondamentale entre l'interprète et son objet. L'herméneutique est un processus dialectique : comprendre un texte est toujours une compréhension plus large de soi-même.   

Del pensiero ermeneutico di Ricoeur, bisogna innanzi tutto sottolineare il risalto dato alla funzione di distanziamento come preliminare necessario a una giusta appropriazione del testo. Una prima distanza esiste tra il testo e il suo autore, poiché, una volta prodotto, il testo acquista una certa autonomia in rapporto al suo autore; inizia un percorso di significato. Un'altra distanza esiste tra il testo e i suoi lettori successivi; questi devono rispettare il mondo del testo nella sua alterità. I metodi di analisi letteraria e storica sono perciò necessari all'interpretazione. Tuttavia il significato di un testo può essere dato pienamente solo se viene attualizzato nel vissuto dei lettori che se ne appropriano. A partire dalla loro situazione, questi sono chiamati a far emergere significati nuovi, in linea con il senso fondamentale indicato dal testo. La conoscenza biblica non deve fermarsi al linguaggio, ma cerca di raggiungere la realtà di cui parla il testo. Il linguaggio religioso della Bibbia è un linguaggio che “fa pensare”, un linguaggio di cui non si cessa di scoprire le ricchezze di significato, un linguaggio che ha di mira una realtà trascendente e che, nello stesso tempo, rende la persona umana conscia della dimensione profonda del suo essere.   

\textbf{2. Utilité pour l'exégèse  }

Qu'en est-il de ces théories contemporaines de l'interprétation des textes ? La Bible est la Parole de Dieu pour tous les âges qui se succèdent dans l'histoire. Par conséquent, on ne peut ignorer une théorie herméneutique qui permet d'intégrer les méthodes de la critique littéraire et historique dans un modèle d'interprétation plus large. Il s'agit de surmonter la distance entre le temps des auteurs et des premiers destinataires des textes bibliques et notre époque contemporaine, afin d'actualiser correctement le message des textes pour nourrir la vie de foi des chrétiens. Toute exégèse des textes est appelée à être complétée par une « herméneutique », au sens récent du terme.   

La nécessité d'une herméneutique, c'est-à-dire d'une interprétation dans notre monde d'aujourd'hui, trouve un fondement dans la Bible elle-même et dans l'histoire de son interprétation. L'ensemble des écrits de l'Ancien et du Nouveau Testament est présenté comme le produit d'un long processus de réinterprétation des événements fondateurs, en lien étroit avec la vie de la communauté des croyants. Dans la tradition ecclésiale, les premiers interprètes de l'Écriture, les pères de l'Église, pensaient que leur exégèse des textes n'était complète que s'ils faisaient ressortir leur sens pour les chrétiens de leur temps dans leur situation. On n'est fidèle à l'intentionnalité des textes bibliques que dans la mesure où l'on essaie de retrouver, au cœur de leur formulation, la réalité de foi qu'ils expriment et si l'on relie cette réalité à la   

L'herméneutique contemporaine est une saine réaction au positivisme historique et à la tentation d'appliquer à l'étude de la Bible les critères d'objectivité utilisés dans les sciences naturelles. D'une part, les événements enregistrés dans la Bible sont des événements interprétés ; d'autre part, toute exégèse des récits de ces événements implique nécessairement la subjectivité de l'exégète. Une bonne connaissance du texte biblique n'est accessible qu'à celui qui a une affinité vécue avec le sujet du texte. La question qui se pose à tout interprète est la suivante : quelle théorie herméneutique permet une compréhension correcte de la réalité profonde dont parle l'Écriture et une expression qui ait un sens pour l'homme d'aujourd'hui ?   

 

Ce sens est exprimé dans les textes. Pour éviter le subjectivisme, il faut donc qu'une bonne actualisation repose sur l'étude du texte et que les hypothèses de lecture soient constamment vérifiées sur le texte.   

L'herméneutique biblique, même si elle fait partie de l'herméneutique générale de tout texte littéraire et historique, est en même temps un cas unique de cette herméneutique. Ses spécificités lui viennent de son objet. Les événements du salut et leur accomplissement en la personne de Jésus-Christ donnent un sens à toute l'histoire humaine. Les nouvelles interprétations historiques ne peuvent être que le dévoilement ou l'exposition de ces richesses de sens. Le récit biblique de ces événements ne peut être entièrement compris par la seule raison. Son interprétation doit être guidée par des présupposés particuliers, comme la foi vécue en communauté ecclésiale et la lumière de l'Esprit. Avec la croissance de la vie dans l'Esprit, la compréhension du lecteur des réalités dont parle le texte biblique grandit également. 

  

\textbf{B. Significations des Ecritures inspirées  }

L'apport moderne de l'herméneutique philosophique et les développements récents de l'étude scientifique de la littérature permettent à l'exégèse biblique d'approfondir la compréhension de sa tâche, dont la complexité est devenue plus évidente. L'exégèse ancienne, qui ne pouvait évidemment pas prendre en considération les besoins scientifiques modernes, attribuait à chaque texte de l'Écriture des niveaux de sens différents. La distinction la plus courante était celle entre le sens littéral et le sens spirituel. L'exégèse médiévale distinguait trois aspects différents au sens spirituel, en rapport respectivement avec la vérité révélée, le comportement à suivre et l'accomplissement final. D'où le célèbre distique d'Augustin de Danemark (XIIIe siècle) : « Littera gesta docet, quid credas allegoria, moralis quid agas,».   

En réaction contre cette multiplicité de sens, l'exégèse historico-critique a adopté, plus ou moins ouvertement, la thèse de l'unicité du sens, selon laquelle un texte ne peut avoir plusieurs sens simultanément. Tout l'effort de l'exégèse historico-critique consiste à définir le sens précis d'un texte biblique donné dans les circonstances où il a été composé. Mais cette thèse se heurte désormais aux conclusions des sciences du langage et de l'herméneutique philosophique, qui affirment la polysémie des textes écrits.   

Le problème n'est pas simple et ne se pose pas de la même manière pour toutes sortes de textes : contes historiques, paraboles, oracles, lois, proverbes, prières, hymnes, etc. Cependant, il est possible de présenter quelques principes, en tenant toujours compte de la diversité des opinions.   

\textbf{1. Sens littéral  }

Il est non seulement légitime, mais indispensable d'essayer de définir le sens précis des textes tels qu'ils ont été composés par leurs auteurs sens dit « littéral ». Saint Thomas d'Aquin affirmait déjà son importance fondamentale (S. Th., I, q. 1, a. 10, ad 1).   

Le sens littéral ne se confond pas avec le sens « littéraliste » sur lequel se fondent les intégristes. Il ne suffit pas de traduire le texte mot à mot pour en saisir le sens littéral. Il faut le comprendre selon les conventions littéraires de l'époque. Lorsqu'un texte est métaphorique, son sens littéral n'est pas celui qui résulte du sens immédiat des mots (par exemple : « Portez vos reins », Lc 12, 35), mais celui qui correspond à l'usage métaphorique des termes (« Avoir une attitude de disponibilité". Lorsqu'il s'agit d'une histoire, le sens littéral n'implique pas nécessairement l'affirmation que les faits racontés se sont réellement produits ; en effet une histoire peut ne pas appartenir au genre historique, mais être le fruit de l'imagination.   

Le sens littéral de l'Écriture est celui exprimé directement par les auteurs humains inspirés. Étant le fruit de l'inspiration, ce sens est aussi voulu par Dieu, l'auteur principal. Cela se discerne grâce à une analyse précise du texte, situé dans son contexte littéraire et historique. La tâche principale de l'exégèse est précisément de conduire à cette analyse, en utilisant toutes les possibilités de la recherche littéraire et historique, afin de définir le sens littéral des textes bibliques avec la plus grande exactitude possible ( Divino afflante Spiritu , EB 550). A cet effet, l'étude des genres littéraires anciens est particulièrement nécessaire ( ibid . 560).   

Le sens littéral d'un texte est-il unique ? En général, oui; mais ce n'est pas un principe absolu, et cela pour deux raisons. D'une part, un auteur humain peut vouloir se référer à plusieurs niveaux de réalité en même temps. L'affaire est courante dans la poésie. L'inspiration biblique ne dédaigne pas cette possibilité de la psychologie humaine et du langage ; le quatrième évangile fournit de nombreux exemples. D'un autre côté, même lorsqu'une expression humaine semble n'avoir qu'un seul sens, l'inspiration divine peut conduire l'expression d'une manière qui produit une ambivalence. C'est le cas de l'expression de Caïphe dans Jn 11, 50. Elle exprime à la fois un calcul politique immoral et une révélation divine. Ces deux aspects appartiennent tous deux au sens littéral, car ils sont tous deux mis en évidence par le contexte.   

Il vaut, en particulier, être attentif à l'aspect dynamique de nombreux textes. Le sens des psaumes royaux, par exemple, ne doit pas être strictement limité aux circonstances historiques de leur production : en parlant du roi, le psalmiste évoquait à la fois une institution royale et une vision idéale de la monarchie, conformément à Le plan de Dieu, de sorte que son texte allait au-delà de l'institution de la monarchie telle qu'elle s'était manifestée dans l'histoire. L'exégèse historico-critique a trop souvent eu tendance à limiter le sens des textes, en le liant exclusivement à des circonstances historiques précises. Elle doit plutôt chercher à éclairer le sens de pensée exprimé par le texte, sens qui, au lieu d'inviter l'exégète à en limiter le sens, lui suggère au contraire d'en percevoir les prolongements plus ou moins prévisibles.

Un courant de l'herméneutique moderne a souligné la différence de situation affectant la parole humaine lorsqu'elle est mise par écrit. Un texte écrit a la capacité d'être placé dans de nouvelles circonstances, qui l'éclairent de différentes manières, ajoutant de nouvelles déterminations à son sens. Cette capacité du texte écrit est efficace surtout dans le cas des textes bibliques, reconnus comme la Parole de Dieu.En fait, ce qui a poussé la communauté croyante à les conserver, c'est la conviction qu'ils continueraient à être porteurs de lumière et de vie pendant des générations. futurs. Le sens littéral est, dès le départ, ouvert à des développements ultérieurs, qui se produisent grâce à des « relectures » dans des contextes nouveaux.

Il ne s'ensuit pas qu'il soit possible d'attribuer une quelconque signification à un texte biblique, en l'interprétant de manière subjective. Au contraire, il faut rejeter comme inauthentique toute interprétation hétérogène par rapport au sens exprimé par les auteurs humains dans leur texte écrit. Admettre des significations hétérogènes reviendrait à dépouiller le message biblique de ses racines, qui sont la Parole de Dieu historiquement communiquée, et à ouvrir la porte à un subjectivisme incontrôlable.   

\textbf{2. Sens spirituel}

Cependant, il n'est pas question de prendre "hétérogène" au sens strict, contrairement à toute possibilité d'épanouissement supérieur. L'événement pascal, la mort et la résurrection de Jésus, a établi un contexte historique radicalement nouveau, qui éclaire d'une manière nouvelle les textes anciens et leur fait subir un changement de sens. En particulier, certains textes qui, dans des circonstances antiques, devaient être considérés comme des hyperboles (par exemple, l'oracle dans lequel Dieu, parlant d'un fils de David, a promis de rendre son trône stable pour toujours : 2Sam 7, 12 -13 ; 1Ch 17 , 11-14), ces textes doivent maintenant être pris au pied de la lettre, car "le Christ, ressuscité des morts, ne meurt plus" (Rm 6, 9). Les exégètes qui ont une notion limitée, « historiciste », du sens littéral croiront qu'il y a ici hétérogénéité. Ceux qui sont ouverts à

Dans des cas de ce genre on parle de "sens spirituel". En règle générale, on peut définir le sens spirituel, compris selon la foi chrétienne, comme le sens exprimé par les textes bibliques lorsqu'ils sont lus sous l'influence de l'Esprit Saint dans le contexte du mystère pascal du Christ et de la vie nouvelle qui en résulte. Ce contexte existe réellement. Le Nouveau Testament y reconnaît l'accomplissement des Ecritures. Il est donc normal de relire les Écritures à la lumière de ce nouveau contexte, celui de la vie dans l'Esprit.

De la définition donnée, plusieurs éclaircissements utiles peuvent être tirés sur la relation entre le sens spirituel et le sens littéral. Contrairement à l'opinion courante, il n'y a pas nécessairement de distinction entre ces deux sens. Lorsqu'un texte biblique se réfère directement au mystère pascal du Christ ou à la vie nouvelle qui en résulte, son sens littéral est spirituel. Et c'est le cas habituel dans le Nouveau Testament. Il s'ensuit que l'exégèse chrétienne parle le plus souvent d'un sens spirituel à propos de l'Ancien Testament. Mais déjà dans l'Ancien Testament, les textes ont dans de nombreux cas une signification religieuse et spirituelle comme sens littéral. La foi chrétienne y reconnaît une relation anticipée avec la vie nouvelle opérée par le Christ.

Lorsqu'il y a distinction, le sens spirituel ne peut jamais être privé de relations avec le sens littéral qui reste sa base indispensable ; autrement, on ne pourrait pas parler de "l'accomplissement" de l'Ecriture. En effet, pour pouvoir parler d'épanouissement, une relation de continuité et de conformité est indispensable. Mais il doit aussi y avoir une transition vers un niveau supérieur de réalité.

Le sens spirituel ne se confond pas avec les interprétations subjectives dictées par l'imagination ou par la spéculation intellectuelle. Elle jaillit du rapport du texte avec certains faits réels qui ne lui sont pas étrangers, l'événement pascal et son inépuisable fécondité, qui constituent le sommet de l'intervention divine dans l'histoire d'Israël, au profit de toute l'humanité.

La lecture spirituelle, faite en communauté ou individuellement, ne découvre un sens spirituel authentique que si elle est maintenue dans ces perspectives. Trois niveaux de réalité sont alors mis en relation : le texte biblique, le mystère pascal et les circonstances présentes de la vie dans l'Esprit.

L'ancienne exégèse, convaincue que le mystère du Christ constitue la clé d'interprétation de toutes les Écritures, s'est efforcée de trouver un sens spirituel dans les moindres détails des textes bibliques, par exemple dans chaque prescription des lois rituelles, en utilisant les méthodes rabbiniques ou inspiré de l'allégorie hellénistique. L'exégèse moderne ne peut accorder une véritable valeur interprétative à ce genre de tentative, quelle qu'ait été leur utilité pastorale dans le passé (cf. Divino afflante Spiritu , EB 553).

L'un des aspects possibles du sens spirituel est le sens typologique, dont on dit généralement qu'il n'appartient pas à l'Écriture elle-même, mais aux réalités exprimées par l'Écriture : Adam est la figure du Christ (cf. Rm 5, 14). ), le déluge est la figure du baptême (1 P 3, 20-21), etc. En effet, le rapport de typologie se fonde ordinairement sur la manière dont l'Écriture décrit la réalité antique (cf. la voix d'Abel : Gn 4, 10 ; He 11, 4 ; 12, 24) et non simplement sur cette réalité. Par conséquent, il s'agit bien alors d'un sens de l'Écriture. 
\textbf{
3. Plein sens  
}
Relativement récente, l'appellation "plein sens" ( sensus plenior ) donne lieu à des discussions. Le sens plein est défini comme un sens plus profond du texte, voulu par Dieu, mais pas clairement exprimé par l'auteur humain. Son existence est découverte dans un test biblique lorsqu'elle est étudiée à la lumière d'autres textes bibliques qui l'utilisent ou dans sa relation avec le développement interne de la révélation.

Il s'agit alors soit du sens qu'un auteur biblique attribue à un texte biblique antérieur à lui, lorsqu'il le reprend dans un contexte qui lui donne un nouveau sens littéral, soit du sens qu'une tradition doctrinale authentique ou un définition conciliaire donne à un texte de la Bible. Par exemple, le contexte de Mt 1, 23 donne tout son sens à l'oracle d'Is 7, 14 sur l' almah qui concevra un fils, en utilisant la traduction de la Septante ( parthenos): "La vierge concevra". L'enseignement patristique et conciliaire sur la Trinité exprime tout le sens de l'enseignement néotestamentaire sur Dieu le Père, le Fils et l'Esprit. La définition du péché originel par le Concile de Trente donne tout le sens de l'enseignement de Paul en Rom 5, 12-21 sur les conséquences du péché d'Adam pour l'humanité. Mais lorsqu'il n'y a pas de contrôle de ce genre, par un texte biblique explicite ou par une tradition doctrinale authentique, le recours à un prétendu sens plein pourrait conduire à des interprétations subjectives dénuées de toute validité.

En définitive, on pourrait considérer le « sens plein » comme une autre façon de désigner le sens spirituel d'un texte biblique, dans le cas où le sens spirituel est distingué du sens littéral. Son fondement est le fait que l'Esprit Saint, auteur principal de la Bible, peut guider l'auteur humain dans le choix de ses expressions de telle manière qu'elles expriment une vérité dont il ne perçoit pas toute la profondeur. Cela se révèle plus pleinement au fil du temps, grâce, d'une part, à d'autres réalisations divines qui manifestent mieux la portée des textes, et grâce aussi, d'autre part, à l'insertion des textes dans le canon de l'Écriture. Ainsi se crée un nouveau contexte qui fait ressortir le potentiel de sens que le contexte primitif laissait dans l'ombre.


\section{Texte}

\paragraph{Historique}
\begin{itemize}
\item {\textit{Epreuve de la catholicité}} Corneille
    \item \textit{L'hellenistation du Christianisme}

    \begin{quote}
        « Nicée n'est pas l'hellénisation, mais la déshellénisation ou la libération de l'image chrétienne de Dieu hors de l'impasse et des divisions où l'entraînait l'hellénisme. Ce ne sont pas les Grecs qui ont fait Nicée, c'est Nicée qui a surmonté les philosophes grecs » \mn{A. GRILLMEIER, dans Comment être chrétien ? La réponse de Hans Küng, Paris, DDB, 1979, p. 128}.
    \end{quote}
    \item \textit{les âges culturels du Christianisme} Une grande période. Mais inclut
\end{itemize}



\section{Héritage}

\paragraph{Etudres 2015}
LE CHRISTIANISME À LA RENCONTRE DES CULTURES
Ludovic Lado, Pierre de Charentenay, Jean-Claude Guillebaud
S.E.R. | « Études »
2015/1 janvier | pages 7 à 18
\begin{quote}
    Il nous reste à éclairer cette résultante : si les autres grandes religions
méritent d’être respectées et reconnues, alors comment définir
le christianisme ? En quoi est-il autre ? Si j’ai réellement la foi, je ne
puis relativiser le message évangélique, mais comment privilégier dès
lors « mon » christianisme sans dédaigner de facto les autres cultes ?
Il n’est pas d’interrogation plus actuelle, plus urgente, plus embarrassante
que cette question de la singularité chrétienne. Nous sommes
    convaincus que, comme « religion de l’Évangile », notre christianisme
est radicalement « autre ». Mais il nous faut reconnaître – comme le
souligne le dominicain Claude Geffré –, que notre vérité « n’est ni
exclusive, ni même inclusive de toute autre vérité d’ordre religieux\sn{Claude Geffré, Le Christianisme comme religion de l’Évangile, Cerf, 2012.} ».
\end{quote}