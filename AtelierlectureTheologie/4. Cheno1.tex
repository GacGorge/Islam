\chapter{Dieu au pluriel, Rémi Cheno}
\mn{Guillaume Gorge le 6/10/22}
\section{Remi Cheno}

\paragraph{dominicain de la province de France} Remi Cheno est né en 1959. Après l'école polytechnique, l'ENSG, il entre au noviciat dominicain en 1994 et continue ainsi ses études, passant de bac +6 à bac +13 après sa thèse de document en théologie dogmatique. Spécialisé en ecclésiologie, il s’est intéressé par la suite plus spécialement à la pneumatologie et à l’eschatologie. 

\paragraph{L'Ideo comme terreau de la question des religions }Il rejoint l’Idéo, \textsc{l'Institut Dominicain des Etudes Orientales} au Caire en septembre 2013.  L'Ideo est un Institut bénéficiant d'une large bibliothèque scientifique sur l'Islam et lieu de recherche et de dialogue avec l'Islam. Il a été secrétaire général de l’Idéo d’octobre 2014 à décembre 2020. Dans cette communauté de spécialistes de l'Islam (Adrien Candiard, Jean Druel,  et maintenant Emmanuel Pisani), il n'étudie pas directement les textes arabes, mais réfléchit davantage sur la théologie (chrétienne) du pluralisme religieux, stimulé par l'environnement intellectuel. Après le succès du livre \textit{Comprendre l'Islam ou pourquoi on y comprend rien} d'Adrien Candiard\sn{\textsc{Adrien Candiard}, \textit{Comprendre l'Islam - ou plutôt : pourquoi on n'y comprend rien }Poche – 11 mai 2016}, Jean Druel\sn{\textsc{Jean Druel}, \textit{Je crois en Dieu ! — Moi non plus}, Paris, Éditions du Cerf, 2017, 160 pages. Le livre est plus une \textit{praxis} du dialogue et de l'incompréhension, en particulier sur les différents niveaux de lecture possibles} et Rémi Chéno écrivent sur le dialogue inter-religieux et comment penser le pluralisme religieux \sn{\textsc{Remi Cheno},\textit{Dieu au pluriel : Penser les religions}, Paris, Éditions du Cerf, 2017, 160 pages.}. 



\paragraph{Un texte écrit près de 30 ans après Dupuis} Nous avons quitté l'apogée de l'optimisme sur le dialogue inter-religieux des années 80 et la lecture d'un progrès du dialogue, d'une vision exclusiviste à une vision pluraliste.


\paragraph{Le contexte des attentats de 2015} Même si les éditions du Cerf sont des éditions dominicaines prolifiques, La production de l'Ideo en 2016 et 2017 aussi intense à destination d'un public large doit être replacée dans le questionnement des attentats de 2015 et le \textit{choc} de français porteurs d'un Islam fortement anti-occidental et violent. C'est que : 
\begin{quote}
     La mondialisation fait cohabiter, parfois paisiblement, souvent de façon chaotique, les religions , les systèmes ou les modèles dans un même voisinage. \sn{p 10}
\end{quote}

 

\section{Dieu au pluriel, penser les religions}
\paragraph{Problématique du livre} Dans ce nouvel environnement que le premier chapitre essaye de qualifier, la question du livre est : \textit{Le dialogue interreligieux a-t-il encore un sens et sous quelles formes ? À quelles conditions ?} Rémi Chéno le fait de façon structurée mais la plus accessible possible. 

\paragraph{Les différentes approches chrétiennes} Le premier chapitre s'applique à définir le monde dans lequel nous sommes, marqué par la condition postmoderne. Puis Rémi Chéno, dans son deuxième chapitre reprend et complète la typologie d'Alan Race\sn{il cite \textit{Christians and religious pluralism, 1983}} des différentes théologies chrétiennes du pluralisme religieux, tout en ne retenant pas la notion de \textit{progrès} qui était présente chez Alan Race. Pour cela, il prend un grand théologien pour chacun des courants : 
\begin{itemize}
    \item l'exclusivisme avec Karl Barth
    \item l'inclusivisme avec Karl Rahner, abandonnant l'ecclesiocentrisme
    \item les théologies pluralistes, abandonnant le christocentrisme avec John Hick et Paul Knitter (regnocentrisme). 
    \item une approche post-libérale dont il annonce dès l'introduction qu'il la développera particulièrement. Il suit particulièrement deux théologiens anglo-saxons,  George Lindbeck  \sn{l'auteur cite \textit{the Nature of Doctrine}} et DiNoia. Cette approche ne cherche pas à trouver un plus petit dénominateur commun mais ce qui \textit{cristallise dans chaque religion} et la rend pertinente au futur de l'humanité, avec une analogie linguistique (religion comme une langue). 
\end{itemize}
Il termine par l'hypothèse qu'au sein d'une personne, on puisse faire l'expérience d'être \textit{bilingue} en différentes religions, avec une religion maternelle mais la possibilité de comprendre le champ culturel de l'autre.  Il ne s’agit pas de justifier de la double
appartenance : le théologien avertit de l’absolue nécessité de la non confusion ;
mais il s’agit de pouvoir goûter l’autre tradition, au point de se sentir comme
l’autre croyant. 

\paragraph{vulgarisation} il fait bien les transitions. Chapitre 1 et 2 sont une introduction. Il décrit la post-modernité, fait une classification et les illustre.

\section{Premier chapitre : la condition Post Moderne}

\paragraph{la problématique : dans quel monde vivons-nous ?} L'hypothèse d'une \textit{culture techno-scientifique commune et universelle}\sn{
   \textit{"Allions nous tous devenir des semblables et nos cultures allaient elles se fondre en une culture technoscientifique commune et universelle ?"} p.9}, qui pouvait sous tendre la modernité (avec sa foi en la raison et dans le progrès) et de récits englobant, n'est plus tenable.
La thèse de Rémi Cheno est de reprendre l'expression de Lyotard de \textit{post-modernité} pour qualifier notre monde. 



\paragraph{La coexistence des différents modèles au XXI\textsuperscript{e}} La mondialisation fait cohabiter, parfois paisiblement, souvent de façon chaotique, les religions , les systèmes ou les modèles dans un même voisinage. \mn{p 10}  

\paragraph{perte de récits unifiants} Les récits de la "modernité" croient en l'émancipation du sujet rationnel et celui de l'histoire de l'esprit universel. Face à la disparition de récits unifiants, On se fait sa petite sauce entre différentes religions et pensées \mn{p. 11}

  
\subsection{Quelles réponses individuelles possibles ? }

\paragraph{la posture intégriste/intégrale} Le but de cette posture est de refaire un récit unifiant de sa vie à partir d'anciens grands récits à mettre à jour  (p. 16)
\paragraph{la réponse identitaire} le but n'est plus de donner du sens (p 17) mais de définir non ce que nous sommes mais en prenant comme définition le groupe lui-même ("ceux qui sont pieux" par opposition "qu'est ce que la piété"). "ils sont chrétiens" "français".
\paragraph{la thèse de l'auteur : vivre aux éclats} 
\begin{quote}
  j'aime ce monde. a condition post moderne n'est pas un fardeau, ... chance d'interactions avec les autres... 
    Elle est une invitation à la rencontre, à l'échange et, peut être, au dialogue. p. 20
\end{quote}

\subsection{Et pour la théologie chrétienne du pluralisme religieux ? }

\paragraph{question classique de la théologie chrétienne du pluralisme religieux} penser l'existence des autres religions (p 22)

\paragraph{Un enjeu pour les religions} Si la modernité n'a pas été "tendre" avec les religions, l'A. indique bien que les dangers pour la religion est maintenant différent. Relégué par la modernité, le problème n'est plus le grand récit de l'athéisme moderne, c'est cette typologie post moderne où les rationalités s'additionnent et se juxtaposent plutôt qu'elles ne se réfutent. 


\section{Quelques aspects critiques} 

\paragraph{Accéleration} Notre monde est certes marqué par la réduction du temps et de l'espace, avec plus de Volatilité, d'Incertude, de complexité et d'ambiguité qui rendent plus complexes l'existence d'un récit, mais notion de degré, pas forcément de rupture.

\paragraph{Réduction du monde}d'une certaine façon, nous faisons tous l'expérience de St François Xavier, rencontrant brutalement les cultures autres, mais depuis notre fauteuil.


\paragraph{Quel postulat de Post-modernité ? Risque de myopie historique} La post-modernité postule que les grands récits unifiants sont morts. Mais n'est ce pas le sentiment de toute personne quand le monde change. "Penser après la Shoah" : n'y a t'il pas la même effrayante perspective que certains récits unifiants (la nation,...) ne sont plus audibles ? Un autre exemple plus présent : l'évolution de la vision du sexe avec des enfants depuis 1968. Certaines valeurs ont pu être un moment considérées comme séquelles identitaires d'un passé révolu ? 

\paragraph{de nouveaux récits fondateurs} Il eest intéressant de voir l'oeuvre de Luther et de Saint Ignace à une époque, la réformation, où les anciens grands récits n'était plus performatif. Ils ont tous les deux \textit{réduit} la foi Chrétienne à son coeur essentiel et pertinent pour l'époque.

\paragraph{Somewhere - Anywhere - village planétaire} On peut se demander si Remi Cheno n'est pas touché par sa capacité à évoluer dans ce nouveau monde comme un anywhere\sn{David Goodhart en opposant Anywhere et Somewhere dans un ouvrage au titre lumineux, Les deux clans. La nouvelle fracture mondiale (The road to somewhere, 2017), Écrit à la lumière du vote sur le Brexit (2016), l'essai renvoie aussi à l'élection de Donald Trump (2016) et à la révolte des Gilets jaunes en France (2018).} : 

\begin{quote}
    La mentalité des Anywhere — ceux qui sont n'importe où, les Partout dans la traduction française, mais on conservera ici le terme anglais — lui semble révélatrice d'un « individualisme progressiste ». «Elle accorde beaucoup de valeur à l'autonomie, à la mobilité et à l'innovation, et nettement moins à l'identité de groupe, à la tradition et aux pactes nationaux (Église, patrie, famille). La plupart des Anywhere voient d'un bon oeil l'immigration, l'intégration européenne et la diffusion des droits humains, autant d'éléments qui ont tendance à diluer les revendications nationales » (p. 19). ... Un groupe social représentant 20 à 25 \% de la population de nos démocraties, qui «  prédomine parmi les décideurs et les faiseurs d'opinion » (p. 48), et comporte un sous-groupe plus radical de 5 \% qu'il appelle les « villageois planétaires » et qui, lui, se recrute principalement « dans l'enseignement supérieur et dans les milieux de la création» (p. 61). \sn{Atlantico. « A\textit{nywhere vs somewhere : le nouveau clivage entre les gagnants et les perdants de la mondialisation au cœur du malaise démocratique }». Atlantico, 13 février 2022. \href{https://atlantico.fr/article/decryptage/anywhere-vs-somewhere---le-nouveau-clivage-entre-les-gagnants-et-les-perdants-de-la-mondialisation-au-coeur-du-malaise-democratique-france-citoyens-emmanuel-macron-2022-christophe-boutin-olivier-dard-frederic-rouvillois}{Atlantico} .
}
\end{quote}

\newpage


\section{PostModernité}

\paragraph{Lyotard} En philosophie, le postmodernisme devient sujet de débat en 1979 avec la publication de l'ouvrage de Jean-François Lyotard, La Condition postmoderne, que l'auteur caractérise par la perte de crédibilité et le déclin des métarécits qui sous-tendent le discours philosophique de la modernité. C'est autour de cette question que va éclater une querelle, dont les protagonistes seront J.-F. Lyotard, Jürgen Habermas et Richard Rorty. Elle a pour enjeu principal la question de la possibilité d'une sortie effective de la modernité. Les trois philosophes s'accordent pour reconnaître que, après Nietzsche et Heidegger, une manière absolue et globalisante d'envisager l'histoire, l'homme et la société, comme le voulaient les idéologies et les philosophies modernes de l'histoire, est devenue irrecevable. Cette convergence ne les empêche pas de s'opposer quant à l'interprétation à donner d'une telle sortie de la modernité. D'après Lyotard, la fin des métarécits de la modernité, c'est-à-dire du discours des Lumières et de celui de l'idéalisme, entraîne la fin aussi bien du subjectivisme que de l'humanisme, comme Michel Foucault l'avait déjà établi de son côté. Les philosophes des Lumières faisaient de l'audace du savoir le moteur de l'émancipation du genre humain tout entier ; quant à l'idéalisme absolu, il faisait dépendre la légitimité de tout savoir de la possibilité de s'inscrire dans la perspective d'une doctrine de la science encyclopédique et universelle. En critiquant les penseurs des Lumières, Lyotard souligne que la raison ne saurait renvoyer automatiquement à une promesse d'émancipation et, surtout, que rien ne garantit la nécessité d'un lien entre les énoncés descriptifs de la science et les énoncés pratiques et prescriptifs visant l'émancipation de l'humanité. \sn{\href{https://www.universalis.fr/encyclopedie/postmodernisme/3-philosophie/#:~:text=En\%20philosophie\%2C\%20le\%20postmodernisme\%20devient,discours\%20philosophique\%20de\%20la\%20modernit\%C3\%A9.}{Post Modernité - E Universalis}}


\section{Mideo - critique d'Emmanuel Pisani }

« Bricolage religieux », juxtaposition des croyances et de pratiques multiples et contrastées, quête d’identités fortes balisant le quotidien, conversions fragmentées, perte des références aux grands récits mythologiques, métissage… le religieux connaît un redéploiement de ses manifestations dans un contexte sociétal caractérisé par la généralisation de « l’individualisme narcissique » (Lasch, Sennet, Lipovetsky). Dans ce contexte sociétal, quelles lumières la théologie apporte-t-elle pour penser la condition post-moderne, le pluralisme religieux, la cacophonie venant de la nouvelle Babel ? Où est Dieu et qu’est-Il ? Le dialogue interreligieux a-t-il encore un sens et sous quelles formes ? À quelles conditions ? L’ouvrage de Rémi Chéno, dominicain, tente d’apporter quelques réponses en sept chapitres dans un petit livre précis et efficace, sachant allier d’une manière surprenante mais réussie, concepts philosophiques et culture pop.

 
2Dans les premiers chapitres, l’A. s’attache à la présentation classique des théologies chrétiennes des religions selon l’approche paradigmatique de l’exclusivisme, de l’inclusivisme et du pluralisme. Ce sont les théologiens Karl Barth, Karl Rahner qui sont présentés pour les deux premiers paradigmes. Pour le troisième, R. Chéno convoque tour à tour Hick, Knitter, Panikkar, Amaladoss, Pieris. Il met en lumière le paradoxe d’une théologie libérale qui invite à la reconnaissance de la faillibilité de chaque religion – à commencer par la sienne – en vue de promouvoir une approche mutualiste, mais qui glisse « vers un impérialisme de la pensée ». Car nos théologiens pluralistes, « à vouloir entrer dans un dialogue mutuel, tendent à réduire la diversité, voire à la rejeter » (p. 104). Quant aux conditions du dialogue, ils imposent la aussi une vision impérialiste puisqu’il faudrait abandonner tout ce qui n’est pas négociable. Le cadre dans lequel est pensé le pluralisme devrait s’imposer à toutes les religions. Or, ce cadre ne reflète-t-il pas des conceptions propres aux traditions de leurs auteurs ? Par ailleurs, la prétention à s’extraire de sa propre tradition religieuse pour contempler les convergences ne revient-elle pas à adopter le point de vue de Dieu lui-même ? C’est en réponse à ses critiques que s’est développée ces dernières années l’approche post-libérale et dont le livre du théologien luthérien américain George A. Lindbeck, The Nature of Doctrine, paru en 1984, pose la première pierre36. Dans ce chapitre au demeurant le plus original, Chéno montre que contrairement à l’approche libérale centrée sur l’expérience du sujet et qui serait le lieu où les croyants se retrouvent, le post-libéralisme appelle à ne pas minimiser les différences, bien au contraire ; dans une approche utilitariste, elles sont mêmes précieuses car ce sont ces différences qui donnent à chaque religion une valeur singulière et vitale. Lindbeck écrit : « Une religion contribuera probablement davantage au futur de l’humanité si elle préserve ses propres caractéristiques et son intégrité que si elle cède aux tendances homogénéisantes qui vont avec l’expressivisme expérientiel libéral » (p. 115). Pour lui, chaque religion a son langage, sa grammaire, et produit une vision du monde propre, singulière. Mais il s’ensuit une question : la vérité d’une doctrine est-elle liée exclusivement à la communauté qui la produit, à sa cohérence interne ? Plus encore, le caractère incommensurable des religions peut-il être postulé indépendamment du terreau culturel de ces religions ? Est-il vrai que le mot « salut » est incommensurable entre l’islam et le christianisme ? Et qu’en est-il du dialogue interreligieux ? Dans la perspective post-libérale, le dialogue interreligieux se fonde sur la reconnaissance de la cohérence interne à chaque religion et à sa prétention à l’exclusivité (insurpassabilité) (p. 129). Pour Chéno, l’approche post-libérale permet de défendre « le caractère providentiel de la diversité des religions37 ». Dans cette perspective, le dialogue interreligieux connaît une nouvelle acuité avec la pluralité des territoires dans lesquels habite l’individu, nonobstant leur caractère incommensurable. Par suite, Chéno soutient la possibilité « d’habiter plusieurs religions à la fois » (p. 147). Mais attention ! Il ne s’agit pas de justifier de la double appartenance : le théologien avertit de l’absolue nécessité de la non confusion ; mais il s’agit de pouvoir goûter l’autre tradition, au point de se sentir comme l’autre croyant. Ainsi, à propos de l’islam, il écrit : « Entendons-nous bien : il reste chrétien, mais il devient progressivement expert du passage, de la traduction, et peut goûter à la fois le caractère insurpassable de ses dogmes chrétiens et celui de la foi musulmane » (p. 147). Une telle perspective a ses exigences et sa cohérence et ne saurait s’assimiler au syncrétisme ou aux logiques du bricolage postmoderne (p. 148).

3L’ouvrage sera utile pour les étudiants en théologie en guise d’introduction à la théologie des religions. Écrit avec pédagogie et un sens avisé de la problématique, il permet de resituer l’histoire de la théologie chrétienne des religions et donne à entendre ses derniers accents pour un discernement théologique.

\paragraph{différents discours} p 14. A opposer à l'unicité du discours du Prophète, économiste, théologien, soufi, amoureux...