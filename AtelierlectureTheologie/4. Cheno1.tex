\chapter{Dieu au pluriel, Rémi Cheno}

\section{PostModernité}

\paragraph{Lyotard} En philosophie, le postmodernisme devient sujet de débat en 1979 avec la publication de l'ouvrage de Jean-François Lyotard, La Condition postmoderne, que l'auteur caractérise par la perte de crédibilité et le déclin des métarécits qui sous-tendent le discours philosophique de la modernité. C'est autour de cette question que va éclater une querelle, dont les protagonistes seront J.-F. Lyotard, Jürgen Habermas et Richard Rorty. Elle a pour enjeu principal la question de la possibilité d'une sortie effective de la modernité. Les trois philosophes s'accordent pour reconnaître que, après Nietzsche et Heidegger, une manière absolue et globalisante d'envisager l'histoire, l'homme et la société, comme le voulaient les idéologies et les philosophies modernes de l'histoire, est devenue irrecevable. Cette convergence ne les empêche pas de s'opposer quant à l'interprétation à donner d'une telle sortie de la modernité. D'après Lyotard, la fin des métarécits de la modernité, c'est-à-dire du discours des Lumières et de celui de l'idéalisme, entraîne la fin aussi bien du subjectivisme que de l'humanisme, comme Michel Foucault l'avait déjà établi de son côté. Les philosophes des Lumières faisaient de l'audace du savoir le moteur de l'émancipation du genre humain tout entier ; quant à l'idéalisme absolu, il faisait dépendre la légitimité de tout savoir de la possibilité de s'inscrire dans la perspective d'une doctrine de la science encyclopédique et universelle. En critiquant les penseurs des Lumières, Lyotard souligne que la raison ne saurait renvoyer automatiquement à une promesse d'émancipation et, surtout, que rien ne garantit la nécessité d'un lien entre les énoncés descriptifs de la science et les énoncés pratiques et prescriptifs visant l'émancipation de l'humanité. \sn{\href{https://www.universalis.fr/encyclopedie/postmodernisme/3-philosophie/#:~:text=En\%20philosophie\%2C\%20le\%20postmodernisme\%20devient,discours\%20philosophique\%20de\%20la\%20modernit\%C3\%A9.}{Post Modernité - E Univ}}

\paragraph{critère d'opérativité - Les types de relation causale} 
Selon les auteurs, quatre types de relation causales sont impliqués dans la construction de la cohérence causale d’un texte. Ces relations causales peuvent différer dans l’importance avec laquelle elles fournissent la nécessité et la suffisance dans les circonstances d’un texte et par conséquent diffèrent dans leur force causale (Tapiero, & al., 2002 ; Trabasso, & al.,1989 ; van den Broek, 1990). Le premier type de causalité est la ‘« causalité physique ’». Elle connecte des événements qui décrivent des changements dans les états physiques des objets ou des personnes. La ‘« motivation ’» réfère aux relations entre un but et ses conséquences tandis que la ‘« causation psychologique ’» rend compte des relations causales qui impliquent des états émotionnels. Enfin, une relation ‘« rend possible ’» décrit le lien entre un événement et une pré-condition nécessaire mais faiblement suffisante pour la conséquence.

Dans ce modèle, les relations sont classées selon les rôles causaux des interrelations des catégories conceptuelles des propositions. Le type de relation causale entre deux événements est déterminé à partir de la règle de décision suivante : pour une paire de propositions dans laquelle A est temporellement antérieur à B et A est nécessaire pour B, déterminer si A contient une information relative à un but. Si l’événement A appartient à la catégorie conceptuelle But alors la relation entre A et B est une motivation ; si A renvoie à une autre catégorie, déterminer si B implique un état interne ou réaction. Si l’événement B est une réaction, alors la relation entre A et B est de type causation psychologique. Si l’événement B ne renvoie pas à un état interne alors il faut déterminer si A est également suffisant dans les circonstances du texte pour B. Si A est suffisant cela signifie que A cause physiquement B. À l’inverse si A n’est pas suffisant alors A rend possible B.

Parallèlement, deux types de relations temporelles ont de l’importance en plus des relations causales. D’une part, la coexistence temporelle, c’est-à-dire quand deux événements sont conjoints et se produisent en même temps et d’autre part, la succession temporelle, qui implique qu’un événement suive dans l’ordre un autre événement.

Ainsi, ce modèle d’analyse causale du discours permet d’identifier les relations entre les états et les actions qui sont décrits tout au long d’un récit. Ces états et actions sont interprétés et catégorisés en fonction de leur contenu et du contenu de leurs relations avec les autres propositions ainsi que par le rôle qu’ils jouent dans un épisode narratif et plus largement dans l’ensemble du récit. Un aspect fondamental de ce modèle repose sur l’identification des relations causales à partir des critères de nécessité et de suffisance d’une catégorie conceptuelle pour une autre catégorie dans les circonstances d’un texte. De plus, la transitivité permet l’assemblage des paires reliées au sein de la chaîne et du réseau causal. En effet, si un événement A cause un événement B qui cause un événement C, ces paires d’événements reliés sont directement assemblées au sein de la chaîne causale. Autrement dit, au sein de la chaîne causale, A précède immédiatement B, et aucune proposition ne peut être insérée entre ces deux événements dans l’histoire. Cependant, la chaîne A cause B cause C n’est pas obligatoirement linéaire. Au contraire, chaque catégorie conceptuelle peut posséder de multiples antécédents cause ainsi que plusieurs conséquences causales. De cette manière, la distance dans la structure de surface du texte ne détermine pas l’opérativité des causes. Ce critère d’opérativité opère spécialement pour la catégorie conceptuelle But. Un but est opérationnel et possède une force causale importante tant qu’il n’a pas été atteint.

Ce modèle implique donc une structure hiérarchique des buts d’un récit, dans lequel les multiples épisodes sont emboîtés, un but pouvant soit, motiver une série d’essais, un but subordonné après plusieurs essais et une issue non satisfaite soit, engendrer une nouvelle série d’essais après l’atteinte d’un but subordonné. Il en résulte un réseau causal plutôt qu’une chaîne linéaire au sein duquel la distance causale entre deux catégories ne dépend pas des autres relations de distance comme les relations temporelles ou référentielles.

Alors que les modèles ‘« structuraux ’» de la représentation mentale des événements d’un récit mettent en évidence l’importance des rôles causaux que les événements occupent dans l’ensemble du réseau (Black & Bower, 1980 ; Trabasso, & al., 1984) ainsi que l’existence des relations causales au sein de la représentation en mémoire (Bloom & al., 1990 ; van den Broek & Lorch, 1993) d’autres modèles ont été proposés pour rendre compte de manière plus précise des processus mis en jeu dans à la construction des relations causales au cours de la lecture (Fletcher & Bloom, 1988 ; van den Boek, 1990). Ces modèles reposent sur l’idée selon laquelle le processus de compréhension est une activité de résolution de problème dans laquelle le lecteur doit découvrir une séquence de liens causaux qui connectent l’ensemble des informations d’un texte.

Ainsi, dans le modèle de l’état courant, Fletcher et Bloom (1988) décrivent les processus qui permettent au lecteur de découvrir la structure causale d’un récit complexe dans le cadre des capacités limitées de la MCT. Les auteurs proposent que l’allocation des ressources attentionnelles au cours de la lecture est contrôlée par la structure causale du texte et, par conséquent, que les propositions les plus utiles pour construire la structure causale sont maintenues en MCT pendant le processus de compréhension. La stratégie de sélection consiste alors à sélectionner et à maintenir au sein du focus en MCT le dernier état de la chaîne causale, et est ainsi définie par les auteurs comme la stratégie de sélection de l’état courant. Cette sélection s’opère en deux étapes. Dans un premier temps, le lecteur doit identifier l’état le plus récemment rencontré qui possède des antécédents causaux dans les propositions précédentes mais qui n’a pas de conséquence. Dans un second temps, il doit sélectionner les propositions à l’intérieur de cet état sans lesquelles ce dernier ne pourrait pas remplir sa fonction causale. Si des connexions causales peuvent être établies entre le dernier état maintenu en MCT et les nouvelles informations textuelles, le processus de compréhension se poursuit sans difficulté. À l’inverse, si aucune relation causale n’est détectée, c’est-à-dire lors d’une rupture de la cohérence causale, la MLT est explorée afin de trouver une connexion causale entre les informations en cours de traitement et le texte antérieurement lu.

Dans la lignée du modèle de Fletcher et Bloom (1988), le modèle de traitement proposé par Graesser et al. (1994) présente spécifiquement les processus sous-jacents à la construction d’un modèle de situation au sein duquel les événements et les actions décrits dans un texte ainsi que les relations causales qui les connectent, occupent également un rôle fondamental. Ce modèle s’inscrit dans une approche constructionniste de l’activité de compréhension qui confère aux processus stratégiques une place aussi fondamentale que le rôle attribué aux processus automatiques dans l’approche du traitement du texte basé sur la mémoire. Il décrit les processus sous-jacents à la construction d’une représentation sous la forme d’un réseau causal et pose les principaux postulats de l’approche du traitement du texte basé sur les explications, approche de la compréhension qui s’oppose à celle du traitement de texte basé sur la mémoire.