\chapter{Cheno : Dieu au pluriel 2}
\mn{13/10/22}


\section{Karl Barth : l'exclusivisme}
 

\paragraph{L'exclusivisme} KB dénonce l'arrogance de son Eglise. Question du mal après la première guerre mondiale
\begin{quote}
    tous ont péché et sont privés de la gloire (Rm)
\end{quote}

\paragraph{relevé par la révélation}

\paragraph{pas de point de contact avec les autres religions} la religion chrétienne n'est pas en surplomb. Mais le dialogue ne l'intéresse pas.

\paragraph{Evolution de sa pensée} Révélation en Christ source de salut. Ouvre un accès de la prise en compte des autres religions avec des sphères impliquées, autour du \textit{Logos}. \mn{Paul VI reprendra cette notion de sphère, protestantes,...}


\begin{itemize}
    \item Exclusivisme christologique : médiation du Christ. Barth est exclusiviste 
    \item Exclusivisme Revelationnel : pas de révélation hors de la révélation chrétienne. 
    \item Exclusivisme eschatologique : pas de salut hors des chrétiens. pour Barth, c'est de la liberté de Dieu.
\end{itemize}

\paragraph{Pour Cheno, l'exclusivisme n'est pas forcément intolérant}. Et il le démontre par l'exemple de Karl Barth. Il déconstruit un préjugé à partir d'un auteur pour déconstruire ce préjugé.
On ne voit pas la problématique très bien au départ mais elle est clair à la fin, "en ce sens, il n'est pas intolérant".

\paragraph{faire attention aux mots} \textit{Pourtant} : montre la problématique. "Karl Barth exclusiviste", "pourtant", "Karl Barth critique son Eglise". 

\paragraph{Soucis de situer Karl Barth dans son contexte} Pasteur réformé suisse, 1886-1968 : sensible à cela. Dans sa jeunesse, il a défini la théologie libérale : 

\begin{Def}[Théologie libérale]
adapter la foi chrétienne à l'humanisme des Lumières, c'est à dire passer alliance avec la raison laïque et la culture mondaine.
\end{Def}
Dans un monde où il y avait une auto-justification du modernisme et du Christianisme. Avec la première guerre mondiale, on s'aperçoit qu'on a oublié la transcendance. 


\paragraph{Démonstration par les citations} p. 50 s'appuie sur le texte de Karl Barth. Citer / Introduire. 

\paragraph{Dialectique déjà utilisée par Luther} Le catholicisme cherche la synthèse. Alors que le protestantisme cherche la dialectique de la "Croix" et la "Résurrection". Vouloir expliquer le paradoxe, c'est se mettre à la place de Dieu. Luther dénonçait la philosophie qui cherche à expliquer. 

\paragraph{Dialectique de Barth}
La parole que l'on reçoit, elle nous juge mais elle nous justifie.
Le chapelet, le pèlerinage : ce sont des oeuvres. 

\paragraph{L'exclusivisme de Barth} Le croyant est le premier qui se reconnaît pécheur. 
\begin{quote}
    p. 53 le christianisme est l'unique vraie religion parce que c'est la seule religion qui se sait dans l'erreur !
\end{quote}

\paragraph{"on peut résumer"} faire cela dans nos travaux. Pour que le lecteur puisse souffler. 

\paragraph{Les autres religions} L'idée de la Religion. face à la sociologie des Religions. on essaye d'interroger le phénomène religieux.

\paragraph{Révélation contre Religion} Religion : auto-justification; La religion doit être renversée pour être relevée. 

\paragraph{Auf-hebung} quelque chose qui est de l'ordre de l'abolition et dépassement. Barth, théologien influencé par Hegel. Aspect prophétique chez Barth. Il est d'abord prédicateur. 

\paragraph{Contexte} on voit bien que Karl Barth est une pensée contextuelle : un pasteur soucieux d'une parole performative. 

\paragraph{Critique du pluralisme qui présente l'exclusivisme comme dépassé}

\paragraph{Actualité de Barth} On ne peut critiquer l'autre qu'en se critiquant d'abord.

\section{Karl Rahner et l'inclusivisme}

\paragraph{Grâce et l'amour}

\paragraph{Aucune référence de Rahner, aucun contexte} Pour Rahner, la question est d'annoncer la foi à l'homme athée. Chéno. Quelle bonne nouvelle pour l'homme ? Comment on argumente ? 

\paragraph{Une partie Critique} bon exemple par \textit{l'interrogation} de la façon dont on peut présenter. 

\paragraph{Quand on répond à des questions profondes, on reste pertinent plus longuement}

\paragraph{Quelles sont les grandes questions de notre temps pour interroger notre foi}