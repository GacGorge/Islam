\chapter{La violence légitimée : de la
révolution aux attentats-suicide}

 
\mn{(14/03/2022) }

\section{Introduction}


Jihadisme : Une dérive de l'islam politique. Même si Banna et Mawdidi

\subsection{Bibliographie}

BENICHOU David, KHOSROKHAVAR Farhad, MIGAUX Philippe \emph{Le jihadisme.
Le comprendre pour mieux le combattre}, Plon, Paris, 2015.

BENSLAMA Fethi et KHOSROKHAVAR Farhad, \emph{Le djihadisme des femmes},
Paris, Seuil, 2017. BONNER, Michael \emph{Le jihad. Origines,
interprétations, combats}, Téraèdre, 2004.

CARRE, Olivier \emph{Mystique et politique : le Coran des islamistes,
commentaire coranique de Sayyid Qutb}, Cerf, Paris, 2004.

KEPEL, Gilles \emph{La revanche de Dieu: chrétiens, juifs et musulmans à
la reconquête du monde}, Seuil, Paris, 1991. Sur l'islam, p. 32-81.

*\emph{Du jihad à la fitna}, Bayard, Paris, 2005.

KEPEL, Gilles ; MILELLI, Jean-Pierre \emph{Al-Qaida dans le texte}, PUF,
Paris, 2008.

LUIZARD Pierre-Jean, \emph{Le piège Da'ech, l'Etat islamique ou le
retour de l'Histoire}, La Découverte, Paris, 2015.





 
  \section{Un pensée fondatrice : Sayyid Qutb (1906-1966)}



   
    \paragraph{Une vie de militant} Milieu rural; des études honorables au Caire pour devenir enseignant. Au départ, il est plutot dans une mouvance réformiste, ouvert aux idées occidentales. Mais il va se lier dans les années 30 avec les frères musulmans. De 1948 à 1950, fait un voyage d'études aux US. Révélateur de la corruption occidentale \sn{L'Amérique que j'ai vue, un article qui va lui poser des problèmes}. Il adhère en 1951 aux frères musulmans (clandestins). 
    
    \subparagraph{soutiens aux officiers libres} coup d'état de Nasser et Sadate en 1952 mais ensuite Nasser va éliminer les frères musulmans.En 1954-1964, Qutb va en prison et se radicalise.
   

   
    \paragraph{Le \emph{Fi zilâl al-Qur'an - \textit{A l'ombre du Coran}} : une méthode de lecture du Coran}. Le principe, c'est le retour aux sources donc on développe un commentaire du Coran. En 1966, il est executé par Nasser, ce qui en fait un Martyr.
   

   
   \paragraph{L'Etat islamique : une utopie} 
   
  \begin{Synthesis}
  Là où il se démarque des prédécesseurs, c'est la rupture avec la raison. POur obeir à Dieu, il n'y a pas à convoquer la Raison. Il est littéraliste. cela converge avec le Wahhabisme.
  \end{Synthesis}
  
  
  Si je fais pour des raisons scientifiques, je ne le fais plus pour plaire à Dieu. Et du coup, si la raison scintifique change, je vais arrêter ma pratique ? IL demande une approche \textit{sacrificielle} sans chercher à comprendre : la figure, c'est celle d'Abraham qui fait le sacrifice de son fils sans chercher à comprendre. Le vrai culte ('ibadat), c'est l'esclavage ('ubudiyyat).
  
  Le Coran est un livre à vivre : pour comprendre le Coran, il faut le vivre. Comme les premiers compagnons étaient des combattants (Moujjadines), il faut être moujjadine pour le comprendre.
  
  \begin{itemize}
      \item La première phase, Mecque, on renonce à l'idolatrie. Dieu seul. \item IL faut fuir, l'exil (Hijjira). Il est impossible de vivre l'Islam dans une société non musulmane. il faut fonder de petites communautés, à l'image de la communauté médinoise. 
      \item puis il faut combattre. Vocation mondiale. \sn{Lire le \textit{l'immeuble Yacoubian}, d'Al Aswani, mis en film qui raconte l'évolution d'un jeune à travers ces différentes étapesF}
  \end{itemize}
  
  \paragraph{Rejet des sociétés actuelles même musulmanes}. A la différence de ces prédécesseurs qui voyaient Medine comme une société parfaite mais inaccessible car le prophète n'est plus là, pour Mawdudi et Qutb, il est possible de refonder Médine en gouvernant par la Loi. Mauwdudi et Qutb ne sont pas juristes et du coup, on un discours idéologique sur la Loi.
  
  
  
 \begin{Def}[munafiq]
  {hypocrite (désigne l'ennemi}
{« interne » à l'umma)}
 \end{Def}
 


\paragraph{Place centrale du \emph{jihad}}

\begin{Synthesis}[Jihad Mondial Permanente]
influence Maoiste ? 
le Jihad est le sixième pilier obligatoire pour les musulmans au moins une fois dans la vie
\end{Synthesis}

Le Grand Jihad n'apparait pas dans le Coran aussi clairement mais à partir du Hadith.
\begin{itemize}
    \item Le grand Jihad : lutte intérieure pour se transformer. \textit{Exercices spirituels}, se fait avec l'aide de la raison. 
    \item le petit Jihad : lutte armée
    \begin{itemize}
        \item défensif : si mon pays est attaqué. Il est individuel (tout homme doit prendre les hommes pour défendre sa foi et son pays)
        \item offensif : devoir collectif, tout homme n'a pas le devoir de prendre les armes, l'Etat musulman a l'obligation de le faire. 
    \end{itemize}
\end{itemize}

\subparagraph{La lecture classique du Jihad} L'islam a toujours reconnu la guerre comme légitime mais encadré.
\begin{Def}[dar al-islam]
\emph{territoire de l'islam}
\end{Def}
\begin{Def}[dar al-kufr]
\emph{territoire des infidèles (mot-à- mot : territoire de
l'impiété)}
\end{Def}
\begin{Def}[dar as-sulh]
\emph{territoire de la paix}
\end{Def}
Des pays même non-musulmans où on a fait alliance : l'alliance est sacrée.
Le reste est \textit{dar al-harb}
\begin{Def}[dar al-harb]
\emph{territoire de la guerre}
\end{Def}


\subparagraph{la relecture de Qutb}
Comme aucun pays n'est vraiment musulman, tout territoire est \textit{dar al-harb}. Nous sommes dans un Jihad defensif puisque l'occident nous attaque par sa culture. Et donc, tout musulman doit faire le jihad de façon individuelle. 
Tant que nous ne serons pas les maîtres du monde, nous risquons d'être attaqués, donc il faut porter le jihad mondial pour \textit{avoir la paix}. 
: 
\paragraph{Une inversion des concepts} Co 2, 217 : \textit{la fitna est plus grave que le meurtre}

\begin{Def}[fitna]

\emph{discorde, querelle (conflit interne au monde musulman)}

\end{Def}

Qutb va relire ce verset coranique en considérant que la fitna est liée à l'Occident.


\paragraph{Dimension spirituelle chez Qutb} Le Jihad va aider l'homme à se purifier. Et c'est ainsi qu'il explique que le succès n'arrive pas tout de suite : parce que l'épreuve purifie le croyant. Il lie le \textit{grand jihad} au petit jihad.

\begin{Synthesis}
Deux influences : 
\begin{itemize}
    \item influence des idéologies contemporaines (Maoisme,;..)
    \item mais aussi la pensée Kharijite 
\end{itemize}
\end{Synthesis}
 \subsection{Textes de Qutb}
\paragraph{Une génération Coranique Unique}

\begin{quote}
    \textbf{Il y a un phénomène historique devant lequel doivent s'arrêter
les détenteurs de l'appel de l'islam en tout lieu et en tout temps.} Et
ils doivent s'y arrêter longuement. En effet, cela a une influence
décisive sur la méthode et l'orientation de l'appel. \textbf{Cet appel a
formé une génération de gens, la génération des compagnons (que la
satisfaction de Dieu soit sur eux), une génération distinguée dans toute
l'histoire de l'Islam et dans l'histoire de l'humanité toute entière.
Depuis il n'a jamais été formé ce type {[}d'homme{]} une autre fois}.
Oui, on trouve des individus de ce type dans le cours de l'histoire,
mais il n'est jamais arrivé que se rassemble un si grand nombre, en un
seul endroit, comme il y avait dans cette première période de la
génération de cet appel. Ceci est un phénomène clair et actuel, il
possède une signification devant laquelle il faut s'arrêter longuement ;
peut-être découvrirons-nous son secret. Le Coran de cet appel est entre
nos mains ; le hadith du Messager de Dieu, (la bénédiction de Dieu et la
paix soient sur Lui), son enseignement pratique, sa noble biographie,
sont aussi entre nos mains, comme ils étaient entre les mains de cette
première génération - laquelle ne s'est pas répétée dans l'histoire-...
Rien n'a disparu {[}aujourd'hui{]}, sauf la personne du Messager de
Dieu, (la bénédiction de Dieu et la paix soient sur Lui). Alors, est-ce
cela le secret ? Si l'existence du Messager de Dieu, (la bénédiction de
Dieu et la paix soient sur Lui) était inévitable pour la mise en place
de cet appel, et pour porter ses fruits ; Dieu n'en aurait pas fait un
appel à tous les hommes, et n'en aurait pas fait un dernier message,
n'en aurait pas fait dépendre les affaires des hommes sur terre jusqu'à
la fin des temps. Mais Dieu (qu'il soit honoré) s'est engagé par la
conservation de l'invocation et il a su que cet appel pouvait continuer
après le Messager de Dieu, (la bénédiction de Dieu et la paix soient sur
Lui) et pouvait porter ses fruits. Il l'a rappelé auprès de lui après 23
ans de mission et a fait rester sa religion depuis après lui jusqu'à la
fin des temps. L'absence de la personne du Messager de Dieu, (la
bénédiction de Dieu et la paix soient sur Lui) n'explique donc pas ce
phénomène et n'en donne pas la raison.
\end{quote}

\begin{Synthesis}
importance des compagnons, génération unique. Coran sola. 
"Tout cela (culture juive, romaine...)à était mélangé avec le tafsir du Coran et la sciece du Kalam comme c'était mélangé avec le fiqh et la croyance, source polluée si bien que cette génération ne s'est pas renouvelé (cf Harnack ?). 

\end{Synthesis}

\begin{quote}
    Cherchons donc une autre cause. \textbf{Regardons la source a laquelle a
puisé cette première génération, peut-être que quelque chose a changé en
elle}. Et regardons dans la voie qui les a formés ; peut-être que
quelque chose a changé en elle pareillement.

\textbf{Cette première source a laquelle a puisé cette génération, c'est
la source du Coran. Rien que le Coran}. Alors, le hadith du Messager de
Dieu, (la bénédiction de Dieu et la paix soient sur Lui), et son
enseignement, n'ont été qu'une des traces de cette source. Et quand on a
interrogé Aïsha (que Dieu soit satisfait d'elle) au sujet de l'éthique
du Messager de Dieu, (la bénédiction de Dieu et la paix soient sur Lui),
elle a dit: ``\textbf{Son éthique était le Coran}''\textbf{Le Coran seul
donc, était la source à laquelle ils ont puisé, par laquelle ils sont
devenus conditionnés, par laquelle ils ont été formés. Ceci n'a pas été
ainsi parce qu'à ce jour, l'humanité n'avait ni civilisation, ni
culture, ni science, ni écrits, ni études... Non ! Au contraire, il y
avait la civilisation romaine, sa culture, ses lois, desquelles l'Europe
continue à vivre (ou de ses extensions)}. Et il y avait les vestiges de
la civilisation grecque, sa logique, sa philosophie, son art et c'est ce
qui continue à irriguer la pensée occidentale jusqu'à aujourd'hui. Et il
y avait de même la civilisation perse, son art, sa poésie, sa
mythologie, ses dogmes et son système de gouvernement. Et d'autres
civilisation lointaines et proches : la civilisation hindoue, la
civilisation chinoise, etc. Et les deux civilisations romaines et perses
étaient au bord de la péninsule arabique, à son nord et à son sud, de
même que les juifs et les chrétiens vivaient au coeur de la péninsule...
\textbf{Ce n'est donc pas à cause de la pauvreté des civilisations
mondiales, des cultures mondiales que cette génération s'est limitée au
Livre de Dieu uniquement, dans la période de formation. il y avait là
une intention spécifique, une direction voulue.}

La colère du Messager de Dieu, (la bénédiction de Dieu et la paix soient
sur Lui) a montré cette intention quand il a vu dans la main de Omar ben
Khatâb (que Dieu soit satisfait de lui), une copie de la Torah. Alors,
il a dit : Si Moïse avait vécu parmi vous, il ne lui aurait pas été
permis autre chose que me suivre.''\textbf{Il y avait donc l'intention
de la part du Messager de Dieu, (la bénédiction de Dieu et la paix
soient sur Lui) de limiter la source à laquelle a puisé cette
génération, dans cette première période de formation, au Livre de Dieu
uniquement, pour que leurs âmes soient pour Lui seul, et pour que leur
comportement se redresse sur sa voie à Lui seul.} D'où le fait qu'il est
devenu furieux quand il a vu Omar ben Khatâb (que Dieu soit satisfait de
lui), puiser à d'autre source. Le Messager de Dieu, (la bénédiction de
Dieu et la paix soient sur Lui) a voulu faire une génération pure de
coeur, pure d'esprit, pure d'appréhension, pure de sentiment, pure de la
formation de toute influence autre que celle de la direction divine que
contient le Saint Coran. \textbf{Cette génération a puisé donc à cette
unique source ; ce pourquoi elle a eu dans l'histoire ce statut
unique... Alors, que s'est-il passé ? Les sources se sont mélangées. Se
sont déversé dans la source à laquelle a puisé les générations
suivantes, la philosophie des grecs et leur logique, la mythologie perse
et ses conceptions, les ``judaïtés'' juive et la théologie chrétienne et
d'autres éléments d'autres civilisations et cultures. Tout cela était
mélangé avec le \emph{tafsîr}\sn{Mot arabe qui signifie explication, commentaire (du verbe fassara, « expliquer ») et qui a pour synonyme sharḥ, tafsīr désigne une forme de commentaires d'ouvrages très divers en matière de science et de philosophie.} du Coran, et la science du \emph{Kalâm},
comme c'était mélangé avec le \emph{fiqh}\sn{fiqh: \emph{droit (musulman)}} et la croyance. Toutes les
générations, après celle-ci ont été formées à cette source polluée, si
bien que cette génération ne s'est jamais répétée}. Et il ne fait pas de
doute que le mélange de cette source première a été un des facteurs
fondamentaux de cette différence entre toutes ces générations et la
génération distincte et unique.

\end{quote}
\begin{Synthesis}
Pollution de la culture
\end{Synthesis}

\begin{quote}
    \textbf{Il y a un autre fait fondamental, autre que la différence de
nature de la source. C'est la manière différente de recevoir ce qui a
formé cette génération unique et distincte}. Ceux de la première
génération \textbf{ne lisaient pas le Coran avec la volonté
{[}d'acquérir{]} une culture ou un savoir, ni avec la volonté du goût et
de la jouissance. Aucun d'entre eux n'a reçu le Coran pour augmenter sa
provision de culture, juste pour la culture, et ne remplissaient une
réserve de propositions scientifiques et juridiques. En effet, il
recevait le Coran pour recevoir un ordre de Dieu concernant ses affaires
et les affaires de la communauté dans laquelle il vivait. Il recevait
cet ordre pour le mettre en pratique dès qu'il l'entendait, comme un
soldat reçoit l'ordre du jour dans un champ de bataille pour le mettre
en pratique sitôt reçu.} En conséquence, il n'y avait aucun d'entre eux
pour {[}chercher{]} à augmenter ces {[}réserves{]} d'un seul coup, parce
qu'il sentait que cela augmentait les devoirs et les fardeaux qu'il
devait charger sur ses épaules et il se limitait à dix versets afin de
les garder et d'y travailler (comme il est dit dans le hadith de Ibn
Mas'ud (que Dieu soit satisfait de lui) \textbf{Cette attitude . . .
cette attitude de recevoir pour la mise en pratique. . . leur ouvrait
dans le Coran les horizons de la joie, horizons du savoir, qui
n'auraient pas été ouverts s'ils avaient misé sur lui avec une attitude
de recherche, d'études et de connaissance. Cela rendait facile le
travail et rendait léger les fardeaux. Le Coran était mélangé avec leur
identité, était transformé dans leurs âmes et leur vie, en un chemin
concret, en une culture dynamique qui ne reste pas à l'intérieur des
esprits ni sur le papier, mais se change en effets et en événements qui
modifie la trajectoire de la vie.} Le Coran ne livre ses trésors qu'à
ceux qui viennent à lui avec cet esprit : esprit de connaissance
engendrant une pratique. Il ne vient pas pour être un livre de plaisir
intellectuel, ni un livre de lettre et d'art, ni un livre d'histoires ou
d'histoire (même si tout ceci est contenu dedans), mais il vient pour
être un chemin de vie, un chemin purement divin. Et Dieu (gloire à Lui!)
les prenait, par cette voie fragmentée, qu'ils ont suivi petit à petit.
``Et nous avons fragmenté le Coran pour que tu le récites aux gens
progressivement et nous l'avons réellement fait descendre'' (17, 106) Ce
Coran n'est pas descendu comme un tout, mais il est descendu selon les
besoins nouveaux et selon la
croissance continue des idées et des conceptions, la croissance continue
de la société et de la vie et selon les problèmes pratiques qui se
présentaient à la communauté musulmane dans sa vie concrète. Le verset
(ou les versets) descendait dans la situation particulière ou pour un
événement spécifique, parlant aux hommes de ce qui était en leur âme et
les éclairant sur ce qu'ils avaient à faire, dessinant pour eux la voie
concrète dans telle situation, corrigeant leurs fautes d'attitude et de
conduite, les rattachant en tout cela à Dieu, leur maître, Le faisant
connaître d'eux à travers ses attributs caractéristiques dans le cosmos.
Alors, ils sentaient qu'ils vivaient ici donc dans le domaine du
Très-Haut, sous l'oeil de Dieu, dans le domaine de son pouvoir. Et dès
lors, ils ont accordé les situations de leur vie selon cette authentique
voie divine. La méthode de recevoir pour la mise en pratique et l'action
est ce qui a façonné cette première génération. Et la méthode de
recevoir pour étudier et pour se réjouir est ce qui a formé les
génération qui ont suivi et il n'y a pas de doute que ce second fait est
aussi fondamental que le premier dans la différence de toutes les
générations par rapport à la génération spécifique et unique.

\end{quote}

 
\begin{Synthesis}
un accès au coran par la pratique. 
\end{Synthesis}


\begin{Def}[jahiliyya]
\emph{ignorance =\textgreater{} société païenne, impie}
\end{Def}
Qu'il applique aux sociétés actuelles.

\begin{quote}
    \textbf{Il y a un troisième fait difficile pour l'attention et la
mémorisation. L'homme qui était sur le point d'entrer dans l'Islam
abandonnait sur le seuil de la porte tout son passé de Jâhiliya}.
\textbf{Il sentait à cette seconde où il venait à l'Islam qu'il
commençait une nouvelle phase, se séparant complètement de la vie qu'il
menait dans la Jâhiliya. Et il s'arrêtait devant tout ce que comportait
sa phase de Jâhiliya ; un arrêt de suspicion, de doute, de méfiance et
de crainte; il sentait que toute cette pollution ne convenait pas pour
l'Islam !} Avec cette sensation, il recevait l'enseignement {[}de
l'Islam{]} nouveau. Et si, une fois, son inclinaison l'emportait sur lui
; si, une fois, ses {[}anciennes{]} habitudes l'attiraient; si, une
fois, il faiblissait devant les obligations musulmanes ... il sentait
immédiatement sa culpabilité et son péché et il reconnaissait du fond de
son âme qu'il avait besoin d'une purification là où il était tombé et il
essayait une nouvelle fois de vivre en accord avec l'enseignement du
Coran. Il y avait donc une attitude complète de séparation entre le
passé du musulman dans la Jâhiliya et le présent dans l'Islam. Cela
résultait d'une séparation complète de ses relations avec la société de
la Jâhiliya de ce qui l'entourait et de ses liens sociaux. Il se
séparait donc définitivement de l'environnement de la Jâhiliya, pour se
connecter finalement à l'environnement de l'Islam. Même s'il était mêlé
avec quelques polythéistes en travaillant dans le monde des affaires et
des échanges quotidiens, cette séparation d'attitude était une chose et
les échanges quotidiens une autre chose. \textbf{Il y avait un acte de
séparation de l'environnement de la Jâhiliya, de ses coutumes, de ses
conceptions, de ses habitudes, de ses liens ; qui résultait de la
séparation de la foi polythéiste pour la foi monothéiste, de la
conception de la Jâhiliya pour la conception islamique sur la vie et
l'existence.} Et cela était dû au fait d'avoir rejoint la nouvelle
assemblée islamique, ses nouveaux meneurs et d'avoir donné à cette
assemblée et à ces meneurs toute allégeance, toute obéissance et toute
communion. Et c'était ce carrefour, c'était le début d'une marche sur le
chemin nouveau, le chemin libre et allégé de toutes les pressions des
coutumes acceptées par la société de la Jâhiliya, et de toutes les
conceptions et des valeurs qui y dominent. Il n'y avait rien d'autre que
ce que le musulman rencontrait de préjudice et de mise à l'épreuve. Mais
il avait lui même tranché et y avait mis un terme, si bien que ni la
pression des conceptions de la Jâhiliya, ni les coutumes de la société
de la Jâhiliya ne pouvait l'atteindre. \textbf{Nous sommes aujourd'hui
dans la Jâhiliya, la même la Jâhiliya que celle qui était contemporaine
à l'Islam ou {[}même{]} plus sombre. Tout ce qui nous entoure est
Jâhiliya, les conceptions des gens, leurs croyances, leurs habitudes et
leurs coutumes, les sources de leur culture, leur art et leur
littérature, leurs normes et leurs lois. A tel point que beaucoup de ce
que nous considérons comme culture islamique, comme références
islamiques, comme philosophie islamique, comme pensée islamique, sont un
produit de cette Jâhiliya.} C'est pour cela que ne se dressent pas en
nos âmes les valeurs de l'Islam et que ne s'éclaire pas
en notre intelligence les conceptions de l'Islam et que la grande
génération des gens de ce type qu'a produit l'islam la première fois ne
se reproduit pas en nous. \textbf{Il nous faut donc dans cette voie du
mouvement islamique, nous dépouiller (dans cette période de gestation et
de commencement) de toutes les influences de la Jâhiliya dans lesquelles
nous vivons et desquelles nous sommes inspirés. Et nous devons retourner
à la source pure qui a inspiré ces hommes, la source garantie qui n'a
pas été mélangée et qui n'a pas été rendue impure}. Nous retournons vers
elle en y empruntant notre conception de la vérité de l'existence
entière et de la vérité de l'existence humaine et de la totalité des
relations entre ces deux existences et entre l'existence parfaite de la
vérité, existence de Dieu (loué soit-il)... Et puis, de cela nous
empruntons nos conceptions de la vie, nos valeurs morales, nos approches
pour juger de la politique, de l'économie et de tous les éléments de la
vie. Et quand on y retourne il faut y retourner avec le sentiment de
``recevoir pour mettre en pratique'' et non avec un sentiment ``d'étude
et de plaisir''. Nous y retournons pour savoir ce qu'il nous demande
d'être et pour l'être. Et sur le chemin, nous rencontrerons, la beauté
artistique dans le Coran, les belles histoires du Coran, les scènes du
jugement dernier {[}qui se trouvent{]} dans le Coran, la logique que
l'on sent dans le Coran et les autres choses que demandent les ``gens de
l'étude et du plaisir''. Mais tout cela nous le rencontrerons, sans que
cela soit notre premier but. Que notre premier but soit de connaître :
qu'est-ce que le Coran veut que nous fassions ? Quelle est la conception
globale qu'il veut que nous concevions ? Comment le Coran veut que
soient nos sentiments pour Dieu? Comment veut-il que soient nos
comportements éthiques, nos conditions, notre organisation pratique de
la vie?Alors, il nous est nécessaire de nous purifier des pressions de
la société de la Jâhiliya, des conceptions de la Jâhiliya, des coutumes
de la Jâhiliya, du commandement de la Jâhiliya, au fond de nous-même. Ce
n'est pas notre préoccupation de nous réconcilier avec la réalité de
cette société de la Jâhiliya, ni d'y adhérer. Parce que le propre de la
Jâhiliya, n'est pas que nous nous réconcilions avec lui. Notre
préoccupation est d'abord de changer notre âme, nous changerons la
société ensuite. Notre première préoccupation est de changer la réalité
de cette société. Notre préoccupation est de changer cette réalité de la
Jâhiliya dans son fondement. Cette réalité {[}est celle{]} qui,
fondamentalement, se heurte de plein fouet au système islamique et aux
conceptions islamiques, et qui nous prive par contrainte et par pression
de vivre comme le veut pour nous la voie divine. \textbf{Le premier de
nos pas sur notre chemin sera de dominer cette société de la Jâhiliya,
ses valeurs et ses conceptions et de ne pas modifier plus ou moins nos
valeurs et nos conceptions pour nous rencontrer ensemble dans un demi
chemin. Nous et elle, sommes dans un carrefour et {[}il suffirait
d'un{]} instant où nous marchions un seul pas {[}avec elle{]} pour que
nous perdions toute la voie et nous perdions le chemin ! Nous
rencontrerons en cela de la peine et de la difficulté et cela exigera de
nous de coûteux sacrifices, mais nous n'avons pas le choix si nous
voulons suivre le chemin de la première génération pour qui Dieu a
établit ce chemin divin et a vaincu la voie de la Jâhiliya. Il est bon
de toujours prendre conscience de la nature de cette voie de la nature
de notre phase et de la nature du chemin qu'il nous faut suivre pour
sortir de la Jâhiliya comme est sortie cette génération distincte et
unique.} \sn{Extraits de \emph{Ma'âlim fi-t-tarîq} (Jalons sur la route), chapitre 2.
Traduction : Henri de La Hougue}
\end{quote}






 % ------------------------------------------------------------------
  \section{{Les dérives  djihadistes}} 
 

   \paragraph{La première génération : la guérilla.} On est typiquement dans la pensée de Qutb : \textit{paramilitaire} contre l'Etat pour s'emparer du pouvoir et pouvoir instaurer l'Etat islamique. Mouvement de Farad (?) en Egypte en 1970 et l'Afghanistan dans les années 80 : se créé l'international jihadiste. A la fin de la guerre, les jihadistes non afghans vont aller dan sles pays où il y a la guerre et exporter les jihadistes : Bosnie, l'Algérie (décennie noire). Au départ, ce ne sont pas des conflits jihadistes mais les jihadistes arrivent et lui donnent cette connotation jihadiste. 
    
   

   \paragraph{La deuxième génération : le terrorisme mondial.}   Le jihadisme de Al Qaida. Il est déterritorialisé, d'influence nihiliste et le mode opératoire, l'attentat suicide : il s'agit moins de prendre le pouvoir que de frapper l'imaginaire occidental. Le conflit se porte en occident. On vise des cibles symboliques (Twin towers, bataclan).
   Ben Laden est un ancien d'Afghanistan. il pensait être accueilli à son retour en héros ce qui n'a pas été le cas. A vécu la guerre du Koweit et l'arrivée des américains comme une trahison. Il crée AlQaida en 1998, \textit{contre les Juifs et les Croisés}. On est contre, les américains et leurs alliés en tout pays possible : car les US ont occupés les lieux saints.
    
   

   \paragraph{Da`esh, troisième génération du djihadisme} On revient à une territorialisation. Eschatologie. Autre point particulier, a attiré des femmes et des jeunes occidentaux. C'est intéressant car cela légitime le discours que le jihadisme profite de l'effondrement de l'idéologie marxiste : du coup, le jihadisme remplace ce discours.   
    
   

   \paragraph{Le martyre : quelle légitimité ?}   \mn{\begin{Def}[Shahid] meurt en combat en martyr
   \end{Def}}
   Cette fibre du Martyr s'est développée surtout dans le Chi'isme (ceux qui sont morts dans l'Islam). L'Iran va sortir cette rhétorique dans les années 80 dans la guerre contre l'Irak. va passer en Moyen-Orient via le Chiisme : en particulier le Hezbollab au Liban. Eux vont être les premiers à faire des attentats suicide, d'abord contre des cibles militaires.
   Puis le Hamas (suicide) va réutiliser ces attentats suicide contre des \textit{civils}. Malheuresuement, deux savants vont légitimer ces attentats suicide contre les civils : 
   \begin{itemize}
       \item Tantawi (Sheikh Al Azhar) : En Israël, tout le monde fait son service militaire donc il n'y a pas de civil en Israël. Donc Jihad defensif.
       \item Qaradawi (vit encore)
   \end{itemize}
   les deux se sont ensuite rétractés en voyant les conséquences de cette fatwa.
    
   
  


\hypertarget{glossaire-4}{%
\subsection{\texorpdfstring{{Glossaire}}{Glossaire}}\label{glossaire-4}}

\subsection{Personnes}
{}

Al-Zawahiri (Ayman) Ben Laden (Usama) Farag (Abd as-salam) Ibn Taymiyya
Mawdudi

Nadawi

Qaradawi (Sheykh) Tantawi (Sheikh)

{Autres noms propres} Al-Jazira/ Al-Jazeera Dar-al-ulum

Hamas

Harakat al-jihad al-islami Hezbollah

kharijites

\subsection{Notions}

dhimmi : « protégés » (statut particulier accordé aux chrétiens et aux
juifs) fasiq: \emph{pécheur}



hakimiyya : \emph{souveraineté}

hijra : \emph{exode}

`ibada : \emph{culte}

ijtihad : \emph{effort personnel d'interprétation (des textes sacrés)}

isnad : \emph{chaîne de transmission (des hadiths)} 


 :


kafir/kufr : \emph{impie/impiété}



na'ib : \emph{délégué} qatl : \emph{meurtre} shahid : \emph{martyr}

shura : \emph{principe de consultation}

taghut : \emph{tyran}

`ubudiyya : \emph{servitude}

umma : \emph{communauté musulmane}

