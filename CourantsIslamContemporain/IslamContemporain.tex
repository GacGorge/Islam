\hypertarget{second-semestre-2021-2022}{%
\section{Second semestre 2021-2022}\label{second-semestre-2021-2022}}

\begin{quote}
Anne-Sophie Vivier-Mureşan

Documents annexes au cours
\end{quote}

\hypertarget{indications-pedagogiques}{%
\subsection{INDICATIONS PEDAGOGIQUES}\label{indications-pedagogiques}}

\begin{enumerate}
\def\labelenumi{\arabic{enumi}.}
\item
  \begin{quote}
  \textbf{Préparation des cours}
  \end{quote}
\end{enumerate}

\begin{quote}
Chaque cours s'appuiera sur la lecture d'un (ou deux) texte(s)
significatif(s) pour le thème abordé. Il est demandé aux étudiants
d'avoir lu le texte à l'avance de façon suffisamment approfondie pour
pouvoir réagir aux questions abordées en cours.
\end{quote}

\hypertarget{approfondissement-des-cours}{%
\subsection{Approfondissement des
cours}\label{approfondissement-des-cours}}

\begin{quote}
En complément de chaque cours, un ou plusieurs documents seront mis en
ligne pour approfondissement (plate-forme e-learning à partir du site :
{https://formation.icp.fr/})
\end{quote}

\hypertarget{validation}{%
\subsection{Validation}\label{validation}}

\begin{quote}
La validation pourra prendre la forme suivante, au choix :
\end{quote}

\begin{enumerate}
\def\labelenumi{\alph{enumi}.}
\item
  \begin{quote}
  {Un dossier sur un thème lié au cours}
  \end{quote}
\end{enumerate}

\begin{quote}
A partir de trois articles universitaires minimum.
\end{quote}

\begin{itemize}
\item
  \begin{quote}
  Introduction : intérêt du thème au regard de l'actualité et/ou de vos
  propres orientations personnelles ; présentation des articles et de
  leurs auteurs. Présentation du plan de votre travail.
  \end{quote}
\item
  \begin{quote}
  Synthèse (ne pas présenter les articles séparément mais faire une
  synthèse à partir des différentes thématiques rencontrées)
  \end{quote}
\item
  \begin{quote}
  Contextualisation de ce thème dans l'islam contemporain, éclairage par
  le contenu du cours (enjeux, dynamiques, etc).
  \end{quote}
\end{itemize}

\begin{enumerate}
\def\labelenumi{\alph{enumi}.}
\setcounter{enumi}{1}
\item
  \begin{quote}
  {Une question d'actualité} (sauf masters)
  \end{quote}
\end{enumerate}

\begin{quote}
A partir de trois articles journalistiques minimum.
\end{quote}

\begin{itemize}
\item
  \begin{quote}
  Introduction : brève historique et/ou recontextualisation de la
  question ; présentation des médias utilisés. Présentation du plan de
  votre travail.
  \end{quote}
\item
  \begin{quote}
  Synthèse (ne pas présenter les articles séparément mais faire une
  synthèse à partir des différentes thématiques rencontrées).
  \end{quote}
\item
  \begin{quote}
  Eclairage à partir des données du cours (compléments d'information,
  regard critique, etc.)
  \end{quote}
\end{itemize}

\hypertarget{modalituxe9s}{%
\subsection{Modalités :}\label{modalituxe9s}}

\begin{quote}
La validation peut se faire par oral ou par écrit (sauf pour les
étudiants de master, pour qui un écrit est requis) :
\end{quote}

\begin{itemize}
\item
  \begin{quote}
  Un oral : présentation du travail en 20 mn, suivies de 10 mn de
  questions
  \end{quote}
\item
  \begin{quote}
  Un écrit : de 5 pages maximum, \textbf{remis le 27 mai 2022} au plus
  tard en format numérique sur Moodle (boîte de dépôt) \textbf{: format
  Word}, pour me permettre d'intégrer ma note et mes remarques.
  \end{quote}
\end{itemize}

\begin{quote}
\textbf{NB} : \textbf{La présentation orale comme écrite devra être
structurée} : introduction, plan rigoureux, conclusion. \textbf{La
présentation orale doit inclure la présentation écrite} du plan de
l'exposé (avec intitulé du sujet ou de l'œuvre choisie et nom de
l'étudiant).



pratiquement, choisir un sujet qui nous intéresse. le proposer à la pause.
Ce qui sera important, c'est de montrer qu'on a assimilé le cours.


\section{Programme des Cours}
\end{quote}

 

\begin{enumerate}
\def\labelenumi{\arabic{enumi}.}
\setcounter{enumi}{1}
\item
  \begin{quote}
  \textbf{{La réforme du droit}} \emph{(11/04/2022)}
  \end{quote}
\end{enumerate}

\begin{quote}
ARMINJON, Constance \emph{Les droits de l'Homme dans l'islam shi'ite.
Confluences et lignes de partage}, Paris, Éditions du Cerf, 2017.

BABES, L. ; OUBROU, T. \emph{Loi d'Allah, loi des hommes}, Albin Michel,
Paris, 2002.

BEN ACHOUR, Yadh \emph{Normes, foi et loi - en particulier dans
l'Islam}, Ceres, Tunis, 1993.

\emph{La deuxième Fâtiha. L'islam et la pensée des droits de l'homme},
PUF, Paris, 2011.

BENKHEIRA, M.H. \emph{L'amour de la Loi - Essai sur la normativité en
islam}, Puf, Paris, 1997.

*CARRE, Olivier \emph{L'Islam laïque ou le retour à la Grande
Tradition}, Armand Colin, Paris, 1993.

*DUPRET Beaudoin \emph{La charia. Des sources à la pratique, un concept
pluriel}, Paris, La Découverte, 2014.

DUPRET, Beaudoin (dir.) \emph{La charia aujourd'hui. Usage de la
référence au droit islamique}, Paris, La Découverte, 2012.

YOUNES Michel (dir.) \emph{La fatwa en Europe. Droit de minorité et
enjeux d'intégration}, Profac, Lyon, 2010.
\end{quote}

\begin{enumerate}
\def\labelenumi{\arabic{enumi}.}
\setcounter{enumi}{2}
\item
  \begin{quote}
  \textbf{{Le soufisme au XXe siècle}} \emph{(09/05/2022)}
  \end{quote}
\end{enumerate}

\begin{quote}
\emph{*Réveils du soufisme en Afrique et en Asie}, dossier de la revue
\emph{Archives de Sciences Sociales des Religions}, n° 135, 2006.

\emph{Confréries soufies en métropole}, dossier de la revue
\emph{Archives de Sciences Sociales des Religions}, n° 140, 2007.

AMSELLE, Jean-Louis \emph{Islams africains : la préférence soufie},
Jean-Louis Amselle, Paris, éd. du Bord de l'eau, 2017.

BALCI, Bayram \emph{Missionnaires de l'islam en Asie Centrale : les
écoles turques de Fethullah Gülen}, Paris, Maisonneuve et Larose, 2003.

CHIH, Rachida \emph{Le soufisme au quotidien : confréries d'Egypte au
XXe siècle}, Paris : Sindbad, Arles, Actes Sud, 2000.

FATHI, Habiba « Les réseaux mystiques au Kazakhstan : entre zhikr et
militantisme ? », \emph{Cahiers d'Asie Centrale}, n° 15-16, 2007, p.
223-261.

GEOFFROY, Eric « Soufisme, réformisme et pouvoir en Syrie contemporaine
», \emph{Egypte/Monde Arabe}, n° 29, 1997, p. 11-22.

*\emph{Le soufisme - Histoire, fondements et pratiques de l'islam
spirituel}, Paris, Eyrolles, 2019.

POPOVIC, A. ; VEINSTEIN, G. (dir.) \emph{Les voies d'Allah : les ordres
mystiques dans l'Islam des origines à aujourd'hui}, Paris, Fayard, 1996.

ROMEY, Alain « Rôle du wahabisme et du réformisme de la Nahda en Algérie
dans le processus d'exclusion et de marginalisation du soufisme »,
\emph{Cahiers de la Méditerranée}, 69, 2004,
\url{http://cdlm.revues.org/index735.html}

Les courants de l'islam contemporain (1)
\end{quote}

\hypertarget{introduction-guxe9nuxe9rale-du-cours}{%
\subsection{Introduction générale du
cours}\label{introduction-guxe9nuxe9rale-du-cours}}

\begin{enumerate}
\def\labelenumi{\Roman{enumi}.}
\item
  \begin{quote}
  \textbf{L'islam, une religion mondiale}
  \end{quote}

  \begin{enumerate}
  \def\labelenumii{\arabic{enumii}.}
  \item
    \begin{quote}
    L'expansion de l'islam
    \end{quote}
  \item
    \begin{quote}
    Répartition démographique
    \end{quote}
  \end{enumerate}
\item ~
  \hypertarget{une-grande-diversituxe9-confessionnelle}{%
  \subsection{Une grande diversité
  confessionnelle}\label{une-grande-diversituxe9-confessionnelle}}

  \begin{enumerate}
  \def\labelenumii{\arabic{enumii}.}
  \item
    \begin{quote}
    Sunnisme, chiisme et autres « confessions »
    \end{quote}
  \item
    \begin{quote}
    Au sein du sunnisme : les différentes écoles de droit
    \end{quote}
  \end{enumerate}
\item ~
  \hypertarget{diversituxe9-des-formes-culturelles.-quelques-exemples.}{%
  \subsection{Diversité des formes culturelles. Quelques
  exemples.}\label{diversituxe9-des-formes-culturelles.-quelques-exemples.}}

  \begin{enumerate}
  \def\labelenumii{\arabic{enumii}.}
  \item
    \begin{quote}
    La mosquée
    \end{quote}
  \item
    \begin{quote}
    Le vêtement féminin
    \end{quote}
  \item
    \begin{quote}
    Approche anthropologique : texte de Clifford Geertz.
    \end{quote}
  \end{enumerate}
\end{enumerate}

\hypertarget{glossaire}{%
\subsection{\texorpdfstring{{Glossaire}}{Glossaire}}\label{glossaire}}

\begin{quote}
{Personnes} Bukhari Ghazali

Ibn Sina (Avicenne) Mu`awiyya

Rumi

{Lieux} Siffîn

{Islam non sunnite} Alévis

Druzes Ghulât Ismaéliens

Kharijites/ibadites Zaydites

{Ecoles de droit sunnites} Hanbalites

Hanéfites Malékites Shafi`ites

{Autres termes} hijab

mihrab minbar qibla

 
\end{quote}
 
 
 
\begin{enumerate}
\def\labelenumi{\Roman{enumi}.}
\item ~
  \hypertarget{duxe9finitions}{%
  \subsection{\texorpdfstring{
  {Définitions}}{ Définitions}}\label{duxe9finitions}}

  \begin{enumerate}
  \def\labelenumii{\arabic{enumii}.}
  \item
    La \emph{shari`a}
  \item
    \begin{quote}
    Le \emph{fiqh}
    \end{quote}
  \end{enumerate}
\item ~
  \hypertarget{l-amuxe9nagement-du-droit}{%
  \subsection{\texorpdfstring{{L' « aménagement » du
  droit}}{L' « aménagement » du droit}}\label{l-amuxe9nagement-du-droit}}

  \begin{enumerate}
  \def\labelenumii{\arabic{enumii}.}
  \item
    \begin{quote}
    Méthode et principes
    \end{quote}
  \item
    \begin{quote}
    La question des \emph{fatwa}
    \end{quote}
  \item
    \begin{quote}
    L'enjeu des musulmans d'Occident et le « \emph{fiqh} des minorités »
    \end{quote}
  \end{enumerate}
\item ~
  \hypertarget{repenser-la-norme}{%
  \subsection{\texorpdfstring{{Repenser la
  norme}}{Repenser la norme}}\label{repenser-la-norme}}

  \begin{enumerate}
  \def\labelenumii{\arabic{enumii}.}
  \item
    \begin{quote}
    Le dynamisme de la \emph{shari`a}
    \end{quote}
  \item
    \begin{quote}
    La \emph{shari`a} sans le \emph{fiqh ?}
    \end{quote}
  \end{enumerate}
\end{enumerate}

\begin{quote}
\textbf{{Glossaire}}

{Personnes}

Al-Qaradawi (Sheikh Yusef) Charfi (Abdelmajid)

Oubrou (Tareq) Rahman (Fazlur) Ramadan (Tariq) Talbi (Mohammed)

{Notions}

dar ash-shahadat : \emph{« maison » (terre) du témoignage}

dar al-da`wa : \emph{« maison » (terre) du témoignage}

fatwa : \emph{opinion sur un point de la loi islamique donnée par un}
mufti\emph{.} furu' al-fiqh : \emph{« branches » du droit (discipline
juridique)}

haqq allah : \emph{droit de Dieu}

haqq al-nas : \emph{droit des gens}

`ibadat : \emph{culte/partie du droit concernant le culte.}

ijma' : \emph{consensus (des juristes, de la communauté).}

{Lieux}

Al-Azhar (Egypte) Qarawiya (Maroc) Zaytuna (Tunisie)

maqasid : \emph{intention, finalité} mu`amalat : \emph{relations/partie
du droit concernant les relations humaines.} naskh : \emph{abrogation}

shari`at allah : \emph{« voie » de Dieu} shari'at al-Masih:
``\emph{voie'' du Christ (= religion chrétienne)}

shari'at Musa: \emph{``voie'' de Moïse (= religion juive)}

talfiq : \emph{éclectisme (fait de choisir librement entre les
différentes écoles juridiques)}

usul al-fiqh : \emph{fondements du droit (discipline juridique)}
\end{quote}

\hypertarget{tareq-oubrou}{%
\subsection{\texorpdfstring{{Tareq
Oubrou}}{Tareq Oubrou}}\label{tareq-oubrou}}

\begin{quote}
\textbf{L'islam est-il une religion de la loi ?}

TAREQ OUBROU : Il y a une autre catégorie d'intellectuels qui pousse le
raisonnement plus loin en considérant que le Coran n'est ni juridique ni
éthique, mais seulement spirituel (car pour eux, même l'éthique est
contraignante).

LEÏLA BABES : Quels intellectuels ? Pouvez-vous citer des noms ?

TAREQ OUBROU : Peu importe les noms. C'est tellement répandu aujourd'hui
chez les musulmans que la foi c'est dans le cœur, dans le sens de la
libération par rapport aux normes rituelles et éthiques. Mais on oublie
qu'une foi qui demeure dans le cœur, qui ne s'exprime pas à travers un
comportement éthique et spirituel cultuel, risque d'étouffer et de
s'éteindre.

Pour eux, le Coran est un message de foi uniquement, presque une forme
de protestantisme poussé à l'extrême. Car lorsqu'il a commandé d'oeuvrer
pour le bien et de prévenir le mal, il n'a pas désigné de quel bien ou
de quel mal il s'agit ni comment réaliser tout cela. Et si l'on suit
cette démarche, on aboutira à la question suivante : y a-t-il une
éthique musulmane ? N'y a-t-il pas une morale universelle, et pourquoi
alors la chercher dans les Sources révélées ? Même le rite n'échappera
pas à de telles interrogations à un moment donné. Par exemple, ni le
nombre ni la forme des prières canoniques, deuxième pilier de l'islam,
ne sont cités dans le Coran. Puisque la prière est très contraignante,
certainement la plus contraignante pour beaucoup, voire pour la majorité
des musulmans, doit-on pour cela l'effacer et la transformer en une
notion allégorique, et donc à chacun sa prière, et on aurait autant
d'islams que de musulmans ? On tombe finalement dans un mysticisme
obscur.

Mais tant qu'on n'a pas compris que le Coran sans la Sunna du Prophète
reste illisible, et qu'il y a des règles d'interprétation issues de ces
mêmes sources, on suivra un enchaînement de questionnements pour aboutir
à l'annihilation pure et simple des valeurs de l'islam. Trouver la loi
contraignante ne doit pas aller jusqu'à abolir des références et des
normes, ce qui plongerait les musulmans encore plus dans le chaos.

Extrait du livre \emph{Loi d'Allah, loi des hommes} de T. Oubrou et L.
Babès, Albin Michel, Paris, 2002, p. 86-87.
\end{quote}

\hypertarget{lhuxe9ritage-fuxe9minin}{%
\subsection{L'héritage féminin}\label{lhuxe9ritage-fuxe9minin}}

\begin{quote}
Vous avez évoqué l'héritage de la femme qui est la moitié de celui de
l'homme. Il faut signaler que le droit successoral en islam a répondu à
des milliers de cas. La règle de la moitié de l'homme donnée à la femme
n'est pas valable pour tous les cas. Nous avons le verset qui donne dans
un cas précis un sixième à la femme et un sixième à l'homme (IV, 11-12).
L'homme qui hérite le double de sa sœur doit subvenir aux besoins de sa
famille, ce qui n'est pas un devoir pour la femme ; ce qu'elle hérite va
dans sa poche, elle peut le faire fructifier, créer une entreprise sans
la permission ni de son frère, ni de son père, ni de son mari, etc. (et,
comme vous le savez bien, la femme en droit musulman est indépendante
économiquement de son mari dans le sens où, s'il subit une faillite, ses
biens restent protégés). Elle exigera de son mari une garantie
matérielle, qu'elle définit. En plus, les femmes ont inventé leur « ruse
» : une partie de la dot est effectivement versée au début du mariage,
l'autre ne devient exigible qu'en cas de divorce... mais cette part
conditionnelle est très élevée.

Ces fictions juridiques (\emph{hiyal}), qui existent dans tous les
systèmes juridiques du monde, permettent de préserver l'esprit de la
shari`a et de remédier aux abus possibles. La femme peut donner la
\emph{zakat} (aumône légale, quatrième pilier de l'islam) --- considérée
comme des restes --- à son mari, ce qui n'est pas permis dans l'autre
sens ; il n'a pas à lui donner de ses restes, tout en sachant que les
biens de sa femme restent intouchables. Elle ne donne pas de
\emph{zakat} sur ses bijoux même si elle en a une tonne... Le juge peut
divorcer le mari si celle-ci porte plainte à cause de son manquement à
ses devoirs matériels.

Il serait donc injuste dans un tel système de donner la même part à la
sœur et au frère. Il y aura toujours mille dispositions juridiques dans
le droit musulman pour garantir l'esprit d'équité, qui

est tout sauf statique. Revenons à ce concept de l'éthicisation, il me
permet d'avancer que si la moitié donnée à la fille dans certaines
conditions lui cause une réelle injustice, la part ajoutée par un
testament peut y remédier. En effet le seul hadith qui interdit le
testament à un héritier désigné par les textes est discutable. Il ne
fait pas l'unanimité chez les traditionnistes critiques. L'énoncé du
hadith est : « Pas de testament pour un héritier légitime. » Les
héritiers légitimes sont ceux qui sont indiqués par le Coran et la Sunna
ou étendus par analogie. Ce hadith rapporté par Nassây (m. en 1277),
Tirmidhi (m. en 892), Abu Dawûd (m. en 888) et Ahmed, ne résiste pas aux
scalpels des traditionnistes critiques. C'est pourquoi Bukhâri ne l'a
pas rapporté dans son Sahîh, car il n'est pas authentique selon ses
règles. Muslim non plus ne l'a pas rapporté. Il existe une autre version
qui dit : « Pas de testament à celui qui hérite légalement sauf si les
autres héritiers consentent », sans toutefois dépasser le tiers de
l'ensemble de la succession ; certains n'ont pas établi de limites. Je
suis de l'avis de Al-Mahdi Al-Murtada qui me permet de stipuler que le
testament pour la fille en plus de sa moitié est possible. Et je vois en
ce sujet que l'abrogation du testament ne signifie pas l'abrogation de
sa permission, mais celle de son imposition qui était obligatoire.
L'imam Shafi'i avance l'\emph{ijmâ'} en cette matière mais je ne vois
pas la base sur laquelle il est fondé. Il ne pouvait pas avancer
l'abrogation par ce hadith, il n'y a pas d'abrogation d'un verset par un
hadith, avis que je

partage.

L'une des raisons légales qui me permettent cet avis est que la
situation financière des filles sous l'effet de l'éclatement des liens
familiaux a bien changé. Elle a acquis une plus grande indépendance
économique, et c'est elle qui prend dans beaucoup de cas la charge de
toute une famille. Dans ces conditions et par le biais du « testament
obligatoire », l'on peut élever la part de la sœur jusqu'à égalité de
celle du frère. C'est à traiter au cas par cas en fonction de la
jurisprudence.

Donc si j'ai évoqué l'intégration des coutumes, mais aussi des
mentalités et des conventions sociales dans le droit islamique, ce n'est
pas pour perpétuer les injustices mais justement pour les lever. Et s'il
y a une mauvaise application de droit musulman en matière d'héritage,
c'est pour une simple raison : l'éclatement des sociétés musulmanes.
C'est pourquoi j'ai parlé de l'éthicisation de la shari'a qui consiste à
moduler l'application du droit sur des bases morales en gardant
justement en vue les grands principes d'équité.

Extrait du livre \emph{Loi d'Allah, loi des hommes} de T. Oubrou et L.
Babès, Albin Michel, Paris, 2002, p. 103-105.
\end{quote}

\hypertarget{mohammed-talbi}{%
\subsection{\texorpdfstring{{Mohammed
Talbi}}{Mohammed Talbi}}\label{mohammed-talbi}}

\begin{quote}
\textbf{L'héritage féminin}

\emph{Pourquoi la question de l'héritage pose-t-elle tant problème, y
compris en Tunisie qui a pourtant fait des efforts considérables pour
moderniser le droit de la famille ?}

C'est le seul problème vraiment délicat. En matière sexuelle, l'islam
est en effet très libéral puisqu'il admet même une certaine forme de
prostitution avec le mariage \emph{mut'a}. Ce libéralisme a d'ailleurs
constitué au Moyen Age un des points forts de la polémique chrétienne
contre l'islam, considéré comme trop laxiste et permissif. Aujourd'hui,
c'est l'inverse. Le problème de l'héritage n'est pas, cela dit,
totalement insoluble. D'abord parce qu'on peut doter les filles de son
vivant pour rétablir l'équilibre. Beaucoup de parents le font déjà. A
plus long terme, si les femmes parviennent à s'imposer davantage dans la
société et à faire aboutir leurs revendications, car les hommes ne leur
feront pas de cadeaux, rien ne dit qu'on n'aboutira pas un jour à un
consensus qui trouverait une solution juridique conforme à l'islam.
L'orientation du Coran va, je vous l'ai dit, dans le sens de
l'émancipation des femmes, telle est la finalité de la révélation. On
peut donc dire que la femme est parvenue aujourd'hui à un haut degré de
maturité, que la conjoncture sociale a changé, qu'elle travaille etc.,
ce qui permet de lui concéder la parité totale avec l'homme. Il existe
trois principes en islam permettant de faire évoluer le droit et de
l'adapter

à la réalité, la \emph{maslaha} c'est-à-dire l'utilité publique, un
concept qui date du II\textsuperscript{e} siècle de l'hégire, la
\emph{zharoura}, la nécessité, c'est un principe fort puisqu'il est dit
que "la nécessité rend permis l'interdit" ; et les \emph{maqassid}, les
finalités de la loi. Ces trois instruments permettent de faire évoluer
cette dernière, mais il faut que la société y soit préparée. Cela pas
été le cas jusqu'à présent. Le jour où cela arrivera, les musulmans
trouveront dans leur patrimoine les éléments nécessaires pour faire
évoluer la loi sans rupture avec la foi.

Extrait de son livre \emph{Un islam moderne}, Cérès, Tunis, 1998, p.
153-154.

Les courants de l'islam contemporain (12)
\end{quote}

\begin{enumerate}
\def\labelenumi{\Roman{enumi}.}
\item ~
  \hypertarget{un-siuxe8cle-difficile}{%
  \subsection{\texorpdfstring{{Un siècle
  difficile}}{Un siècle difficile}}\label{un-siuxe8cle-difficile}}

  \begin{enumerate}
  \def\labelenumii{\arabic{enumii}.}
  \item
    \begin{quote}
    Facteurs politiques
    \end{quote}
  \item
    \begin{quote}
    Facteurs idéologiques et sociaux
    \end{quote}
  \end{enumerate}
\item ~
  \hypertarget{ruxe9sistance-et-renouveau}{%
  \subsection{\texorpdfstring{{Résistance et
  renouveau}}{Résistance et renouveau}}\label{ruxe9sistance-et-renouveau}}

  \begin{enumerate}
  \def\labelenumii{\arabic{enumii}.}
  \item
    \begin{quote}
    Une fin de siècle plus clémente
    \end{quote}
  \item
    \begin{quote}
    Sur la voie de la modernité : les néo-confréries
    \end{quote}
  \end{enumerate}
\item ~
  \hypertarget{soufisme-fondamentalisme-modernisme-des-relations-complexes}{%
  \subsection{\texorpdfstring{{Soufisme, fondamentalisme,
  modernisme : des relations
  complexes}}{Soufisme, fondamentalisme, modernisme : des relations complexes}}\label{soufisme-fondamentalisme-modernisme-des-relations-complexes}}

  \begin{enumerate}
  \def\labelenumii{\arabic{enumii}.}
  \item
    \begin{quote}
    Soufisme et fondamentalisme
    \end{quote}
  \end{enumerate}
\end{enumerate}

\begin{quote}
2. Confrérisme et modernisme : l'exemple des Fethullahci
\end{quote}

\begin{enumerate}
\def\labelenumi{\Roman{enumi}.}
\setcounter{enumi}{3}
\item ~
  \hypertarget{le-soufisme-en-occident}{%
  \subsection{\texorpdfstring{{Le soufisme en
  Occident}}{Le soufisme en Occident}}\label{le-soufisme-en-occident}}

  \begin{enumerate}
  \def\labelenumii{\arabic{enumii}.}
  \item
    \begin{quote}
    Les origines : René Guénon
    \end{quote}
  \item
    \begin{quote}
    Du soufisme « immigré » au soufisme français
    \end{quote}
  \end{enumerate}
\end{enumerate}

\begin{quote}
{Personnes}

Bentounès (Khaled) (1949 -) Guénon (René) (1884-1951) Gülen (Fethullah)
(1938 -)

Ilyas (Muhammad) (1885-1944) Naqshband (Baha'uddin) (m. 1388) Nursi
(Sa`id) (1873-1960)

Schuon (Frithjof) (1907-1998) Skali (Faouzi) (1953 -) Vâlsan (Michel)
(1907-1974) Yasawi (Ahmad) (m. 1166)

{Confréries} `Alawiyya Bektashi Butshishiyya Haqqaniyya
Mevlevi Naqshbandiyya Qubaysiyya Sanusiyya Shadhiliya Shishtiyya
Tijaniyya
\end{quote}

\hypertarget{glossaire-8}{%
\subsection{\texorpdfstring{\hfill\break
{Glossaire}}{ Glossaire}}\label{glossaire-8}}

\begin{quote}
{Autres mouvements} Tablighi jama`at Nurcu

Fethullahci

{Notions}

\emph{bay`a} : allégeance =\textgreater{} en contexte soufi, relation
d'allégeance liant les disciples au maître

\emph{chilla} : retraite (spirituelle)

\emph{da`wa} : « invitation » =\textgreater{} prédication, appel à
entrer dans l'islam

\emph{dhikr (zikr)} : « souvenir » : pratique soufie fondée sur la
répétition du nom de Dieu. \emph{ijaza} : « autorisation »
=\textgreater{} reconnaissance officielle du titre de \emph{shaykh}

\emph{khalifa} : « successeur », « vicaire » =\textgreater{} maître
confrérique (v. \emph{shaykh, pir})

\emph{mawled} (\emph{mouled}) : pèlerinage (annuel) à un tombeau de
saint

\emph{murid} : disciple

\emph{pir} : maître confrérique (v. \emph{shaykh, khalifa})
\emph{sheykh} : maître confrérique (v. \emph{khalifa, pir})
\emph{tariqa} : voie =\textgreater{} confrérie

\emph{tekke} : lieu de rencontre confrérique (v. \emph{zawiyya})
\emph{zawiyya} : lieu de rencontre confrérique (v. \emph{tekke})
\emph{wird} : oraison personnelle
\end{quote}

\hypertarget{une-nuxe9o-confruxe9rie-islamiste}{%
\subsection{\texorpdfstring{{Une néo-confrérie « islamiste
»}}{Une néo-confrérie « islamiste »}}\label{une-nuxe9o-confruxe9rie-islamiste}}

\begin{quote}
Au Maroc, le cheikh Abdessalame Yassine fonde en 1981 une structure de
type confrérique, la \emph{Jamâ`a}, qui a pour vocation la \emph{da`wa},
comprise comme rappel de Dieu à l'ensemble de la société. Cette
\emph{da'wa} a une dimension politique marquée : à terme est visée « la
construction d'une entité politique islamique, qui préparera des
élections islamiques, une constitution islamique et un gouvernement
islamique ». Sa doctrine est exposée dans une œuvre publiée en 1982 :
\emph{Al minhâj al-nabawi} (La voie prophétique). Voici sa pensée
présentée par le chercher Youssef Belal.

Le projet politique d'A. Yassine est largement déterminé par l'idée
qu'il se fait du rapport des croyants à Dieu, c'est-à-dire
essentiellement le rapport mystique. Non seulement la place du cheikh
médiateur entre Dieu et les hommes est indispensable mais la
transcendance vécue lors des rites soufis, le sentiment d'élévation et
de rapprochement de Dieu doit être un sentiment présent à tous les
instants et dans tous les actes des hommes. La structure du livre est
révélatrice à cet égard. Consacrant l'essentiel de son ouvrage aux dix
séances (\emph{khisâl}) qui permettront à l'homme de revivifier sa foi,
A. Yassine reprend largement les thèmes soufis : \emph{al-suhba}
(compagnonnage), \emph{al-dhikr} (remémoration), \emph{al-sidq} (la
sincérité), \emph{al-badl} (le don), \emph{al-`ilm} (le savoir),
\emph{al-jihâd} (la lutte contre l'égo).

Mais il faut bien voir que la pensée d'A. Yassine se déploie constamment
sur le registre de l'éducation, \emph{tarbiyya} et de l'organisation,
\emph{tanzîm}. La tension est permanente entre une éducation soufie et
une action qui se veut révolutionnaire dans le monde.

(\ldots)

L'allégeance à A. Yassine {[}est{]} un acte vital pour les adeptes. Pour
suivre sa voie et son enseignement, il est indispensable de faire preuve
de \emph{sidq}, c'est-à-dire que l'adepte doit suivre tout ce que lui
prescrivent la Jamâ`a et son guide. Ceux qui veulent suivre la voie du
Prophète, c'est-à-dire la voie d'A. Yassine, doivent être conscients que
l'ego, le \emph{nafs}, peut être un obstacle à tout moment. Il faut
combattre le \emph{nafs} qui est le mal (\emph{su'}). Valoriser son
\emph{nafs} est en fait incompatible avec le rôle assigné à la Jamâ`a et
à A. Yassine. Il faut être capable de se dévouer pour la Jamâ'\,`a et
pour être réceptif à l'enseignement du maître il faut que l'âme soit
vierge de l'ego. Laisser le \emph{nafs} triompher c'est avoir le destin
de cet instituteur qui a pris goût à la vie matérielle et qui a préféré
le divertissement (\emph{lahw}) à la \emph{suhba} en se laissant envahir
par d'autres habitudes. Il faut au contraire demander d'achever sa vie
parmi « les frères et les sœurs car la mort dans la \emph{da`wa} est
bien plus haute que celle dans le combat armé.

(\ldots)

Lorsqu'il en vient, après avoir traité du \emph{dhikr} en tant
qu'éducation, à aborder le \emph{dhikr} en tant qu'organisation du
mouvement, il transforme une nouvelle fois des exigences mystiques en
mode d'action politique : « le \emph{dhikr} n'est pas seulement un
travail salutaire au niveau des consciences et des mots qui sortent de
la bouche et des rites extérieurs pratiqués par le croyant. Le
\emph{dhikr} signifie aussi se lever dans les mains de Dieu lors de la
prière. Les soldats de Dieu le pratiquent pour remplir leur devoir
cultuel et parce que c'est un signe de la souveraineté de Dieu dans les
relations de Dieu avec ses soldats et dans les relations des soldats de
Dieu entre eux. C'est s'apprêter à appliquer la Loi de Dieu le jour où
le pouvoir reviendra aux croyants dans tous les domaines du pouvoir, de
la politique, de l'économie, de la société, de la justice et de la
culture du \emph{jihâd} ».

Extraits de « Mystique et politique chez Abdessalam Yassine et ses
adeptes », de Youssef Belal, \emph{Archives des Sciences Sociales des
Religions}, n° 135, 2006, p. 172-173, 181-182.
\end{quote}
