\hypertarget{second-semestre-2021-2022}{%
\section{Second semestre 2021-2022}\label{second-semestre-2021-2022}}

\begin{quote}
Anne-Sophie Vivier-Mureşan

Documents annexes au cours
\end{quote}

\hypertarget{indications-pedagogiques}{%
\subsection{INDICATIONS PEDAGOGIQUES}\label{indications-pedagogiques}}

\begin{enumerate}
\def\labelenumi{\arabic{enumi}.}
\item
  \begin{quote}
  \textbf{Préparation des cours}
  \end{quote}
\end{enumerate}

\begin{quote}
Chaque cours s'appuiera sur la lecture d'un (ou deux) texte(s)
significatif(s) pour le thème abordé. Il est demandé aux étudiants
d'avoir lu le texte à l'avance de façon suffisamment approfondie pour
pouvoir réagir aux questions abordées en cours.
\end{quote}

\hypertarget{approfondissement-des-cours}{%
\subsection{Approfondissement des
cours}\label{approfondissement-des-cours}}

\begin{quote}
En complément de chaque cours, un ou plusieurs documents seront mis en
ligne pour approfondissement (plate-forme e-learning à partir du site :
{https://formation.icp.fr/})
\end{quote}

\hypertarget{validation}{%
\subsection{Validation}\label{validation}}

\begin{quote}
La validation pourra prendre la forme suivante, au choix :
\end{quote}

\begin{enumerate}
\def\labelenumi{\alph{enumi}.}
\item
  \begin{quote}
  {Un dossier sur un thème lié au cours}
  \end{quote}
\end{enumerate}

\begin{quote}
A partir de trois articles universitaires minimum.
\end{quote}

\begin{itemize}
\item
  \begin{quote}
  Introduction : intérêt du thème au regard de l'actualité et/ou de vos
  propres orientations personnelles ; présentation des articles et de
  leurs auteurs. Présentation du plan de votre travail.
  \end{quote}
\item
  \begin{quote}
  Synthèse (ne pas présenter les articles séparément mais faire une
  synthèse à partir des différentes thématiques rencontrées)
  \end{quote}
\item
  \begin{quote}
  Contextualisation de ce thème dans l'islam contemporain, éclairage par
  le contenu du cours (enjeux, dynamiques, etc).
  \end{quote}
\end{itemize}

\begin{enumerate}
\def\labelenumi{\alph{enumi}.}
\setcounter{enumi}{1}
\item
  \begin{quote}
  {Une question d'actualité} (sauf masters)
  \end{quote}
\end{enumerate}

\begin{quote}
A partir de trois articles journalistiques minimum.
\end{quote}

\begin{itemize}
\item
  \begin{quote}
  Introduction : brève historique et/ou recontextualisation de la
  question ; présentation des médias utilisés. Présentation du plan de
  votre travail.
  \end{quote}
\item
  \begin{quote}
  Synthèse (ne pas présenter les articles séparément mais faire une
  synthèse à partir des différentes thématiques rencontrées).
  \end{quote}
\item
  \begin{quote}
  Eclairage à partir des données du cours (compléments d'information,
  regard critique, etc.)
  \end{quote}
\end{itemize}

\hypertarget{modalituxe9s}{%
\subsection{Modalités :}\label{modalituxe9s}}

\begin{quote}
La validation peut se faire par oral ou par écrit (sauf pour les
étudiants de master, pour qui un écrit est requis) :
\end{quote}

\begin{itemize}
\item
  \begin{quote}
  Un oral : présentation du travail en 20 mn, suivies de 10 mn de
  questions
  \end{quote}
\item
  \begin{quote}
  Un écrit : de 5 pages maximum, \textbf{remis le 27 mai 2022} au plus
  tard en format numérique sur Moodle (boîte de dépôt) \textbf{: format
  Word}, pour me permettre d'intégrer ma note et mes remarques.
  \end{quote}
\end{itemize}

\begin{quote}
\textbf{NB} : \textbf{La présentation orale comme écrite devra être
structurée} : introduction, plan rigoureux, conclusion. \textbf{La
présentation orale doit inclure la présentation écrite} du plan de
l'exposé (avec intitulé du sujet ou de l'œuvre choisie et nom de
l'étudiant).



pratiquement, choisir un sujet qui nous intéresse. le proposer à la pause.
Ce qui sera important, c'est de montrer qu'on a assimilé le cours.


\section{Programme des Cours}
\end{quote}

 
\begin{enumerate}
\def\labelenumi{\arabic{enumi}.}
\setcounter{enumi}{2}
\item
  \begin{quote}
  \textbf{{Le soufisme au XXe siècle}} \emph{(09/05/2022)}
  \end{quote}
\end{enumerate}

\begin{quote}
\emph{*Réveils du soufisme en Afrique et en Asie}, dossier de la revue
\emph{Archives de Sciences Sociales des Religions}, n° 135, 2006.

\emph{Confréries soufies en métropole}, dossier de la revue
\emph{Archives de Sciences Sociales des Religions}, n° 140, 2007.

AMSELLE, Jean-Louis \emph{Islams africains : la préférence soufie},
Jean-Louis Amselle, Paris, éd. du Bord de l'eau, 2017.

BALCI, Bayram \emph{Missionnaires de l'islam en Asie Centrale : les
écoles turques de Fethullah Gülen}, Paris, Maisonneuve et Larose, 2003.

CHIH, Rachida \emph{Le soufisme au quotidien : confréries d'Egypte au
XXe siècle}, Paris : Sindbad, Arles, Actes Sud, 2000.

FATHI, Habiba « Les réseaux mystiques au Kazakhstan : entre zhikr et
militantisme ? », \emph{Cahiers d'Asie Centrale}, n° 15-16, 2007, p.
223-261.

GEOFFROY, Eric « Soufisme, réformisme et pouvoir en Syrie contemporaine
», \emph{Egypte/Monde Arabe}, n° 29, 1997, p. 11-22.

*\emph{Le soufisme - Histoire, fondements et pratiques de l'islam
spirituel}, Paris, Eyrolles, 2019.

POPOVIC, A. ; VEINSTEIN, G. (dir.) \emph{Les voies d'Allah : les ordres
mystiques dans l'Islam des origines à aujourd'hui}, Paris, Fayard, 1996.

ROMEY, Alain « Rôle du wahabisme et du réformisme de la Nahda en Algérie
dans le processus d'exclusion et de marginalisation du soufisme »,
\emph{Cahiers de la Méditerranée}, 69, 2004,
\url{http://cdlm.revues.org/index735.html}

Les courants de l'islam contemporain (1)
\end{quote}

\hypertarget{introduction-guxe9nuxe9rale-du-cours}{%
\subsection{Introduction générale du
cours}\label{introduction-guxe9nuxe9rale-du-cours}}

\begin{enumerate}
\def\labelenumi{\Roman{enumi}.}
\item
  \begin{quote}
  \textbf{L'islam, une religion mondiale}
  \end{quote}

  \begin{enumerate}
  \def\labelenumii{\arabic{enumii}.}
  \item
    \begin{quote}
    L'expansion de l'islam
    \end{quote}
  \item
    \begin{quote}
    Répartition démographique
    \end{quote}
  \end{enumerate}
\item ~
  \hypertarget{une-grande-diversituxe9-confessionnelle}{%
  \subsection{Une grande diversité
  confessionnelle}\label{une-grande-diversituxe9-confessionnelle}}

  \begin{enumerate}
  \def\labelenumii{\arabic{enumii}.}
  \item
    \begin{quote}
    Sunnisme, chiisme et autres « confessions »
    \end{quote}
  \item
    \begin{quote}
    Au sein du sunnisme : les différentes écoles de droit
    \end{quote}
  \end{enumerate}
\item ~
  \hypertarget{diversituxe9-des-formes-culturelles.-quelques-exemples.}{%
  \subsection{Diversité des formes culturelles. Quelques
  exemples.}\label{diversituxe9-des-formes-culturelles.-quelques-exemples.}}

  \begin{enumerate}
  \def\labelenumii{\arabic{enumii}.}
  \item
    \begin{quote}
    La mosquée
    \end{quote}
  \item
    \begin{quote}
    Le vêtement féminin
    \end{quote}
  \item
    \begin{quote}
    Approche anthropologique : texte de Clifford Geertz.
    \end{quote}
  \end{enumerate}
\end{enumerate}

\hypertarget{glossaire}{%
\subsection{\texorpdfstring{{Glossaire}}{Glossaire}}\label{glossaire}}

\begin{quote}
{Personnes} Bukhari Ghazali

Ibn Sina (Avicenne) Mu`awiyya

Rumi

{Lieux} Siffîn

{Islam non sunnite} Alévis

Druzes Ghulât Ismaéliens

Kharijites/ibadites Zaydites

{Ecoles de droit sunnites} Hanbalites

Hanéfites Malékites Shafi`ites

{Autres termes} hijab

mihrab minbar qibla

 
\end{quote}
 
 
\begin{enumerate}
\def\labelenumi{\Roman{enumi}.}
\item ~
  \hypertarget{un-siuxe8cle-difficile}{%
  \subsection{\texorpdfstring{{Un siècle
  difficile}}{Un siècle difficile}}\label{un-siuxe8cle-difficile}}

  \begin{enumerate}
  \def\labelenumii{\arabic{enumii}.}
  \item
    \begin{quote}
    Facteurs politiques
    \end{quote}
  \item
    \begin{quote}
    Facteurs idéologiques et sociaux
    \end{quote}
  \end{enumerate}
\item ~
  \hypertarget{ruxe9sistance-et-renouveau}{%
  \subsection{\texorpdfstring{{Résistance et
  renouveau}}{Résistance et renouveau}}\label{ruxe9sistance-et-renouveau}}

  \begin{enumerate}
  \def\labelenumii{\arabic{enumii}.}
  \item
    \begin{quote}
    Une fin de siècle plus clémente
    \end{quote}
  \item
    \begin{quote}
    Sur la voie de la modernité : les néo-confréries
    \end{quote}
  \end{enumerate}
\item ~
  \hypertarget{soufisme-fondamentalisme-modernisme-des-relations-complexes}{%
  \subsection{\texorpdfstring{{Soufisme, fondamentalisme,
  modernisme : des relations
  complexes}}{Soufisme, fondamentalisme, modernisme : des relations complexes}}\label{soufisme-fondamentalisme-modernisme-des-relations-complexes}}

  \begin{enumerate}
  \def\labelenumii{\arabic{enumii}.}
  \item
    \begin{quote}
    Soufisme et fondamentalisme
    \end{quote}
  \end{enumerate}
\end{enumerate}

\begin{quote}
2. Confrérisme et modernisme : l'exemple des Fethullahci
\end{quote}

\begin{enumerate}
\def\labelenumi{\Roman{enumi}.}
\setcounter{enumi}{3}
\item ~
  \hypertarget{le-soufisme-en-occident}{%
  \subsection{\texorpdfstring{{Le soufisme en
  Occident}}{Le soufisme en Occident}}\label{le-soufisme-en-occident}}

  \begin{enumerate}
  \def\labelenumii{\arabic{enumii}.}
  \item
    \begin{quote}
    Les origines : René Guénon
    \end{quote}
  \item
    \begin{quote}
    Du soufisme « immigré » au soufisme français
    \end{quote}
  \end{enumerate}
\end{enumerate}

\begin{quote}
{Personnes}

Bentounès (Khaled) (1949 -) Guénon (René) (1884-1951) Gülen (Fethullah)
(1938 -)

Ilyas (Muhammad) (1885-1944) Naqshband (Baha'uddin) (m. 1388) Nursi
(Sa`id) (1873-1960)

Schuon (Frithjof) (1907-1998) Skali (Faouzi) (1953 -) Vâlsan (Michel)
(1907-1974) Yasawi (Ahmad) (m. 1166)

{Confréries} `Alawiyya Bektashi Butshishiyya Haqqaniyya
Mevlevi Naqshbandiyya Qubaysiyya Sanusiyya Shadhiliya Shishtiyya
Tijaniyya
\end{quote}

\hypertarget{glossaire-8}{%
\subsection{\texorpdfstring{\hfill\break
{Glossaire}}{ Glossaire}}\label{glossaire-8}}

\begin{quote}
{Autres mouvements} Tablighi jama`at Nurcu

Fethullahci

{Notions}

\emph{bay`a} : allégeance =\textgreater{} en contexte soufi, relation
d'allégeance liant les disciples au maître

\emph{chilla} : retraite (spirituelle)

\emph{da`wa} : « invitation » =\textgreater{} prédication, appel à
entrer dans l'islam

\emph{dhikr (zikr)} : « souvenir » : pratique soufie fondée sur la
répétition du nom de Dieu. \emph{ijaza} : « autorisation »
=\textgreater{} reconnaissance officielle du titre de \emph{shaykh}

\emph{khalifa} : « successeur », « vicaire » =\textgreater{} maître
confrérique (v. \emph{shaykh, pir})

\emph{mawled} (\emph{mouled}) : pèlerinage (annuel) à un tombeau de
saint

\emph{murid} : disciple

\emph{pir} : maître confrérique (v. \emph{shaykh, khalifa})
\emph{sheykh} : maître confrérique (v. \emph{khalifa, pir})
\emph{tariqa} : voie =\textgreater{} confrérie

\emph{tekke} : lieu de rencontre confrérique (v. \emph{zawiyya})
\emph{zawiyya} : lieu de rencontre confrérique (v. \emph{tekke})
\emph{wird} : oraison personnelle
\end{quote}

\hypertarget{une-nuxe9o-confruxe9rie-islamiste}{%
\subsection{\texorpdfstring{{Une néo-confrérie « islamiste
»}}{Une néo-confrérie « islamiste »}}\label{une-nuxe9o-confruxe9rie-islamiste}}

\begin{quote}
Au Maroc, le cheikh Abdessalame Yassine fonde en 1981 une structure de
type confrérique, la \emph{Jamâ`a}, qui a pour vocation la \emph{da`wa},
comprise comme rappel de Dieu à l'ensemble de la société. Cette
\emph{da'wa} a une dimension politique marquée : à terme est visée « la
construction d'une entité politique islamique, qui préparera des
élections islamiques, une constitution islamique et un gouvernement
islamique ». Sa doctrine est exposée dans une œuvre publiée en 1982 :
\emph{Al minhâj al-nabawi} (La voie prophétique). Voici sa pensée
présentée par le chercher Youssef Belal.

Le projet politique d'A. Yassine est largement déterminé par l'idée
qu'il se fait du rapport des croyants à Dieu, c'est-à-dire
essentiellement le rapport mystique. Non seulement la place du cheikh
médiateur entre Dieu et les hommes est indispensable mais la
transcendance vécue lors des rites soufis, le sentiment d'élévation et
de rapprochement de Dieu doit être un sentiment présent à tous les
instants et dans tous les actes des hommes. La structure du livre est
révélatrice à cet égard. Consacrant l'essentiel de son ouvrage aux dix
séances (\emph{khisâl}) qui permettront à l'homme de revivifier sa foi,
A. Yassine reprend largement les thèmes soufis : \emph{al-suhba}
(compagnonnage), \emph{al-dhikr} (remémoration), \emph{al-sidq} (la
sincérité), \emph{al-badl} (le don), \emph{al-`ilm} (le savoir),
\emph{al-jihâd} (la lutte contre l'égo).

Mais il faut bien voir que la pensée d'A. Yassine se déploie constamment
sur le registre de l'éducation, \emph{tarbiyya} et de l'organisation,
\emph{tanzîm}. La tension est permanente entre une éducation soufie et
une action qui se veut révolutionnaire dans le monde.

(\ldots)

L'allégeance à A. Yassine {[}est{]} un acte vital pour les adeptes. Pour
suivre sa voie et son enseignement, il est indispensable de faire preuve
de \emph{sidq}, c'est-à-dire que l'adepte doit suivre tout ce que lui
prescrivent la Jamâ`a et son guide. Ceux qui veulent suivre la voie du
Prophète, c'est-à-dire la voie d'A. Yassine, doivent être conscients que
l'ego, le \emph{nafs}, peut être un obstacle à tout moment. Il faut
combattre le \emph{nafs} qui est le mal (\emph{su'}). Valoriser son
\emph{nafs} est en fait incompatible avec le rôle assigné à la Jamâ`a et
à A. Yassine. Il faut être capable de se dévouer pour la Jamâ'\,`a et
pour être réceptif à l'enseignement du maître il faut que l'âme soit
vierge de l'ego. Laisser le \emph{nafs} triompher c'est avoir le destin
de cet instituteur qui a pris goût à la vie matérielle et qui a préféré
le divertissement (\emph{lahw}) à la \emph{suhba} en se laissant envahir
par d'autres habitudes. Il faut au contraire demander d'achever sa vie
parmi « les frères et les sœurs car la mort dans la \emph{da`wa} est
bien plus haute que celle dans le combat armé.

(\ldots)

Lorsqu'il en vient, après avoir traité du \emph{dhikr} en tant
qu'éducation, à aborder le \emph{dhikr} en tant qu'organisation du
mouvement, il transforme une nouvelle fois des exigences mystiques en
mode d'action politique : « le \emph{dhikr} n'est pas seulement un
travail salutaire au niveau des consciences et des mots qui sortent de
la bouche et des rites extérieurs pratiqués par le croyant. Le
\emph{dhikr} signifie aussi se lever dans les mains de Dieu lors de la
prière. Les soldats de Dieu le pratiquent pour remplir leur devoir
cultuel et parce que c'est un signe de la souveraineté de Dieu dans les
relations de Dieu avec ses soldats et dans les relations des soldats de
Dieu entre eux. C'est s'apprêter à appliquer la Loi de Dieu le jour où
le pouvoir reviendra aux croyants dans tous les domaines du pouvoir, de
la politique, de l'économie, de la société, de la justice et de la
culture du \emph{jihâd} ».

Extraits de « Mystique et politique chez Abdessalam Yassine et ses
adeptes », de Youssef Belal, \emph{Archives des Sciences Sociales des
Religions}, n° 135, 2006, p. 172-173, 181-182.
\end{quote}
