\hypertarget{second-semestre-2021-2022}{%
\section{Second semestre 2021-2022}\label{second-semestre-2021-2022}}

\begin{quote}
Anne-Sophie Vivier-Mureşan

Documents annexes au cours
\end{quote}

\hypertarget{indications-pedagogiques}{%
\subsection{INDICATIONS PEDAGOGIQUES}\label{indications-pedagogiques}}

\begin{enumerate}
\def\labelenumi{\arabic{enumi}.}
\item
  \begin{quote}
  \textbf{Préparation des cours}
  \end{quote}
\end{enumerate}

\begin{quote}
Chaque cours s'appuiera sur la lecture d'un (ou deux) texte(s)
significatif(s) pour le thème abordé. Il est demandé aux étudiants
d'avoir lu le texte à l'avance de façon suffisamment approfondie pour
pouvoir réagir aux questions abordées en cours.
\end{quote}

\hypertarget{approfondissement-des-cours}{%
\subsection{Approfondissement des
cours}\label{approfondissement-des-cours}}

\begin{quote}
En complément de chaque cours, un ou plusieurs documents seront mis en
ligne pour approfondissement (plate-forme e-learning à partir du site :
{https://formation.icp.fr/})
\end{quote}

\hypertarget{validation}{%
\subsection{Validation}\label{validation}}

\begin{quote}
La validation pourra prendre la forme suivante, au choix :
\end{quote}

\begin{enumerate}
\def\labelenumi{\alph{enumi}.}
\item
  \begin{quote}
  {Un dossier sur un thème lié au cours}
  \end{quote}
\end{enumerate}

\begin{quote}
A partir de trois articles universitaires minimum.
\end{quote}

\begin{itemize}
\item
  \begin{quote}
  Introduction : intérêt du thème au regard de l'actualité et/ou de vos
  propres orientations personnelles ; présentation des articles et de
  leurs auteurs. Présentation du plan de votre travail.
  \end{quote}
\item
  \begin{quote}
  Synthèse (ne pas présenter les articles séparément mais faire une
  synthèse à partir des différentes thématiques rencontrées)
  \end{quote}
\item
  \begin{quote}
  Contextualisation de ce thème dans l'islam contemporain, éclairage par
  le contenu du cours (enjeux, dynamiques, etc).
  \end{quote}
\end{itemize}

\begin{enumerate}
\def\labelenumi{\alph{enumi}.}
\setcounter{enumi}{1}
\item
  \begin{quote}
  {Une question d'actualité} (sauf masters)
  \end{quote}
\end{enumerate}

\begin{quote}
A partir de trois articles journalistiques minimum.
\end{quote}

\begin{itemize}
\item
  \begin{quote}
  Introduction : brève historique et/ou recontextualisation de la
  question ; présentation des médias utilisés. Présentation du plan de
  votre travail.
  \end{quote}
\item
  \begin{quote}
  Synthèse (ne pas présenter les articles séparément mais faire une
  synthèse à partir des différentes thématiques rencontrées).
  \end{quote}
\item
  \begin{quote}
  Eclairage à partir des données du cours (compléments d'information,
  regard critique, etc.)
  \end{quote}
\end{itemize}

\hypertarget{modalituxe9s}{%
\subsection{Modalités :}\label{modalituxe9s}}

\begin{quote}
La validation peut se faire par oral ou par écrit (sauf pour les
étudiants de master, pour qui un écrit est requis) :
\end{quote}

\begin{itemize}
\item
  \begin{quote}
  Un oral : présentation du travail en 20 mn, suivies de 10 mn de
  questions
  \end{quote}
\item
  \begin{quote}
  Un écrit : de 5 pages maximum, \textbf{remis le 27 mai 2022} au plus
  tard en format numérique sur Moodle (boîte de dépôt) \textbf{: format
  Word}, pour me permettre d'intégrer ma note et mes remarques.
  \end{quote}
\end{itemize}

\begin{quote}
\textbf{NB} : \textbf{La présentation orale comme écrite devra être
structurée} : introduction, plan rigoureux, conclusion. \textbf{La
présentation orale doit inclure la présentation écrite} du plan de
l'exposé (avec intitulé du sujet ou de l'œuvre choisie et nom de
l'étudiant).



pratiquement, choisir un sujet qui nous intéresse. le proposer à la pause.
Ce qui sera important, c'est de montrer qu'on a assimilé le cours.


\section{Programme des Cours}
\end{quote}

 
 

\hypertarget{introduction-guxe9nuxe9rale-du-cours}{%
\subsection{Introduction générale du
cours}\label{introduction-guxe9nuxe9rale-du-cours}}

\begin{enumerate}
\def\labelenumi{\Roman{enumi}.}
\item
  \begin{quote}
  \textbf{L'islam, une religion mondiale}
  \end{quote}

  \begin{enumerate}
  \def\labelenumii{\arabic{enumii}.}
  \item
    \begin{quote}
    L'expansion de l'islam
    \end{quote}
  \item
    \begin{quote}
    Répartition démographique
    \end{quote}
  \end{enumerate}
\item ~
  \hypertarget{une-grande-diversituxe9-confessionnelle}{%
  \subsection{Une grande diversité
  confessionnelle}\label{une-grande-diversituxe9-confessionnelle}}

  \begin{enumerate}
  \def\labelenumii{\arabic{enumii}.}
  \item
    \begin{quote}
    Sunnisme, chiisme et autres « confessions »
    \end{quote}
  \item
    \begin{quote}
    Au sein du sunnisme : les différentes écoles de droit
    \end{quote}
  \end{enumerate}
\item ~
  \hypertarget{diversituxe9-des-formes-culturelles.-quelques-exemples.}{%
  \subsection{Diversité des formes culturelles. Quelques
  exemples.}\label{diversituxe9-des-formes-culturelles.-quelques-exemples.}}

  \begin{enumerate}
  \def\labelenumii{\arabic{enumii}.}
  \item
    \begin{quote}
    La mosquée
    \end{quote}
  \item
    \begin{quote}
    Le vêtement féminin
    \end{quote}
  \item
    \begin{quote}
    Approche anthropologique : texte de Clifford Geertz.
    \end{quote}
  \end{enumerate}
\end{enumerate}

\hypertarget{glossaire}{%
\subsection{\texorpdfstring{{Glossaire}}{Glossaire}}\label{glossaire}}

\begin{quote}
{Personnes} Bukhari Ghazali

Ibn Sina (Avicenne) Mu`awiyya

Rumi

{Lieux} Siffîn

{Islam non sunnite} Alévis

Druzes Ghulât Ismaéliens

Kharijites/ibadites Zaydites

{Ecoles de droit sunnites} Hanbalites

Hanéfites Malékites Shafi`ites

{Autres termes} hijab

mihrab minbar qibla

 
\end{quote}
 