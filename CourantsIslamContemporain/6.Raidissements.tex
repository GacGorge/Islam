
\chapter{{Raidissement : de l'islam politique au salafisme}}
  \mn{(07/03/2022)}
 
 
 \section{bibliographie}
 
\begin{itemize}
\item

  HASSAN AL-BANNA \emph{Five tracts of Hassan al-Banna}, (1906-1949),
  Berkeley, University of California Press, 1978, (180 p.).

\item
  
  MAWDUDI \emph{Comprendre l'islam}, Paris, Association des étudiants
  islamiques en France, 1973.




ADRAOUI, Mohammed-Ali, \emph{Du Golfe aux banlieues : le salafisme
mondialisé}, Paris, PUF, 2013.

BENNOUNE Karima \emph{Votre fatwa ne s'applique pas ici}.
\emph{Histoires inédites de la lutte contre le fondamentalisme
musulman,} Paris, Temps présent, 2018.

NASR, SVR \emph{Mawdudi and the making of Islamic revivalism}, Oxford
university press, 1996.

FEILLARD, Andrée ; MADINIER, Rémy \emph{La fin de l'innocence. L'islam
indonésien face à la tentation radicale de 1967 à nos jours}, IRASEC-Les
Indes savantes, 2006.

GUIDERE, Matthieu \emph{Le printemps islamiste : démocratie et charia},
Paris, Ellipse, 2012. ROUGIER, Bernard (dir.) \emph{Qu'est-ce que le
salafisme ?}, Paris, PUF, 2008.

ROY, Olivier *\emph{Généalogie de l'islamisme}, Hachette, Paris, 1995.

SFEIR Antoine (dir.) \emph{Dictionnaire géopolitique de l'islamisme},
Paris, Bayard, 2009. TERNISIEN, Xavier \emph{Les Frères musulmans},
Fayard, Paris, 2005.
\end{itemize}



 
  \section{{Au Proche-Orient arabe} :
  {émergence des Frères
  Musulmans}}

\paragraph{De Rashid Rida (1865-1935) à Hasan al-Banna (1906-1949)}
\paragraph{ Un islam « radical »}
\paragraph{Un islam actif et mobilisateur}
\paragraph{La question du califat et de l'Etat islamique}

    


   



    



    



  \section{{Le théoricien de l'islam
  politique} : {Abu l-a`la al-Mawdudi
  (1903-1979)}}

\paragraph{ Mawdudi : une vie en rupture}
\paragraph{L'islam comme idéologie}
\paragraph{L'islam comme utopie politique}
\paragraph{L'action politique de Mawdudi}

   



    



    



    


\section{Textes}




  
 

\hypertarget{hasan-al-banna-1906-1949}{%
\subsection{\texorpdfstring{{Hasan al-Banna
(1906-1949)}}{Hasan al-Banna (1906-1949)}}\label{hasan-al-banna-1906-1949}}

\paragraph{Extraits du \emph{Message du 5ème congrès} \sn{ (Le Caire, 1951) Cf.
\emph{Etudes Arabes}, n° 61, p. 35-37}
}
\begin{quote}
    Je voudrais vous parler brièvement de la conception de l'Islam et de
l'image qu'en représentent les Frères Musulmans, pour que soit clair et
manifeste le fondement auquel nous invitons et dans lequel nous sommes
fiers de trouver notre point d'attache et notre origine.


1.
  Nous, Frères Musulmans, considérons que les préceptes de l'Islam et
  ses enseignements universels intègrent tout ce qui touche l'homme en
  ce monde et dans l'autre, et que ceux qui pensent que ces
  enseignements ne touchent que l'aspect cultuel ou spirituel, à
  l'exclusion des autres, sont dans l'erreur. L'Islam est en effet foi
  et culte, patrie et citoyenneté, religion et état, spiritualité et
  action, Livre et sabre. Le noble Coran parle de tout cela, le
  considère comme substance et partie intégrante de l'Islam, il
  recommande de s'y appliquer globalement: c'est ce qu'indique ce noble
  verset: "Parmi ce que Dieu t'a donné, recherche la vie future.
  N'oublie pas ta part de ce bas-monde et sois bon comme Dieu le fut
  envers toi " (Cor. 26, 77).
  
 

...

C'est ainsi que les Frères Musulmans ont fréquenté le Livre de Dieu,
s'en sont inspirés et guidés et sont arrivés à la conclusion que
l'Islam, c'était cette conception totale, à portée universelle et qui
devrait régir tous les aspects de la vie; ceux-ci doivent s'en
imprégner, se soumettre à son pouvoir, suivre ses préceptes et ses
enseignements, les prenant comme référence, dans la mesure où la nation
veut être authentiquement musulmane. Mais si la nation n'est musulmane
que dans son culte, suivant pour le reste d'autres modèles, cette nation
passe à côté de l'Islam. Elle ressemble à ceux que Dieu fustige:
"Croyez-vous donc à une partie du Livre et restez-vous incrédules à
l'égard d'une autre?

Quelle sera la rétribution de celui d'entre vous qui agit ainsi, sinon
d'être humilié durant la vie de ce monde et d'être refoulé vers le
châtiment le plus dur, le Jour de la Résurrection? Dieu n'est pas
inattentif à ce que vous faites" (Cor. 2, 85 b).


2.
  De plus, les Frères Musulmans croient que la base et le soutien des
  enseignements islamiques, c'est le Livre de Dieu - qu'Il soit béni et
  exalté! - et la Tradition du Prophète - que Dieu le bénisse et lui
  donne la paix! - si une nation les prend comme règle de vie, elle ne
  saurait s'égarer. Beaucoup de théories et de sciences en contact avec
  l'Islam et qui s'en sont imprégnées portent la marque des époques qui
  les ont vues naître et des peuples qui leur furent contemporains.
  C'est pourquoi il faut puiser les lois islamiques que la nation prend
  pour référence à cette source pure, la source du premier
  jaillissement. Il importe de comprendre l'Islam comme l'ont compris
  les Compagnons et leurs successeurs de bonne souche - que la faveur de
  Dieu soit sur eux! - Il faut nous attacher à ces préceptes divins et
  prophétiques pour ne pas choisir une ligne de conduite hors de celle
  donnée par Dieu, et ne pas imposer à notre époque la marque d'une
  époque non conforme à cette ligne, l'Islam étant la religion de toute
  l'espèce humaine.
\end{quote}

  \begin{Synthesis}
   Une conception holistique et politique.  "L'Islam est en effet foi et culte, patrie et citoyenneté, religion et état, spiritualité et action, Livre et sabre.
   L'acceptation d'une acculturation "marque d'une époque"
  \end{Synthesis}
 

\paragraph{La profession de foi des Frères musulmans (début des années 1930)}




\begin{enumerate}
\def\labelenumi{\arabic{enumi}.}
\item
  
  Je crois que tout est sous l'ordre de Dieu ; que Muhammad est le sceau
  de toute la prophétie adressée à tous les hommes, que la Rétribution
  {[}éternelle{]} est une réalité, que le Coran est le Livre de Dieu,
  que l'Islam est une Loi complète pour diriger cette vie et l'autre. Et
  je promets de réciter {[}chaque jour{]} pour moi-même une section du
  Coran, de m'en tenir à la Tradition authentique, d'étudier la vie du
  Prophète et l'histoire des compagnons.
  
\item
  
  Je crois que l'action droite, la vertu, la connaissance, sont parmi
  les piliers de l'Islam. Et je promets d'agir droitement en
  accomplissant les pratiques du culte et en évitant les choses
  mauvaises : je me plairai aux bonnes mœurs, j'aurai en horreur les
  mauvaises, je répandrai autant que je peux les usages musulmans, je
  préférerai l'amour et l'attachement plutôt que la rivalité et la
  condamnation, je ne recourrai aux tribunaux que contraint et forcé, je
  renforcerai les rites et la langue de l'Islam et je travaillerai à
  répandre les sciences et les connaissances utiles dans toutes les
  classes de la nation.
  
\item
  
  Je crois que le musulman doit travailler, gagner sa vie, s'enrichir,
  et qu'une part de ses gains revient de droit au mendiant et au
  misérable, et je promets que je travaillerai pour gagner ma vie et
  assurer mon avenir, que j'acquitterai la \emph{zakât} {[}aumône{]} sur
  mes biens en en gardant aussi une part volontaire pour faire la
  charité, que j'encouragerai tout projet économique utile et ferai
  progresser les produits de ma région, de mes coreligionnaires, de ma
  patrie, sans jamais pratiquer l'usure ni l'intérêt ni chercher le
  superflu au-delà de mes capacités.
  
\item
  
  Je crois que le musulman est responsable de sa famille, qu'il a le
  devoir de la conserver en bonne santé, dans la foi, dans les bonnes
  mœurs. Et je promets de faire mon possible en ce sens et d'insuffler
  les enseignements de l'Islam aux membres de ma famille. Je ne ferai
  pas entrer mes fils dans une école qui ne préserverait pas leurs
  croyances, leurs bonnes mœurs. Je lui supprimerai tous les journaux,
  livres, publications qui nient les enseignements de l'Islam, et
  pareillement les organisa- tions, les groupes, les clubs de cette
  sorte.
  
\item
  
  Je crois que le musulman a le devoir de faire revivre l'Islam par la
  renaissance de ses différents peuples, par le retour de sa législation
  propre, et que la bannière de l'Islam doit couvrir le genre humain et
  que chaque musulman a pour mission d'éduquer le monde selon les
  principes de l'Islam. Et je promets de combattre pour accomplir cette
  mission tant que je vivrai et de sacrifier pour cela tout ce que je
  possède.
  
\item
  
  Je crois que tous les musulmans ne forment qu'une seule nation unie
  parla foi islamique et que l'Islam ordonne à ses fils de faire le bien
  à tous ; je m'engage à déployer mon effort pour renforcer le lien de
  fraternité entre tous les musulmans, et pour abolir l'indifférence et
  les divergences qui existent entre leurs communautés et leurs
  confréries.
  
\item
  
  Je crois que le secret du retard des musulmans réside dans leur
  éloignement de la religion, que la base de la réforme consistera à
  faire retour aux enseignements de l'Islam et à ses jugements, que ceci
  est possible, si les musulmans œuvrent dans ce sens, et que la
  doctrine des Frères musulmans réalise cet objectif Je m'engage à m'en
  tenir fermement à ces principes, à rester loyal envers quiconque
  travaille pour eux, et à demeurer un soldat à leur service, voire à
  mourir pour eux.
  
\end{enumerate}


Traduit et cité dans Olivier Carré et Gérard Michaud {[}Michel
Seurat{]}, \emph{Les Frères musulmans} (1928-1982), Paris, Archives
Gallimard/Julliard, 1983, p. 25-26.

\begin{Synthesis}
L'Opus Dei de l'Islam ?

\begin{itemize}
    \item  importance de la Sunna en plus du Coran.  

    \item  Action droit, vertu, connaissance.  \textit{vie digne}

    \item  Importance de la richesse mais zakat, projet économique. POinte anti-chrétienne ?

    \item éducation : point important
    
    \item législation propre de l'Islam. bannière de l'Islam couvrir le genre humain. Combattre pour cela
    
    \item fraternité musulmane, abolir les divergences entre communautés.
    
    \item Retard des musulmans : éloignement de la religion
    
\end{itemize}
\end{Synthesis}

\hypertarget{abu-l-ala-al-mawdudi-1903-1979}{%
\subsection{\texorpdfstring{{Abu l-a'la al-Mawdudi
(1903-1979)}}{Abu l-a'la al-Mawdudi (1903-1979)}}\label{abu-l-ala-al-mawdudi-1903-1979}}


La situation actuelle devant laquelle nous nous trouvons est, en résumé,
la suivante: les gouvernements refusent de suivre la route que la
Communauté musulmane veut suivre, et la Communauté musulmane refuse de
suivre la politique que poursuivent les gouvernements. D'où une lutte
continuelle et acharnée entre les peuples et les gouvernements, dans
tous les pays musulmans. L'Islam contemporain, c'est cela. Des efforts
colossaux sont déployés pour transformer les musulmans en non-musulmans,
et pour cela, tous les moyens possibles sont employés.

D'une façon particulière, le domaine de l'instruction et de l'éducation.
On se sert de méthodes qui sont de nature à condamner toutes les valeurs
islamiques des musulmans, à corrompre leurs mœurs et leurs goûts, à les
détourner de leur héritage culturel et de leurs traditions; elles les
encouragent à suivre une culture qui détruit ce qui reste de leur
morale, et elles répandent chez eux les disciplines occidentales,
destinées à créer en eux des doutes sur l'Islam. Le fin mot de
l'affaire, c'est que les musulmans se trouvent soudain atteints de
faiblesse, de défaillance et d'impuissance, et qu'ils deviendront un
peuple qui a perdu sa personnalité. Et cela n'est pas impossible. Mais
ce qui est impossible, c'est qu'ils renoncent délibérément à l'Islam et
qu'ils créent volontairement un Etat irréligieux.

Quant aux conséquences désastreuses de cette lutte pour les pays
musulmans, il suffit, pour en mesurer l'étendue, d'examiner la part
qu'ont prise ces pays au mouvement de la Renaissance. Voyons- nous un
seul domaine de la vie touché par le progrès? La Turquie, par exemple,
Etat indépendant doté de la souveraineté depuis 1924, jusqu'à quel point
a-t-elle réussi à développer son industrie? A quel stade est-elle
arrivée dans le domaine du commerce? alors que le Japon, qui a atteint
l'indépendance en même temps que la Turquie, est parvenu à un niveau
incroyable de développement matériel. La cause de cette situation ne
saurait échapper aux gens intelligents. La Turquie a dévié de la bonne
route et s'est tournée vers le champ clos de la lutte interne. Ses
gouvernements successifs ont essayé de faire apparaître le peuple turc
comme un peuple non-musulman; mais le peuple turc a refusé de se
transformer en un peuple non-musulman. Au contraire, il a voulu tourner
sa face vers l'Islam, ce qui a déclenché une lutte continuelle et
furieuse entre le gouvernement et le peuple. Comment la Turquie, après
cela, poursuivrait-elle sa marche en avant et s'assurerait-elle le
progrès matériel? La situation a atteint son paroxysme ces derniers
temps, quand la rupture s'est étendue à l'armée elle-même: à ce jour,
six mille officiers ont été relevés de leurs fonctions. Et ce qui est
dit de la Turquie peut être dit également des autres pays musulmans.

Soyez-en sûrs, mes frères! Partout où il y a conflit, lutte entre la
conscience du peuple et la politique du gouvernement, la stagnation y
plante ses griffes, elle empêche le peuple d'avancer, ne serait-ce que
d'un pouce, et les pas du progrès ne se dirigeront pas vers lui. Aucun
gouvernement ne peut promouvoir la force et le bien-être, sans qu'il y
ait harmonie et union entre la conscience du peuple et la politique de
ce gouvernement, de telle sorte que, quelle que soit la politique du
gouvernement, elle soit sanctionnée par les sentiments et les
aspirations du peuple. Et quand cette politique est mise on œuvre, le
peuple déploie tous ses efforts pour la faire réussir.

C'est là le seul moyen de faire du peuple un peuple actif et vigilant.
Mais dans le cas contraire, lorsque le peuple est coupé du gouvernement,
il ne bénéficie pas du moindre progrès. Il est à présumer que le peuple,
même si le gouvernement ne lutte pas pour réaliser ses aspirations, ne
se soulèvera pas contre lui; mais que le peuple s'abstienne d'assister
et de soutenir le gouvernement, cela suffit à conduire le pays vers une
catastrophe certaine. L'insatisfaction du peuple à l'égard de son
gouvernement est donc une chose grave en soi et une situation
inquiétante.

Si ces gens (du gouvernement) persistent à aller à contre-courant, c'est
uniquement par égoïsme, égocentrisme et soumission au pouvoir des
passions. Ils n'ignorent pourtant pas ce que veut le peuple. De plus,
leur expérience passée est la meilleure preuve que ce peuple ne s'est
secoué et n'a joué son rôle héroïque dans les batailles pour la
libération qu'au nom de l'Islam. Ce sursaut du peuple, qui découle de
l'Islam, a conduit les musulmans aux rivages de la liberté et à tenir
une position de force. C'est pourquoi, ces gens-là n'ignorent pas
l'attachement sûr et solide de leur peuple à l'Islam. Mais ayant lié
leur destinée et celle de leurs enfants à l'Occident, à sa civilisation
et à ses mirages, s'étant plongés dans la civilisation occidentale et
ayant mis son empreinte sur leurs coutumes et leurs goûts, ils ne
veulent pas suivre le chemin de l'Islam, et leur égoïsme les empêche
d'agir en musulmans. La logique sur laquelle ils s'appuient est la
suivante: ils ont reçu mission de gouverner leurs peuples déshérités et
malheureux, en toutes circonstances; or l'Islam, religion de ces
peuples, ne leur plaît pas; donc les peuples doivent abandonner l'Islam.
Voilà la règle universelle qu'ils ont adoptée comme fondement de leurs
activités et de leurs orientations.

Tel est l'Islam aujourd'hui. J'essaierai maintenant de définir
brièvement ce que devrait être l'Islam demain.


\hypertarget{lavenir-du-monde-musulman}{%
\subsection{L'avenir du monde
musulman}\label{lavenir-du-monde-musulman}}


L'avenir du monde musulman dépendra de l'attitude que les musulmans
adopteront envers l'Islam. Si le monde de l'Islam s'en tient à son
attitude actuelle, faite d'hypocrisie, de déguisement et de duplicité,
s'il persiste dans son attitude hostile à l'Islam, je crains que les
peuples musulmans ne soient pas capables de maintenir leur indépendance
ni de la préserver de la ruine; au contraire, ils tomberont une deuxième
fois - ce qu'à Dieu ne plaise!- dans les fers de l'esclavage, et ils
iront à la rencontre d'une condition pire que celle dans laquelle ils
ont vécu précédemment, pendant la période de l'esclavage.

Oui, si ceux qui tiennent les rênes du pouvoir dans le monde musulman
retrouvent le droit chemin avant qu'il ne soit trop tard, et s'ils
travaillent à ramener la véritable vie démocratique, si le pouvoir
d'élire les dirigeants fait retour aux masses, de telle sorte qu'elles
puissent choisir qui elles veulent et confier librement les rênes du
pouvoir à ceux qu'elles aiment, et si l'ordre politique, l'économie,
l'enseignement sont en harmonie avec les principes de l'Islam, avec ses
objectifs et sa civilisation, alors je suis convaincu que les peuples
musulmans deviendront très vite une grande puissance dans le monde,
qu'ils seront capables de contrôler le pouvoir dans les instances
internationales et d'avoir le dernier mot.

L'existence du bloc des peuples musulmans ne doit pas être sous-estimée.
Ce bloc, qui s'étend de l'Indonésie à l'est jusqu'au Maroc à l'ouest,
qui est doté de moyens et de possibilités incommensurables, qui jouit
d'un capital énorme de main d'œuvre, s'il se met à suivre les principes
de l'Islam et à agir en commun en s'appuyant sur ce dernier, y aura-t-il
une force, que ce soit dans le monde occidental ou dans le monde
oriental, capable de lui résister ?

Extrait de son livre \emph{al-Islam al-yawm} (l'Islam aujourd'hui)

Trad. Roberto Bellani, parue dans \emph{Etudes Arabes}, n° 52, p. 81-83.
