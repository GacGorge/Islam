
\chapter{{Tendances sécularistes et nationalistes}}
  \mn{(14/02/2022)}
 
 \subsection{Bibliographie}
 
  ABDERRAZIQ, Ali \emph{L'Islam et les fondements du pouvoir}, La
  Découverte/ Cedej, Paris, 1994.
 
*BOZARSLAN, Hamit \emph{Histoire de la Turquie contemporaine}, La
Découverte, Repères, 2004. DEVLIN, John F \emph{The Ba`th Party}, Hoover
Institution Press, Stanford, 1979.

FILALI-ANSARY, Abdou \emph{L'islam est-il hostile à la laïcité ?},
Arles, Actes Sud, 2002. HOURANI, Albert \emph{La pensée arabe et
l'Occident,} Groupe Naufal Europe, Paris, 1991.

PISAI « Courants actuels dans l'Islam: le Ba`t », \emph{Etudes Arabes},
n° 63 \& n° 64, 1982-3.
 




\section{Introduction}
\begin{Def}[Sécularisme]
Une évolution juridique et politique vers un modèle Européen, et un affaiblissement des structures religieuses dans l'Etat et la société
\end{Def}

Ce courant va globalement s'imposer jusqu'aux années 1960/70 avec trois courants : 
\begin{itemize}
    \item Elites politiques qui ont grandi dans des écoles occidentales, missionnaires ou réformistes. Ces élites (Ataturk,..) faisant des études en Europe. Ils vont recommander une\textbf{ sécularisation à l'occidentale} (Bourghiba,...). Les deux pays qui n'ont pas été colonisés (Turquie, Iran) ont été les pays qui ont connu la sécularisation à l'occidentale la plus ferme.
    \item les disciples de 'Abduh qui cherchent à penser la \textbf{sécularisation dans le cadre islamique}
    \item rencontre du premier courant avec la \textbf{pensée socialiste} : nationalisme arabe
\end{itemize}
 ~
   %----------------------------------------------------------------
  \section{Le sécularisme d'importation
  occidentale : le modèle
  turc}

  

  
    
    \subsection{Aux racines : les Jeunes Turcs}
Ils vont être moteurs de la révolution constitutionnelle de 1908\sn{La révolution des Jeunes-Turcs de l'Empire ottoman en juillet 1908 est un soulèvement au cours duquel le mouvement des Jeunes-Turcs restaure la Constitution de l'Empire ottoman de 1876 et inaugure la politique multipartite dans un système électoral à deux étapes sous le parlement ottoman.}. 
Les tribunaux religieux sont placés sous la responsabilité du ministère de la Justice (entre 1908 et 18). 
    
      \subsection{La République de Mustapha Kemal}
\paragraph{Mustapha Kemal ou \textit{Atatürk}} Officier charismatique. il s'empare du pouvoir en 1923. Il crée une république à l'image de la France et en devient le chef. Politique très séculariste. 
\begin{itemize}
    \item Rejet du passé Ottoman après la défaite de 1918, qui se serait affaibli via les influences arabes et persanes, et la place que l'Islam y a joué.
    \item  Permet d'éviter les contre-pouvoirs en particulier confrériques. 
\end{itemize}

\paragraph{Soumettre l'Islam au contrôle de l'Etat}
\begin{itemize}
    \item 1924 : Abolution du Califat et expulsion. On crée une présidence des affaires religieuses qui nomme les imams, administre les mosquée, supervise les \textit{muftis}. Les imams sont des fonctionnaires. Le \textit{Diltib}.
    \item unification de l'enseignement (on ferme toutes les institutions religieuses supérieures : medrese) et on crée des Ecoles d'enseignement supérieur pour Imam d'Etat.
    \item 1925 : confréries religieuses sont interdites
    \item 1926 : code civile suisse introduit
    \item 1928 : l'Islam n'est plus religion d'Etat
    \item 1937 : le principe de laicité dans la constitution.
\end{itemize}

Des mesures symboliques : 
\begin{itemize}
    \item 1925 : interdiction du Fez
    \item 1926 : calendrier grégorien, avec dimanche comme jour férié (1935)
    \item 1928 : alphabet latin. 
    \item 1928 : Sainte Sophie devient Musée
\end{itemize}


\paragraph{moderniser l'Islam}

Rapport en 1926 : "perspective scientifique". Vision positiviste
\begin{itemize}
    \item Des mosquées propres avec des bancs
    \item instruments et musiques sacrées (on veut imposer le modèle chrétien)
    \item former les imams à la philosophie et aux sciences occidentales
\end{itemize}

\paragraph{intégrer l'Islam dans l'idéologie Turque}
Cela passe par un islam turc; remplacer l'arabe par le turc. On arrête l'apprentissage l'arabe et le perse. On traduit donc le Coran et la Sunna en Turc (mais il n'est pas utilisé finalement dans le culte). 
En 1932, l'appel à la prière en turc (mais refus de la population et il fait marche arrière).

\paragraph{L'utilisation de l'Islam comme une composante du nationalisme turque} Entre 1923 et 1930, vaste échange avec la Grèce, le critère n'a pas été un critère linguistique mais un critère religieux (chrétien turcophone). 
En tant que population musulmane sunnite, les kurdes ne peuvent être une minorité, seules les populations chrétiennes, juives,... peuvent être considérées comme une minorité. Les alévites ne sont pas reconnus comme une minorité (musulmans = sunnites). De même, la religion est mentionnée sur la carte d'identité. 
\begin{Synthesis}
\textit{une intégration de la religion dans l'Etat} et non une séparation. 
  
\end{Synthesis}

\paragraph{Ziya Gökalp
(1876-1924)}    Un jeune turc penseur de la Turquie d'Ataturk.
  \begin{quote}
Maintenant\mn{Ziya Gökalp Extraits traduits d'un ouvrage hostile au modernisme musulman:

Maryam Jameelah, \textit{Islam and modernism}

(Md Yusuf Khan, Lahore, 1968), 
p. 101-107} la mission des Turcs ne doit être que celle de découvrir le
passé pré islamique Turc qui est resté ancré dans le peuple et y greffer
la civilisation Occidentale dans sa totalité. Pour égaler les pouvoirs
européens militairement et dans les sciences et l'industrie, notre seule
voie de salut est d'adopter la civilisation Occidentale complètement! \sn{(Ziya Gökalp, \textbf{Turkish Nationalism and Western Civilisation},
New-York, 1959, p. 276.)
}

Parmi les Turcs pré-islamiques, le patriotisme a atteint ses niveaux les
plus hauts. Dans l'avenir, comme dans le passé, le patriotisme doit être
le point de moralité le plus important pour les Turcs parce que la
nation et son âme sont en fin de compte la seule unité qui existe de
soi. La fidélité à la nation doit avoir la priorité sur la fidélité à la
famille ou la religion. Le Turkisme doit donner la priorité la plus
haute à la Nation et à la Patrie. Nous créerons une civilisation
véritable - une civilisation Turque qui suivra la croissance d'une
Nouvelle Vie. Classifier les Turcs, qui sont plus justes et plus beaux
que les Aryens, avec la race Mongole n'a aucune base scientifique. La
race Turque n'a pas dégénéré - comme d'autres races, par l'alcool et le
dérèglement des moeurs. Le sang turc est resté jeune et s'est durci
comme l'acier avec la splendeur du champ de bataille. L'intelligence
turque n'est pas usée; ses sentiments ne sont pas affaiblis. On promet
la conquête de l'avenir à la résolution Turque. (Ibid., pp. 302, 271 et
60.)

La civilisation occidentale est une suite de la civilisation de la
Méditerranée antique. Les fondateurs de la civilisation de la
Méditerranée étaient des peuples Turcs comme les Sumeriens, Scythes, le
Phoeniciens et les Hyksos. Il y a eu un Âge Touranien dans l'histoire
avant les âges antiques car les habitants les plus anciens de l'Asie
Occidentale étaient nos ancêtres. Ainsi nous faisons partie de la
civilisation Occidentale et avons part intégrale à cette civilisation.
(Ibid-, pp. 266- 7)

Quand une nation parvient aux étapes les plus hautes de son évolution,
elle trouve nécessaire de changer aussi sa civilisation. Quand les Turcs
étaient des membres d'une tribu nomade en Asie Centrale, ils ont
appartenu à la civilisation de l'Extrême-Orient. Quand ils ont passé à
l'étape de l'état Sultanesque, ils sont entrés dans le secteur de
civilisation Byzantine. Et aujourd'hui dans leur transition à l'état de
nation en tant qu'Etat séculier, ils sont déterminés à accepter la
civilisation Occidentale. (Ibid., pp. 270-1)

La grande erreur des autorités du Tanzimat\sn{Le mouvement des Tanzimat fut le premier essai de
réforme et de modernisation de l'empire Ottoman vers le milieu du
19\textsuperscript{ème} siècle.} était leur
tentative de créer un amalgame mental composé d'un mélange d'Orient et
d'Occident. Ils n'ont pas réalisé que les deux, avec leurs principes
diamétralement opposés, ne pouvaient pas être réconciliés. La dichotomie
présente dans notre structure politique, le système double de tribunaux,
les deux types d'écoles, les deux systèmes de taxation, deux budgets,
les deux jeux de lois, sont tous les produits de cette erreur ... Toute
tentative de réconcilier Orient et Occident conduit à perpétuer des
conditions médiévales dans l'âge moderne et à essayer de les maintenir
en vie. De même qu'il était impossible de réconcilier des méthodes
janissaires avec un système militaire moderne, de même qu'il était
futile de synchroniser la médecine dépassée avec la médecine moderne,
ainsi est-il inutile de continuer, côte à côte, les vieilles conceptions
de la loi et les nouvelles ; les standards moderne d'éthique et les
traditionnels. Chaque civilisation a sa propre logique, ses propres
standards esthétiques, sa propre perspective du monde. Pour cette
raison, des civilisations différentes ne peuvent pas se mélanger
librement l'une avec l'autre. De nouveau, pour la même raison, quand une
société ne prend pas pour système une certaine civilisation dans sa
totalité, il ne réussit pas davantage à en prendre ses composantes. Même
s'il en prend quelques parties, il ne réussit pas à les digérer et à les
assimiler. Nos réformateurs des Tanzimat, qui ont échoué comprendre ce
point, prenaient toujours des demi-mesures dans ce qu'ils ont essayé de
faire. Avant qu'ils n'aient pris de mesures pour moderniser la
production nationale, ils ont voulu changer les habitudes de
consommation, les vêtements, l'alimentation, le bâtiment et les meubles.
D'autre part, on n'a même pas construit un noyau d'industrie digne des
standards européens parce que
les décideurs de la politique des Tanzimat ont essayé leurs réformes
sans en étudier les conditions et sans fixer des buts et des plans
précis. (Ibid., pp. 270-7).

Le but du Turkisme dans la loi est d'établir un système de loi moderne
en Turquie. La condition la plus fondamentale de notre succès à
rejoindre les rangs des nations modernes consiste à effectuer un
nettoyage complet de toutes les branches de notre structure légale pour
y effacer toute trace de théocratie et de cléricalisme. L'état qui est
libre de ces deux caractéristiques de l'état médiéval est appelé un état
moderne. En premier lieu, dans un état moderne, le droit de légiférer et
d'administrer directement appartient au peuple. Aucune fonction, aucune
tradition et aucun autre droit ne peuvent restreindre et limiter ce
droit. En second lieu, tous les membres de la nation moderne,
indépendamment de leur affiliation religieuse, sont considérés comme
égaux en tous points. Bref, toutes les dispositions existant dans nos
lois qui sont contraires à la liberté, à l'égalité et à la justice ainsi
que toutes les traces de théocratie et de cléricalisme doivent être
complètement éliminées. Le Turkisme est un mouvement séculier et ne peut
accepter que des mouvements de nature séculière. (Ibid., pp. 304-5).

C'est seulement au moyen de sa civilisation que l'Europe a été capable
de défaire les nations Musulmanes et est devenue le Maître du monde.
Pourquoi, alors, devons-nous hésiter à adopter cette même civilisation
qui s'est prouvée si capable de réussir ? Notre foi Musulmane ne nous
fait-elle pas un devoir de rechercher toutes les sortes de science et de
savoir comme notre Saint Prophète lui-même nous l'a dit, "Cherchez la
connaissance même si c'est en Chine," et "l'Étude est la propriété
perdue du croyant ; il doit la prendre partout où il la
trouve"?\sn{ Citations de deux hadiths souvent repris par les
apologètes de l'Islam pour montrer la compatibilité de la foi musulmane
avec la science. L'auteur les exploite d'une façon différente pour
inciter les lecteurs musulmans à s'ouvrir aux valeurs occidentales.} Le Japon est considéré comme puissance européenne
mais nous sommes toujours considérés comme une nation Asiatique à cause
de notre retard à accepter véritablement la civilisation européenne.
(Ibid., pp. 266-7).

La terre où l'appel à la prière résonne en Turc et où ceux qui prient
comprennent la signification de leur religion ; la terre où le Coran est
appris en Turc et où chaque homme, grand ou petit, connaît parfaitement
le commandement de Dieu - Ô fils de la Turquie, cette terre est ta
Patrie!


    
\end{quote}
    
    
    \begin{Synthesis}[Ziya Gökalp]
      Il s'appuie sur une analyse raciale typique du XIX, en valorisant la race turque qui est à l'origine de la civilisation occidentale  (Phénicien, Hyksos,...) par rapport à l'abâtardissement arabe ou perse.
      Puis en proposant une vision comparable à l'ère Meiji du Japon, se calant sur l'Occident, sans essayer un mélange, ce qui explique les mesures symboliques pour rompre avec le passé, ainsi que la sécularisation.
      Légitime par l'Islam les choix qu'il pose.
    \end{Synthesis}
    Ils n'ont pas vraiment pensé le culte musulman (cf la proposition des bancs). Pas une véritable articulation entre valeur occidentale et Islam mais en utilisant l'Islam comme un slogan.
    

    
  


 ~
   %----------------------------------------------------------------
  \hypertarget{les-disciples-de-abduh}{%
  \section{\texorpdfstring{{Les disciples de
  Abduh}}{Les disciples de Abduh}}\label{les-disciples-de-abduh}}

  

  
    
      \subsection{Qasim Amin et le statut des femmes (1863-1908)}
    
    \paragraph{Qasim Amin} voyage en France en 1880. Il a été dans le cercle de \textit{Afghani} à Paris. Revient au Caire - Modernisation de l'universation.
    
    \paragraph{1899 - La libération des femmes} Un livre qui va faire un grand remous. Un diagnostic du déclin de la civilisation musulmane, liée à l'affaiblissement des vertus morales et sociales. Il faut donc renforcer l'éducation à la maison qui se fait par les femmes, dont il faut relever le statut. Education féminine jusqu'au primaire pour éduquer les enfants et travailler (seule façon de garantir ses droits dans le foyer). 
    
    \paragraph{voile} Il pose aussi la question du voile intégrale \mn{lire Naguib Mahfouz, \textit{Impasse des deux palais} qui raconte une femme bourgeoise recluse}, qui empêche la femme bourgeoise de travailler et d'avoir un rôle dans l'Espace public.
    
    \paragraph{Interdiction de la polygamie} 'Abduh s'était positionné contre. Plus que 'Abduh, il s'oppose à la répudiation trop facile et prône une égalité de traitement homme / femme. Il a des arguments religieux pour cela.  Le Coran autorise plusieurs femmes à partir du moment où on est parfaitement équitable, ce qui est impossible. 
    
    \paragraph{Une réaction importante} beaucoup d'oppositions et le livre n'aura pas d'impacts à court terme. A noter néanmoins la figure de \textit{Hoda Sha'rawi} première feministe arabe, Egyptienne : Egypte, centre du féminisme arabe.
  
    
      \subsection{Ali Abderraziq et la laïcité (1888-1966)}
    
    \paragraph{Ali Abderraziq} Egyptien, famille liée à 'Abduh, séjour en Angleterre. Carrière de juriste. 
    
    \paragraph{Abolition du Califat en 1926} Il faut nommer un nouveau Calife, le Sherif de la Mecque par exemple. Un congrès au Caire pour réfléchir au Califat.
    \begin{itemize}
        \item Le Califat est légitime et nécessaire
        \item mais actuellement non réalisable du fait de l'impossibilité d'un pouvoir temporel
    \end{itemize}
  
  \paragraph{1925 : L'islam et les fondements du pouvoir } Le Califat n'a aucune légitimité en Islam. Dans le Coran et la Sunna, pas de réalité politique du Califat. Institution humaine imposée par les armes par les successeurs. Mohammed avait un pouvoir de type charismatique, prophétique. Les hommes reconnaissaient son pouvoir par son charisme. Mais pas de volonté de Dieu de fonder un pouvoir politique. Sinon, Dieu aurait indiqué à Mohammed de nommer un successeur. Les successeurs de Mohammed ont imposé un pouvoir royal en imposant un pouvoir politique et religieux.  Fait historique qui a nuit à l'Islam (collusion entre pouvoir politique et savant, Islam de passivité), cela a empêché le développement d'une pensée politique en Islam. 
  
  \paragraph{La shari'a comme prescriptions éthiques} ne fonde pas un système légal.
  
  \paragraph{Dieu ne se soucie pas de la forme politique du Gouvernement}  Les hommes doivent utiliser leur raison pour définir la forme de gouvernement. 

  \paragraph{unité de l'umma ni possible ni souhaitable} un verset "différentes familles... pour les bonnes oeuvres".

\paragraph{Une remise en cause forte de l'Islam} Remet en cause le statut prophétique de Mohammed et les débuts de l'Islam avec un âge d'or (\textit{les califes bien guidés}). Des fatwas de Al-Azhar qui l'interdisent d'enseigner.  Aujourd'hui, des penseurs religieux le relisent en essayant de penser l'interaction entre Islam et les formes de gouvernance. \sn{lire par exemple le livre de Filali-Ansary, \textit{l'islam est il hostile à la laïcité ?}, Marocain. Voir aussi le texte D\textit{iffusion de la pensée réformiste en Egypte : un témoignage}}
 ~
   %----------------------------------------------------------------
  \hypertarget{islam-nationalisme-et-ruxe9volution}{%
  \section{\texorpdfstring{{Islam, nationalisme et
  révolution}}{Islam, nationalisme et révolution}}\label{islam-nationalisme-et-ruxe9volution}}


  
    
      \subsection{Le nationalisme arabe}
    
  
  \paragraph{discours nationaliste arabe 1920-1930} Dans les élites sécularisées, on va penser l'Islam moins comme une religion qu'une \textit{culture}. Ce qui a créé la nation Arabe, sa culture, l'objet de sa fierté collective. 
  
  \paragraph{Un mouvement qui nait au proche et moyen orient} car le référentiel est d'abord arabe, dans des pays avec une faible vision nationaliste.
  
  \paragraph{Agrégation des chrétiens et minorités arabes non islamique} Selon Al-Bazzaz, L'Islam correspond à la moralité naturelle des arabes nomades de cette époque. Et donc tous ceux qui parlent arabes peuvent s'approprier ce passé islamique. Même les chrétiens parce qu'ils parlent arabe, peuvent s'agréger à ce nationalisme arabe.
   
      \subsection{Islam et révolution : la naissance du parti ba`th}


  \paragraph{Figure de Nasser} avec la rencontre du socialisme
   

 \begin{Def}[ba'th] : \emph{résurrection}
\end{Def}
 
 


 
\paragraph{{Michel `Aflaq
(1910-1989)}} 

\begin{quote}
    \mn{{Michel `Aflaq (1910-1989)}Publié dans \textbf{Etudes Arabes}, N° 63, 1982-2, \emph{Courants
actuels du monde arabe, le Ba't}, p. 100-102.}




Le Ba't arabe est apparu, dans la vie récente des Arabes, au milieu de
l'immobilisme, des reniements, de la recherche de l'intérêt personnel et
en pleine désintégration, le Ba't arabe est apparu comme un mouvement de
foi profonde qui polarise les cœurs purs et sains, qui attire les
volontés fortes et sincères, qui regroupe autour de lui les individus
emplis de l'amour de la Nation arabe, ceux qui ont foi en sa grandeur,
ceux qui ne se laissent pas aveugler par ses imperfections d'aujourd'hui
au point de ne plus voir son essence et les potentialités de son avenir,
ceux chez qui les illusions et les difficultés du présent n'ont pas
réussi à étouffer la volonté de travailler à révéler cette essence et à
réveiller ces potentialités. La croissance du Ba't arabe est une preuve
éclatante de foi, et une affirmation des valeurs spirituelles où la
religion prend sa source.

Mais cette qualité même, cette foi qui caractérise le Ba't arabe, c'est
elle qui lui fait comme une loi de se heurter à tous les mouvements qui
nient la foi ou se voilent sous une foi superficielle et contrefaite.
L'avènement du Ba't, il y a dix ans, fut une déclaration de guerre
ouverte au communisme en tant que mouvement matérialiste, négatif et
porteur de haine, et au nationalisme purement verbal qui était de mode,
qui représente la sécheresse, la stérilité et l'incapacité à créer, qui,
voyant dans la situation présente corrompue la vérité négative, perd
ainsi tout pouvoir clé maîtriser cette réalité. De même, il fallait
absolument s'opposer à la religiosité, courante alors, dans laquelle ces
mêmes défauts
se manifestaient. Le Ba't arabe dès sa création a défié ces phénomènes
malsains et les a tous renvoyés à une seule cause, à savoir la perte de
confiance en soi. Ainsi le communisme n'est qu'éveil factice pour ceux
qui ont perdu tout contact avec l'esprit de leur nation, qui ont
désespéré de toute libération qui viendrait de cette nation elle-même et
se sont satisfaits d'une libération qui viendrait de l'extérieur,
factice et suspecte. Le nationalisme avait accepté le mal comme un état
normal, avait pris son parti de l'égoïsme, de la servitude et du
mensonge comme des valeurs stables de la société, parce que se rebeller
contre ces maux eût exigé de lui qu'il fît confiance en la capacité de
la nation à s'en libérer. La piété avait perdu tout lien avec l'esprit
et les élans qui furent dans le passé la source de la religion, et qui
en firent un mouvement de renaissance, de renouvellement et
d'édification. Elle régressa vers un état de léthargie, de conservatisme
et d'obscurantisme, état qui ouvrit la voie toute grande au pharisaïsme
et à l'exploitation.

Le Ba't arabe appela à une conception nouvelle de la vie nationale et de
la vie en général. La base de cette nouvelle conception est la foi dans
les valeurs spirituelles et humanistes, dans la valeur de l'esprit arabe
authentique. Elle se manifeste dans la rupture définitive d'avec les
maux de la réalité présente, dans la lutte contre ces maux sur une route
montante et difficile qui conduira la Nation arabe, lentement et
laborieusement, à retrouver son âme par le moyen de ce combat jusqu'au
sang contre sa situation présente. C'est pourquoi il n'y a plus place,
dans la conception du Ba't arabe, pour quelque sentiment religieux que
ce soit qui n'assumerait pas cette foi exemplaire. Le Ba't arabe, qui
est un mouvement spirituel et actif, ne peut se séparer de la religion
ni se heurter à elle, mais il rompt avec l'immobilisme, l'égoïsme et
l'hypocrisie.

Le Ba't arabe est un mouvement nationaliste qui se tourne vers tous les
Arabes quels que soient leur confession ou leur "rite", qui considère
comme sacrée la liberté de croyance, qui regarde les religions comme
également sacrées et respectables. Mais, à côté de cela, il voit dans
l'islam un aspect national qui joua un rôle important dans la formation
de l'histoire arabe et du nationalisme arabe. Et le Ba't estime que ce
côté national de l'islam est en relation étroite avec l'héritage
spirituel des Arabes et les caractères propres de leur génie. Le Ba't
arabe fut le premier mouvement à mettre ce lien en évidence et à lui
donner sa forme définitive. Il a ouvert par là une crise qui dure encore
et il a sauvé le nationalisme arabe de deux déviations : celle du
nationalisme abstrait qui conduit fatalement à l'artificiel et à la
misère, et celle du nationalisme religieux qui le conduirait à la
contradiction et à la ruine.

Or l'islam, en tant que religion, est égal aux autres religions dans
l'Etat arabe qui traite tous ses citoyens sur un pied d'égalité et
respecte leur liberté de croyance. L'islam en tant que mouvement
spirituel qui a été intimement mêlé à l'histoire des Arabes, qui a été
imprégné de leur génie, qui a permis l'avènement de leur grande
renaissance, cet islam a une place particulière dans l'esprit du
nationalisme arabe, dans sa culture et dans le mouvement de son éveil.
Toutefois, ce rôle n'est absolument pas imposé, mais il naît de la
liberté, de la puissance de l'esprit, de ce que les Arabes sont
extrêmement attachés et en harmonie profonde et totalement libre avec
leur esprit. C'est en ce sens que le mouvement du Ba't arabe cherche
auprès de l'islam inspiration pour son renouvellement et pour sa révolte
contre les valeurs admises dans la société. Il puise à la source de
l'islam la foi, l'idéalisme et le renoncement à l'intérêt personnel et
aux séductions de ce siècle, dans le but de répandre les principes qui
libéreront les Arabes, aujourd'hui, de leur faiblesse, de leur désunion
et de leur bas niveau spirituel et social. Enfin, le Ba't arabe tire du
dynamisme éternel de l'islam\textit{ la force de résister au courant de la
réalité présente,} malsaine, il y trouve un exemple admirable à suivre
pour ce qui est du zèle sincère envers l'intérêt de la Nation, quand il
s'agit de soigner ses maux avec audace, franchise, sans chercher à
flatter à peu de frais les sentiments superficiels, et sans s'appuyer
sur les forces de l'obscurantisme, de la rancœur et de l'asservissement
de l'âme et de la pensée. Le Ba't croit fermement que cette méthode, qui
est en cohérence avec les principes sublimes qu'il proclame, c'est la
méthode à laquelle le succès est assuré, comme ce fut le cas dans le
passé et comme ce le sera toujours.


\end{quote}

\begin{Synthesis}
  Le mouvement Ba'th est un mouvement qui se veut spirituel (antimatérialiste), arabe mais pas islamique (même si la culture arabe en est imprégné).
  La dynamique spirituelle de l'Islam, c'est la \textit{Révolution}. Il faut combattre les institutions religieuses, les savants qui instrumentalisent la religion. 
\end{Synthesis}
 \begin{Def}[inqilâb]  : \emph{révolution}
\end{Def}
\paragraph{Syrie et Irak} Dans ces pays, le parti Ba'th s'est développé. Pas besoin de prôner l'Islam car en prônant la révolution, on trouve l'Islam authentique. D'où la sécularisation de ces pays. Le fait que Assad et Sadame Hussein appartiennent à des minorités religieuses a probablement joué.  
 ~
   %----------------------------------------------------------------
\hypertarget{glossaire-3}{%
\section{\texorpdfstring{{Glossaire}}{Glossaire}}\label{glossaire-3}}


{Personnes}

`Abd ar-Rahman al-Bazzaz Abdullah Cevdet (journal \emph{Içtihad}) Edmond
Rabbath

Kawakibi Michel Aflaq

Muhammad `Abduh Qustantin Zurayq Rashid Rida

Ziya Gökalp

{Lieux}

Al-Azhar

{Autres noms propres} naqshbandi

sheyh ül islam






 
