
\hypertarget{sabrina-mervin}{%
\section{Annexe : Les Autorités religieuses dans le chiisme duodécimain
contemporain - Sabrina Mervin}\label{sabrina-mervin}}

 \mn{
Référence électronique

Sabrina Mervin, « Les Autorités religieuses dans le chiisme duodécimain
contemporain », \emph{Archives de sciences sociales des religions} {[}En
ligne{]}, 125 \textbar{} janvier - mars 2004, mis en ligne le 22 février
2007, consulté le 17 octobre 2012. URL :
\url{http://assr.revues.org/1033} ; DOI : 10.4000/assr.1033

Éditeur : Éditions de l'École des hautes études en sciences sociales
\href{http://assr.revues.org/}{http://assr.revues.org}

\href{http://www.revues.org/}{http://www.revues.org}

}



 

En 1890, pour renflouer les caisses de l'État iranien, Nâsir al-Dîn Shah
accorda au baron de Reuter, sujet britannique, une concession qui lui
garantissait pour cinquante ans le monopole de la culture, de la vente
et de l'exportation du tabac. Aux yeux des Iraniens, c'était brader les
ressources du pays à des étrangers. La colère gronda, autant dans les
cercles cléricaux que dans les milieux bazaris et parmi la population.
Le Shah refusa d'entendre les injonctions des oulémas comme de plier
devant les manifestations populaires. L'activiste réformiste Jamâl
al-Dîn al-Asadâbâdî al-Afghânî (1), qui venait d'être exilé par le
souverain, poussa le grand chef religieux chiite de l'époque, un Persan
qui résidait à Sâmarrâ', Muhammad Hasan al-Shîrâzî, à réagir. En
décembre 1891, celui-ci promulgua une \emph{fatwâ} déclarant la
consommation de tabac illicite. Le boycott fut suivi dans tout le pays,
jusqu'au sein même de la cour du Shah, qui dut finalement plier et
annuler la fameuse concession (2).

Si le fait constitue un événement historique, traité en tant que tel par
les histo- riens de l'Iran, il est rapporté de façon récurrente par les
oulémas chiites, comme un récit exemplaire. Il leur sert à édifier les
croyants non seulement sur l'efficacité du boycott, c'est-à-dire de la
solidarité nationale, face aux compagnies étrangères, mais aussi sur
leur propre rôle dans la société, leur rapport au pouvoir, et l'étendue
de leur autorité. Ainsi, par exemple, le clerc libanais Muhsin al-Amîn
la raconta à ses amis nationalistes, à Damas, pour les encourager à
boycotter la Régie de l'élec- tricité, détenue par des intérêts
français, dans la Syrie sous mandat des années 1930. « Le Shah lui-même
ne pouvait plus fumer de narguilé, car les femmes du palais les avaient
tous fait disparaître, pour obéir à la \emph{fatwâ} d'al-Shîrâzî... »,
leur expliqua-t-il (3). Voilà pour démontrer l'efficacité de la
solidarité nationale. Quant à l'autorité des clercs, elle est corroborée
par l'anecdote suivante qui complète le récit du boycott du tabac et
circule encore, de nos jours, dans les milieux cléricaux
chiites, par écrit ou oralement. Le directeur de la compagnie
britannique voulut s'enquérir sur celui qui entravait son projet, afin
de mieux le combattre. Il demanda : « De combien d'hommes dispose-t-il ?
». On lui répondit que le clerc n'avait pas d'armée. « À combien se
monte sa fortune ? », poursuivit-il. On lui dit qu'il n'avait pas de
fortune et vivait très simplement. Le Britannique rétorqua alors que,
dans ces conditions, rien ne pouvait être tenté contre cet homme.

C'est non sans fierté que les chiites rapportent cette anecdote. Toute
l'autorité de leurs oulémas réside là : d'un mot, ils peuvent braver le
pouvoir en place, alors que le pouvoir n'a aucune prise sur eux,
puisqu'ils n'ont pas véritablement d'ambi- tion mondaine. On pense à
Khomeini défiant le Shah. Avec lui, l'autorité des clercs fut poussée à
son paroxysme, avec les conséquences que l'on sait, à savoir l'instau-
ration de la République islamique d'Iran. Certes, la figure du clerc
contestataire qui défie le pouvoir n'est pas propre au chiisme,
l'histoire de l'islam sunnite en fournit assez d'exemples, d'Ibn
Taymiyya (m. 1328) à Abd al-Salâm Yâsîn. Le premier s'en prit aux
Mongols, des envahisseurs qui s'étaient emparés du pouvoir et dont il
jugeait l'adhésion à l'islam superficielle. En outre, il blâma aussi les
soufis qui, au sein de sa propre société, s'adonnaient à des pratiques
rituelles non conformes à l'islam rigoriste qu'il prônait (4). Le second
est un cheikh marocain contemporain qui osa s'en prendre au roi Hasan II
en lui adressant un « conseil » (\emph{nasîha}). Il s'ancrait ainsi à la
fois dans la tradition islamique, qui permet au clerc d'admo- nester le
prince, grâce au principe affirmant la nécessité de commander le bien et
d'interdire le mal, et dans une tradition locale de clercs
contestataires, connue de son public. Son autorité sur les croyants fut
renforcée par le statut d'opposant poli- tique qu'il acquit suite à ce
défi, puisqu'il fut astreint à résidence surveillée (5).

Les oulémas chiites n'ont donc pas le monopole de la contestation.
Toutefois, on s'accorde à dire que, en général, les oulémas sunnites ont
eu tendance, au cours de l'histoire, à légitimer les pouvoirs en place,
alors que les chiites ont plutôt adopté une attitude de réserve à leur
égard. Au-delà des similitudes évidentes entre les oulémas des deux
grandes branches de l'islam, il importe de souligner que les chiites ont
des particularismes, autant dans le fondement de leur autorité que dans
les modalités de son fonctionnement. Pour résumer et schématiser les
processus en œuvre, on peut dire que l'autorité des oulémas sunnites,
depuis la centralisation imposée par l'Empire ottoman, l'application des
\emph{tanzîmât} et le contrôle du champ religieux instauré par les États
modernes, tend à se réduire. Alors que c'est l'inverse chez les oulémas
chiites, qui ont conquis leur pouvoir religieux et assis leur autorité
au fil des siècles et, surtout, depuis la seconde moitié du
XIX\textsuperscript{e} siècle. On va voir comment l'épisode de la
\emph{fatwâ} de Shirazi s'insère dans ce mouvement.

Selon la doctrine chiite, les douze imams infaillibles détenaient, de
leur vivant, toute forme d'autorité. En effet, les exégètes chiites
interprètent différemment des sunnites le verset coranique : « Ô vous
qui croyez ! Obéissez à Dieu et obéissez au Prophète et à ceux d'entre
vous qui détiennent l'autorité » (IV, 59). Pour les sunnites, les
détenteurs de l'autorité (\emph{ûlû al-amr}) peuvent être les califes et
les rois ; pour les chiites, ce sont les imams (6). À l'instar du
prophète Muhammad, dont ils transmirent la \emph{sunna}, les imams
étaient, pour leurs fidèles, les guides de la
communauté et les détenteurs des pouvoirs spirituel et temporel (7).
C'est pourquoi les califes et les rois qui lui succédèrent furent
considérés par les chiites comme des chefs injustes ou des oppresseurs.
En outre, dans les doctrines chiites, l'imamat fait partie des
fondements de la religion et complète la prophétie. Comme le prophète et
sa fille Fâtima, les imams sont tenus pour infaillibles : ils ne commet-
tent pas d'erreur. Par ailleurs, selon les anciens \emph{hadîth}
chiites, seuls les imams sont habilités à décliner les normes de la loi
sacrée.

Or, selon les doctrines, le douzième imam « disparut », entra en
occultation, en 874. Vivant, mais caché, il continua, dans un premier
temps, à communiquer ses prescriptions aux fidèles par l'intermédiaire
de quatre agents : ce fut la période de l'occultation mineure. Puis, à
partir de 941, il cessa d'avoir recours à des agents, et le lien avec
ses adeptes fut rompu : on entra dans la période de l'occultation
majeure, qui se poursuit actuellement. Les croyances chiites veulent
que, au terme de cette période, l'imam attendu, le Mahdî ou Qâ'im,
reviendra sur terre pour y restaurer la justice, avant la fin des temps
et le jugement dernier.

Ainsi, à partir de 941, la communauté des croyants se retrouva sans
guide, aussi bien pour les affaires spirituelles que pour les affaires
temporelles. Selon le \emph{hadîth} chiite, « Toute bannière élevée
avant le soulèvement du Qâ'im appartient à un rebelle contre Dieu
(\emph{tâghût)} », tout pouvoir politique était considéré comme inique,
illégitime (8). Or, le temps passant, il devenait de plus en plus
difficile, pour la communauté, de se passer d'autorité, de référence.
Des questions centrales restaient sans réponse : à qui verser les impôts
religieux ? Qui peut diriger la prière du vendredi ou lancer le
\emph{jihâd} ? Qui détient le pouvoir de statuer, de juger, d'arbi- trer
les conflits et de faire appliquer les peines, en l'absence de l'imam ?
Peu à peu, les oulémas procédèrent à une élaboration doctrinale, afin de
s'attribuer les fonc- tions et les pouvoirs de l'imam, et d'agir en son
nom en tant que son délégué (\emph{nâ'ib al-imâm}) (9). En outre, ils
impulsèrent un processus de rationalisation, voire d'idéologisation des
doctrines qui s'effectua par étapes successives, au moyen, notamment, de
l'introduction de différents concepts clefs. C'est ce long processus qui
allait permettre à Khomeini de concevoir sa théorie de \emph{wilâyat
al-faqîh}, « le pouvoir politico-charismatique » ou « guidance » du
théologien-juriste, sur laquelle est fondé l'État islamique en Iran
(10).

L'ouverture de la « porte de l'\emph{ijtihâd} », selon l'expression en
vigueur, est le volet principal de ce processus. L'exercice de
l'\emph{ijtihâd} consiste à extraire les pres- criptions du droit
islamique des quatre sources de ce droit, c'est-à-dire, d'une part, des
textes sacrés que sont le Coran et la \emph{sunna} et, d'autre part,
d'une série de méthodes et de techniques que l'on regroupe autour des
concepts de consensus (\emph{ijmâ`)} et de raisonnement analogique
(\emph{qiyâs}) chez les sunnites, et de raison (\emph{`aql}) chez les
chiites. C'est dire qu'il s'agit d'élaborer les normes de la loi sacrée,
la \emph{charî`a}. Le fait d'exercer l'\emph{ijtihâd} permet donc de
répondre à de nouvelles ques- tions, de réagir à de nouvelles situations
et, plus largement, d'introduire le changement dans les normes tout en
revenant aux textes. Il s'oppose au \emph{taqlîd}, le conformisme
juridique, consistant à reproduire les normes établies par les anciens.

C'est dans ce mouvement d'ouverture que réside une différence
essentielle entre l'histoire des doctrines chiites et celle des
doctrines sunnites. En effet, même si cette théorie doit être modulée et
affinée aujourd'hui, les historiens admettent que le droit islamique
sunnite est théoriquement figé, depuis la fixation de ses quatre écoles
(malékite, hanéfite, hanbalite et chaféite), au XI\textsuperscript{e}
siècle. Depuis cette date en effet, les juristes ont eu une large
tendance au conformisme, hormis les exceptions notoires de grands
savants de l'islam tels Ghazâlî (m. 1111), Ibn Taymiyya (m. 1328) ou
Suyûtî (m. 1505). Cette situation a perduré jusqu'au
XVIII\textsuperscript{e} siècle, lorsque quelques oulémas commencèrent à
prôner l'exercice de l'\emph{ijtihâd}. Puis, le mouvement s'est
intensifié, à partir de la fin du XIX\textsuperscript{e} siècle, quand
des modernistes revendiquèrent la réouverture de sa porte, afin de
mettre l'islam en accord avec l'esprit du siècle. On considère ainsi que
c'est le choc avec la culture envahissante de l'Europe qui incita des
oulémas réformistes à réagir et à entamer une réflexion sur la question.

Les doctrines chiites connurent le mouvement inverse. Alors que la porte
de l'\emph{ijtihâd} se fermait chez les sunnites, les chiites
s'employèrent à l'ouvrir de plus en plus largement et à octroyer des
pouvoirs croissants aux oulémas. Al-Tûsî (m. 1067) donna la première
impulsion à ce mouvement, que poursuivirent les savants de Hilla Ibn
Idrîs (m. 1201), al-Muhaqqiq (m. 1277), al-`Allâma (m. 1325), puis
d'autres du Jabal `Âmil (l'actuel Liban-Sud), Zayn al-Dîn al-`Âmilî dit
« le Second Martyr » (m. 1557) et son fils Hasan (m. 1602). Au même
moment, en Iran, le souverain safavide Shah Tahsmap nommait le juriste
al-Muhaqqiq al-Karakî (m. 1534) représentant de l'imâm (11). Ce courant
du chiisme duodécimain, appelé \emph{usûlî}, se renforça au
XIII\textsuperscript{e} siècle ; il devint majoritaire, et les doctrines
s'affinè- rent (12).

\hypertarget{linstitution-de-la-marjaiyya-pilier-de-lautorituxe9-religieuse}{%
\subsection{L'institution de la marja`iyya, pilier de l'autorité
religieuse}\label{linstitution-de-la-marjaiyya-pilier-de-lautorituxe9-religieuse}}

Un pas décisif fut franchi avec la systématisation de la référence à un
savant habilité à exercer l'\emph{ijtihâd}, en matière de prescriptions
religieuses. Elle fut mise en œuvre par Murtadâ al-Ansârî (m. 1864), qui
institua la fonction de \emph{marja`,} « réfé- rence à suivre », ou «
source d'imitation » pour les croyants (13). Selon cette théorie, les
croyants doivent se conformer aux avis émis par le \emph{marja`}, pour
tout ce qui concerne les questions afférant au droit islamique : d'où un
nouveau sens du terme \emph{taqlîd}, qui désigne désormais, pour les
chiites, le fait de se conformer aux prescriptions d'un \emph{marja`}
vivant, et non pas, comme c'est le cas chez les sunnites, de se
conformer aux écrits des anciens oulémas d'une école juridique donnée.
Les prescriptions, parallèlement, s'étaient élargies à tous les domaines
de la vie sociale et politique. Ainsi les oulémas pouvaient-ils
s'arroger certaines fonctions de l'imam, comme celle de déclarer le
\emph{jihâd} : ce que fit Ja`far Kâshif al-Ghitâ' (m. 1812), lorsqu'il
autorisa Fath `Alî Shah à mener la guerre sainte contre les
Russes (14). Si, par ce geste, le clerc cautionna la politique du
prince, d'autres cessèrent ensuite de composer avec le pouvoir, quitte à
s'opposer à lui. La \emph{fatwâ} que promulgua Muhammad Hasan al-Shîrâzî
fut une première étape. Après cela, des clercs s'investirent dans les
affaires politiques et, notamment, participèrent au mouvement
constitutionnaliste (1906-1911) visant à restreindre le pouvoir du Shah.

Après les travaux d'al-Ansârî, d'autres oulémas précisèrent la doctrine,
quant aux modalités du \emph{taqlîd} et aux critères de choix du
\emph{marja`}, et mirent en place le fonctionnement de l'institution.
Plus tard, des écrits tendirent à montrer qu'elle datait des débuts du
chiisme et dressèrent des listes de \emph{marja`}, à partir des plus
anciens. Même si l'idée a été reprise par des chercheurs, il s'agit bien
là d'une
« tradition inventée » (15).

Selon cette doctrine, tout croyant doit suivre les prescriptions d'un
\emph{marja`}, énoncées par celui-ci dans un traité pratique de droit
islamique, ainsi qu'à ses \emph{fatwâ.} S'il opte pour un \emph{marja`}
selon des règles établies, elles ne sont pas contrai- gnantes et son
choix s'opère donc, au bout du compte, en toute liberté. Les ancrages
ethniques, claniques, locaux et familiaux constituent bien évidemment
des facteurs influents, mais pas forcément décisifs. En outre, il arrive
que des adeptes d'un \emph{marja`} en suivent un autre, pour certaines
questions : ainsi, par exemple, dans les années 1980, bon nombre de
chiites suivaient Kho'i (m. 1992) pour les ques- tions religieuses
classiques, et Khomeini (m. 1989) pour les affaires politiques. Kho'i
étant par ailleurs très rigoriste en matière de voile, puisqu'il
prescrivait aux femmes de se cacher le visage, il leur laissait le
loisir de suivre un autre \emph{marja`,} sur ce point précis ; certaines
se référaient donc à Khomeini en la matière (16). Enfin, le croyant doit
régler ses impôts religieux au \emph{marja`} qu'il suit. Ainsi, il lui
verse la \emph{zakât}, un impôt commun aux grandes branches de l'islam,
mais aussi le \emph{khums}, spécifique au chiisme, se montant au
cinquième de ce qui lui reste lorsqu'il a dépensé ce qu'il lui faut pour
vivre.

Avec l'instauration du système de \emph{taqlîd}, un lien direct est donc
établi entre le croyant et un clerc de haut rang auquel il se réfère
pour suivre les préceptes de la loi, et ainsi assurer son salut. C'est
l'une des raisons pour lesquelles on parle parfois de « clergé »
concernant les autorités religieuses chiites qui, en outre, sont
organisées en une manière de hiérarchie. Toutefois, celle-ci est très
différente de la hiérarchie chrétienne, à laquelle on la compare
parfois, et même du système sunnite mis en place par les Ottomans, dont
se sont inspirés les États musulmans modernes.

Il n'y a aucune procédure de désignation formelle du \emph{marja`} ;
celui-ci n'est ni élu, ni désigné : on dit qu'il « émerge ». En fait,
tout se joue à trois niveaux : celui des cercles des clercs de haut
rang, celui des milieux commerçants et financiers influents, et,
\emph{in fine}, au niveau des croyants qui entérinent, ou non, les avis
des précédents. Ajoutons à cela l'influence éventuelle de l'État, sur
laquelle nous reviendrons. Le \emph{marja`} doit, d'abord, être reconnu
comme le plus savant de son temps et jouir d'une grande réputation dans
l'enseignement des sciences reli- gieuses ; c'est le premier critère de
sélection. Ensuite, il doit faire preuve de piété et de probité morale,
et être capable de percevoir les impôts et de les employer à subvenir
aux besoins des étudiants en sciences religieuses. Voilà donc les
principales qualités requises pour prétendre à la fonction, qui
s'ajoutent aux critères de bases, à savoir : être un homme, de naissance
légitime, d'âge mûr, et doué d'intelligence.

La première condition à remplir pour devenir \emph{marja`,} être « le
plus savant » en sciences religieuses, est difficile à apprécier. Aussi,
des critères plus ou moins objectifs ont été peu à peu mis en œuvre.
Avant tout, il faut compter parmi les oulémas de haut niveau, dûment
habilités à exercer l'\emph{ijtihâd} par leurs maîtres, et donc
détenteurs de certificats l'attestant. Dans les cercles de ces savants,
appelés \emph{mujtahid}, il faut être reconnu comme l'un des meilleurs.
Ce qui ne peut advenir à un jeune clerc, mais requiert âge et
expérience. En outre, la réputation du candidat se « mesure » au nombre
des étudiants qui suivent son enseignement, et à la manière dont ils le
reçoivent : s'ils prennent le soin de noter précisément les cours du
maître, et de les publier, cela ne lui donne que plus de poids. C'est
donc au sein de l'école en sciences religieuses, la \emph{hawza}, et
dans les cercles de ses pairs que le \emph{mujtahid} commence à se
distinguer. S'il répond aux incitations de son entourage à rédiger un
traité pratique de droit islamique, susceptible de devenir un guide pour
les croyants, il se pose alors en candidat à la \emph{marja`iyya}. Des
hommes d'influence (religieuse, sociale, financière) interviennent pour
le promouvoir et, au bout de la chaîne, les fidèles vont décider de se
référer à lui, ou non : l'allégeance populaire est la dernière étape.
C'est ainsi que le \emph{marja`} émerge.

Des liens étroits unissent le \emph{marja`} aux étudiants en sciences
religieuses, donc à la \emph{hawza}. Ce terme désignait d'ailleurs, à
l'origine, le cercle que forment des disciples autour d'un maître, avant
que son sens s'étende à l'école religieuse, puis au système
d'enseignement qu'elle dispense. La plus ancienne \emph{hawza} est celle
que fonda le cheikh al-Tûsî (m. 1067) à Najaf, l'une des villes saintes
du chiisme, aux côtés des autres « seuils sacrés » d'Irak, Karbala,
al-Kâzimiyya et Sâmarrâ', qui abritent des mausolées d'imams (17).
Centres de pèlerinage, ces villes saintes sont aussi des foyers de
savoir, et, particulièrement, Najaf. Celle-ci fut le siège histo- rique
de la \emph{marja`iyya}. En effet, si d'autres villes saintes ont pu, un
temps, voir émerger et abriter un \emph{marja`,} c'est Najaf qui en
compta le plus grand nombre (18). Elle est en cela désormais en
concurrence avec la ville iranienne de Qom, un ancien foyer de savoir
chiite qui fut réactivé, à partir des années 1920. Sa \emph{hawza} fut
ensuite la grande rivale de Najaf, surtout à la faveur de la révolution
iranienne qui lui permit de s'étendre et de se moderniser ; dans le même
temps, la répression qui frappait les chiites, en Irak, provoquait le
déclin de celle de Najaf (19). Après la chute du régime de Saddam
Hussein, celle-ci devrait renaître, et reprendre sa place. Toutefois,
force est de constater que déjà, le \emph{marja`} le plus suivi
aujourd'hui dans le monde chiite est `Alî Sîstânî qui, s'il est persan
d'origine, se réclame néanmoins de la \emph{hawza} de Najaf, où il
réside.

Une autorité supra-étatique convoitée

Au milieu du XX\textsuperscript{e} siècle, la centralisation de la
\emph{marja`iyya} à Najaf entraîna une organisation de l'institution. Le
\emph{marja`} de l'époque, Abû al-Hasan al-Isfahânî, systématisa le
recours à des représentants chargés d'assurer la liaison avec ses
adeptes résidant dans des zones éloignées. Les \emph{marja`} ouvrent
désormais des bureaux, partout où ils sont représentés auprès des
croyants, qui diffusent leurs écrits et leurs \emph{fatwâ}, et récoltent
les impôts religieux. Ce mouvement de centralisa- tion, qui a permis une
certaine bureaucratisation de l'institution, a oscillé avec un mouvement
de décentralisation, qui a engendré le pluralisme de l'autorité reli-
gieuse. En effet si la doctrine présente un \emph{marja` a`lâ}, ou «
référence suprême », dans l'histoire, le consensus autour d'un seul
\emph{marja`} n'a pas toujours été réalisé, et ils furent parfois
plusieurs à assurer la fonction simultanément (20). C'est le cas
aujourd'hui, où, après bien des débats et des écrits sur la question, la
tendance est non seulement à un pluralisme de fait, mais aussi à un
pluralisme souvent affirmé, et revendiqué par les acteurs. Cela, en
réaction à la tentative de centralisation iranienne opérée par Ali
Khamenei au milieu des années 1990.

Les milieux cléricaux chiites sont très attachés à l'indépendance de la
\emph{marja`iyya} par rapport à l'État, qui se fonde à la fois sur le
caractère transnational du chiisme, et sur l'organisation de
l'institution. Son autonomie financière en est un facteur essentiel.
Grâce aux impôts religieux qu'il reçoit, le \emph{marja`} finance les
écoles religieuses qui sont sous son autorité, notamment en versant les
salaires des professeurs, et en allouant des bourses aux étudiants. Il
rémunère par ailleurs ses représentants, et peut rétribuer des clercs.
En outre, il fait construire des mosquées, des \emph{husayniyya} (21),
des hôpitaux, des dispensaires, des orphelinats et autres sociétés de
bienfaisance. Les fonds investis renforcent ainsi son capital
symbolique, à savoir son prestige et son autorité religieuse. En outre,
le système permet à la \emph{marja`iyya} d'assurer la formation des
clercs et d'entretenir une hiérarchie reli- gieuse indépendante -- et,
ce, même en Iran, malgré les pressions du gouvernement islamique. C'est
une grande différence avec le monde sunnite, où les États modernes se
sont employés, en ayant recours à différentes stratégies, à contrôler la
formation des clercs et leur nomination à des fonctions religieuses
officielles (22). Certes, cela n'empêche pas la présence, en parallèle,
d'autorités autoproclamées, qui occupent une partie du champ religieux,
mais celles-ci ne relèvent alors d'aucune hiérarchie, et ont parfois des
statuts précaires. Les milieux cléricaux chiites, à l'inverse,
contrôlent et régulent le champ religieux, par le biais de la
\emph{marja`iyya}. Ce qui n'a pas manqué de susciter des tentatives de
mainmise de la part de certains États.

Ainsi, avant la révolution islamique, l'Iran essayait déjà d'influer,
d'une certaine manière, sur le choix d'un nouveau \emph{marja`.} En
effet, depuis l'accession à la \emph{marja`iyya} de Borûjerdî en 1945,
lorsqu'un \emph{marja`} venait à mourir, le Shah adressait un télégramme
de condoléances au grand clerc qu'il voulait voir lui succéder. Aussi
les chiites attendaient-ils de voir à qui serait destiné le télégramme,
pour connaître la position du souverain (23). L'avènement de la
république isla- mique changea radicalement la situation, au moins à
l'intérieur de l'Iran, puisque Khomeini assura à la fois la
\emph{wilâyat al-faqîh}, la guidance du théologien-juriste, c'est-à-dire
l'exercice du pouvoir politique, et la \emph{marja`iyya}, le pouvoir
spirituel. Toutefois, il ne l'exerça pas de manière absolue, puisque
d'autres \emph{marja`} étaient suivis, même en Iran, notamment Kho'i et
Montazeri. Pressentant, avant de mourir, que sa succession poserait un
sérieux problème, Khomeini révisa sa théorie de \emph{wilâyat al-faqîh},
de sorte qu'il opéra, \emph{de facto}, une séparation entre pouvoir
reli- gieux et pouvoir politique, même si celui-ci continuait à être
tenu par des clercs. C'est d'ailleurs la situation qui prévaut, \emph{de
facto}, dans l'Iran actuel. Ensuite, il nomma à la tête de l'État Ali
Khamenei, un clerc qui ne pouvait prétendre à la \emph{marja`iyya} car
il n'avait pas les qualifications requises en matière de savoir reli-
gieux, puisqu'il n'était même pas \emph{mujtahid}. D'autres que lui
furent donc mis en avant par la république islamique (24). Toutefois, en
1995, Khamenei fut déclaré \emph{marja`.} Face à l'opposition que
suscita cette décision, Khamenei déclara qu'il n'était pas candidat à la
\emph{marja`iyya} en Iran, mais pour le reste du monde chiite. Le
paradoxe de la situation ne manqua pas de frapper les chiites de
l'extérieur, qui se retrouvaient face à une \emph{marja`iyya} imposée
par l'Iran, alors que les sujets iraniens, eux, gardaient la possibilité
de choisir. Ils refusèrent, pour la grande majorité d'entre eux, de se
plier à cette politique d'unification et de centralisation. L'Iran dut
se rétracter et accepter de voir se développer une \emph{marja`iyya}
plurielle, qu'il pouvait tenter de circonscrire à l'intérieur, mais
incontrôlable, hors de ses fron- tières. Le cas de Muhammad Husayn Fadl
Allâh, au Liban, est très significatif : il ne s'aligna pas sur la
direction iranienne, mais s'imposa lui-même comme un \emph{marja`}
indépendant, malgré les pressions qu'il subit (25).

Le régime irakien tenta, lui aussi, de contrôler la \emph{marja`iyya}.
Comme Kho'i, puis ses successeurs, lui échappaient totalement, il promut
un \emph{marja`,} Muhammad al-Sadr, un clerc de Najaf issu d'une
prestigieuse famille d'oulémas. Ce geste fut pris par les chiites,
notamment à l'extérieur, pour ce qu'il était : une ingérence dans les
affaires de la \emph{marja`iyya}. Toutefois, peu à peu, Muhammad al-Sadr
gagna une certaine popularité, à Najaf, et prit ses distances par
rapport au régime qui l'avait mis en place. Tant et si bien qu'il en
vint à le critiquer publiquement, et le paya du prix de sa vie : après
l'avoir assigné à résidence, Saddam Hussein le fit assassiner en février
1999 (26). Depuis, ses partisans sont restés actifs et ont une assise
popu- laire, à Najaf, ainsi qu'une représentation à l'extérieur,
notamment à Sayyida Zaynab, en Syrie, où résident de nombreux réfugiés
irakiens.

Ainsi, les deux États les plus à même d'avoir mainmise sur la
\emph{marja`iyya}, et donc sur la hiérarchie religieuse chiite, ne sont
pas parvenus à l'accaparer ou à l'instrumentaliser. Reste que le fait de
l'abriter confère au moins un certain prestige vis-à-vis des communautés
chiites, et apporte une activité économique non négli- geable au pays.

L'indépendance de la \emph{marja`iyya} par rapport à l'État favorise le
caractère trans- national du chiisme, puisqu'elle réunit sous une même
autorité des réseaux d'adeptes implantés dans différents pays. Ce qui
n'empêche ni l'ancrage des chiites
dans leur région d'origine, ni leur attachement à une identité
nationale. Bien plus, le mouvement d'intégration des communautés chiites
dans les États dont ils sont ressortissants a tendance à s'intensifier.
Au plan de l'organisation des affaires reli- gieuses, le Liban a été le
premier pays (l'Iran n'étant pas à prendre en compte) à permettre à la
communauté chiite d'être représentée par une institution propre, le
Conseil supérieur islamique chiite. Le clerc qui parvint à obtenir sa
fondation en 1967, Mûsâ al-Sadr, était d'ailleurs un iranien récemment
installé au Liban. La création de ce Conseil fut un pas de plus vers la
reconnaissance d'une hiérarchie religieuse chiite interne, qui se charge
de régler les affaires de la cléricature liba- naise, tout en
entretenant des liens, en parallèle, avec la hiérarchie supra-étatique
issue du système de la \emph{marja`iyya} (27).

Enfin, l'indépendance de la \emph{marja`iyya} par rapport à l'État
implique celle de la \emph{hawza} et de la formation des clercs. La
réforme du système de l'enseignement reli- gieux supérieur chiite n'est
donc pas comparable à celle que connurent les grands établissements
sunnites comme al-Azhar, qui se modernisèrent sous la pression de
l'État. Elle se fit plus lentement, suite aux initiatives individuelles
et dispersées de grands \emph{mujtahid} ou de \emph{marja`.} Les
nouvelles générations de clercs chiites ainsi formés ne furent donc pas
promues par l'État, mais par la hiérarchie religieuse. Ce qui explique,
en partie, que les mouvements islamiques chiites sont issus de cette
hiérarchie, contrairement à la tendance prévalant dans les mouvements
sunnites.

Les oulémas chiites : modèles classiques et nouveaux acteurs

Rappelons que les fondateurs des mouvements islamistes sunnites, Hasan
al-Bannâ, Sayyid Qutb, Mawdûdî, étaient de petits intellectuels,
autodidactes en matière de sciences religieuses. Bon nombre de leurs
successeurs ont ce type de profil, même s'ils furent rejoints, ensuite,
par des clercs. Cela ne fut pas le cas des dirigeants des mouvements
chiites : une autorité religieuse autoproclamée avait peu de chance de
se faire entendre sans la reconnaissance des milieux cléricaux. Or,
s'ils sont théoriquement ouverts à tout nouveau venu, à condition qu'il
fasse ses preuves en matière de sciences religieuses et qu'il se
distingue par l'exercice de l'\emph{ijtihâd}, on constate que ceux-ci
sont peu nombreux à sortir du rang. Les grands clercs chiites forment
une aristocratie religieuse qui se reproduit à travers la \emph{hawza}.
La plupart d'entre eux appartiennent à des « familles de sciences » dont
sont issus les oulémas, depuis plusieurs générations. Ces familles
tirent leur légitimité d'une ascendance prestigieuse, un savant qui a
marqué l'histoire du chiisme, par exemple. Bien plus, certaines
proclament être des lignages descendant du prophète Muhammad, généalogie
dûment certifiée par des autorités religieuses à l'appui. Ce sont les
familles de \emph{sayyid}, dont les membres portent ce titre honorifique
et se coif- fent d'un turban noir marquant leur ascendance (28). Si,
dans les milieux sunnites, on a pu observer que suivre un enseignement
en sciences religieuses pour se former à la cléricature était une
stratégie d'ascension sociale, chez les chiites, c'est un moyen de
reproduction de l'élite religieuse (29).

Cette élite religieuse assure par ailleurs sa cohésion par une relative
endo- gamie, en prenant des femmes soit dans les familles de notables,
soit dans les familles d'oulémas. Ainsi, certaines peuvent rester
alliées par des séries d'interma- riage, sur plusieurs générations. En
outre, les affinités se renforcent par le biais des liens qui se tissent
entre le maître et son disciple, et entre des étudiants qui poursui-
vent ensemble, pendant de longues années, un cursus ardu. Fait notable,
ces familles religieuses sont souvent transnationales. La famille
al-Sadr, par exemple, qui est originaire du Liban-Sud (le lignage
apparenté y porte aujourd'hui le nom de Sharaf al-Dîn), a une branche en
Irak, et une autre en Iran.

Les dirigeants des mouvements islamiques « révolutionnaires » étaient
issus de ces grandes familles. Ainsi de Muhammad Bâqir al-Sadr, leader
du mouvement islamique irakien, exécuté en 1980 par le régime ; de son
cousin Mûsâ al-Sadr qui, venu d'Iran, s'installa au Liban où il fut
l'artisan du « réveil » de la communauté chiite avant de disparaître
mystérieusement lors d'un voyage en Libye, en 1978 ; ou bien de
Khomeini, le leader de la révolution iranienne (m. 1989). En outre,
c'étaient de grands clercs, des \emph{mujtahid} ; Muhammad Bâqir al-Sadr
et Khomeini furent même des \emph{marja`}. Ils avaient le profil attendu
pour assurer cette fonction, mais tinrent un discours novateur en la
matière. Le premier appela à une rénovation de la fonction de
\emph{marja`,} le second y apporta une redéfinition. Quant à Mûsâ
al-Sadr, il fut parmi les premiers à avoir reçu une double formation,
universitaire, à la faculté de droit de l'université de Téhéran, et
religieuse, dans les écoles de Qom et de Najaf. D'autres oulémas
iraniens, qui participèrent activement à la révolution, tel Muhammad
Beheshti, assassiné à ses débuts, en 1981, avaient ce nouveau profil.

Le modèle classique a cependant perduré. Le grand \emph{marja`} Kho'i,
qui fut le rival de Khomeini, était non seulement, comme lui, issu d'une
famille religieuse, mais il exerça ses fonctions sans vraiment
s'intéresser aux questions qui consti- tuaient le centre des débats dans
bien des cercles cléricaux, comme le rapport du religieux et du
politique, la nécessité de modernisation du discours religieux et des
institutions, etc. Alî Sîstânî, l'actuel grand \emph{marja`} qui se pose
comme son succes- seur, a la même vision, classique, de sa fonction. On
dit même qu'il ne lit pas la presse, ni ne se penche sur les affaires du
monde (30). Le fait est d'autant plus remarquable que Sîstânî est le
\emph{marja`} le plus suivi dans le monde chiite.

Pour autant, d'autres se positionnent différemment, et le revendiquent.
C'est le cas de Muhammad Husayn Fadl Allâh qui, s'il relève du modèle
classique du clerc chiite pour ce qui est de son origine et de sa
formation, a participé au mouvement islamique à partir des années 1960,
et tient aujourd'hui un discours résolument réformiste et moderniste. Il
clame la nécessité de revoir les qualifications néces- saires au
\emph{marja`} : pour lui, celui-ci doit impérativement, aujourd'hui,
avoir une bonne connaissance des affaires mondaines. On voit donc que
deux conceptions du rôle et de la fonction de \emph{marja`} se dégagent
: le modèle classique du \emph{marja`} atten- tiste, apolitique et
traditionnel ; le paradigme du nouveau \emph{marja`,} qui s'implique en
politique, se veut à l'écoute des changements sociaux, et s'engage dans
la réforme des idées et des institutions religieuses.

Le profil des générations montantes de oulémas est par ailleurs en train
de changer. Si les grands clercs en place aujourd'hui, dans le monde
arabe notamment, appartiennent encore aux anciennes familles de science,
et ont été formés à la \emph{hawza}, de nouveaux acteurs religieux
entrent en scène, suite à l'influence de l'Iran post-révolutionnaire.
D'abord, l'Iran a favorisé la vulgarisation du savoir religieux, en
ouvrant largement ses écoles, de Qom et de Machhad, notamment, aux
Iraniens et aux étrangers. Cela provoqua un décloisonnement des milieux
cléricaux, qui ont été investis par des étudiants en sciences
religieuses issus de familles non spéciali- sées dans la cléricature. En
outre, le suivi d'un double cursus, universitaire et religieux, s'est
développé ; l'Iran a favorisé le phénomène en instaurant un système
d'équivalence entre les deux. En Iran, aujourd'hui déjà, bon nombre de
clercs sont aussi des universitaires, parfois titulaires d'un doctorat ;
à l'inverse, on trouve des intellectuels issus du système universitaire
qui, par ailleurs, ont reçu un enseigne- ment religieux. On a donc des
clercs intellectuels et des intellectuels religieux. Ce phénomène
s'étend aux milieux chiites libanais et irakiens, par exemple. Dans la
\emph{hawza} placée sous les auspices de Muhammad Husayn Fadl Allâh, à
Beyrouth, la moitié des étudiants suit en parallèle un cursus
universitaire. Le directeur de l'école lui-même est un clerc, et il est
docteur en philosophie islamique.

Un « désordre organisé »

L'enseignement dispensé dans les écoles religieuses chiites a été
réformé, depuis les premières critiques formulées par des clercs contre
leur manque d'orga- nisation et de pédagogie, à la fin des années 1920.
Des oulémas parvinrent progressivement à introduire des sciences
profanes au cursus religieux, et à rationa- liser le fonctionnement des
cours, par la mise en place de programmes, de classes, d'horaires fixes.
Ce ne fut pas sans difficulté. En effet, le système de la \emph{hawza}
reposait entièrement sur la totale liberté de l'étudiant de choisir un
maître, et sur le lien qui s'établissait entre maître et disciple. Une
organisation sur le modèle des écoles profanes, modernes, allait contre
ces principes. C'est pourquoi la réforme de la \emph{hawza} rencontra
beaucoup de résistance dans les milieux cléricaux. Aujourd'hui encore,
alors que la plupart des écoles fonctionnent sur un mode réformé (plus
ou moins), des oulémas évoquent ces principes comme les garants de la
formation d'une véritable élite religieuse, non sans nostalgie (31).

Un autre principe cardinal, l'indépendance de la \emph{hawza}, est
menacé par tout processus de réforme allant dans le sens de
l'organisation des écoles et de leur bureaucratisation. L'histoire
récente en fournit un exemple frappant, celui de Kulliyyat al-fiqh, un
collège de droit islamique fondé à Najaf, en 1958. Après des années
d'hésitation due à l'opposition de leurs pairs, un groupe de clercs
réfor- mistes, Muhammad Ridâ al-Muzaffar en tête, ouvrit ce collège dont
l'organisation était calquée sur celle des établissements modernes. Bien
plus, il fut reconnu par l'État, puis rattaché à l'université irakienne.
Tant et si bien que, lors de la répres- sion contre les chiites de 1991,
le gouvernement le ferma.

C'est pourquoi, entre la nécessité de moderniser et l'attachement à un
système souple, ne reposant que sur l'autorité d'un \emph{marja`}, les
chiites oscillent sans cesse. Il

en est de même pour la réforme de la \emph{marja`iyya}. En fait,
l'institution, relativement jeune, est en perpétuel développement et
fait l'objet de nombreuses discussions, entre grands clercs. Néanmoins,
elle reste entièrement centrée sur la personne du \emph{marja`} et sur
son charisme ; celui-ci ne pouvant se charger des fonctions adminis-
tratives, il délègue ses proches, surtout ses fils et ses gendres, qui
s'occupent par exemple des investissements financiers ou de la gestion
des fondations. Tout cela fonctionne donc sur un mode patriarcal et
informel. À plusieurs reprises, il a été question de réformer la
\emph{marja`iyya} pour mieux l'organiser et la rationaliser. D'une part,
il s'agissait de revoir les modalités de désignation du \emph{marja`}
afin de les forma- liser, voire d'instaurer une direction collégiale ;
d'autre part, de mettre en place une structure susceptible d'administrer
l'institution et de la pérenniser après la mort du \emph{marja`} (32).
Autrement dit, c'était permettre un processus de routinisation. Or, ces
différentes tentatives restèrent lettres mortes, les chiites n'étant pas
prêts à se doter d'une institution qui serait comparable au Vatican,
avec un chef élu et une hiérarchie pyramidale, même si le modèle est
tentant, pour certains. Un embryon de hiérarchie a été mis en place en
Iran, notamment par la transformation de titres honorifiques, hujjat
al-islâm, ayatollâh, en grades fondés sur le degré de savoir, mais elle
est assez floue.

L'autorité religieuse chiite demeure donc dans le « désordre organisé »,
pour reprendre une expression utilisée par les clercs et les étudiants
de la \emph{hawza} (33). Cela ne signifie pas qu'elle est incontrôlée.
Si la \emph{marja`iyya} obéit toujours à des règles écrites de base,
publiées dans des ouvrages, elle est aussi soumise à des usages et à des
règles morales, plus ou moins implicites, discutées dans les cercles
cléricaux. Ainsi, par exemple, lorsque les fils d'un \emph{marja`}
prennent trop de libertés ou veulent s'arroger des fonctions revenant à
leur père, on rappelle que la \emph{marja`iyya} ne s'hérite pas, et que
les institutions fondées par le \emph{marja`} ne revien- nent pas à ses
proches, mais au \emph{marja`} suivant. Le débat doctrinal ainsi que
l'autocritique du corps des oulémas permettent de réguler l'institution
de la \emph{marja`iyya,} et laissent la porte ouverte à tout changement
d'un système voulu souple, labile, évolutif.