\section{Programme des Frères Musulmans (1936)}

Après avoir étudié les sentiments qui doivent animer, sur le plan
spirituel, la nation renaissante, nous voulons en conclusion, exposer
certaines suggestions d'ordre pratique. Nous n'évoquerons ici que les
têtes de chapitres car nous savons pertinemment que chaque suggestion
nécessite une étude approfondie et l'attention particulière des
spécialistes; nous savons par ailleurs que les besoins de la nation sont
énormes; nous ne croyons pas que la réalisation des besoins et des
desiderata du pays soit chose facile; nous ne pensons pas non plus que
cette réalisation se fera en une journée ou deux. Nous avons conscience
des difficultés que ces besoins vont surmonter. La tâche nécessitera
beaucoup de patience, beaucoup de sagesse et une volonté tenace.

Mais une chose est certaine: la volonté mènera au succès. Une nation
décidée, œuvrant pour la réalisation du droit, arrivera certainement
avec l'aide de Dieu au but qu'elle désire.

Voici les têtes de chapitres de la réforme fondée sur le véritable
esprit de l'Islam:

I. Dans les domaines politique, juridique et administratif

\begin{enumerate}
\def\labelenumi{\arabic{enumi})}
\item
  interdire les partis et orienter les forces de la nation vers 1a
  constitution d'un front unique;
\item
  réformer le droit en vue de l'adapter entièrement à la législation
  islamique;
\item
  renforcer l'armée, augmenter le nombre des groupements de jeunesse;
  inculquer à cette jeunesse, en partant de l'esprit de la guerre
  sainte, la foi et l'abnégation;
\item
  renforcer les liens entre les pays islamiques et plus particulièrement
  entre les pays arabes, étape nécessaire à l'examen sérieux de la
  question du « Khalifat » défunt;
\item
  propager l'esprit islamique au sein de l'administration afin que tous
  les fonctionnaires ressentent le besoin d'appliquer les enseignements
  de l'Islam
\item
  surveiller la conduite personnelle des fonctionnaires car le côté
  privé et la vie administrative de ces fonctionnaires forment un tout
  inséparable;
\item
  avancer les horaires du travail en été et en hiver, en vue de
  faciliter l'accomplissement des obligations religieuses et en vue
  d'interdire toute veillée inutile;
\item
  condamner la corruption et le trafic des influences; ne doivent
  rentrer en ligne de compte que la compétence et le mérite;
\end{enumerate}

\begin{enumerate}
\def\labelenumi{\arabic{enumi})}
\item
   
  le gouvernement agira conformément à la loi et aux enseignements
  islamiques; l'organisation des cérémonies, des réceptions, des
  réunions officielles, le régime des prisons et des hôpitaux ne doivent
  pas être contraires aux prescriptions de l'Islam. Le roulement dans
  les services doit tenir compte des heures de prières;
   
\end{enumerate}

\begin{enumerate}
\def\labelenumi{\arabic{enumi})}
\setcounter{enumi}{8}
\item
  préparer et utiliser les « azharistes », c'est‑à‑dire les diplômés de
  l'Université d'al‑Azhar aux fonctions militaires et administratives.
\end{enumerate}

II. Dans les domaines social et pratique:

\begin{enumerate}
\def\labelenumi{\arabic{enumi})}
\item
   
  le peuple doit respecter les mœurs publiques: celles‑ci doivent faire
  l'objet d'une sollicitude toute particulière --- sanctionner
  sévèrement les atteintes aux mœurs et à la moralité;
   
\item
   
  trouver une solution aux problèmes de la femme, solution susceptible
  de la faire évoluer et de la protéger conformément aux enseignements
  islamiques. Cette très importante question sociale ne doit pas être
  négligée car elle ferait l'objet de polémiques et d'opinions plus ou
  moins tendancieuses et exagérées;
   
\item
   
  se débarrasser de la prostitution clandestine ou publique et
  considérer la fornication comme un crime abject dont les auteurs
  doivent être punis;
   
\item
   
  prohiber tous les jeux de hasard (jeux, loteries, courses, clubs);
   
\item
   
  interdire l'usage de l'alcool et des stupéfiants ---il faut éloigner
  le peuple de leurs conséquences néfastes;
   
\item
   
  interdire les manifestations portant atteinte à la pudeur, éduquer les
  femmes, donner aux institutrices, élèves, étudiantes, doctoresses, une
  éducation soignée...;
   
\item
   
  prévoir les programmes d'enseignement destiné aux filles ---élaborer
  pour elles un programme d'éducation différent du programme destiné aux
  garçons;
   
\item
   
  les étudiants ne doivent pas être mêlés aux étudiantes--- toute
  relation entre un homme et une femme non mariés est consi­dérée comme
  un crime devant être sanctionné;
   
\item
   
  encourager le mariage et la procréation---élaborer une législation
  sauvegardant la famille et trouvant une solution au problème du
  mariage;
   
\item
   
  fermer les salles de bal, interdire la danse, etc.;
   
\item
   
  censurer les pièces de théâtre et les films, choix sévère des pièces
  et des films;
   
\item
   
  contrôler et choisir les chansons;
   
\item
   
  choisir les conférences, chants et sujets devant être diffu­sés,
  utiliser la radiodiffusion en vue de l'éducation nationale
   
\item
   
  confisquer les pièces et livres malsains, et les journaux ayant un
  caractère grotesque et diffusant des frivolités
   
\item
   
  organiser méthodiquement des centres de vacances;
   
\item
   
  modifier les heures d'ouverture et de fermeture des cafés publics,
  surveiller l'activité des clients qui les fréquentent --- orienter ces
  clients vers le bien, interdire de passer beaucoup de temps dans ces
  cafés;
   
\item
   
  utiliser les cafés comme centres de lecture et d'écriture pour les
  analphabètes, solliciter pour cela le concours des mem­bres de
  l'enseignement primaire et des étudiants;
   
\item
   
  combattre les mauvaises habitudes qui nuisent à l'éco­nomie et à la
  morale de la nation, orienter les masses vers les bonnes habitudes et
  les buts louables tels que les réunions de mariages, les orphelinats,
  les naissances, les manifestations et fêtes le gouvernement en doit
  donner l'exemple;
   
\item
   
  faire juger ceux qui enfreignent les lois de l'Islam, qui ne jeûnent
  pas, ne prient pas, qui insultent la religion, etc.;
   
\item
   
  transférer les écoles primaires des villages à la mosquée et y faire
  toutes les améliorations (recrutement de fonctionnaires, questions
  d'hygiène, intérêt porté aux petits enfants qui doivent apprendre la
  prière, initiation des grands à la science);
   
\item
   
  l'enseignement religieux doit constituer la matière essen­tielle
  destinée à être enseignée dans tous les établissements ainsi qu'aux
  facultés;
   
\item
   
  faire apprendre par cœur le Coran dans les écoles libres ---cette
  condition est essentielle pour l'obtention des diplômes de caractère
  religieux et de caractère philosophique --- dans toute école, on doit
  faire apprendre aux élèves une partie du Coran;
   
\item
   
  élaborer une politique destinée à élever le niveau de l'enseignement,
  à unifier les différentes branches de l'enseignement, à rapprocher les
  différentes branches de culture --- la morale et l'instruction civique
  doivent être enseignées en premier lieu ;
   
\item
   
  intérêt porté à l'enseignement de la langue arabe à tous les échelons
  --- priorité absolue accordée à la langue arabe sur les autres langues
  étrangères (enseignement primaire);
   
\item
   
  étudier l'histoire de l'Islam, celle de la nation, et celle de la
  civilisation musulmane;
   
\item
   
  étudier le meilleur moyen permettant aux gens de s'ha­biller
  progressivement d'une manière identique;
   
\item
   
  combattre les habitudes étrangères (domaine du voca­bulaire,
  habitudes, vêtements, nurses, nourrices, etc.) égyptiani­sation de
  tout (on trouve ces habitudes étrangères dans les classes aisées de la
  société);
   
\item
   
  orienter le journalisme vers le bien, encourager les écri­vains et les
  auteurs qui doivent étudier des sujets spécifiquement musulmans et
  orientaux;
   
\item
   
  sauvegarder la santé publique par tous les moyens de propagande ---
  augmentation du nombre des hôpitaux --- des médecins, des centres de
  santé ambulants;
   
\item
   
  porter une attention particulière aux questions et affaires du village
  (organisation, hygiène, filtrage des eaux, éducation, repos,
  moralité).
   
\end{enumerate}

 
III. Domaine économique:
 

\begin{enumerate}
\def\labelenumi{\arabic{enumi})}
\item
   
  organisation de la « Zakat », conformément à la législa­tion
  islamique, utilisation des fonds de la Zakat dans des buts de
  bienfaisance tels que les asiles d'indigents, de pauvres, orphelins,
  on doit utiliser également les ressources de la Zakat pour le
  renfor­cement de l'armée;
   
\item
   
  interdiction de pratiquer l'usure, orienter les banques vers cette
  interdiction, le gouvernement doit donner l'exemple en abandonnant 1'
  « intérêt » fixé par les banques du prêt et du prêt industriel, etc.;
   
\item
   
  encourager et augmenter le nombre des institutions éco­nomiques, y
  employer les chômeurs, utiliser au profit de la nation les biens que
  possèdent les étrangers dans ces institutions;
   
\item
   
  accorder certaines protections aux ouvriers contre les sociétés
  possédant des monopoles d'exploitation, obliger ces sociétés à obéir à
  la loi, le public doit profiter de tout bénéfice;
   
\item
   
  amélioration du sort des petits fonctionnaires et augmen­tation de
  leurs traitements, abaissement du traitement des hauts fonctionnaires;
   
\item
   
  réduire le nombre de postes destinés aux fonctionnaires, se contenter
  des emplois indispensables au pays, la répartition du travail parmi
  les fonctionnaires doit être faite d'une façon équitable;
   
\item
   
  encourager les travaux agricoles et industriels, amélio­ration de la
  situation du fellah et de l'artisan (dans le domaine de la
  production);
   
\item
   
  attacher une attention particulière aux besoins techniques et sociaux
  des ouvriers, élever leur niveau de vie et améliorer leur sort
   
\item
   
  exploitation de certaines ressources naturelles (terres non labourées,
  mines ignorées, etc.);
   
\item
   
  priorité accordée aux projets dont la réalisation est nécessaire au
  pays.
   
\end{enumerate}

Telle est la profession de foi que nous vous présentons: nous sommes
décidés à obéir et à remettre tout ce que nous possédons à n'importe
quelle organisation ou gouvernement décidé à conduire la nation
musulmane au progrès.

Nous répondrons à tout appel, nous serons des volontaires en toute
circonstance. Nous aurons ainsi accompli notre mission et dit ce que
nous avions à dire.

La religion appartient à Dieu, à Son Prophète, à Son Livre, aux Imams et
à la Communauté. Dieu nous suffit.

Extrait d'une brochure de 1936, \textbf{Vers la lumière},\\
traduite de l'arabe par J. Marel et parue dans \textbf{Orient}, N° 4,
1957, p. 37-62\emph{.}




\section{Les Frères musulmans égyptiens : pour une critique des vœux pieux}

 
\mn{Tewfik Aclimandos} 




Intégrer les islamistes modérés est un impératif si l'on souhaite une
transition démocratique. La modération est définie par certains
critères, et des experts affirment que certains islamistes les
remplissent, dont les Frères musulmans. Cet article tente de démontrer
que pour ce qui concerne ces derniers, le cas est pour le moins
discutable. À travers l'étude de leur discours, l'analyse de leurs modes
et dynamiques de socialisation, de leurs pratiques et leurs idéologies,
on peut souligner la dimension sectaire, et montrer que la séparation --
néces- saire à une véritable démocratisation -- entre prédication et
action politique est bien loin d'être réalisée. On critique finalement
les paradigmes optimistes mobilisant une ruse de la raison.

\textbf{U}n jihadiste hostile aux Frères musulmans, Hânî al Sibâ'i,
affirmait un jour que « les capitales occidentales rêvent d'islamistes
ayant renoncé à la charia et reconnaissant Israël 1 ». Il ajoutait que
les Frères, étant opportunistes, feraient

volontiers ces concessions pour accéder au pouvoir. De nombreux
chercheurs et experts semblent avoir le même rêve. Reste à savoir si,
opportunistes ou non, les Frères sont prêts à faire des concessions. La
solution idéale à la « crise » des régimes arabes, nous expliquent un
nombre croissant de « transitologues 2 » et
de spécialistes de l'islamisme 3, serait l'intégration dans d'éventuels
processus démocratiques de formations islamistes ayant accepté le cadre
de l'État natio- nal et les règles du jeu démocratique, renoncé à la
violence et à l'application des dispositions les plus controversées de
la charia, et reconnu l'existence de l'État d'Israël. Mais cela est-il
possible ?

Sans prétendre formuler un modèle valable pour toutes les configurations
islamistes, je tenterai ici de décrire ce que je sais des Frères
musulmans égyptiens. Je voudrais tester les propositions énoncées plus
haut, fruits de la rencontre de la transitologie, de l'agenda de la
première administration Bush et d'un certain type d'analyse de
l'islamisme, qui s'agrègent désormais dans une sorte de « savoir
conventionnel » s'exprimant dans la presse, les \emph{policy briefs} et
les revues scientifiques, \emph{Foreign Affairs} n'étant pas la moindre.
Ce « savoir conventionnel » se retrouve dans certains cercles
intellectuels de gauche qui prônent un dialogue des civilisations ou un
respect inconditionnel, excluant toute critique de « la religion des
dominés », et érigent les islamistes en représentants de l'islam (on
pense à l'extrême gauche, ou à des intellectuels proches de la revue
\emph{Esprit}). Ce savoir doit être précisé, nuancé ou réfuté. Il faut
auparavant poser des questions élémentaires de méthode, relatives aux
matériaux et aux données disponibles, et réfléchir à leur mode d'emploi
et à la construction de « modèles explicatifs » de la confrérie. Je
résumerai à cette fin ce que l'on sait aujourd'hui des Frères, de leur
organisation, de leur idéologie, puis j'exposerai mes vues sur le débat
en cours. Mes assertions principales sont simples, mais oubliées : avant
de spéculer sur les évolutions possibles, il convient de tenter une
phénoménologie de la confrérie, d'autant plus que cette dernière est un
« donné massif » et donc difficile à manier par décret. Celle-ci semble
induire une « socialisation sectaire », au sens sociolo- gique du terme,
impliquant une socialisation en interne, combinée à une mise à distance
et une dépréciation du réel, que l'on subjectivise en une construc- tion
manichéenne. Il convient aussi d'étudier les textes doctrinaux de la
confré- rie, et la littérature qu'elle produit. Si on ne peut deviner ce
que son lectorat en pense, l'on peut au moins affirmer qu'elle le
marque, et qu'elle reflète au moins la pensée réelle des auteurs.
 

\hypertarget{discours-des-fruxe8res-les-positions-publiques}{%
\section{Discours des Frères : les positions
publiques}\label{discours-des-fruxe8res-les-positions-publiques}}

 
Les analyses sur les Frères aujourd'hui se fondent essentiellement sur
l'étude des discours produits par la confrérie et ses cadres, en plus de
la rituelle mention de son travail social et d'indications assez
sommaires sur le profil sociologique et la supposée psychologie des
militants 4.
Mais quels discours étudie-t-on ? Le plus souvent le programme des
Frères, leur plateforme électorale et les déclarations de ceux qui sont
omniprésents dans les divers espaces médiatiques. L'exégèse subséquente
tend à se féliciter de l'acceptation progressive de la démocratie, à
souligner la « normalité » des Frères -- une force politique « comme les
autres » --, à relever toutes sortes d'in- dices montrant des « progrès
» ou, au contraire, des « manques », des « lacunes » et des «
régressions » 5, expliqués par la présence aux postes clés de théocrates
très vilains et très âgés 6, par la tentative du sommet de préserver la
cohésion interne de la formation et de tenir compte de ce que la base
peut accepter 7, ou par le fait que les régimes arabes ont les
islamistes qu'ils méritent (plutôt que l'inverse) 8.
« d'optimisme » qui parie sur une « ruse de la raison » donnant à
l'action islamiste un « sens » dont les acteurs sont ou ne sont pas
conscients, et qui surestime la portée de toutes sortes de phénomènes
décrétés « nouveaux ». Cette remarque cible deux excellents auteurs,
Patrick Haenni et Raymond W. Baker (dans \emph{Islam without fear. Egypt
and the New Islamists}, Cambridge, Harvard University Press, 2003, ou
encore dans « Invidious comparisons : realism, postmodern Globalism and
centrist Islamic movements in Egypt », \emph{in} J. L. Esposito (ed.),
\emph{Political Islam : Revolution, Radicalism, or Reform ?}, Le Caire,
AUC Press, 1997. Ces deux auteurs ont tendance à ne pas voir qu'une
culture arabe a existé au XXe siècle et que sur ce plan les islamistes
représentent, pour l'instant, une régression. Enfin, les discours
scientifiques sur les islamistes ont en commun d'expliquer l'émergence
de la mouvance par une crise grave de leur société, que l'on décrit ou
construit différemment selon les cas. La revue de littérature proposée
par Haydar Ibrâhîm `Ali, dans son ouvrage \emph{Les courants islamiques
et le problème de la démocratie} {[}en arabe{]}, Beyrouth, Centre
d'études pour l'unité arabe, 1996, fournit des analyses précieuses.
En se contentant d'analyser les positions publiques, on risque la
méprise, surtout sans travail de contextualisation -- or l'entreprise
est difficile car la confrérie, illégale, garde ses secrets. J'affirme
cela en rappelant que l'actuel programme de la confrérie renforce les
thèses des sceptiques (dont je suis). J'admets volontiers qu'il faille
prendre les plateformes et les déclarations des principaux dirigeants,
comme étant \emph{souvent} un indicateur de ce que la confrérie veut
dire (ou faire croire) à ses militants et ses interlocuteurs, bien
disposés ou hostiles, ou de ses priorités du moment, ou encore qu'il
faille les considérer comme l'enjeu et le résultat de luttes intestines,
débouchant soit sur la victoire d'une tendance, soit sur une motion de
synthèse qui perd de vue le destinataire du message. Il est plus délicat
d'y voir une preuve de ce que la confrérie pense, de ses conceptions, à
moins d'estimer qu'elle change sans arrêt d'avis sur toutes sortes de
questions, ou a sur chacune une trentaine d'opinions, ou encore que les
rapports de force en son sein sont très fluctuants, alors même que les
démissions et les modifications des équipes dirigeantes sont rares.
Admettons -- ce que je ne fais pas -- que ces textes soient des
programmes de gouvernement qui engagent ceux qui les émettent : que
fait-on alors quand on y est confronté à la « quadrature du cercle 9 »
ou à des propositions contra- dictoires (par exemple sur la citoyenneté,
le statut des minorités, les libertés publiques, la femme, la
souveraineté) ? La réponse est désolante : chacun retient ce qui
l'arrange et minimise ce qui le dérange (cela vaut aussi pour moi). Cela
dit, on peut suggérer une clé qu'impose le bon sens, sinon en général,
du moins dans le contexte égyptien (où les élections ne se décident pas
sur des programmes et où la revendication démocratique arrive très loin
derrière le chômage, la qualité et le coût de la vie, la corruption dans
les préoccupations de la population 10) : si la confrérie formule des
propositions qui déçoivent tous ses interlocuteurs des microcosmes
politiques et intellectuels, et qui risquent de porter atteinte à son
image, il y a lieu de penser qu'elle -- ou au moins les rédacteurs et
ceux qui ont entériné les textes -- est sincère, puisqu'on ne peut
supposer, sans insulter leur intelligence, qu'ils s'attendaient à autre
chose.

L'étude des positions publiques pose plusieurs problèmes. Expédions le

premier : qui engage la confrérie ? Ici on donne une réponse simple : le
Guide et les deux Vice-Guides. Mais cette réponse n'élimine pas le
problème des volte-face. Le second problème est relatif à la place des
contextes : parler au Parlement, ce n'est pas s'adresser à un
journaliste. Les contextes sont certes une variable significative, mais
ils ne doivent pas forclore toute réflexion sur l'essence éventuelle du
mouvement 11. De plus, les contextes ne sont jamais \emph{connaissables}
avec certitude par un chercheur isolé. La violence islamiste ne peut
être traitée comme une simple « réponse » à une répression étatique :
elle a souvent précédé celle des régimes, notamment ceux de Farouk et de
Sadate,

et l'œuvre de Qutb est certes une réaction à la sauvagerie des services
de Nasser, mais elle est aussi une « systématisation », une mise en
forme d'intuitions ou de questions qui précèdent l'avènement du raïs. En
revanche, il est exact que l'impossibilité d'une alternance pacifique
renforce les partisans de la contestation violente, et il est vrai que
les Frères n'ont pas eu recours au terrorisme depuis une trentaine
d'années. Les contextes peuvent servir de circonstances tantôt
atténuantes, tantôt aggravantes : les chercheurs (dont l'auteur du
présent texte) se souviennent souvent des contextes quand ils servent
leur propos. Le troisième problème, enfin, est quantitatif : le suivi du
dossier Frères (recension \emph{et} traitement de l'information) est
impossible : les prises de parole sont trop nombreuses ! Ces deux
derniers problèmes posent celui de l'extrapolation des données\emph{.}
Rien qu'en se cantonnant à l'étude des décla- rations publiques, la
nécessité d'une sélection rend inévitable l'influence de schémas
hypothético-déductifs présupposés, soulignée par Max Weber 12. Ce rappel
n'implique pas que tous ces schémas se valent -- mais je préfère ceux
que j'utilise 13.

Cela dit, quelques enseignements peuvent être tirés de ce travail : les
voix

émises par les Frères sont plurielles\emph{,} comme le sont les signes
diacritiques les distinguant 14. Néanmoins, sur quelques questions, les
versions des Frères se ressemblent toutes : non à la théocratie, oui à
la mystérieuse et mal définie \emph{marja'iyya} religieuse 15. La
description des « circonstances des arrestations de militants » est
toujours la même, mentionne toujours le thème d'un voisinage terrorisé
la nuit ou à l'aube par la police, hostile à celle-ci, témoin de
l'injustice \emph{(zulm)} qui frappe le juste. L'attachement aux
principes est souvent déclamé,
« norme contraignante », ou encore « instance décisive » ou « source de
la souveraineté ».

mais leur translation en actes concrets est problématique 16. Nombreux
sont les Frères maîtrisant le « parler démocratique » ou le « parler
sécularisé ». Mais peut-on en déduire qu'ils sont sécularisés ou
convertis à la démocratie ? L'exemple le plus simple est l'invocation du
multiculturalisme ou de la liberté religieuse pour justifier le refus de
l'accès des non-musulmans à la magistrature suprême : cette fonction
étant par plusieurs aspects de nature religieuse (incluant la
supervision de l'application de la charia), y accepter un non- musulman
serait agresser les convictions religieuses de ce dernier !
 

\hypertarget{discours-des-fruxe8res-slogans-et-ouvrages-doctrinaux}{%
\section{Discours des frères : slogans et ouvrages
doctrinaux}\label{discours-des-fruxe8res-slogans-et-ouvrages-doctrinaux}}

Dans le champ du discours, deux types de textes semblent plus
significatifs que les déclarations. Les slogans, avec leur force
cognitive, leur symbolisme, ce qu'ils engagent, le dit et le non-dit,
comptent autant voire davantage. Le slogan des Frères, c'est « L'islam
est la solution », ce qui implique au minimum que le régime en place n'a
pas recours à ce remède. Les textes doctrinaux aussi, rédigés par les
membres du Bureau de guidance ou par les théoriciens de la confrérie, ou
encore les enseignements que l'on inculque et les textes que l'on fait
lire aux militants, sont importants. Pour dire les choses brutalement,
en allant dans les librairies islamistes et en achetant les livres qu'on
y propose, surtout ceux dont les auteurs sont des Frères, on a une
vision très différente de l'image que les acteurs de la confrérie
cherchent à donner. Nourri par les témoignages des dissidents, par les
accusations que porte la sécurité de l'État, par certaines pratiques des
Frères, par les multiples dérapages dans la presse de cadres moins
rompus que d'autres au \emph{« talking nice »}, par les mémoires et
souvenirs de certains acteurs, le tableau qui se dégage devrait inciter
à la prudence.

Peu de chercheurs tentent la lecture systématique des productions intel-

lectuelles des Frères. Il faut dire que ces productions sont plurielles,
même si, hormis peut-être Sayyid Qutb, la confrérie n'a pas de grands
penseurs. Ses quelques auteurs sérieux, comme par exemple Farîd `Abd al
Khâliq, sont-ils représentatifs 17 ? Ses « classiques », comme par
exemple les textes d'al Bannâ, n'ont jamais été reniés, mais ne
reflètent plus la situation actuelle. L'historien Jâbir al Ansârî 18
estimait que la pensée arabe est traversée depuis des siècles par trois
démarches : la principale, centrale et centriste, est la \emph{«
tawfîqiyya »}, le concordisme, qui présente deux caractéristiques :
d'une part, sa grande ouver- ture aux apports des autres traditions et
civilisations, qu'elle intègre et absorbe dans son élaboration de la
tradition islamique ; d'autre part, un souci de concilier les contraires
(foi/raison, science/religion, modernité/tradition\ldots), d'apaiser le
conflit en proposant une synthèse. La \emph{« salafiyya »,} plus
importante

depuis la guerre de 1967, est la démarche des zélotes et des puristes,
et correspond peu ou prou à l'islam des marches et du désert, plus
attachée à la sauvegarde d'une pureté mythique, ne cherchant son
inspiration que dans le Coran, la Tradition ou l'exemple des pieux
ancêtres, méfiante à l'égard du rationalisme, des apports étrangers et
de l'herméneutique. La troisième démarche est l'hétérodoxie. Pour sa
part, la confrérie a toujours oscillé entre les deux premiers pôles. Le
second a toutefois été prédominant pendant toute l'histoire des Frères,
mais l'on voit, depuis dix ans, une montée en puissance de la
\emph{tawfîqiyya}, qui se traduit par l'évolution des programmes. Mais
cette \emph{tawfîqiyya} demeure boiteuse et superficielle, et, pour
autant qu'on puisse en juger, salafistes et qutbiens dominent encore 19.
Surtout, il y a toujours eu des Frères partageant la posture
\emph{tawfîqiyya} et que leur période d'influence maximale a sans doute
été le début des années 1950.

Mais il faudrait aussi regarder les textes non politiques de la
confrérie, par exemple ceux relatifs au jihad et à la prédication. Le
problème central réside peut-être dans la ou les théories du jihad, car
l'utilisation extensive de ce concept par les Frères est problématique,
puisqu'elle surévalue, gonfle et privilégie la dimension conflictuelle
et polémologique du politique 20. Mais le problème premier est celui de
la prédication. Elle est radicalement contraire à, voire incompatible
avec l'acceptation des règles du jeu démocratique, puisqu'elle suppose
l'existence d'une vérité détenue par les Frères, d'un camp de « justes
», une avant-garde de censeurs chargée de ramener dans le droit chemin
les mauvais musulmans, et de convertir (ou soumettre) les autres.

Le projet de refonte de l'individu est un premier pas, précédant la
refonte de la famille, de la société, de l'État, du monde. Elle est un
discours sur la société, et aussi, sur un ton souvent infâme, sur
l'autre, sur les autres sociétés et civi- lisations, tenues pour
inférieures, « sales », ennemies de Dieu, à combattre 21. Le rapport au
temps ainsi constitué n'est pas celui de la transition démocratique,
mais de la création de la Cité parfaite, de la Théocratie achevée 22. La
prédication ne reconnaît de légitimité qu'à un point de vue, qu'à un
mode de vie, et recom- mande leur diffusion et la disparition des
autres.

Puisque prédication et discours politique cohabitent chez les Frères, il
faut savoir lequel « sert » l'autre. Cette cohabitation légitime le
reproche de double langage : les programmes politiques des Frères ne
sont-ils qu'une « technique » propre à un champ politique donné, son
recours étant justifié par la nécessité de « faire exister » la «
prédication » dans toutes les couches et institutions de la société 23 ?
L'on peut, au contraire, faire de la prédication une instrumenta-
lisation du symbolique destinée à servir, à mobiliser autour une
politique convaincue par ou résignée à la démocratie. Pour ma part, je
n'y crois pas : le programme Frère, l'importance des sections « fatwa »
et « éducation », les messages qu'elles diffusent, me font penser que la
prédication et sa vision du monde priment pour le moment.

Reste à savoir si ces textes façonnent ou non les vues de la direction,
des cadres et des militants frères. Pour le savoir, on pourrait étudier
les livres que les Frères font lire à leurs militants. L'un des
meilleurs spécialistes du sujet, Tammâm, qui n'a pas encore publié les
résultats de ses travaux, m'a donné quelques indications 24 : les
penseurs de la \emph{tawfiqîyya}, notamment ceux qui ont tenté de penser
la démocratie et ses rapports avec l'islam, ne sont pas au programme de
lecture « Frère » ; l'enseignement demeure conforme au credo classique
des Frères, hostile à la démocratie, estimant que la souveraineté des
hommes est sacrilège, car attribut de Dieu, prônant une théocratie,
malgré les dénégations, et privilégiant le jihad ; les deux Qutb, Sayyid
et Muhammad, figurent dans la liste. Selon une autre source 25, les
militants lisent beaucoup Yûsuf al Qaradâwî, Jum'a Amîn `Abd al `Azîz
26, `Ali `Abd al Halîm Mahmûd 27 et Jamâl Sultân. Sans avoir lu tous les
textes de ces auteurs, je n'y ai presque jamais vu (sauf, dans une
certaine mesure, chez Qaradâwî) de traitement des problèmes politiques
qui se posent aujourd'hui au mouvement, notamment son attitude vis-à-vis
de la démocratie et du pluralisme. Certains minimisent l'impact de ces
lectures sur les militants au motif qu'un lecteur doté de bon sens
s'apercevra qu'elles ne peuvent répondre aux questions et défis que pose
la société moderne. Cette assertion est discutable -- Qutb, par exemple,
combine romantisme et esprit systématique, et son langage et son
message, d'un anti- modernisme très moderne, peuvent exercer la
séduction que l'on attribuait sous

d'autres cieux au marxisme. La question de la réception des textes
enseignés aux militants reste posée, tout comme celle du statut de
l'idéologie au sein de la confrérie (est-elle idéocratique ?), de son
efficience et de son monolithisme. Les Frères participent en tout cas de
l'émergence d'une société de masse,

et sont le « moyen » de l'accès à l'espace public et politique de
couches qui en étaient exclues. Mais les critiques formulées à l'égard
de la « culture de masse » peuvent et doivent être formulées à l'égard
des Frères, d'autant plus qu'ils ont combattu avec férocité toutes les
formes de culture qui ne leur convenaient pas 28, d'avoir très peu --
pas du tout reflète mieux ma pensée -- contribué à la « grande culture
égyptienne » du XXe siècle, et d'avoir une propension à la dilution du
sens de certains mots (notamment « théocratie », « gouvernement civil »
et « citoyenneté »). L'on peut soutenir qu'ils contribuent à perpétuer
la confusion entre religion et politique : limitations à la pensée et à
la liberté d'expression, inégalité entre les membres de l'État-nation,
emploi de la religion dans les questions temporelles, démagogie,
divisions communautaires\ldots{}


\hypertarget{pratiques-des-fruxe8res}{%
\section{Pratiques des Frères}\label{pratiques-des-fruxe8res}}

 
Les « pratiques » des Frères ne sont pas moins difficiles à étudier. Par
exemple, si l'accumulation des exemples individuels permet, sur certains
sujets, de dégager des tendances, sur d'autres on peut soutenir tout et
son contraire. À l'exception de Marie Vannetzel, peu de chercheurs se
sont attelés à la production des monographies indispensables 29. Son
principal argument est que les Frères fonctionnent peu ou prou comme le
Parti national démo- cratique (PND), s'appuyant sur des notables dont le
profil et les ressources changent avec les époques et les lieux,
capables d'assumer des prestations à un public local. Elle souligne
aussi l'importance des réseaux de travail social des Frères. Il est
indubitable qu'ils pallient ainsi les déficiences d'un État qui n'est
plus -- s'il l'a jamais été --, providentiel et protègent les plus
pauvres. Ils contribuent aussi à la construction et à la légitimation
d'un \emph{ethos} islamique et à l'islamisation ou à la réislamisation
d'espaces où rien n'a été fait pour ancrer l'\emph{ethos} démocratique.
L'on a parfois l'impression que les espaces et réseaux de socialisations
frères sont « totaux », qu'un Frère peut (ou doit, selon d'autres
témoignages) organiser sa vie en n'ayant que des partenaires ou
interlocuteurs Frères ; il peut être logé, employé, soigné par des
Frères ; il peut épouser la sœur d'un Frère ; ses interactions
financières peuvent n'avoir lieu qu'avec des Frères. Ceci, bien sûr, ne
veut pas dire que les Frères sont absents des institutions publiques :
au contraire, ils sont très nombreux dans les universités, les
syndicats, les clubs sportifs, qu'ils « islamisent » progressivement.

Les données sur le fonctionnement interne de la confrérie sont
nombreuses,
mais partielles : on peut inférer à partir du règlement interne, ou
tenter de se fonder sur les souvenirs des acteurs (mais ceux-ci couvrent
en général les périodes de la monarchie et de Nasser, et les choses ont
pu changer) ou sur les articles de presse (mais il est malaisé de savoir
quelles sont les sources des auteurs -- souvent la sécurité d'État et
les dissidents). De nombreux journalistes citent aussi des « confidences
» de sources internes au sein de la confrérie -- ceci n'est pas
impossible, que la confrérie cherche à renforcer la crédibilité de
certains spécialistes qui lui sont acquis, ou que des acteurs organisent
des fuites dans le cadre de luttes internes.
 

\hypertarget{comment-devient-on-fruxe8re}{%
\section{Comment devient-on Frère ?}\label{comment-devient-on-fruxe8re}}

 
Dans un journal de gauche, \emph{a priori} peu hostile à la confrérie,
on trouve un article sur les techniques de recrutement de la confrérie
30. C'est elle qui les

« choisit » et non l'inverse : les militants identifient des cibles
potentielles (souvent à l'université ou dans sa cité, car les étudiants
provinciaux s'éloignant

pour la première fois de leurs famille sont des activistes potentiels),
et s'en rapprochent ; une enquête est réalisée sur chaque cible, sur ses
rapports avec sa famille ; puis les militants essaient d'éveiller la
sensibilité religieuse de leur recrue potentielle, de l'encourager à
faire ses prières, à renoncer au tabac, à aller à la mosquée et à lire
le Coran. À ce stade, on ne parle pas de politique, sauf si la cible est
clairement intéressée. Si la personne plaît, on lui présente d'autres
Frères, on fait du sport avec elle. C'est plus tard que l'on explique
que l'islam est une religion « totale » et qu'il faut « agir »,
collectivement. On confie à la recrue des tâches, on la fait jouer à des
sports collectifs, on la jauge. Puis on la convainc de travailler avec
les Frères. L'impétrant assiste à des conférences, étudie la
configuration islamiste, les divers mouvements ; on répond à ses
questions. Mais en général, il est d'ores et déjà décidé : il est devenu
un \emph{« muhibb »} (une personne « qui aime la confrérie » et sur
laquelle on peut compter). Il colle des affiches, assiste à des réunions
avec d'autres \emph{« muhibbs »}, fait du prosélytisme -- mais il n'est
pas encore « Frère ». Il assiste à un cours à la mosquée, où on lui
explique l'entraide et la solidarité dans l'obéissance. Cette période
est conçue pour « parfaire l'éducation », pour vérifier l'intério-
risation des conceptions du mouvement, le respect des prières et des
autres pratiques cultuelles. Cette étape dure de un à deux ans, puis
l'on passe de

\emph{« muhibb »} à \emph{« mu'ayyid »} (« qui appuie ») : techniquement
on n'est pas encore

membre\ldots{} pendant encore 18 mois. Pour la cooptation finale, la
confrérie sollicite l'avis de plusieurs Frères. Selon l'article, les
militants frères sont plutôt d'accord sur ce qu'ils ne veulent pas, et
adhèrent à la règle selon laquelle « les seules constantes sont les
points sur lesquels il y a eu consensus d'ulémas ». Donc on peut trouver
des Frères très libéraux, et d'autres qui ne le sont pas du tout.

La presse égyptienne publie quelquefois des témoignages d'anciens mili-
tants, qu'il convient de manier avec prudence, mais qui confirment,
précisent ou nuancent le tableau 31. Ils posent quelques problèmes (les
indications qu'ils donnent sur le fonctionnement et les règlements
internes ne sont pas toujours compatibles entre elles), mais dans
l'ensemble ils ont des points communs. Le futur militant participe à des
activités religieuses, sans savoir au début qu'il « roule » pour les
Frères. Il néglige sa vie privée, ses liens familiaux, ses amis, il est
totalement immergé dans sa vie frère. Il s'y investit totalement. Selon
les déçus de la confrérie, les Frères apprennent à leurs cadres à
refuser l'autre
et à être extrémistes, même s'ils prétendent le contraire. La confrérie
inculque l'idée qu'elle et ses membres ont toujours raison et que le
reste de l'humanité a toujours tort. On fait tout pour empêcher les
autres courants religieux de prendre la parole dans les amphithéâtres.
En plus les Frères « limitent », ou

« soumettent » leur prédication à des agendas précis et des objectifs
chiffrés à atteindre, quels qu'en soient les moyens, ce qui fait de la
prédication un processus bureaucratique et routinier, ciblant la
quantité au détriment de la qualité.

Les Frères musulmans ont bâti leurs succès en combinant prédication et
action politique, en prônant une idéologie moniste, totalisante,
régissant les différents aspects de la vie individuelle et de la
société. Séparer prédication et politique semble impossible -- ce serait
pour les Frères renoncer à une recette qui gagne et qui constitue, de
surcroît, l'identité du mouvement. On imagine mal l'ampleur du
changement nécessaire : il faudrait changer de culture mili- tante, de
rapport à l'environnement, de critères de sélection des candidats et de
promotion des militants. On devrait changer d'\emph{ethos} -- cela ne se
fait pas par décret, et l'on ne prend d'ailleurs ce type de décrets que
s'il y a crise. Or les Frères prospèrent.
 

\hypertarget{effectifs-et-organigramme}{%
\section{Effectifs et organigramme}\label{effectifs-et-organigramme}}

 
La confrérie est discrète sur sa taille et ses structures. Elle refuse
de chiffrer le nombre de ses militants\emph{.} Les deux dernières années
ont été, selon les responsables sécuritaires, celles d'une fantastique
progression : plus d'un million et demi de personnes paieraient des
cotisations.

Détailler l'organigramme de la confrérie dépasse le cadre de cet article
32 : il nous importe toutefois de souligner le poids prépondérant du
Bureau de guidance, l'instance dirigeante, qui dispose de vastes
prérogatives -- d'autant plus que l'échelon « directement » inférieur,
le \emph{majlis shûra} (assemblée consul- tative), n'a pu se réunir
depuis 1995, à cause des harcèlements policiers. Le Bureau compte une
quinzaine de membres, dont certains, trop âgés ou malades, n'assistent
plus aux réunions, d'autres sont emprisonnés. Le plus connu est sans
doute Abû al-Futûh, considéré comme un démocrate, qui est esseulé. Les
hommes forts sont le Guide suprême `Akif, le second Vice-Guide al
Shâtir, `Izzat et Ghazlân. L'autre Vice-Guide, Muhammad Habîb, semble
avoir moins de relais que le quatuor, mais il cherche probablement à
exploiter le séjour prolongé d'al Shâtir en prison pour placer ses
hommes.

Il convient de relever aussi l'importance de la section « de prédication
et de fatwa »\emph{,} qui gère le quadrillage des mosquées et émet des
avis sur la licéité de telle ou telle mesure. Elle est supervisée par le
mufti de l'organisation,

al Khatîb, membre du Bureau de guidance : consultée sur tout, elle norme
le comportement frère, individuel ou collectif 33. Elle semble plus
importante que le « comité politique », dirigé par al `Iryân, qui gère
les dossiers politiques. Ce dernier est aussi moins important que le «
comité administratif », dirigé par `Izzat, qui transmet -- entre autres
-- les instructions du Bureau de gui- dance aux sections régionales, et
coordonne l'ensemble.

Les Frères ont la réputation d'être une force disciplinée, centralisée
autour du Bureau de guidance. L'obéissance des députés connus pour lui
être affiliés conforte cette impression. Mais il convient de relativiser
: du fait des contraintes de la clandestinité (même relative) et de la «
masse » de la confrérie, il convient de ne pas écarter les témoignages
affirmant que, sur plusieurs sujets, dans plusieurs dossiers, les
sections régionales et les militants disposent d'une grande latitude. On
peut aussi croire al Sharnûbi (le responsable du site web des Frères,
proche d'al Shâtir) quand il affirme :

« Les techniques médiatiques que nous utilisons sont fondées sur le
dialogue direct avec les gens, sur l'assistance, sur la participation à
leurs joies et à leurs malheurs, et ceci personne ne peut l'interdire,
parce que cela a lieu dans toutes les rues égyptiennes, et que c'est
accom- pli par des personnes qui n'en réfèrent pas, pour ce faire, à la
hiérarchie. Ce n'est pas vrai que le Bureau de guidance peut faire
bouger tous les membres de la confrérie par télé- commande 34.»
 

\hypertarget{liduxe9ologie-fruxe8re-elle-se-porte-bien-ne-vous-en-duxe9plaise}{%
\section{L'idéologie frère : elle se porte bien, ne vous en
déplaise}\label{liduxe9ologie-fruxe8re-elle-se-porte-bien-ne-vous-en-duxe9plaise}}

 
On a vu que, selon Tammâm, la doctrine Frère est essentiellement
qutbienne 35. Mais il ajoute que cette idéologie s'est délitée au
contact de la réalité, notamment suite à la décision de renoncer à la
violence, alors que le jihad armé est la conséquence logique de la
lecture qutbienne du monde.
« idéocratie ». En effet, la fatwa est un va-et-vient entre le réel et
la norme, et implique une prise de conscience des problèmes posés par le
premier.
Ce recul du qutbisme s'explique aussi par le souci de brasser large :
les Frères ont décidé que leur formation pouvait « accueillir ce que
l'islam peut accueillir ». Ils comptent donc en leur sein de nombreuses
sensibilités.

Classer les membres du Bureau de guidance ou les courants traversant la
confrérie se fait d'ordinaire en construisant des oppositions binaires :
entre

« démocrates sincères » et « fondamentalistes dogmatiques », entre «
vieille(s) » et « nouvelle(s) » génération(s), entre ceux qui souhaitent
la création d'un parti politique et la séparation entre politique et
prédication, et ceux qui sont attachés à la forme « confrérique » et qui
estiment que la combinaison

« prédication-politique » fonctionne et doit être maintenue. Quelques
remarques s'imposent ici : d'abord, la « nouvelle génération » n'est pas
homo- gène ; ensuite, il y a, au sein de la confrérie, un relatif accord
sur la nécessité d'une démocratisation, comprise comme l'organisation
d'élections libres concurrentielles (comme le montrent les pratiques des
Frères dans les syndicats qu'ils contrôlent) et l'arrêt des harcèlements
policiers. Il est en revanche permis de douter de la conversion de
certains (le Guide suprême notam- ment) au principe de la souveraineté
populaire, ou au respect des libertés fondamentales -- pour ne
mentionner que les points sur lesquels le programme des Frères et des
dérapages, verbaux ou non, ont révélé des réserves et des arrières
pensées. Enfin, ces oppositions ne valent pas ou plus pour le Bureau de
guidance, qui est moins divisé qu'on ne le dit : Abû al-Futûh, qui est
probablement le seul vrai « démocrate » du Bureau (respectueux de la
souveraineté populaire, des libertés publiques, du principe de
citoyenneté) y est marginalisé. Le Bureau comprend en réalité deux
courants idéologiques, les salafistes et les qutbiens. Les positions des
uns et des autres sont souvent proches, même si les salafistes
reprochent souvent aux qutbiens de faire du \emph{« ta'wîl »}
(interprétation, voire surinterprétation). Mais cette distinction entre
qutbiens et salafistes ne doit pas être surestimée : les premiers sont à
peine moins méfiants que les seconds vis-à-vis des apports étrangers à
l'islam, et même chez les anti-qutbiens, le diagnostic de Qutb s'est
imposé : une société qui n'applique pas la charia n'est pas vraiment
musulmane.

Les contraintes idéologiques, le faible poids des démocrates au sein de
la
confrérie, les pesanteurs organisationnelles, l'\emph{ethos} ou, si l'on
préfère, les techniques et les critères de recrutement, les lieux où il
s'effectue, l'endoc- trinement, la formation et la culture que l'on
inculque, les types de socialisation et de réseaux, tout ceci rend
difficile une autonomisation du politique et l'avènement d'une culture
démocratique. Cela vaut aussi pour les interactions avec l'environnement
et les contraintes du double positionnement, sur le champ religieux et
sur le champ politique : le problème de la confrérie est moins les «
forces » démocrates que la concurrence de certains salafistes, qui
ne lui pardonnent pas son acceptation du jeu électoral et ses
concessions à la modernité, en bref sa \emph{tawfîqiyya}.
 

\hypertarget{les-fruxe8res-fer-de-lance-de-la-duxe9mocratisation}{%
\section{Les Frères, fer de lance de la démocratisation
?}\label{les-fruxe8res-fer-de-lance-de-la-duxe9mocratisation}}

 
Il convient maintenant d'examiner les affirmations du « savoir conven-
tionnel » qui semblent provenir de préoccupations que je qualifierais
d'« extérieures » -- je désigne sous ce terme les questionnements
induits par des enquêtes sur les formations islamistes d'autres pays, ou
encore pour répondre à une demande ou à des agendas internationaux, ou
par des raisonnements hypothético-déductifs (certains s'inscrivant dans
des lectures hégéliennes de l'histoire), ou tout simplement par « cécité
créatrice ».
 

\hypertarget{les-fruxe8res-et-le-cadre-national}{%
\subsection{Les Frères et le cadre
national}\label{les-fruxe8res-et-le-cadre-national}}

 
L'on estime en général que les Frères, à l'instar d'autres mouvements
islamistes légalistes, ont accepté le cadre national 36 ; que ces
mouvements sont un moment « indispensable » du processus de construction
nationale. Tout dépend du contenu que l'on donne à ce « cadre national »
ou à « accepter » (« faire avec » semble plus exact). Oui, les Frères
musulmans ont un programme pour gouverner le pays, et raisonnent surtout
en termes et dans un cadre égyptiens. Oui, l'utopie du rétablissement du
califat est de moins en moins mentionnée (mais a-t-elle été totalement
abandonnée ?). Mais affirmer que les Frères sont un moment «
indispensable » de la construction nationale est péremptoire, malgré
l'importance de leur travail social. Outre les effets délé- tères de
leurs pratiques et discours sur le lien national et sur les tensions
confessionnelles, le processus de construction nationale a commencé
avant eux ou sans eux. Târiq al Bishrî en est presque à voir dans les
Frères le stade suprême du nationalisme 37 ! Pour lui, l'histoire
égyptienne est scandée par des moments ou des dialectiques : le moment
\emph{Wafd} a été celui d'un nationa- lisme laïc qui tentait de
recouvrer l'indépendance \emph{politique} ; le moment

« nasséro-gauchiste » a incarné la revendication d'indépendance
\emph{économique} ;

le moment islamiste vise à mettre fin à l'aliénation en faisant
correspondre le régime politique et l'espace social à
\emph{l'authenticité culturelle} du pays, à son
« identité ».

On pourrait critiquer la lecture et la conceptualisation de l'histoire
ainsi instaurée, ou le contenu de la notion d'authenticité 38 ; je
préfère rappeler que les impératifs identitaires ne sont pas ceux de la
démocratie, et qu'un discours de ce type, s'il était tenu en Europe,
serait classé, comme il le mérite, à l'extrême- droite. Et même si l'on
acceptait ce diagnostic, on aurait beau jeu de souligner que l'identité
culturelle égyptienne, si elle existe, relève de la \emph{tawfîqiyya} et
que le mouvement islamiste est loin d'en être le meilleur représentant.
 

\hypertarget{les-fruxe8res-et-la-violence}{%
\subsection{Les Frères et la
violence}\label{les-fruxe8res-et-la-violence}}

 
Aucun acte de violence ne peut être imputé aux Frères depuis 1974, alors
que le comportement des autorités vis-à-vis de la confrérie aurait pu
être un prétexte plausible pour renouer avec les pratiques d'antan. J'ai
écrit en 2005 que la « question était réglée 39 ». J'en suis moins
certain aujourd'hui. Pendant le premier semestre 2006, des informations
inquiétantes évoquant des séjours de formation des militants à
l'étranger ont filtré -- le tout étant de savoir le crédit qu'on accorde
à ce qui émane de la sécurité d'État 40. Pendant la récente guerre au
Liban, le Guide suprême `Akif a affirmé qu'il était prêt à envoyer 10
000 com- battants épauler le Hezbollah, si le gouvernement égyptien l'y
autorisait 41. Les Frères ont reconstitué en 1998 une section
d'éducation physique, pour entraîner leurs militants à la conduite de
manifestations, à l'autodéfense et à la protection des dirigeants. La
vitalité de cette section est attestée par le nombre de camps de
vacances découverts et démantelés les deux dernières années, qui
ressemblent beaucoup à ceux de l'organisme secret créé en 1939 par al
Bannâ, hormis l'absence d'initiation au maniement des armes.

Le 10 décembre 2006, l'opinion apprend qu'une cinquantaine de militants

Frères encagoulés se sont livrés dans l'enceinte d'al Azhar à une
démonstra- tion de leur savoir-faire en sports de combat. Des chaînes de
télé et le quotidien \emph{Al-Masri Al-Yom} en diffusent images et
photos. Les jours qui suivent, les Frères tentent de limiter la casse,
oscillant entre plusieurs stratégies discursives (fuites contrôlées,
minimisation du problème, négation du caractère « martial » de la
démonstration, aveu de « maladresse » assorti d'excuses\ldots). Le Guide
suprême `Akif affirme dans la presse que cette démonstration n'est pas
la première du genre, qu'il y en a eu plusieurs auparavant, et notamment
lors des mani- festations d'appui au Hamas :

« Ce n'était pas un défilé militaire. Si j'en avais organisé un, il eut
été différent. En effet, le lieu et le timing ne sont pas appropriés.

Êtes-vous donc capable d'en organiser un ?

Oui, quand il y a des raisons pour le faire. Mais tant qu'on est dans un
État ayant une constitution et un droit {[}\ldots{]}, nous ne pouvons
envisager un acte portant atteinte au citoyen ou à la patrie
{[}\ldots{]}. Je n'accepte pas de collaborer avec un groupe ayant
recours à la violence 42.»

Pour lui, la démonstration d'al Azhar était un défilé sportif. Mais il
confirme être en mesure d'envoyer 10 000 Frères guerroyer, et même
davantage, si le gouvernement l'accepte. On le voit, la déclaration est
une véritable motion de synthèse : on rassure les partisans d'une
para-militarisation de la confrérie en assumant l'option, et les
légalistes en affirmant qu'elle ne sera pas utilisée sur la scène
intérieure, et on affirme ne pas être à l'origine de l'initiative.

Les Frères n'ont donc plus, depuis trois décennies, commis d'actes
dépassant le « seuil » usuel en Égypte, mais ils disposent des
structures matérielles ainsi que d'un corpus doctrinal pour le faire.
Cette évolution est caractéristique des problèmes que pose la confrérie
au système politique égyptien : cette armée sans armes produit des
combattants potentiels qu'elle tient sous contrôle. Porter un coup
sévère ou décisif à la confrérie (à supposer que cela soit possible)
risquerait de « lâcher dans la nature » des personnes initiées aux
techniques de combat. D'autre part, les Frères occupent un créneau qui
leur permet de concurrencer d'éventuels autres groupes jihadistes,
puisqu'ils offrent un produit, la préparation au combat, qui répond à
une demande sociale -- qu'ils entretiennent et suscitent. Rappelons
toutefois qu'à plusieurs moments cruciaux, les Frères ont été « en
retrait » par rapport à d'autres : on pense notamment à leur position
modérée dans l'affaire de la profanation du Coran à Guantanamo ou lors
de l'épisode des caricatures danoises.
 

\hypertarget{les-fruxe8res-et-la-charia-une-moduxe9ration-discutable}{%
\subsection{Les Frères et la charia : une modération discutable
?}\label{les-fruxe8res-et-la-charia-une-moduxe9ration-discutable}}

 
Même des analystes \emph{a priori} compréhensifs envers les Frères
estiment que leurs discours sur cette question sont très ambigus, qu'ils
n'ont pas donné d'assurances suffisantes 43. La confrérie n'a toujours
pas fait son \emph{aggiornamento} sur la question des aspects les plus
controversés de la charia -- et elle ne le fera probablement pas dans un
avenir proche : en quoi les Frères seraient-ils distincts du PND s'ils
renonçaient à l'utopie (très mobilisatrice en Égypte) d'un ordre
politique et social radicalement autre ? Comment peut-on imaginer
que des militants prenant d'énormes risques personnels, sacrifiant sur
la voie de Dieu carrière et perspectives d'avenir, puissent changer de
logique et d'objectifs du jour au lendemain et se résigner à la
normalité de petits desseins intramondains ? On pourrait imaginer la
chose si les efforts d'intellectuels en vue de la restitution de son
historicité au droit musulman et de la relativisation de sa
sacralisation avaient droit de cité -- mais une personne défavorable à
l'application de la charia ne peut aller au-delà d'une invocation (face
à l'opinion, on ne parle pas des cercles intellectuels) de
l'inopportunité temporaire de celle-ci pour le bien-être de la
communauté.

La stratégie des Frères sur la question des peines corporelles prévues
par le droit islamique a longtemps été l'évitement. On peut penser que
la confrérie est divisée sur le sujet et qu'elle tente d'éviter de se
déchirer, mais je préfère croire qu'elle sait que sur ce point aucune
concession durable n'est possible, et qu'elle ne peut que promettre une
approche graduelle, lente, mais « irré- vocable » : le programme Frère
préconise une approche gradualiste, c'est-à-dire de commencer par un
effort important d'éducation « islamique » et d'inculcation des valeurs
morales (comme si le régime faisait autre chose !), puis d'œuvrer à
l'élimination des « causes » du crime ; ensuite seulement, on appliquera
les peines corporelles, avec une grande sévérité. Optimistes et
pessimistes exploi- teront différemment ce gradualisme !
 

\hypertarget{les-fruxe8res-face-uxe0-la-question-copte}{%
\subsection{Les Frères face à la question
copte}\label{les-fruxe8res-face-uxe0-la-question-copte}}

 
Vis-à-vis de la minorité copte du pays, les positions des Frères
restent, au mieux, ambivalentes. La stratégie, telle que définie par le
dirigeant al Shâtir dans des documents internes saisis en 1992, se
résume en un mot : « rassurer », autant que faire se peut 44. Depuis
plus de deux ans, on annonce régulièrement la publication d'un document
reconnaissant le droit des coptes à la citoyenneté et affirmant que la
capitation (l'impôt sur les non-musulmans, l'un des signes de leur
infériorité, de leur soumission et de la primauté de l'islam) n'est plus
d'actualité. Mais on ne voit rien venir, alors qu'en principe le
compagnon de route Târiq al Bishrî a réussi à fonder « islamiquement »
la citoyenneté et l'égalité entre citoyens de confessions différentes.
Plus grave, lors des débats parlementaires de 2007 sur le remaniement
constitutionnel, les députés Frères ont rejeté le nouvel article 1 qui
faisait de la citoyenneté le principe organisateur de l'État-nation, en
invoquant toutes sortes de prétextes fallacieux. Enfin, le programme
Frère, qui vient d'être rendu public, interdit explicitement aux coptes
l'accès à la Présidence de la République, et, semble-t-il (le texte
n'est pas clair), aux fonctions publiques qui entraînent des obligations
de défense et de promotion de la religion, ce qui devrait inclure la
présidence du Conseil et le commandement des forces armées.

Dans le même ordre d'idées, la confrérie affirme souvent qu'elle n'a
aucune objection contre la création d'un parti copte, affirmation à
première vue bizarre si l'on tient compte de son hostilité affichée à
toute politique de quotas et de

« discrimination positive » en faveur des non-musulmans. Mais affirmer
que les coptes ont droit, en tant que tels, à un parti, revient à dire
que les musulmans et l'islam ont droit à un groupe les
représentant\ldots{} les Frères musulmans. Il s'agit de favoriser une
définition communautariste des enjeux, avec un représentant politique
unique (ou au moins hégémonique) de chaque « religion ». Lequel serait
évidemment, dans le cas de l'islam, les Frères. La confrérie souffle
alter- nativement le chaud et le froid, avec une préférence affichée, et
probablement réelle, pour le chaud. Prudente au niveau national, elle
exploite quelquefois les tensions communautaires et les sentiments
anti-coptes de certaines couches de la population.

Il est plus délicat de parler des « attitudes » des militants et cadres
Frères à propos des coptes, mais on peut se risquer à dire qu'il y a de
très nets pro- grès par rapport aux pratiques d'antan 45, mais que
l'enseignement prodigué à la base reste prisonnier des préjugés et de la
doctrine traditionnels : de surcroît, les Frères donnent souvent
l'impression de penser que les coptes forment une minorité choyée, trop
puissante économiquement, trop bien traitée et geignarde. Ils
reconnaissent la légitimité de certains griefs coptes, mais estiment
qu'ils devraient être dirigés contre l'État. Les dérapages de cadres
Frères sur la question sont toutefois nombreux et parfois graves, et ils
semblent ne pas voir que leurs postures anti-chrétiennes sont très
offensantes (de nombreux discours coptes sont également odieux) 46.
 

\begin{enumerate}
\def\labelenumi{\arabic{enumi}.}
\setcounter{enumi}{43}
\item
   
  Les documents en question ont été publiés par l'hebdomadaire
  \emph{Al-Musawwar} en 1992.
   
\item
   
  Pour ne donner qu'un exemple illustrant la comparaison, j'invite le
  lecteur à consulter les numéros du mensuel frère \emph{Al-da'wa} de
  l'été 1981. Nombreux sont les Frères qui parlent plus facilement et
  plus amicalement aux coptes, les déclarations officielles des
  dirigeants de la confrérie tentent plus systématiquement d'être
  rassurantes, etc.
   
\item
   
  Sur le statut de la femme, les attitudes des Frères varient grandement
  : certains leurs serrent la main, d'autres non ; les épouses de
  certains portent le \emph{hijab} (voile dissimulant les cheveux),
  d'autres le \emph{niqab} (qui fait disparaître le visage). Le
  programme frère affirme que la femme est l'égale de l'homme, qu'elle a
  le droit de travailler, mais que son rôle et sa mission principaux
  sont au foyer. Il ne faut pas lui imposer des missions «contraires à
  sa nature» -- comme par exemple la Présidence de la République --, et
  les travaux et fonctions qu'elle peut occuper sont déterminés dans le
  cadre de la \emph{marja'iyya} islamique\ldots{} En revanche, les
  propositions de détail sont intéressantes quoique vagues, et le
  programme déplore, en conclusion, le fait que la libération «
  excessive » de la femme a entraîné en réaction une radicalisation des
  attitudes misogynes\ldots{} Israël mérite aussi une mention. Si l'on
  excepte une récente déclaration d'al `Iryân (membre du \emph{majlis
  shûra}), qui a affirmé que les Frères au pouvoir reconnaîtraient
  Israël mais s'est vite rétracté, on peut dire que la confrérie a
  adopté avant le Hamas la position de ce dernier: «respecter les
  accords conclus, mais ne pas reconnaître Israël ». Mais d'autres
  responsables frères, tout en réitérant leur hostilité et celle de la
  confrérie à toute reconnaissance de l'État hébreu, ont affirmé qu'en
  cas d'arrivée au pouvoir, un référendum sur cette question serait
  organisé et que l'on laisserait le peuple décider.
   
\end{enumerate}

\hypertarget{les-fruxe8res-et-la-duxe9mocratie}{%
\subsection{Les Frères et la
démocratie}\label{les-fruxe8res-et-la-duxe9mocratie}}

Commençons par l'acquis : la position des Frères n'est plus celle
prévalant sous al Bannâ, qui était hostile aux partis, accusés
d'entériner la division de la Communauté. Désormais, ils acceptent le
multipartisme, l'alternance, l'organisation d'élections libres et
l'arbitrage du peuple. Le consensus des experts veut qu'il s'agisse
d'évolutions importantes mais insuffisantes, ce qu'a confirmé la
publication du programme Frère, qui prévoit une formule à l'iranienne,
où, malgré des élections et une acceptation du multipartisme, la
souveraineté et les décisions de dernier ressort seraient aux mains d'un
organisme d'oulémas. Mais gageons que la confrérie remaniera ce texte
sur ce point et éliminons les thèses simplistes des adversaires des
Frères qui dénoncent la mauvaise foi des acteurs islamistes quand
ceux-ci prônent la démocratie, même si les « faux pas » de la confrérie
et les déclarations de plus en plus inquiétantes de ses cadres, depuis
2006, leur donnent des munitions. Soulignons que certains acteurs de la
mouvance sont sincères quand ils pro- clament leur credo démocratique --
même si d'autres le sont beaucoup moins 47. Il faut aussi être prudent
avant de manier l'argument de l'incompatibilité selon lequel on ne
pourrait être simultanément islamiste et démocrate, comme s'il fallait
absolument choisir entre reconnaître la souveraineté à Dieu et la donner
aux hommes. Il y a tension entre toute religion et la démocratie, en ce
qu'une religion délimite un domaine qu'elle soustrait à la délibération
-- mais les compromis sont possibles 48. Toutefois, quoi que l'on pense
de la cohérence intellectuelle et de la valeur intrinsèque des discours
des « islamistes démo- crates », ils existent et les hommes peuvent
avoir des fidélités contradictoires et des discours peu rigoureux. En
revanche, on ne peut être certain qu'un islamiste démocrate optera
toujours, en cas de conflit, contre la charia et pour la démocratie. Et
l'on sait que les islamistes démocrates ne sont pas majoritaires
au sein de la confrérie.

\textbf{P}our conclure, je voudrais tester maintenant la position
défendue par des spécialistes respectés, qui reconnaissent que
l'acceptation de la démocratie et de ses corollaires par les Frères est
insatisfaisante mais inéluctable, et qu'elle

se fera, même malgré les acteurs. Ainsi, selon Éric Rouleau, puisque les
Frères ont renoncé à la violence ils doivent s'adapter au réel et donc
adopter de nouvelles pratiques, ce qui aura immanquablement des
conséquences sur leur perception des choses, leur discours et leur
idéologie 49. Tammâm et Shirîf Yûnis pensent que les intérêts
économiques et l'embourgeoisement des Frères seront décisifs : le
recours à la violence est de plus en plus improbable, du fait même de
l'importance des effectifs de la confrérie et de ses intérêts finan-
ciers 50. Il y a trop à perdre et les souvenirs des conséquences des
affrontements

avec les régimes successifs incitent à la prudence\textbf{.} L'historien
Shirîf Yûnis, auteur d'une importante thèse sur Qutb, souligne ainsi que
le fait que la logique du capitalisme ou du commerce international ne
permette pas de distinguer les gens en fonction de leur religion, mais
opère selon d'autres critères, finira par faire prévaloir une
modification des « mentalités » 51. Tammâm lui aussi analyse ces 25
dernières années comme une adaptation au réel qui devrait mener à la
démocratie : l'\emph{aggiornamento} consécutif proviendrait pour lui de
la confrontation des Frères aux réalités de l'exercice du pouvoir au
sein des syndicats et ordres professionnels conquis pendant les années
1980 ; les problèmes concrets auraient amené les Frères à revoir, par
petites touches, leurs positions sur plusieurs points 52. Pour lui, les
deux variables clés auraient été les relations internationales et la
présence copte. Mais ceci est-il vraiment probant ? Les Frères se sont
adaptés à l'environnement parce qu'ils n'ont pas pu le restructurer --
mais ceci pourrait changer s'ils prenaient le pouvoir.

Un autre moment important, en ce qui concerne les origines du « nouveau
discours », est la publication d'un collectif autocritique, en 1989,
auquel ont participé plusieurs Frères, et qui a été une première
tentative pour penser (et accepter) la démocratie. Quand Târiq al-Bishrî
(qui n'est pas Frère, mais que la direction lit) publia en 1994 un
article sur la citoyenneté et les coptes, les réactions de certains
Frères, et non des moindres, furent favorables. Ce texte fournit des
arguments à ceux qui, dans la direction, voulaient changer de
  Je voudrais écrire un article intitulé « Mon problème avec les Frères
  démocrates ». Dans une tri- bune d'opinion publiée par le \emph{Daily
  Star Egypt} le 10 novembre 2007, Ibrâhîm al-Hudaybî, membre de
  l'équipe d'Ikhwân On line et étoile montante de la confrérie, explique
  que les gouvernements occi- dentaux ne savent pas faire la différence
  entre terroristes et islamistes (ce qui est faux), qu'ils n'ont pas
  compris que les « islamistes modérés » acceptent la démocratie et
  qu'ils respectent les libertés civiles et les droits de l'homme. Si le
  respect de ces libertés est le critère décisif de la modération, force
  est d'admettre que la confrérie n'est pas modérée : est-elle disposée
  à reconnaître à un athée le droit de proclamer son athéisme ? À un
  musulman de renier publiquement sa foi ? Arrêtons de plaisanter ! Il
  ajoute que même en Occident il n'y a pas de consensus sur la
  définition des droits de l'homme et invoque le relativisme culturel
  pour affirmer que les décisions sur ces questions doivent être fondées
  sur les « valeurs de la majorité » (mais accepterait-il la position
  française sur le voile au nom des valeurs de la majorité ?). Il
  affirme que les islamistes croient en l'égalité de tous les citoyens
  (sauf que les non- musulmans et les femmes sont moins égaux que les
  autres). Il ajoute que les écrits des auteurs « isla- mistes », tels
  al Qaradâwî, al Bishrî et al `Awwâ, montrent le respect islamiste pour
  les droits de l'homme et les libertés. Mais, outre que l'on pourrait
  énumérer leurs dérapages (à l'exception d'al Bishrî), que l'on peut
  critiquer leurs divers travaux, les deux derniers nommés ne sont pas
  Frères\ldots{}
discours à l'égard des coptes. Reste que la doctrine des Frères n'a pas
incor- poré le papier d'al-Bishrî. Ainsi il demeure que la question de
la citoyenneté et de l'égalité des musulmans et des non-musulmans au
sein de la patrie n'a pas été fondée doctrinalement par un théoricien
Frère.

Voilà, exposée en détail, la thèse optimiste crédible. Malgré un support
empirique conséquent, elle est hypothético-déductive, et elle présume
une téléologie discutable ainsi qu'un certain matérialisme. Je ne nie
pas qu'une évolution en ce sens soit possible. Est-elle la plus probable
? L'on constatera d'abord que le texte de Tammâm date de 2004 et que
depuis les évolutions récentes ont été dans le sens d'une régression.
Plus généralement, cette thèse s'obstine à méconnaître les éléments
lourds qui jouent en sens contraire ; ils peuvent être spécifiques aux
Frères, comme leur rapport à l'idéologie, comme leurs modes de
socialisation, sectaire au sens sociologique du terme, comme leur
conception de la Cité idéale et de la charia. Ils peuvent relever du
contexte égyptien et régional : le troisième lieu saint de l'islam est
occupé, ce qui favorise les cultures de guerre ; la libération des
femmes égyptiennes, réelle par plusieurs aspects, provoque un
raidissement des Frères de ces dernières ; la salafisation de la société
et de l'environnement, le délitement du lien national et le ren-
forcement des communautarismes, tout ceci ne joue pas en faveur de la
démo- cratie. L'Occident n'est plus le modèle qu'il était à la Belle
Époque. La liste n'est pas exhaustive. Les « effets » de la
mondialisation ou de la modernisation suscitent des réactions
anti-mondialistes et anti-modernes. Affirmer que les premières gagneront
très vite, c'est formuler, au mieux, un vœu pieux 

Tewfik Aclimandos Cedej, Le Caire

\section{Jihad in Islam - Abul a'la Maududi}
 \includegraphics[width=\textwidth]{CourantsIslamContemporain/ImagesCourantsIslamContemporain/image4.png}
 



\begin{quote}
\textbf{In the Name of Allah, the Merciful and the Most Beneficent}

(\emph{This Address was delivered on Iqbal Day, April 13, 1939, at the
Town Hall, Lahore})
\end{quote}

The word `Jihād' is commonly translated into English as `the Holy War'
and for a long while now the word has been interpreted so that it has
become synonymous with a `mania of religion'. The word `Jihād' conjures
up the vision of a marching band of religious fanatics with savage
beards and fiery eyes brandishing drawn swords and attacking the
infidels wherever they meet them and pressing them under the edge of the
sword for the recital of \emph{Kalima}. The Artists have drawn this
picture with masterly strokes and have inscribed these words under it in
bold letters:

\begin{quote}
`The History of this Nation is a tale of Bloodshed'.
\end{quote}

The irony is that the painters are no other than those benefactors of
ours who themselves have been engaged in an extremely unholy war for
centuries on end. They themselves present the picture of robbers who
armed to the teeth with all kinds of deadly weapons, have set upon the
world pillaging it for the capture of new markets of trade, resources of
raw material, open lands for colonisation and mines
yielding valuable metals, so that they may procure fuel for their
everburning fire of avarice. They fight not for the cause of God but for
the satisfaction of their lust and hunger. For them, it is a sufficient
excuse for invading a nation because the territory of that nation
contains mines, or their lands yield bumper crops, or oil has been
struck there or they can be exploited as profitable markets for their
manufactured goods or that their surplus population can be settled on
the lands belonging to the intended victims. In the absence of all the
other excuses, they consider it a grave crime on the part of a nation if
she happens to live \emph{en route} to a country already captured by
them or the one they plan to capture. Whatever we did is now part of
history, past and gone, but their deeds are a present matter witnessed
by the world day and night. Asia, Africa, Europe and America---which
portion of this planet has been spared from bloodbath resulting from
their unholy war? Their skill is, however, commendable that they have
painted our picture so gory and dark that their own picture was
overshadowed and was completely hidden from the view. Our own simplicity
is amazing too. When we saw this picture of ours painted by the
foreigners, we were so taken aback that we never thought of looking
behind the canvas and seeing the visage of the painter. Instead we
started offering apologies in this manner---Sir, what do we know of war
and slaughter. We are pacifist preachers like the mendicants and
religious divines. To refute certain religious beliefs and convert the
people to
some other faith instead, that is the be-all and end-all of our
enthusiasm. What concern have we with sabres! Yes, indeed, we plead
guilty to one crime, though, that whenever someone else attacked us, we
attacked him in self-defence. Now, of course, we have renounced that
also. The crusade which is waged by swords has been abrogated for the
satisfaction of your honour. Now `Jihad' only refers to waging war with
the tongue and pen. To fire cannons and shoot with guns is the privilege
of your honour's government and wagging tongues and scratching with pens
is our pleasure.

\hypertarget{causes-of-misunderstanding-about-the-holy-war}{%
\subsubsection{Causes of Misunderstanding about the Holy
War}\label{causes-of-misunderstanding-about-the-holy-war}}

In any case, this is a part of political tactics. But from a purely
scholastic standpoint when we analyse the causes due to which the red
nature of the `Holy War for the Cause of God' has become difficult to
understand not only for non- Muslims but Muslims themselves, we discover
two major and basic misconceptions. The first misunderstanding is that
they consider Islam to be a religion in the conventional sense of the
term `religion'. The second misconception is that they take Muslims to
be a `Nation' in the technical sense of this term. These two
misunderstandings have not only mixed up the concept of Jihād' but have
changed the picture of Islam as a whole and have wholly misrepresented
the position of the Muslim people.

In common terminology `religion' means nothing more than a hotch potch
of some beliefs, prayers and

rituals. If this is what `religion' means, then, it should, indeed, be a
private affair. You should be free to entertain any belief and worship
any deity whom your conscience is ready to accept. If you are
over-zealous and ardent devotees of this type of religion, go and preach
it to the whole world and engage yourselves in declamations with the
protagonists of other religions. There is no reason why you should take
up a sword? Do you wish to convert people to your faith by killing them?
We are forced to admit the point that if you regard Islam as a religion
in the conventional meaning of the term and if, indeed, Islam be a
conventional type of religion, the necessity for `Jihad' cannot be
justified.

Similarly, the term `Nation' connotes no more than a homogeneous group
of men who have joined themselves in a distinct entity on the basis of
fundamental and shared traits. A group of people who attain to
nationhood according to this definition of the term, rises or can rise
to arms under two circumstances: either when some other group of people
with the intention of depriving them of their lawful rights attack them
or when they themselves wishing to usurp other people's rights launch an
attack on them. There is an unassailable moral justification for taking
up arms in the first case (although some saintly personages have
declared even armed self-defence a sin). But launching an armed attack
on other people with the purpose of snatching away their lawful rights
can be justified by no one except a few dictators. Even statesmen of
vast Empires like those of Britain and

France dare not to justify this course of action.
So if Islam be a `Religion' and the Muslims are a `Nation'. `Jihad' (on
account of which it has been accorded the dignity of `The Best of all
Prayers' in Islam) becomes useless term. But the truth is that Islam is
not the name of a `Religion', nor is `Muslim' the title of a `Nation'.
In reality Islam is a revolutionary ideology and programme which seeks
to alter the social order of the whole world and rebuild it in
conformity with its own tenets and ideals. `Muslim' is the title of that
International Revolutionary Party organized by Islam to carry into
effect its revolutionary programme. And `Jihād' refers to that
revolutionary struggle and utmost exertion which the Islamic Party
brings into play to achieve this objective.

Like all revolutionary ideologies, Islam shuns the use of current
vocabulary and adopts a terminology of its own, so that its own
revolutionary ideals may be distinguished from common ideals. The word
`Jihad' belongs to this particular terminology of Islam. Islam purposely
rejected the word `\emph{harb}' and other Arabic words bearing the same
meaning of `war' and used the word `Jihad' which is synonymous with
`struggle', though more forceful and wider in connotation. The nearest
correct meaning of the word `Jihād' in English can be expressed as
under:

\begin{quote}
`To exert one's utmost endeavour in promoting a cause'.
\end{quote}

The question is why was the use of this new word preferred to the
exclusion of all older synonyms? The answer to this question is none
else than that the word `war' was and is still being used for struggles
between Nations and States which are waged for the achievement of
individual or national self-interest. The motive forces behind these
conflicts are such individual or collective purposes as are completely
devoid of any ideological bias or support for certain principles. Since
Islamic War does not belong to this category, Islam shuns the use of the
word `war' altogether. Islam has no vested interest in promoting the
cause of this or that Nation. The hegemony of this or that State on the
face of this earth is irrelevant to Islam. The sole interest of Islam is
the welfare of mankind. Islam has its own particular ideological
standpoint and practical programme to carry out reforms for the welfare
of mankind. Islam wishes to destroy all states and governments anywhere
on the face of the earth which are opposed to the ideology and programme
of Islam regardless of the country or the Nation which rules it. The
purpose of Islam is to set up a state on the basis of its own ideology
and programme, regardless of which nation assumes the role of the
standard-bearer of Islam or the rule of which nation is undermined in
the process of the establishment of an ideological Islamic State. Islam
requires the earth---not just a portion, but the whole planet---not
because the sovereignty over the earth should be wrested from one nation
or several nations and vested in one particu-

lar nation, but because the entire mankind should benefit from the
ideology and welfare programme or what would be truer to say from
`Islam' which is the programme of well-being for all humanity. Towards
this end, Islam wishes to press into service all forces which can bring
about a revolution and a composite term for the use of all these forces
is `Jihad'. To change the outlook of the people and initiate a mental
revolution among them through speech or writing is a form of `Jihad'. To
alter the old tyrannical social system and establish a new just order of
life by the power of sword is also `Jihad' and to expend goods and exert
physically for this cause is `Jihad' too.

\hypertarget{for-the-cause-of-godthe-essential-condition}{%
\subsubsection{`For the Cause of God'---the Essential
Condition}\label{for-the-cause-of-godthe-essential-condition}}

But the `Jihad' of Islam is not merely a `struggle'; it is a `struggle
for the Cause of God'. `For the Cause of God is an essential condition
for `Jihad' in Islam. This expression is also part of the special
terminology of Islam to which I have alluded above. Its literal meaning
is `In the way of God'. It is this translation which misled the people
into believing that `Jihad in the way of God' enjoined forcible
conversion of other people to the faith of Islam, for the limited
intellects of the people could take the expression `in the way of God'
to mean nothing else than that. But in the terminology of Islam this
expression bears wider meaning. All such work as is undertaken for the
collective well-being of mankind and in which the functionary has no
vested interest in the present world, his sole interest being to win the
favour of

God, is regarded in Islam as an `act in the way of God'. To take an
instance, if you give away something in charity in anticipation of
receiving some material or moral dividend in this world, it would not be
regarded as an `act in the way of God'. But if it is your desire to win
the pleasure of God by affording assistance to a poor man, this
charitable act would be deemed to have been done `in the way of God'.
Hence the term `in the way of God' is reserved for such deeds only as
are undertaken with perfect sincerity, without any thought of gaining a
selfish end, and executed on the understanding that to afford benefit to
other human beings is a means of winning the pleasure of God and the
sole purpose of human life is to win the favour of the Creator of the
universe.

The condition `in the cause of God' has been attached to `Jihād' for the
same reason. It strictly implies that when a person or a group arises to
carry out a revolution in the system of life and to establish a new
system in conformity with the ideology of Islam, he or they should keep
no selfish motives in mind while offering sacrifices and executing acts
of devotion for the Cause. The aim should not be to knock out an Emperor
and occupy the vacant throne i.e., to become a Caesar replacing another
Caesar. The objectives of the struggle should be completely free from
the taint of selfish motives like gaining wealth or goods, fame and
applause, personal glory or elevation. All sacrifices and exertions
should be directed to achieve the one and the only end i.e., the
establishment of a just and equitable social order among

human beings; and the only reward in view should be to gain the favour
of God. The Holy Qur'an says:

\begin{quote}
`Those who believe fight in the way of God and the unbelievers fight in
the way of Tāghūt (Devil)'. (4: 76)
\end{quote}

The word \emph{Tāghūt} is derived from `\emph{Tughian}' (the deluge)
which bears the meaning `to cross the limit'. When the river crosses its
boundaries we say `the deluge has come'. Similarly, when man
transgresses all lawful bounds and exerts himself to assume the position
of the Lord over human beings or to expropriate more goods than are
rightfully his due, this is called as `fighting in the way of
\emph{Tāghūt}'. In contrast to this `fighting in the way of God' refers
to the struggle for the establishment of God's just order in the world.
The fighter's aim is to abide by the law of God himself and enforce it
among other human beings. In connection with this point, the Holy Qur an
says:

\begin{quote}
`We shall confer dignity in the Eternal world upon those who do not seek
to establish their might in the world and do not wish to create strife.
Success in the world Hereafter awaits those who are God-fearing'. (28:
83)
\end{quote}

It is reported in the Traditions that an individual enquired from the
Holy Prophet (peace be upon him), "What does `war in the cause of Allah'
imply? A man fights to obtain goods. Another engages in battle to secure
a reputation for valour. A third man fights to wreak vengeance upon the
other or is impelled to fight for national honour. Who, among

these men, is a fighter `in the way of God" The Holy Prophet (peace be
upon him) answered: "None. Only he fights in the way of the Lord who
holds no other purpose than the glorification of God".

Another tradition relates: "If a man engaged in battle entertains in his
heart a desire to obtain out of the war only a rope to tie his camel
with, his reward shall be forfeited".

God accepts only such needs as are executed for the purpose of obtaining
His Goodwill and the doers seek to serve no personal or collective
objectives. Hence from the standpoint of Islam, the condition `in the
way of God' is of utmost importance in relation to `Jihad'. Mere
striving is done by all living creatures in the world. Every one is
doing his utmost to secure his purpose. But the most important, nay, the
fundamental ideal among the revolutionary doctrines of that
Revolutionary Party called `Muslims' is to expend all the powers of body
and soul, your life and goods in the fight against the evil forces of
the world, not that having annihilated them you should step into their
shoes, but in order that evil and contumacy should be wiped out and
God's Law should be enforced in the world. After having briefly
elucidated the meaning of Jihad and the significance of the clause `in
the way of God', I wish to explain in brief terms the Revolutionary
Creed which Islam upholds so that it may be easily understood why Jihād
is needed and what is the objective of Jihad?

\hypertarget{the-revolutionary-creed-of-islam}{%
\subsection{THE REVOLUTIONARY CREED OF
ISLAM}\label{the-revolutionary-creed-of-islam}}

\begin{quote}
The Revolutionary Creed of Islam, in a nutshell, is:

`O people! Offer worship to that God alone Who created you'. (2: 21)
\end{quote}

The call of Islam is not addressed to the workers, landholders, peasants
or industrialists; it is directed to the whole of human race. Islam
addresses man in his capacity as human being. If you entertain the
conceit that you are a demi-god, dispel it because none of you has the
right to demand worship and unconditional submission from fellow human
beings. All of you should affirm devotion to one God and in the devotion
to the divine, you should all stand on a level of equality.

\begin{quote}
"Come to a word equal between us and you that we worship none but Allah,
and that we associate no partner with Him, and that some of us take not
others for Lords beside Allah. But if they turn away, then say, `Bear
witness that we have submitted to God". (3: 64)
\end{quote}

This was the call for a universal and complete revolution. It loudly
proclaimed `Sovereignty belongs to no one except Allah.' No one has the
right to become a self-appointed ruler of men and issue orders and
prohibitions on his own volition and authority. To acknowledge the
personal authority of a human being as the source of commands and

prohibitions is tantamount to admitting him as the sharer in the Powers
and Authority of God. And this is the root of all evils in the universe.
God has instilled the correct spirit in man and has shown him the right
way of life. The reason why human beings deviate from this straight path
is that they forget God and consequently forget their own real worth.
This state of affairs inevitably encourages some persons, dynasties or
classes to claim Divine rights for themselves and taking undue advantage
of their might they reduce general humanity to the status of their
creatures. On the other hand also, this forgetfulness of God and of self
leads a portion of mankind to affirm the Divinity of the Mighty of the
World. They acquiesce in the right of the powerful men to issue commands
and their own obligation to carry out those commands with servile
devotion. This is the root-cause of tyranny, conflict and unlawful
exploitation in the world and this is the target upon which Islam
directs its first assault. Islam issues a clarion call:

\begin{quote}
"And obey not the dictate of those who transgress the bounds, who
mischief in the earth and promote not order". (26: 151-152)

"And obey not him whose heart We have made heedless of Our Remembrance,
who follows his low desires, and his case exceeds all (legitimate)
bounds". (18: 28)

"Certainly Allah's curse is on the wrong-doers who obstruct (mankind)
from the path of Allah and seek to make it crooked." (11: 18, 19)
\end{quote}

Islam puts it to the people:

\begin{quote}
"Are many lords differing among themselves better or Allah, the One, the
Almighty." (12: 39)
\end{quote}

If you do not offer devotion to the One God, you shall never be free
from the bondage of these small and false gods; in one form or another
they shall obtain power over you and will inevitably create strife:

\begin{quote}
"Verily, the monarchs, when they enter a land, despoil it, and render
the highest of its people into the lowest". (27: 34)

"And when he captures power he creates strife on earth. He spoils the
fields and annihilates generations. And God disapproves of strife". (2:
35)
\end{quote}

This is not the occasion to go into all the details. I wish to explain
to you in brief terms and I want you to note the point that Islam's call
for the affirmation of faith in one God and offering devotion to Him
alone was not an invitation to follow a creed in the same conventional
sense as the call of other religious creeds. In reality, it was an
invitation to join a movement of social revolution. Its main brunt fell
directly on those classes who, as divines in the religious sphere,
kings, nobles and ruling classes in the political domain and as usurers,
landholders and monopolists in the economic field of life, had reduced
common humanity to the status of their slaves. At some places they had
openly declared themselves to be lords besides Allah. They demanded
obedience and devotion from the people as their hereditary rights or
privileges based on class distinctions and

brazenly declared:

\begin{quote}
"Who besides me is the deity of yours". (28: 38)
\end{quote}

and and and

\begin{quote}
"I am your highest Lord"; (79: 34)

"I give life and cause death"; (2: 258)

"Who is greater in strength than us". (15: 41)

In other places, they had created false gods in the form of idols and
\end{quote}

temples in order to exploit the ignorance of the common people and
taking over behind these idols and temples they hoodwinked mankind to
acquiesce in their own divine rights.

Hence the call of Islam against heresy, polytheism and idolatry, and
invitation to offer worship and devotion to one God only---all this came
into direct conflict with the interests of the Government and of the
classes which either supported its authority or drew support from it. It
was because of this that whenever a prophet (peace be on him)
proclaimed:

\begin{quote}
O people, obey Allah; none is your deity except God; (11: 84)
\end{quote}

the Government of the day hastened to bar his way with all its might and
main and the degenerate exploiting classes opposed him tooth and nail,
for the call of the Prophet was never a metaphysical proposition; it was
a charter of social revolution. Hence the ruling and exploiting classes
smelt the menace of a political upheaval in the very first pronouncement
of a prophet (peace be on him).

\hypertarget{the-characteristic-feature-of-the-revolutionary-creed-of-islam}{%
\subsubsection{The Characteristic Feature of the Revolutionary Creed of
Islam}\label{the-characteristic-feature-of-the-revolutionary-creed-of-islam}}

There is no doubt that all the Prophets of God (peace be on them)
without exception were Revolutionary Leaders, and the illustrious
Prophet Muhammad (peace be upon him) was the greatest Revolutionary
Leader. But the point at which a clear line of demarcation can be drawn
between these God-worshipping Revolutionary Leaders and the general run
of worldly revolutionaries is that these worldly revolutionaries,
however, honest their intentions may be, can never attain to a correct
level of justice and moderation. The revolutionaries of the world either
rise from oppressed classes themselves or stand for upholding the rights
of the oppressed. They, therefore, look at all matters from the
standpoint of these classes alone The natural result is that their
viewpoint is never impartial and purely humane. On the contrary their
outlook is heavily biased in favour of one class and bears hatred and
resentment for the other class. They prescribe a remedy for tyranny
which is itself tyrannical and is revengeful in effect. It is not
possible for them to shake off feelings of vendetta, jealousy and
ill-will and plan an equitable and balanced social order which ensures
the wellbeing of all persons. In striking contrast to this, whatever the
severity of persecution to which the Prophets (peace be on them) were
subjected, whatever the agonies they and their companions had to suffer
at the hands of the oppressors, the Prophets (peace be on them) did not
allow their personal feel-

ings to influence the course of their revolutionary movements. They
acted under direct Guidance of their Lord. Since the Lord is above all
human passions and He has no special connexion with any human group or
class, nor does He entertain any grudge or feelings of animosity against
any other class of human beings, so under His direct guidance the
Prophets (peace be on them) viewed all matters with impartial justice in
order to discover ways for securing collective well-being. They strove
to devise a system in which each individual might feel content to remain
within the limits of his rights, in which every man might fully enjoy
his lawful rights and secure a perfect balance in the relationship
between man and man and man and society. For this reason, the
Revolutionary Movements launched by the Prophets (peace be on them)
never assumed the character of class war They did not effect social
reconstruction so as to secure the dominance of one class over the other
but establish a just pattern of society which afforded equal
opportunities to all human beings for self improvement and for obtaining
material and spiritual excellence.

\hypertarget{the-need-and-objective-of-jihad}{%
\subsubsection{The Need and Objective of
Jihad}\label{the-need-and-objective-of-jihad}}

It is an uphill task to describe in this brief treatise the details of
the social order envisaged by Islam. I hope an occasion to do so will
shortly present itself. Here, confining myself within the limits of the
subject, the only point which I wish to elucidate is this: Islam is not
merely a religious creed

or compound name for a few forms of worship, but a comprehensive system
which envisages to annihilate all tyrannical and evil systems in the
world and enforces its own programme of reform which it deems best for
the well-being of mankind. Islam addresses its call for effecting this
programme of destruction and reconstruction, revolution and reform not
just to one nation or a group of people, but to all humanity. Islam
itself calls upon all the classes which oppress and exploit the people
unlawfully, its call is addressed even to the kings and the noblemen to
affirm faith in Islam and bind themselves to remain within the lawful
limits enjoined upon them by their Lord. Islam impresses upon them that
if they accept this just and righteous system, they will gain peace and
salvation. This system harbours no animosity against any human being.
Our animosity is directed against tyranny, strife, immorality and
against the attempt of an individual to transgress his natural limits
and expropriate what is not apportioned to him by the natural law of
God. Those who affirm faith in this ideology become members of the party
of Islam and enjoy equal status and equal rights without distinction of
class, race, nation or the country to which they belong. In this manner,
an International Revolutionary Party is born to which Qur'an gives the
title of `Hizb Allah' and which alternatively is known as Islamic Party
or the Ummah of Islam'. As soon as this party is formed, it launches the
struggle to obtain the purpose for which it exists. The rationale for
its existence is that it should en-

deavour to destroy the hegemony of an un-Islamic system and establish in
its place the rule of that social and cultural order which regulates
life with balanced and humane laws, referred to by the Qur'an with the
comprehensive term `the word of God'. If this party does not strive to
effect a change in the government and establish the Islamic system of
government, the very basis on which this party exists is knocked out,
for this party comes into existence to secure no other purpose than the
above and there is no use for this party save that it should struggle
for the cause of God. The Holy Qur'an enunciates only one purpose of the
genesis of this party and that is:

\begin{quote}
"You are the best people, raised for mankind, exhorting good and warding
off evil and believing in Allah." (3: 110)
\end{quote}

These men who propagate religion are not mere preachers or missionaries,
but the functionaries of God, (so that they may be witnesses for the
people), and it is their duty to wipe out oppression, mischief, strife,
immorality, high handedness and unlawful exploitation from the world by
force of arms. It is their objective to shatter the myth of the divinity
of demi-gods and false deities and reinstate good in place of evil.

\begin{enumerate}
\def\labelenumi{(\arabic{enumi})}
\item
  \begin{quote}
  "And fight them until there is no persecution and religion is
  professed for Allah." (2: 193).)
  \end{quote}
\item
  \begin{quote}
  "If you do not do (that you are enjoined) there will be mischief in
  the earth and
  \end{quote}
\end{enumerate}

\begin{quote}
tremendous disorder". ( 8: 73)
\end{quote}

\begin{enumerate}
\def\labelenumi{(\arabic{enumi})}
\setcounter{enumi}{2}
\item
  \begin{quote}
  "He is Who sent His Messenger with guidance and the religion of truth,
  he may make it dominant over all religions, even if the polytheists
  resent it". ( 9: 33)
  \end{quote}
\end{enumerate}

Hence this party is left with no other choice except to capture State
Authority, for an evil system takes root and flourishes under the
patronage of an evil government and a pious cultural order can never be
established until the authority of Government is wrested from the wicked
and transferred into the hands of the reformers. Apart from reforming
the world, it becomes impossible for the party itself to act upon its
own ideals under an alien state system. No party which believes in the
validity and righteousness of its own ideology can live according to its
precepts under the rule of a system different from its own. A man who
believes in communism cannot order his life on the principles of
communism while in England or America, for the capitalistic state system
will bear down on him with all its power and it will he quite impossible
for him to escape the retribution of the ruling authority. Likewise, it
is impossible for a Muslim to succeed in his intention of observing the
Islamic pattern of life under the authority of a non-Islamic system of
government. All rules which he considers wrong; all taxes that he deems
unlawful; all matters which he believes to be evil; the civilization and
way of life which, in his view, are wicked; the education system which
seems to him as fatal---all these will be so inexorably imposed on
him, his home and his children that evasion will become impossible.
Hence a person or a group of persons are compelled by the innate demand
of their faith to strive for the extirpation of the rule of an opposing
ideology and setting up a government which follows the programme and
policies of their own faith, for under the authority of a government
professing inimical doctrines, that person or group of persons cannot
act upon their own belief. If these people evade their duty of actively
striving for this end, it clearly implies that they are hypocrites and
liars in their faith.

\begin{quote}
"May Allah forgive you (O Muhammad) Why didst you permitted them (to
remain behind) till had become manifest to you those who were truthful
and who were liars. Those who believe in Allah and the Last Day, will
not seek permission (for exemption) from striving with their riches and
their lives. And Allah knows the righteous Only those will seek
permission from you (to be exempted) who do not believe in Allah and the
Last Day and whose hearts are full of doubts and in their doubts they
waver." (9: 43-45)
\end{quote}

In these words, the Qur'an has given a clear and definite decree that
the acid test of the true devotion of a party to its convictions is
whether or not it expends all its resources of wealth and life in the
struggle for installing its faith as the ruling power in the State. If
you suffer the authority of an inimical doctrine in the State, it is a
proof positive that your

faith is false and the natural result of this is, and can only be this,
that your nominal devotion to the doctrine of Islam will also finally
wear off. To begin with, you will endure the rule of an inimical system
with disdain. Gradually, however, you will learn to live with it until
your contempt will change into a liking for this rule. Finally, it will
come to such a pass that you will serve as a pillar of support for the
establishment and maintenance of the State rule of an opposing ideology.
You will then expend your wealth and life in the struggle for the
installation and upholding the un-Islamic doctrines in place of the
ideology of Islam. Your own resources will be utilised in resisting the
establishment of Islamic ideology as ruling power in the State. At this
stage, no other difference except hypocritical professions of devotion
to Islam, an abominable falsehood and a deceitful title will distinguish
you from the infidels. The Holy Prophet (peace of Allah be upon him) has
clearly explained this fact in the Traditions:

\begin{quote}
"I swear by God Who has Power over my life, you shall have to enforce
good and crub {[}curb{]} evil and arrest the hand of the evil-doer and
turn it by force to do right or the inevitable consequences of the
natural law of God will be manifested in this fashion that the
intentions of the hearts of the evil-doers will influence your hearts
and like them you shall also be damned"
\end{quote}

\hypertarget{a-world-revolution}{%
\subsection{A WORLD REVOLUTION}\label{a-world-revolution}}

It must be evident to you from this discussion that the objective of the
Islamic ` Jihād' is to eliminate the rule of an un-Islamic system and
establish in its stead an Islamic system of state rule. Islam does not
intend to confine this revolution to a single state or a few countries;
the aim of Islam is to bring about a universal revolution. Although in
the initial stages it is incumbent upon members of the party of Islam to
carry out a revolution in the State system of the countries to which
they belong, but their ultimate objective is no other than to effect a
world revolution. No revolutionary ideology which champions the
principles of the welfare of humanity as a whole instead of upholding
national interests, can restrict its aims and objectives to the limits
of a country or a nation. The goal of such an all-embracing doctrine is
naturally bound to be world revolution. Truth cannot be confined within
geographical borders. Truth demands that whatever is right on this side
of the river or the mountain is also right on the other side of the
river or mountain; no portion of mankind should be deprived of the
Truth; wherever mankind is being subjected to repression, discrimination
and exploitation, it is the duty of the righteous to go to their
succour. The same conception has been enunciated by the Holy Qur'an in
the following words:

\begin{quote}
"What has happened to you? Why don't you

fight in the way of God in support of men, women and children, whom
finding helpless, they have repressed; and who pray, "O God! liberate us
from this habitation which is ruled by tyrants". (4: 75)
\end{quote}

Moreover, notwithstanding the national or country-wise divisions of
mankind, human relations and connexions have a universal significance so
that no state can put her ideology into full operation until the same
ideology comes into force in the neighbouring states. Hence it is
imperative for the Muslim Party for reasons of both general welfare of
humanity and self-defence that it should not rest content with
establishing the Islamic System of Government in one territory alone,
but to extend the sway of Islamic System all around as far as its
resources can carry it. The Muslim Party will inevitably extend
invitation to the citizens of other countries to embrace the faith which
holds promise of true salvation and genuine welfare for them. Even
otherwise also if the Muslim Party commands adequate resources it will
eliminate un-Islamic Governments and establish the power of Islamic
Government in their stead. It is the same policy which was executed by
the Holy Prophet (peace of Allah be upon him) and his successor
illustrious caliphs (may Allah be pleased with them). Arabia, where the
Muslim Party was founded, was the first country which was subjugated and
brought under the rule of Islam. Later the Holy Prophet (peace of Allah
be upon him) sent invitations to other surrounding states to accept the
faith and ideology of Islam. When the
ruling classes of those countries declined to accept this invitation to
adopt the true faith, the Prophet (peace of Allah be upon him) resolved
to take military action against them. The war of Tubuk was the first in
the series of military actions. When Hadrat Abu Bakr (may Allah be
pleased with him) assumed leadership of the Muslim Party after the
Prophet (peace of Allah be upon him) have had left for his heavenly
homes he launched an invasion of Rome and Iran, which were under the
dominance of un-Islamic Governments. Later, Hadrat `Umar (may Allah be
pleased with him) carried the war to a victorious end. The citizens of
Egypt, Syria, Rome and Iran initially took these military actions as
evidence of the imperialist policy of the Arab nation. They believed
that, like other nations, this nation had also set out on a course of
enslaving other nations under the yoke of imperialism. It was owing to
this misconception that they advanced under the banners of Caesar and
Khosros to give battle to the Muslims. But when they discovered the
revolutionary ideology of the Muslim Party; when it dawned on them that
Muslim armies were not the champions of aggressive nationalism that they
had no nationalistic objectives; that they had come with the sole object
of instituting a just system; that their real purpose was to annihilate
the tyrannical classes which had assumed divine powers and were
trampling down their subjects under the patronage of despotic Caesars,
kings, the moral sympathies of those downtrodden people turned towards
the party of Islam. They

began to forsake their allegiance to the flags of their own monarchs and
when they were conscripted by force and driven to fight against the
Muslims, they had no heart in the fight. This is the main cause of those
astounding victories won by the Muslims in the early period. It is on
this account also that after the establishment of Islamic governments in
their countries when they saw the social system of Islam in action, they
willingly joined this international party and became the upholders of
its ideology and set out to other countries to spread its message.

\hypertarget{the-terms-offensive-and-defensive-are-irrelevant}{%
\subsubsection{The Terms "Offensive" and "Defensive" are
Irrelevant}\label{the-terms-offensive-and-defensive-are-irrelevant}}

If you carefully consider the explanation given above you will readily
understand that the two terms `offensive' and `defensive' by which the
nature of welfare is differentiated are not at all applicable to Islamic
`Jihad'. These terms are relevant only in the context of wars between
nations and countries, for technically the terms `attack' and `defence'
can only be used with reference to a country or a nation. But when an
international party rises with a universal faith and ideology and
invites all peoples as human beings to embrace this faith and ideology
and admits into its fold as equal members men of all nationalities and
strives only to dismantle the rule of an opposing ideology and set up in
its place a system of government based on its own ideology, then in this
case the use of the technical terms like `offence' and `defence' is not
germane. Even if we stop thinking about these technical terms, the divi-

\hypertarget{jihux101d-in-islam-1}{%
\paragraph{26 JIHĀD IN ISLAM}\label{jihux101d-in-islam-1}}

sion of Islamic `Jihad' into offensive and defensive is not admissible.
Islamic Jihad is both offensive and defensive at one and the same time.
It is offensive because the Muslim Party assaults the rule of an
opposing ideology and it is defensive because the Muslim Party is
constrained to capture state power in order to arrest the principles of
Islam in space-time forces. As a party, it has no home to defend; it
upholds certain principles which it must protect. Similarly this party
does not attack the home of the opposing party, but launches an assault
on the principles of the opponent. The objective of this attack,
moreover, is not to coerce the opponent to relinquish his principles but
to abolish the government which sustains these principles.

\hypertarget{the-status-of-the-dhimmis}{%
\subsection{THE STATUS OF THE DHIMMIS}\label{the-status-of-the-dhimmis}}

\hypertarget{non-believers-under-the-protection-of-an-islamic-government}{%
\subsubsection{(Non-Believers) under the Protection of an Islamic
Government}\label{non-believers-under-the-protection-of-an-islamic-government}}

This also answers the question relating to the status of the votaries of
other faiths and ideologies when an Islamic government has been set up
in their countries. Islamic `Jihad' does not seek to interfere with the
faith, ideology, rituals of worship or social customs of the people. It
allows them perfect freedom of religious belief and permits them to act
according to their creed. However, Islamic `Jihad' does not recognize
their right to administer state affairs according to a system which, in
the view of Islam, is evil. Furthermore, Islamic `Jihad' also refuses to
admit their right to continue with such practices under an Islamic
government which fatally affect the public interest from the viewpoint
of Islam. For instance, as soon as the Ummah of Islam captures state
power it will ban all forms of business prosecuted on the basis of usury
or interest; it will not permit the practice of gambling; it will curb
all forms of business and financial dealings which are forbidden by
Islamic law; it will close down all dens of prostitution and other vices
and for all; it will make it obligatory for non-Muslim women to observe
the minimum standards of modesty in dress as required by Islamic law and
will forbid them to go about displaying their beauty like the days of
igno-

rance; the Muslim Party will clamp censorship on the Cinema. The Islamic
government with a view to securing general welfare of the public and for
reasons of self-defence will not permit such cultural activities as may
be permissible in non-Muslim creeds, but which, from the viewpoint of
Islam are corrosive of moral fibres and fatal. In this connection, if a
man feels inclined to level charges of intolerance at Islam, he should
consider that no creed in the world has shown more tolerance to the
votaries of other faiths as has been practised by Islam. In other
places, protagonists of another faith are so repressed that finding
existence unbearable they are constrained to emigrate from their homes.
But Islam provides full opportunity for self-advancement to the people
of other faiths under conditions of peace and tranquillity and displays
such magnanimity towards them that the world has yet to show a parallel
example.

\hypertarget{the-charge-of-imperialism}{%
\subsection{THE CHARGE OF IMPERIALISM}\label{the-charge-of-imperialism}}

At this point, I must reiterate that Islam regards only that war as
`Jihad' which is fought in the service of Allah---a war to fulfil the
Will of God. When an Islamic government is founded at the conclusion of
this war, the Muslims are categorically barred from assuming the
despotic powers which the old despots wielded upon the people. A Muslim
does not fight and as a Muslim he must not fight to establish a personal
rule and to turn the people of God into his own creatures and to build a
Paradise on earth for himself by expropriating the hard- earned wealth
of the people. This is not a war to fulfil the Will of God, but a war to
fulfil the will of devil; and Islam has no use for such a government.

The `Jihad' of Islam is a dry labour, devoid of pleasure. It is nothing
but a sacrifice of life, wealth and carnal desires. When this `Jihad' is
crowned with victory and an Islamic government is instituted, the
responsibilities of an honest and truly Muslim head of State are so
onerous that sleep during the night and ease during the day time both
are denied to him. But as a reward for these titanic labours he is not
entitled to indulge in pleasures which power and authority may call for
and for the sake of which bids are usually made in the world for
securing governments. A Muslim ruler is not a superior being, distinct
from or any more privileged than the

common man; he cannot sit on the throne of Exaltedness or Highness; he
cannot command any one to prostrate before him; he cannot execute the
slightest move without the sanction of Islamic law; he has no power to
shield any of his relatives, friends or himself against the lawful claim
of the most ordinary man in the community; he cannot take even the most
insignificant thing or even an inch of land from any one else without
justification and he is forbidden by law to draw half a penny more from
the public exchequer as his salary than is necessary for a Muslim of
average means to subsist. This God-conscious head of state cannot occupy
a magnificent palace, nor can he live with pomp and glory nor can he
procure means of pleasure and merriment. At all hours, he is seized with
the fear that one day he will be severely called to account for every
deed he commits in this world and if it is found that he received a
single penny as illicit gains, or snatched away the smallest patch of
land from any one by force, or displayed the slightest measure of pride
or haughtiness, or practised tyranny or injustice in a single instance
or succumbed even for one moment before carnal pleasures, he would be
condemned to endure the most dreadful torture. The world has not seen a
greater fool than the man who truly loves to gain the world and yet is
willing to carry the burden of state responsibility under Islamic law.
The worldly position of a small shopkeeper is far better than the ruler
of an Islamic State. He earns more during the day than the Caliph does
and enjoys a sound

sleep at night. The Caliph neither earns as much as he, nor enjoys peace
during the night.

This is the cardinal difference between the Islamic and un-Islamic
system of government. In an un-Islamic State, the ruling classes
establish themselves as divine powers and exploit the means and
resources of the country to their personal aggrandisement. In striking
contrast to this, the governing class in an Islamic State serves without
any thought of personal motives, and secures no greater personal
advantage for itself than is readily available to the common man.
Compare the scale of salaries granted to civil service cadres under the
government of Islam with the incomes received on account of salaries by
civil servants under modern imperialist governments or imperialist
powers which were contemporary of the Islamic State, you will soon
discover that there is an immeasurably vast difference both in spirit
and nature of the worldly conquests of Islam and the world-wide
dominance of imperialism.

In the Islamic State, the governors of Khurasan, Iraq, Syria and Egypt
were paid a lesser amount of money as salary than is drawn by a
low-grade Inspector today. The first Caliph Hadrat Abu Bakr Siddique
(may Allah be pleassed with him) ran the administration of such a vast
empire on a salary of Rs. 100/- per mensem. Hadrat `Umar's (may Allah
pleased with him) emoluments did not exceed Rs. 150/-per month,
notwithstanding the fact that the coffers of the Islamic State were full
with the wealth of the two empires of the known world. Although
seemingly,

imperialism conquers countries and so does Islam, yet between the two
there is an elemental difference which is equal to the space between
heaven and earth. Verse:

\begin{quote}
"Both fly in space, yet the world of the Eagle is far removed from the
Crow's".
\end{quote}

This, then, is the true meaning of `Jihad', a term about which you have
heard much. If you ask me now where is that Islam, the Muslim Party and
the `Jihad' whose ideology you have enunciated before us and why no
trace of these may be discovered today among the Muslims of the world, I
shall entreat you not to confront me with this question but ask it of
those who have deflected the attention of the Muslims from their real
mission to magical preparations like talismans, incantations,
superstitious rites and supererogatory offerings. Ask it of those who
prescribed short-cuts to salvation, reform and the attainment of the
objective, so that all this may be obtained by no more striving or hard
labour than is necessary for telling the beads or propitiating a soul
lying asleep in a grave. Ask it of those who wrapped up the tenets,
ideology and objectives of Islam and consigned them to the dark corners
and engaged the Muslim mind in the polemics over the most insignificant
things of Divine Faith or visits to the tombs or such other minor issues
with the consequence that the Muslim people lost all sense of their true
identity, the objective of their creation and the real character of
Islam? If they fail to deliver a satisfactory answer,

then put this question to the wealthy, the officials and the ruling
authorities who profess faith in Qur'an and the divine ministry of the
Holy Prophet (peace of Allah be upon him), but believe that they owe to
the Qur'anic injunctions and the guidance of the Holy Prophet (peace of
Allah be upon him) nothing more than holding assemblies for the
recitations of the Qur'an from corner to corner and calling meetings to
celebrate the birth of the Prophet (peace of Allah be upon him) or
sometimes praising God for the beauty of His verse (may God forgive them
and us!).

With regard to the enforcement of the Islamic law and the introduction
of Islamic reforms in practical polity, these gentlemen deem themselves
utterly free from any responsibility. For, as a matter of fact, the soul
of these gentlemen is unprepared to accept the restraints and sustain
the burden of duty imposed by Islam. They are suitors of a very easy
salvation.
