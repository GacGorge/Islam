% --------------------------------------------------------------
\chapter{Panorama de l'Islam dans le monde}

\mn{{Présentation générale. Panorama de l'islam dans le monde (17/01/2022)}}
\paragraph{Introduction : ce cours sera essentiellement sur le monde sunnite} essentiellement. Deux séances sur le Sh'iisme. 
\begin{Synthesis}
Pour le sunnisme, on peut parler d'éclatement de l'Islam, une véritable variété de l'Islam.
\end{Synthesis}
Avoir les clés des différents discours musulmans. 


\section{Aperçu géographique et démographique}

Lors de la bataille de \textit{Siffin} (657), séparation entre : 
\bi
\item Sunnites : 87,4\%
\item Chiites : 11,9\%
\item Kharijites : 0.7\% (Ibadisme en Oman)
\ei 

Le coeur de l'Islam se trouve là où les grands empires musulmans ont existé + Indonésie.


\subsection{les écoles Juridiques}
\begin{figure}
    \centering
    \sidecaption{Atlas de l'Islam dans le monde, Anne Laure Dupont, Autrement, 2005}
    \includegraphics[width=\textwidth]{CourantsIslamContemporain/ImagesCourantsIslamContemporain/CarteSunnisme.png}
 
    \label{fig:my_label}
\end{figure}



\bi 
\item  malékisme : Afrique Nord et ouest.
\item Chafiite : Est de l'Afrique et surtout Egypte 
\item Hanafite : Turquie, asie Centrale, Inde. \textit{les empires turques}
\item Hanbalite : surtout Arabie Saoudite, transformé en \textit{Wahhabisme} au XX, avec une extension au dela de l'Arabie Saoudite en 1960.
\ei 


\subsection{Trois grands courants dans le sh'isme}


\begin{Synthesis}[divergence en Si'isme : les courants]
Des désaccords sur Qui est Imam et sur la \textit{nature de l'Imam}. Il peut être investi de pouvoirs divins.
\end{Synthesis}

\bi 
\item Les duodécimains (ils reconnaissent 12 imams) : les shi'ites Iraniens
\item les Ismaeliens ou septimains (7 imams, l'Aga Khan), Liban.
\item zaydites (5 imams) : Yemen. Au niveau de la doctrine, ils reconnaissent peu de pouvoirs divins aux Imams (proches des Sunnites de ce point de vue).
\ei 
Les Ibadites sont Khajidites. Les druzes et les Alaouites (Alevi en Turc) sont issus du shi'isme (en se proclamant le Maadi) mais ne sont pas considérés comme musulmans par les autres musulmans.  A ne pas confondre à la famille Alaouites au Maroc qui sont tout à fait sunnite (Alaouite veut dire descendant d'Ali). L'Iran reconnait les Alaouites comme shi'ites pour des raisons politiques. 

\subsection{diversité culturelle}
L'islam s'est acculturé aux cultures qu'il a rencontré.
\paragraph{Mosquée}
Le seul élément architectural à une mosquée est la qibla qui indique la Mecque : le Mihrab.

\bi
\item la mosquée bleue a été constituée sur le plan de Sainte Sophie
\item Mosquée de Djenné.
\item La mosquée de Lagos : représente un style baroque brésilien.
\item Xian
\item la grande mosquée de Paris : sur l'image d'une mosquée marocaine
\item Mosquée contemporaine de Créteil

\ei 

\paragraph{L'habit féminin} D'après le Hadith "on ne doit montrer que les mains et le visage". 

\bi
\item les danseuses de cours à Surakarta
\item la burqa, avec grillage sur les yeux, asie centrale et Afghanistan dans certains milieux
\item des vétements de couleur et le shadri, vêtement traditionnel en Asie centrale paysanne voile que l'on met différemment selon qu'on est seul, ...
\item en Afrique, habit d'abord ethnique et non religieux. 
\item En France, les jeunes : le bandeau pour tenir le foulard et le foulard coloré. et certaines ne portent pas le foulard
\ei 

\paragraph{Pourquoi une impression d'uniformisation} La mondialisation crée l'uniformisation et certains courants wahhabites incitent à l'uniformisation sur l'influence saoudite.

\subsection{Observer l'islam}

 
\newlength\q
\setlength\q{\dimexpr .5\textwidth -2\tabcolsep}

\begin{table}[h!]
\sidecaption{\textit{Observer l'Islam} Clifford Geerzt, il a été au Maroc et en Indonésie}
%\begin{tabular}{p[7cm]p[7cm]}
\noindent\begin{tabular}{p{\q}p{\q}}
\toprule
\textbf{Maroc}                                                     & \textbf{Indonésie}                                               \\ 
\midrule
Tribale                                                            & Paysanne                                                         \\
\textit{Audace}                                                    & \textit{Application}                                             \\
une culture préalable moins riche                                  & L'islam est arrivé sur une civilisation hindouiste et bouddhiste \\
Un islam de dévotion aux saints (en particulier le saint guerrier) , austérité morale, pouvoirs magiques, piété agressive & malléable, syncrétique  \\
Uniformisation, facteur de civilisation & diversification, multiforme\\

\bottomrule
\end{tabular}
\end{table}

\subsection{Bibliographie}

\begin{quote}
\textbf{Atlas}

*DUPONT, Anne-Laure \emph{Atlas de l'islam dans le monde}, nouv. éd.,
Autrement, Paris, 2014. GUIDERE, Mathieu \emph{Atlas des pays arabes. De
la révolution aux démocraties}, Autrement, Paris,

2012.
\end{quote}

\hypertarget{ouvrages}{%
\subsection{Ouvrages}\label{ouvrages}}

\begin{quote}
BURESI, Pascal \emph{Géo-histoire de l'islam}, Belin, Paris, 2005.

GEERTZ, Clifford \emph{Observer l'islam : changements religieux au Maroc
et en Indonésie}, La Découverte, Paris, 1992.

*MERVIN, Sabrina \emph{Histoire de l'Islam. Doctrines et Fondements},
nouv. éd., Flammarion, 2016.

DE PLANHOL, Xavier \emph{Les nations du prophète}, \emph{manuel
géographique de politique musulmane}, Fayard, Paris, 1993.
\end{quote}