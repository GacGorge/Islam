\chapter{Matériaux bruts pour validation}


\section{Extrait site internet}
\begin{quote}
En Islam, le Riba est interdit : il s’agit de l’un des plus grands péchés, qui tire son origine du Coran. Mais il est également l’un des plus banalisés. Le Riba, appelé « usure », désigne l’intérêt perçu sur de l’argent prêté. De nos jours, le Riba est partout présent dans le domaine financier, à savoir, dans les comptes d’épargne, dans les prêts à intérêts, dans les agios… Alors, qu’est-ce que le Riba exactement ? Pourquoi est-il interdit ? Comment financer vos projets sans Riba 
\paragraph{Définition du Riba}

Le Riba peut être défini par intérêt - usure. Si le terme « usure » est la traduction la plus fréquemment donnée à cette interdiction de l’intérêt usuraire, il est important de préciser que le mot Ribâ vient cependant du verbe rabâ \& arbâ qui signifie augmenter et faire accroître une chose à partir d’elle-même. 

Dans le contexte financier, il est donc interdit de réaliser des transactions financières basées sur du Riba. Or, lorsque vous possédez un compte courant conventionnel, l’argent qui y transite alimente un circuit basé essentiellement sur l’usure. En effet, dans ce contexte, la banque dispose de vos fonds afin de spéculer. Même principe lorsque vous êtes à découvert : les agios que vous devez payer vous impliquent dans la pratique de l’usure.

Mais ce n’est pas tout : lorsque vous épargnez votre argent sur des comptes de type livret Jeune, Livret A, ou PEL, les intérêts que vous touchez sont également basés sur la pratique du Riba, et cela que vous fassiez don ou non des intérêts.

Enfin, lorsque vous réalisez un prêt bancaire, vous participez au développement de l’usure en payant des intérêts. 

\paragraph{De nombreux versets du Coran interdisent la pratique de l’usure.}     En voici quelques-uns, qui sont particulièrement explicites :

Verset 130 de la Sourate 3, Al-Imran : « Ô les croyants ! Ne pratiquez pas l’usure en multipliant démesurément votre capital. »
Les versets 278 et 279 de la Sourate 2, Al-Baqarah : (V-278) « Ô les croyants, craignez Allah ; et renoncez au reliquat de l’intérêt usuraire, si vous êtes croyants. (V-279) Et si vous ne le faites pas, alors vous recevrez l’annonce d’une guerre de la part d’Allah et de son prophète. Et si vous vous repentez, vous aurez vos capitaux. Vous ne lèserez personne, et vous ne serez point lésés. »
D’après Abu Hurayra, le Prophète (SAWS) a dit : « Evitez les 7 turpitudes ». Lorsque les compagnons demandèrent : « Quelles sont-elles, Ô envoyé d’Allah ? », il mentionna l’usure aux côtés notamment du polythéisme, du meurtre…
D’après Sahih Muslim : Ubâdah Ibn As-Sâmit rapporte que le Messager d’Allah (SAWS) a dit : « Or pour Or, argent pour argent, blé pour blé, orge pour orge, datte pour datte, sel pour sel, de manière égale, de main en main. Si la nature des produits diffère, vendez comme vous le voulez, si c’est de main en main ». \sn{\href{https://firstunion.fr/quest-ce-que-le-riba/}{First Union} ? }
\end{quote}

 
 \paragraph{Abd ar-Rahman ibn Nasir as-Sadi}\href{https://fr.wikipedia.org/wiki/Abd_ar-Rahman_ibn_Nasir_as-Sadi}{Abd ar-Rahman ibn Nasir as-Sadi} Oulema Hanbalite saoudien 1889. influencé par 	
Ibn Taymiyya
Mohammed ibn Abdel Wahhab
 
 \paragraph{ Le ribâ (l’usure) dans l’Islam}
 
 \href{https://www.ajib.fr/le-riba-lusure-dans-lislam/}{Le riba est l'usure dans l'Islam}
 
 \begin{quote}
    


Pour comprendre le sens réel du ribâ, pour connaître son statut légal dans l’islâm, houkm, qui est bien sûr, l’interdiction, ainsi que les méfaits individuels et collectifs qu’il engendre, il faut comprendre quelques points importants.

Parcourons ensemble ces extraite du livre « Le résumé des vertus de la religion musulmane » du grand savant \textit{Abdarrahman As-Sacdî} qui nous éclaireront sur la vue de la législation islamique, \emph{shariah}, sur les transactions, licites et illicites, ainsi que la sagesse du Législateur, Allah, soubhânah, en établissant cette auguste législation.

Nous verrons en particulier, les problèmes qui concernent le ribâ.

\begin{quote}
    
Que signifie le ribâ ?

Le ribâ signifie l’usure. Si l’on entend par « usure » le prêt à taux supérieur à zéro.

Car la définition contemporaine de « l’usure » est « l’intérêt excessif » ou « l’intérêt supérieur au taux légal », tandis que le ribâ est le prêt à intérêt, aussi minime soit-il.

Le prêt islamique légal est donc le prêt à zéro intérêt. Et le ribâ est l’usure dont l’intérêt est supérieur à zéro.

Il y a plusieurs types de Ribâ :
\begin{itemize}
    \item 1/Le prêt à intérêt : le surplus, aussi minime soit-il, perçu, pour le prêt, par le créancier en échange du délai accordé.

 \item 2/Ribâ al-fadhl : le surplus perçu lors de l’échange d’un bien ribawî (relatif au ribâ) tel que l’échange de l’or avec l’or, de l’argent avec l’argent, du blé avec le blé… etc.

 \item 3/Ribâ an-nasî’ah : différer l’encaissement lors d’un échange d’un bien ribawî de même nature, tel que l’or avec l’or, ou de nature différente, mais dont la cause ribawique est commune, tel que l’or avec l’argent.
\end{itemize}


\paragraph{Quelles sont les transactions dites légales dans le \emph{shariah}, halal ?}

La vente rendue licite par la législation islamique ainsi que le louage[1], les sociétés et les différentes transactions ; celles où s’échangent les marchandises entre les gens, les dettes, les utilités… etc.

\paragraph{Pourquoi une transaction est légale dans le \emph{shariah}, dite : halal ?}

La législation parfaite a permis ce type de transaction et l’a autorisé aux hommes, car cela assure leurs intérêts – nécessaires, indispensables ou complémentaires -.

Elle a élargi ainsi considérablement la liberté des hommes afin que leurs affaires et leurs conditions se réforment et que leurs vies s’ordonnent.

Quelles sont les conditions de la légalité d’une transaction ?

\textbf{La législation, \emph{shariah}, posa comme conditions pour que les transactions soient licites :

1-La satisfaction des deux parties ;

2-la clarté du contrat ;

3-la connaissance de l’objet du contrat, de la durée du contrat et des conditions qui en résultent.}



Quelles sont les transactions interdites dans la \emph{shariah}, dites : haram ?

La législation, \emph{shariah}, a interdit tout ce qui comporte un préjudice ou une injustice tel que :

1-Les différents types de jeu de hasard,

2-le Ribâ ;

3- l’ignorance[2], la jahâlah.
\end{quote}
 

Cette division, licite et illicite, halal et haram, prescrite par Allah et révélée à Son Prophète Mouhammed, ainsi qu’à tous les Prophètes avant lui, a pour objectif essentiel de préserver le but premier escompté par ces opérations d’échange entre les humains : l’intérêt mutuel, l’avantage commun à toutes les parties de l’échange et le bénéfice pour tous.

Inversement, tout ce qui n’apporte pas d’intérêt commun et ne fait pas bénéficier toutes les parties qui opèrent la transaction est illicite, haram et interdit.

Car cela va à l’encontre de la raison de cette opération d’échange, la transaction, qui est l’intérêt commun et l’absence de préjudice.

\textbf{Le ribâ fait partie de cette catégorie car son intérêt va dans un sens unique, qui est celui de l’enrichissement du créancier, et il porte un préjudice considérable au débiteur.}

Ceci s’appelle : injustice, iniquité et celui qui en profite ne fait que manger l’argent et les biens des autres injustement.
 \end{quote}
 
 
 \paragraph{Dictionnaire du Qoran}
 

 \section{Demystifying Riba through the Methodology of Muslim Jurists}
 \href{https://www.proquest.com/docview/2352353188?accountid=143046&parentSessionId=Sg7rHgjzHS0M9aF4pv1Nx2pxOOyR5LkKvILKrOVqG1o\%3D&pq-origsite=summon}{Demystifying Riba through the Methodology of Muslim Jurists}\sn{MUHAMMAD ZAHID SIDDIQUE; MUHAMMAD MUSHTAQ AHMAD.
Islamic Studies; Islamabad Vol. 58, N° 2,  (Jun 30, 2019): 169.}
 

 \subparagraph{Summary}
  \begin{quote}
  In the post-colonial world when Muslims tried to restructure their public life in accordance with the shari‘ah, they developed a new discipline known as Islamic economics one of the central constructs of which is prohibition of riba. Unfortunately, the discussion among modern academic circles assumed a wrong methodology, which resulted in mystification of this concept and, hence, in a number of unsettling questions. This paper explains the nature of the mistake committed by modern Muslim scholars and economists. It also outlines the structure of correct methodology, which was laid down by premodern Muslim jurists for understanding the concept of riba and all other legal terms. The paper develops a consistent analytical framework for addressing majority of the questions on the subject of riba and attempts to rectify the mystification created around this concept.
  
  \end{quote}
 \subparagraph{Texte intégral}
 \begin{quote}

Keywords: riba, interest, loan, bay‘, Islamic economics, Islamic legal theory, financial regulations of Islam.

1. Introduction

After losing their political rule to the imperial powers, Muslim societies faced the widespread dominance of interest-based banking system. According to the majority of Muslim scholars and jurists, bank interest (riba) was not allowed, but Muslim societies got engaged in it due to growing spread of interest-based banking in modern societies and the non-availability of interest-free banking. \mn{noter aussi que le développement de la banque s'est fait en parallèle dans les sociétés occidentale. }

Muslim scholars and economists demanded its alternative soon after Muslims got independence from their foreign masters. Commitment to follow religious teachings in the public affairs of life and liberty from the colonial oppressors provided the required room, which resulted in what is now known as Islamic economics in general and Islamic finance/banking in particular.

One of the central concepts of Islamic banking is prohibition of riba, which unfortunately and surprisingly remained controversial among Muslim economists and scholars. Different perspectives about the meaning of riba prevailed in the twentieth century. Majority view holds that both usury and bank interest are equally impermissible in Islam while business profit is allowed.\sn{1 For  detailed  arguments  of  this  position,  see  Ab┴ ’l-A‘la  Maud┴di,  S┴d  (Lahore:  Islamic Publications,  2000),  110–12;  M.  Umer  Chapra,  “The  Nature  of  Riba  in  Islam,”  Hamdard Islamicus  7, no.  1 (1984):  3–24;  Muhammad  Shafi‘, Mas’alah-i  S┴d  (Karachi: Idarat  al-Ma‘arif, 1996), 43–47; Muhammad Ayub, “What  is Riba? A  Rejoinder” Journal of Islamic  Banking and Finance  13,  no.  1 (1996):  7–24;  Muhammad  Taqi Usmani,  The  Historic  Judgment  on Interest Delivered in the Supreme Court of Pakistan (Karachi: Idarat al-Ma‘arif, 1999), 12–16; Mohammad Nejatullah  Siddiqi,  Riba,  Bank  Interest  and  the  Rationale  of  Its  Prohibition  (Jeddah:  Islamic Research  and  Training  Institute, 2004),  45–48;  and  Mahmoud A.  El-Gamal,  Islamic  Finance: Law,  Economics,  and Practice  (Cambridge:  Cambridge University  Press, 2006),  46–52. Within this  category,  there  are  further  two  approaches.  One  approach  that  represents  traditional ‘ulama’ emphasises the resurgence  of only those business contracts that were approved by  the early  Muslim  jurists.  It proposes  profit-and-loss  sharing (PLS)  as an  ideal alternative  to riba. Though it does not deny the permissibility of other than PLS-based financing instruments such as murabahah and ijarah, yet it affirms that equity-based financing method is the primary means of achieving desirable economic objectives. The second approach is pragmatic one. It justifies a more liberal and flexible stance on structuring shari‘ah-compatible transaction forms that looks for financial engineering to meet all demands of modern banking customer.  }

Contrary to the majority view, some modern Muslim scholars dispute that the Qur'anic term riba includes interest paid and charged in the banking system.\sn{ Muhammad Rashid Rida (d. 1935) was among the foremost proponents of this theory. See his al-Riba  wa ’l-Mu‘amalat  fi ’l-Islam  (Cairo: Dar  al-Manar, 2007).  Also see  Sayyid  Yaqub Shah, “Islam  and Productive  Credit,” The  Islamic  Review  47,  no.  3 (1959):  34–37; Fazlur  Rahman, “Riba  and  Interest,”  Islamic Studies  3, no.  1 (1964):  1–43; Timur  Kuran, “On  the Notion  of Economic  Justice  in  Contemporary  Islamic  Thought,”  International  Journal  of  Middle  East Studies  21,  no. 2 (1989): 171–91;  Izzud-Din Pal,  “Pakistan and  the Question  of Riba,”  Middle Eastern Studies 30, no. 1 (1994): 64–78; and ‘Abd al-Karim Athari, S┴d Kiya Hay? (Mandi Baha’ al-Din: Anjuman-i Isha‘at-i Islam, 2008), 8–12}
To them, replacing bank interest with anything else is tantamount to obstructing natural operation of economy and creating inefficiencies because interest is the just reward of capital reflecting its marginal productivity. \sn{ Constant  J.  Mews  and  Ibrahim  Abraham,  “Usury  and  Just  Compensation:  Religious  and Financial Ethics in Historical Perspective,” Journal of Business Ethics 72, no. 1 (2007): 1–15}

3 According to this perspective, there is no need to have anything distinct like “Islamic banking” to begin with because the existing system is already Islamic.


Finally, on the other extreme are \textit{Muslim socialists}\sn{A l'autre extrême ?} who develop their version of Islamic economics based on socialist policy package.

4 Since socialism considers wage as the only legitimate reward of a factor input, the scope of riba is much wider than usury and bank interest according to these scholars. 

It is believed by some that rental earnings on an asset is also included in riba because rent is similar to interest earnings as both are the prices of capital determined by similar market forces. Others are of the view that not only bank interest but also trade or merchant profit is banned under the category of riba.


6 They argue that as lender is forbidden the right to charge interest from poor borrower, so should be the rich industrialists and landlords from appropriating lion's share of value-added on the name of profits.

They assert that loaning riba (riba 'l-qard) covers money lenders and hoarders who charge against time while riba of excess (riba 'l-fadl) is the domain of landlords, merchants, and middlemen who exploit poor workers and make unequal exchanges. 

These differing perspectives are shown in figure 1. Because this last perspective about riba has gained very little popularity among Muslim scholars and masses as compared to the first two, we exclude its analysis from the scope of this paper, though it would be analysed indirectly.

Spectrum of Riba and Its Scope
\
The above differences have left scholars divided on several important questions that demand straightforward answers. Those questions include the following ones:

Spectrum of Riba and Its Scope Liberals Mainstream Muslim Socialists Riba = Usury Riba = Usury, Riba = Usury, (exploitative Interest (all returns Interest, Rent, return on loan) on loan) Profiteering

1. Is bank interest prohibited in the light of the Qur'an and the sunnah? If yes, how?

2. Whether the Qur'anic term riba includes all kinds of interest rates or it relates only to the excessive interest rates?

3. Whether the scope of riba extends to the interest charged and paid on business transactions in the banking system or is restricted to the interest charged on consumption loans only?

4. Does Islam allow loan transactions? If yes, how and in what form?

5. Is paying interest a lesser evil as compared to charging it?

6. Is borrower always mazlum (a losing party) in an interest bearing loan transaction?

7. Does Islam allow indexation of loans on the grounds of inflation?

8. Is credit-sale with higher deferred price as compared to the spot price allowed?

9. Does Islam approve of “time value of money,” especially when charging higher deferred price is allowed in a credit sale?

10. Are future currency contracts permissible in Islam?

11. How and to what extent is salam transaction permissible?

These are but a few questions. We show in this paper that whatever confusion prevails among contemporary scholars on this subject is the outcome of following an inadequate methodology for determining the meaning and scope of riba. In fact, this methodology has mystified the nature of riba, which is otherwise clear when viewed from the methodological view point of the eminent Muslim jurists of the past. The mystification is such that not only it results in confusing answers to these questions but it also begets confusing questions. Unfortunately, the confusion has built up to the extent that the Federal Shariat Court of Pakistan has been struggling to come up with a definition of riba. It is in this background that this paper attempts to explain:

(1) the contemporary Islamic economists' methodology of interpreting and classifying riba; (2) why this methodology is wrong and insufficient; (3) the methodology of understanding riba on the pattern of Muslim jurists of the past; (4) that the methodology given by the Muslim jurists is coherent and compact.

The reader will encounter a number of arguments in this paper that are advanced by those who justify bank interest. Since the paper deals with the legal substance and not with the economic merits of arguments, hence we will restrict ourselves to the legal analysis of those arguments and leave aside their economic analysis and rationale, which require an altogether different methodology. Any legal system has three aspects: (1) what: the legal rulings (i.e., ahkam); (2) how: the rules of deriving those legal rulings (i.e., usul al-fiqh); and (3) why: the underlying rationale(s) and wisdom behind the legal rulings (i.e., hikmah)

It is important not to mix these aspects. The present study deals with the first two aspects of the issue of riba. Moreover, the classification of riba discussed in this paper is primarily based on the methodology of Hanafi jurists for ensuring analytical consistency. We presume that a school of law represents an internally coherent system of interpretation and that mixing up the views of the various schools results in inconsistencies.7 However, views of the other schools have been briefly mentioned in the footnotes wherever required. Finally, the paper does not attempt to show that the Hanafi jurists' approach is superior to all others, rather it explains that the classical jurists' approach (whether Hanafi, Maliki, Shafi‘i or Hanbali) to understanding riba is superior to that of the modern scholars. The methodology of these jurists share several common results that are important in order to answer the above questions.

Following section outlines the method adopted by modern Muslim scholars and economists. The next section discusses problems in this methodology and develops the skeleton for the methodology that is then applied in the coming section, which details out the general rules of riba alongside their resulting implications. The last section concludes the paper by giving a comprehensive definition of riba based on discussions in sections three and four.

\textbf{2. Outline of the Mystifying Methodology}

Imran Ahsan Khan Nyazee explains that the methodology adopted by modern scholars for determining the meaning of riba is the same, though they disagree in their conclusion regarding whether or not bank interest is riba.8 The fundamental problem of their methodology lies in overlooking the inherent link between the Qur'an and sunnah. This methodology of interpreting riba was initiated by Muhammad Rashid Rida (d. 1935) \sn{voir p. \pageref{Theol:Rida} }, which goes as follows:9

Riba is classified into two categories, riba of the Qur'an (also equated with riba 'l-nasi'ah, i.e., interest on loan transaction) and riba of hadith (equated with riba 'l-fadl, i.e., interest on exchange transaction).

Rida begins with literal meaning of the word riba (excess) and then traces some riba-based transactions practiced by Arabs during the time of Prophet (peace be on him). Rida, relying on some commentators of the Qur'an, asserts that the Qur'anic verse regarding riba deals with a specific practice of Arabs known as credit-sale where the payment of price is deferred to a future period while delivery of goods takes place on spot. Because a seller is allowed to charge whatever price he wants in a sale transaction, no riba is involved in the original price negotiated between the two parties-any excess in future price becomes part of the price. However, they used to increase the price excessively whenever the debtor would be unable to settle his debt obligations at the end of payment period. The debtor was given the option, “Will you pay the debt or increase the amount in lieu of delay?”

For Rida, it was this excessive rate (doubling and multiplying) of interest in debt-based transactions added to the original sum at the end of payment period which was prohibited by the Qur'an (he called it riba 'l-jahiliyyah).10 From this, he concluded that the bank interest is not the same riba that was deemed impermissible by the Qur'an because (a) it is neither doubling and redoubling of rates (b) nor the excess is stipulated in the initial period of the banking transaction-he assumes that the initially added interest is part of the principal or original sum just like the original sum in case of credit-sale. Hence, for Rida, only compound interest is prohibited.

Other scholars, supporting Rida's view, added that business loans were not common among Arabs as theirs was a subsistence economy; loans were largely taken by poor people for consumption purposes on interest and whenever they were unable to repay them at due time, excessive interests were added to the original sum. Hence, it was this type of interest that was declared prohibited by the Qur'an and it has nothing to do with the modern commercial loans, which are mutually beneficial for both parties.11

Having ascribed this meaning to the Qur'anic word riba on the basis of some historical traces, Rida then explains the form of riba declared impermissible in the sunnah as a distinct prohibition from that of the Qur'an. He calls it riba 'l-fadl which emerges in the exchange of two counter values of the same or different species and hence also called riba 'l-buyu‘.12 The position of Rida, which may be termed as minority view, is summarised in figure 2.

Thus, Rida dichotomised the two concepts of riba, one attributed to the Qur'an and another to the sunnah. He finally declared the first one as real or explicit riba while latter as lighter or implicit riba.

Though the majority of contemporary scholars did not agree with the conclusion drawn by Rida about legitimacy of bank interest, however they adopted his methodology of classifying riba. The only difference in their opinion is that riba of the Qur'an includes all rates of return on loan and it is not merely restricted to the compound interest of jahiliyyah. To them, business loans were a part of Arab's economy and any contractual return to lender is unfair because this is tantamount to refusing to share business risk with the borrower. We can depict their views in figure three.

Because the sunnah is not linked with the Qur'an in this methodology, both the minority and majority Muslim economists have struggled to explain as to why someone would engage in exchange transactions of the forms mentioned in hadith. Some opined that these transactions are declared impermissible because they may open the path for the “real riba” (i.e., riba of the Qur'an).14 Others assumed that it was meant to discourage the practice of barter exchange and promote market exchange through a medium of exchange.15 Yet another view argues that it eliminates the possibility of benefiting from asymmetric information of the contracting parties.16 The truth is that none of these explanations makes the point.

2.1. The Nature of Debate within Minority and Majority Schools

The debate that has taken place within the followers of this mystifying methodology on the issue of why or why not bank interest is riba may briefly be summarised here. As explained above, Rida asserted that bank interest was not included in the Qur'anic concept of riba of debt because it was different from the riba that was charged by Arabs on credit-sale transaction by doubling and multiplying the price whenever the debtor was unable to settle his debt at due time and asked for relaxation in payment period.18 Rida explained that the Qur'anic verse “Allah has permitted bay‘ and prohibited riba”19 referred to this riba. To strengthen his case, he argued from the verse: “O Believers! Do not devour riba doubled and multiplied and fear God so that you may prosper.”20

This verse complements the former verse in the sense that what was implicit in the first verse was made explicit in the latter-both verses referred to the practice of doubling and multiplying of interest and none of them forbad the bank interest.

How do the majority of scholars respond to this argument? For example Usmani notes the Qur'anic verse:

O you believers! Fear God and give up riba that remains outstanding if you are true believers. Behold! If you do not obey this commandment, then God declares war against you from Himself and from His Prophet. But, if you repent (from riba), then you are entitled to only your principal amounts. Neither should you inflict harm to others, nor others should do harm to you.21

The argument is based on the emphasised words ‘you are entitled to only your principal amounts (ra's al-mal)'. He infers from these words that the rightful entitlement of lenders is the original sum advanced; he cannot charge any increase whether small or large (doubled and tripled). To him, the verse (3:130) forbids a severe form of riba where interest is multiplied, but it does not restrict riba to this specific form. Hence, bank interest falls within the purview of the Qur'anic verse “Allah has permitted bay‘ and prohibited riba.”22 They are also of the view that charging interest on commercial loans was also practiced by Arabs.23

Does the above analysis of mainstream scholars guarantee the prohibition of bank interest? We are afraid it does not. Their arguments rest on two assumptions:

(1) The verses (2:278-79) address the issue of loan-transaction.

(2) Ra's al-mal (principal amount) can only refer to the original principal advanced in loan.

Both of these assumptions are problematic. Following submissions can be made against them:

(a) If the meaning of the verse is to be determined with reference to historical practices, one can equally claim, just like Rida, that the verse is not about loan transaction but about credit sale. In that case, ra's al-mal is not referring to the principal amount lent; rather, it is the deferred future price of the goods sold. On which legal grounds or facts can this claim be dismissed?

(b) Further, this deferred price might include increase over and above spot price. Hence, the future price could consist of two components: spot price plus some additional profit. The sum of these two would constitute ra's al-mal (principal amount) in this transaction (i.e., principal amount in credit sale (ra's al-mal) = spot price + extra profit)

Whenever a debtor was unable to repay full amount, further multiplied increase was added to this original sum, Rida called it interest. This would increase the due amount to: total amount after increase added due to delay, which is interest in addition to ra's al-mal.

Using this structure, one can then argue that the initially added interest in a loan transaction is equivalent to initially added “extra profit,” which becomes part of ra's al-mal. Therefore, entitlement to the ra's al-mal means entitlement to the simple interest, as claimed by Rida.

(a) The only legal justification for ascertaining that the verse is about loan transaction is based on the words “ra's al-mal” (principal amount). But how can it be settled that ra's al-mal here means ra's al-mal of a loan transaction? This question is important because a number of transactions constitute a component of ra's al-mal. For example, there is ra's al-mal both in mudarabah and musharakah contracts. How to exclude these forms of ra's al-mal from the purview of the Qur'anic verse? If someone says, “This verse is about loan, so ra's al-mal refers to that of loan contract and not of mudarabah and musharakah,” he is clearly arguing in circularity. The argument goes like this:

Q: How do we know that the verse is about loan contract?

A: Because the verse talks about ra's al-mal.

Q: How do we know that ra's al-mal here refers to that of loan?

A: Because the verse is about loan contract!

A circular argument is no argument.

(b) Muslim Socialists could maintain that ra's al-mal means principal amount of all business contracts. Therefore, it is not legitimate to charge any excess over and above principal amount, no matter it is mudarabah, musharakah or ijarah.

Not only that the analysis of mainstream scholars does not necessarily imply the prohibition of bank interest, it leads to a set of unsettling arguments that have left Islamic economists bewildering about some basic issues. For example,

(1) Even if it is agreed that ra's al-mal means principal amount of a loan transaction, does it mean “nominal” amount or the “real” (inflation adjusted) amount? Again, what are the legal grounds to settle this issue? Because there are no clear-cut legal grounds available in this methodology, we see scholars are divided on this subject matter-some allow indexation of loan against inflations while others do not.

(2) What about the question of “time value of money?” This question poses challenge for Islamic economists because they, as a rule, approve the practice of charging higher price in credit-sale and murabahah.

(3) The Lawgiver has allowed salam, what are the legal grounds for not extending this permission to currency salam (future currency contracts)?

Undoubtedly, majority view has addressed these issues, but the answers do not seem to be stemming out of a coherent analytical legal system. This approach is often found mixing up legal analysis with economic analysis. This missing coherent analytical legal system is the root cause of most of the mystification that has prevailed all over. It is an unfortunate state of affairs and it is high time to demystify things.

3. Methodological Assumptions of Premodern Muslim Jurists (Fuqaha') for Understanding Riba

To understand the method used by the eminent premodern Muslim jurists for understanding riba, three methodological issues (MI) need to be clarified. They are explained by Nyazee in detail.24

1) Link between the Qur'an and Sunnah

The methodology adopted by the modern Muslim scholars and economists is misleading because it delinks the Qur'an and sunnah. It assumes that the meaning of riba is different in the Qur'an and sunnah, which is not the case. To explain the nature of error made by both the groups, it should be noted that Muslim jurists (fuqaha') classified riba in the category known as mujmal (unelaborated)25 whose meaning and scope cannot be determined without explanation (bayan) of the Lawgiver (Shari‘). The famous hadith (as given in footnote 1) that explains different usurious transactions actually does not add something to the Qur'anic word riba. Rather, it defines its meaning and scope.

Thus, while to the contemporary scholars the meaning of riba is known independent of hadith and they see hadith as adding some more cases to the Qur'anic concept of riba, the jurists say that hadith is the definition of the term riba used by the Qur'an. Thus, riba 'l-nasi'ah and riba 'l-fadl both are included in the Qur'anic concept of riba.

2) Relationship between Loan and Bay‘ (Exchange)

Loan is also classified as a form of exchange transaction (bay‘)26 by Muslim jurists. The scope of this paper does not allow detailed analysis of this assertion.27 For descriptive purposes, it can be seen that a loan of Rs X is an exchange of Rs X today with Rs X after time deferment (and with Rs X + Y if interest payment of Rs Y is included). Figure 4 depicts this nature of loan transaction by illustrating a loan transaction between Mr. A and B:

3) Skeleton of a Coherent Legal System

A coherent shari‘ah-based legal system consists of a set of general rules, called

‘azimah by the jurists, supplemented by some exemptions to these laws, called rukhsah. In the words of Nyazee, ‘Azimah (lit. determination, resolution) is applied to mean a rule that is applied initially and for itself. Such rules form the backbone of the law. As against this, there may be a rule that goes contrary to the requirements of the initial rule, but is permitted by the law. This rule is considered to be a rukhsah (exemption) from the initial rule.28

This classification of ‘azimah (the higher or first order rules) and rukhsah (the lower or second order rules) is important for several reasons.

First, it explains the order in which the rules have to be applied.

Second, it explains why sometimes two opposing cases may be allowed within a given skeleton of law.

Third, the order of rules implies that an exception cannot be extended using any method of argument, whether analytical or analogical. On the other hand, extension of first order rules is legitimate by these methods. In other words, it is not allowed to build a sub-legal system based on exemptions because otherwise it starts negating the primary provisions and objectives of the law-an exemption from the general rule must remain an exemption.

Fourth, because of the logical hierarchy in the operations of ‘azimah and rukhsah, it is clear that an exemption from a rule cannot be used to nullify or change the shari‘ah status (hukm) of any other case that is derived from the general rules. Alternatively put, an exemption (a lower order rule) cannot prevail over the higher order rules.

Fifth, because all rules and exemptions are derived from nusus (the Qur'an and sunnah), hence the only justifiable exemptions are the ones, which are given in nusus (i.e., stated by the Lawgiver Himself).

We call these nusus the “facts” of the shari‘ah-based legal system in this paper. Given these “legal facts,” the task of a jurist is to derive those general rules (‘azimah) from the facts, which render these facts internally consistent and extendible on the one hand and highlight the exemptions (rukhsah), if any, on the other.29 Finally, the general rules and exemptions generate some implications, called ahkam. This skeleton of a shari‘ah-based legal system is illustrated in Figure 5. We apply this skeleton in this paper to elaborate riba.

The relevant “legal facts” used by premodern Muslim jurists to derive general rules and exemptions are quoted at the relevant places in this article. We are now in a position to take on the issue of derivation of the general rules and the implications from those “facts.”

4. Underlying Rules behind the System of Bay‘ in Jurists' Methodology

Our intention in this paper is to reveal that the apparently large and complicated system of legal injunctions (ahkam) is reducible to a few set of rules derived from fewer legal facts. We propose that a majority of ahkam (legal injunctions or provisions) governing economic transactions (buyu‘) can be derived from three broad rules:

1. Rules of riba mentioned in the sunnah. This is not a single rule, rather a set of rules as explained below.

2. Rule about the sale of goods not possessed by a person.

3. Rule about exemption that an exemption is to be treated as exemption.

Before explaining these rules, we first explain the context of the Qur'anic verses that underlies the jurists' methodology of riba to clarify the misconception that the relevant verses of Surat al-Baqarah are about loan transaction and not exchange (bay‘).

4.1. The Context of the Verses of Surat al-Baqarah

The Qur'an states that the disbelievers said, “Verily, bay‘ (sale) is just like riba.” In response to this, it was said, “Allah has permitted bay‘ and prohibited riba.” To understand why the disbelievers said this, consider these three transactions:

(a) A gives B 100 grams of gold in exchange of 110 grams of gold to be paid after one year. This is primarily a sale contract as explained previously (i.e., exchange of 100 grams gold with 110 grams gold with time lag) and involves riba (how, this will be explained in the next section but take it for granted for the moment).

(b) A asks B for 100 grams of gold in exchange of, say, 500 kg wheat at spot. This is a legitimate regular sale contract.

(c) A demands from B 110 grams of gold in exchange of 500 kg wheat for payment of price after one year: this is credit sale contract with higher deferred price as compared to spot price and is also legitimate (this is explained in section 4.3).

The credit sale was a common practice among Arabs and, therefore, they were confused as to why the transaction (a) is impermissible and (c) is permissible while the two are quite similar in nature (i.e., both are credit sales and both involve access payment). In (a), 10 grams of additional gold are paid as counter-value for 100 grams of gold for a delay of one year and similarly 10 grams of gold are paid for a delay of one year in transaction (c). It is for this reason that disbelievers said, “Verily sale is just like riba!” That is, transaction (c) (i.e., the credit sale) is similar to the transaction (a). The technical reason for allowing transaction (c) and forbidding (a) is the similarity of genus which is explained by the sunnah. This is explained in the next sections in detail, but the important point to note here is that the assumption that the Qur'anic term riba is not about sale contract, rather it is about debt, is not implied by these verses.

Thus, the verse says that Allah has approved all forms of buyu‘ (exchange transactions) except those which involve riba.30 The natural question then arises: what is this thing called riba? Has the Qur'an given any definitive description of riba?

One may make one of the two assumptions here. First, the concept of riba was largely a sort of common knowledge for everyone and, hence, it required no legal description by the Qur'an. That common knowledge is traceable by an examination of historical record of Arabs which provides sufficient legal foundations for determining the meaning of riba. As far as the details of riba in the sunnah are concerned, they were additions over and above to that common knowledge of riba and most of these additions were unknown to the Arabs. The liberals and mainstream scholars share this assumption and we believe that this assumption constitutes what we called the “mystifying methodology.”

Second, some forms of riba may be or actually known to the Arabs but these do not set the legal standard against which the Qur'anic concept of riba is to be determined. As it is a legal term, its meaning has to be sought from the Lawgiver. In technical sense, the jurists call it mujmal (unelaborated) for which elaboration (bayan) is sought from the Lawgiver. This elaboration of the legal meaning of the Qur'anic term riba is given by the sunnah. After this elaboration by the Lawgiver, its meaning is determined definitively and it becomes mufassar (elaborated). This is the methodological assumption that the jurists use not only for defining riba but also for other legal terms of the Qur'an, such as salah, zakah, and so on.31 Thus, according to this second assumption, the practices and concepts of Arabs may be referred by the Qur'anic concept riba but it is not the benchmark against which we assign legal meaning to the Qur'anic terms.

For example, the Arabs had some concepts about how to offer salah (prayer), but this information does not define the legal meaning of the Qur'anic term salah nor is this concept limited to this information set. Similar is the case with riba. The Arabs might have been aware of some forms and practices of riba but that does not constitute the legal definition of riba. When the jurists classify a term as mujmal, they mean that this term is a technical legal term and its meaning should be determined with reference to the words of Lawgiver Himself, neither by the linguistics (dictionary) nor by the historically known social concepts and practices that hover around that technical term. It should be emphasised here that considering riba as mujmal does not mean that the Arabs did not know the meaning of this word at all. Nor does it mean that the pre-Islam Arabs did not identify certain transactions as riba-in fact they did and the jurists did consider it part of riba.32

It only means that the meaning of riba in Islamic law is not limited to, and is not based on its usage in the pre-Islam Arabia. The Qur'an and the sunnah added several shades of meaning to this concept. That is why, it became a “technical term” of Islamic law. Hence, its meaning and scope cannot be determined by its dictionary meaning or its practice and understanding by the pre-Islam Arabs. Rather, it must be determined by the Qur'an and the sunnah, like any other legal term such as salah and zakah. Just as we cannot classify concept salah as salah of the Qur'an and salah of hadith, similarly we cannot dichotomise riba. Once it is established that the meaning of riba must not be gathered from pre-Islamic usage and practices but from the Qur'an and the sunnah, the next question is: how to explain the various usages of riba in the Qur'an and the sunnah? The answer, as per the well-established methodology of the jurists, is to consider the sunnah as the elaboration of the mujmal verses of the Qur'an.

This methodology is employed by the jurists for determining the meaning and scope of salah and zakah as well as riba. Let's follow through the path of righteous ones here and have its blessings.

4.2. General Rules of Riba When Transacted Species are Same

Keeping these in mind, one has to understand the classification of riba in the system of Muslim jurists. Because the sunnah defines riba, note the words of hadith,

When you exchange gold for gold, silver for silver, wheat for wheat, rice for rice, dates for dates, and barely for barely, then exchange like for like (in equal measure) and exchange them hand to hand (at spot), else it will be riba.33

To understand what it says, consider these transactions:

1) exchange of 1 gram gold for 1 gram gold on spot;

2) exchange of 1 gram gold for 2 grams gold on spot;

3) exchange of 1 gram gold at spot for 1 gram gold with delay;

4) exchange of 1 gram gold at spot for 2 grams gold with delay.34

As per the hadith, the first transaction is allowed; the second one is disallowed because it involves excess in measurement/quantity (called riba 'l-fadl); the third transaction is also impermissible because the hadith says that the exchange of homogeneous goods is allowed in equal measurement provided it is on spot; therefore, this transaction involves the riba 'l-nasi'ah (i.e., riba of delaying); finally, the fourth transaction involves both types of riba. These transactions provide two guiding rules (R):

R 1.1) Goods of the same species cannot be exchanged immediately unless their measurement (in terms of weight or volume) is same.

R 1.2) Goods of the same species cannot be exchanged with time lag, even with same measurement.

4.2.1. Implications

Five important implications (I) should be noted.

I. 1) Impermissibility of Market for Loanable Funds

Application of rule 1.2 gives the important implication that loan, with or without interest, is prohibited in Islam because, as explained above, a loan is an exchange of homogeneous goods with time lag. Does it mean that loaning is not allowed in Islam under any circumstances? Of course, this implication of the general rule is at odd with a number of legal facts (nusus), which promise reward for offering loan to the needy ones. How to reconcile these apparently contradictory legal facts now? This is where the concept of rukhsah (exemption) is activated by the jurists. Though loaning is against the general rule (‘azimah) given by the Lawgiver, yet it is allowed by Him as an exemption from this prohibition if it takes the form of benevolent giving (tabarru‘ or sadaqah).35 Loan is classified as tabarru‘ if:

(a) it is out of the intention of benevolence to the other person (i.e., the lender consciously bestows upon the borrower the benefits associated with his asset);36

(b) no increase in its value is stipulated, else it would cease to be benevolent and would involve riba 'l-fadl; and

(c) no contractual time limit is stipulated, the lender can ask for his asset anytime he wants.37 Stipulating (legal) time constraint in loaning activity makes it a business transaction as per the application of general rules of shari‘ah and, hence, unlawful because in that case it is simply the exchange of homogeneous goods with time delay, which is not allowed, whether or not interest factor is included. Moreover, making the time period binding would imply that the lender is forced to do, or to continue with, an act of charity. This is against the very nature of charity.

In short, this principle implies that Islamic law does not permit the “market for loanable funds.” It sees loaning as an act of benevolence, especially in favour of one's relatives.38 Stated alternatively, loan is purely a social transaction (a means of tying and strengthening social bonds) in Islam and not a business. It was in this social transaction capacity that the institution of loan prevailed for thousands of centuries not only in Muslim societies but also in other civilisations of the world until the emergence of capitalism in the fifteenth century.39 Note that this important result (impermissibility of the market for loanable funds) does not follow directly from the classification of modern scholars of Islamic economics, as the majority view allows interest-free non-benevolent loans as a general rule and not as an exemption.

This implication answers one of the important arguments in favour of bank interest given by some economists. The argument says that interest should be allowed in shari‘ah because interest is the price of capital and without interest the market for loanable funds cannot be equilibrated. Because we are not dealing with the economic merit of arguments in this paper, we ignore its economic substance and comment on its legal merit only. It is clear from the above implication now that this argument has no shari‘ah basis because shari‘ah does not allow market for loanable funds to begin with, let alone equilibrating it from shari‘ah perspective.

Before moving on to the next implication, the important implication and exemption regarding loan transactions be noted:

I 1.1) A loan transaction is prohibited, whether or not interest factor is added to it.

I 1.2) A benevolent interest-free loan is recommended as an exemption to the general rules of riba by the Lawgiver.

I. 2) Impermissibility of Bank Interest

All forms of bank interest, whether simple or compound, are prohibited by Islam as per Rules 1.1 and 1.2. Similarly, the fact whether loan is made for business or consumption purposes makes no difference to this result. There remains no confusion about these conclusions if the shari‘ah rules are applied with consistency. In fact, the practice of charging interest by the bank includes both kinds of riba and it, therefore, may be stated that it is the most comprehensive form of riba! This can be verified from the figure 6, which depicts detailed structure of riba-based transactions in case of homogenous goods (leaves aside heterogeneous goods for the moment).

I. 3) False Dichotomy between “Giving and Taking” Riba

The recipient of riba is not always the lending party as is usually perceived. It can be seen from above examples that in case of transaction (2) the lender is the beneficiary of riba, but in transaction (3) riba is received by the borrower, and finally both are its recipients in transaction (4). Hence, opinions such as “taking riba is a greater evil than giving it and, hence, paying interest to the bank is a lesser evil” are based on the fallacious assumption that it is only the bank that receives interest in a typical interest-bearing loan transaction. This wrong assumption is the outcome of using the wrong methodology outlined in section two.40

I. 4) Mutually Beneficial Riba is Prohibited

The view that bank interest realised in transaction (4) is or should be permitted (as claimed by liberal Muslim scholars) is implicitly based on the assumption that “two wrongs make one right”-that is, it assumes that mutually enjoyed riba of the lender and borrower can make this transaction acceptable while the matter of fact is that each of them is separately prohibited to begin with.

I. 5) Irrelevance of Time Value of Money

Following the wrong methodology has resulted in another confusing argument that the bank interest should be allowed because of “time value of money.” This argument is based on the presumption that Rs. 1 today is worthier than Rs. 1 tomorrow. Why? Economists believe that this is due to the subjective time preferences of an individual. A rational (i.e., self-interested utility maximising) economic agent is said to have positive time preferences in the sense that consumption today is preferred to consumption tomorrow because the latter is uncertain, which makes him impatient, thus he wants to have it today than tomorrow.

Another reason for having this positive time preference emerges from the institutional arrangements: if I have the option of earning some interest (say Rs. Y) on Rs. 1 by putting it in a bank account today, why should I lend it to someone for free? Putting Rs. 1 in a bank account will make it “Rs. 1 + Rs. Y” for sure (assuming away bank insolvency), say, after one year while lending it to someone will leave it worth Rs. 1. Hence, Rs. Y (which may be expressed in percentage) is the price that should be paid to the lender for a loan of one year, else it would be unfair with him. This argument is more of economic than legal in its substance, however, some comments can be made here to evaluate its legal substance in the light of preceding discussion.

The relevant part of the proposed argument is the second one (the institutional arrangement) because the first one is merely a subjective feeling, which may differ from person to person (as a matter of fact, not everyone prefers to consume more today than tomorrow). The argument presumes that there exists and should exist a well-established legally functional market for loan, which coordinates interest-based loan transactions. But just recall “I.1” that Islam does not approve of the market for loan to begin with. Eliminate this institution of market for loan, and the argument disappears. The point is that the concept of “time value of money” conceived in this economic sense is alien to the discussion of riba. Its validity presumes that there exists a legal institutional market for loanable funds where money is growing continually and, therefore, an individual always has the option of putting his money in that market.

Not only that this assumption is invalid from the point of general rules of shari‘ah as explained, it is also in contradiction with the ontological structure of the universe and economic facts.

The above is not the only format of this argument, it is phrased in some other shades as well. For example, it is stated that money could buy benefits and had the lender not lent it he could have benefitted himself. This implies that lending is an act of sacrificing the benefits associated with money. Therefore, the lender should be compensated for this sacrifice and interest payment is exactly that reward. This reward makes sense given that the borrower takes benefit out of money. The argument is valid to the point that money is beneficial to the lender and that if he makes the choice of not lending it, he can benefit from it. Moreover, it is also true that the borrower enjoys the benefits associated with the money. None of these facts is denied by the shari‘ah rules. But these facts alone cannot formulate the required case for this argument; it requires a moral statement in its premise to derive the desired conclusion.

To see this, note that the argument does not end here, after quoting these facts it then makes a moral assertion: “it is morally (and hence legally) right if money is lent for reciprocal benefits.” Addition of this moral statement is necessary for validating the conclusion that “interest is the just reward for lending.” But this moral assumption contradicts the general rules of the shari‘ah, which are laid down above. Seeking reciprocity in loan is exactly what that changes its status from tabarru‘ to loan as a business transaction and, hence, it becomes nothing but riba. The argument here is quite straightforward:

The owner of money is granted the right of benefitting from his money by shari‘ah rules; he is given the option of making a conscious choice of transferring the benefits associated with his money to another person as an exception to the general rules by the shari‘ah, but there is neither any general rule nor any exemption from the Lawgiver that assigns him the right of lending money in the name of the so-called “mutual benefits” (refer to I. 4 above). Legally speaking, this involves both riba 'l-fadl (because the homogeneous goods are exchanged at different rates) and riba 'l-nasi'ah (because time stipulation is invoked-the lender asks for the excess of measurement for parting with the benefits of his money for a specified time).

Another variant of this argument comes with the heading of “effects of inflation on money.” We deal with it in the next section.

4.3. General Rules of Riba when Transacted Species are Different

What about the exchange of heterogeneous goods? The last words of the hadith are as follows: “If these species differ, then exchange as you like as long as it is from hands to hand.”

They give an immediate rule:

R 1.3) Goods of the different species can be exchanged with difference in measurement.

This rule says that such goods can be exchanged at different rates as far as measurement is concerned. In other words, riba 'l-fadl does not apply in case of heterogeneous goods. Is riba 'l-nasi'ah (prohibition of time delay in payment) also not applicable in this case? Apparently, it seems that it is not because of the words of hadith, “exchange should be on spot.” This has an odd implication that credit sale (sale of goods against money where payment is deferred to future time period) is not permissible under the shari‘ah rules. This is so because credit-sale is an exchange of heterogeneous goods with time lag. But the legal facts reveal that the Lawgiver has allowed credit-sale.41 How to explain this? Is credit sale also an exemption to the general rule, like a loan transaction? The answer is: “No, it falls within the general rules.”

To see how credit-sale is permissible within general rules, one needs to dig deep into the issue of the underlying cause (‘illah) that the Muslim jurists derived from the sunnah to understand the system of riba. The relevant question facing jurists was: Is prohibition of riba restricted only to the six goods named in the hadith or is it extendible to other goods? The answer of the jurists is, yes, it is extendible and for this extension they derived the underlying cause due to which riba was declared prohibited by the Lawgiver. Keeping aside the technical details and arguments, it should be noted that some of the goods are measured in terms of weight while others are measured in terms of volume. In the hadith under discussion, gold and silver were weighable while the other four items were volumeable at the time of Prophet (peace be on him).42 Based on this classification, the jurists derived two further rules:

R 1.4) when species are different but their method of estimation is the same (such as gold vs silver or wheat vs rice), unequal quantities can be exchanged, provided that the exchange is immediate;

R 1.5) when species are different and their method of estimation is also different (such as gold vs wheat), unequal quantities can be exchanged with time delay.43

Thus, the credit sale is allowed due to the application of Rule 1.5. To see this, consider these combinations of transactions:

1) Exchange of 1 gram gold at spot for 2 gram silver on spot (method of estimation same)

2) Exchange of 1 gram gold at spot for 2 gram silver in future (method of estimation same)

3) Exchange of 2 kg wheat at spot for 1 gram gold/silver on spot (method of estimation different)

4) Exchange of 2 kg wheat at spot for 1 gram gold/silver in future (method of estimation different)

The first transaction is allowed but the second is not because when species are measured by same method (i.e., “weight” in this case), then difference in the measurement (fadl) is allowed but deferment (nasi'ah) is not permissible. The third and the fourth transactions are allowed because here not only the transacted species are different but also their method of measurement (one was measured in “weight” while the other in “volume”).

In short, when both of the similarity factors (i.e., species and method of measurement) are found, then both fadl (excess of measurement) as well as nasi'ah (excess of time delay or time deferment) are prohibited. When similarity of measurement is found alone, then fadl is allowed but nasi'ah is prohibited. Finally, when none is found, both fadl and nasi'ah are allowed. Figure 7 depicts all of these rules completely (discussion about the last layer of boxes on the right-hand side of this figure is coming next).

The preceding discussion shows that the hadith explaining the nature of riba was not about the actual practices of Arabs that begged some economic explanations with which Muslim scholars have been struggling. Rather, it stipulated the rules of exchange. It says, “If at all you make exchange transactions, here are the governing rules.” Thus, all transactions that correspond to these general rules are allowed while those in contradiction with them are prohibited (however, some are exempted by the Lawgiver).

4.3.1. Implications

Following implications are derived from the above rules. It is important to note that the first two transactions mentioned in sub-section 4.2 belong to the case when method of estimation of the heterogeneous goods is same while the latter two cover the cases when their method of estimation is different.

I. 6) Placement of Regular and Credit-Sale

Transaction (3) is categorised as regular sale transaction (usually termed bay‘) by the jurists. On the other hand, transaction (4) covers credit sale, which may take two forms: with or without extra profit margin as compared to the spot sale. Because both measurement as well as payment time differential are allowed in this case, hence credit sale of both forms is allowed.

I. 7) Placement of Currency Exchange

The remaining two boxes are relating to the exchange of currencies (termed as bay‘ al-sarf by the jurists). A detailed description of these requires an appreciation of some more technical classifications44 made by the Muslim jurists. However, they are beyond the scope of this paper. Suffice to say that the jurists divided all tradeable species into two: (a) currency items, which are used as means of exchange; they included gold and silver (though other goods may also be treated as currency in this system) and (b) non-currency items, (goods that are exchanged, and are not medium of exchange). They roughly included all but gold and silver.45 Given this division, the jurists broadly mention four types of transactions (buyu‘):

(1) Non-currency item in exchange of non-currency item-called barter exchange.

(2) Spot or delayed currency (say gold) in exchange of spot non-currency (say wheat) item.

(a) If both of them (gold and wheat) are exchanged on spot, it is called regular sale of goods, and

(b) if the currency price (gold) is delayed, this is called credit sale.

(3) Delayed non-currency item (say rice) in exchange of spot currency item (say gold). Here, the price of the good is paid on spot while its delivery is delayed. This is called bay‘ al-salam (advance payment) by the jurists.

(4) One currency (gold) in exchange of another currency (silver)-known as bay‘ al-sarf.

Rules regarding the first two have been discussed above. Here, we have to make some submissions regarding this fourth type of transaction. Because this transaction comes under the umbrella of “different species with same method of measurement,” it is clear from figure 7 that the excess of measurement is allowed in this transaction while time deferment is not. This gives two further rules under rule (1.4):

1.4a) If different currency items (such as gold and silver) are exchanged, then it is allowed to exchange them at any rate;

1.4b) if different currency items (such as gold and silver) are exchanged, then it is not allowed to exchange them with time deferment.

If it is accepted that modern currencies are just substitutes of gold and silver, then two further important results emerge from this discussion:

I 7.1) Future Currency Contracts are Prohibited

Rules (1.4a) and (1.4b) imply that the spot currency transactions are allowed while their future contracts (known as currency salam in Islamic finance literature) are prohibited in Islam as they come under the purview of riba 'l-nasi'ah.

I 7.2) Indexing of Loans is Prohibited

Indexing the value of the currency loans against some underlying assets (say gold) on the ground of inflationary pressures is not allowed. It is often argued that since the value of currency decreases over time due to the presence of inflation, hence an extra-payment equal to the rate of inflation, over and above the original sum given in loan, should be allowed in favour of the lender to keep his purchasing power. Again, because the economic merit of this argument is beyond the scope of discussion in this paper, we restrict only to its legal merit. If it is accepted that one rupee is legally nothing but equivalent of 1 unit of gold or silver (whatever that unit be), then Rules 1.1 and 1.2 (governing the loaning contract in gold or silver currencies) should automatically become operational.

Those rules imply that (a) loaning in the form of currency item is allowed if and only if equal measurement (whatever the unit of measurement) is returned; else it would be riba 'l-fadl; and (b) it is a loan made out of benevolence and not business intention (having time stipulation); else it would be riba 'l-nasi'ah. Hence, adding an extra amount to loan transaction in the name of “indexation” is but both, riba 'l-fadl (because of the excess of measurement) and riba 'l-nasi'ah (because the increase is time bound). 46 Again, let simplicity and sanity prevail.

I. 8) Placement of Salam

To see how the jurists accommodated salam in this scheme, note that there is nothing in the set of rules 1 (from 1.1. to 1.5) which forbids it. However, according to rule 2 (given at the start of this section), selling what one does not possess is not permissible and this is exactly what a salam transaction involves. Thus, a salam transaction should not be allowed as per the general rules of shari‘ah. We are once again faced with the same issue: salam is permitted in the “legal facts;” how and where to place it in the legal skeleton of the shari‘ah? Is there another general rule, which governs its permission as we saw in case of credit sale or is it an exemption from the general rule just like loan? The jurists' answer is the following: Salam is permitted as rukhsah-exemption from the general rules-by the Lawgiver.47

Because it is an exception, as per rule 3, it would be allowed only as “one of its kind” (sui generis) and cannot be used as justificatory mode for deriving more comparable transaction forms (e.g., currency salam). An exception to the general rule remains exception and does not turn into a rule for other cases because then it ceases to be an exception and creates a situation of self-contradictory general rules, which is not acceptable in any legal system. Thus, salam transaction is allowed as an exception for those transactions where (a) a currency item is exchanged against a non-currency item and (b) non-currency item is deferred while the currency-item has been paid at spot.48 This is what the exception is all about; one cannot extend this exception to the transaction types where currency items are exchanged with each other because that would violate condition (a) of the exception case.49

Figure 8 shows a map of interplay among legal facts (nusus), general rules, exemptions, and the derived implications related to riba and bay‘ that are discussed in this paper. This diagram shows that a rather complex looking system of ahkam (implications) showing up at the ending layer boxes of figure 8 emerge out of a set of general rules, which are derived to make underlying legal facts compatible with each other.

5. Conclusion: The Definition of Riba

We conclude this paper by elaborating a comprehensive definition of riba that can be inferred from the discussions in this paper. Let's quote it from al-Sarakhsi:50

Riba in its literal meaning is excess... and in the technical sense (in the shari‘ah), riba is the stipulated excess without a counter-value in bay‘ (sale).51

Let's explain it noting several points about this definition:

(1) Muslim jurists do not introduce the word loan in the definition of riba because they categorise loan transaction under exchange (bay‘). Not appreciating this point resulted in the misconception that since the fiqh conception of riba does not deal with the subject of bank loans, it needs to be inferred directly from the Qur'an.

(2) Riba is excess, either in the form of quantity (qadr) or in the form of benefits of delay (nasa'). The first is called riba 'l-fadl while the latter is called riba 'l-nasi'ah.

(3) This excess is without any counter-value permitted by the shari‘ah. Thus, the excess of quantity paid in lieu of time delay in case of interest-bearing loan is not allowed because these two cannot be the legitimate counter-values (see I. 4).52 For a substance to be counted as counter-value, it must be recognised by the general rules of the shari‘ah to begin with.53

(4) The excess is stipulated in exchange. If the excess is granted voluntarily, it would not be riba.

We started off with specific questions in the introduction. The appendix lists down the answers to these questions in the light of the above definition of riba. It can be seen that once the discussion about riba is placed on the right track, right and clear cut answers start emerging automatically.

Appendix: Questions and their Answers that Follow from the above Analysis

Notes

1 For detailed arguments of this position, see Abu 'l-A‘la Maududi, Sud (Lahore: Islamic Publications, 2000), 110-12; M. Umer Chapra, “The Nature of Riba in Islam,” Hamdard Islamicus 7, no. 1 (1984): 3-24; Muhammad Shafi‘, Mas'alah-i Sud (Karachi: Idarat al-Ma‘arif, 1996), 43-47; Muhammad Ayub, “What is Riba? A Rejoinder” Journal of Islamic Banking and Finance 13, no. 1 (1996): 7-24; Muhammad Taqi Usmani, The Historic Judgment on Interest Delivered in the Supreme Court of Pakistan (Karachi: Idarat al-Ma‘arif, 1999), 12-16; Mohammad Nejatullah Siddiqi, Riba, Bank Interest and the Rationale of Its Prohibition (Jeddah: Islamic Research and Training Institute, 2004), 45-48; and Mahmoud A. El-Gamal, Islamic Finance: Law, Economics, and Practice (Cambridge: Cambridge University Press, 2006), 46-52. Within this category, there are further two approaches.

One approach that represents traditional ‘ulama' emphasises the resurgence of only those business contracts that were approved by the early Muslim jurists. It proposes profit-and-loss sharing (PLS) as an ideal alternative to riba. Though it does not deny the permissibility of other than PLS-based financing instruments such as murabahah and ijarah, yet it affirms that equity-based financing method is the primary means of achieving desirable economic objectives. The second approach is pragmatic one. It justifies a more liberal and flexible stance on structuring shari‘ah-compatible transaction forms that looks for financial engineering to meet all demands of modern banking customer.

2 Muhammad Rashid Rida (d. 1935) was among the foremost proponents of this theory. See his al-Riba wa 'l-Mu‘amalat fi 'l-Islam (Cairo: Dar al-Manar, 2007). Also see Sayyid Yaqub Shah, “Islam and Productive Credit,” The Islamic Review 47, no. 3 (1959): 34-37; Fazlur Rahman, “Riba and Interest,” Islamic Studies 3, no. 1 (1964): 1-43; Timur Kuran, “On the Notion of Economic Justice in Contemporary Islamic Thought,” International Journal of Middle East Studies 21, no. 2 (1989): 171-91; Izzud-Din Pal, “Pakistan and the Question of Riba,” Middle Eastern Studies 30, no. 1 (1994): 64-78; and ‘Abd al-Karim Athari, Sud Kiya Hay? (Mandi Baha' al-Din: Anjuman-i Isha‘at-i Islam, 2008), 8-12 3 Constant J. Mews and Ibrahim Abraham, “Usury and Just Compensation: Religious and Financial Ethics in Historical Perspective,” Journal of Business Ethics 72, no. 1 (2007): 1-15.

4 See Ghulam Ahmad Parvaiz, Nizam-i Rububiyyat (Lahore: Idara-i Tulu‘-i Islam, 1978).

5 Rafi‘ Allah Shihab, Kirayah-i Makanat ki Shar‘i Haithiyyat (Lahore: Kitab Ghar, 1981).

6 Ziaul Haque, “The Nature and Significance of the Midieval and Modern Interpretations of Riba,” The Pakistan Development Review 32, no. 4 (1993): 933-46.

7 For details, see Imran Ahsan Khan Nyazee, Theories of Islamic Law: The Methodology of Ijtihad (Islamabad: Islamic Research Institute, 1994), 9-12.

8 Nyazee, The Concept of Riba and Islamic Banking (Islamabad: Institute of Advanced Legal Studies, 1995), 11-19. Imran Ahsan Khan Nyazee (b. 1945) is a well-known scholar and a prolific writer on the subject of Islamic law and is a former Professor of law in International Islamic University, Islamabad. His major works include Theories of Islamic Law; Islamic Jurisprudence; Islamic Law of Business Organization; and The Concept of Riba and Islamic Banking. He also translated some of the classical texts on Islamic law and jurisprudence, including: Hidayah of Marghinani; Bidayat al-Mujtahid of Ibn Rushd; Amwal of Abu ‘Ubayd; and first two volumes of Muwafaqat of Shatibi.

9 Rida, al-Riba wa 'l-Mu‘amalat fi 'l-Islam, 69ff.

10 Period before the advent of the Prophet (peace be on him) is referred to as jahiliyyah (i.e., the period of uncivilised state of affairs).

11 Fazlur Rahman, “Riba and Interest,” 7-8.

12 In this regard, a hadith reads, “The Prophet said, ‘While exchanging gold for gold, silver for silver, wheat for wheat, barley for barley, dates for dates, and salt for salt, exchange like for like, in equal measure, and exchange from hand to hand. If these species differ, then sell as you like as long as it is from hand to hand.'” Muslim b. al-Hajjaj, Sahih, Kitab al-musaqah, Bab al-sarf wa bay‘ al-dhahab bi 'l-wariq naqdan.

13 Adopted from Nyazee, Concept of Riba.

14 Maududi, Sud, 118-19.

15 Chapra, “Nature of Riba in Islam,” 3.

16 Siddiqi, Riba, Bank Interest and the Rationale of Its Prohibition, 49-50.

17 Adopted from Nyazee, Concept of Riba.

18 Rida, al-Riba wa 'l-Mu‘amalat fi 'l-Islam, 69-70.

19 Qur'an 2:275.

20 Ibid., 3:130.

21 Ibid., 2:278-79.

22 Ibid., 2:275.

23 For details, see Shafi‘, Mas'alah-i Sud, 106-120 and Siddiqi, Riba, Bank Interest and the Rationale of Its Prohibition, 38-40.

24 Nyazee, Concept of Riba, 35-36.

25 Mujmal is a term used by Muslim jurists to refer to a Qur'anic term that begs its explanation through the words of Lawgiver (i.e., God and His Prophet [peace be on him]). One cannot interpret mujmal either by looking its meaning in the dictionary nor can its meaning be determined through historical practices at the time of revelation of the Qur'an. Mujmal can be elaborated only by the Lawgiver. Another example of mujmal is the Qur'anic term salah (prayer) which cannot be interpreted literally.

26 Bay‘ means exchange of counter values, and is not restricted to sale of goods/services. Abu Bakr b. Mas‘ud al-Kasani (d. 587/1191), the illustrious Hanafi jurist, defines bay‘ as “exchange of property with property” and then elaborates that the concept includes not only ordinary sale but also barter, exchange of currencies, advance payment and many other forms of exchange. Bada'i‘ al-Sana'i‘ fi Tartib al-Shara'i‘, ed. ‘Ali al-Mu‘awwad and ‘Adil ‘Abd al-Mawjud (Beirut: Dar al-Kutub al-‘Ilmiyyah, 1997), 6:532-33. Abu 'l-Hasan ‘Ali b. Abi Bakr al-Marghinani (d. 593/1197), author of the authoritative Hanafi manual al-Hidayah, also explicitly asserts that qard (loan) begins as an act of charity but becomes an exchange transaction in the end. al-Hidayah fi Sharh Bidayat al-Mubtadi (Beirut: Dar Ihya' al-Turath al-‘Arabi, n.d.), 3:60

27 See Nyazee, Concept of Riba, 45-46.

28 Ibid., 49.

29 The Hanafis use the methodology of istihsan (juristic preference) for ensuring harmony and analytical consistency within the law when general rules and legal facts seem to contradict. If something appears prohibited in the light of the general principles of law, but has been explicitly permitted by one of the texts (i.e., legal facts), the Hanafi jurists take the position that it is permissible as an exception to the general principle. They use the rule, “prohibited under qiyas but permissible under istihsan” for this purpose. Exceptions to the general principles are made on the basis of the text, consensus, necessity or some other “covered principle” (qiyas khafi), which needs to be uncovered. Muhammad b. Abi Sahl al-Sarakhsi is worth quoting here: “This [istihsan] is the evidence coming in conflict with that apparent principle (qiyas zahiri), which comes into view without one's having looked deep into the matter.

Upon a closer inspection of the rule and the resembling principles, it becomes clear that the evidence that is conflicting with this apparent principle is stronger and it is obligatory to follow it. The one who chooses the stronger of the two evidences cannot be said to be following his own personal caprices.” Muhammad b. Abi Sahl al-Sarakhsi, Tamhid al-Fusul fi 'l-Usul, ed. Abu 'l-Wafa' al-Afghani (Beirut: Dar al-Kutub al-‘Ilmiyyah, 1993), 2:200-202. Another important point made by al-Sarakhsi is that when the jurist uses istihsan and prefers the stronger rule, he abandons the weaker one and as such it is not permissible for him or his followers to follow the latter. He goes on explaining that when istihsan is carried out on the basis of a concealed or covered principle (qiyas khafi), the established rule does not amount to be an exception but becomes a general principle in itself.

Interestingly, not only the Hanafi jurists but also the Maliki jurists explicitly employ the principle of istihsan for resolving the apparent anomaly found in the legal facts where one set of nusus prohibits a loan transaction and another set of nusus allows it. They hold that it is prohibited as an exchange transaction but allowed as an act of charity.

30 Al-Sarakhsi interprets this verse as the following: “Trade is of two kinds: permitted (halal), which is called bay‘ in the law; and prohibited (haram), which is called riba. Both are types of trade. Allah informs us, through the denial of the disbelievers, about the rational difference between sale (bay‘) and riba, and says, ‘That is because they said, “Sale is like riba.”' He, then, distinguishes between prohibition and permission by saying, ‘And Allah has permitted sale and prohibited riba.' Through this, we came to know that each one of these is trade, but only one form is permitted.” Al-Sarakhsi, al-Mabsut, ed. Hasan Isma‘il al-Shafi‘i (Beirut: Dar al-Kutub al-‘Ilmiyyah, 1997), 12:1-2.

31 The famous Hanafi jurist Abu Bakr al-Jassas al-Razi (d. 370/980) says, “In the law (shari‘ah), it (riba) is applied to meanings in which it was not used in the language. This is indicated by the fact that the Prophet (peace be on him) termed nasa' as riba in the tradition of Usamah b. Zayd (God be pleased with him). He said, ‘Verily, riba is in nasi'ah.' ‘Umar b. al-Khattab, (God be pleased with him) said that riba had different forms and out of these salam in teeth, that is, in animals, is not concealed. ‘Umar also said that the verse of riba was one of the last to be revealed, and the Prophet (peace be on him) was taken away before he could elaborate the details for us. Therefore, give up riba and the suspicion of riba. It is established from this that riba became a technical term, for had it been governed by its original meaning in the language, it would not have been obscure for ‘Umar, who was fully aware of the names used in the language, being a native speaker.

This (the conversion of the word into a technical meaning) is also indicated by the fact that the Arabs were not aware of the sale of gold for gold and silver for silver with a delay (nasa') as riba, but this is riba in the technical meaning. If this (meaning of riba) is as we have explained it, then, it became like all the other unelaborated (mujmal) words that are in need of an elaboration (bayan). These are terms that have been transferred from the language to the law and assigned meanings to which the word was not originally applied in the language, like salah, sawm, and zakah. Such words are in need of a bayan and it is not proper to employ them in legal reasoning for the prohibition of any of the contracts, unless an evidence has been adduced to show that such a meaning is employed by the law.

The Prophet (peace be on him) elaborated on many occasions the intention of Allah in a verse, by way of an explicit statement or in response to a query (tawqif), and through these he has indicated the evidence (dalil). The (legal) meanings are, therefore, not lost to those who have knowledge when they employ legal reasoning.... In the technical sense, the word riba is assigned several meanings. The first is the one that was prevalent among the people of the jahiliyyah. The second is excess in the same species out of things measured and weighed, according to the view expressed by our (Hanafi) jurists.... The third is nasa' (delay), which is of several types.” Ahmad b. ‘Ali al-Razi al-Jassas, ed. Muhammad al-Sadiq al-Qamhawi, Ahkam al Qur'an (Beirut: Dar Ihya' al-Turath al-‘Arabi, 1992), 2:183-84. Al-Sarakhsi is also worth quoting here:

“Mujmal is the word the meaning of which is not understandable except by asking the one who used this word.... An example of mujmal is the saying of the Almighty: “He prohibited riba” as riba literally means excess but we know that this is not meant here because sale has been permitted for the purpose of excess. Rather, riba here means prohibition of a sale due to an excess without a counter-value stipulated in the contract; and this excess is either in the form of increase in measure or by way of delay.... It is obvious that this elaboration is not known by literal analysis. Rather, it needs a separate source. Hence, it is mujmal with respect to its intended meaning. The same is the case of salah and zakah. They are also mujmal because their original literal meaning is prayer and growth, but because of their use in specific legal acts, their intended meaning cannot be gathered from their literal analysis.” al-Sarakhsi, Tamhid al-Fusul fi 'l-Usul, 1:168-69.

32 See al-Jassas, Ahkam al-Qur'an, 2:183-84.

33 Muslim, Sahih, Kitab al-buyu‘, Bab bay‘ al-ta‘am bi Mithlih.

34 One can simply substitute “Rs.” for “gram gold” in these transactions if Rs. (currency) is treated as substitute of gold and silver currency.

35 The famous Hanafi manual Hidayah explains the position of a loan transaction in the following words: “It is an act of charity in the beginning and that is why it is not valid from a person who does not have the capacity to do charity, such as a minor or a guardian (of a minor). However, at the end, it becomes a contract of exchange because it turns into exchange of dirhams with dirhams with delay, and that is riba.” See al-Marghinani, 3:60. The commentators explain, “This necessitates invalidity of loan but the shari‘ah has recommended it and the whole ummah agrees on its validity; hence, we hold that it is valid but not binding (and can be terminated at will by any party).” Akmal al-Din Muhammad b. Mahmud al-Babarti, al-‘Inayah sharh ‘ala al-hidayah (Bulaq: al-Matba‘ah al-Kubra al-Amiriyyah, 1316 AH), 5:273. The same position is upheld by Maliki jurists.

Thus, the famous Andalusian Maliki jurist Abu Ishaq al-Shatibi (d. 790/1388) says, “There are many examples of istihsan in the law, such as loan, which is riba in reality because it is exchange of dirham with dirham with delay; but it has been permitted because it benefits and facilitates the needy.” Ibrahim b. Musa al-Shatibi, al-Muwafaqat fi Usul al-Shari‘ah, ed. Abu ‘Ubaydah Mashhur b. Hasan (al-Khobar: Dar Ibn ‘Affan, 1997), 5:194-95.

36 Jurists apply the rules of ‘ariyah (commodate-loan) on these transactions because it is the nearest match for qard and the only way to legally justify a qard transaction. Al-Kasani, 10:600.

37 In much the same way as time period cannot be stipulated in a contract of ‘ariyah because no one can be compelled to do or continue with an act of charity (tabarru‘). Al-Marghinani, al-Hidayah, 3:60. In other words, making the condition of time-period binding changes the nature of the transaction and it no longer remains tabarru‘.

38 This has some income distributional as well as social consequences.

39 For an analysis of the idea of “debt as a social construct” and the transformation of this social construct to the impersonal market form, see David Graeber, Debt: The First 5,000 Years (New York, NY: Melville House Publishing, 2011, 308-60.

40 This false dichotomy is also not consistent with a number of “legal facts” (nusus). For example, in a hadith the Prophet (peace be on him), after explaining the rule of exchange among six goods, said, “Whosoever paid more or demanded more, indulged in riba.” Muslim, Sahih, Kitab al-musaqa, Bab al-sarf wa bay‘ al-dhahab bi 'l-wariq naqdan. Both are treated equally because both are the participants of “market for loan” which is not allowed.

41 The validity of credit-sale is inferred from many facts. These include the general permissibility of sale transactions such as the words of the Exalted, “Allah has permitted sale” (2:275). The jurists hold that all sales are permitted except those which have been prohibited specifically, such as sales involving uncertainty (gharar) or which stand prohibited by the operation of other principles of law, such as the prohibition of riba. The analysis in text explains that credit sale does not fall under the prohibition of riba.

42 This is the Hanafi position. The other schools classify these six items in different ways, but interestingly all classify them into two categories. The below table summarises their positions:

School Position on Gold and Silver Position on other Four Items Hanafi weighable (mawzunat) volumeable (makilt) Hanbali weighable (mawzunat) volumeable (makilt) and countable (ma'dudt) Shfi'i currency (thaman) edibles (mat'umat) Maliki currency (thaman) storable edible items (mat'umat)

The net result is that all the four schools agree on the applicability of the rules of riba on gold and silver (though for different reasons) and they come up with the impermissibility of loan transaction. For the Hanafis, they are also applicable on all items that are either weighed or volumeable (whether they are food items or not, does not matter); the Hanbalis agree with the Hanafis but add a third category of the counted items; for the Shafi‘is, the rules of riba are applicable on food items (whether they are weighed, measured or counted does not matter); the Malikis agree with the Shafi‘is but add a proviso that these food items must be such that people generally prefer to store them. These differences have interesting implications for extending the rules of riba to cases other than the six items specifically mentioned in the traditions. For details, see Nyazee, Concept of Riba, 83-88.

43 Interestingly, although the four schools have determined different ‘illah (cause) for the operation of riba on gold and silver, yet a loan transaction even if interest-free remains prohibited for all the four schools. Thus, for the Hanafis and the Hanbalis gold and silver must be exchanged on spot because they are weighable items, the Malikis and the Shafi‘is deem it necessary because gold and silver are currency items. Resultantly, despite disagreement on the ‘illah of riba, all the four schools agree that a loan transaction is prohibited as an exchange transaction and permitted only as an act of charity.

44 These include the terms ‘ayn, dayn, and thaman. For an elaboration of the meaning of ‘ayn and dayn, see Nyazee, Concept of Riba, 54-57.

45 The jurists treat gold and silver as thaman (price/currency) in exchange with all other items. Even when they are exchanged with each other (as in the contract of sarf), both of them are treated as thaman. That is why they are called thaman mutlaq (absolute thaman). Fungible items (mithliyyat) are deemed thaman if they are exchanged with non-fungible items (qimiyyat). When a fungible item is exchanged with another fungible item, such as when wheat is exchanged with barley, the parties are at liberty to consider any one of them as thaman but they have to specify it in the contract. For details, see al-Kasani, Bada'i‘ al-Sana'i‘, 7:216-17.

46 The last two implications are based on the widely accepted assumption that modern currencies are just like gold and silver currencies and should be treated as their substitutes. See Ghulam Rasul Sa‘idi, Sharh Sahih Muslim (Lahore: Farid Book Stall, 1998), 4:350-361; and Muhammad Taqi Usmani, Islam aur Jadid Ma‘ishat-o Tijarat (Karachi: Ma‘arif-i Islami, 1999). Changing this assumption can change the implications. The alternative to this view is to accept that modern money is a promise of payment, which implies that it is an acknowledgement of debt. In that case, Islamic rules of hawalah (endorsement) transaction will be applicable. Accepting this position can allow the indexation of loans since money is now treaded as value of something which it promises and, therefore, a loan can be linked to the underlying promised asset.

However, accepting the premise that “modern money is debt” leads to the result that exchange of currencies is not allowed even on spot because of another general rule of the shari‘ah, namely, “prohibition of exchanging debt for debt” (bay‘ al-kali' bi 'l-kali'). The prohibition is reported in by many scholars of hadith. For instance, see ‘Ali b. ‘Umar al-Daraqutni, Sunan, Kitab al-buyu‘, Bab nahy ‘an bay‘ al-kali' bi 'l-kali'; Muhammad b. ‘Abd Allah al-Hakim, al-Mustadrak ‘ala 'l-Sahihayn, Kitab al-buyu‘, Bab nahy ‘an bay‘ al-kali' bi 'l-kali'. Thus, one cannot maintain both of these positions simultaneously; either he has to allow indexation of loans or he has to allow exchange of currencies. For details, see Nyazee, Concept of Riba, 96-114.

47 The jurists cite traditions of various Companions who report that the Prophet (peace be on him) prohibited them from selling what they did not possess but gave exemption for salam. Al-Kasani, Bada'i‘ al-Sana'i‘, 7:101-02.

48 Some other conditions are also applicable for the validity of this transaction but they do not relate to our subject matter here. For their details, see al-Kasani, Bada'i‘ al-Sana'i‘, 7:103ff.

49 A misconception prevails regarding the nature of riba due to a tradition, “there is no riba except in nasi'ah (deferred payment transactions).” These words of Ibn ‘Abbas constitute reason that can explain the adoption of wrong methodology by the contemporary Muslim scholars. It is inferred from this tradition that the primary form of riba deals with loan transaction, which is riba 'l-Qur'an. However, several points invalidate this inference as indicated by al-Sarakhsi. See al-Sarakhsi, al-Mabsut, 12:11-12. First, the words of the hadith are quoted from Ibn ‘Abbas who initially had this opinion but he reverted from this position later on when the hadith of riba was brought to his knowledge by Abu Sa‘id al-Khudri. Second, the ahadith of riba are quoted by several Companions of the Prophet (peace be on him) through several sources. Therefore, they cannot be ignored out rightly in favour of this isolated narration.

Third, hence, it is necessary to place these words of Ibn ‘Abbas appropriately within the legal structure of the shari‘ah. Thus, al-Sarakhsi points that the words relate to the exchange of heterogeneous goods measured similarly, because in that case there is no riba except in deferment.

50 Apçna²'

51 Al-Sarakhsi, al-Mabsut, 12:109.

52 That is why, the definition of riba in al-Durr al-Mukhtar, a later Hanafi text, is given as follows: “Riba is an excess without any counter-value recognised by shari‘ah, in favour of one of the parties in a transaction.” Muhammad Amin b. ‘Abidin, Radd al-Muhtar ‘ala 'l-Durr al-Mukhtar Sharh Tanwir al-Absar, ed. ‘Adil Ahmad ‘Abd al-Mawjud and ‘Ali Muhammad Mu‘awwad (Riyadh: Dar ‘Alam al-Kutub, 2003), 7:398-401.

53 For example, if a female sells her body in exchange of mangoes, this would not be legitimate. Nor will it be legitimate if A lends Rs 1,000 to B on the condition that B will repay Rs 1,000 plus a swine.

Appendix: Questions and their Answers that Follow from the above Analysis No Questions Answers 1 Is bank interest prohibited in the light Yes, it is prohibited, because it is of the Qur'n and the sunnah? violation of rules 1.1 and 1.2. 2 Whether the Qur'nic term rib It includes all forms of interests. includes all kinds of interest rates or it This is a necessary implication of relates only to the excessive interest rules 1.1 and 1.2. rates? 3 Whether the scope of rib extends to It extends to all kinds of loans, the interest charged and paid on commercial or consumption, as business transactions in the banking shown by the application of rules 1.1 system or it is restricted to the interest and 1.2. charged on consumption loans only? 4 Does Islam allow loan transactions? If Loan is against the general rules of yes, how and in what form? Islam. However, it is permitted by the Lawgiver as an exemption to the rule if it takes the form of tabarru'. 5 Is paying interest a lesser evil as No, it is not. The assumed compared to charging interest? dichotomy is wrong, as it has been clarified by I.3. 6 Is borrower always malm (a losing No, the borrower can also be the party) in an interest-bearing loan receiver of rib as per rules 1.1 and transaction? 1.2 (see I.3). 7 Does Islam allow indexation of loans No, it does not. The demand for on the grounds of inflation? loan indexation is invalidated by rules 1.4a and 1.4b. 8 Is credit sale with higher deferred price Yes, it is validated by the application as compared to the spot price allowed? of rules 1.4 and 1.5. 9 Does Islam approve of "time value of No, it does not. In fact, the concept money," especially when charging is alien to the subject matter of rib, higher deferred price is allowed in a provided both the concept of time credit sale? value of money and rules of rib are used appropriately. 10 Are future currency contracts No, they are not. It is violation of permissible in Islam? rule 1.4b. 11 How and to what extent is salam Rule 2 implies that salam is against transaction permissible? the general rules of the shar'ah but allowed as an exemption by the Lawgiver, hence, should remain exemption as per rule 3.
 \end{quote}
 
 
 \paragraph{Chapitre 8. Islam et assurances}
 LES CAPITAUX DE L’ISLAM  | Gilbert Beaugé
 \href{https://books.openedition.org/editionscnrs/871?lang=fr}{ Islam et assurances}
\begin{quote}
    
L’assurance est-elle légale du point de vue de la Shari’a ? Ce débat aujourd’hui encore est très vif dans les pays musulmans\sn{1 En fait, quatre éléments ont permis de classer les contrats d’assurance parmi les contrats interdits par le clergé islamique : a) l’imprécision (gharar), b) le jeu de hasard (gimar), c) l’usure (riba) et d) l’échange de choses équivalentes (bai al sain bil-dain). Le travail le plus complet à ce sujet est celui de Muhammad Baltagi, Uqud al-ta’ min wigha al-figh al islami Koweit, 1982. K. Krüger donne une bonne vue d’ensemble du droit privé des États régis par le droit égyptien. Certains problèmes comme celui du riba sont analysés de façon plus détaillée, cf. Recht van de islam, 1987.}. La fonction sociale de l’assurance est tout particulièrement perçue par la population musulmane, dans la mesure où les catégories classiques du droit musulman intègrent cette valeur. Cette fonction suppose l’utilisation du principe d’assurance et, même dans un pays aussi conservateur que l’Arabie Saoudite, on ne rencontre aucune réserve à l’encontre de l’assurance sociale\sn{2 Introduit par décret royal, pp. 98 sv., n° 746 du 5.09.1969.}. En revanche, on continue à y critiquer l’assurance privée et, de temps à autre, cette critique réapparaît dans d’autres pays islamiques.

 
2 L’assurance peut se définir comme les précautions terrestres que l’on prend à l’encontre des coups du destin et des pertes matérielles, autant d’épreuves envoyées par Allah comme le sait tout musulman pieux. De ce point de vue, l’assurance peut apparaître comme le renoncement à la croyance en une prédestination, l’abandon de l’espérance en une miséricorde ou, pour le moins, comme leur limitation. A ces problèmes de nature religieuse ou morale qui, dans des périodes difficiles, comportent également un élément politique, s’ajoute, pour le musulman, la difficulté à concevoir la notion de risque et à la démarquer de celles de jeu ou de pari, afin de pouvoir en saisir la signification fonctionnelle, en rupture avec les contrats spéculatifs qu’interdit la Shari’a\sn{3 Cf. à ce sujet Al Amin Al Darir, Al Gharar wa-athauhu fil-uqud fil figh al islami.}. La discussion sur ce thème, qui se fonde sur des textes classiques de la Shari’a remontant au Moyen Age, s’est développée à la suite de l’implantation des compagnies d’assurance européennes dans l’empire ottoman, au cours de la deuxième moitié du xixe siècle, et des tentatives d’innovation des réformateurs de l’Islam, au début de ce siècle. Une opposition de type nationaliste et pan-islamique se manifesta alors à l’encontre de ces institutions capitalistes et occidentales, qui jouera un rôle non négligeable.
 
\subparagraph{HISTORIQUE DE LA NOTION D’ASSURANCE DANS LES PAYS DU MONDE ARABE}

3 La première réflexion approfondie sur la légalité de l’assurance en pays islamique tint compte de façon singulière des exigences musulmanes. Dans son ouvrage Radd al Muhttar, Ibn Abidin, représentant de l’école officielle de droit Hanafi dans l’empire ottoman, suggère en effet le compromis suivant : il serait licite d’établir des contrats d’assurance portant sur les risques encourus à l’intérieur du royaume islamique – le Dar al Islam – à condition que ces contrats soient conclus avec une compagnie d’assurance ayant son siège hors des pays de l’Islam, dans le pays des infidèles. La prise en charge du risque devrait se faire au siège de la compagnie\sn{4 Cf. l’étude complète de C.A. Nallino, « Belle assicurazioni in diritto musulmano banafita », in Oriente moderne, vol. VII, Rome, 1947. pp. 446 sv.}.

4 On estimait ainsi satisfaire suffisamment les besoins pratiques en assurances nécessités par le trafic international des marchandises, à une époque où Constantinople, Beyrouth et Alexandrie, ports par où transitait le coton, étaient les centres essentiels du commerce britannique du Levant, alors que le commerce français était davantage orienté vers les côtes d’Afrique du Nord.
 
5 Ce n’est que relativement tard, vers 1890, bien après la fondation de banques ou la création de filiales bancaires (1850), que fut créée une compagnie d’assurance à Alexandrie5. L’initiative en fut prise par trois Libanais qui reprirent la filiale d’une petite société anglaise s’occupant d’une affaire d’assurance-vie en Égypte, pour l’étendre, à partir de Beyrouth, à l’ensemble de la Syrie. Au début du siècle, au moment même où des sociétés françaises tentaient de s’établir en Afrique du Nord6, d’autres sociétés anglaises et françaises suivirent le mouvement, en Égypte puis dans les pays du Levant, tandis que la péninsulte arabique demeurait pratiquement à l’écart jusqu’au lendemain de la Seconde Guerre mondiale.

6 Le marché était peu porteur et étroit. Il s’agissait d’opérations d’assurance-vie, d’assurance-incendie et d’assurance-transport pour quelques biens importants. Ces assurances étaient souscrites par des banques étrangères ayant un intérêt particulier à le faire : par exemple, la Banque Ottomane travailla avec Eagle Star dans le domaine du crédit et de l’assurance hypothécaire. En raison de l’inflation qui suivit la Première Guerre mondiale, les polices d’assurance-vie perdirent leur valeur et il fut difficile de remplir de nouveaux portefeuilles dans un pays ruiné par la guerre. Les compagnies d’assurance ainsi que les courtiers s’orientèrent dès lors vers l’assurance des dommages et des transports à court terme. L’activité de ces entreprises était plus importante dans les pays sous influence britannique comme l’Égypte, le Soudan, la Palestine et l’Iran, alors que l’influence française dominait en Afrique de Nord ainsi que dans les pays du nord Levant, Syrie et Liban.


7 Si l’on prend l’exemple de l’Égypte, on constate qu’un système national d’assurance se mit en place, après la Première Guerre mondiale, avec la participation de trusts européens. Son évolution fut lente car le marché était étroit7 et il était nécessaire de former un personnel local. Le Conseil de surveillance introduit en 1939 maintint dans les limites du raisonnable une évolution qui, entre temps, s’était renforcée. Les pays sous influence française connurent également ce processus, sur les détails duquel nous reviendrons. Pour la Lybie, le marché italien eut une influence déterminante sur le système de l’assurance. Il connut une évolution moins marquée, mais parallèle.

 
8 La notion d’assurance privée s’imposa timidement comme instrument de la vie économique moderne dans les pays musulmans arabes. Les réserves de nature religieuse ne disparurent que progressivement : dans le secteur bancaire tout comme dans le secteur de l’assurance, les modernistes cherchèrent d’abord à obtenir des compromis. Ils tentèrent de prouver que les institutions et les pratiques que générait la vie moderne étaient compatibles avec les règlements de la doctrine islamique, mais ne tinrent pas suffisamment compte des réalités politiques et économiques8. Curieusement, le système d’assurance sociale, alors faiblement développé, fut tenu à l’écart de cette discussion. Il était, et reste toujours appréhendé, moins comme une assurance que comme une aide de l’État. Par ailleurs, jusqu’au tout début de la motorisation, les bases manquaient pour que soit mis en place un important volume d’affaires, aussi bien dans le domaine de l’assurance des personnes que dans celui de l’assurance des dommages. Dans les décennies qui suivirent la Seconde Guerre mondiale, les défenseurs du patrimoine islamique national s’opposèrent aux partisans d’une économie occidentalisée dirigée, ou tout au moins influencée, par des pays occidentaux. Sans se situer au cœur des débats, le système d’assurance fut tout de même soumis à examen critique : l’enjeu était d’établir s’il était ou non compatible avec les prescriptions de la Shari’a.
 
9 Ces discussions se développèrent à la fin des années cinquante, stimulées par un mouvement de création de compagnies nationales d’assurance et culminèrent lors d’une semaine de discussions à Damas\sn{Une telle manifestation avait eu lieu à Paris en 1951. Un des principaux orateurs, le professeur syrien Mustapha Ahmed al Zarga, démontra la compatibilité totale de l’assurance sous toutes ses formes avec la Shari’a, tandis que le professeur égyptien Abu Zahra n’envisageait que des associations mutualistes. L’ensemble des références figure dans un rapport complet de Zahra paru à Damas en 1962 : Aqd at-ta’min wa-mauqifal-shari’a al-islamiyya minhu.} : d’éminents juristes prirent position sur la question de l’assurance et l’on s’interrogea également sur la notion de risque et la manière de le définir et de le préciser dans un contexte économique. Le droit islamique, vers lequel il s’agissait de jeter des ponts, n’offrait sur ce terrain que peu de possibilités. Il fut très difficile, entre autre, de vaincre l’opposition unanime à tous les contrats sur les risques, assimilés aux jeux et aux paris\textsuperscript{10}. L’idée moderne d’une communauté organisée pour faire face à son destin\textsuperscript{11} – dont les bateaux et caravanes pouvaient fournir un exemple – et au sein de laquelle chacun participe de manière appropriée à la réparation des dommages, ne figure pas dans le droit islamique. 

\newpage
Sous cette forme, l’idée de risque demeure étrangère et incertaine\sn{ Le terme général est gharar, l’incertitude. \begin{Def}[gharar]
Gharar al-uqud signifie contrat à risque. Ils ont un rôle dogmatique. On range aussi sous cette dénomination les contrats portant sur des prestations que l’on peut préciser, mais qui ne l’ont pas été lors de la signature.\end{Def} Cf. art. « gharar », ainsi que le commentaire de l’article 484f « äegyptisches Zivilgesetzbuch von Sanhuri » (cf. supra, note 14). voir \href{https://fr.wikipedia.org/wiki/Abd_el-Razz\%C3\%A2q_el-Sanhour\%C3\%AE}{Sanhouri et code civil Egyptien}} au musulman, qui aimerait voir codifier non seulement ses devoirs religieux, mais également ses relations vis-à-vis du monde qui l’entoure. La seule chose sensiblement comparable à une assurance personnelle est le financement d’une rente à vie, sorte de rente viagère (umra) prélevée sur les bénéfices retirés de la terre ou de tout autre bien. Toutefois, une telle institution a suscité de nombreuses réserves de la part des érudits musulmans, dans la mesure où la durée de vie du destinataire restait incertaine13.

LES TENDANCES ACTUELLES
10 Actuellement, le problème des dommages est au cœur des discussions, sans qu’il y ait toutefois de différenciation structurelle entre l’assurance des personnes et celle des dommages. L’assurance trouve sa justification première dans l’idée mutualiste d’une aide réciproque contre ce qu’il est convenu d’appeler, de nos jours, les aléas de l’existence.


11 Dans son commentaire sur le nouveau Code civil égyptien, entré en vigueur en 1956, Sanhuri14, père spirituel de cet ouvrage de lois, s’interroge une fois de plus sur la légitimité des opérations d’assurance. Il se refuse à les justifier par analogie avec des contrats comportant des éléments identiques (caution et garantie), mais il fait du contrat d’assurance un contrat autonome de type nouveau. Dans la mesure où, dans le système du droit islamique, il n’existe pas de numerus clausus et étant donné que ses éléments constitutifs sont déjà reconnus, rien ne s’oppose à ce que l’on admette ce type de contrat. Quant à l’imprécision, motif d’annulation pour un tribunal suprême de la Shari’a en Égypte, il la justifie par la nécessité économique d’une telle institution et par la légitimité statistique. Il résulte de cet examen que l’assurance n’est pas une opération de type spéculatif. Pour ce qui est de l’interdiction du riba et de la perception d’un intérêt, il soutient la thèse que le riba Nasi, c’est-à-dire le report de paiement (crédit) devrait être autorisé dans la mesure où il est nécessaire. Quant à la deuxième forme du riba – l’échange injustifié de prestations différentes –, comme par exemple l’assurance-vie lors d’un décès prématuré après paiement de quelques primes seulement, il soutient qu’il s’agit d’une opération qui pourrait être autorisée pour des raisons d’intérêt social supérieur, si cela s’avérait nécessaire.
\end{quote}
\begin{Synthesis}
Assurance : au début, légal car pas depuis Dar al Islam : peut être repris ?
Ensuite, la principale question est l'imprecision juridique, mais elle ne peut être facilement réduite, même par tontine ou autre mécanismes mutualisant (d'ailleurs caravane pas bon non plus) ?
Sanhuri : c'est nouveau donc licite. Imprécision : corrigé par actuariat ! + raison sociale supérieure. A
\end{Synthesis}

\begin{quote}
    


12 L’argument mutualiste – ta awuni ou tabaduli\sn{What does \TArabe{ تبادلي }(tabaduli) mean in Arabic?

reciprocal : 96\% of use and mutual in 4\%} – fut l’argument essentiel avancé en faveur de l’autorisation de l’assurance et, en tout premier lieu, de l’assurance-dommage. Sur ce point, on peut parler d’un accord de tous les érudits musulmans : le système traditionnel de la société d’assurance mutuelle s’offre alors pour résoudre ces problèmes. Il apporte également une solution à la question des intérêts résultant d’une accumulation de capital dans la mesure où les fruits du capital sont répartis entre les adhérents. L’interdiction de l’usure perd ainsi, sur un plan moral, de sa force explosive. Certains docteurs surmontèrent les réticences à l’égard des contrats sur les risques en faisant prévaloir que l’élément dominant dans les sociétés d’assurance mutuelle est positif, puisqu’il s’agit d’un soutien dans la détresse et le malheur. Il serait donc convenable d’accepter une part d’incertitude, élément secondaire dans un contrat d’assurance par rapport aux notions d’aide et de soutien, et qui ne nuit pas à l’efficacité du contrat. En revanche, on tend à éliminer les autres formes d’assurance parce que la mentalité de profit, associée à la structure des sociétés par actions, vise à procurer des bénéfices aux actionnaires et ne saurait être justifiée par sa fonction d’entraide15.
 
 \begin{Synthesis}
 Il est intéressant de voir le même développement des mutuelles en France dans un cadre \textit{éthique}
 \end{Synthesis}
13 Après la Seconde Guerre mondiale, les marchés nationaux reçurent une impulsion décisive du développement rapide de la motorisation, occasion pour les compagnies de recettes provenant des primes obligatoires de responsabilité civile dans l’assurance des véhicules. Dans certains pays comme ceux de la péninsule arabique, l’industrie pétrolière, en pleine expansion, avait besoin d’assurer non seulement des investissements croissants sur les lieux de production mais aussi son environnement économique et social. On s’efforça donc de mettre en place une participation appropriée aux assurances pour les transports pétroliers. Quant aux compagnies aériennes nationales, elles s’assurent pour les risques inhérents au transport aérien auprès de compagnies nationales dans la mesure où celles-ci sont aptes à répondre à leur demande16.
 
14 En raison de la faible capacité caractéristique de tous les marchés arabes, la plus grande partie des risques sont couverts par une réassurance qui, il y a encore quelques dizaines d’années, était le domaine réservé des compagnies européennes. Depuis, les marchés nationaux arabes ont tenté, avec quelque succès, de mettre en place un marché commun arabe de la réassurance. Cependant, celui-ci ne dispose que d’une capacité limitée17. Ces efforts sont soutenus efficacement par l’Association des Assurances Arabes, fondée en 1963, dont les membres sont recrutés auprès des entreprises de l’ensemble des États arabes, depuis la Mauritanie jusqu’au Golfe d’Oman. Notons pour mémoire qu’un autre groupe important a été créé quelque temps après : il s’agit de l’assurance afro-asiatique, compagnie de réassurance dont le siège est au Caire. Par des colloques internationaux, mais également par des cours de formation pour cadres, on s’efforce désormais d’atteindre un haut niveau de technicité dans le domaine de l’assurance et de promouvoir une coopération dans le travail.
 
15 Cette coopération interétatique – expression du panarabisme formulée de manière la plus expresse au sein de la Ligue Arabe – avait été précédée par des aménagements internes sur les marchés arabes qui visaient à la mise en place d’une législation relative aux contrats, assortie d’un contrôle étatique. Cela fut le cas dans les États qui connaissaient une économie libérale, mais se réduisit à quelques fonctions de surveillance dans les pays où l’assurance était un monopole d’État18.

16 Depuis qu’en 1987 a été créée en Arabie Saoudite une compagnie nationale d’assurance19, il n’existe pratiquement plus un seul pays arabe qui n’ait son propre système d’assurance, fondé essentiellement sur l’assurance-véhicule. Par contre, l’assurance-vie n’a pas suivi l’évolution : les suites d’une souscription sont toujours remises en cause par l’inflation ; de plus, l’utilisation d’une devise forte n’est pas possible dans la mesure où – dans la plupart des pays – la loi prévoit expressément que les réserves constituées par les primes seront placées dans le pays20.

\begin{Synthesis}
Nécessité fait force de loi : l'assurance Auto en Arabie Saoudite
le développement du marché est comme en France lié à l'opacité du contrat (cf assurance vie et jeu au début)
\end{Synthesis}
17 Le mouvement qui milite en faveur d’une suppression de la clause d’intérêt sur les marchés arabes, parti d’Arabie Saoudite, a également gagné le terrain des assurances. Dans les pays qui interdisent l’intérêt, comme l’Arabie Saoudite, les difficultés proviennent du traitement des dépôts des réassureurs – même dans le cas de l’assurance-dommage – dans la mesure où ceux-ci n’entrent pas dans le cadre des exceptions pour capitaux étrangers. Dans le domaine de l’assurance-vie, une solution reste envisageable par la coopération avec un fond d’investissement islamique.

\begin{Synthesis}
Voir le hanbalo-wahhabisme et son influence et la gestion de l'assurance auto dans ce cadre
\end{Synthesis}

UN EXEMPLE D’ASSURANCE ISLAMIQUE
18 Un modèle intéressant a été développé par les banques islamiques qui travaillent sans prélever d’intérêt. Leur modèle principal pour le placement de leurs capitaux est la forme bien connue de la mudaraba de participation. De par sa structure, on peut la comparer à une société en participation, à un fond de placement où le déposant a une participation aux bénéfices proportionnelle au montant de son placement et où les pertes se font au détriment de celui-ci. Si on intègre la mudaraba dans le système d’une compagnie d’assurance, on obtient la combinaison suivante :

19 Il s’agit d’une caisse de décès ou, plus exactement, d’associations d’épargnants garantissant l’épargne, même en cas de décès prématuré. Les compagnies, filiales de la banque islamique, se nomment sharikat al-takaful al-islamiyya, leur slogan publicitaire est la participation réciproque garantie (mudaraba al tadamum), c’est-à-dire l’épargne garantie entre les musulmans. Les buts de la société sont présentés comme suit :

Organisation d’un soutien et d’une solidarité réciproques au sein de la communauté islamique.

Suppression dans la communauté musulmane de l’usure (riba), qui est un fléau ainsi que l’atteste le verset du Coran 2,278. « Fidèles ayez confiance en Dieu et laissez s’éloigner de vous ce qui demeure encore du riba, si vous êtes croyants. Si vous ne le faites pas, Dieu et le Prophète vous déclareront la guerre. »

20 Toute personne âgée de 20 à 50 ans peut devenir sociétaire de la compagnie. Chaque année, celle-ci verse une prime définie, dont la plus grande partie est placée à la banque islamique. Quant aux 20 \% restants, ils sont versés librement (!) par le sociétaire à un deuxième fond destiné à soutenir les familles de ceux que le destin a frappés avant qu’ils n’achèvent leur programme d’épargne. Quand l’affiliation cesse et en cas de survie, on verse au sociétaire la quote-part sur ses parts d’investissement ainsi que sa quote-part sur les bénéfices des parts fixes et, éventuellement, sur les bénéfices de ses parts au fond de soutien de l’assurance décès. Les deux fonds opèrent donc d’après le principe de la mudaraba, sans intérêts ni participation aux bénéfices des sommes investies21. Le plus grave problème posé par cette forme de placement est la garantie d’une liquidité suffisante nécessaire pour un règlement rapide des dommages importants. Les parts de l’assuré, tout comme les bénéfices réalisés par le fonds, sont soumis au Zakat ainsi qu’à l’impôt prélevé à la source (respectivement 2.5 et 5-10 \%).

21 Les banques islamiques opèrent dans les pays arabes depuis un bon nombre d’années. Sur le volume effectif d’affaires réalisé par une société observant rigoureusement les principes de la Shari’a, on connaît très peu de choses. On sait par contre que les difficultés dans le domaine de la réassurance internationale, mais également les questions de liquidité pour remboursement de dommages importants, n’ont pas été résolues de manière satisfaisante. Une sérieuse tentative dans ce sens a été la création d’une compagnie d’assurance saoudienne caractérisée par la garantie mutualiste et gérée en conformité avec les règles de la Shari’a. Son champ d’activité se limite à un secteur ne comprenant pas l’assurance-vie, dans la mesure où cet aspect est pris en charge par les filiales des banques islamiques. Seule l’expérience montrera si cette société peut, dans le cadre de ses missions internationales, ne pas se conformer aux exceptions d’usage faites pour la circulation des capitaux internationaux.
 

22 Mentionnons en conclusion que la création des instituts financiers islamiques correspondait également à l’intention de stimuler la circulation de l’argent et des capitaux dans la masse de la population. En tant que facteur économique, l’importance de l’assurance est cependant relativement faible. Le volume global des primes est estimé à moins de 1.5 \% du volume des primes dans le monde, alors que le nombre des entreprises est évalué à 10 \% des entreprises mondiales, pourcentage d’autant plus remarquable que de nombreux États ont nationalisé l’assurance ce qui réduit d’autant le nombre d’entreprises. La majeure partie des primes provient de l’assurance-automobile et de l’assurance du commerce extérieur, auxquelles il faut ajouter l’assurance de quelques grands projets industriels. L’absence d’une véritable critique de l’assurance de la part des milieux fondamentalistes est sans doute liée à son manque d’extension. La publicité pour les produits de l’assurance est faible : la notion de prévoyance n’ayant pas été encouragée chez l’individu, la part de l’assurance-vie dans le volume global des primes est largement inférieur à la moyenne. Ce serait cependant méconnaître les faits que d’en rendre l’islam responsable : l’assurance-automobile obligatoire, introduite par la plupart des pays, a été acceptée par la population, sans réserves et sans critiques. Dans quelques pays, mais surtout dans les pays de la péninsule arabique, le montant de l’assurance est pris en considération pour le paiement des indemnités : en Arabie Saoudite, il est de l’ordre de 60 000 RS et s’élève à 100 000 RS pendant la période de ramadan.

23 De nombreux signes permettent donc de penser que la notion d’assurance, à partir des bases actuelles, s’imposera progressivement. S’interroger sur l’étendue des réalisations, c’est se poser une question d’ordre économique. Les réserves de nature religieuse peuvent concerner l’application pratique de l’assurance, mais non son institutionnalisation en tant que phénomène.

\subparagraph{NOTES}
 



5 Certaines agences de sociétés britanniques travaillaient en Égypte dès les années 1850. Cependant, la poussée véritable se situe après l’occupation anglaise de 1882. Pour plus de détails, cf. Basim A. Paris, Insurance and Reinsurance in the arab world, Lon-don, 1983, pp. 43 et sv.

6 La première société à s’installer dans la région fut bien La Espanola, à partir de 1879, au Maroc. La première vague de création se situe après la Première Guerre mondiale, une deuxième suivra au début des années 60.

7 Après la Seconde Guerre mondiale, il n’y avait que cinq compagnies d’assurance en Égypte. En 1967, s’y ajouta une compagnie de réassurance et plus tard vinrent deux compagnies en joint-venture dans la « zone libre ».

8 Cf. à ce sujet I. Goldziher, Vorlesungen über den Islam, 2e éd, Heidelberg 1925, p. 259, ainsi que les Fetwas citées dans la revue moderniste islamique. Pour l’assurance vie, cf. Al Manar, Le Caire, vol. VI, p. 938, pour l’assurance-marchandise, vol. VIII, pp. 588 sq. Ainsi que vol. VI p. 717 et vol. X p. 330 et sv. Pour la question de l’intérêt et son autorisation dans les caisses d’épargne, cf. Ali Cikri et les conférences de Rida. Pour l’évolution d’ensemble actuelle, cf. E. Klingmüller, « Betrachtungen zur Reislamisie-rung im Recht », in FS H. Hübner, Cologne 1984, pp. 84 et sv.



10 Le muqamara et le rihan, déjà prohibés au début de l’islam, sont expréssement interdits par le Coran (Sourate 2,219 et 5,90). Les exégètes n’ont eu de cesse de déterminer quelles étaient les comportements de la vie quotidienne que pouvaient recouvrir ces concepts. Cf. sur ce point article « qimar », in L. Milliot, Introduction à l’étude du droit musulman, Paris, 1953, p. 653.

11 La protection lors des accidents de voyage (daman khatar al tariq) comme cas particulier d’une caution (mafala) utilisée pour la justification de l’assurance et, par voie analogique (giyas) l’institution du bai-am wafa qui ménage au vendeur une possibilité de rachat, l’acquéreur ayant dans l’intervalle l’usufruit du bien, mais acceptant le risque d’un dommage. Sur tout ceci, cf. Milliot, op. cit., p. 953.



13 Pour plus de détail, cf. Al Dharir, op. cit. p. 633, avec citations complètes de la littérature figh classique. Curieusement, y est mentionnée une prise de position hostile de la confrérie musulmane parue dans son propre journal, en 1941.

14 Abd al Razzag Ahmed Al Sanhuri, dans son commentaire « al-wasit » 1re éd, Le Caire 1964, vol. 7, p. 1088 sq. Cf. également son oeuvre systématique dans laquelle il tente de justifier le maintien des valeurs de l’islam et de trouver un compromis avec les exigences de la vie moderne, Masadir al-haqq fil-fighal-islami. 3e éd., Le Caire, 1967, vol. 3 pp. 32 et sv.

15 Cf. Isa Abduh, in Al-uqud al shari’a, Études pour le congrés sur le droit islamique (Riyadh novembre 1978), publié sur place. Par ailleurs, pour justifier certaines formes nouvelles de contrat, on faisait appel à la clause, déjà répandue au Moyen Age, consistant à se référer à l’utilité publique et à la nécessité d’une utilisation fondée sur l’intérêt général.

16 Lors du dernier congrés de la General Arab Insurance Federation (Damas, mai 1988), fut décidée la création d’un pool aérien ainsi qu’une collaboration plus étroite dans le domaine des risques lourds.

17 La franchise absolue dans une société arabe peut atteindre de 1 à 2 \% de la somme assurée. L’intervention d’une coassurance ou d’une réassurance dans le pays n’augmente pas la capacité du marché de la réassurance. La majeure partie des risques lourds est couverte par une réassurance facultative sur le marché international, ce qui n’est possible que parce que les réassurances apportent leurs connaissances techniques et leur expérience pour juger des différents risques. Cf. également Abdul Zahra Abdul-lah Ali, Insurance development in the arab world, London, 1985, chap. III. Dans ce chapitre sont publiés des chiffres qui ne se font pas encore l’écho de la guerre Iran-Irak.

18 Les marchés nationalisés sont l’Algérie, l’Irak, la Libye, la Mauritanie, la Somalie, la Syrie et le Sud Yémen, alors que les marchés du Maroc, de Tunisie, du Nord Yémen et du Soudan sont dominés par des sociétés contrôlées par l’État, mais fonctionnant comme des sociétés privées. Actuellement, on trouve un marché relativement libre en Égypte, au Qatar, dans les Émirats ainsi que dans le Sultanat d’Oman. Récemment a été fondée en Arabie Saoudite une société d’assurance-dommage soutenue par l’État.

19 Sur un plan formel, la société d’assurance est une société par action qui porte le nom de société nationale reposant sur la réciprocité (ta’min ta’awun). Dans ses statuts, il est expréssement mentionné qu’elle ne doit pas enfreindre les préceptes de l’islam et qu’elle doit tout particulièrement respecter l’interdiction du riba. De manière active ou passive, la société ne doit donc pas travailler avec intérêts. Ceci vaut aussi bien pour les mouvements de capitaux à l’intérieur du pays, que pour les transferts hors du pays (art. 4, alinéa 3 des statuts). Le capital action entièrement versé se monte à 500 millions de RS. Les statuts sont reproduits dans Umm al-Qurra du 12.04.1986.

20 Le clergé musulman a toujours quelques réticences quant à la création d’une société d’assurance-vie. Même dans le cas d’une compagnie assurant les biens matériels, il convient de tenir compte des préceptes du Coran, d’une part pour ce qui est de la structure en société mutuelle, d’autre part pour ce qui est du respect de l’interdiction de l’intérêt qui rend pratiquement impossibles les affaires de réassurance avec dépôts. Les placements de capitaux ne peuvent se faire qu’auprès des banques islamiques, des concessions au marché international des capitaux n’ont, à l’exception de la SAMA, jamais été faites jusqu’à ce jour.

21 Cf. à ce sujet H.P. Kindt, « Das islamische Versicherungswesen », in Versicherungswirtschaft, Karlsruhe, 1985, pp. 585 sq.
\end{quote}

AUTEUR
Ernst Klingmüller



\paragraph{les mutuelles en France}
\href{https://www.cairn.info/revue-les-tribunes-de-la-sante1-2016-3-page-61.htm}{Les mutuelles, un acteur puissant mais peu écouté
François Charpentier}
\begin{quote}
    Dans un pays catholique où, pour des raisons religieuses, le principe de l’assurance était encore prohibé au XVIIIe siècle et où l’individu devait s’en remettre à la providence divine, la mutualité est apparue très tôt comme une alternative possible au besoin de protection contre la maladie et la vieillesse. Les historiens font remonter les premières mutuelles au XIVe siècle. Nous retiendrons pour notre part que l’essor des mutuelles au sens moderne du terme – des collectivités d’hommes et de femmes qui s’organisent entre eux sur des bases professionnelles ou locales pour se protéger – est lié à l’urbanisation galopante de la société, elle-même liée à la révolution industrielle.

La concurrence des sociétés d’assurance
3Pour autant, l’apparition des mutuelles dans le paysage social français n’a pas été un long fleuve tranquille. On rappellera d’abord que l’essor des mutuelles s’est opéré au moment où les compagnies d’assurance obtenaient un droit légal à l’existence avec la création, en 1787 par Louis XVI, sous la pression de banquiers genevois, donc protestants, d’une Compagnie royale d’assurance sur la vie. Bien sûr, il s’agissait d’une institution publique, mais fonctionnant sur le modèle d’une société privée collectant de l’épargne pour faire face aux besoins de ses membres quand se réalisait l’aléa. Quant à la recherche du profit, elle constituait l’élément moteur de cette innovation.

4On retiendra que cette royale initiative a eu pour mérite de rendre à l’homme la liberté de maîtriser les événements de son existence. Elle a sécularisé la société, qui passe alors « du moral au légal, du religieux au laïc [2]
[2]
L. Lautrette, Le Droit de la retraite, PUF, coll. Que sais-je ?… ». Elle a ouvert la voie à la pratique de l’actuariat. Elle a inscrit la prévoyance dans le champ d’intervention de l’État en favorisant la mise en place d’un projet à finalité sociale, puisque cette « prévoyance sociale » doit se développer au profit des professions les plus exposées : marins, pêcheurs, maçons, charpentiers, couvreurs. Elle a anticipé, d’une certaine façon, « le passage des assurances commerciales aux assurances sociales professionnelles [3]
[3]
B. Gibaud, Mutualité, assurances (1850-1914), Economica, 1988. ».

5D’emblée la concurrence a donc été vive entre, d’une part, les assurances développant la prévoyance libre et, d’autre part, des mutuelles prônant la fraternité et la solidarité et, de ce fait, gérées bénévolement par leurs membres qui ne faisaient pas du profit leur raison d’exister. Au contraire même puisque la règle voulait que les surplus accumulés et non utilisés soient rétrocédés aux adhérents sous forme de baisse de cotisation ou de majoration des remboursements.

6En résumé, comme le formule Patricia Toucas-Truyen dans son Histoire de la Mutualité et des assurances : « Les débats éthiques autour du développement de l’assurance sur la vie font clairement apparaître des différences […] de finalités entre caisses de secours fraternelles ou corporatives et assurances : les assurances permettent aux classes aisées d’accroître leurs biens et de les transmettre à leurs héritiers ; les caisses de secours offrent à leurs membres la possibilité d’atténuer les conséquences de la maladie et de la vieillesse. Pour les uns, il s’agit d’obtenir l’assurance d’un capital confortable jusqu’à la fin de leur vie et, au-delà, transmissible à leur postérité. Pour les autres, l’épargne réalisée n’assure guère que la survie [4]
[4]
P. Toucas- Truyen, Histoire de la Mutualité et des assurances,…. »
\end{quote}
 \section{Bibliographie}


« Ad-Dourra Al Moukhtasara » : Le résumé des vertus de la religion musulmane. Par : Abdarrahman As-Sacdî. Traduit par Tamime Khemmar.

Jomier Jacques. L'imam Mohammad 'Abdoh et la Caisse d'Epargne (1903-1904). In: Revue de l'Occident musulman et de la
Méditerranée, n°15-16, 1973. Mélanges Le Tourneau. II. pp. 99-107;

\cite{Chapra:Riba}

\chapter{Plan pour Riba}

\section{Introuction}
\href{https://icp.summon.serialssolutions.com/2.0.0/link/0/eLvHCXMwrV1db9MwFL0a6wNICMYAUQYoT3w8tHPt3Lh5TNukYWoH2jpN4sWyHRehiTK1Rdoj_4F_yC_BN3UjOvGEeIuU2FJ0fK_PTc65BhC8yzq3coKbpzLlzm8QLkYjetazftrqcK6lZ7TkG87H6cciPj8Tkz0ot9aYTbuI5vsbBUqdvinetVkd_yHMIQuRr7FYkGtxlrCuJ5ct-tGA-9AajM4-FU2S5lifnecrECQTPu6KfP46V5O0WwTAzQ4lbYitCIm2eAhXjb_HXne_VF8bf_Wtho__4z0P4EEgsFG2WXGPYM8tDuHu1t-8OoSDIKnzD4XE8RiOZ2UehTMTz2fR9EOZTaf5KHo_OI2ywehiMsnK6C1K9uvHz0Twd0_goshnw7ITjmrofOZIBaidG4Pk16skam4SURkrekJTzWn8LdnrSVt5coGoNTIzN8xJak7vHClNxVO4r0nSv1jX1r_qGUSVpx-JdTK2lsVWCy37fmDKEysSNMy24Q3hoSgS10ttdTAUfFs46mmlMk7NAf0OnLbhdQ2Zut508FB6eUWCNonq8nSsyuH05PKkHKukDUdbTFWI5ZXqI9JZPv24DbJGpplmp4DqK0JFESqqRkXdqHyYFXT5_J9HHsG9zVdsUr29gP318rt7CXf8knoVlvRvpSXyjQ}{The Economist  MOHAMMED IBN ABDULLAH (570-632)}
The absence of a dynamic market economy in many Islamic societies has encouraged the inference that the values of Islam are not compatible with capitalism. However, an examination of the biography and commercial record of Islam's founder, the Prophet Mohammed, refutes this presumption. Mohammed ibn Abdullah was a scion of an elite dynasty of religious, civic and commercial leaders in Mecca. He abandoned his successful business career in Mecca and fled to Medina at the age of 52, where he realised his vision of an Islamic society. In Medina, Mohammed implemented policies for competition, consumer protection and market regulation. Mohammed's approach to fair trading explains his ban on usury, as distinct from a proscription on borrowing. Mohammed's achievements as an economist and market reformer earn him a place in the history of economic thought.

\href{https://icp.summon.serialssolutions.com/2.0.0/link/0/eLvHCXMwnV1Lj9MwEB61e0BIqwXKrggsyJe9rEhJGidO9rI81MChhwroiUPk-KEt0CrbhxD_nhknaWhRLxwcR8pETjLO-LP9zQxANBoG_oFNMDYT2cjgAGF4XEahQtRPQ11spUBES37D44_ZNOdfPkeTHqStawyxLB1N0G3qI14qf5o3OGojyI75bXXvU_oo2mZtcmmQLUZI4Hr3-84i78f45u3uZu1Ch5jBp3l1FtLJ3vhUUxSxXpnKqP3NUtGY1PwRfNt58qhqONeLnSf1QWjH_3mjx3DWQFP2ru5LT6BnlgN40DLjB9CfyF9PYTZboxqYXGpGmcAYGRScCjsF37CG4bxdO4G8DefBalo9my9ZF5eETTtHz3OY5eOvHz75TW4GnwDIxk-1sFGiuI6VUok0QnKtlOCpDU2cIajUJeWcyLhKAi04XrUmQ6gQSWtNmdroAk4lcfiXG-frp58B04g3EmUEVyrgSkZSpHFQZqNERUlcBsqD61Y1RVXH4ii6qMukx4JYeqTHIvDgwn3mnWT7jT1467S5u_BDVt_L7drgv11gmyM8_MZCi7hYzbGEWCpXh3Fxt1l48LrtCH89CLVPCzRFo6f6QSptPbj6R9wJ1vdgi6GTPSInDuWeH3u1F_CwXnomYtElnGxWW_MS-tgrX7kf4g9iMhFq}{
Usury and Just Compensation: Religious and Financial Ethics in Historical Perspective}
Usury is a concept often associated more with religiously based financial ethics, whether Christian or Islamic, than with the secular world of contemporary finance. The problem is compounded by a tendency to interpret riba, prohibited within Islam, as both usury and interest, without adequately distinguishing these concepts. This paper argues that in Christian tradition usury has always evoked the notion of money demanded in excess of what is owed on a loan, disrupting a relationship of equality between people, whereas interest was seen as referring to just compensation to the lender. Although it is often claimed that hostility towards 'usury' has been in retreat in the West since the protestant Reformation, we would argue that the crucial break came not with Calvin, but with Jeremy Bentham, whose critique of the arguments of Adam Smith, upholding the reasonableness of the laws against usury, led to the abolition of the usury laws in England in 1854. There has to be a role for law, whether Islamic or secular, in regulating financial relationships. We argue that by retrieving the necessary distinction between demanding usury as illegitimate predatory lending and interest as legitimate compensation, we can discover common ground behind the driving principles of financial ethics within both Islamic and Christian tradition that may still be of relevance today. By re-examining past ethical discussions of the distinction between usury and just compensation, we argue that the world's religious traditions can make significant contributions to contemporary debate.
\section{Contexte Coranique}

\paragraph{difficulté d'application de la Riba} \href{https://www-jstor-org.icp.idm.oclc.org/stable/23264404}{The Qadi, the Big Merchant and Forbidden Interst (ribā)} While the imposition of interest on loans is expressly forbidden by the Quran, over the generations Muslims in fact found it difficult to observe this prohibition and lent to and borrowed from one another. The Muslim big merchants (tujjār), who held large amounts of liquid capital, were prominent among the lenders. The religious prohibition on the one hand, and everyday constraints on the other, caused a certain cognitive dissonance among many ulema, as manifested in the religious (shar'i) courts. The records of these courts in various regions in the Middle East from the sixteenth to the beginning of the twentieth century include cases in which judges (qadis) exempted borrowers from full or partial repayment of interest on the grounds that a demand for payment of interest violates the precepts of Islam. The present article provides examples of such cases and discusses the possible effects of the courts' retroactive annulment or amendment of contracts on the development of capitalist economies in Muslim countries during the nineteenth century when Middle Eastern economies began integrating into the global economic system. Inter alia, the discussion of this question sheds new light on the historians' critiques of Max Weber's observation regarding 'Kadijustiz' (qadi's justice) and its effect on the development of modern capitalism.
\section{Un premier mouvement, modernisme}

\section{une réaction islamique : Takaful}

\section{Une difficulté à appliquer}


\paragraph{Insurance} \href{https://www-jstor-org.icp.idm.oclc.org/stable/27650571?pq-origsite=summon}{Islamic Insurance: National Features and Legal Regulation} The present paper studies Islamic insurance (takaful) as opposed to conventional one. The first part of the paper covers, among other things, such issues as nature and historic roots of Islamic insurance, early forms of Islamic insurance and narrates the disputes among Muslim scholars concerning the compatibility of insurance with Islamic Shariah. The second part deals with history and emergence of Islamic insurance in the modern financial market, as well as the practice of Islamic insurance in different coutries. The third part discusses the feasibility of Islamic insurance in Russia in the current legal framework. The paper contains comprehensive glossary of related terms.
\paragraph{Assurance Auto}

\paragraph{Tawarruq} \href{https://www-jstor-org.icp.idm.oclc.org/stable/43294670?pq-origsite=summon&seq=1}{Debate on "Tawarruq": Historical Discourse and Current Rulings}

One of the controversial products used by the Islamic financial sector is organized tawarruq. As the substance of this product resembles that of an interest-based loan, there is a debate on its permissibility from a Shari'ah perspective. While discussions on tawarruq have arisen due to the emergence of the practice in Islamic finance, there have been deliberations on this transaction in the past starting immediately after the emergence of Islam. The aim of the article is to provide an overview of the historical discourse on tawarruq and examine the rulings on it by two contemporary jurisprudential bodies to assess the practice of the transaction in Islamic finance. The discussions show that the current rulings concur with the majority view of the past scholars. The practice of organized tawarruq by the Islamic financial industry, however, appears to be inconsistent with both the contemporary and historical rulings.

\section{Rissalat}

\cite{Abdou:Rissalat}

\paragraph{manière de démontrer le besoin qu'ont les hommes de la prophétie}
\begin{quote}
   On a dit : Pourquoi Dieu n •a -t-il pas déposé dans la conscience même
de l'homme le savoir dont celui-ci a besoin ? Pourquoi ne l'a-t-il pas fait
son propre guide qui se dirigerait soi-même vers l'action juste et vers le
chemin droit conduisant au but assigné dans l'autre vie? Quelle est
cette miséricorde étrange qui prend une voie détournée pour nous
guider et nous instruire ? Celui qui parle ainsi exagère les attributions
de la raison et n • apprécie pas comme il convient le sujet dont nous parlons,
c'est-à-dire le genre humain, tel qu'il est, avec les différents éléments
spirituels et intellectuels qui le composent, et qui ont pour conséquence
des degrés divers dans la formation intellectuelle des individus. Il
néglige le fait que chaque individu n'est pas apte à acquérir spontanément
tous les états de la connaissance, mais qu'au contraire son progrès
est basé sur la recherche et sur l'étude. Car s'il pouvait satisfaire
ses besoins par l'instinct, comme le font les autres animaux, il ne serait,
plus de son espèce mais bien un animal, comme l'abeille et la fourmi,
ou un ange, habitant des cieux.
\end{quote}

\begin{quote}
    Nul ne met en doute que chaque membre d'une société a _besoin
des autres membres; et chaque fois que l'individu accroître ses exigences
dans la vie il ressent plus fortement le besoin de recourir au concours
de ses semblables. Ainsi se développent les besoins et à leur suite s'étendent
les relations de la famille à la tribu, de celle-ci à la nation, et finalement
au genre humain tout entier, comme le montre notre époque . Ces
besoins qui créent dans le sein de chaque nation (surtout dans le sein de
celles qui méritent vraiment ce nom) des relations et des rapports
spéciaux la distinguant des autres nations, sont le besoin de se procurer
sa subsistance, celui de profiter des biens de la vie, celui d' acquérir
les choses désirables et d'éloigner de soi celles qui déplaisent. 
Si la vie de l'homme se déroulait selon les lois de la nature, telles
que nous les voyons appliquées aux autres êtres vivants, les besoins
que nous venons de citer auraient été parmi les facteurs les plus puissants
de l'amour entre les individus, le facteur qui aurait fait sentir à chacun
que son existence dépend de celle de son groupe, que ce groupe est pour lui comme une faculté nouvelle dont il dispose pour acquérir les
choses utiles et éloigner de soi les choses nuisibles. L'amour est la source
de la paix et de la quiétude dans les coeurs ; quand deux êtres s'aiment,
c'est lui qui pousse chacun d'eux à agir dans l'intérêt de l'autre, à le défendre en cas de danger.  C'est l'amour qui main tient l'harmonie au
sein des peuples, qui est l'essence même de leur existence ; les lois naturelles qui nous régissent ont créé un lien étroit entre l'amour et le besoin,
car l'amour n'est que le besoin que nous éprouvons de nous rapprocher de la personne ou de la chose aimée, et, en croissant, il se transforme en passion.

\ldots
Si par contre, l'intérêt se mêle aux relations amicales, et si chacun des amis exige un prix pour son amour, celui-ce se change en esprit d'exploitation, il se reporte sur l'effet utilitaire et se transforme chez l'un des amis en un abus de la force, et chez l'autre, en peur avilissante, en dissimulation et en hypocrisie.
p.66
\end{quote}

\begin{quote}
    
    « l 'homme
a été créé avec u11 caractère inconstant; quand le malheur l'atteint
il est abattu, et quand il acquiert quelque bien il devient insolent. >;
(C. ch. 70, v. 19 à 21). Les individus sont diversement doués au point
de vue de l'intelligence, de l'activité, de l'assiduité et de la volonté. 11 y
en a qui sont inférieurs à la moyenne, soit _par faiblesse d'esprit, soit
par paresse, tout en la dépassant d ans l'intensité de leurs désirs qu'ils
cherchent à satisfaire avec passion et avidité ; ils considèrent leurs
semblables comme un moyen pour subvenir à leurs besoins, mais ils
se représentent la jouissance que leur procurerait l'usage exclusif de
tout ce qui est dans la main d'autrui et ne se contentent pas d'utiliser
les fruits de leur propre travail en les échangeant contre ce qu'ils désirent ;
et comme ils veulent vivre sans travailler, le meilleur moyen, d 'après
eux, est de s'appliquer à toute espèce de ruses afin de profiter des autres
sans être eux-mêmes d'aucune utilité. Ces sentiments s'emparent d'eux
au point qu'ils s'imaginent qu'il n'y a pas de mal à priver de l'existence
celui qui s 'oppose à leurs désirs et à le supprimer après l'avoir
dépouillé. Chaque fois que la mémoire et l'imagination les poussent à
éviter quelque chose qui leur inspire de la crainte ou à atteindre un
objet qui leur fait envie, leur intelligence leur ouvre une porte de la
ruse ou leur découvre une voie de la violence ; alors le rapt remplace
l'échange pacifique, la dispute prend la place de l'union et la conduite
de l'homme rie s 'appuie plus que sur l'astuce et la violence.
p 68
\end{quote}

\begin{quote}
   Est-il possible que les sociétés humaines, dont la bonne marche et
l'existence même, dépendent de la collaboration et de l'aide que les
hommes s'accordent entre eux, puissent subsister dans ces conditions?
Est-ce que les facteurs que nous venons de faire ressortir ne seront pas
la cause de leur disparition? Certes oui, etc 'est pourquoi le genre humain
a surtout besoin, pour conserver son existence, de l'amour ou d'un
sentiment qui le remplace.
A différentes époques il y a eu des penseurs qui ont fait appel à
l'équité ; ils ont pensé, et même quelques mystiques ont exprimé cette
pensée par de nobles paroles, que l'équité remplace l'amour. Cette
assertion ne manque pas de sagesse, mais qui est-ce qui peut établir les
règles de l'équité et amener la totalité des hommes à se soumettre ?
p 69
\end{quote}

\begin{Synthesis}
Abdou montre qu'on peut accéder à l'équité par des voies naturelles mais que pour les masses, et que par ailleurs la corruption des élites, c'est mieux d'avoir une loi externe, celle qui lutte contre la plus grande détresse, celle qui lui inspire l'inconnu.
\end{Synthesis}

\begin{quote}
    cieux et à
Lui retournera toute chose. " (Cor. ch. 3, v. 100 à 105.) Après ces
exhortations, qui jettent le trouble dans le coeur de ceux qui transgressent
les commandements de Dieu et qui confirment les châtiments pour ceux
qui s'en écartent ou les pratiquent imparfaitement, le Coran indique
qu'il n'y a pas de condition meilleure que celle des hommes qui ordonnent
le bien et défendent le mal: • Vous êtes les meilleurs parmi les hommes,
vous ordonnez le bien, vous défendez le mal et vous croyez en Dieu.
 
(Cor. ch. 3, v. 106.) Dans ce verset le fait d'ordonner le bien et de
défendre le mal est mentionné avant la foi en Dieu, bien que la foi soit la base même sur laquelle s'appuient les bonne oeuvres. 
p121
\end{quote}

\begin{quote}
L'Islam impose aux riches de consacrer une partie déterminée de leurs biens en faveur des pauvres, par ce sacrifice les riches élèvent une digue contre l'indigence, ils adoucissent la détresse des endettés, ils émancipent ceux qui plient sous le joug de la servitude, ils viennent en
aide aux voyageurs \sn{Payer les dettes des débiteurs malheureux. affranchir les esclaves et construire des
caravansérails constituaient dans les pays musulmans, des actes de générosité particulièrement
recommandes par la religion.}.
 
Il nous invite par-de,sus tout à dépenser notre bien pour les oeuvres
charitables et souvent il en fait l'expression de la foi et la manifestation
d'une bonne conduite; il déracina par là, du coeur des pauvres, la rancune
et la haine contre ceux que Dieu a favorisé des biens terrestres ; il leur
inspira l'amour pour les riches, tout comme il fît naître dans le coeur de
ceux-ci la pitié pour les malheureux ; ainsi il développa la confiance dans
le coeur de tous les hommes. Quel remède plus efficace contre les maux
dont souffre la société: « C'est une faveur que Dieu accorde à qui il veut,
car Dieu est d'une bienfaisance sans bornes. » (Cor. ch. 57, v. 21.)
\textbf{L' Islam a fermé les deux portes du mal, il a bouché les deux sources
qui minent l'intelligence et détruisent la richesse, en frappant les boissons
enivrantes, les Jeux de hasard et l'usure, d'une interdiction absolue qui
n'admet pas d'infraction.}

Enfin l'Islam n'a pas laissé une seule des vertus principales sans
en parler, une seule source de bonnes oeuvres sans la vivifier une seule
loi de. l'ordre sans la préciser. Il prépara, pour l'homme arrivé à sa
maturité, l'émancipation de l'esprit, l'indépendance de la raison dans
ses recherches, et, comme conséquence de cette émancipation et de
cette indépendance, l'épanouissement de ses facultés naturelles, le réveil
de sa volonté, son élan sur la voie de l'effort. Celui qui lit le Coran,comme
1I_ doit être lu, y trouve, sous ce rapport, des trésors inépuisables et des
richesses sans fin.
p 122

\end{quote}


\section{L'imam Mohammad 'Abdoh et la Caisse d'Epargne (1903-1904)}

\mn{Jomier Jacques. L'imam Mohammad 'Abdoh et la Caisse d'Epargne (1903-1904). In: Revue de l'Occident musulman et de la
Méditerranée, n°15-16, 1973. Mélanges Le Tourneau. II. pp. 99-107;}
\cite{Jomier:AdbouCaisseEpargne}
\begin{quote}
    L'on pense couramment dans les milieux d'orientalistes que l'Iman
Mohammad *Abdoh aurait approuvé certaines opérations financières au sujet des
quelles l'Islam traditionnel restait réticent. L'on s'appuie sur l'autorité d'Ignace
Goldziher qui, dans ses Vorlesungen ùber den Islam publiées en 1910, écrivait :
"Le mufti égyptien Cheikh Muhammad 'Abduh, mort en 1905, a trouvé le
moyen, dans une savante fatwa sur la matière, de présenter comme licite pour la
société musulmane, au point de vue de la loi religieuse, la Caisse d'Epargne et le
gain des dividendes" (1).
Par contre les études parues en Proche Orient ne parlent guère d'une telle
fatwa alors que d'autres consultations juridiques de Mohammad *Abdoh ont eu un
retentissement considérable. Pourquoi cette différence d'attitudes ?
\end{quote}

\begin{quote}
    1 ) En ce qui concerne les dividendes, je n'ai jusqu'ici rien rencontré dans les
oeuvres de Mohammad 'Abdoh qui en proclame formellement la licéité. Par contre
les principes invoqués à deux reprises, dans une fatwa qu'il a donnée à
l'instigation de la Compagnie d'assurance-vie Gresham (1903) et dans la
modification de la loi sur la Caisse d'Epargne, soumise à son approbation (1904), vont
dans le sens de la légitimation. Dans les perspectives du réformisme musulman,
cette question d'ailleurs est loin d'être insoluble car une société qui émet des
actions peut être assimilée à une société en commandite. Les dividendes
représentent alors la part des bénéfices proportionnelle au capital engagé... 
al-Banna, fondateur des Frères Musulmans en Egypte, avait admis la licéité de ce
genre d'opérations et des Frères avaient, peu avant 1 948, fondé quelques ateliers
ou sociétés financées par ce procédé (2). La difficulté principale vient de ce que la
langue arabe n'a pas de mot spécial pour désigner les dividendes et que le terme
fâ'ida a des relents de ribâ.
\end{quote}

\begin{quote}
    2) En second lieu, de quelle fatwa s'agit-il ? Jusqu'ici les efforts pour
retrouver le texte précis d'une fatwa de l'Iman concernant les Caisses d'Epargne et
les dividendes sont restés vains. Dans sa traduction arabe des Vorlesungen de
Goldziher, le Dr Mohammad Youssef Mousa a une note pour dire qu'il n'a pas vu
personnellement cette fatwa. Il en ignore même le contenu. Il suppose que le
demandeur de la fatwa n'a pas touché à la question des intérêts déterminés
(tah'dîd al-fâ'ida). "Ce que nous savons, ajoute-t-il, c'est que le prêt à intérêt
(fâ'ida) est de l'usure (ribâ) interdite et que personne n'a le pouvoir de le rendre
licite" (3). La traduction du passage de Goldziher cité ici plus haut est d'ailleurs
infléchie dans le sens de la Caisse d'Epargne ; la mention des dividendes a disparu.
Ceux-ci sont examinés dans la phrase suivante et désignés par une périphrase.
\end{quote}

\begin{quote}
    Il existe bien une fatwa donnée par l'Iman Moh'ammad 'Abdoh, en tant que
mufti d'Egypte, en 1903. Ce fut à la requête d'un particulier mais en fait pour la
Compagnie d'Assurances sur la vie Gresham. La Compagnie d'Assurances al-Chark,
au Caire, en conservait une copie afin de la montrer à ses clients, le cas échéant.
La traduction française de cette copie se trouve dans mon étude Le Commentaire
Coranique du Manor (4). Le texte arabe en avait d'ailleurs été publié dans la revue
al-Manâr à l'instigation de lecteurs tunisiens alertés par les journaux al-Wat'an et
al-Maghrib. Ces deux feuilles en avaient parlé, avaient donné le texte ; peu après,
la fatwa avait été l'objet d'un article dans al-Zahra de Tunis. La revue al-Manâr
disait tenir le texte de ce que les deux premiers journaux avaient publié (5).
Il s'agit dans la fatwa d'un contrat passé entre un client et une société. Le
client effectue des versements réguliers et périodiques par tranches successives
prévues d'avance ; la compagnie les encaisse et se charge de faire fructifier cet
argent. Finalement, à l'échéance, la société remboursera le total des versements
effectués augmenté des bénéfices résultant de la fructification. La réponse admet
la licéité d'une telle opération en des termes qui, au fond, s'appliqueraient à toute
société en commandite.
\end{quote}

\begin{quote}
    La revue al-Manâr affirme que la Compagnie d'Assurances Gresham a rajouté
entre parenthèse son propre nom à côté du terme général de société. Pourtant
dans l'extrait de fatwa montré par la compagnie al-Chark et provenant des
Tribunaux Charéis, cette parenthèse figure nettement. Le ton sur lequel la revue
parle de cette fatwa est plus que réservé. Elle n'en nie pas l'authenticité alors que
riman Moh'ammad 'Abdoh était encore vivant et tout proche. On peut tenir la
fatwa pour authentique car l'Iman aurait aussitôt inspiré une protestation s'il y
avait eu un faux (6). Il y a eu probablement dans l'utilisation de ce texte quelque
chose de tendancieux qui a indisposé les musulmans. La revue proteste en effet,
signalant que si la réponse correspond bien à la question du demandeur, cette
question est loin de s'appliquer au cas des assurances sur la vie. Aurait-on profité
de la fatwa pour en tirer des conclusions indues? La question reste toujours
mystérieuse.
\end{quote}

\begin{quote}
    Quant à une fatwa concernant" directement la Caisse d'Epargne, nous en
ignorons l'existence. Le 5 décembre 1903, la revue al-Manâr (p. 717) affirme
qu'une telle fatwa (objet de commentaires de la part du public) n'existait
absolument pas. L'histoire des débuts de la Caisse d'Epargne montre seulement
que Moh'ammad *Abdoh a joué un rôle lors de la rédaction d'un nouveau texte de
loi, à ce sujet, texte remaniant complètement la rédaction de la loi primitive. Il fit
partie d'une commission d'Azhariens chargés d'élaborer le projet et la loi lui fut
soumise, en tant que mufti, avant sa promulgation. Et ceci nous conduit au
troisième élément de la phrase de Goldziher : la Caisse d'Epargne.
3) C'est en effet la Caisse d'Epargne elle-même qui offre la piste de
recherches la plus intéressante. Les textes officiels de lois, leur histoire, les
principes mis en jeu méritent en effet que nous nous y arrêtions.
\end{quote}

C'est très intéressant parce que face aux scrupules de 3000 musulmans de toucher les intérêts (fa'ida), il a fallu une note du Manâr.
\begin{quote}
    La note du Manâr (5 décembre 1903)
Cette note fournit un certain nombre d'informations sur la pensée de
Moh'ammad 'Abdoh. La modification de la loi sur la Caisse d'Epargne partit d'une conversation privée entre lui et le directeur des Postes d'alors (Çaba Pacha
probablement). Ils parlèrent des trois mille usagers de la Caisse d'Epargne qui ne
retiraient pas les intérêts auxquels ils avaient droit alors que les autres usagers le
faisaient (11). Dans la conversation, le mufti maintenait fermement le principe de
l'interdiction absolue de l'usure (al-ribâ) ; mais il ne classait pas le cas de la Caisse
d'Epargne avec ceux d'usure. Il l'assimilait à celui d'une société en commandite
(chirkat al-mod'âraba).
A la suite de cette conversation, deux commissions d'azhariens furent
formées pour étudier la question. L'une d'elles était présidée par le mufti.
\end{quote}

\begin{quote}
    La modification de la loi (loi du 14 février 1904)
La difficulté principale venait de ce que dans une société en commandite la
part des bénéfices varie suivant l'état des affaires tandis que dans le cas de la
caisse d'épargne cette part est fixée une fois pour toutes. Voici comment la
formulation de la loi tourna les difficultés.
L'article premier est rédigé de façon à souligner le caractère de "société créée
pour faire fructifier en commun les dépôts de ses clients" que l'on veut donner à
la caisse d'épargne. Il est prévu dans l'article que le client devra signer un
formulaire imprimé dans lequel il déclarera donner au Directeur Général de la
Poste tout pouvoir pour faire fructifier les sommes déposées, d'une façon licite et
en excluant toute opération usuraire. Il déclare permettre au Directeur de la Poste
de joindre ses dépôts à ceux des autres clients pour les faire fructifier en commun
à condition de recevoir une part de bénéfices (al-ribh') proportionnelle à ses
versements. On notera que dans cet article comme dans tout le reste de la loi le
mot de \textit{fâ'ida}, intérêt, est totalement absent.
\end{quote}

\begin{quote}
    L'article deux stipule entre autres ce qui suit. Il évite de parler de tant pour
cent, formule qui rappelle trop les intérêts. Il note que la part des bénéfices ne
dépassera pas un pour quarante du capital et le surplus, s'il y en a, reviendra à
l'administration postale à titre de compensation pour les services rendus et les
frais que ceux-ci comportent. L'assimilation à la société en commandite est ainsi
mise en accord avec le fait que la proportion des bénéfices touchés est fixe. On
notera que la proportion de un pour quarante est familière aux oreilles des
musulmans pieux puisque dans un certain nombre de cas le montant de la Zaka
atteint ce chiffre.
\end{quote}

Voir la Zaka ?

\begin{Def}[Zakat]
La zakat est le troisième pilier de l'islam et son essence même révèle l'importance de la participation sociale dans l'univers musulman. La zakât est clairement un impôt sur l'avoir et la propriété qu'il faut comprendre, d'abord, comme une obligation devant Dieu. Ce prélèvement purifie sur le plan religieux, sacré et moral le bien de celui qui le possède.
Les différents types de biens soumis à la Zakat
Sont soumis à la zakat quatre types de biens 3 :

Avoirs/biens et fortune (espèces, métaux précieux, dépôts ou titres bancaires) ou zakat al maal
Les récoltes
Fonds de commerce (sur tout bien destiné à la vente)
Les bestiaux (ovins, bovins, ou encore camélidés)
Non soumis à la zakat :

Terrain, immeuble, bâtiment (non destinés à la vente)
Mobiliers, vêtements, voitures, etc.
Hypothèque
Bijoux personnels (pour les femmes suivant l'école Chafiite leurs parures en or sont exemptés de zakat, dans les trois autres écoles, tous les bijoux sont exemptés de zakat sauf l'or et l'argent)
\end{Def}

\paragraph{un pour quarante}
 La zakat constitue 2.5 \% du chiffre annuel épargné. (Elle ne s'applique pas sur les bijoux personnels en or pour les femmes chafiites).

La zakat est obligatoire sur l'argent économisé et qui a été immobilisé un an durant  
 Tous les ans les musulmans doivent se renseigner sur le prix du gramme d'or du pays où ils résident et le multiplier par 85 pour connaître le seuil d'imposition de la zakat sur leur argent personnel. Le minimum imposable est estimé en dollars et en d’autres billets de banque à l’équivalent de 20 mithqal d’or ou 1400 mithqual d’argent selon leur prix du moment où vous devez acquitter la zakat en dollar ou en d’autres monnaies.
 
 \begin{quote}
     La seconde note du Manâr (18 mars 1904)
II s'agit d'une question posée par un lecteur (réel ou fictif ? ) et qui concerne
cet amendement législatif du 14 février 1904. Il est noté dans la réponse que les
nouveaux textes législatifs ont été approuvés par le Mufti (c'est à dire Moh'ammad
'Abdoh). La note explique que les opérations prévues par la loi sont entièrement
différentes des opérations usuraires condamnées par le Coran. L'Administration
des Postes n'emprunte pas parce qu'elle serait dans le besoin : son but est
uniquement d'être au service de tous. Cette façon d'agir est à ranger dans la
catégorie de la vente et non pas dans celle de l'exploitation d'un homme dans le
besoin et tous en profitent. Le dépôt des sommes à la Caisse d'Epargne rentre
donc dans la catégorie juridique des contrats.
 \end{quote}
 
 \paragraph{Texte de la fatwa et de la note du manâr du 18 mars 1904}
 \begin{quote}
     Réponse. L'amendement législatif auquel vous faites allusion a été élaboré
d'après les vues d'une commission d'Oulémas d'al-Azhar. Celle-ci avait été formée
par le Khédive afin que les opérations de la Caisse d'Epargne soient conformes
aux principes du droit religieux. Ces Oulémas donnèrent leurs conclusions et les
envoyèrent au gouvernement. Celui-ci les soumit au Mufti et lorsqu'elles eurent
été approuvées par lui, le gouvernement les fit appliquer. Voilà ce qui est de
notoriété publique. Quant à nous, nous n'avons pas à donner de fatwa sur ce
qu'ils ont écrit, pour manifester notre opinion sur la question. Mais en tout cas,
nous ne pensons pas qu'il soit mal de mettre en pratique ces conclusions. Car nos
critiques visent uniquement celles des ruses juridiques propres aux Oulémas
casuistes et formalistes (aux Oulémas al-rusûm\sn{Intelligent / rusé ?} comme dit al-Ghazali) qui vont
contre les véritables buts de la loi religieuse tels qu'ils sont solidement établis par
le Coran et la tradition. Ainsi par exemple les ruses pour échapper au paiement de
la Zaka, ou les ruses concernant l'usure véritable, celle dont le Coran justifie
l'interdiction par le fait qu'il demande : "vous ne lésez point et vous n'êtes pas
lésés" (Coran, 2,279). Et ailleurs la différence entre l'usure et le commerce est
expliquée par ces mots du Coran : "Dieu vous a permis la vente et il vous interdit
l'usure" (Coran 2,275). Car passer un contrat pour une opération qui profite à
celui qui cède et à celui qui acquiert est une vente ou une opération de commerce.
Et de même ce texte qui explique la raison de l'interdiction par ces mots : "O
vous qui croyez ! Ne vivez pas de l'usure multipliant de double en double"
(Coran 3,130).
Et tout cela parce qu'il y avait à Médine et ailleurs des Juifs et des païens
qui prêtaient aux gens dans le besoin à des conditions honteuses d'usure, comme
nous y ont habitués en Egypte les Juifs et les messieurs étrangers (al-khawâgât),
avec la ruine des maisons qui en est la conséquence. La raison de cette
interdiction de l'usure est de faire cesser l'injustice et de conserver les vertus de
solidarité et de bonté mutuelle. Ou si vous préférez, pour que le riche n'exploite
pas le pauvre qui a besoin de lui (comme le dit l'Iman Mohammad 'Abdoh). C'est
en effet le sens du verset coranique : "Vous aurez vos capitaux, ne lésant point et
n'étant pas lésés" (Coran 2,279).

Il ne vous échappera pas que cette façon d'agir sert à la fois l'intérêt de celui
qui cède et de celui qui acquiert : elle est miséricordieuse pour eux et son absence
signifierait un manque de profit pour tous les deux. Elle n'est donc pas concernée
par les raisons données dans ce verset "ne lésant point et n'étant pas lésés". Elle
en est tout le contraire. Car une façon d'agir qui a pour but la vente et l'achat,
non pas le prêt à des créanciers dans le besoin, est à ranger dans la catégorie de la
vente et non pas dans celle de l'exploitation d'un homme dans le besoin. Il ne
vous échappera pas que l'Administration des Postes est un service gouvernemental
riche et qu'elle fait fructifier les sommes qui sont déposées dans les caisses
d'épargne (p. 29). Le déposant en profite aussi bien que les employés de
l'administration et du gouvernement. Aucun d'eux ne lèse l'autre. Aussi le plus probable
est-il que les conclusions de la commission ne sont pas une ruse légale : il s'agit
seulement d'une véritable association entre ceux qui apportent l'argent et d'autres
qui le font fructifier. Aussi je ne vois aucun empêchement à ce qu'on mette en
pratique l'amendement qu'elle propose en sorte qu'au point de vue du fiqh, l'on
considère l'affaire comme un contrat [. . .]".N.B. Nous ne traduisons pas les
dernières lignes qui ne font que se répéter.
(Revue al-Manâr, tome 7, pp. 28-29)
Jacques JOMIER
 \end{quote}