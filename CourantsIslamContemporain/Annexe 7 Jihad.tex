\hypertarget{tuxe9moignages-de-combattants-de-lei}{%
\section{Témoignages de combattants de
l'EI}\label{tuxe9moignages-de-combattants-de-lei}}
\mn{
\href{http://www.oasiscenter.eu/fr/articles/revolutions-arabes/2015/02/03/aller-retour-d-un-foreign-}{\underline{http://www.oasiscenter.eu/fr/articles/revolutions-arabes/2015/02/03/aller-retour-d-un-foreign-}}
\underline{fighter}
}
\hypertarget{aller-retour-dun-foreign-fighter}{%
\subsection{Aller retour d'un foreign
fighter}\label{aller-retour-dun-foreign-fighter}}

L'obligation morale de faire quelque chose pour défendre son peuple des
atrocités commises par des régimes autoritaires : voilà l'impulsion qui
pousse en 2011 Sam Najjari, de père libyen et de mère irlandaise, à
quitter une vie aisée et « intégrée » à Dublin pour combattre en Libye
et puis en Syrie.

Une expérience singulière dans le phénomène hétérogène des foreign
fighters qui implique aujourd'hui des centaines de jeunes européens,
poussés par des motivations différentes.

Marialaura Conte \textbar{} mardi 3 février 2015

Sam Najjair, 35 ans et d'origine libyenne et irlandaise, n'éprouve
aucune difficulté à raconter son histoire de combattant étranger, son
voyage aller retour. Il est motivé par la volonté d'expliquer ce qui
pousse un jeune à quitter son milieu « occidental » afin de partir pour
une guerre lointaine et puis, après l'immersion dans la violence, à
rentrer chez lui. Même si son expérience reste particulière, elle
éclaire les motivations de beaucoup d'autres comme lui.

Né et élevé à Dublin dans une famille musulmane, de père libyen et de
mère irlandaise, une des premières femmes à se convertir à l'Islam dans
son pays, Sam a été éduqué au respect des valeurs religieuses, mais sans
fondamentalismes particuliers. Si bien que jeune adulte, il s'est
éloigné de la pratique religieuse, comme beaucoup de ses amis, d'autres
intérêts ayant pris la première place. Mais à un certain moment, le lien
avec la patrie lointaine de son père, la Libye, a joué un rôle décisif.

« J'avais 9 ans quand je suis allé pour la première fois en Libye,
j'étais petit et très impressionnable. J'ai appris rapidement l'arabe,
en six mois environ, ce qui m'a permis d'entrer en grande empathie avec
le pays et avec ceux qui, comme moi, avaient des origines différentes.
Après ce premier séjour, je suis rentré à Dublin où j'ai passé toute mon
adolescence.

Ce fut lors de mon voyage d'adulte que j'ai eu la perception évidente de
ce que cela signifie vivre sous une dictature. À l'époque, il y avait
l'embargo et j'ai pu voir ce qu'étaient les punitions collectives de
Kadhafi. Après deux ans, je suis rentré à Dublin, où j'ai vécu
normalement, tournant le dos à cette expérience ; je n'étais même plus
pratiquant, j'ai vécu ma vie et j'ai obtenu une licence en informatique,
parce que ces années-là le secteur de l'informatique était en ébullition
en Irlande ».

\hypertarget{et-puis-le-printemps-arabe...}{%
\subsection{Et puis le printemps
arabe...}\label{et-puis-le-printemps-arabe...}}

« Puis 2011 est arrivé : cette année-là, j'ai eu la possibilité de me
racheter après une longue période durant laquelle j'avais vécu sans rien
faire, et la Libye a eu l'occasion d'essayer de résoudre ses problèmes.
Pendant des mois, j'ai suivi les événements à la télévision et j'étais
très inquiet parce que je savais ce que le régime était capable de faire
et qu'il pouvait tuer de nombreux innocents. Ce qui me choqua le plus
fut l'intervention des mercenaires serbes arrivés en Libye, payés par
Kadhafi pour combattre afin de défendre la dictature. Je n'avais pas
beaucoup étudié l'histoire et je ne connaissais pas les mécanismes d'une
intervention armée. Ma vie a pris un virage lorsque j'ai découvert le
nombre de mercenaires qui combattaient dans nos villages, payés avec de
l'argent, mais aussi de la drogue et de l'alcool. C'en était trop, je ne
pouvais plus rester sans rien faire devant les atrocités dont parlaient
les journaux télévisés. Je me souviens avoir fait un pas en arrière et
avoir dit : « Mes amis sont là. Je dois faire quelque chose ! J'y vais!
». Je devais tout quitter et partir. C'était le rappel

de la patrie, je devais combattre les injustices. J'étais prêt à
effectuer n'importe quel service, mettre à disposition ma connaissance
de l'anglais pour parler avec les médias, m'occuper de la distribution
des biens alimentaires, et même à utiliser des armes ».

\hypertarget{de-liduxe9al-au-ruxe9el-quand-vous-uxeates-arrivuxe9-luxe0-au-front-que-sest-il-produit}{%
\subsection{De l'idéal au réel : quand vous êtes arrivé là, au front, que
s'est-il produit
?}\label{de-liduxe9al-au-ruxe9el-quand-vous-uxeates-arrivuxe9-luxe0-au-front-que-sest-il-produit}}

« Quand tu arrives là, tu plonges dans quelque chose de totalement
nouveau. Dans un conflit militaire, tu es nu, tu es habillé mais c'est
comme si tu étais nu : tu n'as pas d'armes et tu n'es pas entraîné. J'y
suis allé, j'ai rejoint une brigade à laquelle appartenait mon
beau-frère et je me suis entraîné avec eux pendant des mois. Dans ce
genre d'entraînement, tu es jeté immédiatement dans le conflit, parce
que c'est la meilleure manière d'apprendre. Par exemple, j'ai appris
beaucoup sur le langage du corps durant les interrogatoires. Après deux
mois, j'étais au combat et j'y suis resté huit mois, à partir de juin
2011. Lorsque nous sommes arrivés aux portes de Tripoli pour libérer la
ville (j'étais avec environ quatre cents hommes), nous avons compris la
portée du conflit, mais aussi que les choses empiraient. Puis après
quelque temps, j'ai décidé d'aller en Syrie pour rejoindre les groupes
rebelles contre Assad ».

\hypertarget{combien-votre-uxe9ducation-musulmane-a-t-elle-influencuxe9-votre-duxe9cision-de-partir}{%
\subsection{Combien votre éducation musulmane a-t-elle influencé votre
décision de partir
?}\label{combien-votre-uxe9ducation-musulmane-a-t-elle-influencuxe9-votre-duxe9cision-de-partir}}

« Mon voyage en Libye comme celui en Syrie n'avaient aucun fondement
religieux. Je suis parti, comme des milliers d'autres hommes, motivé par
un esprit patriotique, et pas pour des raisons religieuses. Je suis allé
en Syrie, avec d'autres brigades libyennes, pour donner ce que nous
avions appris de la révolution de notre pays : ce n'étais pas une guerre
« normale », mais une révolution et il fallait donc la combattre comme
telle. C'est pour cette raison que nous sommes partis au nord de la
Syrie. Je voulais voir le peuple syrien libéré de la dictature. Voilà la
motivation, pas la religion ».

\hypertarget{quel-est-votre-rapport-avec-la-pratique-de-la-violence}{%
\subsection{Quel est votre rapport avec la pratique de la violence
?}\label{quel-est-votre-rapport-avec-la-pratique-de-la-violence}}

« Je considère la violence comme le mal nécessaire. De nombreuses
personnes deviennent violentes parce que la violence se présente à leur
porte, c'est une réaction nécessaire. L'EIIL ne représente pas l'Islam,
c'est une dérive dangereuse. En 2012, lorsque je suis arrivé en Syrie,
l'EIIL n'était pas encore impliqué complètement dans le conflit, mais il
commençait à recruter des jeunes. Je crois que c'est ce qui s'est
produit avec de nombreux combattants, comme certaines histoires de
personnes rencontrées le prouvent : de laïcs et libéraux, en passant par
l'expérience dramatique de la guerre, ils deviennent religieux et
fanatiques. Parce que si tu te trouves dans une situation horrible comme
la guerre, tu es en danger, et quand tu entends les gens parler d'un
au-delà, au fond tu voudrais te sentir en paix en quelque sorte, et tu
commences à penser qu'ils ont peut-être raison. Peu importe ta soit,
quand tu te retrouves dans une situation où ta vie est en danger, tu
voudrais au-moins être du côté des bons. C'est pour cette raison que les
personnes deviennent religieuses une fois arrivées là- bas.

L'EIIL est devenu attrayant parce qu'il prend les hommes vulnérables de
pays en conflit et qu'il leur donne des compagnons et des instruments
pour se défendre dans un lieu qui ressemble au film d'horreur le plus
terrible. Le pourcentage des personnes qui partent déjà avec un objectif
fondamentaliste est très faible, mais après, chacun devient extrémiste.
C'est difficile à expliquer, mais quand tu te retrouves là-bas, tu as
tellement peur que tu ne peux rien faire d'autre que de penser que tu es
là pour la bonne cause. Tu ne sais pas qui sont tes amis et tu ne
connais pas tes ennemis, en tant que combattant étranger, tu ne connais
même pas le pays et tu dois être attentif parce que les dynamiques sont
différentes, tu n'es pas chez toi, tu pourrais être vendu, enlevé. Le
principal c'est de faire attention : on ne peut pas attribuer la même
étiquette à tout le monde, un journal italien m'a collé l'étiquette de
jihadiste, mais je ne le suis pas! ».

\hypertarget{avez-vous-eu-des-probluxe8mes-quand-vous-uxeates-rentruxe9-chez-vous-comment-avez-vous-uxe9tuxe9-accueilli}{%
\subsection{Avez-vous eu des problèmes quand vous êtes rentré chez vous ?
Comment avez-vous été accueilli
?}\label{avez-vous-eu-des-probluxe8mes-quand-vous-uxeates-rentruxe9-chez-vous-comment-avez-vous-uxe9tuxe9-accueilli}}

« Cela aussi est très intéressant, parce que cela montre combien les
systèmes médiatiques et l'opinion publique sont changeants. En 2012,
nous qui avions combattu les révolutions nous étions considérés comme
des héros, même par les médias occidentaux. Lorsque j'étais en Syrie
l'EIIL n'existait pas, il y avait encore la révolution. Mais maintenant
tout a changé : il y a une guerre en Syrie aujourd'hui, totalement
différente d'il y a trois ans. Aujourd'hui, je ne pourrais jamais
partir, je ne pourrais même pas combattre dans le même camp qu'en 2012
parce qu'on m'arrêterait une fois rentré en Occident ».

\href{http://www.oasiscenter.eu/fr/articles/revolutions-arabes/2015/02/03/\%C3\%A9gorgez-les-}{\underline{http://www.oasiscenter.eu/fr/articles/revolutions-arabes/2015/02/03/\%C3\%A9gorgez-les-}}
\underline{m\%C3\%A9cr\%C3\%A9ants-pour-obtenir-la-satisfaction-du-mis\%C3\%A9ricordieux}

\hypertarget{uxe9gorgez-les-muxe9cruxe9ants-pour-obtenir-la-satisfaction-du-misuxe9ricordieux}{%
\subsection{« Égorgez les mécréants pour obtenir la satisfaction du
Miséricordieux »
!}\label{uxe9gorgez-les-muxe9cruxe9ants-pour-obtenir-la-satisfaction-du-misuxe9ricordieux}}

Ce sont de véritables coups de sabre : les paroles du testament que
Islam Yakan, jeune jihadiste égyptien, a lancées sur Twitter peu avant
de mourir dans une attaque suicide en Syrie, frappent par leur
radicalisme. S'appuyant sur des références constantes au Coran et aux
hadiths, elles débordent de haine envers ceux qu'il appelle mécréants,
et manifestent une tension extrême pour atteindre le prix qu'est le
Paradis. Elles aident à comprendre de l'intérieur le phénomène du
jihadisme.

Rédaction \textbar{} mardi 3 février 2015

On a beaucoup parlé de Islam Yakan en Égypte au cours de l'année qui
s'est écoulée : un jeune homme de famille aisée parti combattre le jihad
en Syrie dans les rangs de l'État islamique. En septembre dernier, un
journal égyptien avait reconstruit son parcours, recueillant les
témoignages de parents, voisins et amis. Mais c'est le jeune homme
lui-même qui a révélé les détails de son enrôlement dans un récit publié
sur le site justpaste.it. Islam raconte comment, d'étudiant en droit peu
motivé à l'université `Ayn Shams du Caire, il est devenu combattant du
jihad en passant par un emploi d'entraîneur dans un centre sportif et la
prédication islamique dans les rues de la capitale : « Toute ma vie
était faite d'entrainement, de prédication, de mosquée et de
mémorisation du Coran ». L'idée du jihad s'insinue progressivement en
lui, jusqu'à devenir une obsession : « Nous avons vu la condition des
musulmans et de l'Islam en différents points du monde, de la Syrie à la
Birmanie et à la Palestine, et l'humiliation, l'asservissement, et la
faiblesse dans lesquelles ils se trouvent, et nous avons pensé
instinctivement au jihad et au combat, même si nous ne savions pas en
quoi cela consistait : nous en avions entendu parler uniquement dans les
récits et dans les livres, et peut-être à la télévision ou sur internet
avec Osama Bin Laden {[}\ldots{]} Nous avons commencé à parler du jihad,
mais c'était pour nous comme un rêve, une légende ».

Un jour, un de ses amis part pour la Syrie : à partir de ce moment-là
naît en lui aussi et brûle le désir de tout laisser pour aller
combattre, même si, à l'époque, il ne savait « rien du jihad, ni de son
fondement, ni de ce que prévoit à son sujet le credo islamique ». Il
demande des explications à un shaykh, mais ses réponses ne le satisfont
pas. Islam revient chez lui, allume la télévision sur la chaîne
consacrée au Coran, et entend le verset : « Dis : Si vos pères, vos
fils, vos frères, vos épouses, vos clans, les biens que vous avez
acquis, un négoce dont vous craignez le déclin, des demeures où vous
vous plaisez, vous sont plus chers que Dieu et son Prophète et la lutte
dans le sentier de Dieu : attendez-vous à ce que Dieu vienne avec son
Ordre. -- Dieu ne dirige pas les gens pervers » (Cor.

9,24). Mots fulgurants pour Islam : en ce moment même, sa décision est
prise. Après s'être procuré

avec quelque difficulté les documents nécessaires, il part pour la
Turquie, d'où on le fait ensuite passer en Syrie. Islam Yakan est mort
le 1er décembre dernier à Kobane en se faisant exploser dans une voiture
bourrée d'explosifs. Avant de mourir, il a diffusé sur Twitter son «
testament » chargé sur justpaste.it, puis repris par bon nombre de
journaux égyptiens.

Le texte, que nous reproduisons ci-dessous, est un collage de citations
du Coran et de traditions prophétiques, rapprochées pour expliquer la
noblesse de la mort sur la « voie de Dieu » et inviter d'autres
musulmans à suivre son exemple.

Sa biographie et ses propos (ou du moins la forme qu'ils ont assumée du
fait de la personne qui les a publiés sur Internet) font apparaître
combien le jeune homme cherchait à donner un sens définitif à sa vie, et
l'a trouvé dans le choix de l'action violente la plus abominable, comme
le confirme la suite impressionnante des actes qu'il encourage pour
punir les mécréants et obtenir la satisfaction divine.

\hypertarget{testament-spirituel-de-islam-yakan-diffusuxe9-via-twitter-adressuxe9-uxe0-ses-compagnons-de-combat.}{%
\subsection{Testament spirituel de Islam Yakan diffusé via Twitter, adressé
à ses compagnons de
combat.}\label{testament-spirituel-de-islam-yakan-diffusuxe9-via-twitter-adressuxe9-uxe0-ses-compagnons-de-combat.}}

Louange à Dieu seul et prière et paix sur celui après lequel il n'est
plus de prophète.

« Ceci est le message et le testament du serviteur de Dieu Islam Yakan,
appelé Abû Salma. {[}\ldots{]}

Avant tout je vous demande votre indulgence pour tout mal que je
pourrais avoir commis par mes paroles ou mes actions, consciemment ou
inconsciemment. Je déclare devant Dieu que je pardonne tous ceux qui
m'ont fait du tort, si graves qu'aient pu être leurs offenses, celles
que je connais ou celles que j'ignore. Le Très-Haut dit : « Nous n'avons
donné l'immortalité à nul homme avant toi.

Seraient-ils immortels, alors que tu mourras ? » (Cor. 21,34). Et
encore: « Dis : Oui, la mort que vous fuyez va vous rejoindre. Vous
serez ensuite ramenés devant celui qui connaît parfaitement ce qui est
caché et ce qui est apparent. Il vous informera de ce que vous faisiez »
(Cor. 62, 8). « Où que vous soyez, la mort vous atteindra, même si vous
vous tenez dans des tours fortifiées » (Cor. 4, 78). La mort rassemble
le musulman, l'athée et le mécréant ; elle arrive à l'improviste, et ne
fait pas de différence entre le petit et le grand, le malade et le bien
portant, selon le terme que Dieu a prescrit pour nous. {[}\ldots{]}

Le Très-Haut a dit : « Nous n'avons pas créé par jeux les cieux, la
terre et ce qui est entre les deux. Nous les avons créés en toute
Vérité, mais la plupart des hommes ne savent pas » (Cor. 44,38-39).
Sachez donc que nous n'avons pas été créés pour vivre comme des moutons
et manger, boire et nous reproduire. Nous avons été créés pour une fin
et un but qui est de professer le monothéisme dans notre cœur, dans nos
paroles et dans nos actions. Tel est le message que Dieu a envoyé à
travers tous ses messagers. {[}\ldots{]}

Tu obéiras aux ordres de Dieu, tu combattras sur Son sentier pour rendre
suprême sa Parole et faire régner sa loi, et tu vivras sous Son ombre
puissante et généreuse. Ou bien tu seras tué en marchant en avant et non
en reculant, fidèle au Très-Haut, et alors tu seras un martyr et c'est
ce qu'il y a de mieux pour vous, si vous le saviez. Le Très-Haut a dit :
« Dieu a acheté aux croyants, leurs personnes et leurs biens pour leur
donner le Paradis en échange. Ils combattent dans le chemin de Dieu :
ils tuent et ils sont tués. C'est une promesse faite en toute vérité
dans la Tora, l'Évangile et le Coran. Qui tient son pacte mieux que Dieu
? Réjouissez-vous donc de l'échange que vous avez fait : voilà le
bonheur sans limites ! » (Cor. 9,111).

Annonce à celui qui a choisi la voie de l'intégrité et le sentier du
jihad que maintenant Dieu lui prescrit d'accomplir sa hijra vers l'État
islamique. Celui-ci a un Calife, le prince des croyants Ibrâhîm

Ibn `Awwâd al-Badrî, auquel il doit prêter allégeance. L'Envoyé de Dieu
--la paix de Dieu soit sur lui -- a dit dans son hadîth authentique : «
Qui meurt sans avoir prêté allégeance {[}au Calife{]} mourra dans
l'ignorance ». {[}\ldots{]}

O Frères du monothéisme qui vous trouvez en différents points du monde,
et en particulier vous du Sinaï, Dieu est témoin de l'amour qu'en Lui
j'éprouve pour vous. Je demande au Très-Haut que, par cet amour, il nous
réunisse à l'ombre de Son trône, dans le Paradis le plus élevé. Veillez,
soyez patients, soyez fermes dans la lutte et craignez Dieu afin que
vous puissiez prospérer. Combattez les ennemis de Dieu parmi les
mécréants adorateurs de la croix, les juifs et les tyrans apostats qui
gouvernent les arabes, leurs armées et leurs partisans. Égorgez-les de
vos épées, ouvrez-leur la tête avec vos projectiles, faites sauter en
l'air leurs corps avec vos ceintures explosives, et n'oubliez pas les
voitures piégées, car ce sont les armes les plus terribles et les plus
dévastatrices et qu'elles comptent parmi les actions privilégiées pour
obtenir la satisfaction du Miséricordieux. Sachez que Dieu vous
soutiendra si vous agissez pour soutenir sa religion et ses serviteurs
qui sont opprimés, parce que le Très-Haut a dit : « O vous qui croyez :
si vous aidez Dieu, il vous secourra et il affermira vos pas. -- Malheur
aux incrédules car Dieu rendra vaines leurs œuvres. Il en est ainsi,
parce qu'ils ont éprouvé de l'aversion pour ce que Dieu a révélé : il
rendra vaines leurs œuvres ». (Cor. 47,7-9).

Sachez que « le Paradis est à l'ombre des épées », et ne désirez d'autre
but que le martyre.

Ma mère, mon père, et mes frères ! Je demande à Dieu de me pardonner,
moi et eux aussi, et d'en avoir pitié parce qu'ils m'ont élevé depuis
mon enfance ; je leur recommande de prendre les distances des tyrans
infidèles qui gouvernent le pays sans la Loi de Dieu. Le Très-Haut a dit
: « Les incrédules sont ceux qui ne jugent pas les hommes d'après ce que
Dieu a révélé » (Cor. 5,44). Eux au contraire gouvernent avec la
constitution, la démocratie et la laïcité et sont amis des mécréants,
juifs et chrétiens et d'autres idolâtres. Le Très-Haut dit : « Celui
qui, parmi vous, les prend pour ami, est des leurs » (Cor. 5,51). Je
leur recommande d'émigrer dans l'État islamique pour y vivre le reste de
leurs jours sous la Loi de Dieu Tout-Puissant : ils comprendront alors
les choses de leur religion que les tyrans leur ont tenues cachées dans
leur pays. Je leur recommande de trouver leur réconfort dans la Sunna du
Prophète -- que Dieu le bénisse et lui donne la paix -- quand ils
apprendront mon martyre, si telle sera la volonté de Dieu. Ils ne
devront pas déchirer leurs vêtements ni se frapper le visage, ils ne
devront pas pleurer pour moi, ni réciter les prières des païens, ils ne
devront pas porter le deuil au-delà des trois jours, ni recevoir les
condoléances, parce que tout cela n'appartient pas à la religion de Dieu
et j'en suis innocent. Qu'ils se rappellent les paroles du Très-Haut : «
Ne dites pas de ceux qui sont tués dans le chemin de Dieu : ils sont
morts ! Non ! Ils sont vivants, mais vous n'en avez pas conscience. Nous
nous éprouvons par un peux de crainte, de faim, par de pertes légères de
biens, d'honneurs ou de récoltes. Annonce la bonne à ceux qui sont
patients, à ceux qui disent, lorsqu'un malheur les atteint : nous sommes
à Dieu et nous retournons à lui. Voilà ceux sur lesquels descendent des
bénédictions et une miséricorde de leur Seigneur. Ils sont bien dirigés
» (Cor. 2,154- 157).

Je m'adresse enfin à mon cher shaykh, le commandeur des croyants Ibrâhîm
ibn `Awwad : nous t'avons prêté allégeance et nous nous sommes engagés à
t'écouter et à t'obéir sans nous ménager, dans le bien et dans
l'adversité, dans la prospérité et dans les difficultés, et à instaurer
la religion de Dieu, à combattre le jihad contre tes ennemis, et à ne
pas contester l'autorité de ceux qui, parmi les tiens, détiennent
l'autorité, à moins qu'ils ne commettent des impiétés évidentes et dont
Dieu nous montre les preuves. Continue ce que tu es en train de faire,
n'adule pas les mécréants et les apostats, ne te vante pas du succès que
Dieu t'a concédé, et tu ne devras pas craindre Ses embuches. Crains Dieu
en ton âme, en tes sujets et en tes troupes, et rappelle-toi que la
victoire que Dieu concède à ses serviteurs dépend de la mesure avec
laquelle ses serviteurs œuvrent pour Lui et pour sa religion.

Si nous agissons autrement, nous ne pouvons espérer que Dieu nous aide.
Rappelle-toi ce qu'a dit le Très-Haut : « Lorsque ces gens eurent oublié
ce qui leur avait été rappelé, nous leur avons ouvert les portes de
toute chose ; mais après qu'ils eurent joui des biens qui leur avaient
été accordés, nous les avons emportés brusquement et ils se trouvèrent
désespérés. Tout ce qui restait de ce peuple injuste fut alors
retranché. Louange à Dieu, le Maitre des mondes ! » (Cor 6, 44-45).

« Vous vous souviendrez de ce que je vous dis : je confie mon sort à
Dieu. Dieu voit parfaitement ses serviteurs ! » (Cor. 40,44).

\begin{quote}
\textbf{Extrait de Al-Masry al-Yowm, 1 décembre 2014, traduction Oasis}
\end{quote}



\hypertarget{quelques-textes-des-leaders-dal-qaida}{%
\section{Quelques textes des leaders d'Al
Qaida}\label{quelques-textes-des-leaders-dal-qaida}}


\subsection{Usama ben Laden}

Chacun d'entre vous sait quelle injustice, quelle oppression, quelle
agression subissent les musulmans de la part de l'alliance judéo-croisée
et de ses valets ! A tel point que le sang des musulmans n'a plus aucun
prix, que leurs biens et leur argent sont offerts en pillage à leurs
ennemis. Leur sang coule en Palestine, en Irak et au Liban (les
horribles images du massacre de Qana sont encore présentes dans tous les
esprits), sans compter les massacres du Tadjikistan, de Birmanie, du
Cachemire, d'Assam, des Philippines, de Pattani, de l'Ogaden, de
Somalie, d'Erythrée, de Tchétchénie et de Bosnie-Herzégovine, où les
musulmans ont été victimes d'abominables boucheries. Et tout cela au vu
et au su du monde entier, pour ne pas dire en raison du complot des
Américains et de leurs alliés, derrière l'écran de fumée des Nations
Injustes Unies. Mais les musulmans se sont rendu compte qu'ils étaient
la cible principale de la coalition judéo-croisée, et toute cette
propagande mensongère sur les droits de l'Homme a laissé la place aux
coups portés et aux massacres perpétrés contre les musulmans sur toute
la surface de la terre.

La dernière calamité à s'être abattue sur les musulmans, c'est
l'occupation du pays des deux sanctuaires, le foyer de la maison de
l'islam et le berceau de la prophétie, depuis le décès du Prophète et la
source du message divin, où se trouve la sainte Kaaba vers laquelle
prient l'ensemble de musulmans, et cela par les armées des chrétiens
américains et leurs alliés ! Il n'y a de force et de puissance qu'en
Dieu ! {[}\ldots{]}

Lorsque les devoirs s'accumulent, il faut commencer par le plus
important : repousser cet ennemi américain qui occupe notre territoire,
c'est, après la foi, le premier des devoirs, rien ne peut le surpasser,
comme l'ont affirmé les oulémas. Ainsi, le cheikh de l'islam, Ibn
Taymiyya, a écrit : « Quant au combat pour la défense, c'est de défendre
à tout prix ce qu'il y a de plus sacré avec la religion. C'est un devoir
collectif. Car l'ennemi agresseur qui corrompt la religion et la vie,
rien n'est plus pressant, après la foi, que de le repousser, sans
condition, autant que faire se peut » (« Les morceaux choisis
scientifiques », annexe aux Grandes Fatwas, 4/608).

On ne peut repousser l'envahisseur qu'avec l'ensemble des musulmans, ces
derniers doivent donc ignorer ce qui les divise, provisoirement, car
fermer les yeux sur leurs divisions ne peut pas être plus grave que
d'ignorer l'impiété capitale qui menace les musulmans. {[}...{]}

Il y a quelques jours, les agences de presse ont transmis une
déclaration du ministre de la Défense américain, croisé et occupant,
dans lequel il disait qu'il n'avait retenu qu'une leçon des explosions
de Riyad et d'al-Khobar : ne pas reculer devant les lâches terroristes.
Eh bien, nous disons à ce ministre qu'il y a là de quoi faire rire même
une mère accablée par la perte de son enfant, car cela ne fait que
montrer la frayeur dont vous êtes saisi. Où était cette prétendue
bravoure, à Beyrouth, après l'attentat de 1403 {[}1983{]} qui a fait de
vos 241 Marines des paillettes éparpillées et des membres
déchiquetés\textsuperscript{32}, et où est cette prétendue bravoure à
Aden dont vous êtes partis précipitamment, vingt- quatre heures après
les deux attentats\textsuperscript{33} ? {[}...{]}

Je t'affirme, William, que ces jeunes-là aiment autant la mort que vous
aimez la vie, qu'ils ont hérité de l'honneur, de la fierté, de la
bravoure, de la générosité, de la sincérité, du courage et de l'esprit
de sacrifice, de père en fils, et leur endurance au combat se vérifiera
lors de l'affrontement, car ils ont hérité de ces qualités de leurs
ancêtres depuis l'anté-islam\textsuperscript{35}, avant que l'islam ne
les ancre en eux. Et ces jeunes gens dont vous prétendez qu'ils sont
lâches {[}...{]} ont porté les armes pendant dix ans en Afghanistan, en
jurant de poursuivre leur lutte contre vous jusqu'à ce que vous partiez,
vaincus, battus et penauds, si Dieu le veut, tant que le sang coulera
dans leurs veines et que les larmes s'écouleront de leurs yeux.

Ô Dieu, conforme la situation de ta nation en honorant ceux qui
t'obéissent et en avilissant ceux qui te désobéissent, conforte-la en
ordonnant le Bien et pourchassant le Mal.

Prière et salut sur ton adorateur et messager. Mohammad, sa famille et
tes compagnons.

Notre ultime prière sera : Louange à Dieu, seigneur des mondes !

Extraits de « Déclaration de jihad contre les Américains qui occupent le
pays des deux lieux saints » (1996). Traduction parue dans \emph{Al
Qaïda dans le texte}, G. Kepel et J-P Millelli dir., PUF, Paris, 2005,
p. 51-57.


\hypertarget{abdallah-azzam}{%
\subsection{Abdallah Azzam}\label{abdallah-azzam}}

 
Établir une société musulmane sur un territoire est pour les musulmans
une nécessité comme l'eau et l'air, et ce territoire n'existera que par
un mouvement islamique organisé qui s'engage dans le jihad, en actes et
en paroles, et fait du combat son objectif et sa protection. Le
mouvement islamique ne sera capable d'établir la société islamique que
grâce à un jihad popu- laire général, dont le mouvement sera le cœur
battant et le cerveau brillant, pareil au petit détonateur qui fait
exploser une grande bombe, en libérant les énergies contenues de l'oumma
et les sources de bien qu'elle retient dans son tréfonds. Les compagnons
du Prophète (que Dieu les agrée !) n'étaient qu'une poignée par rapport
aux musulmans qui renversèrent le trône de Chosroès et ternirent la
gloire de César.

Et même les tribus qui avaient apostasié l'islam sous le califat d'Abou
Bakr, Omar les envoya, après qu'elles se furent repenties, au combat
contre les Perses, et Toulayha ibn Khouwaylid al-Assadi1, qui avait
auparavant prétendu être prophète, fut l'un des héros de la bataille
d'al-Qadissiyya2, au point que Saad3 l'envoya en mission d'espionnage
chez les Perses où il fit preuve d'un grand courage.

Quant à la poignée d'officiers qui pensent pouvoir établir une société
musulmane, c'est une illusion et un leurre qui risque de répéter la
tragédie que vécut le mouvement islamique sous Gamal Abd al-Nasser.

Le mouvement jihadiste populaire, avec son long chemin à parcourir,
l'amertume des souffrances endurées, l'ampleur des sacrifices et
l'énormité des pertes, est de nature à purifier les âmes et à les élever
de la réalité {[}...{]}, les intérêts s'éloignent des conflits médiocres
au sujet des sous, des besoins à court terme, et {[}...{]} les haines
s'effacent et les âmes se policent, la caravane remonte de la vallée
encaissée vers le sommet élevé, loin des marais en putréfaction et des
combats de la jungle. {[}...{]}

La société islamique a besoin de naître, mais la naissance se fait dans
la douleur et la peine.

Extraits de « Rejoins la caravane », 1987. Traduction parue dans
\emph{Al Qaïda dans le texte}, G. Kepel et J-P Millelli dir., PUF,
Paris, 2005, p 169.
 

\hypertarget{ayman-al-zawahiri}{%
\subsection{Ayman al-Zawahiri}\label{ayman-al-zawahiri}}

 
La démocratie est une nouvelle religion, car si en islam la législation
vient de Dieu (qu'il soit exalté !), cette capacité en démocratie en
incombe au peuple. Il s'agit bien d'une nouvelle
religion qui repose sur la déification du peuple et qui lui confère le
droit de Dieu ainsi que Ses attributs ; cela revient à associer des
idoles à Dieu et à tomber dans l'impiété, car Dieu (qu'il soit exalté !)
a dit : \{Le jugement n'appartient qu'à Dieu. Il a ordonné que vous
n'adoriez que Lui.) (\emph{Youssouf})\textbf{4}.

La souveraineté en islam n'est qu'à Dieu, alors que dans la démocratie
elle est au peuple, c'est pourquoi Abou al-Ala al-Mawdoudi a dit de la
démocratie qu'elle est « une divinisation de l'homme {[}...{]} C'est le
pouvoir des masses » (\emph{L'Islam et la civilisation moderne}, p. 33).
En démocratie, le législateur est le peuple représenté par la majorité
des députés au Parlement.

Ces députés sont des hommes et des femmes, des chrétiens, des
communistes et des laïcs, ce qu'ils légifèrent devient une loi, qui
s'impose à l'ensemble du peuple, par laquelle on prélève des impôts et
par laquelle on exécute des gens. Prenons l'exemple de la constitution
égyptienne qui énonce, dans son troisième article, que « La souveraineté
n'appartient qu'au peuple qui est la source de tous les pouvoirs », et
dans son quatre-vingt sixième article : « Le conseil du peuple
--- le parlement --- détient le pouvoir législatif. » Ils ont fait ainsi
du peuple l'égal et le semblable de Dieu (qu'il soit exalté !). Dieu
(qu'il soit exalté !) a dit : \{Ont-ils des divinités qui auraient
établi pour eux des lois religieuses que Dieu n'aurait pas sanctionnées
7\} (\emph{Le conseil})\textbf{5}.

Selon l'une de ses définitions, la religion est un système de vie des
hommes, qu'il soit vrai ou faux, car Dieu a dit : \{A vous, votre
religion ; à moi, ma Religion\}\textbf{6}. Il (qu'il soit exalté !) a
donc appelé l'impiété des impies une religion ; par conséquent, ces
êtres humains qui légifèrent pour les gens en démocratie sont des idoles
(associées à Dieu) adorées à la place de Dieu, ce sont les seigneurs
cités par Dieu (qu'il soit exalté !) : \{Ne nous prenons pas les uns les
autres pour des seigneurs à la place de Dieu\} (\emph{La famille
d'Imrane})\textbf{7}.Y a-t-il une plus grande impiété? D'Adi ibn Hatim
(que Dieu l'agrée !), un chrétien converti à l'islam : «Je suis allé
voir le Messager de Dieu (que Dieu lui accorde salut et bénédiction !)
alors qu'il psalmodiait la sourate intitulée Le désaveu et en était à ce
verset : \{Ils ont pris leurs docteurs et leurs moines comme

seigneurs au lieu de Dieu ?\}\textbf{8}.

Je lui ai alors dit : "O messager de Dieu, nous ne les avons pas pris
comme seigneurs."

- "Si" a-t-il dit, "ne vous permettent-ils pas ce qui est interdit, qui
devient licite pour vous, et n'interdisent-ils pas ce qui est permis,
qui devient illicite pour vous ?"

-"Oui" lui ai-je dit.

-"Cela revient à les adorer" » (hadith, considéré comme bon, rapporté
par Ahmad et al- Tirmidhi\textbf{9}).

Al-Aloussi\textbf{10} a dit dans son exégèse de ce verset : « La plupart
des exégètes ont dit : ici "seigneurs" ne signifie pas "dieux", mais
qu'ils obéissent à leurs ordres et respectent leurs interdictions »
(cité dans \emph{A l'ombre du Coran}, vol. 3, p. 1642).

Nous citerons brièvement ce qu'a dit M. Sayyid Qotb (que Dieu ait son
âme !) à propos de la parole de Dieu (qu'il soit exalté !) : \{Nul parmi
nous ne se donne de seigneur, en dehors de Dieu\}\textbf{11}. «Cet
univers dans son entier ne vit et ne prospère que par l'existence d'un
Dieu unique, qui en est l'organisateur : \{Si des divinités, autres que
Dieu, existaient, le ciel et la terre seraient corrompus\}\textbf{12}.
La plus manifeste des caractéristiques divines par rapport à l'humanité
est l'adoration des créatures, ainsi que la législation dans leur
existence, et l'équilibre entre elles. Donc, si quelqu'un prétend
posséder cela, il revendique pour lui-même la plus manifeste des
caractéristiques divines et s'érige en Dieu des hommes à la place de
Dieu, et rien n'égale la corruption qui se répand lorsque les dieux se
multiplient sur terre, lorsque les hommes adorent les hommes, lorsqu'une
des créatures prétend avoir le droit d'être obéie en tant que telle,
ainsi que le droit de légiférer, d'établir des valeurs et des règles en
tant que telles ; tout cela, c'est prétendre à la divinité, même s'il ne
dit pas comme le Pharaon : \{Je suis votre seigneur, le Très- Haut
!\}\textbf{13}.

Cela revient à associer d'autres dieux à Dieu, et à l'impiété, c'est la
corruption sur la terre, et quelle corruption ! {[} ...{]} »

Extraits de « Conseil à l'umma de rejeter la fatwa du Sheikh Ben Baz
autorisant l'entrée au parlement » (1990). Traduction parue dans
\emph{Al Qaïda dans le texte}, G. Kepel et J-P Millelli dir., PUF,
Paris, 2005, p. 267.
 
 
