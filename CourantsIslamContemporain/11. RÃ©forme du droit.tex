\chapter{La réforme du droit}


\mn{11/4/22}
\section{Bibliographie}

 
ARMINJON, Constance \emph{Les droits de l'Homme dans l'islam shi'ite.
Confluences et lignes de partage}, Paris, Éditions du Cerf, 2017.

BABES, L. ; OUBROU, T. \emph{Loi d'Allah, loi des hommes}, Albin Michel,
Paris, 2002.

BEN ACHOUR, Yadh \emph{Normes, foi et loi - en particulier dans
l'Islam}, Ceres, Tunis, 1993.

\emph{La deuxième Fâtiha. L'islam et la pensée des droits de l'homme},
PUF, Paris, 2011.

BENKHEIRA, M.H. \emph{L'amour de la Loi - Essai sur la normativité en
islam}, Puf, Paris, 1997.

*CARRE, Olivier \emph{L'Islam laïque ou le retour à la Grande
Tradition}, Armand Colin, Paris, 1993.

*DUPRET Beaudoin \emph{La charia. Des sources à la pratique, un concept
pluriel}, Paris, La Découverte, 2014.

DUPRET, Beaudoin (dir.) \emph{La charia aujourd'hui. Usage de la
référence au droit islamique}, Paris, La Découverte, 2012.

YOUNES Michel (dir.) \emph{La fatwa en Europe. Droit de minorité et
enjeux d'intégration}, Profac, Lyon, 2010.
 


\section{Définitions}


\paragraph{Une question qui nait au début du XIX} avec les réformistes. 
\begin{Synthesis}
Il y a deux raisons : 
\begin{itemize}
    \item idée de progrès venu de l'Occident et des lumières
    \item Une nouvelle école de droit pour unifier le monde musulman ('Abduh)
\end{itemize}
\end{Synthesis}
    
\paragraph{Abduh, un projet inabouté} Le seul qui a cherché à réformer l'école de droit.

\paragraph{Un projet qui se déplace} C'est la question de l'application de la loi religieuse. l'islam politique a pour but d'appliquer la loi et non de la réformer. Dans les années 70, l'apothéose (Pakistan, Egypte, Soudan...). 

\paragraph{La question de la réforme du droit réapparait dans les années 90} 
\subparagraph{nouvelles techniques} Il y a de plus en plus de questions, liées à de nouvelles techniques : 
\begin{itemize}
    \item Bioéthique
    \item économique
    \item conquête spatiale
\end{itemize}

 \subparagraph{Dans un contexte minoritaire, avec des relations avec des non musulmans} A la fois, ils veulent un cadre normatif et à la fois ils ont intégré aussi les valeurs de l'Occident (homme / femme, respect des parents, liberté sexuelle), contexte laic (comme pratiquer le Ramadan, 5 prières par jour). 
 
 \paragraph{Toute pensée se construit par rapport à une altérité} L'islam est en réaction par rapport à cette temporalité. A cette question de l'Occident qui se pose pour toute culture, il y a le problème de la colonisation qui pose le problème de façon plus prégnante. Le seul lieu qui peut \textit{penser réellement} l'Occident, c'est l'Iran.
 
 
 
 
 \begin{itemize}
     \item Réponse fondamentaliste, salafiste : on va surinvestir la Sunna, dans les normes, et en particulier le modèle prophétique : comment le prophète se lavait les dents
     \item Aménagement du droit : assez traditionnel. Ne touchons pas aux 4 écoles du droits. on garde le principe d'une norme islamique mais il faut l'aménager via \textit{fatwa}, jurisprudence, permettant de répondre à de nouvelles questions.
     \item Réforme, va remettre en cause le rapport même à la norme. Ils refusent de penser que la norme pensée dans le droit musulman est suffisante. Jusqu'à remettre en cause le principe même d'un droit normatif musulman.
 \end{itemize}
 
 
  \paragraph{La shari'a}
  
  \begin{Def}[\emph{shari`a}]
    \emph{shari`a}, la voie, d'origine divine, forcément révélée, c'est la voie que Dieu donne aux homme pour aller vers lui : la \emph{shari`a} de Moise,... Ce qui explique pourquoi les musulmans ont laissé les juifs et les chrétiens appliqués leur propre shari'a et règles juridiques.
  \end{Def}
    
Elle contient : 
\begin{itemize}
    \item les articles de foi, "le dogme"
    \item les actes cultuels (les pilliers de l'Islam)
    \item les principes éthiques (qu'est ce que le bien, le mal ?)
    \item quelques normes juridiques qui régulent les relations humaines. 
\end{itemize}
  
  Les sources de la shari'a sont le \textit{Coran} et la \textit{Sunna}.
  
  
  
  
    \paragraph{Le \emph{fiqh}} Un concept parallèle.
    
    \begin{Def}[\emph{fiqh}]
    Compréhension intelligente [de la shari'a]. Le fiqh n'est pas lié à Dieu mais rattaché à un homme, ex : Hanaf, Chafi,...
    \end{Def}
    
    
  
\section{L' « aménagement » du
  droit}  
  
  
  
  
    
    \subsection{Méthode et principes}
    
  
    \paragraph{Qu'est ce qu'on entend par la réforme de Droit} Trois niveaux de discours possible : 
    \begin{itemize}
        \item On adapte le Fiqh mais en restant dans les 4 écoles
        \item on remet en cause le principe même le Fiqh
        \item on remet en cause les normes juridiques de la shari'a pour ne garder que les aspects dogmatiques et éthiques.
    \end{itemize}
    
    \subsection{Adaptation du Fiqh}
    On ne remet pas en cause les 4 écoles mais on va montrer la malléabilité du droit.
    \paragraph{Tareq Oubrou} Apparaissant d'abord conservateur, il est maintenant connu pour ses prises de position publiques en faveur d'un islam libéral. Il affirme notamment que le Coran serait mal interprété, notamment par méconnaissance du texte et de l'histoire contemporaine de Mahomet. \sn{\href{https://fr.wikipedia.org/wiki/Tareq_Oubrou}{Tareq Oubrou}} \label{Theo:TareqOubrou} \pageref{tareq-oubrou}  
    \begin{Ex}
    [femme adultère]
    On ne remettra pas en cause la sentence de mort de la femme adultère mais on montre que la shari'a a prévu des critères très strictes : 4 témoins avec 4 endroits différents, pour éviter une telle lapidation.
    \end{Ex}
    
    \paragraph{La question des \emph{fatwa}} Normalement, toujours lié à un contexte d'un fidèle. Cet avis juridique peut ensuite faire jurisprudence. Elle n'a pas de force contraignante à la fatwa, elle n'a de force que de l'autorité de celui qui l'émet.
    
    \paragraph{Eclatement de la fatwa du fait d'internet} 
    Du coup, un \href{https://fr.wikipedia.org/wiki/Conseil_europ\%C3\%A9en_pour_la_fatwa_et_la_recherche}{\textit{conseil européen de la fatwa et de la recherche}} \sn{plutôt une ligne conservatrice, frères musulmans} pour agréger les fatwa.
    
  
    
    \paragraph{L'enjeu des musulmans d'Occident et le « \emph{fiqh} des minorités »}
    les musulmans de l'Islam comme vitrine de l'Islam.
    Un enjeu majeur, promouvoir un islam ouvert, en dialogue avec l'Occident, mais sûr de lui.
    Certains voient l'Islam en occident comme terre de mission \emph{Dar al-da'wa}. 
    
    \begin{Ex}
    [Islam online] 1999 des pages en arabe mais aussi en anglais destiné aux musulmans en occident. 
    Qaradawi\sn{\href{https://fr.wikipedia.org/wiki/Youssef_al-Qarad\%C3\%A2w\%C3\%AE}{Qaradawi}. Il a en particulier autorisé le prêt à intérêt pour que les musulmans puissent s'installer dans le pays} fut le premier président du conseil européen de la fatwa.
    \end{Ex}
    
 \section{Repenser la
  norme}   
    Falzur Rahman et Mohammed Talbi (+ 2017).
    
    toute loi a besoin de s'appuyer sur des principes moraux. C'est seuelement s'il croit en Dieu et qu'il se sait jugé par Dieu que l'homme va agir en homme éthique.
    Il faut donc un droit aux fondements religieux. 
    Ils appliquent donc le principe du \textit{fiqh} mais ils critiquent deux choses.
  
  \paragraph{Mohammed Talbi} et l'école Tunisienne. Il critique le principe d'une école médiévale. On condamne sa sacralisation.  
  
  \paragraph{Falzur Rahman} va plus loin. Il critique la façon dont le droit musulman s'est constitué, erroné. En particulier, Chakni et son \textit{principe d'analogie}, qui ne fonctionne pas. La façon dont on a intégré les normes coraniques, en enlevant leur contexte, a dévié leur sens. Et ensuite, on le fait de façon anarchique. Il faut tout reprendre à 0.
  
  \paragraph{les solutions} ils se retrouvent. Il faut retrouver le principe de \emph{maqasid}, les finalités de la Révélation. Quelle est l'intention de la norme ? la dimension \textit{vectoriel } de la révélation.
  \begin{Ex}
    [La place de la femme.]
    Avant, aucune place pour les femmes. Dieu a voulu une égalité parfaite devant Dieu. Mais vu le contexte, la shari'a a avancé vers la progression homme/femme mais n'est pas encore à une égalité absolue.
    Donc pratiquement, en 1998, en Tunisie, égalité homme/femme de l'héritage (qui n'a pas abouti du fait des résistances).
  \end{Ex}
    Le dynamisme de la \emph{shari`a}
    
  
    \subsection{La \emph{shari`a} sans le \emph{fiqh ?}}
    
    Idée qu'il puisse y avoir un droit religieux sans la révélation. 
    \paragraph{Pour eux, la shari'a ne peut être que sur un plan éthique/pédagogique} et non juridique.  Du coup, les quelques normes dans le Coran (Témoignage, héritage,..) s'interprètent comme une \textit{pédagogie divine}, pour aider la société de l'époque qu'il voulait vraiment une évolution vers l'égalité des femmes. 
    
    \paragraph{Abdelmajid Charfi} Tunisien. Le Fiqh, d'abord sur Omar, qui voulait changer les relations tribales par des relations musulmanes. Puis, lors de la rencontre avec Bysance et la Perse, qui vont faire rentrer dans la Shari'a des choses qui n'était pas dans le Coran.
    
    \paragraph{principe : l'homme est libre et responsable} en utilisant sa raison, il peut développer son droit. 
    \mn{Voir le texte en ligne sur le canon de la prophétie}
    
    \paragraph{Scellement de la prophétie}. On peut le comprendre de plusieurs façons : 
    \begin{itemize}
        \item Image d'une maison que l'on ferme de l'intérieur. C'est ce qu'on fait les musulmans;
        \item mais on peut le comprendre de l'extérieur : vous êtes à l'extérieur et Mohammed clot la prophétie. Désormais, l'homme n'a plus besoin de révélation pour faire sa propre loi.
    \end{itemize}
    
    \begin{Def}[Coraniste]
    Penseurs qui disent qu'ils faut se débarrasser des Hadiths. On ne prend donc que le Coran comme réference et pas la Sunna.
    \end{Def}
    Un courant très développé en Tunisie. Il vient d'éditer une première édition critique du Coran.
    
    \paragraph{Une certaine reconstruction de l'époque préislamique} Fatima était une riche veuve, donc elle avait des biens. Alors que la tradition musulmane indique un monde preislamique était sombre.
  



\section{{Glossaire}}

\paragraph{Personnes}

Al-Qaradawi (Sheikh Yusef) Charfi (Abdelmajid)

Oubrou (Tareq) Rahman (Fazlur) Ramadan (Tariq) Talbi (Mohammed)

\paragraph{Notions}

dar ash-shahadat : \emph{« maison » (terre) du témoignage}

dar al-da`wa : \emph{« maison » (terre) du témoignage}

fatwa : \emph{opinion sur un point de la loi islamique donnée par un}
mufti\emph{.} furu' al-fiqh : \emph{« branches » du droit (discipline
juridique)}

haqq allah : \emph{droit de Dieu}

haqq al-nas : \emph{droit des gens}

`ibadat : \emph{culte/partie du droit concernant le culte.}

ijma' : \emph{consensus (des juristes, de la communauté).}

\paragraph{Lieux}

Al-Azhar (Egypte) Qarawiya (Maroc) Zaytuna (Tunisie)

maqasid : \emph{intention, finalité} mu`amalat : \emph{relations/partie
du droit concernant les relations humaines.} naskh : \emph{abrogation}

shari`at allah : \emph{« voie » de Dieu} shari'at al-Masih:
``\emph{voie'' du Christ (= religion chrétienne)}

shari'at Musa: \emph{``voie'' de Moïse (= religion juive)}

talfiq : \emph{éclectisme (fait de choisir librement entre les
différentes écoles juridiques)}

usul al-fiqh : \emph{fondements du droit (discipline juridique)}

\paragraph{Tareq Oubrou} \label{tareq-oubrou}


\textbf{L'islam est-il une religion de la loi ?}\sn{Extrait du livre \emph{Loi d'Allah, loi des hommes} de T. Oubrou et L.
Babès, Albin Michel, Paris, 2002, p. 86-87.}
\begin{quote}
TAREQ OUBROU : Il y a une autre catégorie d'intellectuels qui pousse le
raisonnement plus loin en considérant que le Coran n'est ni juridique ni
éthique, mais seulement spirituel (car pour eux, même l'éthique est
contraignante).

LEÏLA BABES : Quels intellectuels ? Pouvez-vous citer des noms ?

TAREQ OUBROU : Peu importe les noms. C'est tellement répandu aujourd'hui
chez les musulmans que la foi c'est dans le cœur, dans le sens de la
libération par rapport aux normes rituelles et éthiques. Mais on oublie
qu'une foi qui demeure dans le cœur, qui ne s'exprime pas à travers un
comportement éthique et spirituel cultuel, risque d'étouffer et de
s'éteindre.


Pour eux, le Coran est un message de foi uniquement, presque une forme
de protestantisme poussé à l'extrême. Car lorsqu'il a commandé d'oeuvrer
pour le bien et de prévenir le mal, il n'a pas désigné de quel bien ou
de quel mal il s'agit ni comment réaliser tout cela. Et si l'on suit
cette démarche, on aboutira à la question suivante : y a-t-il une
éthique musulmane ? N'y a-t-il pas une morale universelle, et pourquoi
alors la chercher dans les Sources révélées ? Même le rite n'échappera
pas à de telles interrogations à un moment donné. Par exemple, ni le
nombre ni la forme des prières canoniques, deuxième pilier de l'islam,
ne sont cités dans le Coran. Puisque la prière est très contraignante,
certainement la plus contraignante pour beaucoup, voire pour la majorité
des musulmans, doit-on pour cela l'effacer et la transformer en une
notion allégorique, et donc à chacun sa prière, et on aurait autant
d'islams que de musulmans ? On tombe finalement dans un mysticisme
obscur.

Mais tant qu'on n'a pas compris que le Coran sans la Sunna du Prophète\sn{il s'oppose aux Coranistes}
reste illisible, et qu'il y a des règles d'interprétation issues de ces
mêmes sources, on suivra un enchaînement de questionnements pour aboutir
à l'annihilation pure et simple des valeurs de l'islam. Trouver la loi
contraignante ne doit pas aller jusqu'à abolir des références et des
normes, ce qui plongerait les musulmans encore plus dans le chaos.


\end{quote}

\begin{Synthesis}
Nier l'importance de l'éthique et de la loi dans l'Islam, c'est une forme de protestantisme. la Sunna du Prophète, main dans la main avec Coran.
\end{Synthesis}

\paragraph{L'héritage féminin}\label{lhuxe9ritage-fuxe9minin}\sn{Extrait du livre \emph{Loi d'Allah, loi des hommes} de T. Oubrou et L.
Babès, Albin Michel, Paris, 2002, p. 103-105.}

\begin{quote}
    Vous avez évoqué l'héritage de la femme qui est la moitié de celui de
l'homme. Il faut signaler que le droit successoral en islam a répondu à
des milliers de cas. La règle de la moitié de l'homme donnée à la femme
n'est pas valable pour tous les cas. Nous avons le verset qui donne dans
un cas précis un sixième à la femme et un sixième à l'homme (IV, 11-12).
L'homme qui hérite le double de sa sœur doit subvenir aux besoins de sa
famille, ce qui n'est pas un devoir pour la femme ; ce qu'elle hérite va
dans sa poche, elle peut le faire fructifier, créer une entreprise sans
la permission ni de son frère, ni de son père, ni de son mari, etc. (et,
comme vous le savez bien, la femme en droit musulman est indépendante
économiquement de son mari dans le sens où, s'il subit une faillite, ses
biens restent protégés). Elle exigera de son mari une garantie
matérielle, qu'elle définit. En plus, les femmes ont inventé leur « ruse » : une partie de la dot est effectivement versée au début du mariage,
l'autre ne devient exigible qu'en cas de divorce... mais cette part
conditionnelle est très élevée.

Ces fictions juridiques (\emph{hiyal}), qui existent dans tous les
systèmes juridiques du monde, permettent de préserver l'esprit de la
shari`a et de remédier aux abus possibles. La femme peut donner la
\emph{zakat} (aumône légale, quatrième pilier de l'islam) --- considérée
comme des restes --- à son mari, ce qui n'est pas permis dans l'autre
sens ; il n'a pas à lui donner de ses restes, tout en sachant que les
biens de sa femme restent intouchables. Elle ne donne pas de
\emph{zakat} sur ses bijoux même si elle en a une tonne... Le juge peut
divorcer le mari si celle-ci porte plainte à cause de son manquement à
ses devoirs matériels.

Il serait donc injuste dans un tel système de donner la même part à la
sœur et au frère. 
\begin{Synthesis}
inégalité mais des "ruses" pour préserver l'esprit de la shari'a. La shari'a est équitable car il ne faut pas oublier les bijoux qui ne sont pas soumis à la zakat. Mais les bijour ne sont plus une part importante de la richesse.
\end{Synthesis}

Il y aura toujours mille dispositions juridiques dans
le droit musulman pour garantir l'esprit d'équité, qui
est tout sauf statique. Revenons à ce concept de l'éthicisation\sn{\begin{Def}[éthicisation] Donner un caractère éthique, ou plus éthique \end{Def} }, il me
permet d'avancer que si la moitié donnée à la fille dans certaines
conditions lui cause une réelle injustice, la part ajoutée par un
testament peut y remédier. En effet le seul hadith qui interdit le
testament à un héritier désigné par les textes est discutable. Il ne
fait pas l'unanimité chez les traditionnistes critiques. L'énoncé du
hadith est : \begin{quote}
    « Pas de testament pour un héritier légitime. »
\end{quote} 
Les
héritiers légitimes sont ceux qui sont indiqués par le Coran et la Sunna
ou étendus par analogie. Ce hadith rapporté par Nassây (m. en 1277),
Tirmidhi (m. en 892), Abu Dawûd (m. en 888) et Ahmed, ne résiste pas aux
scalpels des traditionnistes critiques. C'est pourquoi Bukhâri ne l'a
pas rapporté dans son Sahîh, car il n'est pas authentique selon ses
règles. Muslim non plus ne l'a pas rapporté. Il existe une autre version
qui dit :
\begin{quote}
    « Pas de testament à celui qui hérite légalement sauf si les
autres héritiers consentent »
\end{quote}, sans toutefois dépasser le tiers de
l'ensemble de la succession ; certains n'ont pas établi de limites. Je
suis de l'avis de Al-Mahdi Al-Murtada qui me permet de stipuler que le
testament pour la fille en plus de sa moitié est possible. Et je vois en
ce sujet que l'abrogation du testament ne signifie pas l'abrogation de
sa permission, mais celle de son imposition qui était obligatoire.
L'imam Shafi'i avance l'\emph{ijmâ'} en cette matière mais je ne vois
pas la base sur laquelle il est fondé. Il ne pouvait pas avancer
l'abrogation par ce hadith, il n'y a pas d'abrogation d'un verset par un
hadith, avis que je
partage.
\begin{Synthesis}
 une analyse des Hadiths permettant de laisser à chacun le choix de rééquilibrer pour les filles par "testament obligatoire"(dans une certaine mesure ?). C'est important du fait de l'indépendance économique des femmes (donc on intègre l'aspect culturel pour faire évoluer la pratique)
\end{Synthesis}
L'une des raisons légales qui me permettent cet avis est que la
situation financière des filles sous l'effet de l'éclatement des liens
familiaux a bien changé. Elle a acquis une plus grande indépendance
économique, et c'est elle qui prend dans beaucoup de cas la charge de
toute une famille. Dans ces conditions et par le biais du « testament
obligatoire », l'on peut élever la part de la sœur jusqu'à égalité de
celle du frère. C'est à traiter au cas par cas en fonction de la
jurisprudence.

Donc si j'ai évoqué l'intégration des coutumes, mais aussi des
mentalités et des conventions sociales dans le droit islamique, ce n'est
pas pour perpétuer les injustices mais justement pour les lever. Et s'il
y a une mauvaise application de droit musulman en matière d'héritage,
c'est pour une simple raison : l'éclatement des sociétés musulmanes.
C'est pourquoi j'ai parlé de l'éthicisation de la shari'a qui consiste à
moduler l'application du droit sur des bases morales en gardant
justement en vue les grands principes d'équité.


\end{quote}


\paragraph{Mohammed Talbi}
 \label{mohammed-talbi}


\textbf{L'héritage féminin}\sn{Extrait de son livre \emph{Un islam moderne}, Cérès, Tunis, 1998, p.
153-154.}
\begin{quote}
    \emph{Pourquoi la question de l'héritage pose-t-elle tant problème, y
compris en Tunisie qui a pourtant fait des efforts considérables pour
moderniser le droit de la famille ?}

C'est le seul problème vraiment délicat. En matière sexuelle, l'islam
est en effet très libéral puisqu'il admet même une certaine forme de
prostitution avec le mariage \emph{mut'a}. Ce libéralisme a d'ailleurs
constitué au Moyen Age un des points forts de la polémique chrétienne
contre l'islam, considéré comme trop laxiste et permissif. Aujourd'hui,
c'est l'inverse. Le problème de l'héritage n'est pas, cela dit,
totalement insoluble. D'abord parce qu'on peut doter les filles de son
vivant pour rétablir l'équilibre. Beaucoup de parents le font déjà. A
plus long terme, si les femmes parviennent à s'imposer davantage dans la
société et à faire aboutir leurs revendications, car les hommes ne leur
feront pas de cadeaux, rien ne dit qu'on n'aboutira pas un jour à un
consensus qui trouverait une solution juridique conforme à l'islam.
L'orientation du Coran va, je vous l'ai dit, dans le sens de
l'émancipation des femmes, telle est la finalité de la révélation. On
peut donc dire que la femme est parvenue aujourd'hui à un haut degré de
maturité, que la conjoncture sociale a changé, qu'elle travaille etc.,
ce qui permet de lui concéder la parité totale avec l'homme. Il existe
trois principes en islam permettant de faire évoluer le droit et de
l'adapter
à la réalité, 

la \emph{maslaha} c'est-à-dire l'utilité publique, un
concept qui date du II\textsuperscript{e} siècle de l'hégire, 

la
\emph{zharoura}, la nécessité, c'est un principe fort puisqu'il est dit
que "la nécessité rend permis l'interdit" ; 

et les \emph{maqassid}, les
finalités de la loi.  \label{TroisPrincipesEvolutionsShariA}

\begin{Synthesis}
On part des principes contre les principes. On retrouve le \textit{maqassid}. Maintenant que le contexte change, il faut changer la loi pour garder la finalité.
\end{Synthesis}

Ces trois instruments permettent de faire évoluer
cette dernière, mais il faut que la société y soit préparée. Cela pas
été le cas jusqu'à présent. Le jour où cela arrivera, les musulmans
trouveront dans leur patrimoine les éléments nécessaires pour faire
évoluer la loi sans rupture avec la foi.
\end{quote}

 

