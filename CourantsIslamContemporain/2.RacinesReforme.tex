\chapter{Les racines de la réforme : le
renouveau islamique des XVIIe-XVIIIe
siècles}
\mn{(24/01/2022)}

Un leçon sur les racines de cette diversité, et de tous ces courants au sein du monde Sunnisme. Tout cela part du désir de "\textit{Réforme}" qui part à un retour au source. Et cela peut donner des solutions très diverses, soit des lectures libérales ou des lectures fondamentalistes (un peu comme les évangéliques protestants).
Plutôt \textit{Renouveau} que \textit{Réforme}.

%------------------------------------------------
\section{La crise interne du monde musulman}
\subsection{Déclin des grands empires}

 
 \begin{Synthesis}[Date Symbolique]
 1798 : Napoléon en Egypte, date symbolique de la rencontre du monde musulman avec la modernité occidentale.
 \end{Synthesis}
 Cette lecture n'est pas fausse et va donner lieu au \textit{réformisme} mais c'est une lecture occidentalo-centrée.
 En réalité, cela a commencé plus tôt. Un désir né courant XVII et interne au monde musulman.
 
Les trois grands empires musulmans de l'époque rentrent en crise : 

\begin{table}[h!]
    \sidecaption{Les trois grands Empires musulmans}


%\newlength\q
\setlength\q{\dimexpr .25\textwidth -2\tabcolsep}
\footnotesize%
\noindent\begin{tabular}{p{\q}p{\q}p{\q}p{\q}}
\toprule
                & Fondation           & Apogée                            & Fin                                                                                \\
\midrule
Empire Safavide & 1501 (Ismaïl   1er) & Abbas Ier (1588-1629)             & 1736                                                                               \\
Empire Moghol   & 1526 (Babur)        & Akbar (1556-1605)                 &  1857 (existence symbolique après 1761)  \\
Empire Ottoman  & 1299 (Osman 1er)    & Soliman le Magnifique (1520-1566) & 1922                                        \\
\bottomrule
\end{tabular}


    \label{tab:my_label}
\end{table}
 
 
 
 \paragraph{Raisons économiques} Les portugais contournent l'Afrique et contournent les empires musulmans.  
 \paragraph{crises politiques} au sein de ces empires, trop grands. 
 \paragraph{Des voisins puissants} Ces Etats vont perdre des territoires face à des puissances qui ne sont pas musulmanes. L'empire Safavide est attaqué par les Ousbeks. L'Empire Moghol en XVIII siècle s'effondre devant l'empire Marathe (Hindou). L'empire Ottoman resiste mais décline depuis 1686 (siège de Vienne). 
 \begin{itemize}
     \item  {Affaiblissement des sultanats Indonésiens} face aux Hollandais. Java est annexé en 1800. 
  \item {En Chine} les musulmans Chinois (Hui) subissent des pressions des Manchous.
   \item  {Afrique} Les bambaras conquièrent Djenné, Tombouctou et Gao. Les portugais en Océanie. 
  \item {En Asie Centrale}, avancée des Russes (XIVII : Kazakstan et reste de l'Asie Centrale au XIX).
 \end{itemize}

 \paragraph{A la fin du XVI} 1591 : deuxième millénaire de l'Islam; sentiment millénariste. \mn{Calendrier lunaire, donc décalage}
 
 
 
\subsection{Problèmes doctrinaux et religieux}

\paragraph{Un raidissement des écoles juridiques au XVII}
Elles s'accusent les unes les autres de ne pas être juste, sentiment de division du monde sunnite.

\paragraph{Décadence du monde Soufi} On a tendance à considérer les soufis comme à la marge du monde musulman. Pourtant, elles ont joué un rôle social déterminant voire politique. Or, à l'époque, elles sont en crise.
\begin{itemize}
    \item {En faisant des Etats dans l'Etat} 
\item {Laxisme spirituel} Un soufi n'est plus tenu de suivre les prescriptions de l'Islam.
\item {Un rôle de plus en plus important du Sheykh}, le maitre spirituel de la Confrérie, qui transmet la \emph{Baraka}.
\item {Des pratiques spéctaculaire} Le fakirisme. le \emph{Dhikr}, rappel du nom de Dieu et l'on atteint l'Etat de transe, et vont marcher sur des braises. 

\end{itemize}



%------------------------------------------------
\section{L'aspiration au
  renouveau}
\subsection{Millénarisme}


 \begin{Def}[rapport au temps Involutif]
 L'Islam se tourne vers le passé, en Islam, les prophètes viennent toujours rappelés ce qui a été donné à l'origine. et le problème est que les hommes déforment le message transmis. Donc Dieu envoie de nouveaux prophètes pour restaurer le message originel. \textit{le temps corrompt.}
 
 \end{Def}
 vs Evolutif. 
 
 Ce qui fait que Mohammad est le dernier des prophètes, c'est que le message est transmis intact, pour la première fois, donc plus besoin de nouveaux prophètes. Autant, la \textit{pratique} peut se corrompre. 
 Un hadith dit : 
 \begin{quote}
     Slt un dixième de la pratique suffira pour sauver le monde.
 \end{quote}
 
 \begin{Def}[Moujaddid]
 de la racine JDD, nouveau, celui qui renouvelle l'Islam pour le mettre tel qu'il était à l'origine. 
 \end{Def}

 
 \begin{Def}[tajdid]
 Tajdid, le renouveau, qui doit venir régulièrement. Chaque siècle Dieu envoie un \textit{Moujaddid} selon un Hadith du Prophète.
 \end{Def}
 
 
 Il peut y en avoir plus mais à chaque siècle, il y a un grand moujaddid, savant, qui ne fait pas forcément au sein de la communauté. 
  \begin{Ex}
 Ghazali est considéré comme un Moujaddid
 \end{Ex}
 
 
 \subsection{Quelques grandes figures de la \textit{pre-Reforme}}
 

\paragraph{Ahamad Sirhindi} (Inde, 1564-1624) , essaye de rénover l'Islam et réforme une confrérie, la \emph{Naqshbandiyya} réformée (mujaddidi). 
\paragraph{Muhaddidi} Restaurer le monde musulman et restaurer l'Islam dans sa pureté originelle. Il va opposer 
\begin{itemize}
    \item le \emph{taqlid}, l'imitation (péjorative, servile, non réfléchie) et
    \item l'\emph{ijtihad}, l'effort d'interprétation du Coran et de la Sunna. Retourner aux sources scripturaires pour restrouver le sens authentique de l'Islam
    \item pour éliminer la \emph{bid'a}, très péjoratif, l'innovation
\end{itemize}  

On peut distinguer deux courants : un courant plus libéral et l'un plus fondamentaliste :
\begin{itemize}
    \item le rapport à la Tradition : fondamentaliste ne vont considérer que le Coran et la Sunna, en critiquant les élaborations savantes de l'Islam. 
    \item ce sur quoi va porter l'\emph{ijtihad}, soit limité sur les versets peu clairs du Coran, soit plus large pour le savant et les différentes sciences qui vont être convoquées pour l'ijtihad.
\end{itemize}


\paragraph{Muhammad 'Abd Al-Wahhab } 'Arabie, 1703-1792). Voir p. \pageref{Theol:AlWahhab}

\paragraph{Shah Wali Allah} (Inde, 1703-1762), \label{Theo:waliAllah} a eu les mêmes professeurs que Al-Wahhab ! et l'un des premiers à traduire le Coran dans une langue vernaculaire (en persan). Effort d'acculturation dans le contexte Moghol et Indou. 




 
%------------------------------------------------
\section{Les voies du renouveau} 
\subsection{Les confréries soufies}

La création de néo-confrérie
\begin{itemize}
    \item lutter contre un laxisme moral
    \item lutter contre le pouvoir du Sheykh.
    \item insister sur le côté social, en prenant en charge la société (vs gnosticime ?).
\end{itemize}
\mn{Soufi : \pageref{Def:SoufiNaqchabandiya}, \pageref{Def:Soufiqādirīya} }




\paragraph{Ahmad Al Tijani} (Algérie 1737, 1815). Et fonde la \emph{\textbf{tijaniyya}}, un groupe toujours très présent en Afrique du Nord. Normalement un Sheykh insiste sur la chaine d'initiation qui remonte au prophète. Lui, va être initié directement en rêve par le Prophète.

\paragraph{Abdurrauf al-Singkili} (Aceh, Indonésie, 1615-1693)

\paragraph{Abdelkarim al-Samman} (Soudan, 1718-1776) => sammaniyya

Au sénégal, aussi une confrérie de type réformée. 
\mn{A completer carte Carte moderne. Il y avait 24 confréries soufis en Arabie avant le Wahhabisme}




%---------------------------------------
\subsection{Les centres de pèlerinage : La Mecque et Médine}

La Mecque et Médine jouent un rôle essentiel, du fait du Hajj. Du fait de l'Empire Ottoman, il y a pacification des parcours du pélerinage, par ailleurs incités par l'Empire.
Et par ailleurs, des centres de formation s'implémentent à la Mecque.  On en profite pour étudier. Se croisent des savants de différentes origines.
Non seulement les centres d'étude mais aussi les confréries, avec leur Sheykh les plus importants s'installant à Médine ou la Mecque. Ils sont initiés à l'enseignement Moujaddidi et vont réformer la confrérie à laquelle ils appartiennent, selon un modèle qui se répète.




%------------------------------------------------
\section{Un aspect particulier : les jihads aux marges du monde   musulman} 


\begin{figure}[h!]
    \centering
        \sidecaption{Le « renouveau » des marges du monde musulman (XVIIIe-XIXe
siècle)  \emph{L'atlas de l'islam depuis 1500}, F. Robinson, Nathan, 1987  
Des petits Etats vont se créer sur des bases confrériques et dont la mission de faire le jihad.
}
 \includegraphics[width=\textwidth]{CourantsIslamContemporain/ImagesCourantsIslamContemporain/image1.jpeg}

    \label{fig:le-renouveau-des-marges-du-monde-musulman-xviiie-xixe-siuxe8cle}
\end{figure}


\subsection{Situation particulière des zones marginales}


Pourquoi on les retrouve aux marges du monde musulman ? Il y a un besoin de purification et l'islamisation par rapport à des populations hétérogènes. Mais aussi, parce que ces zones ne sont pas controlées par les grands empires musulmans qui ne tolèrent pas ces confréries.
Le jihad se porte d'abord sur les musulmans eux-mêmes, pour qu'ils deviennent musulmans puis contre les puissances extérieures ou non musulmanes qui controlent ces zones. 

\paragraph{Une structure Etatique} Le Sheykh est le chef, \emph{Commandeur des Croyants} - on fait référence aux premiers temps musulmans, on prête allégeance, et avec une approche puritaine, très forte pratique. \mn{Des analogiques avec DAESH, qui se réfèrent eux aussi aux temps de l'Islam}

\paragraph{Une accélération de l'Islamisation}


\subsection{Quelques exemples}

\paragraph{Spectre Temporel 1680-1920}{1680, Etat de Bondou} Afrique de l'Ouest jusqu'e l'Etat Mahdiste (Somalie) 1899-1920.

\paragraph{Chine - Ma Ming-hsin} Réveil religieux des communautés chinoises sous la pression des Manchous. Ma Ming-hsin (1719-1781) lorsqu'il est à la Mecque, il fonde une confrérie réformée, la \textit{Nouvelle Secte} et entre en opposition contre les autorités locales, qui appelle en soutien la dynastie Manchoue des Ming. Ming-Hsin est décapité. Il y aura d'autres révoltes plus tard, du Kan-Su et du Chen-Si (1862) et du Yunnan (1856)

\paragraph{Indonésie - Jajji Miskin} Le mouvement Padri à Sumatra (1803-1845). Après le Hajj, en 1803, il prend le contrôle de villages et va imposer sa vision de l'Islam, interdit les pratiques populaires et proclament le Jihad contre les autres villages et les pouvoirs. 1820 : les Hollandais reviennent dans la région et on a une transformation de ce mouvement contre un Jihad contre les Hollandais. Eliminé par les Hollandais.

\paragraph{Nigéria - Califat de Sokoto} Uthman da Fodio (1755-1817) issu d'une famille de savants musulmans, l'enseignement lui vient par la fraderyya\sn{revoir} et crée une communauté qui le reconnait comme \textit{commandeur des Croyants}, s'oppose aux pouvoirs locaux et va être défait par la Grande Bretagne.


\begin{Synthesis}
\begin{itemize}
Ce renouveau : 
    \item Avant la rencontre de l'Occident
    \item lié à des thèmes de crises internes
    \item des thématiques que l'on retrouvera : retourne aux sources scripturaires
    \item C'est assez fascinant de voir la circulation des idées, liée autour du \textit{Hajj}
\end{itemize}


\end{Synthesis}

%------------------------------------------------

 



 



\paragraph{Soufis du Badakhshân : un renouveau confrérique entre
l'Inde et l'Asie centrale}
\mn{Alexandre Papas, \emph{Cahiers d'Asie Centrale}, n° 11-12, 2004, p.
87-102 (extraits -- texte expurgé)}
 
 
\subparagraph{Éléments de biographie d'un soufi
badakhshânais} 
\mn{le Badakhshân est entre l'Afghanistan et le Tajikistan}
\begin{quote}
Mawlânâ Mîr Ghiyâs al-Dîn \label{Theo:MawlanaMirGhiyasAlDin} naît en 1117/1705-06 dans la petite localité
de Hisârak, située au cœur du district de la ville de Jirm. Le
grand-père de Mawlânâ a émigré du village de Dahbîd, non loin de
Samarcande, en direction du Badakhshân. La \emph{silsila}\sn{Génération} de la famille
remonte au Prophète et, sur dix générations, au frère d'un grand saint
Kubrawî et découvre une généalogie \textit{soufie}. C'est donc au sein d'une des
grandes familles \emph{muhâjir}\sn{Emigré, en référence aux premiers musulmans qui ont suivi le Prophète à Médine} de l'aristocratie religieuse du
Badakhshân que naît Ghiyâsî.

{[}\ldots{]}

C'est précisément pour l'Inde -- destination qui concurrence la
Transoxiane savante, en particulier Boukhara, surtout depuis le XVIe
siècle -- que part le jeune Ghiyâsî âgé de 14 ans en quête d'initiation \sn{soufi}.
Cet exil de l'adolescent fait l'objet de deux contes hagiographiques :
 

\begin{itemize}
\item
 
  Le futur shaykh de Ghiyâsî, Shâh Walî Allâh, qui a quitté Sirhind pour
  entreprendre un pèlerinage au mausolée de Bahâ' al-Dîn Naqshband\sn{le fondateur de la confrérie réformée Naqshandayyi} non
  loin de Boukhara, stationne au Badakhshân, à Jirm, chez le père même
  de Ghiyâsî. Le shaykh demande alors à ce dernier de lui amener ses
  enfants, mais perçoit par claire-vue qu'on lui dissimule le jeune
  Mawlânâ qui, en état de \emph{majzûb}\sn{Ils peuvent faire des choses indécentes} « ravi en Dieu », suscite la honte de
  sa famille. À sa vue le shaykh indien cite un vers de Ni'mat Allâh
  Walî Kirmânî. Et au jeune Ghiyâsî de prononcer miraculeusement le
  second vers du distique. Walî Allâh annonce alors qu'à son retour de
  Boukhara il prendra le jeune homme comme disciple et l'emmènera en
  Inde.
  
\item
 
  Dès l'âge de 9 ans Mawlânâ refuse les conseils de sa famille et se
  distingue des autres enfants. Plusieurs nuits, au cours de rêves, lui
  apparaît un homme illuminé qui lui enjoint de partir pour l'Inde où
  lui est promise la rencontre d'un grand saint soufi. Malgré le refus
  de ses parents qui souhaitent marier leur fils, Ghiyâsî parvient à
  quitter le Badakhshân quelques années plus tard. Parvenu à Lahore et
  après un nouveau rêve révélateur, il attend de nombreux jours au
  couvent de Khwâja Khwândamîr, un khalîfa de la Naqshbandiyya, jusqu'au
  jour où il rencontre Shâh Walî Allâh.
 
\end{itemize}

 
Restent les faits : après une formation classique en \emph{madrasa} à
Delhi où le novice rencontre ses premiers maîtres, il devient à Lahore
durant douze années le disciple d'un fameux maître Naqshbandî Mujaddidî.
Il interrompt une unique fois son initiation lorsque le shaykh lui
confie la mission de se rendre au Cachemire afin d'aller chercher un
homme qu'on prétend thaumaturge et que Walî Allâh souhaite convertir à
l'islam et initier à sa \emph{tarîqat}\sn{confrérie}. Au terme de ses douze années de
noviciat, le shaykh lui enjoint de retourner à sa terre natale pour
propager la confrérie. De retour au Badakhshân, Ghiyâsî âgé de trente
ans environ et qui a obtenu le rang de \emph{mawlawiyyat}\sn{maître}, fait office
d'enseignant à la \emph{madrasa} Jâmi'-i Islâmî du district de Jirm. Il
est ensuite convié à Fayz Âbâd à la cour de Sultân Shâh, laquelle abrite
des savants et des poètes venus d'Inde et d'Iran, dont certains
acquièrent grande réputation. C'est là que Ghiyâsî compose son œuvre
poétique et mystique. C'est également là, de son \emph{khânaqâh}\sn{couvent soufi, Ḵānqāh ou ḵānāqāh, fut d'abord un lieu destiné à abriter les spécialistes et savants religieux musulmans (‘ulamâ’), une sorte d'équivalent des couvents chrétiens. Ces établissements ont été ensuite réservés aux soufis.}, qu'il
dirige son enseignement, suivis par de nombreux disciples venus de
toutes les régions alentours. Le soufi badakhshânais devient aussi le
directeur spirituel de Sultân Shâh. Et lorsque ce dernier est capturé à
Qunduz par les ouzbeks du Qataghân en 1179/1765, le vieux maître
conseille durant trois ans le fils et suppléant du shah emprisonné, Mîr
Muhammad Shâh. D'un tel succès et d'une telle influence, Ghiyâsî
apparaît comme l'un des principaux promoteurs de la Mujaddidiyya dans le
Nord de l'Afghanistan.

{[}\ldots{]}
\end{quote}

\begin{Prop}
On retrouve des noms : Walî Allâh (cf p. \pageref{Theo:waliAllah}; l'importance du réseau confrérique; et rôle social (il devient le conseil du sultan).
\end{Prop}

\subparagraph{Savants et soufis au croisement du
Badakhshân}
Le Pamir est un sanctuaire pour les savants hétérodoxes ou bannis (cf Unwân banni d'Egypte du fait de sa lutte sociale).

\begin{quote}
Malgré les obstacles géographiques et en dépit des troubles politiques
qui affectent la fin du XVIII\textsuperscript{e} siècle, le Badakhshân
reçoit la visite de savants religieux qui, pour certains, décident de
s'y installer et interrompent définitivement leur voyage vers l'Inde ou
vers la Transoxiane. Il faut rappeler ici que le Pamir a, de façon
continue dans l'histoire, servi de sanctuaire à des individus frappés
d'ostracisme ou fuyant la répression dans leur région natale. Mais au cours du
XVIII\textsuperscript{e} siècle le sanctuaire se mue en lieu de
renaissance où fleurissent \emph{madrasa}\sn{lieu d'enseignement religieux} et couvent soufis. Le
\emph{Armaghân-i Badakhshân} mentionne le cas de deux étudiants de
Peshawar, Mîr Ahmad Mujaddidî dit `Izhâr' (m. 1259/1843) et son frère
Muhammad Anwar qui, à une date indéterminée durant le règne de Mîr
Muhammad Shâh, se rendent d'abord à Boukhara afin d'obtenir les
compétences du rang de \emph{`âlim}\sn{savant} et qui, lors de leur retour,
s'installent et officient au Badakhshân, pour le premier à Jirm, pour le
second à Bahârak.

Un autre cas de figure est celui, analogue à Ghiyâsî, de ces
badakhshânais qui partent se former aux sciences religieuses, et
éventuellement au soufisme, dans les grands centres de savoir du monde
musulman, proches ou lointains. De ce point de vue, le cas le plus
intéressant -- et malheureusement le plus douloureux faute
d'informations suffisantes et dans l'absence de vestiges de son œuvre --
est celui de Sayyid Abâ al-Hasan « `Unwân ». Né à Jirm en 1123/1711, il
quitte le Badakhshân pour Boukhara afin d'acquérir une formation
théologique. De là `Unwân se rend au pèlerinage de La Mecque et à
Médine, puis il s'installe durant 18 ans en Egypte, probablement au
Caire, où il poursuit son acquisition des sciences islamiques classiques
et commence à enseigner. Mais `Unwân ne se contente pas de dispenser un
enseignement religieux, il prend parti pour les classes populaires
égyptiennes et contre leur oppression par les propriétaires terriens.
C'est du moins la réputation qu'il gagne selon le \emph{Armaghân-i
Badakhshân}, et qui lui vaut d'être banni d'Egypte. `Unwân part alors
pour Istanbul, rejoint Boukhara et reprend son enseignement. `Unwân, qui
prône l'unité de la Communauté des Croyants (\emph{umma}), décide
d'aller prêcher la concorde (\emph{âshtî}) dans le Caucase, peut-être au
moment de l'activisme Naqshbandî de Shaykh Mansûr dans les années 1780.
Mais il renonce à son projet et entre dans une retraite spirituelle
jusqu'à sa mort en 1206/1791, sans être retourné au Badakhshân.
{[}\ldots{]}

\end{quote}

\begin{Prop}
 Dans le deuxième paragraphe, `Unwân se forme à la Mecque et Médine (rôle du pélerinage), rôle social en Egypte et pas seulement religieux. L'importance aussi de l'Unité, l'\textit{Umma}. 
\end{Prop}
 
\subsection{Liste des neo confréries}

\paragraph{Naqshbandiyya}: La Mecque, Damas, Yémen, Inde, Asie Centrale \label{Def:Naqshbandiyya}
                     => Caucase, Chine Occidentale + Orientale, Sumatra.

\paragraph{Qadiriyya}: Proche Orient, Irak, Inde, Asie centrale 
                    => Indonésie (Java, Aceh), Caucase, Afrique

\paragraph{Khalwatiyya}: Egypte => nombreuses branches en Afrique :

\begin{itemize}
    \item \textit{Tijaniya}: Algérie => Afrique de l’Ouest
    \item \textit{Sammaniyya}: Soudan
    \item \textit{Idrisiyya}: Maroc => Sanusiyya (Lybie), Sahiliyya (Somalie),
    \item \textit{Murghaniyya} (Erythrée)
\end{itemize}- 
 
%-----------------------------------------------
\subsection{bibliographie}

\begin{quote}


AZRA, Azyumardi \emph{The Origins of Islamic Reformism in Southeast
Asia}, Asian Studies Association of Australia/KITLV Press, Leiden, 2004.

PAPAS, Alexandre \emph{Soufisme et politique entre Chine, Tibet et
Turkestan}, J. Maisonneuve, Paris, 2005.

*ROBINSON, Francis \emph{Atlas de l'Islam depuis 1500}, Nathan, Paris,
1987 (dispo à la FELS)

*VOLL, John R. « Foundation for renewal and reform », in John L.
Esposito ed., \emph{Oxford History of Islam}, Oxford University Press,
1999, pp. 509-547.
\end{quote}