\chapter{Introduction}


\mn{Anne-Sophie Vivier Muresan \url{as.viviermuresan@icp.fr}
Anthropologue, THèse sur l'Iran, l'Islam en France
}
Ce cours porte sur les courants de l'Islam, depuis le XVII\textsuperscript{ème} siècle.

\paragraph{Processus}
Lire les textes


\paragraph{Validation} uniquement sur un thème lié au cours. utiliser Recherche+
Un écrit en format Word.



\section{Chronologie indicative}

 
\paragraph{1744 } Alliance entre Muhammad Ibn ‘Abd el-Wahhab, fondateur de la doctrine wahabbite, et Ibn Sa‘ud, ancêtre de la dynastie saoudienne actuelle.

\paragraph{1798} : Expédition en Egypte de Napoléon. Date-symbole habituellement retenue pour marquer le début de l’intensification des relations réciproques entre Occident et Orient musulman.
\paragraph{1826-1831 }: l’Etat égyptien envoie un groupe de 40 personnes étudier en France. La modernité est alors comprise comme appropriation des sciences développées en Occident.
\paragraph{1830 } : Prise d’Alger par la France. Début de la main mise de l’Occident sur le monde arabe (colonisation et mandats).
\paragraph{1839 } : Début des « réformes » modernisatrices (tanzimat) dans l’Empire Ottoman. La modernité est pensée en termes de réformes sociales et politiques sur le modèle occidental.
\paragraph{1884 } : Jamal ad-din al-Afghani et Mohammad Abduh fondent à Paris la revue Al ‘Urwa al Wuthqa. Débuts du mouvement réformiste musulman.
\paragraph{1924 } : prise de La Mecque et de Médine par les descendants d’Ibn Sa’ud et de Muhammad Ibn Abd al Wahhab. L’Arabie Saoudite devient le foyer du fondamentalisme musulman, qu’elle propagera surtout à partir des années 1970.
\paragraph{1924 } : abolition du califat par Mustapha Kemal (Atatürk). Recherche d’un accord au sein du monde sunnite pour l’élection d’un nouveau calife, sans suite.
\paragraph{1929 } : Fondation des Frères Musulmans par Hassan al-Banna, en Egypte. Vise l’ « islamisation par le bas », par l’éducation religieuse et la da‘wa (activité missionnaire).
\paragraph{1941 } : Fondation de l’association Jama‘at al-islami par Mawdudi au Pakistan. Débuts de l’islam politique (islamisme).
\paragraph{1947 } : Création du Pakistan, premier Etat moderne fondé sur une définition d’abord musulmane de la Nation.
\paragraph{1979-80 } : Révolution iranienne et instauration de la République islamique. En parallèle, montée en puissance des islamistes dans de nombreux pays musulmans.
\paragraph{1985 } : Pendaison de Mahmud Taha, au Soudan, accusé de blasphème pour son interprétation novatrice de la Révélation coranique. Emergence des « nouveaux penseurs de l’islam », qui cherchent à rompre avec la visée réformiste et à repenser entièrement le rapport de l’islam à la modernité.
\paragraph{2001 } : Attentat du 11 septembre. L’islam politique, qui a perdu sa légitimité au sein du monde musulman, se radicalise, se sectarise, et prend la voie du terrorisme.

\paragraph{2007 } : « Lettre des 138 » : 138 théologiens musulmans affirment leur désir de dialogue et de paix dans une lettre adressée aux responsables religieux chrétiens. C’est la première fois qu’une voix commune de cette ampleur prend corps en islam.
\paragraph{2014 } : Victoires de Da’esh (Etat Islamique) en Irak et en Syrie. Prétentions à l’instauration d’un califat – refusé par l’ensemble des savants et des institutions sunnites officielles.
 