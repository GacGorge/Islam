\chapter{Introduction}


\mn{Anne-Sophie Vivier Muresan \url{as.viviermuresan@icp.fr}
Anthropologue, THèse sur l'Iran, l'Islam en France
}
Ce cours porte sur les courants de l'Islam, depuis le XVII\textsuperscript{ème} siècle.

\paragraph{Processus}
Lire les textes


\paragraph{Validation} uniquement sur un thème lié au cours. utiliser Recherche+
Un écrit en format Word.

% --------------------------------------------------------------
\chapter{Panorama de l'Islam dans le monde}

\paragraph{Introduction : ce cours sera essentiellement sur le monde sunnite} essentiellement. Deux séances sur le Sh'iisme. 
\begin{Synthesis}
Pour le sunnisme, on peut parler d'éclatement de l'Islam, une véritable variété de l'Islam.
\end{Synthesis}
Avoir les clés des différents discours musulmans. 


\section{Aperçu géographique et démographique}

Lors de la bataille de \textit{Siffin} (657), séparation entre : 
\bi
\item Sunnites : 87,4\%
\item Chiites : 11,9\%
\item Kharijites : 0.7\% (Ibadisme en Oman)
\ei 

\subsection{les écoles Juridiques}
\mn{Atlas de l'Islam dans le monde, Anne Laure Dupont, Autrement, 2005}

\bi 
\item  malékisme : Afrique Nord et ouest.
\item Chafiite : Est de l'Afrique et surtout Egypte 
\item Hanafite : Turquie, asie Centrale, Inde. \textit{les empires turques}
\item Hanbalite : surtout Arabie Saoudite, transformé en \textit{Wahhabisme} au XX, avec une extension au dela de l'Arabie Saoudite en 1960.
\ei 


\subsection{Trois grands courants dans le sh'isme}


\begin{Synthesis}[divergence en Si'isme : les courants]
Des désaccords sur Qui est Imam et sur la \textit{nature de l'Imam}. Il peut être investi de pouvoirs divins.
\end{Synthesis}

\bi 
\item Les duodécimains (ils reconnaissent 12 imams) : les shi'ites Iraniens
\item les Ismaeliens ou septimains (7 imams, l'Aga Khan), Liban.
\item zaydites (5 imams) : Yemen. Au niveau de la doctrine, ils reconnaissent peu de pouvoirs divins aux Imams (proches des Sunnites de ce point de vue).
\ei 
Les Ibadites sont Khajidites. Les druzes et les Alaouites (Alevi en Turc) sont issus du shi'isme (en se proclamant le Maadi) mais ne sont pas considérés comme musulmans par les autres musulmans.  A ne pas confondre à la famille Alaouites au Maroc qui sont tout à fait sunnite (Alaouite veut dire descendant d'Ali). L'Iran reconnait les Alaouites comme shi'ites pour des raisons politiques. 

\subsection{diversité culturelle}
L'islam s'est acculturé aux cultures qu'il a rencontré.
\paragraph{Mosquée}
Le seul élément architectural à une mosquée est la qibla qui indique la Mecque : le Mihrab.

\bi
\item la mosquée bleue a été constituée sur le plan de Sainte Sophie
\item Mosquée de Djenné.
\item La mosquée de Lagos : représente un style baroque brésilien.
\item Xian
\item la grande mosquée de Paris : sur l'image d'une mosquée marocaine
\item Mosquée contemporaine de Créteil

\ei 

\paragraph{L'habit féminin} D'après le Hadith "on ne doit montrer que les mains et le visage". 

\bi
\item les danseuses de cours à Surakarta
\item la burqa, avec grillage sur les yeux, asie centrale et Afghanistan dans certains milieux
\item des vétements de couleur et le shadri, vêtement traditionnel en Asie centrale paysanne voile que l'on met différemment selon qu'on est seul, ...
\item en Afrique, habit d'abord ethnique et non religieux. 
\item En France, les jeunes : le bandeau pour tenir le foulard et le foulard coloré. et certaines ne portent pas le foulard
\ei 

\paragraph{Pourquoi une impression d'uniformisation} La mondialisation crée l'uniformisation et certains courants wahhabites incitent à l'uniformisation sur l'influence saoudite.

\subsection{Observer l'islam}

 
\newlength\q
\setlength\q{\dimexpr .5\textwidth -2\tabcolsep}

\begin{table}[h!]
\sidecaption{\textit{Observer l'Islam} Clifford Geerzt, il a été au Maroc et en Indonésie}
%\begin{tabular}{p[7cm]p[7cm]}
\noindent\begin{tabular}{p{\q}p{\q}}
\toprule
\textbf{Maroc}                                                     & \textbf{Indonésie}                                               \\ 
\midrule
Tribale                                                            & Paysanne                                                         \\
\textit{Audace}                                                    & \textit{Application}                                             \\
une culture préalable moins riche                                  & L'islam est arrivé sur une civilisation hindouiste et bouddhiste \\
Un islam de dévotion aux saints (en particulier le saint guerrier) , austérité morale, pouvoirs magiques, piété agressive & malléable, syncrétique  \\
Uniformisation, facteur de civilisation & diversification, multiforme\\

\bottomrule
\end{tabular}
\end{table}


\section{Chronologie indicative}

 
\paragraph{1744 } Alliance entre Muhammad Ibn ‘Abd el-Wahhab, fondateur de la doctrine wahabbite, et Ibn Sa‘ud, ancêtre de la dynastie saoudienne actuelle.

\paragraph{1798} : Expédition en Egypte de Napoléon. Date-symbole habituellement retenue pour marquer le début de l’intensification des relations réciproques entre Occident et Orient musulman.
\paragraph{1826-1831 }: l’Etat égyptien envoie un groupe de 40 personnes étudier en France. La modernité est alors comprise comme appropriation des sciences développées en Occident.
\paragraph{1830 } : Prise d’Alger par la France. Début de la main mise de l’Occident sur le monde arabe (colonisation et mandats).
\paragraph{1839 } : Début des « réformes » modernisatrices (tanzimat) dans l’Empire Ottoman. La modernité est pensée en termes de réformes sociales et politiques sur le modèle occidental.
\paragraph{1884 } : Jamal ad-din al-Afghani et Mohammad Abduh fondent à Paris la revue Al ‘Urwa al Wuthqa. Débuts du mouvement réformiste musulman.
\paragraph{1924 } : prise de La Mecque et de Médine par les descendants d’Ibn Sa’ud et de Muhammad Ibn Abd al Wahhab. L’Arabie Saoudite devient le foyer du fondamentalisme musulman, qu’elle propagera surtout à partir des années 1970.
\paragraph{1924 } : abolition du califat par Mustapha Kemal (Atatürk). Recherche d’un accord au sein du monde sunnite pour l’élection d’un nouveau calife, sans suite.
\paragraph{1929 } : Fondation des Frères Musulmans par Hassan al-Banna, en Egypte. Vise l’ « islamisation par le bas », par l’éducation religieuse et la da‘wa (activité missionnaire).
\paragraph{1941 } : Fondation de l’association Jama‘at al-islami par Mawdudi au Pakistan. Débuts de l’islam politique (islamisme).
\paragraph{1947 } : Création du Pakistan, premier Etat moderne fondé sur une définition d’abord musulmane de la Nation.
\paragraph{1979-80 } : Révolution iranienne et instauration de la République islamique. En parallèle, montée en puissance des islamistes dans de nombreux pays musulmans.
\paragraph{1985 } : Pendaison de Mahmud Taha, au Soudan, accusé de blasphème pour son interprétation novatrice de la Révélation coranique. Emergence des « nouveaux penseurs de l’islam », qui cherchent à rompre avec la visée réformiste et à repenser entièrement le rapport de l’islam à la modernité.
\paragraph{2001 } : Attentat du 11 septembre. L’islam politique, qui a perdu sa légitimité au sein du monde musulman, se radicalise, se sectarise, et prend la voie du terrorisme.

\paragraph{2007 } : « Lettre des 138 » : 138 théologiens musulmans affirment leur désir de dialogue et de paix dans une lettre adressée aux responsables religieux chrétiens. C’est la première fois qu’une voix commune de cette ampleur prend corps en islam.
\paragraph{2014 } : Victoires de Da’esh (Etat Islamique) en Irak et en Syrie. Prétentions à l’instauration d’un califat – refusé par l’ensemble des savants et des institutions sunnites officielles.
 