\chapter{Le soufisme au XXe siècle}

\mn{(09/05/2022)}
\section{Bibliographie}


\begin{quote}
\emph{*Réveils du soufisme en Afrique et en Asie}, dossier de la revue
\emph{Archives de Sciences Sociales des Religions}, n° 135, 2006.

\emph{Confréries soufies en métropole}, dossier de la revue
\emph{Archives de Sciences Sociales des Religions}, n° 140, 2007.

AMSELLE, Jean-Louis \emph{Islams africains : la préférence soufie},
Jean-Louis Amselle, Paris, éd. du Bord de l'eau, 2017.

BALCI, Bayram \emph{Missionnaires de l'islam en Asie Centrale : les
écoles turques de Fethullah Gülen}, Paris, Maisonneuve et Larose, 2003.

CHIH, Rachida \emph{Le soufisme au quotidien : confréries d'Egypte au
XXe siècle}, Paris : Sindbad, Arles, Actes Sud, 2000.

FATHI, Habiba « Les réseaux mystiques au Kazakhstan : entre zhikr et
militantisme ? », \emph{Cahiers d'Asie Centrale}, n° 15-16, 2007, p.
223-261.

GEOFFROY, Eric « Soufisme, réformisme et pouvoir en Syrie contemporaine
», \emph{Egypte/Monde Arabe}, n° 29, 1997, p. 11-22.

*\emph{Le soufisme - Histoire, fondements et pratiques de l'islam
spirituel}, Paris, Eyrolles, 2019.

POPOVIC, A. ; VEINSTEIN, G. (dir.) \emph{Les voies d'Allah : les ordres
mystiques dans l'Islam des origines à aujourd'hui}, Paris, Fayard, 1996.

ROMEY, Alain « Rôle du wahabisme et du réformisme de la Nahda en Algérie
dans le processus d'exclusion et de marginalisation du soufisme »,
\emph{Cahiers de la Méditerranée}, 69, 2004,
\url{http://cdlm.revues.org/index735.html}
\end{quote}

\section{Introduction}
\begin{Synthesis}
A su s'adapter; une réalité de l'islam.
\end{Synthesis}

\paragraph{Soufisme}
Quête intérieur de Dieu

\paragraph{Organisation en Confréries} à partir du XIIè, structuration en confréries. C'est cela qu'on étudiera dans son chapitre :
\begin{itemize}
    \item marqué par la \textsc{hiérarchie}, avec à sa tête : \textit{sheykh}, \textit{piz}, \textit{marabout}. En bas, les frères. Un pacte d'allégeance au \textit{sheykh} (comme le monachisme chrétien).
    \item notion \textsc{d'initiation}, avec des rites d'initiation. Et on a une chaine (\textit{silsila}) entre le sheykh et Mohammed et Ali.
    \item des \textsc{exercices spirituels} avec des invocations à dire des milliers de foi par jour. invocation du nom de Dieu. 
    \item des \textsc{des pratiques collectives}, sikr (dhikr), avec des transes (\textit{tekke}).
    
\end{itemize}

\section{XXè : siècle difficile pour le Soufisme}
Anawati annonce sa disparition.

\subsection{pourquoi ce déclin ?}

\paragraph{répression des gouvernements} par exemple, en Turquie et le sécularisme.

\paragraph{parfois disparition} en particulier interdiction des mawlids.
\begin{Def}[\textit{mawlid}]
commemoration de la mort du sheykh fondateur de la confrérie. Tous ceux qui peuvent vont s'y rendre. C'est essentiel pour la confrérie
\end{Def}

\paragraph{Aussi dans le Caucase et les balkans} du fait de l'interdiction socialiste.

\paragraph{Dans les pays wahhabistes} En Arabaie Saoudite, 25 confréries soufis représentés en Arabie Saoudite au XVIIIè. En 1920, les wahhabites ont tout détruit. 
Associationisme : On adore le sheykh à l'égal de Dieu
 
 
 \paragraph{En Iran} en 1980, persécution des confréries pour des raisons de contre-pouvoir.
 
\subsection{Facteurs idéologiques et sociaux}

\paragraph{même Abduh considérait que c'était une régression}

\paragraph{Sociologique} : on était dans une confrérie familialement. Ces structures traditionnelles ont éclatées avec l'exode rural. Dans les langues locales. Alors que dans les villes, on va parler Arabe ou une langue vernaculaire.

\paragraph{Un rôle de charité et d'éducation} qui va être pris par l'Etat soit par les courants de l'Islam musulman (cf frères musulmans).\sn{En ce sens, une confrérie n'est pas une secte car elle ne coupe pas de la société}


% - ---------------------------------------------------  
\section{Résistance et renouveau} 
Alors qu'on annonçait leur mort, elles ont finalement résisté.

\subsection{Une fin de siècle plus clémente}

\paragraph{A partir des années 70/80} au moment où on annonçait la mort du soufisme, ces Etats laics ont recherché la légitimité religieuse face à l'Islam politique : les confréries paraissent des contre-poids contre cet islam politique.
\begin{Ex}
Tunisie : "rêve de Ben Ali", il faut que tu reconstruises la x. 

\end{Ex}
\paragraph{Elles se sont implémentées dans les villes}
Nashqband : fonde la Nashqbandia.
Yasawiyya


\subsection{Sur la voie de la modernité : les néo-confréries}
Comment les confréries se sont adaptées.
\paragraph{Touchent les classes moyennes de ville} en changeant leur mode de fonctionner.

\paragraph{Changement de discours} respect du culte, coranisation du vocabulaire pour bien montrer l'orthodoxie coranique.  On va parfois éliminer des pratiques qui paraissent non orthodoxes (divination, amulettes).

\paragraph{Reinvestissement du champ social} des réseaux éducatifs qui leur sont propres, des hopitaux,... des centres d'accueil. 
\begin{Ex}
Au Pakistan, Indonésie ou Egypte, des aides pour sortir de la drogue, en s'appuyant sur le dhikr (invocation).
\end{Ex}


\paragraph{changement du mode de recrutement} Ces confréries vont devenir prosélytes. Méthode moderne de communication (show télévisé). Une adhésion qui devient individuelle (et non pas familiale). On se passe d'initiation (ikhwan). 

\paragraph{La relation au Sheykh change} Avant le lien physique avec le Sheykh était essentiel, pour le passage de la \textit{barakha}. Désormais, on peut suivre sur internet la pensée du Sheykh. Le rôle du délégué est minoré, avec un rapport direct au Sheykh. La progression passe par l'enseignement que la prescence physique. 

\paragraph{Féminisation} développement de branches féminines. 
\begin{Ex}
En Syrie, la \textit{qubaysiyya} quasi exclusivement féminine, très importante dans les classes supérieures.
Depuis 2011 ?
\end{Ex}





 % - ---------------------------------------------------  
\section{Soufisme, fondamentalisme, modernisme : des relations complexes}

Rapport entre le modernisme, le fondamentalisme et soufisme. 
\paragraph{Discours de réislamisation des sociétés} qui peut être assez proches de celui du discours de l'Islam. 
\begin{Ex}
Qubaysiyya avait l'ambition de reislamiser la Syrie par les femmes
En Syrie, on a eu des Sheykh qui ont aidé à l'implémentation des frères musulmans.
A Hama, 1982, les confréries ont aidé les frères musulmans contre l'Etat
\end{Ex}

\paragraph{la structure soufis des frères musulmans, exemple de porosité entre islam politique et soufisme}
Les frères musulmans eux-mêmes ont leur structure copiée sur le soufisme. Banna avait proposé des \textit{exercices spirituels}, allégeance, \textit{sheykh}

\paragraph{En France, rôle fondamental du Tabligh dans la réislamisation}. \mn{voir Tabligh p. \pageref{sec:Tabligh}} 1927 en Inde, M. Ilyâs, pour islamiser les basses castes, face au Christianisme. Importance du culte. Il faut dire les prières. Normalement, on peut dire la prière entre l'appel à la prière (ajhan) et la suivante. Or, pour les Tabligh, il faut faire sa prière à l'heure exacte, ce qui ne favorise pas l'emploi (approche sectaire).  Or, M. Ilyâs était d'une famille soufi (Chichtiyya) et il a gardé la structure soufi (en particulier il a repris la retraite).


\subsection{Soufisme et fondamentalisme}

\subsection{Confrérisme et modernisme : l'exemple des Fethullahci}
 
 
 
\paragraph{Fethullah Gûlen, et le mouvement nurcu}

\paragraph{A l'origine un modernité Sa'id Nursi} (1885-1944). Ataturk va le contraindre à résidence. Il va fonder un islam de type réformiste, nurcu, \textit{mais avec l'ojectif de la reislamisation de la société}

\paragraph{refondé par Fethullah Gülen (1938-)} jusqu'à 2013, il va fonder un véritable empire sur tout ce qui est éducatif, des cités universitaires, des bourses étudiants (avec un ainé dans l'appartement qui enseigne les jeunes de l'appartement). Y compris dans la diaspora et la France.
Empire dans les média et financier.  Au départ, il était allié de Erdogan avec qui il partageait la volonté de reislamiser la Turquie.

\paragraph{répression} En 2013, Erdogan a trouvé qu'il était trop puissant et après le coup d'état en 2016, il les a accusés de ce coup d'état avec une forte persécution. 

\paragraph{Un islam réformiste et moderniste} démocrate, avec la volonté de développer le dialogue inter-religieux. Mouvement nationaliste, Turquie. Mariage extrêmement endogame. Seulement 7\% des filles se marient en dehors de la communauté turque !

\subsection{le soufisme en occident}
 \paragraph{Les origines : René Guénon} 1884-1951. \sn{\href{https://fr.wikipedia.org/wiki/Ren\%C3\%A9_Gu\%C3\%A9non}{Guénon dans Wikipedia}}
 \begin{itemize}
     \item  Une Tradition unique des religions, primordiale, avec un Dieu unique. 
     \item cette Tradition s'est essaimée dans les cultures et les religions.
     \item par l'initiation, on peut remonter à la Tradition unique (et du coup le rapport maître/ disciple). 
 \end{itemize}
Il s'oppose au discours moderniste. Il vient d'une lecture ésotérique (lecture chiffrée). La référence ultime est \textit{Ibn Arabi}, Dieu est dans toute chose, s'anéantir en Dieu.  Et Guénon considère que le soufisme a gardé le mieux le principe de l'initiation, et en particulier la gnose soufie. De fait, les premières conversions à l'Islam en France ont été dans le soufisme à cause de Guénon. \textit{une influence certaine}.

     
\paragraph{Du soufisme « immigré » au soufisme français}   

A partir des années 1970 et de l'afflux des immigrés des pays subsahariennes. Au départ, diffus, ne cherchant pas à pénétrer la société française.
Par contre, :
\begin{itemize}
    \item Naqshbandiyya (Turquie) : essaye de se faire connaître
    \item la Alawiyya (Algérie) : Le Sheykh Ben Founès. C'est lui qui a fondé les scouts musulmans en France. 
    \item la Boutchichiyya (Maroc) : très présent au Forum 104\sn{104 rue de Vaugirard. Propose des ateliers d'introduction au Niqs}. Représenté en France par Fawzi Skali. Adapte son discours, au lieu de parler de Dieu, on va parler d'énergie divine.
\end{itemize}




 
\section{Annexes}
\begin{quote}
{Personnes}

Bentounès (Khaled) (1949 -) Guénon (René) (1884-1951) Gülen (Fethullah)
(1938 -)

Ilyas (Muhammad) (1885-1944) Naqshband (Baha'uddin) (m. 1388) Nursi
(Sa`id) (1873-1960)

Schuon (Frithjof) (1907-1998) Skali (Faouzi) (1953 -) Vâlsan (Michel)
(1907-1974) Yasawi (Ahmad) (m. 1166)

\paragraph{Confréries} `Alawiyya Bektashi Butshishiyya Haqqaniyya
Mevlevi Naqshbandiyya Qubaysiyya Sanusiyya Shadhiliya Shishtiyya
Tijaniyya
\end{quote}
\subsection{glossaire}

\begin{quote}
{Autres mouvements} Tablighi jama`at Nurcu

Fethullahci

{Notions}

\emph{bay`a} : allégeance =\textgreater{} en contexte soufi, relation
d'allégeance liant les disciples au maître

\emph{chilla} : retraite (spirituelle)

\emph{da`wa} : « invitation » =\textgreater{} prédication, appel à
entrer dans l'islam

\begin{Def}[\emph{dhikr (zikr)}] : « souvenir » : pratique soufie fondée sur la
répétition du nom de Dieu. 
\end{Def}
\emph{ijaza} : « autorisation »
=\textgreater{} reconnaissance officielle du titre de \emph{shaykh}

\emph{khalifa} : « successeur », « vicaire » =\textgreater{} maître
confrérique (v. \emph{shaykh, pir})

\emph{mawled} (\emph{mouled}) : pèlerinage (annuel) à un tombeau de
saint

\emph{murid} : disciple

\emph{pir} : maître confrérique (v. \emph{shaykh, khalifa})
\emph{sheykh} : maître confrérique (v. \emph{khalifa, pir})
\emph{tariqa} : voie =\textgreater{} confrérie

\emph{tekke} : lieu de rencontre confrérique (v. \emph{zawiyya})
\emph{zawiyya} : lieu de rencontre confrérique (v. \emph{tekke})
\emph{wird} : oraison personnelle
\end{quote}

\hypertarget{une-nuxe9o-confruxe9rie-islamiste}{%
\subsection{\texorpdfstring{{Une néo-confrérie « islamiste
»}}{Une néo-confrérie « islamiste »}}\label{une-nuxe9o-confruxe9rie-islamiste}}

\begin{quote}

Au Maroc\sn{
Extraits de « Mystique et politique chez Abdessalam Yassine et ses
adeptes », de Youssef Belal, \emph{Archives des Sciences Sociales des
Religions}, n° 135, 2006, p. 172-173, 181-182.}, le cheikh Abdessalame Yassine fonde en 1981 une structure de
type confrérique, la \emph{Jamâ`a}, qui a pour vocation la \emph{da`wa}, comprise comme rappel de Dieu à l'ensemble de la société. Cette \emph{da'wa} a une dimension politique marquée : à terme est visée « la construction d'une entité politique islamique, qui préparera des
élections islamiques, une constitution islamique et un gouvernement islamique ». Sa doctrine est exposée dans une œuvre publiée en 1982 : \emph{Al minhâj al-nabawi} (La voie prophétique). Voici sa pensée présentée par le chercheur Youssef Belal.

\begin{Synthesis}
Un projet politique
\end{Synthesis}

Le projet politique d'A. Yassine est largement déterminé par l'idée qu'il se fait du rapport des croyants à Dieu, c'est-à-dire essentiellement le rapport mystique. Non seulement la place du cheikh médiateur entre Dieu et les hommes est indispensable mais la transcendance vécue lors des rites soufis, le sentiment d'élévation et
de rapprochement de Dieu doit être un sentiment présent à tous les instants et dans tous les actes des hommes. La structure du livre est révélatrice à cet égard. Consacrant l'essentiel de son ouvrage aux dix
séances (\emph{khisâl}) qui permettront à l'homme de revivifier sa foi,
A. Yassine reprend largement les thèmes soufis : \emph{al-suhba}
(compagnonnage), \emph{al-dhikr} (remémoration), \emph{al-sidq} (la
sincérité), \emph{al-badl} (le don), \emph{al-`ilm} (le savoir),
\emph{al-jihâd} (la lutte contre l'égo).

Mais il faut bien voir que la pensée d'A. Yassine se déploie constamment sur le registre de l'éducation, \emph{tarbiyya} et de l'organisation, \emph{tanzîm}. La tension est permanente entre une éducation soufie et une action qui se veut révolutionnaire dans le monde.


\begin{Synthesis}
Combat permanent. Glissement entre un islam mystique et un islam politique. 
Allégeance : on renonce à son Ego et on s'ouvre à Dieu. Mais peut donner prise à une obéissance aveugle de ce que demandera le sheykh.
\end{Synthesis}
(\ldots)

L'allégeance à A. Yassine {[}est{]} un acte vital pour les adeptes. Pour suivre sa voie et son enseignement, il est indispensable de faire preuve de \emph{sidq}, c'est-à-dire que l'adepte doit suivre tout ce que lui prescrivent la Jamâ`a et son guide. Ceux qui veulent suivre la voie du
Prophète, c'est-à-dire la voie d'A. Yassine, doivent être conscients que l'ego, le \emph{nafs}, peut être un obstacle à tout moment. Il faut combattre le \emph{nafs} qui est le mal (\emph{su'}). Valoriser son \emph{nafs} est en fait incompatible avec le rôle assigné à la Jamâ`a et
à A. Yassine. Il faut être capable de se dévouer pour la Jamâ'\,`a et pour être réceptif à l'enseignement du maître il faut que l'âme soit vierge de l'ego. Laisser le \emph{nafs} triompher c'est avoir le destin
de cet instituteur qui a pris goût à la vie matérielle et qui a préféré le divertissement (\emph{lahw}) à la \emph{suhba} en se laissant envahir
par d'autres habitudes. Il faut au contraire demander d'achever sa vie
parmi « les frères et les sœurs car la mort dans la \emph{da`wa}\sn{l'appel à l'Islam} est
bien plus haute que celle dans le combat armé.

(\ldots)

Lorsqu'il en vient, après avoir traité du \emph{dhikr} en tant
qu'éducation, à aborder le \emph{dhikr} en tant qu'organisation du
mouvement, il transforme une nouvelle fois des exigences mystiques en
mode d'action politique :
\begin{quote}
    « le \emph{dhikr} n'est pas seulement un
travail salutaire au niveau des consciences et des mots qui sortent de
la bouche et des rites extérieurs pratiqués par le croyant. Le
\emph{dhikr} signifie aussi se lever dans les mains de Dieu lors de la prière. Les soldats de Dieu le pratiquent pour remplir leur devoir
cultuel et parce que c'est un signe de la souveraineté de Dieu dans les relations de Dieu avec ses soldats et dans les relations des soldats de
Dieu entre eux. C'est s'apprêter à appliquer la Loi de Dieu le jour où le pouvoir reviendra aux croyants dans tous les domaines du pouvoir, de
la politique, de l'économie, de la société, de la justice et de la culture du \emph{jihâd} ».
\end{quote}

\end{quote}
\begin{Synthesis}
Il reprend tout le vocabulaire et pratique de la confrérie dans une optique différente. On ne peut pas opposer simplement les uns avec les autres. 
\end{Synthesis}