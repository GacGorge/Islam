\chapter{La nouvelle pensée islamique}
\mn{\emph{(04/04/2022)}}


\section{Bibliographie}


\begin{itemize}
\item
  
  AL AJAMI, \emph{Que dit vraiment le Coran}, Erick Bonnier Editions,
  2020.
  
\item
  
  ABOU ZEID Nasr Hamid \emph{Critique du discours Religieux}, Sindbad,
  Acte Sud, Paris, 1999.
  
\item
  
  AL-BANNA, Gamal \emph{L'Islam, la Liberté, la Laïcité et le Crime de
  la tribu des « il nous a été rapporté »}, trad. D. Avon et A. Elias,
  L'Harmattan, Paris, 2013.
  
\item
  
  ARKOUN, Mohamed \emph{Lectures du Coran}, Maisonneuve \& Larose,
  Paris, 1982.
  
\item
  
  BAJRAFIL, Mohammed \emph{L'islam de France, l'an I. Il est temps
  d'entrer dans le XXIe siècle}, Ed. Plein Jour, 2015.
  
\item
  
  BENKIRANE Réda \emph{Islam, à la reconquête du sens}, éd. Le Pommier,
  2017.
  
\item
  
  CHARFI, Abdelmajid \emph{L'islam entre le message et l'histoire},
  Albin Michel, Paris, 2004.
  
\item


\emph{La pensée islamique, ruptures et fidélités}, Albin Michel, 2008.


 
\item
  
  CHEBEL, Malek \emph{Manifeste pour un islam des lumières}, Hachette,
  Paris, 2004.
  
\item
  
  ESACK Farid \emph{Coran, mode d'emploi}, Albin Michel, Paris, 2004.
  
\item
  
  LAMRABET Asma, \emph{Islam et femmes : les questions qui fâchent},
  Casablanca, En toutes lettres, 2017.
  
\item
  
  SANGARE, Youssouf \emph{Repenser le Coran et la tradition islamique --
  Introduction à la pensée de Fazlur Rahman}, Al Bouraq, 2017.
  
\item

  TALBI Mohamed \emph{Plaidoyer pour un Islam moderne}, Cérès/Tunis,
  DDB/Paris, 1998.
  

\item BENZINE, Rachid \emph{Les nouveaux penseurs de l'Islam}, Albin Michel,
Paris, 2004. FILALI-ANSARY, Abdou \emph{Réformer l'islam ? Une
introduction aux débats contemporains}, La
Découverte, Poche, Paris, 2005.

 \item HOFFNER, Anne-Bénédicte \emph{Les nouveaux acteurs de l'islam}, Paris,
Bayard, 2017.

\item RENAUD, Etienne "Mahmud Taha et la «seconde mission de l'Islam»",
\emph{Se Comprendre}, 85/07, 1985.

\item ROUSSILLON\textbf{,} Alain \emph{La pensée islamique contemporaine},
\emph{acteurs et enjeux}, Téraèdre, Paris, 2005.

\item SALEH Waël, \emph{A la recherche d'un aggiornamento de l'islam. Des
voies contemporaines}, Paris, L'Harmattan, 2018.

\end{itemize}



\section{Les nouveaux penseurs : essai
  de
  présentation}
  
  
\section{Nouvelles approches critiques}

\section{le statut de la Parole de Dieu : un Coran créé ?}


\section{la question de l'interprétation}





\subsection{Glossaire}

 
{Personnes}

Abu Zayd (Nasr Hamid) Al-Khuli (Amin) Arkoun (Mohamed) Charfi
(Abdelmajid) Engineer (Ali)

Esack (Farid) Hossein (Taha) Iqbal (Muhammad)

Khalafallah (Muhammad Ahmad) Rahman (Fazlur)

Shabestari (Muhammad) Sorush (Abdelkarim)

{Notions}
\begin{Def}[tafsir]
\emph{commentaire (exotérique) du Coran}
\end{Def}


\begin{Def}[ta'wil]
\emph{herméneutique ésotérique du Coran}
\end{Def}
 

\subsection{Fazlur Rahman
(1919-1988)}

 
Pour le Coran lui-même, et par conséquent pour les Musulmans, le Coran
est la Parole de Dieu (kalâm Allâh). Mohammed, lui aussi, était
absolument convaincu de recevoir le Message de Dieu, le Tout-Autre (nous
essaierons plus loin de comprendre plus précisément le sens de cette
Altérité absolue), à tel point qu'il s'est basé sur la force de cette
conviction pour rejeter quelques-unes des assertions historiques les
plus fondamentales de la tradition Judéo-Chrétienne concernant Abraham
et les autres Prophètes. Cet "Autre", par un certain canal, a "dicté" le
Coran avec une autorité absolue. La voix surgissant des profondeurs de
la vie parlait distinctement, impérieusement, sans méprise possible. Non
seulement le mot \textbf{"}\emph{qur'ân}", signifiant "récitation",
l'indique clairement, mais le texte du Coran lui-même déclare en
plusieurs endroits que le Coran est révélé verbalement et non pas
seulement son "sens" et ses idées. Le terme coranique pour "Révélation"
est \emph{wahy} dont le sens est assez proche d' "inspiration", à
condition que ce mot ne soit pas censé exclure nécessairement le mode
verbal (par "Parole", bien entendu, nous ne voulons pas dire le Son). Le
Coran dit: "Dieu ne parle à aucun humain (c'est-à-dire en paroles
audibles) excepté par \emph{wahy} (c'est-à-dire par idée-parole,
inspiration) ou derrière un voile, ou bien Il envoie un Messager (un
ange) qui parle par \emph{wahy}...Nous t'avons ainsi révélé un Esprit
qui est à nos ordres". (Q. 42,51-52)

Pendant le 2° et le 3° siècles de l'ère islamique, cependant, de
violentes différences d'opinion, des controverses, en partie causées par
l'influence des doctrines chrétiennes, s'élevèrent parmi les Musulmans
au sujet de la nature de la Révélation. L'"orthodoxie" islamique
émergente qui en était au point crucial de formuler son contenu précis,
mit alors l'accent sur l'origine externe de la Révélation Prophétique
afin d'en sauvegarder son "Altérité", son "Objectivité" et son caractère
verbal. Le Coran lui-même maintenait certainement son altérité vis-à-vis
du Prophète. Il déclare en effet: "\emph{L'Esprit fidèle l'a fait
descendre sur ton coeur pour que tu sois au nombre des avertisseurs}"
(Q. 26,194), et encore: "\emph{Dis: Qui est l'ennemi de Gabriel (qu'il
le soit), car c'est lui qui a fait descendre sur ton coeur... le Livre}"
(Q. 2,97). Mais il manquait à l'orthodoxie (en fait, à toute la pensée
médiévale) d'avoir les instruments intellectuels nécessaires pour
allier, dans sa formulation du dogme, l'Altérité et le caractère verbal
de la Révélation d'une part, et, d'autre part, son lien intime avec
l'oeuvre et la personnalité religieuse du Prophète, c'est-à-dire qu'il
lui manquait la capacité intellectuelle de dire, à la fois, que le Coran
est entièrement la Parole de Dieu et aussi, dans un sens ordinaire, la
parole de Mohammed. Le Coran affirme clairement les deux idées, car s'il
insiste sur le fait qu'il est descendu sur le "coeur" du Prophète,
comment peut-il lui être extérieur ? Ceci, bien sûr, n'implique pas
nécessairement que le Prophète ne percevait pas un personnage extérieur,
comme la tradition le dit, mais il est remarquable que le Coran lui-même
ne fait aucune mention d'un personnage dans ce contexte: ce n'est qu'en
lien avec certaines expériences (habituellement liées à l'Ascension du
Prophète), que le Coran parle du prophète comme de quelqu'un ayant vu un
personnage ou un esprit ou quelque autre objet "à la lointaine limite"
ou "à l'horizon" bien qu'ici encore, comme nous l'avons dit,
l'expérience soit décrite comme étant d'ordre spirituel. Mais
l'orthodoxie, par les \emph{hadith} ou les "traditions" transmises du
Prophète, en partie interprétées correctement, et en partie inventées,
ainsi que par une discipline théologique largement basée sur le
\emph{hadith}, fit de la Révélation prophétique un phénomène ne frappant
que son oreille, uniquement extérieur à lui, considérant l'ange ou
"l'esprit qui vient sur le cœur" comme un agent totalement externe.
L'image que l'Occident moderne garde de la Révélation prophétique se
base largement sur cette formulation orthodoxe plutôt que sur le Coran,
d'ailleurs la foi du Musulman ordinaire fait de même.

... Lorsque la perception intuitive de la morale chez Mohammed atteint
son plus haut point et s'identifie avec la morale elle-même, la Parole
est donnée avec l'inspiration elle-même. Le Coran est donc la pure
Parole divine, mais évidemment, il est également relié intimement à la
personnalité interne du Prophète Mohammed, dont la relation au Coran ne
peut être conçue mécaniquement comme un disque enregistré. La Parole
divine passe par le cœur du Prophète.

\emph{Extrait de son livre \textbf{Islam} (Doubleday Anchor Book, New
York, 1968, 331 pp.), p. 25-28.}