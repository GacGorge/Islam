\chapter{La nouvelle pensée islamique}
\mn{\emph{(04/04/2022)}}


\section{Bibliographie}


\begin{itemize}
\item
  
  AL AJAMI, \emph{Que dit vraiment le Coran}, Erick Bonnier Editions,
  2020.
  
\item
  
  ABOU ZEID Nasr Hamid \emph{Critique du discours Religieux}, Sindbad,
  Acte Sud, Paris, 1999.
  
\item
  
  AL-BANNA, Gamal \emph{L'Islam, la Liberté, la Laïcité et le Crime de
  la tribu des « il nous a été rapporté »}, trad. D. Avon et A. Elias,
  L'Harmattan, Paris, 2013.
  
\item
  
  ARKOUN, Mohamed \emph{Lectures du Coran}, Maisonneuve \& Larose,
  Paris, 1982.
  
\item
  
  BAJRAFIL, Mohammed \emph{L'islam de France, l'an I. Il est temps
  d'entrer dans le XXIe siècle}, Ed. Plein Jour, 2015.
  
\item
  
  BENKIRANE Réda \emph{Islam, à la reconquête du sens}, éd. Le Pommier,
  2017.
  
\item
  
  CHARFI, Abdelmajid \emph{L'islam entre le message et l'histoire},
  Albin Michel, Paris, 2004.
  
\item


\emph{La pensée islamique, ruptures et fidélités}, Albin Michel, 2008.


 
\item
  
  CHEBEL, Malek \emph{Manifeste pour un islam des lumières}, Hachette,
  Paris, 2004.
  
\item
  
  ESACK Farid \emph{Coran, mode d'emploi}, Albin Michel, Paris, 2004.
  
\item
  
  LAMRABET Asma, \emph{Islam et femmes : les questions qui fâchent},
  Casablanca, En toutes lettres, 2017.
  
\item
  
  SANGARE, Youssouf \emph{Repenser le Coran et la tradition islamique --
  Introduction à la pensée de Fazlur Rahman}, Al Bouraq, 2017.
  
\item

  TALBI Mohamed \emph{Plaidoyer pour un Islam moderne}, Cérès/Tunis,
  DDB/Paris, 1998.
  

\item BENZINE, Rachid \emph{Les nouveaux penseurs de l'Islam}, Albin Michel,
Paris, 2004. FILALI-ANSARY, Abdou \emph{Réformer l'islam ? Une
introduction aux débats contemporains}, La
Découverte, Poche, Paris, 2005.

 \item HOFFNER, Anne-Bénédicte \emph{Les nouveaux acteurs de l'islam}, Paris,
Bayard, 2017.

\item RENAUD, Etienne "Mahmud Taha et la «seconde mission de l'Islam»",
\emph{Se Comprendre}, 85/07, 1985.

\item ROUSSILLON\textbf{,} Alain \emph{La pensée islamique contemporaine},
\emph{acteurs et enjeux}, Téraèdre, Paris, 2005.

\item SALEH Waël, \emph{A la recherche d'un aggiornamento de l'islam. Des
voies contemporaines}, Paris, L'Harmattan, 2018.

\end{itemize}



\section{Les nouveaux penseurs : essai
  de
  présentation}
  
  \subsection{Synthèse}
  
  \begin{Synthesis}
  Nous sommes dans un contexte Post moderne. A la différence moderne, universaliste et objectif, en post modernité, il n'y a qu'un accès relatif et subjectif à la vérité.
  Nous sommes tous des personnes raisonnables mais la façon d'accéder à la vérité, passe par le filtre de la subjectivité. 
  Le Coran passe par le filtre de la propre de subjectivité de Mohammed. Le texte est déjà une interprétation. Il y a une remise en cause du Coran \textit{Verbatim}.  
  \end{Synthesis}
  
Rachid Benzine \sn{voir p. \pageref{Theo:Benzine}}: 
\begin{quote}
    les modernistes, ou les nouveaux penseurs
\end{quote}  

Il s'est développé à partir des années 1980 en réaction à l'Islam politique et le wahhabisme, l'entrée de la violence en Islam, la cloture.

\paragraph{dans un cadre post moderne}
Ils vont chercher des nouvelles voies. La grande différence avec les modernistes, c'est qu'ils pensent la fin du XXeme. Ce n'est plus la positivité d'Auguste Comte, c'est le cadre de la post modernité. On reconnaît que la raison est construite historiquement, sociologiquement. L'accès à la vérité est donc partiel. Ce qui demeure et s'accentue, c'est la subjectivité.

\paragraph{L'herméneutique} ou Interprétation. L'accès à la vérité ne peut se faire que via l'herméneutique.

\begin{Def}[herméneutique]
Il n'y a que des interprétations à la vérité auxquelles on accède
\end{Def}

\paragraph{Conséquences} 
\begin{itemize}
    \item Il n'y a pas une essence de l'Islam. \textit{il n'y a que ce que les croyants musulmans en disent}.
    \item refus de chercher un islam pur à l'origine. Il n'y a que des islams contextualisés
    \item Importance plus faible du droit, de la norme. Ce qui compte, c'est l'éthique. L'Islam est moins pensé comme une religion normative mais un moyen pour l'individu d'entrer en relation avec le divin.
\end{itemize}

\paragraph{Rapport complexe à l'occident} Ils peuvent en particulier critiquer l'aspect absolu des droits de l'homme qu'il convient de contextualiser en Islam.

\paragraph{Un même type de formation} Plutôt formés en sciences humaines ou littératures, où ils ont découvert l'herméneutique : philosophie, histoire, ... acquis en occident. Ils prennent très au sérieux les études en sciences islamiques. 

\paragraph{En opposition avec les élites} religieuses (qu'ils critiquent) ou étatiques (face à leur souhait démocratique). Exil parfois nécessaire. En Tunisie et au Maroc, on pouvait avoir cette réflexion.

\paragraph{Du monde entier} 
\begin{itemize}
    \item Arabe
\begin{itemize} 
    \item Arkoun (Mohamed) (Algérie)
    \item Abu Zayd (Nasr Hamid) (Egypte)
    \item Khalafallah (Muhammad Ahmad) (Egypte)
\end{itemize}
\item Iran
\begin{itemize} 
    \item Sorush (Abdelkarim)
    \item Shabestari (Muhammad)
\end{itemize}
\item Pakistan
\begin{itemize} 
    \item Rahman (Fazlur)
    \item Engineer (Ali)
\end{itemize}
\item Afrique du Sud
\begin{itemize} 
    \item Esack (Farid) (théologie de la libération)
\end{itemize}
\end{itemize}
A classer : 
   Charfi
(Abdelmajid) 

 Hossein (Taha) Iqbal (Muhammad)

 

 
  
\section{Nouvelles approches critiques}

\subsection{L’approche littéraire}

Rappel du statut du Coran en Islam Classique : voix de Dieu et Mohammed n'est que l'enregistreur. Ce texte est vérité.

'Abduh va suggérer que le Coran ne doit pas être pris comme un document historique. Le but du Coran est d'exhorter les fidèles, sans avoir un soucis de précision historique.

\paragraph{Al-Khuli (Amin) 1895-1966} Enseigne à Al-Azhar. mais il a fait quelques années d'étude en Europe. Le Coran doit être compris et analysé comme un texte littéraire : 
\begin{itemize}
    \item Sémantique : les mots à l'origine. \sn{Ce n'est pas tout à fait nouveau. des les premiers siècles de l'islam, travaux sur la poésie pre-islamiques, la grammaire et dictionnaires}
    \item forme littéraire
    \item le contexte culturel et social. Cela est nouveau.
\end{itemize}

\paragraph{Khalafallah (Muhammad Ahmad) 1916-1998 } Disciple de Al-Khuli. Savant et étude de lettres à l'université du Caire (université moderne). Thèse en 1948 sur \textit{l'art du récit dans le Coran}. Il va aller jusqu'à dire que certains récits dans le Coran n'ont pas de valeur historique mais uniquement littéraire. 

\begin{Ex}
Par rapport aux récits des \href{https://fr.wikipedia.org/wiki/Sept_Dormants_d\%27\%C3\%89ph\%C3\%A8se#:~:text=Les\%20Sept\%20Dormants\%20d'\%C3\%89ph\%C3\%A8se,r\%C3\%A9cit\%20au\%20contexte\%20arabo\%2Dmusulman.}{sept dormants d'Ephèse}, va montrer que dans certains versets, il y en a 5 et dans d'autres 6. Cela montre que le but de Dieu n'est pas de faire un récit historique
\end{Ex}

Pour Khalafallah, il distingue dans le Coran différentes formes littéraires : 
\begin{itemize}
    \item Historique
    \item Parabolique : ce qui compte, c'est le message
    \item mythique (comme dans les sept dormants)
\end{itemize}

\subparagraph{une forte réaction à Khalafallah} car dès le Coran, on reprochait au Prophète de ne transmettre que des mythes. La réponse de Khalafallah a été de dire que le point du prophète était de défendre que ces textes étaient révélés. Thèse refusée et \textit{fatwa} interdisant à Al-Khuli et Khalafallah d'enseigner les sciences coraniques. Thèse néanmoins publiée.


\subsection{Critique historique}

\paragraph{Mohammed Arkoun 1928 - 2010} \label{theol:Arkoun5} En France en 1954, agrégé d'arabe\sn{cf p. \pageref{theol:Arkoun1}}. Il est marqué par la pensée des IX-X siècle. Sa compréhension de l'Islam (comme dans toute religion) : 
\begin{itemize}
    \item \textbf{la religion forte} réponse théologique crédible aux grandes questions de l'homme : le bien, le mal, ce qui se passe après la mort,... C'est ce qui compte pour toute grande religion
    \item \textbf{religion individuelle} la façon dont une religion se déploie à l'intérieur de chaque individu dans sa vie intérieure et sa relation à Dieu
    \item \textbf{la religion forme} les formes prises par la Religion dans une situation sociologiques et historiques données. Pour lui secondaire. 
\end{itemize}

\subparagraph{Une sacralisation de la religion forme} en Islam :
\begin{itemize}
    \item sur le plan politique, le \textit{califat}
    \item sur le plan politique, la \textit{shari'a}, loi religieuse
    \item et \textit{l'umma} éventuellement pouvant être pensé comme un seul état.
\end{itemize}
Si on touche à l'un de ces trois points, on touche à ce qui est considéré comme coeur à l'islam

\subparagraph{il faut une archéologie pour déconstruire ces formes}  L'Islam aurait pu être autre que ce qu'il est aujourd'hui. Ce sont des rapports de force qui ont imposé le califat, la shari'a et l'umma. 
\subparagraph{Une cloture dogmatique} au X-XI eme siècle. Ce qui est pensable et ce qui est impensable (tabou). Par exemple :
\begin{itemize}
    \item on oublie la genèse du texte du Coran. Simplification alors qu'on sait
    \item Coran incréé.
\end{itemize}
Il y a en parallèle un imaginaire, celui de l'âge d'or des 4 premiers califes, qui permet de justifier cette fermeture.

Arkoun va même jusqu'à distinguer le fait coranique de la révélation et le document historique et littéraire, le \textit{corpus officiel clos} qui est lui le résultat d'un rapport de force. 
\begin{Ex}
Osman a fait brûler des corpus hétérodoxes.
\end{Ex}


\paragraph{Différence entre les modernistes et les réformismes} Les modernistes critiquent la trahison des premières générations, à l'opposé de ce qui est reçu par la tradition musulmane d'un âge d'or.



\section{le statut de la Parole de Dieu : un Coran créé ?}



\subparagraph{Le statut du Prophète }

\paragraph{F. Rahman 1919-1988} Indo Pakistanais. Formation moderne. Inde puis Oxford (th. Philosophie). Enseigne au Canada. Rentre au Pakistan en 1961 mais il va rencontrer une très forte contestation des savants pakistanais. Ils vont l'accuser d'apostasie. Exil aux US en 1968 où il connaît une célébrité.

\paragraph{Une vision du Coran où le Coran passe par la langue et l'imaginaire du Prophète}. Voir texte. Un Coran purement extérieur du fait des controverses chrétiennes (lo-
gos,...) La parole de Dieu mais elle est aussi parole du prophète (dans
son coeur) et pas seulement enregistreur. Il reconnaît une subjectivité du
Prophète. Elle s’incarne dans son imagination, sa langue,...


\paragraph{Coran Parole de Dieu}\sn{Fazlur Rahman
(1919-1988) \emph{Extrait de son livre \textbf{Islam} (Doubleday Anchor Book, New
York, 1968, 331 pp.), p. 25-28.}}

 \begin{quote}
     

Pour le Coran lui-même, et par conséquent pour les Musulmans, le Coran
est la Parole de Dieu (kalâm Allâh). Mohammed, lui aussi, était
absolument convaincu de recevoir le Message de Dieu, le Tout-Autre (nous
essaierons plus loin de comprendre plus précisément le sens de cette
Altérité absolue), à tel point qu'il s'est basé sur la force de cette
conviction pour rejeter quelques-unes des assertions historiques les
plus fondamentales de la tradition Judéo-Chrétienne concernant Abraham
et les autres Prophètes. Cet "Autre", par un certain canal, a "dicté" le
Coran avec une autorité absolue. La voix surgissant des profondeurs de
la vie parlait distinctement, impérieusement, sans méprise possible. Non
seulement le mot \textbf{"}\emph{qur'ân}", signifiant "récitation",
l'indique clairement, mais le texte du Coran lui-même déclare en
plusieurs endroits que le Coran est révélé verbalement et non pas
seulement son "sens" et ses idées. Le terme coranique pour "Révélation"
est \emph{wahy} dont le sens est assez proche d' "inspiration", à
condition que ce mot ne soit pas censé exclure nécessairement le mode
verbal (par "Parole", bien entendu, nous ne voulons pas dire le Son). Le
Coran dit: "Dieu ne parle à aucun humain (c'est-à-dire en paroles
audibles) excepté par \emph{wahy} (c'est-à-dire par idée-parole,
inspiration) ou derrière un voile, ou bien Il envoie un Messager (un
ange) qui parle par \emph{wahy}...Nous t'avons ainsi révélé un Esprit
qui est à nos ordres". (Q. 42,51-52)


Pendant le 2° et le 3° siècles de l'ère islamique, cependant, de
violentes différences d'opinion, des controverses, en partie causées par
l'influence des doctrines chrétiennes, s'élevèrent parmi les Musulmans
au sujet de la nature de la Révélation. L'"orthodoxie" islamique
émergente qui en était au point crucial de formuler son contenu précis,
mit alors l'accent sur l'origine externe de la Révélation Prophétique
afin d'en sauvegarder son "Altérité", son "Objectivité" et son caractère
verbal. Le Coran lui-même maintenait certainement son altérité vis-à-vis
du Prophète. Il déclare en effet: "\emph{L'Esprit fidèle l'a fait
descendre sur ton coeur pour que tu sois au nombre des avertisseurs}"
(Q. 26,194), et encore: "\emph{Dis: Qui est l'ennemi de Gabriel (qu'il
le soit), car c'est lui qui a fait descendre sur ton coeur... le Livre}"
(Q. 2,97). Mais il manquait à l'orthodoxie (en fait, à toute la pensée
médiévale) d'avoir les instruments intellectuels nécessaires pour
allier, dans sa formulation du dogme, l'Altérité et le caractère verbal
de la Révélation d'une part, et, d'autre part, son lien intime avec
l'oeuvre et la personnalité religieuse du Prophète, c'est-à-dire qu'il
lui manquait la capacité intellectuelle de dire, à la fois, que le Coran
est entièrement la Parole de Dieu et aussi, dans un sens ordinaire, la
parole de Mohammed. Le Coran affirme clairement les deux idées, car s'il
insiste sur le fait qu'il est descendu sur le "coeur" du Prophète,
comment peut-il lui être extérieur ? Ceci, bien sûr, n'implique pas
nécessairement que le Prophète ne percevait pas un personnage extérieur,
comme la tradition le dit, mais il est remarquable que le Coran lui-même
ne fait aucune mention d'un personnage dans ce contexte: ce n'est qu'en
lien avec certaines expériences (habituellement liées à l'Ascension du
Prophète), que le Coran parle du prophète comme de quelqu'un ayant vu un
personnage ou un esprit ou quelque autre objet "à la lointaine limite"
ou "à l'horizon" bien qu'ici encore, comme nous l'avons dit,
l'expérience soit décrite comme étant d'ordre spirituel. Mais
l'orthodoxie, par les \emph{hadith} ou les "traditions" transmises du
Prophète, en partie interprétées correctement, et en partie inventées,
ainsi que par une discipline théologique largement basée sur le
\emph{hadith}, fit de la Révélation prophétique un phénomène ne frappant
que son oreille, uniquement extérieur à lui, considérant l'ange ou
"l'esprit qui vient sur le cœur" comme un agent totalement externe.
L'image que l'Occident moderne garde de la Révélation prophétique se
base largement sur cette formulation orthodoxe plutôt que sur le Coran,
d'ailleurs la foi du Musulman ordinaire fait de même.

... Lorsque la perception intuitive de la morale chez Mohammed atteint
son plus haut point et s'identifie avec la morale elle-même, la Parole
est donnée avec l'inspiration elle-même. Le Coran est donc la pure
Parole divine, mais évidemment, il est également relié intimement à la
personnalité interne du Prophète Mohammed, dont la relation au Coran ne
peut être conçue mécaniquement comme un disque enregistré. La Parole
divine passe par le cœur du Prophète.
 \end{quote}


\begin{Synthesis}
Un Coran purement extérieur du fait des controverses chrétiennes (logos,...)
La parole de Dieu mais elle est aussi parole du prophète (dans son coeur) et pas seulement enregistreur. Il reconnaît une subjectivité du Prophète. Elle s'incarne dans son imagination, sa langue,...
\end{Synthesis}

\subsection{L’ « abaissement » de la Parole }
\paragraph{Abu Zayd (mort en 2010)} Littérature arabe en Egypte et étude aux USA. Approche littéraire. Une langue est forcément porteuse d'une culture donnée, une langue particulière. Il va jusqu'à parler d'une \textit{incarnation de la Parole}, une langue humaine. Si Dieu a choisi l'arabe, il faut étudier ce texte en Arabe comme n'importe quel texte littéraire. 

\subparagraph{Contextualisation culturelle}

\section{la question de l'interprétation}

\subsection{Approche littéraire} 
Avec Abu Zayd.
Tout texte demande à être reçu et interprété, il ne vit pas en soi. C'est le passage, la transformation par la subjectivité de l'auditeur. Et ce texte coranique est \textit{marqué par l'interprétation du Prophète}. Il va comparer le Coran à un organisme vivant qui est riche d'une infinité de sens et qui va se donner selon la compréhension de l'auditeur.

\paragraph{des critères pour interpréter}


\begin{itemize}
    \item Sens : sens qu'avait le texte pour le prophète et ses compagnons
    \item signification : sens pour aujourd'hui
\end{itemize}
Le critère est de s'assurer de la \textit{cohérence entre sens et signification}. 


\subsection{Approche scientifique }

\paragraph{A. Sirush 1945-} Approche par la philosophie des sciences. Il a découvert la philosophie des sciences; Thèse en Angleterre. Très enthousiaste au début de la révolution en Iran. Il s'est exilé aux US (Harvard).

\paragraph{Un parallèle entre nature/science et religion} A. Sirush fait un parallèle entre nature et science d'une part et Religion et ce qu'il nomme religiosité d'autre part:
 
\begin{tabular}{p{5cm}p{1cm}p{5cm}}
Nature &  & Religion   \\
\textit{{objectif, absolu} }      \\                        
          $ \downarrow $  &  &         $ \downarrow $  \\
Science &  & Religiosité\\
\textit{{partielle, mouvante}}  & & \textit{ce que nous comprenons de la religion} \\
\end{tabular}
 


\paragraph{théorie de l'extension et de la contraction de la connaissance religieuse}, la religiosité. Tout est lié, toute découverte sur les sciences a un impact sur notre philosophie et l'anthropologie, et donc ensuite sur notre compréhension.


\begin{Ex}
L'Eglise s'est opposée à l'idée que la terre tournait autour du soleil. Car cela changeait notre regard de la place de l'homme et de la création, au centre et au sommet de la Création.
\end{Ex}

\paragraph{un processus continu} de connaissance religieuse. Notre compréhension de la religion évolue. Il faut être dans une attitude de questionnement pour que le Coran parle. 
\begin{quote}
    Si l'homme n'a aucune question à poser au religieux, il ne vient apprendre de ce religieux. \ldots Même chose pour le Coran.
\end{quote}

\paragraph{Pluralisme de pensée}
Cela a des implications politiques et démocratiques. Dialogue.

\subsection{Glossaire}

 
{Personnes}



{Notions}
\begin{Def}[tafsir]
\emph{commentaire (exotérique) du Coran}
\end{Def}


\begin{Def}[ta'wil]
\emph{herméneutique ésotérique du Coran}
\end{Def}
 
