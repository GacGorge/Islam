\chapter{Le chiisme au XXe sècle : de la Réforme à la Révolution}

\mn{28/3/22}
\section{bibliographie}



 
 
 
  KHOMEYNI, Ruhollah \emph{Gouvernement islamique}, Institut pour
  l'édition et la publication des œuvres de l'ayatollah Khomeyni,
  Téhéran, 1996.
 
 
 
  SHARIATI, Ali \emph{Histoire et destinée} {[}textes choisis{]},
  Sindbad, Paris, 1982.
 

 
\emph{Fatima est Fatima}, Albouraq, Beyrouth, 2009.

ADELKHAH, Fariba \emph{La révolution sous le voile}, Karthala, Paris,
1991.

*ADELKHAH, Fariba ; BAYART, Jean-François ; ROY, Olivier \emph{Thermidor
en Iran}, Ed. Complexes, Bruxelles, 1993.

*DIGARD, Jean-Pierre ; HOURCADE, Bernard ; RICHARD, Yann \emph{L'Iran au
XXe siècle}, Fayard, Paris, 2007.

KIAN-THIEBAUT, Azadeh \emph{La République islamique d'Iran : de la
maison du Guide à la raison d'Etat}, Michalon, Paris, 2005.

KHOSROKHAVAR, Farhad \emph{L'utopie sacrifiée, sociologie de la
Révolution iranienne}, Presses de la Fondation Nationale des Sciences
Politiques, Paris, 1993.

*KHOSROKHAVAR, Farhad et ROY, Olivier : \emph{Iran, comment sortir d'une
révolution religieuse ?}, Seuil, Paris, 1999.

MERVIN, Sabrina \emph{Un réformisme chiite. Ulémas et lettrés du Gabal
`Amil (actuel Liban-Sud) de la fin de l'Empire ottoman à l'indépendance
du Liban}, Karthala-CERMOC - IFEAD, Paris-Beyrouth-Damas, 2000.

MERVIN, Sabrina (dir.) \emph{Les mondes chiites et l'Iran}, Paris,
Karthala-IFPO, 2007.
 


\subsection{Introduction}

Le Chiisme ne doit pas être pensé en vase clos, elle est en interaction avec les pensées qui l'entoure.

\section{Le réformisme chiite}
\label{le-ruxe9formisme-chiite}

  \paragraph{Lien très fort avec au centre Najjaf et Kerbala} En terme d'influence, et de centres du Chiisme, ce sont les villes saintes de Najjaf (tombeau de Ali) et Kerbala (tombeau de Hussein). Importance du pélerinage sur leurs tombeaux. 
  Pour être plus qu'un recteur de mosquée, il faut aller à Najjaf. 
  
  
     \subsection{Un mouvement parallèle au réformisme sunnite}

  \paragraph{Le réformisme chiite} comme le réformisme sunnite, le renouveau passe par :
  \begin{Def}[tajdid] 
\emph{renouvellement}.
\end{Def}

  \begin{itemize}
      \item   l'acceptation de la modernité et la raison au centre.
      \item   Il passe aussi par une purification de la foi : \emph{tawhid}, unicité de Dieu.
      \item Cela passe par un retour aux sources, Coran et sunna (chiite)
      \item Enfin, le désir de retrouver \emph{l'umma} et refaire l'unité avec le monde sunnite.
C'est d'abord un réformisme pensé par les religieux.
  \end{itemize}

\paragraph{Al-Amin (Muhsin)} réformisme libanais.Al-Sayyed Mohsen al-Amin (b.1284/1867-d.1371/1952), also transliterated Muhsin al Amin, was a Shia scholar, biographer, traditionist, and jurist. He was born in Jabal Amil, Lebanon. His most important work is A'yan al-Shi'a.[1] \sn{\href{https://en.wikipedia.org/wiki/Al-Sayyed_Mohsen_al-Amin}{Al Amin}}

\paragraph{Baqir al-Sadr (Muhammad) Borujerdi} né le 23 mars 1875 à Boroudjerd et mort le 30 mars 1961 à Qom \sn{\href{https://fr.wikipedia.org/wiki/Seyyed_Hossein_Tabatabai_Borujerdi}{Borujerdi}}


\paragraph{Changement de l'éducation} au début du XX.

 \begin{itemize}
     \item  manuel scolaire (avec une doxa réformisme)
     \item histoire des religions, psychologie
 \end{itemize}
Ils vont se heurter à beaucoup de résistances et cela ne va aboutir qu'à moitié.


  
 
\subsection{Particularités du réformisme chiite : \emph{ijtihad, taqlid} et
    \emph{bid`a}}

Dans le sunnisme, face à l'imitation stérile des savants précédents, \emph{le taqlid}, il faut faire l'effort d'interprétation des textes du Coran (\emph{ijtihad}). On dit souvent que les portes de \emph{ijtihad} se sont fermées au XIe siècle.
Très différent dans le chiisme : 
\begin{itemize}
    \item le taqlid est positif : il s'agit d'imiter le \textit{marja}
    \item l'\textit{ijtihad} s'est toujours poursuivi par les Ayatollahs \sn{seuls eux peuvent faire l'ijtihad}. Un rapport positif à la raison dans le chiisme : On a toujours étudié la philosophie grecque. De facto, dans le chiisme, le réformisme était plutot à lutter contre l'ijtihad face à des ayatollahs considérés comme sclérosés. 
\end{itemize}

\paragraph{refus de toute médiation}Face au Wahhabisme, qui condamne toute innovation (bid'a) au titre du \textit{tawhid} et donc toute médiation (dévotion aux saints, soufisme, rôle du clergé chiite). Or la dévotion aux saints est clé dans la pratique chiite\sn{rôle économique et religieux de cette dévotion aux imams}.

    \subsection{Réforme de la doctrine et des pratiques}

\paragraph{rationalisation des imams}  On va introduire une forme de rationalisation sur les imams : pouvoir surhumain des imams, préexistence des imams,...
  au profit de leur vie exemplaire sur le plan moral qu'il faut imiter : on ne parle plus d'intercession mais du rôle de modèle.
  
 \paragraph{Schizophrénie entre une théologie épurée et une pratique } et piété populaire, qui n'a pas été condamnée et qui continue : culte des imams, voeux aux imams,...
 
 \paragraph{Des pratiques qui ont reculé} la pression de la tombe : on va sentir le poids de la terre qui va faire rejeter le lait de la mere aux ongles. Donc il y avait un vrai commerce de cadavres qui allaient se faire enterrer à Kerbala (pour éviter sa pression).
 Mais Al Amin n'a pas réussi à faire évoluer les pratiques de Achoura pour des questions de moralités (travestis, flagellation,). 
 

\section{Genèse et destin du chiisme politique : Révolution iranienne et République islamique}
\label{genuxe8se-et-destin-du-chiisme-politique-ruxe9volution-iranienne-et-ruxe9publique-islamique}


 
 
 
   \subsection{Un contexte bien particulier : l'Iran pahlavi}
    
    C'est le roi, \textit{shah} d'Iran. Au début du XX, on a une révolution constitutionnelle en 1906, qui limite (un peu) le pouvoir du Shah en introduisant une constitution.
    
    En 1921, on a la prise de pouvoir par Reza Khan, officier, dans une démarche très séculariste. Mais il décide de rester dans le cadre monarchique pour ne pas froisser les Ayatollahs. Il décide donc de créer une nouvelle dynastie, la dynastie \textit{Palhavi}.
    
    \paragraph{Des réformes sécularistes} Un réseau d'écoles publiques, modernisés. Création de tribunaux séculiers et introduit un code civil occidental, à la place du système juridique islamique qui existait auparavant.
    
    En 1935, interdiction du voile dans l'espace publique. Mais résistance à cette mesure et le voile n'est finalement interdit que dans les administrations.
    
    \paragraph{Mohammed Reza Shad (1941)} son fils au pouvoir. En 1963, il fait la \href{https://fr.wikipedia.org/wiki/R\%C3\%A9volution_blanche}{révolution blanche}, qui comporte plusieurs volets : réforme agraire, droit de vote au femme. Cela lui vaut l'hostilité du clergé (qui était grand propriétaire). Par ailleurs, il entre dans une forme de mégalomanie, notamment en 1971 (2500 ans de monarchie ininterrompue). 
    En 1975, il adopte un calendrier achéménide au lieu d'un calendrier musulman. 

    \paragraph{le clergé prend la tête de l'opposition}
    
   \subsection{Ruhollah Khomeyni (1902-1989) et le \emph{velâyat-e faqih}}
    
    \paragraph{Ruhollah Khomeyni } \sn{\href{https://fr.wikipedia.org/wiki/Rouhollah_Khomeini}{Khomeyni sur Wikipedia}} dynastie d'Ayatollahs, bourgeoisie du bazaar (commerce) et propriétaires terriens. Il est même nommé Marja en 1962. Il enseigne dans l'université de \textit{qom}. Dès les années 60, il est à la tête de l'opposition. Jugement en 1963, peine de mort commué à l'exil. 14 ans en Irak mais indispose le pouvoir irakien et s'installe en France en 78.
    
    
    \paragraph{Delégitimisation du Shah} Applique au shah, le \textit{taghut} (idole, pharaon) \sn{\begin{Def}
    [taghut] : \emph{idole ; tyran}.
    \end{Def}}. face à cette délegitimisation,  le pouvoir revient aux clercs.
    \begin{Def}[velayat-e faqih]
    le pouvoir aux clercs
    \end{Def}
    
    \paragraph{Une rupture théologique} Dans le chiisme, il y a une idée d'une corruption profonde dans le monde et qu'il n'est pas possible d'imposer le règne de Dieu sur terre. Seule le mahdi peut le faire. Face à Khomeyni, beaucoup de marjas se sont imposés à lui et à sa volonté d'avoir un rôle politique.
    
    \sn{Quand le religieux se melent de politique, c'est toujours le religieux qui est asservi par le politique.}
    
    
    
\paragraph{Le \emph{velayat-e faqih} : Ruhollah Khomeyni (1902-1989)}



\begin{quote}
    

\emph{Méthode du gouvernement islamique. Ses différences avec les autres
gouvernements.} \sn{Extraits de \emph{Pour un gouvernement islamique}, 1969 (traduction
française chez Fayolle, 1979).}

Le gouvernement islamique ne ressemble à aucun autre gouvernement
actuellement en vigueur. Il n'est pas despotique. Le chef de l'État
n'est pas un despote qui se joue des biens et de la vie du peuple et en
fait ce qu'il désire, qui tue celui qu'il veut, et enrichit ou ennoblit
qui il veut, distribuant de-ci de-là les terres et les biens du peuple.
Le Prophète, Ali et les califes n'avaient pas ce genre d'attributions.
Le gouvernement islamique n'est ni despotique ni absolutiste, il est
constitutionnel \sn{rôle de la loi : cf Banna, \textit{le Coran est notre Constitution}}, bien entendu pas au sens habituel du terme, où les lois
sont approuvées par des personnes et une majorité : constitutionnel, au
sens où les dirigeants sont tenus à un ensemble de « conditions »
définies dans le Coran et dans la Sunna du Prophète à la fois en ce qui
concerne l'exécutif et l'administration. Ces conditions ne sont rien
d'autre que les lois islamiques, celles-là mêmes qui doivent être
observées et appliquées. De cette manière, le gouvernement islamique est
le gouvernement de la Loi divine sur le peuple.

C'est ce qui constitue la différence fondamentale entre le gouvernement
islamique et les autre gouvernements constitutionnels, monarchiques et
républicains ; un autre fait capital est que dans ces régimes, les élus
du peuple ou le monarque sont les législateurs, tandis que dans l'Islam,
le seul législateur est Dieu \sn{Idéalisation de la loi islamique}, le Législateur sacré.

Personne n'a le droit d'émettre des lois, et aucune loi n'est applicable
si ce n'est celles du Législateur. Voilà pourquoi dans le gouvernement
islamique, au lieu de l'assemblée législative qui représente
habituellement l'un des trois pouvoirs, il existe une assemblée de
planification qui a pour rôle d'organiser les divers ministères au
regard des lois islamiques et de déterminer, à l'aide de ces plans et
sur tout le territoire, la manière d'accomplir les services publics.

L'ensemble des lois islamiques réunies dans le Coran et dans la Sunna
ont été acceptées par les musulmans et ceux-ci leur obéissent. Ceci
facilite la tâche du gouvernement qui devient du même coup le
coordinateur du peuple. Tandis que dans les autres régimes
constitutionnels, la majorité de ceux qui se font passer pour les
représentants de la majorité du peuple approuvent ce qui leur plaît au
nom de la loi, et ensuite s'imposent au peuple tout entier.

Le gouvernement de l'Islam est le gouvernement de la Loi. Dans cette
méthode de gouvernement, la souveraineté revient exclusivement à Dieu,
et la Loi constitue l'ordre et le décret de Dieu. La Loi de l'Islam,
Ordre de Dieu, règne d'une façon absolue sur tous et sur l'État islami-
que. Tout les hommes depuis le Prophète, jusqu'à ses califes et au
commun des mortels, sont définitivement soumis à la Loi, loi qui est
envoyée par Dieu et expliquée dans le Coran et par le Prophète. Si
celui-ci a pris la charge du califat, ce fut sur l'ordre de Dieu. Il est
Calife de Dieu sur terre et non pas calife sur sa propre initiative dans
l'intention de devenir le chef des musulmans. Lorsque des risques de
conflits se firent jour dans la communauté, étant donné le caractère
récent des conversions à l'Islam, Dieu s'est révélé au Prophète et l'a
engagé à annoncer le califat de toute urgence, ainsi, en plein désert.
Mahomet désigna alors Ali comme Calife en obéissant à la Loi, non pas
parce que celui-ci était son gendre ou qu'il avait rendu des services,
mais parce que lui-même en avait reçu la mission divine et qu'il
s'inclinait devant l'ordre divin. Dans l'Islam, le gouvernement signifie
l'obéissance à la Loi, et seule la Loi exerce son autorité sur la
société. Là où une certaine limitation des attributions a été donnée au
Prophète et aux Imams, c'est l'œuvre de Dieu. Chaque fois que le
Prophète a exprimé quelque chose ou annoncé une loi, c'était en
obéissance à la Loi divine, Loi à laquelle tout le monde sans excep-
tion doit obéir, le gouvernant comme le gouverné. Obéir au Prophète est
également un ordre de Dieu qui dit : « Obéissez au Prophète. » La
soumission aux responsables du gouvernement ou imams est également un
ordre de Dieu qui dit: «Obéissez aux Imans qui sont issus de vous.»

L'opinion des personnes, fût-ce celle du Prophète, n'a pas de prise sur
la Loi divine. Tous se plient à la volonté de Dieu. {[}\ldots{]}


\begin{Synthesis}
C'est une nomocratie, plus qu'une théocratie
\end{Synthesis}

\emph{L'Exercice du pouvoir du faqih, par les textes}

\emph{Les faqih justes, véritables successeurs des prophètes}

Dans l'un des \emph{ravâyat} les plus authentiques, il est dit : « Ali
rapporte les paroles du Prophète : Dieu ! Pardonne à mes successeurs.
(\emph{Il a répété cette parole trois fois}.) --- Qui sont tes
successeurs ?, lui a-t-on demandé. --- Ceux qui viendront après moi et
qui citeront et enseigneront au peuple mes hadith et ma Sunna. »

Par conséquent, il ne fait aucun doute que ce \emph{ravâyat} ne concerne
pas les rapporteurs de hadith, c'est-à-dire ceux qui les rédigent ; en
effet, un écrivain ne peut être calife du Prophète.

Il faut entendre par le mot calife qui est employé dans ce
\emph{ravâyat}, le \emph{faqih} de l'Islam. La diffusion des lois et
l'éducation du peuple seront à la charge des \emph{faqih} justes, car
s'ils ne le sont pas, ils ressembleront aux juges qui ont inventé des
\emph{ravâyat} contre l'Islam à la manière de Samarat Ebn-e-Djandâb.
S'il n'y a pas de \emph{faqih}, le peuple ne pourra pas connaître le
\emph{feqh}, la Loi de l'Islam. Alors deviendra possible la propagation
de milliers de \emph{ravâyat} fabriqués par les agents des oppresseurs
et les \emph{âkhond} courtisans, destinés à faire l'apologie des
sultans. {[}\ldots{]}

\emph{Les pleins pouvoirs des faqih}

Par conséquent, « les faqih sont les confidents des prophètes »,
signifie que les \emph{faqih} ont le droit de prendre en charge tout ce
qui était du domaine des prophètes. Comme ceux-ci, ce sont des hommes
qui ne doivent dévier d'aucune loi et qui sont purs et désintéressés des
biens de ce monde, comme il est dit à la fin du hadith cité plus haut :
« Tant qu'ils n'entrent pas dans le monde » (dans le bourbier de la
recherche des biens matériels !). Si donc un \emph{faqih} pense aux
biens terrestres, ce n'est pas un \emph{faqih} juste, et il n'est pas le
confident du Prophète, ni l'exécutant des lois islamiques. Seul le
\emph{faqih} juste est en mesure d'appliquer les Lois, d'établir les
principes islamiques, d'administrer les peines et les châtiments, de
veiller aux frontières et à l'intégrité territoriale de la communauté
musulmane, bref d'être le Magistrat suprême du gouvernement. Comme le
Prophète, il peut établir les principes islamiques et appliquer les
lois. {[}\ldots{]}


 
\end{quote}
    
    
    
   \subsection{ `Ali Shariati (1933-1977) et la naissance du chiisme révolutionnaire}
   
   \paragraph{Ali Shariati (1933-1977)} intellectuel laic \sn{\href{https://fr.wikipedia.org/wiki/\%27Al\%C3\%AE_Shar\%C3\%AE\%27at\%C3\%AE}{Ali Shariati sur Wikipedia}} existencialisme, théologie de la libération, marxisme,...
   Il va penser le shiisme comme \textit{la religion des opprimés}. La révolution pour imposer la justice sociale et libérer le peuple.
   \begin{Def}[musta'zafin]
   Oppression. Un concept utilisé pour les persecutions religieuses qu'Ali Shariati va appliquer au contexte socio-économique
   \end{Def}
   Fatima va être mise en avant pour montrer qu'elle est femme religieuse. Dans la société islamique, les femmes ont perdu mais ont gardé un rôle important.
   
   La grande différence avec Khomeyni,n c'est que les clercs n'ont aucun rôle : les savants ont trahis Ali et Husseyn et ont maintenu le peuple dans l'oppression.
   
   Pour Khomeyni, l'état est basé sur la loi islamique (comme Mawdudi). Alors que pour Shariati, c'est la révolution.
   
   \paragraph{Un très grand mécontentement social} Face à l'échec de la réforme agraire mal faite, un exode rural. Khomeyni récupère le discours contestataire de Shariati. 
   

   
 \subsection{ La République islamique}
     
     
\paragraph{Notion de martyr}    , de Achoura, reprise par les discours politiques de la notion de martyr. 
   En 2009, un vrai changement de ce discours puisqu'un contre-discours a appliqué Achoura,  Yazid étant le gouvernement islamique.
   
   
\paragraph{1979 : constitution islamique d'Iran}   
    
\begin{figure}[h!]
    \centering
       \sidecaption{De la République islamique d'Iran : on voit à quel point c'est verrouillé. le conseil des gardiens filtre les candidats à la présidence, peut mettre son veto aux lois}
    \includegraphics[width=\textwidth]{CourantsIslamContemporain/ImagesCourantsIslamContemporain/image3.jpeg}
  
    \label{fig:my_label}
\end{figure}
 

 





\hypertarget{glossaire-6}{%
\section{ {Glossaire}}\label{glossaire-6}}

 
\paragraph{Personnes}



Khameney Khatami

Khomeyni (Ruhollah) Mohammed Reza Shah Shariati (`Ali)

{Lieux} Jabal Amil Karbala Najaf

Neauphle-le-Château

\paragraph{Notions}

ayatollah : « \emph{signe de Dieu » (sur terre) ; grade supérieur dans
le clergé chiite}. marja'-ye taqlid : \emph{« source d'imitation » ;
plus haut grade dans le clergé chiite.} 



tawhid : \emph{unicité (divine).}

mujtahid : \emph{celui qui pratique l'ijtihad.}

musta`zafin : \emph{les opprimés.}


