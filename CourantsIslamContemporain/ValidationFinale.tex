\chapter{Evolution de l'interdiction du Riba et Gharar}

Bonjour,
pour la validation du cours courant contemporain en Islam, je souhaitais vous proposer de regarder l'évolution de la prohibition de Riba ou du Gharar depuis le 19eme, en particulier ce qu'Abduh a pu en dire et la vision wahhabite.
Pourquoi : de même que le christianisme a dû revoir les concepts médiévaux d'interdiction de l'usure, lors du développement de la banque et de l'assurance au XIXème, la question s'est posée à l'Islam.
Or, dans mon expérience professionnelle, j'ai vu passer beaucoup de mémoires d'actuariat récent finance "Takaful" et qui parle de l'Islam avec un grand I et dérive des principes considérés comme acquis et les développent de façon concrete mais avec une marge d'interprétation pas toujours très assumée (voir par exemple le début de
\href{http://www.ressources-actuarielles.net/EXT/ISFA/1226-02.nsf/0/8c814ff5f2bae57ec1257e1a004407b6/\%24FILE/Memoire_ISFA_Tontines_et_Takaful_Bendimerad_Version_avec_Couverture.pdf}{Document Actuariat} ).
\begin{quote}
    [assurance vie].... incompatibilités avec le droit musulman : Gharar, Haram, Riba et Maysir [...]
A l'issue de l'analyse précédente, l'idée même d'une assurance islamique semble être un contresens. Pourtant, la religion musulmane encourage l'individu à prendre des mesures pour réduire l'ampleur des désastres qui pourraient l'affecter. D'après un Hadith1 authentifié, le prophète conseille à un croyant de placer sa confiance en Dieu et d'attacher son chameau plutôt que de se limiter uniquement à placer sa confiance en Dieu en offrant la possibilité au chameau libre de s'échapper. L'Islam ne s'oppose donc pas à l'idée de vouloir minimiser les risques et par conséquent elle ne s'oppose pas à faire usage de la loi des grands nombres. Elle exclut certes la spéculation et l'incertitude ainsi que le taux d'intérêt. En revanche, elle compte parmi ses principes la coopération et l'entre-aide mutuelle ainsi que le partage équitable des risques et des bénéfices. Toutes ces bases ont permis de concevoir un modèle alternatif à celui de l'assurance conventionnelle.
\end{quote}
 

Enfin, l est (peut être ?) intéressant de regarder la géographie de la finance takaful, forte en moyen-orient et surtout en Malaisie. Eventuellement, voir la "corrélation" avec les différents mouvements étudiés dans le cours en fonction du rapport à la modernité et l'occident.
J'ai vu plus d'articles de théologie sur l'interdiction que sur celle du Gharar, donc probablement plus facile ?
Qu'en pensez vous ?

%---------------------------------------------------------------------------------------------------------------
\section{Introduction}

%---------------------------------------------------------------------------------------------------------------
\section{Vision dans l'Islam Classique de l'interdiction de Riba}

\subsection{une interdiction de la Riba, usure que l'on retrouve aussi en Occident}

\paragraph{Saint Thomas d'Aquin}

\subsection{un nouveau contexte avec la naissance du Capitalisme}

\paragraph{Raison de la nécessité de Riba : l'industrialisation} augmente la capacité productive en cas d'investissement souvent important, ainsi que les risques.


%---------------------------------------------------------------------------------------------------------------
\section{Elements théologiques apportés}

\subsection{Une réponse prudente mais positive de la part des autorités religieuses Ottomanes}
\subsection{position d'Abduh}

%---------------------------------------------------------------------------------------------------------------
\section{Naissance de la finance et de l'assurance Islamique}

\subsection{Ce que l'on voit : Malaisie,...} 

\paragraph{une pratique supposant une \textit{shari'ah} qu'on ne peut interroger}

\subsection{Quelle est la théologie sous-jacente}