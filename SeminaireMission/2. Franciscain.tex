\chapter{les Franciscains et les expériences missionnaires}

\mn{Alberto présente}

\paragraph{Démarche} Jean Comby écrit \textit{2000 ans d'histoire de Mission. }démarche d'historiens pour montrer que la mission fait partie de l'histoire du christianisme. 


\section{Synthèse des textes}
\paragraph{François d'assise prêche l'évangile au Sultan d'Egypte (1219)} justification de la croisade par l'Evangile "arraché l'oeil". Surprise pour les musulmans qu'une religion d'amour puisse faire la guerre.

\paragraph{l'acharnement à l'évangélisation des franciscains au Maroc (1220)} cherchent le martyr. Rapport au témoignage et au martyr difficilement compréhensible aujourd'hui.

\paragraph{Correspondance entre Innocent V et le Khan Guyuk} \mn{le danger pour l'occident était réel. la lettre du Pape est de 1247 et Guyuk se dirigeait vers l'ouest quand il trouva la mort. Son successeur prit bagdad quelques années plus tard, provoquant un effroi dans la population} mention de la \textit{loi naturelle de parenté qui unissent les hommes entre eux}.  Une ecclésiologie du pape à l'époque : \textit{vicaire du Christ}, d'abord vis à vis des rois chrétiens puis du monde entier. Le pape va s'occuper que les juifs suivent la thora, les paiens suivent la loi naturelle.


\paragraph{Réponse de Guyuk} Conquêtes militaires vus comme bénédiction divine : "seul par ordre de Dieu"... \textit{si vous n'observez pas l'ordre de Dieu et contrevenez à nos ordres, nous vous saurons nos ennemis.}

\paragraph{Débat interreligieux à la cour de Mangou Khan 1253} \mn{Tuluy père de Möngke et fils préféré de Gengis Khan, épousa la princesse Soyughaqtani et conserve auprès d'elle une église nestorienne, leurs fils Möngke, Kubilai, Houlagou et Ariq Boqa sont élevés dans l'esprit de la foi chrétienne, mais la yassa mongole leur interdit d'être baptisé} Edward Conze rapporte dans son ouvrage sur le bouddhisme deux mots de Möngke, qui, alors qu'il favorisait nestoriens, bouddhistes et taoïstes au nord de l'Inde, vers 1250, montre l'ouverture de son esprit politicien. Au Franciscain Guillaume de Rubrouck, il déclara : « Nous croyons qu'il n'y a qu'un seul Dieu […]. Mais, comme Dieu a donné à la main plusieurs doigts, Il a donné de même aux hommes plusieurs voies », alors qu'il disait aux bouddhistes que leur mouvement était comme la paume de cette main dont les doigts étaient les autres religions.

\paragraph{pour réfuter l'Islam, il faut connaître le Coran (1141)} Pierre le vénérable ordonne la traduction du Coran (Robert de Ketton), car pour combattre l'Islam, il faut \textit{agir contre elle, c'est à dire écrire}. 


\paragraph{Le recours à la force de St Thomas d'Aquin} On ne peut contraindre quelqu'un à la foi mais on peut obliger quelqu'un par la force qui s'est converti à le contraindre même physiquement à accomplir ce qu'ils ont promis et à garder la foi. 



\paragraph{formation missiologique (c1300) Raymond Lulle} \href{https://fr.wikipedia.org/wiki/Raymond_Lulle}{Raymond Lulle} Missionnaire : saint, docte, connaissant la langue de l'autre, prêt à mourir.. 
 un contexte très moderne de dialogue inter religieux dans les îles de Méditerranée.  Missionnaire : saint, docte, connaissant la langue de l'autre, prêt à mourir.. 


\paragraph{Une règle de vie adaptée pour les prédicateurs de l'Evangile (1226)} adaptation de la règle mendiante (possibilité de garder de l'argent). Règle qui sera par la suite généralisée.


\paragraph{Un franciscain à Pékin (Khanbaliq -1305)}  persécution des Nestoriens. construction d'une église. 6000 personnes baptisés. Roi Georges\sn{8. Le nom « Zhu’an » du fils de Georges représente sans doute une forme médiévale/
dialectale (italienne méridionale) du nom « Jean », telle que Giuann/Giuànne (nous
remercions Pier Giorgio Borbone pour la suggestion; cf. Paolillo 2009a, p. 90). La
conversion de Georges est affirmée aussi dans la lettre de Pérégrin de Castello (1318).
L’authenticité de cette lettre n’est cependant pas tout à fait assurée et, même si elle
est authentique, elle ne peut pas être considérée comme un témoignage indépendant
concernant Georges car le récit se base sans doute sur ce que Pérégrin a entendu
de Jean (éd. Moule 1921, p. 111, van den Wyngaert 1929, p. 365 ; trad. Moule 1920,
p. 540, Id. 1930, p. 208, Hosten 1930b, p. 438, Dawson 1955, p. 232).} de \textit{la race du prêtre Jean} se convertit du Nestorianisme au catholicisme. Soutien implicite du Khan?


\paragraph{Lettre de pérégrin de Castello - Zayton 1318} en 1318, l'Eglise de Zayton, faute de prêtre, risque de ne plus être irriguée. Zayton : canton

\paragraph{André de Pérouse, évêque que Zayton 1326} Lettre teintée de tristesse ("tous sont morts dans la paix, je suis resté seul"). Mais qui montre la possibilité de convertir des "idolâtres". Pas de difficulté de conversion : "il existe dans ce vaste empire des hommes de toutes les nations qui sont sous le ciel, des religieux...

\section{Analyse}

\begin{Synthesis}

une période où la rencontre inter-religieuse est d'abord l'islam : le credo et donc la théologie. Une Religion finalement assez proche du Christianisme : pas de mystère, religion universelle. \sn{Alberto : le dialogue inter religieux devrait se faire sur la base théologique}
\begin{itemize}
\item  Effort d'inculturation partielle : traduction du Coran; langue de l'autre
\item  compétition avec les chrétiens locaux (Nestoriens) malgré leur connaissance plus fine de la culture locale
\item un courage fort; une abnégation réelle. Des ordres précheurs adaptés à cette évangélisation.

\end{itemize}


\end{Synthesis}

\paragraph{une diversité de texte} avec un public spécifique : propagande, ou transcrire un fait,... \textit{A qui ce texte est destiné ?}

\paragraph{des contextes différents} parfois pacifiques, parfois plus radicales. 
\begin{Ex}
Par exemple, le premier texte, raconte le discours de François par Célano dans le contexte de la béatification. Alors que le deuxième texte est écrit par Bonaventure pour légitimer l'\textit{ordre franciscain}.
\end{Ex}

\paragraph{les ordres précheurs} avant, on était dans le témoignage des chrétiens qui vivaient en paix dans le pays. les bénédictins convertissent des "chrétiens"  en proposant un mode de vie. Les ordre précheurs importent une réelle nouveauté.



\paragraph{Qu'est ce qu'on entend par Mission ?} Si on regarde les franciscains au Maroc, ils suivent un schéma christique (ensiengement non compris, martyr) mais ils manquent de \textit{guérison, de soin} pour l'autre, même les chrétiens. 

\paragraph{Comment on désigne l'autre ?} qu'est ce qui va nous orienter vers l'autre ?

\paragraph{la mission pensée d'abord en interne} les cathares,... et ensuite on envisage selon les mêmes critères ("hérésie") en appliquant la même vision à l'exterieur de la chrétienté.

\paragraph{Question de la mission dans un contexte occidental} Dans un contexte occidental, la mission se veut universelle. Est ce lié au Christianisme ou à une vision occidentale (qui pourrait venir de la pensée grecque ?) ?
Cela veut dire quoi \textit{témoigner} jusqu'à la mort ? 
