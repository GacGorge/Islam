\chapter{Rites Chinois}
\mn{la querelle des rites 1552-1773 présentée par Etiemble}
\section{introduction}

\paragraph{la querelle des rites} Longue série de discussions qui se développèrent au xviie et au xviiie s. entre missionnaires catholiques, sur le point de savoir s'il convenait de permettre ou d'interdire aux Chinois nouvellement convertis toute participation aux rites chinois traditionnels (culte des ancêtres, etc.).

La discussion débuta à la mort de Matteo Ricci (1610), jésuite, qui avait autorisé les rites en question, espérant les christianiser progressivement. Certains missionnaires (dominicains et franciscains notamment) les jugeaient idolâtriques. Durant tout le xviie s., les débats entre les deux conceptions de l'action missionnaire furent vifs. Rome se prononça (1645-46) dans un sens défavorable aux rites, mais on continua à débattre l’interprétation à donner à ces décisions. La discussion se poursuivit jusqu’en 1742, où Benoît XIV interdit strictement les rites aux nouveaux chrétiens. La querelle avait mis surtout aux prises les jésuites, défenseurs des rites, et les missionnaires des Missions étrangères de Paris, adversaires. Elle fut orchestrée en France par les philosophes et les jansénistes : les premiers en firent une machine de guerre contre l’Église, les autres s’en servirent pour attaquer la « morale » des jésuites. Il semble bien qu’en Chine les lettrés n’aient accordé aux rites qu’une importance civile et politique, tandis que le peuple chinois en faisait un acte religieux. Une partie du malentendu venait de là. Mais il s’agissait aussi de l’opposition entre deux conceptions missionnaires : fallait-il conserver tout ce qu’on pouvait des habitudes mentales et des pratiques traditionnelles des Orientaux, ou exiger une conversion tellement absolue qu’elle devenait une sorte d’occidentalisation ? En 1939, le pape Pie XI, constatant que l’évolution des mœurs avait nettement enlevé aux rites tout caractère religieux ou superstitieux, permit aux Chinois chrétiens de les pratiquer.

\paragraph{Les divergences entre missionnaires}
Après 1633, la Querelle des rites est aussi la partie visible des différends qui séparent les ordres de missionnaires. Les Dominicains lancent l'offensive en se plaignant que les Jésuites permettent à leurs convertis les rites aux Ancêtres et à Confucius. En 1639, une enquête commence de la part de Rome pour faire la lumière sur ce que permettent les Jésuites en Chine.

Les décrets des papes se suivent et se contredisent.

En 1645, un décret du pape Innocent X déclare les cérémonies comme superstitieuses et idolâtriques.

En 1656, un décret inverse du nouveau pape Alexandre VII considère une partie des cérémonies, dont les hommages aux Anciens, comme des coutumes civiles.

Finalement en 1669, Clément IX déclare le premier décret encore valide. Une confusion certaine règne entre les diverses proclamations.

En 1693, un mandement est proposé par Mgr Maigrot, des Missions étrangères de Paris, et soutenu par son supérieur, Jacques-Charles de Brisacier : c'est l'élément déclencheur de la crise.

Il contient une proposition : utiliser Tianzhu pour Dieu, interdire la tablette impériale dans les églises, interdire les rites à Confucius, condamner les cultes et tablettes des Ancêtres et encore quelques précisions. Et tout cela au moment même où Kangxi décrète l'Édit de tolérance.

Au sein des Jésuites et des autres ordres de missionnaires, les avis sont partagés. Dans le groupe des personnes favorables aux rites, on retrouve les missionnaires qui sont depuis plus longtemps en Chine, et donc influencés par les intellectuels chinois. De même, ceux qui admirent Ricci, poursuivant donc ses recherches en sinologie et les contacts avec les élites, sont plutôt favorables aux rites. Un autre groupe de missionnaires, qui travaille plus à la christianisation par le bas et qui probablement fait face plus qu'aux rites officiels à toutes sortes de superstitions locales, est favorable au mandement.

Mais il ne faut pas oublier que les Chinois n'apprécient certainement pas que des missionnaires s'opposent à leur rites et traditions. Une justification pour la permission des rites est le fait que ces derniers vont peu à peu disparaître, mais que les garder au début facilite selon certains les conversions.

\paragraph{Condamnation papale définitive}
Interdiction de 1704
Un décret de Clément XI en 1704 condamne définitivement les rites chinois. Il reprend les points du Mandement. C'est à ce moment qu'est instauré par l'empereur le système du piao : pour enseigner en Chine les missionnaires doivent avoir une autorisation -le "piao"- qui leur est accordée s’ils acceptent de ne pas s’opposer aux rites traditionnels. Mgr Maigrot, envoyé du pape en Chine, refuse de prendre le piao, et est donc chassé hors du pays.

L'empereur Kangxi est impliqué dans le débat. L'empereur convoque l'accompagnateur de Mgr Maigrot et le soumet à une épreuve de culture : ce dernier ne réussit pas à lire des caractères et ne peut discuter des Classiques. L'empereur déclare que c'est son ignorance qui lui fait dire des bêtises sur les rites. De plus, il lui prête plus l'intention de brouiller les esprits que de répandre la foi chrétienne. Les Chinois commencent à percevoir le manque d'unité dans le message des missionnaires. Kangxi juge impertinents les jugements émis par des gens peu cultivés. Une nouvelle délégation est conduite par Mgr Mezzabarba. Il devait faire accepter le Mandement par les Jésuites en Chine. L'accueil est poli, mais peu à peu la pression s'exerce envers Mezzabarba pour qu'il approuve les rites.

Autorisation de 1721
Une bulle papale en 1721, de Benoît XIII, accorde les huit permissions requises par les Jésuites et retransmises par Mezzarbarba. Yongzheng succède à Kangxi comme empereur et interdit le christianisme en 1724. Seuls les Jésuites, scientifiques et savants à la cour de Pékin, peuvent rester en Chine.

\Section{les textes étudiés}

\paragraph{Le décret de Clément XI - 1704}
\begin{itemize}
    \item Pas de mot \textit{Tien }
    \item des interdictions différentes ? \textit{Il ne saurait être permis}(VII) vs  \textit{on ne peut permettre} ?
\end{itemize}

\paragraph{le mandement de Mgr de Tournon}
\textit{Nous glorifioions Dieu d'un même coeur et dans un même langage, lui qui n'est pas un Dieu de dicorde}
"nous en sommes certains par la déclaration que nous en a faite le Patriarche d'Antioche \ldots et à qui nous sommes obligés de croire (114)

Clair : c'est non.

\paragraph{Lettre de Mgr d'Ascalon, vicaire Apolstolique pour le Kiang Si, Augustin}

\textit{grand scandale des fidèles et encore plus des infidèles dont cette ville est remplie. }

\subparagraph{ces messieures du séminaire de Paris}

\paragraph{les malheurs de Mgr Maigrot} en Irlande. "et pour y déplorer pendant cette attente la malheureuse situation de ceux qui défendent la vérité et la cause de Dieu et de son Eglise, au milieu des efforts continuels que font les père de la Société pôur les traverser". 


\paragraph{Mémoire pour ROme sur l'Etat de la Religion chrétienne dans la Chine} \textit{faussement exposé ce qui se passe en Chine}
\paragraph{Plaidoyer} protestation des jésuites. Insiste surtout sur les changements de doctrine.  

\paragraph{L'Empereur K'ang hi} d'abord tolérant puis se réduit, devant les risques sur la paix civile.


\Section{Analyse}

\paragraph{Une vision différente selon les élites et le peuple} les élites voyaient dans le confucianisme un rite civil, alors que le peuple le voyait religieusement. Cela peut expliquer la différence de point de vue

\paragraph{Michel Serres Le Vatican et la Chine} \sn{href{https://www.cairn.info/revue-etudes-2013-2-page-161.htm}{Etudes}}
\begin{quote}
    12Par parenthèse, j’ai longtemps déploré la décision pontificale, consécutive à la Querelle des Rites, beaucoup plus tardive et coupant court à la conversion de la Chine au christianisme, obtenue par les jésuites, autour de Matteo Ricci, au xviie siècle ; l’histoire du monde en eût été bouleversée. Je pensais qu’admettre ou refuser le culte des ancêtres n’avait été, dans le débat, que prétexte pour un tel refus. Je m’aperçois seulement aujourd’hui de l’importance décisive du décret final.

13Le culte des ancêtres marquait, en effet, que la société chinoise était fondée sur la famille ; refuser qu’il soit ajouté à la pratique du christianisme, pour accepter le peuple chinois dans la confession, révèle, au contraire, que l’Église, consciente de son histoire, ne voulut pas revenir en arrière, vers la société archaïque qu’elle avait contribué à faire disparaître dans les lieux de son influence. La Querelle des Rites et sa conclusion négative montrent une sorte d’expérience cruciale dans la thèse ici défendue.

14Dans ces conditions, les historiens susdits se posent la question des causes qui poussèrent l’Église du Moyen Âge à une telle conduite. Et, comme d’habitude, ils évoquent des intérêts, en particulier d’économie. Certains disent qu’elle se conduisit, sur ce point, en captatrice d’héritages ; en quelque sorte, elle cherchait à prendre la place des légataires.

15Je ne sache pas qu’ils aient cherché ces causes dans la tradition ecclésiale elle-même, c’est-à-dire dans les Évangiles et la Théologie.
\end{quote}


\section{résumé}

Ouvrage d'Entiemble. 
Vision plutot de FX de la première époque que de la deuxième époque, celui avant de la rencontre japonaise.
De fait, Matteo Ricci ouvrira à une pratique chinoise adaptée.

Faut il garder la culture locale quite à être \textit{borderline} par rapport à la théologie.


\paragraph{pas uniquement un pb de forme} Chine ont une vision, comprehension du monde différente. Mais foi est elle véritable ?

Saint Augustin : 

\paragraph{verbe tolérer}utilisé au début du XIX, mais à l'époque pas forcément très clair.

\paragraph{vertical vs horizontal} verticalité chinoise, horizontale occidentale (???)

\paragraph{faible utilisation des Evangiles}


