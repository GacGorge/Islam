\chapter{Pourquoi le dialogue ? }
\mn{3ème séance : Pourquoi le dialogue ? 29 septembre 2022
Jacques DUPUIS, « Le dialogue interreligieux, praxis et théologie », dans Vers
une théologie chrétienne du pluralisme religieux, Paris, Cerf, 1997, p. 543-582.}



\begin{Synthesis}
Les autres traditions religieuses participent de la réalité du Règne de Dieu.
le dialogue interreligieux
fait partie de la mission évangélisatrice de l’Église
\end{Synthesis}

\begin{quote}

Nous avons noté dans le chapitre précédent que les chrétiens
et les membres des autres traditions religieuses participent
ensemble à la réalité du Règne de Dieu et sont destinés à
l'édifier ensemble au cours de l'histoire jusqu'à sa plénitude
eschatologique : ils sont comembres et cobâtisseurs avec Dieu
de son Règne sur terre. Nous ajoutions qu'on trouve peut-être
ici ce qui - dans une perspective théologique chrétienne -
constitue le fondement le plus profond du dialogue interreligieux
entre les chrétiens et les « autres». Il n'est donc pas surprenant
qu'une théologie chrétienne du dialogue interreligieux
adopte, de préférence, une \textbf{perspective «régnocentrique».}


Pareille perspective coïncide en outre avec celle de Jésus
lui-même. Nous avons rappelé que le Règne de Dieu était au
centre de la vie et de la mission de Jésus, de son message et
de son action ; le Règne était, pour le Jésus historique, « la réalité
véritablement dernière» (J. Sobrino 1) . Il doit en être de
même pour l'Église également, s'il est vrai que l'Église est
destinée à prolonger la mission de Jésus lui-même. L'évangile
selon Marc montre cela de manière frappante. Au début de son
évangile, Marc offre un résumé programmatique de la mission
de Jésus : il allait et « proclamait l'Évangile» en disant : « Le
Règne de Dieu s'est approché» (Mc 1, 14-15). Selon le même
Marc, le Christ ressuscité envoie ses disciples par le monde,
leur enjoignant de « proclame[r] l'Évangile» (Mc 16, 15) du Règne de Dieu, qui est maintenant advenu par le mystère de
sa Pâque.


L'Église, avons-nous montré dans le chapitre précédent,
n'est pas à son propre service, mais au service du Règne de
Dieu. Le Règne de Dieu est l'horizon de toute son « activité
missionnaire ». Cela est très bien exprimé dans les « Thèses
sur le dialogue interreligieux » ( 1987) de la Commission théologique
consultative de la Fédération des Conférences épiscopales
asiatiques: \begin{quote}
    « Le point de convergence [focus] de la
mission évangélisatrice de l'Église est la construction du
Règne de Dieu, et de l'Église au service du Règne. Le Règne
est donc plus large que l'Église. L' Église est le sacrement du
Règne. Elle le rend visible, elle lui est ordonnée, elle le promeut;
mais elle ne lui est pas identique» (6, 3 1).
\end{quote}

\end{quote}

\subsection{le dialogue interreligieux
fait partie de la mission évangélisatrice de l’Église}


\begin{quote}
Ce qu'il faut montrer à présent est que le dialogue interreligieux
fait partie de la mission évangélisatrice de l'Église.
Cela n'a pas toujours été perçu dans la théologie de la mission,
même durant les dernières décennies. C'est en fait une acquisition
récente des années postérieures au concile Vatican II,
dont les antécédents doivent être brièvement rappelés.
Le terme «évangélisation» a tendu à remplacer celui de
« mission » dans la théologie de la mission avant les années du
Concile; le Concile a utilisé les deux termes, souvent en les
combinant. Toutefois, dans les documents conciliaires, « évangélisation
» reste un concept étroit, pratiquement identifié avec
l' «annonce» de l'Évangile destinée à inviter les « autres » à
se joindre à la communauté de l'Église. Le Concile - stimulé
par l'encyclique de Paul VI \textit{Ecclesiam suam} (1964) - a lancé
un vibrant appel en faveur du dialogue avec les membres des
autres traditions religieuses (NA 4; GS 92); mais il n'y est dit
nulle part que le Concile considère le dialogue interreligieux
comme une dimension de la mission évangélisatrice de l'Église.
Quels que soient l'importance ou le mérite devant être attribués
au dialogue, en termes de sa relation avec l'évangélisation, il ne représente qu'une première approche des autres, à
laquelle le terme théologique préconciliaire de « pré-évangélisation» \mn{Pré-évangélisation : Matteo Ricci et le calendrier solaire; laiors que le calendrier lunaire est à la base des \textit{superstitions} chinoises} pourrait encore être appliqué.


Cela veut dire que considérer le dialogue comme un élément
intégrant de l' « évangélisation » représente un changement
qualitatif significatif dans la théologie post-conciliaire de
la mission. Cela fait partie du développement, dans les années
qui suivirent Vatican II, d'une notion large et globale de
l'évangélisation, dont le dialogue - avec d'autres éléments -
est une dimension intégrante. Le changement ne s'est toutefois
pas produit sans hésitations ni contretemps. Témoin le fait
que l' exhortation apostolique \textit{Evangelii nuntiandi }(1975) de
Paul VI - le « pape du dialogue» - est restée totalement silencieuse
sur le sujet. Dans cette exhortation papale, les «autres»
étaient considérés seulement comme « bénéficiaires » de la
mission évangélisatrice de l'Église - toujours conçue principalement
en termes de l' «annonce» de l'Évangile et des activités
ecclésiales qui lui sont liées. La percée, comme on le verra ci-après,
s'est effectuée avec des documents des années 80 et 90.
\end{quote}

Dialogue pensé comme mission à partir de Jean-Paul II. Avant, transmission (les "autres").
\begin{Def}[Evangélisation]
«Évangélisation», ou mission évangélisatrice de l'Église, « se réfère à la mission de l'Église dans son ensemble» (DA 8).
En Vatican II, « évangélisation » reste un concept étroit, pratiquement identifié avec
l' «annonce» de l'Évangile.     
\end{Def}
\begin{quote}
Toutefois, avant d'aller plus loin, quelques éclaircissements
concernant les termes seront encore une fois utiles. Nous examinerons
principalement l'évangélisation, le dialogue, et l'annonce.
Les définitions de ces termes, telles qu'elles sont
proposées ici, sont empruntées principalement au document
Dialogue et annonce (1991), déjà mentionné auparavant \sn{Texte dans PONTIACIUM CONSILIUM PRO DIALOGO INTER RELIGIONES, Bulletin,
77; 26 (1991/2), p. 260-302.}.
\paragraph{Évangélisation}
\begin{quote}
   «Évangélisation», ou mission évangélisatrice de l'Église, « se réfère à la mission de l'Église dans son ensemble» (DA 8),étant en fait constituée de divers éléments.  
\end{quote}



 

\paragraph{Dialogue}
 
En ce qui concerne
le «dialogue», une distinction doit être faite entre dialogue
comme attitude ou esprit, et dialogue comme élément distinct,
de plein droit, de la mission évangélisatrice de l'Église.
\begin{quote}
    L' « esprit de dialogue » se réfère à une « attitude de respect et
d'amitié, qui imprègne ou devrait imprégner toutes les activités
qui constituent la mission évangélisatrice de l'Église» (DA 9).
\end{quote}    

En tant qu'élément spécifique et intégral de l'évangélisation,
le dialogue veut dire 
\begin{quote}
    «l'ensemble des rapports interreligieux,
positifs et constructifs, avec des personnes et des communautés
de diverses croyances, afin d'apprendre à se connaître et à
s'enrichir les uns les autres [DM 3], tout en obéissant à la
vérité et en respectant la liberté de chacun. Il implique à la fois
le témoignage et l'approfondissement des convictions religieuses
respectives» (DA 9).
\end{quote}

\paragraph{Annonce}
\begin{quote}
    L' «annonce», d'autre part, « signifie la communication du
message évangélique, le mystère du salut réalisé pour tous par
Dieu en Jésus-Christ avec la puissance de l'Esprit Saint. C'est
une invitation à un engagement de foi en Jésus-Christ, une
invitation à entrer par le baptême dans la communauté des
croyants qu'est l'Église» (DA 10).
\end{quote}

L'importance de ces notions, si nous voulons éviter les
confusions et les malentendus, apparaîtra ci-après. Pour l'instant,
il suffit d'indiquer ce qui suit à titre d'éclaircissements
ultérieurs. L' « esprit de dialogue» doit informer chaque aspect
ou élément de la mission évangélisatrice. \textbf{Ainsi, l' «annonce»
de l'Évangi!e, par laquelle les membres des autres traditions
religieuses sont invités à devenir - librement - des disciples de
Jésus dans l'Église, doit être faite dans un «esprit de dialogue».} \mn{Qu'est ce que cela veut dire pratiquement ? se laisser transformer par la rencontre ?}


Le dialogue, toutefois, en tant qu'élément spécifique
de l'évangélisation, est distinct de l'annonce ; il n 'a pas pour
but - comme nous le verrons clairement - la «conversion»
des autres au christianisme, tandis que, bien sûr, il implique
nécessairement de la part de l'évangélisateur le témoignage de
sa vie - sans lequel aucune activité évangélisatrice, quelle
qu'elle soit, ne peut être ni sincère ni crédible.
Le chapitre sera divisé en deux sections. La première montrera
la place du dialogue interreligieux dans la mission évangélisatrice,
comme élément distinct et . intégral de cette
mission, de plein droit. La seconde examinera les défis que
pose la rencontre inter-religieuse à la mission évangélisatrice
de l'Église, ainsi que les fruits et les avantages qui dérivent du
dialogue pour la foi et la théologie chrétiennes.

\end{quote}

\section{Une revue du magistère récent}

\begin{quote}
    L'importante contribution\sn{Une collection importante de documents du magistère pontifical sur le
dialogue interreligieux a été publiée par le Conseil pontifical pour le dialogue
interreligieux. Voir F. GtOIA (éd.), Il dialogo interreligioso ne/ magis·
tero pontificio (Documenti /963-/993).} de l'enseignement du pape Jean-Paul
II à une théologie des religions a consisté en sa constante
affirmation de la présence et de l'action de l'Esprit de Dieu
panni les membres des autres religions. De cette façon, il a
donné un fondement théologique à la signification du dialogue
interreligieux dans la mission de l'Église. Ainsi, s'adressant
aux peuples d'Asie en 1981, le pape a souligné une fois de
plus le thème de l'Esprit Saint \sn{Adresse sur Radio Veritas, Manille. Le texte se trouve dans AAS 73
(1981 ), p. 391-398 ; trad. fse dans DC 78 (1981), p. 281-282.}. L'Eglise, a-t-il affirmé, 
\begin{quote}
    « ressent
à notre époque un profond besoin d'entrer en contact avec
toutes ces religions».
\end{quote}
 Ce qui semble rassembler et unir les
chrétiens et les croyants d'autres religions d'une manière particulière
est la reconnaissance d'un besoin de prière. Le pape
exprimait sa conviction que l'Esprit de Dieu est présent dans la
la prière de chaque personne qui prie, chrétien ou autre \sn{Voir texte cité p. 263.}
Il concluait :

\begin{quote}
    « Tous les chrétiens doivent donc s'engager dans
le dialogue avec les croyants de toutes les religions, de manière
que leur compréhension et leur collaboration mutuelles puissent
s'accroître; de manière que les valeurs morales soient renforcées;
de manière que Dieu soit glorifié dans toute la
création. Des moyens doivent être développés pour faire en
sorte que ce dialogue devienne partout réalité, mais tout particulièrement
en Asie, le continent qui est le berceau des
anciennes cultures et des anciennes religions» (5 5) .
\end{quote}
Tandis que l'appel au dialogue se fait ici plus pressant que
dans des documents précédents, la doctrine reste la même.
Une reconnaissance de la présence agissante de l'Esprit chez 
les autres transforme le dialogue inter-religieux en une tâche
importante et en un besoin ressenti par l'Eglise. Mais cette
doctrine n'est pas encore exposée explicitement en termes de
mission et d'évangélisation.
La situation est la même si nous considérons le discours du
pape à la curie romaine (22 décembre 1986), dans lequel il
disait voir dans la Journée mondiale de prière pour la paix, à
Assise (27 octobre 1986), une « illustration visible, une leçon
de choses», de ce que signifie l'engagement de l'Église dans
le dialogue interreligieux, recommandé par le Concile (7\sn{Texte dans Assise. Journée mondiale de prière pour la paix (27 octobre
1986). p. 147-155.}). Ici,
plus clairement que jamais auparavant, le pape pose le fondement 
théologique d'un tel dialogue :  dialogue, en se référant au mystère
de l'\textit{unité} qui existe déjà entre les chrétiens et ceux qui restent
«orientés» vers l'Eglise (8). L' « unité universelle » est
fondée sur l'origine et la destinée commune de toute l'humanité
en la création (3), sur l'unité du mystère de la rédemption en Jésus-Christ (4-7), et sur la présence active de l'Esprit de Dieu dans la prière sincère des membres des autres traditions
religieuses (11).


\subsection{Unité}
Il y a donc d'abord une \textit{unité radicale} provenant de la
création:
\begin{quote}
    «Il n'y a qu'un seul dessein divin pour tot être
humain qui vient en ce monde (voir Jn l, 9) » (3). 

« Les différences
sont un élément moins important par rapport à l'unité
qui, au contraire, est radicale, fondamentale et déterminante »
(3). \sn{face à l'absolutisation de la culture et de l'individualisation, rappel de l'unité du genre humain.}
\end{quote}
 11 y a ensuite l'unité fondamentale établie sur le mystère
de la rédemption universelle en Christ (4). A Ja lumière de ce
double mystère d'unité, « ]es différences de tout genre, et en
premier Jieu les différences religieuses, dans ]a mesure où eHes
sont réductrices du dessein de Dieu, se révèlent [ ... ] comme
appartenant à un autre ordre. [EUes] doivent être dépassées
dans le progrès vers Ja réalisation du grandiose dessein d' unité
qui préside à Ja création» (5). Malgré ces différences, ressenties
parfois comme des divisions insurmontables, tous ]es
hommes « sont indus dans le grand et unique dessein de Dieu,
en Jésus-Christ» (5). « L'unité universe]]e fondée sur 1' événement
de Ja création et de la rédemption ne peut pas ne pas laisser
une trace dans la vie réelle des hommes, même de ceux qui
appartiennent à des religions différentes» (7). Ces « semences
du Verbe » répandues chez ]es autres constituent le fondement
concret du dialogue interreligieux promu par le Concile » (7).
Appartient également à ce fondement l'influence de l'Esprit
sur toute prière authentique, car « l'Esprit Saint [ ... ] est mystérieusement
présent dans le coeur de tout homme» (l l). L'Église,
pour sa part, est « appelée à travailler de toutes ses forces
l'évangélisation, la prière, Je dialogue) pour que disparaissent
entre les hommes les fractures et ]es divisions qui les éloignent
de Jeur principe et fin et qui les rendent hostiles les uns aux
autres» (6).
Le fondement théologique du dialogue interreligieux, tel
que le voit Je pape, est énoncé avec une grande clarté dans ce
texte. Pourtant, le document ne déclare nulle part explicitement
que Je dialogue interreligieux fait partie de la mission
évangélisatrice de l'Église.  
Pour en trouver une claire affirmation,
nous devons nous tourner vers le document publié par
le Secrétariat pour les non-chrétiens, Dialogue et mission
(1984). Le document du Secrétariat s'intéresse surtout au rapport
entre le dialogue et la mission (DM 5). 11 faut noter - et regretter
- que dans l'Introduction du document ce rapport soit
encore conçu en termes d'une dichotomie entre évangélisation
et dialogue. Mention est faite de « la présence simultanée, au
sein de la mission, des exigences inhérentes à l'évangélisation
et au dialogue », et des difficultés qui peuvent en surgir (DM 7).
Toutefois, cette impression de dichotomie est vite dissipée.
Dans la première partie, qui porte sur la mission, Je document
explique que ]a mission de l'Église « est unique, mais elle
s'exerce de manières diverses selon les conditions dans lesquelles
Ja mission est engagée » (DM 11 ). Le document entend
tracer « les modalités et les différents aspects de la mission »
(DM 12). 11 Je fait dans un passage qui, sans prétendre être
exhaustif, énumère cinq « éléments principaux» de la « réalité
unitaire, mais complexe et articulée» de la ·mission évangélisatrice
de l'Église. L'importance de ce texte exige qu'il soit
cité largement :
\begin{quote}
    
\end{quote}
La mission est, d'abord, réalisée par la simple présence et le
témoignage efficace de la vie chrétienne [voir EN 21]; même-sron
·doit reconnaître que « nous portons ce trésor dans des vases d'argile
» (2 Co 4, 7), que l'écart est toujours impossible à combler entre
la manière dont le chrétien vit réellement et ce qu'il affirme être.
Il y a, ensuite, l'engagement effectif au service des hommes ainsi
que toute l'action pour la promotion sociale, pour la lutteëontre la
pauvreté et les structures qui la favorisent.
Il y a, en plus, la vie liturgique, la prière et la contemplation qui
sont des témoignages éloquents d'une relation vivante et libératriëe
avec le Dieu vivant et vrai qui nous appelle dans son Royaume et
dans sa gloire [voir Ac 2, 42].
Il y a, aussi, le dialogue grâce auquel les chrétiens rencontrent les
croyants d'autres ·traditions- religieuses poùr riïarêbëTTnsem61e à la
iëëlierche de la vérité et pour collaborer en des oeuvres d'intérêt
commun.
Il y a, enfin, l' annonce. et la catéchèse, lorsqu • on proclame la
Bonne Nouvelle, qu'on en approfondit les repercussions sur la vie et
les cultures.
.,Tous ces éléments entrent dans le cadre de la nùssion [DM 13].
« Tous ces éléments entrent dans le cadre de la mission»,
mais la liste n'en est pas complète. Nous ferons quelques
observations. La proclamation de l'Évangile par l'annonce et
la catéchèse vient à la fin, et à juste titre, car la mission ou
l'évangélisation doit être vue comme un processus dynamique 1•
Ce processus culmine en effet dans l'annonce de Jésus-Christ
par le kèrugma (annonce) et la didachè (catéchèse). Selon le
même principe, toutefois, la phrase « la vie liturgique, la prière
et la contemplation » aurait dû être insérée après l'annonce de
Jésus-Christ, à laquelle ces éléments sont dire~tement liés
- tout comme en Ac 2, 42, auquel le texte renvoie - et dont
ils sont l'aboutissement naturel. L'ordre aurait alors été: présence,
service, dialogue, annonce et sacramentalisation - Ies
deux dernières correspondant aux activités ecclésiales qui,
selon la vision plus étroite, mais traditionnelle, constituaient
l'évangélisation.
Dans la perspective plus large adoptée par le d~ument, ~a
« réalité unitaire» de l~é.vangé.lisation est présentee a la fois
comme « complexe et articulée » : elle est u~ pro~~-s~~s. Cela
signifie que, tandis que tous les éléments du processus sont
des aspects de l'évangélisation, tous n'ont pas la Ilême place
ni la même signification dans la mission de l'Eglise. Par
exemple, le dialogue interreligieu~pr~çède 1!.Q~l~~Et)' ~n
Q!!ÇÇ. Le premier peut être ou non suivi de la seconde,.rn~i~
le processus d' ~\'angélisation n'est porté à son terme qu~ s1
1 •~2!!Ç(? suit le dialogue: annon~e et, ~acr~menta!i~at~on /) •.
représentent le sommet de la nussion evangehsatrtce de I Eghse. !
Ayant insisté une fois de plus sur l' «importance» du dialogue
dans la mission (DM 19), le document étudie le dialog~e
de plus près dans la seconde partie. Le dialogue est en lUI:
même une expression distincte de l'évangélisation; c'estau~si
« une attitude et un esprit», et donc « la norme et le style indispensables
de toute mission chrétienne et de chacune. de ses
formes, qu'il s'agisse de la simple présence et du témoignage,
ou du service ou d'annonce directe» (DM 29). Tous les aspects
de la mission énumérés auparavant doivent être « imprégné[s]
de l'esprit de dialogue» (ibid.). Le dialogue, en ~t que dimension
distincte de l'évangélisation, se différencie amsi nettement
de l' « esprit de dialogue» qui doit informer toutes les
expressions de la mission évangélisatrice de l'Église.
Le dialogue interreligieux lui-même, comme tâche s~écifique
de l'évangélisation - qui est « inséré dans le dyn:ii:111sme
global de la mission de l'Église» (DM 30) - peut r~vetlr plusieurs
formes : le dialogue de la vie, ouvert et accessible à tous
(DM 29-30); le ·dial9gùe en un engagement commun aux
oeuvres de justice et de libération humaine (DM 31-32); le
dialogue ,intellectuel où des spécialistes s'engagent en des
échanges sur leurs patrimoines religieux respectifs, en vue de
promouvoir la communion et la fraternité (DM 33-34); enfin,
sur Je plan Je plus profond, le partage d'expériences religieuses,
de prière et de contemplation, dans une recherche commune
de l' Absolu (DM 35). Toutes ces formes de dialogue I sont,
pour le partenaire chrétien, autant de moyens d' oeuvrer à la
« transformation des cultures par l'Évangile» (DM 34), autant
d'occasions de partager avec les autres, d'une manière existentielle,
les valeurs de l'Évangile (DM 35).
Nous devons à présent nous tourner d'abord vers l'encyclique
Redemptoris missio et, ensuite, vers le document Dialogue
et annonce pour demander comment ils conçoivent la
placedu dialogue interreligieux dans la mission évangélisatrice
de l'Eglise et son rapport avec l' «annonce» de l'Évangile 2.
Dans le chapitre v de Redemptoris missio sont traitées les
diverses « voies de la mission» qui, est-il tout d'abord noté,
est« une réalité globale mais complexe qui s'accomplit de différentes
manières» ( 41 ). L'ordre dans lequel les différentes
modalités de la mission sont mentionnées et expliquées a ici
son importance. La première forme d'évangélisation est le
témoignage: souvent, c'est la seule façon possible d'accomplir
la mission (42). En deuxième lieu vient l'annonce de JésusChrist
qui « a, en permanence, la priorité dans la mission » ;
toutes les formes d'activité missionnaire y tendent (44). Dans
la réalité complexe de la mission, la première annonce a un
rôle central et irremplaçable (44). La priorité de l'annonce par
rapport aux autres activités ne doit toutefois pas être entendue
comme étant de l'ordre temporel, mais comme logique et
idéale. La façon concrète de procéder dépendra des circonstances
; plus loin, Redemptoris missio note que le dialogue
peut parfois être « 1 'unique manière de rendre un témoignage
sincère au Christ» (57). La façon dont toutes les autres formes
d'activité missionnaire « tendent à [la] proclamation» (44) ne
fait l'objet d'aucune explication ultérieure.
En troisième lieu sont mentionnés la « conversion chrétienne
» vers laquelle est ordonf!ée l'annonce, et le baptême
qui introduit les croyants dans l'Eglise (46). Redemptoris mis-sio insiste sur le fait qu'on ne peut se dispenser de l'annonce
sous le faux prétexte du prosélytisme, car toute personne a le
droit d'entendre la Bonne Nouvelle (46); on ne peut pas non
plus séparer la ,conversion au Christ du baptême, car le Christ
a voulu que l'Eglise fût le «lieu» où les personnes « peuvent
effectivement le rencontrer» ( 4 7). Ainsi, la fondation de nouvelles
communautés et le développement de nouvelles Églises
particulières, qui sont liés à la conversion et au baptême, sont
mentionnés en quatrième place; il s'agit d' « un objectif principal
et déterminant de l' activité missionnaire» ( 48).
Jusqu'à présent, les éléments suivants ont été mentionnés en
une succession organique : témoignage, annonce, conversion
et sa sacrament~lisation dans le baptême, et établissement et
croissance de l'Eglise. Redemptoris missio traite ensuite rapidement
des « communautés ecclésiales de base» en tant que
force d'évangélisation et d'extension missionnaire (51), et de
l'inculturation du message évangélique dans les diverses
cultures des peuples (52-54). Ce n'est qu'après celles-ci - et
avant de parler de développement et de promotion humaine
(58-59) qui, sera-t-il dit, ont un « lien étroit» avec l'annonce
de l'Évangile (59) - que Redemptoris missio traite explicitement
du dialogue interreligieux (55-57). Les points principaux
à ce sujet peuvent être groupés sous trois rubriques: 1, dialogue
et évangélisation; 2, dialogue et proclamation; 3, le but du
dialogue.
1. Le dialogue interreligieux, affirme \textit{Redemptoris missio},
\begin{quote}
  « fait partie de la mission évangélisatrice- de l'Église-» (55);
« il en est une expression» (ibid.) ; il est, en outre, « un chemin
vers le Royaume» (57).  
\end{quote}
 Ces affirmations semblent impliquer
une large notion de l'évangélisation. Le dialogue
interreligieux et l'annonce apparaissent comme « deux éléments
» ou deux expressions distinctes de l'évangélisation.
Entre les deux, il n'y a aucune opposition, mais à la fois lien
étroit et distinction. Cela est expliqué de la manière suivante :

\begin{quote}
    l faut que ces deux éléments demeurent intimement liés et en même temps distincts, et c'est pourquoi on ne doit ni les
confondre, ni les exploiter [\emph{Nec immodice instromentorum instar adhibenda}] ni les tenir pour équivalents comme s'ils
étaient interchangeables» (55).
\end{quote}
Que le dialogue ne puisse être «exploité» signifie, catégoriquement,
qu'il ne peut être réduit à un moyen en vue de la
proclamation ; d'autre part, il est dit que « le dialogue ne
dispense pas de l'évangélisation» - là, peut-on observer en
passant, Redemptoris missio semble retomber dans une vision
étroite de l'évangélisation qui, à présent, est identifiée implicitement
avec l'annonce.
2. Malgré le lien intime entre dialogue et annonce (55),
celle-ci garde « en p~rmanence la priorité» dans la mission
évangélisatrice de l'Eglise (voir 44). Le dialogue n'en dispense
pas (55). Sur ce point, Redemptoris missio rappelle une
lettre du pape aux évêques d'Asie: 
\begin{quote}
    « Le fait que les adeptes
d'autres religions puissent recevoir la grâce de Dieu et être
sauvés par le Christ en dehors des moyens ordinaires qu'il a
institués n'annule [ ... ] pas l'appel à la foi et au baptême que
Dieu veut pour tous les peuples» (55 1). La raison en est que
« l'Eglise est la voie ordinaire du salut et qu'elle seule possède
la plénitude des moyens du salut» (55).
\end{quote}

3. Le dialogue est compris comme « méthode et comme
moyen en vue .. d'Uile conriaissan·cë et d'un enrich1ssëment rec1-
proques » (55). Dieu 
\begin{quote}
    « ne manque pas [ ... ] de manifester -sa
présence de beaucoup de manières, non seulement aux individus
mais encore aux peuples, par leurs richesses spirituelles
dont les religions sont une expression principale et essentielle
» (55).
\end{quote}
 Dans le dialogue interreligieux, l'Eglise cherche
à découvrir les \textit{ semences du Verbe} et les « rayons de la
Vérité» qui se trouvent dans les personnes et dans les traditions
religieuses de l'humanité (56). Elle est incitée « à découvrir
et à reconnaître les signes de la présence du Christ et de
l'action de l'Esprit, et aussi à approfondir son identité et à
témoigner de l'intégrité de la révélation dont elle est dépositaire
pour le bien de tous» (56). Le dialogue, enfin, « tend à la
purification et à la conversion intérieure» (56). Il s'agit ici non
·pas ·de la conversion des autres au christianisme, mais de la
conversion à Dieu des deux partenaires du dialogue, le chrétien
et l'autre.
Passant à Dialogue et annonce2, la matière directement en
cause peut être regroupée sous quatre chefs : l, le fondement 
théologique du dialogue ; 2, le dialogue dans la mission évangélisatrice
de l'Église; 3, le but du dialogue; 4, dialogue et
annonce.
1. Le document rappelle le « mystère de l'unité», fondé sur
l'origine et la destinée communes du genre humain en Dieu,
sur le salut universel en Jésus-Christ et la présence active de
l'Esprit en tous les hommes (28), dont Jean-Paul II avait parlé
dans son discours à la curie romaine en 1986. « II découle, de
ce mystère d'unité, que tous ceux et toutes celles qui sont sauvés
participent, bien que différemment, au même mystère de
salut en Jésus-Christ par son Esprit. Les chrétiens en sont bien
conscients, grâce à leur foi, tandis que les autres demeurent
inconscients du fait que Jésus-Christ est la source de leur salut.
Le mystère du salut les atteint, néanmoins, par des voies
connues de Dieu, grâce à l'action invisible de l'Esprit du
Christ» (29). Suit un texte important, déjà cité 1, dans lequel
Dialogue et annonce assigne un rôle positif aux traditions
elles-mêmes dans le salut de leurs membres: c'est «dans la
pratique sincère de ce qui est bon dans leurs traditions religieuses
» qu'ils répondent positivement à l'offre que Dieu leur
fait de la grâce (29).
Les membres des autres religions, donc, ne sont pas sauvés
par le Christ en dépit ou en dehors de leur propre tradition,
mais dans celle-ci et, en quelque mystérieuse manière, par elle.
Cela ne signifie toutefois pas que tout, dans les autres traditions,
peut conduire au salut de leurs membres. En fait, identifier
en elles « ces éléments de grâce capables de soutenir la
réponse positive de leurs membres à l'appel de Dieu» est une
tâche difficile, qui exige du discernement (30). Tout n'est pas,
en elles, le résultat de la grâce, et elles ne contiennent pas non
plus que des valeurs positives; car le péché a été à l'oeuvre
dans le monde, et les traditions « sont aussi le reflet des limitations
de l'esprit humain, qui est parfois enclin à choisir le
mal» (31).
2. Pour montrer la place du dialogue interreligieux dans la
mission de l'Église, Dialogue et annonce rappelle d'abord la
doctrine de Vatican II sur l'Église comme sacrement universel,
c'est-à-dire comme signe et instrument du salut (LG 1, 48; DA 33)~ Concernant la relation « mystérieuse et complexe»
entre l'Eglise et le Royaume, le document cite Jean-Paul II
déclarant que « le Royaume est inséparable de l'Église car tous
deux sont inséparables de la personne et de l' oeuvre de Jésus
lui-même» (34). Les membres des autres traditions religieuses
sont orientés (ordinantur, voir LG 16) vers l'Église, comme
vers le sacrement dans lequel le Royaume est présent « en
mystère» ; mais ils « partagent déjà en quelque sorte la réalité
signifiée par le Royaume» (35).
En fait, ajoute Dialogue et annonce dans un autre texte
important déjà cité 1, le Royaume de Dieu est, dans l'histoire,
une réalité plus large que l'Église, même si la réalité historique
du Royaume en dehors de l'Église « trouvera son achève,
ment » en elle et dans le monde futur (35). D'autre part,
l'Église sur terre est toujours en pèlerinage (36); par conséquent,
malgré sa sainteté « par institution divine», elle a toujours
besoin de renouvellement et de réforme, non seulement
dans ses membres mais comme institution (ibid.). Quant à la
vérité divine, alors que la plénitude de la révélation de Dieu
en Jésus-Christ lui a été confiée (DV 2), l'Église, toutefois,
comme le fait remarquer Vatican II (voir DV 8), « tend
constamment vers la plénitude de la vérité, jusqu'à ce que
soient accomplies en elle les paroles de Dieu» (37).
Cette situation de l'Eglise permet de voir plus facilement
« pourquoi et dans quel sens le dialogue interreligieux est un
élément intégrant de la mission évangélisatrice de l'Église»
(38). La raison fondamentale de l'engagement de l'Église à
dialoguer « n'est pas simplement de nature anthropologique :
elle est aussi théologique» (ibid.). Comme l'enseignent Paul VI
et Jean-Paul Il, l'Église doit entrer en un dialogue de salut
avec tous les hommes, de même que Dieu a entrepris un dialogue
séculaire de salut avec le genre humain, qui continue
même aujourd'hui (ibid.). « Dans ce dialogue de salut, les chrétiens
et les autres sont tous appelés à collaborer avec l'Esprit
du Seigneur ressuscité, Esprit qui est universellement présent
et agissant » ( 40).
3. En ce qui concerne le but du dialogue interreligieux, le
document a ceci à dire : le dialogue interreligieux ne vise pas
simplement la compréhension mutuelle et les relations amicales; en lui, chrétiens et autres sont invités à « approfondir J
les dimensions religieuses de leur engagement et à répondre, /
avec une sincérité croissante, à l'appel personnel de Dieu et au !
don gratuit qu'il fait de lui-même, don qui passe toujours, /
comme notre foi nous le dit, par la médiation de Jésus-Christ
et l' oeuvre de son Esprit» (ibid.). Ainsi, le but du dialogue
interreligieux est « une conversion plus profonde de tous à
Dieu» ; comme tel, il possède « sa propre valeur» ( 41 ). « Le
dialogue sincère implique, d'une part, que l'on accepte l'existence
de différences et même de contradictions, et, d'autre
part, que l'on respecte la libre décision que les hommes prennent
en accord avec les impératifs de leur conscience» (ibid.).
4. Sur le rapport entre dialogue interreligieux et annonce, le
document contient cette importante affirmation : « Le dialogue
interreligieux et l'annonce, sans être sur le même plan, sont
tous les deux des éléments authentiques de la mission évangélisatrice.
Tous les deux sont légitimes et nécessaires. Ils sont
intimement liés mais non interchangeables [ ... ]. Les deux
domaines, certes, demeurent distincts, mais [ ... ] la même et
unique Église locale, la même et unique personne peuvent être
diversement engagées dans l'un et l'autre» (77).
Concrètement et effectivement, la manière de remplir la
mission de l'Église dépend des circonstances particulières;
elle exige de la sensibilité à l'égard de la situation, une attention
aux « signes des temps par lesquels l'Esprit de Dieu parle»,
et du discernement (78). Mais en toute situation, la mission de
l'Église s'étend à toutes les personnes. En effet, l'Église en
dialogue peut être considérée comme ayant « un rôle prophétique
par rapport aux religions » elles-mêmes, du fait que son
témoignage rendu à l'Evangile leur pose des questions; mots
elle peut à son tour se trouver interpellée. Ainsi « les membres
de l'Église -êtlês adeptés des autres religions se retrouvent
comme compagnons sur le chemin commun que ~oute l'humanité
est appelée à parcourir» (79). En outre, « l'Eglise encourage
et stimule le dialogue interreligieux, non seulement entre
elle-même et d'autres traditions religieuses, mais aussi entre
ces traditions religieuses elles-mêmes» (80). C' ëst une des 
façons dont elle remplit son rôle comme « sacrement, c'est-
à-dire à la fois le signe et le moyen de l'union intime avec
Dieu et de l'unité de tout le genre humain» (LG 1). Le dialogue
interreligieux fait donc réellement partie du dialogue de
salut initié par Dieu (80).
L'annonce, d'autre part, « tend à conduire les humains à une
connaissance explicite de ce que Dieu a fait pour tous,
h~mmes et femmes, en Jésus-Christ en devenant membres de
l'Eglise» (81). L'attention aux mouvements de !'Esprit et le
discernement sont nécessaires quant au moment où l'Eglise est
appelée à remplir cette tâche. La proclamation doit en outre
être faite de manière progressive, suivant le rythme de la croissance
de ceux qui la reçoivent dans l'obéissance de la foi
(ibid.).
A~nonce et dialogue sont« deux façons d'accomplir l'unique
m1ss1on de l'Eglise» (82). La manière concrète de remplir la
mission dépendra des circonstances. Mais il faut se rappeler
que « le dialogue [ ... ) ne constitue pas toute la mission de
l'Eglise, qu'il ne peut pas simplement remplacer l'annonce,
mais reste orienté vers l'annonce; c'est en celle-ci en effet
que le processus dynamique de la mission évangélisatrice de
l'Eglise atteint son sommet et sa plénitude» (ibid.).
Avec les diverses étapes du dialogue, en fait, « les interlo~
uteurs éprouvent un grand besoin tant d'informer que d'être
mformés, de donner comme de recevoir des explications, et de
se poser réciproquement des questions. Les chrétiens engagés
dans le dialogue ont alors le devoir de répondre aux attentes
de leurs partenaires concernant le contenu de la foi chrétienne
et de porter témoignage de cette foi lorsqu'ils y sont appelés,
de donner raison de l'espérance qui est en eux (voir 1 P 3,
15) » (82). Dans cette situation dialogique, ils auront l'espoir et
le désir de partager avec les autres leur joie de connaître et de
suivre Jésus-Christ, Seigneur et Sauveur - un désir qui, dans
la mesure où ils ont un profond amour pour le Seigneur Jésus,
sera motivé non pas simplement par l'obéissance au commandement
du Seigneur, mais par leur amour pour lui (83). Mais
il faut trouver normal que les adeptes des autres religions
soient animés d'un désir similaire de-partager leur propre foi;
« tout dialogue implique la réciprocité et vise à bannir la peur
et l'agressivité» (ibid.). En tout cela, les chrétiens doivent
« être préparés à aller là où l'Esprit les mène, de par la providence
et les desseins de Dieu». « C'est !'Esprit qui guide la
mission évangélisatrice de l'Église» ; à nous, il appartient
d'être attentifs à ses suggestions. Mais, « que l'annonce soit
possible ou non, l'Église poursuit sa mission dans le plein respect
de la liberté, par le dialogue interreligieux, ainsi que par
le témoignage et le partage des valeurs évangéliques» (84).
Si l'on veut établir une rapide comparaison entre Redemptoris
missio et Dialogue et annonce au sujet du dialogue interreligieux
et de son rapport avec l'annonce, les principales
différences qui émergent, à côté des points de contact, sont les
suivantes.
1. Une première différence significative consiste dans la
diverse accentuation donnée au dialogue interreligieux dans
les deux documents. En Redemptoris missio, dialogue (et promotion
humaine) sont mentionnés dans le chapitre « Voies de
la mission » ( ou « formes d'évangélisation » ), après des points
tels que les communautés ecclésiales de base et l' inculturation.
L'accent reste mis de façon prédominante sur l'annonce, qui
est considérée comme l'objet de l'activité missionnaire proprement
dite, c'est-à-dire de la mission aux nations (34), et qui
a,« en permanence, la priorité» (44). En comparaison, Dialogue
et annonce insiste davantage sur le dialogue interreligieux. Là
où l'intention principale de Redemptoris missio est de réaffirmer
avec force l'actualité et l'urgence de l'annonce, le souci
premier de Dialogue et annonce est que la signification du
dialogue ne soit pas sous-estimée.
2. En outre, la perspective de Redemptoris missio apparaît
plus ecclésiocentrique en comparaison avec celle de Dialogue
et annonce, qui est plus christocentrique et régnocentrique.
D'après Redemptoris missio, « l'activité missionnaire spécifique
[ ... ) a [ ... ) pour caract~re propre d'être une action d'annonce
du Christ et de son Evangile, d'édification de l'Eglise
locale et de promotion des valeurs du Royaume » (34) ; « la
mission ad gentes a comme obj~ctif de fonder des communautés
chrétiennes, d'amener des Eglises à leur pleine maturité»
(48). Ainsi, dans une perspective çlairement ecclésiocentrique,
l'accent est sur l'édification de l'Eglise. Par contraste, la perspective
de Dialogue et annonce est plus christocentrique et
régnocentrique : se reliant à son prédécesseur de 1984 (13 ), le
document définit la mission simplement en termes d'évangélisation,
et la réalité complexe de l'évangélisation comme
comprenant, entre autres éléments, le dialogue interreligieux
et l'annonce (voir 8, 82 mentionnés ci-dessus).
3. Venons-en au rapport entre dialogue et proclamation. On
a vu que tant Redemptoris missio que Dialogue et annonce
affirment clairement que, dans la mission évangélisatrice de
l'Église, ils constituent deux éléments distincts, à ne pas
confondre ni séparer (voir RM 55 et DA 77, cités ci-dessus).
Redemptoris missio dit qu'ils ne peuvent être «exploités», ce
qui signifie que 1~ dialogue ne peut être réduit à un « moyen »
en vue de l'annonce. Dialogue et annonce affirme de même
que le dialogue « possède sa propre valeur» ( 41 ). Quant à leur
corrélation dans la mission de l'Église, Redemptoris missio
affirme la « priorité en permanence» de l'annonce, en vertu de
laquelle « toutes les formes de l'activité missionnaire tendent
à cette proclamation » ( 44 ). Comme on l'a noté plus haut, cette
priorité ne doit pas s'entendre comme temporelle, comme si la
proclamation devait, en toute circonstance, précéder les autres
formes d'évangélisation; car il sera dit par la suite que le dialogue
interreligieux est souvent « l'unique manière de rendre
un témoignage sincère au Christ et un service généreux à
l'homme» (57). La « priorité en permanence» est d'un ordre
d'importance logique et idéal: la proclamation a un « rôle central
et irremplaçable» (44). Dialogue et annonce, pour sa part,
affirme sans ambiguïté que« le dialogue interreligieux et l'annonce
[ne sont pas] sur le même plan» (77); mais leur rapport
y reçoit une explication plus théologique, à savoir que le dialogue
« reste orienté vers l'annonce ; c'est en celle-ci en effet
que le processus dynamique de la mission évangélisatrice de
l'Église atteint son sommet et sa plénitude» (82).
4. Enfin, on peut se demander si et dans quelle mesure
( Redemptoris missio et Dialogue et annonce sont allés au-delà
/ de ce qui avait été affirmé dans le passé par l'autorité doctrinale
centrale en ce qui concerne les sujets examinés. On peut
dire ceci : Vatican II a recommandé le dialogue avec les autres
' traditions religieuses (NA 2; GS 92), mais sans dire qu'!l était
une partie intégrante de la mission évangélisatrice de l'Eglise.
C'est ce qu' affirment clairement aussi bien Redemptoris missio
que Dialogue et annonce, suivant les traces de Dialogue et
mission. En outre, malgré une certaine ambiguïté dans la terminologie
de Redemptoris missio, les deux documents développent
un concept large de l'évangélisation, qu'on ne trouvait
pas encore dans Vatican II ; tous deux affirment, bien que de
façon différente, que le dialogue ne peut être réduit à un
«moyen» en vue de l'annonce, mais qu'il a valeur en luimême.
De cette façon, et en d'autres manières, Redemptoris
missio et Dialogue et annonce, avec leurs accentuations et
nuances distinctes, constituent un pas en avant dans la doctrine
de l'Église sur l'évangélisation, le dialogue et l'annonce.
\end{quote}

\section{« La mission \textit{est} le dialogue » ?}