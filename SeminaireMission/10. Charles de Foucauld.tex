\chapter{Charles de Foucauld et l'évangélisation}

\section{Synthèse}

\subsection{Charles de Foucauld : Biographie et éléments de contexte}

\paragraph{Une carrière d'officier} Né en 1858, Charles de Foucauld commence une carrière d'officier. Rétif à la hiérarchie et noceur invétéré, il quitte l'armée. Il décide alors d'explorer le Maroc, exploration qui lui vaut un succès d'estime dans la société savante et la publication de son livre \textit{Reconnaissance au Maroc}.

\paragraph{Conversion en 1886 et vie cachée du Christ} Il se convertit en Octobre 1886 dans l'Eglise Saint Augustin, en se convertissant à l'Abbé Huvelin qui devient son directeur spirituel. Là, il commence à suivre le Christ dans sa vie cachée, à l'abbaye Notre-Dame-des-Neiges, puis à Nazareth puis à Béni-Abbès.

\paragraph{Tamanrasset} En 1905, il commence une vie d'ermite à Tamanrasset puis à l'Assekrem. Il meurt assassiné en 1916.

\subsection{Spiritualité et mission}

\paragraph{Documents étudiés} En préalable, un mot sur les trois documents étudiés : 
\begin{itemize}
    \item deux articles de Pierre Sourisseau, archiviste de la cause de canonisation de Charles de Foucauld
    \item une anthologie de textes de Charles de Foucauld compilée par Michel Lafon par thème. A ce propos, une attention méthodologique est de mise quand on a ce type de sources, avec la recommandation d'aller autant que possible aux textes originaux directement pour situer les passages dans leur contexte.
\end{itemize}



\paragraph{Une évangélisation liée à une spiritualité profonde} Chez Charles de Foucauld, l'urgence, le soucis de l'évangélisation s'enracine dans une spiritualité forte, celle d'imiter Jésus dans sa vie caché Nazareth, vie humble.  Ce n'est pas uniquement la Croix qui sauve mais l'incarnation du Christ.

\paragraph{Une ouverture à l'universel} Cette imitation de Jésus, à travers un chemin concret est le moyen de devenir \textit{frère universel}. Il s'agit d'une christologie ascendante, partant du Jésus \textit{historique}, enracinée dans l'Evangile, du bas vers le haut : Dieu agit dans les humbles, et dans la rencontre des autres,  le royaume de Dieu se construit.  

\paragraph{Le renouveau de l'Amour à la fin du XIX} Nous notons par rapport aux textes précédents, y compris celui récent du Cardinal Lavigerie, un changement de tonalité : l'amour est fortement présent chez Charles de Foucauld, comme chez Thérèse de l'Enfant Jésus, marque d'un contexte culturel nouveau.

\paragraph{Conséquence en terme de mission} Par l'insistance sur la grâce et non les oeuvres, Charles de Foucauld comprend l'efficacité de la mission non pas en terme de chiffres de convertis. Elle devient alors une mission conversationnelle, respectant la foi de l'autre, juif ou musulman. Charles de Foucauld cite comme exemple missionnaire St François d'Assise et sa rencontre avec le Sultan et la \textit{Regula Non Bullata}. Cette mission qui est d'abord présence à côté de frères humains,  découle de sa mystique. Elle fut peut être renforcée par le texte \textit{Ius Commissionis}, qui établit un monopole missionnaire par région : Charles de Foucauld aura des liens amicaux avec les Pères Blancs mais il expérimente une mission basée sur la présence et la rencontre et non une mission active ”prosélyte” réservée aux Pères Blancs.

\paragraph{Une activité en relation} L'image d'Epinal de l'Ermite de Tamanrasset n'est pas tout à fait juste, tant il est en lien, à la fois avec les populations locales (Tamanrasset est un carrefour et Charles de Foucauld se plaint d'un trop plein d'activités), mais aussi avec un vaste réseau via sa correspondance active. Il crée aussi  les « Frères et Sœurs du Sacré-Cœur de Jésus ».



\subsection{Fécondité}
De façon étonnante, l'Evangélisation de Charles de Foucauld a été à court terme un échec, sans conversion de son vicant. Néanmoins, sa fécondité est toujours notable, dans les pratiques missionnaires (nous citons Monchanin en Inde et la Mission de France) 

Et de façon plus récente, le pape François le cite longtement dans la conclusion de l'encyclique \textit{Fratelli Tutti} : 

\begin{quote}
    286. Dans ce cadre de réflexion sur la fraternité universelle, je me
suis particulièrement senti stimulé par saint François d’Assise, et
également par d’autres frères qui ne sont pas catholiques : Martin
Luther King, Desmond Tutu, Mahatma Mohandas Gandhi et beau-
coup d’autres encore. Mais je voudrais terminer en rappelant une
autre personne à la foi profonde qui, grâce à son expérience intense
de Dieu, a fait un cheminement de transformation jusqu’à se sen-
tir le frère de tous les hommes et femmes. Il s’agit du bienheureux
Charles de Foucauld.
287. Il a orienté le désir du don total de sa personne à Dieu vers
l’identification avec les derniers, les abandonnés, au fond du désert
africain. Il exprimait dans ce contexte son aspiration de sentir tout
être humain comme un frère ou une sœur,[286] et il demandait à
un ami : « Priez Dieu pour que je sois vraiment le frère de toutes
les âmes [⋯] ».

[287] Il voulait en définitive être « le frère universel
».

[288] Mais c’est seulement en s’identifiant avec les derniers qu’il
est parvenu à devenir le frère de tous. Que Dieu inspire ce rêve à
chacun d’entre nous. Amen !
\end{quote}
\mn{le 22/11/22 Jacques}

\section{Charles de Foucauld}
\mn{biographie de René Bazin, 1921. }
Né à Strasbourg, 1858. 
Eloigné de la foi. Officier. Au Maroc.

\paragraph{Oct 1886} Abbé Huvenin. 
Suivre sa vie dans sa vie de Nazareth
Trappe 7 ans, syrie, puis Nazareth.
46 ans, prêtre, en Algérie puis Tamanrasset

\paragraph{frère universel} pour aller le plus loin. 
\begin{quote}
    je voudrais être assez bon pour qu'on dise, si c'est lui un \textit{serviteur}, quel est donc le maître
\end{quote}

\paragraph{Une vocation à être entouré}

\paragraph{Une mission profondément mystique} Les textes montrent sa spiritualité

% ----------------------------------------------
\section{Rencontrer pour annoncer - Conseils missionaires de Charles de Foucauld}



\paragraph{Pierre Sourisseau} pendant près de 40 ans, a travaillé à la canonisation de Ch. de Foucauld. Biographie de Ch. de Foucauld, 2016. \mn{Les lumières d'un Phare, Charles de Foucauld}

\paragraph{Importance de la mission} Pas seulement une question de méthode, imiter Jésus dans toutes ses dimensions. 

\paragraph{s'adapter à chacun}
\paragraph{là où l'Evangile n'a pas été annoncé} 
Eloignés

\section{Prier avec Ch. de Foucauld}

\paragraph{Michel Lafond} disciple de Péliguere. Prière et méditation. 

\paragraph{Anthologie} en terme de méthode, il fat une synthèse par thème. C'est péché. Ch. de Foucault a pu écrire avant sa conversion. Donc, importance de citer plus précisément. Serait intéressant de situer chaque texte plus précisément dans sa vie.

\paragraph{Marie} Visitation, Nazareth. Enfouissement. Présence "tabernacle".

\paragraph{Frère universel} nous sommes tous fils du très haut, et le point est que nous nous aimions tous : Mt 25. 

\paragraph{Imitation de Jésus Christ} Imitation et donc mission. 


\paragraph{noblesse du travail manuel}

\section{Charles de Foucauld devant l'évangélisation}


\paragraph{Conclusion}
Grande machine : quelque chose de nouveau qui apparaît. 

Etre là par rapport à sa vision 

\paragraph{pas de conversion de son vivant}

\paragraph{cite St François d'Assise et St François de Sales} mais sa source, c'est la vie de Jésus et sa vie cachée. 

p 88. depuis 5 ans, je ne pouvais faire mieux que la vierge Marie et sa vie cachée.

Parallèle avec St François et sa rencontre au Sultan. Regula non bullata. cf \textit{non en le préchant de bouche, mais en le priant.} \textit{ceux qui n'ont pas reçu mission}

\paragraph{Prier l'Evangile par notre vie}
Missionnaire : vivre intérieurement la vie cachée de Nazareth. 

\paragraph{importance de Ius Commissionis } Missionnaire en abandonnant la mission. passivité. 

\paragraph{curiosité des autres} \textit{conversationnel}. La mission découle indépendemment de lui. Grande modernité : Paul VI. 
Matteo Ricci : on s'intéresse aux puissants, Ici les pauvres.
Ils recueillent une culture orale.  Mt 25 : tout petit. 

\paragraph{Mission de France} \textit{France, pays de mission ?}. Mission, ce n'est pas aller qu'à l'étranger et des classes sociales dechristianisé, comme les ouvriers. 

\paragraph{Dimension d'exemplarité} Il s'est quand même fait connaître. Quelle stratégie ?

\paragraph{Dimension eremetique}

\paragraph{des ressources d'Evangile}

\paragraph{on passe d'une logique de pouvoir à une logique de responsabilité}

\paragraph{Deuxième conversion} les femmes touaregs ont pris le lait pour lui sauver la vie. A basculé vers frère universel. Avant cette conversion, il avait envie d'annoncer l'évangile. le fait de recevoir.

\paragraph{C'est bazin, qui a dit qui a donné son image d'éremetisme} En fait, a toujours été en lien (écriture). Pas une vie d'ermite. 

\paragraph{véritable mystique et pourtant profondément dans le monde}

\paragraph{missionnaire} a la fin de sa vie, "je ne veux pas être appelé missionnaire". la mission a échappé à sa vie : si le grain de blé ne meurt pas... il a une méditation là dessus.
C'est Dieu qui est missionnaire, ce n'est pas l'église.

\paragraph{comme sainte thérese, on pense la mission dans une théologie de la grâce et non des oeuvre}

\paragraph{Eremetique} Il voulait faire une communauté, il a même invité Louis Massignon (qui n'a pas voulu). 

\paragraph{Tamarasset} Carrefour. Il a tout le temps du monde, et fatigué, il s'endormait devant le saint sacrement.


\paragraph{Imitation de Jesus} et non le Christ. Cette humanité de Jésus.  Discours de Paul VI à la fin du Concile. A t on parlé trop de l'homme ? "en parlant de l'homme, on parle de l'image du Chirst, en voyant les pauvres, on voit le christ, en voyant le Christ, on voit le Père".

\paragraph{On insistait que la croix sauve} mais il dit que c'est l'incarnation pour sauver. L'incarnation : sauver. "Il est Dieu sauve dans ces 30 ans de vie cachée". Il prend en compte toute sa vie. le salut vient par le Dieu qui nous rejoint. 
Habité par le salut toute sa vie. La grande question. 


\paragraph{lien avec les pères blancs} A creuser mais il semblerait que cela se passe bien. Serait bien de creuser le sujet. 


\paragraph{Respect de la foi des musulmans et des juifs} Certains mots sont datés (idolatres) mais ne l'emploie pas pour les musulmans et les juifs.


\paragraph{Ambiguité sur la pensée de Charles de Foucauld sur le prosélytisme} son expérience et sa mission est "silencieuse" mais pas de mot sur le prosélytisme.
"un prosélytisme en sourdine", sans le dire : prière, pénitence, exemple, bonté et amitié. 

\paragraph{la prière de Charles de Foucauld : l'abandon} intéressant car soumis à Dieu. La médiation ecclésiale ne joue pas trop. Assez libre. 

\paragraph{la conversion ne change pas son caractère} qui reste radical et peu respectueux de la hierarchie.

\paragraph{Fratelli Tutti} \begin{quote} \mn{[286] Cf. Charles de Foucauld, Méditations sur le Notre Père (23 janvier 1897).

[287] Id., Lettre à Henry de Castries (29 novembre 1901).

[288] Id., Lettre à Madame de Bondy (7 janvier 1902). C’est ainsi que saint Paul VI aussi le désignait, en louant son engagement : Populorum progressio (26 mars 1967), n. 12 : AAS 59 (1967), p. 263.}
    286. Dans ce cadre de réflexion sur la fraternité universelle, je me suis particulièrement senti stimulé par saint François d’Assise, et également par d’autres frères qui ne sont pas catholiques : Martin Luther King, Desmond Tutu, Mahatma Mohandas Gandhi et beaucoup d’autres encore. Mais je voudrais terminer en rappelant une autre personne à la foi profonde qui, grâce à son expérience intense de Dieu, a fait un cheminement de transformation jusqu’à se sentir le frère de tous les hommes et femmes. Il s’agit du bienheureux Charles de Foucauld.

287. Il a orienté le désir du don total de sa personne à Dieu vers l’identification avec les derniers, les abandonnés, au fond du désert africain. Il exprimait dans ce contexte son aspiration de sentir tout être humain comme un frère ou une sœur,[286] et il demandait à un ami : « Priez Dieu pour que je sois vraiment le frère de toutes les âmes […] ».[287] Il voulait en définitive être « le frère universel ».[288] Mais c’est seulement en s’identifiant avec les derniers qu’il est parvenu à devenir le frère de tous. Que Dieu inspire ce rêve à chacun d’entre nous. Amen !
\end{quote}


\paragraph{On devient universel à travers un chemin concret} On était dans une théologie du surplomb vers les hommes (contrefort de l'enfer, XVIII). Ici, on a une théologie ascendante du bas vers le haut : Dieu agit dans les humbles, et par sa fraternité, le Règne, dans la rencontre des autres, et c'est comme cela que le royaume de Dieu se construit. Itinéraire terrestre de Jésus.

\paragraph{une théologie s'ouvrant à l'amour fin du XIX} fin du jansénisme dans sa vision moraliste, Dieu juge plus qu'amour (Madame Grandet, 1830, rejet du théatre). 
Ste Thérèse : ce n'est pas le missionnaire qui convertit. Le missionnaire ouvre le coeur. 
Thème de l'amour à la fin du XIX. \mn{cela vaudrait le coup de voir l'origine de l'amour}
\begin{quote}
    L'amour consiste, non à sentir qu'on aime mais à vouloir aimer \sn{A Massignon, 1916}
\end{quote}


\paragraph{Abbé Huvenin} dompte sa fougue. 
\begin{quote}
    il fait de la religion un amour (Huvenin de Foucauld)
\end{quote}

\paragraph{quelle fécondité ?} Voir  \href{https://fr.wikipedia.org/wiki/Jules_Monchanin}{Montchanin} en Inde.



- Jacques : Ius Commissionis.  Missionnaire quand on ne peut pas être missionnaire. Cadre pas possible. Etre tel que je sais

- les sources

\section{synthèse}

\paragraph{Charles de Foucauld}
vie rapide (Abbé huvenin dont nous avons souligné l'importance


\paragraph{les textes étudiés et leurs auteurs}

\section{quelques éléments de discussion}

\paragraph{Imiter Jésus dans sa vie caché} Nazareth, kenose, importance de l'Evangile



\paragraph{Amour} contexte de la fin du XIX : changement de tonalité

\paragraph{conséquence en terme de mission} au dela de l'échec, une mission conversationnelle. Il ne lui appartient pas de faire une mission active "proselyte".

\paragraph{pas isolé, mission}

\paragraph{Une fécondité à long terme} comme le sang des martyrs, une telle approche non productiviste parait absurde. Mais fruit fratelli tutti?.
Montchanin en Inde
