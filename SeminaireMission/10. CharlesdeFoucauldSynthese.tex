\chapter{Charles de Foucauld et l'évangélisation}

\section{Synthèse}

\subsection{Charles de Foucauld : Biographie et éléments de contexte}

\paragraph{Une carrière d'officier} Né en 1858, Charles de Foucauld commence une carrière d'officier. Rétif à la hiérarchie et noceur invétéré, il quitte l'armée. Il décide alors d'explorer le Maroc, exploration qui lui vaut un succès d'estime dans la société savante et la publication de son livre \textit{Reconnaissance au Maroc}.

\paragraph{Conversion en 1886 et vie cachée du Christ} Il se convertit en Octobre 1886 dans l'Eglise Saint Augustin, en se confessant à l'Abbé Huvelin qui devient son directeur spirituel. Là, il commence à suivre le Christ dans sa vie cachée, à l'abbaye Notre-Dame-des-Neiges, puis à Nazareth puis à Béni-Abbès.

\paragraph{Tamanrasset} En 1905, il commence une vie d'ermite à Tamanrasset puis à l'Assekrem. Il meurt assassiné en 1916.

\subsection{Spiritualité et mission}

\paragraph{Documents étudiés} En préalable, un mot sur les trois documents étudiés : 
\begin{itemize}
    \item deux articles de Pierre Sourisseau, archiviste de la cause de canonisation de Charles de Foucauld
    \item une anthologie de textes de Charles de Foucauld compilée par Michel Lafon par thème. A ce propos, une attention méthodologique est de mise quand on a ce type de sources, avec la recommandation d'aller autant que possible aux textes originaux  pour situer les passages dans leur contexte.
\end{itemize}



\paragraph{Une évangélisation liée à une spiritualité profonde} Chez Charles de Foucauld, l'urgence, le soucis de l'évangélisation s'enracine dans une spiritualité forte, celle d'imiter Jésus dans sa vie cachée à Nazareth, dans sa vie humble.  Ce n'est pas uniquement\textit{ la Croix qui sauve} mais l'incarnation du Christ.

\paragraph{Une ouverture à l'universel} Cette imitation de Jésus, à travers un chemin concret est le moyen de devenir \textit{frère universel}. Il s'agit d'une christologie ascendante, partant du Jésus \textit{historique}, enracinée dans l'Evangile, christologie du bas vers le haut : Dieu agit dans les humbles. Dans la rencontre des autres,  le royaume de Dieu se construit.  

\paragraph{Le renouveau de l'Amour à la fin du XIX} Nous notons par rapport aux textes précédents, y compris celui récent du Cardinal Lavigerie, un changement de tonalité : l'amour est fortement présent chez Charles de Foucauld, comme chez Thérèse de l'Enfant Jésus, marque d'un contexte culturel nouveau.

\paragraph{Conséquence en terme de mission} Par l'insistance sur la grâce et non les oeuvres, Charles de Foucauld comprend l'efficacité de la mission non pas en terme de chiffres de convertis. Elle devient alors une mission conversationnelle, respectant la foi de l'autre, juif ou musulman. Charles de Foucauld cite comme exemple missionnaire St François d'Assise et sa rencontre avec le Sultan et la \textit{Regula Non Bullata}. Cette mission qui est d'abord présence à côté de frères humains,  découle de sa mystique. Elle fut peut être renforcée par le texte \textit{Ius Commissionis}, qui établit un monopole missionnaire par région : Charles de Foucauld aura des liens amicaux avec les Pères Blancs mais il expérimente une mission basée sur la présence et la rencontre et non une mission active ”prosélyte” réservée aux Pères Blancs.

\paragraph{Une activité en relation} L'image d'Epinal de l'Ermite de Tamanrasset n'est pas tout à fait juste, tant il est en lien, à la fois avec les populations locales (Tamanrasset est un carrefour et Charles de Foucauld se plaint d'un trop plein d'activités), mais aussi avec un vaste réseau via sa correspondance active. 



\subsection{Fécondité}
De façon étonnante, l'Evangélisation de Charles de Foucauld a été à court terme un échec, sans conversion de son vivant, ni oeuvre directement créé (les Frères et Sœurs du Sacré-Cœur de Jésus seront créés à partir de sa spiritualité mais pas directement par Charles de Foucauld). Néanmoins, au delà des frères et soeurs du Sacré-Coeur de Jésus, sa fécondité est  notable, dans les pratiques missionnaires (nous citons Monchanin en Inde et la Mission de France).

Et de façon plus récente, le pape François le cite longtement dans la conclusion de l'encyclique \textit{Fratelli Tutti} : 

\begin{quote}
    286. Dans ce cadre de réflexion sur la fraternité universelle, je me
suis particulièrement senti stimulé par saint François d’Assise, et
également par d’autres frères qui ne sont pas catholiques : Martin
Luther King, Desmond Tutu, Mahatma Mohandas Gandhi et beau-
coup d’autres encore. Mais je voudrais terminer en rappelant une
autre personne à la foi profonde qui, grâce à son expérience intense
de Dieu, a fait un cheminement de transformation jusqu’à se sen-
tir le frère de tous les hommes et femmes. Il s’agit du bienheureux
Charles de Foucauld.
287. Il a orienté le désir du don total de sa personne à Dieu vers
l’identification avec les derniers, les abandonnés, au fond du désert
africain. Il exprimait dans ce contexte son aspiration de sentir tout
être humain comme un frère ou une sœur,[286] et il demandait à
un ami : « Priez Dieu pour que je sois vraiment le frère de toutes
les âmes [⋯] ».

[287] Il voulait en définitive être « le frère universel
».

[288] Mais c’est seulement en s’identifiant avec les derniers qu’il
est parvenu à devenir le frère de tous. Que Dieu inspire ce rêve à
chacun d’entre nous. Amen !
\end{quote}