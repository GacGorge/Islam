\chapter{Neminem Profecto}

\mn{Instruction de la Propagande du 23 nombre 1845}


\section{Etude du texte}

\subsection{Contexte}
\paragraph{Période difficile} post révolutionnaire. Pendant les révolutions, entre 1789 et 1815, toute communication est interrompue entre l'Europe et les missions. Rome est totalement destabilisé. Le Pape s'est enfui et prisonnier en 1799. Meurt à valence. La \textit{propagande a fide} est supprimée. Toutes les congrégations disparaissent. 30 000 prêtres se sont exilés. Les anglais s'installent en Inde car les français n'y sont plus. 
\paragraph{Couper la tête des rois} fait que l'accueil des élites est moins net.

\paragraph{Les missionnaires qui restent décident de déléguer le maximum de pouvoir aux religieux locaux} et codification des Eglises. A partir des lettres, formalisation des instructions de Rome pour pouvoir le déléguer.



 
\paragraph{Pauline Jaricot} lien avec les missions

\paragraph{politique centralisatrise de Rome} Comme Pie VI est mort à Valence. Clément VI : on l'a vu car il est venu à Fontainebleau. Rome : centre des missions.

\paragraph{Gregoire XVI} Pape en 1831 : au moment des révolutions. Gregoire a horreur du mot révolution et liberté : "on va couper la tête des prêtres". On ne peut regarder le terme liberté sans penser à la revolution.
Désormais, la papauté n'est plus liée à la politique.
Différentes terres de mission.
En Europe, il est vu comme un pape rigide (qui condamne Lamnenais par rapport aux différentes libertés (de conscience..)) mais sur le plan de la mission, c'est le grand organisateur des activités missionnaires, avec une souplesse de regard sur l'activité missionnaire.
Va accepter les prêtres de basses castes. 

\paragraph{Jean Luquet et Marion Bresillac} Ont réflechi sur la formation des prêtres en Inde. Deux prêtres des MEP. \href{https://fr.wikipedia.org/wiki/Melchior_de_Marion-Br%C3%A9sillac}{Marion Bresillac} fonde les \href{https://www.missions-africaines.net/}{SMA} (Société missionnaire Africaine de Lyon) dans la spiritualité de \textit{Neminem Profecto}.

\subsection{Forme}

\paragraph{pas de citation de l'évangile}


\paragraph{Insistance sur les lettres des premiers évèques} multiplier les évèques locaux (pas forcément directement ce qu'on peut lire de St Ignace d'Antioche) ?

\paragraph{possible influence romantique} avec le retour à la source qui marque le romantisme.


\subsection{Civilisation Chrétienne}
\paragraph{formation des écoles des filles en Chine}

\paragraph{civilisation} importance de l'éducation aux \textit{lumières du Christ}. Cf Marion-Brésillac au Dahomey où il y avait encore des sacrifices humains. 

\paragraph{importance de la formation} contre Réforme : dans chaque diocèse, on a un séminaire pour former. on réplique cela dans les missions.


\paragraph{formation des jeunes filles}

\paragraph{formations des prêtres indigènes dès le plus jeune âge}


\subsection{préparation}
\paragraph{Lien avec les instructions de 1659}




\paragraph{des évèques et pas des prêtres}

\paragraph{des prêtres et évèques indigènes} \textit{former d'abord à la science et à la piété de jeunes indigènes qu'on devrait ensuite initier aux ordres sacrés. p 125} collèges nationaux et à Rome. 

\paragraph{la perte du texte de 1659} seule note de bas de page. Apporté par Luquet à la Propagande, qui l'avait perdu.

\subsection{Plan}


\paragraph{Ouverture}
Affirmation centrale de cette Instruction :
Pour la propagation et l'établissement de la religion catholique,
deux moyens principaux : des évêques et un clergé indigène.
\paragraph{I - Défense et illustration de cette affirmation
}
A) - Démonstration brève par le recours à 'Écriture et plus longue par les
Pères de l'Église.

B ) - Les pontifes Romains ont suivi la même voie, spécialement depuis les
trois derniers siècles : ils ont multiplié le plus possible le nombre des évêques.

C ) - En même temps les pontifes Romains ont poussé les évêques à la formation
d'un clergé indigène.

D ) - Rappel des principaux documents de Rome depuis le XVIIe siècle jusqu'à
Grégoire XVI, notamment les Instructions de 1659.

E ) - Des réponses diverses ont été faites face aux efforts du Siège apostolique
dans cette voie, mais la situation actuelle semble favorable à un établissement
plus solide de la foi et de la hiérarchie catholique, d'où la présente Instruction.

\paragraph{II - Huit points ordonnés et décrétés d'une manière expresse et absolue}


I - De nouveaux évêques et la formation par division de nouvelles Églises hiérarchiquement
constituées.

II - Devoir impérieux de former un clergé indigène et, pour cela, fonder des
séminaires.

III - Prendre les moyens pour que certains clercs indigènes deviennent capables
d'être revêtus du caractère épiscopal.

IV - Les prêtres indigènes ne doivent pas être employés comme de simples
auxiliaires et il doit y avoir égalité entre européens et indigènes.

V - Les catéchistes ne doivent pas être des substituts à un clergé indigène, à
la fondation duquel on ne travaillerait pas. Quand ils sont nécessaires, veiller
à la qualité de leur formation.

VI - Pour l'usage des cérémonies orientales, suivre ce que Benoît XIV a fixé
en 1755.

VII - Ne pas s'immiscer dans les affaires et dans les questions de la politique
séculière.

VIII - Ce que chaque chef de mission doit encore mettre en oeuvre : des associations
et institutions de prière et de pénitence ; les oeuvres de miséricorde ;
l'instruction civile et religieuse, sans oublier les jeunes filles ; agir sur la culture
locale ( travaux et arts ) ; trouver des ressources locales ; tenir souvent des assemblées
synodales.

\section{Réception}

\paragraph{premiers évèques : en Inde} Mgr Jérôme Fernandes, ancien évêque de Quilon dans l’Etat du Kerala, est décédé, le 26 février 1992, à l’âge de 93 ans, après 63 ans de sacerdoce et 54 ans d’épiscopat. Premier évêque indien de rite latin du Kerala, nommé en 1937, Mgr Jérôme, rompant avec la tradition occidentale, portait une soutane safran, la couleur distinctive des ascètes hindous. Dans les années 50, bien avant le second Concile du Vatican, il avait fait figure de précurseur en bâtissant églises et autels en style indien. La statue de la Vierge Marie vêtue d’un sari, qu’il fit sculpter pour le petit séminaire de Quilon, est connue dans toute l’Inde.\href{https://fr.wikipedia.org/wiki/Valerian_Gracias}{Valerian Gracias }

\paragraph{Synodes diocésains} A l'époque, uniquement les clercs. Il y a eu des synodes en Inde,...
A travers ces synodes, les syntheses sont envoyées à Rome et permettront une évolution de la vision de la mission. 


