\chapter{François-Xavier avant le Concile de Trente}

\section{AUX COMPAGNONS VIVANT A ROME
(EX.I, 160-177; S.11, 406-410)}

Cette longue lettre présente le grand intérêt de nous présenter
plusieurs aspects du choc spirituel et culturel que fut la rencontre
du monde indien et de la Chrétienté occidentale. En arrivant sur
la Côte du Dekkan, saint François Xavier fut frappé par l'ignorance
religieuse de ceux qu'un baptême sans doute hâtif avait rendus
Chrétiens : au Cap Comorin, les atavismes et les traditions hindoues résistent bien chez les néophytes. Comment parvenir à les
couper complètement et de leurs concitoyens hindous et de leur
passé ? Telle est bien la question que se pose /'Apôtre de l'Asie.
La solution par lui trouvée peut paraître « médiévale » ou au
contraire d'une inquiétante « modernité » selon l'idée que chacun
se fait de la marche de l'histoire humaine. Sur lequel de ces deux
versants doit-on placer une institution aussi importante que l'Inquisition
? Et ce que Xavier organise ? Il fait encadrer la population
par un réseau dense de militants, de responsables et même d'agents
de renseignement, fait surveiller les parents par leurs enfants, entreprend
de casser la transmission des usages et des croyances en
tablant sur la jeune génération contre l'ancienne. Ne sommes-nous
pas en terrain connu, celui de la guerre subversive ? S'il veutfaire
du passé table rase, c'est évidemment parce qu'il en a une bien
mauvaise opinion. Selon lui, il n'y a rien de bon à tirer de la religion
païenne qu'il définit comme étant un mélange de satanisme,
d'ignorance et de crédulité. Les Brahmanes ne valent rien. Rien ici
ne préfigure ce qui sera plus tard l'attitude de la Compagnie de
Jésus envers les civilisations asiatiques, Robert de Nobili ou Matthieu
Ricci. La sincérité de Xavier et la chaleur avec laquelle il évoque
à la fin son amitié pour ses Frères ne manquent pas de racheter
son antipathie envers l'Inde, sa caste sacerdotale et sa culture

\begin{quote}
Cochin, le 15 janvier 1544
+
Ihus ,
La grâce et l'amour du Christ notre Seigneur soient toujours en
notre aide et en notre faveur. Amen.
1. Voici deux ans et neuf mois que je suis parti du Portugal et
depuis, de mon côté, je vous ai écrit trois fois, en comptant cette
lettre-ci ; je n'ai reçu de vous que quelques-unes depuis que je me
trouve ici, en Inde ; elles ont été écrites le 13 janvier de l'année
1542 et Dieu notre Seigneur sait la consolation que j'en ai reçue.
Il doit y avoir deux mois qu'on me les a remises ; si elles sont arrivées
si tard en Inde; c'est parce que le navire qui les transportait
a hiverné à Mozambique.
2. Messire Paul, François de Mansilhas et moi, nous sommes en
bonne santé. Messire Paul se trouve à Goa, au collège de Sainte
Foi, où il a la charge des étudiants de cette maison. François de
Mansilhas et moi, nous sommes parmi les Chrétiens du Cap Comorin.
Voici plus d'un an que je suis chez ces Chrétiens, à propos desquels
je dois vous dire qu'ils sont nombreux et qu'il se fait chaque jour

\end{quote}

\begin{quote}

beaucoup de nouveaux Chrétiens. Sitôt arrivé sur cette côte
où ils habitent, J ai tâché de savoir d'eux quene·connaissance ils
o~t du Christ not~~ Sei~eur ; à propos des articles de foi, je leur
ai ~emandé ce q~ Ils croient, ou ce qu'ils ont de plus maintenant
qu Ils sont chrétiens que lorsqu'ils étaient païens et je n'ai pas
trouv~ d'autre réponse sur leurs lèvres que ceci, à savoir qu'ils sont
chréti~ns et que,. ne comprenant pas notre langue, ils ne connaissent
~ notre LOI, ru ce qu'il faut croire. Comme ils ne me comprenaient
pas, et que moi non plus, je ne les comprenais pas, parce
qll;~ l~W:,langue ~aternelle est le malabar 1 et la mienne le bis~
aien , J ~semblai ceux d'entre eux qui sont les plus instruits et
Je cherchai des personnes comprenant notre langue 3 et la leur.
Après nous être à grand peme réunis de longs jours durant nous
avo_ns traduit les prières, tout d'abord la façon de faire le si~e de
cro~ en confessant que les trois Personnes sont un seul Dieu ;
ens~te, le Credo, les commandements, le Pater Noster, l'Ave
Maria, le Salve Regina et la confession générale, de latin en malabar.
Après les. avo~ trad_uites dans leur langue et les avoir apprises
par coeur, Je m en sws allé par tout le village 4 une cloche à la
main, pour ~ssembler !ous les enfants et tous les hommes que je
~ouvais réumr. Une fois assemblés, je les ai instruits deux fois par
Jour. Je leur ai enseigné les prières pendant la durée d'un mois leur
ayant donné cet ordre que les enfants enseigneraient à leurs ~ères
et mères,_et à tous ceux de leur maison et de leur voisinage ce qu'ils
ont appns à l'école.
3. Le ~anche, j'assemblais tous les habitants du village, hommes
aussi bien que felll;°les, grands et petits, pour réciter les prières
dans leur langue ; ils y ont montré beaucoup de plaisir et ils
Y sont venus avec beaucoup d'allégresse. Après avoir commencé
par_ conf~sser un seul Dieu, trine et un, ils ont récité le Credo à
P!e!ne VOIX d~s leur langue et au fur et à mesure que je le leur ai
~eclté tous m _ont donné les réponses. Une fois le Credo terminé,
Je le reprenais tout seul ; disant chaque article séparément et
m'arrêtant à chacun des dou~e, je les admonestais en leur expliqu~
t que le nom de« Chrétiens » ne signifie rien d'autre que de
crorre fermement et sans doute aucun ces douze articles ; puisqu'ils
ont proclamé qu'ils sont chrétiens, je leur demandais s'ils croient
fermement en chacun de ces douze articles. De la sorte, tous ensem-
\end{quote}