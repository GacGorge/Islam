\chapter{Les modèles de mission dans le Nouveau Testament}

\mn{Dynamique de la mission Chrétienne David Bosch 20/9/22}

\begin{Synthesis}
Concile de Jérusalem : pas une concession, une position commune pour \textit{préserver l'unité}.
\end{Synthesis}

\begin{quote}
 Selon cette dernière opinion, le fait que la rencontre ait
eu lieu à Jérusalem et que les apôtres y aient joué un rôle crucial, a été
accepté de bonne grâce par Paul et la délégation d'Antioche. Ceux-ci
n'ont pas fait simplement un geste de conciliation. La position de
Jérusalem et des apôtres correspondait à une identité commune (cf.
Holmberg 1978: 26-28). La« concession» n'a pas été exploitée par
les «colonnes» de Jérusalem: tout comme Paul, ils redoutaient la
division et étaient prêts à tout pour sauvegarder l'unité: Le désir e
s'écouter était mutuel. Les chefs de Jérusalem ne voulaient nen faire
qui mette en péril les communautés chrétiennes non juives, dont
plusieurs existaient déjà depuis plus de dix ans (Hengel 1986: 116).
Ils étaient capables de distinguer entre la piété de la Torah et le coeur
de la vie chrétienne. Tous ces facteurs ont préparé la voie à l'entente
dans ce concile, que Meyer (1986: 1oi) caractérise comme étant « la
décision politique capitale de l'Église du I°'  siècle». . . .
Bien entendu, avant et après le concile, les hebrawi chrétiens
restèrent obstinément critiques à l'idée d'une mission indépendante de
la Torah (Meyer 1986 : 99). Au concile, ils formaient une minorité.
Mais dans les années qui suivirent, ils gagnèrent toujous pus
d'influence dans l'Église de Jérusalem. Le récit de Luc sur la s1tuat1on
conflictuelle qui prévalait lors de la dernière visite de Paul à Jérusalem
(Act 21: 17-26) semble bien le confirmer (Hengel 1986: 116s).
Les divergences à propos des conclusions du concile persistèrent
jusqu'à la Guerre juive. Il est vrai que, avant même la destruction du
temple et la chute de Jérusalem, la plupart des chrétiens juifs avaient
quitté la Judée. Au moment où la guerre éclata, le mouvement des
sadducéens était déjà en perte de vitesse. Avec la destruction du
temple, les sadducéens perdirent leur dernier appui. Les remous de la
guerre précipitèrent leur fin, mais aussi celle des zélotes et des
esséniens en tant d'organisations séparées. Seuls les pharisiens
survécurent à la crise, en partie grâce à leur enracinement dans les
synagogues, dispersées à travers le pays juif et au loin. Aussitôt après
la guerre, ils arrivèrent à établir leur autorité, pratiquement sur tout le
judaïsme. Dès qu'ils eurent le dessus, Ils introduisirent des restrictions
pour les chrétiens juifs qui étaient encore membres e synagogues
locales. Il devint toujours plus difficile de rester pratiquant Juif et
même temps que chrétien. Autour de 85 ap. J.-C., cela devint
impossible. Les \textit{Dix-Huit Bénédictions}\sn{La 12ème bénédiction : Pour les meshummadim, qu'il n'y ait pas d'espérance ! Extirpe vite le règne de la superbe, de nos jours ! Puissent les nosrim périr en un instant ! Qu'ils soient effacés du livre de la vie ! Qu'ils ne soient pas inscrits au nombre des justes ! Béni sois-tu, Seigneur, qui humilies les insolents ! }, promulguées par. les pharisiens
à leur nouveau centre de Jamnia, comportent une disposition qm
frappait d'anathème les chrétiens ( « nazaréens ») et les hérétiques
(minim) et les excluait des synagogues.
\end{quote}

\begin{quote}
    Pour le mouvement de Jésus, ce ne fut pas la fin. Il venait de relever
sn premier défi : ou bien rester dans les limites du judaïsme ou bien
vivre dans la logique du ministère même de Jésus en en franchissant
les barrières. Il choisit cette dernière voie. Le sens missionnaire de la
communauté mettait ses membres dans l'impossibilité de faire autrement.
Leur horizon s'étant dégagé à l'infini, il n'y avait plus moyen
de revenir en arrière. L'Église avait fait un bond définitif vers la vie et
elle l'avait fait au bon moment (Dix 1953 : 55).
    
\end{quote}
\begin{Synthesis}
La question de l'annonce par la thora n'est pas entièrement tranchée par le concile de Jérusalem. C'est l'éjection des judeo-chrétiens des synagogues qui force l'Eglise à devenir \textit{paulinienne et missionnaire}
\end{Synthesis}

\paragraph{La pratique missionnaire de Jésus et l'Église primitive}
\begin{quote}

Essayons de relier ensemble quelques éléments majeurs du
ministère missionnaire de Jésus et de l'Église primitive.
1. D'abord et avant tout, la mission chrétienne primitive se rattache
à Jésus lui-même. Il reste toutefois impossible d'insérer Jésus dans un
cadre bien circonscrit. Schweizer a raison de l'appeler « l'homme qui
n'entre dans aucune formule» (1971 : 13):
\begin{quote}
    Ce qu'il dit et fit heurta ses contemporains. Ils auraient compris et toléré
un ascète qui aurait éliminé Je monde au profit du futur royaume de Dieu.
Ils auraient compris et toléré un homme féru d'apocalypse qui n'aurait
vécu qu'en espérance sans le moindre intérêt pour les affaires du monde.
(\ldots) Ils auraient compris et toléré un pharisien qui, d'urgence, aurait
sommé les gens d'accueillir le Royaume de Dieu ici et maintenant en
obéissant à la loi pour participer au Royaume à venir. Ils auraient compris
et toléré un réaliste ou un sceptique qui se serait tenu dans cette vie les
pieds sur terre, se déclarant agnostique quant aux attentes du futur. Mais ils
ne pouvaient admettre un homme qui annonce que le Royaume de Dieu
prend place parmi les hommes au moyen de ses paroles et de ses actes et
qui, malgré cela, refuse, en une prudence incompréhensible, d'exécuter des
miracles concluants ; qui guérit des individus, mais refuse de mettre fin à
la misère de la lèpre ou de la cécité ; qui parle de détruire le temple et d'en
bâtir un nouveau, mais qui ne boycotte même pas le culte de Jérusalem
comme la secte de Qumran, qui, elle, inaugurait un nouveau culte, purifié,
dans le couvent du désert ; un homme qui, par-dessus tout, parle de
l'impuissance de ceux qui ne tuent que le corps, mais refuse de chasser les
Romains du pays (E. Schweizer 1971 : 25s).
\end{quote}

C'est une perspective à garder à l'esprit lorsqu'on analyse la mission
de Jésus.
2. La mission chrétienne primitive était politique, même
révolutionnaire. Ernst Bloch, le philosophe marxiste, a dit une fois
qu'il était difficile d'imaginer une révolution sans la Bible. 
\end{quote}

\begin{quote}
    A quoi Moltmann ajoute, faisant écho à Ac 17,6s qu'"\textit{il est encore plus difficile,, avec la Bible, de ne pas provoquer de
révolution"}. Dans son étude exhaustive en trois volumes sur la métaphysique
politique de Solon (W siècle av. J.-C.) à Augustin (Ve siècle ap. J.-C.),
le juriste allemand Arnold Ehrhardt a démontré la nature subversive de
la foi et des écrits des premiers chrétiens (1959: 5-44). Comme il fait
autorité en jurisprudence et politique romaines et grecques de
l' Antiquité, Ehrhardt était capable d'identifier bien des termes et des
comportements des premiers chrétiens qui devaient être carrément
séditieux pour l'époque, même si nous ne les ressentons plus ainsi.
Ceci ne concerne pas seulement le mouvement de Jésus des années 30
en Palestine, mais aussi Paul, Luc et les autres auteurs du Nouveau
Testament. Le mouvement chrétien des premiers siècles était radicalement
révolutionnaire « et il devrait bien l'être aussi aujourd'hui»,
mais, précise Ehrhardt, il faut se rappeler que les révolutions ne
s'évaluent pas en termes de terreur et de destruction, mais plutôt sous
l'aspect des horizons nouveaux qu'elles ouvrent (p. 19). L'Église
primitive a dégagé de tels horizons dans son extension missionnaire à
travers le monde gréco-romain. En rejetant tous les dieux, elle a détruit
les fondements métaphysiques des théories politiques dominantes. Les
chrétiens confessaient Jésus comme Seigneur de tous les seigneurs
sous les formes les plus diverses, qui se profilaient dans le contexte
politico-religieux du temps. C'est la manifestation politique . la plus
révolutionnaire qu'on puisse imaginer dans l'Empire romain des
premiers siècles de l'ère chrétienne. Dans l'éclairage du \textbf{règne de Dieu}
·introduit par Jésus, et qui englobait tout, \textbf{il était impensable de prendre
la religion pour une « affaire privée» ou de séparer le «spirituel» du}
«matériel».

3. Le caractère révolutionnaire de la mission chrétienne primitive se
manifesta, entre autres, par de nouveaux modes de relations à l'intérieur de la communauté. Juifs et Romains, Grecs et barbares,
libres et esclaves, riches et pauvres, femmes et hommes, s'accueillaient
mutuellement comme frères et soeurs. C'était un mouvement sans
pareil, une véritable « impossibilité sociologique » (Hoekendijk
1967a: 245). Il n'est pas surprenant que la communauté chrétienne
primitive ait causé tant d'étonnement dans l'Empire romain et au
dehors. Les réactions n'étaient pas toujours positives. Effectivement, la
communauté chrétienne et sa foi étaient tellement différentes de tout
ce qu'on connaissait dans le monde antique que, pour ceux du dehors,
elles n'avaient aucun sens. Suétone qualifie le christianisme de
« nouvelle superstition malveillante » ; Tacite l'appelle « futile et
insensé», il reproche aux chrétiens leur« haine de la race humaine» et
il les traite de « gens réprouvés» parce qu'ils considéraient
dédaigneusement les temples comme des morgues, qu'ils ne faisaient
aucun cas des dieux et se moquaient des objets sacrés 

\end{quote}

Aspect révolutionnaire de la foi chrétienne 
\begin{quote}
\sn{références dans
    Harnack 1962 · 267 -270 ; excellente vue d'ensemble
pa1ennes sur les chrétiens   des premiers siècle dans - Wilken 1980 \textit{passim} }
. Ce que les chrétiens étaient et faisaient se situait hors de la grille commune aux philosophes de l'époque. En même temps, il faut 
remarquer que, pendant le premier siècle, les chrétiens furent critiquer 
pour des raisons sociales plutôt que politiques. L'action se dirgiea contre eux slt lorsque le christianisme commença à assumer son identité propre et menaça de devenir un mouvement puissant (cf. Malherbe 1983, 21s).
 
 
 
 Selon la \textit{lettre à Diogène, } les chrétiens ne se distinguaient pas du reste de l'humanité par leur langage, leurs coutumes, eur habitat. Il restait cependant toujours une certaine \textit{distance} entre eux et le monde environnant. Ils se maintenaient dans le monde comme dans une prison, et pourtant ce sont eux qui maintenaient le monde en son intégrité.
\end{quote}

Cette position révolutionnaire a pu être vu comme positive. 
\begin{Synthesis}
But des chrétiens de Sauvegarder le monde en pratiquant l'amour. 

Maranatha : intense espoir non encore réalisé
\end{Synthesis}

\begin{quote} \mn{p.68}
    kahneman : 
    \begin{quote}
    
    Il n'a nullement apporté le paradis terrestre et ce qu'il a fait réellement l'a
mené finalement à la croix. Il a a introduit le règne de Dieu dans le monde 
des démons, mais il na pas poussé à qu'il s'accomplisse totalement et 
universellement. Il a dressé des signes pour montrer que ce règne s'était
approché, et que le combat contre les puissances et les autorités de le
 
temps avait commencé.
\end{quote}
 
 \ldots
 
 l'Eglise primitive a continué le ministère de Jésus, dressant des signes du règne naissant de Dieu. les chrétiens n'avaient pas vocation d'en faire davantage mais pas moins non plus.\sn{
Pas très clair ce passage sur les signes. }
\ldots
des signes contestés. \ldots
C'est bien ainsi qu'une mission authentique s'est toujours présentée : dans la faiblesse. Paul défiant toute logique, déclare : \begin{quote}
    \textit{lorsque je suis faible, c'est alors que je suis fort (2 Co 12,10)}
\end{quote}
[\ldots]

témoignage et martyr vont de pair
\end{quote}


\paragraph{Les défaillances de l'Église primitive}

\begin{quote}
     
\end{quote}

\begin{quote} \mn{p.68}
 Les communautés chrétienns étaient aussi loin d'être idéales que nos    Églises d'aujourd'hui. Il ne s'agit pas là d'un développement tardif
de la fin du I'" siècle : certains défauts se manifestent dès le début. On
découvre des rivalités évidentes très tôt parmi les disciples de Jésus. Un
exemple : Jacques et Jean demandent des sièges honorifiques au
royaume de Jésus (Marc 10: 35-41), provoquant l'indignation des
autres disciples. On note encore, surtout chez Marc, plusieurs cas où
les disciples manquent de discernement et de foi (cf. Breytenbach
1984: 191-206). Et le livre des · Actes, loin de présenter un tableau
idéal de l'Église primitive, ne cache pas qu'il existait des tensions, des
lacunes et des péchés chez les premiers chrétiens, y compris chez des
dirigeants.

Je n'en dirai pas plus sur les défaillances du christianisme primitif
en général. Par contre, je désire attirer l'attention sur certaines
faiblesses plus spécifiques des premiers chrétiens dans le domaine de
la mission - des faiblesses qui risquent de détruire, à des degrés divers,
la cohérence du premier stade du paradigme.


1. \textbf{J'ai déjà signalé que Jésus n'avait pas eu l'intention de fonder
une nouvelle religion.}\sn{Voir baptisé au nom du Fils \ldots} Ceux qui le suivaient ne portaient pas de nom
pour se distinguer d'autres groupes, ils n'avaient pas de confession de
foi particulière ni aucun rite révélant un caractère distinctif, ni de
centre géographique pour leurs activités (Schweizer 1971 : 42 ;
Goppelt 1981 : 208). Les Douze formaient l'avant-garde de tout Israël
et au delà, implicitement, de l'humanité entière. La communauté
rassemblée autour de Jésus jouait un rôle de \textit{pars pro toto}, de
communauté chargée de stimuler les autres. Jamais cette communauté
ne devait se dissocier des autres.
Or, sa vocation ne s'est pas maintenue longtemps à ce niveau élevé.
Tôt déjà, les chrétiens ont commencé à prendre conscience de ce qui
les distinguait des autres plutôt que de leur propre vocation et de leur
responsabilité à l'égard des autres. Ils ont préféré se mettre à vivre
comme groupe religieux à part plutôt que de s'engager pour le règne
de Dieu. Je cite Alfred Loisy (1976: 166) : \textit{« Jésus a annoncé le
Royaume et c'est l'Église qui est venue.»} Au cours des âges, la
communauté de Jésus est devenue une nouvelle religion, le
christianisme, un nouveau principe de division de l'humanité. Et il en
a été ainsi jusqu'à aujourd'hui.


2. La deuxième défaillance de l'Église primitive est étroitement liée
à la première: \textit{cessant d'être un mouvement, elle est devenue une
institution. }Il y a, selon H.R. Niebuhr (à la suite de Bergson), des
différences essentielles entre une institution et un mouvement: l'une
est conservatrice, l'autre progressiste ; la première est plus ou moins
passive, soumise à des influences extérieures, le second est actif, il
exerce une influence plutôt qu'il n'en subit; l'une est tournée vers le
passé, l'autre vers l'avenir (Niebuhr 1959: lls). Ajoutons que l'une
est craintive, l'autre est prêt à prendre des risques ; l'une protège ses
frontières. l'autre les franchit .
\end{quote}
\begin{Def}[Institution]
La chose instituée (personne morale, groupement, régime).
Mais ici, question sur la dialectique de la sociologie des religions plaquée sur l'Eglise primitive.
\end{Def}

\begin{Synthesis}
si Jésus n'a pas voulu créer une religion, c'est que les chrétiens devaient UNIR le monde, \textit{avant-garde de tout Israel et de l'humanité entière}

d'un mouvement (prophète - Paul), l'Eglise est devenue institution (Évêque - Jérusalem). 
\end{Synthesis}
\begin{quote} \mn{p.70}
    Pour discerner la différence entre institution et mouvement, il suffit de comparer la communauté de Jérusalem à celle d'Antioche ds les
années 40 du I{e} siècle. L'esprit de pionnier de l'Église d'Antioche
provoqua aussitôt une inspection par celle de Jérusalem: 1 est évident
que le souci de l'équipe de Jérusalem n était pas la mission, mais la
consolidation ; non la grâce, mais la loi ; non le franchissement des
frontières, mais leur fixation ; non la vie, mais la doctrine ; non le
mouvement, mais l'institution.
La tension entre ces deux manières de voir conduisit, comme nous
l'avons vu, à la convocation du « concile des apôtres» n l'an 47 ou
48. D'après le récit de Luc (Act 15) et aussi d'après celui de Paul (Gal
2), le point de vue des non-juifs prévalut à ce oet. Cependant la
situation resta indécise, et la tendance du christianisme  primitif à
s'institutionnaliser devint à long terme irrésistible - non seulement
dans les communautés de chrétiens juifs, mais certainement aussi chez
les non-juifs.

Dans un premier temps, il semble que deux types A différents de
ministères se sont développés: le\textbf{ ministère local} d éveques (?u anciens)
et de diacres, et le \textbf{ministère mobile } d'apôtres de prophetes et
d'évangélistes. Le premier tendait à faire du  christianisme  primitif une
institution, le second gardait la dynamique du mouvement. A Antioche,
les premières années, il y eut constamment une tension créatrice
entre ces deux types de ministères. Paul et Barnabas furent en même
temps des dirigeants de l'Église locale et des ministères itinérants, et 11
semble qu'en revenant à Antioche ils reprenaient tout naturellement
leurs responsabilités dans la congrégation. Mais ailleurs (t certainement
plus tard aussi à Antioche), les Églises s'institutionnalisèrent peu
à peu et s'occupèrent moins du monde exterieur. Bientot, elles durent
fixer des règles pour le déroulement de leurs cultes cf. 1 Cor ,11 • 2-
33 ; 1 Tim 2 : 1-15), elles établirent des critères de 1 homme d Église
idéal et de sa femme (1 Tim 2: 1-13), elles firent face à des cas de
refus d'hospitalité envers des émissaires d'Églises et de lutte pour le
pouvoir (3 Jean; cf. Malherbe 1983 : 92-112). Au fi! du. temps, es
chrétiens consacrèrent toujours plus d'énergie aux afares mternes de
l'Église et à la lutte pour subsister comme groupe religieux à part.


3. Nous avons déjà fait allusion au trois1ème aspect de défaillance
dans l'Église primitive: elle s'est montrée incapable, à long terme, de
faire en sorte que les juifs s'y sentent chez eu. Fnée come
mouvement religieux à l'oeuvre exclusivement parmi les Jutfs, el!e. s est
muée dans les années 40 du r' siècle, en un mouvement des Jutfs et
non-jifs à la fois, et elle a abouti à ne plus proclamer son message
qu'aux non-juifs. ,
Deux événements ont servi de catalyseurs à cet . égard, . l un
religieux-culturel (l'affaire de la circonciion des convertis· non Juifs),
l'autre sociopolitique (la destruction de Jerusalem et du temple en 70). APrès la guerre, le judaisme pharisaique devint bcp trop  
    xénophobe pour tolérer autre chose qu'une orientation dure,
exclusivement juive. Les chrétiens juifs furent obligés de choisir entre
l'Église et la synagogue, et il semble qu'ils furent nombreux à choisir
cette dernière. En outre, vu les circonstances, il devint pratiquement
impossible de recruter de nouveaux convertis parmi les juifs.
Dans les années 50, Paul se sentait encore passionnément et
inconditionnellement chargé de convertir les juifs. Quelques décennies
plus tard, longtemps même après la Guerre juive, Matthieu et Luc
tentèrent tous deux de démontrer « la nécessité de la mission auprès
des juifs et la préséance permanente d'Israël» (Hahn 1965 : 166). A
la longue, l'intérêt fléchit. Au comportement anti-chrétien du
judaïsme, l'Église répondit par l'anti-judaïsme.
\end{quote}

Alors que Jésus n'avait pas créé de religion, les juifs ne se sentant pas chez eux dan sle Christianisme les expulsèrent.
bizarre cette idée qu'une "secte" puisse accueillir la "religion" ? 


\paragraph{Pouvait-on faire autrement ?}

\begin{quote}
    En évoquant la mission de l'Église primitive, nous ne pouvons que
déplorer aujourd'hui les trois défaillances que nous venons de mettre
en lumière. Il faut toutefois se demander si elles étaient réellement
évitables étant donné les circonstances où se trouvait l'Église à ce
moment. \textbf{Elles ne l'étaient sans doute pas.}
D'abord il faut se demander si on est en droit d'attendre d'un
mouvement qu'il continue à exister uniquement comme mouvement.
Ou bien un mouvement se désintègre, ou alors il s'institutionnalise -
c'est une simple loi sociologique. N'importe quel groupe qui a débuté
comme mouvement et a trouvé moyen de survivre, l'a fait en
s'institutionnalisant progressivement: les vaudois du Piémont, les
moraves, les quakers, les pentecôtistes et encore beaucoup d'autres. Il
ne pouvait pas en être autrement du mouvement chrétien primitif. A
long terme, il ne pouvait pas survivre uniquement en ayant un chef
charismatique entouré d'artisans du bas peuple venant de la périphérie
de la société. En fait, cet aspect n'a duré que les premiers mois du
ministère public de Jésus. Certains spécialistes du Nouveau Testament
du siècle dernier (comme Adolf Deissmann) et des théologiens
marxistes ont pensé que la très grande majorité des chrétiens venaient
des basses couches de la société et que le christianisme était donc
essentiellement un mouvement prolétarien, mais certaines études
récentes vont dans un tout autre sens. Aujourd'hui les théologiens
admettent qu'à Corinthe, sans doute, l'Église se recrutait parmi le bas
peuple, mais ils pensent que ce n'était pas le cas pour la plupart des
autres Églises (cf. Malherbe 1983: passim; Meeks 1983: 51-73).

\end{quote}

Les chrétiens ne seraient pas tous de couches populaires. 

\begin{quote} \mn{p. 72}
  D'éminents membres du judaïsme traditionnel avaient eux aussi
porté un vif intérêt au mouvement de Jésus à son premier stade. Deux
noms s'imposent: Joseph d' Arimathée et Nicodème. On peut bien les critiquer tous deux non leur indécision et les blamer de leur lenteur à rallier publiquement la cause de Jésus, de leur sens excessif de  respectabilité bourgeoise, mais en a-t-on le droit? Après tout, Joseph
et Nicodème ont tous deux fait le pas avant Pâques, sans savoir que
Jésus allait ressusciter des morts. C'était une démarche osée vu leur
situation, avec toutes les responsabilités de cadres qu'ils assumaient
(cf. Singleton 1977 : 31). On peut bien dire que c'était un pas timide,
qu'ils auraient dû abandonner femme et enants, et le Sanhédrin (dont
ils étaient membres) pour suivre Jésus parmi les hameaux de la Galilee,
mais est-on en droit de l'attendre de leur part? Bien peu de gens
peuvent être en même temps à la périphérie et au centre. Et même
lorsque c'est le cas, c'est en général pour très peu de temps.


Quoi qu'il en soit, les Josephs et les Nicodèmes ont contribué à
aplanir le passage d'un mouvement charismatique à une institution
religieuse. En ce sens, ils auront aussi aidé à garantir la survie du
mouvement. Sans eux, en termes de raison et de sociologie, le
mouvement de Jésus aurait été soit absorbé par le judaïsme, soit il
aurait disparu, « ne laissant que le souvenir d'un bizarre mouvement
millénariste » (Singleton 1977 : 28).

Il n'est pas possible à un mouvement purement et exclusivement
religieux d'être en même temps une réalité qui traverse les siècles en Y
exerçant une influence dynamique. Notre critique ne s'en prend donc
pas au fait que ce mouvement soit devenu institution, mais qu ette
transformation lui ait tant fait perdre de sa fougue. Les convictions
ardentes des premiers adhérents se sont cristallisées en des codes, en
des institutions solidifiées et en des dogmes pétrifiés. Le prophète est
devenu prêtre de l'institution, le charisme une onction, e l'amour
une routine. L'horizon du monde s'est rétréci aux hmttes de la
paroisse locale. L'impétueux torrent missionnaire des premières
années s'est laissé endiguer en un calme ruisseau, ou même n étang
stagnant. C'est cette évolution qu'il y a lieu de déplorer. Instttuton et
mouvement ne devraient jamais s'exclure mutuellement: Église et
mission non plus. 
La deuxième défaillance de l'Église primitive dans le domame de
la mission est sa rupture d'avec les juifs. Il faut de nouveau nous
demander si cette évolution était évitable. Comment l'Église primitive
aurait-elle pu faire autre chose que suivre fidèlement la logique du
ministère de Jésus et néanmoins adopter la loi juive comme moyen de
salut? De même, comment le judaïsme aurait-il pu rester identique à
lui-même et s'ouvrir aux nations par une mission exempte des
prescriptions de la loi? Dans ces circonstances pouvait-on à long
terme, faire autre chose que se séparer ? En outre, après les évenements
de la Guerre juive qui ont balayé le judaïsme, peut-on encore blamer
le judaïsme pharisien de s'être réformé en un club religieux
xénophobe et d'avoir conçu les Dix-Huit Bénédictions?
\end{quote}

L'institutionalisation du Christianisme a "sauvé" le christianisme qui aurait disparu s'il était resté simple mouvement.
De même par rapport aux juifs, fidélité à l'enseignement de Jésus de ne pas imposer la loi juive aux non chrétiens. Mais du coup, renoncement à cette loi pour les juifs ? impossible.

\begin{quote}
Sociologiquement et donc raisonnablement, on ne  peut que repondre réponsulemnt non. En fait, le sort en était déjà jeté au cours du
    ministère terrestre de Jésus de Nazareth. Quarante ans plus tard, à la fin
de la Guerre juive, le destin du judaïsme et celui du christianisme
étaient définitivement scellés : ils iraient chacun de leur côté.


Pour des chrétiens, il n'est guère réjouissant de raconter ce tournant
de l'histoire, surtout à la lumière des relations ultérieures entre
chrétiens et juifs. Il faut bien admettre que le germe de l'anti-judaïsme
a déjà été semé très tôt. L'apôtre Paul qui pouvait souhaiter être frappé
d'anathème et séparé de Christ pour ses frères israélites (Rom 9: 3s),
va jusqu'à accuser les juifs d'avoir tué Jésus, de déplaire à Dieu, d'être
les ennemis de tous les hommes et de mettre ainsi le comble à tous
leurs péchés, appelant sur eux la colère éternelle de Dieu (1 Thess
2: 15s). Ce comportement a servi de modèle aux opinions ultérieures
sur le judaïsme. Dans l' Apocalypse, par deux fois, l'assemblée
religieuse juive est comparée à « une synagogue de Satan » (2 : 9 ;
3 : 9). L' Épître de Barnabas (en 113 environ) et le \textit{Dialogue avec le
juif Tryphon de Justin}\sn{pas vraiment, un dialogue qui essaye de viser la nouveauté du christianisme} (en 150 environ) ont exclu les juifs du champ
de vision de l'Église, les appelant la nation la plus impie et la plus
misérable sur terre, le peuple du diable, séduit par un ange maudit dès
les origines et n'ayant même pas le droit de se réclamer de l'Ancien
Testament (références in Harnack 1962: 66s). Tertullien et Cyprien
concèdent que quelques juifs peuvent être convertis individuellement.
Les édits anti-juifs de l'empereur Théodose en 378 ne font même plus
cette concession. Voici le commentaire de Harnack (1962: 69):
\begin{quote}
Une pareille injustice, commise par l'Église non juive à l'égard ru
judaïsme, est absolument sans précédent dans les annales de l'Histoire.
L'Église non juive l'a éliminé partout; elle lui a ravi son Livre sacré;
elle-même, simple métamorphose du judaïsme, a coupé tous les liens avec
la religion dont elle était issue. La fille a commencé par dépouiller sa
mère, ensuite elle l'a repoussée.
\end{quote}

Le Nouveau Testament doit être perçu comme document
missionnaire, ai-je dit au début de ce chapitre. Le profil missionnaire
du document et de l'Église primitive s'est maintenant dessiné plus
clairement. Il y a une certaine ambivalence dans la nature et la portée
de la mission, j'en conviens, et il semble qu'il en ait été ainsi dès le
début. Mais au cours de nos investigations, nous avons tout de même
pu dégager quelques éléments sûrs et permanents. La mission de
l'Église plonge ses racines dans la révélation de Dieu par l'homme de
Nazareth qui a vécu et oeuvré en Palestine, a été crucifié à Golgotha et,
selon la foi de l'Église, est ressuscité des morts. Pour le Nouveau
Testament, la mission s'est décidée dès qu'on a compris que le temps
eschatologique avait paru, mettani le salut à portée de tous et menant à
son accomplissement final (Hahn 1965 : 167s). 
\begin{quote}
La mission est « le
service de l'Église, rendu possible par la venue du Christ et l'aube de l'événement eschatologique du Salut. L'Eglise se met en route avec confiance et espérance à la rencontre de l'avènement de son Seigneur, 
    elle est chargée de manifester au monde entier l'amour de Dieu et son
oeuvre rédemptrice» (Hahn 1965 : 173 ; cf. Hahn 1980 : 37).
\end{quote}
 Les
témoins du Nouveau Testament rendent possible la réalisation d'une
communauté de gens qui, au milieu de leurs tribulations, gardent les
yeux fixés sur le règne de Dieu en priant pour sa venue, en faisant acte
de disciples, en proclamant sa présence, en oeuvrant pour la paix et la
justice au milieu de la haine et de l'oppression, en dirigeant leur
espérance et leur action vers l'avenir libérateur de Dieu (cf. Lochmann
1986: 67).
En étudiant soigneusement le Nouveau Testament, nous pouvons
obtenir plus de clarté sur ce que la mission signifiait alors et sur ce
qu'elle peut signifier aujourd'hui. Nous allons revenir sur nos pas
pour écouter le témoignage de trois auteurs du Nouveau Testament,
Matthieu, Luc et Paul, qui représentent chacun l'un des modèles du
paradigme de la mission chrétienne primitive, pour découvrir
comment ils ont interprété la mission dans leurs communautés, et pour
nous inspirer de leur approche imaginative dans notre propre
engagement.
Il est bon d'expliqur brièvement pourquoi j'ai choisi de me
concentrer sur les trois témoins néotestamentaires mentionnés. Il aurait
évidemment été possible de donner un aperçu du Nouveau Testament
tout entier et même d'autres écrits du christianisme primitif. Je me suis
décidé pour deux raisons de m'en tenir à l'évangile de Matthieu, de
Luc, aux Actes et aux lettres de Paul. Premièrement, un traitement
consciencieux de tout le matériel que. nous possédons du l"' siècle
exigerait plus qu'un volume et empêcherait une discussion en
profondeur des questions actuelles de la missiologie. Deuxièmement,
et c'est peut-être plus important, je pense que les trois auteurs
néotestamentaires choisis sont, dans l'ensemble, représentatifs de toute
la pensée et la pratique missionnaire du l"' siècle. Je m'en explique en
quelques mots.
Matthieu a écrit, lui juif, à une communauté chrétienne
principalement juive. Tout le propos de son oeuvre était de pousser sa
communauté à s'engager de façon missionnaire dans son entourage.
C'est pourquoi l'entreprise missionnaire protestante des deux siècles
derniers en a appelé, à juste titre, à l' Impératif missionnaire de
Matthieu quand elle avait à rendre compte de son extension aux
peuples du globe. Malheureusement, cette référence à l' Impératif
missionnaire n'a pas tenu compte du fait que cette péricope ne peut se
comprendre que si on la relie intimement à l'évangile tout entier.
\end{quote}

\begin{quote}
J'ai choisi Luc parce qu'il n'a pas seulement écrit un évangile,
comme Marc, Matthieu et Jean, mais en réalité une oeuvre en deux
parties : l'évangile de Luc et les Actes des apôtres. Dans nos bibles,
comme l'évangile de Jean est intercalé entre l'évangile et les Actes, on
oublie facilement que l'évangile et les Actes ont été écrits comme une seule oeuvre et qu'ils devraient être lus dans cette unité. Par la  structuration en deux volumes, Luc voulait démontrer l'unité
fondamentale entre la mission de Jésus et celle de l'Église primitive. C'est ce qui justifie amplement d'inclure l'évangile de Luc et les Actes dans un tour d'horizon de ce genre. 
Quant aux lettres de Paul, le choix se passe de commentaire. On ne saurait concevoir une analyse de la pensée et de la pratique missionnaire de l'Eglise primitive sans étudier les écrit et l'activité de 
l'« apôtre des païens ».
\end{quote}

Trois corpus complémentaires : Matthieu (annonce aux Juifs), Luc (Actes et Ev : lien entre Eglise missionnaire et mission du Christ) et Paul (apôtre des païens).


\section{Pistes}

\paragraph{Une histoire théologique} Théologie qui reconstruit l'histoire. Marie-Françoise Baslez montre en particulier que le martyr vient de la citoyenneté donné à tous. Une alternative, c'est qu'il y atoujours eu une institutions, et en particulier le lien par les lettres entre Evêque. un regard pessimiste sur l'institution (protestant). 

\paragraph{Retour aux sources ?} Au nom de quoi et de qui ? 
Souligner l'unité dans le Règne de Dieu. souligner l'humilité, la faiblesse. St Paul.

\paragraph{Martyr} Les martyrs ont eu un role fondamental en \textit{sacralisant} les lieux où ils sont morts. Pour une religion non sacralisée, crée un lieu (ex : Lyon, période atypique de persécutions. Mais les martyrs ont crée une vénération et des conversions). \mn{question de sociologie sur la fertilité des conversions du père Jacques Hamel, prêtre auxiliaire de Saint-Étienne-du-Rouvray}  


\paragraph{Quelle mission implicite ?}

\paragraph{Contexte} Kahneman : dialectique : on peut accéder à Jésus mais discontinuité entre l'Eglise primitive, en étant fidèle à Jésus.  Comme Loisy, l'Eglise évolue, mais avec continuité : fidélité.

\paragraph{Tournant herméneutique} On doit reinterpréter le message. Comme Paul face à Athènes, il ne peut utiliser les mots de Jésus.

\paragraph{Eglise primitive} Il y a une vision un peu idéalisé de cette Eglise. Bosch montre bien les "failles", mais elle garde, dans son idéalisation, un pouvoir évocateur. Par exemple, \textit{Marie de l'incarnation} au Canada, insiste sur le fait de faire \textit{Eglise primitive}. 

\paragraph{Le message chrétien circule} Les marchands, l'armée circule. Et pas uniquement prophète. Préférer le terme communauté à institution. 

\paragraph{Règne de Dieu} L'Eglise doit avoir le regard sur le règne de Dieu.

