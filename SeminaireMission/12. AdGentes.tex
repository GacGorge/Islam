\chapter{Ad Gentes - Concile Vatican II}


\section{Préparation}
\paragraph{Plan}
Préambule (1)

Principes doctrinaux (2-9)

L’œuvre missionnaire elle-même (10-18)
\begin{itemize}
    \item Introduction (10)
  \item Le témoignage chrétien (11-12)
  \item La prédication de l’Évangile et le rassemblement du Peuple de Dieu (13-14)
  \item La formation de la communauté chrétienne (15-18)
\end{itemize}
 
Les Églises particulières (19-22)

Les missionnaires (23-27)

L’organisation de l’activité missionnaire (28-34)

La coopération (35-41)

Conclusion (42)

\subsection{Lecture attentive}
\begin{itemize}
    \item \paragraph{Principe doctrinaux} : 
    \begin{itemize}
        \item  Dieu veut nous rassembler en un peuple\mn{Jn 11, 52Et ce n'était pas pour la nation seulement; c'était aussi afin de réunir en un seul corps les enfants de Dieu dispersés.}
        \item la mission du fils : Dieu s'engage dans l'histoire humaine.
        \item Mission du saint esprit : \textit{pousse l'Eglise à s'étendre. }
        \item l'Eglise envoyée par le Christ.  \textit{C'est de là (Allez baptiser...) que découle pour l'Eglise le devoir de propager la foi et le salut apportés par le Christ. } 
        \item l'activié missionnaire : Évêques.  \textit{la mission est unique et la même partout, en toute situation, bien qu'elle ne soit pas menée de la même manière du fait des circonstances}
    \end{itemize}
   \item les Eglises particulières
    
\end{itemize}

\paragraph{Rôle des évêques} Après des textes sur l'activité missionnaire centralisée, rôle des évêques. Car la mission ne se pense plus comme les \textit{missions} (quand il n'y a pas d'Eglise en un lieu)

\paragraph{Insistance à l'universalité} \textit{la mission est unique et la même partout, en toute situation, bien qu'elle ne soit pas menée de la même manière du fait des circonstances} -6)

\begin{quote} AG 16
    Ces exigences communes de la formation sacerdotale, même pastorale et pratique, selon les dispositions du Concile [47], doivent se combiner avec le zèle à prendre en considération le mode particulier de penser et d’agir de son propre peuple. Les esprits des élèves doivent donc être ouverts et rendus pénétrants pour bien connaître et pouvoir juger la culture de leur pays ; dans les disciplines philosophiques et théologiques, ils doivent saisir les raisons qui créent un désaccord entre les traditions et la religion nationales, et la religion chrétienne [48]. De même, la formation sacerdotale doit tenir compte des nécessités pastorales de la région ; les élèves doivent apprendre l’histoire, le but et la méthode de l’action missionnaire de l’Église, et les conditions particulières d’ordre social, économique, culturel de leur propre peuple. Ils doivent être éduqués dans un esprit d’œcuménisme et préparés comme il convient au dialogue fraternel avec les non-chrétiens [49]. 
\end{quote}


\paragraph{Transformation} Il n'est pas rare que les groupes humains au sein desquels l'Eglise existe, ne soient complètement transformées pour des raisons diverses. des situations nouvelles peuvent en résultaer (6)

\paragraph{Divisions des chrétiens} 6

\begin{Synthesis}
    L’activité missionnaire n’est rien d’autre et rien de moins que la manifestation du dessein de Dieu, son épiphanie et sa réalisation dans le monde et son histoire, dans laquelle Dieu conduit clairement à son terme, par la mission, l’histoire du salut. 9
\end{Synthesis}
