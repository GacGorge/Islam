\chapter{Maximum Illud, Rerum Ecclesiae, Fidei Donum}

\section{Introduction}



Nous étudions 3 textes sur la mission, écrits par 3 papes dont les pontificats (1914-1958) se situent au cœur du XXème siècle sanglant (2 guerres mondiales et les totalitarismes). 
\begin{itemize}
    \item Le 1er texte de Benoit XV - (1914-22)- date de 1919 ; il s’agit d’une lettre apostolique Maximum illud (MI) mais qui a quasi une valeur d’encyclique ; c’est une sorte de charte des missions modernes qui pose des problèmes approfondis par la suite et qui fixe l’ambition d’implanter le christianisme sur toute la Terre. 

    \item Le 2nd texte est rédigé par Pie XI (1922-39) : il s’agit de l’encyclique sortie en 1926 Rerum Ecclesiae (RE) qui approfondit MI et se préoccupe du développement et de la formation du clergé indigène. 

    \item Enfin, le 3ème et dernier texte est aussi une encyclique, écrite par Pie XII, Fidei donum (FD), peu avant sa mort en 1957. Il s’agit d’un texte qui donne la possibilité au clergé diocésain de partir en mission à l’étranger. 
\end{itemize}

Ces 3 textes témoignent de l’évolution de la situation de la mission en France et en Europe au cours du XXème siècle et de la sortie de ce que C.Prudhomme appelle « l’âge d’or ». Voyons-les à présent les uns après les autres…



---------------------------------------------
\section{Maximum Illud}

Ce n'est pas une encyclique mais une lettre. Le pape est Benoit XV (sept 1914 - \href{https://fr.wikipedia.org/wiki/Conclave_de_1914}{étrange conclave}. \sn{Cardinal Belge "Nous éviterons de parler de la guerre" et le cardinal Allemand : "nous éviterons de parler de la paix"... Ambiance} ).


\paragraph{mission et guerre} Les missionnaires deviennent brancardiers. Les soeurs soignent les algériens,... En 1918, les missionnaires repartent mais ils sont portés par un nationalisme guerrier.
Le pape écrit donc un texte :
\begin{quote}
    vous êtes au service de l'Eglise universelle. 
\end{quote}

\paragraph{contexte d'autonomisation}
Le père Lebbe, lazariste avait fait un rapport pour "siniser l'Eglise de Chine" et plus généralement créer des églises autochtones.
Le missionnaire n'est pas un professionnel : il doit transférer ses pouvoirs au clergé local. 
Importance de la formation du clergé locale. Les prêtres venant d'Occident sont à égalité.

\paragraph{Approche post coloniale} L'Eglise en 1919 se prépare au départ des colonies.

\paragraph{Importance aux langues et la culture}


\paragraph{Qualification des autres} "barbares", "infidèle", "plus par l'instinct que la raison". On est condescendant. L'Europe se pense comme la civilisation. 

\paragraph{Un vocabulaire de conquête} "Position déjà acquise",.. le champ lexical est assez guerrier. 
Aujourd'hui, on n'emploie plus ces termes.


%--------------------------------------
\section{Rerum Ecclesiae}


\subsection{Présentation et biographie de l’auteur, Pie IX (1857-1939)}	

Pie XI, né Ambrogio Damiano Achille Ratti en Italie, est le Pape de l’entre-deux guerres (1922 à 1939). Si l’on jette un regard global sur son pontificat, très riche, c’est le Pape qui fait la transition entre chrétienté et monde moderne sécularisé. C’est certainement ce qui explique quelques paradoxes dans ses positions : 
\begin{itemize}
    \item  viscéralement anti communiste (il évoque dès sa 1ère encyclique, la lutte des classes comme « l’ulcère mortel des nations ») et il publie en mars 1937 Divini Redemtoris qui condamne le communisme et le bolchevisme  ; MAIS il sortira néanmoins une encyclique sociale Quadragesimo anno (QA) qui est la suite de Rerum novarum, 40 ans après, dans laquelle il affirme le principe de subsidiarité. Dans ces 2 encycliques il confirme l’importance de mettre en place un syndicalisme chrétien et par ailleurs, il encourage la naissance de l’Action Catholique, notamment la JOC.                        
  \item  Dans 4 textes qui eurent un grand retentissement, il condamne le nationalisme exacerbé (Encyclique Ubi Arcano en 1922), l’action française (Allocution consistoriale en déc 1926), le fascisme (Encyclique Non abbiamo bisogno en 1931) et le nazisme (Encyclique Mit brenender sorge mars 1937), et il sort ostensiblement de Rome quand Hitler vient dans la capitale italienne ; MAIS il s’entend néanmoins avec Mussolini avec qui il met fin à la querelle de Rome en signant en 1925 les accords du Latran. 
  \item  Sa vision de la religion est dit « intégraliste » c’est-à-dire que « son catholicisme intégral se conçoit comme une contre-culture alternative tant au libéralisme qu’au socialisme apportant ses propres réponses aux besoins de la société contemporaine, sans transiger avec la doctrine et les « erreurs [du] temps », autrement dit avec le modernisme. » Il freine par ailleurs des 4 fers l’ouverture de l’Eglise vers l’œcuménisme en ayant une vision unioniste, MAIS il prêche une égalité parfaite entre toutes les races, il évoque « l’universelle famille humaine » des peuples, qui « sont liés entre eux par des rapports de fraternité » (Ubi Arcano en 1922) et qui ont droit au respect, à la vie, à la prospérité, à la justice, et à la paix. Et devant le risque que représente la traque nazie des juifs, il n’hésite pas, selon la formule qui fera florès à se qualifier de « sémite ». 
\end{itemize}

Difficile donc de classer ce Pape, d’autant qu’au bout du compte, il restera surtout comme « le pape des missions » , comme nous allons le voir à présent.

\subsection{Bibliographie consultée pour l’exposé}
 	

1.	Pie XI, Encyclique Rerum Ecclesiae sur les missions catholiques, 28 février 1926
2.	Benoit XV, Lettre apostolique Maximum illud, 1919
3.	BOUCHER André (Mgr)/ Institut Pie XI, L’action missionnaire, Paris, Librairie Bloud & Gay, 1931, 226p
4.	PRUDHOMME Claude, « Chapitre 23 - La France et les missions catholiques, XVIIIe-XXe siècles » (p 375 à 390) in Alain Tallon & Catherine Vincent (Sous dir), Histoire du christianisme en France, Collection U, Armand Colin, 2014, 448p 
5.	BRIA (Ion), Philippe CHANSON, Jacques GADILLE & Marc SPINDLER, Dictionnaire œcuménique de missiologie – cent mots pour la mission ; Articles « 11 Clergé indigène (p50 à53) » par Paule Brasseur« 12 Colonisation (p53 à 56)» par J.Françis Zorn & « « 69 Papauté et missions (p252 à 255) » par Claude Prudhomme, Paris CERF/Genève Labor et Fides/Yaoundé CLE, 2001, 394p
6.	DELACROIX (ss dir Mgr), Histoire universelle des Missions catholiques- T3 Les missions contemporaines (1800-1957) ; Chap 5 « l’avènement des jeunes églises Benoit XV, Pie XI et Pie XII » (p126 à 165), Paris, Librairie Grund, 1957, 447p
7.	COSTANTINI Cardinal, Réforme des Missions au XXème siècle, Paris, Casterman, 1960, 282p
8.	DESOUCHE Marie-Thérèse, « Pie XI, le Christ Roi et les totalitarismes » Dans Nouvelle revue théologique (Tome 130), 2008/4 , p 741 à 759
9.	ESSERTEL Yannick, Evangélisation et cultures – Essai d’histoire et d’anthropologie d’une pédagogie missionnaire du 1er au XXème siècle, CERF Patrimoines, 554p, 2021, Cote 297 411
10.	OOMS Toon, « Une ère nouvelle pour les missions - Le plaidoyer de Jean Bruls (1911-1982) pour un nouveau modèle missionnaire », dans Nouvelle revue théologique 2022/3 Tome 144, pages 452 à 470
11.	HILDESHEIMER Françoise, Une brève histoire de l'Église - Le cas français IVe-XXIe siècles ; chap9 « Modernistarum callidissimum artificium »,  Coll. Champs Histoire, Flammarion, 2019, 477p
12.	Ouvrage collectif, Théo – Nouvelle encyclopédie catholique, Droguet-Ardant/Fayard, 1989, 1236p
13.	Site du Vatican, notamment consultation des biographies des papes Benoit XV, Pie XI, Pie XII
14.	WIKIPEDIA, articles « Missions catholiques aux XIXe et XXe siècles » ; « Benoit XV » ; « Pie XI » ; « Pie XII » ; « encyclique Rerum Ecclesiae » « Lettre apostolique Maximum illud »


\subsection{Nature du texte et contexte historique et textuel}
 

L’Encyclique Rerum Ecclesiae est la 8ème encyclique de Pie XI, Pape particulièrement prolixe en encycliques  . C’est, nous l’avons dit, le « Pape des missions ». Il intervient dans une période particulièrement intense :  le personnel missionnaire passe de 1903 à 1925 de 15000 à 30000. Tout est doublé : les missions, les baptisés. Le Pape se sent donc appelé à organiser ces missions. Particulièrement actif dans les 8 premières années de son pontificat, il adopte une attitude presque militaire en centralisant dès la 1ère année de son pontificat, à Rome, en 1922, l'Œuvre de la Propagation de la Foi (créée par Pauline Jaricot à Lyon). En 1923, c’est la constitution des premières églises indigènes aux Indes et l’année d’après, le concile plénier de Shanghaï. Et en décembre 1924, il organise l’exposition missionnaire du Vatican temporaire (jusqu’au 10 janvier 1926) qui deviendra ensuite permanente, comme il le dit explicitement dans l’encyclique (création en novembre 1926 d’un musée ethnologique des missions au palais du Latran). Puis sort le 28 février 1926 l’encyclique Rerum Ecclesiae. Soucieux de l'ouverture du clergé aux indigènes, il sacre en 1926 les six premiers évêques chinois et l’année suivante le 1er Evêque japonais. Et fin 1927 il institue Ste Thérèse de Lisieux comme patronne des missions. Enfin en 1929, il réorganise et coordonne l’action de ses 3 grands instruments de la Mission, conformément à ce qu’il annonce dans l’encyclique Rerum Ecclesiae : les grandes œuvres pontificales de la propagation de la Foi (qui subvient aux besoins généraux de la Mission) ,  la Sainte Enfance (qui s’occupe des enfants des pays de mission) et St Pierre-apôtre (qui met en place et prend en charge les séminaires et la formation des prêtres indigènes). 

4.	L’encyclique

	4.1 L’objet et l’effet recherché

C’est dans le contexte décrit précédemment que s’inscrit l’encyclique Rerum Ecclesiae. Elle porte le sous-titre suivant : « sur le développement à donner aux missions ». Elle s’adresse donc en premier chef aux responsables de Mission et de façon générale à tous les missionnaires. Mais cela va au-delà car Pie XI souhaite mobiliser tous les catholiques.
Il s’agit essentiellement premièrement de « booster » les missions en impliquant tout le peuple catholique ; il s’agit, deuxièmement, de réfléchir à la mise en place d’une part et à la formation, d’autre part, d’un clergé indigène à vaste échelle.

On ressent une certaine urgence dans cette encyclique : Pie XI se vit clairement comme un pape dont « le but [est] de répandre la lumière de l’Evangile et les bienfaits de la culture et de la civilisation chrétiennes aux peuples qui "étaient assis dans les ténèbres et dans l’ombre de la mort" », pour reprendre le tout début de son encyclique (§1).  

L’enjeu majeur, pour l’Eglise, en effet, c’est à la fois de développer, mais aussi de pérenniser ces avancées missionnaires sur des terres nouvellement christianisées. 
Le contexte missionnaire est le suivant : implicitement, en Afrique, en Asie,  les religions sont clairement en concurrence les unes les autres et il est clair que Pie XI ne souhaite se faire doubler ni par l’islam, ni par les protestants. L’heure n’est ni à l’œcuménisme (en tous cas, avec son approche unioniste, le Pape est très réticent vis-à-vis des prises d’initiatives qui se font jour) ni au dialogue interreligieux. Mais explicitement, c’est contre le paganisme qu’il se bat : il déplore le fait que les païens soient quasi 1 milliard (§4)

4.2	 L’argumentation 

Souvent, l’Eglise est assimilée au colonisateur. Il y a donc un grand risque que celle-ci soit considérée comme une occupation étrangère par des missionnaires occidentaux. D’où l’urgence ressentie par le Pape de développer et former un clergé insulaire. Pie XI, pour le dire, reprend l’image de la vigne : « Pourquoi le clergé indigène serait-il empêché de cultiver son champ ? ». 

De plus, il se joue aussi un enjeu qui a à voir avec une certaine vision humaniste. De même que Lavigerie a été à la pointe du combat contre l’Esclavage, il est nécessaire de se battre aussi pour parvenir à une totale égalité entre le clergé européen et le clergé insulaire. C’est ce qu’évoque le Pape en parlant de stricte égalité dans la « dignité sacerdotale ».  Mais est-ce vraiment l’argument humaniste qui l'emporte ? N'est-ce pas plutôt une forme de réalisme au nom de l’efficacité missionnaire ? 
Car Pie XI reprend dans son encyclique un argument de Benoit XV figurant dans Maximum Eliad : « le prêtre, par sa naissance, son tempérament, ses sentiments et ses intérêts est en contact étroit avec son propre peuple, il est au-delà de toute controverse à quel point il peut être précieux pour inculquer la Foi dans l’esprit de son peuple. Le prêtre autochtone comprend mieux que n’importe quel étranger comment procéder avec son propre peuple. » (§21). Il rajoute un argument supplémentaire qui va dans le même sens : la connaissance imparfaite de la langue.

Par ailleurs, est pointé aussi le fait que l’Europe a elle aussi besoin de ses prêtres (23§). Il est d’ailleurs intéressant de noter cet argument à la lumière du 3ème texte étudié aujourd’hui de Pie XII : Fidei Donum qui donnera à l’inverse la possibilité aux prètres diocésains à répondre aux appels des missions d'outremer, notamment en Afrique et Amérique latine, tout en restant attachés à leur diocèse d'origine.

Le gros du développement du texte (à partir du §19), et le plus important, est consacré à la formation et au statut de ce clergé indigène. Il est affirmé -ce qui ne va pas de soi au moment où sort l’encyclique-  l’égalité dans la formation initiale, l’égalité dans les responsabilités, dans les missions données aux uns et aux autres. Et surtout, il est clairement préconisé l’ouverture de séminaires sur place. Il faut instruire les séminaristes indigènes de toutes les sciences modernes (sciences sacrées et profanes » est-il précisé §21). 

Pie XI évoque aussi l’opportunité de la création « de congrégations entièrement nouvelles, qui correspondraient mieux au génie et au caractère des indigènes et qui seraient plus en accord avec les besoins et l’esprit des différents pays. » Cela ne l’empêche pas de donner la Possibilité pour les indigènes de rallier des congrégations anciennes y compris des contemplatifs. 

Enfin il souhaite optimiser l’organisation de la mission, et implanter le christianisme en créant sur place des écoles, des hôpitaux, etc… 
 
4.3	La valeur théologique du texte : une christologie de conquête et une ecclésiologie missionnaire qui s’inscrit dans la durée

Le vocabulaire est un vocabulaire de conquête, somme toute pas très loin de l’expansionnisme colonial. Il s’agit de « gagner au Christ tous ceux qui ne le connaissent pas ou sont encore sans troupeau. »  Dans cette vision, le Pape se considère comme le « vicaire du Christ » (cela a été dit par Pie XI dans des discours)
Notons au passage au §5 un passage qui éclaire la vision intégraliste du Pape et qui témoigne d’une christologie très particulière : il importe pour lui en effet : « que nous essayions de mettre sous la domination du doux Christ autant d’autres hommes que possible ». Le Pape et partant l’Eglise en mission, s’apparentent à une armée. La christologie de Pie XI est fondée sur une vision hiérarchique au sommet de laquelle il y a le Christ. Et juste en dessous le Pape.  On rappelle que c’est Pie XI qui a instauré en 1925 la célébration du Christ-Roi qui met en avant la puissance du Jésus-homme …Toutefois il ne s’agit pas de prendre les armes mais de convertir. Et pour cela est mise en avant « l’autorité particulière du Pape sur les missions. » . Pie XI se situe clairement toujours dans une logique de chrétienté, mais ce sont bien les derniers soubresauts qui tomberont après la 2GM.

Par ailleurs, concernant l’Eglise, ce qui est fondamentalement en jeu, c’est un changement de paradigme ecclésiologique et missionnaire : il s’agit d’avoir une vision de la Mission qui soit une approche temporaire, un état transitoire et passager. Il n’est plus question que s’implantent durant des siècles, comme cela avait pu se passer auparavant, des missionnaires étrangers qui s’inculturent. On n’en est plus là. Les missionnaires ont vocation à tout mettre en œuvre pour que se développe et se mette en place un clergé indigène, autochtone qui pourra, si on peut dire, passer la 2ème couche de l’évangélisation.

5.  Appréciation et discussion critique 

Il s’agit d’une encyclique qui amplifie, précise et applique la lettre apostolique du prédécesseur de Pie XI, Benoît XV Maximum Illud (du 30 novembre 1919 ) considérée comme « la grande charte des missions modernes ». Il y a donc beaucoup d’idées qui se trouvaient déjà en germe dans ce premier document comme la formation d’un clergé insulaire. Mais qui sont précisées. 

Si l’on fait un bilan de la situation dans les années 30, soit 5 ans après cette encyclique, selon le livre L’action missionnaire, on parvient aux chiffres suivants : 3734 prêtres indigènes et 8279 prêtres étrangers soit 31% d’autochtones. On voit donc que la politique que veut faire triompher Pie XI ne se concrétise pas massivement dans les faits. Elle sera stoppée par l’ouragan de la seconde guerre mondiale puis de la décolonisation.

Et si l’on prend, pour finir, un regard contemporain, force est de constater que dans tous les pays du monde à présent, se sont mis en place des clergés et des séminaires autochtones. On peut donc dire que de ce point de vue la vision de Pie XI a, de ce point de vue, triomphé.  Un triomphe tel que, par une totale ironie de l’histoire, c’est à présent ce "clergé indigène" -pour conserver un titre qui a extrêmement mal vieilli- qui vient aujourd’hui au secours des paroisses françaises ; et qui, sans lui, faute de vocation sacerdotale dans notre pays, ne seraient plus en mesure de fonctionner…               
FIN DE L’EXPOSE
