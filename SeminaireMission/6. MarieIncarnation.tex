\chapter{Marie de l'Incarnation}


\section{Marie de l'incarnation}


\paragraph{grand siècle} Bossuet, descartes, Pascal, Versailles, Molière. 

\paragraph{siècle de la réforme}On l'appelle aussi siècle des âmes. Au XVIeme, siècle de la réforme, jésuites. On voit la fondation des nouveaux : lazaristes (missionnaires). spiritualité : ignatienne, Carmel, Ecole française (Christo centrique de Pierre de Berulle). \textit{Imitation de Jésus Christ}, livre lu dans les noviciat des Ursulines. Importance pour la mission de FX. Les Carmélites sont arrivés en France au début du XVII


\paragraph{1599-1672}

Marie Guyart est née en France, à Tours. Ses parents, Florent Guyart et Jeanne Michelet, sont maîtres-boulangers et ils ont eu sept enfants. C'est un foyer catholique où les enfants sont encouragés à s'instruire. Au cours des trois premières décennies de sa vie, elle vivra au sein du monde des artisans et de la moyenne bourgeoisie commerçante de Tours.
\paragraph{Tours} ville importante. 

\paragraph{Jeunesse, grâces mystiques}
On sait peu de choses précises de son éducation. Elle fréquenta l'école. Elle avoue avoir reçu une « bonne éducation » qui lui « avait fait un bon fonds dans [s]on âme pour toutes les choses du christianisme et pour les bonnes mœurs »7. À l'âge de sept ans, elle a une première grâce mystique qui la conduit à se donner au Christ. Elle fait alors un rêve marquant qu'elle racontera elle-même bien plus tard (1653) :

\begin{quote}
 « En mon sommeil, il me sembla que j’étais dans la cour d’une école champêtre, avec quelqu’une de mes compagnes... Ayant les yeux levés vers le ciel, je le vis ouvert et Notre-Seigneur... en sortir et qui par l’air venait à moi qui, le voyant, m’écriai à ma compagne : “Ah! Voilà Notre-Seigneur ! C’est à moi qu’il vient !” […] Mon cœur se sentit tout embrasé de son amour. Je commençai à étendre mes bras pour l’embrasser. Lors, lui, le plus beau de tous les enfants des hommes, avec un visage plein d’une douceur et d’un attrait indicible, m’embrassant et me baisant amoureusement me dit : “Voulez-vous être à moi?” Je lui répondis : “Oui” — Lors, ayant ouï mon consentement, nous le vîmes remonter au ciel »8.   
\end{quote} 

Vers l’âge de 14 ans, elle est attirée par la vie cloîtrée. Elle manifeste son désir d'entrer chez les bénédictines de Beaumont, un ordre religieux établi dans la région. Ses parents, qui ne comprennent pas son aspiration à la vie religieuse, la marient à 17 ans avec le maître ouvrier en soie Claude Martin. De leur union naîtra Claude le 1er avril 161910. Six mois plus tard, elle devient veuve à 19 ans alors que la petite fabrique est en faillite. Elle se retrouve avec des biens à liquider et des dettes sur les bras, en plus d'un enfant à élever. Elle décide de retourner chez son père. On lui fait sentir qu'un nouveau mariage réglerait ses problèmes matériels. Mais l'appel de Dieu et de la solitude est bien trop fort.

\paragraph{Conversion : dans du sang}
Le 16 mars 1620, elle vit une expérience mystique qu'elle appelle sa « conversion » : l'irruption du Christ dans sa vie. Elle se confesse au premier religieux qu'elle rencontre et se sent transformée. Elle aspire à une vie de recluse, mais sa sœur Claude, mariée à Paul Buisson, marchand, l'invite en 1621 à vivre chez elle. Elle accepte cette offre pour assurer sa subsistance et celle de son fils. Marie désire y mener une vie d’abnégation et de servitude. Pourtant, ses talents d’administratrice sont reconnus et le couple espère qu'elle les aidera à consolider leur entreprise de transport fluvial en difficulté. Elle prend parfois le rôle de gérante lorsque les deux patrons en titre sont hors de la ville. On ira jusqu'à lui confier la direction de l'entreprise en 1625. Cette même année, les grâces mystiques la conduisent à l'union au Christ. Elle ne peut entrer en religion parce qu'elle doit élever son fils Claude, mais elle fait déjà à cette époque vœu de chasteté, de pauvreté et d'obéissance.

\paragraph{Elle quitte son fils à 11 ans} mais lien spirituel

\paragraph{Religieuse missionnaire à Québec}


 
Le 25 janvier 1631, elle entre au couvent des Ursulines de Tours. Si elle rêve de devenir missionnaire, il n'est pas normal à l'époque qu'une femme, une religieuse de surcroît, fasse le voyage outre-mer pour devenir enseignante. Finalement, sa rencontre avec une autre femme, riche et pieuse, Madeleine de la Peltrie, sera déterminante car elle obtiendra les fonds nécessaires à la fondation de son monastère à Québec.

En 1639, elle part avec deux autres Ursulines, Marie Madeleine de la Peltrie et une servante, Charlotte Barré, pour fonder un monastère à Québec. L'objectif est de veiller à l'instruction des petites Amérindiennes. Elle cherche à convertir au catholicisme les filles qui lui sont confiées : d'abord les Montagnaises et les Abénaquises, puis les Huronnes et les Iroquoises.

\paragraph{Arrivée à Québec à 1639}

\paragraph{difficulté à l'assimilation}
Pourtant, elles auront de la difficulté à franciser les Amérindiennes qui résistent parfois à l'assimilation. Avec le déclin démographique qui bouleverse la population amérindienne et une réticence de plus en plus grande des parents amérindiens à confier leurs filles aux Ursulines, Marie de l'Incarnation devra s'éloigner de son rôle de missionnaire pour se consacrer davantage à l'instruction des jeunes filles françaises de la colonie.


 
Même si elle est cloîtrée, Marie de l'Incarnation joue un rôle actif dans la vie de la colonie. En 1663, elle est témoin d'un tremblement de terre à Québec. Elle narre l’événement dans l'abondante correspondance qu'elle a avec son fils. L'ursuline voit dans la catastrophe un signe de Dieu punissant le commerce d'alcool entre les colons et les Amérindiens. Elle se voit aussi mêlée à une épidémie de vérole qui atteint durement les peuples autochtones : son monastère se voit transformé en hôpital à quelques reprises. Elle commente aussi abondamment les guerres franco-iroquoises et la destruction de la Huronnie.

Marie de l'Incarnation, par ses écrits, est considérée par plusieurs historiens comme étant l'auteur de la première mention en français, et non plus en latin, de l'identité canadienne des colons, en vertu d'une lettre datée du 16 octobre 166611.

Elle meurt de vieillesse le 30 avril 1672 à Québec à l'âge de 72 ans. Elle est associée à la vie de la petite colonie française fondée à Québec, en 1608, qui, sans elle et ses compagnes, aurait difficilement survécu.


\paragraph{femme passionnée} épistoliaire, ethnologue, portée universelle. 

\section{Mère Marie de Saint Joseph}

\paragraph{Biographie} Mère Marie de Saint-Joseph \sn{\href{https://fr.wikipedia.org/wiki/Marie_de_Saint-Joseph}{Mère Marie de Saint Josph}}, née Marie de la Troche de Saint-Germain (1616-1652) est une religieuse catholique canadienne d'origine française. Elle fit partie des premières Ursulines à émigrer au Canada.


Marie de la Troche de Saint-Germain naît en Anjou le 7 septembre 1616. Lorsqu'elle eut atteint ses neuf ans, elle est conduite par sa mère au monastère des Ursulines de Tours. Elle aime beaucoup la lecture, surtout les vies de saints. Saint François-Xavier, l'apôtre des Indes, l'attirait plus que tout autre, parce qu'il avait travaillé à la conversion des infidèles à l'autre bout du monde. À quatorze ans, Marie de la Troche demande à ses parents la permission d'entrer au noviciat des religieuses qui lui avaient donné son éducation. Après beaucoup d'hésitation, les parents la lui accordent. Marie de la Troche prend l'habit sous le nom de Saint-Bernard, qu'elle devait modifier plus tard en celui de Saint-Joseph.
\begin{quote}
    J'étais ravie d'étonnement, écrit Mère Marie de l'Incarnation, de voir en une fille de quatorze ans, non seulement la maturité de celles qui ont plus de vingt-cinq, mais encore la vertu d'une religieuse déjà bien avancée. Rien de puéril ne paraissait en sa jeunesse, elle gardait ses règles dans une si grande exactitude, qu'on eut dit qu'elle était née pour ces actions... En un mot, son esprit toujours également joyeux, la rendait très aimable et très agréable à toute la communauté, et elle veillait si soigneusement sur soi-même, qu'il ne fallait pas lui donner deux fois des avis sur une même chose, elle se tenait même pour avisée et pour reprise des fautes qu'elle voyait corriger en ses compagnes.
\end{quote}


\section{Lettre CXL -1652 sur Marie de l'Incarnation}

\subsection{Contexte}
\paragraph{Vision du Quebec : équivalent au retrait au désert} \textit{pais si barbare}

\paragraph{vie parfaite} vie de sa naissance jusqu'à sa mort. Il ne s'agit pas d'un récit, mais plutot une description de la vie de la religieuse idéale. \textit{pour servir d'exemple à celles qui nous succederons}. C'est un \textit{code de conduite}. 


\paragraph{Elle s'efface devant la mission} Cherche la souffrance \textit{afin de lui être plus semblable}. Freiner. Version janséniste.

\paragraph{Disiciplinée}

\subsection{Spiritualité}



\paragraph{Rencontre Reine} et Archevèque. mais traitement identique. (453) \textit{la religion rend tous ses sujets égaux (457}


Relever le courage de ceux qu'elle voieoit abattus (451)

Instruire les ignorants (451)

nettoyer les filles Sauvages\sn{de Sylva : forêt. Quitter la foret et se sédentariser. } (451)
\paragraph{Bible} \textit{Elle ressentoit leurs biens et leurs maux plus que tout ce qui l'eut pu toucher en ce monde.(412)} 

\paragraph{spiritualité de la relation }{à Dieu dans ses entretiens familiers (412)}
\textit{Ne parloit jamais à une personne qu'elle n'en fut touchée (452)}

\paragraph{Spirtualité ignatienne}
\textit{voir combien elle en étoit détachée dans les occasions où la grâce le devoit emporter sur la nature (78-547)}
\textit{fermeté dans sa vocation dès le commencement (80-457) \textit{en avertissoit en secret s'il y avait quelque chose à redire en leur conduite (54-458)}}, \textit{intérieur et extérieur (84-458)} \textit{\textbf{Elle se vouloit tenir à l'ordinaire (96-464)}}
Croisement de spiritualité. Jésuites présents et fondant l'Eglise de la ville de Québec.


\paragraph{Souffrance} Toutes les dates par rapport à sa mort (relecture) 


\paragraph{Expérience de la nuit} \textit{et enfin elle s'est trouvée dans des délaissemens si extrêmes qu'il sembloit que Dieu l'eut entièrement abandonnée. Ce qu'elle souffroit dans l'extérieur, étoit sans comparaison plus insupportable que ce qu'elle enduroit dans le corps : mais Dieu qui l'affligeoit d'un côté, la soutenoit de l'autre; car elle rendoit des soumissions héroiques à sa divine Majesté pour honorer les délaissements de son très cher et très aimé fils dans la croix. (93-462)}
\paragraph{Relation au Christ - apparition} \textit{Ma fille gardez l'extérieur, et moi je garderai le dedans (454)} \textit{Embrassements} \textit{crepe (454)} \textit{embrassa}Identique Therese d'Avila


\subsection{mission}
\paragraph{Apprentissage de la culture de l'autre}
parler en peu de temps les langues Huronnes et Algonguines.

\paragraph{Francisation} \textit{depuis elles sont demeurées couvertes (87-459)}
 \textit{elles y portent cette façon modeste, en sorte qu'on peut dire que cette chère mère a mis la pudeur parmi les femmes et les filles sauvages (87-459)} Contraste avec les instructions de 1659 sur l'acculturation. 

\paragraph{sauver les âmes} et les corps ? Et les Groupes ? Le corps, c'est ce qui nous rassemble. L'âme, c'est ce qui est individuel. 

\paragraph{Sanctification par la mission ou sanctification parallèle}

\paragraph{Colonisation} de peuplement par les jésuites. Et intégration des indiens. Très original. Cloture : construit et reconstruit. Nouvelle chrétienté ("beauté") que l'on va créer.  Nouvelle terre promise, beaucoup plus belle.

\paragraph{Présence évangélique au service de ceux qui sont à l'image de Dieu} La présence des femmes, en communauté, est missionnaire. Elles portent quelque chose d'exceptionnel. La présence du Christ caché, souffrante, qui par elle-même évangélise. La présence du Christ concret \textit{incarné} est assez neuf. 



\paragraph{Méthode d'évangélisation des bénédictins} attirer les populations semi nomades des forets. organisation sociale et économique. société chrétienne.  gouvernance. 

\paragraph{Une vision positive de l'indien} à l'image de Dieu, regard positif. Amour total au nom du Christ. 

\paragraph{Eloge de la vie Cachée - C'est par l'humilité qu'on attire} attraction pas séduisante (460). 

\paragraph{MRP} mortification renoncement Pénitence