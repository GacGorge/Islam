\chapter{Instructions du cardinal Lavigerie}

\mn{acheter livre de référence}

\section{Cardinal de Lavigerie}

A marqué le siècle. 

\paragraph{Terre de rivalité} A la fois dans l'Eglise universelle, et les missions en Afrique du Nord et l'Afrique équatoriale (sud soudan, Tanzanie, Burundi, jusqu'au Kenya). Rivalité entre Belge, Allemagne et Angleterre. 

\paragraph{Un basque} tétu, déterminé. Commence ses études à Bayonne, puis St Sulpice, études au Carmes. Erudition en histoire (premiers siècles du Christianisme) et son attachement à l'Eglise universelle.

\paragraph{1856 Oeuvre des Ecoles d'Orient} Ecole au Levant (Egypte,...). Il va faire un voyahe au Levant. Choc de voir le travail dans les hopitaux des femmes.
\begin{quote}
    Rôle des femmes dans le monde musulman : bien vivre ensemble en Orient.
\end{quote}

\paragraph{1861 - nommé à la Rote} Second Empire, il est nommé à la Rote (états pontificaux). Relation amicale avec Léon XIII.

\paragraph{1863 - Evèque de Nancy} Travailleur redoutable. il fonde une congrégation : \textit{les filles de charité de Notre Dame}. 

\paragraph{Comboni} Il rencontre le Père Comboni, \textit{l'Afrique doit être évangélisé par les Africains}. 

\paragraph{1866 - Evèque d'Alger} On imaginait qu'il ferait carrière au Vatican. Il étonne en acceptant d'aller à Alger \mn{Songe à Tours auprès du tombeau de Saint Martin}. Mac Mahon : vous n'y allez pas pour convertir les musulmans.
Lavigerie arrive dans un moment de grande famine. Il crée  :
\begin{itemize}
    \item les pères blancs.
    \item Soeurs blanches, agricultrices et hospitalières
    \item les frères agriculteurs
\end{itemize}

Sa conception de la mission, c'est Saint Martin, vivre au milieu d'une population qui ne partage pas sa religion. Et s'inspire aussi des bénédictins. Frappé par l'évangélisation des bénédictins : composante spirituelle (dans le monastère), sociale (village, autour du monastère). 

Il va fonder des villages chrétiens autour des orphelins. Les pères blancs, des médecins.

\paragraph{traverser le Sahara} Il envoie en 1876, trois missionnaires avec des touaregs. Les touaregs vont assassiner les trois missionnaires. il va alors profiter du canal de Suez pour envoyer des missionnaires par bateau du côté oriental de l'Afrique : Livingstone,...
En 1878, première caravane qui part dans l'Afrique équatoriale. 

\paragraph{Recréer l'Eglise de Saint Augustin} Evèque d'Hippone. Il reçoit la responsabilité des missions d'Afrique. Projet faramineux, que s'il a la responsabilité de l'évangélisation de cette région \sn{alors qu'elles étaient sous la responsabilité des spiritains \mn{\href{https://www.spiritains.org/qui-sommes-nous/notre-histoire/}{Libermann et les spiritains}}}.

\paragraph{Zanzibar} lieu d'esclaves. Les trafiquants se font de vrais royaumes. Vont s'opposer : les missionnaires, spiritains et les missionnaires anglais (Livingstone).

\paragraph{Méconnaissance de la région de Lavigerie} Préparer les missionnaires à toutes les possibilités, y compris le martyr. Dangereusité de ces voyages\sn{voir Stanley}.
Les pères blancs vont lutter contre l'esclavage. Ils font venir les femmes pour s'occuper des personnes rachetées.

\paragraph{mort en 1892} Lire les funérailles de Lavigerie : funérailles nationales. La troisième république va changer la règle napoléonienne qui empêchait l'appropriation des terres indigènes par les métropolitains. Désormais, appropriation des terres et  réaction anti-française. 


\subsection{Introduction au texte}

\paragraph{Instructions aux trois premiers pères blancs} le 21 avril. Instruction mars 1878. 

\paragraph{Traité de Berlin 1884} répartition de l'Afrique entre Anglais, Belges, allemands et Français.

\section{Analyse du texte}

\paragraph{solidité de la vocation missionnaire} Vous n'êtes pas un voyageur, un robinson. 

\paragraph{mportance de l'obeissance} Un électron libre.  Lettre de Saint Ignace de Loyola, référence de tous les ordres du XIX. Il faut écouter la règle, règle de vie, pour nourrir le missionnaire et l'esprit de famille de la congrégation. 


\paragraph{Ce sont des instructions à des pionniers} pret à affronter toutes les épreuves. Donc, insiste sur les vertus.

\paragraph{Rachete des esclaves, formés à Malte pour devenir médecin} Grandes personalités dans les missions des Pères Blancs.

\paragraph{Argent} Cf \href{https://fr.wikipedia.org/wiki/Pauline_Jaricot}{Pauline Jaricot} (Lyon Fourvière) :  les missions étrangères de Paris ont de sérieuses difficultés financières. Pour récolter de l'argent, Pauline et ses réparatrices fondent une association structurée en dizaines, centaines, mille, chacun devant donner un sou par semaine7 pour la propagation de la foi chrétienne. C'est en 1822 que cette association devient officiellement l'Œuvre pontificale de la propagation de la foi.
L'œuvre jouera un rôle de première importance dans le développement du mouvement missionnaire français au xixe siècle.  À la fin du xixe siècle, l'œuvre sera présente dans tous les pays de la chrétienté.

\paragraph{Islam vs relgiions africaines} Rester dans le divin africain : miracle. 


\paragraph{Ius Commissionis} Un important effort missionnaire est constaté sous le pontificat de Grégoire XVI. Par le bref apostolique \href{https://fr.wikipedia.org/wiki/Multa_praeclare}{Multa praeclare}, le pape libère les territoires missionnaires du contrôle du 'Padroado' portugais. Par la 'Propaganda Fide', dicastère romain pour l'évangélisation, il relance le travail missionnaire à partir de 1840.

Des missionnaires sont envoyés auprès des Indiens d’Amérique du Nord, tandis que de nouveaux diocèses sont créés, aux États-Unis. Sur le continent asiatique, la Chine et l’Inde mobilisent l’effort des congrégations religieuses. En Océanie, la prise de possession des archipels polynésiens par les puissances européennes favorise l’élan missionnaire. Le continent africain, notamment l’Abyssinie, est également l’objet de l’intérêt du souverain pontife.

\textit{Ius Commissionis}   

\paragraph{Question sur la vision hierarchique} Evangélisation par le ruissellement des têtes. 


\paragraph{Questions des différentes sociétés protestantes} règle de prudence. Marquer une distance pour éviter les problèmes. Mais de fait, le congrès d'Edimbourg\mn{\href{https://museeprotestant.org/notice/la-conference-missionnaire-mondiale-edimbourg-1910/}{Le Congrès d'Edimbourg} a eu lieu en 1910 est une conférence missionnaire mondiale, et été considérée par certains comme le « berceau » du mouvement œcuménique. Elle a revêtu une importance capitale pour les Églises anglicanes et protestantes. Ce fut le premier aperçu d’une « Église chrétienne vraiment mondiale ».} à l'origine du mouvement oecuménique, vient de l'observation du contre témoignage que donne la division des chrétiens dans le cadre des missions.

\paragraph{l'expérience missionnaire} peut avoir un va et vient et influencer les évolutions de pastorales en Europe. Ici, on voit le sujet de la polygamie. p 84 : "contre la faiblesse commune" et le jansénisme.


\paragraph{contexte anti-religieux} de la troisième république. La Raison (franc maçon) triomphera contre le "milliard des catholiques", "les banquiers protestants" et les "juifs capitalistes". Contexte très dur entre 1880 et 1904. Expulsion. Combes, Ferry.
il faudra attendre 1914 pour que cela se calme.



