
\chapter{Introduction}

\mn{Catherine Marin / Xavier Gué}

Vers une théologie chrétienne de l'Islam : était le thème des précédents séminaires.
Maintenant, de l'histoire à la théologie.
Partir des textes du magistère, et faire un travail inductif sur la théologie sous-jacente.
\begin{Synthesis}
Quelle est notre pratique missionnaire aujourd'hui ?
\end{Synthesis}

Un séminaire, c'est un travail universitaire, si tout le monde participe.

\section{Pourquoi il faut repenser la catégorie de mission ? Entre crise et résurgente}

\subsection{des crises}

Pendant des siècles, 
\begin{quote}
Le cadre théorique était simple : Mt 28, 18-20, "allez et faites des disciples". 
\end{quote}

Bosch : 
\begin{itemize}
\item propagation de la foi
\item Emergence du royaume de Dieu
\item nouveaux croyants
\item nouvelles Eglises
\end{itemize}

\begin{Def}[mission]
\textit{missio } : Dieu envoie.

\end{Def}
Peu à peu, c'est l'Eglise qui envoie. Telle une conquete. Cette conception est entrée en crise au XX sous des coups internes et externes : 
\begin{quote}
depuis la fin de Vatican II, on assiste à de tels changements... qu'il est difficile d'imaginer une stabilité de l'article
82 - article \textit{théologie de la Mission} dans \textit{catholicisme}
\end{quote}

\paragraph{les facteurs de la crise} 
\subparagraph{la décolonisation} la mission comprise comme \textit{domination}. L'idée de richesse à la source de la colonisation se transforme en \textit{civilisation} et donc on apporte aussi la religion, partie de la civilisation Européenne.

\paragraph{Vatican II}
\begin{quote}
On peut être sauvé sans être baptisé, par l'esprit saint, être associé à la mort et resurrection du Christ d'une certaine manière.
GS 65
\end{quote}

Les missionnaires apportaient le salut (FX sauvant les âmes). 

\subparagraph{l'esprit saint précède la mission}

\subparagraph{Notra Aetate} Respect des autres Traditions religieuses. Aucune référence au magistère. \textit{inattendu.}

\begin{Synthesis}
Le christianisme était pensé comme universel, c'est à dire adapté à toutes les cultures.
supériorité du Christianisme prouvé par son succès (autoréférence ?).
\end{Synthesis}
\NA change la vision. Fin de l'optimisme où le monde deviendrait chrétien. 

\subparagraph{On passe de la figure du missionnaire à l'Eglise mission} Ad Gentes  : l'origine de l'Eglise est missionnaire car le Christ est missionnaire. 

\paragraph{le décret \textit{Ad Gentes}} \href{https://www.vatican.va/archive/hist_councils/ii_vatican_council/documents/vat-ii\_decree\_19651207\_ad-gentes\_fr.html}{Ad Gentes}, Un texte de compromis, avec des tensions internes. Demande des pères de lier organiquement le décret des missionnaires à \LG. Texte publié en 1965. Tendance à spécialiser les missions aux pays de mission. Alors que d'autres voulaient aussi inclure l'évangélisation aux défavorisés.



\section{Un certain retour missionnaire}

\subsection{La nouvelle Evangélisation} 
Sous l'impulsion de Paul VI et Jean-Paul II, avec deux évènements : la création du Conseil Pontifical pour la Nouvelle Evangélisation, en 2012, et le synode sur la nouvelle Evangélisation.

\paragraph{Nouvelle Evangélisation} Le terme Évangélisation est récent dans l'Eglise, d'abord utilisé pour désigner les Luthériens. L'Eglise préférait le terme de \textit{mission}. Le concile Vatican I n'emploie qu'une seule fois évangélisation. Alors que Vatican II l'utilise régulièrement.

\paragraph{Exhortation dans le monde moderne - 1975} 
\begin{quote}
En effet, \ldots des temps nouveaux d'évangélisation.
\end{quote}

A Nova Uta, Jean-Paul II : la nouvelle croix de bois, levée... En ces temps nouveaux, l'évangile est de nouveau annoncé. La nouveauté vient de la nouveauté des Temps.
\textit{Redemptoris Mission} les pays de tradition chrétienne ont perdu des chrétiens vivants avec la sécularisation. Card. Suard, 1947 : Évangélisation nécessaire dans le monde ouvrier.


\subparagraph{François abandonne le terme Evangélisation pour Mission} dans \textit{Evangelii Gaudium}

\begin{quote}
\textsc{III. La nouvelle évangélisation pour la transmission de la foi}

14. À l’écoute de l’Esprit, qui nous aide à reconnaître, communautairement, les signes des temps, du 7 au 28 octobre 2012, a été célébrée la XIIIème Assemblée générale ordinaire du Synode des Évêques sur le thème La nouvelle évangélisation pour la transmission de la foi chrétienne. On y a rappelé que la nouvelle évangélisation appelle chacun et se réalise fondamentalement dans trois domaines.[10] 
\begin{itemize}
\item En premier lieu, mentionnons le domaine de la pastorale ordinaire, « animée par le feu de l’Esprit, pour embraser les cœurs des fidèles qui fréquentent régulièrement la Communauté et qui se rassemblent le jour du Seigneur pour se nourrir de sa Parole et du Pain de la vie éternelle ».
 [11] Il faut aussi inclure dans ce domaine les fidèles qui conservent une foi catholique intense et sincère, en l’exprimant de diverses manières, bien qu’ils ne participent pas fréquemment au culte. Cette pastorale s’oriente vers la croissance des croyants, de telle sorte qu’ils répondent toujours mieux et par toute leur vie à l’amour de Dieu. 
\item  En second lieu, rappelons le domaine des « personnes baptisées qui pourtant ne vivent pas les exigences du baptême »,[12] qui n’ont pas une appartenance du cœur à l’Église et ne font plus l’expérience de la consolation de la foi. L’Église, en mère toujours attentive, s’engage pour qu’elles vivent une conversion qui leur restitue la joie de la foi et le désir de s’engager avec l’Évangile.

\item  Enfin, remarquons que l’évangélisation est essentiellement liée à la proclamation de l’Évangile à ceux qui ne connaissent pas Jésus Christ ou l’ont toujours refusé. Beaucoup d’entre eux cherchent Dieu secrètement, poussés par la nostalgie de son visage, même dans les pays d’ancienne tradition chrétienne. Tous ont le droit de recevoir l’Évangile. Les chrétiens ont le devoir de l’annoncer sans exclure personne, non pas comme quelqu’un qui impose un nouveau devoir, mais bien comme quelqu’un qui partage une joie, qui indique un bel horizon, qui offre un banquet désirable. \textit{L’Église ne grandit pas par prosélytisme mais « par attraction ».}[13]
\end{itemize}

\end{quote}
\begin{Synthesis}
Toute Eglise est missionnaire. Pour François, nous sommes \textit{disciple missionnaire} car c'est intrinsèque.
\end{Synthesis}
\mn{Travailler le thème de disciple missionnaire en Cvx ? influence Jésuite ? Matteo Ricci ? Question}

\paragraph{L'Evangélisation de Rue} sous l'influence des milieux anglo saxons, on va aborder l'évangélisation directe, avec une approche marketing\mn{cf le Congrès Mission : technique marketing au service de l'annonce. Mais quelle théologie ?}.

\paragraph{Conclusion} Regarder l'approche historique pour mieux comprendre théologiquement ce qui est en jeu.

David Bosch. p19. 

\section{L'approche de la prédication}

\subsection{Histoire}
Adapte la mission à la situation économique : les moines n'utilisaient pas le terme de mission. Saint Martin de Tours a évangélisé parce que c'était normal \mn{paysan et paien même racine}.


\paragraph{Comment passe-t-on du profane au sacré} {La vallée des Saints : Sacralisation} du passage du profane au sacré. 

\paragraph{Professionnalisation de la \textit{mission} avec les jésuites}  au XVI. 
Lettres de François-Xavier. On passe d'un baptême d'abord en Inde au passage, au Japon, par le préalable de la compréhension de la culture.

\paragraph{Colonisation} Les Eglises issues de la colonisation se sont appropriées le christianisme avec leur caractère propre. Cf Eglise du Vietnam.


\paragraph{Va et vient historique}
Dans le temps, on va s'inspirer du passé pour réfléchir à la manière d'être missionnaire. Par exemple, les \textit{réductions} jésuites en Amérique du Sud. Les Chrétiens prenaient en charge l'aspect religieux et sociale et économie. Ces réductions se sont inspirées des moines. Avec le village autour des moines : va et vient tout simple autour d'un monastère \textit{pour la plus grande gloire de Dieu}.

Le Cardinal de la Vigerie : \textit{"il faut faire comme les moines du XIX" }: équilibre de vie : coeur spirituel, social et économique. 

\mn{
Deux mille ans d'évangélisation et de diffusion du christianisme Broché – Illustré, 20 janvier 2022
de Jean Comby (Auteur), Claude Prudhomme (Auteur)}

\section{Indications méthodologiques}

\paragraph{Sources}
\paragraph{Mise en récit} cf Ricoeur : trois phases 
\begin{itemize}
\item j'ai découvert des choses
\item je mets en ordre la vie d'un missionnaire
\item récit en lui-même, récit partiel car on n'a qu'une vision partielle de ce qui s'est passée.
\end{itemize}

\begin{Ex}
Surin : 10 ans sans publier. ne pas extrapoler.
\end{Ex} 

\begin{Def}[Story / History]
En Français, on n'a que histoire.
\end{Def}

Il y a ce qui est écrit et ce qui n'est pas écrit. Par exemple, on sait qu'il s'est passé quelque chose d'important près du lieu de mission. le missionnaire n'en parle pas. pourquoi ?

\paragraph{Historicité d'un fait}
Tout être humain porte une histoire. Il faut toujours être vigilant sur notre analyse de l'histoire. On ne peut pas juger de ce qui s'est passé il y a trois cents ans. Michel de Certeau : \textit{l'écart}. Il faut éviter le jugement car on fait le court circuit du \textsc{contexte}, politique, historique, spirituel.

\begin{Ex}
Missionnaires au XIXeme, marqué par l'outrage de la Révolution et la nécessaire \textit{réparation}. Les missionnaires sont beaucoup plus rigides au XIX qu'au XVII / XVIII. 
\end{Ex} 


\paragraph{Témoignage}
\begin{quote}
18 Jésus s’approcha d’eux et leur adressa ces paroles : « Tout pouvoir m’a été donné au ciel et sur la terre.
19 Allez ! De toutes les nations faites des disciples : baptisez-les au nom du Père, et du Fils, et du Saint-Esprit,
20 apprenez-leur à observer tout ce que je vous ai commandé. Et moi, je suis avec vous tous les jours jusqu’à la fin du monde. » Mt 28
\end{quote}

\paragraph{Annonce} lié au témoignage : individuel mais communautaire. Des congrégations religieuses. La communauté où est né le missionnaire. 

Henri René Marrou
\begin{quote}
le Christianisme ne crée pas les civilisations; il les sauve
\end{quote}
Et de fait, des grammaires de 30 langues africaines; arts.  


\subsection{Que va-t-on englober dans l'histoire des missions ?}

\paragraph{Une professionalisation depuis le XVI}

\paragraph{Mais battue en brèche par les femmes et les acteurs locaux}

\paragraph{Histoire de l'envoi / histoire de la réception} Il y a aussi le pouvoir local : si un Roi local ne veut pas de missionnaires, il n'y aura pas de missionnaires. cf un Roi en Guinée. 

\paragraph{Histoire du christiansime}

\paragraph{retour entre le témoignage des missionnaires revient sur les vieilles terres chrétiennes} On se nourrit dans ces vieilles terres.

Marrou : écrire l'histoire, sortir de soi. Prise de conscience de l'écart : ne pas juger ce qui était par rapport à ce qui était. On fait de l'histoire pour mieux comprendre le présent.
\begin{quote}
Christianisme comme histoire; pédagogie 
Danielou
\end{quote}

\section{A lire}

Hugues Didier - Correspondance de François Xavier , 1532 -1552, Desclée de Brouwer. 

