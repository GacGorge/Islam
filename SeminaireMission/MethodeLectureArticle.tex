\chapter{Méthodologie~: Pour lire et exposer un article}

\paragraph{Le plan des exposés}

\paragraph{11. Le temps de l'investigation}

\paragraph{111. Première lecture «~affective~» ou «~réactive~»}

Vous parcourez le texte, tentez de saisir d'où il part et où il va en
comparant l'introduction et la conclusion et en repérant les éléments
d'information donnés par les sous-titres, ce que vous aimez ou non à
première vue. Vous relevez votre premier intérêt et vos incompréhensions
ou désaccords, dans l'idée que, très généralement l'auteur a en fait
déjà répondu à vos objections.

Après cette première approche, vous pouvez vous engagez dans une étude
approfondie du texte.

\paragraph{112. Lecture pas à pas}

Vous commencez par numéroter les paragraphes page par page en adoptant
la convention suivante~: la première ligne en haut d'une page marque
toujours le commencement d'un paragraphe, même si elle commence au
milieu d'une phrase.

Vous lisez paragraphe par paragraphe et vous notez le contenu essentiel.
Ce travail que tous doivent faire pour préparer la séance n'est pas
nécessairement destiné à être restitué, mais est essentiel pour préparer
l'étape suivante.

\paragraph{113. Etablissement de la problématique de l'auteur}

Après la phase de lecture pas à pas, vous construisez la question à
laquelle l'auteur s'affronte~:

\begin{itemize}
\item
  pourquoi l'auteur se bat-il~?
\item
  quel problème essaie-t-il de régler, d'éclairer~? En général, il n'est
  pas difficile de trouver exposée la problématique en toute lettre dans
  le texte lui-même -- parfois même de manière redondante~;
\item
  en fonction de quel contexte culturel, social, culturel, économique,
  politique l'auteur construit-il sa problématique~? Notez les
  événements déterminants auxquels il se réfère et les auteurs qu'il
  évoque -- plus ou moins explicitement -- comme ses alliés ou ses
  adversaires -- et le cas échéant, renseignez-vous sur eux.
\end{itemize}

\paragraph{114. Etablissement de la thèse}

En fonction de la problématique de l'auteur, vous établissez la thèse de
l'article ou du texte étudié~: quelle solution l'auteur apporte-t-il à
sa question~? quelle perspective établit-il~? Là encore, il n'est pas
difficile de trouver cette thèse exposée de manière explicite dans le
texte lui-même.

\paragraph{115. Etablissement de l'argumentation}

Dans une seconde lecture, vous reconstruisez la logique argumentative en
fonction de laquelle l'auteur établit sa thèse.

\paragraph{116. Vérification}

Vous sélectionnez l'un ou l'autre extrait du texte (au maximum une page)
qui vous apparaît particulièrement décisif. Vous les ferez travailler à
l'ensemble du groupe, afin de vérifier dans les faits la justesse de
votre exposé.

\paragraph{117. Appréciation critique}

Enfin, vous concluez en, évaluant si l'auteur a atteint son but, dans
quelle mesure sa question était bien posée, sa thèse éclairante et
originale, sa démonstration convaincante. Et vous faites valoir l'apport
que votre exposé constitue pour la recherche du séminaire.

\paragraph{12. Le temps de l'exposition (20 mn)}

\paragraph{121. Biographie de l'auteur}

Efforcez-vous de présenter rapidement en quelques lignes l'auteur de
l'article étudié, et si possible, de situer plus brièvement les auteurs
avec lesquels il débat.

\paragraph{122. Bibliographie}

Notez l'ensemble des textes et références auxquels vous avez eu recours
pour préparer l'exposé, y compris vos sources pour la biographie et les
sites internet visités.

\paragraph{123. La problématique (cf. supra) (ce n'est pas un résumé).}

\paragraph{124. La thèse (cf. supra)}

\paragraph{125. L'argumentation (cf. supra)}

Vous pouvez vous appuyer sur vos notes de lecture pas à pas et sur le
texte sélectionné pour la vérification.

\paragraph{123. Appréciation et discussion critique}

\paragraph{2. Les comptes-rendus (ou synthèses)}

Les comptes-rendus permettent de garder une mémoire des avancées de la
recherche commune.

L'étudiant chargé du compte-rendu de la séance veillera à relever de
manière synthétique~:

\begin{itemize}
\item
  les acquis recueillis lors de la séance de travail~: rappel de la
  problématique, de la thèse, principales conclusions~; principaux
  éléments du débat~; critiques adressées à l'auteur~;
\item
  les acquis pour la suite du séminaire~: nouvelle manière de
  questionner, réponses obtenues à des questions antérieurement
  soulevées, découvertes d'horizons nouveaux etc.
\end{itemize}

Le compte-rendu prendra la forme d'une note de deux pages maximum qui
sera distribué à l'ensemble des participants au séminaire et rapidement
discuté en début de la séance suivante.


\section{Présenter un texte lors d’un séminaire}


\begin{Prop}
Attention !
-	La présentation ne doit pas dépasser 20 mn
-	Il ne s’agit pas de faire le résumé d’un texte que tous les participants ont déjà lu
\end{Prop}


\paragraph{1.	Présentation et biographie de l’auteur 
}
Efforcez-vous de présenter rapidement en quelques lignes l’auteur du texte étudié, et si possible, de situer plus brièvement les auteurs avec lesquels il débat.

Il ne s’agit pas de reprendre Wikipédia, mais plutôt de montrer en quoi l’auteur et son parcours nous aident à mieux comprendre le texte étudié.

\paragraph{2.	Bibliographie}

Eventuellement, notez l’ensemble des textes et références auxquels vous avez eu recours pour préparer l’exposé, y compris vos sources pour la biographie et les sites internet visités. 

\paragraph{3.	Nature du texte}

Quel genre de texte est-ce (article, lettre, traité, sermon, etc..) ? Quel est son genre littéraire ?

\paragraph{4.	Le contexte historique et textuel}

Situez la production du texte dans son contexte historique (date de la publication). A quelle occasion a-t-il été rédigé (suite à quel événement), dans quel contexte culturel et social, etc. ? A qui est-il adressé ?

Situez le texte dans son environnement littéraire, s’il est extrait d’un ouvrage ou d’un corpus. Présentez rapidement l’ouvrage, indiquez ce qui précède et ce qui suit le texte choisi, etc. Est-ce une traduction ? \mn{s'il y a 700 pages, dire les chapitres avant et après}

\paragraph{5.	Si c’est un article de théologie}


	\subparagraph{5.1 La problématique }

Déterminer la problématique en vous inspirant des questions suivantes :
Après la phase de lecture pas à pas, vous construisez la question à laquelle l’auteur s’affronte :
-	pourquoi l’auteur se bat-il ?
-	quel problème essaie-t-il de régler, d’éclairer ? En général, il n’est pas difficile de trouver exposée la problématique en toute lettre dans le texte lui-même – parfois même de manière redondante ;
-	en fonction de quel contexte culturel, social, culturel, économique, politique l’auteur construit-il sa problématique ? Notez les événements déterminants auxquels il se réfère et les auteurs qu’il évoque – plus ou moins explicitement – comme ses alliés ou ses adversaires – et le cas échéant, renseignez-vous sur eux. 

\subparagraph{5.2	 La thèse ou plutôt hypothèse}

En fonction de la problématique de l’auteur, vous établissez la thèse (ou hypothèse) de l’article ou du texte étudié : quelle solution l’auteur apporte-t-il à sa question ? quelle perspective établit-il ? Là encore, il n’est pas difficile de trouver cette thèse exposée de manière explicite dans le texte lui-même. 

\subparagraph{5.3	 L’argumentation }


Présentez la logique argumentative en fonction de laquelle l’auteur établit sa thèse (passe de l’hypothèse à la thèse vérifiée).

\paragraph{6.	Si c’est une lettre, une instruction ou un autre type de texte}

\subparagraph{	6.1 L’objet et l’effet recherché}

Déterminer l’objet du texte et, éventuellement l’effet recherché (sur le lecteur).

\subparagraph{6.2	 L’argumentation}

Comment le texte mobilise-t-il des arguments, des images et des exemples pour toucher son lecteur ?

\subparagraph{6.3	 Valeur théologique du texte}

Quels sont les présupposés théologiques du texte ?


\paragraph{7.  Appréciation et discussion critique }

Critiquez le texte : l’argumentation est-elle convaincante ? Pourquoi ?


\subsection{de l'histoire à la théologie}
il est important de répondre à la question : quelle théologie ? vocabulaire employé ? alors qu'il n'est plus employé. Le sens des mots. ex :  Indigène, personne né aux Indes.
