\chapter{le Saint dans le soufisme}


\subsection{
La théorie 
Actuelle }

Des 
Mollahs \sn{Des savants}
 ci après 
Est ce que tous les shiites 
Font ça ? Je ne sais pas, mais je sais que c'est toujours la théorie 
Que ce terme 

\paragraph{Origine du terme soufisme}
Soufisme qui  veut dire laine, on pense que.
Les premiers doivent 
être les premiers musulmans et refoulés 
Voilà   c'est l'origine du terme 

\paragraph{pas marginal} . le 
Soufisme, c'est pas un phénomène 
Marginal de l'islam 
Il fait partie 
Intégrante de cette religion et trouve ses fondements dans le Coran et dans l'exemple du Prophète 

\paragraph{deux rapports à Dieu}
\begin{itemize}
    \item  . dans l'islam, il y a deux façons d'envisager le rapport à  Dieu 
Un rapport juridique  C'est à  dire se fie à  la loi 
Et à  la charia 
Et qui  implique de se méfier des 
Expériences subjectives 
\item 
Et puis il y a un 
Rapport 
Mystique qui , cette fois, implique une l'expérience personnelle 
\end{itemize}


\paragraph{rapport mystique}
 à  côté de l'islam, des 
Oulémas et des docteurs de la loi, a 
Développé un autre 
Islam 
Un islam ?   c'est cette 
Stabilité qu'il vise à  établir un autre type de 
Relation entre l'homme  

\paragraph{Dieu lointain } Allah, qui  est un Dieu très lointain puisque, comme le dit le Coran :
\begin{quote}
    Dieu peut bien se passer des univers  - sourate trois, verset 97 
\end{quote}
 
\begin{Def}[soufisme]
    C'est une voie d'accès directe à  Dieu qui  repose sur un ensemble de 
Pratiques et qui  implique 
L'amour  à‡a, c'est une notion essentielle dans le soufisme 
L'amour, la prière, la méditation, 
Et le renoncement au monde mental
\end{Def}
 
 \paragraph{géographiquement large}
Dans presque toutes les sociétés musulmanes, sauf peut être 
En Arabie Saoudite  

\paragraph{manifestation}
 . il se manifeste de différentes façons 
\begin{itemize}
    \item La première, c'est d'être savant c'est l'adhésion à  des théories mystiques,  des théories qui  sont 
qui  peuvent être en contradiction 
avec les théories qui  sont produites par les oulémas. qui  ont développé des théories qui  sont là  pour expliquer 
ce Coran mystérieux 
Ou même expliquer qui  est le 
Prophète   une vision ésotérique 
du Coran 
\item Le deuxième niveau, c'est être disciple, c'est l'appartenance à  un ordre mystique
\item la dévotion 
Populaire au sein 
à  la fois des saints qui  sont vivants ou qui  sont des 
Saints morts du passé.  En cherchant, c'est l'intercession du saint pour résoudre des problèmes quotidiens 
\end{itemize}


\paragraph{au début ascétisme}   
Né dès les premiers siècles de l'hégire et à  cette époque de cette époque, il y a des hommes qui  sont 
Reconnus pour leur 
Ascétisme . 

\paragraph{décadence des moeurs}
   de leurs contemporains et de 
Califes .
Une façon de rejeter la décadence 
des moeurs.   . les premiers soufis, on les appelle 
Ascètes ou pauvres en dieu  - \textit{
de fakirs 
} L'idée de refuser les désirs de l'homme à  Dieu, l'idée de détachement 



\paragraph{le premier ascète : Ali} gendre et le cousin du prophète 
Il est reconnu comme le premier des ascètes, le premier des mystiques.

Le deuxième grand mystique est \textbf{Al Basri},  au VIII siècle et c'est lui qui  va proposer la première 
Interprétation Spirituelle 
du Coran.
\sn{Il est le premier à  parler dans ses écrits 
à  la fois de la crainte de Dieu et du désir de l'homme de s'en rapprocher }

Le troisième grand 
Mystique, c'est \textbf{Moubarack} 
 
On est à  la fin du VIII  siècle.  C'est lui qui  va écrire le premier 
Traité pour essayer de définir ce qu'est l'ascétisme et l'adoration 
de Dieu.
Et les deux  Le premier a décrit 
Une pratique qui  est spécifique 
Et une pratique spécifique à  la mystique qui  est le dhikr.  
\begin{Def}[Dhikr]
    Rappel et ré remémoration. c'est le rite soufi par excellence.
\end{Def}
\paragraph{Moubarak} Chacun des 
Ordres 
Soufis 
va proposer  sa propre interprétation. Moubarak, le premier parle de cet acte, de cette pratique, cette pratique. Il parle 
de la crainte de Dieu 
Et du désir de 
s'en rapprocher.

\paragraph{Abou}

\paragraph{POesie, mode d'expression du soufisme}
Tout 

\paragraph{Rabi'a Al-'Adawiyya} 
une des rares femmes célèbres du soufisme, parce qu'il y en a certainement eu d'autres. 
Elle est mort en prison, mais c'est la 
seule 
qui  est restée dans l'histoire.
Rabia,   va introduire 
Un idée de pur 
Amour, un \textit{amour désintéressé que Dieu n'a pas 
exigé }
Et pour lequel il n'aura une récompense par la mort 
Vous comprenez cette idée que c'est une idée qui  est très 
Présente à  partir d'elle dans le 
Soufisme, l'idée d'un amour désintéressé 
Est célèbre 
Pour avoir voulu 
atteindre 
le feu de l'enfer et mettre le feu au paradis. 


\paragraph{l'Hispanie ?} ivresse, extase, anéantissement de Dieu. 
Nouvelle tendance de la mystique qui  repose sur l'expérience de 
L'ivresse.

\paragraph{gens du blâme}
Un malâmati, ou melâmî  (de l’arabe \TArabe{ملامة} 
 , « malâma », blâme, critique) est un soufi qui, par souci de sincérité, va faire exprès d’avoir un comportement presque contraire à ce qu’il est vraiment, même si ça doit lui causer des ennuis et le discréditer publiquement. Cette attitude singulière basée sur le rejet de tout formalisme ou extériorité de la spiritualité se développa à partir du Khorassan (nord-est de l’Iran) au ixe siècle. ‘Abd’l Rahmân al-Sulami (936-1021), qui en fut l'un des principaux protagonistes, explique que « la voie du blâme »
 
Par des comportements 
immoraux, \textit{sans reconnaissance de Dieu} pour se consacrer en 
toute humilité à  
l'amour de Dieu. 
 
\begin{Ex}
Il travaille avec des fakirs, des 
Renonçants au Pakistan, qui  
Vivent dans des cimetières, drogue. et ils portent des tenues qui  
Sont faites de 
Tissu rapiécé. ils ne suivent pas du 
tout  les piliers de l'islam, ni le  ramadan. 
  Tout ce qu'on va leur donner dans la journée à  avoir les nourritures va être redistribués instantanément   
\end{Ex}

\paragraph{On s'éloigne de la Charia mais pas du Coran}
On a différentes 
tendances 
soufis qui  vont coexister, coexistent 
Mais elles se fondent sur 
des éléments 
communs qui  sont la récitation et 
la méditation 
du Coran. Il y a un refus de la 
Charia, mais pas du 
Coran.  
\paragraph{Interprétation ésotérique du Coran}
Il y a l'idée 
que l'ascension céleste du prophète est 
l'archétype de l'expérience spirituelle. 
 le Prophète devient un modèle 
Chez les mystiques et 
Notamment son ascension céleste qui  est une image 
L'archétype de l'expérience 
Spirituelle  Et on avait vu avec le Prophète monte jusqu'au point de Dieu  Qu'est ce qu'être expert ?
Les soufis, et  ce qu'il faut faire débat 
Entre 
Les juristes et les soufis, c'est que les soufis 
Ne respectent pas 
de façon littérale la charia.

\paragraph{Grands maitre du soufisme}pour se lier à l'islam orthodoxe.
\paragraph{Al Halaj} crucifé en Irak 1122. Et il est considéré comme 
le 
Martyr mystique.
Vanté d'avoir réalisé l'union avec Dieu   . il a formulé cette idée d'union avec Dieu en disant :
\begin{quote}
    Je suis le vrai, le vrai et le miracle, 
\end{quote}
et sa Poésie : 
 \begin{quote}
    Je suis 
Devenu celui que j'aime\sn{sous entendu Dieu }, 
Et celui que 
J'aime et devenu  Moi 
\end{quote}
 Par son 
Martyre, il 
Incarne la quintessence de 
l'expérience mystique.  
  
Il y a 
énormément 
de mystiques qui vont se réclamer de Al Hajaj même s'il a laissé des poèmes mais peu 
de textes, ce n'est 
Pas un théoricien.
\paragraph{Al Ghazali} 
C'est un auteur 
qui  va essayer de réconcilier le camp des soufis et celui des juristes.Auteur très profond, prolifique. Dans son \textit{Revivification des sciences religieuses 
},  l'origine de la connaissance, les sciences 
Religieuses 
Et la charia.  Il propose une théorie soufie qui  ne s'oppose pas à  celle des docteurs de la Loi. Il 
Développe 
Une théorie de la connaissance qui  est fondée sur la 
pratique du soufisme. 
Il maîtrise les sciences des juristes, et il essaye 
de faire coincider qui  ce que lui apporte son savoir de 
Juriste. Certainement quelqu'un d'une 
Intelligence 
Supérieure et un 
Gros travailleur. Il propose une théorie 
de la connaissance 
qui  est fondée sur la pratique soufie.

\paragraph{Ibn Arabi}
 mort 
en 1240. 
Le plus grand maître. 
Il va influencer 
tous les soufis après lui. 
  pensée 
complexe et touffue. 
 . on lui attribue 
Même la paternité d'un concept de \textit{monisme ontologique}, \textit{l'unicité de 
L'être 
} Ou unité de l'existence 
 
 
Accusé de remettre en cause 
 l'unicité de Dieu. Il remet aussi l'idée de transcendance : \textit{Dieu est dans tout}. 

\paragraph{la pensée de Al Arabi} 

 
L'anéantissement 
Du moi en Dieu. 
Après avoir atteint ce stade,  le fait de pouvoir rester Dans cet état ou le soi est 
Anéanti et on reste dans une espèce 
D'unité avec Dieu.

\paragraph{être 
Parfait } A chaque époque Dieu envoie un Prophète et des saints. Les Saints vont ressembler au prophète de leur temps : certains à Jésus, des saints qui  vont plutôt ressembler à  Muhammad, des saints qui  vont prendre comme modèle le prophète 
Noé.



\paragraph{Roumi } Il meurt en 1273 . Savant; poète. Pensent 
à travers la poésie. Persan. OUvrage célèbre : Matmadhi. 24 000 vers magnifiques. Le chagrin de l'être aimé. 
C'est ce départ d'un autre mystique 
avec lequel vous avez une relation d'amitié. Quand il meurt; il décrit son 
chagrin. 
\paragraph{danse}
Il va donner naissance aux  \textit{derviches tourneurs} célèbre par sa pratique de la danse extatique. 
  c'est la deuxième  pratique soufie : la danse.
 A  partir du 13e siècle, ordre des  tariqa, qui  veut dire 
La voie, le chemin.

Pour devenir maître, il a parcouru sa propre voie, son propre chemin.
Les disciples 
Devenant à  leur tour un maître 
En tout cas pour les disciples.  

\begin{Def}[silsila]
    Chaîne initiatique qui lie le disciple à celui qui a fondé l'ordre.
    Cette « chaîne initiatique »  est appelé arbre généalogique (\emph{Shajrah'i Tayyaba}).
\end{Def}
\begin{Ex}
    vous êtes 
X et vous êtes lié à  votre maître qui  est lui même lié à  
son maître.
 
On remonte jusqu'à  
Ali 
\end{Ex}



\paragraph{Création des ordres} A  partir du XIIIè  siècle, 
les idées soufies,  ont 
commencé 
à  pénétrer les 
cercles 
des oulémas . Au 17è 
Siècle, les oulémas 
sont moins réfractaires 
Au soufisme.  Il y a plus d'échanges entre les soufis et les oulémas, il y a une vulgarisation des savoirs et  une 
Naissance des ordres. Tous les ordres tiennent leur nom du maître spirituel. Objectif :  l'enseignement d'un maître,  ensemble 
de connaissances ésotériques, des rites et des exercices mystiques.


 
Plutôt des scissions que des fusions dans les ordres pour avoir des ordres     
\begin{Ex}
Shishtia en Asie du Sud, va avoir des sous-groupes.     
\end{Ex}


\paragraph{Les principales confréries}

\begin{itemize}
    \item Naqshbandiyya (Turquie)\sn{des noms proches parce que c'est 
Une façon de faire  
Un substantif d'après 
Un nom d'une personne } : essaye de se faire connaître
    \item la Alawiyya (Algérie) : Le Sheykh Ben Founès. C'est lui qui a fondé les scouts musulmans en France. 
    \item la Boutchichiyya (Maroc) :   Adapte son discours, au lieu de parler de Dieu, on va parler d'énergie divine.
\end{itemize}
\paragraph{Asie du Sud: impact des castes}
En Asie du Sud, on a une société hierarchisée. 
La société s'organise 
 en Asie du Sud, on a une société hiérarchisée  Je pense que vous avez certainement tous entendu 
Parler du système des 
Castes 
Et des 
Intouchables. En haut de 
La pyramide, à  part les 
Descendants du prophète. 
Qu'on appelle autopraxes, 
 qui  suivent, qui  suivent la pratique orthodoxe, la loi 
Coranique, recrutent parmi les hautes castes :  sont reconnus comme honorables. 
Et puis, à  l'opposé, il y a des 
Ordres hétéropraxes, sans loi, qui  ne suivent pas la 
Charia. Ils sont ils recrutés parmi les basses castes 
de la société asiatique.  Ils ont des comportements qui  sont vus comme excentriques, 
usage de drogues, Mendicité, d'itinérance, la pratique 
de la musique 
et de la magie. 


Mais il y a une quarantaine 
d'ordres. Au début du XIXè siècle, sous l'Etat colonial;  un nouvel ordre 
qui  s'est développé 
En Inde qui  s'appelle \textit{Tariqa ou Ahmadiyya.  }
A lancé 
Une guerre sainte contre les sikhs et les 
Anglais, et qui  visait 
à  ramener 
L'orthodoxie des premiers 
Califes en Inde.  

Du point de vue des musulmans, si c'est considéré comme des ordres soufis, même si l'expérience mystique parait éloignée.

\section{Lien entre les hommes au sein des confréries}

Nous proposons ici d'étudier les
liens qui unissent les hommes au sein des confréries. 
Ce type d'institution 
 chaque ordre 
 a ses propres règles et ses 
usages.
\begin{Def}[Haddad]
     règles 
usages soufis
\end{Def}
Il y a une grande variété de règles mais certains invariants.

\paragraph{Maître 
Le disciple et Dieu }
Maître montre la voie pour aller à Dieu au disciples, le \textit{mujhid}, le \textit{désirant}. 


\begin{Def}[taramaq]
    faire des miracles pour des maîtres.
\end{Def}
\begin{itemize}
    \item transmet ses pouvoirs
    \item gère un lieu / loge soufie.
\end{itemize}


Et moi je suis Gambetta  Avec les termes de vocabulaire, le maître 
Est un guide qui  
Montre la bonne voie  C'est Tariqa qui  montre la bonne voie 
Pour parvenir à  
Dieu et montre cette bonne voie par sa guidance 
à  un disciple, à  un disciple qui  est à  la 
Fois 
Appelé 
Le cheminant 
 celui qui  progresse 
Sur le 
Chemin, mais aussi qui  est appelé désirant, mouride, mouride  C'est le désir,  celui qui  désire atteindre Dieu 
 ça, c'est l'esprit du soufisme  Et  
Même si dans l'aménagement 
de Y ou 
Dans 
L'appareil il y a, il y a moins et moins de réflexion et d'amour  Il y a quand 
Même cette relation entre le maître guide 
Pour parvenir à  Dieu 
Allah le cheminant sur son nom, sur son chemin, il doit 
Franchir différents à  différentes étapes et ces étapes à  
Franchir  Elles sont de deux sortes   . là   là , on va rentrer dans 
Un autre système de pensée qui  n'est pas le 
Nôtre   le premier 
Type 
D'étapes 
à  franchir, c'est ce 
Qu'on appelle le al, le al 
qui  est l'état spirituel 
qui  s'installe dans le coeur du novice 
 tout au 
Long de sa formation 
Le novice le désirant 
Le cheminement 
Le disciple 
Mène un combat 
Contre lui même 
Par le coeur grà¢ce à  ce combat et transforme son à¢me 
Et son 
Coeur 
Pour parvenir, dans le but de parvenir à  la 
Fusion en Dieu 
Et à  
Son parcours, il va connaître un certain 
Nombre d'étapes spirituelles qui  s'installent 
Dans son coeur 
 . ces étapes spirituelles 
Suivant les hauteurs sont plus ou moins nombreuses, mais je veux, je peux vous donner des exemples  Par exemple 
Le premier élément 
C'est l'attention 
Le deuxième état, c'est la certitude  Le troisième, c'est l'espérance   c'est ces idées  Dans d'un état, ça renvoie, à  des notions comme celle ci la tension, la certitude, l'espérance 
 ça 
C'est la première sorte 
D'étapes à  franchir  C'est le crà¢ne 
L'état spirituel  Le deuxième, c'est ce 
Appelle la 
Primaire 
éventuelle 
Et le monde ?
Il vous ?
 moins de phénomènes  Mais, car 
à‡a revient bien, l'idée d'une 
Station, une 
Demeure spirituelle ou 
Dieu 
établit le monde, le novice 
Pour un temps, pour un moment 
 selon les auteurs, là  aussi, il y a pour différents 
Types de stations   . pour certains auteurs, il y en a sept  Pour deux 
Il y en a plus de 1000   . des stations peuvent donner les types de stations  Comme il y a le repentir, le respect de la charia, le renoncement 
La pauvreté, la patience, l'abandon de Dieu 
Tout ça, c'est des stations 
 . cette conception 
Des étapes 
Affranchie 
Elle 
Trouve sa résonance avec 
Le voyage 
Spirituel de nomades  Le miracle 
Et  . ? On voit l'idée de 
L'échelle qui  
Conduisit 
Muhammad sur et lui même Muhammad  Durant son cheminement 
Il s'était guidé 
Par la voie 
Symbolique  C'est l'idée sur la 
Voie, excusez 
Moi de laisser guider 
Sur la voie par son maître  Si Dieu va vers vous, suivez  . ce maître, ce guide 
Il est aussi un spécialiste, un spécialiste des sciences 
Et 
Des théories, et 
 il enseigne  Il enseigne 
Ses 
Connaissances 
Des livres, des conseils et à  travers son exemplarité 
Il est une source 
Normative 
Pour le comportement 
de ses disciples   . il est lui 
Même 
Investi 
Du pouvoir du saint fondateur de son ordre 
Pouvoir qui  provient d'aller  . 
Ce pouvoir s'appelle ce pouvoir 
Il reçoit et 
Il reçoit plusieurs mains et notamment le pouvoir de faire 
de faire des miracles 
Ce qu'on appelle Karama, c'est 
Les prophètes 
Aussi ont la capacité de faire 
Des miracles, mais ce n'est pas le 
Même terme en arabe 
qui  est utilisé 
 ce maître 
Il va exercer auprès ses disciples 
En fonction de l'opinion et en dirigeant 
Des rites propitiatoire 
Rituels pendant lesquels il assure le transfert 
Matériel 
de sa puissance 
à  ses disciples 
Cette puissance 
qui  aideront ses disciples 
à  
Poursuivre leurs objectifs 
Ce maître 

\paragraph{Loge soufie}
le maître est le 
Propriétaire d'un sanctuaire du sanctuaire où se trouve le saint 
qui  a fondé à  l'origine de cet ordre. 
\begin{Def}[hanka]
    C'est un lieu, hospice,
  comme un monastère vec les mêmes règles 
\end{Def}


\paragraph{Amis / Affiliée }. C'est ce qu'on appelle l'ami 
Ou le fidèle, c'est l'ami ou le fidèle. 
Et eux, ils ne font pas l'objet 
D'un rite d'initiation. 
Ils se contentent de recevoir la puissance du maître en en venant le visiter.
La plus courante courante

\paragraph{Murid / fakir} \textit{désirant}. Et le pauvre, le pauvre, une sorte de fakir. Le fakir qu'on appelle aussi l'élève. 

\paragraph{Murid}
Ils font un serment d'allégeance avec un maître spirituel 
Mais ils restent dans le monde. On peut les considérer comme 
des oblats et il reconnaît le maître comme directeur 
Spirituel.

\paragraph{Fakir}
Par contre, dans la 
catégorie 
de l'élève ou du faqih, il faut voir là  
Ou sont réunis les 
Renonçants. Ils font 
le voeu de consacrer 
leur vie au but du fondateur de l'ordre. \textit{Initiation} :
Il forme le voeu pour toute vie. C'est là  ou on vit dans des hospices ou dans des cimetières  Comme je l'ai dit, et on obéit 
  ce que son maître 
Désire 
Père, on en perd 
Sa liberté et on est soumis 
à  la volonté de son maître   . habits
de formules 
Liturgiques, de 
Formules incantatoires 

\paragraph{fous rassouf} il y a une 
    dernière catégorie 
 
qui  n'a pas de maître 
qui  n'est pas initié, mais dont le 
cmportement 
Indique qu'il est proche 
de Dieu. 
Rabbi traduire 
Pour ramener à  Dieu oui 

\paragraph{Pourquoi rentre t on dans un ordre}
Membre d'une entité, un ordre qui  devient par choix personnel.   Mais on observe du point de vue sociologique que le plus souvent c'est 
un héritage qui  se 
transmet de père en fils 
\begin{Ex}[Asie du Sud]
  C'est que souvent les aspects, les affiliations et les ordres sont multiples  On appartient à  
Plusieurs ordres à  la fois et aucune 
Condition de religion ou de confession n'est exigée 
Si je prends le cas de l'Asie du Sud, là  ou je travaille 
On a des 
Maîtres musulmans 
qui  ont des plans d'eau 
Et on a des maîtres à  vous reconnus 
Comme des soufis qui  ont 
Des maîtres et qui  ont des disciples musulmans, probablement des maîtres musulmans qui  ont des disciples à  nous  Et on a des maîtres 
Un gourou reconnu comme des maîtres soufis qui  ont des disciples musulmans   
\end{Ex} 

\paragraph{le dhikr} on se rencontre le jeudi soir, pour le dhikr.  
Pour la fête d'anniversaire de la mort du fondateur de l'ordre.

\paragraph{Circumambulation} n observe sur la tombe des saints pour leur 
Ou leur anniversaire de naissance. Rituels qui  ressemblent 
Sont calqués sur les images de La Mecque\sn{Tourne autour de la tombe du saint comme on tourne autour de la Kaaba } 
  
Pour certaines 
confréries,  , surtout ceux qui  sont éloignés 
de la Mecque 
, en 
allant plusieurs fois faire le 
Pèlerinage 
Sur la tombe du fondateur de l'Ordre ou de l'Ordre auquel 
On appartient, cela remplace un 
Pèlerinage 
à  la Mecque,  . 

\subsection{Devenir Saint}

\paragraph{le saint} \TArabe{قديس}
C'est saint  C'est ça qu'on vient 
C'est comme ça   c'est fondateur 
de confrérie 
de l'ordre soufi 

La Reconnaissance par les gens de du pouvoir d'une personne d'un pays considéré comme de Dieu, qui  va 
Déterminer si c'est un saint. Généralement ça passe par les miracles    . On peut aussi avoir des témoignages 
de gens qui  rêvent que le saint demande à  ce qu'on lui rende un culte.

Le Saint n'est pas pas forcément le fondateur, cela peut être un de ses disciples. Cela permet de ne pas avoir à aller au tombeau du fondateur : je peux aller au tombeau du saint. 

\mn{Pour aller voir les saints ?  
C'est des sanctuaires, sont 
Des lieux ouverts   pour moi, c'est ouvert de l'intérieur   . souvent, c'est avec une grande amplitude horaire.  on enlève ses chaussures  Il faut avoir une tenue correcte et on rentre dans le sanctuaire.  
}


\paragraph{Respect pour le maître} le maître est un \textit{saint potentiel} et on a donc un respect pour le maître. Mais certains maîtres, parce qu'ils n'ont pas fait grand chose, ne vont pas avoir un culte. D'autres vont se "spécialiser". 
\begin{Ex}[saint en Inde du Sud]
    Par exemple, un saint qui guérit la lèpre ou de dépigmentation. Les gens de la localité vont le voir pour tous les problèmes. Mais les gens viennent de plus loin pour sa \textit{spécialité}, la lèpre.C'est des pèlerins 
de toute l'Inde du Nord qui  font jusqu'à  
2003 1000 kilomètres pour venir ce saint pour être guéri 
de la lèpre 

\end{Ex}

\paragraph{dérive sectaire} 
Le gros problème que pose 
Le soufisme, c'est cette dérive vers le culte des saints ou on n'est plus 
Seulement dans une relation d'enseignement entre le maître 
et le disciple, avec rupture d'égalité, alors qu'en Islam, tout le monde est égal devant Dieu. 
En Asie du Sud, le disciple rend 
Un culte, il se tourne plus 
Vers Dieu et il 
Se tourne vers son maître. c'est 
L'amour du maître qui peut de prendre la place de Dieu. Donner des associés à  Dieu et  de remettre 
en cause 
L'unicité 
de Dieu.
Les initiés sont inspirés directement par Dieu. Comprendre par Dieu, capable de louer Dieu selon les propres 
Attributs. ils Voient dans l'essence divine.

\paragraph{Initiation publique}
Un rituel d'initiation qui  est qui  est à  la fois privé et public. Il y a des jours dans l'année et une date dans l'année ou on décide de se faire initier. Ils ont les cheveux rasés et les sourcils rasés car tous les poils ont été rasés. 
Après, il y a 
le comportement : On peut porter le costume de Fakir, mais il y a un comportement qui permet de les démasquer.
\begin{Ex}
  cas d'une 
Personne dont c'est une femme qui  appartient 
à  une caste 
Commerçante très riche du Pakistan. Avec son mari ils font faillite.  elle 
Va dans cette ville de 
Pèlerinage ou je travaille. Le fakir lui demande : 
\begin{quote}
    Si tu veux retrouver la fortune, tu dois construire ma maison  
\end{quote}
Elle fait confiance, fait construire la maison et depuis, elle a retrouvé la richesse.
\end{Ex}
Les miracles tournent autour des maladies et des exorcismes. 
Les fakirs sont assez réputés pour exorciser des djinn et des fantômes, des morts. 

\paragraph{ticket pour le paradis}  Pour les disciples qui  le suivent, 
Ce sera de voir au moment du Jugement 
Dernier 
Rassembler 
Derrière lui tous ses disciples et pouvoir les faire 
Entrer au 
Paradis à  la suite de Ali. Assurance qu'ils pourront 
entrer au paradis. Chaque année il y a une cérémonie ou on rejoue l'alliance avec son maître : ils reçoivent un ticket, un ticket pour 
Aller au 
Paradis. 

\paragraph{Femmes Fakir} En principe, non  Il n'y a pas de femmes Fakirs. Parce que elles sont impures et qu'elles sont 
d'impureté 
des menstruations.  Elles peuvent pas rentrer dans les sanctuaires mais là , le groupe que j'étudie, le maître accepte de plus en plus des femmes. La dernière 
Fois, j'y suis allé  Il y avait trois femmes qui  avaient été initiées, mais c'est des femmes qui  
qui  sont très 
masculines physiquement. Ex une femme n'a jamais été réglée : 
Ils se considèrent leur rituel 
d'initiation.  Ils le considèrent comme homme
   et ils changent 
de sexe   à  la fin du 
rituel. Ils sont les épouses 
du saint pour lequel le saint qui  représente l'ordre pour lequel se sont 
Initiés et ils préfèrent être des femmes 
Dans un monde d'hommes, vers le monde des saints plutôt que des hommes dans un monde de femmes qui  est le monde. 
Il y a toute cette question d'inversion des sexes. Il faut savoir que dans le cadre de 
l'Asie du 
Sud, le seul Dieu que la femme peut sanctifier, c'est son époux. 
 
