\chapter{Qu'est ce que l'homme pour l'Islam ?}
\mn{Delphine Ortis \newline Chercheur(e) indépendant Ethnologie / Anthropologie sociale
Delphine.ortis@gmail.com}
\section{Approche et validation}

\paragraph{Approche}
\begin{itemize}
\item diversité de l'islam
\item dignité de l'homme
\item figures d'autorité : 
\begin{itemize}
\item le prophète et le Sunnisme
\item l'\Imam dans le Chiisme
\item le Saint dans le soufisme
\end{itemize}
\item Eschatologie
\end{itemize}

\paragraph{Validation du Cours}Presenter un support de cours et partager un thème sur l'article sur 2 pages et on le présente : 
\begin{itemize}
\item Revue des Mondes Musulmans et de la Méditerranée\sn{\href{https://journals.openedition.org/remmm/}{RMMM}}
\item Revue de l'extrême orient
\end{itemize}
Résumé de l'article. On situe l'auteur. C'est quoi son sujet ? Qu'est ce qu'il a voulu démontré ? Comment ? 
Commentaire personnel 


\chapter{Diversité de l'Islam}

On pense au sunnisme. il faut penser aux sectes, les séparations religieuses mais il y a aussi les différences culturelles.
\paragraph{Des schismes à la mort du Prophète} Sans fils et sans testament, la succession de Mohammed est problématique. Les Califes sont ses compagnons de la même tribu et même souvent à son clan. 

\subparagraph{les Califes bien dirigés} Les premiers califes sont des compagnons mais ne sont pas des \textit{ gens de la maison}.
 
\subparagraph{Fatima, seule enfant du prophète} épouse Ali, le cousin germain du prophète. Seuls eux sont \textit{gens de la maison}

\subparagraph{Uthman} 3eme calife est assassiné en 655. Du coup, trois groupes : 
\begin{itemize}
\item sunnite
\item chiite
\item Kharijite, les \textit{séparés}
\end{itemize}

Ces trois groupes vont élaborer des doctrines différentes sur le califat.

\paragraph{Les inégalités reconnues dans l'Islam}
Inégalité : 
\begin{itemize}
\item Croyant / non Croyant
\item homme / femme
\item homme libre / esclave (écrit dans le Coran)
\end{itemize}
Dans les croyants, on peut mettre les gens du livre. 


\paragraph{Les Kharijites} même un esclave noir peut devenir Calife\mn{Dans le désert Algérien, on trouve des kharijites}. En disant cela, les kharijites sinsistent sur l'individu. On le juge juste sur son discours. 
Ce groupe se scinde : 
\begin{itemize}
\item Ibadite (Zanzibar et Oman). Il se caractérise par un très fort égalitarisme et une rigueur morale.
\item autres ? 
\end{itemize}

\paragraph{Le Chiisme} Les chiites insistent sur le fait que les califes doivent être des \textit{gens de la maison}, Ali et ses descendants. Le premier \Imam est Ali, puis Hassan, puis Husayn. On arrive au 6ème \Imam Jafar al Sadiq. Il désigne Ismael, son fils ainé, qui décède avant son père. Comme Jafar n'a pas nommé de successeur, la communauté chiite se scinde en différents courants : 
\begin{itemize}
\item les septimains (7) ou Ismaelien. Ismael n'est pas mort, dans une sorte d'occultation. Il réapparaitra à la fin du temps.  
\item les duodécimains : disent que le successeur est \textit{Musa}, le second fils de Jafar. Le \textit{Mahdi} est le douzième \Imam, Mohammed al-Madhi.
\end{itemize}

\begin{Def}[Mahdi]
Le Mahdi, celui qui revient à la fin des temps et qui va lutter avec Jésus contre l'armée du mal :
\begin{itemize}
\item Il n'est pas né
\item pour les septimains, c'est \textit{Ismael}, qui n'est pas mort
\item pour les duodécimains, c'est le 12ème \Imam
\end{itemize}
\end{Def}
