\chapter{Figure d'autorité}


\section{première figure : Mohammad}
\paragraph{autorité : bien comprendre le lien avec Mohammad} Dans l'enseignement du coran et la Sunna, l'idée d'\textit{imiter le prophète} et de lui accorder \textit{amour et vénération}

\paragraph{débat sur la figure du prophète} 

\begin{itemize}
    \item simple messager. le législateur, dont la loi sha'ria doit être suivie
    \item pour certains, sa réalité est supratemporelle, intemporelle, composée de lumière. Sa présence est toujours actuelle dans le monde et on peut donc espérer de lui une \textit{intercession dans l'au delà}
\end{itemize}

\paragraph{juste un prophète serviteur ou un prophète Roi ?} qui gouverne. 

\paragraph{distinction entre prophète et envoyé} \mn{\href{https://fr.wikipedia.org/wiki/Messagers_de_l\%27islam}{wikipedia}}
\paragraph{\Rasul}  Dieu envoie un envoyé de Dieu, un apôtre (\Rasul) \TArabe{رسول}. \mn{le message est toujours le même mais ce sont les hommes qui comprennent mal, sur l'unicité de Dieu et le jugement dernier}. Il est le représentant d'un peuple
On trouve {\Rasul} dans trois sourates du Coran.

\paragraph{\Nabi} connaissance de la divinité. Idée que les prophètes sont la source de toute connaissance supérieure. Il n'est pas le représentant d'un peuple. Sont \Nabi Jésus, ... c'est un statut plus faible que celui de \Rasul.

\paragraph{Les caractères des prophètes}
Quand ils sont sous le coup de leur révélation, ils peuvent s'abstraire de leur nature humaine.
Naturellement portés vers le bien.
Ils agissent en faveur de la religion.
Ils ont une noble ascendance.
Ils font des prodiges en parole et action. Le plus grand prodige, c'est le \textit{Coran}
Intelligence / créativité.



\paragraph{littérature populaire} va parler des miracles du prophète, pas forcément orthodoxe. Au début de l’Islam, peu d’intérêt pour Mohammed.

\paragraph{historique}  peu d’information sur lui. On ne sait pas quand il est né. On fête sa naissance, le 12 \textit{Rabi sul Awwal}\mn{mois lunaire}.
Il a été présenté comme \textit{illétré}. Il était commerçant. On peut douter de cet illétrisme car ce n'est pas compatible avec son métier.
Son illétrisme désigne le \textit{prophète du peuple} qui n'appartenait pas aux \textit{gens du livre}\mn{Les musulmans se pensent dans la continuité du Christianisme, dans la continuité du judaïsme.}. 

\begin{Def}[Contribule]
Appartenir à une tribu
\end{Def}
Il passe du statut de Contribule au statut d'envoyé de Dieu, quand il est à la Mecque. Puis chef de guerre, chef d'état dans la ville de Médine. 
A la Mecque, il annonce que le Règne de Dieu est proche. Réformateur religieux qui prône un strict monothéisme. 

On sait qu'il est persécuté par les polythéistes Mecquois.  
\paragraph{Ascension} Son ascension (Miraj) : 27 rajab. Il accomplit cette ascension, avec comme véhicule \textit{Bourak}, un âne avec une tête de femme. Il va à Jérusalem en rêve. Il prend son élan pour monter dans les cieux. Le ciel est divisé en 7. il monte les 7 strates et voit la \textit{lumière divine}.


\paragraph{Représentation de l'homme} L'interdiction est récente, avec les mouvements de réforme du XIX et dans le sunnisme. L'interdiction porte sur l'image de Dieu.

\paragraph{L'exil à Yatrib}, la ville \emph{medina} du prophète. Hégire : 622 après JC. 
