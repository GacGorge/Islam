\chapter{L’islam et les femmes}


\section{Introduction}
\mn{14/11/19 à l'ICP.   Table ronde animée par Emmanuel Pisani.}
 
Place des femmes : question transversale dans toute la société ; décision et pouvoir. Amazonie : l’Eglise Catholique n’a pas laissé les femmes voter. Pas une spécialité de l’Islam.
Question spécifique à l’Islam : pas uniquement une question de rite : dans bcp de pays, le statut personnel de la femme est une lecture médiévale, archaique. Pas uniquement religieux mais aussi social.
Mais l’Islam bouge. Pas récent ; une vraie évolution ; 

 

Présentation des intervenants
\begin{itemize}
    \item Kahina Bahloul : première imam de France ; mosquée fatima ; pas de salle pour prier le vendredi. Doctorante Islamologie : Ibn Arabi ; dans son activité de prédication ; association : « parle moi d’Islam ». Charles Peguy.Mystique musulmane. Esprit globale du Coran. Historico-critique de la Sunn’a. 
    \item Hicham Abdel Gawad : musulman ; « pas un théologien » mais « religiologue ». Master à Louvain ; doctorant : « pédagogie du Croire chez les jeunes musulmans ». Méthode historico critique.  Daniel Marguerat. 
    \item Jamel El Hamri. Docteur sur Malek ben Ali. Islam et altérité. Approche savante et confessante. 
\end{itemize}






 Mohamed Arkoun \label{theol:Arkoun1},
dit de la condition de la femme \sn{\cite{ArkounEveillera}} :
\begin{quote}
    « A l’orée du XXI siecle, il n’est plus possible de parler de la condition feminine … [en parlant depuis le Coran]. On en est toujours là pourtant ».
\end{quote}


\subsection{L’absence du nom de femmes}
Partons du Coran : 
\begin{itemize}
    \item 	Sur les femmes, un mystère, une étrangeté. Il est bien question des femmes mais le nom des femmes n’est jamais cité. La femme est… l’épouse d’un homme. Nommer, essentiel. 
\item	Un seul prénom féminin : Miriam. Difficulté fondamentale ? 
\item	Captatio : « partir de cette étrangeté ».

\end{itemize}

 
précaution. parfois question mal posée. Dichotomie entre les hommes et les femmes. Essentialiser la femme musulmane. Alors qu’en fait ce sont des cheminements.
\begin{itemize}
    \item Pas un féminisme ; mais une diversité de féminisme.
\item 	Occultation sur le sort des femmes. Dans les zahouia et les mosquées, des femmes remarquables, portant le voile ou non, pas dans la dichotomie tradition/modernité.
\item 	Question difficile : dans une perspective Hégelienne, les femmes doivent s’émanciper comme les femmes occidentales. Mais cette émancipation peut surprendre
\item 	Ex : la femme française a voté après la femme turque. 
\end{itemize}

Le Coran n’est pas un livre d’histoire. On cite peu de noms. Un créateur / des créatures.  Dans la tradition musulmane, croyance que le prophète a assumé le premier le texte coranique. Dans la première société musulmane, on s’adresse à Khadija.
Sens que le Coran ne donne pas le nom de Khadija mais uniquement la tradition (EP)
Kahina : 
Interroger le texte ; Coran mais aussi Hadiths et Sunn’a. 
D’après Mohamed Arkoun\label{theol:Arkoun2}, le Coran est une construction jusqu’au Corpus officiel clos. Collecte des textes, controverse, François Deroche, College de France : la forme de certains versets, les clausules ; ces clausules ont changé. Tout le monde doit s’interroger sur le statut du Coran : 
\begin{itemize}
    \item Parole de Dieu, incréé, de toute éternité ; consubstantielle de la nature divine. Pose Question. 
    \item Concernant le texte de la femme, on voit très bien comment le texte est le fruit de son époque. 
    \item	Les noms cités sont les prophètes ; tous les noms sont des noms de prophètes, y compris Marie. Le texte Coranique nous dit qu’une femme a un statut de prophétesse. Grande source d’inspiration pour les femmes
 \item	Absence de noms de femmes ; fonde la nécessité de contextualiser le Coran ; mais aussi ouvre au statut de prophétesse.  De ce manque, vous y voyez du plein. 
 \item	Sur la question de l’égalité, essentielle. Question névralgique. Question du projet modernité ; c’est celui qui a commencé à la renaissance d’émancipation de l’individu. Le projet modernité n’est pas achevé et du coup, des questions sur sur l’acquisition de nouveaux droits même en Occident. 
 \item	Post modernité : résistance à ce projet de modernité avec les structures qui résistent.
\end{itemize}
	

\section{Qu’en est il de l’égalité, de l’égalité de dignité ? }
au-delà de ces différences, égalité de dignité.

Hicham : pas forcément pertinent. En tant que professeur d’Islam en Belgique, Moelenbeek. Comment aborder ces questions avec les jeunes ?

Qu’est-ce que le Coran ? si Parole de Dieu dictée ; ce qui s’y trouve est aussi important que ce qui ne s’y trouve pas. Dignité formidable ou non selon qu’on est cité ou non.
Abulah Hub : cité. Une dignité pour cet homme, même si péjoratif (sobriquet ?)
Abu bakr : meilleur ami de Mohammed ; jamais cité
Moise : 145 fois
Jésus :  25 fois

Le Coran est auto référentiel. Il se désigne comme \textit{diq’h} \mn{ref ?} ; \textit{Tamzin} : descente. Il se définit non pas par sa nature mais par sa modalité de manifestation. 250 fois mais 0 fois comme parole de Dieu. Seul Jésus est désigné comme \textit{Parole de Dieu} + 2 fois comme « Thora » et une dernière fois ? \mn{A RETRAVAILLER}
Est-ce qu’il n’est pas paradoxal que le Coran ne s’appelle pas parole de Dieu ?

\section{« frappez les »}

Comment entendre aujourd'hui la Sourate 4 ; verset 24.  « frappez les » ?
\begin{Ex}
Ousnia, 18 ans.  « est ce que le Coran autorise à taper sa femme ? ». comment répondre à Ousnia ? Dans les explications traditionnelles ; on a le droit de taper mais avec un stylo. « oui, mais moi ce qui me gêne, c’est qu’il leve la main sur moi ». Fondamentalement, l’asymétrie gène.
\end{Ex}

Le mécanisme exegétique du « verset abrogeant abrogé » \sn{voir p. \pageref{{mansûkh}}} peut permettre de travailler ce verset, qui serait abrogé dans la pratique.

Kahina : il faudrait que les femmes revendiquent cette dignité.

Jamel : 
Herméneutique : 
\begin{itemize}
    \item -	Contexte 3ème siècle : les avancées / les regressions de la société arabe. 
-	Evolution sur ces versets de l’analyse coranique.
-	Modernité : une rupture. Un projet. 
-	Monde musulman se perçoit en retard par rapport au monde. 
\end{itemize}

\subsection{le voile} \label{voile}
La question du voile pollué: 
\begin{itemize}
    \item 	Fantasme du voilement par rapport à un fantasme du dévoilement.
\item	Jadam el cheer : à l’intérieur des communautés musulmanes, mal perçu. 
\item	Partir du vécu des musulmans et pas des textes.
\item	Manque l’oulema, qui s’appuie sur le politique pour assurer le droit.
\end{itemize}

Question de l’égale dignité : oui. Mais on part dans la société patriarchale.
-	Polygamie, Héritage
-	Mais sur un certain nombre de promesses de l’Islam, certains promesses (esclavage) n’ont pas été conservées.

Kahina
Dans le texte coranique, il parle d’une égale dignité ; ils sont tenus de leurs actes devant Dieu. Il n’y a rien. Pierre Lory \sn{\cite{LoryDignite}} ; plusieurs exegèses classiques. L’Homme Coranique et c’est ce que dit Hahina Ouadou. Si on lit avec un prisme non sexué le Coran, égale dignité de l’homme et de la femme (et de toute la Création). 

Des lectures d’un même verset à chaque période. Les textes sont les textes ; ils peuvent être lus et relus. Ce verset 4, 34 a plusieurs interprétations mais avec rigueur. 
Kahina
Khiwama - Jimmama \mn{Revoir } : inférieorité de la femme 
Mohammed Hamdou : la femme pas égale à l’homme car elle accouche donc elle est plus vulnérable. 
 l’imamat de femme : pas possible du fait du khiwama.
Moise : Imam ; guide ; direction. Mais le Coran ne précise pas ce qu’est un Imam.
Un Hadith : une femme Imam ; le prophète lui a désigné un muezzin (appel à la prière). 
Ecole la plus littéraliste : les femmes prière surrogatoire mais pas devant ; derrière.
Ecole halafite : reconnait l’imamat mais uniquement pour les femmes.
Ecole Malekite ne reconnait pas l’imamat féminin (majoritaire dans le magreb). Mais il n’y a pas eu Consensus.

Hicham
Kierkegaard : pourquoi Abraham accepte d’immoler son fils ?
Idem pour moi ; lecture victorielle : bof pour répondre à Ousnia. Rusnia.
\begin{quote}
    « Dieu dit : ne forcez pas vos femmes à la prostitution ; mais si elles sont contraintes, Dieu leur pardonnera »
\end{quote}
. Comment on arrive à cette dignité ?
Pas d’autres choix que de fermer le texte au moins temporairement.
Du coup, on a regardé la vie de Khadija. Le prophète s’est réfugié dans les bras de khadija après la première vision. Khadija a cru en mohamed avant que mohamed croit en lui-même. Insatisfaction du prophète 
Aicha : a levé une armée après la mort de Mohamed. Carrière militaire.
Rabia el khabia : poetesse ; seau d’eau ; éteindre le paradis. Pour que l’on aime Dieu pas par crainte de l’enfer et le souhait de Paradis. Une réflexion profonde, comme kant, sur le sens de l’éthique.
A partir de ces exemples, aider Ousnia à rester une femme musulmane de façon unique. 


Abrogeant abrogé : je ne le défend pas. Mais si j’avais été une femme, je l’aurais utilisé mais comme la théologie a été faite par les hommes.
Hadiths authentiques : selon un auteur ou un autre, renforce ou affaiblit tel ou tel Hadiths. Bourali ; édité au XVeme siècle. 
Saint Jean Damascène. VIIeme. On ne trouve pas de traces de l’histoire d’un mariage avec une femme de 6 ans. Pourquoi alors inventer cette histoire tardive d’un Hadith où Aicha se serait marié à 6 ans à Mohamed. Polémique anti Chiite (très anti Aicha) pour expliquer que Aicha n’a pas pu être tenté par satan car auprès de Mohammed dès le début.


