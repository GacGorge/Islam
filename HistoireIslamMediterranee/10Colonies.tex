\chapter{Chrétiens et musulmans au XXe siècle}

\mn{}

\section{Quelques références bibliographiques}

\begin{itemize}
    \item 
 
Abitbol
Michel, Histoire du Maroc Paris, Perrin, coll « Tempus », 2014
    \item 
Bessis
Sophie, Histoire de la Tunisie de Carthage à nos jours, Paris, Taillandier, 2019
    \item 
Bonin
Hubert, L'empire colonial français de l'histoire aux héritages XX e XXI e siècles, Paris, Armand Colin, « Collection U », 2018
    \item 
Bouchène
Abderrahmane, Peyroulou Jean Pierre, et al dir Histoire de l'Algérie à la période coloniale 1830 1962 Paris, La Découverte,
2012
    \item 
Chouikha
Larbi et Gobe Eric Histoire de la Tunisie depuis l’indépendance, Paris, La Découverte, coll « Repères », 2015
    \item 
Droz
Bernard, Histoire de la décolonisation au XX e siècle, Paris, Le Seuil, « L'Univers historique », 2006
    \item 
Georgeon
François, Vatin Nicolas et Veinstein Gilles dir Dictionnaire de l’Empire ottoman Paris, Fayard, 2015 2 e édition en poche aux
éditions du CNRS, coll « Biblis », 2022
    \item 
Hourani
Albert, Histoire des peuples arabes Paris, Seuil, coll « Points », 1993
    \item 
Phan
Bernard, Colonisation et décolonisation ..(XVI e XX e siècle Paris, Presses Universitaires de France, « Quadrige », 2017
\end{itemize}

\section{Quelques dates importantes (1)}

\begin{itemize}
    \item 
 
Mai
juin 1830 : expédition d’Alger
    \item 
Mai 1837 : traité de Tafna
    \item 
1839 : rupture du traité de Tafna
    \item 
Novembre 1848 : Algérie devient «
territoire français »
    \item 
Avril 1863 :
Senatus Consulte sur la propriété foncière
    \item 
Juillet 1865 :
Senatus Consulte sur l’état des personnes et la naturalisation
en Algérie
    \item 
1881 : Tunisie devient protectorat français (Traité du Bardo : mai 1881)
    \item 
1912 : Maroc devient protectorat français (Traité de Fès : mars 1912)

\item Indépendance du Maroc : 3 mars 1956
\item
Indépendance de la Tunisie : 20 mars 1956
\item
Indépendance de l’Algérie : 3 juillet 1962
\end{itemize}

\section{Senatus Consulte}

\begin{quote}
    Extrait du
Senatus Consulte sur l’état des
personnes et la naturalisation en Algérie (1865)
 
Article premier
L'indigène musulman est Français ; néanmoins il continuera à être régi par la loi musulmane. Il peut être admis à servir dans
le s armées de terre et
de mer. il peut être appelé à des fonctions et emplois civils en Algérie. Il peut, sur sa demande, être admis à jouir des dro its de citoyen français ; dans
ce cas il est régi par les lois civiles et politiques de la France.
 
Article 2
L'indigène
israëlite est Français ; néanmoins il continue à être régi par son statut personnel. Il peut être admis à servir dans les armées de ter re et de
mer. il peut être appelé à des fonctions et emplois civils en Algérie. Il peut, sur sa demande, être admis à jouir des droits de citoyen français ; dans ce
cas il est régi par la loi française.
 
Article 3
L'étranger qui justifie de trois années de résidence en Algérie peut être admis à jouir de tous les droits de citoyen français.
 
Article 4
La qualité de citoyen français ne peut être obtenue, conformément aux articles 1, 2 et 3 du présent sénatus
consulte, qu'à l'âge de vingt et un ans
accomplis ; elle est conférée par décret impérial rendu en conseil d'Etat.
\end{quote}

\in