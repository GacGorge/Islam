\chapter{Jean Damascène}

\section{Jean Damascène}

\paragraph{635 : ouverture de la ville de Damas} par le grand père de Jean Damascène.  

\paragraph{du temps d'Umar II, celui de la deuxième capitulation} Contexte plus dur vis à vis des chrétiens. Jean va devenir moine et quitter ses postes d'administration. Umar II, avec ce nouveau pacte, va islamiser les principaux postes omeyyades. 

\paragraph{Texte en Grec} reste la langue du trait d'union entre les cultures. Peut aussi permettre une plus faible accessibilité des musulmans au texte.


\section{De Haeresibus}
\paragraph{les sources de la connaissance} une partie philosphie, une sur les hérésies, et un chapitre sur la foi orthodoxe.


\paragraph{Ce n'est pas une hérésie mais une superstition} C'est un délire, alors que les hérésies ont un fond rationnel.

\paragraph{il ne les appelle pas musulman} mais sarrasin. 

\paragraph{croyance que l'Islam} est né de l'arianisme et du culte d'Aphrodite : Aphrodite car permissivité sexuelle.

\paragraph{Définit la profession de foi musulmane} Il connait par coeur le coran ou au moins une partie. Pourquoi a t il basculé dans une vision très négative de l'Islam ?


\begin{quote}
    Il y a aussi la superstition - encore vigoureuse - des Ismaélites qui abuse le monde et annonce la venue de l'Antéchrist. Elle tire son origine d'Ismaël, l'enfant qu'Agar avait donné à Abraham, et c'est pour cela qu'on les appelle Agaréniens et Ismaélites. On les nomme aussi \textit{Sarakénoi} (Sarrasins) parce que Sara les a renvoyés les mains vides: en effet, Agar dit à l'ange: 
    \begin{quote}
        "Sara m'a renvoyée les mains vides"\sn{Fausse éthymologie: Sarra-kenoï (Sara - Vides). En fait Gn 21,10-24 nous montre Abraham; chassant Agar en lui donnant du pain et de l'eau.}
    \end{quote}. Ces gens-là, donc, étaient des idolâtres et ils vénéraient l'étoile du matin et Aphrodite qu'ils nommaient Habar dans leur langue. Ce mot signifie "grand"; ainsi, jusqu'aux jours d'Héraclius, il n'y a pas de doute qu'ils étaient des idolâtres. A partir de cette époque, un faux prophète apparut parmi eux.

\paragraph{Mohammed et moine arien}
Il s'appelait Mameth. Ayant acquis par hasard quelque connaissance de l'Ancien et du nouveau Testaments, il aurait rencontré un moine Arien; à la suite de quoi, il élabora son hérésie personnelle. Après avoir faussement donné au peuple l'impression qu'il était un home de Dieu, il répandit la rumeur selon laquelle une Ecriture lui était descendue du Ciel. C'est ainsi qu'après avoir mis par écrit dans son livre un ensemble de déclarations qui ne méritent que le ridicule, il les leur donna pour qu'ils les observent.

\paragraph{Définition de Foi - Croyance du Coran sur Jésus}
Il dit qu'il existe un Dieu créateur de tout, qui n'a pas été engendré et qui n'engendre pas . Il dit que le Christ est le Verbe de Dieu et son Esprit, qu'il est une créature et un serviteur", qu'il est né sans semence de Marie , la soeur de Moïse et d' Aaron. En effet. dit-il, le Verbe de Dieu et son Esprit vinrent en Marie et elle donna naissance à Jésus qui était un prophète et un serviteur de Dieu. Les Juifs, ayant violé la Loi, voulaient le crucifier, et après l'avoir arrêté, ils crucifièrent son ombre6. Mais le Christ, lui-même, prétendent-ils, n'a pas été crucifié et n'est pas mort, car Dieu l'a élevé auprès de Lui au Ciel car Il l'aimait. Et voici ce qu'il dit: quand le Christ monta au ciel, Dieu le questionna et lui demanda: 
\begin{quote}
    "O Jésus,\sn{Paraphrase de Cor. 5,116  On remarquera les différences entre notre texte et celui
· du Coran: Jean Damascène; fait preuve d'une grande familiarité avec le texte du Coran, mais il n'a pas la permission d'en posséder un exemplaire. A cette époque les manuscrits en sont rares et chers, mais, en outre, un \textit{dhimmî} (minoritaire protégé) ne doit pas avoir le Coran. C'est donc de mémoire qu'il cite le texte.
} as tu dit: \textit{'Je suis Fils de Dieu et Dieu ?'}. Et Jésus - à ce qu'ils disent - répondit: \textit{"Prends pitié de moi, Seigneur; tu sais que je n'ai rien dit de tel, et \textbf{je ne me vanterai pas non plus d'être ton serviteur; mais des hommes qui se sont fourvoyés ont écrit }que j'avais fait une telle affirmation et ils ont dit des mensonges contre moi et ils se sont trompés}" Et, selon eux, Dieu lui répondit: \textit{"Je savais bien que \textbf{tu ne dirais pas une telle chose}" }
\end{quote}   
\paragraph{Crédibilité du Coran comme songe}
Et quoiqu'il inclue dans son écrit beaucoup d'autres absurdités ridicules, il insiste que cela lui est venu de Dieu. Et nous demandons: "Et qui est celui qui rend témoignage que Dieu lui a donné les écritures ?". Et parce qu'ils sont surpris et incapables de répondre\sn{Selon les critères founis par la Bible (Dt13,2 5 ; 18,11-20), on ne doit croire à la parole d'un prophète que si son enseignement est conforme à la Révélation antérieure et ses dires confirmés par des miracles et des prophéties accomplies. Les musulmans mettront un certain temps à élaborer une réponse: par exemple en présentant Mohammed comme annoncé dans la Bible. Sur ce thème, voir le livre d'un chrétien converti à l'Islam: \textit{Kitâb al-dîn wal-dawla de 'Alî al Tabarî}}, nous leur disons que Moïse a reçu la Loi au Mont Sinaï, à la vue de tout le peuple quand Dieu apparût dans la nuée et le feu, dans l'obscurité et la tempête; et que tous les prophètes, à partir de Moïse et à sa suite, prédirent que le Christ viendrait et que le Christ serait Dieu, et que le Fils de Dieu devait venir en prenant chair et qu'il serait crucifié, qu'il mourrait et qu'il serait le juge des vivants comme celui des morts. A la suite de quoi, nous leur demandons: "Comment se fait-il que votre prophète n'est pas venu de cette façon, confirmé par d'autres qui lui portent témoignage, et que, de même qu'Il a donné la Loi à Moïse au vu de tout le peuple, pendant que la montagne fumait, Dieu ne lui a pas donné l'écriture en votre présence de sorte que vous puissiez vous aussi en avoir l'assurance - car c'est bien ce que vous prétendez ?" A cela, ils répondent que Dieu fait tout ce qui lui plaît. "Ceci, répondons-nous, nous le savons aussi; mais comment l'écriture est-elle descendue sur votre prophète ? voilà ce que nous demandons". Alors, ils répondent que l'écriture descendit sur lui pendant qu'il dormait. En nous moquant, nous leur disons alors que, puisqu'il a reçu l'écriture dans son sommeil et qu'il n'a pas eu connaissance de ce qui se passait, c'est bien à son sujet que se réalise la proverbe populaire \sn{Le proverbe n'est pas cité dans le texte. On suggère celui-ci tiré de Platon: "Tu
me contes des songes l" (Cf. Chase, Saint John, p.155, n.106).
}.
\end{quote}

\paragraph{la véracité d'une révélation vient de la présence de témoin} argument faible. Il donne un des trois arguments positifs de l'apologie chrétienne : l'accomplissement des prophéties en plein jour, le tombeau vide, les miracles,... Ici, ce sont les prophéties. 

\paragraph{Aujourd'hui} C'est la justesse de la vie vécue qui est retenue : cela permet un autre dialogue avec l'Islam.


    

\paragraph{Les Prophètes annonçaient la Trinité}
\begin{quote}
Quand à nouveau nous leur demandons: " Alors que, dans votre écriture, il vous demande de ne rien faire ni de ne rien accepter sans prendre des témoins\sn{Cor,222,·4,6.15,41, 5,106 \& ss, 24,4 \& ss,13, 65,2,}, comment se fait-il que vous ne lui ayez pas dit: 'Prouve d'abord par des témoins que tu es un prophète et que tu es venu de Dieu, et quelle est l'écriture qui te rend
témoignage' ", ils restent silencieux à cause de la honte. "Puisqu'il ne vous est pas pennis d'épouser une femme sans témoins, ni d'acheter quoi que ce soit, ni d'acquérir quelque propriété. vous n'acceptez même pas d'avoir un âne ou un animal sans des témoins car vous avez des femmes, des propriétés et, des ânes à l'aide de témoins; par contre, c'est seulement votre foi et votre écriture que vous obtenez sans témoin". Et ceci parce que celui qui vous l'a transmise n'a aucune attestation d'où que ce soit, et personne de connu n'a témoigné d'avance à son sujet; bien au contraire, lui, c'est dans son sommeil qu'il a reçu cela. Par ailleurs, ils nous appellent les "\textbf{associateurs}" parce que, disent-ils, nous mettons à côté de Dieu un associé en disant que le Christ est le.Fils de Dieu et Dieu. A quoi nous répondons que \textit{"C'est cela que les prophètes et l'Ecriture nous ont transmis; et vous, vous prétendez que vous acceptez les prophètes"}. Si donc nous disons, à tort, que le Christ est le Fils de Dieu, eux aussi ont eu tort puisqu'ils nous l'ont transmis et enseigné. Et quelques uns d'entre eux soutiennent que c'est nous qui avons ajouté ces choses en interprétant les prophètes de façon allégorique \sn{C'est déjà l'accusation de falsification de l'Ecriture par fausse interprétation
(tal)rif ma'oawî).
}. D'autres prétendent que les Juifs, par haine, nous ont trompés par des écritures supposées venir des prophètes, pour que soyons condamnés\sn{Falsification textuelle (tal,irîf al-laf ).}.
   
\end{quote}

\paragraph{Union mystique à Dieu} Si c'est un Dieu un, que devient l'union mystique ? Si c'est Dieu trinité, que devient l'union mystique ?

\paragraph{Pour répondre à l'associationisme, il utilise le Jésus du Coran} Et déclare les musulmans les mutilateurs. Il y aura une réponse mutazilite \textit{subordationisme} pour éviter l'associanisme radical qui est gardé en Chiisme sous la forme de la \textit{lumière mohamédienne}. Mais dans la querelle du Coran incréé, un retour à une vision stricte de l'associatisme.


\paragraph{Associateur et mutilateur de l'Esprit et du Verbe}
\begin{quote}
De nouveau nous leur répondons: "Puisque vous dites que le Christ est le Verbe et l'Esprit de Dieu, comment nous reprochez-vous d'être des associateurs ? En effet, le Verbe et l'Esprit sont chacun inséparable de celui dont il tire son origine; si donc le Verbe est en Dieu, il est évident qu'il est aussi Dieu. Si, au contraire, il est en dehors de Dieu, alors Dieu, selon vous, n'a ni verbe ni esprit. Ainsi, en voulant éviter de donner à Dieu des associés, vous le mutilez. Car il vaudrait mieux admettre qu'il a un associé plutôt que de le mutiler et de le présenter comme s'il était une pierre, un morceau de bois ou quelque objet inanimé. Ainsi donc, en nous accusant faussement, vous nous appelez des associateurs; nous alors. nous vous appelons des mutilateurs de Dieu".
\end{quote}
\paragraph{La Ka'ba comme pierre d'Abraham}
\begin{quote}
    
Ils nous calomnient aussi comme des idolâtres parce que nous vénérons la croix qu'ils méprisent; et nous leur répondons: \begin{quote}
    ''Comment se fait-il que vous vous frottiez contre une pierre par votre \textbf{habathan}\sn{Ka'ba} et que vous exprimiez votre adoration de la pierre en lui donnant des baisers. Quelques uns répondent que c'est parce qu'Abraham s'est unis à Agar sur cette pierre; d'autres, parce qu'il a attaché son chameau autour de cette pierre alors qu'il allait sacrifier Isaac.
\end{quote} Alors nous leur répondons: 
\begin{quote}
    "Vu que l'Ecriture dit qu'il y avait là une colline boisée et du bois mort qu'Abraham coupa pour y coucher Isaac en vue de l'holocauste; vu qu'il laissa des ânes en arrière avec ses serviteurs, d'où viennent ces balivernes que vous dites ? Le lieu dont vous parlez n'a ni bois ni forêt et les ânes n'y parviennent pas." 
\end{quote}

Ils restent alors tout embarrassés, mais prétendent pourtant que la pierre est bien celle d'Abraham . Alors, nous leur répliquons: \begin{quote}
    "Supposons qu'elle soit celle d'Abraham , comme vous le dites de façon si stupide: n'avez-vous point honte de la baiser uniquement parce qu'Abraham s'y est uni à une femme, ou parce qu'il y a attaché son chameau, alors que vous nous reprochez de vénérer la croix du Christ par laquelle le pouvoir des démons et la tromperie du diable ont été détruits ?"
\end{quote} 

En fait, ce qu'ils appellent "la pierre" est la tête d'Aphrodite qu'ils avaient l'habitude de
vénérer et qu'ils appellent \textbf{Haber}. Ceux qui s'y connaissent peuvent encore y voir des traces d'inscription.
\end{quote}

\paragraph{Dans la Ka'ba} pierre où Abraham a attaché son âne. et où il aurait eu d'Agar, Ismael.

\paragraph{Aphrodite} Un historien byzantin reprendra cette critique d'un temple d'Aphrodite.

\paragraph{puis attaque de la polygamie}. 
Il connaissait bien les commentaires de Zayd. 

\begin{quote}
    

\paragraph{la Femme}
Comme il a été dit, ce Mamed a composé bon nombre de fables oiseuses sur chacune desquelles il a mis un titre; ainsi par exemple, le discours de \textit{La Femme}\sn{Cor. 4, et en particulier Cor. 4,3.} dans lequel il légifère clairement que l'on peut avoir quatre femmes et un millier de concubines si l'on veut, autant que l'on puisse en maintenir en plus des quatre épouses; et que l'on peut divorcer de celle qu'on veut, et, si on le désire, en prendre une autre. Il a fait cette loi à cause du cas suivant: Mamed avait un compagnon appelé Zayd. Cet homme avait une belle femme pour qui Mamed se prit d'amour\sn{Cor. 33,37}. En fait le texte coranique montre Mohammed encourageant Zayd à
garder sa femme.cf Sinna : 
\begin{quote}
Le Prophète épousa aussi Zaynab bint Jahch. Elle lui fut donnée en mariage par son frère Abû Ahmad ibn Jahch. Le Prophète lui donna en dot quatre cents dirhams. Avant lui, elle avait été l'épouse de Zayd ibn Hâritha, affranchi du Prophète. C'est à son sujet que Dieu a révélé : Puis, quand Zayd eut cessé tout commerce avec son épouse, nous te l'avons donnée pour femme. (Coran, 33, 37.)
Hichâm, Ibn. Sinna 
\end{quote}

Alors qu'ils étaient assis ensemble, Mamed lui dit: "\textit{0 toi, Dieu m'a ordonné de prendre ton épouse}". ET l'autre-de lui répondre: "Tu es un apôtre; fais ce que Dieu t'a dit; prends ma femme". Ou plutôt, pour dire l'histoire depuis le début, il lui dit: "Dieu m'a commandé (de te dire) que tu devrais divorcer d'avec ton épouse"; et il divorça d'avec elle. Plusieurs jours plus tard, il dit: "Maintenant Dieu m'a ordonné de la prendre". Et alors, après l'avoir prise et avoir commis l'adultère avec elle, il fit la loi suivante\sn{Cor. 2,230.}: \begin{quote}
    "\textit{Quiconque le désire peut renvoyer sa femme. Mais si, après le divorce, il veut revenir à elle, que quelqu'un d'autre l'épouse d'abord. Car il ne lui est pas permis de revenir à elle, à moins que quelqu'un d'autre ne l'ai d'abord épousée. Et même si un frère divorce, que son frère l'épouse s'il le désire}". 
\end{quote}
Voici le genre de préceptes qu'il donne dans son discours: "\textit{Laboure la terre que Dieu t'a donnée et embellis-la; fais ceci, et fais le de telle et telle façon"} (pour ne pas répéter ses paroles dans ce qu'elles ont d'obscène)\sn{Cor. 2,223.}.

\paragraph{le texte (pas dans le coran) de la Chamelle}
Il y a aussi le texte du Chameau de Dieu où l'on parle d'une chamelle venue de Dieu qui avait l'habitude de boire toute la rivière au point qu'elle ne pouvait plus passer entre deux montagnes parce qu'il n'y avait pas assez de place pour la laisser passer. Il y avait des gens qui vivaient à cet endroit, et, à ce qu'il dit: un jour, c'était eux qui buvaient l'eau, le jour suivant - c'était la chamelle\sn{Cor. 26,155} Quand elle buvait l'eau, elle les nourrissait en leur offrant son lait au lieu de l'eau. Mais ces gens étaient méchants: il se rassemblèrent et tuèrent la chamelle. Il y avait, cependant, une petite chamelle qui était son petit et, dit-il, qui cria vers Dieu quand sa mère fut tuée. Alors Dieu la prit par devers lui. Et nous leur disons: \begin{quote}
    "D'où venait cette chamelle ?" Et ils répondent qu'elle venait de Dieu. Et nous disons: "Y eut-il un autre chameau qui s'accoupla avec elle?" Et ils disent: "Non". "Comment eut elle un petit alors ? , demandons-nous. Nous voyons, en effet, que cette chamelle n'avait ni père, ni mère, ni généalogie, et qu'après avoir mis bas, il lui arriva malheur.
\end{quote} 
Mais dans votre histoire on ne trouve ni le chameau qui couvrit la chamelle, ni l'endroit où fut emporté ce petit chameau ! Votre prophète, dont vous dites que Dieu lui a parlé, pourquoi ne s'est-il pas informé au sujet de la petite chamelle: où pature-t-elle? qui la trait? qui boit de son lait? Ou bien alors, comme sa mère, serait-elle tombé entre les mains d'hommes méchants qui l'ont tuée ? A moins qu'avant vous, elle soit déjà entrée au paradis et que d'elle coule la rivière de lait dont vous parlez ? Car vous dites qu'au paradis, vous aurez trois rivières où coulent l'eau, le vin et le lait\sn{Cor. 47}. Car si votre chamelle précurseur est en dehors du paradis, il est clair qu'elle est morte de faim et de soif, ou que d'autres gens vont profiter de son lait; et votre prophète se targue en vain d'avoir parlé avec Dieu, puisque le mystère de la chamelle ne lui a pas été révélé. Si, au contraire, elle est au paradis, elle boit encore de l'eau et vous allez, par manque d'eau, vous retrouver à sec au milieu des délices du paradis. Et puisqu'il n'y a plus d'eau (la chamelle l'a toute bue), si vous désirez boire du vin de la rivière d'à côté, vous allez, en n'arrêtant pas d'en boire, brûler au-dedans, tituber d'ivresse et vous assoupir. Au réveil, la tête lourde et encore intoxiqués d'avoir trop bu de vin, vous allez manquer les plaisirs du paradis. Comment se fait-il que votre prophète n'ait jamais pensé que tout cela pourrait vous arriver au paradis ? Il ne s'est jamais soucié de trouver où vivait actuellement la chamelle; et vous ne lui avez jamais posé la question quand, au sortir de ses songes, il vous prêchait sur les trois· rivières. Mais nous ·vous assurons, en toute certitude, que votre merveilleux chameau est déjà entré bien avant vous dans les âmes des ânes où vous-même demeurerez comme des animaux. C'est là qu'il y a la ténèbre extérieure de l'enfer éternel; un feu qui rugit, un ver toujours éveillé et les démons de l'enfer.
\paragraph{la Table}
Mamed parle aussi du texte de \textit{La Table}. Il dit que le Christ demanda à Dieu une table et qu'elle lui fut donnée. Car, dit-il, il lui a été révélé: \begin{quote}
    "Je t'ai donné à toi et à tes compagnons une table incorruptible"\sn{Cor. 5,112}.
\end{quote}

Il y a encore le texte de La Vache\sn{Cor. 2}, et bien d'autres contes futiles dignes du ridicule, qu'il vaut mieux taire tant il sont nombreux. Il. fit une loi d'après laquelle eux et leurs femmes doivent être circoncis, et il leur demanda de ne pas observer le Sabbat, de ne pas être baptisés, et de manger des nourritures interdites par la Loi, tout en s'abstenant d'en prendre d'autres (qui sont permises par elle): enfin il leur défendit absolument de boire du vin.


\end{quote}


\paragraph{Cela bouge} deux Jean Damascène : le premier haut fonctionnaire, connaissant bien l'Islam et le deuxième Jean Damascène. 