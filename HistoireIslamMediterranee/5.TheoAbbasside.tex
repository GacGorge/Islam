\chapter{Approche théologique dans un monde musulman éclaté (VIII-XI).}

\mn{Séance 5 (Charbel (VIIIe Attalla). ) du 14 février : Approche théologique Les
 relations islamo chrétiennes dans un monde musulman éclaté (VIII-XI
siècles).}

\section{Introduction - prolégomènes}
Les Manichéens, une religion qui a eu énormément de fidèles (première religion ?).
\begin{Def}[manichéisme]
\end{Def}

\paragraph{Une identité arabe plus qu'une identité islamique au Ième siècle} L'identité musulmane a été construite à l'époque du Calife Omeyyade Abd al Malik (685-705). Et le Coran aurait été établi en 685 \sn{Coran des historiens}. Uthman en 646, a déjà une version. Mais certains traditionalistes citent Abu Bakr comme celui qui construit le Coran, pour affirmer son autorité.

Abd al Malik va ajouter la deuxième Shaada : "et Mahomet son prophète". 

\paragraph{Omar II, 743} va radicaliser la vision de XXX. Calife de l'année 100, approche millénariste.
Bcp de ses décisions vont être prises par sa vision des temps derniers. Pour comprenser, il va être moins tolérant.

\paragraph{Sources critiques} Pas bcp d'archéologie. 

\paragraph{Himyanite} Dans l'actuel Yémen (Ouest), des tribus qui passent d'abord au judaisme puis chistianisme  puis islamn. 


\paragraph{L'islam nait dans un milieu de petites villes} pas complétement désertique.


\paragraph{des mariages chrétiens et musulmans} sans qu'il soit nécessaire de changer notre pratique. 


\paragraph{le dhimmi} une tolérance. Mais un facteur économique important. \sn{\textit{Dhimmi et financement des conquêtes} de Charbel Attalla}. Quelque chose de realPolitik. 



\section{Abu Qurra}


\paragraph{Un disciple de Jean Damascène} 740-825, Evèque Melkite de Harran, lieu de traduction du grec en arabe en Mésopotamie. Abbasside : pendant l'âge d'or.  Le \textit{père du dialogue dogmatique}. On va sortir d'une polémique anti-autre et on va entrer dans un début théologique, pour consolider sa propore foi et pousser l'autre à se convertir. 

\paragraph{A connaître, le nom des califes Abbassides} Hârûm al-Rachid (786-809), Al Amîn (809-814) et Al-Mam'um (813-833). 

\paragraph{Al-Mam'um (813-833) et l'épreuve} Al-Mihma, l'épreuve. Al-Mam'um est le calife motazilite politique qui va rechercher à imposer cette vision rationaliste.

\paragraph{L'époque du débat} et certains l'appellent l'âge d'or de l'Islam. 


\paragraph{le Credo du mutazilisme}
\begin{itemize}
    \item Coran créé\sn{vs la vision de l'ange Al Jibril sur le Coran Incréé. Parmi les 7 gammes musicale, la gamme du Hijaz est triste, pour la récitation du Coran. Même gamme pour les chrétiens pour la semaine sainte. Pourquoi cette gamme ? parce que l'appel à la conversion doit avoir les larmes}. Il est proche de l'inspiration chrétienne. Privilégier le sens métaphorique et symbolique. 
    \item Responsabilité et libre arbitre et non seulement le destin. 
    \item Dieu n'est pas responsable du mal
    \item les noms de Dieu sont à dégager d'une manière symbolique
    \item l'enfer n'est pas éternel. \sn{Isaac le Syrien va prier pour la conversion de Satan}
\end{itemize}
Cette première théologie  est le fruit de la rencontre avec les Chrétiens et la pensée grecque.  


\paragraph{12 traités } sur la liberté, trinité, la Vraie Religion,...\textit{prouver qu'il a un fils},.. la mort volontaire du Christ. Les 12 traités nous indiquent les termes du débat.  

\paragraph{Deux discussions} L'une lors de la cour de Al-Mamun. En tant d'évêque, il participait à la cour de Al-Mamun. 


\paragraph{La Trinité}
\begin{quote}
    \mn{Traduction inspirée de A. Ducellier, Le Miroir de l'Islan (Collection Archives, Julliard-Gallimard), p. 155-157.}
La pensée des Agarènes et tout leur souci consiste à nier la divinité du Verbe de Dieu, et de toutes parts ils rassemblent leurs forces pour montrer qu'il n'est pas Dieu ni le Fils de Dieu. En effet, leur faux prophète, qui avait été à l'école d'un Arien, leur a transmis cette doctrine athée et impie. Cela explique que l'un d'entre eux qui se vantait du poids de ses propos au cours d'une réunion publique, interrogea l'évêque en ces termes: 

\begin{itemize}
\item \textsc{Comment nommes-tu le Christ,} Théodore\sn{Nom de l'évêque, Théodore Abu Qurra} ? 
    \item \textit{L'Evêque}: Dieu et Fils de Dieu.
\textit{Le Sarrasin}: Dieu donc pourrait avoir un fils ?
\item \textit{L'Evêque}: Il est impossible que Dieu n'ait pas un Fils.
\item \textit{Le Sarrasin}: Pourquoi ? Comment cela ?
\item \textit{L'Evêque}: Est-il possible à Dieu d'être sans pouvoir ?
\item \textit{Le Sarrasin}: Certes non.
\item \textit{L'Evêque}: Et sur qui Dieu exerce-t-il son pouvoir ?
\item \textit{Le Sarrasin}: Sur ses créatures.
\item \textit{L'Evêque}: Par conséquent, son pouvoir est accidentel, et non naturel, on pourrait dire qu'il a un pouvoir adventice et acquis, récent et on coéternel. En effet, avant ses créatures, comme tu le dis, il ne commandait pas; c'est donc que ses créatures furent cause de son pouvoir, qu'elles l'ont choisi pour exercer sur elles le pouvoir d'un 
maître, cc qu'il n'était pas auparavant... A une telle série d'absurdités viendra du reste s'en ajouter une autre encore plus absurde.
\item \textit{Le Sarrasin}: Laquelle ?
\item \textit{L'Evêque}: C'est que le pouvoir d'un roi terrestre devrait être considéré comme incomparablement meilleur et plus vénérable que le pouvoir de Dieu.
\item \textit{Le Sarrasin}: Comment cela ?
\item \textit{\textit{L'Evêque}}: Parce que le roi terrestre commande à des êtres de même substance et vivant dans le même temps que lui, alors que Dieu a pouvoir sur des choses de tout autre essence et qui lui sont de beaucoup postérieures.
\item \textit{Le Sarrasin}: Je ne comprends pas ce que tu dis !

\end{itemize}
\end{quote}

Il a lu les motazilites. La coexistence d'un fils pré etermel est une condition de la toute puissance de Dieu. Sinon, la puissance de Dieu serait accidentelle, c'est à dire qu'elle commencerait au moment des créatures et en plus de nature différente entre Dieu et ses créatures. 


\begin{quote}
  \begin{itemize}

\item \textit{L'Evêque}: Si quelqu'un s'avançait et s'adressait au roi en lui disant.
"Salut, roi des ânes !", que mériterait-il à ton sens ?
\item \textit{Le Sarrasin}: Le dernier des châtiments.
\item \textit{L'Evêque}: Quelles choses sont, par nature, plus proches l'une des autrès: Dieu et ses créatures ? ou le roi terrestre et les ânes ?
\item \textit{Le Sarrasin}: Le roi et les ânes, car ce sont, l'un et les autres, des créatures, de même nature et également esclaves. Mais continue donc,
- je te le demande; parle: Sur quoi donc s'exerce le pouvoir de Dieu, puisque, comme tu le dis, il ne s'exerce pas sur ses créatures ?
\item  \textit{L'Evêque}: Ceux sur qui s'exerce le pouvoir sont de trois sortes: ceux qui sont plus grands que le maître, ceux qui lui sont inférieurs et ceux qui lui sont égaux. Prétendre que Dieu commande à de plus grands que lui est un blasphème, car il n'y a rien de plus grand que Dieu; qu'il commande à de plus petits que lui est sans valeur, en raison des absurdités déjà mentionnées. Reste à parler des éléments égaux qui, comme lui, n'ont pas de commencement.
\item \textit{Le Sarrasin}: Et qui donc est égal à Dieu et, comme lui, sans commencement, et pourtant subit son pouvoir?
\item \textit{L'Evêque}: Ceux sur qui s'exerce le pouvoir sont de trois sortes:
\begin{itemize}
    \item ceux qui choisissent ce pouvoir,
    \item  ceux qui sont contraints (tyrannisés),
    \item  et ceux qui, sans le choisir, ne sont pas non plus tyrannisés.
\end{itemize}

Mais Dieu ne commande pas à ceux qui le veulent bien: un pouvoir dépendant du choix d'autrui est un pouvoir acquis, indigne (de
Dieu) comme nous l'avons dit.
Mais il ne commande pas non plus à des gens tyrannisés, ce qui serait inconvenant, avec toutes les autres absurdités qui en découlent.
Reste a dire que Dieu ne commande ni à des gens qui le veulent, ni à des gens qui le refusent, mais à des gens qu'il gouverne par pouvoir naturel.

\item \textit{Le Sarrasin}: Mais quel est celui que Dieu gouverne par pouvoir naturel ?
 
\item \textit{L'Evêque}: Il y a trois sortes de gouvernants:
\begin{itemize}
    \item   celui qui est élu,
    \item le tyran,
    \item le maître naturel.
\end{itemize}

Dieu, lui, ne commande pas grâce au suffrage de ses sujets: comme nous l'avons dit, cela conduit à des absurdités dans le cas de Dieu. Mais il ne commande pas non plus en tyran: en effet la tyrannie est absolument étrangère à Dieu27. Reste alors la troisième manière, qui revient à dire que Dieu commande bien au-dessus de tout suffrage et pur de toute tyrannie, c'est-à-dire naturellement:
\item \textit{Le Sarrasin}: Mais quel est celui qui est, par nature, sous le pouvoir de Dieu?
\item \textit{L'Evêque}: Son Fils, par qui Dieu est reconnu comme maître et comme Père sans commencement. Et cela parce que tout chose naturelle est antérieure à la volonté librement choisie.
\item \textit{Le Sarrasin}: Que veux-tu dire ?

\item \textit{L'Evêque}: Avant de vouloir respirer, nous respirons; avant de vouloir entendre, nous entendons; avant de vouloir voir, nous voyons. Eh bien donc, ô négateur de la divinité du Verbe de Dieu, voici démontré que le Fils est consubstantiel à Dieu, qu'il est comme lui sans commencement et éternel, qu'il participe, avec son père, au principe inhérent à la divinité.
\end{itemize}

\end{quote}

\section{Textes}


\subsection{Al Jahiz (776-869)}

\paragraph{Les chrétiens ne sont pas ce que l'on croit}
\begin{quote}
    

Si le peuple musulman savait que les Chrétiens, en particulier les Byzantins, n'ont ni science, ni littérature, ni vies profondes, mais qu'ils sont seulement habiles. de leurs mains dans la tournure, l'ébénisterie, la sculpture. le tissage des étoffes de soie, il ne les compterait plus parmi les gens cultivés et supprimerait leurs noms du Livre des philosophes et des sages, car La Logique, le traite de La Génération et La Corruption, La Météorologie, et d'autres ouvrages sont d'Aristote qui n'était ni chrétien, ni byzantin; l'Almageste est l'ouvre. de Ptolémée qui Hétait ni chrétien, ni byzantin, la Géométrie. Euclidienne est d'Euclide qui n'était ni chrétien, ni byzantin : La Médecine est de Gallien qui n'était ni byzantin, ni chrétien; il en est de même des ouvrages de Démocrite; Hippocrate, . Tous ces hommes appartenaient à un peuple qui a disparu, mais dont le génie a laissé; des traces (profondes): ce sont les Grecs, Leur religion n'était pas celle des Chrétiens, leur littérature n'avait rien de commun avec la leur. Les Grecs étaient des savants, les. Byzantins sont des artisans: Ceux-ci ont mis la main sur les livres grecs grâce au voisinage des deux peuples et à la proximité de leurs deux pays. ils se sont attribués certains de ces livres et. en ont adapté d'autres à leur religion: Pour, ceux des ouvrages qui sont trop célèbres et pour les sciences dont tout \item le monde sait qu'elles sont d'origine grecque, ne pouvant changer les noms de leurs auteurs, ils ont prétendu que les Grecs étaient une des tribus qui constituait le peuple romain.
"C'est pourquoi, ils proclament la supériorité de leur religion sur celle des Juifs et' ils méprisent celle des Arabes et des Hindous, si bien qu'ils vont jusqu'à prétendre que nos savants et nos philosophes n'ont fait que suivre la trace des leurs. Voilà ce qu'il en est !

\paragraph{Même convertis à l'islam, ce sont des hypocrites}

Leur religion, que Dieu te soit miséricordieux, a des analogies avec l'athéisme et concorde sur certains points avec les doctrines des matérialistes. Les Chrétiens sont un des facteurs de l'inquiétude morale et du doute. Ce qui le prouve dest que dans aucune relicion que la leur il n'y a autant d'hérétiques et autantle monde sait qu'elles sont d'origine grecque, ne pouvant changer les noms de leurs auteurs, ils ont prétendu que les Grecs étaient une des tribus qui constituait le peuple romain.
C'est pourquoi, ils proclament la supériorité de leur religion sur celle des Juifs et ils méprisent celle des Arabes et des Hindous, si bien qu'ils vont jusqu'à prétendre que nos savants et nos philosophes n'ont fait que suivre la trace des leurs. Voilà ce qu'il en est 1 Même convertis à l'islam, ce sont des hypocrites:
Leur religion, que. Dieu te soit miséricordieux, a des analogies avec l'athéisme et concorde sur certains points avec les doctrines des matérialistes. Les Chrétiens sont un des facteurs de l'inquiétude morale et du doute. Ce qui le prouve d'est que dans aucune religion que la leur il n'y a autant d'hérétiques et autant d'adeptes plus enclins au doute et dont la foi soit plus vacillante. Il en est ainsi de tous ceux qui, possédant peu d'aptitudes intellectuelles, se mêlent néanmoins d'approfondir les questions métaphysiques. Na-t-on pas constaté également que la plupart des hérétiques qui ont été mis à mort, parmi ceux qui pratiquaient ostensiblement la religion musulmane, étaient ceux dont les parents etaient Chrétiens ? Et de nos jours, si l'on voulait dénombrer ceux dont la foi est douteuse, on trouverait que le plus grand nombre d'entre eux est de descendance chrétienne.
\end{quote}

\section{Ali b. Rabban al-Tabari (m. 855)} 

\paragraph{Un chrétien qui se convertit à 70 ans} ouvert. Ecrit un livre pour montrer aux chrétiens l'aspect raisonnable de l'Islam. Se convertit sous le Calife traditionaliste, Al Mutawaqil, et qui impose une forte pression à la conversion.


\paragraph{Même quand il y a débat, on affermit son semblable avant de convertir l'autre} 

\paragraph{Orientations pour le dialogue au 9° siècle }
\mn{Kitab al-din wal-dawla, Ed. A. Nuweihed, Dar al-afag al jadida, Beyrouth, 1979, 239 p., p. 34-36 Réfutation aux Nazaréens} 
\begin{quote}
    

Dans son Livre parfait, Dieu dit: "Dites: Nous croyons en Dieu, à ce qui nous a été révélé, à ce qui a été révélé à Abraham, à Ismail; à Isaac, à Jacob et auseitribus; à ce qui a été donné à Moise et à Jésus : à ce qui a été donné aux prophètes. de la part. de leur: Seigneur.
Nous n'avons de préférence.pour aucun
dentre eux; nous sommes soumis à Lui" (Cor.. 2;136).
Et Il a dit: "Le Prophète a cru à ce qui est descendu sur lui de la part de isonSeigneur. Lui et les croyants; tous ont cru en Dieu, en ses anges, en ses Livres et en ses prophètes. Nous ne faisons pas de différence entre ses prophètes",
eto: (Cor., 2,285).
Quant à ceux qui associent à Dieu d'autres divinités, ou qui Lui attribuent un partenaire; Il dit:
\begin{quote}
    "Dis: Lui, Dieu est Un ! Dieu !.
L'impénétrable ! Il
m'engendre:pas; il n'est pas engendré ; nul n'est égal à Lui !" (Cor. 112) 
\end{quote}
Il dit encore: "Dis: O Gens du Livre ! Venez à une parole commune entre nous et vous: nous n'adorons que Dieu ; nous ne lui associons rien ; nul parmi nous ine se donne de Seigneur, en dehors de Dieu. S'ils se détournent, dites-leur:
Attestez que nous sommes vraiment soumis". (Cor. 3,64) Il dit aussi: "Est-ce que celui qui a fondé son édifice' sur la crainte révérencielle de Dieu et pour lut plaire n'est pas meilleur que celui qui a fondé son édifice sur le bord d'une berge croulante, rongée par une eau gui fait' crouler la batisse et son bätisseur dans le feu de la Géhenne ? - Dieu ne dirige pas un peuple
injuste": (Cor. 9,109)
C'était ces points que la prédication (de Mohammed) visait, c'était sur eux af alftoit avec eux au'il commenca à promulguer Quant à ceux qui associent à Dieu d'autres divinités, ou qui Lui attribuent ¿un' partenaire; Il dit: "Dis: Lui, Dieu est Un I Dieu. 1.
wengendre pas ; il n'est pas engendré; nul n'est égal à Lui !" (Cor. 112)
: L'impénétrable ! I
Il. dit encore: "Dis: O Gens du Livre ! Verier à une parole commine entre nous'et vous: nous n'adorons que Dieu ; nous ne lui associons' rien ; nul parmi nous'ne se donne de Seigneur, en dehors de Dieu S'ils se détournent, dites-leur:
Altestez que nous sommes vraiment soumis". (Cor. 3,64) Il dit aussi: "Est-ce que celui qui a fondé son édifice sur la crainte révérencielle de Dieu et pour lui plaire n'est pas meilleur que celui qui a fondé son édifice sur le bord d'une berge croulante, rongée par une eau qui fait crouler la bâtisse et son bâtisseur dans le feu de la Géhenne ? - Dieu ne dirige pas un peuple injuste". (Cor. 9,109) C'était ces points que la prédication (de Mohammed) visait, c'était sur eux qu'il fondait l'édifice de son appel, et c'était avec eux qu'il commença à promulguer la législation de sa religion et les exigences de sa doctrine que refusèrent de recevoir les Arabes polythéistes et les détenteurs du Livre révélé. Ils ont caché son nom et changé son portrait que contenaient les livres de leurs prophètes - la paix soit sur eux I Cela, je vais le démontrer, l'arracher à son secret, le dévoiler, pour que le lecteur puisse le voir clairement et s'affermir dans sa confiance et sa joie dans la religion de l'Islam.



Pour ce faire, je vais emprunter une voie plus directe et plus fructueuse que celle qu'ont empruntée d'autres auteurs sur le même sujet. Certains d'entre eux ont : abrégé, raccourci et tronqué leur présentation, au point de ne pas l'expliquer suffisamment; d'autres ont écrit des poèmes contre, les Gens du Livre, sans même connaître leurs Livres; d'autres encore ont noirci les premières pages de leur livre avec des harangues s'adressant aux Musulmans plutôt qu'aux polythéistes, puis ont entrepris de coucher leurs arguments d'une manière laborieuse et compliquée. Leur adversaire pourrait dire avec raison que ces auteurs ressemblent à quelqu'un qui ramasse du bois dans l'obscurité et qui prend sans les distinguer des brindilles et des bûches, ou encore à quelqu'un qui est emporté par un torrent et qui crie soudain. en mêlant paroles articulées et hurlements stridents; il semblerait qu'ils ont discuté non pas pour démontrer le. vrai mais pour le cacher, non pas pour. éclairer, mais pour aveugler, non pas pour faciliter les choses mais pour les compliquer.
\end{quote}
Critique du Motazilisme, pas toujours très clair. 

\begin{quote}
    
Celui qui entreprend d'écrire un livre sur un sujet aussi élevé, capable de guider, d'éclairer et de nourrir des gens de toutes religions, doit le faire de façon compréhensible et facile : il doit discuter et débattre avec son adversaire et non l'attaquer et l'insulter; il doit être intelligible et non pas obscur; courtois et non pas offensant; il doit se montrer aimable, habile a tirer au clair ce qui est obscur et à présenter des arguments et des réparties qui touchent l'adversaire et le conduisent a abandonner ses positions doctrinales et sa foi. En agissant ainsi à l'égard de son adversaire, il en fera sa monture, le percera de ses flèches, le conduira par la bride.
C'est ce que j'ai essayé de faire, avec l'aide de Dieu, le Très-Haut, rendant mes. phrases faciles pour que le lecteur les comprenne et ne reste pas dans le doute.
Aux "adeptes des religions protégées™\sn{Dhimma}, je n'ai laissé sans l'aborder, le réfuter et le résoudre, aucun argument, ancun problème difficile, aucun point de désaccord. le l'ai fait grâce à l'aide et au secours de Dieu, et avec l'approbation de son Calife, Imâm Ja'far Al-Mutawakkil 'alâ Allâh, Commandeur des Croyants. - que Dieu prolonge sa vie - qui m'a guidé et m'a fait profiter. des mots tombés de sa bouche\sn{le livre se présente comme écrit à la demande du Calife qui organise une campagne d'islamisation de la société et de l'administration} C'est son intention et son désir que de tels livres se répandent et se perpétuent pour renforcer les motifs de crédibilité de la foi, faire triompher ses arguments et montrer ainsi, son zèle pour la religion à tous ceux qui l'ignorent et qui ne reconnaissent pas combien Dieu a fait prospérer l'Islam et ses adeptes pendant son règne, combien Il les a comblés de ses bienfaits ou combien Il s'est fait reconnaître à tous par la douceur de son administration les rendant plus nombreux, plus prospères et plus honorés.


    
\end{quote}
\paragraph{Ad interna}
Utilisation de l'apologie pour prêcher sa propre communauté ? Dans une période "bloquée", le genre apologétique est un moyen de discuter en interne, au sein de la communauté musulmane.

C'est une réponse à ceux qui veulent retrouver l'apologétique pour le dialogue.

\paragraph{Les raisons du refus de croire}


\begin{quote}
 
J'ai trouvé que les gens qui se sont opposés à l'Islam l'ont fait pour quatre 
\begin{itemize}
    \item 1. A cause de doutes sur l'histoire du Prophète - Dieu le bénisse...
    \item 2. Par mépris et insolence, collective,
    \item 3. Par tradition et coutume,
    \item 4. Par folie ou stupidité.
\end{itemize}
\end{quote}
Quatre raisons des chrétiens par rapport à l'islam.

\begin{quote}
    
Par ma vie, s'ils avaient discerné et saisi la vérité de cette histoire, l'auraient pas rejetée. Mais puisqu'ils ont recherché les desseins de Dieu tout t s'opposant à son plan, nous devons viser à leur prouver la vérité de cette histoire,48. Lit. "Les gens de la Dhimma". Il s'agit du statut: politique des juifs et des chrétiens dans la Cité islamique. Sur ce statut, voir les documents, p. 49 et ss.
49. Le livre se présente comme écrit à la demande du Caliphe qui organise une campagne dislamisation'de la société et de l'administration.

les délivrer du doute, leur expliquant, dans son principe et ses applications, l'étude des traditions, de leurs faiblesses et de leurs convergences, le moyen de distinguer les vraies des fausses, et les raisons ou les motifs pour lesquels les peuples ont accepté leurs prophètes et accueillis leurs missionnaires.
Ensuite nous établirons un parallèle entre notre message et le leur, entre les témoins qui nous l'ont transmis et ceux qui leur ont transmis le leur; si les raisons que nous avons de croire en notre Prophète50 sont les mêmes que celles quils ont de croire dans le leur, ils n'auront aucune excuse devant Dieu et devant leur conscience pour refuser de croire en notre Prophète tout en continuant de croire dans le leur en effet, si deux adversaires présentent la même preuve à I'appui d'une même revendication, leurs droits sont identiques, et ce qui est dû à Mun est aussi nécessairement dû à l'autre61,
\end{quote}

 


