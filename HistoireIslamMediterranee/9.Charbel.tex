\chapter{mission islamo-chrétienne à partir de la chute de Bagdad}

\mn{Charbel 28/3/23}

\subsection{rappel historique}

un islam élcaté et divisé
la parenthèse al Hakim

\subsection{rappel du cours}
\begin{itemize}
  
\item reconquista musulmane arrêtée par l'invasion mongole (1220 - 1260)
\item le Jihad dans un tel tcontexte
\item les mouvement missionnaires chrétiens du 11 et 12e siècles
\item de la chute de Bagdad à la prise de COnstantinople, l'explosion missionnaire avec les ordres mendiants.
  \item 
\end{itemize}

\subsection{les dominicains intègrent la culture} 

\paragraph{langue, culture, contextualisation}Apprennent la langue et la culture. \sn{\textsc{{Aujourd'hui on veut discuter avec l'autre sans passer par la culture}} Olivier Roy (à lire)}

\paragraph{Raymond Marti} Grand islamologue

\paragraph{Saint Thomas d'Aquin}
Question d'un théologien qui entre en dialogue. Pourquoi Saint Thomas n'a pas compris que le dialogue était la source même de Dieu ?  
Quelques pistes : perspective théléologique, la vérité à découvrir au fur et à mesure.

Il va lire Jean Damascène et Raymond Marti, l'islamologue.
il écrit
\begin{itemize}
    \item la somme contre les gentils - 1264, à la demande de Raymond Painafor pour l’apologie à l'usage des dominicains partant en Espagne.
    \item contre les infideles (?) 15 pages; pour un ecclésiastique d'Antioche.  Refuse d'utiliser des éléments rationnels pour démontrer la foi. Contre les objections des musulmans dans leurs polémiques. 
    
\end{itemize}

\paragraph{POur Saint Thomas, la raison ne donne pas la foi} il pose juste des questions. 


\paragraph{Pour aller vers l’autre, il faut être droit dans ses bottes théologiquement}


Il y a des fondements qui vont engendrer du positif dans le futur : habiter chez l’autre. 
Sinon, on reste dans le vague, le témoignage spirituel sans engager le théologique

\paragraph{vision de St Thomas assez dure contre l'Islam} morale musulmane accomodante, conversion par la force. 

\paragraph{Guillaume de Tripoli} lui aussi dominicain, à Tripoli, franco-italien né au Liban, à Tripoli. La force de Guillaume est d'être né dans des quartiers avec des musulmans. Il parlait arabe dès son enfance.
\begin{itemize}
    \item adaptation, premier principe du dialogue : \textit{s'adapter au mode de vie et à l a langue}.
    \item pas de contrainte.  Il trouvait très négative les retombées des croisades pour la mission. 
    \item éviter l'apologétique; la foi ne passe pas par le dialogue de la raison.
    \item la victoire du Christ; sens qu'il faut y aller avec le verset : \textit{je vous précède en Galilée}. 
\end{itemize}

\section{et du côté musulman, y a t'il une mission}

\begin{Def}[darwa]
pas vraiment la mission.
\end{Def}

\paragraph{le plus proche de la mission, ce sont les soufis} Les musulmans ont converti par les conquêtes, \textit{par la force et le glaive} dit S. Thomas même s'il faut garder un sens de la mesure.

\subsection{les franciscains}

\paragraph{différent avec les dominicains} ce n'est pas une démarche intellectuelle, d'inculturation.
Pour Saint François, il n'y a pas d'autres règles que l'Evangile. Dans cette règle, il va aborder la question du martyr.

\paragraph{martyr} les missionnaires partaient pour la mort. Pour les frères qui cherchaient le martyr, \textit{ce n'est pas le but de l'action missionnaire mais un risque}.

\paragraph{deux manières de la mission} S. François :
\begin{itemize}
    \item l'attraction à travers le témoignage de la vie
    \item l'annonce directe. \textit{Lorsque le missionnaire voit que cela plait au Seigneur.}
\end{itemize}

Le centre de la foi, c'est le \textit{credo}.

\paragraph{revision du pape} pour le rendre plus juridique. Règle \textit{parmi les sarrasins}.
Dans la première règle, on trouve qu'il est trop vague car il fait de la théologie. Beaucoup de martyrs franciscains. 1220 à Marrakech, Ceuta 1227. Comme si la vie culturelle semblait les épargner.

\paragraph{Raymond Lulle} Tiers ordre franciscain. le seul franciscain qui va développer une certaine apologie rationnelle. 

\paragraph{La rencontre d'El Malik et de François} Il va pour annoncer l'Evangile et convertir El Malik Al kan (?). Mais la sainteté de François va toucher Al Malik sans qu'il se convertisse. Quelque chose de mystérieux dans la rencontre. \sn{Christian de Chergé en parle un peu : quand on porte le christ dans son coeur, chaque rencontre ne fait que vibrer ses entrailles. l'autre qui porte le projet de Dieu. }
5ème croisade. Le sultan d'Egypte était connu pour ses dispositions conciliantes. Il proposait de rendre Jérusalem aux croisés s'ils se retiraient. 
Malheureusement, le légat du pape, Pélage, va refuser l'offre de conciliation. Et la guerre va être un désastre.
En 1220, François va aller discuter avec lui. Al Malik va le protéger de l'animosité des lettrés musulmans de l'époque.

A la 6ème croisade, ce même sultan va rencontrer en 1229 Frédéric II qui régnait sur la Sicile et là encore, obtention de Jérusalem et d'autre trève. 
Ce même Frédéric avait des lettrés musulmans et des troupes musulmanes. \mn{\textit{rencontre miracle : là où la guerre abonde, le dialogue surabonde}}.

\section{Chute de Constantinople}
\paragraph{Chute de constantinople 1453} En 1467, chute de la Grèce, Albanie,... Syrie,... 15xx
De nouveau, une préoccupation pour les Européens et surtout les Autrichiens. avec le siège de Vienne (XVII). 

\paragraph{Humanisme et anthropocentrisme} face à un islam qui reste théocentrique. 
Etat nations. On passe au dialogue entre les \textit{francs} et les \textit{turcs}, \textit{espagnols} et \textit{turcs}. Moins de poids religieux. Pauvreté spirituel, zèle missionnaire très diminué.

\paragraph{2 attitudes}
\begin{itemize}
    \item Dialogue : même dans les périodes les plus sombres, le dialogue continue\sn{George de Trébizone, 1460 propose 5 conférences pour la paix. Jean de Ségovie, Nicolas de Sue et son \textit{rève d'unité religieuse} }
    \item Apologie : 
\end{itemize}



\paragraph{De pax fidei} : attitude synchrétiste. Nicolas va dire que toutes les religions s’enracinent. Une conférence mondiale de toutes les religions à Jérusalem : des chrétiens, musulmans et hindou. Il va réussir\sn{voir en quoi} : une seule religion dans une diversité de culte.     

\paragraph{pentarchie} les 5 patriarchats, Jérusalem, Alexandrie, Antioche, Rome et Constantinople, avec 4 dans l’empire Ottomans. Et aujourd’hui, même au Liban, nous subissons les conséquences de ces 4 siècles, avec la constitution des millets et des capitulations.  Va créer des grandes familles féodales avec des titres ( ?) si on payait.  Au début du XX, les maronites vont être affamés. Les français annexent la plaine de la bekaa.

\paragraph{Evangile de Barnabé}

\paragraph{XVIII : un regard objectif de l’Islam en Europe} sans visée missionnaire et apologétique. Le premier à étudier, Marachi (?). 1704 : traduction des mille et une nuit. Traduction du Coran (1649 ?). Voltaire va aussi philosopher sur l’Islam (imposture, religion naturelle sans dogme,…) Boulinvilliers (?) va exalter l’islam. D’autres vont mystifier l’islam comme une société musulame mythique (\textit{lettres persanes }de Montesquieu)

Combien l’islam a impacté la théologie chrétienne. On ne peut pas faire de la théologie sans connaître l’islam ( ;)) et les interactions avec l’Islam à travers l’islam.                  










