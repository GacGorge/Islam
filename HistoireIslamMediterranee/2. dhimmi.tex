\chapter{Le statut de dhimmi de la
naissance de l’islam aux Omeyyades}

\mn{Séance 2 (MarieCarmen Smyrnelis) du 24 janvier}


\section{Bibliographie}

\begin{itemize}
    \item CAHEN Claude,
Islam, des origines au début de l’Empire ottoman , Paris, Hachette, coll. « Pluriel », 2011.
    \item 
GAUDEUL
Jean Marie, Disputes ? Ou rencontres ?, l’islam et le christianisme au fil des siècles, Rome,
PISAI, coll. « Studi arabo islamici », n 12, 1998, 2 volumes.
    \item 
HOURANI Albert,
Histoire des peuples arabes , Paris, Seuil, coll. « Points », 1993.
    \item 
LEWIS Bernard,
Les Arabes dans l’histoire , Paris, Flammarion, « Champs », 1993.
    \item 
RODINSON Maxime,
Les Arabes, Paris, PUF, coll. « Quadrige », 2002.
\end{itemize}


\section{Les débuts de l’islam}

\subsection{la péninsule arabique à la veille de l'apparition de l'islam}

\paragraph{une lutte entre l'empire bysantin et l'empire sassanide}
\begin{Def}[Empire]
Ce qui caractérise l'empire, c'est l'hétérogéneité des peuples qui la compose. 
\end{Def}

 \subsection{Repères chronologiques}
 \begin{itemize}
   \item	570-580 : naissance de Mahomet à La Mecque
\item 	622 : installation à Médine (Hégire)
\item 	630 : conquête de La Mecque
\item 	632 : mort de Mahomet
 \end{itemize}


\paragraph{les Rasidun (632-661) : successeurs de Mahomet}
\begin{itemize}
  \item 	Abu Bakr (632-634)
\item 	Umar (634-644)
\item 	Uthman (644-656)
\item 	Ali (656-661)
\item 	les Omeyyades (661-749) => transfert de la capitale à Damas
\end{itemize}


\paragraph{sous les rasidun}
\item 	633-637 : conquête de la Syrie
\item 	639-642 : conquête de l’Egypte
\item 	634-652 : conquête de la Perse


\section{Les Rasidun les « bien guidés », successeurs de Mahomet et califes}

 
\section{La dynastie des Omeyyades}

\paragraph{}
\item 	696 : Chute de Carthage
\item 	710 : débarquement en Espagne
\item 	732 : bataille dite de Poitiers (Omeyyades mis en échec par Charles Martel)
 
\section{Le statut de dhimmi}
