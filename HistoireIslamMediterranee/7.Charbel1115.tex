\chapter{Les relations islamo-chrétiennes entre XIe et XVe siècles}
\mn{Charbel Attallah}

\section{Deux cas différents : l'Afrique de Nord et l'Espagne}


% --------------------------
\paragraph{pourquoi disparition des chrétiens en Afrique}
Au Xè, il n'y a plus de chrétiens. 
Quelques raisons possibles : 
\begin{itemize}
    \item Pas de théologiens marquants en Afrique à cette époque.
\end{itemize}
Cette disparition rapide fait penser que le christianisme était un produit colonial. 


\paragraph{En Andalus} un peu différent, la quasi-disparition des chrétiens vient probablement de l'attirance de la culture musulmane. 
Les Evèques sont \textit{collabo}. Pour la sécurité, on peut sacrifier son identité. 
cf Avaro (854) à Cordoue \mn{cf p. \pageref{par:Alvaro}}. On passe de l'admiration de la culture musulmane à la haine.


\paragraph{Résistance des martyrs de Cordoue} en 850-51, des jeunes qui affirment leur foi et sont tués par les qadis. Entraîne d'autres \href{https://fr.wikipedia.org/wiki/Martyrs_de_Cordoue}{martyrs}. Seule l'intervention des Évêques sur demande des califes calmera ces martyrs. Mais au Nord de l'Espagne, ces martyrs seront affirmés.
Il s'agissait de dénoncer le laxisme de certains chrétiens. Qu'est ce qui animaient ces martyrs andalous ? 

\begin{Synthesis}
    L'acceptation et la compromission des chrétiens en Al Andalus, génère une réaction de martyrs. 
\end{Synthesis}

% --------------------------
\section{Dialogue par dessus des frontières : la correspondance entre Umar II et Léon}
\paragraph{Califat Abbasside} pas de persecution chrétienne insistante. AlMansur autour de l'an 1000 est la notable exception, qui détruit St Jacques de Compostelle (997). 
\paragraph{Fatimide en 900} En 996, autre épisode violent, Al Akim, destruction de l'Eglise du St Sepulcre. mais ne qualifie pas l
\paragraph{Umar II, calife Omeyyade} auteur de capitalisation anti chrétienne.
\paragraph{Ecrit autour de 1000} Avant, des écrits apologétique à visée interne. Umar prend l'initiative d'écrire à l'empereur de Byzance. Probablement pseudographique. Dans cette correspondance.
Les arguments ne sont pas très originaux. Cela donne un bon exemple de controverse et laisse penser que c'est pseudographique. 

\begin{itemize}
    \item Trinité
    \item raison : Islam voulu par Dieu, la preuve rapidité de la victoire
\end{itemize}

Réponse de Léon : 
\begin{itemize}
    \item Victoire eschatologique de Jésus face à a Satan
\end{itemize}

\paragraph{première étape du dialogue islamo-chrétien} On est dans la polémique


\paragraph{les thèmes qui reviennent} la trinité, 
les chrétiens attaque l'immoralité de Mohammed (Zeinab,...) . Face à ce mouvement, au IX, un mouvement de sacralité de Mohammed.

\paragraph{Concentre les stéréotypes des siècles qui ont précédé} importance de l'authenticité des Ecritures (thème qui se développera au XIe).

\paragraph{Salut de l'autre}
Hors de l'Eglise point de salut et 
\begin{quote}
    Si vous ne m'aimez pas, Dieu ne vous aimera pas 3,55
\end{quote}
On a du mal à penser le salut pour l'autre.

\begin{Synthesis}
    une lettre epigraphique mais la première lettre de dialogue, qui reprend certes les apologétiques respectives mais les articulent.
\end{Synthesis}
% --------------------------
\section{L'après dialogue des philosophes}


avec Elie de Nisibe, on a une sortie, reflux de la vague philosophique dans le dialogue islamo-chrétien.

\paragraph{Touchant} A travers une expérience existentielle, un début de changement de paradigmes dans la rencontre  : plus partage. 

\paragraph{Elie de Nisibe} Elie de Nisibe (975-1046) Archevèque. 
% --------------------------
\subsection{Elie de Nisibe}
    

\paragraph{Un évêque, un ministre ...
et un miracle}


\begin{quote}
Il advint que le Vizir (que Dieu ait pitié de lui !) entra à Nisibe le vendredi
26 Gumâdâ !, de l'année dernière, à savoir l'année 720. J'entrai chez lui le samedi
suivant, alors que je ne l'avais jamais vu auparavant. Il me fit approcher de lui, m'honora et me fit asseoir à côté de lui.


Après avoir invoqué Dieu en sa faveur et l'avoir félicité de sa venue, je me
levai pour m'en aller. Il me fit alors arrêter et me dit: "Sache que depuis longtemps
Je désire avoir des rencontres avec toi et te demander beaucoup de choses. Je veux
donc que ta venue et ton départ de chez moi aient lieu selon mon désir". Je lui
répondit qu'il serait écouté et obéi, invoquai Dieu en sa faveur et m'assis. Après
m'avoir mis à l'aise et m'avoir tenu compagnie, après s'être informé de mes
nouvelles et de la marche de mes affaires et après avoir mentionné les savants et les
hommes de science, il me dit:


Sache que mon opinion au sujet des Chrétiens était \textit{autrefois} l'opinion de
qui a établi qu'ils sont impies et polythéistes21 . Mais \textit{maintenant}, je doute de leur
impieté et de leur polythéisme, à cause d'un miracle étonnant dont je fus témoin,
venant de leur religion. Et je doute également de leur monothéisme à cause de
certaines choses auxquelles ils croient qui obligent à douter qu'ils soient
monothéistes.
\end{quote}

\textit{une autre forme de monothéisme}. 
\paragraph{une rencontre liée à une expérience existentielle}

\begin{quote}    
Je dis: "De quoi donc le Vizir (que Dieu prolonge ses jours !) a été témoin
et qui l'oblige à douter de leur impiété ? Et qu'est-ce qui. dans leurs croyances. le
contraint de douter de leur monothéisme ?"
\end{quote}

\paragraph{Récit du miracle}

\begin{quote}
    Le Vizir me dit: "Quant à ce dont j'ai été témoin et qui m'a obligé à douter
de leur impiété, c'est ceci: Alors que je me trouvais la première fois à Diarbekir, je
me suis dirigé vers Badlîs, pour régler des affaires survenues alors.
A mon arrivée là, une grave maladie m'assaillit, qui me fit perdre mes
forces. Mon appétit cessa et je désespérai de moi-même. Je sortis de Badlîs pour
retourner à Maâfâriqîn, en sorte que, si Dieu le Très-Haut avait décrété à mon
sujet l'inévitable, ceci m'advint en cette ville, ou du moins près d'elle.
Or mon être n'acceptait rien, en fait de nourriture ou de boisson; et la
fatigue de la chevauchée m'avait causé une énorme peine. Je me mis à parcourir
chaque jour une petite distance, tandis que ma faiblesse augmentait, que mes forces
diminuaient et que la maladie s'aggravait et devenait plus pénible.
Je parvins alors à un couvent sur le chemin, connu sous le nom de Couvent
de Saint Mârî. J'étais plus faible que jamais, tandis que la maladie était plus forte
que jamais.
Sitôt arrivé au couvent, considérant la faiblesse dans laquelle je me
trouvais, je fis demander un peu de boisson et la pris, dans l'espérance que cela
soutint mes forces. Mais à peine était-elle parvenue dans mon estomac, que je la
rejetai. La faiblesse de mon être augmenta, et je désespérai de moi-même. Tous
ceux qui étaient avec moi furent inquiets.
Alors le moine chargé du service du couvent vint à moi et invoqua Dieu en
ma faveur. Il me fit apporter un peu de grenades et demanda aux serviteurs de les
égrener et de m'en présenter pour que j'en prenne un peu. Ils lui firent alors savoir
que j'étais incapable de parler ou d'entendre quelqu'un parler, que mon être
n'acceptait aucun aliment et que mon estomac ne retenait plus de boisson, pour ne
rien dire du reste.
Le moine insista auprès d'eux et leur dit: "J'aimerais que vous lui en
portiez, afin qu'il prenne de ces grenades. ne fut-ce qu'un peu: il en tirera profit,
grâce à la bénédiction de ce lieu.
Je fis donc signe à l'un des serviteurs d'accepter sa proposition, car je
m'accrochais à la santé. Je mangeai un peu de ces grenades, qui restèrent dans mon
estomac et le soutinrent. Du coup, j'en pris régulièrement, un peu à la fois, si bien
que mon être reprit force et que mon appétit se réveilla.
Or le moine avait cuisiné des lentilles pour les serviteurs, et les leur
apporta. Je les vis manger, j'en demandai et mangeai avec appétit. \sn{Elie de Nisibe, Kitab al-Majalis. Le livre des discussions. Jamais avec un ton polémique}
\end{quote}
\begin{quote}
    Voila donc ce qui me contraint de croire que les Chrétiens ne sont ni des impies ni des polytheistes.

    Quant à ce qui me contraint de croire, en ce qui les concerne qu'ils sont polytheistes, c'est le fait qu'ils croient que Dieu est une substance en trois hypostases. Ainsi adorent-ils trois dieux, et confessent ils trois seigneurs. Ils croient aussi que Jésus (qui est, selon eux, l'homme assumé de Marie) est éternel et incrée.

    Je dis les Chrétiens n'adorent pas trois dieux et ne croient pas que l'homme assumé de Marie est éternel et incréé.

    Le vizir dit : ne disent ils pas que Dieu est une substance en trois hypothases : Père, Fils et Esprit-Saint ?

    Je dis : Certes, c'est ainsi qu'il disent !
    Il dit : n'acceptent-ils pas le Credo que les 318 (pères de Nicée) ont décrété et mis par écrit ?

    Je dis : Certes, nous l'acceptions et l'exaltons.

    Il dit : Votre affirmation que "Dieu est en trois hypostases. Père et Fils et Esprit Saint", est impiété 
\end{quote}

\paragraph{Bcp de musulmans dans les sanctuaires chrétiens} bcp de femmes voilées à ND du Liban, ...
Bcp de guérisons de musulmans. Le miracle est un acte gratuit, fruit d'une providence universelle. 

\paragraph{forte symbolique et mystique} Question sur l'historicité du fait des symboles. 


\paragraph{Vizir muta'zilite} Un grand débat pour les savants musulmans, ce sont les attributs divins. \mn{\pageref{par:AttributdivinMutazilite}} Pour les mutazilites, les attributs sont inséparables de l'essence. Et du coup, la trinité pensée comme attributs.

\paragraph{Elie de Nisibe} nestorien ? Humanocentrique. 

\begin{Synthesis}
On passe d'une apologique interne à un échange de lettres.
On a en parallèle le dialogue entre philosophes.
Ici, avec Nisibe, on a un dialogue existentiel sur les miracles.
\end{Synthesis}


\paragraph{C'est le monotheisme qui sauve même si vous n'admettez pas la prophétie de Mohammed} On est l'année 1000. C'est vraiment assez sidérant, et une annonciation de Nostra Aetate.

\paragraph{Les trois grands paradigmes qui avancent pour arriver à ce miracle} Cette conversation paisible (7 soirées), 


% --------------------------
\section{Au début des Croisades}

\subsection{Lettre du Moine de France (vers 1076)}

\mn{Cf. A Turki, « La lettre du "moine de France »à Al-Muqtadir billâh,
roi de Saragosse, et la réponse d'Al-BâJ1. le faqîh andalou"
(présentation, tex1e arabe, traduction), in: Al-Andalus.. T.XXXI (1966).
p.73-15336}
\begin{quote}

Au nom de Dieu, le Compatissant, le Miséricordieux. Dieu bénisse notre
Seigneur Mohammed et sa famille .,
    Lettre du moine de France (\textit{Dieu la détruise !})
à
Al-Muqtadir billâh, seigneur de Sarragosse.


Au cher ami qui, nous l'espérons, deviendra un intime très proche, lui à
qui Dieu a conféré le pouvoir du royaume de ce monde, au noble roi, de la part
du plus humble des moines qui désire la repentance et la foi au Christ Jésus, fils de
Dieu, notre Seigneur.
. quand nous est parvenue la nouvelle de ta haute position dans le monde, ô
prince puissant, et celle de ta perspicacité à trouver un sens aux vicissitudes d'ici-bas,
nous avons décidé d'entrer en communication avec toi et de t'inviter à préférer
le Royaume eternel a celui qui passe et prend fin. Tu as déjà lu la lettre que nous
t'avons adressée précédemment, et à laquelle tu as répondu d'une façon que les gens
du monde trouvent honorable mais qui ne correspondait pas à notre désir d'une
réponse spirituelle.
C'est pour cette raison que nous avons tardé à te répondre tant nous avions
peur d'entreprendre un travail épuisant et stérile. Et en vérité, le Tout-Puissant qui
a choisi ses amis des avant la création du monde  et ne les a pas prédestinés à la
destruction\sn{Ep 1,4} a illuminé ton coeur\sn{1Th 5,9 ?} et lui a fait connaître la foi en Dieu qui te sauve, le Compatissant, le Miséricordieux, qui pardonne et te guide vers Sa connaissance. Il ne nous est donc pas possible de tarder davantage à faire tout notre
possible pour mener cette affaire a son terme, avec son aide. pour que tu en viennes
a communier avec nous à son Royaume, si tel est ton choix.


2. Pour cette raison, nous t'avons envoyé quelques-uns de nos frères qui
t'apporteront une Parole de Dieu selon ce que Dieu leur donnera dans ce but. Ils
expliqueront en ta présence la vérité de la religion du Christ. notre Seigneur, le seul
en qui nous devions croire41 et le seul dont nous attendons notre salut42. Il est
Dieu qui s'est fait un voile de notre forme humaine pour nous délivrer de la
perdition du diable par son sang innocent.


3. Sur ce sujet, ô noble roi, nous pourrions parler longuement si nous n'avions pas peur que tu éprouves de la peine en l'écoutant. Il s'agit là de la preuve
de la religion chrétienne et de la démonstration de son éminence bien qu'il soit au-dessus des capacités de la Raison humaine d'en comprendre l'essence. Le Royaume
de Dieu est trop grand et trop glorieux, pour que l'entendement humain puisse le saisir ou l'atteindre par raisonnement theologique. Et pourtant, c'est un des signes
miraculeux du Dieu Tout-Puissant qu'il dilate le coeur des fils d'Adam et y introduit
un esprit de connaissance pour que la Foi s'établisse dans leurs âmes. Et puisque la
terre était autrefois habitée par l'erreur et le monde souillé par l'idolâtrie, le Tout-Puissant
a Juge bon d instaurer, a la fin des âges, une époque nouvelle et de rendre
au monde la rectitude dont il avait été privé par Adam notre premier père.
C'est vers ce but qu'étaient guidés nos pères. bien avant Abraham , Isaac et
Jacob, et c'est cela qu'ont expliqué les prophètes après eux. Voilà la promesse que
Dieu avait faite et confirmée avant et après la révélation de la Loi (Torah), de
révéler la Sainte Incarnation. Ce plan n'est d'ailleurs pas contenu seulement dans
nos Écritures mais aussi dans celles des Juifs et de nos contradicteurs de la façon la
plus claire et la plus évidente.


4. Satan, le maudit, qui, dans sa jalousie pour Adam. avait exposé les gens
de ce monde à la mort, chercha. à corrompre cette sainte religion après la venue des
Apôtres qui gardèrent les habitants de la terre par leur prédication, et après la
victoire remportée sur le démon par les martyrs qui versèrent leur sang pour Dieu,
partout dans le monde, en obéissance à sa Sainte Loi. Comme il ne parvenait pas à
séduire les gens de ce monde pour les ramener à l'antique erreur de l'idolâtrie, il
entreprit de tromper les Fils d'îsmaël à propos du Prophète dont ils reconnurent la
mission conduisant ainsi de nombreuses âmes au châtiment de l'enfer. Dans le
passé, \textit{lblîs} avait péché et égaré l'humanité méritant ainsi de recevoir un sévère
châtiment au Jour où Dieu, notre Seigneur Jésus-Christ. ressuscitera les morts.
Maintenant il a péché de plus belle en causant la perte de ces grandes nations.
\end{quote}


% --------------------------
\section{Rencontres amicales}


