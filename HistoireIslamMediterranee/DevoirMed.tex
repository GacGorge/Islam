\chapter{Y a-t-il eu au Moyen Âge un dialogue entre l’islam et le christianisme ?}

%---------------------------------------------------------------
\begin{comment}

\paragraph{Instruction} 5000 signes, très synthétique. Fiche de lecture, idées principales du texte (plan haché : donc on peut rassembler les idées qui vient d'une conférence). Qui est Rémi Brague ? Quelle publication ? Arriver à la fois à faire des liens avec une séance de théologie ou d'histoire ?  En particulier lire Convivencia. Texte d'un chrétien. Deuxième partie est libre et on se permet sur 1/3 des signes, comment on a lu le texte "pas assez approfondi",... Se relire et éviter les fautes. 


Cette fiche de lecture, entre 4000 et 5000 caractères (espace compris), devra :  

- présenter brièvement  l'auteur de l'article

- mettre en évidence les idées fortes du texte dans un contexte d'intérêt croissant des chercheurs pour les relations entre musulmans et non-musulmans (cf. gros programme de recherche dirigé par John Tolan évoqué aujourd'hui)

- Votre conclusion, dense, vous permettra de faire un commentaire personnel sur les apports et limites de l'article, en le situant par rapport à la réflexion plus large que nous menons en cours et que vous entreprenez, de votre côté, par vos lectures personnelles
\end{comment}

%---------------------------------------------------------------


\paragraph{Rémi Brague est spécialiste de la philosophie médiévale arabe et juive.} Il  est connu pour son livre  \textit{l'Europe, voie romaine}\cite{brague_europe_2009}. 

 
Dans une conférence donnée en 2005 \cite{lejbowicz_y_2005}, il s'interroge sur la possibilité que les relations entre Islam et monde chrétien au moyen-âge aient pu déboucher sur un dialogue réel entre ces deux religions. 
%---------------------------------------------------------------
Il articule sa démonstration autour de trois idées principales : \begin{comment}
    : l'asymétrie entre monde Islamique et monde Chrétien, l'absence de tolérance religieuse au Moyen-âge, et le caractère largement polémique des \textit{dialogues} entre religions
\end{comment}



\paragraph{1. Asymétrie entre Islam et monde chrétien.} Islam et christianisme ont été forcés à cohabiter mais essentiellement en Islam. Du fait de leur ordre d'apparition, le rapport à l'autre est différent; Les chrétiens savent qu'ils ne connaissent pas l'islam alors que les musulmans croient qu'ils connaissent le christianisme. \begin{comment}
    Les chrétiens sont curieux de l'Islam alors que le christianisme est pour l'islam une \textit{vieille histoire}, connue mais dépassée. 
\end{comment}


\paragraph{2. Pas de réelle tolérance religieuse.} On ne peut guère parler de liberté de pensée dans le monde musulman (ni dans le monde chrétien) au moyen-âge.  Les communautés juives et chrétiennes sont tolérées mais humiliées sous le régime de la \textit{dhimma} qui s'applique aux religions dites du Livre. 

\paragraph{3. Pas de dialogue véritable entre religions.} La forme de la \textit{disputatio} entre des représentants de différentes religions est généralement un genre de fiction. Les vrais dialogues entre chrétiens et musulmans qui n'ont pas de visée apologétique sont rares.

\paragraph{}
L'exposé est à la fois stimulant et étayé, malgré la concision due au format de conférence.  Nous proposons de reprendre certains points et de les éclairer par les recherches récentes présentées en cours. 
\begin{comment}
    trois points en lien avec le cours : partant du mythe de la \textit{convivencia} dans un Islam fragmenté, nous souhaitons souligner l'existence d'un dialogue au niveau de la culture, et à un niveau plus religieux, la transformation de la théologie chrétienne et musulmane au contact l'une de l'autre.
\end{comment}


\paragraph{Les recherches récentes montrent une réalité contrastée de la \textit{convivencia} ou vivre-ensemble.} 
\begin{comment}
    Or, avec l'émergence de l'empire Abbasside, l'Islam se  fragmente avec trois califes à Cordoue, au Caire ou à Bagdad dans ce qui est pourtant considéré comme l'âge d'or de l'Islam.
\end{comment} 
Des juifs et des chrétiens ont été des acteurs importants de l'âge d'or de la civilisation arabe,
\begin{singlequote}
  [\ldots] à tel point qu'on présente parfois l'entente inter-religieuse  comme une sorte de « paradis perdu », une « culture de tolérance ». C'est le mythe de la \textit{convivencia} [\ldots] Comme de nombreux mythes, celui-ci se base sur une partie de vérité et une partie d'oubli.  \cite[p. 102]{tolan_nouvelle_2022}
\end{singlequote}
Ainsi, les recherches récentes sur la \textit{dhimma} montrent que si celle-ci interdit théoriquement la création de nouvelles églises, les contre-exemples abondent. 
A d'autres moments de l'histoire, malgré la protection théorique de la \textit{dhimma}, les chrétiens furent persécutés, même dans la mythique \textit{Al-andalus} sous les règnes des très rigoristes Almohades.

\paragraph{L'existence d'un dialogue au niveau de la culture.} La religion n'est pas le facteur ultime de la civilisation et est aussi influencée par de nombreux facteurs culturels, politiques ou économiques\cite[p. 40]{brague_sur_2023}. Avant de chercher un dialogue proprement théologique, le mythe de la \textit{convivencia} signale une hybridation culturelle (au sens large)  entre Chrétienté et Islam  : fonctionnaires à Damas ou traducteurs chrétiens à Bagdad, chrétiens en Andalousie au milieu du IXe adoptent la culture arabe, au grand dam de certains, comme le chrétien Alvaro.   Cette hybridation interroge chacun sur son identité propre, y compris religieuse (cf \cite[p. 84]{brague_sur_2023})


\paragraph{Un dialogue théologique à distance.} 
La confrontation à une autre religion - sous toutes ses formes et pas uniquement la \textit{disputatio} - oblige le théologien à clarifier  son discours, à entrer dans le questionnement propre de l'autre, sa manière de pensée : Averroes influençant la pensée thomiste, les mu'tazilites utilisant des outils théologiques venant probablement des chrétiens comme la théologie négative \cite{Candiard_theologie_2022}, ou traitant de questions (libre-arbitre,  attributs de Dieu,...), dont les polémiques avec les chrétiens pourraient être l'origine (cf critique de Jean Damascène \cite[p.216]{damascene_ecrits_1992}). 
\begin{comment}
L'\textit{ examen des religions} d'Ibn Kammuna \cite{ibn_kammuna_examen_2012} montre la proximité des outils théologiques utilisés pour défendre chaque religion.
Enfin, le lecteur moderne de  est frappé par sa distance culturelle avec ce texte, m.
\begin{singlequote}
   Nous leur répondons : ”Puisque vous dites que le Christ est le
Verbe et l’Esprit de Dieu, comment nous reprochez-vous d’être des \textit{associateurs}
? En effet, le Verbe et l’Esprit sont chacun inséparable de celui dont il
tire son origine ; si donc le Verbe est en Dieu, il est évident qu’il est aussi
Dieu. Si, au contraire, il est en dehors de Dieu, alors Dieu, selon vous, \textit{n’a ni
verbe ni esprit}. Ainsi, en voulant éviter de donner à Dieu des associés, vous
le mutilez.
\end{singlequote}
\end{comment}
\paragraph{Des exemples rares mais attestés de vrais dialogues théologiques.}
La recommandation d'at-Tabari n'a de sens que dans le cadre de vraies \textit{disputatio}:
\begin{singlequote}
    {discuter et débattre avec son adversaire et non l’attaquer et l’insulter ; il doit être intelligible et non pas obscur ; courtois et non pas offensant.\cite[p. 34-36]{tabari_refutation_1979}}
\end{singlequote}
Rémi Brague cite une \textit{disputatio} tolérante à Fez en 1394. Trois siècles plus tôt, en 1026-27, nous avons la trace d'une \textit{disputatio} rapportée par Elie de Nisibe. L'origine de ce dialogue vient de la curiosité du vizir suite à sa guérison miraculeuse dans un monastère chrétien. Le vizir s'adresse ainsi à Elie : 
\begin{singlequote}
    En effet, je crois que tout chrétien monothéiste est digne de louange et de la récompense finale, même s’il ne reconnaît pas la prophétie de Muhammad Ibn ‘Abdallâh (paix sur lui !). Mais c’est une condition de toute interrogation que d’approfondir la question et de faire des objections. N’attribue donc pas à mes paroles d’autre intention que l’interrogation, et non pas un quelconque autre motif \cite{ElieNisibe}.
\end{singlequote}

 

\paragraph{En guise de conclusion...} La conférence de Rémi Brague est stimulante mais peut être nuancée sur certains points. D'abord, les travaux récents des chercheurs comme J. Nolan montrent une réalité complexe des rapports entre chrétiens et musulmans. La rareté (et non l'absence) des \textit{disputatio} religieuses n'est qu'une facette du dialogue inter-religieux, qui passe aussi par la confrontation à un cadre culturel différent et à d'autres outils ou questions théologiques. 

\paragraph{...quelques pistes pour le dialogue inter-religieux contemporain.}

Aux nombreuses raisons avancées pour expliquer \textit{l'ankylose} de la pensée islamique après le XIIIe\cite{brague_sur_2023}, nous voudrions en proposer une autre:  l'Europe, avec la création des universités, a établi un vrai dialogue "sans confusion" entre théologie et les autres "\textit{sciences}", en premier lieu la philosophie. Ce dialogue universitaire, d'une grande fécondité, n'a pas eu son équivalent en Islam. Cependant, la confrontation uniquement avec les \textit{sciences} met en risque la dimension \textit{transcendantale} du christianisme. 

De façon symétrique, la pensée msulmane semble souffrir que la période perçue comme son âge d'or (VIIIe-XIIIe) soit très proche de son origine\cite[p.84]{brague_sur_2023} même si cette période est prolongée par les théologiens \cite{CandiardTheoIslam} et que des penseurs comme Al-Afgani ou Abduh ont montré au XIXè que le dialogue avec la culture de leur temps était possible. Or, au XIè siècle, lors de cet âge d'or, Ibn Kammuna \cite{ibn_kammuna_examen_2012} présente les trois religions monothéistes avec le substrat philosophique et les outils théologiques de son époque, d'une grande proximité d'une religion à l'autre. On peut donc avancer que les outils théologiques  utilisés par les théologiens chrétiens ou juifs contemporains (historico-critique, herméneutique, ...) pourraient être aussi pertinents pour la pensée musulmane d'aujourd'hui.  




\begin{comment}
  \begin{singlequote}
    il a tendance à considérer que son apogée s'est situé à son début et que son histoire postérieure est celle d'un éloignement constant à partir de ce sommet. Il attribue à Mahomet des déclarations en ce sens comme :\textit{ «Le pire dans les choses, c'est ce qu'elles ont de nouveau» }  
\end{singlequote}  
\end{comment}

 
  \section{  Une vision trop corsetée de l’islam}

\mn{John Tolan - Sur l’islam de Rémi Brague, Gallimard, 400 pages, 24 euros
}


Spécialiste de la philosophie arabe médiévale, Rémi Brague livre, dans Sur l’islam, une
série de réflexions sur la tradition musulmane, dont il distingue quatre sens : un rapport à la divinité, une religion, une civilisation, et l’ensemble des croyants marqués par l’héritage d’une civilisation – quatre sens qui se chevauchent et se confondent parfois. L’auteur montre la richesse de la tradition philosophique et scientifique en langue arabe, ainsi que l’importance de cette tradition dans l’Europe latine à partir du XIIe siècle. Mais, à ses yeux, parler de « dette » de l’Occident envers le monde arabe n’a pas de sens : les penseurs arabes ont simplement acca- paré, selon lui, les traditions intel- lectuelles grecques, perses et autres pour les transformer.

Le professeur émérite à Paris-I entend par ailleurs déconstruire les « discours édulcorés » de ceux qui présentent l’histoire des rapports entre souverains musulmans et sujets juifs et chrétiens de manière selon lui idéalisée, ou qui affirment que la violence et le djihad seraient étrangers au « vrai » islam, par essence pacifique : pour appuyer son propos, il cite de nombreux textes, allant des hadiths (recueils d’actes ou de paroles du Prophète et de ses compagnons) aux juristes et théologiens du Moyen-Age, qui légitiment la violence contre les ennemis de l’islam.

Si l’érudition de l’auteur et sa connaissance des textes apportent richesse et complexité à ces propos, elles contribuent dans le même temps à les rendre obscurs pour les non-initiés. Ainsi, évoquant les idéologies de conquête, il saute d’Al-Ghazali au XIe siècle à Sayyid Qutb au XXe siècle, sans expliquer que ce dernier était un théoricien des Frères musulmans. Ailleurs, il passe du voyageur ottoman Evliya Celebi au XVIIe siècle à Riad Sattouf au XXIe siècle. Ces sauts dans le temps supposent, de la part du lecteur, une connaissance fine de contextes historiques bien différents, contextes que l’auteur ne maîtrise pas toujours. Il parle par exemple d’ « Alphonse le Sage dans la Tolède fraîchement reconquise » , qu’il situe au XIIe siècle – en fait, Tolède fut conquise en 1085 par Alphonse VI de Castille, alors qu’Alphonse X « le Sage
» (1221-1284) régna surtout depuis Séville.
 
 \subsection{Manque de contextualisation historique}


Certaines erreurs sont plus gênantes. L’auteur « propose une loi très simple : chaque religion tolère – plus ou moins – celle ou celles qui l’ont précédée » . Loi très simple, en effet, mais assez éloignée de la réalité historique. Il affirme en particulier que « dans la chrétienté médiévale, il y avait une place pour les juifs» . C’est faire fi de la longue histoire de la persécution des juifs en Europe, du Moyen Age au XXe siècle ; de nombreux rois européens expulsèrent les juifs de leur territoire (dont le roi de France Philippe IV le Bel, en 1306). De l’époque médiévale jusqu’au milieu du XXe siècle, les juifs étaient mieux acceptés dans des Etats musulmans que dans l’Europe chrétienne.

On pourrait dire la même chose pour les chrétiens, car en terre d’islam florissaient une pluralité d’Eglises et de doctrines chrétiennes, ce qui n’était guère possible en Europe latine de tradition catholique. Aujourd’hui, les choses se sont inversées : il est souvent plus facile pour un musulman de pratiquer sa religion à sa guise, d’y réfléchir et de l’étudier en Europe ou en Amérique de
Nord que dans certains pays musulmans. Les raisons de ce bouleversement sont nombreuses et sujettes à débat ; mais l’absence de perspective historique dans l’ouvrage ne permet pas d’entrevoir cet état de fait, encore moins d’y apporter de la lumière.

\paragraph{djihad concept tardif}
Une meilleure contextualisation aurait en outre permis de nuancer certains propos. Rémi Brague souhaite mettre l’accent sur « la continuité dans les principes et les pratiques » de l’islam. Mais les pratiques et les doctrines dont il parle n’ont émergé que graduellement au cours des premiers siècles de l’islam. C’est par exemple à l’époque omeyyade (661-750) que fut défini le djihad de manière à justifier les conquêtes islamiques. Et si le droit, pour M. Brague, serait « le cœur même de l’islam » , cette discipline ne s’est formée que progressivement en islam – intégrant, entre autres, des éléments de droit romain et perse. 
 \paragraph{le statut de Dhimmi gagné de hautes luttes} Ainsi, le statut de dhimmi (« protégé ») n’était pas « octroyé » par les autorités musulmanes aux juifs, chrétiens et zoroastriens, comme le prétend l’historiographie musulmane que suit M. Brague ; il s’agit plutôt d’une co-construction, dans laquelle les autorités religieuses non musulmanes prirent activement part pour affirmer leur juridiction sur leurs ouailles, en négociation permanente avec les autorités califales.

En dépit de la grande érudition de l’auteur et de la finesse de certains de ses propos, l’image de la religion musulmane qui en sort est d’un légalisme rigide, qui évolue peu dans le temps. Un islam bien morne, que de nombreux musulmans auraient du mal à reconnaître, et qui correspond en bonne partie à celui promu par des wahhabites et autres salafistes. Là encore, un ancrage historique plus fort aurait permis de
remettre en perspective les luttes actuelles internes au sein de l’islam.
 


