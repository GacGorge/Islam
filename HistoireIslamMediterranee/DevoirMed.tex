%\chapter{Y a-t-il eu au Moyen Âge un dialogue entre l’islam et le christianisme ?}

%---------------------------------------------------------------
\begin{comment}

\paragraph{Instruction} 5000 signes, très synthétique. Fiche de lecture, idées principales du texte (plan haché : donc on peut rassembler les idées qui vient d'une conférence). Qui est Rémi Brague ? Quelle publication ? Arriver à la fois à faire des liens avec une séance de théologie ou d'histoire ?  En particulier lire Convivencia. Texte d'un chrétien. Deuxième partie est libre et on se permet sur 1/3 des signes, comment on a lu le texte "pas assez approfondi",... Se relire et éviter les fautes. 


Cette fiche de lecture, entre 4000 et 5000 caractères (espace compris), devra :  

- présenter brièvement  l'auteur de l'article

- mettre en évidence les idées fortes du texte dans un contexte d'intérêt croissant des chercheurs pour les relations entre musulmans et non-musulmans (cf. gros programme de recherche dirigé par John Tolan évoqué aujourd'hui)

- Votre conclusion, dense, vous permettra de faire un commentaire personnel sur les apports et limites de l'article, en le situant par rapport à la réflexion plus large que nous menons en cours et que vous entreprenez, de votre côté, par vos lectures personnelles
\end{comment}

%---------------------------------------------------------------
 \paragraph{}
Dans une conférence donnée en 2005 \cite{lejbowicz_y_2005} et développée dans son livre \textit{Sur l'Islam} \cite{brague_sur_2023}, Rémi Brague  s'interroge sur la possibilité que les relations entre Islam et monde chrétien au moyen-âge aient pu déboucher sur un dialogue réel entre ces deux religions. 

\paragraph{Rémi Brague est spécialiste de la philosophie médiévale arabe et juive.} Son oeuvre la plus connue, \textit{Europe, la voie Romaine} \cite{brague_europe_2009}, soutient la thèse originale que la civilisation Européenne, héritière de Rome, ne se définit pas par ce qu'elle crée en propre mais par sa capacité à reprendre, transmettre et diffuser ce que Grecs et Juifs ont créé. 
\paragraph{}
%---------------------------------------------------------------
Pour répondre à sa question, il articule sa démonstration autour de trois idées principales : l'asymétrie entre monde Islamique et monde Chrétien, l'absence de tolérance religieuse au Moyen-âge, et le caractère largement polémique des \textit{dialogues} entre religions.



\paragraph{Asymétrie entre Islam et monde chrétien} Du fait des conquêtes militaires arabes, islam et christianisme ont été forcés de cohabiter. Cet état de fait crée une asymétrie entre monde chrétien et musulman : Les chrétiens sont du "dedans" et du "dehors" pour les musulmans alors que, hormis quelques exceptions dont la notable \textit{Reconquista} espagnole, il n'y a pas de musulmans en monde chrétien. Du fait de leur ordre d'apparition, le rapport à l'autre est différent : Les chrétiens savent qu'ils ne connaissent pas l'islam alors que les musulmans croient qu'ils connaissent le christianisme : les chrétiens sont curieux de l'Islam alors que le christianisme est pour l'islam une \textit{vieille histoire}, connue mais dépassée. 


\paragraph{Pas de réelle tolérance religieuse} hormis quelques rares exceptions sur lesquelles nous reviendrons, on ne peut parler de liberté de pensée dans le monde musulman (ni dans le monde chrétien). En monde musulman, les communautés juives et chrétiennes sont tolérées mais humiliées sous le régime de la \textit{dhimma}. Un système équivalent existait pour les juifs en monde chrétien.  La \textit{dhimma} ne doit pas être comprise comme une tolérance religieuse, sous peine d'anachronisme. 

\paragraph{Pas de dialogue véritable entre religions} Dans le monde musulman, l'arabe, langue commune, a facilité le dialogue entre religions.  Mais l'islam,   dès le début de son existence, doit se battre contre les chrétiens et le Coran est ainsi le premier livre apologétique anti-chrétien. Puis du fait de la foi musulmane  très condensée, elle évolue vers un discours à destination non des chrétiens mais des convertis qui pourraient garder un fond chrétien. L'apologétique chrétienne va attaquer quant à elle la figure prophétique de Muhammad. cette littérature est destinée à décourager les chrétiens d'abandonner leur foi. La forme du dialogue entre des représentants de différentes religions est alors un genre littéraire de fiction. Les vrais dialogues entre chrétiens et musulmans qui n'ont pas de visée apologétique sont rares, l'auteur citant l'exception notable d'une \textit{disputa} tolérante à Fez en 1394, date tardive, dialogue à partir du \textit{Livre de la Trinité} de Raymond Lulle.

\paragraph{}
L'exposé est à la fois stimulant et étayé, malgré la forme de conférence qui oblige à une concision certaine. Nous proposons de développer trois points en lien avec le cours : partant du mythe de la \textit{convivencia} dans un Islam fragmenté, nous souhaitons souligner l'existence d'un dialogue au niveau de la culture, et à un niveau plus religieux, la transformation de la théologie chrétienne et musulmane au contact l'une de l'autre.

\paragraph{Islam fragmenté et mythe de la convivencia} Remi Brague \cite[p. 40]{brague_sur_2023} souligne que la religion n'est pas le facteur ultime de la civilisation : l'\textit{Islam} en tant que civilisation est certes liée à l'\textit{islam} comme religion mais elle est aussi influencée par de nombreux facteurs culturels, politiques ou économiques. Or, avec l'émergence de l'empire Abbasside, l'Islam se  fragmente avec trois califes à Cordoue, au Caire ou à Bagdad dans ce qui est pourtant considéré comme l'âge d'or de l'Islam. Des juifs, chrétiens et d'autres en sont moteurs 
\begin{singlequote}
  [\ldots] à tel point qu'on présente parfois l'entente inter-religieuse  comme une sorte de « paradis perdu », une « culture de tolérance ». C'est le mythe de la \textit{convivencia}, le vivre-ensemble. [\ldots] Comme de nombreux mythes, celui-ci se base sur une partie de vérité et une partie d'oubli.  \cite[p. 102]{tolan_nouvelle_2022}
\end{singlequote}
Ainsi, alors que la \textit{dhimma} interdit théoriquement la création de nouvelles églises, le premier émir de Cordoue accepte que les chrétiens construisent une nouvelle église en  dédommagement de l'expopriation de la cathédrale St Vincent et, fait remarquable, accepte même de la partager le temps de la construction de la nouvelle église. La \textit{dhimma} est donc un système lourd et humiliant mais finalement  appliquée de façon souple en fonction des circonstances. 
A d'autres moments de l'histoire, malgré la protection théorique de la \textit{dhimma}, les chrétiens seront persécutés, même dans la mythique \textit{Al-andalus} sous les règnes des très rigoristes Almoravides puis les Almohades.

\paragraph{L'existence d'un dialogue au niveau de la culture} Avant de chercher un dialogue proprement théologique, le mythe de la \textit{convivencia} signale  d'abord des liens entre personnes (le Calife de Cordoue participant à une fête équestre chrétienne, un évêque à la cour de l'émir ...) mais, de façon plus profonde une hybridation culturelle (au sens large)  entre Chrétienté et Islam : ainsi, l'Islam Ommeyyade reprenant à Damas de nombreux fonctionnaires chrétiens pour diriger son nouvel empire, ou bien la dynastie abbasside, s'appuyant essentiellement sur des chrétiens pour la traduction des oeuvres philosophiques grecques.

 De manière symétrique, les chrétiens en terre d'Islam adoptent la culture arabe, au grand dam de certains, comme le chrétien Alvaro en Andalousie au milieu du IXe, s'offusquant que les chrétiens ne parlent plus latin mais connaissent poésie et contes arabes.

Arrêtons-nous sur cette interaction entre musulmans et chrétiens : adopter la culture de l'autre comme les chrétiens mozarabes ou simplement son organisation politique comme le califat de Damas, c'est accepter, comme dans tout dialogue, de se laisser toucher, transformer, questionner sur son identité propre, y compris religieuse, comme l'écrit l'A. dans son livre \textit{sur l'islam} : 
\begin{singlequote}
    Mais à y regarder de plus près, l'originel n'est pas nécessairement le plus vrai. Ce qui vient se déposer sur la couche primitive peut très bien en accentuer les formes et, donc, les rendre plus clairement visibles. Déjà dans notre expérience quotidienne, il arrive souvent que ce que nous avons vécu ne déploie tout son sens que dans le souvenir que nous en conservons. Un fort émouvant sermon du cardinal Newman expose admirablement cette idée en contexte religieux. \cite[p. 84]{brague_sur_2023}
\end{singlequote}


\paragraph{Un dialogue théologique à distance} Par ailleurs, si les vraies \textit{disputa} théologiques sont effectivement rares, on note néanmoins des auteurs comme at-Tabari qui recommande de
\begin{singlequote}
    {discuter et débattre avec son adversaire et non l’attaquer et l’insulter ; il doit être intelligible et non pas obscur ; courtois et non pas offensant.\cite[p. 34-36 cité dans le cours ]{tabari_refutation_1979}}
\end{singlequote}
La confrontation à une autre religion - même à distance - oblige d'abord le théologien à clarifier  son discours, mais aussi à entrer dans le questionnement propre de l'autre religion, sa manière de pensée. On connaît l'influence d'Averroes sur la pensée thomiste. De manière symétrique, le  mu'tazilisme, courant théologique rationaliste, a traité de questions théologiques (libre-arbitre,  attributs de Dieu,...), dont les polémiques avec les chrétiens pourraient être l'origine : par exemple,  dire que le Coran est incréé, c'est risquer de l'\textit{associer} à Dieu, alors que l'\textit{associationisme} est la principale critique adressée aux chrétiens. Jean Damascène déjà le reprochait implicitement : 
\begin{singlequote}
   Nous leur répondons : ”Puisque vous dites que le Christ est le
Verbe et l’Esprit de Dieu, comment nous reprochez-vous d’être des \textit{associateurs}
? En effet, le Verbe et l’Esprit sont chacun inséparable de celui dont il
tire son origine ; si donc le Verbe est en Dieu, il est évident qu’il est aussi
Dieu. Si, au contraire, il est en dehors de Dieu, alors Dieu, selon vous, \textit{n’a ni
verbe ni esprit}. Ainsi, en voulant éviter de donner à Dieu des associés, vous
le mutilez.\cite[p.216]{damascene_ecrits_1992}
\end{singlequote}

Par ailleurs, certains outils théologiques que vont utiliser les mu'tazilites, comme la théologie négative, viennent des
traditions grecque et chrétienne \cite{Candiard_theologie_2022}. De même, si Ibn Kammuna \cite{ibn_kammuna_examen_2012} a écrit son\textit{ examen des religions} dans un climat de tolérance exceptionnel (les trente ans qui suivent la prise de Bagdad par les Mongols (1258)), son examen reprend sur la forme et le fond des disputes théologiques anciennes entre ces religions dont le lecteur moderne ne peut que remarquer la proximité culturelle. 


 

\paragraph{En guise de conclusion, quelques pistes pour un dialogue contemporain} La conférence de Rémi Brague est stimulante. Nous avons voulu complété par quelques nuances car l'absence de réelles \textit{disputa} religieuses ne signifiant pas l'absence de dialogue religieux mais comme "à distance", par l'intermédiaire culturel ou des questions théologiques que porte l'autre religion. 
L'épilogue du livre \cite{brague_sur_2023} essaye d'analyser les raison de \textit{l'ankylose} de la pensée islamique après le XIIIe. Aux nombreuses raisons mentionnées, nous voudrions en ajouter une autre : L'Europe, avec la création des universités, a établi un vrai dialogue "sans confusion" entre théologie et les autres "\textit{sciences}", en particulier la philosophie - A la différence de la \textit{falsafia} assimilée en islam à un courant théologique musulman précis. Et ce dialogue universitaire, d'une grande fécondité, a finalement éclipsé le dialogue précédent entre Islam et christianisme. Cependant, le dialogue uniquement avec les \textit{sciences} porte le risque d'une perte de la dimension \textit{transcendantale} du christianisme. Un dialogue contemporain à nouveau frais avec l'Islam peut être de ce point de vue une opportunité pour le christianisme.

De façon symétrique, l'A remarque qu'une difficulté actuelle de l'islam est que la période perçue comme son âge d'or (VIIIe-XIIIe) est très proche de son origine : 
\begin{singlequote}
    il a tendance à considérer que son apogée s'est situé à son début et que son histoire postérieure est celle d'un éloignement constant à partir de ce sommet. Il attribue à Mahomet des déclarations en ce sens comme :\textit{ «Le pire dans les choses, c'est ce qu'elles ont de nouveau» }  \cite[p.84]{brague_sur_2023}
\end{singlequote}
En s'ouvrant au dialogue avec la culture contemporaine et le dialogue avec le christianisme (ce que des théologiens musulmans ont fait au XIXè comme le Cheikh Abdou d'Al-Azhar), l'islam peut déployer tout son sens en réveillant certains souvenirs.
 
  