
\chapter{II. Courants, sensibilités, figures de l’islam de France (ii)}

\mn{LUNDI 6 DECEMBRE 2021 (3e
cours)}


A) « Islams consulaires » : les pays d’origine dans l’équation de l’islam de
France
B) La création du CFCM et ses vicissitudes
C) L’émergence d’un courant réformiste
-Bibliographie de la partie :
BRUCE Benjamin, Governing islam abroad : the Turkish and Moroccan Muslim fields
in France and Germany, thèse de doctorat en sciences politiques, Sciences Po Paris,
sous la direction de Catherine Withol de Wenden, 2015,
https://www.theses.fr/2015IEPP0001 (une synthèse sera présenté par l’enseignant)
ZEGHAL Malika, « La constitution du Conseil Français du Culte Musulman :
reconnaissance politique d'un Islam français ? », Archives de sciences sociales des
religions [En ligne], 129 | janvier - mars 2005, mis en ligne le 09 janvier 2008, consulté
le 17 septembre 2020. URL : http://journals.openedition.org/assr/1113 ; DOI :
https://doi.org/10.4000/assr.1113
BAYLOCQ Cédric, « L’islam réformiste en France : des débats numériques à l’espace
socio-religieux », https://www.oasiscenter.eu/fr/islam-reformiste-en-france-debats-etprojets
Pour le lundi suivant : test de connaissance à mi-parcours (modalités à venir)