\chapter{Introduction}
\mn{Premier Cycle – Institut catholique de Paris, Théologicum en ligne
Cédric BAYLOCQ S., anthropologue, chargé de cours à l’ICP, chercheur associé au
LAM (IEP de Bordeaux) et au CISMOC (Université Catholique de Louvain)
cedric.baylocq@gmail.com
c.baylocq@chens.icp.fr}


\section{Présentation du cours} 
\mn{\href{https://vimeo.com/645176303/b0facd8d46}{lien Video} ici}
L’islam est considéré comme la seconde religion de France par
le nombre de fidèles (de 4 à 5 millions de personnes s'y identifieraient à des degrés
divers, soit environ 8\% de la population française). Ces chiffres sont soumis à enjeux.

Mais le champ religieux musulman
de la nation-mère de la laïcité est composé de nombreuses sensibilités (islam dit
"consulaire", soufisme, tabligh, salafisme, frérisme, tendance réformiste émergente...).
Il est traversé par des dynamiques locales, nationales et transnationales complexes et
fait face à des enjeux multiples, depuis la problématique de la représentation
(Fédérations, CFCM, CRCM, départementalisation...) ou du financement du culte,
jusqu'à la formation des imams, en passant par celle, plus récente, de la prévention de la
radicalisation. Après un préalable relevant de l’histoire de la présence musulmane en
France (moyen-âge, guerres mondiales, immigration de travail…, 1ere séance), ce cours
s'attachera à dresser le panorama des différentes sensibilités qui composent l'\textbf{islam en/de
France} (2e et 3e
séances), sous l’angle des sciences sociales, de ses tendances les plus
anciennes aux mouvances contemporaines. La troisième partie du cours resserrera la
focale sur les mouvances identitaires/radicales (4è séance), et la quatrième partie
présentera les grands enjeux contemporains du culte musulman en France, les objets et
modalités du dialogue entre pouvoirs publics et représentants du culte musulman
(Instances de dialogue de 2015 et 2016 et leurs suites, notamment), dans le contexte
spécifique et structurant de la séparation entre l’État et les cultes (5e
et 6e
séances).
Modalités d’évaluation : Interrogation à mi-parcours (post 3e
cours) et oral final en janvier 22.




\subsection{Pédagogie et méthodologie du cours}
De l’histoire aux sciences sociales. Contextes
historiques, données, chiffres, recherches anciennes et récentes sur l’islam de France.

\paragraph{Supports :} 
\bi
\item 1) Plan de cours 
\item 2) Powerpoint 
\item 3) Vidéo enregistrée 20/30 min. de
présentation de chaque séance 
\item 4) Articles académiques, liens, vidéos, articles de presse
à lire.
\ei


Compétences à acquérir à l’issue de l’enseignement : 
\bi
\item connaissance des différents
courants et des dynamiques qui traversent les communautés musulmanes de France ;
\item connaissance des enjeux et débats contemporains sur l’islam en/de France. 
\ei


\subsection{Usuels de base} (en sus de la bibliographie indicative par parties, voir ci-dessous)
\bi
\item ARKOUN Mohammed (dir.), Histoire de l’islam et des musulmans en France, Albin
Michel, 2006 (dont sélection d’articles enseignant envoyés au format pdf).
\item GODARD Bernard, La question musulmane en France, Fayard, 2015.
\item GODARD Bernard, TAUSSIG Sylvie, Les Musulmans en France. Courants,
institutions, communautés : un état des lieux, Robert Laffont, 2007.
\item LESCHI Didier, Misères de l'islam de France, Le Cerf, 2017
\item INSTITUT MONTAIGNE, « Portraits des musulmans de France », Sondage IFOP pour
l’Institut Montaigne, 2016 + essai EL KAROUI Hakim, « Un islam français est
possible », \url{https://www.institutmontaigne.org/publications/un-islam-francais-estpossible}
 \item FERET Corinne, GOULET Nathalie, REICHARDT André, « De l'Islam en France à un
Islam de France, établir la transparence et lever les ambiguïtés », rapport d'information,
Sénat, 2016, \url{http://www.senat.fr/commission/missions/islam_en_france/index.html}
\ei



