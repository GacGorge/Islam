
\chapter{III. Tabligh, salafisme et djihadisme en France : définitions, convergences et
divergences}
\mn{LUNDI 20 DECEMBRE 2021 (4e
cours)}

A) Le courant tabligh : un piétisme qui prépare le terrain
B) Le courant salafiste : simple orthodoxie ou voie vers la radicalisation ?
C) L’avènement d’une génération de djihadistes français (définitions, chiffres,
rapports)
-Bibliographie de la partie :
AMGHAR, Samir, Le salafisme d’aujourd’hui. Mouvements sectaires en Occident,
Michalon, 2011.
CARRE Olivier, SEURAT Michel, Les Frères Musulmans (1928-1982), L'Harmattan,
2001 (1983)
MICHERON Hugo, Le jihadisme français. Quartiers, Syrie, Prisons, Gallimard, 2020.