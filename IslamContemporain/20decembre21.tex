
\chapter{III. Tabligh, salafisme et djihadisme en France : définitions, convergences et
divergences}
\mn{LUNDI 20 DECEMBRE 2021 (4e
cours)}

% --------------------------------------------------
\section{ Le courant tabligh : un piétisme qui prépare le terrain}



% --------------------------------------------------
\section{Le courant salafiste : simple orthodoxie ou voie vers la radicalisation ?}




% --------------------------------------------------
\section{ L’avènement d’une génération de djihadistes français (définitions, chiffres,
rapports)}




%----------------------------------------------------
\section{A parallel society is developing in parts of Muslim Britain}

\marginnote{
As a new book by Ed Husain explains
Britain
Jun 5th 2021} 

Ed Husain’s new book, “Among the Mosques”, is a fascinating addition to this tradition, taking readers inside religious institutions that most non-Muslims only experience as domes on the horizon. The country’s first two mosques were founded in Liverpool in 1887, in a terraced house, and in Woking in 1889, on a grander scale. There are now almost 2,000 serving a Muslim population of more than 3m. Some heavily Muslim areas such as Blackburn’s Bastwell district have several in the same street. But what goes on inside? And what is their relationship with wider society?

Mr Husain is the ideal man to answer these questions. The son of an Indian father and a mother who migrated from what is today Bangladesh, he won a prize for reciting the Koran as a child and spent much of his 20s in the Middle East perfecting his Arabic. He has written two books on Islam and has a broad intellectual hinterland. He wrote a phd thesis under the supervision of the conservative British philosopher, Roger Scruton, and has worked for a number of think-tanks including the Council on Foreign Relations in America.


Mr Husain discovered much to be pleased about. Britain has absorbed a big Muslim population better than its ancient foe, France. On May 6th London re-elected its first Muslim mayor, Labour’s Sadiq Khan. Several young politicians such as Naz Shah, mp for Bradford West, represent the modern face of the religion.

There is also a darker story. The British establishment that presided over the immigration which followed the second world war expected Islamic migrants to melt into wider society and relax their religious views. But in parts of the country Muslim communities are distancing themselves from wider British society and adopting stricter versions of their faith.

This is particularly true in the old mill towns of Yorkshire and Lancashire, which now contain parallel societies, where the faithful can live their day-to-day lives without mixing. Mosques run schools and pronounce on Islamic law. Restaurants offer gender segregation under the polite name of “family seating”.

These societies are dominated by a clerical class that extends its influence into secular society by, for example, endorsing candidates for Parliament. Mr Husain visited mosque after mosque that taught a highly literal interpretation of Islam, sometimes clinging to arguments that are being dropped in the Middle East. He saw shops displaying books that advocate stoning gays or keeping wives in purdah or waging jihad. Sayyid Qutb (p.\pageref{theol:SayyidQutb}), Osama bin Laden’s favourite philosopher, appeared often.

Many of these clerics belong to religious groupings with roots far from these shores. Saudi Wahhabis pour money into British mosques and offer all-expenses-paid scholarships to young British Muslims. More surprising is the importance of the Deobandis \sn{Le deobandi ou deobandisme est une école de pensée musulmane sunnite, très présente en Asie du Sud (Pakistan, Inde et Afghanistan). Apparue dans les Indes britanniques en 1867 en réaction à la colonisation, elle tire son nom de la ville de Deoband, dans l'État de l'Uttar Pradesh dans le nord de l'Inde, qui a vu naître sa première école1. Se réclamant de Abu Hanifa, juriste musulman du viiie siècle fondateur de l'école hanafite, elle prône un islam traditionaliste et apolitique ainsi qu'une lecture littéraliste des textes.L'école deobandi a aussi bien été l'une des sources de pensée des talibans afghans que du Tablighi Jamaat. Le nombre d'étudiants inscrits au Pakistan dans des madrasas déobandi serait en croissance rapide (multiplication par deux en 2007). Cette école est souvent en conflit avec l'école barelvie, également très présente en Asie du Sud.}. 

Mr Husain claims more than half of the country’s mosques now belong to the movement, which began in India and seeks to rebuild the caliphate from the ground up, convert by convert. Dewsbury, a historic market town in Yorkshire, is the European capital of the largest Muslim organisation in the world, the Tableeghi Jamaat, the movement’s evangelical arm.

Why does this matter? Religious minorities have always clung together, the better to preserve their faith. Look at the Quakers during the Industrial Revolution or Orthodox Jews in Manchester or London today. Isn’t “a parallel society” just a derogatory name for a flourishing subculture? And isn’t the Catholic church also an example of foreign influence? It is no business of the state to make windows into people’s souls.

There are nevertheless good reasons to be worried. One is the paradox of toleration. There are limits to how much liberal societies can tolerate people who call for gays to be stoned or who denounce Ms Shah as “a dog” because she fails to wear a hijab. The radicalised version of Islam being preached by clerics not only promotes intolerance but also fosters extremism.

A second is the paradox of diversity. The welfare state that liberals hold dear depends for its legitimacy on people feeling that they have a common identity. Robert Putnam, a Harvard sociologist, has demonstrated that support for the provision of public goods falls sharply if people think that the recipients are strikingly different from them. It is hard to be more strikingly different than the parallel communities of Dewsbury and Bradford.
\subsection{State failure}
The third is more practical. Britain is witnessing a struggle for the soul of Islam. But the state has repeatedly acted as if it is on the side of the forces of reaction rather than those of enlightenment. It has kowtowed to self-proclaimed community leaders, mistaking hardline beliefs for “authenticity”. It has tolerated schools such as Darul Uloom, in Rochdale, that combines gcse instruction with requiring students to memorise the Hadiths, including ones about beating wives and stoning homosexuals. And it has failed to make a compelling case for Britishness. Mr Husain points out that many Muslim children get a warts-and-all account of British history from their schools, while hearing constant praise for Turkey and Saudi Arabia in their madrassas. The trauma of Brexit has created a palpable desire to cure many of the social and geographic divisions that threaten to divide the country into warring tribes. Mr Husain makes a compelling case that that quest should not ignore the world of the mosque. 
% --------------------------------------------------
\section{Pour aller plus loin}
-Bibliographie de la partie :
AMGHAR, Samir, Le salafisme d’aujourd’hui. Mouvements sectaires en Occident,
Michalon, 2011.
CARRE Olivier, SEURAT Michel, Les Frères Musulmans (1928-1982), L'Harmattan,
2001 (1983)
MICHERON Hugo, Le jihadisme français. Quartiers, Syrie, Prisons, Gallimard, 2020.



% --------------------------------------------------


\section{Faudra-t-il sauver les salafistes ?}

Le gouvernement français a voulu lancer en octobre 2019 une offensive
contre l'islamisme et les courants radicaux, rapidement relayée par un
emballement médiatique qui a échappé à tout contrôle. Or, l'ennemi
désigné n'a nullement été identifié selon des termes juridiques, pas
plus que ses torts. On lui reproche sa piété rigoureuse, son voile, sa
pratique du jeûne de Ramadan, sa barbe fournie, son refus de toucher les
femmes, ce qui le rapproche dangereusement de n'importe quel fidèle
conservateur.

L'offensive vise donc une manière de concevoir la piété musulmane, et
nullement une qualification criminelle ou une atteinte à l'ordre public.
C'est dire que nous sommes confrontés à un « délit de sale gueule »,
lequel échappe à la tradition juridique républicaine, délit qui est
indiscernable, sans limite, extensible, mais politiquement pratique
auprès d'une opinion chauffée à blanc par les attentats et
l'immigration.

\subsection{Un engagement d'abord religieux}

Si l'islamiste ainsi décrit ressemble évidemment
au~\href{https://www.la-croix.com/Religion/Islam/Quest-salafisme-2018-10-14-1200975866}{\underline{salafiste}},
c'est oublier un peu vite que l'écrasante majorité des~\emph{salafi~}--
ceux qui sont attachés au modèle des « anciens » (les~\emph{salaf}),
c'est-à-dire les compagnons du Prophète -- se veulent quiétistes : leur
mode d'action est la prédication et l'action missionnaire
(la~\emph{da`wa}). Le salafiste souhaite d'abord vivre un islam épuré et
intégriste -- au sens d'intégral -- dans le cadre de sa famille et de sa
communauté.

Ce mouvement est distinct d'un engagement politique, de sorte que les
salafistes sont rarement liés aux Frères musulmans, qui eux forment un
mouvement politique. Si la matrice religieuse et idéologique du
salafisme imprègne les mentalités djihadistes, elle ne se confond pas
avec celles-ci, ni dans la pensée, ni dans les faits. La radicalisation
concerne donc à des degrés différents et sous des formes incomparables
les sympathisants du salafisme et les partisans du djihadisme de Daech.
Les premiers ont un engagement d'abord religieux, tandis que les autres
sont mus à la fois par la volonté de puissance, des facteurs politiques,
sociaux et religieux.

\subsection{L'autodidacte de l'islam présente plus de risques que le
salafiste}

L'hostilité des salafistes envers les courants djihadistes a été prouvée
à de nombreuses reprises par des déclarations publiques et surtout en
fournissant du renseignement de qualité auprès des services de police.
Le meilleur ennemi du terroriste est souvent le~\emph{salafi}, et
l'autodidacte de l'islam présente plus de risques que le salafiste.

En outre, le salafisme n'a pas été désavoué par les représentants du
culte musulman pour la simple raison que ce courant n'est pas une
idéologie : il faudrait donc lui enlever son~\emph{isme}~final et
l'appeler, selon la tradition religieuse, la~\emph{salafiya~}; il s'agit
d'un vieux courant légitime de l'islam, qui a fourni des générations
d'imams et de lettrés attachés au sens littéral du Coran et de la Sunna.

\subsection{Un « écosystème » étroit mais rassurant}

Il est évident que le salafisme représente une alternative culturelle et
sociale au modèle français, modèle égalitaire, inclusif, ouvert (au
moins en théorie). Les quelques salafi que j'ai connus -- des convertis
à 25 ou 30 \% d'entre eux -- vivaient dans un étroit triangle
géographique. Parce qu'ils souhaitent faire les cinq prières à leur
heure, sans les décaler, et ce dans une salle de prière, ils sont
contraints de vivre et de travailler non loin d'une mosquée. Ils passent
ainsi de leur habitation au lieu de travail et à la salle de prière,
lesquels se situent nécessairement dans un « écosystème » étroit mais
rassurant. Ils ne peuvent guère être exigeants sur le plan
professionnel.

\href{https://www.la-croix.com/Religion/Le-Coran-peut-etre-interprete-2021-01-25-1201136852}{Le
Coran peut-il être interprété ?}

Le salafisme, qui représente au moins 40 000 individus, est socialement
dangereux car il impose l'auto-ségrégation, le refus des contacts avec «
ceux qui n'en sont pas ». C'est la raison pour laquelle les spécialistes
des questions de sécurité se refusent à les impliquer dans la lutte
contre le djihadisme. Salafistes et terroristes participeraient à une
même matrice intellectuelle, celle du bien contre le mal, une sorte de
vision sectaire du monde. La différence vient du rapport à la violence :
assumé chez les djihadistes, rejeté chez les salafistes. Leur
fondamentalisme présente l'avantage d'une certaine forme de morale : à
Sartrouville les quartiers salafisés ont vu s'effondrer la toxicomanie
et la délinquance, avec le soutien de la mairie.

\subsection{Confondre l'approche culturelle avec la lutte contre le
terrorisme}

Ces courants ne peuvent être incriminés sur le plan sécuritaire. On
confond donc l'approche culturelle avec la lutte contre le terrorisme. À
moins de changer tout le droit européen, la première doit être menée par
l'éducation, la philosophie, la raison, le débat ; quant à la seconde
elle doit s'appuyer sur le droit et sur des qualifications pénales, et
non sur de vagues impressions de « radicalisation », notion qui n'a
toujours pas été appréhendée de façon rigoureuse en termes sociologiques
et psychologiques.

Comme la guerre d'Algérie nous l'enseigne, une telle manière de
concevoir l'action politique va aboutir à l'effet inverse de celui
recherché : le renforcement de la méfiance collective, le repli
communautaire du côté musulman, l'action violente du côté des « anti »,
et, finalement, la fragmentation sociale et l'insécurité.

\subsection{Islam : les fumées de la radicalisation}

Olivier Hanne, médiéviste (université de Poitiers), chercheur en
islamologie, estime qu'il est très difficile de définir le parcours type
d'une personne radicalisée. Le dernier de trois articles consacrés à
l'islam en France. 
 

Qui parle d'islam aujourd'hui pense aussitôt à la radicalisation. En
2015, on estimait entre 8 000 et 10 000 le nombre de Français
radicalisés. Leurs profils sont si variés qu'il est difficile de donner
des catégories fixes : les mineurs représentent 25 \% des cas, les
femmes 27 \%, les personnes signalées sont plutôt jeunes (entre 16 et 30
ans), leur niveau scolaire est généralement faible, même si l'on
rencontre des diplômés.

La plupart travaillent. Internet représente pour tous ces individus un
passage obligé, même s'il se concrétise différemment : terrain initial
de la radicalisation, facteur de renforcement ou vecteur unique de
l'expression radicale, le partage des contenus djihadistes sur Internet
n'a pas du tout la même fonction chez une adolescente connectée, un
salafiste convaincu et un combattant expérimenté déjà parti en Syrie.

\subsection{Les autorités font feu de tout bois}

De toute évidence, l'attraction pour la radicalité religieuse n'est pas
nécessairement liée à un phénomène de rupture sociale. Les failles de la
société contemporaine (éclatement des familles, déclin des autorités et
des idéologies, chômage, ghettoïsation) créent un terreau facilitateur,
mais nullement déterminant. La frustration individuelle alimente le
recours à des convictions extrêmes, voire le passage à l'acte
terroriste, mais n'est qu'un facteur parmi tant d'autres.

Les autorités font feu de tout bois pour tenter de faire face à une
radicalisation multiforme. En avril 2015, le premier ministre français,
Manuel Valls, annonçait l'ouverture d'une dizaine de centres de
prévention de la radicalisation, dont la plupart furent un échec. Des
sites Internet officiels sont créés et proposent des fiches techniques
contre la radicalisation et le terrorisme, dont le contenu est souvent
simple, voire binaire. Ainsi sur le site
français~\emph{stop-djihadisme.gouv.fr}, un bandeau intitulé «
Radicalisation djihadiste, les premiers signes qui peuvent alerter »
énonce pêle-mêle : « ils se méfient des anciens amis qu'ils considèrent
maintenant comme des impurs » ; « ils changent brutalement leurs
habitudes alimentaires » ; « ils arrêtent d'écouter de la musique car
elle les détourne de leur mission » ; « ils ne regardent plus la
télévision et ne vont plus au cinéma ». Autant de signes extérieurs qui
se rapprochent de l'adolescente anorexique\ldots{} L'efficacité de ces
dispositifs a d'ailleurs été très contestée dès 2015.

\subsection{L'État, tenté d'être omniprésent}

Toute l'entreprise de déradicalisation définit en creux le modèle
positif occidental : monde de loisirs, de consommation, d'épanouissement
personnel et professionnel. Le vocabulaire de la radicalisation masque
le rejet de ce modèle culturel. Et les pouvoirs publics d'hésiter à
appeler leur objectif par son vrai nom : le reconditionnement mental.

Le danger de la déradicalisation se situe dans l'élargissement des
intrusions de l'État : en voulant réinsérer, l'État pénètre dans
l'intimité des individus afin de redéfinir le religieux et lui redonner
une place acceptable. Or, l'État a-t-il compétence pour définir ce
qu'est l'islam, le « bon » islam ? Ne sachant cerner la menace, l'État
est tenté d'être omniprésent, sans en avoir la capacité légale. La
déradicalisation pourrait relever de la posture intellectuelle.

Le problème vient sans doute des hésitations du vocabulaire. Car,
après-tout, qu'est-ce que la radicalisation ? Au
XIX\textsuperscript{e}~le mot anglais~\emph{radical}~était employé pour
désigner les partis politiques britanniques exigeant une réforme
démocratique libérale. Transféré tel quel en France, on l'appliqua aux
partis de gauche, laïques et libéraux qui voulaient réformer la société.

\subsection{Réactions épidermiques}

Le verbe « radicaliser » fut employé régulièrement dans les années
1960-1970 dans une acception politique avec l'idée de « devenir plus
intransigeant, se durcir » ou « plus extrême ». Le premier sens était
donc politique et pas nécessairement négatif. Se déradicaliser était un
synonyme pour « se compromettre ». Appliqué à l'islamisme, le verbe
impose une redéfinition complète des termes : à partir de quand
juge-t-on l'islam intransigeant ou extrême ? par rapport à quelle norme
? à quelle moyenne ?

Les réactions épidermiques qui ont suivi le meurtre de l'enseignant de
Conflans-Sainte-Honorine en octobre 2020 sont tristement révélatrices :
les imams doivent s'exprimer ! les musulmans doivent désavouer le
terrorisme et faire allégeance à la France ! Mais quand ils le font,
c'est encore insuffisant, déloyal et mensonger. Le gouvernement proposa
même qu'ils prient pour la République au cours de la prière collective
du vendredi. Nos références sur la question religieuse restent
tragiquement celles de la Révolution française : comme il y eut les «
prêtres jureurs », adhérant à la loi, contre les « prêtres réfractaires
», obstinés dans leur obéissance à Rome, de la même façon il nous faut
des « imams jureurs », intimement républicains. L'État se retrouve donc
juge des reins et des cœurs.

