\chapter{Conclusion}
%\chapter{Débats et enjeux contemporains (ii) : imamat, financement, prévention de la radicalisation, émergence de courants réformiste}{Débats et enjeux contemporains (ii) }
\mn{ (10/01/22) (6e COURS)}


\paragraph{Les débats les plus récents}; formation des imams, représentation (CFCM), lutte contre le « séparatisme islamiste » Les débats les plus récents ont porté sur la représentation du CFCM, remplacés par le ForiF \sn{lire Saphire News}

\paragraph{Quid des ‘chances’ du champ musulman réformiste en France} (cf. citation liminaire de Berque)?
quelle est la chance d'un véritable d'Islam de France ? vis à vis du Maroc ... Ces dernieres années, notamment suite aux attentats, un développement de cet Islam réformiste en France.

Même le mouvement Muhaziliste est considéré comme un mouvement réformiste. Voix pour un Islam Eclairé. 

-Quelles relations Etat laïque / Culte musulman à venir ? Quel modèle ?

 
\paragraph{Crispations perpétuelles ou enfantement au forceps d’un islam de France ?} Un Islam de France est en train d'émerger.
 
 
 \paragraph{Découvertes personnelles}
ce sont les termes : 
\bi
\item \textbf{Islam en France} : les premiers contacts; Eléments d’histoire de la présence musulmane en France : Moyen-Age, soldats
nord-africains (période coloniale), 
\item \textbf{Islam de France} : Grande mosquée de Paris, et immigration de
travail.
\item \textbf{Islam Français} : plus récent, en construction, un islam qui intègre les spécificités françaises

\ei
 \bi 
 \item Un lien avec l'Islam plus ancien, et en particulier par l'armée, une chance que n'ont pas les autres pays Européens. une flexibilité par rapport à la loi 1905 dans le passé qu'on ne retrouve pas aujourd'hui (cf rapport du Sénat \sn{\cite{Senat:financementIslam}} .
 \item 



\section{Bibliographie Générale}
Amghar, Samir, Le salafisme d’aujourd’hui. Mouvements sectaires en Occident, Michalon, 2011.
Arkoun Mohammed (dir.), Histoire de l’islam et des musulmans en France, Albin Michel, 2006.
Carré Olivier, Seurat Michel, Les Frères Musulmans (1928-1982), L'Harmattan, 2001 (1983)
Godard Bernard, La question musulmane en France, Fayard, 2015.
Godard Bernard, et Sylvie Taussig, Les Musulmans en France. Courants, institutions, communautés : un état des lieux, Robert Laffont, 2007.
Geisser Vincent, Marongiu Oméro, Smaïl Kahina, Musulmans de France, la grande épreuve, L’atelier, 2017
Képel Gilles, Les Banlieues de l’islam, Seuil, 1987.
Krieger-Krynicki Annie, Les musulmans en France, Maisonneuve et Larose, 1985.
Sellam Seddak, La France et ses musulmans, Fayard, 2006.
Sèze Romain, Être imâm en France. Transformation du clergé musulman en contexte minoritaire, Cerf, 2013.
Mouline Nabil, Les Clercs de l’islam. Autorité religieuse et pouvoir politique en Arabie Saoudite, XVIIIe–XXIe siècle, PUF, 2011.
Wikctorowicz Quintan, «Anatomy of the Salafi Movement», Studies in Conflict and Terrorism, 29-3, 2006, pp.207-239.
 \subsection{Rapports}
\begin{itemize}
    \item « La Fabrique de l’islamisme », Institut Montaigne, septembre 2018 \url{https://www.institutmontaigne.org/publications/la-fabrique-de-lislamisme}
    \item « Portraits des musulmans de France », Sondage IFOP pour l’Institut Montaigne, 2016 + essai « Un islam français est possible », \url{https://www.institutmontaigne.org/publications/un-islam-francais-est-possible} 
    \item Rapport du Sénat : \cite{Senat:financementIslam}  
    \item \cite{INED:teo}

\end{itemize}
