LUNDI 3 JANVIER 2022 (5e COURS)
IV. Débats et enjeux contemporains (i) : les nouvelles bases du dialogue Etat-culte
musulman (2015-2020)
A) La laïcité : une contrainte pour l’organisation du culte musulman en France ?
B) Les Instances de dialogue Etat-culte musulman et les problématiques
soulevées (2015-2018)
LUNDI 10 JANVIER 2022 (6e
cours)
IV. Débats et enjeux contemporains (ii): imamat, financement, prévention de la
radicalisation, émergence de courants réformistes
A) Les vicissitudes de la formation des imams
B) La problématique du financement (hallal, hajj, zakat, fonds étrangers…)
C) Une prévention ‘religieuse’ de la radicalisation ?
4
D) L’organisation de l’Aïd el kébir : un exemple de dialogue Etat-culte
musulman
-Bibliographie de la partie :
OUBROU Tareq, avec Cédric Baylocq et Michaël Privot, Profession imam, Albin
Michel, coll. Spiritualités, 2009 (2e
ed. 2015)
GEISSER Vincent, MARONGIU Oméro, SMAÏL Kahina, Musulmans de France, la
grande épreuve, L’atelier, 2017.
-BAYLOCQ Cédric, HOARAU Margaux, Aïd el Kébir. Modalités d’organisation et
d’encadrement de l’abattage, La documentation française/Ministères de l’agriculture et
Ministère de l’intérieur, 2016. https://www.interieur.gouv.fr/Publications/Cultes-etlaicite/Guide-pratique-de-l-Aid-el-Kebir