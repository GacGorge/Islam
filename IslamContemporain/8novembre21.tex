LUNDI 8 NOVEMBRE 2021 (1
er cours)
Introduction : présentation de l’enseignant, du cours, du plan, de la bibliographie,
devoirs hebdomadaires et modalités d’évaluation.
I. Eléments d’histoire de la présence musulmane en France : Moyen-Age, soldats
nord-africains (période coloniale), Grande mosquée de Paris, et immigration de
travail.
A) Premiers contacts (Moyen-Âge, Septimanie, Fraxinet, Moussais/Poitiers…)
B) Sujets coloniaux, soldats, travailleurs, puis citoyens
C) Evolution de la présence musulmane en France (chiffres, nombre de lieux de
culte…)
-Bibliographie de la partie :
ARKOUN Mohammed (dir.), Histoire de l’islam et des musulmans en France, Albin
Michel, 2006.
KRIEGER-KRYNICKI Annie, Les musulmans en France, Maisonneuve et Larose,
1985.
SELLAM Seddak, La France et ses musulmans, Fayard, 2006.
-Support multimédia :
Documentaire « Nous, Français musulmans », Arte, Janvier 2020, 2 parties X 52’ :
https://www.arte.tv/fr/videos/084758-001-A/nous-francais-musulmans-1-2/ et
https://www.arte.tv/fr/videos/084758-002-A/nous-francais-musulmans-2-2/
Emission « L’islam et la République » (52’), de Questions d’islam, France culture,
pres. Ghaleb Bencheikh, invité Haouès Seniguer (Mcf Univ. Lyon);
https://www.franceculture.fr/emissions/questions-dislam/lislam-et-la-republique