\chapter{Eléments d’histoire de la présence musulmane en France}
\mn{LUNDI 8 NOVEMBRE 2021 (1
er cours)}

Eléments d’histoire de la présence musulmane en France : Moyen-Age, soldats
nord-africains (période coloniale), Grande mosquée de Paris, et immigration de
travail.


\section{Premiers contacts (Moyen-Âge, Septimanie, Fraxinet, Moussais/Poitiers…)}

\section{Sujets coloniaux, soldats, travailleurs, puis citoyens}



\section{Evolution de la présence musulmane en France (chiffres, nombre de lieux de
culte…)}

\subsection{La société des Habous à la Mecque}
Cautre impulsion conduisant à la création de la Mosquée de
Paris est la mise en place, par le Quai d'Orsay, de la Société des
habous en 1916-1917. Celle-à est liée à la diplomatie de guerre
menée par la France au Proche-Orient. Pour contrer les menaces
sur les colonies à population musulmane que l'appel du calife au
\emph{jihâd} fait peser, la France a contracté une alliance avec le cherif
Hussein de la Mecque qui déclenche la révolte arabe contre les
Turcs dam le Hedjaz le 10 juin 1916. C'est la grande mission de
Si Kaddour Ben Ghabrit (186~-1954), un Algérien entré au service
de l'administration du protectorat au Maroc en 1892 et
agent diplomatique français de grande envergure.
Le chef du gouvernement Aristide Briand, envoie Ben Ghabrit à
la tête d'une délégation de pèlerins nord-africains après avoir
décidé la réouvertur du pèlerinage de La Mecque le 2 aout 1916.
Dans le but de péreniser la présence française, il est statué finalement
sur 1a question de l hôtellerie des pèlerins à la Mecque.
Celle-ci avait été évoquée en 1915 par la Commission interministerielle
des affaires musulmanes mais le gouvernement Général
de l'Algérie s'y était opposé. De la même manière il s'oppose
au projet de « Village kabyle » comprenant notamment une mosquée.
envisagé par la chambre de commerce de Marseille en
1917. Mais le Parlement vote un crédit de 500000 francs le
31 janvièr 1916 et Pierre de Margerie (1861-1942), ancien secrétaire
général de la conférence d'Algesiras pour le Maroc (1906)
et directeur des Affaires politiques au Quai d'Orsay pendant toute
la guerre, organise le dispositif exécutoire qui débouche sur
l'achat d'un batiment à La Mecque et la création de 1a Société des
babous comme organisme propriétaire et administrateur.
Seul-un bien habous, en effet, c'est-à-dire une propriété de type
religieux inalienable, insaisissable et imprescriptible, autorisait
la détention indirecte par le gouvernement français d'un bien
immobilier dans l'enceinte de La Mecque. Et un tel bien habous
devait être administré par une Société des babous (on dirait
aujomd'hui une association). C'est Pierre de Margerie qui en
désigne lui-même les membres, parmi lesquels Si Kaddour Ben
Ghabrit qu'il charge de la presidence de la comission chargée de la création. 

\begin{Synthesis}
On voit comment
s'imbriquent le devoir religieux qui incombe à la France en tant
que puissance musulmane et les impératifs diplomatiques.
Cette société des Habous est encore propriétaire des lieux de la Mosquée de Paris
\end{Synthesis}

\subsection{Loi française et inspiration marocaine : l'institut musulman}
Edouard Herriot, partie radical, à la suite de l'effort de guerre de la main d'oeuvre venus des colonie, a un respect de libre penseur pour l'Islam : "Encourageons cet islam qui s'éveille ou se réveille". En 1920, il fait voté 500 000 francs pour la mosquée de Paris "vraie maison de l'Islam".

Il est étonnant de noter qu'après la loi de séparation de l'Eglise et l'Etat, une telle subvention ait pu être votée. Un sénateur, Dominique Delahaye, note : "puisqu'on parle des musulmans, il serait bientôt temps de traiter les catholiques aussi bien que les musulmans".

L'architecte choisi est Maurice Tranchant de Lunel, appartenant à l'entourage Lyautéen de Rabat et qui a beaucoup voyagé. La mosquée s'inspire de l'architecture marocaine, mais aussi de l'Espagne d'al Andalus et de l'Inde musulmane. 

A leurs côtés,
Maurice Co1rat représente le gouvernement. Il est l'auteur de
1a fameuse formule, attribuée parfois à tort à Lyautey :
\begin{quote}
    « Quand
il s' ériga-a, au-dessus des toits de la ville, le minaret que vous
allez construire à cette place, il ne montera vers le beau ciel
nuancé de l'Île-de-France qu'une prière de plus, dont les tours
catholiques de Notre-Dame ne seront point jalouses. »
\end{quote}

L'inauguration de la Mosquée de Paris a lieu le 15 juillet 1926 en présence du président Gaston Doumergue et du sultan du Maroc. Le discours de Si Kaddour Ben Ghabrit 
\begin{quote}
    
Ma confusion le dispute à
ma fierté de voir ici réunis le plus haut représentant de la nation
~ et Sa Majesté le sultan Maroc. Cette réunion est symbo1ique. Elle marque que 1a France, fidèle à une politique plusieurs
fois sécu1aire affirme d'éclatante manière la sympathie qu'elle ressent pour les musulmans de toutesorigines qui sont pour elle également des amis. Cet homage de haut et noble libéralisme aura, a déjà eu le plus grand retenstissement dans le monde musulman car il démontre que la France hospitalière à toutes les races ne l'est pas moins à toutes les idées, à toutes les religions. \end{quote}

Gaston Doumergue n'hésite pas à citer un \emph{hadith} du Prophète consistant à définir le meilleur islam : 
\begin{quote}
    c'est celui du croyant dont les musulmans n'ont à redouter ni la main ni la langue
\end{quote}
et à conclure : "cet islam-là est aussi le notre".


il y eu des réactions comme Charles Maurras : 
\begin{quote}
    Cette mosquée en plein Paris ne me dit rien de
bon. Il n'y a peut être pas de réveil de l'islam, auquel cas tout ce
que je dis ne tient pas et tout ce que l'on fait se trouve être la plus vaine des choses. Mais, s'il y a un réveil de l'islam, et je ne
crois pu que l'on en puisse en douter, un trophée de la foi coranique sur cette colline Sainte-Geneviève où tous les plus grands docteurs
de la chrétienté enseignèrent contre l'islam ~ plus
qu'une offense à notre passé : une menace pour notre avenir. 
\end{quote}

\begin{figure}
    \centering
    \includegraphics[width=\textwidth]{Images/MosqueeFrejus.jpg}
    \caption{La Mosquée de Fréjus, édifiée en 1928 et de style malien}
    \label{fig:Frejus}
\end{figure}
\section{Bibliographie de la partie}
ARKOUN Mohammed (dir.), Histoire de l’islam et des musulmans en France, Albin
Michel, 2006.
KRIEGER-KRYNICKI Annie, Les musulmans en France, Maisonneuve et Larose,
1985.
SELLAM Seddak, La France et ses musulmans, Fayard, 2006.
-Support multimédia :
Documentaire « Nous, Français musulmans », Arte, Janvier 2020, 2 parties X 52’ :
https://www.arte.tv/fr/videos/084758-001-A/nous-francais-musulmans-1-2/ et
https://www.arte.tv/fr/videos/084758-002-A/nous-francais-musulmans-2-2/
Emission « L’islam et la République » (52’), de Questions d’islam, France culture,
pres. Ghaleb Bencheikh, invité Haouès Seniguer (Mcf Univ. Lyon);
https://www.franceculture.fr/emissions/questions-dislam/lislam-et-la-republique