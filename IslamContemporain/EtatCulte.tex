\chapter{IV. Débats et enjeux contemporains : les nouvelles bases du dialogue Etat-culte
musulman (2015-2020)}
\mn{LUNDI 3 JANVIER 2022 (5e COURS)}


%----------------------------------------------------------------
\section{La laïcité : une contrainte pour l’organisation du culte musulman en France ?}

\mn{non traité, voir vidéos}
%----------------------------------------------------------------
\section{Les Instances de dialogue Etat-culte musulman et les problématiques
soulevées (2015-2018)}

Sous Bernard Cazeneuve (min. intérieur); consultations à l’échelle locale via les préfectures, des milliers de représentants musulmans, société civile, acteurs associatifs, intellectuels, etc.. Synthèse faite par les préfécture et le Bureau Central des Cultes fait une synthèse, puis envoyé au ministre au format complet + synthèse . Puis invitations de près de 200 personnes à Beauvau. Elargir la discussion, pointer les pb et la recherche de solutions au-delà du seul CFCM et ses difficultés. Permet de comprendre le montage d'un dialogue Etat - entité non étatique.

\paragraph{1ere Instance (15 juin 2015)}: 6 mois après Charlie et Hyper Casher. Panorama des problématiques :
\begin{enumerate}
    \item table ronde protection des lieux de culte, actes antimusulmans et image de l’islam. A la suite de Charlie, des actes antimusulmans. Cela a abouti à un système de video surveillance.
    \item table ronde sur les édifices du culte (construction, gestion, financement) : pratique de gestion des lieux de culte (pas uniquement islam). \emph{Gestion et construction des lieux de culte : guide pratique}
    \item table ronde sur les cadres religieux (formation et statut des imams et aumôniers) : un thème lancinant qui va devenir central. Renforcement de l'ICP : 31 DU (bloc : histoire de la france et laicité, 2 : droit des cultes, 3 : sciences du fait religieux). Il permet à des publics assez variés de créer des liens au niveau départemental : des imams, des étudiants et des agents de l'Etat. à l'été 2018 (vérifier), les DU deviennent obligatoires pour les aumoniers militaires, de prison et d'hopitaux.
    \item table ronde sur les pratiques rituelles (Funéraire, Aïd, hadj, halal)  \mn{\url{https://agriculture.gouv.fr/aid-el-kebir}, page 85, \textit{ les conditions du succès et les acteurs} de l'abattage rituel de l'Aid.
    
1) Une préparation anticipée,

2) Une préparation impliquant l’ensemble des acteurs : responsables
musulmans locaux, associations, CRCM, collectivités territoriales,
entrepreneurs, professionnels de l’élevage, transporteurs, responsables de marché et de centres de rassemblements, abatteurs et
préfectures.

3) Une communication adaptée des porteurs de projets à destination
des riverains.

4) Un suivi de la préfecture auprès des différents acteurs participant
au bon déroulement de la fête.

5) Une optimisation des flux au niveau régional et interrégional
afin de saturer les abattoirs pérennes existant avant d’envisager le
montage d’un abattoir temporaire

6) La désignation d’un interlocuteur unique

7) L’accompagnement des collectivités territoriales dans l’organisation et la mise en œuvre.

8) Un accueil de la clientèle garantissant un déroulement fluide de l’abattage en toute sécurité.

9) Une politique de sanctions fermes contre l’abattage clandestin

10) Une connaissance pointue des procédures par les porteurs de
projets d’abattoirs.} : des problèmes d'arnaque ont été soulevés sur ces 4 instances de la pratique musulmane. Par exemple, pour l'Aid el Kebir, un guide pratique de l'abattage rituel. Ce qui est intéressant, c'est que ces tables rondes regroupent des personnes différentes. Modalités d'installation temporaire. Comment registres normatifs pouvaient s'articuler ensemble (normes religieuses, sanitaires, dignité animale).
\end{enumerate} 
 Chaque atelier comprenait 40 personnes environ et une synthèse devant le ministre et des membres du CFCM,... regroupe des élus locaux, préfectures, Bureau des cultes,...

Des avancées concrètes sur de nombreux sujets…


\paragraph{2e Instance (21 mars 2016)} : 4 mois après Bataclan. plus axé lutte contre la radicalisation (post attentats novembre 2015). Plus axé sur la radicalisation. 
\bi 
    \item Atelier 1 - Mobilisation dans les territoires : partage d’expérience
    \item Atelier 2 - La prévention de la radicalisation en prison : renforcement sensible du nombre des aumoniers de prison et leur formation.
    \item Atelier 3 - Quels discours pour prévenir la radicalisation ? : 40 représentants du culte musulman pour définir un contre discours. L'UOIF a mis en place une formation (?) mais certain ont considéré que les effets n'étaient pas suffisants.
    \item Atelier 4  -Les jeunes, acteurs de la prévention de la radicalisation  : deux volets, le religieux et le social. Code de la radicalisation de certains jeunes dans les banlieues. De idées pour que les jeunes participent à la création du lien social. 
\ei 



\paragraph{3e Instance (12 décembre 2016)}: 6 mois après Nice. Financement, organisation du culte musulman et formation des imams. Toujours sous Bernard Cazeneuve. 

 \bi 
    \item \textit{Atelier 1 - Quels projets pour la Fondation de l’islam de France ?}\begin{Def}[Fondation pour un Islam de France]
la FiF, Dirigé par Jean-Pierre Chevènement puis Ghaleb Bencheikh (voir p.\pageref{Theol:Bencheikh}), abondé par Caisse des Dépots, fondation ADP, SNCF. Non cultuel mais culturel.
\end{Def}
.
Dotation de 10 M\EUR{} annoncé par E. Macron aux Mureaux. En particulier, \textit{Lumière d'Islam}\sn{avec des spécialistes universitaires sur l'Islam}.
    \item \textit{Atelier 2 - Quelle formation des imams en France ?}. Le gros morceau. des directeurs de DU. Beaucoup de discussions avec une pléthore d'offres centrées sur les disciplines traditionnelles, plutôt que les disciplines dialogiques comme le \textit{kalam} \sn{voir p. \pageref{Def:kalam} pour les développements théologiques} mais manque d'une structure unitaire, financé par l'Etat d'islamologie. Un DU d'islamologie à l'INALCO vient d'ouvrir \sn{\href{http://www.inalco.fr/formations/formation-distance/diplome-civilisation-islamologie}{Islamoligie à l'INALCO}} et un autre à l'EPHE \sn{\href{https://www.ephe.psl.eu/formations/master/master-etudes-europeennes-mediterraneennes-et-asiatiques-eema/islamologie-et-mondes-musulmans-histoire-sources-doctrines}{Islamologie et mondes musulmans : histoire, sources, doctrines à l'EPHE}}.
    \item \textit{Atelier 3 - Quels besoins de financement pour le culte musulman  en France ? Quels projets ?} Intérêt d'une structure nationale qui centraliserait un prélèvement sur le Hagg, permettrait un contrôle des pratiques des agences du voyage.  Les acteurs économiques (en particulier les agences) ont été dans la défensive. Principe d'un groupe de travail du CFCM a été acté mais jamais mis en pratique. Néanmoins un meilleur contrôle. 
    \item \textit{Atelier 4 - Quelles ressources nationales pour le financement du culte musulman ?}
\ei 



\paragraph{Genèse de la départementalisation du culte musulman}
Puis en septembre 2018, le nouveau ministre de l’intérieur, G.Collomb, reprend l’idée mais uniquement à l’échelle départementale; Assises territoriales de l’islam de France (ATIF), dans les préfectures, avec focus sur les sujets tels que \textit{Gouvernance des lieux de culte} + \textit{financement du culte} + \textit{formation des ministres du culte} (derechef)… Le modèle de départementalisation type « conseil des imams du Rhône » semble s’affirmer…

\paragraph{le FORIF}

\begin{Def}[Forum de l'Islam de France - FORIF]
créé en décembre 2021, a vocation à remplacer le CFCM 
\end{Def}




 

% --------------------------------------
\section{Bibliographie de la partie}
\begin{itemize}
    \item OUBROU Tareq, avec Cédric Baylocq et Michaël Privot, Profession imam, Albin
Michel, coll. Spiritualités, 2009 (2e
ed. 2015)

\item GEISSER Vincent, MARONGIU Oméro, SMAÏL Kahina, Musulmans de France, la
grande épreuve, L’atelier, 2017.

\end{itemize}


 





