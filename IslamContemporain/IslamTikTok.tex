\section{Influenceur Tiktok musulmans}

\mn{La Croix Grand format
Sur TikTok, la nouvelle vague des influenceurs musulmans
Texte: Pierre Sautreuil.  26/01/2022 à 17:29 Modifié le 27/01/2022 à 11:07}

Une nouvelle génération d’influenceurs musulmans touche une audience considérable sur le réseau social TikTok avec des vidéos ultracourtes promouvant une approche rigoriste de l’islam. L’hégémonie de ce discours sur l’application préférée des jeunes inquiète les autorités. Mais pourquoi un tel succès ?
 
À 27 ans, Charlène n’avait jamais franchi les portes de la mosquée de Lannion (Côtes-d’Armor) jusqu’à ce jour de novembre 2021, où elle a décidé de s’y rendre « en cachette ». Après des mois d’hésitations, elle a fini par contacter l’imam pour lui faire part de son désir de conversion. « Cette envie est venue pendant le premier confinement, après une période difficile », confie-t-elle au téléphone.

Aujourd’hui musulmane pratiquante, cette employée de boulangerie, non baptisée mais scolarisée dans des établissements catholiques, dit apprécier le « cadre » que lui offre l’islam. Charlène prie quotidiennement, mange halal et ne boit plus d’alcool. « Je suis les règles qui me semblent les plus justes, qui conviennent le plus à ma personnalité », déclare-t-elle. « Mais je n’imagine pas arrêter de fêter Noël, ou les anniversaires de mon petit garçon. Je sais que ce n’est pas bien, mais on fait tous des bêtises ! » « Pas bien » ? Où a-t-elle entendu que célébrer un anniversaire était un péché ? La réponse tient en deux mots : sur TikTok.
 

Si Charlène s’est convertie, c’est d’abord grâce à Redazere, un « tiktokeur », comme on appelle les créateurs de contenus sur ce réseau social chinois de partage de vidéos courtes. Yeux verts, barbe claire, cet Algérien de 26 ans établi au Québec publie chaque jour des vidéos ultracourtes dans lesquelles il dispense des « rappels » (messages pour remémorer aux fidèles un point de théologie ou de pratique), entrecoupées d’images de la série coréenne Squid Game ou de ses vacances au Mexique.

\paragraph{« Sur TikTok il y a aussi des comptes chrétiens. Mais ce sont les musulmans qui m’ont interpellée. »}

Charlène, convertie à l’islam

Selon l’humeur du jour et les questions formulées en commentaires, ce jeune homme charismatique indique quelle invocation prononcer à l’approche d’un examen, il prône la charité, lance des défis participatifs – comme citer le plus de noms d’Allah en trente secondes –, ou encore affirme qu’écouter de la musique et regarder les femmes n’est pas permis en islam.
 
 \begin{quote}
     « Sur TikTok il y a aussi des comptes chrétiens, poursuit Charlène. Mais ce sont les musulmans qui m’ont interpellée. Surtout Redazere. C’est sa personnalité, son énergie… Il est captivant. »
 \end{quote}
 Pour lui témoigner sa gratitude, elle a publié trois jours après sa conversion un message sous sa dernière vidéo : « Tu m’as donné envie de me convertir, merci beaucoup. » Le commentaire a été liké 2 685 fois : un échantillon de la communauté qu’agglomère le tiktokeur.

Avec 1,7 million d’abonnés, il est sans conteste le premier influenceur religieux francophone de TikTok. Ce qui, vu le dynamisme fulgurant de l’application – la plus téléchargée au monde depuis deux ans –, n’est pas loin d’en faire le plus puissant influenceur religieux de la francophonie.

\paragraph{Discours hégémonique}
Redazere est représentatif, dans la forme et le discours, d’une nouvelle génération d’influenceurs musulmans qui a émergé sur TikTok depuis 2020. Ilhan.st, Comprends Ton Dîne, Savoir Islam, Hicham R2F, AbuayahTV, Ismaël Abou Nour… Sous ces pseudonymes, on découvre de jeunes hommes plus ou moins formés en religion, pas nécessairement des imams, mais qui partagent une même connaissance parfaite des codes et de l’esthétique de ce réseau social. Longues de dix secondes à une minute, leurs vidéos taillées pour la viralité mélangent sermons, traits d’humour, mises en scène, références à la culture pop, et emprunts au langage du développement personnel.

 

Derrière cette modernité dans le ton et la forme, on retrouve chez la plupart d’entre eux un discours rigide, voire rigoriste, sur ce qui est halal et haram (licite et illicite), dans le quotidien ou dans les relations, en particulier avec les femmes. Si leur degré de conservatisme varie, ils forment, à force de se citer et de se répondre, un écosystème dont le parti pris littéral dans l’approche de la tradition islamique est aujourd’hui hégémonique sur TikTok.


Ainsi, Redazere considère que se raser la barbe est haram (illicite), de même que « regarder les femmes » ou s’embrasser avant le mariage. « Faire la bise ou des câlins à sa cousine du sexe opposé » n’est « pas halal ». Les relations hors mariage sont illicites et « éloignent de Dieu ». Le port du voile est une obligation. Célébrer Noël, même sans y croire, de même que mettre un sapin chez soi, est proscrit. Au Nouvel An, l’influenceur recommande de ne pas faire le compte à rebours de minuit « pour ne pas ressembler aux non-croyants », qu’il qualifie de « koufar » (mécréants) dans un commentaire.

\paragraph{« Salafisme mainstream »}
Difficile, en entendant pareils propos, de ne pas penser à la polémique soulevée le 1er janvier par le très populaire rappeur Gims, qui avait demandé à ses fans de ne plus lui souhaiter la bonne année, au motif que les compagnons de Mohammed, le prophète de l’islam, ne fêtaient pas le Nouvel An. Difficile aussi d’apposer une dénomination commune sur ces discours. Si les experts joints par La Croix s’accordent à les qualifier de salafistes, ils remarquent une dilution de ce courant littéraliste à mesure qu’il rencontre un plus large public.

« Aujourd’hui, ce qui marche, c’est un salafisme mainstream [...], porté par des gens qui savent adapter leur discours. »

 
\begin{quote}
    « On n’est plus du tout dans ce salafisme vraiment dur et parfois proche du djihadisme qui a commencé à s’imposer dans les années 2000 », remarque Benjamin Hodayé, chercheur associé à l’Institut Montaigne. « Aujourd’hui, ce qui marche, c’est un salafisme mainstream, moins radical, capable de rejoindre d’autres courants conservateurs, et porté par des gens qui savent adapter leur discours. » Ainsi, s’ils se heurtent occasionnellement aux règles communautaires de TikTok (le compte de Redazere a été temporairement suspendu en novembre 2021), il est rarissime que ces propos tombent sous le coup de la loi, comme ce fut le cas pour un imam de Villiers-le-Bel (Val-d’Oise) condamné en 2020 pour « apologie du terrorisme ».

\end{quote}
 

Les autorités s’inquiètent néanmoins de cette banalisation des thèses salafistes sur le réseau social préféré des jeunes. « On retrouve une logique islamiste et séparatiste redoutablement efficace, qui vise à séparer la communauté des croyants de la communauté des “mécréants” », estime Christian Gravel, secrétaire général du comité interministériel de prévention de la délinquance et de la radicalisation, qui anime depuis un an une unité de contre-discours républicain (UCDR) visant à contrecarrer les propos séparatistes et extrémistes. Avec un succès réduit.

 

Pour l’heure, seules 6 000 personnes sont abonnées au compte « République_gouv » sur TikTok. « Nous sommes conscients que notre action restera limitée en l’absence d’une responsabilisation accrue des plateformes numériques, et d’une mobilisation de la société civile », ajoute Christian Gravel. Joint par La Croix, TikTok affirme veiller au strict respect de « ses règles communautaires » (lire les repères ci-contre).

« Allah est le plus savant »
Un compte comme celui d’Ilhan.st permet de réaliser combien ces discours en ligne ont évolué. Fort de 469 000 abonnés sur TikTok, ce jeune homme de 20 ans au sourire onctueux et aux lunettes rondes fait carton plein en présentant depuis sa chambre les invocations à prononcer « lorsque tu es triste » ou « lorsque tu es face à une situation difficile ». Ami des jours maussades et des matins pluvieux, il insiste dans ses rappels sur la « bienveillance » et la « motivation », et invite ses abonnés à penser « toujours positif ».


Ce discours lisse saupoudré de cœurs et d’émojis enrobe cependant à l’occasion des postures plus raides, comme l’interdiction de la musique ou la rencontre entre jeunes gens en âge de se marier sans la supervision d’un tiers. « En tant que musulmans (…), on ne peut pas se permettre de ressembler à des non-musulmans », lance-t-il en réponse à un commentaire critique. « L’islam, c’est un mode de vie. Et n’oubliez pas : l’islam modéré n’existe pas. L’islam que l’on vit, c’est l’islam de l’époque d’an-Nabi (le Prophète), et on ne peut pas se permettre de le modifier. »Interrogé sur l’existence des dinosaures par un abonné, le tiktokeur botte en touche, arguant qu’il y a « des choses plus essentielles à savoir », et qu’« Allah est le plus savant ».
 
Plus professoral, Hamid S., 37 ans, est le visage de Comprends Ton Dîne (le mot arabe dîn est généralement traduit par « religion »), qui rassemble plus de 500 000 abonnés. Cet ancien imam formé dans les écoles religieuses du Maroc et de Mauritanie est arrivé sur TikTok en 2020 et a fait de sketchs sur les interrogations religieuses du quotidien sa marque de fabrique. Une mise en scène motivée par la difficulté de capter l’attention si volatile du public sur ce réseau social.

« En une minute seulement, il faut faire visualiser et comprendre le message », explique Hamid S. au téléphone. « C’est totalement différent de YouTube où on trouve du contenu plus travaillé, à destination d’un public plus éduqué. Ici c’est un public jeune, qui n’a pas réellement de base religieuse mais beaucoup de fausses croyances. Sur TikTok, c’est les bases. De l’introduction. »

\paragraph{« Euphémisation de l’apostasie »}
Pour ce faire, Hamid S. et les quatre autres personnes qui animent « Comprends Ton Dîne » privilégient une approche raide de la tradition islamique. Les fêtes d’anniversaire sont déconseillées car « basées sur des fondements païens qui sont en contradiction avec notre religion ».

Une autre vidéo le montre décrochant un poster de l’équipe de France de football, au motif qu’il serait « interdit en islam d’accrocher toute sorte de photo qui serait composée de créatures humaines ou animales ». En matière de football toujours : porter un maillot sur lequel figure une croix, comme celui du FC Barcelone, « n’est pas autorisé en islam, tout simplement car la croix du Barça est un symbole religieux ».


Au téléphone, Hamid S. nie toute radicalité. « Je suis un pédagogue, je ne prends pas position, je ne qualifie personne d’égaré », affirme-t-il, bien qu’il ait employé ce terme à propos des confréries soufies. Il se dit mal compris lorsque l’on mentionne une vidéo dans laquelle il avait suggéré que les musulmans de France seraient « punis » pour s’être « trop intégrés », et qu’ils feraient peut-être mieux d’arrêter de voter.

« J’ai simplement dit que je constatais qu’on s’est impliqués dans la politique mais qu’on reste stigmatisés », se défend-il. Quelques jours après cette interview, La Croix a constaté que la vidéo n’était plus disponible sur son compte TikTok, de même qu’une autre pastille dans laquelle il qualifiait l’assimilation à la société française d’« euphémisation de l’apostasie ».
 

Outre ces nouvelles têtes, on retrouve aussi sur TikTok des visages déjà connus. Ce sont les « imams 2.0 », prédicateurs aujourd’hui trentenaires ou quadragénaires, passés pour certains par le salafisme, et qui ont acquis à partir de 2010 une popularité considérable en s’impliquant sur Facebook et en publiant leurs prêches sur YouTube.

« Sur TikTok, les comptes qui marchent portent un discours simpliste, culpabilisateur et communautaire [...]. C’est l’islam niveau maternelle. »
 

Ces vidéos réapparaissent sur TikTok, débitées en tronçons de quelques dizaines de secondes par des « comptes de rappels » parfois très suivis (jusqu’à 1,1 million d’abonnés pour le plus important identifié par La Croix), qui recyclent des vidéos déjà existantes sur YouTube, ou diffusent des hadiths incrustés sur des paysages de rêve, au rythme de chants islamiques.

« C’est l’islam niveau maternelle »
Mais bien qu’ils constituent un bruit de fond permanent sur TikTok, les imams 2.0 restent paradoxalement en marge du renouvellement qui s’y opère. Les trois imams français les plus populaires sur YouTube ont tous critiqué ce réseau social où les vidéos de danse et de musique sont omniprésentes. « C’est pas normal de voir quelqu’un qui fait des rappels se mettre torse nu ! », s’offusquent l’imam de Brest Rachid Eljay et l’imam de Roubaix Abdelmonaïm Boussenna, suivis respectivement par 2,5 millions et 800 000 abonnés sur Facebook, et manifestement déroutés par les nouveaux codes qui s’expriment sur l’application.
 
« TikTok, c’est fait pour qu’au bout de quatre-cinq vidéos tu tombes sur quelque chose de catastrophique, qui brûle les yeux et le cœur », tacle le très conservateur imam Nader Abou Anas dans une vidéo YouTube de septembre 2021. Seul des trois à entretenir un compte officiel sur TikTok, il n’y publie pas de contenu original, mais des fragments de prêches déjà parus sur d’autres plateformes.
 
« Perso, je suis largué », soupire un imam très suivi sur les autres réseaux sociaux. Il déplore surtout la difficulté de faire passer un message nuancé sur une plateforme conçue pour des vidéos d’une dizaine de secondes. « Mon public est composé d’adultes insérés qui veulent approfondir leurs connaissances, poursuit-il. Sur TikTok, les comptes qui marchent portent un discours simpliste, culpabilisateur et communautaire, mais avec une corde émotionnelle et un format qui parlent aux jeunes, voire aux très jeunes. C’est l’islam niveau maternelle. »

« À cause de tes vidéos, j’ai peur »
De fait, il suffit de lire les commentaires sous les vidéos pour se convaincre de la jeunesse de l’auditoire. Les adolescents se montrent particulièrement sensibles à ces discours qui offrent des explications totalisantes et jouent sur la culpabilité. Beaucoup de collégiens s’inquiètent : comment accomplir les cinq prières quotidiennes en dépit des obligations scolaires ? Comment faire si les parents boivent de l’alcool ? Peut-on continuer à lire des mangas ?

« Reda, à cause de tes vidéos j’ai peur. J’ai que 13 ans et je trouve (…) que je gagne trop de péchés, que je fais pas assez pour Allah. »
 

Sous une vidéo dans laquelle Redazere affirme qu’il est « super grave » d’effectuer seulement quatre prières sur cinq, un adolescent a écrit : « Reda, à cause de tes vidéos j’ai peur. J’ai que 13 ans et je trouve (…) que je gagne trop de péchés, que je fais pas assez pour Allah. »

 
La capacité de diffusion des vidéos sur TikTok est d’autant plus grande qu’il n’est pas nécessaire d’y rechercher un contenu pour le voir apparaître. Le fil « Pour Toi » du réseau social suggère en effet constamment aux utilisateurs de nouvelles vidéos sélectionnées par l’algorithme sur la base de leurs interactions précédentes.

 
« Avant, il fallait être versé dans le salafisme pour aller chercher ces contenus, commente Damien Saverot, doctorant à l’École normale supérieure. Aujourd’hui, on a des influenceurs ne se réclamant pas ouvertement du salafisme qui le diffusent auprès de personnes qui ne le connaissent pas, en prétendant qu’il s’agit simplement de l’islam. Cela contribue à imposer le salafisme comme norme de référence dans l’imaginaire collectif. »

Référentiel de la bonne pratique
La jeune convertie de Lannion, Charlène, avait découvert les contenus de l’influenceur Redazere dans ses « Pour Toi » au printemps 2020. « Je ne pense pas que je serais allée chercher ces vidéos par moi-même », reconnaît-elle. Même chose pour Talia (prénom modifié), étudiante en droit à Saint-Étienne (Loire).

Élevée dans une famille musulmane originaire d’Algérie, cette jeune femme de 20 ans n’a pas découvert l’islam sur TikTok, mais aime y recevoir les rappels percutants, conçus par et pour des jeunes comme elle, qui l’aident à « garder un lien avec la religion ». Elle apprécie tout particulièrement les vidéos de « Comprends Ton Dîne », qui lui apprennent, dit-elle, comment distinguer les croyances religieuses des superstitions maghrébines, « comme quand on me disait que c’était pas bien de se doucher le soir ».

 
Talia n’a pas adopté toutes les prescriptions énoncées par l’influenceur. Elle ne remet cependant pas en question leur validité comme référentiel de la « bonne » pratique. « Pour le vote c’est vrai, il a raison (de suggérer que les musulmans devraient s’abstenir, NDLR), mais moi je vais voter, parce que je suis en France », affirme la jeune femme, qui ne porte pas le voile, mais dit y reconnaître une obligation.

Quand on lui demande comment elle fait le tri parmi les normes mises en avant par les tiktokeurs, elle dit se fier à son libre arbitre et à ses « propres recherches ». « Les petits, quand ils regardent ça, ils y croient automatiquement, mais à mon âge c’est pas possible, rigole-t-elle. Si j’avais 15 ans, je suivrais comme une idiote. »

TikTok, un nouveau géant au défi de la modération
\begin{itemize}
 \item Créé par la société chinoise ByteDance en 2016, TikTok a connu une croissance spectaculaire à la faveur de la crise sanitaire. L’application a dépassé le milliard d’utilisateurs dans le monde en 2021. Les Français étaient 13,9 millions à l’utiliser en août 2021, selon Médiamétrie.

\item TikTok a été critiqué dans le passé pour la faiblesse de sa politique de modération. Le réseau affirme supprimer tout contenu ne respectant pas ses « règles communautaires » relatives à l’extrémisme violent, aux comportements haineux, à la promotion d’activités illégales, mais aussi aux contenus violents, à caractère sexuel, ainsi qu’aux actes de harcèlement.

\item D’après l’entreprise, plus de 81,5 millions de vidéos ont été supprimées pour infraction à ses règles communautaires ou conditions de service entre avril et juin 2021, «dont 94,1 \% avant même de faire l’objet d’un signalement ».
\end{itemize}
