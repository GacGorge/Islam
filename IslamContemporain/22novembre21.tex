
\chapter{II. Courants, sensibilités, figures de l’islam de France (i)
}

\mn{LUNDI 22 NOVEMBRE 2021 (2e
cours)}

A) Les premières associations cultuelles (FNMF, RMF, GIF, UOIF, FNGMP,
FFAIACCA…)
B) Figures de l’imamat : des grands théologiens à l’entrepreneur musulman
-Bibliographie de la partie :
GODARD Bernard, et TAUSSIG Sylvie, Les Musulmans en France. Courants,
institutions, communautés : un état des lieux, Robert Laffont, 2007 (partie glossaire).