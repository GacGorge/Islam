\chapter{Imamat}


\mn{Guillaume Gorge le 10 janvier 2022}
\section*{Sujet}

\begin{quote}
    A partir 1) des documents joints, 2) des éléments du cours, 3) de toutes sources complémentaires que vous jugerez utiles (mention spéciale aux livres mentionnés dans la bibliographie), vous indiquerez en 2 à 4 pages maximum les raisons des tensions, entre les différents acteurs du culte musulman autour de la problématique de l'imamat (formation, instance représentative, récente charte etc.).
\end{quote}

\subsection*{la volonté de l'Etat de structurer l'Imamat}

A partir des années 80, avec la fin de l'immigration de travail et son remplacement par une immigation dans le cadre du regroupement familial, plus stable, la question de la régulation de l'Islam en France est devenue une question importante pour l'Etat, avec la création de la commission des Six Sages, le CORIF, la Charte du culte musulman, l'\emph{Istichara} et plus récemment le CFCM. 
Ce besoin de régulation a pu évoluer et répond aujourd'hui à différents objectifs : 

\paragraph{lutter contre l'Islam dit consulaire}, fruit de la délégation de cette régulations aux pays d'origine des immigrés (l'Algérie, Maroc et Turquie). Cette délégation, si elle fut pertinente au début de l'implémentation de l'Islam en France métropolitaine, en particulier pour contourner la séparation des religions de l'Etat, n'est plus pertinent quand les musulmans résidents en France sont en majorité des Français nés en France \sn{Le sondage de l'Institut Montaigne, \emph{Un Islam de France est possible} - 2016 indique que 50\% des personnes qui se déclarent musulmanes sont français de naissance, 25\% sont français par acquisition et 26\% sont de nationalité étrangère}. 

\paragraph{éviter le communautarisme et les discours intégristes terreau de la violence} La France est régulièrement la cible d'attentats islamiques depuis 1995. L'importance d'un terreau propice à la violence a été progressivement mieux comprise: Ainsi en 2012, on a pu encore qualifier l'auteur des attentats de Toulouse, Mohammed Merah de \textit{loup solitaire}. L'enquête a cependant montré l'importance de la nébuleuse l'ayant entraîné dans sa radicalisation puis soutenue dans son action terroriste. Il est donc important de lutter contre ce terreau, par des \textit{lois} mais aussi en luttant contre la dimension proprement \textit{idéologique} de ce terreau islamique, ce qui, dans un contexte de séparation des Religions et de l'Etat, ne peut être fait que par les musulmans eux-mêmes. 
\paragraph{se doter d'{interlocuteurs représentatifs}}  pour discuter des problématiques pratiques (Hajj, nourriture Halal, voile,...).

\subsection*{Une ambition renouvelée avec le discours des Mureaux d'Emmanuel Macron}


\paragraph{une implication assumée de l'Etat}
Dans son discours des Mureaux sur le thème de la lutte contre les séparatismes le 2 Octobre 2020,  Emmanuel Macron a souligné une ambition renouvelée sur l'Imamat avec l'ambition d'un \textit{"Islam des Lumières"}.



\begin{quote}
Donc, je ne pense pas qu'il faille une forme d'islam gallican, non. Mais il nous faut aider cette religion dans notre pays à se structurer pour être un partenaire de la République pour ce qui est des affaires que nous avons en partage. Et c'est normal. Les autres religions se sont ainsi structurées, d'abord parce que c'est leur histoire, parfois, je dirais, leur structure même, et nous avons appris à vivre ensemble. Mais là aussi, nous nous devons la lucidité.
Lorsque la loi de 1905 a été votée, l’islam n’était pas une religion si présente dans notre pays. Et elle s’est beaucoup développée ces dernières années aussi à travers les vagues migratoires qui sont plutôt celles du 20ème siècle. {Et donc, nous sommes face à une réalité dont l’organisation ne correspond pas à nos propres méthodes, à nos propres truchements. Nos interlocuteurs, aujourd’hui, n’assument pas une vraie responsabilité cultuelle. Et donc, il est très difficile pour le ministre en charge des cultes, des préfets, des maires, de savoir à qui ils parlent lorsqu’ils veulent pouvoir évoquer les sujets qui relèvent du culte et ont un impact sur notre vie en société et parfois sur l'ordre public. Parce que le culte n'y est pas ainsi organisé.}

    
\end{quote}

\paragraph{Former des Imams et des intellectuels qui défendent les valeurs de la France}
Pratiquement, cela passe pour le chef de l'Etat par la formation : 

\begin{quote}

Enfin l’ambition de former et promouvoir en France une génération d’imams mais aussi d’intellectuels qui défendent un islam pleinement compatible avec les valeurs de la République est une nécessité. 
\end{quote}

\paragraph{labeliser les Imams en s'appuyant sur le CFCM} et signer la Charte, l'Etat appuyant par le biais de soutien financier ou de  mesures coercitives en cas de non-respect des lois.
\begin{quote}
Et donc ce dont nous sommes convenus avec le Conseil français du culte musulman, c’est que d’ici à 6 mois au plus tard celui-ci allait finaliser un travail largement commencé depuis plusieurs mois et indispensable. Un travail qui consiste premièrement à labeliser des formations d’imam dans notre pays. Deuxièmement, à assumer une responsabilité cultuelle qui sera celle de la certification des imams. Troisièmement, d’écrire une charte dont le non-respect entraînera révocation des imams. [...]
Ce que je vous décris là, ça n’est pas l’État qui le fera, en vertu même du principe de séparation, ce sera au Conseil français du culte musulman. 
\end{quote}


\subsection*{Les raisons structurelles de l'échec de CFCM}
Le \textit{Conseil Français du Culte Musulman} a été officiellement créé  en 2003 sous les impulsions successives des ministres de l'Intérieur Jean-Pierre Chevènement (cadre de la consultation, \textit{l'istichâra}) et Nicolas Sarkozy. Avant les élections de 2003, la composition du bureau est décidée, avec neuf sièges pour les fédérations, 5 pour les mosquées et 2 pour les personnalités qualifiées. Le scrutin de 2003 est un succès et voit l'UOIF et le FNMF grandes gagnantes. Néanmoins rapidement, la structure se révèle incapable de répondre aux attentes de l'Etat et est marqué par une paralysie structurelle.


\textit{Le CFCM a été incapable de relever ce défi, pour des raisons structurelles et repérées depuis plusieurs années.}


\paragraph{le CFCM doit traiter de sujets dépassant le strict cadre du culte} Dès sa création \sn{Rapport Montaigne, Ibid}, la conception strictement cultuelle du CFCM a été élargie à des questions théologiques. Ainsi, les débats sur le voile en 2003-2004 ont affaibli l'institution et montré les tensions entre les différences théologiques des pays d'origine.
\begin{quote}
    l'absence d'instance théologique à laquelle il pourrait se référer met à jour la prégnance de \textit{effet diasporique} qui traverse l'islam français, et la difficulté à définir une ligne théologique commune. \sn{Rapport Montaigne, Ibid}
\end{quote}
Une musulmane témoigne ainsi du manque d'une théologie de l'Islam tenant compte du contextefrançais \sn{Arte 2020 - Nous, français musulmans} : 
\begin{quote}
    je demande à un imam marocain ou d'Arabie Saoudite ce que je dois faire avec le voile et que si je n'enlève pas le voile, je perds mon travail. il me répond "tu n'as pas besoin de travailler car ton mari gagne suffisamment d'argent", sans se rendre compte que le travail est non seulement un moyen financier mais un élement important d'estime de soi.
\end{quote}


\paragraph{Et  pourtant le CFCM est une structure sans instance théologique} Les représentants de fédérations qui occupent la majorité du bureau du CFCM n'ont pas des profils de prédicateurs ni de théologiens, mais celui \textit{d'entrepreneurs musulman laïc}  et/ou de recteurs de mosquée à l'image des anciens et actuels présidents du CFCM, Anouar Kbibech, ingénieur SFR, Mohammed Moussaoui, universitaire ou Chems-eddine Mohamed Hafiz \sn{ recteur de la Grande Mosquée de Paris}, avocat franco-algérien.  Cette mutation des figures de l'Islam s'est faite \textbf{au détriment de l'imam qui dirige la prière du vendredi.}
Face à ce manque, souligné par exemple dans le rapport Montaigne , le CFCM s'est doté en 2016 d'un conseil théologique, mais cette instance, voulue informelle pour éviter les sempiternelles querelles de préséance et de modalités d’élection des dirigeants, n'a connu qu'un succès d'estime à sa création et n'a rien produit depuis à notre connaissance.



\paragraph{Un renforcement de l'Islam consulaire} La structure du bureau du CFCM, sous l'impulsion de Nicolas Sarkosy, a renforcé le poids de l'Islam Consulaire et de ses fédérations. La coalition entre UOIF, FNMF et la GMP se révèle éphémère et le CFCM devient le théatre de lutte d'influence, les rivalités entre Etats d'origine se trouvant importés dans son enceinte \sn{Rapport Montaigne, Ibid}. Ce constat est souligné par Tareq Oubrou dans sa Tribune au "Monde"\sn{\emph{Le monde 22 novembre 2020 - Qui seront les imams homologués ? Les limites du futur conseil national des imams}}.
\begin{quote}
    le CFCM est une institution fragile. Les premières à le savoir, ce sont les fédérations elles-mêmes qui le composent, elles qui, aujourd’hui comme hier, passent leur temps à s’y disputer. L’Etat le sait aussi.
Créée sous l’impulsion de Nicolas Sarkozy par des fédérations sous influences des pays étrangers (Maroc, Turquie, Algérie) et mêmes financée par eux, l’instance n’est pas opérationnelle.
\end{quote}




\paragraph{Une structure finalement non représentative des mosquées} Non seulement les  imams et théologiens sont peu représentés dans le bureau du CFCM, mais se pose la question de la la légitimité des fédérations a représenter les lieux de culte : Aujourd’hui, selon M. Lasfar, « toutes fédérations confondues, on ne dépasse pas cinq cents imams », auxquels s’ajoutent les trois cents imams détachés par la Turquie, l’Algérie et le Maroc, et qui sont appelés à disparaître d’ici à 2024. A comparer à l'estimation de 2800 imams en France\sn{Arte 2020}. De même, 1 100 mosquées n’ont pas participé à la dernière élection du CFCM. 

\subsection*{La mort du CFCM en décembre 2021}

\paragraph{L'échec du Conseil de l'Imamat}
Pour contourner la fragilité du CFCM, Emmanuel Macron a régulièrement réuni directement les présidents des fédérations musulmanes de France, comme le 18 novembre 2020. Sous sa "pression", ces fédérations ont accepté les principes d’un conseil national des imams. Celui-ci aurait été composé d’eux-mêmes, qui ne sont pas imams, et de neuf imams, un par fédération, choisi par son président, répliquant le poids et donc la probable paralysie du CFCM.

Mais très rapidement, des théologiens et imams représentant une approche plus  réformatrices se sont exprimés :  démission de Mohamed Bajrafil de sa charge d'Imam , tribune remarquée dans "Le Monde" d'imams et théologiens proches de l'\textit{Association musulmane pour l’islam de France} \sn{Tribune sous la direction de
Tareq Oubrou, imam de la grande mosquée de Bordeaux,  parue dans Le monde 22 novembre 2020 - \textit{Qui seront les imams homologués ? Les limites du futur conseil national des imams}}, souligne l'erreur de créer le  conseil national des imams à partir du CFCM: 
\begin{quote}
    « Aucun imam ou théologien de la scène musulmane française n’a été consulté. Or, un conseil national des imams, c’est l’affaire des imams, pas des laïcs ! »
\end{quote}

\paragraph{Une charte signée dans la douleur et sous la pression forte de l'Etat}
Par ailleurs, quatre des neufs fédérations (fédérations turques en particulier) ont initialement refusé de signer la Charte des principes pour l’islam de France, montrant la résistance de l'Islam consulaire à abdiquer son influence \sn{\emph{Charte de l’islam de France : une signature à retardement}, la Croix, le 04/01/2022 Mélinée le Priol}. provoquant l’ire du gouvernement et l’émoi des quatre fédérations devenues sécessionnistes trois mois plus tard.
3 fédérations turques et Tabligh ont finalement accepté de signer cette charte le 25 décembre 2021 en la qualifiant  de «compromis perfectible» mais uniquement après la déclaration de Gerald Darmanin de la mort du CFCM quelques semaines plus tôt. Et un certain nombre d'ambiguités persistent dans le communiqué des 3 fédérations du 24 décembre ,2021, qui n'aborde par  le refus de l’islam politique ou celui de discours nationalistes en défense des régimes étrangers, pierre d’achoppement, notamment pour le CCMTF, organiquement lié à Ankara par le biais de la Diyanet, la Direction turque des affaires religieuses.

\paragraph{La conséquence : Acter la mort du CFCM}
L’organisation prochaine du Forif a été annoncée par le ministre de l’intérieur Gérald Darmanin dès le 9 décembre – trois jours avant que ce dernier acte unilatéralement de la « mort » du CFCM, déclarant que cette instance n’était « plus l’interlocuteur de la République ».  Mohammed Moussaoui, président du CFCM depuis deux ans, reconnaît désormais que la structure créée par Nicolas Sarkozy en 2003 « n’est plus viable dans son format actuel » et reconnait avoir échoué à mener cette réforme nécessaire « de l’intérieur » en raison des dissensions internes au CFCM : 

\begin{quote}
    « Après ma rencontre mercredi soir avec Gérald Darmanin puis le discours du président de la République à l’Élysée, je considère qu’il y a une possibilité de faire émerger, de façon coordonnée avec les pouvoirs publics, une nouvelle organisation du culte musulman ». 
    « Cette nouvelle organisation doit être le fruit d’une concertation associant le Forif et les structures départementales du culte musulman. »
\end{quote}




Début Janvier 2022,\sn{\textit{Création du Forif -L’organisation du culte musulman entre dans une nouvelle ère } - La Croix - Mélinée Le Priol, le 06/01/2022}  la mise en place d’un Forum de l’islam de France (Forif) se précise lors des "voeux" aux représentants des cultes. Pour l’instant simple « instance de dialogue » entre le culte musulman et l’État, elle pourrait déboucher sur une nouvelle instance représentative.

\subsection*{les difficultés à surmonter pour le Forif}

Les enjeux pour le Forif sont donc de dépasser les limites du CFCM : 
\bi
\item s'assurer que la pression financière et juridique de l'Etat est suffisante pour imposer la recherche de labelisation / agrément auprès du conseil de l'Immamat. La démarche interventioniste d'Emmanuel Macron et Gérald Darmanin, que l'on pourrait qualifier de "droite" \sn{selon la typologie du rapport Montaigne : \begin{quote}
Alors que la gauche prône un modèle collaboratif, la droite, à l'inverse[...] ont fait preuve d'un dirigisme bien
plus important ainsi que d'un attachement à l'islam consulaire.
\end{quote}} est assumée \sn{discours des Mureaux} : \begin{quote}
    Nous devons aller jusqu'au bout. Nous allons donc renforcer les contrôles, mettre dans la loi les principes en vertu desquels il sera permis de dissoudre les associations et assumer que, en vertu de nos principes républicains et sans attendre que ce soit le pire, on puisse dissoudre des associations dont il est établi qu'elles portent ces messages, qu'elles contreviennent à nos lois et nos principes. Avant la
dissolution, il y a le financement. Toute association sollicitant une subvention auprès de l'État ou d'une collectivité territoriale devra signer un contrat de respect des valeurs de la République et des exigences minimales de la vie en société, pour reprendre la formule du Conseil constitutionnel.
\end{quote}
\item limiter le poids des fédérations et de l'Islam consulaire et le risque de copinage, par un poids fort des imams. Ce point est explicitement mentionné comme un objectif de la nouvelle structure, avec 80 et 100 participants choisis dans les départements et non au niveau régional ou national, inadapté pour le culte musulman très décentralisé. La question sera néanmoins la représentativité de ces participants, un enjeu majeur à long terme de la nouvelle instance.
\item lutter contre les imams virtuels, qui dirigent des  « mosquées virtuelles » et qui proposent une version plus "identitariste". Pour cela,  un discours attractif de l'\emph{Islam de France} est nécessaire, articulant éthique et théologiee et "puisée dans l’interprétation des textes canoniques musulmans analysés dans le contexte de notre modernité occidentale"\sn{tribune au « Monde », le grand imam de Bordeaux, soutenu par dix imams et islamologues}.
\item le sujet de la formation des imams, avec une implication publique possible en s'appuyant par exemple sur l'INALCO ou l'université de Strasbourg sous régime concordataire.
\item un dernier point recommandé par l'AMIF est de structurer le financement, par la viande Halal\sn{le rapport Montaigne souligne l'importance de la nourriture Halal pour les musulmans de France} et l'agrément des pélerinages. Cette question financière ne semble pas une priorité pour Emmanuel Macron.

\ei 
   
 



 

\newpage
%------------------------------------------------------------------------------

\section*{conseil national des imams }

\mn{\url{https://www.lemonde.fr/politique/article/2020/11/19/emmanuel-macron-discute-de-la-creation-d-un-conseil-national-des-imams-avec-le-cfcm_6060320_823448.html }
Par Cécile Chambraud et Olivier Faye 
Publié le 19 novembre 2020 à 10h05 
}

Emmanuel Macron discute de la création d’un « conseil national des imams » avec le CFCM
Le président de la République a reçu à l’Elysée, mercredi 18 novembre, les représentants du Conseil français du culte musulman. 

Le plan d’Emmanuel Macron contre le « séparatisme islamiste » se précise. Alors que le projet de loi « confortant les principes républicains » a été transmis, mardi 17 novembre, aux présidents des deux Chambres du Parlement, le chef de l’Etat a reçu à l’Elysée, mercredi, en fin d’après-midi, les représentants du Conseil français du culte musulman (CFCM) pour aborder la question de la formation des imams et de leur certification. Un sujet épineux puisque, selon un proche de M. Macron, « le fait d’avoir des imams autoproclamés a pu créer des troubles à l’ordre public ».

Le CFCM est donc venu avec une proposition à soumettre au président de la République et au ministre de l’intérieur, Gérald Darmanin, lui aussi présent lors de ce rendez-vous : créer un « conseil national des imams ». Une sorte de conseil de l’ordre, qui définirait des critères d’éligibilité afin de pouvoir devenir imam et délivrerait des cartes officielles aux personnes concernées. Si l’Elysée assure ne pas se mêler de ce projet au nom du respect du principe de laïcité, la suggestion avancée mercredi est considérée par un proche d’Emmanuel Macron comme « intéressante et aboutie ». « Le conseil pourrait révoquer la carte d’imam en cas de manquements », croit savoir un conseiller.
\paragraph{Des fédérations qui ne s’entendent pas}
L’idée d’une forme d’homologation des imams et de leur formation a été mentionnée par Emmanuel Macron dans son discours des Mureaux, dans les Yvelines, le 2 octobre. Il avait alors donné « six mois » au CFCM pour organiser une labellisation des formations religieuses, une certification des imams et pour rédiger une charte dont le non-respect entraînerait la révocation de l’imam. Un épisode parmi d’autres donne une idée de la difficulté qu’ont les fédérations musulmanes à se mettre d’accord. En mars 2017, déjà sous la pression du gouvernement dans le sillage des attentats, la précédente direction du CFCM avait fini par rédiger une telle charte. Mais il a suffi que son président d’alors l’annonce pour qu’aussitôt une bonne partie des fédérations s’en désolidarisent, ruinant le travail accompli. Or, il se trouve que, depuis trois semaines, le CFCM est de nouveau en proie aux tiraillements entre fédérations, qui ont réagi en ordre dispersé aux appels au boycottage des produits français après l’assassinat de Samuel Paty.


 
L’hypothèse d’un conseil des imams qui régulerait l’accès à la fonction pose, en outre, davantage de questions qu’elle n’apporte de réponses, chaque mosquée étant libre de recruter qui bon lui semble pour exercer ce ministère. Dans ce contexte, qui peut donner au CFCM le monopole d’une certification des imams, qui, de toute façon, n’existe pas ? En quoi les mosquées se sentiraient-elles concernées par ce label ? D’autant que les pouvoirs publics ne peuvent pas s’impliquer trop loin dans cette question proprement cultuelle sans enfreindre la loi de séparation des Eglises et de l’Etat.
\paragraph{« Sortir des ambiguïtés »}
Lors de l’entretien de mercredi, Emmanuel Macron a aussi demandé aux représentants du CFCM de travailler avec le ministère de l’intérieur à une charte des valeurs républicaines, qui engagerait l’ensemble de ses membres. « Le président n’exclut pas que certaines fédérations [du CFCM] ne signent pas cette charte. Il a dit qu’il en tirerait toutes les conséquences », rapporte un proche. Si l’Elysée dit se féliciter du travail accompli ces dernières semaines par le CFCM, le chef de l’Etat a tout de même tenu un « discours de vérité » à ses représentants, selon la présidence de la République. « Il faut sortir des ambiguïtés, avoir une adhésion massive des différentes fédérations aux valeurs de la République », leur a ainsi déclaré M. Macron. « Il faut un islam dans la République, qui fait la séparation entre le religieux et le politique, et qui se sorte de l’influence étrangère », juge-t-on au sommet de l’Etat.
Cette mise en garde résume toute l’ambiguïté de la partie qui se joue. Parmi les fédérations du CFCM qui étaient représentées à cette rencontre, cinq sont liées à l’Algérie, au Maroc ou à la Turquie. D’autres sont soupçonnées par l’exécutif « d’ambiguïtés » envers « les valeurs de la République », au premier rang desquelles Musulmans de France (ex-Union des organisations islamiques de France – UOIF, proche des Frères musulmans) et les deux fédérations liées à la Turquie. « Nous n’avons pas reçu les islamistes à l’Elysée, mais le CFCM, avec l’ensemble de ses fédérations, parmi lesquelles il peut y avoir des personnes avec des propositions ambiguës », défend un proche d’Emmanuel Macron, qui cherche à tuer dans l’œuf toute polémique éventuelle sur le sujet. Un nouveau rendez-vous sous le même format est prévu dans deux semaines pour faire un point sur l’avancée du travail sur la charte des valeurs républicaines. Une rencontre qui doit avoir lieu à quelques jours de la présentation du projet de loi en conseil des ministres, prévue le 9 décembre.

\section*{Pourquoi Mohamed Bajrafil renonce à être imam après vingt et un ans d’exercice}
\mn{\url{https://www.lemonde.fr/societe/article/2020/11/25/pourquoi-mohamed-bajrafil-renonce-a-etre-imam-apres-vingt-et-un-ans-d-exercice_6061088_3224.html} Par Cécile Chambraud 
Publié le 25 novembre 2020 à 15h00}
\label{Theo:Bajrafil1}

Le théologien, qui prêchait depuis quatorze ans à la mosquée d’Ivry-sur-Seine, est lassé du gâchis autour de l’organisation du culte musulman en France. 

Il arrête. On ne verra plus Mohamed Bajrafil, 42 ans – dont vingt et un ans passés à servir comme imam dans des mosquées d’Ile-de-France – prêcher le vendredi à la mosquée d’Ivry-sur-Seine (Val-de-Marne), où il officiait depuis quatorze ans. L’annonce, le 18 novembre, de la création d’un conseil national des imams par le Conseil français du culte musulman (CFCM), sous la pression d’Emmanuel Macron, a fini de le convaincre de renoncer.
Trop de calomnies, de féodalisme, d’occasions manquées. Trop de pression sans que rien n’avance dans « l’islam de France ». L’islam consulaire, l’extrémisme, l’hystérisation du débat public auront eu raison de l’énergie de cet avocat d’un « islam du XXIe siècle », pour reprendre le sous-titre de l’un de ses livres (Islam de France, l’An I, Plein Jour, 2015). Venant d’un acteur en vue dans les milieux musulmans, cette décision apparaît comme un symptôme du désolant gâchis qui entoure la question de l’organisation cultuelle de l’islam dans notre pays.
\subsection*{Pris en étau, les musulmans se cherchent une nouvelle « voix » }
Mohamed Bajrafil a annoncé par écrit sa décision à quelques amis, vendredi 20 novembre. « J’ai eu ma dose. Depuis, c’est comme si je renaissais », témoigne-t-il au Monde. Sa lettre est dénuée d’amertume. Mais il y évoque les rigidités qui usent un imam se voulant un esprit libre, indépendant des mosquées qui l’accueillent et des fédérations qui contrôlent une moitié d’entre elles. Né aux Comores en 1978, arrivé en France en 1999, il raconte comment, « chimère », « mi-arabe, mi-noir », il s’est vite trouvé coincé dans un réseau dominé par les fédérations algérienne et marocaine, et plus récemment turque. « Les Subsahariens sont marginalisés », résume-t-il au Monde.

\subsection*{« Des guignols »}
L’islam consulaire, c’est-à-dire la délégation par l’Etat aux fédérations liées à l’Algérie, au Maroc et à la Turquie, des commandes institutionnelles du culte musulman, par le biais du CFCM, est en ligne de mire. « Au nom de quoi ce doit être un Algérien ou un Marocain qui dirige le CFCM ?, s’insurge-t-il. Macron laisse l’islam au bled. Dans ces conditions, parler d’islam de France, c’est une foutaise. Un oxymore. » Pour sa part, il s’est engagé dans le projet d’Association musulmane pour l’islam de France (AMIF), porté par l’essayiste Hakim El Karoui. « Ce qui m’a plu, c’est que ce serait la fin de l’islam consulaire », précise-t-il.
Baigné dès son plus jeune âge dans l’étude intensive de la religion au sein d’une famille qui a produit l’un des grands muftis du pays, lecteur en public du Coran dès l’âge de 6 ans, Mohamed Bajrafil est venu suivre ses études supérieures en France, jusqu’à obtenir un doctorat de linguistique. En vingt et un ans, il a officié dans plus d’une douzaine de mosquées, sans jamais avoir droit à « un seul bulletin de paie ». Au mieux, il était défrayé pour ses trajets. « Je ne dois rien à personne. Des guignols qui ne nous ont même pas consultés pour ce conseil national des imams, comme si on était des gamins, viennent maintenant me dire: “On va te labelliser”… C’est sans moi. »
 
Ces dernières années, la tension s’est accrue autour des débats sur et dans l’islam. Elle n’a pas peu contribué à sa décision. Mohamed Bajrafil s’est d’abord retrouvé « dans la short list de Daech », menacé de mort pour son orientation jugée réformiste. Il a été contraint de se mettre en disponibilité de l’éducation nationale pour raison de sécurité – il enseignait le français et l’histoire – et de changer d’académie (il est aujourd’hui contractuel). Il a essuyé « les insultes des salafistes ».
\paragraph{« Je ne veux plus me mettre dans la gueule du loup »}
En France, des tenants d’une laïcité belliqueuse et « islamo-suspicieuse » l’ont ensuite accusé d’être un « crypto-Frère musulman » et de pratiquer la taqiya\sn{Le mot taqîya, parfois orthographié taqiyya et takia, provient de l'arabe \TArabe{ تقيّة }(taqīyya) qui signifie « prudence » et « crainte ». Ce terme désigne, au sein de l'islam, une pratique de précaution consistant, sous la contrainte, à dissimuler ou à nier sa foi afin d'éviter la persécution. Cette pratique est connue dans le monde chiite et autorisée dans le sunnisme. Elle possède un fondement coranique, provenant notamment de la sourate 3:28.
On trouve aussi la taqîya dans l'ésotérisme musulman, en particulier dans le monde chiite, où elle est liée à la non-divulgation de données ésotériques relatives à l'imamat.
Dans les années 1990, le mot « taqîya » a reçu une autre interprétation : des auteurs l'utilisent pour désigner une dissimulation de la foi dans un but de conquête. Cette interprétation est contestée par d'autres auteurs. Selon cette interprétation, il s'agirait alors d'une pratique utilisée par des mouvements djihadistes extrémistes tels qu'Al-Qaida et l'État islamique.}, la dissimulation de ses vraies opinions. Sur les réseaux sociaux, des trolls lui ont mené la vie dure. 
\begin{quote}
    « Toute voix qui veut travailler à une meilleure intégration de l’islam dans l’espace républicain, loin des extrêmes, est à abattre, dénonce-t-il. Par cette accusation, on a voulu discréditer ma parole. Calomniez, calomniez, il en restera toujours quelque chose. Je ne suis lié à personne. Je n’ai aucun lien avec l’UOIF [l’Union des organisations islamiques de France, aujourd’hui Musulmans de France, dans la mouvance des Frères musulmans]. C’est tellement gros ! Mais comment prouver que c’est faux ? »
\end{quote}

A force d’être pris en tenailles, la fatigue s’est installée. L’appréhension aussi parfois. « J’ai découvert récemment que tous les vendredis, ma famille aux Comores s’asseyait et pleurait, témoigne-t-il. Ma sœur m’a expliqué : “On a peur que quelqu’un s’en prenne à toi”. » « Je ne veux plus me mettre dans la gueule du loup pour des gens qui ne sont même pas reconnaissants, a-t-il décidé. Je reste fidèle à ma religion, à mon pays, la France, mais je ne suis plus imam. » Mohamed Bajrafil continuera à écrire des livres, à faire de la théologie et à travailler à la conciliation de l’islam et de la culture française. A d’autres de « prendre le relais » au minbar.

\section*{Qui seront les imams homologués ? Les limites du futur conseil national des imams}
\mn{\url{https://www.lemonde.fr/societe/article/2020/11/22/qui-seront-les-imams-homologues-les-limites-du-futur-conseil-national-des-imams_6060684_3224.html } Par Cécile Chambraud 
Publié le 22 novembre 2020 à 03h03 - Mis à jour le 22 novembre 2020 à 05h19 }


Réunies par Emmanuel Macron mercredi, les fédérations musulmanes ont accepté les principes d’un tel organisme. Mais ses mécanismes d’homologation et son crédit posent question. 

Tareq Oubrou n’en revient toujours pas. Cela fait près de deux ans que l’imam de Bordeaux travaille, avec des collègues, à la création d’un « conseil national des imams ». Auteur de plusieurs livres dans lesquels il développe les implications d’un islam acculturé au contexte européen, qui ont contribué à sa notoriété, il est convaincu de l’utilité d’une telle instance.
Mais pourquoi donc l’exécutif a-t-il choisi, pour la créer, des personnes dont aucune n’est imam ? Des présidents de fédérations dont les plus importantes sont liées à des pays de l’influence desquels on prétend justement vouloir s’extraire ? Pourquoi ne lui a-t-on pas même passé un coup de téléphone, ni aux autres religieux avec qui il travaille ? 
\begin{quote}
    « Aucun imam ou théologien de la scène musulmane française n’a été consulté, s’insurge-t-il. Or, un conseil national des imams, c’est l’affaire des imams, pas des laïcs ! »
\end{quote}

Mercredi 18 novembre, réunis à l’Elysée par Emmanuel Macron, les neuf présidents des fédérations musulmanes représentées au Conseil français du culte musulman (CFCM) ont accepté les principes d’un conseil national des imams. Celui-ci serait composé d’eux-mêmes, qui ne sont pas imams, et de neuf imams, un par fédération, choisi par son président. Autant dire qu’ils auront la complète maîtrise de cette instance, qui fonctionnera de la même manière que le CFCM, c’est-à-dire par consensus. Raison pour laquelle cet organisme n’a jamais pu faire aboutir les différents projets auxquels il a travaillé au cours des années, comme une charte du halal, une charte des imams, la mise en route d’un conseil théologique, qui existe mais ne s’est pour ainsi dire jamais réuni.
 
Anouar Kbibech, le président du Rassemblement des musulmans de France (RMF), a rappelé au chef de l’Etat que, cinq ans auparavant, jour pour jour, juste après les attentats de novembre 2015, il se trouvait place Beauvau avec le ministre de l’intérieur, Bernard Cazeneuve. Alors président du CFCM, Anouar Kbibech avait proposé ce jour-là un mécanisme de certification des imams par le CFCM. « Mais il n’y a pas eu de consensus à l’époque. Et on a perdu cinq ans », regrette-t-il aujourd’hui.
\paragraph{Qu’est-ce qui a changé depuis ?} « Aujourd’hui, il y a un dialogue direct du président de la République et du ministère de l’intérieur avec les présidents de fédération, explique-t-il. Cela permet de dépasser les bisbilles internes. » Autant dire qu’il a fallu « l’immense pression » promise par Emmanuel Macron dans son discours des Mureaux (Yvelines) pour que, individuellement et hors cadre du CFCM, les neuf fédérations donnent leur accord de principe à un texte de cadrage.
\subsection*{« Ils vont s’autolabelliser »}
Encore faut-il préciser que l’accord n’est pas complet. 
\begin{quote}
    « Il reste à s’entendre sur les critères d’évaluation des trois catégories d’imams, celui qui conduit les prières quotidiennes, le prédicateur du vendredi et le conférencier, indique Amar Lasfar, le président de Musulmans de France (MF), l’ex-Union des organisations islamiques de France. Ce n’est pas encore le cas à 100 \%. » 
\end{quote}
 
« Il n’y a pas de blocage, mais il reste du travail », confirme M. Kbibech. Les présidents de fédération ont promis au chef de l’Etat de parvenir à un accord dans les quinze jours, avant une prochaine réunion à l’Elysée.
\paragraph{Qui seront les imams homologués ?} Chaque ministre du culte devra faire une demande individuellement, bien évidemment sur la base du volontariat. Or, plus d’une moitié d’entre eux sont rattachés à des mosquées qui ne participent pas à l’élection du CFCM. Pourquoi iraient-ils demander cette certification à des fédérations qui leur sont étrangères ? Et à un organisme ignoré de la grande majorité des musulmans qui fréquentent les quelque 2 500 mosquées et salles de prière ?
Aujourd’hui, selon M. Lasfar, « toutes fédérations confondues, on ne dépasse pas cinq cents imams », auxquels s’ajoutent les trois cents imams détachés par la Turquie, l’Algérie et le Maroc, et qui sont appelés à disparaître d’ici à 2024. 
\begin{quote}
    « Les fédérations vont labelliser leurs propres imams sur des critères d’appartenance “ethnique”, raille M. Oubrou, qui prêche pour un conseil national indépendant des fédérations. Ils vont s’autolabelliser. » 
\end{quote}
Et « ces imams feront fuir tous les jeunes », qui ne goûtent guère une estampille d’« imam officiel », ajoute un acteur de l’organisation de l’islam.
Si l’on veut effectuer sérieusement ce travail d’homologation, vérifier les qualifications et les parcours, il faut un minimum de financement. Or, le CFCM en manque, ou du moins il ne s’est pas donné les moyens d’en avoir, la décision revenant, là encore, aux fédérations. « Le conseil national devra avoir ses propres moyens, indépendamment du CFCM, estime M. Kbibech. Chaque fédération devra mettre au pot. » En auront-elles la volonté ?
\subsection*{Des fédérations visées}
Mercredi, Emmanuel Macron a également exigé des neuf fédérations qu’elles s’accordent sur une charte « des valeurs républicaines », rédigée avec le ministère de l’intérieur – il a chargé Gérald Darmanin de boucler l’affaire –, qu’elles devront signer. Cette fois, il s’agit d’une sorte de « label » de républicanisme qui serait décerné par l’Etat. Cette charte comprendra une clause sur le fait que l’islam est une religion et pas un mouvement politique, et une autre sur la non-ingérence d’Etats étrangers. Si l’une d’entre elles ne la signait pas ou ne la respectait pas, a précisé l’Elysée, des « conséquences » en seraient tirées. De quel ordre ? Compte tenu du contexte, on peut penser à une menace voilée de dissolution.

Lors de la rencontre de mercredi, le chef de l’Etat n’a désigné personne. Mais chacun a bien compris qu’il s’agissait d’un test pour les deux fédérations turques et pour MF, dont les affinités avec les Frères musulmans lui valent souvent d’être accusée de promouvoir un islam politique. « A aucun moment on ne s’est sentis désignés, assure M. Lasfar. Il nous a tenu un discours franc, mais c’était à toutes les fédérations. Moi, j’ai été clair : les musulmans de France sont des citoyens d’abord. La charia des musulmans, c’est la loi de la République. Ces questions nous sont posées encore et encore, mais il y a longtemps que, nous, nous les avons évacuées. »


\section*{Tareq Oubrou : « Aux laïcs la gestion administrative de l’islam de France, aux religieux la question religieuse »}
\mn{\url{https://www.lemonde.fr/idees/article/2020/12/01/tareq-oubrou-aux-laics-la-gestion-administrative-de-l-islam-de-france-aux-religieux-la-question-religieuse_6061726_3232.html } Tribune 
Tareq Oubrou, Grand imam de Bordeaux Publié le 01 décembre 2020 à 05h25 - Mis à jour le 01 décembre 2020 à 09h20 Temps de Lecture 7 min. }



L’idée, proposée par le chef de l’Etat, d’un conseil national des imams est bonne, mais ce dernier ne peut être l’émanation du Conseil français du culte musulman (CFCM), instance peu représentative, souligne, dans une tribune au « Monde », le grand imam de Bordeaux, soutenu par dix imams et islamologues.

 L’Etat a récemment annoncé, par la voie du Conseil français du culte musulman (CFCM), la création d’un conseil national des imams [chargé de labelliser les formations religieuses des imams]. Si l’idée d’avoir un conseil des imams est bonne, les voies et moyens suivis ne sont pas les bons. D’abord, parce que ce conseil s’appuie sur le CFCM, alors que tout le monde sait qu’il est une institution fragile. Les premières à le savoir, ce sont les fédérations elles-mêmes qui le composent, elles qui, aujourd’hui comme hier, passent leur temps à s’y disputer. L’Etat le sait aussi.
Créée sous l’impulsion de Nicolas Sarkozy par des fédérations sous influences des pays étrangers (Maroc, Turquie, Algérie) et mêmes financée par eux, l’instance n’est pas opérationnelle. D’ailleurs, les Français de confession musulmane ne s’y reconnaissent pour ainsi dire pas. 1 100 mosquées n’ont pas participé à la dernière élection du CFCM dont la labellisation n’aura aucun poids parmi les fidèles. Rappelons au passage qu’aujourd’hui les citoyens sont libres. Grâce à la protection de l’Etat de droit, ils peuvent en tant que musulmans s’organiser en associations cultuelles en dehors de toute institution nationale officielle et désigner ceux qu’ils estiment les plus compétents pour être leurs imams. Ce qui veut dire que, labellisation ou pas, il y aura toujours des imams qui n’auront pas besoin d’être reconnus par une quelconque instance, surtout par le CFCM, et ne le voudront même pas.
\subsubsection*{Influences étrangères}
Les fédérations du CFCM – labellisateur non labellisé –, appelées à former le conseil national des imams, sont dirigées par des laïcs et pas par des religieux. Ils gèrent des associations, des fédérations. Ils font davantage de la politique que de la religion. Leur demander de créer un conseil des imams en l’absence des principaux concernés, c’est comme demander à des juges de créer un ordre des avocats. Les présidents de fédérations ne connaissent souvent rien à la théologie ni au droit canon musulman. Ils vont siéger à côté d’imams désignés par eux-mêmes, des imams qui dépendent d’eux. Ils auront donc prééminence sur eux, alors que ces présidents de fédérations dépendent pour la plupart de pays étrangers. Que va-t-il se passer ? On va officialiser en France les interprétations canoniques en vigueur dans ces pays, par la voix de gens sans savoir religieux crédible ! Et on demande en même temps, au travers de la charte des principes républicains, à ces mêmes représentants de l’influence étrangère de lutter… contre l’influence étrangère sur l’islam en France ! Comprenne qui pourra. Au lieu de demander au CFCM de constituer un conseil national des imams (« CNI »), l’Etat aurait dû exiger du CFCM qu’il se réforme, réforme qui se fait attendre depuis des années !

Mais, dépassons ce sujet et imaginons, dans une France imaginaire, que le problème des imams des mosquées soit réglé par le « CNI », que fera-t-on du vrai problème, celui de toute une armée d’« imams virtuels » qui dirigent des « mosquées virtuelles » entières, ces « communautés musulmanes connectées », constituées d’une jeunesse assoiffée de religion, si possible dans sa version la plus identitariste ? Comment labelliser ces « imams de fait » très influents, alors que leur aura et leur succès reposent largement sur le fait qu’ils prêchent en dehors des institutions officielles et souvent contre elles ?
\subsection*{Absence de déontologie}
L’intention de l’Etat est compréhensible, elle est même louable. Mais la démarche n’est pas bonne. Que va-t-il en effet se passer ? Les fédérations qui composent le CFCM vont habiliter leurs propres imams, comme elles le font déjà pour les aumôniers. Or, quelle est la situation des aumôneries ? Très mauvaise. Mis à part quelques exceptions très estimables, on trouve parmi les aumôniers « labellisés » par le CFCM des gens qui ne connaissent pas grand-chose à la religion ni même parfois aux piliers de la foi musulmane. Certains ont en outre du mal à accepter le principe de l’allégeance à la nation ! Tout cela à cause d’un mode de désignation dont la seule règle est le copinage ; et la seule déontologie, l’absence de déontologie. Une sorte d’auto-labellisation clientéliste est à l’œuvre, clientélisme auquel le conseil des imams, lié au CFCM, n’échappera pas. L’Etat connaît cette situation et s’en plaint. Pourquoi crée-t-il les conditions de sa reproduction avec les imams ?
Il ne suffit pas de déclarer la guerre au conservatisme traditionaliste ou au salafisme wahhabite, il faut lui proposer une alternative éclairée
L’Etat comprend qu’il y a un problème mais, et c’est frustrant pour lui, le problème est religieux et sa résolution incombe aux seuls musulmans. Ce problème, c’est la théologie. Est-ce que ce « CNI » aura le courage et les outils intellectuels théologico-canonico-éthiques pour répondre à cette problématique fondamentale ? Pourra-t-il penser une « théologie préventive », par exemple ? Quand on sait que certaines fédérations qui composent le CFCM ont des instituts qui enseignent des doctrines canoniques médiévales dont l’importation non critique engendre des comportements problématiques, voire dangereux, on sait bien que la réponse est négative. Il ne suffit donc pas de déclarer la guerre au conservatisme traditionaliste ou au salafisme wahhabite, il faut lui proposer une alternative éclairée, puisée dans l’interprétation des textes canoniques musulmans analysés dans le contexte de notre modernité occidentale.
Si la question théologique n’est pas traitée d’abord et avant tout, même la décision courageuse du gouvernement de stopper les visas des imams détachés ne mettra pas fin au problème. Car les discours des imams étrangers seront simplement remplacés par ceux des imams français qui véhiculeront les mêmes lectures problématiques, avec un français sans accent, voire vernaculaire. Le fond restera inchangé : la diffusion des mêmes doctrines en vigueur dans les pays musulmans d’origine.
\subsection*{Réformer le CFCM}
Alors que faire ? Tout d’abord il faudrait commencer par une réforme structurelle et statutaire du Conseil français du culte musulman et reformuler son lien avec les conseils régionaux du culte musulman (CRCM), lesquels pour l’instant ne servent à rien. Sa tâche devra se limiter à la gestion administrative et temporelle du culte, place éminente néanmoins, car le CFCM a vocation à être une pièce maîtresse dans le puzzle de la représentativité. Mais, il ne doit pas être la seule.
 
L’autre pièce est le conseil national des imams de France, une association indépendante, constituée uniquement d’imams, faite par des imams pour les imams, à laquelle un groupe d’entre eux travaille depuis dix-huit mois. Ce conseil collaborera avec le CFCM, lequel pourra proposer quelques imams pour siéger dans l’organe dirigeant qui devra se pencher sur l’élaboration d’un consensus doctrinal minimal ; agréer des imams – et non les « labelliser », nous ne sommes pas des produits de grande consommation ! – ; participer à la désignation des aumôniers et animer la réflexion intellectuelle musulmane.
La troisième pièce du dispositif, c’est l’organisme qui s’occupe du financement. C’est le « nerf de la guerre », qui permettra de favoriser l’émancipation progressive des musulmans de France à l’égard des pays étrangers, de leur influence politique comme théologique. C’est la mission de l’Association musulmane pour l’islam de France (AMIF), à la fois indépendante du CFCM et du conseil des imams afin d’éviter tout conflit d’intérêts. Elle peut cependant admettre en son sein des membres désignés par le CFCM et par le conseil des imams, notamment pour des questions d’ordre éthique. L’AMIF attend d’ailleurs depuis huit mois que les dirigeants du CFCM veuillent bien désigner leurs représentants au sein de son conseil d’administration, comme ils s’y étaient engagés. La nature des liens restera à préciser entre ces trois instances, indépendantes les unes des autres, pour tenir un certain équilibre et éviter ainsi toute subordination ou conflit d’intérêts. A côté d’elles, enfin, la Fondation pour l’islam de France (FIF) doit œuvrer pour la diffusion de la culture musulmane en France et soutenir la formation profane des imams.
Article réservé à nos abonnés Lire aussi « Sur l’organisation de l’islam de France, les solutions d’Emmanuel Macron reposent sur trois idées fausses » 
L’Etat doit accompagner cette structuration, comme il l’a fait en d’autres temps avec les autres cultes. Il peut faciliter le dialogue entre musulmans qui doit maintenant aboutir, car les palabres n’ont que trop duré. Mais, qu’il ne se trompe pas d’interlocuteurs : aux laïcs la gestion administrative du culte, aux religieux la question religieuse, à la société civile la question financière. Et à l’Etat le respect de la lettre et de l’esprit de la loi de 1905.
Tareq Oubrou, grand imam de Bordeaux, avec le soutien de : Mohamed Ayari, imam et président du conseil des imams des Alpes-Maritimes ; Mohamed Bajrafil, théologien et islamologue ; Nacer Bensalem, imam de la mosquée de Montreuil (Seine-Saint-Denis) ; Mahmoud Doua, Imam de la mosquée de Cenon (Gironde) ; Omar Dourmane, imam prédicateur à la mosquée de Brunoy (Essonne), professeur des sciences islamiques à la Faculté des sciences islamiques de Paris (FSIP) ; Azzeddine Gaci, conférencier et imam à Lyon ; Hassan Izzaoui, imam de la mosquée de Limoges ; Mohammed El Mahdi Krabch, imam et aumônier référent des hôpitaux de l’Hérault ; Abdelali Mamoun, imam et théologien ; Tahar Mahdi, imam à Conflans-Sainte-Honorine (Yvelines).



\subsection*{L'enseignement de la théologie islamique en France et la formation des
imams}
\mn{Montaigne}

L'analyse de l'offre et de la demande d'enseignement de l'islam à
l'université, depuis les années 1980 et 1990, fait apparaître
\textbf{deux dynamiques d'offre répondant à deux types de demandes.}


Une offre publique composite et incomplète


\textbf{Une dynamique de préconisation d'offre publique a émergé, sous
impulsion politique et universitaire, soutenue par les recommandations
de divers rapports publics :} création d'un enseignement religieux dans
les écoles en Alsace et en Moselle, préconisée par la Commission de
réflexion sur l'application du principe de laïcité dans la République
(Rapport Stasi, 2003), la Commission sur les relations des cultes avec
les pouvoirs publics (Rapport Machelon, 2006) et la Commission sur le
port du voile intégral sur le territoire national (Rapport Gérin, 2010).

Des universitaires (Mohamed Arkoun et Etienne Trocmé) ont inlassablement
tenté de sensibiliser les dirigeants politiques français et de formuler
une réponse à la demande d'enseignement religieux dans le cadre de la
loi de 1905.

Cette dynamique d'offre publique, mue par une volonté de création tantôt
d'une faculté de théologie musulmane à Strasbourg, tantôt d'un institut
islamique (Pierre Joxe et Alain Boyer en 1987), voire d'une École
nationale d'études islamiques fondée sur le modèle de l'École normale
supérieure (commission Stasi en 2003), pour former des enseignants à
l'enseignement du fait religieux et plus particulièrement l'enseignement
de l'islam et de la théologie islamique dans le supérieur.

\textbf{En 1997,} Jean-Pierre Chevènement relance l'idée d'un « institut
universitaire des hautes études de l'islam » ou des « études supérieures
islamiques » pour former des cadres musulmans au sein de l'Institut
national des langues et civilisations orientales (INALCO).

Entre 2005 et 2010, les facultés d'Aix-en-Provence, de Paris IV -- La
Sorbonne, puis de Paris 8 - Saint-Denis, ainsi que plusieurs
établissements publics, ont
tenté d'organiser des formations destinées aux futurs imams. La seule
initiative qui ait vu le jour est celle de l'Institut Catholique de
Paris, qui a créé, en 2008, un diplôme universitaire (DU) :
«Interculturalité, laïcité, religions ».

\textbf{En 2009,} le master d'islamologie de l'Université de Strasbourg
est ouvert. Cette formation offre un cadre général de formation d'imams
républicains.

\textbf{En 2016, une palette de treize offres publiques disparates et
éclatées de formations, d'enseignements et de diplômes existe.} Après
Paris, Lyon, Strasbourg, puis Montpellier, Aix et Bordeaux, sept nouveau
DU ont vu le jour en septembre 2015 à Sceaux, Paris 1, Lille, Toulouse,
Mayotte, Nantes et La Réunion.


Une offre privée de formation assez disparate


Il existe une dynamique d'offre privée, développée à l'initiative de
particuliers, d'associations ou de collectifs musulmans désirant
répondre à une demande de formation théologique des imams, de «
catéchèse musulmane » et d'exégèse du texte coranique.

Ainsi :


\begin{itemize}
\item
  en 1990, l'UOIF crée, à Saint-Léger de Fougeret, l'Institut européen
  des sciences humaines (IESH). Les frais de scolarité sont de 6 000
  euros par an ;
\item
  en 1993 est créée, sous l'impulsion financière saoudienne,
  l'Université islamique de France, à Mantes-la-Jolie ; elle est devenue
  en 1995 l'Institut d'études islamiques de Paris ;
\item
  en 1994, Dalil Boubakeur et Charles Pasqua sont à l'initiative de
  l'Institut Ghazali de formation des imams ;
\item
  en 1999 est créé l'Institut international des sciences islamiques
  (ISSI), qui propose une formation fondée sur le rite malékite ;
\item
  en 1999 l'International Institute of the Islamic Thought (IIIT),
  ouvert aux États- Unis en 1981, ouvre un établissement français :
  l'institut international de la pensée islamique à Saint-Ouen, en
  Seine-Saint-Denis ;
\item
  en 2001 est fondé l'Institut français des études et sciences
  islamiques (IFESI), à Boissy-Saint Léger, dans le Val-de-Marne ;
\item
  en octobre 2002, la Grande Mosquée de Paris relance son cursus de
  formation des imams, afin de former théologiquement des imams et des
  aumôniers femmes ;
\end{itemize}





\begin{itemize}
\item
  en 2006 est créé à Lille l'Institut Avicenne des Sciences humaines
  (IASH), qui a pour ambition de former les imams. L'Institut Avicenne
  réserve l'accès à son cursus aux seuls candidats justifiant de
  l'exercice de la fonction d'imam ou de responsable associatif depuis
  au moins six mois.
\end{itemize}

La formation des cadres religieux musulmans en Europe


Les modes de formation des cadres religieux musulmans en Europe
dépendent des statuts des cultes nationaux dans chaque pays.

\textbf{En Belgique,} le culte musulman est reconnu par l'État depuis
1974. L'Exécutif des musulmans de Belgique, équivalent du CFCM français,
a proposé en 2006 la création d'un statut des ministres du culte
musulman et la mise en place d'une formation à l'imamat de quatre à cinq
ans (théologie et formation civile et civique). Ce projet, demeuré
lettre morte, a été relancé en 2013.

En 2007, a été créée sur une initiative privée, une Faculté des sciences
islamiques de Bruxelles. Celle-ci a en 2008 signé une convention avec
l'université islamique européenne de Rotterdam, proche du mouvement turc
Nursi, et dispose d'un département en charge de la formation des imams.
Les diplômes qu'elle délivre ne sont pas reconnus par l'État.

\textbf{En Allemagne,} le ministère fédéral de l'Enseignement supérieur
s'est engagé en 2010 à financer pendant cinq ans des supports de postes
de professeurs dans des départements de théologie et de pédagogie
religieuse islamique. Les universités de Tübingen, Munster, Osnabruck,
Francfort sur le Main et Giessen ont également accompagné la création
d'instituts de théologie islamique en leur sein. Les pouvoirs publics
allemands ont toujours refusé la création d'une faculté libre de
théologie musulmane et ont au contraire privilégié l'intégration de
l'enseignement de la théologie islamique dans l'université publique.

\textbf{Au Royaume-Uni,} l'université dispense un enseignement de
théologie non confessionnelle. La formation des ministres du culte
s'appuie sur des enseignements dispensés dans les \emph{« private halls
»} : rattachés à une université, ces structures d'enseignement délivrent
des diplômes au nom de l'université. La plupart des \emph{« private
halls »} ont été fondés par des autorités religieuses.
Celles-ci fixent les programmes et sélectionnent leurs étudiants.
Toutefois, la puissance publique, dans un objectif de sauvegarde et de
préservation de la qualité et des standards scientifiques des
universités et des collèges britanniques, a chargé la \emph{Quality
Assurance Agence of Higher Education} d'évaluer la qualité de
l'enseignement dispensé par ces \emph{« private halls »}.

L'\emph{Islamic College} fondé de Londres, en 1998, délivre des diplômes
de théologie validés par la Middlesex University, dans le cadre d'un
partenariat passé entre les deux établissements

\textbf{En Suisse,} la puissance publique a affiché sa volonté de voir
formés en Suisse les imams et professeurs de religion intervenant dans
les écoles. Néanmoins, il n'existe pas d'institut suisse de théologie
musulmane : la formation des futurs cadres est assurée en France, à
l'Institut européen des sciences humaines (IESH) de Château Chinon,
fondé par des membres de l'UOIF.

En 2009, a été créé un certificat de formation continue « islam,
musulmans et société civile », décerné par l'université de Fribourg et
financé par l'Office fédéral des migrations. Il s'agit d'une formation
ambitieuse, qui comprend sept modules : épistémologie des sciences
islamiques ; gestion, management associatif ; finance et éthique ; islam
et médias ; histoire et civilisation de l'islam entre texte et contexte
; laïcité religions et politique ; diversité intégration et travail
social ; santé publique, pratiques religieuses et aumôneries. Toutefois,
le trop faible nombre d'inscrits à cette formation a conduit à sa
suppression.


\subsection*{Création du Forif -L’organisation du culte musulman entre dans une nouvelle ère }



 Un mois après l’annonce par le ministre de l’intérieur Gérald Darmanin de la « mort » du Conseil français du culte musulman (CFCM), la mise en place d’un Forum de l’islam de France (Forif) se précise. Pour l’instant simple « instance de dialogue » entre le culte musulman et l’État, elle pourrait déboucher sur une nouvelle instance représentative.
\mn{Mélinée Le Priol, le 06/01/2022 à 19:34 Modifié le 06/01/2022 à 19:36}


Une nouvelle étape a été franchie, au soir du 5 janvier, à l’Élysée, dans la lente agonie du Conseil français du culte musulman (CFCM) et la restructuration de l’islam de France. Face aux différents représentants des cultes, qu’il recevait en présence du premier ministre et du ministre de l’intérieur, Emmanuel Macron a entre autres confirmé la création d’un Forum de l’islam de France (Forif).



Ce n’est pas, en soi, une nouveauté : l’organisation prochaine d’un tel « forum » avait été annoncée par le ministre de l’intérieur Gérald Darmanin dès le 9 décembre – trois jours avant que ce dernier acte unilatéralement de la « mort »du CFCM, déclarant que cette instancen’était « plus l’interlocuteur de la République ». Sauf que, cette fois, un dialogue a pu avoir lieu avec les principaux intéressés. Au point que Mohammed Moussaoui, président du CFCM depuis deux ans, reconnaît désormais que la structure créée par Nicolas Sarkozy en 2003 « n’est plus viable dans son format actuel ».

\begin{quote}
    « Après ma rencontre mercredi soir avec Gérald Darmanin puis le discours du président de la République à l’Élysée, je considère qu’il y a une possibilité de faire émerger, de façon coordonnée avec les pouvoirs publics, une nouvelle organisation du culte musulman », déclare ce professeur de mathématiques d’origine marocaine, qui dit avoir échoué à mener cette réforme « de l’intérieur » en raison des dissensions internes au CFCM. « Cette nouvelle organisation doit être le fruit d’une concertation associant le Forif et les structures départementales du culte musulman. »
\end{quote}


\paragraph{Fausse unanimité}
L’objectif affiché par l’exécutif est clair : en finir avec la tutelle des fédérations de mosquées affiliées à des pays (Maroc, Turquie, Algérie) et, in fine, avec « l’islam consulaire ». « Si certaines fédérations prennent part au Forif, ce ne sera qu’en tant que gestionnaires de lieux de culte ou d’instituts de formation (la Grande Mosquée de Paris, par exemple, contrôle l’Institut Al-Ghazali, NDRL), pas en vertu de leur représentativité supposée », assure-t-on au ministère de l’intérieur. De fait, la majorité des 2 500 lieux de culte musulmans ne sont affiliés à aucune fédération.

\paragraph{Les représentants du culte musulman revendiquent chacun le Conseil national des imams}

Après avoir constaté la « fausse unanimité » du CFCM, dont quatre des neuf fédérations ont claqué la porte en mars dernier à cause de désaccords sur la Charte des principes pour l’islam de France, le gouvernement veut désormais s’appuyer sur des acteurs de terrain : imams, responsables associatifs… Ils sont un millier à avoir pris part, au printemps dernier, à la troisième édition des Assises territoriales de l’islam de France. C’est dans leur sillage que s’inscrit le Forif.

Mais celui-ci sera plus resserré, avec entre 80 et 100 participants : une vingtaine pour chacun des quatre groupes de travail déjà mis sur pied par le Bureau central des cultes (BCC). Portant sur la formation des cadres religieux, les aumôneries, les actes anti-musulmans et l’application de la loi confortant le respect des principes de la République, ces groupes ont commencé à se réunir ces deux derniers jours. Ce sera à nouveau le cas régulièrement après le premier « forum » qui doit se tenir à Paris, fin janvier ou début février. Un rendez-vous national appelé à devenir annuel.

\paragraph{Apprendre des erreurs du passé}
C’est bien vers un remplacement du CFCM par le Forif que l’on semble s’acheminer. Mais prendra-t-il à terme, lui aussi, la forme d’une instance représentative, jouant le rôle d’interlocuteur auprès des pouvoirs publics ? Sur ce point, le ministère de l’intérieur reste évasif : « Le Forif est à tout le moins une instance de discussion ; nous verrons plus tard comment ses participants veulent porter la question de la représentation. »

«Il ne faut pas croire que c’est au CFCM que se joue l’avenir de l’islam de France»
Parmi les musulmans, des critiques émergent déjà : les uns redoutent que le Forif soit une structure « fourre-tout » peu maniable, les autres qu’il soit aussi peu représentatif que le CFCM, ses participants ayant été choisis dans les départements par les préfets. Beaucoup veulent en tout cas « aller de l’avant » et apprendre des erreurs du passé.

« Quand on construit une maison, on ne commence pas par le toit mais par les fondations : c’était ça, le problème du CFCM », estime Kamel Kabtane, recteur de la Grande Mosquée de Lyon et artisan de la « départementalisation » des structures au niveau du Rhône. Sa présence, mercredi, à l’Élysée, aux côtés de Mohammed Moussaoui et du recteur de la Grande Mosquée de Paris Chems-Eddine Hafiz, était une première. Le signe de la volonté de l’État d’élargir le panel de ses interlocuteurs.