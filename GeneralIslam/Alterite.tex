\chapter{Islam et Altérité}




\section{Islam, pourquoi cette sévérité avec les autres croyants et les
incroyants ?}

\emph{Explication~}

« Mécréants », « infidèles » : les terroristes islamistes s'en sont pris
violemment, ces dernières années, à tous ceux qu'ils jugent hors de
l'islam « authentique ». Une intolérance fondée sur une lecture
littérale du Coran. 

\mn{ La Croix
  Mélinée Le Priol,~
  le~28/01/2021 à 13:04~
}
Selon la théologie musulmane, l'islam est la religion originelle de
l'humanité.

\subsection{Que dit la tradition ?}

Selon la théologie musulmane, l'islam est la religion originelle de
l'humanité.~\emph{« Tout homme est né musulman »,}~dit un hadith
attribué
au~\href{https://www.la-croix.com/sacralite-prophete-lislam-2020-11-06-1101123195}{\underline{prophète
Mohammed}}. L'homme est né pour adorer Dieu : certes, il a reçu une
dignité plus haute que les autres créatures, mais celle-ci est
conditionnée à sa soumission au Dieu unique. Plus un homme applique la
loi divine (\emph{charia}), plus il devient humain. Quant au « mécréant
» (\emph{kâfir}), qui refuse de suivre la charia, il se situe en quelque
sorte à un degré inférieur d'humanité.

Cette sévérité envers les non-musulmans s'appuie sur la lecture du texte
coranique qui s'est imposée à partir du IX\textsuperscript{e}~siècle,
lors de la transformation de l'islam en un empire soucieux de se
légitimer. Confortée par des hadiths rédigés à cette époque, elle
dépeint une vérité unique et non négociable. Elle insiste sur les
versets du Coran particulièrement virulents envers les polythéistes,
païens ou idolâtres, qualifiés d'\emph{« associateurs
»}~(\emph{mouchrikoun}) car ils « associent » à Dieu d'autres divinités.

Quant aux athées,~\emph{« ils appartiennent, selon la théologie
musulmane, à une catégorie de mécréance encore inférieure aux
polythéistes, aux juifs et aux chrétiens »,~}explique l'islamologue
Abdessamad Belhaj, chercheur au Centre interdisciplinaire d'études de
l'islam dans le monde contemporain de l'Université catholique de
Louvain. Même si des institutions comme le Haut Conseil des oulémas du
Maroc ou la Maison de la fatwa en Égypte considèrent que les apostats ne
peuvent plus être condamnés à mort, cette peine reste appliquée dans une
dizaine de pays, comme l'Afghanistan ou
la~\href{https://www.la-croix.com/Monde/Afrique/prisons-Mauritanie-calvaire-dun-apostat-2019-09-30-1201051050}{\underline{Mauritanie}}.

\subsection{ Pourquoi juifs et chrétiens bénéficient-ils d'un statut
spécifique ?}

Selon la tradition musulmane, chrétiens et juifs font l'objet d'un
traitement différent des autres non-musulmans : ils bénéficient dans le
droit islamique d'une protection juridique particulière (\emph{dhimma})
toutefois accompagnée d'injonctions humiliantes, comme l'interdiction de
monter à cheval ou de construire des lieux de culte dépassant ceux des
musulmans.
 

\emph{« Le Coran est très ambivalent au sujet des ``gens du Livre''
»,~}rappelle
l'historien~\href{https://www.la-croix.com/Culture/Livres-et-idees/historiens-decryptent-Coran-avant-lislam-2019-11-27-1201063090}{\underline{Guillaume
Dye}}, professeur à l'Université libre de Bruxelles (1) Il a codirigé avec Mohammad Ali Amir-Moezzi, Le Coran des
historiens, \mn{\cite{Amir:CoranHistoriens}} . Selon la
sourate 5, juifs et chrétiens ne doivent pas être pris pour~\emph{«
alliés »~}(5, 51) mais, quelques versets plus loin, on lit qu'ils ne
seront~\emph{« point affligés »~}(5, 69). Les chrétiens se voient
reprocher de nier l'unicité de Dieu mais du respect est exprimé pour les
prêtres et les moines, qui~\emph{« ne s'enflent pas d'orgueil ».}

\begin{Def}[tahrif]
Selon une théologie dite de la falsification (\emph{tahrif}), les juifs
et les chrétiens ont altéré le message transmis par leurs prophètes
respectifs (Moïse, Jésus), message qui n'était autre que l'islam. Le
Coran, lui, corrige cette déviation en transmettant fidèlement le
message révélé à un ultime prophète, Mohammed.
\end{Def}
 À Médine, celui-ci aurait
signé une~\emph{« Constitution »~}disposant que les juifs, notamment,
pouvaient pratiquer leur religion en sécurité, mais ces relations se
sont rapidement détériorées.

\subsection{Quelles pistes pour une «théologie du pluralisme»?}

Les attentats visant des « mécréants » en terrasse à Paris, les
persécutions contre les Yézidis ou les chrétiens en Irak, sont autant de
conséquences d'une lecture littéraliste du Coran encouragée par l'essor
du salafisme saoudien à partir des années 1970. D'autres lectures ont
pourtant existé dès les premiers siècles de l'islam. Contrairement à la
doctrine sunnite traditionnelle, l'exégèse rationaliste a par exemple
conclu très tôt à une~\emph{« égalité entre tous les êtres humains, tous
étant dotés de la même raison les rendant aptes à comprendre la parole
de Dieu »,~}rappelle l'islamologue Pierre Lory, directeur d'études à
l'École pratique des hautes études (EPHE).

Pour Abdessamad Belhaj, tout l'enjeu est aujourd'hui de refonder le
rapport à l'altérité sur la base de l'éthique, et de\emph{~« mettre
l'homme au cœur de la théologie »}. Pour cela, certaines valeurs
présentes dans l'islam gagneraient à être redécouvertes, comme celles du
soin, du don et du service à l'humanité, longtemps éclipsées selon lui
par l'autorité et la loyauté à la communauté musulmane ou à la tribu.


%----------------------------------------------------
\section{Liberté de conscience en Egypte}
\mn{Dominique Avon, thèse sur l'IDEO;  directeur d’études à l’Ecole pratique des hautes études (EPHE), où il occupe la chaire « Islam sunnite », raconte la genèse de ce texte. conférence du 17/5/21}

\begin{Def}[Intégralisme]
\begin{itemize}
    \item aucun contexte
    \item religion a raison face au politique
\end{itemize}
\end{Def}

 
\begin{itemize}
    \item Introduction dans la langue arabe de liberté religieuse a entrainé attraction et rejet
\item Nations Unis
\item Pourquoi elle est contestée
\end{itemize}

\begin{Prop}
Liberté de conscience : liberté individuelle
Liberté religieuse : liberté individuelle et collective : on ne fait pas religion seule
\end{Prop}


\subsection{Nations Unis}
\begin{quote}
    
\end{quote}
 L’article 18 de la Déclaration universelle des droits de l’homme, adoptée aux Nations unies en 1948, a instauré un principe inédit : la centralité de l’individu, sujet ultime du droit, au-delà de ses appartenances. Dans son nouveau livre,  Mais La Liberté de conscience, qui devrait s’imposer comme le livre de référence sur le sujet, est bien plus que cela.

Quand, où et dans quelles  conditions est apparue l’idée de la liberté individuelle ? Comment penser sa spécificité ? Comment – et pourquoi – la préserver, en un temps où, du radicalisme musulman aux diverses tentations illibérales, elle est sans cesse attaquée ou subvertie ? L’historien, en analysant des milliers de sources, de l’Antiquité à nos jours, partout dans le monde, reconstitue avec une minutie et une ampleur fascinantes le destin sinueux d’une espérance.

Comment avez-vous été amené à conduire une enquête de cette ampleur ?
Je suis parti de problèmes de traduction. Quand j’enseignais au Liban, au milieu des années 2000, j’ai constaté que, par exemple, la Constitution libanaise emploie l’expression « liberté de conscience » dans sa version française mais, en arabe, il n’est question que de « liberté de doctrine  religieuse », une notion plus restrictive, qui n’inclut pas la possibilité de l’athéisme : chacun est censé appartenir à une religion. J’ai voulu comprendre sur quels fondements tout cela repose, et j’ai commencé à remonter de plus en plus loin et à élargir la question à l’ensemble des cultures. Je me suis aperçu qu’il fallait à la fois certaines circonstances, et une anthropologie particulière. Voilà pourquoi le livre marche sur deux pieds : l’histoire  politique et religieuse, et une analyse des outils conceptuels qui ont permis aux penseurs d’aborder la liberté de conscience.
 
 
 % ---------------------------------------
\section{de l'Altérité par rapport à la polémique de Gims sur la bonne année}

\mn{M. Barjafil \url{https://www.saphirnews.com/Le-calendrier-musulman-a-failli-etre-romain--Reponse-a-la-polemique-des-voeux-de-Maitre-Gims_a28530.html}}
Venons-en à la polémique à proprement parler. Des textes, d’hier comme d’aujourd’hui, existent qui interdisent de souhaiter de bonnes fêtes aux non-musulmans, surtout si ces dernières sont religieuses, c’est indéniable. D’Ibn Taymiya (m. 1328) à Ibn Bāz (théologien contemporain d’une grande influence dans le milieu salafo-wahhabite, mort en 1999), en passant par Ibn Al-Qayyim (m. 1350), disciple du premier, ce ne sont pas les auteurs qui soutiennent ce type de propos qui manquent. Pire, il en existe même qui interdisent d’être le premier à saluer son voisin juif ou chrétien. Est-ce clair maintenant ? Regardez un des livres de référence du droit musulman, Al-maǧmūʿ, de l’imam Al-Nawawī (m. 1277). Vous y trouverez, pages 468-469 volume 4, ceci : « Il est interdit (de dire « Paix à vous » (Assalāmu ‘alaykum), aux mécréants, non-musulmans, s’entend. Ceci est l’avis sain, et ce qu’a tranché la majorité. » Il va, plus loin dans la même page, affirmer ceci : « Et si on dit "salam" à quelqu’un et que l’on s’aperçoit qu’il est non-musulman, il nous est recommandé de nous faire rendre notre salam, en disant : "Je reprends mon salam !" »

Mon but n’est pas, loin s’en faut, de dénigrer cet imam, encore moins de dire qu’il n’était pas musulman. Qui suis-je ? Il est simplement ici de dire deux choses. La première est que notre bon sens doit travailler tout le temps, certes. Mais encore plus face à pareilles affirmations qui nous coupent du monde, y compris, parfois, des nôtres, si nous les adoptons.

Osons le mot : c’est cela le vrai séparatisme. Pas celui, souvent exagéré, sinon imaginaire, prétendu ici et là, dans les médias et par certains politiques, qui aboutit à des décisions absconses et aggrave la situation, parce qu’il stigmatise et amalgame indistinctement, à mon avis, tout en prétendant la régler. Car depuis quand on éduque les gens à coup de lois ? Depuis quand on lutte contre une idéologie par les biceps ? On dit en arabe que « c’est avec du fer qu’on arrive à tordre le fer ». Il faudrait donc une idéologie alternative, interne à l’islam, qui s’attaque à ce « séparatisme », qui commence par toucher les musulmans eux-mêmes. Dans ledit livre et sur la même page, il est rapporté, en effet, que le musulman que l’on considère comme hérétique, on ne le salue, de préférence, pas. Nous sortirions de nos contradictions et métrions à l’abri quiconque serait tenté par de telles billevesées, si nous acceptions qu’elles existent.

La seconde est de rappeler à tous que les réalités vécues par Al-Nawawī sont différentes des nôtres. En effet, il y a fort à parier qu’il n’avait pas comme sage-femme Valérie, comme pédiatre Philippe, comme médecin traitant de sa famille Michèle, comme professeure Dorothée... Or, c’est le cas aujourd’hui du musulman français, anglais ou européen. Ce qui m’amène à affirmer que si l’imam Al-Nawawī avait vécu nos réalités, il n’aurait très certainement pas dit pareilles choses, car « la conscience humaine est fille de son environnement ». Ce n’est pas être fidèle à Al-Nawawī, lui qui a vécu dans un contexte de guerres incessantes entre musulmans et chrétiens, que de transférer sa production telle quelle en 2022, en France, où le musulman vit comme susdit. C’est même criminel.