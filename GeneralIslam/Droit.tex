\chapter{Droit en Islam}

\begin{Def}[Waqf - Habous]
Le Waqf (arabe :\TArabe{ ﻭﻗﻒ } pl. : awqaf \TArabe{ أوقاف)} est, dans le droit islamique, une donation faite à perpétuité par un particulier à une œuvre d'utilité publique, pieuse ou charitable, ou à un ou plusieurs individus. Le bien donné en usufruit est dès lors placé sous séquestre et devient inaliénable. Au Maghreb, le waqf est appelé Habis (en arabe :\TArabe{ ﺣﺒﺲ }pl. : habous\TArabe{ الحبوس)}.
\end{Def}



Si la zakât est obligatoire pour tout musulman solvable, le waqf, dont la possibilité n'est évidemment offerte qu'aux seuls possédants, est facultatif. Il procède en tout cas, dans le droit traditionnel, du même esprit de subordination de l'usage de la propriété privée au bien général de la Cité . Dans tous les cas, il s'agit d'une obligation charitable (Coran, v. 92, s. 3).