\chapter{Colloque ISTR Mars 2023 Rabat}


\section{Synthèse pour les amis de l'IDEO}

\subsection{Etre chrétien minoritaire dans un pays musulman : l’expérience marocaine}

En quoi l’expérience chrétienne au Maroc nous intéresse-t-elle ? Quelle intuition a eu le pape François pour nommer cardinal, l’archevêque de Rabat, l’une plus petits diocèses du monde, où la grande majorité est musulmane ?  
Je voudrais partager ici quelques réflexions rapportées du colloque organisé le 16 et 17 mars par l’institut de théologie Al-Muwafaqa de Rabat pour ses 10 ans. Cet institut de théologie parrainé par l’Institut Catholique de Paris et la faculté Protestante de théologie de Strasbourg a vocation de former dans un même lieu les cadres et pasteurs dont ont besoin l’Eglise du Maroc et les pays subsahariens. 

Le premier étonnement vient de la découverte de la vitalité de cette Eglise du fait des migrations subsahariennes d’étudiants et de travailleurs depuis une vingtaine d’années.  A la différence du judaïsme maghrébin, présent lors du colloque et qui doit reconstruire son identité dans un contexte d’extrême faiblesse numérique, les chrétiens sont très minoritaires mais forment une église vivante avec une culture subsaharienne. C’est un défi en termes d’accueil pour l’Église héritée de l’époque coloniale, qui vit avec une simplicité heureuse d’être bousculé et de devenir peu à peu Eglise africaine. Ainsi, le visiteur Européen qui choisit d’aller à la messe à Notre Dame de Lourdes à Casablanca (« l’église de la Vierge de Gad Elmaleh ») aura la surprise après plus d’une heure de messe dans une église bondée, de commencer une procession de 20 mn chantée et dynamique pour aller mettre son offrande devant l’autel.  
Pour former ces pasteurs dont les Églises au Maroc et en Afrique ont besoin, Al-Muwafaqa vit de plusieurs intuitions : 
\begin{quote}
  \item	vivre une expérience œcuménique, ce qui n’est pas forcément évident à vivre naturellement (Jessica, Pentecôtiste béninoise et ancienne étudiante de l’institut nous partagea avec simplicité son interrogation initiale  sur le fait que les « catholiques puissent-ils être de vrais chrétiens » ?) 
  \item		vivre une théologie incarnée dans la culture. Ce qui veut dire en terre marocaine, se former à l’Islam.
\end{quote}

A l’issue de ce colloque, quelques conclusions peuvent être tirées qui sont pertinentes au-delà du Maroc :
\paragraph{D’abord, le phénomène de sécularisation}
 que l’Église connaît en Europe, touche aussi l’Islam du Maroc, sans être un phénomène linéaire et avec des phénomènes paradoxaux (comme le développement du voile). Plusieurs évolutions du droit islamique ont eu lieu suivant les aspirations de la société. Quand ces dernières sont fortes, l’ijtihad, l’interprétation des textes coraniques, devient plus créative. Ainsi, sur la réforme du droit successoral, l’ijtihad a permis d’envisager la suppression du Taasib \label{Def:Taasib}, l’héritage des oncles au détriment des filles à la mort de leur père, car elle n’est plus acceptée socialement et la contrepartie demandée aux oncles (« prendre soin de leurs nièces ») ne correspond plus à aucune réalité sociale. En revanche, les théologiens salafistes s’opposent toujours à une stricte égalité de l’héritage entre hommes et des femmes (« règle du double héritage pour les hommes ») car cette inégalité est encore acceptée socialement par certains marocains. Par ailleurs, les nombreuses propositions religieuses sur internet (« islam TikTok ») peu contrôlables par l’Etat Marocain, cassent l’unité de l’Islam marocain pourtant fort de la légitimité du Roi, Commandeur des Croyants.  De plus en plus, les Marocains font leur propre « menu » religieux, marquant le repli de la religion dans la sphère privée. 

Le colloque a aussi montré l’état des recherches théologiques au Maroc autour de la Charte de Médine, qui scelle l’alliance du Prophète Mohammed et des tribus juives . Cette charte permet de penser un vivre-ensemble ancré dans la Tradition musulmane et dépassant le statut de dhimmis, certes « protégés » mais pas pleinement citoyens. 
\paragraph{Au Maroc, j’ai pu être témoin d’une Église minoritaire}
 mais qui assume sereinement cette minorité : avec une certaine visibilité, comme l’institut Al-Muwafaqa, et vivant du service et de l’hospitalité, en particulier des migrants de toutes religions.  Plusieurs débats ont rejeté le terme de  minoritaire  qui nous force à nous définir par rapport à une majorité et qui méconnait que toute identité est plurielle et en dialogue : chrétien catholique mais aussi français, ingénieur,… Cette question présente un enjeu théologique pour les chrétiens : Déjà la lettre de Diognète à la fin du IIè siècle  mentionnait que \textit{les chrétiens ne se distinguent des autres hommes ni par le pays, ni par le langage, ni par les coutumes.} Il s’agit donc de vivre une identité affirmée et ouverte, qui ne nie pas la faiblesse numérique mais ne s’enferme pas dans une posture victimaire : pour les chrétiens, former un corps n’est pas une fin en soi mais c’est se mettre au service du Royaume de Dieu dans ce monde. \textit{Le risque n’est pas d’être minorité mais d’être insignifiant (Pape François).}

\section{Cardinal Cristóbal - Rabat}

Le Cardinal Cristóbal LÓPEZ ROMERO est l'archevêque de Rabat. Né le 19 mai 1952 à Vélez-Rubio (Espagne). Religieux Salésien de Don Bosco depuis le 16 août 1968, il a été ordonné prêtre le 19 mai 1979 à Barcelone.

\begin{quote}
    Je suis Christobal, fils de Dieu, mon titre le plus important.
\end{quote}

\paragraph{CERNA} Conférence Episcopal de la réion Nord Afrique : Lybie, Tunisie, Algérie, Maroc. 

\paragraph{Rabat} 36 prêtres, fidei donum ou religieux. Tanger (15 prêtres). 30 000 km au Maroc. 0,1\% des marocains.
Mais une Eglise jeune. Subsaharien, une église de 35 ans. Plus d'hommes que de femmes. Eglise de passagers, noirs. 

\paragraph{Une église où les jeunes peuvent la rester}
Equipe ACI

\paragraph{Une église insignifiante mais significative}
signe pour la société et pour l'Eglise.

\begin{Def}[dépression religieuse]
En Espagne, tentation de frustration. Ici, à Rabat, la grâce d'être minoritaire. 
\end{Def}
\paragraph{Préparez l'examen} On connait déjà les questions de l'examen final, donc préparons-nous.

\paragraph{Conversion} Changer de religion, ce n'est pas forcément se convertir. Pour épouser une musulmane, il faut dire la \textit{shahada}. 
\begin{quote}
    un jeune a décidé de rompre avec son amie le jour où il a découvert qu'il devait renoncer à sa foi chrétienne.
\end{quote}


\paragraph{Mon objectif, c'est une Eglise pour construire le royaume de Dieu} Pas prosélyte pour un objectif quantitatif, car alors je suis \textit{Pepsi Cola qui cherche à prendre des parts de marché de Coca Cola}.

\paragraph{Adveniat Regnum Tuum} Il faut que le levain soit plus petit que la pâte. 

\paragraph{Eglise incarnée au Maroc} L'Eglise veut être marocaine. Nous voulons donner de la saveur à la société marocaine. 

\paragraph{Eglise de dialogue et de rencontres} Une Eglise de \textit{sortie}. 
Christian de Chergé sur la visitation : 
\begin{quote}
    Visitation. J'imagine assez bien que nous sommes dans cette situation de Marie qui va voir sa cousine Élisabeth et qui porte en elle un secret vivant qui est encore celui que nous pouvons porter nous-mêmes, une Bonne Nouvelle vivante. Elle l'a reçue d'un ange. 
    
    C'est son secret et c'est aussi le secret de Dieu. Et elle ne doit pas savoir comment s'y prendre pour livrer ce secret. 
    
    Va-t-elle dire quelque chose à Élisabeth ? Peut-elle le dire ? Comment le dire ? Comment s'y prendre ? Faut-il le cacher ? 
    
    
    Et pourtant, tout en elle déborde, mais elle ne sait pas. D'abord c'est le secret de Dieu. Et puis, il se passe quelque chose de semblable dans le sein d’Élisabeth. Elle aussi porte un enfant. 
    
    
    Et ce que Marie ne sait pas trop, c'est le lien, le rapport, entre cet enfant qu'elle porte et l'enfant qu’Élisabeth porte. Et ça lui serait plus facile de s'exprimer si elle savait ce lien. 
    
    Mais sur ce point précis, elle n'a pas eu de révélation, sur la dépendance mutuelle entre les deux enfants. Elle sait simplement qu'il y a un lien puisque c'est le signe qui lui a été donné : sa cousine Élisabeth. 

    
    Et il en est ainsi de notre Église qui porte en elle une Bonne Nouvelle - et notre Église c'est chacun de nous - et nous sommes venus un peu comme Marie, d'abord pour rendre service (finalement c'est sa première ambition) ...
    
    
    mais aussi, en portant cette Bonne Nouvelle, comment nous allons nous y prendre pour la dire ... et nous savons que ceux que nous sommes venus rencontrer, ils sont un peu comme Élisabeth, ils sont porteurs d'un message qui vient de Dieu. 
    
    
    Et notre Église ne nous dit pas et ne sait pas quel est le lien exact entre la Bonne Nouvelle que nous portons et ce message qui fait vivre l'autre. 
    
    Finalement, mon Église ne me dit pas quel est le lien entre le Christ et l'Islam. Et je vais vers les musulmans sans savoir quel est ce lien. Et quand Marie arrive, voici que c'est Élisabeth qui parle la première. 
    
    Pas tout à fait exact car Marie a dit : as salam alaikum ! Que la paix soit avec vous ! Et ça c'est une chose que nous pouvons faire. Cette simple salutation a fait vibrer quelque chose, quelqu'un en Élisabeth. Et dans sa vibration, quelque chose s'est dit... qui était la Bonne Nouvelle, pas toute la Bonne Nouvelle, mais ce qu'on pouvait en percevoir dans le moment.
    
    D'où me vient-il que l'enfant qui est en moi a tressailli ? Et vraisemblablement, l'enfant qui était en Marie a tressailli le premier. En fait, c'est entre les enfants que cela s'est passé cette affaire-là … Et Élisabeth a libéré le Magnificat de Marie. 
    
    
    Finalement, si nous sommes attentifs et si nous situons à ce niveau-là notre rencontre avec l'autre, dans une attention et une volonté de le rejoindre, et aussi dans un besoin de ce qu'il est et de ce qu'il a à nous dire, vraisemblablement, il va nous dire quelque chose qui va rejoindre ce que nous portons, montrant qu'il est de connivence ... et nous permettant d'élargir notre Eucharistie, car finalement, le Magnificat que nous pouvons, qu'il nous est donné de chanter: c'est l'Eucharistie. 
    
    
    La première Eucharistie de l’Église, c'était le Magnificat de Marie. Ce qui veut dire le besoin où nous sommes de l'autre pour faire Eucharistie : pour vous et pour la multitude ... 
\end{quote}
Christian de Chergé  : j'ai hâte de voir Dieu pour voir comment il regarde l'histoire.
Quand on parle de dialogue, il s'agit moin sd'un dialogue sicientifique que d'un dialogue de vie entre chrétiens et musulmans. \textit{Nous pouvons vivre ensemble Chrétiens et musulmans}.

\paragraph{Une Eglise samaritaine} Une Eglise diaconale, au service. Les jeunes marocains ont un déficit d'espérance. Un facteur psychologique : impression d'être emprisonné (physiquement, entouré).

\paragraph{Eglise pont} \textit{pont} : nous sommes des pontifes. Nous devons établir des ponts. \textit{Kenitra} : \textit{Alcantara}, le pont. 

\paragraph{Passionné} par le Christ, et par le Maroc. Et alors \textit{passionnante}. 

\paragraph{Se laisser accueillir} quand on se laisse accueillir, on transmet le Christ.


\section{Soeur Gaby}

\paragraph{Soeurs} désir d'être présente aux marocains. Petites soeurs de Jésus, les franciscaines (à Midal). Nous avons beaucoup fondu. 
Dans l'est du Maroc, soeurs haitiennes et congolaises. Pas d'évangélisation directe. 
\paragraph{Présence pour les petites soeurs de Jésus} on aime bien cette spiritualité de la présence.

\paragraph{Soutien des subsahariens}

\paragraph{Moines de Mides} après Tibhirine dans le  \href{https://telquel.ma/2019/04/04/lheritage-des-moines-de-tibhirine-au-maroc-raconte_1633845}{moyen Atlas}.

\subsection{Petites soeurs des Pauvres}

\paragraph{Petite soeur Madeleine}

\paragraph{Charles e Foucault} vie desordonnée. 
\begin{quote}
    Islam : ... continuelle présence de Dieu... forte impression sur moi. 
\end{quote}

\begin{quote}
    celui qui a attrapé la passion de Jésus, ne peut guérir
\end{quote}
Beniabet.


\paragraph{Soin du scorbut} Il est touché par le scorbut en 1908. Les femmes touareg lui donnent le lait de chèvre. Cette guérison est pour lui un changement. 

\paragraph{Soeur magdeleine} \href{https://fr.wikipedia.org/wiki/Magdeleine_Hutin}{Soeur Magdeleine} 1939, fondation des petites soeurs. Désir de rejoindre les nomades; Après 14-18, sa mère est seule et risque de chaise roulante, "sauf si elle va au Sahara". Elle part dans la région de Boughari à l'Ouest d'Alger. Cherche une vie contemplative. 

\paragraph{les amis musulmans de Magdeleine} Influence des amis musulmans, ex : 
\begin{Ex}
    Par exemple, prendre soin de ses parents jusqu'à la mort
\end{Ex}

\begin{Synthesis}
L'islam est comme la matrice des soeurs de Jésus.
\end{Synthesis}
Ouverture en dehors de l'Islam mais marquée par l'Islam.
\begin{quote}
    L'Islam est une puissance énorme, ... mérite l'offrande d'une congrégation entière. 
\end{quote}

\paragraph{Dans la grande simplicité de la vie ordinaire}

\subsection{Dialogue avec les musulmans}

\paragraph{Pas un échange sur la foi} De temps en temps, certains essayent de nous situer : "c'est quoi votre jeune ?".

\paragraph{Prière} On parle l'arabe dialectal. Il nous manque des mots, 
Nous rencontrons les autres au niveau où nous nous situons nous mêmes : être disponible aux événements de la rue,...

\paragraph{Fraternité de Casa} Groupe de femmes européennes, épouse de marocains. 
\begin{Ex}
3 filles d'une de ses femmes. 
Eucharistie. 
Très touchée par les bougies autour du corps.
Ils ont prié la Fathiha. 
\end{Ex}


\paragraph{Fraternité de Fès} Avec les amies soufies. Amitié, terreau du dialogue. 

\paragraph{On se bagarre pour la fraternité} Pas besoin d'être religieux ou religieuse.

Le beau de l'autre revèle l'autre en moi. 


\section{visite Cathédrale}
Respect des cultures
\begin{itemize}
    \item Saint françois 
   \item Marrakech et les prisonniers
   \item  les vitraux non figuratifs
   \item style art brut et influence arabe
\end{itemize}





\section{Migrants au Maroc}

\mn{13/3/2023 Père Exelmans }

\paragraph{la frontière marocaine}
Les migrants passent par la Montagne.

\paragraph{Ceuta} une première barrière de 3m à crochet et une barrière de 5m.

\paragraph{Accueil à Casa} pas de jugement. Humaniser la migration. Pastorale des migrants (même si 95\% sont musulmans)

\paragraph{toutes les questions} ne pas juger; écouter ces jeunes, au contact de pourquoi. Se laisser transformer par cette rencontre.
\mn{le bon samaritain}

\paragraph{Autorisation implicite de le faire} mais discrètement. 

\paragraph{Culture de l'Islam} donner le vendredi un couscous. Alors que les catholiques sont plutôt au changement structurel.

\paragraph{les femmes invisibles} traite. 

\paragraph{Oujda}


\section{Visite de Casablanca}

\paragraph{Mosquée Hassan II}
visite de la mosquée, qui introduit moins à la théologie musulmane que l'anthropologie, la pratique réelle : 
\begin{quote}
\item importance de se laver, on ne choisit pas celui qui est à côté de moi
\item  chaussures respect mais aussi la possibilité     
\end{quote}

\paragraph{Respect du souverain} Je note de la part de mes interlocuteurs un respect du souverain : "sa majesté",...
Dans la mosquée, on trouve la généalogie, depuis Mahomet, Fatima, Hassan jusqu'à Hassan II;

\paragraph{visite de la grande synagogue de Casa} Accueil très chaleureux. Mohamed VI protecteur des juifs (cf attaque du \href{https://fr.wikipedia.org/wiki/Parti_de_la_justice_et_du_d%C3%A9veloppement_(Maroc)}{PJD}
  : \textit{parti de la justice et du développement }s’est plaint des accords de 2000. 
\paragraph{les juifs au Maroc}  
2000-  juifs environ au Maroc
30 synagogues. 

\paragraph{Grande synagogue de Casa}
Phrases de la Mishnah à traduire. Touché par la définition de la richesse, celui qui sait se contenter de ce qu'il a.

\begin{quote}
    On a enlevé le 11e me commandements de remplir sa feuille d’impôt
\end{quote}

\section{Islam au Maroc}

\paragraph{Sunnite}

\paragraph{Islam contextualisé}

\paragraph{Islam territorialisé / patrimonialisé} Maric fait le chioix de se contruire à un islam national territorialisé qui se soustrait à l'Orient (Majrer) et se met à l'abri de la matrice salafiste / salafo wahhabite.


\paragraph{tryptique} rite malikite, théologie asharite, soufisme sunnite. Cette construction idéologique, confortée par la formalisation très tôt.
Tryptique contesté.

\paragraph{Islam au Maroc} analyse synchronique.

\paragraph{Berbère - Amazigh} demande de \textit{sa Majesté le roi} de parler d'\href{https://fr.wikipedia.org/wiki/Berb%C3%A8res}{Amazigh}. 

\paragraph{Réforme du champ religieux}
En 2003, attentats.  Contrôle de l'Etat (verrouillage politique - le Roi) et paradoxalement dynamique (dévérouillage théologique). 

\paragraph{Histoire de l'Islam - les Amazighen} \mn{Islam Berbère, Louis Garder} bien que largement arabisée, la population du Maroc appartient dans sa très grande majorité à l'ethnie berbère, les Imazighen (\textit{l'homme libre}). Depuis 20 000 ans.  Ces Groupes étaient chrétiens, juifs ou animistes (à tendance schismatique ). Le christianisme a été supplanté par l'islam à partir de la fin du VIIè siècle, jusqu'à sa disparition au XII avec les fanatiques almohades.

\paragraph{Apostasie des berbères} D'après l'historien Ibn Kuldoum, 12 apostasies. Le Maroc a échapé au crontrôle califales, avec des petits états amazighs indépendants. des Kharéjites. La dynastie Berghouata, une forme d'islam berbère, qui utilisait les idées et les croyances de l'islam de l'Est avec les traditions locales.  Si la révolte kharijite des ancêtres.

\paragraph{les Idrissides et Fatimides} au 8 et 9è, affirmant descendre du Prophète, ont régné au Maroc.
Islam Chiite va marquer le Maroc \mn{cf les prénoms chiites Ali, Fatima, Zohar, Zainab, El Hassan, El Houcine et Idrisse mais aussi la \textit{célébration d'Achoura}.}

\paragraph{double processus arabisation et islamisation} mais garde sa spécificité, islam pluriel. Aujourd'hui toujours (50\%) parlent Amazighs. Arabisé et islamisé au VIle siècle, le Maghreb berbère a toujours su défendre ses particularités.
\begin{itemize}
    \item  	Distinguer l'islamisation de l'arabisation, la première se fit à un rythme bien plus rapide que la seconde. La Berbérie devient musulmane en moins de deux siècles (Vile - Ville siècles), alors qu'elle n'est pas encore aujourd'hui entièrement arabisée,
  \item 	La plupart des tribus berbères ont adopté l'islam, mais ont conservé leurs lois coutumières
\end{itemize}

L'islam Marocain traditionnel est caractérisé par un syncrétisme entre l'islam et les traditions berbères locales issues de l'ancienne religion (la tanewya).

tanewaya : Animisme fonctionnel et monophysiste.

\paragraph{La parenthèse schismatique Boughata}  2-3è siècle de schismatique : un Coran. Mais impensable.

\paragraph{Retour au sunnisme avec les Almohavide} Revenu au sunnisme après une parenthèse schismatique, il s'attache (avec dynastie amazighe des almoravides) au rite malékite où se mêlent traditions locales et rigorisme dogmatique.


\paragraph{les Almohades} Mais Une autre dynastie berbère, issue du Rif, au nord du Maroc, celle des Almohades (I|47-1269),
ou \textit{Mouwahidoune}, les « Unitaristes ».


\paragraph{Ibn Toumart} \label{theo:IbnToumart1}  Ibn Toumart , mahdi « chef suprême » :refaire sans concession l'unité de la oumma (communauté des musulmans ).
Aux cinq piliers de l'islam , il en ajouta un sixième (pas canonique) , le \textit{jihad} (guerre sainte) : la livrer impérativement aux mauvais musulmans avant même d'attaquer les infidèles.
 II a renversé les Almoravides, combattu l'appartenance du Maghreb au rite malékite.
 Mais confirmant la thèse de Louis Gardet, Henri Laoust constate :\begin{quote}
     « Action missionnaire et action coercitive ne réussiront jamais à venir pleinement à bout des forces vives de l'islam maghrébin: l'attachement à l'école de l'imam Malik et aux forces populaires du soufisme ».
 \end{quote} 

\paragraph{islam métissé et complexe}


\paragraph{dynasties marocaines} reconstitution d'une continuité
 Idrissides (789 - Xe siècle)
 Almoravides (1069 - [147)
 Almohades (1|47 - 1248)
 Mérinides 1248 - 1548)
Saadiens (1548 - 1660)
  Alaouites (1660 à nos jours)

%-----------------------------------
\subsection{Le projet de la nationalisation et territorialisation de l'islam}

\paragraph{Indépendance en 1956 : état nation} le protectorat français de 1912 à 1956.

\paragraph{Religion d'Etat}
   comme dans presque tous les pays musulmans, l'Islam est en fait au Maroc le fondement de la nation, de l'Etat et de la société.
 L'option politique:
projet
de « nationalisation » pour ne pas dire d'autochtonisation de l'islam marocain sur le malikisme pour le mettre à l'abri de la vague déferlante du salafisme international »l'école juridique (maddhâb) malékite, opposition à toutes les déviations doctrinales , qui cherchèrent à supplanter le sunnisme » Pour des raisons politiques, la monarchie a, dès l'indépendance, choisi de s'appuyer sur des forces sociales et culturelles conservatrices.




\mn{Salafisme wahhabisme ? y a t'il un autre salafisme ?}

\paragraph{Corpus normatif légal} Il y a un corpus et une pratique politique. 
Les modes d'insertion de la dimension normative de la religion musulmane dans le dispositif institutionnel du royaume du Maroc sont multiples.
Ils concernent aussi bien le corpus légal qu'une pratique politique référant à l'islam comme mode de légitimation 

Cette insertion intéresse principalement trois domaines : la définition du rôle et de la fonction des oulémas, la gestion et l'encadrement de l'usage des mosquées et la formation des techniciens du culte

\paragraph{Le nouveau règne (en place depuis 1999)} En 1999,  dès son intronisation Mohammed VI a marqué le coup en allant se recueillir sur les tombeaux de deux saints : Moulay Idris de Zerhoun et son fils enterrés à Fès, représentés comme les fondateurs de la nation marocaine.
 Le geste est symbolique et renvoie à une lecture traditionnelle du contenu et du profil de la religiosité marocaine.
 Il renvoie aussi à une sorte de prééminence du chérifisme (appartenance à la lignée du Prophète) considéré comme un particularisme très positif
spiritualité des saints, qui n'est pas opposé à l'orthodoxie. 

% ----------------------
\subsection{Constantes et référentiels religieux de l'islam marocain}

\paragraph{Sunnite, ashâarite, malikite, soufisme}
la doctrine des «Ahl-sunna wa al-jamãa»,  sunnite (Islam dit orthodoxe et majoritaire, fondé sur le Coran et la tradition) itjihad de la doctrine ashâarite, le rite malikite,  le soufisme sunnite,  Imarat Al-Mouminine aux conditions de l'Imamat

  un islam à la fois orthodoxe et adapté au contexte.
\paragraph{l'islam du juste milieu}
Un label théologico-politique , souvent construit selon des modalités autoréférentielles Maîtrise du sens du texte et du contexte. 
la Rabita Mohammadia des Ouléma   (texte officiel)
\begin{quote}
    Le juste milieu en islam est le plus haut degré de la compréhension des textes religieux 
\ldots
     la modération est basée, avant tout, sur une maîtrise du sens du texte et du contexte.
  Une implémentation d'une loi ou d'une législation religieuse doit être élaborée en fonction de l'époque dans laquelle on vit afin d'éviter les extrémismes de tous bords
\end{quote}
 




\paragraph{Sécularisation} Etude sociologique montre un processus de sécularisation, post islamique. 

\paragraph{Ijtihad} effort pour répondre aux questions nouvelles

\paragraph{Imarat Al Mouminine} Le Roi est le \textit{commandeur des Croyants}



\paragraph{paradoxe du juste milieu} 
\begin{Def}[al wasatiyya]
    Islam du \textit{juste milieu} (auto-référentiel) : orthodoxe et adapté au contexte
\end{Def}
\paragraph{la tolérance a fait son temps}
Discours du trône, 2003.
\begin{quote}
    
 « Depuis quatorze siècles en effet, les Marocains ont choisi d'adopter l'Islam parce que, religion du juste milieu, il repose sur la tolérance, honore la dignité de l'homme, prône la coexistence et récuse l'agression, l'extrémisme et la quête du pouvoir par le biais de la religion.
C'est à la lumière de ces enseignements que nos ancêtres ont édifié une civilisation islamique et un État indépendant du Califat du Machrek (Orient ), se distinguant par son attachement à la commanderie unique des croyants, par son ouverture en matière de culte et par l'exclusivité du rite malékite ».
\end{quote}

\subsection{le Rôle du Roi}
\paragraph{Commanderie des croyants} des leurres que la réalité. Source de dynamique que de blocage. \textit{en soutenant l'islamité, permet de maximiser les réformes}

\paragraph{pas de réalité divine} mais de sa dignité de \textit{sharîf}. 


L'article 3 de la constitution marocaine stipule que: «L'Islam est la religion de l'Etat, qui garantit à tous le libre exercice des cultes».
\begin{quote}
     art. 41 de la Constitution de 201| : « le roi est constitutionnellement « Commandeur des Croyants » Amir Al-Mouminine   la nouvelle Constitution :Chef de l'État, son Représentant suprême, Symbole de l'unité de la Nation, Garant de la pérennité et de la continuité de l'État et Arbitre suprême entre ses institutions »


 
\end{quote}
 Le sultan n'est pas un « religieux » au sens des oulémas, ni un saint, il et la monarchie qui se définit autour de n'a donc pas qualité divine,sa personne n'est pas une monarchie « de droit divin ».
  Il tient sa qualité religieuse de sa généalogie, qui explicite sa parenté avec la personne du Prophète.
  Le sultan est en effet un sharif, c'est-à-dire un descendant du Prophète (ce qui constitue une parade à un danger chiite éventuel)

\paragraph{légitimité}

 l'islam est une source essentielle de légitimation de la monarchie.
  Le régime mobilise la religion, depuis les années 1970, pour contrecarrer certains adversaires politiques, qu'ils soient du gauche ou islamistes.
  Le contrôle du champ religieux apparaît par conséquent comme une nécessité pour ces régimes en quête de stabilité et de légitimation   Les modes d'insertion de la dimension normative de la religion musulmane dans le dispositif institutionnel le corpus légal et pratique politique référant à l'Islam comme mode de légitimation



%-----------------------------------
\subsection{Rite malekite}

\paragraph{Histoire}

 Malgré son rigorisme originel, le malékisme a dû s'insinuer dans un espace anthropologique où de multiples coutumes préexistaient (en matière familiale et en droit de la propriété notamment), y composer avec le complexe maillage confrérique qui s'est développé quasi concomitamment à l'arrivée de l'islam au Maroc.
  Cette école juridique islamique a donc été dès les débuts de son exportation et jusqu'à nos jours l'objet de réélaboration.



\paragraph{pas n'importe lequel}
Structure anthropologique précédente, avec la femme très forte.

\paragraph{maraboutisme et soufisme} considéré comme peu orthodoxe. 


\paragraph{Caractéristiques}

\begin{itemize}
    \item très forte demande de s'adapter
    \item caractéristiques
    \begin{itemize}
        \item   malikisme: Rigorisme, mais adaptation du à la demande sociale et politique:
       \item maslaha :recherche de l'intérêt général de la collectivité.
       \item La protection de l'intérêt public (al-masälih al-
mursala)
     \item  istihsan (choix de la solution la plus « soft »)   La prise en considération des divergences
  Privilégie:
       \item I-les finalités de la Loi islamique révélée (al magasid) 
  sur les sources du droit islamique (al usan.
       \item 2- la coutume locale 
  (amal) sur la doctrine , légale islamique (figh),
    \end{itemize}

\end{itemize}

\begin{Def}[maslaha - malekisme]
    Recherche de l'intérêt général de la collectivité.
\end{Def}

mais surtout il privilégie les \textsc{les finalités} de la Loi islamique révélée (\textit{al maqâsid}) sur les sources du droit islamique.


\paragraph{Al Shatibi} On est en train de ressusciter sa pensée après une période d'oubli.

\begin{Ex}[suppression de la polygamie]
    validation théologique mais encore plus importante validation populaire
\end{Ex}

  Assertions dans discours du roi cette doctrine tient compte des « desseins et finalités des préceptes de l'islam, et aussi par son ouverture sur la réalité ».
  Le souverain renvoie à la notion de magasid. ( XIv siècle
Al Shatibi) : dans certains cas, une application stricte de disposition de la charia pourrait se retourner contre par le biais du magasid,

% ----------------------
\subsection{Asharisme}

% ----------------------
\subsection{Soufisme}

\begin{Synthesis}
    le soufisme sunnite :une mystique/spiritualité qui modère la foi
\end{Synthesis}

\paragraph{Présence ancestrale}
 Un soufisme (voie intérieure/contemplation) qui prend ses racines initiales dans l'orthodoxie sunnite.
  Au Maroc, le soufisme a quasiment toujours existé notamment à travers l'école de l'Egyptien al Junayd   Vers les Xle et Xlle siècles, des turug se structurent.
 Le soufisme, via ces confréries et leurs Zaouïas (lieux saints), imprégna l'ensemble de la population marocaine de ses valeurs spirituelles 
 
 \paragraph{chorfas, descendants du Prophète} L'essor de ces turugs est également intimement lié à l'image très positive des \href{https://fr.wikipedia.org/wiki/Ch%C3%A9rif}{chorfas} aux yeux des Marocains (idrissides).
 modéré forte dimenion spirituelle, tourné vers l'amour de Dieu.
de la wassatiyya (le juste
marocaine rimant avec tolérance)

\paragraph{dimension politique}
Confréries soufies ou
Zaouias sont
protégées par le Palais, qui s'appuie sa légitimité et en faire sur elles pour salafisme wahhabite (généralement considérés comme une menace, soit que  pour contre la  véritable « religion marocaine »). 


« voie marocaine du soufisme » conservatrice sans être rigoriste ni radicale

\mn{conflit entre les marocains et les algériens se renforce}
\paragraph{culte des saints - Sidi}

\paragraph{Objet politique} indispensable pour éviter le légalisme et radicalisme par un aspect spirituel.
Deux branches soufis : 
\begin{itemize}
    \item tijaniya
    \item Quadiriyya Boutchichiyya
\end{itemize}


\paragraph{instrumentalisation des soufis}

Il y a près de six ans, dans une lettre lue lors de l'ouverture du troisième forum international des disciples de la Tariga Tijaniya à Fès, le roi Mohammed VI a loué le rôle du soufisme dans la diffusion de la sécurité spirituelle et des valeurs d'amour et d'harmonie afin de « barrer la route aux chantres du radicalisme, du terrorisme, de la dissension, démembrement et des du doctrines mystifiantes ».
Evolution de l'Islam confrérique: (soufisme)   Rénovation de la confrérie soufie Boutchichiya\sn{La Qadiriyya Boutchichiyya a joué un röle incroyablement important dans la déradicalisation de la jeunesse et de l'augmentation de la tolérance au sein de l'Islam marocain.} .
Il lui a assuré une présence et une visibilité médiatique aux côtés de la monarchie.


\paragraph{On n'est plus dans les soufis de stricte observance} renouvellement théologique, soufisme moderne. 
\begin{Ex}
    Théologie de la prospérité. Voir les capitalistes au Maroc.
\end{Ex}


Renouvellement
  le soufisme continue d'attirer les jeunes. La principale raison de cette attirance est que le progrès et le changement infiltrent la spiritualité soufie marocaine " Boutchichiyya est présenté comme étant contemporain dans une recherche de négation du stéréotype du soufisme, archaïque ou superstitieux, et a a réussi à recruter dans toutes les catégories démographiques.
  Elle a su s'adapter à l'esprit du temps, recrutant de nombreux 
  intellectuels, universitaires, bourgeois


\subsection{un Islam populaire}



\paragraph{De facto, c'est un Islam qui fait partie de la particularité du Maroc}


l'instar de maints pays musulmans, la pratique soufie au Maroc est mêlée au maraboutisme.
  culte des saints: un moyen d'adaptation de l'islam aux particularismes locaux.
age forme originale de piété populaire
Yès ancrée, et cela malgré les
Contestations de certains oulémas

» Pèlerinage, les festivals (les moussems) l'honneur d'un saint   Le culte maraboutique revêt deux formes : le culte du saint vivant et le culte du saint défunt.
  Les mouvements soufis vont également se développer à travers des mouvements mystico-religieux à l'ancrage social complexe comme les Aïssawa à Meknès ou les Hamadcha à Fès ou les Gnaouas à

% -------------------------------------
\subsection{Evolutions}

\paragraph{Attentat}
 Le tournant du Il septembre et surtout les attentats qui ont frappé Casablanca en mai 2003 ont transféré la religion de l'espace de légitimité à l'espace de sécurité.
  Cette nouvelle donne a rendu obligatoire une politique religieuse volontariste.
 
 \paragraph{indépendance religieuse} Désormais, le Maroc, à travers son Roi, affiche son particularisme et restaure son indépendance religieuse en confirmant son attachement au rite malikite dans une perspective de distanciation avec le rite hanbalite, plus rigoriste et proche du wahhabisme
 \begin{quote}
      Discours du roi : Les Marocains, en effet, sont restés attachés aux règles du rite malékite qui se caractérise par une \textbf{souplesse} [nous soulignons] lui permettant de prendre en compte les desseins et les finalités des préceptes de l'islam, et aussi par son ouverture sur la réalité.
      Ils se sont employés à l'enrichir par l'effort imaginatif de l'\textit{ijtihad}, faisant de la sorte la démonstration que la modération allait de pair avec l'essence même de la personnalité marocaine qui est en perpétuelle interaction avec les cultures et les civilisations
 \end{quote}

\paragraph{l'« essence de la personnalité marocaine »}
 ce n'est pas tant pour la réifier dans une ontologie doctrinale et ethnoculturelle spécifique que pour affirmer que celle-ci se trouve être \textit{« en perpétuelle interaction avec les cultures et les civilisations ».}

 \paragraph{La condamnation d'importation de « rites cultuels étrangers »}
   Elle ne saurait ainsi se concevoir comme une crispation identitaire mais semble la délicate équation diplomatique marocaine qui consiste à contenir l'influence du « rite étranger » hanbalo-wahhabite tout en se gardant de froisser le riche partenaire , saoudien, monarchie sunnite à l'instar du Maroc.

  l'approche sécuritaire ne suffirait pas à elle seule , il fallait porter le combat sur le terrain religieux.
 Le roi , s'appuyant sur la légitimité de son autorité spirituelle il est le Commandeur des croyants - et de descendant du Prophète, il engagea une profonde réforme des affaires religieuses.

\paragraph{I-Fondement institutionnel}

  La restructuration de la chose religieuse passe en premier lieu par la restructuration du ministère des Habous et des affaires islamiques.
  \paragraph{II- Deuxième fondement : l'encadrement}
   Le pilier institutionnel a été renforcé par l'encadrement, deuxième fondement de la stratégie de restructuration du champ religieux. Première mesure prise dans ce cadre, la nomination des membres des Conseils des Oulémas,
     \paragraph{III-
L'éducation islamique}
   troisième
fondement
 le royaume chérifien : renouveler le champ religieux en vue de prémunir le Maroc contre les velléités d'extrémisme et de terrorisme, et de préserver son identité qui porte le sceau de la pondération, la modération et la tolérance»

 \paragraph{Affirmation du triptyque marocain (malikisme-acharisme-soufisme)}
 piliers de la wasatyyia , un aggiornamento visant l'immunisation du champ religieux contre toute dérive et pour renforcer l'unité doctrinale

Réorienter l'éducation religieuse pour concurrencer le hanbalo-wahhabisme décrit comme étranger aux traditions religieuses du Maroc   visant ce qu'il nomme le \textit{« courant réductionniste littéraliste »} tout autant que les \textit{« jihadistes takfiristes »}. ce qui expliquerait l'« hostilité au soufisme » La consolidation de la sécurité spirituelle des citoyens



\paragraph{il y avait un salafisme quiétiste} 

\begin{Def}[Sécurité spirituelle]
    concept marocain contre l'islam wahhabisme (qu'on appelle radical \mn{du fait des liens avec l'Arabie Saoudite})
\end{Def}

\paragraph{discours de sa Majesté}



\paragraph{Wahhabisme} extra-exclusiviste (\textit{takfir})




\subsection{Stratégie de contrôle de la religion pour une liberté théologique}



\begin{Ex}
    Est ce qu'on souhaite Noël à un Chrétien ? Clair au Maroc. rapport d'affection. 
\end{Ex}



\paragraph{impact sociologique} avant 2003, on disait des choses dans les mosquées (propos haineux contre chrétiens et juifs possibles). Le fait de ne plus le dire change la vision du monde 


\mn{question par rapport à l'efficacité de la formation avec Internet}

 \paragraph{seul l'Etat peut faire des fatwas}




 \paragraph{Question des minorités religieuses}



 Face à la montée de l'intégrisme et du fondamentalisme, La gestion des mosquées est soumise à un contrôle étroit, contrôle des prêches   la mosquée retrouve sa vocation première: un espace d'apprentissage, de mémorisation et de déclamation du Coran un Conseil supérieur des oulémas, seul habilité à prononcer des fatwas, fut institué:
d'éviter les
interprétations inappropriées ou malintentionnées des textes sacrés.
  les imams furent repris en main et leur formation verrouillée.
  mais aussi restauration des synagogues et des cimetières juifs

\paragraph{formation}
L'Institut Mohammed-VI pour la formation des imams, morchidines et morchidates \sn{féminin !} 
 accueille chaque année des étudiants d'Europe et d'Afrique pour qu'ils s'imprègnent de « l'islam à la marocaine ». Cette formation est destinée à les « immuniser » - ainsi que leurs futurs fidèles - contre l'extrémisme » et de  défendre la pérennité de l'islam malélite en face du développement hégémonique de l'islam wahhabite dans le monde (une vision salafiste et totalitaire de la religion).
 \begin{itemize}
     \item La question de la maslaha, la recherche de l'intérêt général, est fondamentale dans la pensée malékite.
     \item  Le soufisme vient aussi gommer la rigidité d'une approche strictement légaliste de l'islam.
     

\item Enseigner les prescriptions de la Charia et préceptes de la religion islamique dans le respect des principes du juste milieu et de la modération y renouveau du patrimoine culturel, soufi et spirituel marocain 
\item la recherche sur la doctrine achaârite, le \textit{fiqh} malikite et le soufisme sunnite,
\item Analyse (déconstruction) du discours extrémiste violent dans ses différentes déclinaisons.
\item Inventaire des concepts utilisés et à partir desquels ce une lecture déformée instrumentalisée, tels que le jihad, l'ezcommunication, la Oumma, la chariaa, la gouvernance divine (Hakimirra),
 \end{itemize}
  






\paragraph{Oulemas} Les ouléma ont historiquement eu une importante politique significative au Maroc bien que jamais réellement organisés en tant que corps.
  Au fil des années, le rôle des oulémas en tant que contre-pouvoir s'est réduit de plus en plus.
Le conseil des Oulémas est une conséquence directe de leur rôle historique. 
Discours de sa majesté appelle les oulémas à adhérer à un pacte scellant leur loyauté et définissant le contenu d'un islam modéré qu'ils sont appelés à défendre et à promouvoir.

  \paragraph{post attentat} Suite aux attentats perpétrés à Paris et qui ont coûté la vie à plusieurs innocents sous le prétexte du Jihad au nom de Dieu.
  Le Conseil Supérieur des Ouléma émet une Fatwa sur la question du 
  Jihad en Islam
Le Coran interdit : \begin{quote}
    Celui qui tuerait un homme non coupable d'un meurtre ou un délit sur la terre, c'est comme s'il avait tué tous les hommes.
\end{quote}

\paragraph{Jihad} Effort sur soi , plusieurs formes (Ecriture...)
  Même dans ce cas de figure, poursuit la Fatwa du Conseil, la proclamation du jihad relève du ressort exclusif du Grand Imam à qui l'Islam a donné le droit exclusif de le proclamer. L'islam ne permet, par conséquent, à aucun individu ou groupe de proclamer le Jihad de leur propre chef.

\paragraph{Condition féminine}
 un lieu de transformations religieuses, un islam progressiste ?
y le rôle de la femme dans la mise à niveau du champ religieux.
  Les femmes sont formées pour être «mourchidates» (guides), mais elles ne peuvent pas encore diriger la prière.
  Il ne s'agit pas seulement d'un travail d'éducation à l'islam, qui comprend l'explication du droit de la famille et de certaines sourates du Coran, mais aussi d'une guidance socio-religieuse   L'institution de femmes ouléma et de mourchidates a eu un impact exponentiel sur la résorption de obscurantisme dans les milieux religieux feminins.

  2018 :le Conseil supérieur des Oulémas a exprimé un avis favorable à la réforme qui, en rupture avec une pratique pluriséculaire, allait permettre aux femmes d'exercer le métier de \textit{adûl}, l'auxiliaire de justice chargé de rédiger les actes légaux dans le domaine chariatique.
  La décision d'ouvrir cette fonction aux femmes, quasi-inédite dans le monde musulman, a été prise fin janvier par le roi Mohammed VI, "commandeur des croyants » 
  
  
  L'arrivée de femmes notaires en droit musulman a provoqué des remous dans la sphère salafiste contraire à la charia.

\paragraph{Droit de la famille}
  En 2004, la réforme de la Moudawana (code du droit de la famille 
  marocain)
 Ce code définit notamment les normes concernant le mariage, le divorce, la filiation ou le droit de succession.
La libération des femmes d'un pouvoir autoritaire qui, au Maroc, est à la fois patriarcale et théologique   Les hauts dignitaires religieux a dû prendre acte nolens volens, en est la preuve : le malikisme marocain n'empêche pas une forme de pluralisme normatif de se développer.
  abolit la règle de la vilaya dans la conclusion du mariage pour la femme majeure (la tutelle d'un membre mále de sa famille)  
  
  \paragraph{polygamie} La polygamie n'a pas été supprimée, mais assortie de restrictions sévères, soumise à l'autorisation du Juge et à des conditions légales draconiennes, en laissant du seul ressort du juge le soin de s'assurer de l'inexistence de présomption d'iniquité.
  
 Le texte confère en outre à la femme le droit de conditionner son mariage par l'engagement du mari à ne pas prendre d'autres épouses. Il est, en plus, fait obligation au mari, qui veut prendre une »deuxième épouse, d'aviser sa première femme de ses

\subsection{Des réformes à venir}

\paragraph{code de la famille}
Le Palais annonce de nouvelles réformes du code de la famille, appelé à instaurer davantage d'égalité entre les hommes et les femme
\begin{quote}
    « A cet égard\sn{2022 Med VI}, Nous nous attachons à ce que cet élan réformateur soit mené en parfaite concordance avec les desseins ultimes de la Loi islamique (Charia) et les spécificités de la société marocaine.
    [\ldots]
    Nous veillons aussi à ce qu'il soit empreint de modération, d'ouverture d'esprit dans l'interprétation des textes, de volonté de concertation et de dialogue, et qu'il puisse compter sur le concours de l'ensemble des institutions et des acteurs concernés »
\end{quote}
\begin{quote}
    « Je ne peux, en ma qualité de Commandeurs de croyants, autoriser ce que Dieu a prohibé, ni interdire ce que le Très-Haut a autorisé ».
\end{quote}
 
\paragraph{L'égalité dans l'héritage, une revendication sensible}

 Le code de la famille ira-t-il jusqu'à instaurer l'égalité dans l'héritage entre les hommes et les femmes  ? Non.
 La loi est ouverte à tout le reste, tout ce qui ne fait pas l'objet de 
textes sacrés précis, explicites. Ce qui ouvre la possibilité de 
 réviser le \textit{Taásib} (héritage par agnation) \mn{Le \textit{Taásib} \label{ref:Taasib}  permet aux oncles et / ou 
  aux cousins de partager l'héritage d'une ou plusieurs filles (sans 
  frère) à la mort de leur père,} une tradition qui est souvent 
  considérée comme injuste. Le taassib fait quasiment l'unanimité contre lui aujourd'hui. Il fera sans doute partie des sujets qui seront discutés en premier.

 Ces réformes juridiques furent l'occasion de débats socio-politiques intenses, l'affrontement traditionalistes et les modernistes entre les Réforme
théologique:

\begin{itemize}
    \item   I-l'argument de l'immuabilité du Coran (un texte inaltérable car descendant du droit divin) pour défendre une exegese ("ijtihad") très orthodoxe, et justifier ainsi leur résistance au changement des lois libertés individuelles et à l'égalité des sexes).


\item II-ouverture de la tradition malékite, ijtihad des textes . la condition d'une lecture égalitariste contextualisée des textes de l'islam, et
) Pour eux, la Moudawana est une création de l'homme, inspirée par la Charia bien entendu, mais ouverte à l'interprétation et la révision.
\item 
Féminisme islamique

\end{itemize}

\paragraph{Jouer dans l'espace permis par l'ijtihad}
L'État a toujours essayé de trouver un équilibre entre, d'un côté, la nécessité de ne pas heurter la légitimité religieuse que le commandeur des croyants doit préserver, et de l'autre, l'ouverture à la modernité(conventions internationales sur l'égalité entre l'homme et la femme).

\paragraph{Féminisme islmaique d'Etat}Pour pouvoir satisfaire simultanément les revendications des féministes et celles des islamistes, le pouvoir a été amené à produire de toute urgence un « féminisme islamique d'État ».
Les idées de justice sociale et d'égalité des genres avancées par le féminisme islamique sur la base d'une lecture renouvelée du Coran
\paragraph{les limites de l'aggiornamento : Asma Lamrabet}
l'affaire qui a impliqué en mars 2018 Asma Lamrabet le revele tres clairement. Médecin biologiste de profession, activiste par vocation, Lamrabet a publié en 2017  « Islam et femmes : les questions qui fâchent » \sn{ éditions En toute lettres, Casablanca }  où elle couvre : polygamie, inégalité dans le droit successoral, tutelle masculine, port obligatoire du voile, etc.
L'ouvrage n'accuse pas l'islam en soi, mais bien l'élaboration théologico-juridique qui a véhiculé au cours des siècles une vision « misogyne » et « patriarcale » de la religion, et qui aujourd'hui contraint les femmes au rôle de « dernières gardiennes du temple de la tradition.

  La relecture du Coran souhaitée et promue par cette 
  doctoresse-théologienne avait reçu
une
reconnaissance officielle dès 2011, quand Lamrabet est devenue directrice du Centre d'Études et de Recherches sur les questions féminines au sein de la Rábita.

 Mais la coexistence avec les oulémas de cette Institution n'a pas été facile; elle est devenue impossible quand Lamrabet est intervenue à la présentation d'un livre sur le problème de la disparité de genre en matière d'héritage, et qu'elle a souscrit, avec une centaine d'autres personnalités marocaines, une pétition qui invitait à dépasser le régime successoral prévu par la jurisprudence islamique.

  Les salafistes se sont insurgés, l'accusant de « déviance » et « ignorance » ; l'aile la plus conservatrice des oulémas marocains a réagi elle aussi avec irritation.
  Accablée d'insultes et de menaces sur les réseaux sociaux, et soumise probablement à de fortes pressions, Lamrabet a présenté sa démission de la Râbita, démission que ses supérieurs ont rapidement acceptée.
  \paragraph{L'internationalisation de l'offre religieuse et la fragmentation de la 
  demande religieuse}
  \begin{itemize}
      \item l'étranger et l'ouverture du paysage médiatique vont amener les Marocains à s'affranchir de plus en plus des modes traditionnels de socialisation religieuse.
      \item La disponibilité d'une offre religieuse diversifiée et non totalement contrôlée par l'État va amener les Marocains à composer leur propre « menu » religieux et à s'autoriser des syncrétismes d'abord au sein même de la religion musulmane (dans le sens d'un rapprochement avec le chiisme) et probablement avec les autres religions dans le cadre d'un processus sécularisé situant la pratique spirituelle dans la sphère privée.
  \end{itemize}


\paragraph{Apostasie}
 En avril 2013, une fatwa du Conseil supérieur des Oulémas affirmant que le musulman que l' apostasie mérite la peine de mort. 
 Une fatwa avait toutefois semé le trouble dans le pays. Sur la base d'arguments classiques tirés de la tradition Islamique, elle qualifiait d' « apostat » tout citoyen marocain musulman désireux de changer de religion - en cas de non réponse à l'injonction de repentir que lui adresseraient les autorités - le rendait passible de la peine capitale.

\paragraph{une évolution sur l'apostasie}
  Dans un document intitulé \textit{Sabil al-Oulémas} (La voie des savants), le Conseil supérieur des oulémas du Maroc a modifié sa position sur l'apostasie : revient sur sa fatwa de 2016 (peine de mort) 
  
  Pour justifier ce revirement, les savants marocains ont redéfini l'apostasie non pas comme une question religieuse, mais comme un sujet politique plus proche de la 'haute trahison'», ils distinguent « apostasie politique », passible de la peine capitale, et « apostasie intellectuelle », qui relèverait de la liberté individuelle.
Les apostats ne sont plus passibles de la peine de mort.

 Mais ce n'est pas un document officiel et il ne s'agit que de l'avis de cinq oulémas du Conseil supérieur » La décision des oulémas marocains représente une position partagée par beaucoup de penseurs musulmans réformistes, mais c'est la première fois qu'elle est assumée de manière aussi explicite par une institution religieuse officielle
  le délit d'apostasie n'existe pas en droit marocain.
Seul existe un délit d'« ébranlement de la foi d'un musulman » qui concerne davantage les actions de prosélytisme susceptibles de provoquer des troubles dans l'espace public (l'article 220 du Code pénal).
\paragraph{Conférence de Marrakech
2016}
conférence internationale sur les minorités religieuses.
Thème \textit{"Les droits des minorités
religieuses en terre d'Islam: Le cadre juridique et l'appel à l'action"}
\begin{quote}
    Message du roi Mohammed
VI aux participants
:"Rien ne nous paraît justifier, au royaume du Maroc, que des minorités religieuses soient privées de l'un quelconque de leurs droits 

Nous n'acceptons pas que ce déni de droit soit commis au nom de l'islam, ou à l'encontre d'un musulman, quel qu'il soit.
Nous nous chargeons de préserver les droits des musulmans non-musulmans distinction entre eux. Nous les protégeons aussi en tant que citoyens en vertu de la Constitution »
\end{quote}
 

\paragraph{un événement, le congrès des minorités
2017}  Congrès inédit organisé à Rabat les représentants des minorités religieuses ont appelé à une clarification des textes juridiques sur la liberté de culte au Maroc, pays où l'islam est religion d'Etat.
Son but ultime est que la Constitution marocaine reconnaisse la liberté de conscience.

Le roi avait semblé faire un pas vers les minorités religieuses marocaines en novembre 2016, en déclarant à la presse à Madagascar qu'il était « Commandeur des croyants, des croyants de toutes les religions ». 

\subsection{Paradoxe entre une visibilité accrue de l'Islam et une sécularisation}
\paragraph{Post-Islamisme} Augmentation des signes religieux des jeunes sans projet politique

\paragraph{Islamisme de-islamisé}

\paragraph{sharia dynamique} concept iranien


\paragraph{Stratégie d'inclusion ou banalisation} vis à vis du PJD, stratégie d'inclusion. Banalisation vis à vis du salafisme. Printemps arabe (les jeunes n'ont pas mis en avant les religieux). Même si ce sont les mouvements islamiques qui en ont profité (113 sièges). Ont été délégitimé par le fait de ne pas interdire l'alcool, accord avec Israel, forcer le voile pour les extrémistes, 

\begin{Synthesis}
    \begin{itemize}
        \item Le mot Laicité a une très mauvaise publicité, pensé contre la religion. Mais sociologiquement, accommodement avec une laicité sociale
        \item  Ramadan : pose la question de l'espace publique. Des changements.
        \item un verouillage religieux pour une liberté théologique
    \end{itemize}
\end{Synthesis}

\subsection{Bibliographie}
\begin{itemize}
    \item  Diffuser un « islam du juste milieu a / Les nouvelles ambitions de la diplomatie religieuse africaine du Maroc. Cédric Baylocq. Aziz Hiaoua dans Afrique contemporaine 2016/1 (n° 257), pages 113 à 120 
        \item  L'évolution du champ religeux marocain au défi de la mondialisation Mohamed Tory. Dans Revue internationale de politique comparée 2009/1 (Vol. 16), pages 63 à 01
)     \item  Le code de la famille marocaine en question et bilan et perspectives de réforme , Abdessamad Dialmy, in L'islamisme marocain entre révolution et intégration »,Archives de sciences sociales des religions, 1101 2000, 5.   \item   Travaux en cours dans l'islam Marocain 25/10/2018 Michele     \item   Religion et politique au Maroc aujourd'hui MALIKA ZEGHAL     \item  Espace public et croyances religieuses au Maroc,Hassan Rachik AYADI Mohammed, RAcHIK Hasan, Tozy Mohamed,     \item  L'Islam ou quotidien.
Enquéte sur les voleurs et les protiques religieuses au Maroc Casablanca, Prologues, collection  Religion et société, 2007, 272, Revue des mondes 

\end{itemize}



\section{Visite de Salé}

Notre Guide, Mohammed est passionnant. Il est responsable de retrouver l'aspect originel du lieu. Musulman convaincu (il porte la \href{https://fr.wikipedia.org/wiki/Zabiba}{zabiba}, la marque de la prière assidue).
\paragraph{Humilité dans l'Islam} Les côtes de la salle de prière : pas dessiné car la création c’est Dieu qui le fait se façon parfaite. On laisse des fautes exprès.

\paragraph{Marabout} veut dire Saint au Maroc. Ce n'est pas la définition subsaharienne du marabout : c'est aussi le monument autour.
\begin{quote}
    Le marabout pas islamique mais on fait quand même 
\end{quote} 
On constate moins de marabouts depuis 1820,  marque la partie économique 

\paragraph{cimetière musulman}
Cimetière normalement rien sauf une petite pierre pour marquer le lieu 

\paragraph{République de Salé} Un Emir. \mn{Roger coindreau les corsaires de sale}





 
\paragraph{Ministre des Habous} Al Mowafaqa : concordance, concordance avec la volonté de Dieu. Connaître et se reconnaître.
Humilité : Emotion de ce souvenir : 
\begin{quote}
Quand j’ai vu le Pape François, je lui ai demandé de prier pour moi.  Il m’a répondu : priez pour moi
\end{quote}

\paragraph{}

\section{Colloque Al-Muwafaqa sur les minorités  religieuses en Afrique}

\subsection{Introduction  - Minorités religieuses en Afrique méditerranéenne et subsaharienne, le défi la force et la grâce d’être minoritaires}
\mn{Père Daniel Nourissat, Anne-Marie Teeuwissen}
\paragraph{Des chrétiens des Eglises subsahariennes}
\paragraph{Assistants pastoraux} le réveil des Eglises du fait de cette venue
Apostolat des semblables par les semblables

\paragraph{Former ces assistants}
\paragraph{différences de formation entre protestants et catholiques} Vivre ensemble 10 jours sur la montagne ; estime réciproque a grandi. 
\paragraph{éviter la théologie hors sol} 

\paragraph{Crise climatique et afrique} Achille Mbe ( ?) : pluraliste épistémique ; animiste offre un pluralisme. Habiter le monde, c’est co-habiter, humain et non humain ; hospitalité fondamentale. 

\paragraph{Unversel itératif} intégrer les particularités. Se répétant au sein de contexte spécifique que l’universalité devient présente. Toujours  à réinventer. Venders ( ?)

\paragraph{Dar El Hadith El Hassania} Interconnecté : dès que l’un est touché, les autres le sont.
\mn{Bouchra Chakir}
Charte de Médine : 
\begin{quote}
Première charte du vivre ensemble
\end{quote}
Ethique universelle (Hans Kung) : éthique à revoir par rapport à l’actualité
Lévinas : « je suis responsable de tout »
Principe responsabilité (Kant). S’intéresser aux autres comme nous nous intéressons à nous-mêmes. C’est cela à l’Islam. 
Moral : fondements moraux des religions à retrouver.
Règle d’or « soit une fin et non un moyen »

\subsection{Minorité et minoritaire}

\paragraph{minorité} objectif vs minoritaire : perception. 

\paragraph{diversité des minorités} Apport des sciences humaines croisés avec la théologie. 

\subsection{Minorité dans la pensée contemporaine : vers un changement paradigmatique risqué et prometteur }
\mn{Père Fadi Daou, libanais - Français}
\paragraph{concept de minorités} différentes perspectives. Pensée engagée.

\begin{itemize}
\item linguistique
\item international
\item géopolitique
\item constitution irakienne de 2005
\item Question en monde islamique
\item question en monde chrétien
\end{itemize}

\paragraph{Pleine citoyenneté} et arrêter de penser la minorité qui discrimine. 
Déclaration de la fraternité humaine : 
\begin{quote}
\end{quote}
Texte fort qui montre comment le terme de minorité est attaqué

\paragraph{minorité : grâce ou déformation de la foi} depuis le XXI, changement de paradigme : 

\paragraph{qui sont les minorités ?} deux facteurs mis en avant : identité détermine la minorité, et la discrimination. Force du local par rapport à la force de globalisation.

\paragraph{identité} concept ambivalent. On confond avec appartenance. Proposition que l’identité de chacun est unique mais elle est constituée de plusieurs appartenances. Conflit entre la narrativité de chacune de ces appartenances.  « je suis croyant, chrétien, maronite » : parfois majoritaire, parfois très minoritaire. 
Amin Malouf : identité meurtrière. Je ne suis pas d’accord. 
Importance du contexte pour comprendre l’identité. Parfois la minorité devient minorité, parce que une appartenance est problématique.
\begin{quote}
… défarorisé… souhaitent conserver leur nom
Minority …  Group
\end{quote}

\begin{itemize}
\item reconnaissance ; tension entre une reconnaissance et un cadre qui protège les minorités. 
\item posture
\end{itemize}

\paragraph{Cadre juridique international}

Un individu n’est jamais minoritaire mais au niveau socio-politique, il faut que le système permette que les appartenances particulières puissent d’exprimer ; sans risque social
\begin{Ex}
Les pays du Nord ne sont pas à l’aise avec cette proposition
\end{Ex}
Fukuyama\href{https://fr.wikipedia.org/wiki/La_Fin_de_l%27histoire_et_le_Dernier_Homme}{1992} pense à la filiation libérale  : \mn{1992 - la fin de l’histoire et le dernier homme } 
En 2018, il a changé et écrit « politique identitaire… ». Les gens sont en \textbf{tribus}. Les gens votent moins selon leur valeur mais selon leur identité, la menace majeure aujourd’hui pour la démocratie. 
\begin{Ex}
Brexit : le vote a été beaucoup plus en fonction des angoisses sur l’identité britannique qu’un problème de droit. 
\end{Ex}
\paragraph{Habermas}
Fukuyama a en parti raison mais ce n’est pas toute la raison de cette réalité. Le déclin du nationalisme (Habermas) est aussi une raison. Habermas fait le lien entre déclin du nationalisme et religieux\sn{Entre Naturalisme et Religion. Les Défis de la démocratie, Paris, Gallimard, coll. « NRF Essais », 2008, 380 p.}. 

Non pas une posture identitaire mais adopter la perspective des autres citoyens. Mais ce n’est pas toujours réel donc l’apparition de minorité. Il y a toujours des controverses ; mais si certains groupes n’ont pas le droit de parler de la controverse ou n’ont pas les moyens, il y a l’apparition des minorités.
On a une politique sans culture explique l’augmentation des extrêmes. 

\begin{Ex}[Irak]
2005 Constitution
\textit{Composantes} \textit{Almoqaiwat}. 
L’idée est de reconstruire l’Irak non pas sur le nationalisme, avec un double marqueur religieux (Sunnite et Chiite) et ethnique (Kurde sunnite). « composante ethno-religieuse ».
Beaucoup de penseurs ont apprécié cette dimension. L’avantage de composante est qu’elle n’est pas quantitative.
En 2012-14, vu comme positive. Mais aujourd’hui, les Yazidis ont demandé d’être appelés \textit{minorité}, pour avoir la protection de la loi internationale. Composante est positive si la société prend le terme au sérieux. Si on est marginalisé, il vaut mieux le terme minorité. 
\end{Ex}
\paragraph{déplacement du discours musulman}
Marrakech 2016 : « le droit des minorités dans les territoires musulmans ». est devenu le texte de Marrakech « le droit des minorités dans les sociétés majoritairement musulmanes ». Un déplacement très intéressante.
Très intéressant cette déclaration, à partir de la réinterprétation de la \textit{Ṣaḥīfa} ou \href{https://theses.hal.science/tel-02079610/document}{Charte de Médine} : \sn{ l’historicité de ce texte est discutée dans le \href{https://theses.hal.science/tel-02079610/document}{lien} . }

\begin{quote}
    27. Les juifs de Banū ‘Awf forment une umma avec les mu’minūn.148Que les juifs aient leur religion
« dīn »149 et que les muslimūn aient la leur, cela s’applique aussi bien à leurs clients qu’à eux-mêmes, à l’exception de celui qui aurait mal agi ou qui commettrait une transgression ; il n’attirera le mal que sur lui-même et sur sa famille.
\end{quote}

La \textsc{Déclaration de Marrakech}\sn{voir discussion d'E. Pisani dans le \href{https://journals.openedition.org/mideo/1655}{MIDEO}} de 2016 : 
\begin{quote}

« Citoyenneté, entière cohérence avec la théologie musulmane. Avec vous juifs,… nous formons une seule et même umma. » (texte de Médine)
\end{quote}
Et le roi de commenter : 
\begin{quote}
2 piliers : citoyenneté inclusive\mn{\href{https://www.right2city.org/fr/themes/citoyennete-inclusive/}{\textsc{Citoyenneté}} \textsc{inclusive}
Une ville/établissement humain  dans laquelle tous les habitants (qu'ils soient permanents ou temporaires) sont considérés comme des citoyens et sont traités sur un pied d'égalité.} ; liberté religieuse
\end{quote}

\paragraph{qu’est ce que l’umma} communauté religieuse ou bien la nation, la communauté politique ?

\paragraph{déclaration de Al-Azhar} Etat national constitutionnel (et non musulman) 2011. 
\paragraph{la liberté, complémentarité} Al-Azhar. 2017. Avec le pape François. Sur la minorité, source du texte d’Abu Dhabi. 

\begin{itemize}
\item de dhimmitude vers la citoyenneté inclusive
\end{itemize}
\begin{quote}
Il ne faut pas aller au Hadj quand on voit les gens reviennent du Hadj
\end{quote}
Comment accompagner ce mouvement musulman ? 

\paragraph{Chrétiens} la première posture, posture d’équilibre. Ce n’est pas anodin. 
Lettre à Diognete
\begin{quote}
Les chrétiens ne se distinguent pas par… leur coutume. Ils résident dans leur propre patrie comme des étrangers domiciliés…. Toute terre étrangère leur est une patrie… et leur patrie est une terre étrangère ».
\end{quote}
Texte très important, sur l’Eglise locale. Question de l’inculturation. Si on est une église de passage, pas d’inculturation, on n’est pas inséré. Karl Rahner : « Aujourd’hui, l’Eglise rentre dans son troisième âge historique.  Et le deuxième âge commençant en 70 ». Avant 70, Eglise biculturel, judéo-grecque.
L’Eglise locale est tout le contraire d’une Eglise d’exil, elle est inculturée. 

\paragraph{Eglise diaconale} deuxième posture. On ne peut pas être minoritaire si on n’est pas un peuple. Dynamique diaconale tournée vers l’extérieur

\paragraph{Engagement intégral} La posture identitaire (chrétiens en Syrie) peut être opportuniste.   A l’opposé, engagement intégral. Cf 1992 texte orient… ? 
\begin{quote}
La présence signifie que nous sommes au milieu du monde, un signe du monde, pour les autres, et non pas contre ou en marge de la société.
\end{quote}

Dans l’Eglise catholique, il y a des synodes du Moyen-Orient, 2012, Benoit XVI, aucune citation de \textit{minorité}. Pas être traité come citoyen mineur.
Le CERNA parle d’une d’une église ayant le droit de citer (citoyen ?).

\paragraph{conclusion}
Christian de Chergé : apostolat de la réponse. 
Pape / Al Azhar : citoyenneté inclusive qu’on arrive à être croyant, citoyen sans se sentir minoritaire.

\paragraph{le concept de narratif minoritaire}  déconstruction du narratif minoritaire car faire une narrativité négative est souvent une construction qui n’ouvre pas au dialogue et à l’échange \mn{voir les identités narratives d’extrême droite de l’ostracisation}
\paragraph{citoyenneté inclusive} Le fait minoritaire est l’échec de la diversité. 
\begin{Ex}
La non reconnaissance de ma diversité (ex : je dois travailler à Noel)  coute à la société ; reconnaitre la diversité.  
\end{Ex}

\begin{Synthesis}
Citation sur la fraternité humaine disant que le concept de la pleine citoyenneté et non le terme discriminatoire de minorité, terrain de la discorde. Déplacement des discours religieux, avec un effort islamique d’enraciner le citoyen dans la charte de Médine ; Côté catholique, plusieurs postures possibles, exil ou minorité; une critique des évêques d’Afrique.
Engagement intégral dans le champ religieux de la citoyenneté inclusive dans le champ politique
\end{Synthesis}
% ---------------------------------------------------------
\section{Minorité et construction identitaire dans le premier testament }
\mn{Jeudi, 1e section de l’après midi Jean Kougala, recteur de Al-Mowafaqa. Ancien testament.}
\paragraph{paradoxe dans le concept de minorité} Moyen d’expression de soi d’auto-identification. Assumer positivement la minorité. 

\paragraph{le mot de minorité n’est pas employé dans le nouveau testament} très anachronique dans le mot mais pas la réalité.
\paragraph{Premier testament : minorité comme mécanisme de défense et comme moyen d’auto-identification} « je ne suis pas n’importe qui » justement car « je suis petit ». 
Dt 26, 5-9, texte du paysan au prêtre en offrant la première récolte
\begin{quote}
Mon père était un Araméén errant.
\end{quote}
\paragraph{identité narrative} on le retrouvera dans les psaumes, prophètes… cf Ps 78
\begin{itemize}
\item minorité comme errance
\item 
\item minoritaire chez soi
\end{itemize}
\paragraph{Errance} une personne sans domicile fixe : fragilité, pauvreté,… mais il y a un sens positif : la mobilité. Une errance voulue, choisie. Hébreu : Ivri veut dire traverser. Etymologie programmatique, qu’on peut traduire par errance. Mot hebreu : abad, au participe présent : errant, se perdre (mourir, disparaitre). Identité de non crispation. Nous ne savons pas trop à qui se rattache ce père. 
\paragraph{Exil en Egypte} Vous vous considérez comme un étranger : auto-construction identitaire. 1R3,1 : salomon s’est marié avec la fille de Pharaon. « se faire gendre avec quelqu’un ». Salomon sous la coupe de Pharaon. 

\paragraph{Minorité comme domination} On est minoritaire mais on domine. Pour le deutéronomiste, la royauté est une erreur. Et l’autre notion est celle de \textsc{Reste}. Pour Amos, c’est négatif : les rescapés sont un signe du jugement de Dieu mais Isaïe voyait un signe de la grâce de Dieu. Pas d’unanimité. 

\paragraph{conclusion} l’expérience de la minorité est essentielle dans la Bible. Interprétation positive dans le premier testament : élection et Alliance. Je suis important parce que je suis minoritaire parmi les nations.

\paragraph{dans le NT} « quand je suis faible, c’est là que je suis fort ». 1 Co : clairement au début de l’expérience ecclesiale, ce sentiment de « reste » mais qui diminue au fur et à mesure de la formalisation de la communauté.
 

\begin{Synthesis}
Errance : programmé dans le nom même
Exil
Hébreux, minoritaire sur leur propre sol.
Cela fait une interprétation théologique d’une Interprétation positive de la minorité, petit reste.
\end{Synthesis}
% -----------------------------------------------------------------------------
\section{Juifs et Chrétiens sous les Almohades}
\mn{Fouad Ben Ahmed}
\paragraph{Dynastie Almohade}
Dynastie Almohade et la révocation du statut des dhimmis. Pas de statut légal pour les juifs et les chrétiens.  Est-ce que ce contexte a fait écho, à influencer en théologie ? je n’ai rien trouvé : Ibn Rush n’a pas parlé du statut de Dhimma mais a parlé des peuples du livre. Dans le détail métaphysique, vision chrétienne et la vision Pythagore (monde généré et corrompu). 
\paragraph{abolition de la dhimma}
Les chrétiens et juifs doivent quitter le territoire ou se convertir. Discrimination vis-à-vis des musulmans non almohades.
Annulation du Pacte de Dhimma.
Les juristes doivent traiter des musulmans esclaves des chrétiens. Ibn Rush :
« toutes ces questions sont inconnues car il n’y en a pas d’esclavage chez les chrétiens. »
Il connait la position des chrétiens sur l’esclavage.  
Pas de réponse univoque vis-à-vis des chrétiens
« du ciel du monde » : les deux religions ont deux visions du monde. 
\begin{quote}
-	Monde aristore ni généré ni corrompu
-	Monde généré platonicien
-	Monde Pythagore généré et corrompu. Chrétien. 
\end{quote}
Vision positive des chrétiens et juifs
Ibn Rush : 
Relation essence et attribut des asharites. Doivent admettre que les substances subsistent en lui-même. Pluralité des attributs divins : non, les chrétiens ont tort. 
Substantialité : pas la multiplicité en dehors de la pensée. 
L’incohérence de l’incoherence : 
Les chrétiens ne croient pas que les attributs ne soient p. 

Les trois , un en acte, trois en puissance

Ibn Rush moins intéressé par les juifs. 

\paragraph{cosmologie} il n’y a rien dans les textes coraniques que le monde n’était pas généré (Aristote). Démonstration philosophique d’un au dela (monde corruptible) mais le grand désaccord est sur le mode de cette resurrection : corps / âme… aucun texte là-dessus. Il y avait une interprétation latine que l’âme reste éternelle et non corruptible. Mais quand on lit les textes arabes, relativisation de la capacité relationnelle de chaque croyant. Même pensée que Al Ghazali sur ce point de la résurrection, spirituelle, imaginaire, corporelle. Pour les soufi, négation de la résurrection corporelle. Pour Ibn Rush, hésitation entre résurrection de l’âme ou corporelle.  
\paragraph{voltaire}
Voltaire sur les religions en Angleterre, dont le nombre permet d’éviter le despotisme

\paragraph{Conclusion} Discussions sur demande ; considérer l’essence / attributs des chrétiens ; 
Abrégé de la république de Platon. Une exception. Sinon, discussion théorique pure et pas sur le statut des juifs et chrétiens. 

 
\begin{Synthesis}
XIIe – Almohades. N’exerçaient pas slt un pouvoir mais une idéologie religieuse qui niait le droit à la différence religieuse et suppression de la dhimmi.
Ibn Rush ne fait pas écho à cette situation mais expose les points de vue islamiques, chrétiens et juifs.
Absence accord des musulmans sur la resurrection et corruptibilité du monde. 
\end{Synthesis}
% ----------------------------------------------------------------------------------------------
\section{Etre minoritaire et Catholique en Algérie}
\mn{ Oissila Saaidia }
\paragraph{nouvelle Eglise Française métropolitaine} et non Eglise d’Afrique, Tertulien, Augustin. 
\paragraph{une minorité dominante, les catholiques algériens 1830-1959} Algérien : catholique en Algérie, comme les Corréziens, par opposition aux indigènes d’Algérie . 
Une vie religieuse animée dans les grandes villes. Mais hors des villes, seules les grandes fêtes sont fêtées. Diocèse concordataire, proche de Rome. 

\paragraph{Création de pèlerinage Algérien : Augustin, Notre Dame de Santa Cruz d’Oran, d’Afrique…} Augustin dépasse l’Algérie. Ferveur mariale. Ardemment invoquée, y compris pour les musulmanes. Des cordeaux de la vierge bénis par les prêtres. 1908 : pèlerinage à Lourdes. 
\paragraph{catholicisme pluriel} populaire, intellectuel et progressiste, toutes les tendances sont là. 

\paragraph{changer le cadre colonial avant la guerre}
Ils passent d’une minorité dominante, à une minorité minoritaire.

\paragraph{Post indépendance, les chrétiens décident de réfléchir à leur position} Se regrouper.  Deux possibilités : soit une Eglise d’Embassade, soit une Eglise du pays. Algérianité de l’Eglise ? l’indépendance a mis citoyenneté = arabité = Islam. 
\paragraph{les catholiques} pieds rouges : coopérants. Mais reste uniquement le clergé dans les années noires. Migrants. En 2012, 500 fidèles / 20 prêtres.   
\paragraph{conclusion} influence française avant et après l’indépendance, même si Afrique subsaharienne. Rôle des bibliothèques, présence des chrétiens en Algérie à repenser. Question de la mission vis-à-vis des evangélistes qui convertissent. Redécouvrir un aspect de l’Eglise catholique 


\begin{Synthesis}
Situation de minorité dominante : contexte coloniale. Une majorité part avec l’indépendance. Le choix fait alors par l’Eglise de vivre l’enfouissement sous la contrainte sociale. Catholique : comment vivre la catholicité ? Pour les protestants, tous algériens, comment être citoyens avec leur nouvelle appartenance au christianisme.
Gestion de la diversité dans une même société. Question de la liberté religieuse qui implique la liberté de conscience.
\end{Synthesis}
\section{Jessica, étudiante Al Mowafaqa}
Pasteure, bénino gabonaise.
Convertir les autres catholiques, « sont ils vraiment chrétiens ? » : mon objectif au début mais formellement interdit. Et j’ai découvert leur foi.
\paragraph{Cain et Abel} lors de la bagarre, Abel reconnait les traits de soi et arrête de se battre et meurt.  Souvent, nous nions la fraternité de l’autre.
L’incarnation du Christ : imiter le Christ ; solidaire de l’humanité ; sans distinction aucune. Institut prophétique. 



% --------------------------------------------





\section{Les perceptions quels regards portés sur ce qui se passe et se vit ?}
\mn{}
\subsection{Quand la minorité conteste la majorité : retour sur une disputation wahhabo-soufie au Sénégal}
\mn{Seydi Diamil Niane - \href{https://www.mizane.info/seydi-diamil-niane-entre-la-sharia-et-la-haqiqa-il-ne-saurait-y-avoir-opposition/}{sharia et de la haqiqa}
}

\paragraph{wahhabisme comme minorité} Le Sénégal est majoritairement soufi.  « je suis musulman et mon marabout est x ». Islam confrérique. Vers le XIè siècle, le Sénégal accepte l’Islam mais à un niveau politique et au début du XX, la popularisation vient par les soufis. Les musulmans sénégalais partaient à la Mecque avec la découverte du wahhabisme : 
\begin{quote}
Marcel Gardet « une nouveauté… comment confrérie inavouée en sorte ? »
\end{quote}
\paragraph{Ahmed ‘Armadou} étudie en Arabie Saoudite. Le premier à avoir critiqué le soufisme. « le soufisme pense que leurs maîtres sont des dieux ». Raisons pour lequel il voit les dérives soufies. 
\paragraph{réplique des soufis} Al tijaniyya. La plus importante des confréries soufis en Afrique. Mais il y a aussi d’autres soufis (coloré).  Cheick xxx va répondre à Ahmed ‘Armadou. 3 critiques, dont la troisième
\begin{Ex}
[critique de la méthode]  Lire un livre dans la première édition : non. As-tu lu le manuscrit : non. Alors comment connais tu le livre ? mon maitre me l’a dit. Mais si tu sais que le livre a été écrit après le maitre, ce sont des affabulations. 
\end{Ex} 

\paragraph{disputation} Emergence du Wahhabisme au Senegal : la confrontation a pris le tour intellectuel de la disputatio. Les soufis ont été attaqués pour la première fois. Force la majorité à préciser certaines options. La minorité peut bénéficier à la majorité. 

\paragraph{Question de la définition de l’orthodoxie} Continuité de l’Islam subsaharien par rapport à l’Islam mondial.

\paragraph{question sur le wahhabisme minoritaire} Comme la pratique orthodoxe est largement commune à celle du wahhabisme (cf Culte des Saints), peut on vraiment dire que le wahhabisme est minoritaire. 
Débat intéressant qui montre que l’orthodoxie est aujourd’hui vue comme celle du Hanbalisme (ex : refus de l’intercession) alors qu’historiquement la théologie acharite reconnaissait l’intercession des saints. 


\subsection{judaïsmes maghrébins entre patrimonialisation et effacement}
\mn{Karima Dirèche, CNRS - Aix}

\paragraph{Indonésie} 270m musulmans.  Cet état a établi une égalité de droits malgré la disproportion des populations. 

\paragraph{idée générale : histoire propre au Maghreb} Ces judaismes ne peuvent être que Maghrébins mais un histoire tumultueuse depuis la colonisation.  Plus ancienne Synagogue (VI après JC à Djerba). Mouvement suite à l’inquisition. 
\paragraph{civilisation arabo-berbère judeo-musulmane} Mais un statut ambivalent de dhimmi. Principales conclusions de la diversité ethno-religieuse.  Les juifs algériens devenus citoyens en 1870 n’auront pas la même vision que les Tunisiens et Marocains. En Tunisie, les Ottomans donnent une égalité de droits aux juifs en xxxx


\paragraph{colonisation} A transformé la coexistence musulmane et juive. Première chose en 1830 c’est la suppression de la dhimmi. En 1870, cadeau de la citoyenneté, alors que les musulmans restaient des citoyens de seconde zone.  A été considéré comme un moyen de sortir de la dhimmitude.

\paragraph{Education} ecole française et école Israelite Universelle. On adopte les prénoms, les vêtements… : occidentalisation et intégration au modèle français, promotion sociale.  Les juifs et musulmans s’éloignent de plus en plus.  Le destin de la communauté juive. 
\paragraph{Octobre 1940 loi anti juive} Les juifs de Lybie sont déportés en 1938. La suppression du décret Crémieux, est traumatisante. En Tunisie, les Allemands contrôlent le territoire et persécutent les juifs de 41 à 43.

Au Maroc, « si mes sujets doivent porter une étoile jaune, je serai la premier à la porter » Mohammed V. Contexte assez exceptionnel. Le nombre des juifs au Maghreb représente 500 000, soit la moitié du monde arabe. Et 250 000 au Maroc.
\paragraph{Israel 1948} L’agence juive est très efficace, avec une émigration très importante. Hassan II : chaque juif à l’étranger est un ambassadeur pour le Maroc ». 

\paragraph{Indépendance et Nationalisme} Les nationalismes de façon générale ne respectent pas les minorité. Avec le contexte palestinien, une vraie réaction anti-juif au Maghreb avec le départ de pratiquement tous les juifs. 

\paragraph{Tunisie et Maroc : des micro communautés juives} mais n’a pas empêché la conservation patrimoniale et culturelle. Question en Tunisie sur la sécurisation des lieux de culte
\begin{Ex}
Timbre poste de la synagogue de Djerba
\end{Ex}
Le Maroc valorise par ses actions patrimoniales et la constitution de 2011 et pose la question de la liberté religieuse. 3 magistrats rabbins qui appliquent la loi mosaique sur les juifs. La maison de la mémoire à Essaouira. Tunisie et Maroc ont une même politique d’intégration et patrimonialisation.

\paragraph{plus de juifs Algériens en 1962}  Du côté des Algériens, une acrimonie : les juifs ont accepté l’assimilation, ils se sont donc coupés de la nation. 

\paragraph{quelle identité narrative pour les juifs marocains aujourd’hui} Aujourd’hui, la troisième génération parle de « déplacés », un attachement profond au Maroc.
\paragraph{repositionner la migration juive dans un contexte post colonial et non colonial} pour éviter l’ultra dolorisme. Ce qui est intéressant avec les juifs marocains, c’est que le lien est resté entre juifs marocains et juifs marocains en Israel. L’Etat marocain fait bcp d’effort vis-à-vis de ses marocains juifs à l’étranger. En Algérie, on n’a pas cela. Un pèlerinage très important juif en Algérie n’a pu se faire. 

\subsection{Migrations africaines et transmission religieuse chrétienne au Maroc}
\mn{Sophie Bava – Migrations africaines}

\paragraph{2000 : blocage des frontières entraine une installation des africains subsahariens au Maroc} Deux campagnes de titularisation du Maroc, la dernière en 2013. 
\paragraph{Communauté de sang} lenteur, manque d’argent, crainte du lendemain, propres expériences y compris religieuse (et on passe d’une religion héritée à une religion choisie). 
La religion permet d’éviter les mauvaises fréquentations. 
\begin{quote}
Au début, on priait en cachette. Mais maintenant on laisse notre lumière allumée. On ne dérange personne. 
\end{quote}

\paragraph{dynamisation de l’Eglise par ces migrants, une bénédiction}
\paragraph{Christianisme bouillonnant} Importance du Royaume. La mobilité ouvre des vocations. 
\begin{Ex}
Pasteur de la main xxx : Pacico « tu peux partir, je pars ». 
\end{Ex}
\paragraph{et si le christianisme marocain était la chance du christianisme} Les Eglises des maisons à contrôler.  Un excès de foi à canaliser qu’il manque en Europe. De nouvelles structures à créer : inégalité sociale,  Nord sud, islamo chrétien. 

\paragraph{Institut Al Mowafaqa} catholiques et protestants. Rapidité de sa mise en marche.  Dialogue inter-religieux. Intégrer le volet culturel (africain, ouverture à la société marocaine).
\begin{Synthesis}
Le vrai dialogue nous change. Al-Mowafaqa. 
\end{Synthesis} 

\paragraph{FOREM : Eglise de Maison} tout le monde n’était pas d’accord.

\paragraph{Reprise}
Réhabilitation / Emergence de ce christianisme. Méfiance courtoise de l’Etat (migration : transporte non des corps mais aussi une culture)

\paragraph{Changement du regard de la société marocaine sur la réalité chrétienne ?} Est-ce que la réalité subsaharienne chrétienne change la vision du christianisme vu comme « occidental » ? Difficile de dire.  On voit de plus en plus de liens, plus simples. Il manque des travaux sur la vision des marocains sur le christianisme. 		

\paragraph{Africanisation de l’Eglise} On n’est pas dans la Foi mais dans des processus d’africanisation.  Pluralité des dénominations religieuses. Sentiment d’appartenance et de tension, préservation : « faire tenir une idéologie religieuse ». 				
\subsection{Michele Lapka} CI. Protestante. Désir de connaître Dieu. En CI, nous sommes conservateurs, nous ne connaissons que le « nom de Jésus ». Me connaitre moi-même, dans ma foi et connaitre l’autre. Mowafaqa : un laboratoire pour la paix. 

% -----------------------------------------------------------------------------------------------------------------------
\section{ Quel sens donner à l’expérience d’être minoritaires ?}
 

% -----------------------------------------------------------------------------------------------------------------------
\subsection{du deuil de la chrétienté à une nouvelle manière d’habiter la société : l’in-existence des catholiques en France}
\mn{Xavier Gué
Vendredi Après midi animé par Rachid Saadi (Al Mowafaqa) et Anouk Cohen (Centre Jacques Berque). Problème de la prière à 15h}

\paragraph{in-existence} Romano Guardini , {théologien allemand}, essence du Christianisme. Vision Paulinienne. Ils doivent animer le monde comme le sel, tout en s’effaçant. 
En France, de nombreux monuments rappellent le christianisme. In-existence, veut signifier cette tension : 
\begin{itemize}
\item Chrétienté : loi inspirée par le christianisme. Favorise le vivre ensemble et la manière de vie commune.
\item quand elle est minoritaire ; des disciples sont sels, signes, dans un monde qui n’est pas façonné par elle.
\end{itemize}

\paragraph{Eglise communion} ou Eglise comme Corps, réalité sociale. Et c’est parce qu’elle est corps qu’elle pose problème : elle n’est pas une simple culture, elle \textit{résiste}.

\paragraph{histoire} Le terme de communion pas utilisé dans le NT pour décrire l’Eglise. Ac « communion fraternelle ». Ac 2,44 : « tous ceux qui étaient devenus croyants, mettaient tout en commun ». Fraternité. Ce qui fait communion, c’est l’eucharistie.
\begin{quote}
Communion au corps et au sang du christ. Eglise communion de communion.
( ?)
\end{quote}

\paragraph{Eglise comme solidarité} Marie Françoise Bathé\sn{comment le monde est devenu chrétien} donnait de la visibilité aux chrétiens. Toilette mortuaire des morts lors de la maladie Alexandrie 251. Alors que les autres religions chassaient les malades.

\paragraph{individualisme lié partiellement au phénomène intérieur à l’Eglise} En devenant la société, elle perd son caractère social pour un rôle sociétal. « un curé vaut mieux que deux gendarmes » (au XIX). Mais aussi pour permettre aux fidèles d’obtenir le salut (individuel). Eschatologie collective (« le règne de Dieu ») et une eschatologie individuelle (« jugement à la mort ».\sn{1336 Benoit XII dit ce qui se passe après la mort. Les saints verront Dieu. Les âmes en péché mortel vont en enfer. Les autres, le purgatoire. }
Au Concile de Trente, on ne se pose pas la question du salut collectif mais du salut individuel. Eglise était comme un Royaume anticipé sur terre. Dans l’Encyclopédie Catholique de 1924, pas d’entrée à Règne de Dieu. Comment prier « que ton règne vienne » alors que l’Eglise est le Règne ? 
En 1938, Henri de Lubac avait vu le problème : « catholicisme ». L’individualisme détruisait la foi. 

\paragraph{la sécularisation de la société chrétienne et individualisation de la Foi} Nombreuses études sur ce processus. 1905 : neutralité de l’état. 1968 : émancipation de la société. Récit du progrès soutenant l’émancipation. Portier : renforcement du magistère sur les fidèles.

\paragraph{contexte extra-moderne : déconstruction et reconstruction du religieux} \mn{Portier} Mais Portier n’a pas tenu compte des causes internes au christianisme. Deux influences de la théologie post libérale (Lindbeck) et le mouvement maurassien. 
\begin{itemize}
\item 
Culture minoritaire renforcée et transmis à la génération suivant (Exilé, terme adapté pour l’Eglise américaine). 
\begin{Ex}
Près de Tours, un catholand.
\end{Ex}
Pourtant Jn 1,11 : « Le Verbe vint au monde et \textit{les siens} ne l’ont pas reconnu ». 
\item habiter le monde en disciple du Christ, pas un corps sectaire. Lettre à Diognete. Rassemblement (Eglise) et dispersion dans le monde en vue de l’unité du monde. 
« Le père a tant aimé le monde… » solidaire des hommes de notre temps.  Le Christianisme témoigne du Christ quand il défend ceux qui ne sont pas eux (cf Mgr Saliège défendant les Juifs).  Non pas un corps auto-référencé mais un corps pour les autres. 
\end{itemize}

\paragraph{minorité, chance pour l’oecuménisme} un passage, lieu de tentation. Les jeunes sont plus identitaires et se recentrent sur leurs propres traditions. La démarche synodale peut être un moment où l’on va progresser dans une écoute mutuelle, avec un style ecclésial différent et moins autoritaire. 

\paragraph{Corps du Christ est là pour le Royaume de Dieu d’abord et donc pour le salut du monde} engagement. Elle doit faire corps, quand elle se rassemble autour du Christ et se disperser. Le Corps du Christ n’arrête pas de disparaitre.  C’est un corps qui résiste. 
% -----------------------------------------------------------------------------------------------------------------------
\subsection{Entre trajectoires de minorités musulmanes des premiers siècles et théologie de l’alterité}
\mn{Farid El Asri, UIR, Rabat, islamologie et judeologie}

\paragraph{minoritaire dans quel sens} L’islam mohamedien ne commence pas avec l’Islam, qui ne commence pas au VII. Prise de filiation : « père ». 

\paragraph{approche par la réappropriation de la foi musulmane par l’étymologie}
\paragraph{Jumallyyia- Islam : sans rupture, l’importance de la filiation} rupture dogmatique entre pre-islamisme puis monothesime. On voit bien qu’il faut une grille de lecture éthique : vendetta, l’expérience mohamedienne vient policer les mœurs. Polissage éthique. Logique de continuité. Racine tribale. 1/3 du coran raconte des histoires, 1/3 raconte des histoires de filiation. Il s’agit d’enraciner l’islam dans une filiation monothéisme. 
Islam : « acceptation de l’évidence de la réalité ». \mn{mais alors quelle résistance, Lindbeck ?  }

\paragraph{ambassadeurs chrétiens à médine : relation}
\paragraph{} rite funéraire juif : « c’est une âme ». 
\paragraph{le voyage nocturne} enracinement dans les aeuils : ascension graduelle.  Le prophète : Noé, cité dans le Coran 60 fois. 

\paragraph{dimension de kufr} dimension apparaissant dans un contexte mecquois, émerge le concept de kufr, dans un concept de minoritaire et douloureux : la Mecque est « kufr ».
A Médine, Kufr devient hypocrisie (on est minoritaire mais on veut défendre son intérêt et on le fait via un statut religieux).
\paragraph{muslim vs imane} Imane : intériorisation du message alors que le muslim fait le minimum.  
\paragraph{projet de la Umma de Médine} la Umma de Médine, c’est la communauté de Médine, avec les juifs qui ont des intérêts communs, pragmatiques.  Minorité : processus de detribusation, déracinés. Les ultra-modernes sont des déracinés. Une religion liquide 

\paragraph{l’approche étymologique et à compléter par la prise en compte de la spécificité de l’environnement } qu’est ce qui est universel dans le message musulman. La question de l’essence « Me, myself and I ». Alors que pour les musulmans doivent se rappeler qu’il n’y a pas de verbe être en musulman.

\paragraph{on peut faire la démonstration que parce qu’on est à l’intérieur } (de l’islam). 

\paragraph{affirmation ontologique normalement dans la verticalité avec Dieu ; la norme/loi étant dans l’horizontalité} or, aujourd’hui, on se définit ontologiquement par la loi. « je suis très salafiste dans mon approche » : interroger la dynamique première à travers le sens épistémologique. 
\begin{Ex}
Genèse : fascinant de voir comment on définit la Genèse sans se référer au texte coranique car le vocabulaire est sur-positivé. 
\end{Ex}

\paragraph{Jacques Elhul} Puissance, impuissance et Non-puissance : capacité de faire des choses et refuser de le faire. C’est l’exemple du Christ. Et en particulier, il ne fait pas tous les miracles, il fait les miracles quand l’amour est en jeu. Les franciscains font le choix de la non-puissance.

\paragraph{P. Ricoeur} Altérité renforce l’identité ipse qui se construit par relation. Mais il y a aussi une identité idem. On a besoin des deux.
\paragraph{dialectique} au sens de Kierkegaard, les pôles théologie et philosophie se nourrissent l’un l’autre, pas au sens de Hegel, où théologie et philosophie sont dépassées.  
% -----------------------------------------------------------------------------------------------------------------------
\subsection{Etre minoritaires : destin, opportunité ou grâce ? }
\mn{Frédéric Rognon, Faculté de théologie Protestante de Strasbourg, }
\paragraph{defi n’est pas le destin, force n’est pas opportunité} La force est permanente, l’opportunité s’offre à un moment et pas à un autre. Par ailleurs « ou ».
\paragraph{diversité des situations minoritaires} pas de réponses univoques. Minorités éthiques, religieuses, ethno-religieuses. Cf Apartheid, minorité dominante et persécutante .
\begin{Ex}
Canaque : minorité 39\% autochtone.
\end{Ex}
\paragraph{micro-minorité ou minorité ?} pas pareil d’être 2\% (protestants en France). La défense  à la laïcité est elle inhérente au protestantisme ou est elle liée à la minorité ?
On peut devenir de persécutés (Huguenot) à persécuteur (Sud Africain).
On voit donc qu’il n’y a pas de soi.

\paragraph{3 termes évitent la pensée trinitaire}

\paragraph{Etre minoritaire, destin} dukei, puissance supérieure aux Dieux pour les grecs. Victimes du destin qu’acteur de leur histoire. Chez les stoïciens, le destin était si fort que le philosophe devait mettre son vouloir au destin : « me vaincre plutôt que la fortune. Rien n’est en notre pouvoir que notre pensée » (Descartes). 
Les faits sont têtus, donc le fait minoritaire est un destin. Ceoendant, le destin ne doit pas être vu comme un drame épouvantable : plus ou moins lourd. « amor fati » : aimer son destin, de Nietzsche. Le destin peut même devenir aimable. Ecole stoicienne : « j’ai fait une heureuse navigation : grâce à ce naufrage, j’ai créé le stoïcisme ». Ricoeur : sort et ratification du sort. Le cercle existentiel : 
\paragraph{Etre minoritaire, une opportunité ?} un Kairos : 85 fois vs Chronos 45 fois dans la Bible. La Bible est moins le temps qui s’écoule (Chronos) qu’une chance à saisir (Kairos).  Une opportunité, à discerner et un engagement fort. \sn{cvx}
Malgré la minorité, on peut laisser passer l’opportunité et continuer la bureaucratie et non se focaliser sur ce qui a de la valeur. Saisir l’opportunité pour se convertir à ce que Dieu nous appelle à être. Intelligence collective et synodale pour saisir l’opportunité. 
\paragraph{Etre minoritaire, une grâce ?} la Grâce n’exige aucune condition. Charis ou Gratia, aucun effort particulier ; reçu avec action de grâce car elle ne peut être rendu. Charis et Kairos : reçu ou saisi.
Mais pas forcément bon marché. Boenhuffer, « la grâce est gratuite mais elle a un coût ». C’est ici que s’articule opportunité et grâce : payer le cout de la grâce : en rendre grâce de la joie d’être minoritaire.
2 CO 12, 9
\begin{quote}
Ma grâce te suffit…
Quand je suis faible, c’est la que je suis fort
\end{quote} 
Cela vient après l’écharde dans la chair. Eclairage précieux sur le fait minoritaire. Renversement des valeurs : la faiblesse devient fort quand le Seigneur vient l’habiter. La prix de la grâce : vivre en \textit{ vertu de cette grâce }

\paragraph{conclusion} les trois termes pensés comme des idéaux type. Ces trois types ne s’excluent pas, renoncer au destin comme fardeau,  une opportunité de renoncer à nos fantasmes de toute puissance et payer le cout de la grâce : vivre de cette grâce.

\paragraph{Un kairos pour l’oecuménisme} Vatican II, Pape François\sn{François, ce pape que les protestants aiment, Reforme, 2023}. 

\paragraph{Regard ou posture sur la situation dans laquelle on est} D. Boernhoffer considère que Luther a survalorisé la grâce. Il insiste sur le coût de la grâce de l’engagement.  Martin Mueller, Eglise confessante (« quand ils sont venus chercher les juifs, je n’ai rien dit car je n’étais pas juif… »), 8 ans en camp de concentration. 

\paragraph{mythe de l’homogéneité} parallèle entre les africains au Maroc et les musulmans en France. Quand l’alterité devient visible et n’est plus « exotique », c’est là que vient le problème. On est dans une civilisation de l’œil, tout passe par le voir. 

\paragraph{le rapport au Coran pour les musulmans est le même que le rapport des chrétiens au Christ} verbe de Dieu. On méconnait l’analyse du Coran qui a toujours existé, en intégrant les sciences anthropologiques et sociologiques au temps mohammedien. Eviter les cloisonnements disciplinaires. 

% ----------------------------------
\subsection{Discussions}

\paragraph{minorité franciscaine présente au maroc depuis 800 ans}

% ----------------------------------
\section{Conclusion}
\mn{Cardinal Cristobal - Karen}
\subsection{Defi}
\paragraph{défi de Al Mowafaqa} revendiquer le droit à la citoyenneté. Insister sur ce qui nous unit. 
Des tentations : 
\begin{itemize}
\item se refermer dans une bulle, ne pas sortir (« vivre au Maroc comme si on était en France »)
\item Complexe de supériorité ou infériorité. 
\end{itemize}

\paragraph{Je suis important car je suis minoritaire} Karen : peur d’être noyée dans un monde d’alterité : est ce que je vais garder mon alterité si je vais rencontrer l’autre.  Bcp de difficultés pour les chants ce soir. Qu’est ce que je vais chanter avec les musulmans, Allah Akbah. Est-ce que je vais garder mon identité. 
\paragraph{Notre défi est de créer des espaces sécurisés où je serai respecté.  }
\paragraph{comment écrire un narratif chrétien d’ouverture vers l’autre ? } assumé par nos paroisses. 
\paragraph{la question des finances de Muwafaqa}

\subsection{Force}
\paragraph{dialogue}
\begin{itemize}
\item La force de l’identité
\item Le courage de l’alterité, 
\item La sincérité des intentions
\end{itemize}
\mn{Karen}
\paragraph{notre force, ce sont nos relations } Nous sommes enfants. Quand on reconnait ces relations. Notre credo au Maroc : Rassemblé (Père), nourri (Fils), et envoyé (Esprit)
\mn{Cristobal}
\paragraph{notre force, c’est le sel, le levain, la lumière} à condition d’être authentique. La semence : parabole de Dieu. La force de la faiblesse (1 Co).

\subsection{grâce}
\paragraph{la grâce et nous pousse d’être un petit troupeau, en sortie, en dialogue} Au Maroc, je me sens accueilli. Et cet accueil permettra à un marocain d’aller au ciel (Mt 25).
Et laisser la minorité des franciscain.
\mn{Karen}
\paragraph{hum- (us) } illusion d’être au centre du monde. Nous sommes poussés à être décentrés. Cela laisse de l’espace pour l’autre. 
\paragraph{Etre don pour le Maroc} être bénédiction pour le Maroc.

\begin{Synthesis}
Le risque n’est pas d’être minorité mais d’être insignifiant. Pape François.
\end{Synthesis}






