\chapter{coupes divinatoires}

Deux bols magique en cuivre et laiton XIXe
COUPE DIVINATOIRE dit bol «magique «en alliage de cuivre partiellement étamé, de forme circulaire aux bords évasés. La paroi extérieure est gravée d'une fleur de lotus à huit pétales calligraphiés au coeur en forme d'une étoile à cinq branches, et une longue inscription le long du rebord externe. L'intérieur est décoré d'un cartouche inscrit, de deux motifs de tughra, d'un sceau de propriétaire en forme d'amande «sahib Tador (Théodore ?) «et d'un cachet ARMENIEN daté 1875». Le rebord est gravé d'une longue frise épigraphique sur deux lignes. Empire ottoman, datée 1875.
Haut. : 4,9 ; Diam. : 15,1 cm




Bol talismanique ou bol magique coupe de forme circulaire à bords évasés, ombiliquée en laiton anciennement étamé incrustation de pâte noire gravé à l'intérieur de mihrabs, inscriptions en écriture naskh et nasta'liq d'invocation religieuse et verset coranique, patine d'usage. Iran XIX-XXe.
Haut. : 4 ; Diam. : 13,5 cm