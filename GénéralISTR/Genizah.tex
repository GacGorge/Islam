
\chapter{A trier}


\section{Notes diverses à repositionner}







Important de connaître un auteur pour avoir un avis objectif.





\begin{itemize}

  
\item  Est-ce que j'utilise la raison, l'analogie, la coutume locale ? c'est
  cela les différences.
\item  Si la coutume est la laicité, je dois en tenir compte pour mon avis.

  \begin{itemize}
  \item    Le shavinisme qui intègre le ur, la coutume,
  \item     le hanafisme, aussi~
  \item    la où on le ferait moins, c'est le hanbalisme.
  \end{itemize}
\end{itemize}



  Pour quitter l'Islam, la peine est celle de la Loi locale. En France,
  si on définit l'islam comme une loi, on dit son aversion. Tout le
  chapitre sur la loi, «~islam = loi~», est un schématisme redoutable.
  Ce n'est pas qu'un texte législatif. Malheureusement, il y a tout un
  courant dans l'Islam qui encourage cette lecture caricaturale. Dans
  les pays occidentaux, on peut combattre avec les idées l'islam
  radical.
  
  \subsection{Aristote}
  Penser raisonnablement : éviter le relativisme (l'Islam n'existe pas) mais ne pas non plus en faire une essence.
  
  \subsection{Islam compliqué}


Rachid Benzime

Islam, compliqué à lire le coran sans clé herméneutique

Passé colonial~en France~:

champ sémantique~: passer d'un champ indigène, à immigré, à ~musulman.
La religion devient le marqueur identitaire.

Pisani~:

Compliqué l'islam~; islam~: complexe

Edgar Morin~: c'est quoi la complexité d'un fait social. Si complexe,
reponse complexe

Ici, les arrières pensées~: on croit connaitre de l'islam alors que ce
n'est qu'une réalité.

Les musulmans comme «~citadelles assiégées~»

Difficile pour les musulmans de voir une certaine réalité car l'islam
quelque chose de beau.

«~un terroriste qui se dit musulman, on n'a pas le droit de lui dire
qu'il n'est pas musulman~»

Derrida~: «~il faut bien séparer l'Islam de l'Islam~».

Accepter que l'islam est pluriel, alors que l'islam est vécu par les
personnes comme unique.

Pisani~:

Macron~: l'islam est en crise.

Ce n'est pas possible pour les musulmans~: «~l'islam ne peut pas être en
crise~» - méta religieux.

Mohammed Arkoun~: le fait islamique. Dieu est absent.

Trop de représentations dans le champ «~Islam~». Mot trop chargé.

What is Islma.
