\chapter{introduction à l'année universitaire ISTR 2023}

\paragraph{objectifs ISTR} approche dialogale

Objectif de la visite du pape au retour de Mongolie : 
\begin{singlequote}
    
\end{singlequote}

\begin{itemize}
    \item culture et spiritualité d'Asie, seule formation de ce type en France
    \item connaître l'Islam et connaître les musulmans, elle rencontre la réalité en France. 
    \item culture contemporaine. La mission de l'Eglise doit respecter la culture du temps.
\end{itemize}


\paragraph{les colloques}

\begin{itemize}
    \item comment redéfinir ce qu'est une religion au XXIè ? 
    \item journée d'étude sur la question : maître / disciple dans le judaisme et christianisme. 27 mai 
    \item interpréter le Coran. 
\end{itemize}



\section{création de l'ISTR}
\mn{Catherine Marin}

\paragraph{1907 - Mgr Baudrillard} Grande figure de la catho jusqu'à la seconde guerre mondiale. Crée une chaire d'histoire des religions. Mgr Alexandre Leroy, spiritain. Le public c'est des personnes qui partent en mission (religieux et laics).
L'abbé Paul Debreuil (1879) : \textit{problème et conclusion de la théologie des religions}.

\paragraph{fondation des Instituts Catholiques} en 1875.  

\paragraph{Georges Voyot} Années 1930. Grande personnalité
Mais on peut rester dans une logique apologétique. 1936 : série de conférences de Mgr Monservain contre l'anti-sémitisme en expliquant ce qu'est le judaisme.


\paragraph{fait religieux, post seconde guerre mondiale} On passe dans une logique de rencontre.

\paragraph{le fait marxiste et indépendantiste}

\paragraph{la sécularisation}

Dans un contexte d'exode rural et d'industralisation, qui change les paroisses. En France, 1930, moitié d'urbains, moitié de ruraux.
En 1960, slt 16\% de ruraux. Cest le temps du déracinement. Disparition du religieux. 

\paragraph{entrée en résistance ?} \textit{Rerum Ecclesiae} créé en 1947 : sociologie religieuse. 

\paragraph{Henri Bernard Maïstre, sj} Mgr Blanchet, recteur en 1946. Il va créer un centre de recherche des religions à partir d'une lecture sociologique. Fusion avec la section sociologique en 1963.

Pompidou : soutient l'ICP. 

\paragraph{le concile} Maistre, conseiller de Mgr Blanchet au concile.
Nouvelle approche des religions : direction nette au sujet des religions. Partir de S Paul sur les religions naturelles. Ce travail va aboutir à des textes autour du concile.

\paragraph{entrer en conversation}

\paragraph{Père Guillou} crée l'ISTR. Refuse M. de Certeau. Mais il y aura P. Hayek, de Lubac, Danielou. "athéisme moderne", "athéisme marxiste". 

En 1968, Danielou, ferme avec souplesse.





