\chapter{Ecologie et Religions - Colloque IDEO 2022 }
\mn{pour célébrer les 20 ans de la bibliothèque - Le Caire}

\section{M. Aziz Hilal}

\paragraph{Aziz Hilal} il est philosophe, il est membre de membre de l'IDEO, philosophe, il enseigne l'arabe et la philosophie au lycée français du Caire et il est un passionné d'Al Farabi. Il a fait sa thèse de doctorat sur Al-Farabi. Il en est un lecteur précis et qui essaie de faire justice à ces textes qui parfois sont lus de manière un peu biaisée.

\paragraph{Qu'est ce que l'écologie ? Qu'est ce que la création en soi ?}

\paragraph{Thèse d'Hercule}Je commence tout de suite par une thèse qui me passionne. C'est celle de Thomas Bower que je découvre et qui s'appelle cette thèse \textit{Der Kultur der Ambiguitate}. Tour de l'ambiguïté. La culture de l'ambiguïté pour dire que l'islam, en quelque sorte, avant un peu les textes qui ont été fixés une fois pour toutes était un continuum.
Ça veut dire qu'il y a sur la nature. Vous allez voir qu'on voit tout trouver. D'abord, on parle de protection de la nature ou de l'environnement. Le mot qui veut dire environnement, c'est a n'existait pas, il n'était pas utilisé dans une autre signification. On parlait de beit. Cela veut dire la maison et par extension, c'est fonder une maison.
Alors c'est très important. Partir par là. Le mot nature n'existe pas comme tel dans le Coran par exemple. Alors qu'est ce qui existait ? Existait un mot qui compte, qu'on a trouve, qu'on a soustra (?), qu'on a traduit par la nature. Mais ce n'est pas une traduction exacte. Je ne sais pas la peine, ça veut dire c'est pas la peine de rentrer dans le détail lexicographique album de la nature que Dieu a donné à chacun de nous au début de la création, est ce qui décide un peu de la nature de chaque homme. Al feltra
Est ce que ce qui est présent, ça peut être en fait en pour parler de la nature, il n'y avait pas le signifiant. Bien sûr, il y avait le signifié. Je commencerai par un verset coranique que tout le monde utilise pour dire que finalement, on trouve dans le Coran quelque chose de le côté de l'environnement. Dans la sourate deux, vers 30,
\begin{quote}
    2 : 30 - Lorsque Ton Seigneur confia aux Anges: "Je vais établir sur la terre un vicaire "Khalifa". Ils dirent: "Vas-Tu y désigner un qui y mettra le désordre et répandra le sang, quand nous sommes là à Te sanctifier et à Te glorifier?" - Il dit: "En vérité, Je sais ce que vous ne savez pas!".
\end{quote}
  Là je traduis. Donc quand Dieu a dit je vais établir sur terre, il y avait le mot terre, pour la nature sur terre indiquée ou bien un sous réseau. Et les anges ont riposté. Seniors, vous allez mettre quelqu'un qui va semer le désordre, vous allez mettre quelqu'un - je traduis presque mot à mot -.
vous allez mettre quelqu'un qui va se porter et qui va répandre le son alors que nous, les anges, on vous glorifie et on vous sanctifie. Et Dieu a fini par dire :
\begin{quote}
     je sais ce que vous ignorez.
\end{quote}
 Alors cette histoire de Dieu a établi sur terre un vicaire, un sous régent, ça veut dire lui a confié un dépôt sacré pour s'en occuper en quelque sorte.
L'homme, dans un autre verset, était un petit peu naïf. Il a accepté ce dépôt sacré parce qu'il a été au départ confié à toute la nature. Nous avons exposé le dépôt sacré Amana à à tous, sur tous les cieux et les terres. Ils ont refusé, mais l'homme a accepté.
Alors l'homme a accepté d'être le vainqueur, d'être le sous régent pour s'occuper de la terre. Même si les Anges ont un peu mis en garde le Seigneur. Attention, peut être que vous allez mettre quelqu'un qui va répandre le sang et  qui va mettre le désordre, ça veut dire qu'il va mettre du désordre dans la nature.
Dans le Coran, il  y a toute tous les éléments de la nature, mais il n'y a pas la nature en tant que tel, Il n'y a pas le mot, il le prouvera. Après, en philosophie, on ne le trouve pas dans le Coran. Il y a tous les éléments de la nature. La nature est quelque chose, la nature, il y a quelque chose de le Coran, de ce qu'on appelle.
Tous les Grecs ont appelé le pharmacant. Remarquons que ça veut dire un format quand ça veut dire Dans le caducée d'Hermès, vous avez les deux serpents qui se font face, un qui représente bien sûr la bienfaisance et l'autre qui représente la toxicité. C'est la nature qu'on trouve dans le Coran. La nature est quelque chose d'agréable.
La nature est quelque chose de recèle beaucoup de beauté.   Mais en même temps, vous voyez la beauté en même temps, cette beauté là. En même temps, ces lampes là, ils sont vraiment des projectiles pour chasser les démons. Vous voyez le sens de Fatma quand c'est comme ça, la nature, la nature qui va être glorifiée.
D'un côté, ça veut dire tous les grands peuples qui sont dans l'histoire, Dans le Coran, ils ont arrivés à un point presque de maîtrise de la nature de la mettre à leur merci. Si on parle par exemple de peuples, de tabou, de peuples, d'autres, les peuples de Al, c'est des gens qui ont fondé une ville extraordinaire qui s'appelle Ams, aux colonnes.
Paraît il que c'est une ville à plusieurs étages, même à des sous sols, et on ne sait pas où il est. Mais qu'est ce que ça veut dire ? Les éléments de la nature ont été. On a fait que les éléments de la nature pour la nature obéissent au même temps, parce que ces peuples là n'ont pas compris le message divin, n'ont pas écouté les envoyés de Dieu.
C'est la nature elle même qui va se retourner contre eux. Et on trouve que la nature, le peuple Al, par exemple ont été décimés par un vent de sept jours, un vent violent. Vous voyez les deux côtés de la nature, mais la nature belle finalement, et la de ce modèle 

\paragraph{Paradis}et ce modèle, bien sûr, c'est le paradis avec les rivières limpides, avec le miel des rivières de miel, avec les rivières de lait, avec les quatre voies sont connues pour les Arabes.
Vous savez que c'est un milieu austère. Et quand on parle de miel et d'eau extraordinaire, ça, c'est l'image de la nature parfaite. Cette nature parfaite. Finalement, vu qu'on vivait dans un désert, qu'on vivait dans un milieu austère, c'était presque le modèle et le rêve. Et tout ce qui pourrait exister sur le sur terre et éphémère. Et c'est ce paradis, cette nature idéale, cette nature extraordinaire.
C'est cette nature qui va être célébrée par de grands poètes et les Arabes auront quitté le désert. Bien sûr, les musulmans ont quitté le désert, ils sont allés au l'Andalousie et en temps, le tout d'apprécier la nature. Les meilleurs poètes d'aventures se trouvent l'Andalousie, ça veut dire cette terre qui va de Saragosse à au sud de Porto, la ligne qui va un peu, surtout au neuvième Xᵉ siècle.
Après, on va, descendra pas vers le bas. Il y a un grand poète xxxxx, qui a vécu à Saragosse et après qu'il va partir vers Valence quand il y avait avoir le mouvement de la Reconquista. Mais c'est bien sûr qu'était loin de toute l'histoire de guerre, est un vrai, pourrait être un vrai troubadour. D'ailleurs, il a vu quelque chose des troubadours et raconte une histoire extraordinaire et disait au Andalou : "profitez vous, profitez de cette nature.
Personne d'autre ne l'a pas vu là. Un paradis sur terre". 

Alors je vais vous donner le poème en arabe, je le traduis.  vous avez les gens de l'Andalousie, vous avez l'eau, vous avez des ombrages, vous avez des rivières et vous avez des arbres.
Je crois que le paradis, c'est vous qui est la posséder encore. Et si on m'avait demandé à choisir, c'est ce paradis que je choisirais. Ça veut dire si on m'avait donné le choix entre les deux paradis, paradis céleste, le paradis terrestre choisirait le paradis terrestre et après lui, je continue en traduisant. Il dit finalement N'ayez pas peur, le feu de la géhenne est loin de vous, parce qu'aux hommes, quelqu'un qu'on a introduit au paradis ne peut goûter le tout ne peut subir les flammes de la géhenne.
Alors c'est sûr. Mais bon, quand on l'Andalousie, on l'appelle d'ailleurs à , on l'appelle maintenant le paradis perdu.  


C'est cette nature qu'on a perdu, qu'on pleure quand tous les poètes ont vraiment pleuré. 

\paragraph{Philosophie} Troisième volet Je pense maintenant à la philosophie seule en matière. Je veux parler de la philo, de la nature en philosophie bien sûr. Chez les juifs, c'est dans la culture arabo musulmane, pas les autres.  la nature n'a rien à voir avec la nature telle qu'elle a été développée chez les Grecs.
Ça veut dire la nature a ses propres lois et est indépendante. 

\section{Emmanuel Pisani - Al-Ghazali et Ecologie}
\label{theol:AlGazali25}

    \paragraph{Les Ecrits de Ghazali sur l'Ecologie sont rares}Pour une protection possible de la nature, de la manière dont il convient de penser la relation de l'homme à son environnement, à la création des données. Dans quelle mesure le Ghazali pourrait il permettre l'éclosion d'une pensée écologique islamique qui partirait de sa théologie en trois en trois points ? Le premier, c'est d'abord en vue d'un constat : nonobstant l'importance de notre auteur, Les écrits sur la Création de Ghazali dans l'islam sont très rares : elles ne sont pas inexistantes justement - x ont attiré mon attention sur un de ses articles publié récemment où il est cité.
Elles ne sont pas inexistantes, mais elles sont rares, marginales. Alors pourquoi est ce que ce serait dû à la nécessité ? Le signe que la nécessité de restreindre les appétits de l'homme dans son usage des biens de la création n'est jamais abordée dans ses écrits ? Il est vrai qu'il n'y a pas de \textbf{théologie de la nature} qui appellerait à une attention ou une protection chez lui.

\paragraph{Question non posée : la création au service de l'homme}

    

Alors Al-Ghazali ne connaît pas la question l'anthropocène, capacité qu'a l'homme à modifier son environnement, L'extermination des dodo ? Il ne connaît pas. 
Je dirais même que pour lui, la création, au contraire, est un réservoir infini de biens disponibles pour l'homme. Et c'est le signe même de la manifestation, de la bienveillance, de la bonté, de la bienfaisance de Dieu à l'égard de l'homme que de lui donner.
Tout est là pour l'homme à son service. Et si Al-Ghazali envisage l'activité désordonnée de l'homme, elle ne saurait de toute manière avoir des répercussions décisives sur la création. 

\paragraph{Théologie de l'action}

  Pourtant, et c'est mon deuxième point, il y a chez lui une théologie d'action qui n'est pas sans incidence pour notre sujet. En effet, et dans sa réflexion, c'est un théologien.
Il part donc de Dieu. Il constate que méditer sur l'essence de Dieu qui est Dieu, ce n'est pas possible pour l'homme . Une méditation sur l'essence divine et pour mieux se faire comprendre. Peut être que je vous ai déjà perdus en disant cela. 
\begin{Ex}[Raisonnement de la mouche]
Il va donner l'exemple prosaïque de la mouche. Là, je vous l'ai retrouvé.
Si on lui disait à la mouche que son créateur  ne possédait ni des ailes, ni même ni pied, ni capacité à voler, elle le trouverait plus imparfait qu'elle même qu'elle n'est. 
\end{Ex}
Or, malheureusement, déplore Al-Ghazali, ce raisonnement, le raisonnement de la mouche et celui de la majorité des hommes. Et comme il n'écrit pas seulement pour des savants. Il est donc préférable, pour parler de Dieu, non pas de partir d'une méditation de son essence, mais de ce qu'il crée et donc de la création.
\paragraph{Nous sommes insensible à la beauté de la Création}
Aussi, il l'invite à une méditation sur la création. Et dans cette optique, la création, vous le voyez, n'est pas seulement uniquement la disposition de l'homme pour satisfaire ses besoins, mais elle est là pour \textsc{permettre à l'homme de retrouver le chemin du Créateur et admirer la grandeur de Dieu}, grandeur qui suscite la gratitude et l'adoration. Alors, bizarrement, déplore Al-Ghazali, les hommes trouvent beau devant un tableau peint par des hommes par des mains d'hommes, Mais ils restent insensibles à l'artisan des merveilles de la création.
\paragraph{C'est Dieu qui écrit dans la Création}
Chez lui, c'est grâce à lui, la création acquiert un véritable statut théologique. La méditation sur les choses créées vient ainsi tempérer l'image d'une pensée où la création ne serait là que comme un réservoir disponible à l'homme pour sa consommation et son confort. La création, au contraire, acquiert une qualité qui implique une attention, parce qu'au delà de la chaîne des causalités, il s'agit de découvrir qui est Dieu écrit.
Tous les êtres vivants sont des effets de la puissance de Dieu et une des lumières de son essence. Il n'y a rien de plus obscur que le néant et de plus lumineux que l'existence. L'existence de toute chose est une des lumières de l'essence de Dieu. 
\paragraph{la méditation transforme}La \textit{méditation} est donc pour le Ghazali, le chemin de l'illumination. Elle permet d'accéder à la connaissance divine, ce qui n'est pas sans incidence profonde pour l'homme.
Les passions de son cœur sont alors transformées. Les facultés de la vision intime intérieure développée par ces lumières divines auxquelles il accède. La perspective est clairement celle du soufisme. La méditation sur le créé permet voir ce pas visible du premier coup d'œil et l'ouvre à la possibilité un spirituel plus élevé. Mais quid ? Quid de son comportement à l'égard du Christ ?

\paragraph{Ascèse si méditation du Christ}
Si ce chemin d'illumination passe par la méditation du Christ ?\textsc{ Il est inséparable. Mesdames et messieurs, je vous le dis  d'une certaine ascès}e. Pour souligner le lien, Al-Ghazali s'appuie sur une tradition prophétique qui dit que la méditation sur les choses créées est la moitié de l'adoration. Mais le peu de nourriture, l'adoration entière. Ainsi, pour lui, la méditation de la création met sur la voie de l'adoration, adoration qui n'a que plénitude par la saisie.

\paragraph{l'ascèse régule nos appétits}
C'est mon troisième point. L'ascèse, en effet, est au cœur du dispositif du chemin spirituel qui conduit vers Dieu. Chez Al Ghazali, l'ascèse régule nos appétits et plus encore dans une perspective fonctionnaliste. Alors Ghazali va justifier la nécessité d'éprouver la fin. Et mesdames et messieurs, il va distinguer dix bonnes raisons d'éprouver la fin. Alors, je vous les livre ce soir, avant notre cocktail dînatoire.
\begin{itemize}
    \item Premièrement, elle purifie le cœur et oui, elle anime le tempérament et à la fuite, le regard intérieur. Alors que la satiété provoque l'apathie, elle aveugle le cœur et donc brûle le cerveau. Ainsi, l'ascèse ouvre à la connaissance des réalités divines.  
    \item Deuxièmement, la purification du cœur prépare à l'émotion de l'énoncé de l'énoncé de Dieu. Souvent, on prononce le Dieu, mais sans s'en délecter.
Mais celui qui a faim, quand il convoque le nom de Dieu, il le fait avec délectation.    \item Troisièmement, la fin suscite la crainte de Dieu. Elle brise lames dans le sens de son impudence, de son orgueil, de sa rébellion à l'égard de Dieu ou de son oublie de Dieu. Mais la foi permet à l'homme de voir son impuissance, sa faiblesse et elle redonne toute sa place à Dieu.
    \item Quatrièmement, la faim permet de se souvenir des épreuves qui attendent ceux qui sont éprouvés par Dieu. Elle rappelle les tourments de l'au delà et elle crée une solidarité avec ceux qui ont faim ou. Pourquoi tiens tu autant à la faim alors que tu possèdes les trésors de la terre ? Le Prophète répondait J'ai peur de manger à ma faim et d'oublier ceux qui souffrent.
Ce qui en souffre.     \item Cinquièmement, elle permet d'affaiblir les passions liées aux actes de désobéissance. Au fond, la faim de l'homme, la faim permet de contrôler tous ces désirs.     \item Sixième sens permet de repousser le sommeil, de s'habituer à la vie.     \item Septième. Nous, elle facilite l'assiduité à l'adoration. On gagne du temps à ne pas se nourrir et se tend. Pour quoi faire ?
Enfin, pour jouer sur son téléphone portable. Nous sommes au XIIᵉ siècle, mais ce temps pour adorer, pour se recueillir, pour se souvenir du nom de Dieu. On gagne du temps à ne pas faire les courses et à préparer les repas, à se laver les mains et la bouche, à aller plusieurs fois aux toilettes parce que l'on a trop bu et pour manger ?
Oui, avec Al-Ghazali, on est toujours au cœur du quotidien et de tous ces moments.    \item Huitième Non, cela donne la santé car trop de nourriture alourdit le corps et le rend malade. L'ascèse épargne des maladies au corps, épargne les cures de maladies comme l'impulsivité ou d'autres vices.     \item Neuvième un Manger peu. On fait des économies, les prix augmentent. Mangeons moins     
\item et 10ᵉ enfin.
Manger peu ou pas.
\end{itemize}

\paragraph{donner, c'est conserver dans les coffres de la grâce divine}
 L'altruisme nous rend bons envers les orphelins et les pauvres. Toujours avec un esprit très prosaïque, À la Al-Ghazali écrit que la nourriture dont il se nourrit finit dans les latrines, alors que la nourriture dont il fait l'aumône est conservée dans les coffres de la grâce divine. Alors, pour conclure, nous étions partis de la vision consumériste de la création chez Al-Ghazali et visiblement de la difficulté à élaborer une théologie islamique de l'écologie.


 \paragraph{La Création, chemin vers Dieu}
Nous avons vu cependant, à partir de ces livres spirituels qui invitent à voir toute chose comme comportant une sagesse, qu'il s'agisse des étoiles, des mers, des plantes ou des animaux. Pour lui, la création n'est pas d'abord à la disposition des estomacs et des passions des hommes, mais elle est un chemin pour parvenir adieu à la connaissance de Dieu.
Certes, certains abusent, gaspillent, ont un comportement ingrat. Ainsi écrivent critiques. Cassé la branche d'un arbre sans but précis est une ingratitude. Pour autant, il poursuit en disant que l'usage des arbres pour les hommes est conforme à la volonté divine, selon le principe que si l'homme et l'arbre sont tous deux périssables, l'homme est plus noble que l'âme, et il est sage que ce qui est moins noble contribue à la pérennité de ce qui est plus noble.
Si on devait s'arrêter à ce niveau de réflexion, on retrouverait la vision d'une création mise simplement au service de l'homme le plus noble des créatures. 
\paragraph{Aller plus loin en ajoutant l'idée de justice}
Mais il faut aller plus loin et elle va plus loin en y adjoignant l'idée de justice. Ainsi, utiliser l'arbre du voisin, qui donc ne m'appartient pas, est une injustice. Or, rappelle t il, le propriétaire ultime de toute chose créé, c'est qui ?
C'est Dieu ? Ainsi, indépendamment de la main divine, de la bonté de Dieu, celui qui prend au delà de ce dont il a besoin se comporte comme un homme injuste. Le questionnement de Al-Ghazali sur l'écologie n'aura donc pas été vain, et il me semble que l'on peut voir dans ces concepts, dans ces exemples, dans cette réflexion spirituelle, les braises encore chaudes pour penser à partir de cette éthique islamique de la responsabilité, une écologie islamique.
Et commencer à voir un peu mieux notre quotidien des lectures que nous faisons du Coran, de la poésie andalouse, des philosophes, des textes mystiques. 




\section{Adrien Candiard}

\paragraph{Adrien Candiard}
    Nous continuons notre exploration du patrimoine islamique classique avec le frère Adrien. Adrien , qui soutient très bientôt le mois prochain sa thèse de doctorat sur un auteur Très controversé, Ibn-Taymiyya.   Sa passion, Adrien, c'est d'essayer de comprendre comment pensent les gens qui pensent autrement et c'est forcément rationnel. Et donc on peut faire confiance que même les auteurs les plus en apparence délirants, extrémistes, incompréhensibles, ont une rationalité. Il est passionné par cette quête de la rationalité. Comment pensent les autres ? 

    \paragraph{risque de l'anachronisme}
Emmanuel m'a demandé d'intervenir aujourd'hui pour vous présenter ce que peut avoir à nous dire sur les problèmes écologiques contemporains. Le corpus islamique que je fréquente est celui de la théologie islamique médiévale. Je n'ai accepté qu'avec obéissance ou avec un peu de réticence ou peut être par goût du paradoxe, seulement pour les questions de méthode, qui est manuel rappelait le risque de faire des anachronismes tout le temps en lisant des auteurs avec des questions qui ne se posent absolument pas.

\paragraph{le Kalam et son aridité}
Pour éviter quand même des anachronismes complets. Je ne parlerai pas d'interdit aujourd'hui parce que là, vraiment, il y a matière. Mais surtout, rien n'évoque moins l'écologie que les arides traités de théologie islamique, cette science qu'on appelle la science du Kalam, de la parole et qui, à partir du IXᵉ siècle, va s'efforcer de faire rentrer la foi islamique fondée sur le Coran et les hadiths, les traditions prophétiques dans un cadre conceptuel rigoureux, tellement rigoureux à vrai dire, pour ceux qui ont déjà ouvert des traités de Kalam, que dans ces discussions extrêmement techniques, il n'y a pas un guide technique et polémique non plus d'ailleurs parce que plusieurs écoles vont s'affronter, parfois très rudement. 

\paragraph{la Kalam à la différence du Coran est abstrait :  pas de place pour les ours blancs}Et bien dans
ces traités là, il n'y a pas tellement de place pour les écureuils, pour les ours blancs ou pour les autres mastodontes. Plus précisément, on y navigue en général dans un univers d'entités abstraites d'où notre monde concret, avec sa météo, avec ces insectes et sa qualité de l'air, est extrêmement absent. Il y est question de création, bien sûr, dans ces traités, le terme, qui est très employé, mais le terme renvoie à l'acte créateur de Dieu et non pas tellement au ou concret créé par lui dans sa matérialité.
Ce monde concret et matériel est bien présent. Pourtant, dans le Coran, il nous l'a rappelé et fait voir le Coran, d'ailleurs, développe aussi des images apocalyptiques qui peuvent nous donner à réfléchir aujourd'hui. Mais on en trouve à peu près rien dans les textes du Kalam. Ça ne veut pas dire qu'il ne se préoccupe pas du Coran, au contraire, mais il va concentrer son attention sur d'autres aspects du texte sacré des musulmans, et notamment une question de première importance qui saute aux yeux si vous ouvrez le Coran.

\paragraph{Omnipotence de Dieu dans le Coran} Si vous commencez à lire le Coran, c'est un élément qui va vous apparaître immédiatement, même si vous n'êtes pas un spécialiste dans le Coran.
L'omnipotence de Dieu est affirmée de façon absolument massive, totale, inlassablement répétitive. Dieu est puissant sur toute chose. La formule revient comme un refrain. Je crois que je vais trouver une petite quarantaine d'occurrences de cette formule, précise, et bien des versets vont en préciser le sens. Dieu fait ce qu'il veut, il crée ce qu'il veut, il fait le bien.
Il crée le bien, mais aussi le mal. Il châtie et récompense à son gré. Rien n'arrive qu'il ne l'ait voulu.  Il envoie aux hommes bénédiction, mais aussi fléaux et calamités.
\paragraph{Comment réconcilier liberté de l'homme ?}Et cette omnipotence divine, affirmée dans sa version la plus extensive, va poser au premier théologien de l'islam un certain nombre de difficultés, et en particulier parce qu'elle entre en concurrence avec la liberté et la responsabilité de l'homme.
Si Dieu, en effet, crée toute chose et en particulier chacune de nos actions, nous les bonnes comme les mauvaises. Alors, au nom de quoi va t il nous juger ? Au dernier jour, il enverra en enfer les méchants, mais pour des actions qu'il a lui même créées. Bizarre, c'est quand même pas très juste cette tension entre la responsabilité humaine et la toute puissance de Dieu.
\paragraph{Anecdote du Prêt}
On la voit magistralement résumée dans une petite anecdote que je vous raconte, qui est rapportée par un théologien très important du Xᵉ siècle, la charia, qui raconte une petite histoire qui aurait eu lieu en Irak entre un certain Soheib qui avait emprunté de l'argent à un certain Mimoun mais on ne connaît pas les deux par ailleurs. Et Mimoun, évidemment, demande à Soheib de lui rembourser son argent.
t Soheib lui répond \textit{Je te le rendrais si Dieu le veut}. Mimoun n'est pas complètement rassuré et s'appuyant sur une certitude qui est que Dieu veut qu'on rende à chacun ce qui lui revient, lui dit : \textit{Dieu veut que tu me rendes mon argent}. Soheib, s'appuyant sur la toute puissance divine dit : \textit{si Dieu voulait que je demande ton argent, je l'aurais déjà}.

Eh oui ! Ce qui n'est pas sans fin, tout ça, vous voyez. Dans cette petite discussion, on voit l'enjeu essentiel de cette toute puissance divine et on voit qu'en fait les deux ont des bonnes raisons à faire valoir un peu problématique. Alors ils vont aller chercher à s'en remettre à l'arbitrage d'un sage qui est alors en prison et qui va par une réponse écrite, ménager la chèvre et le chou. Ce qui ne permet pas de trancher. 

\paragraph{Conséquence pour la réflexion écologique, chaine naturelle de causes et d'effets}
Cette question théologique et morale est passionnante et je pourrais en parler littéralement des heures, mais je comprend que c'est assez mobiliser l'énergie de théologiens médiévaux. Mais si je vous en parle ce soir, c'est parce que cette envahissante toute puissance de Dieu n'a pas seulement des effets sur la morale humaine. Elle risque en effet de rendre également impossible toute réflexion écologique.
La réflexion écologique, en effet, suppose comme premier préalable l'existence d'une chaîne naturelle de causes et d'effets. Si on réunit la COP 27 , c'est bien parce qu'on constate les effets de l'action de l'homme sur le climat, sur la biodiversité et qu'on espère que d'autres actions ont d'autres effets, ont des effets correcteurs, et cetera Cela nous paraît peut être assez évident cette idée qu'il y a des causes et des effets, mais cette évidence n'est pas partagée.
En effet, la fascination pour la toute puissance divine a pu pousser des penseurs de l'islam, parfois des intellectuels immenses, à minimiser, voire à nier cette chaîne des causes qui régit notre monde. Au profit de quoi ? De l'action immédiate de Dieu, créateur de toute chose à tout instant. Et donc, si ce verre tombe, si je le lâche, ça ne veut pas dire qu'il tombe parce que je le lâche dans cette optique là, parce que ça ferait de moi le créateur d'une réalité.
Cela ferait de moi le créateur de la chute du gobelet. Et certains auteurs vont donc dire non, Pas du tout. Si c'est Dieu qui crée à la fois mon acte de lâcher et la chute du gobelet, si Dieu ne voulait pas activement faire tomber le gobelet, eh bien il ne tombera pas. J'aurais beau le lâcher, le gobelet ne tombera pas.
\paragraph{pour certains, refus de faire une loi de l'univers qui s'impose à Dieu}
Bien sûr, vous avez déjà observé qu'à chaque fois, je pourrais le faire 25 fois devant vous. À chaque fois que je lâche, il tombe et les penseurs dont je parle en sont absolument conscients. Évidemment, ils ont remarqué que ça marchait comme ça, mais il refuse cependant d'en faire une loi de l'univers, la gravité en l'occurrence, qui s'imposerait à Dieu.
Il préfère voir dans la chute du gobelet, à chaque fois qu'on lâche une habitude de Dieu. Dieu qui d'ordinaire fait tomber les gobelets quand on les lâche. Mais c'est toujours un effet de sa libre volonté. C'est lui qui le choisit à chaque fois. Mais voilà, on pourrait tout aussi bien vouloir le contraire, vouloir que ça ne tombe pas sans que cela contrevient à une loi naturelle, physique.
Alors, personne ne le formule comme je vais le faire au Moyen Âge. Mais à l'évidence, une telle conception totalisante de la puissance divine stérilise la possibilité d'une pensée écologique. Si le réchauffement climatique, par exemple, est le résultat de l'action de l'homme et même du péché de l'homme, là dessus, on peut développer une théologie de Pascal au réchauffement climatique. On peut le combattre.
\paragraph{Combattre ou accepter le changement climatique ?}
Peut être même doit on le combattre. S'il est en revanche le résultat de la volonté active de Dieu, alors ce réchauffement, il faut l'accueillir dans la foi, peut être comme une épreuve imposée à l'humanité. Mais on ne peut pas vraiment lutter contre lui, car ce serait alors lutter contre la volonté de Dieu. La seule attitude raisonnable est alors ou pas précisément, comme on le dit parfois, le fatalisme, parce qu'il ne s'agit pas d'une fatalité, mais l'abandon, la contemplation de Dieu dans la réalité qui est devant moi, sans cesse, dans ses beautés comme dans ces calamités qui expriment l'insondable et l'adorable volonté de Dieu.
Mais la théologie islamique classique, la théologie du kalam, a, dans certaines de ses écoles,  essayé de produire des des éléments conceptuels qui rendent pensable cette envahissante toute puissance de Dieu contre la chaîne naturelle des causes et des effets. Mais le même Kalam a aussi vu naître en son sein les plus farouches opposants à cette conception la toute puissance divine, à savoir les représentants de l'école mu'tazilite.

\paragraph{Mu'tazilisme contre toute puissance divine}
Ce courant théologique, né au milieu du VIIIᵉ siècle de l'ère chrétienne, avec en gros l'arrivée au pouvoir des califes abbassides et d'ailleurs, sans doute la première véritable école de théologie au sens propre en islam, qui a cherché à rendre raison conceptuellement de la foi islamique. Ces théologiens ont la réputation parfois trompeuse en Occident, d'être des rationalistes. Je dis trompeuse, car nous entendons facilement par là qu'ils pourraient être un peu athées s'ils sont rationalistes.
Et ce n'est absolument pas le cas. Il s'agit bien de théologiens qui croient que le Coran est la Parole de Dieu, mais qui croit aussi que la raison humaine est un instrument fiable pour comprendre cette parole, pour l'interpréter. l'École mutazilite n'a pas toujours bonne réputation chez les musulmans d'aujourd'hui, du fait d'une importante controverse sur la nature du Coran.
 Mais cette école, qui a du reste disparu du monde sunnite au début du XIIIᵉ siècle, a marqué la pensée théologique islamique à qui elle a donné d'importants instruments de pensée rationnelle. D'ailleurs, depuis le XIXᵉ siècle, un certain nombre d'intellectuels musulmans y font référence et en ont certainement salué l'audace, la liberté de pensée.

\paragraph{pas des partisans du libre arbitre car ne pas penser du point de vue morale} Alors, sur la question de la toute puissance divine. On a voulu souvent faire en Occident des auteurs pour les partisans du libre arbitre de l'homme. Ce n'est pas Dieu qui crée les actes humains, mais bien l'homme qui les produit et qui porte donc la responsabilité de ces mauvaises actions. Alors ce n'est pas complètement faux, mais c'est lire la pensée unique uniquement du point de vue moral, vu la moralité des actions humaines.
Et si ce n'est pas, ce n'est pas complet parce que évidemment ça compte sa dimension morale, mais l'ambition des mu'tazilites est beaucoup plus vaste. Ils ne disent pas seulement que l'homme est capable de faire le bien ou le mal. Ils disent bien une créature est capable de produire un véritable effet dans la création. Ce qui est en jeu, donc plus large, c'est l'autonomie de la création que dureront capables de causes et des faits.
Sans cela, pour eux, la création de Dieu serait incomplète. Elle serait presque fictive. Dieu n'aurait créé qu'un vaste terrain de jeu où sa volonté occuperait toute la place. Au lieu de cela, disent les mots Tassili. Dieu, qui a créé un véritable monde, un monde où entrent en jeu les lois de la physique et la volonté des vivants qui ne sont pas de simples illusions masquant la seule force agissante et la toute puissance divine.
Pour les élites, cette autonomie du monde créé n'est pas une manière d'arracher quelque chose à la puissance de Dieu. Ce n'est pas en concurrence, c'est au contraire la seule manière de comprendre la création qui respecte la grandeur de Dieu et sa bonté. Il n'y a pas, si vous voulez vous créer des espèces de Playmobils avec lesquelles ils joueraient comme un gamin un peu arbitraire et capricieux.
Il a voulu créer des êtres libres et un monde capable de tenir debout. C'est ce qui fait qu'il est véritablement Dieu. Les moutons des élites, je conclus là dessus, ne se doutent pas que l'outillage qu'ils construisent pour articuler la toute puissance de Dieu à laquelle ils croient évidemment, et l'autonomie de la création. Ils ne se doutent pas que cela est indispensable pour bâtir une théologie de la création et de l'écologie.
Leurs préoccupations ne sont évidemment pas du tout les nôtres, mais leur souci de respecter la création dans son statut ontologique nous donne des clés pour respecter cette même création. Dans sa réalité concrète. C'est la richesse presque inépuisable de la théologie islamique. C'est pour ça que j'ai plaisir à la travailler. Malgré son aridité, Les débats qui ont animé les premiers siècles de l'islam ont vu s'exprimer tant de grands esprits, avec tant de liberté qu'on y trouve des ressources propres, inexploitées, souvent contre les impasses du monde contemporain.


 
 


\section{Fabien Révol}


\paragraph{Fabien Révol}
    Fabien Revol est venu mais moi je ne connais pas. Il n'est pas membre de Hideo. C'est un extérieur. Je fais sa connaissance aujourd'hui à venir réagir, réagir à ce que nous disons, à ce que nous faisons ici. Donc, Fabien, vous êtes vous à une formation de biologie et puis une formation de théologie et de philosophie. Vous dirigez le Centre interdisciplinaire d'éthique à l'Université catholique de Lyon et vous vous intéressez à la théologie chrétienne, faut il le préciser ?
Parce que là où nous sommes, le mot théologie évoque plus le kalam que la théologie chrétienne à la théologie de la création. 

\paragraph{Théologies islamiques et chrétiennes}Il m'a été demandé de mettre en lumière des résonances entre la théologie islamique et la théologie chrétienne, toutes deux interrogées par les questions et les thèmes écologiques.
Alors, il y a des résonances à mettre en lumière et sur lesquelles j'ai essayé de mettre un peu le projecteur. L'enjeu de ces résonances, c'est de trouver aussi quelles formes d'originalité peuvent nous être aujourd'hui utiles dans la perspective de la sauvegarde de la création, en résonance cette fois ci avec les enjeux écologiques contemporains, comme par exemple avec la COP 27.

\paragraph{Emerveillement de la Création}
Alors, en préalable, pour comprendre un peu ce que je vais raconter, il me faut identifier un enjeu théologique à finalité éthique. Concernant l'écologie et un enjeu qui me concerne particulièrement dans mon travail. Ce sont les représentations de la nature issues du discours religieux, car ces derniers vont configurer des mentalités et des cultures des civilisations, fussent elles d'origine chrétienne ou islamique.
En ce sens, la thèse du frère Emmanuel est importante dans sa lecture de Ghazali. Je cite cet appel à retrouver les merveilleux mondes de ce qui est donné à l'homme dans la création et sans doute le fondement pour aujourd'hui d'une attitude écologique. Et c'est une thèse pertinente sur cette question de l'émerveillement qui implique la question du regard que l'on porte sur la nature.

\paragraph{Importance du regard sur la nature }
La manière dont on regarde la nature n'est pas innocente. Ensuite, dans la manière dont nous allons nous comporter par rapport à elle, que l'on considère une nature purement de manière profane ou une nature en relation avec Dieu comme création. Mais alors, aujourd'hui, les résultats du GIEC et de l'EEP sur la biodiversité ou les différents travaux des différentes COP, ce n'est pas l'émerveillement qui préside.
C'est plutôt l'imaginaire de la catastrophe qui est censée faire fonctionner les leviers de l'engagement écologique. L'engagement, l'émerveillement est il suffisant ? Est il suffisamment puissant comme levier pour supplanter cette heuristique de la peur à court terme ? Certainement pas. Alors, qu'est ce qu'on fait là ? Si ce n'est pas le cas. Mais justement, ce qui est intéressant ce soir, c'est que finalement, nous ne travaillons pas sur les ressources à court terme pour lutter contre la crise écologique, mais sur les ressources de Sens qui ont été évoquées par mes collègues.
Efficientes dans le long terme et dont il est urgent de bien commencer à travailler leur réception dès aujourd'hui. En effet, les représentations, ça peut être dangereux dans les mentalités humaines. Si je prends un exemple, ce qui a été donné par le frère Emmanuel, il a dit Pour lui, la création est au contraire comprise comme un réservoir infini de biens disponibles pour l'homme.


Mais si on prend ça au pied de la lettre, c'est dramatique en termes de relation à la nature, parce que ça veut dire qu'on peut épuisé toutes les ressources avec l'idée qu'il y en aura toujours et que, parce que Dieu va les fournir sans arrêt. Et donc là, on prend quelques risques pour l'avenir de l'humanité sur la terre.
Donc le site, cette représentation, elle est aussi contextualisée. C'est celle d'un médiéval qui vit dans un monde qui lui semble immense, avec des ressources qui lui apparaissent comme infinies. Et donc c'est cette représentation, Eh bien, elle conditionne sa façon de penser le rapport de Dieu à la création, mais aussi le rapport de l'humain à la création. Donc, je vous dis ça pour dire que les représentations de la nature ne sont pas anodines, tant sur le plan religieux que sur le plan strictement écologique.
Pour penser le rapport humain à la création. Alors, un deuxième propos préalable, c'est peut être pour parler d'un premier, d'un grand absent des représentations dans le discours de mes collègues de ce soir. Je m'excuse de commencer par ça, mais au moins ce sera fait. Justement, il manque la question de l'interdépendance dans les critères qu'ils ont soulevé. Mes collègues vont parler d'écologie.
Le frère Emmanuel a soulevé trois critères. Le frère Adrien Adrien avait un présupposé qui est celui de l'éthique présupposée que l'écologie est d'abord une question éthique. Eh bien, en fait non, c'est le présupposé écologique dans l'histoire de l'écologie, il est d'abord scientifique. Et s'il y a un critère pour penser l'écologie qui est important pour aujourd'hui, et en particulier dans la manière d'articuler théologie et écologie, c'est la question de l'interaction d'interdépendance.
Pourquoi ? Parce que l'écologie se définit dans sa définition originelle de 1866 comme étant la discipline qui s'intéresse relation des êtres vivants entre eux et avec leur milieu. Donc, cela veut dire que l'imaginaire originel de l'écologie en 1866, c'est celui de la science et de la science descriptive, une science descriptive dont on découvre qu'elle a une portée et des portées éthiques absolument incontournables et fondamentales aujourd'hui.
Alors, c'est une question que je pose un peu en aparté à mes très estimés collègues de l'idée ou qu'est ce qui se passe quand on interroge la tradition islamique avec la problématique de l'interaction d'interdépendance, la relation écologique comme clé de compréhension du réel de notre monde ? En d'autres termes, qu'est ce que ça serait si on prenait ? Le tout est lié de l'encyclique Laudato si du pape François et qu'on interroge les traditions islamiques du point de vue théologique.
Je n'attends pas évidemment de réponse ce soir. Je remercie Aziz à Lille d'avoir, à l'oral, tenté un possible rapprochement sémantique avec le conte, avec différents concepts. Il a dit quelque chose que j'ignorais la nature. Le mot nature est pas présent dans le Coran, Dans la Bible non plus d'ailleurs. Donc c'est une forme de premier rapprochement. Ça veut dire qu'une réflexion chrétienne sur le concept de nature doit partir de quelque chose qui n'est pas présent en tant que tel dans ses sources.
C'est la philosophie éthique qui nous a amenés. Et l'autre rapprochement qui m'a intéressé, c'est celui qui est celui d'environnement. Le mot n'existe pas vraiment non plus. Et par contre, il y a un mot qui est utilisé en arabe, si j'ai bien compris, mais qui parle plus de la maison. Et c'est formidable. l'Environnement, c'est un concept qui de plus en plus par tabou, mais en tout cas enfin, en tout cas, ce n'est pas de très bonne presse chez les gens qui parlent vraiment d'écologie.
Parce que le mot écologie est basé sur le concept de maison. Wiko logo en grec, le discours sur la maison, sur l'habitat. Donc si c'est une bonne nouvelle finalement, alors sous cet angle de représentation de la nature, la maison est une issue du discours religieux. J'aimerais relever trop brièvement si lieu de résonance, quelques minutes chacun. D'abord le thème de l'intendant de la création.
Deuxièmement, celui du livre de la nature. Troisièmement, l'autonomie des réalités terrestres. Quatrièmement, la présence de Dieu dans la création. Cinquièmement, la question de l'ascèse en rapport avec la sobriété. Et finalement, la destination est scatologique. Des créatures. Une autre façon de parler de la fin du monde. Alors, d'abord, la question de la représentation de Dieu dans la création, l'Intendant de Dieu dans la Création.
Je remercie Aziz d'en avoir parlé tout à l'heure parce que dans les textes, personne n'en avait parlé. Donc je suis content qu'à l'oral ce soit évoqué et parce que c'est justement un lieu de de résonances très fort avec la tradition chrétienne pour penser le rôle de l'être humain dans la création. Le Coran est effectivement à au moins deux reprises la sourate deux et la sourate six parle de cette instauration du califat humain sur la création, la lieutenance, la vice régence, l'intendance, où, le successeur même de Dieu dans la création, et cela a des implications éthiques très importantes.
C'est une idée reprise dans les hadiths aussi, et de ce fait, au niveau éthique, les condamnations du gaspillage dans le Coran sont compris. Le gaspillage est compris comme un méfait contre la création. Et Dieu n'aime pas que l'homme maltraite la terre. À la sourate deux ou à la source de. Cette idée de méfaits contre la création est présent, alors ça renvoie à la responsabilité humaine.
Dans le premier chapitre de la Genèse, dans la Bible qui est créé à l'image de Dieu, puisque être créé à l'image de Dieu dans cette tradition du Proche-Orient antique, c'est recevoir aussi un mandat de service du peuple qui est ainsi confié au nom, au nom de la divinité, au nom de Dieu, et ce service est confié à toute l'humanité.
Toute l'humanité est faite et créée à l'image de Dieu, et c'est cette, cette création à l'image de Dieu. Un pour implication le service politique de l'intendance de toute la création. Donc je pourrais développer là aussi des œuvres sur sur ce sujet, comme le frère pourrait parler des heures de l'occasion dans la tradition musulmane. Mais moi je le sais, c'est un thème vraiment que je trouve important.
Pour un peu, déminer le terrain de la domination biblique de la création qui est une interprétation purement cartésienne de Genèse un. Et c'est un thème très présent dans la recherche en théologie de l'écologie aujourd'hui. Deuxième point le livre de la nature. C'est ce que j'appelle la dimension théophanie de la création. Le livre de la nature, c'est un vieux et un vieux concept qui remonte à la fin.
Le mot remonte au Moyen Âge, mais l'idée qui recouvre est très ancienne chez les Pères de l'Église. On est dans cette idée que la création nous dit quelque chose de Dieu. Ainsi, il le dit, je le cite la nature, aussi loin qu'elle se trouve dans l'ordre de l'être, n'en reçoit pas moins la lumière de l'être, ce qui permet à chaque chose en particulier l'homme de se tourner vers son principe premier.
Mais ça, c'est tout à fait convergent et congruence avec la tradition chrétienne. Qui est qui ? Qui nous dit depuis même les textes bibliques, que la création est témoin du Créateur, celui qui veut avoir un autre aspect de la connaissance, de la création, en complément où une lumière projetée par les Écritures sur la création pour la connaissance de Dieu.
Il a là cette capacité par son intelligence, de reconnaître les signes de l'existence de Dieu, de sa grandeur, de sa splendeur, de sa beauté. Donc la matière n'est pas indigne d'être le support et la médiation de la lumière divine qui se laisse voir par les yeux humains. Et ça, cela participe en fait de la bonté. Un thème qui va revenir assez souvent maintenant, de la création montée qui va contribuer à construire ce que le pape François appelle dans son encyclique Laudato si sur la sauvegarde de la maison commune, la valeur propre et la valeur intrinsèque des créatures ou de la création.
Quant à lui, le frère Emmanuel a dit je cite ainsi la médiation des actes de Dieu et de nécessité. Et si la création était à la disposition de l'homme ? Ce n'est pas seulement pour satisfaire ses appétits et ses besoins de toutes sortes, mais c'est aussi pour lui permettre de retrouver le chemin du Créateur et d'admirer sa grandeur qui doit engendrer de sa part gratitude et adoration.
Ici, on a plus spécifiquement des résonances bibliques à travers le livre de la nature, des résonances qui sont elles mêmes évoquées par le pape François dans Laudato si. Dans son chapitre sur la bonne nouvelle de la création, le chapitre deux de l'encyclique au numéro 84 de l'encyclique, je cite par exemple Tout l'univers matériel est un langage de l'amour de Dieu, de sa tendresse démesurée envers nous.
Le sol, l'eau, les montagnes, tout est caresse de Dieu au numéro 85. Un peu plus loin, il dit encore Cette contemplation de la création nous permet de découvrir à travers chaque chose un enseignement que Dieu va nous transmettre. Parce que pour le croyant, contempler la création, c'est aussi écouter un message, entendre une voix paradoxale et silencieuse. Nous pouvons affirmer qu'à côté de la révélation proprement dite qui est contenue dans les Saintes Écritures, il y a donc une manifestation divine dans le soleil qui resplendit comme dans la nuit qui tombe.
Ce qui fait écho aux propos d'Assise tout à l'heure, le frère Emmanuel rajoute toutes choses par sa nature, dit dans sa langue la majesté de Dieu. Si ce n'est pas un écho du psaume 18, les cieux proclament la gloire de Dieu. Le firmament raconte l'ouvrage de ses mains. La formulation d'Assise est néanmoins proche de celle que l'on trouve dans la tradition théologique chrétienne, occidentale et orientale.
Dieu se fait connaître donc par ses œuvres, et notamment la tradition franciscaine qui est à convoquer ici. Pour saint François d'Assise, toute créature est une occasion de rencontre du Créateur, car d'après saint Bonaventure, son disciple, elle est porteuse des vestiges du Dieu trinitaire. On pourrait développer sur le rapport entre relations écologiques et vestiges de trinité dans la création, comme le pape François fait dans Laudato si, Mais je n'ai pas le temps.
Je voudrais passer maintenant sur le thème sur le thème de l'autonomie des réalités terrestres. Le frère Adrien propose comme enjeu de réflexion écologique les enjeux de la causalité. Alors on agit en chrétien. Je vous assure que c'est une grosse question, déjà, en particulier dans le registre du dialogue entre science et religion. Mais si je résume votre thèse Frère Adrien, vous nous dites que vous nous proposez de penser que si la causalité est réelle dans le monde, alors oui, la responsabilité humaine, par l'engagement de sa liberté est possible.

Et donc, dire que l'on peut lutter contre la crise écologique a un sens, surtout dans un contexte qui suppose que l'être humain reçoive une vocation à l'aide, à l'intendance ou à la sauvegarde de la création. Alors je suis heureux d'apprendre que le courant des élites semble apte à penser une telle piste de réflexion écologique en régime islamique. Et c'est tout à fait en résonance avec ce que propose déjà depuis quelques décennies l'Église catholique avec le concile de Vatican II.
Au paragraphe 36 de Gaudium et spes, qui s'intitule justement Autonomie des réalités terrestres, je cite le paragraphe 36 sous paragraphe deux ci Par autonomie des réalités terrestres, on veut dire que les choses créées et les sociétés elles mêmes ont leurs droits, ont leurs lois et leurs valeurs propres que l'homme doit peu à peu apprendre à connaître, à utiliser et à organiser une telle exigence d'autonomie pleinement et pleinement légitime.
Non seulement elle est revendiquée par les hommes de notre temps, mais elle correspond à la volonté du Créateur. C'est en vertu de la création même que toutes les choses sont établies selon leurs ordonnances et leurs lois et leurs valeurs propres, que l'homme doit peu à peu apprendre à connaître, à utiliser et à organiser. Donc, on voit bien ce rapport entre causalité de la nature dans son ordre et la capacité de l'être humain à agir dessus en toute responsabilité.
Mais l'exercice de la toute puissance divine en régime chrétien se comprend donc différemment qu'en régime islamique, en tout cas dans le régime de la branche principale du Kalam. Mais c'est toute puissance de bien effective en régime chrétien. Le concile continue, même si, par autonomie du temporel, on veut dire que les choses créées ne dépendent pas de Dieu et que l'homme peut en disposer sans référence au Créateur.
La fausseté de tels propos ne peut échapper à quiconque reconnaît Dieu. En effet, la créature, son créateur, s'évanouit. Du reste, tous les croyants, à quelque religion qu'ils appartiennent, ont toujours entendu la voix de Dieu et sa manifestation dans le langage des créatures, et même l'oubli de Dieu, rend opaque la créature elle même. Donc la toute puissance divine. Elle est à comprendre en une référence à la foi directe.
Si Dieu cesse son activité de soutient de la création, la création retourne dans le néant. Mais c'est une activité qui permet, tout en tenant la création dans sa main, de laisser la création fonctionner selon ses lois propres. Des lois créées par Dieu. Le frère Adrien continue en citant la branche secondaire du Kalam que l'on semble retrouver aujourd'hui. Dieu a créé un véritable monde où entrent en jeu les lois de la physique et la volonté des vivants qui ne sont pas de simples illusions masquant la seule force agissante et la toute puissance divine.
 
Content d'apprendre cela ! Alors, en théologie catholique notamment, on résout, on résout autant que faire se peut ce problème de l'autonomie et de l'acte créateur par la doctrine, une doctrine d'origine aristotélicienne, celle des causes secondes en regard de la cause première, c'est à dire que Dieu, en tant que cause première de toute la création, soutient dans son existence l'exercice effectif des causes secondes, tout en étant présent et actif à l'intérieur d'elle comme soutient et énergisant quelque part.
Alors, quand il dit quand le frère Adrien dit. Mais leur souci de respecter la création dans son statut ontologique nous donne des clés pour respecter cette même création dans sa réalité concrète. J'aimerais bien que ce soit le cas. Moi aussi, j'ai envie de dire ce serait et ce serait bien que cette notion philosophique ait un impact dans la vie concrète des gens.
Mais j'ai bien peur qu'en termes de représentations, ce ne soit pas et c'est un travail long de sédimentation des représentations dans la nature, comme je l'ai évoqué dans mon dans mon introduction. Quatrième point je vais y aller brièvement parce que finalement, il a été évoqué que très peu, mais par mes collègues. C'est le thème de la présence de Dieu dans la création.
Quand, Assise tout à l'heure, nous a parlé de la lumière divine présente sur toutes choses, elle est présente aussi en toutes choses. Elle se manifeste par les créatures elle même. C'est une, et donc non seulement Dieu peut se donner à connaître, mais en plus ils seront, ils seront présents dans cette lumière. Ils parlaient de cette représentation émanant de la création par différents niveaux ou degrés d'émanation jusqu'à la réalité matérielle.
Cela rejoint Nyx Dieu serons présents à travers ces créatures et se donnent à connaître, mais aussi une présence d'intimité. C'est une, donc une présence de Dieu qui appartient également au corpus de la Bonne Nouvelle de la Création défendue par le pape François dans Laudato si. Dans la perspective de l'écologie intégrale. Je cite le paragraphe 88 de Laudato si.
Les évêques du Brésil ont souligné que toute la nature, en plus de manifester Dieu, est un lieu de sa présence. En toute créature, habite son esprit vivifiant qui nous appelle à une relation avec lui. La découverte de cette présence stimule en nous le développement des vertus écologiques. Mais en disant cela, n'oublions pas qu'il y a aussi une distance infinie entre la nature et le Créateur et que les choses de ce monde ne possèdent pas la plénitude de Dieu.
Nous avons ici une présence de Dieu qui affirme la transcendance de Dieu, une transcendance radicale. Et donc on peut dire que aussi, pour les chrétiens, Dieu est toujours plus grand. Je vois que le micro se rapproche, donc c'est bientôt la fin. Il faut que je vous parle encore un petit peu du thème du paradis qui a été évoqué par Aziz et c'était très intéressant de voir que dans son papier, la nature parfaite, elle n'existe pas.
Elle n'est pas de ce temps, elle est, c'est une réalité qui n'est accessible que dans le paradis. Et le mot paradis en grec, signifie le mot jardin. Ça, c'est intéressant de voir que ce qu'il y a une forte convergence, en tout cas dans les représentations de ce que sont les temps de la création nouvelle Les temps du paradis, sous la forme d'une, d'une création réconciliée, d'une création parfaite.
Alors, ce qui est ce qui est ce que je voudrais relever ici, c'est que on ne se rend pas compte, quand on dit ça, que Dieu donne une destination à la création, une destination et scatologique. C'est à dire que ce n'est pas qu'un jardin de décorations qui nous est donné pour l'agrément, c'est que c'est la création qui est aujourd'hui devant nous, même si elle est appelée à une certaine destruction dans une mort, comme par exemple nous le dit la Bible.
Eh bien, il y a une création nouvelle qui qui nous est promise, qui nous parle d'une forme de transformation, de la création en un jardin qui va accomplir la perfection de cette nature qui est en fait une qui est actuellement imparfaite mais qui est appelée à une transformation. Les chrétiens vont dire vont parler de divinisation de la création, c'est à dire que le jardin, il vit de la vie même de Dieu, il est investi de la vie même de Dieu.
Donc cette perfection n'est pas à l'origine, elle n'est pas dans le présent, mais elle est dans un achèvement et une transformation par une action divine. Donc je passe rapidement pour arriver quand même à mon dernier point sur la sobriété et l'ascèse en relation avec la raison d'être des créatures pour un dernier parallèle avec la théologie de sa vie proposée par le frère Emmanuel, c'est une théologie de l'émerveillement qui a une portée éthique.
Et je cite Encore une fois, le frère Emmanuel vient freiner la vision d'une création qui serait qu'au service de l'homme, elle acquiert au contraire une qualité qui implique une reconnaissance, une attention. Pourquoi je me suis ennuyé à vous parler des représentations de la création, à vous parler de la nature comme lieu d'une théophanie, la présence de Dieu dans la création de sa destination et écologique, c'est pour susciter l'émerveillement et pour susciter un changement de regard qui va vous permettre, en fait, de décaler votre façon de penser la nature et la façon dont vous êtes en relation avec elle, et notamment en faisant redécouvrir ce que le frère Emmanuel appelle reconnaissance une attention.
En fait, le sens de la valeur propre et intrinsèque des créatures. C'est le sens de la bonne nouvelle de la création proposée par le pape François dans Laudato si. La finalité de cette reconnaissance, c'est bien l'ajustement de nos relations humaines avec créatures dans la reconnaissance des limites de chacune. Et pour le Ghazali, apparemment, cela passe par l'exercice de l'ascèse.
Excusez moi, c'est le même mot exercice et ascèse en grec. Alors c'est cette vision ascétique du rapport aux créatures. Elle est aussi très présente dans la tradition chrétienne et plus, mais plutôt dans la tradition orientale et orthodoxe, et qui est une ressource spirituelle fortement mobilisée. D'ailleurs, actuellement depuis les années 80 et 90, dont la lutte contre la crise écologique avec ses ressources spirituelles proposées par le patriarche Bartholomée premier et le sens de de cette ascèse en régime chrétien.
Si je reprends les propos du patriarche Bartholomée, c'est s'entraîner à vivre en ressuscités,  c'est mettre en œuvre une forme d'ascèse. Mais qu'est ce que c'est que cette ascèse ? Eh bien, c'est la mise en œuvre pratique d'une réflexion et d'une intégration du sens des limites, de la création, du sens des limites de l'être humain dans la création et de son rapport à Dieu.
L'hubris de l'humain, l'absence de limites de l'humain. C'est quand il se prend pour Dieu qui est celui qui est absolu. Eh bien, retrouver le sens de la limite, c'est retrouver le sens du statut de créature et retrouver les limites des autres créatures afin de pouvoir avoir une relation juste et ajustée avec elle. Et c'est ça le sens de la sobriété dont parle le pape François dans Laudato si.
Retrouver ce sens d'une ascétique au sens joyeux du terme, à finalité écologique. Alors, l'originalité, effectivement, ici, c'est éprouver le manque ou éprouver la faim. Et ça ? Peut être est ce quelque chose qui peut rentrer dans un dialogue un peu dialectique avec la perspective, la sobriété heureuse, parce que la sobriété, elle va dire on peut arriver à trouver satisfaction en ayant une meilleure connaissance de ses besoins réels.
Et dans cette articulation entre besoin et satisfaction, on peut vraiment développer un sens de la joie de vivre. Et donc peut être que c'est un lieu à creuser, cette place de la faim qui n'est pas nécessairement présent aujourd'hui dans les courants de sobriété heureuse. Alors, pour conclure, merci aux organisateurs d'avoir permis cette rencontre. Car si l'écologie a une vertu, c'est d'accomplir sa propre nature.
La mise en relation pour former des écosystèmes vivants. Ce soir, ça marche aussi en théologie. En d'autres termes, je suis convaincu que l'écologie est un lieu privilégié pour faire avancer le dialogue interreligieux aux différents niveaux de ces exercices tant pratiques que spéculatifs. Je le vis déjà de manière intense en contexte œcuménique. Merci de me le faire goûter. Dans le cadre de ce dialogue avec Islam et merci de votre attention.




\section{Conclusion}

\begin{quote}
    Merci beaucoup. Cher Fabien, D'abord pour la reprise de cours, on écoute bienveillante et attentive de chacune de nos trois interventions pour cette reprise théologique, pour aussi nous avoir éclairé sur sur l'apport de la théologie chrétienne contemporaine et notamment du Magistère et du pape François. C'était aussi important dans le contexte de la COP COP 27. Je crois aussi que tu as fait ressortir bien des questions, un certain nombre de questionnements et je crois que c'est ça la nature de la recherche.
La recherche. Bien sûr, elle apporte des réponses, heureusement, mais elle soulève aussi beaucoup et parfois davantage de questions. Et elle invite à poursuivre pour suivre ce chemin. Merci, merci beaucoup et on essaiera évidemment de poursuivre le champ et le questionnement que tu as soulevés. Un dernier mot, je voudrais le donner donner la parole à notre frère, à notre cher ami et frère Jean qui est donc le président de l'Association des amis de l'IDEO et qui nous a fait l'honneur d'être présent ce soir avec nous.
Jean, je te laisse le micro bien non ? C'est avec avec beaucoup, beaucoup d'humilité que je dis quelques mots. Je vous rassure tout de suite, je serai très bref parce qu'on a beaucoup parlé d'ascèse. Mais je sais qu'il y a une certaine contradiction avec le buffet qui nous attend d'abord. D'abord, je crois que la première chose que je dois dire, c'est que je suis sûr d'être, d'exprimer le sentiment de tout le monde.
Un immense merci, un immense merci pour la qualité exceptionnelle des interventions que nous avons eues ce soir et ce merci, je crois et aussi dans un sens à porte, un sens de gratitude parce que je crois que vous avez apporté beaucoup. En tout cas, moi je l'ai ressenti et qu'on se sent un peu plus intelligent, peut être un peu moins bête.
Après vous avoir entendu ce soir apporter vos réflexions sur sur un sujet qui maintenant est devenu central dans la perception à la fois des individus et des citoyens, et les politiques des gouvernements ? Quelles que soient les critiques qui sont apportées sur les insuffisances ou les retards de ces de ces politiques. Au delà de ce merci. Je voudrais souligner que cette réunion de ce soir est l'illustration parfaite de la qualité exceptionnelle de ce lieu et de ceux qui l'animent.
Nous avons tous collectivement la chance d'avoir, avec l'idéologue, un endroit exceptionnel, exceptionnel par la qualité de la réflexion, exceptionnelle aussi par l'ambition humaine qui est derrière qui est celle du dialogue, de la compréhension, du respect de l'autre dans un monde. Je crois qu'il est juste de qualifier de plus en plus intransigeants et de moins en moins apte à l'écoute de l'autre.
Donc ce que vous faites est tout à fait essentiel. Alors cette gratitude est bien évidemment morale. Mais tu m'as invité à faire un appel à ce qu'elle revête aussi un caractère concret. Donc Jean Jacques a rappelé que les amis idéaux avaient été créés en partie au moment de la décision de construire ce bâtiment. Elle a été rendue possible par un legs d'une extrême générosité d'un frère qui a permis la mise en place d'un processus d'investissement.
Je le répète, je le souligne fortement d'investissements qui ne sont pas des investissements spéculatifs, mais au contraire qui sont entourés de toutes les garanties morales et éthiques dans les sommes investies, mais qui nous permettent d'apporter une contribution importante au fonctionnement de l'IDEO. Nous parlons de bibliothèque, une contribution notamment très importante pour l'acquisition d'ouvrages dont les amis de l'IDEO contribuent massivement à l'enrichissement du fonds de la bibliothèque de l'IDEO.
Alors, il est simple pour tous ceux qui ne sont pas membres des amis de l'IDEO bien, vous pouvez dès ce soir pour une modeste contribution que vous je le répète, je l'espère, vous répéter chaque année contribuer au rayonnement et à l'enrichissement. Le rayonnement de cette institution et à l'enrichissement de cette bibliothèque et peut être quand même revenir très brièvement sur ce qu'a souligné Jean-Jacques.
C'est bien sûr un investissement qui a été fait. C'est bien sûr des difficultés concrètes, mais cela a été aussi une merveilleuse aventure humaine que la construction de ce bâtiment. Et donc votre contribution en adhérant aux Amis d'idéaux, c'est aussi participer à une très belle aventure humaine. Voilà, merci de votre attention. Merci infiniment pour ces très beaux mots qui, me vont au cœur et qui j'espère aussi touchent vos cœurs à tous.
\end{quote}