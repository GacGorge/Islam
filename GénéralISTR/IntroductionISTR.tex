\chapter{Rentrée ISTR 2022}

\section{Introduction }
\mn{Xavier Gué, Septembre 2022}
AU sein de la théologie de l'ICP, l'ISTR (Institut de Sciences de Théologie des Religions) affirme son intérêt sur l'autre :
\begin{itemize}
    \item clé de ces religions
    \item regard distancié 
    (Sociologie, Symbole rite des religions)
    
\end{itemize}

mais pas uniquement de la connaissance des religions. Le concile Vatican II va promouvoir un regard positif sur les autres traditions religieuses.
\paragraph{
Sorte de paradoxe de l'ISTR}
\begin{itemize}
    \item Dialogue : l'accueil des autres religions (Nostra Aetate)
    \item Annonce et mission (Ad Gentes : V2 qui promeut l'annonce)
\end{itemize}
Ce paradoxe, on pourrait l'éteindre. 

\subsection{Quelle mission pour l'ISTR}
pas seulement des informations mais modeler les personnes pour qu'elles puissent vivre heureuses dans ce monde.
Christoph Theobald \sn{sur l'alterité de l'autre}
\begin{quote}
    tacitement, on suppose que si la personne avait une connaissance interne du Christianisme, il 
\end{quote}
Il ne faut pas confondre Mission et prosélytisme.

la seconde tentation, ce serait d'absolutiser l'individualisation de l'autre. Nous tombons alors dans la privatisation de la Religion ou une tentation communautarisme.

L'institut se place dans un ni-ni : 
\begin{itemize}
    \item ni prosélytisme
    \item ni aucun rapport aux autres
\end{itemize}
Le mot serait de \textit{l'hospitalité}.  
\begin{quote}
    N'oubliez pas l'hospitalité; car, en l'exerçant, quelques-uns ont logé des anges, sans le savoir. He 13,2
\end{quote}
Dieu nous surprend.


Jésus rencontra de nombreuses personnes et les mit en confiance : \textit{Ma fille, ta foi t'a sauvée}. Ce n'est pas Jésus qui sauve ici, c'est la foi, la confiance créée en et par Jésus.

Chemin singulier de l'ISTR à former à l'hospitalité.


\section{Calendrier}
Début : 12 septembre
Vacances : 
19 novembre : journée d'étude ISTR sur le dialogue islamo-chrétien
15 décembre : Forum des masters

8 - 14 janvier : voyage d'étude ISTR à Rome. Catherine Marin. Travail dans les archives.

Reprise le 16 janvier. 

11 février : "Marie chez les écrivains et penseurs contemporains"

18-27 février :


12-18 mars : Voyage d'étude ISTR et colloque à Rabat (Maroc).

24 mai : journée d'étude ISTR : les relations de maîtres à disciples dans les religions et les spiritualité d'Asie. Approches critiques.

Charbel Atala. 

\section{Licence Canonique}

\begin{itemize}
   \item ouvre à l'enseignement
\item mémoire 1 et 2 : 40 crédit sur 120 crédits
\item à soutenir avant le 30 juin
\item certificat Al Mowafaqa à Rabat
\end{itemize}


B106 Armelle. Mail. prévenir en cas d'hybride. En cas d'absence, validation.  istr.theologicum@icp.fr

\subsection{Sujets potentiels}

\paragraph{Exercices spirituels} questions sur la vision de la religion que cela donne. Philosophie grecque. différence avec mystique. Matteo Ricci décide de ne pas suivre les bonzes centrés sur les Exercices. \sn{\href{https://www.persee.fr/doc/assr_0335-5985_1973_num_36_1_2069}{La Politique de conversion de Matteo Ricci en Chine} sur Matteo Ricci et les Bonzes }

\paragraph{importance du rite} lecture chinoise. 

\section{la théologie de la mission, une tâche oecuménique}

\mn{François Moog Rentrée canonique 2022}
 
\begin{Def}
Des modèles qui évoluent.

Penser théologiquement l'évolution de la mission que l'on constate. 
\end{Def}

Le Concile \textit{Ad Gentes} montre la difficulté de penser le modèle classique (première annonce,... jusqu'à clergé autochtone). Et en parallèle, la question de \textit{France, terre de mission}.

\begin{itemize}
    \item des racines plus profondes. Cf cathéchisme des enfants anciens
\end{itemize}

\paragraph{les hypothèses du cours}
\begin{itemize}
    \item On parle de la mission et non pas des techniques missionnaires. Il s'agit de la pensée \textit{théologiquement}
    \item La penser \textit{oecuméniquement} pour dépasser la tentation des prendre les pratiques des autres : le but n'est pas d'importer les mega churchs. Peut enrichir la réflexion théologique.
\end{itemize}


\subsection{Tradition Protestante}
\mn{Gilles Vidal, Pasteur IPT Montpellier}

Le singulier : dans votre tradition chrétienne (au singulier), une difficulté pour les protestants qui se pensent au pluriel.
5 principes :
\begin{itemize}
    \item \textit{Sacerdoce universelle du Croyant}
    \item sola scriptura, ...
\end{itemize}

Des dénominations(anglican,...) et des courants (libéral, charismatique,...). On se placera plutot au niveau des courants.  

\begin{Def}[Evangélique]
peut être une dénomination ou un courant : ici courant qui se caractérise par 4 points : biblicisme, crucicentrisme, militantisme, et x ?
et un critère historique depuis \href{https://fr.wikipedia.org/wiki/John_Wesley}{John Wesley}.
\end{Def}

\paragraph{une approche historique}

\paragraph{les réformateurs} implanter des Eglises : rechercher la protection politique pour péreniser l'Eglise. Il y a une certaine \textit{passivité} en action missionnaire : ni Luther ni Calvin n'en parle. La mission est close après la \textit{pentecote}.
\begin{quote}
    L'Evangile trouve lui-même son chemin même si les peuples ne sont pas prets à le recevoir. Luther, 1523
\end{quote}
Action de Dieu sans intervention humaine : il s'agit de ne pas gagner son salut par les oeuvres; l'homme n'y est pour rien.

POur Calvin, forte activité epistolaire mais théorie classique de l'\textit{occasionalisme} : porte ouverte ou fermée, comme la prison de Paul. Toujours la méfiance de l'oeuvre. Anti-jésuite ("des sauterelles").

\paragraph{XVII}
XVII\sn{Anabaptistes au XVII avaient aussi une activité missionnaire} : Dans le piétisme, face à l'adhésion aux dogmes luthero calvinistes, petites églises avec une influence personnelles.
Les frères Morave, activité missionnaire, des missionnaires deux par deux.

Mission en Afrique, Asie,... : Signes exterieures de conversion, importance secondaire à la \textbf{connaissance} des Ecritures. La mission ne nait pas du centre mais de groupe \textit{borderline}. 

\paragraph{XIX}
On distingue au XIX la mission et les \textit{missions}. Historiquement, la création des \textit{sociétés de mission}, se base sur la comparaison par William Carrey des sociétés commerciales : 
\begin{itemize}
    \item un comité
    \item des dons
    \item des comptoirs
\end{itemize}
\paragraph{Caractère préventif}
Il fait la distinction entre Evangélisation et Mission. Préserver les africains et asiatiques des vices européens. \mn{Il y a aussi une question sur l'abolisation de l'esclavage, avec la question du salut.}
\begin{quote}
   La polynésie : paradis pour les philosophes français et cauchemars pour les pasteurs Anglais. 
\end{quote}

L'Europe est déjà évangélisé. A charge pour l'Europe de pratiquer(?). 


\paragraph{réaction au XX} La mission est vue négativement : 
Mission est trop connectée avec la \textit{colonisation} en univers protestant. Graduation entre mission et pauvreté.

\subsubsection{le penser théologiquement}
\paragraph{Textes fondateurs}
Avec deux textes qui structurent : 
\begin{itemize}
    \item Jn 3, 16
   \item Mt 18, 16-20 : the great commitment
\end{itemize}

Deux mandats :
\begin{itemize}
    \item le mandat culturel : \textit{cultivez la terre}
    \item le mandat missionnaire
    \item la promesse du ROyaume de Dieu
\end{itemize}
\paragraph{concurrence des sociétés de mission}. Ruineux, du coup, des \textit{conférences de mission}. En 1910,  
Deux mandats :
\begin{itemize}
    \item le mandat culturel : \textit{cultivez la terre}
    \item le mandat missionnaire
    \item la promesse du Royaume de Dieu
\end{itemize}
\paragraph{concurrence des sociétés de mission}. Ruineux, du coup, des \textit{conférences de mission}. 

\paragraph{En 1910, à Edinbourg}, 1200 sociétés de mission se réunissent pour régler le problème de  concurrence. Marqué par le millenariste : en une generation, tout le monde sera christianisé.
Une ligne de fracture entre ceux qui sont respectueux des chrétiens en Amérique du sud et Russie et ceux qui sont pour une mission tout asimuth.
A Edinbourg, nait l’\textsc{œcuménisme} : la doctrine sépare donc on décide de ne pas en parler (deux branches, foi et constitution et la branche du christianisme et sociale : life and work en 1925). On va travailler sur la mission, les questions doctrinales et les questions sociales.
L’œuvre missionnaire va être intégrée dans les deux autres. En 1948, création du conseil Œcuménique. Les orthodoxes rejoignent dès les années 20 le mouvement œcuménique. 

\paragraph{1952 : conférence de Willingen}. Mission Dei. La mission c’est la mission de Dieu, avec une théologie de l’apostolat, où on écarte tout triomphalisme. Kerygma (annonce), koinonia (communion), diakonia (service). Conférence de Mexico, remet en cause l’axe Nord Sud. Mission : action sociale. Challengé par Billy Graham et John Stock vont créer le mouvement de Lausanne par les evangéliques : l’action sociale est seconde par rapport à l’annonce. 

\paragraph{1920 : question de la relation chrétiennes avec les autres religions}. Vision exclusiviste (hors Eglise, point de salut) et une vision pluraliste qui peut reconnaitre dans les autres religions une part de vérité.  \textit{No other Name} (que Jésus). Et un autre publie \textit{no other Name ?}
Tendance récente : Inclusivité de la mission : rôle de l’Esprit dans la mission. Rôle de la guérison. Préoccupation écologique, avec les Luthero-réformés un peu gênés devant « la vie », grand concept un peu vague.
En France, les Luthéro-réformés qui insistent sur le témoignage : accueillir, ouverture. On est passé d’une mission missile à une mission d’hospitalité et d’accueil. Pour les evangéliques, ils reprennent didasko (enseignement) et le discipulat (mot jésuite XVII), condition de disciple. Eglise missionnelle, son essence est d’être missionnaire. 
\begin{Synthesis}[Mission comme Traduction / Transmission / Transformation]
3 T : \textsc{Traduction} (dans les cultures), nécessité d’une herméneutique interculturelle qui inclut les marges \\
\textsc{Transmission} : agir envers les pauvres, des tiers lieux. Il faut transcender le cadre des institutions (témoignage chrétien dans un monde inter religieux). \\
\textsc{Transformation} : Michel Certeau, la transformation du missionnaire. Transformation écologique. \href{https://acteurs.epudf.org/wp-content/uploads/sites/2/2021/02/textes-complementaires-vidal_evolution_de_la_figure_du_missionnaire-11477.pdf}{Benjamin Simon}
\end{Synthesis}


\section{Tradition orthodoxe}
\mn{Georges El Hage, Libanais, orthodoxe arabe, thèse sur la pensée politique d’Origine. ISEO}

\paragraph{Vitesse de la mission} la théologie de la mission orthodoxe, c’est d’arriver en retard. La question : pourquoi aller à la mission alors que tout le monde revient. Le contexte des pays orthodoxes, sous le communisme, ou en cas de persécution, la question de la mission ne se pose pas.
\paragraph{une Eglise anti coloniale} Dans un contexte anti coloniale, favorable des orthodoxes en Afrique. Une question sur la russification ou héllenisation de l’Afrique : pas possible.

\paragraph{Une mission externe ou interne} faut il mieux une nation grecque ou romaine pieuse ? Est-ce que l’Eglise dépend d’un territoire ou dépasse le territoire. Pour les orthodoxes, l’Eglise est Communion, mouvement centripete et centrifuge. Un des paradoxes, appel à tout le monde à la communion mais ne la donne à personne. 

George Roth, le \textit{pneumos spermatikos}.  Reveiller le Christ endormi dans le Coran. 

\paragraph{freins de la mission} Frein de nationalisme. Pietisme individualiste (chacun sauve son âme et s'en fiche, plus de communion), clericalisation \sn{il n'y a que les moines deviennent saint en Orthodoxie}

\paragraph{De nouvelles tensions dans le monde orthodoxe} Jeunes théologiens roumains,... se formant à St Serge à Paris en 1952 : \textit{fédération mondiale des mouvements de Jeunesses orthodoxes } Sysdesmos. \sn{un exemple intéressant du rôle Européen pour l'Islam}

\begin{Synthesis}
Chaque tradition va chercher son charisme propre, par exemple l'Eucharistie pour les orthodoxes.

\end{Synthesis}


\section{Tradition Catholique}

\mn{Xavier Debilly, Mission de France et ICP. These sur Marie Dominique Chenu et théologie et mission}

Père Chenu et Pasteur Groux. Chenu : artisan de V2, signes de temps, mais aussi dans sa préparation dès la fin du XX. Dialogue culturel et inter-religieux. AJOC et Action catholique.
Force est de constaté que des acteurs du concile portaient le souci de la mission catholique et l’oecuménisme. Rencontre et Dialogue, sens dans l’histoire de l’Eglise et dans le signe des temps.

\paragraph{Articuler théologie et pratique}

\paragraph{Importance de Vatican II} positionnement de l'Eglise vis à vis du monde, reconnaissance des autres religions, ... nous trouvons la trace dans les textes du Concile mais aussi des textes des papes.
l'un des tournants majeurs du concile, c'est la \textit{perception de la mission}. Un passage de : 
\begin{itemize}
    \item LG 1 : "l'Eglise ... sacrement". 1964
    \item Ad Gentes, 2
    \begin{quote}
        
    \end{quote}
    on note le passage du ET au EST. 
\end{itemize}
\textsc{L'Eglise nait de la mission}
\begin{quote}
    on ne demande à l'Eglise pourquoi elle est missionnaire
    \sn{le fondement théologique des Missions, lubac, 1942}
\end{quote}
L'Eglise n'a pas élaboré d'abord ses dogmes pour ensuite partir en mission. Le témoignage



Lumen Fidei : 2013 mise en mot grec : la mission nous transforme. L’Eglise nait de la mission. 
Les prolongements Théologiques
Pendant 16 siecles, la mission n’était pas une action (on parlait d’annonce, prédication,…), ce n’est qu’au XVI, qu’on a parlé de mission et d’activité missionnaire de l’Eglise.
La mission n’est pas le simple mandat reçu mais une véritable incarnation, le verbe fait chair parmi nous. (Ad Gentes : introduction au texte du décret).
La mission du Saint Esprit, très important après V2. A permis un renouvellement des pratiques. On quitte une vision « quantitative » et « efficace ».

\begin{Synthesis}
V2 : du matériau pour penser pour la mission. La tradition conciliaire nous position
\end{Synthesis}

Eglise n’est pas pour elle-même mais pour autrui : elle est excentrique
Pape François : être référé au maître qui nous envoie et à ceux et celles à qui nous sommes envoyés.

Pratiques : ferveur théologale du Concile. \textbf{Ne faut il pas considérer la mission comme une inquiétude fondamentale et non une pratique ? } Est-ce le sujet d’un message et donc une pratique ?
Notre rapport à l’altérité : est ce que l’autre est l’objet d’attention ou sujet de relation. Affirmation théologique : si nous croyons que Dieu veut faire de nous une relation aimante et miséricordieuse, 

Cardinal Billé – 2001 conférence : « cette société à aimer ». 

\paragraph{Mission comme Eglise hospitalière}
La conversion du missionnaire de Michel Certeau. Itinéraire spirituel et théologique de ceux qui vivent l’épreuve de la rencontre (Afrique). 
« un dialogue ne s’exprime pas…. Les interlocuteurs « 
Relation : je prends le risque d’une réciprocité en tant que missionnaire.  
\paragraph{Hospitalité}
\begin{quote}
 \textit{Dans cette perspective anthropologique, en quoi l’hospitalité chrétienne diffère-t-elle }?    \sn{Voir Theobald :\href{https://www.la-croix.com/Journal/Lhospitalite-valeur-universelle-lhumanite-2018-06-02-1100943813}{Hospitalité, valeur universelle de l'humanité}}
 
 
Christoph Theobald : D’une part, parce que toutes les rencontres de Jésus, surtout dans l’Évangile de Luc, sont des récits d’hospitalité. Il s’agit de relations caractérisées par la capacité à se dessaisir de soi au profit de la présence à l’autre, ici et maintenant. Ainsi, lorsqu’il envoie ses 72 disciples, Jésus les met en situation de devoir demander l’hospitalité (Lc 10, 7). Ce qui signifie que le propre du missionnaire, c’est d’être accueilli. Il me semble que l’Église aujourd’hui en Europe doit se mettre à nouveau en situation d’être accueillie : parfois, elle se sent tellement chez elle qu’elle veut imposer les traditions chrétiennes comme des présupposés.

\textit{Mais quand on parle d’hospitalité chrétienne, s’agit-il d’« accueillir le Christ » ou d’« accueillir au nom du Christ » ?}


Christoph Theobald : Penser faire quelque chose « au nom du Christ » – formule de l’évangéliste Luc – signifie que l’on se considère comme envoyé par lui. Et c’est effectivement bien souvent la motivation des chrétiens qui accueillent, qu’ils le disent ou pas. Mais il s’agit aussi d’accueillir tout hôte en étant persuadé qu’il est le Christ lui-même (Mt 25, 40).

\textit{Et quelle est l’autre spécificité de l’hospitalité chrétienne ?}


Christoph Theobald : C’est le fait que non seulement l’accueil est inconditionnel, avec priorité aux plus démunis (aveugles, boiteux…), mais aussi qu’il est poussé jusqu’au bout, c’est-à-dire jusqu’à l’accueil de l’ennemi. En ce sens, la Cène est le lieu ultime de l’hospitalité puisque Judas, qui va livrer Jésus à la mort, est totalement accueilli – ce qui rappelle, au passage, que la violence la plus cruelle provient souvent des plus proches. Ainsi, il y a identification entre hospitalité et sainteté : est saint celui qui accueille et qui aime jusqu’à courir le risque de mourir. À travers l’hospitalité, la tradition chrétienne manifeste ce qu’elle est et témoigne d’un Dieu à la fois hospitalier et hôte.. 
\end{quote}
Le missionnaire n’est pas le pivot de la relation. Pas comme uniformité mais comme commission. Mais relation aussi marquée par l’échec, le récit du crucifié ressuscité pas audible. Mystère car cela nous échappe. 
Le terme de \textit{Révélation}, Dieu pouvant se révéler dans un discernement moral, nous permet une liberté d'écouter l'autre. Dimension mystique, nuit de la foi. Et même une gratuité. 

\subsection{Questions}

\paragraph{On a besoin de pratiques} mais est ce que les pratiques ne sont pas comme des rites, des moyens de « contrôler » ce qui n’est pas du contrôle dans le cadre de la rencontre ?

\begin{quote}
    31.\sn{Deus Caritas 31} L’augmentation d’organisations diversifiées qui s’engagent en faveur de l’homme dans ses diverses nécessités s’explique au fond par le fait que l’impératif de l’amour du prochain est inscrit par le Créateur dans la nature même de l’homme. Cependant, cette croissance est aussi un effet de la présence du christianisme dans le monde, qui suscite constamment et rend efficace cet impératif, souvent profondément obscurci au cours de l’histoire. La réforme du paganisme tentée par l’empereur Julien l’Apostat n’est que l’exemple initial d’une telle efficacité. En ce sens, la force du christianisme s’étend bien au-delà des frontières de la foi chrétienne. De ce fait, il est très important que l’activité caritative de l’Église maintienne toute sa splendeur et ne se dissolve pas dans une organisation commune d’assistance, en en devenant une simple variante. Mais quels sont donc les éléments constitutifs qui forment l’essence de la charité chrétienne et ecclésiale ?
    [\ldots]
    c) De plus, la charité ne doit pas être un moyen au service de ce qu’on appelle aujourd’hui le prosélytisme. L’amour est gratuit. Il n’est pas utilisé pour parvenir à d’autres fins[30]. Cela ne signifie pas toutefois que l’action caritative doive laisser de côté, pour ainsi dire, Dieu et le Christ. C’est toujours l’homme tout entier qui est en jeu. Souvent, c’est précisément l’absence de Dieu qui est la racine la plus profonde de la souffrance. Celui qui pratique la charité au nom de l’Église ne cherchera jamais à imposer aux autres la foi de l’Église. Il sait que l’amour, dans sa pureté et dans sa gratuité, est le meilleur témoignage du Dieu auquel nous croyons et qui nous pousse à aimer. Le chrétien sait quand le temps est venu de parler de Dieu et quand il est juste de Le taire et de ne laisser parler que l’amour. Il sait que Dieu est amour (cf. 1 Jn 4,8) et qu’il se rend présent précisément dans les moments où rien d’autre n’est fait sinon qu’aimer. Il sait – pour en revenir à la question précédente – que le mépris de l’amour est mépris de Dieu et de l’homme, et qu’il est la tentative de se passer de Dieu. 
\end{quote}
Un vrai lien entre pratique et théologie proposé par Benoit XVI.

Est-ce que je preche pour ma paroisse ou le Christ ?

\paragraph{Danse} Danse comme moyen d'exprimer les concepts, y compris théologiques. Comment rejoindre l'autre dans sa culture et l'expression de ces concepts. Danse comme forme d'expressoin. 

\paragraph{Se mettre à l'écoute des victimes} Comme théologien, invité à partir de ce qu'on entend des autres. Et non pas édulcorer et mettre de côté, ce que l'on ne veut pas entendre. 

\paragraph{Synchrétisme} Question traduction et acceptation de chacun, sans tomber dans la sujectivité totale : on doit pouvoir reconnaitre en l'autre une personne digne de discuter

\paragraph{Missio Dei} vient d'Ignace.
la mission, c'est quelque chose de moderne (XVI). Si l'Eglise est vraiment missionnaire, alors qu'est ce qui s'est passé avant ?

\paragraph{Kerygme est premier} au sens essentiel (Pape François\sn{Evangelii Gaudium}), le Kerygme est le socle. Après question du pape rançois sur l'originalité de transmettre le kerygme.

\begin{quote}
    164. Nous avons redécouvert que, dans la catéchèse aussi, la première annonce ou “kérygme” a un rôle fondamental, qui doit être au centre de l’activité évangélisatrice et de tout objectif de renouveau ecclésial. Le kérygme est trinitaire. C’est le feu de l’Esprit qui se donne sous forme de langues et nous fait croire en Jésus Christ, qui par sa mort et sa résurrection nous révèle et nous communique l’infinie miséricorde du Père. Sur la bouche du catéchiste revient toujours la première annonce : “Jésus Christ t’aime, il a donné sa vie pour te sauver, et maintenant il est vivant à tes côtés chaque jour pour t’éclairer, pour te fortifier, pour te libérer”.
    Quand nous disons que cette annonce est “la première”, cela ne veut pas dire qu’elle se trouve au début et qu’après elle est oubliée ou remplacée par d’autres contenus qui la dépassent. Elle est première au sens qualitatif, parce qu’elle est l’annonce principale, celle que l’on doit toujours écouter de nouveau de différentes façons et que l’on doit toujours annoncer de nouveau durant la catéchèse sous une forme ou une autre, à toutes ses étapes et ses moments.\sn{\href{https://www.vatican.va/content/francesco/fr/apost_exhortations/documents/papa-francesco_esortazione-ap_20131124_evangelii-gaudium.html#4._Une_\%C3\%A9vang\%C3\%A9lisation_pour_l\%E2\%80\%99approfondissement_du_kerygme}{Evangelii Gaudium} Importance aussi dans le même texte de la créativité.}
\end{quote}

\paragraph{Catéchèse : enseignement, s'appuyant sur mémoire, intelligence et volonté} En Asie, importance de la mémoire. La volonté, scoutisme.

\paragraph{Théorie et pratique}
\paragraph{Penser la mission en fonction des vainqueurs et vaincus} Metz. Du coup, penser le martyr du missiojnnaire comme un moyen de sortir de cette opposition vainqueur/vaincu.

\paragraph{bibliste} mission du coté de l'être et non pas de l'action. pas mal de conséquence. Comment l'exégèse peut définir la mission ? 3 T : parlent aux biblistes. Comment être fidèle à la parole en assumant la diversité ?

\paragraph{Critère d'interprétation de la traduction / trahison} dans certaines traditions polynésiennes, on mange avec les cochons. Et donc, la parabole de fils prodigue fait rire. Mais du coup, on va parler, au lieu d'agneau de Dieu, de \textit{porc de Dieu}, inaudible pour nous.


\paragraph{Liturgie : lieu de l'alterité} On parlait d'oecuménisme, la liturgie est elle un obstacle au chemin missionnaire, à l'oecuménisme ? 
\paragraph{Polarité fondamentale} entre un message à transmettre  et l'écoute/dialogue. On peut parler d'une tension féconde ou plus simplement des pôles. Il y a 50 ans, la liturgie était aussi habitée par ses sujets : Il faut faire une liturgie attractive ? risque de l'instrumentaliser et non de la reconnaître pour elle-même. Or, une réaction de la salle qui dit que la liturgie est \textit{missionnaire} y compris sur son aspect sensible, esthétique. L'esprit précède les missionnaires. Au moment de FX, Eurocentrisme, au risque d'oublier l'Esprit Saint. 
