\chapter{Lire les classiques Chinois}

 

\textbf{Comment lire les classiques chinois ? (French Edition)}

\textbf{Vermander, Benoît}

\section{Prologue}


\subsection{Période axiale}
  \mn{ p. 5 · E. 59  } \begin{quote}
Lorsque la notion même de « période axiale » est acceptée au moins comme
hypothèse de travail ( elle est parfois rejetée dès l'abord ) , les
percées qu'on lui associe sont généralement les suivantes : ( a ) les
penseurs de la période auraient contribué à « problématiser le monde »
plutôt qu'à le recevoir tel quel , ce par le fait d'articuler la réalité
sociale autant que naturelle autour de questions portant sur son «
pourquoi » et son « comment » ; ( b ) c'est ainsi qu'aurait émergé la «
conscience critique » , l'examen des opinions reçues , la remise en
cause d'au moins certaines des structures sociales ; ( c ) ce
questionnement
aurait nourri un souci éthique ( à moins que ce dernier ait contribué à
l'émergence de la conscience critique ) , souci qui fut accompagné
souvent de la formation d'une vision utopique -- la cité idéale de
Platon , la pleine libération des individus dans le bouddhisme , le rêve
d'un ordre social gouverné par les rites , l'éducation morale , chez
Confucius , l'annonce de la venue du Royaume de Dieu chez les prophètes
d'Israël ; ( d ) le surgissement critico - éthique aurait été porté par
des personnalités incarnant les « sauts » entrepris , personnalités dont
le nom est transmis aux générations qui leur succèdent ; ( e ) ces
personnalités sont souvent représentées ( et semblent s'être considérées
elles - mêmes ) comme des « Renonçants » , des individus dont les choix
de vie mettent en question l'ordre dominant : dans la République , le
philosophe de la cité idéale de Platon renonce aux possessions privées
et à la vie de famille pour gouverner la Cité avec justice

\end{quote}  

\href{https://fr.wikipedia.org/wiki/William_Marx}{William Marx} dans sa leçon inaugurale sur les bibliothèque et la notion de Canonisation note : 
\begin{quote}
    Le Christianisme n'a pas de récit fondateur du canon comme celui d'Esdras pour le judaïsme ou en Islam de la nuit du destin. En revanche, il a la Pentecôte, qui reprend des signes non seulement de Shavouot, mais aussi de Soukkot, la fête du don de la loi [\ldots]
    Il est intéressant de noter que le terme de canonisation utilisée par l'Eglise catholique renvoie à la canonisation des Saints, et donc la canonisation de l'action.
\end{quote}
\subsection{Divination, mathématique et concepts}
\mn{- p. 17 · E. 258  } \begin{quote}  

Élaborer des ramifications entre les principaux viscères du corps humain
et les états climatiques ( le chaud , le froid , le sec , l'humide ) ,
les émotions , les notes de musique , les couleurs , tout cela transcrit
dans une arithmétique de
base cinq » , n'est pas simple opération de l'esprit mais permet bien
d'observer pragmatiquement un ensemble de correspondances . En même temps, ces analogies deviennent moralement signifiantes. C’est ainsi que s’opère une évolution du divinatoire au combinatoire et à l’éthico-politique.   
A la fois matrice de ce processus et l'accompagnant tout au long de sa
formation textuelle , le \textit{Classique des mutations ( Yijing )} construit un
métalangage de l'univers à partir d'une réflexion mathématisée sur les
processus de divination.
il s'essaie à décrire la façon dont les phénomènes se suivent et
s'agencent de façon nécessaire . C'est , à l'extrême , une mathématique
de tous les phénomènes possibles , fondée sur l'idée que tout n'existe
que dans l'échange , le passage , la fluidité : la gradation est un
autre nom du contraste , et la logique des transformations ( des
changements par quoi une forme entre en contraste avec celles qui la
précèdent et la suivent ) , non pas celle des oppositions , régit
l'univers et la pensée .

\end{quote} \mn{ p. 18 · E. 268 } \begin{quote}

Au ciel se composent les images {[} xiang {象} {]} , sur la terre se
composent les formes {[} xing 形 {]} , et les transformations se font
visibles {[} Yijing , Xici 繫辭 I . 1 {]} 28 .

\end{quote} 

\paragraph{le rite comme éducation} chez Confucius
\mn{p. 20 · E. 304  } \begin{quote}

Le vecteur de l'éducation ainsi dispensée , ce sera le rite
: l'éducation de l'intérieur par l'extérieur . À noter pourtant que « le
rite » ( chinois ) et « la loi » ( en contexte grec ou hébreu ) ne se
trouvent pas dans un strict régime d'opposition : ils régulent tous deux
les conduites extérieures jusqu'au moment où les prescriptions se
retrouvent être inscrites « au fond des cœurs » .

\end{quote} \mn{ p. 24 · E. 367 } \begin{quote}

La détermination d'un corpus de classiques fonde une culture dans sa
spécificité tout en lui donnant d'affirmer son l'universalité latente .
Hans - Georg Gadamer fait de l'extensibilité ,  en principe illimitée, de la durée pendant laquelle une œuvre communique son sens le trait qui désigne un « classique »
\end{quote} 
 
\section{Chapitre I : Qu'est-ce qu'un classique chinois ?}

\mn{ p. 37 · E. 786 } \begin{quote}

Du reste , dès les premiers rencontres entre lettrés chinois et
missionnaires occidentaux , les uns et les autres comprennent que le
continent qui leur fait face est fondé sur une armature de classiques
dont la maîtrise est essentielle pour la réussite culturelle ,
religieuse et sociale , et ils feront de la connaissance réciproque de
ces textes la clé d'une rencontre possible , d'une entrée dans la pensée
de l'Autre , que ce soit pour l'apprécier ou l'utiliser , pour
convaincre ou dominer l'interlocuteur .

\end{quote} \mn{p. 38 · E. 805  } \begin{quote}
On trouve cette acception chez Yang Xiong 揚雄 (53 AEC-18) :

 
 \begin{quote}
     : « Un texte qui ne se mesure pas aux Classiques n’est pas un texte. Une parole qui ne se mesure pas aux Classiques n’est pas une parole. Ce sont des excroissances qui prolifèrent  
 \end{quote}
La formule, très concise, pourrait aussi être comprise comme signifiant : « Un texte qui n’est pas (qui ne tend pas au) Classique n’est pas un texte. » Ou même plutôt : \textit{« Un texte non référé à une norme n’en est pas un. »} Le registre sémantique du passage 11 rappelle celui qui entoure le terme déjà évoqué κανών : l’instrument qui norme (la règle du charpentier en grec) donne naissance à un ensemble textuel établi par rapport à lui – ensemble qui, de normé, devient alors normatif.
 
\end{quote} \mn{p. 44 · E. 926  } \begin{quote}

Pouvoir {[} di {]} d'en haut {[} shang {]} » surplombe le monde
surnaturel ( ne parlons pas d'un « panthéon » ,car on n’y trouve pas l’individuation des divinités qu’on attache d’ordinaire à ce terme).

 

\end{quote} \mn{ p. 45 · E. 951 } \begin{quote}

L'autre chapitre des Documents sur lequel nous nous arrêterons
brièvement est celui du Hongfan 洪 範 , expression que l'on peut
traduire par « Grand plan » ou , sans doute plus justement , par «
Modèle universel » . Il présente l'ensemble des phénomènes ( éléments ,
conduites , matières politiques , calendrier , sources de bonheur et
malheur ) sur des combinaisons fondées sur les chiffres Cinq et Neuf .
Sa composition apparaît si rigoureuse que certains formalistes russes
ont spéculé sur son nombre de caractères et leur inscription à
l'intérieur d'un carré magique 30

\end{quote} 

\paragraph{les annales historiques} attribuées à Confucius : essayer de dicerner les signes des temps. 
\mn{ p. 56 · E. 1164 } \begin{quote}
les lecteurs de ces Annales vont s’employer à faire émerger des sens, à trouver l’expression du mandat du Ciel dans cette succession d’événements qu’ils vont lire comme une sorte de rébus. C’est dire que ce sont les commentaires suscités par l’œuvre qui en font tout l’intérêt.
[..]
Au travers de la sèche prose des Printemps et
  Automnes , le Gongyang s'emploie à trouver « le sens profond de paroles
subtiles"
\end{quote}


Importance du processus de mémoire.

 \mn{p. 59 · E. 1228  } \begin{quote}

Au travers de cet hommage rendu par les historiographes du Zuozhuan à
leurs prédécesseurs se donne à lire leur conviction : l'enregistrement
et la qualification fidèles des événements sont questions de vie ou de
mort . Cela parce que l'enchaînement des actions , des prédictions que
l'on en tire , de la vérification ou non de ces prédictions constitue la
trame de l'ordre du monde , et qu'en altérer le récit , c'est forcément
en distordre la marche .

\end{quote}

\paragraph{Anectdote : matériau de la connaissance historique} car elle offre un aperçu soudain sur une réalité plus profonde et étendue, comme la citation. \mn{ p. 59 · E. 1233 } \begin{quote}
Du fait de la nature même de ces deux extraits, on aura noté la place que joue, dans le Zuochuan, le genre littéraire très particulier de l’anecdote.
 
Elle est révélatrice d'une tendance plus générale : l'anecdote , dans
l'historiographie chinoise , deviendra le matériau même de la connaissance historique.
 

\end{quote} 
\subsection{le sacrifice comme lieu de l'unité}
Une partie des textes classiques visent à expliquer la tenue en cas de non respect du sacrifice. Cette importance peut étonner.
\mn{p. 65 · E. 1345  } \begin{quote}

C'est en distinguant qu'on peut droitement unir .
Le

mode de découpe de l'animal sacrifié puis la façon dont on répartit les
parties découpées constituent des opérations destinées à symboliser et
réaliser la cohésion du groupe , dans un modèle qui associe exigence
d'unanimité et affirmation de principes hiérarchiques stricts . 

\end{quote} \mn{ p. 66 · E. 1353 } \begin{quote}
Le sacrifice
était donc image inversée de l'ordre social effectif . La supériorité
sociale était étroitement associée à la capacité de donner .

\end{quote} \mn{  p. 66 · E. 1356 } \begin{quote}

Les anomalies rituelles relevées par les textes fonctionnent alors comme
informations historiques d'importance particulière . La crise rituelle
est toujours symptôme de crise politique .

\end{quote} \mn{ p. 66 · E. 1357 } \begin{quote}

le Liji traduit une anxiété face à tous les dysfonctionnements possibles
de l'activité rituelle , et , de ce fait , du travail social .

\end{quote} \mn{p. 67 · E. 1368  } \begin{quote}

Le rituel est donc affecté par une tension entre son but avoué -- édifier ,
maintenir un ordre -- et le fait qu'il soit par essence une entreprise
risquée , soumise à divers aléas .

\end{quote} 
\subsection{La grande Etude - Daxue}
\mn{p. 67 · E. 1384  } \begin{quote}

Tentons après tant d'autres une traduction de cette célébrissime entrée
en matière -- sachant qu'aucun lecteur ni commentateur n'arrive à faire
vraiment abstraction de la lecture qu'en ont effectué Zhu Xi et ses
continuateurs . 
\begin{quote}
    La voie de la Grande Étude 96 , c'est d'éclairer {[} le
principe de {]} vertu qui {[} lui - même {]} éclaire {[} toute chose {]}
, c'est d'aimer {[} ou bien de « renouveler » , selon les versions {]}
le peuple , et c'est de ne s'arrêter qu'une fois la plus haute
perfection atteinte . Sachant le terme , on prend sa détermination . Une
fois déterminé , on peut s'apaiser {[} jing 靜 97 {]} . Une fois apaisé
, on peut s'ancrer dans la paix . Quand on est ancré dans la paix , on
peut discerner . Ayant discerné , on peut atteindre le but . Les plantes
ont racine et branches , les affaires un commencement et une fin . Celui
qui sait la succession des affaires , il est tout proche de la voie .
大學之道,在明明德,在親[新]民,在止於至善。知止而后有定,定而后能靜,靜而后能安,安而后能慮,慮而后能得。物有本末,事有終始,知所先後,則近道矣。


\end{quote}


\end{quote} \mn{ p. 69 · E. 1411 } \begin{quote}

La partie gauche du caractère , sa clé sémantique , c'est le jade ( yu 玉 )
. Associée à l'autre partie du caractère , cette clé donne à l'ensemble
un tour plus abstrait : elle parle d'une structure sous - jacente aux
phénomènes qui prennent corps . Les Chinois ont trouvé dans les nervures
du jade la plus marquante des expressions des forces vitales qui , à
l'interne , structurent l'organisme.

\end{quote} 
\paragraph{L'invariable milieu, à l'autre extremité des quatre livres}
\mn{ p. 69 · E. 1420 } \begin{quote}
Comme pour la Grande Étude, voyons à traduire le premier paragraphe du texte :
\begin{quote}
La loi du Ciel 101, c’est ce qu’on appelle « la nature » 102. Se conformer à la nature, c’est ce qu’on appelle « la voie ».
\ldots

   Aussi , l'homme de bien se tient sur ses gardes même quand il ne
distingue rien {[} d'inquiétant {]} , il craint et tremble même quand
rien ne frappe son oreille . 
 
\ldots

Aussi , l'homme de bien reste sur ses gardes {[} même {]} dans la
solitude 104 . N'émettre {[} fa 發 {]} ni joie ni colère ni tristesse ni
allégresse , c'est ce qu'on peut appeler l'équilibre {[} le milieu {]} .

\end{quote} 
\end{quote}

Relecture

\paragraph{Xunzi : la discrimination fait l’homme
 }
\mn{ p. 72 · E. 1482 } \begin{quote}

On a argué , à mon sens fort justement , que le caractère fen 分 (
séparer , diviser ) constitue une clé -- sinon la notion clé -- du Xunzi
112 . Le fait que l'on compte 113 occurrences du caractère dans
l'ouvrage est un premier signe de son importance .

\end{quote} \mn{ p. 73 · E. 1493 } \begin{quote}

le principe de base sur lequel Xunzi fonde sa théorie politique est que
les désirs ( par essence subjectifs ) sont par nature illimités , tandis
que les biens sont objectivement limités ( yu duo er wu gua 欲 多 而 物
寡 113 ) . Savoir comment répartir et partager , c'est bien là l'essence
de l'art politique 114 . Xunzi pose ici l'accent sur la notion de gong
公 : gong ( habituellement traduit par « chose publique ») consiste à diviser quelque chose avec impartialité, sans afficher égoïsme ou partialité (si 私).
 

\end{quote} \mn{ p. 74 · E. 1510 } \begin{quote}
Le rituel, manifestation privilégiée des distinctions ainsi effectuées, devient alors le conduit par quoi ce qui était au départ
de l'ordre de la nature ( xing 性 ) se transforme en artifice ( wei 偽 )
, ce dernier terme revêtant un sens éminemment positif
 
\end{quote} 
Sens positif de l'artifice chez xunxi.
\begin{Def}[artifice]

\end{Def}
\mn{ p. 74 · E. 1522 } \begin{quote}

Xunzi trace une distinction entre les niveaux atteints respectivement
par le Lettré ( shi 士 ) , l'homme de bien ( junzi 君子 ) et le sage (
shengren 聖人 ) , ce dernier seul capable de fusionner en un tout la
perfection des dispositions à l'interne et celle atteinte par les
observances extérieures .



\end{quote}


Cf vie digne et vie parfaite

\mn{p. 75 · E. 1527  } \begin{quote}

Les châtiments contribuent à assurer l'équilibre entre hiérarchie et
impartialité .

\end{quote} \mn{ p. 76 · E. 1550 } \begin{quote}

Xunzi voyait bien la nature humaine ( xing 性 ) comme mauvaise ( e 惡 )
, il discernait dans l'artifice rituel couplé à la droite éducation la
route par laquelle s'ouvrait la possibilité de dépasser l'inné , jusqu'à
rendre chacun capable de devenir -- en théorie -- semblable aux sages souverains du temps passé. Route trop dangereuse pour le Hanfeizi, quand les États font face à des risques imminents (la principauté dont Han Fei est un dignitaire éminent va disparaître trois ans après sa mort).
 
L'art politique ne consiste pas à éduquer le peuple , que ce soit par
les rites ou par la vertu , mais à instaurer un ordre durable au moyen
des châtiments et récompenses attachés à des lois valables pour tout un
chacun . Notons que
les Légalistes systématisent ici une vue politique apparue bien
auparavant , mais à laquelle les penseurs classiques avaient jusqu'alors
été à même de s'opposer : le Zuozhuan relate qu'en 536 AEC , l'État de
Zheng 鄭 fonde dans le bronze un code des châtiments . Il reçoit une
algarade d'un ministre de l'État de Jin 晉 , qui rappelle à tous que les
anciens rois comptaient , pour assurer l'ordre public , sur le rituel ,
le sens de la justice , l'attrait des positions officielles , l'exemple
donné par leur dévouement constant \ldots{} Ils savaient que l'existence
même d'un code intangible ne pouvait que corrompre la fibre morale d'un
peuple 116 . Pourtant , vingt - trois ans après cet épisode , l'État de
Jin cède à la même tentation 117 \ldots{} Nous approchons là de la fin
des récits du Zuozhuan :

\end{quote} \mn{p. 77 · E. 1580  } \begin{quote}
Ou encore comme le roi-sage taoiste,  \begin{quote}
    il
agit par le non agir ( wei wu wei 為 無為 ) 
\end{quote}, mais s'il est en mesure
de procéder ainsi c'est parce qu'il a institué l'artifice d'un corps de
loi simples , compréhensibles , adaptées au temps présent et cependant
universelles par leur inspiration , des lois qui , en quelque sorte ,
gouvernent à sa place -- pour autant qu'elles soient conformes à l'ordre
naturel : Les choses ont leur usage , les talents leur emploi . Lorsque
tout est en place , alors on n'a pas à s'activer du haut vers le bas .
Ordonnez à un coq de proclamer l'aube , ou à un rat de chasser les
souris ! {[} Hanfeizi , Yangquan , 1 {]} .

\end{quote} \mn{p. 79 · E. 1614  } \begin{quote}

Voyant un teinturier teignant la soie , Mozi dit tout en soupirant : Ce
qu'on teint en bleu devient bleu , ce qu'on teint en jaune devient jaune
. Quand on change {[} la teinture {]} dans quoi l'étoffe est immergée ,
change pareillement sa couleur . Trempée cinq fois , sa couleur
nécessairement changera cinq fois . Lorsque l'on teint , on ne saurait
être trop prudent {[} Mozi , Suoran , 1 {]} . 子 墨子 言 見 染 絲 者 而
歎 曰 : 「 染 於 蒼 則 蒼 , 染 於 黃 則 黃 。 所 入 者 變 , 其 色 亦
變 。 五 入 必 而已 , 則為 五色 矣 。 故 染 不可 不慎 也 。 」

\end{quote} \mn{p. 82 · E. 1666  } \begin{quote}

La partialité insère des négations dans une réalité de soi toute marquée
par l'affirmation . En contraste : l'attitude inclusive ( impartiale )
est toute positive :
\begin{quote}
    « Je fais pour l’autre comme pour moi-même » (wei bi wei you wei ji ye 為彼猶為己也, Mozi, Jian’ai, III, 2).
 
\end{quote}
\end{quote} \mn{p. 83 · E. 1695  } \begin{quote}

Pareille étincelle pourrait être par exemple la « découverte » faite par
Mozi que l'amour n'apporte bénéfice effectif que pour autant qu'il
s'abstienne de discriminer , de hiérarchiser . Il est plus aisé
d'imaginer que pareil basculement fut à l'origine de l'attraction
exercée par Mozi que d'y voir un développement « naturel » ( on ne voit
pas du reste en quoi il serait tel ) d'un appel initial à la simple
bienveillance .

\end{quote} 
\paragraph{des écrits pour se taire}
\mn{ p. 84 · E. 1701 } \begin{quote}

EN décalage peut - être plus grand encore que les textes mentionnés à
l'instant , dans la mesure où leurs principes épistémologiques diffèrent
profondément de ceux défendus par les lettrés ( shi 士 ) , le Daodejing
道德 經 ( ou Laozi 老子 ) et le Zhuangzi 莊子 comptent parmi les textes
les plus influents de la pensée chinoise -- et universelle .
\end{quote}