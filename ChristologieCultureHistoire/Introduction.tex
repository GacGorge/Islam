\chapter{Introduction}

\mn{Xavier Gué, 2022, Christologie, Culture dans
l’histoire 
Mardi de 17h à 19h les 18/01 ; 25/01 ; 1/02 ; 8/02 ; 15/02 ; 8/03 ; 15/03 ; 22/03 ; 29/03 ;
5/04 ; 12/04 ; 19/04.
 }
 
 \section{Plan}
 
 \begin{itemize}
     \item A. Prolégomènes I : déconstruction
     \item B. Prolégomènes II : une théorie de la tradition christologique.
     \item C. La christologie façonnée dans la tradition d’Israël : Les christologies du NT
     \item D. La christologie se définit dans le monde grec I (IIe
-IIIe s.)
     \item E. La christologie se définit dans le monde grec II (IVe
-Ve s.)
     \item F. La christologie se définit dans le monde gréco-romain III (Ve s.)
     \item G. La christologie médiévale et moderne façonnée par la « romanité »
     \item H. La christologie dans une culture moderne marquée par l’histoire
     \item I. La christologie anthropologique dans un monde moderne sécularisé
 \end{itemize}


\section{Validation du cours}
Merci de suivre les normes universitaires, voir le document joint : « Normes de présentations
de mémoires »
Vous avez deux possibilités pour valider le cours :
\paragraph{1. Travail à partir de l’ouvrage de M. SACHOT}\mn{\cite{Sachot:InventionChrist}}
M. SACHOT, L’invention du Christ. Genèse d’une religion, Paris, Odile Jacob, 2011
.
a) Choisir une partie de l’ouvrage
- Le mouvement chrétien, homélie du judaïsme, p.13-100
- Le mouvement chrétien, philosophie dans le monde hellénistique, p. 101-162
- Le christianisme, religion romaine, p.163-225.
b) En faire un résumé (5 pages)
c) Evaluation personnelle (2-3 pages)
\paragraph{2. Rédaction d’une dissertation de 8 pages}
Vous choisissez une question montrant comment la christologie est interrogée par une
culture/autre tradition religieuse. Vous travaillez en vous appuyant sur un article ou un autre
texte.
Votre sujet et le texte doivent être au préalable validés par le professeur.
Votre travail écrit doit être déposé sur l’ENT (espace dédié) et la date limite pour la
remise de votre travail est le 8 mai 2022

\paragraph{Alternative} regarder une approche christologique dans une culture, par exemple perse ou chinoise. 
bien montrer la culture chinoise et comment elle s'adapte à cette "langue", cette culture.
%----------------------------------------------------------------------------------------------------------------------------------------------------------------------
\chapter{Prolégomènes I : déconstruction}

%----------------------------------------------------------------------------------------------------------------------------------------------------------------------
\section{Introduction générale}

%----------------------------------------------------------------------------------------------------------------------------------------------------------------------
\section{La Christologie commme un savoir sur le Christ}
\paragraph{Post-Modernité}
\begin{Def}[Post-modernité]
Contexte où les grands récits fondateurs ne sont plus opérants, créant des vérités partiels
\end{Def}
\mn{J.F. Lyotard, Lipovesky \textit{Les post-modernes se situent dans la perspective de surmonter le désenchantement du monde, après la désagrégation des repères culturels ou religieux, le relativisme des sciences, la crise de l'idée de progrès, l'humanité confrontée aux faillites écologiques, économiques et sociales, et l'échec patent des utopies révolutionnaires.}}

risque de relativisme


\paragraph{universalité vs contre-culture}. Soit on dit que le Christ est l'unique sauveur, on développe une contre-culture chrétienne, qui n'est plus en dialogue avec le monde, soit on vise l'universalité mais avec le risque de relativisme du Christ.


Bernard Sesboué  :
\begin{quote}
    Question de l'unicité du Christ. Jésus de Nazareth, unique médiateur.
    Ac 4,12
    Ces affirmations neo-testamentaires semblent refuser le dialogue avec les autres religions.
    Question théologique la plus forte du XX\textsuperscript{ème}
    \sn{Sesboué, Introduction à la Théologie, 2017, p. 202-203}
\end{quote}

\paragraph{Passer par l'histoire} reconnaître que la christologie s'est toujours construit dans un dialogue avec les cultures. A chaque génération, il nous faut reinterpréter notre foi, redire notre foi en Christ dans le contexte d'aujourd'hui.

\begin{Ex}[Consubstantiel]
peut être plus juste mais incompréhensible.
\end{Ex}

\paragraph{Jésus-Christ ne va pas de soi}, la christologie vient d'un dialogue.

\section{La christologie comme un savoir sur le Christ}

Culture comme langage, si une langue vit sans culture, elle meurt.

\subsection{Le traité du verbe incarné}
\paragraph{Le traité du verbe incarné} On enseignait non pas la christologie mais le traité du Verbe incarné. Le terme christologie est apparu au XX. On pensait que dès le début, l'identité de Jésus était bien définie. Avec les temps modernes, on redécouvre l'\textit{écart} entre Jésus et le Christ. Pour connaître qui est le Christ, il faut non seulement une catéchèse mais un dialogue, une contemplation. Historiquement, cela s'est aussi passé comme cela.

\paragraph{Historiquement, une approche d'en haut} Christologie descendante, partant de la vision de Dieu. Dès le début. Christologie johannique, épiphanique. On ne tient pas compte de l'histoire, l'histoire est le support de cette manifestation. Entre Jésus et Dieu, pas de discontinuité, Jésus est le Christ dès le début et il l'a révélé au monde.  

\paragraph{Le problème est que sa mort et sa résurrection n'ont rien à faire avec le Christ}. Son histoire ne nous dit pas qui il est puisque nous savons ce qu'il est dès le début. Or, on ne peut pas dire qui est un homme avant sa mort, \textit{du fait de sa liberté}. L'identité narrative \sn{Ricoeur, Temps et Récit. L'homme dit qu'il est à travers son récit.} 

\paragraph{Cette christologie va exploser avec la modernité}


%----------------------------------------------------------------------------------------------------------------------------------------------------------------------
\section{la prise de conscience entre Jésus historique et le Christ de la Foi}




\subsection{L’origine d’une conscience nouvelle}
\paragraph{Question récente} jusqu'au XVIII, on pensait que l'Evangile racontait l'histoire. On ne faisait pas de différence entre Jésus et le Christ. Or, Jésus-Christ est \textit{kerygme} et acte de foi.

\paragraph{On sépare le Christ des Evangiles} par rapport au christ des dogmes, puis le Christ des Evangiles du Jésus de l'histoire. Prise en compte de la distance. Ces questions sont récentes mais pas récentes-récentes. 
\bi 
\item au XVI : Lelio Sozzini (1525, 1562), le Sozzinialisme, mouvement qui remet en question la Trinité (et a donné ensuite le mouvement unitarien, Michel Servet\sn{\href{https://fr.wikipedia.org/wiki/Michel_Servet}{Notice Wikipedia de M. Servet}}, brulé par les Calvinistes à Genève). Dieu est unique. Tout un travail sur les sources scripturaires pour contrer ce mouvement.
\item Reimarus \sn{\href{https://fr.wikipedia.org/wiki/Hermann_Samuel_Reimarus}{Reimarus - notice Wiki}}. Une partie de ces oeuvres publiée par Lessing en 1778, un christ politique pour prendre le pouvoir, mais il est arrêté et crucifié. Les disciples continuent la lutte mais de façon spirituelle, en en faisant une religion. Reimarus part des \textit{contraduction de la foi}. 
\ei 

\subsection{La recherche historique sur la vie de Jésus au XIXe}
Pour répondre à ces questions, plusieurs périodes : 

\paragraph{LebensJesusForschung} Recherche qui va mobiliser beaucoup d'énergie. On va opposer deux approches, soit on rebâtit notre foi sur l'histoire (LebensJesuForschung),  cachée par les dogmes et les traditions, soit on part du Jésus de la Foi et c'est cela qui compte. Enlever la gangue théologique et mythique. Idée de la pêche : le vrai Jésus est au milieu et autour des couches, les disciples, l'Eglise, mon curé : il faut percer les différentes couches pour arriver au noyau. 

\paragraph{Théologie libérale} au sens de reprendre sa liberté au nom de la raison. Geert Theissen, p. 47. 
\begin{quote}
    
    un nimbe... qui le transfigure. 
\end{quote}
\begin{Ex}
on a un peu la même chose pour de Gaulle : tout le monde se réfère à lui, on en a fait un mythe. 
\end{Ex}

\paragraph{L'échec des vies de Jésus} Après avoir déconstruit, il faut reconstruire mais la difficulté et de ne pas projeter dans la reconstruction sa propre vision. 
\begin{Ex}[vision de F. Schleiermacher]
Jésus était tellement en lien avec Dieu qu'il vivait le Royaume de Dieu intérieurement, qu'il a transmis à ces disciples.
\end{Ex}

\paragraph{Réaction : idée religieuse} ou christologique du sauveur. Approche idéaliste. Jésus  ne fait qu'incarner un idéal qu'il nous faut incarner à notre tour. 
\begin{Ex}[E Kant]
L'homme agréable à Dieu qui a mis en action la morale universelle. Maintenant qu'on connait la morale, on n'a plus besoin du Christ. On le détache de son histoire. 
\end{Ex}
David Strauss et au XX, R. Bultmann, développent ceci, mais avec le risque de l'idéologie. 


\subsection{La « deuxième quête » : de 1953 à 1985}

\paragraph{Une troisième periode} On ne peut pas s'arrêter au Kerygme, il faut montrer le lien entre le Christ de la Foi et le Christ historique. La grande figure est Käsemann (1953-1985). Il ne s'agit pas de faire une biographie historique mais de s'assurer du passage du Jésus de l'histoire au Christ de la Foi. On développe une méthode pour valider ce qui est authentique.

\paragraph{Les critères d'historicité}
\begin{Prop}[Les critères de validité]
Différents critères ont été développés : 
\bi
\item Critère d'embarras. 
\item Critère de discontinuité, ce qui n'est pas enseigné dans le judaisme de son époque ni les premières communautés 
\item Critère d'attestation multiple 
\ei 
\end{Prop}




\subparagraph{critère d'embarras}\sn{D'après PAGOLA, J. A, \emph{Jésus. Approche historique}, Paris 2013,
510-511.}
Les faits, les comportements ou les paroles de Jésus, qu'il aurait été
difficile aux chrétiens d'inventer postérieurement, car cela leur aurait
créé des difficultés, jouissent d'une grande crédibilité historique. On
observe à l'occasion comment ce matériel « embarrassant » est atténué,
voire supprimé au fur et à mesure de la transmission. \textit{Ex : Jean-Baptiste. Comment Jésus a pu se faire baptiser par Jean ?}, \textit{l'arrivée imminente du Royaume de Dieu}

\subparagraph{critère de discontinuité}
On peut raisonnablement attribuer à Jésus les actes et les paroles qui
ne peuvent dériver du judaïsme de son époque ni de l'Église primitive.
Ce critère est particulièrement utile pour connaître ce qu'il y a de
plus original et de plus irréductible dans son message et dans ses
comportements, mais il ne couvre pas tout. Il serait absurde de
l'utiliser de manière absolue et exclusive, car nous serions confrontés
à un Jésus « irréel », réduit au minimum, artificiellement isolé de son
peuple et déconnecté du mouvement auquel il a donné naissance.: \textit{Abba pour s'adresser à Dieu, repas avec des publicains et des pécheurs, }
\subparagraph{critère d'attestation multiple}
Les actes et les paroles de Jésus jouissent d'une plus grande
crédibilité historique lorsqu'ils ont été conservés dans plus d'une
source littéraire indépendante (par ex. dans Marc, la source Q, Jean,
Paul) ; et sous plus d'une forme littéraire (aphorismes, paraboles,
récits de guérison). Mais une tradition peut être authentique même si
elle n'est recueillie que par une seul source (Mc 14,36 : invocation
araméenne Abba). : quand on a une référence en Marc, Q et Jean, Paul. et sous plus d'une forme littéraire : Jésus au Temple, guérison, Annonce du Royaume. 
\subparagraph{critère de cohérence}
Après avoir réuni un ensemble de matériaux en conformité avec les
critères précédents, il est possible de retenir d'autres faits ou
paroles de Jésus s'ils s'intègrent dans la « base de données » déjà
établie, car ils ont de grandes chances d'être historiques.

Ce critère concerne également, la cohérence de la condamnation de Jésus
: « en regard de l'issue violente à laquelle aboutit la carrière de
Jésus, les paroles et les actes qui permettent d'expliquer
l'exaspération contre lui, son procès et sa crucifixion, acquièrent un
fort coefficient de probabilité historique » \sn{(Durand, 48 qui reprend J.
Meier, II, 15).}

\subparagraph{critères secondaires}

On peut citer d'autres critères secondaire qui n'offrent pas les mêmes
garanties que les précédents : des vestiges d'expressions araméennes,
des sémitismes (qui pourraient provenir de chrétiens juifs) ; une «
couleur local » palestinienne (qui peut aussi refléter le contexte d'un
groupe de chrétiens palestiniens) un récit marqué de détails concrets et
vivants (critère insuffisant).




\paragraph{Christologie fondamentale} ou christologie d'en bas. On part du Jésus de l'histoire et comment notre Foi est fondée sur Jésus. Dans la vie de Jésus, il y a une \textit{christologie cachée}. La discontinuité entre Jésus et le Christ annoncé par les apôtres n'est pas totale. 
\begin{Ex}
Jésus par exemple parlait avec autorité. Sa résurrection confirme son autorité. 
\end{Ex}

Certes, certains aspects étaient cachés mais sont révélés par la résurrection.


\subsection{La « Third Quest » de 1985 à aujourd’hui}
\paragraph{Limites de l'approche} En cherchant la discontinuité, on risque d'opposer le Christ à un judaisme légaliste et finalement deshumanisée \sn{cf Marguerat. Un crypto-anti judaisme.}

\paragraph{un troisième moment : redécouvrir le contexte juif} à partir des années 1985. Senders (1977) dans le monde anglosaxon, insiste sur l'enracinement de Jésus au sein même de la Foi juive. Pas un débat contre le judaisme mais un débat au sein du Judaisme.  A cela, s'ajoute un antijudaisme latent \sn{\textit{The Aryan Jesus: Christian Theologians and the Bible in Nazi Germany}
Susannah Heschel}
1905 : Wellhausen 
\begin{quote}
    Jésus n'est pas un chrétien mais un juif.
\end{quote}
Cette phrase résonne comme un coup de tonnerre. 
Dans les années 80, on redécouvre le judaisme de Jésus. 


%----------------------------------------------------------------------------------------------------------------------------------------------------------------------

 \section{L’interprétation vivante que fait Jésus de sa propre tradition religieuse}
\subsection{Jésus, un homme dont la vie est commandée par sa foi au Dieu d’Israël}

\paragraph{Un exemple, Geert Theissen} écrit une thèse : 
\begin{quote}
« Jésus historique vivait dans un mythe. Il attendait l'irruption du
règne de Dieu et se présentait lui- même comme le représentant de ce
règne de Dieu {[}\ldots{]} Il historicisait par là un mythe du temps de
la fin. Une attente qui se référait à un temps incertain de l'avenir
(proche) se changeait alors dans une expérience actuelle dans
l'histoire. L'unité liant ensemble mythe et histoire commençait donc
déjà chez le Jésus historique » (Theissen, 49).
\end{quote}

\begin{Def}[mythe]
Dieu intervient dans l'histoire.
\end{Def}

\subsection{Le mythe dans la prédication du Jésus historique : l’annonce du règne de Dieu}
\begin{quote}
    

« La prédication de Jésus contient dans son noyau un mythe : un mythe du
temps de la fin qui est pour le monde un temps déterminant où Dieu
s'imposera contre toutes les autres puissances surnaturelles, Satan et
les démons, pour transformer la situation actuelle instable qui est
celle du malheur en une situation de salut » (Theissen, 51).
\end{quote}

\begin{quote}
    « ce mythe n'est rien d'autre que du monothéisme juif conséquent : à la
fin Dieu sera le Dieu un et unique, à côté duquel il n'y aura plus
d'autres puissances susceptibles de limiter sa seigneurie -- et il
réalisera enfin son salut en Israël et dans la création tout entière.
L'annonce du règne de Dieu est une dramatisation mythique du premier
commandement, avec cette différence simplement que l'exode hors de
l'Egypte a été relayé par l'exode hors de la situation oppressante du
présent et la référence au règne de Dieu qui fait irruption » (Theissen,
51).
\end{quote}


\begin{quote}
    « L'avenir mythique est présent dans l'activité de Jésus d'une triple
manière : Il est présent par \emph{la victoire sur le mal}. Jésus
interprète ses exorcismes comme le fait que le règne de Dieu l'emporte
contre Satan et ses puissances (Mt 12,28). Satan est déjà tombé du ciel
(Lc 10,12). Il s'agit incontestablement de deux affirmations mythiques,
mais elle sont mises en rapport avec des expériences historiques et
concrètes : ici avec des exorcismes et les guérisons. L'avenir mythique
est présent en outre comme \emph{l'accomplissement du passé}. Ce que les
générations avaient attendu avec impatience advient maintenant en
présence de témoins oculaires (Mt 13,16s. : « en vérité, je vous le dis,
beaucoup de prophètes et de justes ont souhaité voir ce que vous voyez
et ne l'ont pas vu, entendre ce que vous entendez et ne l'ont pas
entendu ! »). Le temps est accompli, le règne de Dieu est devenu proche
(Mc 1,14s.). Enfin, l'avenir mythique est présent dans le temps présent
comme \emph{un germe caché}. Le règne de Dieu est `au milieu de vous'
(ou `en vous') (Lc 17,20s.). Il s'impose comme une semence qui `se
développe' rapidement, on ne sait comment, jusqu'à la moisson (Mc
4,26-29) » (Theissen, 52-53).
\end{quote}


Quelques éléments : 
\begin{itemize}

\item Comprendre Jésus dans sa Foi au Dieu d'Israël. Jésus vivait dans un \textit{mythe}, celui de l'Apocalypse juive : il attendait le Règne de Dieu et se comprend dans ce Règne de Dieu, avenir proche.  Jésus concrétisait ce mythe. (Texte 49).Jésus n'a pas voulu une nouvelle religion, c'est une revitalisation du Judaisme, il a vécu radicalement sa Foi au Dieu d'Israël. 
\item Jésus annonce le \textit{Règne de Dieu}. Dimension mythique (intervention de Dieu), d'un monde de malheur en monde de Salut. A la fin, le Dieu sera... Dramatisation mythique du premier commandement. 
\item Une deuxième métaphore, l'image de Dieu comme Père. On remarque que Dieu annonce le Règne de Dieu mais pas comme d'un Roi mais d'un Père. "Notre Père, que ton Règne vienne". 
\item Dans l'AT, Règne de Dieu associé à la victoire sur les paiens. Mais chez Jésus, règne de Dieu est là \textit{sans que les paiens soient vaincus} : "coexistence possible entre Pilate et le Règne"; un afflux de tous les paiens. Seul l'ennemi, c'est Satan. \textsc{Une démilitarisation}.
 
\end{itemize}

\subsection{La transformation « historique » du mythe}



\begin{quote}
    « Chez Jésus, le règne de Dieu est déjà présent de façon cachée, sans
que les païens aient été vaincus. Le règne de Dieu et la domination des
Romains peuvent coexister un certain temps dans le temps présent. C'est
pourquoi la fusion des temps du présent et de l'avenir signifie
davantage qu'un changement dans la forme. Cela est confirmé par
l'attente qui porte sur le futur : pour le futur également, aucune
victoire sur les païens n'est attendue, mais un afflux de tous les
païens (\ldots) venus de tous les points cardinaux. (\ldots) Seuls Satan
et ses démons sont vaincus par le règne de Dieu (Mt 11,28par) »
(Theissen, 55)
\end{quote}

\begin{quote}
    « Le bouleversement par lequel s'inaugure le règne de Dieu est un
bouleversement au niveau métaphysique -- la fin de la domination des
démons -- et un bouleversement à l'intérieur du peuple : c'est aux
pauvres (Mt 5,3), aux enfants (Mc 10,14) qu'appartient le royaume de
Dieu ; les publicains et les prostituées y entreront avant les pieux (Mt
21,32). (\ldots) Il s'agit bien plutôt d'une `démilitarisation' »
(Theissen, 55).
\end{quote}



\begin{Synthesis}[Vision de Theissen]
Jésus vit dans la Foi d'un Dieu unique du judaisme, avec à la fin la victoire de Dieu. Tension entre l'histoire d'Israël et que Dieu intervient dans l'histoire (mythique). Ses conflits avec ses contemporains se situaient au sein du Judaisme. 
historisation du mythe de Règne de Dieu. Expérience concrète. Parabole : image d'une réalité mythique pas encore présente. 
\textbf{Jésus va devenir un mythe}
\end{Synthesis}

\subsection{Mythe et autocompréhension de Jésus}

\begin{quote}
   « C'est à travers les moyens de sa religion seulement que Jésus pouvait
exprimer le rôle qu'il s'attribuait à lui-même (\ldots) Ce qui est
déterminant pour l'autocompréhension de Jésus, ce n'est pas tel ou tel
titre, mais l' `historicisation' du mythe du temps de la fin dans
l'ensemble de son activité. Elle entoura sa personne \emph{d'un éclat
surnaturel}. Elle lui conféra un rôle déterminant dans le drame entre
Dieu et l'homme dans le temps présent : il faisait du règne de Dieu une
expérience historique dans le  présent : des expériences concrètes du présent devenant une présence
mythique réelle du règne de Dieu, tandis que des paraboles et des
actions symboliques, devenaient les images d'une réalité mythique qui
n'était pas (encore) présente » (Theissen, 72-73). 
\end{quote}

\begin{quote}
  « Cet éclat surnaturel -- généré par un mythe dans lequel vivaient Jésus
et ses disciples -- était la cause de ce charisme par lequel Jésus
fascinait ses adeptes et irritait ses adversaires » (Theissen, 72).
Theissen considère donc Jésus comme un charismatique juif revivifiant la
tradition juive. « Sa prédication était rigoureusement monothéiste »
(Theissen, 72).  
\end{quote}

 \begin{Synthesis}
Processus historique de la Christologie : un processus qui conduit à la vérité. Penser la manière dont on fait ce processus. La théorie de la Tradition et de la transmission théologique. 
La Théologie Chrétienne de la religion a existé avant que le terme existe, comment ce regard, plutôt négatif, a pu évoluer, avec une critique interne. 
\end{Synthesis}
Nous verrons que la première culture où Jésus a été annoncé est la culture juive.


%----------------------------------------------------------------------------------------------------------------------------------------------------------------------

 \section{Conclusion}
\subsection{En résumé : une plus grande prise en compte de la judaïté de Jésus}

%----------------------------------------------------------------------------------------------------------------------------------------------------------------------

\section{Eléments bibliographiques :}

BERNARD, D., \emph{Les disciples juifs de Jésus du Ier siècle à Mahomet.
Recherches sur le mouvement ébionite}, Paris 2017.

DETTWILER, A. (éd.), \emph{Jésus de Nazareth. Études contemporaines},
Genève 2017.

EHRMAN, B. D., \emph{Jésus avant les évangiles. Comment les premiers
chrétiens se sont rappelé, ont transformé et inventé leurs histoires du
Sauveur}, tr. par J.-P. PRÊVOST, Paris 2017.

GOWLER, D., \emph{Petite histoire de la recherche du Jésus de
l'Histoire}, Paris 2009.

HURTADO, L. W., \emph{Le Seigneur Jésus Christ. La dévotion envers Jésus
aux premiers temps du christianisme}, Paris 2009.

MARGUERAT, D., \emph{Vie et destin de Jésus de Nazareth}, Paris, Seuil,
2019.

MEIER, J. P., \emph{Un certain juif. Jésus. Les données de l'histoire}.
I, \emph{Les sources, les origines, les dates}. II, \emph{La parole et
les gestes}. III, \emph{Attachements, affrontements, ruptures}. IV,
\emph{La loi et l'amour}, Paris 2004-2009.

MIMOUNI, S. C., \emph{Le judaïsme ancien et les origines du
christianisme}, Paris 2017. PAGOLA, J. A., \emph{Jésus. Approche
historique}, Paris 2013.

THEISSEN, G., \emph{La religion des premiers chrétiens}, tr. par J.
HOFFMANN, Paris -- Genève 2002.

 
 


%--------------------------------------------------------------------------------------------------------------
\section{Annexes}

\subsection{De Jésus à la christologie : reconstruction historique}
 
 
\emph{Le contexte historique et
culturel} : Le monde et la religion juifs (et païens) en Galilée,
Samarie et Judée
$$\downarrow$$
\emph{Les éléments historiques
concernant la vie de Jésus} : Jésus né en 4 avant JC (Jean-Baptiste,
prédication du Royaume et guérisons)
$$\downarrow$$
\emph{Les éléments historiques
concernant la mort de Jésus} (en avril 29) : Jésus est crucifié sous
Ponce Pilate. Causes ?
 
$$\downarrow$$

 
\paragraph{{[}résurrection{]} : un événement qui échappe à l'histoire tout
en changeant l'histoire. Mais c'est à la lumière de la résurrection que
la christologie est vraiment
née.} 
$$\downarrow$$


 
\emph{Expérience de quelques
disciples} : Jésus est vivant. En quoi consiste cette expérience ?
Apparition (évangiles), vision (Paul), prise de conscience et foi ?
$$\downarrow$$
\emph{Les disciples de Jésus commencent à annoncer publiquement que
Jésus est ressuscité} : C'est un fait historique.
$$\downarrow$$
\emph{Histoire des traditions}
: des traditions orales sur l'histoire et le destin de Jésus circulent
dans l'Empire.
 $$\downarrow$$
\emph{La rédaction des écrits autour de Jésus (par des juifs)} : de 50 à
150 environ. On pense que les évangiles ont été rédigés entre 60 (Marc)
et 90 (Jean) donc par la deuxième génération voire la troisième des
disciples. Mais il y a d'autres écrits dits « évangiles ».
$$\downarrow$$


Le christianisme tend à se distinguer du judaïsme (moitié du IIe siècle
?)
 
$$\downarrow$$

\emph{Canon des Ecritures} (IIIe s.) : on distingue clairement les
écrits canoniques des écrits dits apocryphes.
 
 
