\chapter{Introduction}

\mn{Xavier Gué, 2022, Christologie, Culture dans
l’histoire
 }
\paragraph{Post-Modernité}
\begin{Def}[Post-modernité]
Contexte où les grands récits fondateurs ne sont plus opérants, créant des vérités partiels
\end{Def}
\mn{J.F. Lyotard, Lipoveski}

risque de relativisme


\paragraph{universalité vs contre-culture}. Soit on dit que le Christ est l'unique sauveur, on développe une contre-culture chrétienne, qui n'est plus en dialogue avec le monde, soit on vise l'universalité mais avec le risque de relativisme du Christ.


Bernard Sesboué  :
\begin{quote}
    Question de l'unicité du Christ. Jésus de Nazareth, unique médiateur.
    Ac 4,12
    Ces affirmations neo-testamentaires semblent refuser le dialogue avec les autres religions.
    Question théologique la plus forte du XX\textsuperscript{ème}
    \sn{Sesboué, Introduction à la Théologie, 2017, p. 202-203}
\end{quote}

\paragraph{Passer par l'histoire} reconnaître que la christologie s'est toujours construit dans un dialogue avec les cultures. A chaque génération, il nous faut reinterpréter notre foi, redire notre foi en Christ dans le contexte d'aujourd'hui.

\begin{Ex}[Consubstantiel]
peut être plus juste mais incompréhensible.
\end{Ex}

\paragraph{Jésus-Christ ne va pas de soi}, la christologie vient d'un dialogue.

\section{La christologie comme un savoir sur le Christ}

Culture comme langage, si une langue vit sans culture, elle meurt.

\subsection{Le traité du verbe incarné}
\paragraph{Le traité du verbe incarné} On enseignait non pas la christologie mais le traité du Verbe incarné. Le terme christologie est apparu au XX. On pensait que dès le début, l'identité de Jésus était bien définie. Avec les temps modernes, on redécouvre l'\textit{écart} entre Jésus et le Christ. Pour connaître qui est le Christ, il faut non seulement une catéchèse mais un dialogue, une contemplation. Historiquement, cela s'est aussi passé comme cela.

\paragraph{Historiquement, une approche d'en haut} Christologie descendante, partant de la vision de Dieu. Dès le début. Christologie johannique, épiphanique. On ne tient pas compte de l'histoire, l'histoire est le support de cette manifestation. Entre Jésus et Dieu, pas de discontinuité, Jésus est le Christ dès le début et il l'a révélé au monde.  

\paragraph{Le problème est que sa mort et sa résurrection n'ont rien à faire avec le Christ}. Son histoire ne nous dit pas qui il est puisque nous savons ce qu'il est dès le début. Or, on ne peut pas dire qui est un homme avant sa mort, \textit{du fait de sa liberté}. L'identité narrative \sn{Ricoeur, Temps et Récit. L'homme dit qu'il est à travers son récit.} 

\paragraph{Cette christologie va exploser avec la modernité}

\subsection{la prise de conscience entre Jésus historique et le Christ de la Foi}

\paragraph{Question récente} jusqu'au XVIII, on pensait que l'Evangile racontait l'histoire. On ne faisait pas de différence entre Jésus et le Christ. Or, Jésus-Christ est \textit{kerygme} et acte de foi.

\paragraph{On sépare le Christ des Evangiles} par rapport au christ des dogmes, puis le Christ des Evangiles du Jésus de l'histoire. Prise en compte de la distance. Ces questions sont récentes mais pas récentes-récentes. 
\bi 
\item au XVI : Lelio Sozzini (1525, 1562), le Sozzinialisme, mouvement qui remet en question la Trinité (et a donné ensuite le mouvement unitarien, Michel Servet\sn{\href{https://fr.wikipedia.org/wiki/Michel_Servet}{Notice Wikipedia de M. Servet}}, brulé par les Calvinistes à Genève). Dieu est unique. Tout un travail sur les sources scripturaires pour contrer ce mouvement.
\item Reimarus \sn{\href{https://fr.wikipedia.org/wiki/Hermann_Samuel_Reimarus}{Reimarus - notice Wiki}}. Une partie de ces oeuvres publiée par Lessing en 1778, un christ politique pour prendre le pouvoir, mais il est arrêté et crucifié. Les disciples continuent la lutte mais de façon spirituelle, en en faisant une religion. Reimarus part des \textit{contraduction de la foi}. 
\ei 

Pour répondre à ces questions, plusieurs périodes : 

\paragraph{LebensJesusForschung} Recherche qui va mobiliser beaucoup d'énergie. On va opposer deux approches, soit on rebâtit notre foi sur l'histoire (LebensJesuForschung),  cachée par les dogmes et les traditions, soit on part du Jésus de la Foi et c'est cela qui compte. Enlever la gangue théologique et mythique. Idée de la pêche : le vrai Jésus est au milieu et autour des couches, les disciples, l'Eglise, mon curé : il faut percer les différentes couches pour arriver au noyau. 

\paragraph{Théologie libérale} au sens de reprendre sa liberté au nom de la raison. Geert Theissen, p. 47. 
\begin{quote}
    
    un nimbe... qui le transfigure. 
\end{quote}
\begin{Ex}
on a un peu la même chose pour de Gaulle : tout le monde se réfère à lui, on en a fait un mythe. 
\end{Ex}

\paragraph{L'échec des vies de Jésus} Après avoir déconstruit, il faut reconstruire mais la difficulté et de ne pas projeter dans la reconstruction sa propre vision. 
\begin{Ex}[vision de F. Schleiermacher]
Jésus était tellement en lien avec Dieu qu'il vivait le Royaume de Dieu intérieurement, qu'il a transmis à ces disciples.
\end{Ex}

\paragraph{Réaction : idée religieuse} ou christologique du sauveur. Approche idéaliste. Jésus  ne fait qu'incarner un idéal qu'il nous faut incarner à notre tour. 
\begin{Ex}[E Kant]
L'homme agréable à Dieu qui a mis en action la morale universelle. Maintenant qu'on connait la morale, on n'a plus besoin du Christ. On le détache de son histoire. 
\end{Ex}
David Strauss et au XX, R. Bultmann, développent ceci, mais avec le risque de l'idéologie. 

\paragraph{Une troisième periode} On ne peut pas s'arrêter au Kerygme, il faut montrer le lien entre le Christ de la Foi et le Christ historique. La grande figure est Käsemann (1953-1985). Il ne s'agit pas de faire une biographie historique mais de s'assurer du passage du Jésus de l'histoire au Christ de la Foi. On développe une méthode pour valider ce qui est authentique.
\begin{Prop}[Les critères de validité]
Différents critères ont été développés : 
\bi
\item Critère d'embarras. \textit{Ex : Jean-Baptiste. Comment Jésus a pu se faire baptiser par Jean ?}, \textit{l'arrivée imminente du Royaume de Dieu}
\item Critère de discontinuité, ce qui n'est pas enseigné dans le judaisme de son époque ni les premières communautés : \textit{Abba pour s'adresser à Dieu, repas avec des publicains et des pécheurs, }
\item Critère d'attestation multiple : quand on a une référence en Marc, Q et Jean, Paul. et sous plus d'une forme littéraire : Jésus au Temple, guérison, Annonce du Royaume. 
\ei 
\end{Prop}

\paragraph{Christologie fondamentale} ou christologie d'en bas. On part du Jésus de l'histoire et comment notre Foi est fondée sur Jésus. Dans la vie de Jésus, il y a une \textit{christologie cachée}. La discontinuité entre Jésus et le Christ annoncé par les apôtres n'est pas totale. 
\begin{Ex}
Jésus par exemple parlait avec autorité. Sa résurrection confirme son autorité. 
\end{Ex}

Certes, certains aspects étaient cachés mais est révélés par la résurrection.

\paragraph{Limites de l'approche} En cherchant la discontinuité, on risque d'opposer le Christ à un judaisme légaliste et finalement deshumanisée \sn{cf Marguerat. Un crypto-anti judaisme.}

\paragraph{un troisième moment : redécouvrir le contexte juif} à partir des années 1985. Senders (1977) dans le monde anglosaxon, insiste sur l'enracinement de Jésus au sein même de la Foi juive. Pas un débat contre le judaisme mais un débat au sein du Judaisme.  A cela, s'ajoute un antijudaisme latent \sn{\textit{The Aryan Jesus: Christian Theologians and the Bible in Nazi Germany}
Susannah Heschel}
1905 : Wellhausen 
\begin{quote}
    Jésus n'est pas un chrétien mais un juif.
\end{quote}
Cette phrase résonne comme un coup de tonnerre. 
Dans les années 80, on redécouvre le judaisme de Jésus. 

\subsection{Un exemple de la troisième quête : Geert Theissen }
\paragraph{Un exemple, Geert Theissen} écrit une thèse : 
\bi
\item Comprendre Jésus dans sa Foi au Dieu d'Israël. Jésus vivait dans un \textit{mythe}, celui de l'Apocalypse juive : il attendait le Règne de Dieu et se comprend dans ce Règne de Dieu, avenir proche.  Jésus concrétisait ce mythe. (Texte 49).Jésus n'a pas voulu une nouvelle religion, c'est une revitalisation du Judaisme, il a vécu radicalement sa Foi au Dieu d'Israël. 
\item Jésus annonce le \textit{Règne de Dieu}. Dimension mythique (intervention de Dieu), d'un monde de malheur en monde de Salut. A la fin, le Dieu sera... Dramatisation mythique du premier commandement. 
\item Une deuxième métaphore, l'image de Dieu comme Père. On remarque que Dieu annonce le Règne de Dieu mais pas comme d'un Roi mais d'un Père. "Notre Père, que ton Règne vienne". 
\item Dans l'AT, Règne de Dieu associé à la victoire sur les paiens. Mais chez Jésus, règne de Dieu est là \textit{sans que les paiens soient vaincus} : "coexistence possible entre Pilate et le Règne"; un afflux de tous les paiens. Seul l'ennemi, c'est Satan. \textsc{Une démilitarisation}.


\ei

\begin{Def}[mythe]
Dieu intervient dans l'histoire.
\end{Def}
\begin{Synthesis}[Vision de Theissen]
Jésus vit dans la Foi d'un Dieu unique du judaisme, avec à la fin la victoire de Dieu. Tension entre l'histoire d'Israël et que Dieu intervient dans l'histoire (mythique). Ses conflits avec ses contemporains se situaient au sein du Judaisme. 
historisation du mythe de Règne de Dieu. Expérience concrète. Parabole : image d'une réalité mythique pas encore présente. 
\textbf{Jésus va devenir un mythe}
\end{Synthesis}

Cf Loisy : Jésus a annoncé le Règne de Dieu et est venu l'Eglise. 

Jésus-Baptiste ? Mais rôle du Temple renforcé ? 

\begin{Synthesis}
Processus historique de la Christologie : un processus qui conduit à la vérité. Penser la manière dont on fait ce processus. La théorie de la Tradition et de la transmission théologique. 
\end{Synthesis}
Nous verrons que la première culture où Jésus a été annoncé est la culture juive.


