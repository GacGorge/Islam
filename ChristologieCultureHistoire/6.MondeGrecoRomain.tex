\chapter{La christologie face au monde religieux gréco-romain }
\mn{Christologies et cultures dans l’histoire 6 La christologie face au monde religieux gréco-romain La christologie du médiateur }

%--------------------------------------------
\section{Bibliographie}
AUGUSTIN, La cité de Dieu, Livres VI – X, Œuvre de saint Augustin 34, tr. par G. COMBES et notes de G. BARDY, Paris, Desclée de Brouwer, 1959.
TEXTES : La Cité de Dieu sera abrégée par CD. 
FITZGERALD, A. D. (dir), Encyclopédie saint Augustin, « christologie », Paris 2005. 


MADEC, G., Le Christ de saint Augustin. La Patrie et la Voie, Paris 2001. 


REMY, G., Le Christ médiateur dans l’œuvre de saint Augustin, Paris, Librairie H. Champion, 1979, 2 vol. (voir Compte-rendu de J.-P. WELSCH dans Revue théologique de Louvain 11 (1980), 362-366). 

REMY, G., « la christologie d'Augustin : cas d'ambiguïté », Recherches de science religieuse 2008 96 (2008) 401-425.  

VAN BAVEL, T., Recherches sur la christologie de saint Augustin. L’humain et le divin dans le Christ d’après saint Augustin, Editions universitaires Fribourg, Fribourg 1954. 

VERWILGHEN, A., Christologie et spiritualité selon Saint Augustin : l'hymne aux Philippiens, Beauchesne, Paris 2000.


%--------------------------------------------
\section{Introduction}


%--------------------------------------------
\section{La christologie dans l’œuvre d’Augustin }


\paragraph{la christologie d'Augustin n'est pas a priori originale} A la différence d'une réflexion sur la grâce et la liberté, sur la trinité, ses écrits sur le Christ ne semble pas particulièrement originaux. Il meurt en 430, un an avant le Concile Christologique d'Ephèse. Il faut néanmoins souligner dans un traité de \textit{Leporius} \sn{Vers la fin de son épiscopat, saint Augustin vit arriver un moine gaulois du nom de Leporius. Soutenant des doctrines erronées notamment à propos de l'Incarnation. L’évêque d’Hippone entreprit de le ramener à l'orthodoxie, ce qu’il réussit magnifiquement puisque c’est dans les termes les plus affectueux qu’il parle de cette édifiante conversion.  Réunis à Carthage, ces évêques reçurent de Leporius une profession de foi, écrite et signée : le Libellus emendationis sive satisfactionis..}. Dans le \emph{Libellus emendationis sive satisfactionis} de Leporius, on décerne généralement l'influence de Saint Augustin. \sn{\href{http://www.documentacatholicaomnia.eu/04z/z_0426-0426__Leporius_Monachus__Libellus_Emendationis_Sive_Satisfactionis__MLT.pdf.html}{le Libellus}}

\paragraph{Christ médiateur} St Augustin va développer une pensée non dogmatique autour du Christ médiateur ou de voie. On n'est plus dans le Christ \textit{logos}, mais toujours aspect relationnel.




\begin{quote}
    « la théologie mythique, la théologie physique, la théologie civile. Si l’usage latin le permettait, nous appellerions la première ‘fabulaire’, disons ‘fabuleuse’ : mythique en effet dérive du grec muthos qui signifie ‘fable’. Quant à la seconde, l’habitude du langage admet déjà qu’on l’appelle ‘naturelle’.  A la troisième, la théologie civile, Varron lui-même donne un nom latin » (AUGUSTIN, La Cité de Dieu, VI, V, 1).  
\end{quote}



 


%--------------------------------------------
\section{Le contexte : dialogue et confrontation avec la culture et les religions païennes }


\label{Def:Religio}
\subsection{La théorie de la religion selon Augustin}




\begin{quote}
    « Le chemin de la vie bonne et heureuse n’est autre que la vraie religion, qui adore le Dieu unique et le reconnaît avec une piété très pure, comme principe de tous les êtres, origine, achèvement et cohésion de l’univers » (AUGUSTIN, De vera religion I ; BA 8, p. 23). 
\end{quote}

La vraie religion, c'est le \textit{chemin} de la \textit{vie bonne et heureuse}. Un champ lexical un peu différent que celui de vérité. La religion c'est ce qui va nous donner le bonheur. 
    Une vision proche de la philosophie, avec une recherche commune du bonheur, bonheur terrestre mais aussi éternel.
    
    \paragraph{Une définition de la religion}
    la religion est ce qui nous permet de nous \textit{unir à Dieu} et donc de recevoir le bonheur. C'est une définition qu'il reprend de Lactance (re-ligare) mais reprend la vision grecque de la relation : le multiple doit intégrer l'un.  D'où l'importance de la médiation et du chemin. Tout ce qui nous permet de rejoindre Dieu (ascèse, culte,...) est religion. 
    
    
 





\subsection{La chute de Rome en 410 et la rédaction de La Cité de Dieu}
\paragraph{Culture paienne}
Sagesse neoplatonicienne. 

\paragraph{Empire tardif} Contexte de la chute de Rome en 410 devant les goths d'Alaric. Effet psychologique important. Augustin écrit en 416-418 la \textit{cité de Dieu}.

\paragraph{Une attaque contre le christianisme} Pour beaucoup de Romains, la punition de la chute de Rome vient de l'abandon des Dieux pour adopter le Dieu des Chrétiens. Mais il y a une autre interprétation plus subtile, la diffusion de la morale chrétienne aurait provoqué un déclin du sentiment de citoyenneté, car la religion romaine est une religion civile, de la cité. Or, le christianisme, visant le Royaume, aurait affaibli le lien entre les citoyens romains et Rome : amour de ses ennemis, vertus d'humilité et de patience. Certaines valeurs du Christianisme ne poussaient pas à la résistance de l'Empire.

\begin{quote}
    les chrétiens... très fort pour prier... moins pour se défendre.

\end{quote}

\paragraph{Est ce que les chrétiens sont des citoyens fiables ? } Oui et non

\paragraph{La Cité de Dieu comme Apologie} Saint Augustin va retourner l'argument et critiquer les paiens : 
l'abandon du culte paien ne peut expliquer la chute de Rome. puis supériorité du culte chrétien.

\begin{quote}
    « Le sacrifice visible est donc le sacrement, c-à-d le signe sacré du sacrifice invisible » (CD X, V.) Le véritable sacrifice est une œuvre qui nous unit à Dieu. 
\end{quote}

\begin{quote}
    « En présence de ce sacrifice suprême et véritable tous les faux sacrifices se sont évanouis » (CD X, XX). 
\end{quote}


% ----------------------------------------
\subsection{La typologie des religions selon Varron et adoptée par Augustin}

\paragraph{Varron} 1er sicèle avant Jesus Christ. 


\begin{quote}[théologie mythique]
 On nomme mythique la théologie qu’on trouve surtout chez les poètes, naturelle celle des philosophes, civile celle des peuples.   
\end{quote}

\begin{quote}
    « On nomme \textit{mythique} la théologie qu’on trouve surtout chez les poètes, naturelle celle des philosophes, civile celle des peuples. La première contient beaucoup de fictions contraires à la dignité et à la nature des êtres immortels […] ‘La deuxième sorte de la théologie que j’ai distinguée est celle sur laquelle les \textit{philosophes} ont laissé un grand nombre de livres où ils se demandent : que sont les dieux ? où résident-ils ? quelle fut leur origine ? leurs qualités ? […] ‘C’est la troisième espèce, dit-il, celle que dans les villes les citoyens et surtout les prêtres doivent connaître et mettre en pratique. On y trouve quels dieux chacun doit officiellement honorer, par quels rites et quels sacrifices’ » (CD VI, V,1-3). 
\end{quote}

Ce qui est intéressant, c'est qu'Augustin ne parle pas des religions à mystère (Mistra,...). Peut être n'a t il pas été en contact ? Bcp de religions qui ne sont pas publiques mais cachées.







A de nombreuses reprises dans la CD, désigne le Christ comme étant le « Médiateur », mais aussi « voie » et cette médiation passe par l’humilité. 



\subsection{Des fausses religions }
 
Saint Augustin montre les autres religions :  
 

\begin{quote}
    « De ces dix premiers livres, les cinq premiers ont été écrits contre ceux pour qui les dieux doivent être adorés en vue des biens de cette vie ; les cinq autres, contre ceux pour qui le culte des dieux doit être pratiqué en vue de la vie qui vient après la mort » (CD X, 32,  p. 559). 
\end{quote}

La vraie religion est celle du Christ, vraie médiateur.

%--------------------------------------------
\section{Le Christ modèle la « nouvelle » religion  }


\subsection{La concentration christologique}


\subsection{L’assomption du vocabulaire religieux dans le christianisme }
 
%--------------------------------------------
\section{Le Christ comme « Médiateur » et « voie » }


\subsection{L’emploi du terme médiateur par Augustin}


\subsection{Origine des termes  « médiateur » et  « voie » dans le NT}

\begin{quote}
    Gal 3,19-20 : « La Loi a été promulguée par les anges par la main d’un médiateur (Moïse ?). Or, ce médiateur n’est pas médiateur d’un seul. Et Dieu est unique. » ? 
\end{quote}

\begin{quote}
    1 Tm 2,5 : « Il n’y a qu’un seul Dieu, un seul médiateur aussi entre Dieu et les hommes, un homme : Christ Jésus, qui s’est donné en rançon pour tous ».  
\end{quote}
\begin{quote}
    He 8,6 : « En réalité, c’est un ministère bien supérieur qui lui revient, car il est médiateur d’une bien meilleure alliance »  
\end{quote}

\begin{quote}
    He 9,15s : « Voilà pourquoi il est médiateur d’une alliance nouvelle, d’un testament nouveau ; sa mort étant intervenue pour le rachat des transgressions commises sous la première alliance, ceux qui sont appelés peuvent recevoir l’héritage éternel déjà promis »  
\end{quote}
\begin{quote}
    He 12,24 : « Jésus, médiateur d’une alliance neuve ». 
\end{quote}

\begin{Synthesis}
Comment saint Augustin va introduire cette représentation du Christ, pas considéré comme consubstantiel au père, mais comme chemin, voie, \textit{'odos}. 
\end{Synthesis}
A force d'être pris, le sentier devient chemin. Jésus est qualifié comme chemin (Jn, 14, 26, chemin, vérité et la Vie). Jésus a tracé le chemin. Dans les Ac, on a l'expression de la voie (Ac, 8, 2) : \textit{Tao} ! \sn{voir  p. \pageref{ch:christologieChine}} en tant qu'il est le nouveau style de vie : nettement moral et éthique et existentielle que cultuel par rapport aux rites de la religion romaine. 

\subsection{Le Christ médiateur dans le dialogue avec le néoplatonisme}

Augustin dialogue dans  \CD avec les néoplatoniciens\sn{Apulée, Porphyre} sur la nature \textit{théandrique}. Le christ peut combler la distance, le pont entre Dieu et l'homme. La médiation est soit : 
\begin{itemize}
    \item soit par la dimension métaphysique
    \item soit par la nature simple de l'humanité de Dieu
\end{itemize}

 \paragraph{La dimension « métaphysique » de la médiation : l’union en Christ des deux natures } 
 
 \paragraph{Néoplatonisme} Pour les néoplatoniciens,le problème est de passer du : 
\begin{table}[h!]
\centering
\begin{tabular}{lll}
 & UN / Esprit        &  \\
 & Logos              &  \\
 & Multiple / matière & 
\end{tabular}
\end{table}

Il y a des intermédiaires dans la dégradations passant du un au multiple : Il y a des \textit{démons}...
Mais pour passer du multiple au UN : par la purification des âmes s'élèvent,  ce n'est que la partie spirituelle.

\paragraph{Pourquoi les démons ne sont pas des médiateurs}

\begin{quote}
    « Tous ces philosophes (…) ont pensé qu’il fallait rendre hommage à plusieurs dieux » (CD VIII, 12). « Le ciel est la demeure des dieux ; la terre est le séjour des hommes ; l’air celui des démons  (…) Les démons partagent avec les dieux l’immortalité du corps ; et avec les hommes, les passions de l’âme » (CD VIII, 14). 
\end{quote}

On a une vision : 
\begin{table}[h!]
\centering
\begin{tabular}{lll}
 & Ciel        &  Dieu\\
 & Air (au milieu)              &  il y a les démons\\
 & Terre & les hommes
\end{tabular}
\end{table}

\paragraph{les démons sont intermédiaires mais ne peuvent nous apporter le bonheur}
les démons partagent l'immortalité du corps avec les Dieux et avec les hommes les passions de l'âme.
Mais pour Saint Augustin, le salut, le but de la vie est d'être heureux. or, les démons ont l'immortalité (que nous n'avons pas), et ils sont malheureux comme nous. Il ne peuvent donc nous sauver.

\begin{quote}
    « Quant à l’éternité, est-ce donc un bien sans le bonheur ? Mieux vaut la félicité dans le temps qu’une éternité de misère » (CD VIII, 16).  
\end{quote}



\begin{quote}
    « Les démons profitent des avantages de leur séjour et de l’agile subtilité de leurs corps, pour suspendre, pour détourner le progrès de nos âmes, et loin de nous ouvrir la voie qui mène à Dieu, ils la sèment de pièges. » (CD IX, 18). 
\end{quote}
 
 \paragraph{La dimension « physique » de la médiation : Le corps du Christ } 
Les bon esprits sont les anges mais ils ne sont pas les médiateurs :
Quand aux anges, ils n'ont rien de commun avec nous et ne peuvent nous sauver : ils ne peuvent faire pont.
 

\begin{quote}
    « Les bons anges ne peuvent (…) pas occuper une position intermédiaire entre les mortels malheureux et les immortels bienheureux, puisqu’ils sont eux-mêmes et bienheureux et immortels » (CD IX, XV,1) 
\end{quote}

\begin{Synthesis}
les anges ne sont pas intermédiaires en tant que médiateur : on ne peut suivre son ange gardien pour être sauvé.
\end{Synthesis}

\begin{quote}
    « Des êtres immortels et bienheureux, de quelque nom qu’on les appelle, mais qui cependant ont été faits et créés, ne peuvent servir d’intermédiaires pour conduire à l’immortel bonheur les malheureux mortels qu’une double différence sépare d’eux » (CD IX, XXIII,3) 
\end{quote}

\paragraph{Du coup, quel médiateur ?} Il déduit le Christ à partir de la question du salut. \sn{voir aussi réflexion de Levinas sur le concept d'homme Dieu \label{LevinasHommeDieu}}

\begin{quote}
    « Si (…) tous les hommes tant qu’ils sont mortels, sont nécessairement aussi malheureux, il faut chercher un intermédiaire qui soit non seulement homme, mais encore Dieu. Car il pourra ainsi par l’entremise de sa bienheureuse immortalité acheminer les hommes de leur mortalité misérable à l’immortalité bienheureuse (…) Il a donc fallu que le médiateur entre Dieu et nous possédât une mortalité transitoire et une béatitude permanente (…) » (CD  IX, XV,1). 
\end{quote}

\paragraph{Mortalité transitoire} Il introduit l'histoire de Jésus dans cette formule. L'humanité historique du Christ et sa dimension corporelle.
 
 \subsection{dimension corporelle de la médiation}
 
\paragraph{Dans la vision neoplatonicienne, aucun Dieu ne se mélange à l'homme}. Les philosophes pensaient un tel rapport qu'en se détachant de son corps et alors l'âme pouvait monter vers Dieu. Le corps de l'homme souille Dieu.

\paragraph{Saint Augustin dit qu'ils se sont arrêtés en chemin} Ils ont vu Dieu mais ils refusent le chemin, le corps du Christ, le chemin, de l'homme à Dieu. 

\begin{quote}
    « Le Dieu vraiment supérieur à toutes choses n’en est pas moins présent (…) d’une manière intelligible et ineffable à l’intelligence des sages, quand ils se sont détachés du corps autant qu’il est possible sans plus être pour Dieu une occasion de souillure » (CD IX, XVI, 1) 
\end{quote}


\begin{quote}
    « La grâce de Dieu ne pouvait plus gracieusement se recommander qu’en ce Fils unique que Dieu qui, sans subir lui-même aucun changement,  s’est revêtu de l’homme et a donné aux hommes par la médiation de l’homme l’esprit de son amour, afin que par cet amour on puisse, de la région des hommes, venir à lui (…) 
    Assurément vous avez de l’âme intellectuelle qui sans nul doute est humaine, une si haute idée qu’elle peut, selon vous, devenir consubstantielle à cette Intelligence paternelle en qui vous reconnaissez le Fils de Dieu. Qu’y a t-il donc d’incroyable que l’une de ces âmes intellectuelles, d’une manière ineffable et exceptionnelle ait été prise pour le salut de beaucoup ? Le corps est uni à l’âme pour constituer l’homme total et complet ; nous le savons par le témoignage de notre nature ; si ce n’était un fait d’expérience courante, ce serait sans nul doute un fait assez difficile à croire ; car il est plus facile d’ajouter foi à l’union même de l’humain avec le divin, du changeant avec l’immuable, et d’ailleurs d’un esprit avec un esprit, ou selon votre manière de parler, d’un incorporel avec un incorporel, qu’à \textit{l’union d’un corps avec un incorporel} » (CD X, XXIX). 
\end{quote}


\paragraph{de la difficulté à penser un corps avec un incorporel} Il est plus facile de penser que l'âme s'unisse à Dieu que de penser que Dieu s'unit au corps. Mais l'âme pourtant s'unit au corps et cela devrait réfléchir. Les Grecs ont du mal à penser l'esprit/âme avec le corps.  C'est par \textsc{le corps que le Christ est médiateur et non en tant que Verbe}. Le Verbe est souverainement éternel et heureux. 

\begin{quote}
    « Ce n’est pas en tant que Verbe qu’il est médiateur, car le Verbe est souverainement immortel et souverainement heureux, est loin des mortels malheureux. Il est médiateur en tant qu’homme, montrant par là même que pour atteindre celui qui est le bien non seulement bienheureux mais béatifiant, il ne faut pas chercher d’autres médiateurs que nous croirions chargés de disposer les degrés de notre montée, puisque le Dieu bienheureux et béatifiant, devenu participant de notre humanité \textsc{nous a fourni un moyen rapide de participer à sa divinité}\sn{cf Sainte Thérèse de l'enfant Jésus et l'ascenseur } (…) Aussi quand il a voulu pour être médiateur se mettre au-dessous des anges  en la forme d’esclave, est-il resté au-dessus d’eux en sa forme de Dieu, étant voie de la vie dans les régions inférieures, parce qu’il est la vie même dans les supérieures » (CD IX, XV,2) 
\end{quote}

C'est la totalité de l'homme qui est sanctifié.


 \paragraph{La dimension « humble » de la médiation : la forme d’esclave du Christ et le sacrifice } 




  \begin{quote}
      « Pour la guérir, comme aucun rapport n’est possible entre l’immortelle pureté d’en haut et les êtres mortels et impurs d’en bas,  il faut évidemment un médiateur, non pas tel cependant qu’il ait un corps immortel proche des réalités d’en haut et une âme malade semblable aux chose d’en bas (…), mais adapté à notre bassesse par la mortalité de son corps, de sorte que l’immortelle justice de son esprit qui le maintient dans les hauteurs non quant à la distance mais par sa parfaite ressemblance avec Dieu, apporte à l’œuvre de purification et de notre libération une aide vraiment divine. Incapable assurément comme Dieu d’être souillé, écartons l’idée qu’il ait pu craindre d’être souillé par l’homme dont il s’est revêtu ou par les hommes en compagnie desquels il a vécu comme homme. \textsc{Car il est grand ce double enseignement que pour notre salut nous a donné l’Incarnation : que la véritable divinité ne saurait être souillée par la chair }; qu’il ne faut pas juger les démons meilleurs que nous parce qu’ils n’ont pas de chair. Et voilà, tel que l’annonce la sainte Ecriture, le médiateur entre Dieu et les hommes : l’homme Jésus-Christ ; par sa divinité, il est toujours égal à son Père, par son humanité, il est devenu semblable à nous » (CD IX, XVII). 
  \end{quote}

Il s'agit de ne pas sauver que l'âme mais tout l'homme. Esclave et sacrifice \sn{voir l'humilité de Dieu}
\paragraph{Distance créée par le péché}
Le mal n'est pas lié à la matière mais le mal, c'est l'orgueil. Il parle de métaphysique, la matière. mais ce que ne voit pas Porphyre, c'est que la matière n'est pas mauvaise, c'est le péché qui est mauvais. Il ne s'agit pas de quitter la chair puisqu'elle n'est pas mauvaise. 
Le Christ s'humilie en prenant chair, contraire du péché, et ainsi nous sauve. Dans le chemin, médiation éthique.

\begin{quote}
    « Cette voie purifie l’homme tout entier le prépare, mortel à l’immortalité en toutes les parties qui le constituent. Et pour qu’il n’y ait point à chercher une purification pour la partie de l’âme que Porphyre appelle intellectuelle, une autre pour celle qu’il appelle ‘spirituelle’ et  une autre pour le corps, le Purificateur et le Sauveur très véritable et très puissant a assumé l’homme tout entier. » (CD X, XXXII, 2). 
\end{quote}
Du coup, il ne devrait pas y avoir de notion de pureté et d'impureté dans le christianisme.

Si on refuse le Christ dans son humilité, on n'est pas dans le bon chemin. 

\begin{quote}
    « Le véritable Médiateur – celui qui ayant pris la forme d’un esclave est devenu à ce titre médiateur entre Dieu et les hommes, l’homme Jésus-Christ – sous la forme de Dieu, reçoit le sacrifice comme son Père avec lequel il est lui aussi un seul Dieu ; cependant sous la forme d’esclave, il a mieux aimé être le sacrifice que le recevoir, pour que personne n’estimât même à cette occasion, qu’on puisse sacrifier à quelque créature. Ainsi est-il prêtre : c’est lui-même qui offre, et il est lui-même l’oblation. Et il a voulu que soit sacrement quotidien de cette réalité, le sacrifice de l’Église qui, étant le corps dont il est la tête, apprend à s’offrir ellemême par lui » (CD X, XX). 
\end{quote}
  \begin{quote}
      « Il l’a méprisé dans cette chair même que le Christ a prise en vue du sacrifice de notre purification.  Il n’a pas compris ce grand mystère par suite de cet orgueil que le bon, le vrai Médiateur a abattu par son humilité en se montrant aux mortels dans cette mortalité que n’avaient pas les médiateurs malfaisants et trompeurs (les démons) (…). Aussi le bon et véritable Médiateur a-t-il montré que le mal, c’est le péché, non la substance ou la nature de la chair » (CD X, XXIV). 
      
        \end{quote} 

%--------------------------------------------
\section{ Le Christ, fondateur d’une « religion » ? }





\begin{Synthesis}
Parce que la vraie religion cherche le bonheur, Saint Augustin a montré que le Christianisme propose le médiateur. Hors de cette voie, personne n'a été délivrée et ne sera délivrée.
\end{Synthesis}

  \begin{quote}
      Les hommes trouvent en Christ « la plus miséricordieuse des purifications, celle de l’intelligence, celle de l’esprit et celle du corps. Car s’il a assumé l’homme tout entier, sans le péché, c’est pour le guérir de la peste des péchés tout ce qui constitue l’homme » (CD X, XXVII). 
  \end{quote}
  
  \begin{quote}
      « Hors de cette voie qui n’a jamais fait défaut au genre humain, soit au temps où ces événements étaient prédits comme devant s’accomplir, soit au temps où ils sont annoncés comme déjà accomplis, personne n’a été délivré, personne n’est délivré, personne ne sera délivré » (CD, X, XXXII, 2). 
  \end{quote}
  
  D'une certaine façon, il est encore plus exigeant que \textit{hors de l'Eglise point de Salut}. Concentration christocentrique de \CD. La voie est traduite sur le plan religieux.
  

\begin{quote}
     « Ils ne veulent pas (…) que nous les honorions comme nos dieux, mais que nous adorions avec eux leur Dieu et le nôtre ; ils ne veulent pas que nous leur offrions des sacrifices, mais que nous soyons avec eux un sacrifice offert à Dieu » (CD X, XXV). 
\end{quote}


\begin{quote}
    Le médiateur est celui qui est en même temps le sacrifice et le prêtre : « Elle est donc vaincue au nom de celui qui a assumé l’homme et a mené sans péché une vie humaine, pour que la rémission des péchés s’opérât en lui, prêtre et sacrifice, médiateur entre Dieu et les hommes, l’Homme-Jésus-Christ  par qui, purifiés de nos péchés, nous sommes réconciliés avec Dieu » (CD X, XXII). 
\end{quote}
\begin{quote}
      « Voilà donc cette voie universelle de la délivrance de l’âme ! C’est elle qu’annoncèrent les saints anges et les saints prophètes. S’adressant d’abord là où ils le purent à un petit nombre d’hommes et particulièrement au peuple hébreu […] ils en ont donné des signes : le tabernacle, le temple, le sacerdoce et les sacrifices » (CD X, XXXII, 2). 
  \end{quote}
  \paragraph{s'il est la voie, le Christ est il devenu Religion ?}
  
 
  
  La démonstration de Saint Augustin est en deux parties : 
  \begin{itemize}
      \item   discrédit sur toutes les religions
  \item aspiration à toute la religion en Christ : le Christ absorbe en lui-même la religion car il est le médiateur par excellence. UIl n'est pas seulement celui qu'on adore mais celui en qui on adore.
  \end{itemize}

  \subsection{Le Christ exprimé en termes religieux (inculturation)}
  
  Saint Augustin introduit le concept de religion dans le Christianisme. Dans \CD 10, il intègre les termes grecs en latin.
  
  latria (idolatrie) se traduit par \textit{hommage}. \textit{thrèskeia} est traduit par \textit{Religio}, en ce sens d'un lien qui nous rattache à Dieu. \textit{theosebeia} par \textit{Dei cultus}
  
  \begin{quote}
    « Le terme grec \textit{latreia} se rend en latin par \textit{servitus}, mais avec le sens d’un hommage rendu à Dieu  le grec \textit{thrèskeia} par \textit{religio}, mais avec le sens d’un lien qui nous rattache à Dieu ; et le mot grec \textit{theosebeia} par deux mots latins :\textit{ Dei cultus} : ce qui désigne pour nous le culte exclusivement réservé à Dieu, au vrai Dieu qui rend ‘dieux’ ceux qui l’honorent » (CD, X,I,3). 
\end{quote}



Regardons le NT : 
4 occurrences de thrèskeia, 1 seule occurrence de theosebeia, traduit par crainte de Dieu, profession de piété (1 Tm 2,8)
Dans Col 1,18, thrèskeia est utilisé pour \textit{culte des anges}. La "religion pure", c'est d'aller voir les veuves ? 
latreia est plus utilisé, vient de \textit{salaire} et appliqué au culte paien dans le NT : victime animale, ou victime vivante. On n'est pas dans le rite mais plutot dans l'apostolat.

Dans \CD, 10-3, très beau texte où il spiritualise le culte.

\begin{quote}
    « A (Dieu) nous devons le service appelé en grec \textit{latreia}, soit dans le rites sacrés, soit en nous-mêmes. Car tous ensemble et chacun, nous sommes son temple (…) Quand notre cœur s’élève vers lui, il est son autel ; son Fils est le prêtre par qui nous le fléchissons. A lui nous immolons des victimes sanglantes quand nous combattons jusqu’au sang pour sa vérité ; pour lui nous brûlons l’encens le plus suave lorsqu’en sa présence nous sommes enflammés d’un pieux et saint amour ; à lui nous vouons et nous rendons les dons qu’il nous faits et nos personnes mêmes ; nous publions et consacrons la mémoire de ses bienfaits en des fêtes solennelles et à des jours fixés, de peur qu’au cours du temps ne se glisse en nous un ingrat oubli ; à lui nous sacrifions sur l’autel de notre cœur une hostie d’humilité et de louange au feu d’une fervente charité. Pour le voir comme il pourra être vu et pour nous unir à lui, nous nous purifions de toute souillure des péchés et des mauvaises convoitises, et par son nom nous nous consacrons. Car il est lui-même la source de notre béatitude et le terme total de notre aspiration » (CD X,III, 2). 
\end{quote}  
  
 \paragraph{une transposition de la geste de la Lettre aux Hébreux} Pour Saint Augustin, il spiritualise beaucoup en utilisant les termes romains mais en changeant leur sens, moralisation du culte à Dieu comme dans la lettre aux Hébreux. 
  Malgré le changement de sens, rattache la foi chrétienne aux autres religions. Dans la lettre aux Hébreux, il y avait une discontinuité mais on utilise les références juives. Ici, il fait le même travail mais avec le vocabulaire romain. Il transpose le raisonnement de la lettre aux Hébreux en l'appliquant à Rome. 
  
  \paragraph{le mot Alliance plus juste ?} On utiliserait dans le contexte de He \textit{l'Alliance} plutôt que \textit{religio}. Le risque en utilisant \textit{religio}, c'est de récupérer tout ce que ce mot charrie dans le monde romain.
  
  
  
  
\subsection{Christologisation de l’idée de médiation}

\paragraph{Le Christ est médiateur en tant qu’il vrai Dieu et vrai homme } 
La médiation : 
  \begin{itemize}
     \item dans le neoplatonisme, par des intermédiaires
      \item dans les religions paiennes, le sacrifice
  \end{itemize}
  L'un des titres que l'on retrouve dans toute l'oeuvre de Saint Augustin, c'est le titre de \textsc{médiateur}, l'un des concepts les plus féconds de Saint Augustin. Il va sortir le terme de médiateur du contexte de l'Alliance vers celui des religions. On transforme la vision de \textit{He} en le mettant dans un contexte des religions du monde romain.  \sn{on en retrouvé en Allemagne des textes de Saint Augustin et dans un sermon de 404, il passe de la critique du paganisme à celui du rôle de médiateur}. 
  Ainsi le verbe de Dieu, proche de nous, se fait chemin, la patrie et voie vers Dieu.
  
  \paragraph{un siècle avant, le concile de Nicée avait mis en avant la divinité de Jésus} A peine un siècle plus tard, Saint Augustin souligne son humanité, son humilité. 
  
  \paragraph{Par quel intermédiaire on accède à Dieu} \emph{Mesites}, intermédiaire, pont entre l'homme et Dieu. Dans l'AT, on a : le Roi, le prêtre, le prophète, les anges,... 5 occurrences de \emph{mesites}\sn{code strong 3316 Galates 3 : 19		Pourquoi donc la loi ? Elle a été donnée ensuite à cause des transgressions, jusqu'à ce que vînt la postérité à qui la promesse avait été faite; elle a été promulguée par des anges, au moyen d'un médiateur (mesites).} dont 3 dans \textit{He}.  \sn{il y a vraiment une influence forte de \textit{He} dans la reflexion sur le rôle de médiateur du Christ chez Augustin.}
\begin{quote}
    Galates 3 : 20		Or, le médiateur (mesites) n'est pas médiateur d'un seul, tandis que Dieu est un seul.

\textbf{1 Timothée 2 : 5		Car il y a un seul Dieu, et aussi un seul médiateur (mesites) entre Dieu et les hommes, Jésus-Christ homme,}

Hébreux 8 : 6		Mais maintenant il a obtenu un ministère d'autant supérieur qu'il est le médiateur (mesites) d'une alliance plus excellente, qui a été établie sur de meilleures promesses.

Hébreux 9 : 15		Et c'est pour cela qu'il est le médiateur (mesites) d'une nouvelle alliance, afin que, la mort étant intervenue pour le rachat des transgressions commises sous la première alliance, ceux qui ont été appelés reçoivent l'héritage éternel qui leur a été promis.

Hébreux 12 : 24		de Jésus qui est le médiateur (mesites) de la nouvelle alliance, et du sang de l'aspersion qui parle mieux que celui d'Abel.
\end{quote}
Dans Ga, c'est Moise qui est le médiateur, sinon, c'est Jésus. Georges Remy a compté 135 fois 1 Tm 2,5 ! 
\begin{Synthesis}[médiateur dans la lettre aux Hebreux]
l'idée de médiateur est lié en He à l'Alliance. Non pas dès son incarnation où la nature divine rejoint la nature humaine, mais c'est en raison de son sacrifice de réconciliation. 
\end{Synthesis}
  

 \subparagraph{l'unique sacrifice aimable à Dieu}Le christ n'est pas intermédiaire mais \textit{médiateur}, en tant qu'il s'est fait chair. Augustin réconcilie les deux visions : 
   
  \begin{quote}
    « Notre corps également, quand nous le mortifions par la tempérance, est un sacrifice (…) Si donc le corps, cet être inférieur dont l’âme se sert comme d’un serviteur ou d’un instrument, est un sacrifice quand cet usage bon et droit est rapporté à Dieu, combien plus l’âme ellemême est-elle un sacrifice quand elle se réfère à Dieu» (CD X, VI).  « Les vrais sacrifices sont les œuvres de miséricorde soit envers nous-mêmes soit envers le prochain, que nous rapportons à Dieu. L’unique but de ces œuvres est de nous délivrer du malheur et par suite, nous procurer le bonheur » (CD X, VI). 
\end{quote}

le vrai sacrifice, ce sont les oeuvres de miséricorde, l'amour du prochain. Mt 25.

\paragraph{Le Christ est médiateur en tant qu’il s’est fait chair }    

Va se développer la dimension sociale.

\begin{quote}
    « La voici la très glorieuse Cité de Dieu, celle qui connaît et adore un seul Dieu, celle que nous ont annoncée les saints anges en nous invitant à en faire partie et en souhaitant que nous soyons en elle leurs concitoyens » (CD X, XXV).  
\end{quote}

\begin{quote}
    « Toute œuvre qui contribue à nous unir à Dieu dans une sainte société, à savoir toute œuvre rapportée à ce bien suprême grâce auquel nous pouvons être véritablement heureux. Voilà pourquoi la miséricorde elle-même qui nous fait secourir notre semblable, si on ne la pratique pas en vue de Dieu, n’est pas un sacrifice.  Car même accompli ou offert par l’homme, le sacrifice n’en est pas moins une chose divine (…) En conséquence, l’homme consacré par le nom de Dieu et voué à Dieu, en tant qu’il meurt au monde pour vivre à Dieu, est un sacrifice » (CD X, VI). 
\end{quote}
\subsection{De la christologie du médiateur à la christologie sociale/inclusive}    

\begin{quote}
    « Cette Cité rachetée tout entière, c-à-d l’assemblée et la société des saints, est offerte à Dieu comme un sacrifice universel par le Grand Prêtre qui, sous la forme d’esclave, est allé jusqu’à s’offrir pour nous dans sa passion, pour faire de nous le corps d’une si grande Tête. C’est en effet cette forme qu’il a offerte, c’est en elle qu’il s’est offert, parce que c’est grâce à elle qu’il est le Médiateur, en elle qu’il est prêtre, en elle qu’il est sacrifice. » (CD X, VI). 
\end{quote}
\begin{quote}
    « Deux amours ont donc fait deux cités : l’amour de soi jusqu’au mépris de Dieu, la cité terrestre ; l’amour de Dieu jusqu’au mépris de soi, la Cité céleste » (CD XIV,28). 
\end{quote}





