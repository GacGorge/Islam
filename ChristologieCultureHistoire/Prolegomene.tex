\hypertarget{christologies-et-cultures-dans-lhistoire-2-proluxe9gomuxe8nes-ii}{%
\chapter{Prolégomène II : Théorie de la tradition christologique}\label{christologies-et-cultures-dans-lhistoire-2-proluxe9gomuxe8nes-ii}}





\hypertarget{eluxe9ments-bibliographiques}{%
\section{Eléments bibliographiques}\label{eluxe9ments-bibliographiques}}


ALETTI, J.-N., \emph{Jésus-Christ fait-il l'unité du NT ?}

ALETTI, J.-N., « Mystère » dans Dictionnaire critique de théologie.
CONCILE VATICAN II, \emph{Constitution dogmatique} Dei Verbum.

GUARDINI, R., \emph{L'essence du christianisme}.

PANNENBERG, W., \emph{Esquisse d'une christologie}, Paris, Cerf,
1993\textsuperscript{2}. VINCENT DE LERINS, \emph{Comminotorium}.

ZUMSTEIN, J., \emph{L'évangile selon st Jean}.


%-----------------------------------------------------------------------------------------------------

\section{Introduction} 




  On ne peut pas éluder la question Christologique de la façon dont on a été amené à poser la question : \textit{herméneutique}
%-----------------------------------------------------------------------------------------------------
  \section{Une théorie de la tradition christologique}
  

    \subsection{Qu'entend-t-on par théorie de la tradition christologique ?}
    

\paragraph{Tradition au sens de transmission} 
\begin{Def}[Tradition Christologique]
  Transmet ou on annonce le Christ en présentant dans un contexte particulier Jésus le Christ.
\end{Def}
Le Christ est toujours transmis.
\begin{Def}[Théorie]
  Rechercher les règles, les lois.
\end{Def}
  
  \paragraph{Quelle est la logique ?} Il n'y a évidemment pas de questions si on postule qu'il n'y a pas d'évolution. Mais si cette annonce est \textit{dialogale}, elle va évoluer.
  
  \paragraph{A partir de l'histoire de la Christologie} Quelles sont les Lois qui sous-tendent cette histoire.
    
    \subparagraph{Phase fondamentale ou originelle} Phase où l'on va réfléchir au Christ à la suite de sa mort et sa Résurrection, dans le contexte de son époque (Judée, I er siècle).
    
    \subparagraph{Phases ultérieures} Faut il répéter les mêmes choses ou réintérpéter ? Comment penser la façon dont ces christologies ultérieures ont concilier la phase fondamentale (le NT) et la culture de l'époque ?

\subsection{Une théorie classique du développement théologique}

On a progressivement penser les choses.
    
      
      \paragraph{la théorie d'un développement homogène}
      
      \subparagraph{Vincent de Lérins} Depuis le XVI, on a redécouvert un texte de Vincent de Lérins (V) \sn{Vincent de Lérins, \textit{Commonitorium}} : développement organique des vérités de la Foi, toujours dans le même sens. Il développe un moyen de voir comment fonctionne les vérités de Foi par rapport aux nouveautés, hérétiques :  \emph{Quod ab omnibus, quod ubique, quod semper ?} (par tous, partout, toujours) : universalité, antiquité, unanimité. 
      
      \begin{quote}
          «  Ce  progrès  constitue  vraiment  pour  la  foi  un  progrès  et  non  une  altération  ;  le  propre  du progrès  étant  que  chaque  chose  s’accroît  en  demeurant  elle-même,  le  propre  de  l’altération qu’une  chose  se  transforme  en  une  autre.  Donc  que  croissent  et  que  progressent  largement l’intelligence,  la  science,  la  sagesse  (…)  mais  à  condition    que  ce  soit  exactement  selon  leur nature  particulière,  c’est-à-dire  dans  le  même  dogme,  dans  le  même  sens,  dans  la  même pensée.  Qu’il  en  soit  de  la  religion  des  âmes  comme  du  développement  des  corps.  Ceux-ci déploient  et  étendent  leurs  proportions  avec  les  années,  et  pourtant  ils  restent  constamment  les mêmes  (…)  Rien  de  nouveau  n’apparaît  chez  l’homme  âgé  qui  aurparavant  déjà  n’ait  été caché  dans  l’enfant»  (Vincent  de  Lérins,  Commonitorium, ch. 23).   
      \end{quote}
      \subparagraph{des critères difficiles} Ces critères sont compliqués à appliquer en pratique, mais cela donne des directions. 
      \begin{Ex}[Immaculée Conception]
        pas toujours cru, comme Thomas d'Aquin.
      \end{Ex}
  
      \subparagraph{Développement homogène} Existe-t-il un progrès du dogme ? Progrès et non une altération ? progrès : chaque chose progresse en \textit{demeurant elle-même} : le progrès est organique. Une altération : on change de nature. Progrès : dans le même dogme, dans le même sens.
      
      \subparagraph{Maurice Blondel (XX) } on passe de l'implicite vécu à l'explicite connu. Toutes les vérités de Foi, elles sont vécus avant d'être connus. 
      
      \subparagraph{Dei Verbum : modèle subjectivo-ecclésiale} DV8 : On utilise le mot "prise de conscience" plutot que progrès organique. Dans le contexte optimiste des 30 glorieuses..
      \begin{quote}
Cette Tradition qui vient des Apôtres progresse dans l’Église [12], sous l’assistance du Saint-Esprit ; en effet, la perception des réalités aussi bien que des paroles transmises s’accroît, soit par la contemplation et l’étude des croyants qui les méditent en leur cœur (cf. Lc 2, 19.51), soit par l’intelligence intérieure qu’ils éprouvent des réalités spirituelles, soit par la prédication de ceux qui, avec la succession épiscopale, ont reçu un charisme certain de vérité. Ainsi l’Église, tandis que les siècles s’écoulent, tend constamment vers la plénitude de la divine vérité, jusqu’à ce que soient accomplies en elle les paroles de Dieu.
      \end{quote}
 
    
      
      \paragraph{les limites d'une telle théorie}
      
    \subparagraph{Monde clos}      On est toujours dans un cercle clos, sans véritable dialogue avec le Monde. \textit{Autoréférencés}. Tout progrès n'est que le développement de ce qui est en germe. Est-ce suffisant de penser comme cela ?
    
    \subparagraph{Une approche qui garde sa pertinence en soulignant l'aspect définitif de la Révélation en Christ} On avait une approche théorique, de \textit{déduire les choses}, descendantes, déductives et descendantes. 
    \begin{Ex}
      En latin, et on l'applique au chinois même s'il n'y a pas de terme de Dieu en chinois.
  
    \end{Ex} 
    
    
    \subparagraph{Urs von Balthasar} Il éclaire des aspects tout à fait nouveau de la vérité infinie. Des choses nouvelles peuvent apparaître. 
    
    
    \subsection{Introduire une dimension dialogale dans la théorie du
    développement organique}
    
    Il faut rééquilibrer l'approche.
    \paragraph{Ecart entre l'annonce du Christ et l'histoire de Jésus} Cet écart n'est il pas le modèle de tous les écarts d'application du Christ dans une nouvelle culture. 
    Il nous faut recevoir d'une nouvelle manière le Christ; une christologie. 
    La représentation que nous avons du Christ doit mourir pour renaître à de nouvelles représentations. Toujours une continuité mais purifiée. 
  
  \paragraph{Difficulté à transmettre le Christ sans transmettre la culture} On a souvent transmis la culture occidentale en transmettant l'Evangile. Deux contre-exemples intéressants : 
  \begin{itemize}
      \item \textsc{Matteo Ricci} face à une culture Chinoise beaucoup plus sophistiquée, Matteo Ricci décide de s'acculturer
      \item l'intérêt de regarder les \textit{hérésies} (cf Arianisme et Barbares) pour voir si le discours sur le Christ doit s'adapter, en revenant au mystère du Christ.
  \end{itemize}
  
  \paragraph{Revenir au mystère du Christ}
%-----------------------------------------------------------------------------------------------------
  \section{Le mystère du Christ }

A l'origine de la Christologie, il y a \textit{l'expérience du Christ comme mystère}, dans le sens Paulinien, la venue salvifique de Dieu, sa Révélation : 

  
  \begin{Def}[mystère]
  le mystère n'est pas ce qui nous est caché, mais Dieu se révèle comme mystère, c'est à dire dans sa dimension paradoxale, quelque chose qui nous saisit et pas quelque chose que nous pouvons saisir.
  \end{Def}
   \begin{quote}
       Le mystère en christianisme est un fait qui relève de l'histoire du salut. Le Nouveau Testament, et particulièrement saint Paul, emploie le terme "mystère de Dieu" (Col 2, 2) pour parler de "toute l'histoire sainte, depuis la venue du Christ ici-bas jusqu'à sa Parousie. L’Évangile est la \textit{révélation} de ce mystère [...]". Le mystère, dans la foi chrétienne, n'est pas ce qu'on ne peut comprendre, mais ce qu'on n'a jamais fini de comprendre, et qui ne peut être compris de façon ultime que dans la foi. \sn{Wikipedia}
   \end{quote}
    
\subsection{L'expérience de la résurrection de Jésus}
    
    \paragraph{Jésus est le Juste} celui qui est ajusté au Père
    
    \paragraph{Eschatologie} Pas une action divine parmi d'autres, une action eschatologique (définitive) qui met en jésus un terme à l'histoire. "on célèbre la fin du monde chaque année à Pâques." 
    \paragraph{Règne de Dieu} Le jour du Ywhw, sachant l'attente apocalyptique juive. Désormais le Règne de Dieu est advenu. Alors, si Dieu est maitre de l'histoire, alors l'histoire dit qui il est. Et la Résurrection de Jésus d'entre les morts, nous dit qui Dieu est : 
  \begin{quote}
    Rm  10,9  :  «  Si  tes  lèvres  confessent  que  Jésus  est  Seigneur  (divinité)  et  si  ton  cœur  croit  que Dieu l’a  ressuscité  des  morts, tu seras  sauvé».   
    
    Jn 14,9  :  «  Qui  me  voit, voit  le  Père  »   
    
    Jn 10,30  :  «  Le  Père  et  moi  nous  sommes  un  » 
\end{quote}
    
    \begin{Prop}[Unité de Jésus]
    Le révélateur ne peut pas être différent au Révélé, d'où Jésus est un avec son Père
    \end{Prop}
    Dieu n'a pas seulement ressuscité un homme, il s'est révélé par cet acte : \textit{auto-révélation de Dieu}. Si Dieu se révélait par un intermédiaire, on n'aurait pas une connaissance de Dieu mais de son intermédiaire.
    \paragraph{Jésus est Seigneur} Affirmation de l'unité du Christ et de Dieu.
    
    
\subsection{L'expérience de l'unité de cet homme crucifié, Jésus, avec Dieu}
    
    \paragraph{Résurrection} On pense souvent uniquement à ce qu'il arrive à Jésus : réveil, se mettre debout. Mais il y a aussi une possibilité d'être nouvelle pour le Christ.
    \begin{quote}
        «  Dans  la  résurrection  de  Jésus,  une  nouvelle  possibilité  d’être  homme  a  été  atteinte,  une possibilité  qui  intéresse  tous  les  hommes  et  ouvre  un  avenir,  un  avenir  d’un  genre  nouveau pour  les  hommes  (…)  Ou  bien  la  résurrection  du  Christ  est  un  événement  universel  ou  elle n’est  pas, nous  dit  Paul  »  \sn{Ratzinger, \textit{ Jésus  de  Nazareth}**, 278}.   
    \end{quote}
   Si cela ne nous concerne pas, tant mieux pour lui mais pour nous ? Expérience de Paul que le Christ est uni à tous les hommes dans cette expérience. 
   \begin{quote}
       1  Co  15,16.20  :  «  Car  si  les  morts  ne  ressuscitent  pas,  le  Christ  non  plus  n’est  pas ressuscité… Mais  non,  le  Christ  est  ressuscité  d’entre  les  morts,  prémices  de  ceux  qui  se  sont endormis  ». 
   \end{quote}
    

\paragraph{Le maillon : les disciples} qui commencent historiquement à témoigner que le Christ est ressuscité. 


\subsection{L'expérience de l'unité dynamique de cet homme, Jésus, avec
    tous les hommes}
    

\paragraph{Unité} entre le Christ et Dieu est attestée. Mais l'Unité entre le Christ et tous les hommes, une expérience dynamique, marquée par l'histoire. \textit{Tout homme est enclos en lui}. Un évènement collectif attendu (la Ressurection des morts) se manifeste en un seul individu, le Christ. Comment le penser ? Les temps sont accomplis, proximité du Royaume en Jésus-Christ, plus besoin de se marier... et les années passent. 

\paragraph{Le salut est universel} et nous avons à l'annoncer. pas à rendre le salut universel. 

\paragraph{Repas avec ses disciples}, figure de la communion avec tous les hommes. 
Lc 24, 31
Jn 21
1 Co 15 "pour la multitude"

\paragraph{Toute l'humanité rentre dans la vie divine} 1Co15, 45 : "dernier Adam". Jésus devient le \textit{type } de l'humanité. Adam annonçait l'humanité nouvelle qui est le Christ, premier né de toute créature (Col 1,15),  
  
 \begin{quote}
     «  Par  son  incarnation,  le  Fils  de  Dieu  s’est  en  quelque  sorte  uni  lui-même  à  tout  homme  »  (GS 22, 2). 
 \end{quote}
 
 Idée de cette union de toute l'humanité, non pas par force. Cette proximité s'exprime sous le forme de l\textit{alliance}, de tout homme avec Dieu, \textit{inclusion } de tout homme.
    
\subsection{Le mystère du Christ comme mystère du salut}
    
    \paragraph{Expérience d'un double lien}, lien entre le Christ avec Dieu, et lien entre le Christ et tout homme. Le lien \textit{salvifique} de Jésus.
    La Résurrection fait voir de façon fulgurante ce lien qui nous sauve, et c'est ainsi que le Royaume de Dieu vient. Les disciples font cette expérience de la Résurrection, originale et vont l'essayer de l'exprimer : "il se fait voir".
    Cette expérience s'impose aux témoins. 
    
    On ne comprendrait pas la mission des disciples non pas par l'ordre de Jésus (de toutes les nations, faites des disciples), pourquoi il y a ce mandat ? la réalité, c'est que Jésus est proche de tous les hommes. \textsc{Cette Union avec le monde est présente dès la Résurrection et avant la rencontre}. Christ est ressuscité : on n'a pas à les convaincre, le Christ est déjà présent dans leur vie. Les disciples avaient trouvé la perle, ils ont été bousculés, témoins de l'évènement.
    
    \begin{quote}
            «  Puisque  le  Christ  est  mort  pour  tous  et  que  la  vocation  dernière  de  l’homme  est  réellement unique,  à  savoir  divine,  nous  devons  tenir  que  l’Esprit-Saint  offre  à  tous  (…)  la  possibilité d’être  associé  au mystère  pascal  »  (GS  22,5). 
    \end{quote}

     
  
    
\subsection{Le mystère du Christ et la règle de foi christologique
    (matrice)}
    
  
     \paragraph{Jésus est le Médiateur}, vraiment uni à Dieu et à tous les hommes. La bonne nouvelle, c'est qu'en étant en communion avec Dieu, nous recevions la vie, et la vie en plénitude. Dans les lettres de Paul, le mystère dans Col et Eph est équivalent à la bonne nouvelle
     
     \begin{quote}
         «  En  Col  et  Ep,  lettres  plus  tardives,  sans  supprimer  le  mot  ‘Evangile’  de  son  vocabulaire, Paul  lui  adjoint  un  autre  vocable,  celui  de  ‘mystère’,  pour  décrire  le  contenu  de  son  annonce  : ‘Priez  aussi  pour  moi,  afin  qu’il  me  soit  donné  d’ouvrir  la  bouche  pour,  avec  assurance,  faire connaître  le  mystère  de  l’Evangile’  (Ep  6,19).  S’il  ne  s’étend  pas  davantage  sur  le  sens  de cette  expression,  c’est  parce  qu’il  a  plus  longuement  parlé  ‘du  mystère’  dans  les  chapitres précédents  (Ep  2-3),  en  soulignant  ses  dimensions  christologique  et  ecclésiologique.  Ce  qui vient  d’être  dit  sur  Ep  vaut  a  fortiori  pour  Col,  où  la  composante  christologique  du  mystère  est 2 Christologies  et  cultures  dans  l’histoire  ISTR  2021-2022 3 dominante  [voir  Col  1,24  –  2,5],  et  qu’on  peut  résumer  ainsi  :  Christ  parmi  les  nations  (Col 1,27) \sn{J.-N. Aletti,  Jésus-Christ  fait-il  l’unité  du NT  ?  p. 31}. 
     \end{quote}
    
     
     \paragraph{Unité Salvifique de Dieu} parce qu'il est Un avec Dieu et \textit{un avec tous les hommes}
     
     \paragraph{Reconnaissance paradoxale du Christ ressuscité} à la fois dans sa corporeité et dans sa liberté.
     \begin{quote}
         Mystérieuse existence du ressuscité; ... corporeité (il mange du poisson), montrant son union avec nous, et sa profonde liberté \sn{REVOIR J. Ratzinger, Jésus de Nazareth}
     \end{quote}

    \paragraph{La théologie n'est pas à son service} La question de la transmission du mystère du Christ se pose donc, alors que le mystère est doublement paradoxal, comment éviter que le mystère s'affadisse. 
    
    
    
%-----------------------------------------------------------------------------------------------------
\section{La christologie : transmettre le mystère du Christ pour faire croire en Christ} 

  Si le Christ nous sauve, être avec lui nous permet d'être en communion avec lui, comment transmettre ?
  
  \paragraph{Nous ne l'avons pas vécu, expérimenté} 
  
    
    
\subsection{A l'origine de la christologie : de l'évidence à
    l'in-évidence}
    
  
  \paragraph{De l'évidence à l'inevidence} A l'origine de la Christologie, c'est évident. mais la parousie se fait attendre. Quand les premiers disciples ont dû s'adapter. 1 Th 4, 13-18 \sn{Thessalonicien : premier récit du NT}
  \begin{quote}
     1  Th  4,13-18  :  «  Nous  ne  voulons  pas,  frère,  vous  laisser  dans  l’ignorance  au  sujet  des  morts, afin  que  vous  ne  soyez  pas  dans  la  tristesse  comme  les  autres,  qui  n’ont  pas  d’espérance.  Si  en effet  nous  croyons  que  Jésus  est  mort  et  qu’il  est  ressuscité,  de  même  aussi  ceux  qui  sont morts,  Dieu,  à  cause  de  ce  Jésus,  à  Jésus  les  réunira.  Voici  ce  que  nous  vous  disons,  d’après une  parole  du  Seigneur  :  nous,  les  vivants,  qui  seront  restés  jusqu’à  la  venue  du  Seigneur, nous  ne  devancerons  pas  du  tout  ceux  qui  sont  morts….Réconfortez-vous  donc  les  uns  les autres  par cet  enseignement  »   
  \end{quote}
  Par la suite, il faudra se rendre à l'évidence, le Seigneur n'est pas venu. On élabora le récit Christologique, pour aider les générations suivantes à attendre la Parousie, moins kerygmatique et plus pédagogique. Pourquoi on a mis par écrit uniquement au bout d'une génération ?
  
  \paragraph{La conclusion de l'Ev. de Jn} est exemplaire. Jn 20, 30-31
  
  \begin{quote}
      Jn  20,30-31  :  «  Jésus  a  opéré,  sous  les  yeux  de  ses  disciples  bien  d’autres  signes  qui  ne  sont pas  rapportés  dans  ce  livre.  Ceux-ci  l’ont  été  pour  que  vous  croyiez  que  Jésus  est  le  Christ,  le Fils  de  Dieu, et  pour que, en croyant, vous  ayez  la  vie  en son nom  ». 
  \end{quote}
    
    
    \paragraph{On passe du voir à la Foi / Croire}
    Ainsi, lorque le disciple arrive au tombeau : 
    \begin{quote}
        Jn 20,8  :  «  il  \textbf{vit}  et  crut  »   
    \end{quote}
    En revanche, St. Thomas n'a pas vu, il passe au \textit{croire} : 
    \begin{quote}
            Jn 20,29  :  «  Parce  que  tu m’as  vu, tu as  cru  ;  bienheureux ceux qui, sans  avoir vu, ont  cru  »   
    \end{quote}
    Zumstein, c'est l'expérience du tombeau vide pour les premiers disciples, doit être mis en récit pour les futurs croyants. Un discours a besoin d'être implémenté pour ceux qui n'ont pas vu. Elle rend évident ce qui n'était pas évident : \textsc{La christologie nait d'un éloignement}. Ainsi, la conversion de Paul : 
   \begin{quote}
       Ac  26,13.15-16  :  «  Vers  midi,  je  vis,  ô  roi,  venant  du  ciel  et  plus  éclatante  que  le  soleil  qui resplendit  (…)  Le  Seigneur  dit  :  ‘Je  suis  Jésus,  que  tu  persécutes.  Mais  relève-toi  et  tiens-toi debout.  Car  voici  pourquoi  je  te  suis  apparu  :  pour  t’établir  serviteur  et  témoin  de  la  \emph{vision dans  laquelle  tu viens  de  me  voir}  ». 
   \end{quote}
    Mais après cela ne va plus s'imposer, écluse.
    
    
\subsection{La double tâche de la christologie}

\paragraph{la première interprétation est l'interprétation elle-même} pour les premiers chrétiens. 

\paragraph{mais ensuite il faudra renouveler cette interprétation}
  \begin{quote}
      «  Dieu,  qui  parla  jadis,  ne  cesse  de  converser  avec  l’Epouse  de  son  Fils  bien-aimé,  et  l’Esprit Saint,  retentit  dans  l’Église  et,  par  l’Église,  dans  le  monde,  introduit  les  croyants  dans  la  vérité tout  entière  »  (DV  8). 
  \end{quote}

%-----------------------------------------------------------------------------------------------------
  \section{Penser la logique de la tradition
  christologique}

  
  
  
    
    \subsection{Principe dialogal et principe introspectif de la christologie}
    
  
    
    \subsection{Les deux phases de la tradition christologique}
    

    
      
      \paragraph{La première phase : le NT}
      
    
      
      \paragraph{La seconde phase : l'histoire des christologies}
      
    
  
    
    \subsection{Plan du cours}
    
  



%-----------------------------------------------------------------------------------------------------
\section{Textes} 

\begin{quote}
    « Ce progrès constitue vraiment pour la foi un progrès et non une
altération ; le propre du progrès étant que chaque chose s'accroît en
demeurant elle-même, le propre de l'altération qu'une chose se
transforme en une autre. Donc que croissent et que progressent largement
l'intelligence, la science, la sagesse (\ldots) mais à condition que ce
soit exactement selon leur nature particulière, c'est-à-dire dans le
même dogme, dans le même sens, dans la même pensée. Qu'il en soit de la
religion des âmes comme du développement des corps. Ceux-ci déploient et
étendent leurs proportions avec les années, et pourtant ils restent
constamment les mêmes (\ldots) Rien de nouveau n'apparaît chez l'homme
âgé qui aurparavant déjà n'ait été caché dans l'enfant» \sn{Vincent de
Lérins, \emph{Commonitorium}, ch. 23.}
\end{quote}

\begin{quote}
    « Cette Tradition qui vient des apôtres se poursuit dans l'Église, sous
l'assistance du Saint- Esprit : en effet, la perception des choses aussi
bien que des paroles transmises s'accroît, soit par la contemplation et
l'étude des croyants qui les méditent en leur cœur (\ldots), soit par
l'intelligence intérieure qu'ils éprouvent des choses spirituelles, soit
par la prédication de ceux qui, avec la succession épiscopale, reçurent
un charisme certain de vérité. Ainsi l'Église, tandis que les siècles
s'écoulent, tend constamment vers la plénitude de la divine vérité,
jusqu'à ce que soient accomplies en elle les paroles de Dieu » (DV 8).
\end{quote}

\begin{quote}
    Rm 10,9 : « Si tes lèvres confessent que Jésus est Seigneur (divinité)
et si ton cœur croit que Dieu l'a ressuscité des morts, tu seras sauvé
».
\end{quote}

\begin{quote}
    Jn 14,9 : « Qui me voit, voit le Père »

Jn 10,30 : « Le Père et moi nous sommes un »
\end{quote}

\begin{quote}
    « Dans la résurrection de Jésus, une nouvelle possibilité d'être homme a
été atteinte, une possibilité qui intéresse tous les hommes et ouvre un
avenir, un avenir d'un genre nouveau pour les hommes (\ldots) Ou bien la
résurrection du Christ est un événement universel ou elle n'est pas,
nous dit Paul » \sn{(Ratzinger, \emph{Jésus de Nazareth}**, 278).}
\end{quote}

\begin{quote}
    1 Co 15,16.20 : « Car si les morts ne ressuscitent pas, le Christ non
plus n'est pas ressuscité\ldots Mais non, le Christ est ressuscité
d'entre les morts, prémices de ceux qui se sont endormis ».
\end{quote}

\begin{quote}
    « Par son incarnation, le Fils de Dieu s'est en quelque sorte uni
lui-même à tout homme » (GS 22, 2).
\end{quote}

\begin{quote}
    « Puisque le Christ est mort pour tous et que la vocation dernière de
l'homme est réellement unique, à savoir divine, nous devons tenir que
l'Esprit-Saint offre à tous (\ldots) la possibilité d'être associé au
mystère pascal » (GS 22,5).
\end{quote}

\begin{quote}
    « En Col et Ep, lettres plus tardives, sans supprimer le mot `Evangile'
de son vocabulaire, Paul lui adjoint un autre vocable, celui de
`mystère', pour décrire le contenu de son annonce : `Priez aussi pour
moi, afin qu'il me soit donné d'ouvrir la bouche pour, avec assurance,
faire connaître le mystère de l'Evangile' (Ep 6,19). S'il ne s'étend pas
davantage sur le sens de cette expression, c'est parce qu'il a plus
longuement parlé `du mystère' dans les chapitres précédents (Ep 2-3), en
soulignant ses dimensions christologique et ecclésiologique. Ce qui
vient d'être dit sur Ep vaut a fortiori pour Col, où la composante
christologique du mystère est
dominante {[}voir Col 1,24 -- 2,5{]}, et qu'on peut résumer ainsi :
Christ parmi les nations (Col 1,27) \sn{J.-N. Aletti, \emph{Jésus-Christ
fait-il l'unité du NT ?} p. 31.}
\end{quote}

\begin{quote}
   1 Th 4,13-18 : « Nous ne voulons pas, frère, vous laisser dans
l'ignorance au sujet des morts, afin que vous ne soyez pas dans la
tristesse comme les autres, qui n'ont pas d'espérance. Si en effet nous
croyons que Jésus est mort et qu'il est ressuscité, de même aussi ceux
qui sont morts, Dieu, à cause de ce Jésus, à Jésus les réunira. Voici ce
que nous vous disons, d'après une parole du Seigneur : nous, les
vivants, qui seront restés jusqu'à la venue du Seigneur, nous ne
devancerons pas du tout ceux qui sont morts\ldots.Réconfortez-vous donc
les uns les autres par cet enseignement » 
\end{quote}

\begin{quote}
    Jn 20,30-31 : « Jésus a opéré, sous les yeux de ses disciples bien
d'autres signes qui ne sont pas rapportés dans ce livre. Ceux-ci l'ont
été pour que vous croyiez que Jésus est le Christ, le Fils de Dieu, et
pour que, en croyant, vous ayez la vie en son nom ».

Jn 20,8 : « il vit et crut »

Jn 20,29 : « Parce que tu m'as vu, tu as cru ; bienheureux ceux qui,
sans avoir vu, ont cru »
\end{quote}



\begin{quote}
   Ac 26,13.15-16 : « Vers midi, je vis, ô roi, venant du ciel et plus
éclatante que le soleil qui resplendit (\ldots) Le Seigneur dit : `Je
suis Jésus, que tu persécutes. Mais relève-toi et tiens-toi debout. Car
voici pourquoi je te suis apparu : pour t'établir serviteur et témoin de
la vision dans laquelle tu viens de me voir ». 
\end{quote}

\begin{quote}
    « Dieu, qui parla jadis, ne cesse de converser avec l'Epouse de son Fils
bien-aimé, et l'Esprit- Saint, retentit dans l'Église et, par l'Église,
dans le monde, introduit les croyants dans la vérité tout entière » (DV
8).
\end{quote}


