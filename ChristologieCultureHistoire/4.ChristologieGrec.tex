

 
\chapter{Ière partie : La christologie se définit dans le monde grec
(IIe-IIIe s.)}

 
\mn{Christologies et cultures dans l'histoire
4} 
La christologie du Logos

\section{Eléments bibliographiques :}

BOVON, F., GEOLTRAIN, P., KAESTLI, J.-D. (dir.), \emph{Écrits apocryphes
chrétiens I et II}, La Pléiade, Gallimard, Paris 1997 et 2005.

BULTMANN, R., \emph{Le christianisme primitif dans le cadre des
religions antiques}, Paris 1969. CLÉMENT D'ALEXANDRIE, \emph{Le
Pédagogue I et III}, Sources Chrétiennes, Paris 1960 et 1970.

DORÉ, J., « Les christologies patristiques et conciliaires » dans
LAURET, B. -- REFOULÉ, F. (dir.),

\emph{Initiation à la pratique de la théologie ** Dogmatique} I, Paris
1982, 185-262.

FÉDOU, M., \emph{La voie du Christ, Genèses de la christologie dans le
contexte religieux de l'Antiquité du IIe siècle au début du IVe siècle},
Paris 2006.

IRENÉE DE LYON, \emph{Contre les Hérésies. Dénonciation et réfutation de
la prétendue gnose au nom menteur}, tr. par A. ROUSSEAU, Paris
1991\textsuperscript{3}.

JONAS, H., \emph{La religion gnostique}, tr. par L. ÉVRARD, Paris 1978.

KLAUCK, H.-J., \emph{L'environnement religieux gréco-romain du
christianisme primitif}, Paris 2012.

MAHÉ, J-P., POIRIER, P-H. (dir.), \emph{Écrits gnostiques. La
bibliothèque de Nag Hammadi}, La Pléiade, Gallimard, Paris 2007.

MACMULLEN, R., \emph{Christianisme et paganisme du IV\textsuperscript{e}
siècle au VIII\textsuperscript{e} siècle}, Paris 1998. POUDERON, B.,
\emph{Les apologistes grecs du II\textsuperscript{e} siècle}, Paris
2005.

RATZINGER, J., \emph{La foi chrétienne hier et aujourd'hui}, Paris
1996\textsuperscript{2}.
 
\hypertarget{introduction}{%
\section{Introduction}\label{introduction}}

\paragraph{Un monde avec des croyances variées} S. Augustin décrira :
\begin{quote}
    une religion civique, politique, 
    une religion mythologique
    une religion, une sagesse philosophique
\end{quote}
Le Christianisme devait s'adapter pour répondre aux besoins et au contexte, traduire dans une autre langue. \textit{traduttore traditore} certes.

\paragraph{la christologie s'adapte}
\paragraph{Phase 1 : traduction} On va adopter le langage de l'autre (image,...) non pas pour faire plaisir pour pour l'annonce de la Bonne Nouvelle. 
\paragraph{Phase 2 : on se démarque} on montre les spécificités chrétiennes : rester fidèle à la Révélation. 
\paragraph{Phase 3 : on demeure marqué par les concepts fondamentaux de l'autre} La christologie prend forme dans une nouvelle culture. Jamais une opération lisse. On oublie certains aspects de la christologie initiale pour en déployer d'autre.
\begin{Synthesis}
Si le Christ est universel, il doit être transmissible.
\end{Synthesis}

En s'inculturant dans le monde grec, la théologie va se \textsc{spiritualiser} et s\textsc{universaliser}. On va adopter une matrice de la culture grecque, la \textsc{la philosophie}, la raison, le logos, la vérité. \mn{Matrice pour la culture chinoise ? unicité ?}

Comment la bonne nouvelle fait son irruption dans le monde grec ?
Trois étapes : 
\begin{itemize}
    \item La confrontation avec la Religion civique : penser le monothéisme.
    \begin{Def}[hénotheisme]
    polytheisme organisé : Un dieu suprême avec des dieux par \textit{administration}. 
    \end{Def}
    \item La confrontation avec la culture gnostique en rappelant la réalité de l'incarnation. On ne sort jamais indemne du combat et de la lutte
    \item La théologie du logos, universalité. 
\end{itemize}

\section{{La confrontation avec le monde religieux païen : polythéisme
  et mythologie}}
  
 \subsection{Le choix du logos contre le mythos}
   
\paragraph{Le logos} les chrétiens ont choisi de s'inculturer dans le monde grec à partir de la raison, du logos, choix conscient des chrétiens alors qu'ils pouvaient choisir le mythos. \sn{Thèse de Ratzinger} 
\begin{quote}
    La prédication chrétienne se trouvait dans un monde saturé de Dieu,... comme Israël face aux autres Nations. La Chrétienté primitive a fait un choix, le Dieu des philosophes contre le Dieu des Religions, choix audacieux. \sn{RATZINGER, J., \emph{La foi chrétienne hier et aujourd'hui}, Paris
1996}
A aucun de ces Dieux, mais nous vénérons le Dieu des philosophes.
\end{quote}
C'est une réponse contre les théologiens libéraux comme Harnack, qui voulaient retrouver la source originelle, cachée par le monde hellenistique.

\paragraph{En se coupant de la vérité} et en s'attachant aux coutumes.

\paragraph{Monogène : deux Dieux ?} Les pères apologistes vont essayer de répondre à ces questions. 
Athénagore d'Athènes \sn{ATHENAGORE, \emph{Supplique}, X,1-5., fin II} va répondre à ceux qui disent que Jésus est le fils comme les autres religions. La confrontation avec les religions païennes
\begin{quote}
« Ce n'est pas à la façon des poètes qui, dans leurs fables, présentent
les dieux comme n'étant en rien meilleurs que les hommes, que nous avons
conçu notre doctrine d'un Dieu qui soit aussi un Père ou celle du Fils ;
mais le Fils de Dieu est le \emph{Logos} du Père en idée et en énergie :
tout a été fait par son opération et son intermédiaire, puisque le Père
et le Fils ne font qu'un. Et comme le Fils est dans le Père, et le Père,
dans le Fils, dans une unité et une puissance spirituelle, le Fils de
Dieu est \textsc{l'intelligence et la raison (logos) du Père.}
(\ldots) Il est le premier à être issu du Père, non pas qu'il soit né

-- car dès l'origine Dieu, qui est \textsc{intelligence éternelle},
portait en lui son Logos, puisqu'il est éternellement raisonnable
(logikos) -, mais parce que alors que toute la matière était dénuée de
qualité, comme une terre non travaillée (\ldots) \textsc{il a procédé
du Père pour leur servir d'idée et d'énergie} (\ldots). »

\end{quote}
On ne dit pas que le Logos est le fils de Dieu. Défend aussi l'universalité. 

\paragraph{mais le culte chrétien semble contredire Athenagore}. Le culte peut être \textit{polylatrie}. \textit{Celse}, Philosophe romain : 
\begin{quote}
    ils rendent un culte excessif à son Fils.
\end{quote}
Origène va répondre à Celse, \sn{Origène, \textit{contre Celse, VIII}} :
\begin{quote}
car nous lui ajoutons foi. lorsqu'il dit: 
\begin{quote}
    Avant qu'Abraham fût, je suis; et encore, 
    
    Je suis la vérité (Jean, VIII, 58 et XIV,6)
\end{quote} 
Il n'y a aucun parmi nous d'un esprit assez grossier pour croire que la vérité ne fût pas un être qui subsistât avant la venue de Jésus-Christ. Ainsi nous adorons le Père de la vérité, et le Fils qui est la vérité, les considérant comme deux choses à l'égard de leur \emph{hypostase} (ou subsistance), mais comme une seule et même chose à l'égard de leur accord, de la conformité de leurs sentiments et de la parfaite union de leur volonté. De sorte que qui a vu le Fils qui est le rejaillissement de la gloire et le caractère de l'hypostase (ou la subsistance) de Dieu, a vu Dieu en voyant celui qui est l'image de Dieu (II Cor.,IV, 4) . Celse veut encore que parce que nous rendons nos hommages et à Dieu et à son Fils, il suive de là que, selon nous, ce n'est pas Dieu seul qu'il faut servir, mais que l'on doit aussi servir ses ministres.
\end{quote}



\subsection{Contre l'idée d'un engendrement charnel, l'engendrement
    spirituel du Fils}
    
  
    
    \subsection{La véritable image contre les images païennes}
    
    \paragraph{les statues, L'image de Dieu est spirituelle} réflexe iconoclase, car il y a de l'idolatrie. Clément d'Alexandrie \sn{CLEMENT D'ALEXANDRIE, \emph{Protreptique}, X, 98,1-4.}.
\begin{quote}
    La véritable image de Dieu est spirituelle.
\end{quote}    
Face aux images paiennes, présenter le Christ, image de Dieu : 
\begin{quote}
« `Image de Dieu' est son Logos\ldots, et image du Logos est l'homme
véritable, l'esprit qui est dans l'homme, et qui est dit, à cause de
cela, avoir été fait `à l'image' de Dieu et `à sa ressemblance',
assimilé au divin Logos par l'intelligence de son cœur et , par là,
raisonnable. Mais les statues à figures humaines ne sont qu'une image
terrestre de l'homme tel qu'on le voit, né de la terre, et elles
n'apparaissent que comme une reproduction passagère bien éloignée de la
vérité. »
\end{quote}
    
      
      \paragraph{Le Logos façonne des images de Dieu}
      
      \paragraph{L'image de Dieu dans le monde, ce sont les croyants} Ce sont les êtres vivants. Plus on est proche du logos, plus notre raison est assimilée au logos divin, et plus nous devenons ressemblant à Dieu et nous devenons des icônes.
      
      \paragraph{Homme devient Dieu par l'action intérieure du logos} Le Christ va rendre visible Dieu, car il est le \textit{logos incarné}. Et nous, de façon dégradée, nous rendons visible Dieu.
      
      \paragraph{L'homme intérieur}\sn{Clément dit gnostique} Ce dialogue parle aux paiens. 

\subsection{Conclusion}

Michel Feydou : 
\begin{quote}
    Refus de l'idolatrie s'enracine dans le logos. 
    l'image la plus parfaite : le Christ. \ldots vrais autels, véritables statues.
\end{quote}
  
  Mais il y a un risque, celui de la gnose, quand on spiritualise la christologie, à mettre en avant le logos et pas la \textit{sarx}, la chair.
  
 ~
  \hypertarget{la-religiosituxe9-gnostique-et-la-christologie-gnostique-du-logos}{%
  \section{La religiosité gnostique et la christologie gnostique du
  Logos}\label{la-religiosituxe9-gnostique-et-la-christologie-gnostique-du-logos}}

  
  
  
    
    \subsection{L'origine obscure du gnosticisme}
    
  
    
    \subsection{Qu'est-ce que la gnose ?}
    
  
    
    \subsection{La doctrine gnostique}
    

    
    
    
      
      L'idée de Dieu
      
    
      
      Le drame divin
      
    
      
      l'anthropologie gnostique
      
    
  
    
    \subsection{Le Christ comme le sauveur gnostique}
    

    
    
    
      
      Le salut gnostique
      
    
      
      La christologie docète
      
    
  
    
    \subsection{Conclusion}
    
  
 ~
  \hypertarget{la-christologie-orthodoxe-du-logos}{%
  \section{La christologie « orthodoxe » du
  Logos}\label{la-christologie-orthodoxe-du-logos}}

  
  
  
    
    \subsection{Le Logos un avec le Père, créateur et principe du monde}
    
  
    
    \subsection{Universalité du Logos et le thème des « semences du Verbe »}
    

    
    
    
      
      Le thème du Logos disséminé : Justin et les autres
      
    
      
      Influence stoïcienne et originalité chrétienne
      
    
  
    
    \subsection{Le Logos est le Sauveur car il nous introduit à la véritable
    gnose}
    

    
    
    
      
      Le Logos comme « pédagogue » de l'homme
      
    
      
      La gnose ou connaissance véritable
      
    
  


\hypertarget{conclusion-la-question-de-la-relation-entre-le-logos-et-juxe9sus-au-cux153ur-de-la-question-christologique}{%
\section{Conclusion : La question de la relation entre le Logos et
Jésus au cœur de la question
christologique}\label{conclusion-la-question-de-la-relation-entre-le-logos-et-juxe9sus-au-cux153ur-de-la-question-christologique}}



\textbf{Le gnosticisme}

CLEMENT D'ALEXANDRIE, \emph{Extraits de Théodote} 78,2.

« Qui étions-nous ?

Que sommes-nous devenus ? Où étions-nous ?

Où avons-nous été jetés ?

Vers quel but nous hâtons-nous ? D'où sommes-nous libérés ?

Qu'est-ce que la génération ? Et la régénération ? »

« Eugnoste, NH III,3 » dans \emph{Ecrits gnostiques}, 592-594

« Celui qui est indicible, nulle principauté l'a connu, nulle autorité,
nulle puissance subalterne, nulle créature depuis la fondation du monde
si ce n'est lui seul. Celui-là est immortel ; il est immortel ; il est
éternel parce que sans engendrement {[}\ldots{]} Il est inengendré parce
que sans principe {[}\ldots{]} Il est innommé. Il n'a pas d'apparence
humaine {[}\ldots{]} Il est illimité. Il est insaisissable. Il est en
permanence Un. Il est inconcevable, étant seul à se concevoir. Il est
incommensurable. Il est impénétrable. Il est tout entier intellect,
pensée et délibération, réflexion et discours intérieur et puissance ».


\hypertarget{apocryphe-de-jean-ou-livre-des-secrets-de-jean-2615-2715-dans-ecrits-gnostiques-223.}{%
\section{\texorpdfstring{« Apocryphe de Jean (ou Livre des secrets de
Jean) 26,15-27,15 » dans \emph{Ecrits gnostiques},
223.}{« Apocryphe de Jean (ou Livre des secrets de Jean) 26,15-27,15 » dans Ecrits gnostiques, 223.}}\label{apocryphe-de-jean-ou-livre-des-secrets-de-jean-2615-2715-dans-ecrits-gnostiques-223.}}


Ennoia, la pensée de l'Esprit

« C'est lui (l'Esprit) qui se pense lui-même dans propre lumière qui
l'entoure. C'est lui qui est la source d'eau vivre, la lumière pleine de
pureté. La source de l'Esprit s'écoula, venant de l'eau vive de la
lumière. Et {[}il{]} organisa tous les éons et leurs ordres. En toutes
formes il pensa sa propre image en la voyant dans l'eau de lumière pure
qui l'entoure.

Et son Ennoia devint une œuvre, se manifesta et se tint devant lui dans
le flamboiement de la lumière. Elle est la puissance manifestée
antérieurement à toutes choses.

Elle est la Pronoia de toutes choses qui brille dans la lumière, l'image
de l'Invisible. Elle est la puissance parfaite, Barbélô, l'éon parfait
de gloire qui glorifie (l'Esprit) pour l'avoir manifestée. Et quand elle
le pense, elle est Prôtennoia, son image ».

IRENEE DE LYON, \emph{Contre les hérésies}, I, 21,4. Voir aussi
\emph{l'Evangile de la vérité} NHC I/3 24,28-25,19.

« La `rédemption' parfaite, c'est la connaissance même de la Grandeur
inexprimable : puisque c'est par l'ignorance que sont sorties la
déchéance et la passion, c'est par la gnose que sera aboli tout l'état
des choses issu de l'ignorance. C'est donc bien la gnose qui est la
`rédemption' de l'homme intérieur. Cette `rédemption' n'est ni
somatique, puisque le corps est corruptible, ni psychique, puisque l'âme
aussi provient de la déchéance et n'est que l'habitacle du pneuma ; elle
est donc nécessairement pneumatique ; de fait, par la gnose est racheté
l'homme intérieur ou pneumatique ».

IGNACE D'ANTIOCHE, \emph{Aux Tralliens} IX, 1-2.

« Soyez donc sourds quand on vous parle d'autre chose que de
Jésus-Christ, de la race de David, {[}fils{]} de Marie, qui est
\emph{véritablemen}t né, qui a mangé et a bu, qui a été
\emph{véritablement} persécuté sous Ponce Pilate, qui a été
\emph{véritablement} crucifié, et est mort, aux regards du ciel, de la
terre et des enfers, qui est aussi \emph{véritablement} ressuscité
d'entre les morts ».

IGNACE D'ANTIOCHE, \emph{Aux Smyrniotes} I-III et IV, 2.

« Et il a véritablement souffert, comme aussi il s'est véritablement
ressuscité, non pas, comme disent certains incrédules, qu'il n'ait
souffert en apparence\ldots Car si c'est en apparence que cela a été
accompli par notre Seigneur, moi aussi c'est en apparence que je suis
enchaîné »


\hypertarget{la-christologie-du-logos}{%
\section{La christologie du Logos}\label{la-christologie-du-logos}}


JUSTIN, \emph{Apologie}, 46,3-4.

« Ceux qui ont vécu selon le Logos sont chrétiens, même s'ils ont été
tenus pour athées, comme par exemple, chez les Grecs, Socrate,
Héraclite, et d'autres pareils, et, chez les Barbares, Abraham, Ananias,
Azarias, Misaël, Elie et quantité d'autres, dont nous renonçons pour
l'instant à énumérer les œuvres et les noms (\ldots). Dès lors aussi,
ceux qui, parmi les hommes des temps passés, ont vécu loin du Logos,
furent mauvais, ennemis du Christ, meurtriers de ceux qui vivaient selon
le Logos, tandis que ceux qui ont vécu et qui vivent encore selon le
Logos sont chrétiens, sans crainte et sans inquiétude ».

CLEMENT D'ALEXANDRIE, \emph{Stromates}, V, 3, 18, 6-8.

« Quand je dis : philosophie, je n'entends pas celle du Portique, ou de
Platon, ou d'Epicure, ou d'Aristote. Mais tout ce qui a été dit de bon
dans chacune de ces écoles, et qui nous enseigne la justice accompagnée
de connaissance religieuse, c'est cet ensemble que j'appelle
philosophie. »

CLEMENT D'ALEXANDRIE, \emph{Stromates}, VII, 11,68, 1-5.

« Aussi le gnostique, qui se définit par l'amour pour le Dieu réellement
un, est-il réellement l'homme parfait et l'ami de Dieu, placé au rang de
fils. Tels sont en effet les titres de la noblesse d'origine, de la
connaissance et de la perfection correspondant à la vision de Dieu, le
privilège suprême que reçoit l'âme gnostique, devenue parfaitement pure,
jugée digne de voir éternellement face à face, d'après la parole, le
Dieu Tout-Puissant. Devenue en effet toute entière spirituelle, elle
atteint ce qui lui est apparenté, et demeure dans l'Église spirituelle,
pour le repos qui vient de Dieu. »

