

 
\chapter{Ière partie : La christologie se définit dans le monde grec
(IIe-IIIe s.)}

 
\mn{Christologies et cultures dans l'histoire
4} 
La christologie du Logos

\section{Eléments bibliographiques :}

BOVON, F., GEOLTRAIN, P., KAESTLI, J.-D. (dir.), \emph{Écrits apocryphes
chrétiens I et II}, La Pléiade, Gallimard, Paris 1997 et 2005.

BULTMANN, R., \emph{Le christianisme primitif dans le cadre des
religions antiques}, Paris 1969. CLÉMENT D'ALEXANDRIE, \emph{Le
Pédagogue I et III}, Sources Chrétiennes, Paris 1960 et 1970.

DORÉ, J., « Les christologies patristiques et conciliaires » dans
LAURET, B. -- REFOULÉ, F. (dir.),

\emph{Initiation à la pratique de la théologie ** Dogmatique} I, Paris
1982, 185-262.

FÉDOU, M., \emph{La voie du Christ, Genèses de la christologie dans le
contexte religieux de l'Antiquité du IIe siècle au début du IVe siècle},
Paris 2006.

IRENÉE DE LYON, \emph{Contre les Hérésies. Dénonciation et réfutation de
la prétendue gnose au nom menteur}, tr. par A. ROUSSEAU, Paris
1991\textsuperscript{3}.

JONAS, H., \emph{La religion gnostique}, tr. par L. ÉVRARD, Paris 1978.

KLAUCK, H.-J., \emph{L'environnement religieux gréco-romain du
christianisme primitif}, Paris 2012.

MAHÉ, J-P., POIRIER, P-H. (dir.), \emph{Écrits gnostiques. La
bibliothèque de Nag Hammadi}, La Pléiade, Gallimard, Paris 2007.

MACMULLEN, R., \emph{Christianisme et paganisme du IV\textsuperscript{e}
siècle au VIII\textsuperscript{e} siècle}, Paris 1998. POUDERON, B.,
\emph{Les apologistes grecs du II\textsuperscript{e} siècle}, Paris
2005.

RATZINGER, J., \emph{La foi chrétienne hier et aujourd'hui}, Paris
1996\textsuperscript{2}.
 
\hypertarget{introduction}{%
\section{Introduction}\label{introduction}}

\paragraph{Un monde avec des croyances variées} S. Augustin décrira :
\begin{quote}
    une religion civique, politique, 
    une religion mythologique
    une religion, une sagesse philosophique
\end{quote}
Le Christianisme devait s'adapter pour répondre aux besoins et au contexte, traduire dans une autre langue. \textit{traduttore traditore} certes.

\paragraph{la christologie s'adapte}
\paragraph{Phase 1 : traduction} On va adopter le langage de l'autre (image,...) non pas pour faire plaisir pour pour l'annonce de la Bonne Nouvelle. 
\paragraph{Phase 2 : on se démarque} on montre les spécificités chrétiennes : rester fidèle à la Révélation. 
\paragraph{Phase 3 : on demeure marqué par les concepts fondamentaux de l'autre} La christologie prend forme dans une nouvelle culture. Jamais une opération lisse. On oublie certains aspects de la christologie initiale pour en déployer d'autre.
\begin{Synthesis}
Si le Christ est universel, il doit être transmissible.
\end{Synthesis}

En s'inculturant dans le monde grec, la théologie va se \textsc{spiritualiser} et s\textsc{universaliser}. On va adopter une matrice de la culture grecque, la \textsc{la philosophie}, la raison, le logos, la vérité. \mn{Matrice pour la culture chinoise ? unicité ?}

Comment la bonne nouvelle fait son irruption dans le monde grec ?
Trois étapes : 
\begin{itemize}
    \item La confrontation avec la Religion civique : penser le monothéisme.
    \begin{Def}[hénotheisme]
    polytheisme organisé : Un dieu suprême avec des dieux par \textit{administration}. 
    \end{Def}
    \item La confrontation avec la culture gnostique en rappelant la réalité de l'incarnation. On ne sort jamais indemne du combat et de la lutte
    \item La théologie du logos, universalité. 
\end{itemize}

\section{{La confrontation avec le monde religieux païen : polythéisme
  et mythologie}}
  
 \subsection{Le choix du logos contre le mythos}
   
\paragraph{Le logos} les chrétiens ont choisi de s'inculturer dans le monde grec à partir de la raison, du logos, choix conscient des chrétiens alors qu'ils pouvaient choisir le mythos. \sn{Thèse de Ratzinger} 
\begin{quote}
    La prédication chrétienne se trouvait dans un monde saturé de Dieu,... comme Israël face aux autres Nations. La Chrétienté primitive a fait un choix, le Dieu des philosophes contre le Dieu des Religions, choix audacieux. \sn{RATZINGER, J., \emph{La foi chrétienne hier et aujourd'hui}, Paris
1996}
A aucun de ces Dieux, mais nous vénérons le Dieu des philosophes.
\end{quote}
C'est une réponse contre les théologiens libéraux comme Harnack, qui voulaient retrouver la source originelle, cachée par le monde hellenistique.

\paragraph{En se coupant de la vérité} et en s'attachant aux coutumes.

\paragraph{Monogène : deux Dieux ?} Les pères apologistes vont essayer de répondre à ces questions. 
Athénagore d'Athènes \sn{ATHENAGORE, \emph{Supplique}, X,1-5., fin II} va répondre à ceux qui disent que Jésus est le fils comme les autres religions. La confrontation avec les religions païennes
\begin{quote}
« Ce n'est pas à la façon des poètes qui, dans leurs fables, présentent
les dieux comme n'étant en rien meilleurs que les hommes, que nous avons
conçu notre doctrine d'un Dieu qui soit aussi un Père ou celle du Fils ;
mais le Fils de Dieu est le \emph{Logos} du Père en idée et en énergie :
tout a été fait par son opération et son intermédiaire, puisque le Père
et le Fils ne font qu'un. Et comme le Fils est dans le Père, et le Père,
dans le Fils, dans une unité et une puissance spirituelle, le Fils de
Dieu est \textsc{l'intelligence et la raison (logos) du Père.}
(\ldots) Il est le premier à être issu du Père, non pas qu'il soit né

-- car dès l'origine Dieu, qui est \textsc{intelligence éternelle},
portait en lui son Logos, puisqu'il est éternellement raisonnable
(logikos) -, mais parce que alors que toute la matière était dénuée de
qualité, comme une terre non travaillée (\ldots) \textsc{il a procédé
du Père pour leur servir d'idée et d'énergie} (\ldots). »

\end{quote}
On ne dit pas que le Logos est le fils de Dieu. Défend aussi l'universalité. 

\paragraph{mais le culte chrétien semble contredire Athenagore}. Le culte peut être \textit{polylatrie}. \textit{Celse}, Philosophe romain : 
\begin{quote}
    ils rendent un culte excessif à son Fils.
\end{quote}
Origène va répondre à Celse, \sn{Origène, \textit{contre Celse, VIII}} :
\begin{quote}
car nous lui ajoutons foi. lorsqu'il dit: 
\begin{quote}
    Avant qu'Abraham fût, je suis; et encore, 
    
    Je suis la vérité (Jean, VIII, 58 et XIV,6)
\end{quote} 
Il n'y a aucun parmi nous d'un esprit assez grossier pour croire que la vérité ne fût pas un être qui subsistât avant la venue de Jésus-Christ. Ainsi nous adorons le Père de la vérité, et le Fils qui est la vérité, les considérant comme deux choses à l'égard de leur \emph{hypostase} (ou subsistance), mais comme une seule et même chose à l'égard de leur accord, de la conformité de leurs sentiments et de la parfaite union de leur volonté. De sorte que qui a vu le Fils qui est le rejaillissement de la gloire et le caractère de l'hypostase (ou la subsistance) de Dieu, a vu Dieu en voyant celui qui est l'image de Dieu (II Cor.,IV, 4) . Celse veut encore que parce que nous rendons nos hommages et à Dieu et à son Fils, il suive de là que, selon nous, ce n'est pas Dieu seul qu'il faut servir, mais que l'on doit aussi servir ses ministres.
\end{quote}



\subsection{Contre l'idée d'un engendrement charnel, l'engendrement
    spirituel du Fils}
    
  
    
    \subsection{La véritable image contre les images païennes}
    
    \paragraph{les statues, L'image de Dieu est spirituelle} réflexe iconoclase, car il y a de l'idolatrie. Clément d'Alexandrie \sn{CLEMENT D'ALEXANDRIE, \emph{Protreptique}, X, 98,1-4.}.
\begin{quote}
    La véritable image de Dieu est spirituelle.
\end{quote}    
Face aux images paiennes, présenter le Christ, image de Dieu : 
\begin{quote}
« `Image de Dieu' est son Logos\ldots, et image du Logos est l'homme
véritable, l'esprit qui est dans l'homme, et qui est dit, à cause de
cela, avoir été fait `à l'image' de Dieu et `à sa ressemblance',
assimilé au divin Logos par l'intelligence de son cœur et , par là,
raisonnable. Mais les statues à figures humaines ne sont qu'une image
terrestre de l'homme tel qu'on le voit, né de la terre, et elles
n'apparaissent que comme une reproduction passagère bien éloignée de la
vérité. »
\end{quote}
    
      
      \paragraph{Le Logos façonne des images de Dieu}
      
      \paragraph{L'image de Dieu dans le monde, ce sont les croyants} Ce sont les êtres vivants. Plus on est proche du logos, plus notre raison est assimilée au logos divin, et plus nous devenons ressemblant à Dieu et nous devenons des icônes.
      
      \paragraph{Homme devient Dieu par l'action intérieure du logos} Le Christ va rendre visible Dieu, car il est le \textit{logos incarné}. Et nous, de façon dégradée, nous rendons visible Dieu.
      
      \paragraph{L'homme intérieur}{Clément parle de la vraie gnose} Ce dialogue parle aux paiens. 

\subsection{Conclusion}

Michel Feydou : 
\begin{quote}
    Refus de l'idolatrie s'enracine dans le logos. 
    l'image la plus parfaite : le Christ. \ldots vrais autels, véritables statues.
\end{quote}
  
  Mais il y a un risque, celui de la gnose, quand on spiritualise la christologie, à mettre en avant le logos et pas la \textit{sarx}, la chair.
  
 ~
  \hypertarget{la-religiosituxe9-gnostique-et-la-christologie-gnostique-du-logos}{%
  \section{La religiosité gnostique et la christologie gnostique du
  Logos}\label{la-religiosituxe9-gnostique-et-la-christologie-gnostique-du-logos}}

  
  
  
    
    \subsection{L'origine obscure du gnosticisme}
    
    \mn{COurs du 15/2/22}
  
    
    \subsection{Qu'est-ce que la gnose ?}
    
   \subparagraph{On le connait par ses contradicteurs} Irénée de Lyon. Mais les manuscrits de Nag Hammadi nous apprend aussi bcp de choses.
   
   \subparagraph{Harnack} La gnose serait récente et une dérive hellenistique du christianisme.
   
   \subparagraph{Bultmann} Prologue de Saint Jean a un susbtrat gnostique \sn{Das Evangelium von Johannes}. Jean contre le gnosticisme en montrant l'incarnation. cf  "Apocryphe de Jean" \sn{(ou Livre des secrets de
Jean) 26,15-27,15 » dans \emph{Ecrits gnostiques},
223.}

   
   
       \begin{quote}
        
 


Ennoia, la pensée de l'Esprit

« C'est lui (l'Esprit) qui se pense lui-même dans propre lumière qui
l'entoure. C'est lui qui est la source d'eau vivre, la lumière pleine de
pureté. La source de l'Esprit s'écoula, venant de l'eau vive de la
lumière. Et {[}il{]} organisa tous les éons et leurs ordres. En toutes
formes il pensa sa propre image en la voyant dans l'eau de lumière pure
qui l'entoure.

Et son Ennoia devint une œuvre, se manifesta et se tint devant lui dans
le flamboiement de la lumière. Elle est la puissance manifestée
antérieurement à toutes choses.

Elle est la Pronoia de toutes choses qui brille dans la lumière, l'image
de l'Invisible. Elle est la puissance parfaite, Barbélô, l'éon parfait
de gloire qui glorifie (l'Esprit) pour l'avoir manifestée. Et quand elle
le pense, elle est Prôtennoia, son image ».



    \end{quote}
   \paragraph{Influences orientales} La gnose ne serait pas la première hérésie chrétienne mais lui préexisterait. Hans Jonas décrit la gnose comme une \textit{religiosité particulière} qui serait très présente autour du Ie siècle. Elle se distingue de la Foi juive vetero-testamentaire (croyance en la Création) et la philosophie stoicienne qui se sent en harmonie avec le monde (non pas un \emph{chaos} mais un \emph{Cosmos} organisé par le \emph{logos}. Ici, la gnose est plus pessimiste sur le monde et en fin d'Antiquité, serait très présente.
   
   \subparagraph{Forme religieuse syncrétiste} métissé.
   
    
    \subsection{La doctrine gnostique}
    
    \begin{Def}[Gnose]
    de \emph{gnosis}, la connaissance en Grec, nébuleuse qui affirme que le salut se fait par la connaissance.
    \end{Def}
    Attention, le mot connaissance a plus de sens pour les Grecs : initiation, sagesse,...

    Clément nous transmet les questions d'un bon gnostique\sn{\textsc{Clément d'Alexandrie}, \emph{Extraits de Théodote} 78,2.} : 
    
    \begin{quote}
        


« Qui étions-nous ?

Que sommes-nous devenus ? Où étions-nous ?

Où avons-nous été jetés ?

Vers quel but nous hâtons-nous ? D'où sommes-nous libérés ?

Qu'est-ce que la génération ? Et la régénération ? »

« Eugnoste, NH III,3 » dans \emph{Ecrits gnostiques}, 592-594
    \end{quote}
      
      \subparagraph{Face à  une religion essentiellement rituelle}, en regardant ces questions, on voit que les questions vont de la dégradation à la regénération : un état originel mais nous avons été jeté dans un monde, malheureux mais pas totalement sans espoir. Il existe une possibilité de revenir à l'état originel, via un \textit{chemin}, la \emph{gnose}, la connaissance de notre véritable perception et de la réalité des choses.
      
      \subparagraph{parenté avec le Christianisme} Chute et remontée. Mais vision négative du monde, un dualisme, état sans corps. Et choix initial du christianisme lors de la rencontre du monde grec de choisir une voix spirituelle et philosophique
      
      \paragraph{L'idée de Dieu} 
      \sn{Ecrit de Nag Hammadi, Pleiade} 
      
      \begin{quote}
          « Celui qui est indicible, nulle principauté l'a connu, nulle autorité,
nulle puissance subalterne, nulle créature depuis la fondation du monde
si ce n'est lui seul. Celui-là est immortel ; il est immortel ; il est
éternel parce que sans engendrement {[}\ldots{]} Il est inengendré parce
que sans principe {[}\ldots{]} Il est innommé. Il n'a pas d'apparence
humaine {[}\ldots{]} Il est illimité. Il est insaisissable. Il est en
permanence Un. Il est inconcevable, étant seul à se concevoir. Il est
incommensurable. Il est impénétrable. Il est tout entier intellect,
pensée et délibération, réflexion et discours intérieur et puissance ». \sn{NH 3,3}
      \end{quote}
    
      
      \paragraph{Le drame divin}
      
      Drame au sein même de la divinité. L'origine du mal s'explique à l'intérieur de la divinité, la création n'est pas voulu, un accident ou une erreur. 
      \subparagraph{Dévolution du Divin} la seule réalité initiale, c'est le monde divin. Tout un mythe sur la rupture dans le divin, et cette rupture, cette chute, c'est l'homme.
      
      \begin{Def}[eon]
          les émanations de Dieu, qui se dégradent, jetées à l'extérieur.
      \end{Def}
    Un couple d'Eon, \emph{Sophia} et \emph{Esprit} qui sortent et de leur union, est créé  \emph{Yaldabaöth}, le démiurge qui créé le monde. 
      Il crée le monde avec des puissances et des autorités. Mais garde une trace du divin. 
     \paragraph{l'anthropologie gnostique}
     le corps est une prison (plus gnostique que platonicien). 
     A l'âme de l'homme, on va l'enfermer dans un corps. Car dans l'âme de l'homme, il y a quelque chose de divin et on l'enferme dans une sorte de prison.
     
     \subparagraph{Il y a quelque chose de lumineux dans l'homme} mais il ne peut pas en sortir. Vision tripartite de l'homme. Le corps de l'homme n'est pas bon pour lui mais l'empêche de se souvenir qu'il n'est pas de ce monde.
     
     \subparagraph{les pneumatiques, les psychiques et les charnels}. Le pneumatique doit sortir du corps.
    
  
    
    \subsection{Le Christ comme le sauveur gnostique}
    
    Le Christ est le \textit{révélateur} de ce que nous sommes. 

      \paragraph{Le salut gnostique}
      Irénée \sn{\textsc{Irénée de Lyon}, \emph{Contre les hérésies}, I, 21,4. Voir aussi
\emph{l'Evangile de la vérité} NHC I/3 24,28-25,19.} : 
    \begin{quote}
       

« La `rédemption' parfaite, c'est la connaissance même de la Grandeur
inexprimable : puisque c'est par l'ignorance que sont sorties la
déchéance et la passion, c'est par la gnose que sera aboli tout l'état
des choses issu de l'ignorance. C'est donc bien la gnose qui est la
`rédemption' de l'homme intérieur. Cette `rédemption' n'est ni
somatique, puisque le corps est corruptible, ni psychique, puisque l'âme
aussi provient de la déchéance et n'est que l'habitacle du pneuma ; elle
est donc nécessairement pneumatique ; de fait, par la gnose est racheté
l'homme intérieur ou pneumatique ».

    \end{quote}
      Le vocabulaire "résonne" pour des chrétiens mais de façon particulière.
      Si on se connaît vraiment soi-même, on est déjà sauvé (livre de Thomas). 
      Ces lumières qui nous habitent vont sortir de notre condition grâce au Christ et vont revenir à leur origine, le \textit{plérôme}.
      
      \subparagraph{Ignorance} un défaut pour les grecs. Ce qui pose question sur l'ignorance du Christ.
      
      \paragraph{La christologie docète}
      C'est la christologie gnostique : le Christ n'est pas vraiment homme.
      \begin{Def}[docétisme]
          de \emph{dokein}, sembler, paraître, le christ est apparu sous l'apparence d'un corps mais n'est pas mort et donc n'est pas ressuscité.
      \end{Def}
      Il a dû se déguiser.
      
      \subparagraph{Cosmogonie} 7 cieux \sn{cf 2 Co 11, 7 : troisième ciel}, avec des douanes, le plérôme au centre. Pour arriver jusqu'au monde, la descente du sauveur, il doit se déguiser pour tromper les puissances célestes en se déguisant comme elles.
      
      \subparagraph{Déguisé depuis le baptême} le Logos est apparence, j'ai revêtu Jésus, une enveloppe. Jésus est un corps.
  
        Saint Irénée : 
        \begin{quote}
           Les Anges qui occupent le ciel  inférieur, celui que nous voyons, ont  fait  tout  ce que  renferme le  monde  et se sont partagé  entre  eux  la  terre  et  les  nations  qui  s'y  trouvent.  Leur  chef  est  celui  qui  passe  pour  être  le  Dieu  des  Juifs. Celui-ci  ayant  voulu  soumettre  les  autres  nations  à  ses  hommes  à  lui,  c'est-à-dire  aux  Juifs,  les  autres  Archontes se  dressèrent  contre  lui  et  le  combattirent.  Pour  ce  motif  aussi  les  autres  nations  se  dressèrent  contre  la  sienne. Alors  le  Père  inengendré  et  innommable,  voyant  la  perversité  des  Archontes,  envoya  l'Intellect,  son  Fils premier-né  — c'est  lui  qu'on  appelle  le  Christ  —  pour  libérer  de  la  domination  des  Auteurs  du  monde  ceux  qui croiraient  en  lui.  Celui-ci  apparut  aux  nations  de  ces  Archontes,  sur  terre,  sous  la  forme  d'un  homme,  et  il accomplit  des  prodiges.  Par  conséquent,  il  ne  souffrit  pas  lui-même  la  Passion,  mais  un  certain  Simon de  Cyrène fut  réquisitionné  et  porta  sa  croix  à  sa  place.  Et  c'est  ce  Simon  qui,  par  ignorance  et  erreur,  fut  crucifié,  après avoir  été  métamorphosé  par  lui  pour  qu'on  le  prît  pour  Jésus  ;  quant  à  Jésus  lui-même,  il  prit  les  traits  de  Simon et,  se  tenant  là,  se  moqua  des  Archontes.  Étant  en  effet  une  Puissance  incorporelle  et  l'Intellect  du  Père inengendré,  il  se  métamorphosa  comme  il  voulut,  et  c'est  ainsi  qu'il  remonta  vers  Celui  qui  l'avait  envoyé,  en  se moquant d'eux,  parce  qu'il  ne  pouvait  être  retenu  et  qu'il  était  invisible  à  tous.  Ceux  donc  qui  "savent"  cela  ont été  délivrés  des  Archontes  auteurs  du  monde.  Et  l'on  ne  doit  pas  confesser  celui  qui  a  été  crucifié,  mais  celui  qui est  venu  sous  une  forme  humaine,  a  paru  crucifié,  a  été  appelé  Jésus  et  a  été  envoyé  par  le  Père  pour  détruire, par  cette  "économie",  les  œuvres  des  Auteurs  du  monde.  Si  quelqu'un  confesse  le  crucifié,  dit  Basilide,  il  est encore  esclave  et  sous  la  domination  de  ceux  qui  ont  fait  les  corps  ;  mais  celui  qui  le  renie  est  libéré  de  leur emprise et connaît  l' "économie" du Père inengendré.  \textit{Irénée, III, 1}
        \end{quote}
    
    La résurrection : le baptême est le début de l'illumination, de l'ascension : à partir du moment où l'on prend conscience de qui nous sommes, on commence à se tourner vers le Père. Une compatibilité entre Gnosticisme et Christianisme.
    Au II, on avait équivalence entre Esprit et Logos.
    
    \begin{Synthesis}
        Le gnosticisme n'attache pas d'importance particulière à l'Unicité du Christ. Le caractère eschatologique de la venue du Christ est secondaire pour eux.
    \end{Synthesis}
    
    
    \paragraph{Réaction Chrétienne}
Très rapidement, l'Eglise va s'opposer au gnosticisme. Par exemple, Ignace d'Antioche (deb II) \sn{\textsc{Ignace d'Antioche}, \emph{Aux Tralliens} IX, 1-2.} :
\begin{quote}
« Soyez donc sourds quand on vous parle d'autre chose que de
Jésus-Christ, de la race de David, {[}fils{]} de Marie, qui est
\emph{véritablemen}t né, qui a mangé et a bu, qui a été
\emph{véritablement} persécuté sous Ponce Pilate, qui a été
\emph{véritablement} crucifié, et est mort, aux regards du ciel, de la
terre et des enfers, qui est aussi \emph{véritablement} ressuscité
d'entre les morts ».
\end{quote}
4 fois \textit{véritablement}. il montre ici qu'il est véritablement homme. Parce qu'il est dans un climat où on spiritualise le Christ.

Il ajoute \sn{\textsc{Ignace d'Antioche}, \emph{Aux Smyrniotes} I-III et IV, 2.}
\begin{quote}
« Et il a véritablement souffert, comme aussi il s'est véritablement
ressuscité, non pas, comme disent certains incrédules, qu'il n'ait
souffert en apparence\ldots Car si c'est en apparence que cela a été
accompli par notre Seigneur, moi aussi c'est en apparence que je suis
enchaîné\sn{Il va être tué à Rome} » 
\end{quote}

\paragraph{Une certaine perméabilité} La Tradition Chrétienne doit s'interroger sur les reprises du gnosticime dans la Foi Chrétienne, en particulier sur la négativité du Corps. Même si on réfute, on est toujours marqué par l'adversaire. 
\begin{quote}
    Des éléments gnostiques... intégrés dans le système chrétien. ... assecha le mouvement gnostique. \sn{KLAUCK, H.-J., L’environnement religieux gréco-romain du christianisme primitif, Paris 2012}
\end{quote}

    
    \subsection{Conclusion}
    
  
 ~
  \hypertarget{la-christologie-orthodoxe-du-logos}{%
  \section{La christologie « orthodoxe » du
  Logos}\label{la-christologie-orthodoxe-du-logos}}

  
  4 points : 
  \begin{itemize}
      \item Ce logos est un avec Dieu
      \item Logos par qui Dieu crée
      \item il est universel
      \item il est sauveur de tout l'homme
  \end{itemize}
  Certains pères de l'Eglise vont essayer de créer une Christologie à partir du Logos.
  
    
    \subsection{Le Logos un avec le Père, créateur et principe du monde}
  
  \paragraph{Intelligence du Père} Et donc lié au Père.
  
  \begin{quote}
V. — Dieu était dans le principe, et nous avons appris que le principe, c’est la puissance du Logos. Car le maître de toutes choses, qui est lui-même le support substantiel de l’univers, était seul en ce sens que la création n’avait pas encore eu lieu; mais en ce sens que toute la puissance des choses visibles et invisibles était en lui, il renferme en lui-même toutes choses par le moyen de son Logos. Par la volonté de sa simplicité, sort de lui le Logos, et le Logos, qui ne s’en alla pas dans le vide, est la première œuvre du Père. \begin{quote}
    « C’est lui, nous le savons, qui est le principe du monde. Il provient d’une distribution, non d’une division. Ce qui est divisé est retranché de ce dont il est divisé, mais ce qui est distribué suppose une dispensation volontaire et ne produit aucun défaut dans ce dont il est tiré. »
\end{quote} Car, de même qu’une seule torche sert à allumer plusieurs feux, et que la lumière de la première torche n’est pas diminuée parce que d’autres torches y ont été allumées, ainsi le Logos, en sortant de la puissance du Père, ne priva pas de Logos celui qui l’avait engendré. Moi-même, par exemple, je vous parle, et vous m’entendez, et moi qui m’adresse à vous, je ne suis pas privé de mon logos parce qu’il se transmet de moi à vous, mais en émettant ma parole, je me propose d’organiser la matière confuse qui est en vous, et comme le Logos, qui fut engendré dans le principe, a engendré à son tour, comme son œuvre,[33] en organisant la matière, la création que nous voyons, ainsi moi-même, à l’imitation du Logos, étant régénéré et ayant acquis l’intelligence de la vérité, je travaille à mettre de l’ordre dans la confusion de la matière dont je partage l’origine. Car la matière n’est pas sans principe ainsi que Dieu, et elle n’a pas, n’étant pas sans principe, la même puissance que Dieu, mais elle a été créée, elle est l’œuvre d’un autre, et elle n’a pu être produite que par le créateur de l’univers.

      Tatien\sn{ TATIEN, Discours aux Grecs V, 1, o. c..}
  \end{quote}
  
  On commence à penser (mais à préciser) la Création et le Logos.
  
  Théophile d'Antioche \sn{Évêque d’Antioche sous Marc , il a laissé trois livres adressés à un païen, autrement inconnu, Autolycus. Comme Justin, Théophile avait écrit contre les hérésies : Contre Marcion, Contre Hermogène (traités perdus, mais dont se servit Tertullien, selon toute apparence, dans ses ouvrages homonymes).} : 

    
    \begin{quote}
        
(10)… Dieu engendra son Verbe (Logos)\sn{Théophile d’Antioche, À Autolycus 2, 10.22 [27]}, qui était immanent (endiathetos) en son sein, et le produisit avec sa Sagesse avant toute chose. Il eut ce Verbe comme ministre de toutes ses œuvres, et par son intermédiaire il a tout fait. On l’appelle Principe, parce qu’il est le Principe et le maître de tout ce qui a été créé par son intermédiaire. C’est lui, Esprit de Dieu, Principe et Sagesse et Force du Très-Haut, qui descendait sur les prophètes et racontait par leur bouche ce qui concerne la création du monde et tout le reste : les prophètes n’existaient pas quand le monde fut, mais la Sagesse de Dieu demeurant en lui, mais le Verbe Saint de Dieu qui est sans cesse présent avec lui. Voilà pourquoi la Sagesse, par la bouche du prophète Salomon, prononce ces mots : \begin{quote}
    « Quand il organisa le ciel, j’étais avec lui ; et tandis qu’il consolidait les assises de la terre, je l’assistais dans ce travail » (Pr 8, 27-29).
\end{quote}
(22)… Dieu, le Père de toutes choses, n’est pas localisable et ne se trouve pas dans un lieu, car il n’y a pas de lieu où il cesse d’être ; mais son Verbe, par lequel il a créé toutes choses, qui est sa Puissance et sa Sagesse, s’est revêtu de la figure du Père et Seigneur de l’Univers : c’est lui qui venait dans le Paradis sous la figure de Dieu et qui s’entretenait avec Adam. Car l’Écriture Sainte elle-même nous enseigne qu’Adam disait qu’il avait entendu sa voix. Quelle autre voix serait-ce que le Verbe de Dieu, qui est aussi son Fils ? non dans le sens où poètes et mythographes disent que des fils des dieux naissant d’unions charnelles, mais suivant ce que la vérité rapporte du Verbe qui existe toujours immanent (endiathetos) au sein de Dieu.
Avant que rien ne fût, il tenait conseil avec lui qui est son intelligence et son sentiment. Et quand Dieu décida de faire tout ce qu’il avait délibéré, il engendra ce Verbe au-dehors (prophorikos), premier-né de toute créature (Col 1, 15), sans être privé lui-même du Verbe, mais ayant engendré le Verbe et s’entretenant toujours avec son Verbe.
D’où l’enseignement que nous donnent les Saintes Écritures, et tous les inspirés, entre autres Jean, quand il dit : \begin{quote}
    « Dans le principe était le Verbe ; et le Verbe était en Dieu » (Jn 1, 1)
\end{quote} . Il montre qu’au début, il n’y avait que Dieu, et qu’en lui était le Verbe. Puis il dit : \begin{quote}
    « Et le Verbe était Dieu ; tout par lui a existé et sans lui n’a pas existé une seule chose » (Jn 1, 1-3)
\end{quote}. Le Verbe est donc Dieu et il est né de Dieu ; et, chaque fois que le veut le Père de toutes choses, ce Père l’envoie à tel lieu ; il s’y rend, s’y fait entendre et voir, comme son envoyé, et se trouve dans un lieu.
 

    \end{quote}
    Distinction ferme entre le Dieu Père et son Verbe. Trop ferme sans doute, jusqu’à faire du Verbe le ministre du Père dans la création et la révélation (comme Justin, Théophile réfère au Logos les théophanies de l’Ancien Testament). D’où la saveur subordinatienne, accentuée par l’impression que, avant sa génération, le Verbe immanent n’était en Dieu qu’un attribut impersonnel. Mais cette impression est sans doute fausse, et l’érudition contemporaine est plus nuancée touchant le prétendu simplisme des conceptions de Théophile.
    
    
    Logos proféré (engendré), c'est l'intelligence du Père que l'on va proférer, sortir de nous même, et qui devient \emph{Parole}.
    
    \paragraph{Origène} reprend une génération plus tard cette approche \sn{traité sur les principes} ;
    \begin{quote}
        ... Nous disons que la Parole et Sagesse est née de Dieu sans que rien ne sorte corporellement. [comme la volonté procède de l'Intelligence] \mn{revoir}
    \end{quote}
    
    \subparagraph{Différence avec Plotin} Chez Plotin, le Logos descend de l'UN alors que dans le Christianisme, il n'y a pas d'inégalité.
    
    
    \subparagraph{Ce qui est compliqué} c'est de penser le passage du Un, (Dieu) au multiple (peuple) et la médiation. Dans la pensée neoplatonicienne... Origène repense 1/Multiple avec le Christ, comme premier de la nouvelle création. 
    
    
    
    \subsection{Universalité du Logos et le thème des « semences du Verbe »}
    
    \paragraph{Montrer que le Christianisme n'est pas une religion régionale} mais universelle. La matrice grecque,c'est le Logos, qui permet d'universaliser Jésus.
    Si le Christ est Logos, on peut affirmer qu'il est partout dans le monde.
    
   
    
      
      \paragraph{Le thème du Logos disséminé : Justin et les autres}
      
     \subparagraph{Semence du logos ou du verbe} Justin \sn{165, Naplouse, conversion (mais avec le Christ, pas une rupture mais un accomplissement), Présente une apologie à l'empereur} et Clément d'Alexandrie.
    
    \subparagraph{Justin} développe une théorie proche des chrétiens anonymes \sn{JUSTIN, \emph{Apologie}, 46,3-4.} 
    \begin{quote}
« Ceux qui ont vécu selon le Logos sont chrétiens, même s'ils ont été
tenus pour athées, comme par exemple, chez les Grecs, Socrate,
Héraclite, et d'autres pareils, et, chez les Barbares, Abraham, Ananias,
Azarias, Misaël, Elie et quantité d'autres, dont nous renonçons pour
l'instant à énumérer les œuvres et les noms (\ldots). Dès lors aussi,
ceux qui, parmi les hommes des temps passés, ont vécu loin du Logos,
furent mauvais, ennemis du Christ, meurtriers de ceux qui vivaient selon
le Logos, tandis que ceux qui ont vécu et qui vivent encore selon le
Logos sont chrétiens, sans crainte et sans inquiétude ».
    \end{quote}
    Tout homme peut devenir Chrétien sans le savoir mais d'une certaine façon, c'est sa vie. 
    
\subparagraph{Clément d'Alexandrie} 

\begin{quote}
    

« Quand je dis : philosophie\sn{CLEMENT D'ALEXANDRIE, \emph{Stromates}, V, 3, 18, 6-8.}, je n'entends pas celle du Portique, ou de
Platon, ou d'Epicure, ou d'Aristote. Mais tout ce qui a été dit de bon
dans chacune de ces écoles, et qui nous enseigne la justice accompagnée
de connaissance religieuse, c'est cet ensemble que j'appelle
philosophie. »
\end{quote}
    multiplication d'écoles mais la Philosophie avec une unité, c'est ce qui est \emph{Bon} dans cette école. 
    
    \begin{Synthesis}
        L'image de la semence permet de dire que cela pousse bien ou non. Clément ne peut regarder le Logos qu'en rassemblant la gerbe de la moisson des hommes. 
    \end{Synthesis}
      
      \paragraph{Influence stoïcienne et originalité chrétienne}
      
      \subparagraph{La semence} principe immanent du monde, la raison qu'il organise, \emph{Logoi spermatikoi}. Le logos universel qui régit le monde doit être rassemblé.
      
      \subparagraph{pour les chrétiens} le \emph{logos} c'est une personne, Jésus Christ, alors que pour les Stoiciens c'est un principe. A noter que Justin et Clément ne cite pas le prologue de St Jean.
      
    
  
    
    \subsection{Le Logos est le Sauveur car il nous introduit à la véritable
    gnose}
    
      
      \paragraph{Le Logos comme « pédagogue » de l'homme} Clément d'Alexandrie christianisme la gnose, Philantropie du Logos. il va comparer le pédagogue à la Loi (Torah). Cette Loi s'accomplit dans le Logos. Toute l'histoire est sous l'influence du Logos.
      
      \begin{quote}
         la philosophie pour les Grecs, comme la Loi pour les Juifs, préparation du Logos en Christ (Clément d'Alexandrie, \textit{Stromates})
      \end{quote}
      Progression dans la gnose et la connaissance. 
      
    
      
      \paragraph{La gnose ou connaissance véritable}
      
      L'ignorance est liée au péché dans la pensée grecque. Et  \textit{la vérité vous rendra libre\sn{Jn 8,32}}. 
    
  \begin{quote}
      

« Aussi le gnostique\sn{CLEMENT D'ALEXANDRIE, \emph{Stromates}, VII, 11,68, 1-5.}, qui se définit par l'amour pour le Dieu réellement
un, est-il réellement l'homme parfait et l'ami de Dieu, placé au rang de
fils. Tels sont en effet les titres de la noblesse d'origine, de la
connaissance et de la perfection correspondant à la vision de Dieu, le
privilège suprême que reçoit l'âme gnostique, devenue parfaitement pure,
jugée digne de voir éternellement face à face, d'après la parole, le
Dieu Tout-Puissant. Devenue en effet toute entière spirituelle, elle
atteint ce qui lui est apparenté, et demeure dans l'Église spirituelle,
pour le repos qui vient de Dieu. »
  \end{quote}
Toute une montée de l'âme qui se purifie pour voir Dieu, cette connaissance transformante.

\begin{Synthesis}
    On voit que l'acculturation grecque doit prendre en compte l'importance de la connaissance et le fait à travers le terme de \textit{logos}
\end{Synthesis}




\hypertarget{conclusion-la-question-de-la-relation-entre-le-logos-et-juxe9sus-au-cux153ur-de-la-question-christologique}{%
\section{Conclusion : La question de la relation entre le Logos et
Jésus au cœur de la question
christologique}\label{conclusion-la-question-de-la-relation-entre-le-logos-et-juxe9sus-au-cux153ur-de-la-question-christologique}}


\paragraph{Rôle historique de Jésus} Les apologistes parlent du \textit{logos} mais ne font pas référence à Jésus de Nazareth. Comment articuler la Vérité et l'homme concret ? Comment concilier ce qui s'oppose, l'immortel et le mortel, l'esprit et la Chair ? Par le Logos, on donne une réponse. 
Mais la pensée grecque tend à opposer Dieu (Immortel,...) et l'homme (mortel...) et opposition des deux natures (à part l'âme). On va voir comment la christologie du III-IV va être en partie commandé par cette opposition ?
Il faut intégrer l'histoire de Jésus et son histoire concrête. 

 
\begin{Synthesis}
    le concept de Logos a permis de sortir du polytheisme mais risque qu'on ne sauve que l'âme
\end{Synthesis}







