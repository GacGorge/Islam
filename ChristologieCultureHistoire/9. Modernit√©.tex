\chapter{La modernité humaniste détermine la christologie  Une christologie « anthropologique »}

\section{Eléments bibliographiques} 

BRAGUE, R., Le règne de l’homme. Genèse et échec du projet moderne, Gallimard, Paris, 2015. 

CLAVERIE, P.,  Lettres et messages d’Algérie, Paris 1996. 

GILLEPSIE, M. A., « Humanism and the Apotheosis of Man » dans ID., The Theological Origins of Modernity, The University of Chicago Press, Chicago – Londres, 2008, 69-100. 

LUBAC (de), H., Le drame de l’humanisme athée, Paris 1944. 

MAURICE, E., La christologie de Karl Rahner, Paris 1995. 

RAHNER, K., « Problèmes actuels de christologie » dans Ecrits théologiques I, Paris 1959, 113-181. 

RAHNER, K., « Réflexions théologiques sur l’incarnation » dans Ecrits théologiques III, Paris 1963,81-101. 

RAHNER, K., Traité fondamental de la foi, Paris 1983. 

SESBOÜÉ, B., Hors de l’Église pas de salut. Histoire d’une formule et problèmes d’interprétation, Paris 2004. 




\section{Introduction }





\paragraph{Au centre l'Homme : tournant anthropologique} Face à un moyen-âge qui est théocentrique, la théologie est la règne des sciences, avec l'anthropocentrie, on va passer aux sciences humaines. 

\paragraph{Une Christologie Anthropologique}

\section{L’humanisme comme le mouvement culturel de la modernité } 

\subsection{L’humanisme italien}   

\paragraph{François Pétrarque 1304 - 1374} Les Papes sont à Avignon, Grande Peste. Début de la Renaissance Italienne. 1336 : il monte le mont Ventoux près d'Avignon et fait une expérience spirituelle. Il médite les \textit{confessions} de Saint Augustin (Chapitre 10) et il se rend compte du monde intérieur.

\paragraph{le projet de Pétrarque} Vie humaine pleine de sens commence par son intériorité, voyage intérieure. Approche individualiste. 
\paragraph{Des figures } En particulier l'Antiquité. Non seulement des saints mais aussi Cicéron, Socrate,... en reintégrant les figures greco-romaines.
Cette affirmation était extraordinaire.

\paragraph{Approche néo-platonicienne du Christianisme} On s'éloigne de l'image d'Adam pécheur pour se rapprocher de l'Homme à l'image de Dieu (Gn 1).  Cela a longtemps été vu comme anti-Chrétien. Cf Etienne Gilson (XX):
\begin{quote}
    Renaissance : pas le Moyen Age plus l'homme, mais le Moyen âge sans Dieu et donc sans l'homme.
\end{quote}

\paragraph{Lien entre humanisme et Christianisme} Humanisme : on voulait mettre ensemble l'Antiquité Paienne, avec la vie monastique, la charité Chrétienne. Synthétiser les deux mondes. Pétrarque admirait les deux mondes, au service de l'Homme et de son individualité. Ils n'étaient pas anti-religieux ou anti-Chrétien. 

\begin{Synthesis}
L'homme n'est pas irrémédiablement déchu mais à l'image de Dieu : un chemin pour l'individualité.
\end{Synthesis}
Mais en devenant plus optimiste sur l'homme, on devient pélagianiste, en oubliant la grâce qui n'est plus centrale. Pélage : le Christ est un exemple et je vais l'imiter. La grâce, c'est de pouvoir imiter le Christ.

\paragraph{Salutati} un autre italien qui va insister sur la volonté. Pour défendre la dignité humaine, on pense que les grands héros de l'antiquité peuvent être sauvés. Alors que Dante mettait Socrate en enfer. Si Socrate était damné, alors injustice de Dieu. Mais si Socrate est sauvé, comment penser le sacrifice du Christ ?

\paragraph{Lorenzo Valla +1447} Un être voulant plus que rationnel. On peut de la contemplation à l'action. Les hommes veulent non seulement contempler le monde mais le transformer. Donner \textit{une forme au monde}. Tout \textit{logos} est une forme de \textit{poesis}, une création. L'homme participe à la Création du monde.



\subsection{L’humanisme d’Erasme } 

\paragraph{Philosophia Christi} Un travail sur l'homme intérieur. 

\begin{Synthesis}
Trouver Dieu à travers l'homme et sa vie intérieure
\end{Synthesis}
\subsection{L’humanisme et la question du salut } 

On va ensuite arriver aux droits de l'homme, partant de cette culture, d'abord vu par l'Eglise comme en opposition avec les droits de Dieu.

 
\paragraph{L'angoisse de l'Enfer} Pourquoi Dieu envoie-t-il autant de personnes en enfer ? Une peur de manquer son salut. Mais l'humanisme a peu à peu refuser ce discours sur les fins dernières.

\paragraph{Jean Jacques Rousseau ... curé savoyard} Après la découverte des nouveaux pays, on va se poser la question du salut de ces populations.

\paragraph{Un humanisme qui s'éloigne peu à peu de l'Église} La vision du monde se fait autour de l'homme. On remet en cause cette idée que Dieu peut châtier l'homme. Par capillarite, d'abord les élites puis la population.

\begin{Synthesis}
Une culture nouvelle naît d'évènements extérieurs (nouveau monde) mais aussi d'éléments internes, et va donner naissance à un monde sécularisé, sans référence à Dieu.
\end{Synthesis}

%----------------------------
\section{L’humanisme chrétien et le Christ }

Il faudra attendre le XX\textsuperscript{è} pour que l'Eglise donne droit à l'humanisme, en particulier après la seconde guerre mondiale.
\paragraph{Vatican II} va entrer en dialogue avec ce monde moderne. L'Eglise marque ainsi la fin de la Chrétienté. \textit{Gaudium et Spes}

\paragraph{Un Christianisme qui ne va pas contre l'homme}

\subsection{Le drame de l’humanisme athée} 
\paragraph{Essai de \textit{de Lubac}} Il reprend Auguste Comte, Nietzsche, Dostoïevski, et en se coupant de la relation à Dieu, l'homme se détruit lui-même. Les totalitarisme du XX sont nourris de cet humanisme sans Dieu :
\begin{quote}
    « Il\sn{(De Lubac, Drame de l’humanisme, avant-propos, p. 10) } n’est pas vrai que l’homme (…) ne puisse organiser la terre sans Dieu. Ce qui est vrai, c’est que, sans Dieu, il ne peut en fin de compte que l’organiser contre l’homme. L’humanisme exclusif est un humanisme inhumain. » 
\end{quote}

\subsection{L’humanisme proposé par Paul VI}

\paragraph{discours conclusif du 7 décembre 1965} apologie de l'humanisme Chrétien, comme Augustin et Lactance parlaient du Christianisme comme véritable religion. 
\begin{quote}
    « L’humanisme laïque et profane enfin est apparu dans sa terrible stature, et a (…) défié le Concile. La religion du Dieu qui s’est fait homme s’est rencontré avec la religion (…) de l’homme qui se fait Dieu »\sn{contre la dimension prométhéenne de l'humanisme}. « Tout cela (…) a-t-il peut-être fait dévier la pensée de l’Église en Concile vers les positions anthropocentriques prises par la culture moderne ? » (Paul VI, discours du 7 décembre 1965). 
\end{quote}
Le concile s'est occupé non de l'homme abstrait mais concret, "phénoménal". 

Paul VI montre qu'on n'a pas trop parlé de l'homme mais l'homme est central du christianisme.

\paragraph{Comment rejoindre Dieu à partir de l'homme}


\begin{quote}
    « Reconnaissez-lui au moins ce mérite, vous, humanistes modernes, qui renoncez à la transcendance des choses suprêmes, et sachez reconnaître notre nouvel humanisme : nous aussi, nous plus que quiconque, nous avons le culte de l'homme. » (Paul VI) 
        
    \end{quote}
    \begin{quote}
    « La religion catholique est pour l’humanité (…) Elle est la vie, par l’explication que notre religion donne de l’homme ; la seule explication, en fin de compte, exacte et sublime ».
    
    « Pour connaître l’homme, l’homme vrai, l’homme tout entier, il faut connaître Dieu.  (…) La religion catholique est la vie, parce qu’elle décrit la nature et la destinée de l’homme ; elle donne à celui-ci son véritable sens (…) elle infuse à la vie cette énergie mystérieuse qui la rend vraiment divine ». 
    
    « A travers le visage de toute homme (…) nous pouvons et devons reconnaître le visage du Christ \sn{Il parle de l'homme Concret}(Mt 25,40) (…) et si sur le visage du Christ nous pouvons et devons reconnaître le visage du Père céleste : \begin{quote}
        ‘Qui me voit (…) voit aussi le Père’ (Jn 14,9)
    \end{quote}, notre humanisme devient christianisme, et notre christianisme se fait théocentrique, si bien que nous pouvons également affirmer : pour connaître Dieu, il faut connaître l’homme » (Paul VI). 
\end{quote}

\paragraph{Double mouvement} Pour connaitre l'homme, il faut passer par Dieu; mais pour accéder à Dieu, il faut passer par l'homme.

\paragraph{l'humanité est un chemin vers Dieu mais la véritable humanité, c 'est celle du Christ} L'homme Jésus qui entre en dialogue, vient le guérir \textit{comme le bon samaritain}, son service, la proximité des pécheurs.

\paragraph{Lavement des pieds}

\paragraph{Non un transhumanisme} qui se construit elle même mais qui se met au service de l'autre.
\paragraph{Une morale transcendante} Si c'est l'homme qui décide de sa morale, ce sont les puissants qui vont la décider (morale bourgeoise, pour Marx). La morale, c'est la prescription, ce \textit{que l'on doit faire}. 


\subsection{La christologie de Gaudium et spes} 

\paragraph{Théologie de Gaudium et Spes, Eglise dans son rapport au monde} Si le Concile n'a pas fait de christologie, il a néanmoins proposé une interprétation renouvelée du Christ dans le monde moderne (Aggiornamento). Les mots de la foi de façon renouvelée pour que cela touche les gens.

\begin{quote}
    « En réalité, le mystère de l’homme ne s’éclaire vraiment que dans le mystère\sn{On ne parle pas de l'homme mais de son mystère} du Verbe incarné. Adam, en effet, le premier homme, était figure de celui qui devait venir, le Christ Seigneur. Nouvel [novissimus = dernier] Adam, le Christ dans la révélation même du mystère du Père et de son amour, manifeste pleinement l’homme à lui-même et lui découvre la sublimité de sa vocation » (Concile Vatican II, Gaudium et spes 22,1). 
\end{quote}

\paragraph{Christ qui accomplit l'homme} Non pas Christ rédempteur mais Christ qui accomplit.

\paragraph{Tension entre \textit{figure} et \textit{réalité}} La réalité de l'homme apparaît avec le Christ. On parle d'Adam et du nouvel Adam, Adam, \textit{figure}; et Nouvel Adam \textit{réalité}. Montre ce que l'homme est appelé à devenir.

\paragraph{Dans GS, ce qui prédomine, c'est la continuité} entre l'homme et le Christ. La traduction de \textit{nouvel Adam} n'est pas très juste, car elle marque une discontinuité. Novissimus : ultime, pas de discontinuité.

\begin{quote}
    1 Co 15, 45 \TGrec{ἔσχατος Ἀδὰμ} Dernier Adam (eschatos)
\end{quote}

\begin{quote}
    « ‘Image du Dieu invisible’ (Col 1,15), il est l’Homme parfait qui a restauré dans la descendance d’Adam la ressemblance divine, altérée dès le premier péché (…) par le fait même, cette nature a été élevée en nous aussi à une dignité sans égale » (GS 22,2). 
\end{quote}

\begin{quote}
    « Le Verbe de Dieu (…) s’est lui-même fait chair (…). Homme parfait, il est entré dans l’histoire du monde, en l’assumant et la récapitulant en lui. C’est lui qui nous révèle que ‘Dieu est charité’ (1 Jn 4,8) et qui nous enseigne en même temps que la loi fondamentale de la perfection humaine, et donc de la transformation du monde, est le commandement nouveau de l’amour » (GS 38,1). 
\end{quote}
L'homme n'est pas un concept éternel, une essence, mais l'homme concrêt, intégré dans l'histoire.


\paragraph{Humanisation} Proche de la doctrine de la deification chez les grecs, devenir plus homme, en Christ.
\begin{quote}
    « Quiconque suit le Christ, homme parfait, devient lui-même plus homme » (GS 41,1)
\end{quote}
Le Christ n'est pas seulement un homme mais l'homme parfait. Ce qui permet de fonder un humanisme chrétien.

\section{La christologie de K. Rahner en réponse à une société sécularisée}

\subsection{La remise en cause du mythe de l’incarnation }
\paragraph{Une culture qui s'autonomise} du fait des guerres des religions, besoin de fonder une \textit{vivre ensemble} à partir de la philosophie naturelle, la raison et non des \textit{croyances}. Monde un peu technique. Mais aujourd'hui, dans la post-modernité, on est passé à un autre monde, puisque la figure d'expert n'a plus d'autorités.


\paragraph{les choses de la foi ne sont plus des miracles} repenser la formulation des affirmations pour qu'elles ne soient pas rejetées a priori. Par exemple, le \textit{dogme de l'incarnation}, central, ne peut plus être compris de façon mythique. \sn{John Hick, \textit{le Mythe du Christ incarné}}.

\paragraph{Une théologie transcendantale} pour rendre pensable aux hommes contemporains les formulations de la foi. Il va montrer les conditions de penser de la Foi.

\paragraph{contre la mythologisation des expressions de la Foi} d'un Dieu qui descend sur terre en s'habillant avec un vêtement des hommes, une interprétations gnostiques et mythiques :

\begin{quote}
    [Les énoncés de la tradition ne doivent]\sn{Rahner, Réflexions théologiques sur l’incarnation, Ecrits Théologiques, III, Paris 1963,  90.} « pas donner l’impression mythologique que Dieu, travesti dans la livrée [vêtement] d’une nature humaine, qui ne lui adhérait que de l’extérieur,  est descendu sur terre pour régler une situation qu’il ne pouvait plus dominer de son ciel » 
\end{quote}
L'homme n'est qu'un outil pour que Dieu agisse. Mais cela donne un sens à l'humanité. Il y a dans les hérésies une réduction du médiateur à un instrument.

Apollitarisme, monophysisme : empute l'homme, qui n'est pas assumé totalement.

\begin{quote}
    Selon Rahner, est mythologique « toute représentation de l’Incarnation d’un Dieu qui ne voit dans son ‘humanité’ que l’habit, la livrée dont il se ‘sert’ pour signaler sa présence parmi nous, sans que l’humain trouve, précisément dans ce fait d’être assumé par Dieu, son originalité et son autonomie les plus hautes. On voit alors qu’il y a dans les hérésies christologiques, de l’Apollinarisme au monothélisme, une infrastructure intellectuelle soutenue par le même sentiment mythique fondamental.  Que ce sentiment ait déjà la vie si dure dans la formulation conceptuelle devrait nous rendre attentifs à sa survie probable, même en l’absence de professions de foi théoriques, dans l’idée que, en fait, d’innombrables chrétiens se font de l’Incarnation, soit qu’ils y ‘croient’, soit qu’ils la refusent » (Rahner, Ecrits Théologiques, I, 125 n.1). 
\end{quote}

\subsection{Réinterpréter : « le Verbe s’est fait chair » Jn 1, 14}

\begin{quote}
    \TGrec{Καὶ ὁ λόγος σὰρξ ἐγένετο, καὶ ἐσκήνωσεν ἐν ἡμῖν – καὶ ἐθεασάμεθα τὴν δόξαν αὐτοῦ, δόξαν ὡς μονογενοῦς παρὰ πατρός – πλήρης χάριτος καὶ ἀληθείας.}
    Et la Parole est devenue chair, et elle a habité parmi nous (et nous avons contemplé sa gloire, une gloire telle qu'est celle du Fils unique, venu du Père) pleine de grâce et de vérité. Jn 1, 14
\end{quote}


\paragraph{Considérations sur la méthode }

on ne peut pas connaître le verbe sans connaître le Christ Jésus et réciproquement. Aspect dynamique et vivant (\textit{cercle herméneutique}\sn{Dans les termes de Wolfgang Stegmüller (1986, p. 28) : « Le cercle herméneutique semble être le noyau rationnel, qui vient de la thèse sur la distinction ou la supériorité des sciences humaines par rapport aux sciences naturelles, après avoir éliminé tous les facteurs irrationnels. »}) 

L'homme est \textit{capax Dei}, pour accueillir le verbe de Dieu



\paragraph{L’homme est un mystère } Il part de la nature humaine, car nous expérimentons l'homme. Il parait donc plus facile de partir de l'homme. Nous pouvons voir ce qui est essentiel de l'homme de ce qui est accidentel, ce qui change.  Mais qu'est ce qui reste ?

\begin{Def}[Définition de l'homme]
Animal rationnel (zoon logikon). Le propre de l'homme, ce n'est pas d'avoir une âme, mais raisonnable.
\end{Def}

La dimension rationnelle, pour Rahner, est sans limite. 
Il montre que l'homme est toujours ouvert à quelque chose de plus grand. La liberté est liée à l'ouverture à l'homme. L'homme est capable d'un désir pour quelque chose dont il n'a pas besoin. Dans l'âme humaine, il y a quelque chose de sans limite. 


\begin{quote}
    « Nous pensons que cette réduction du médiateur à un moyen utilisé par Dieu existe chaque fois que la nature est considérée comme un pur instrument de la personne, instrument qui n’a plus alors aucune signification pour une personne divine » (Rahner, ET I, 125 n. 2)
\end{quote}

\begin{quote}
    « L’homme est référence au Dieu incompréhensible, c’est pourquoi il n’est pas définissable. \textit{L’Incompréhensible est donc le ce par quoi nous pouvons nous saisir, la condition de possibilité de la connaissance de l’homme.} « L’accueil ou le refus du mystère, ce mystère que nous sommes en tant que référence pauvre au mystère de la plénitude, constitue notre existence » (Rahner, \textit{Traité fondamental de la foi}, 246).  
\end{quote}
L'homme est défini par sa relation à l'infini. Dialectique de Hegel de l'infini et du fini. Pour qu'il y ait du fini, il faut de l'infini : le fini, c'est qu'on découpe dans l'infini. 

\paragraph{Le processus d’humanisation}  

\begin{quote}
    « [La nature humaine] se réalise et réussit à un degré insurpassable et au sens le plus radical là où cette nature qui ainsi se livre au mystère de la plénitude se dessaisit (à ce point) d’elle-même » (Rahner, 86).  
\end{quote}
Plus nous approchons du mystère de Dieu, plus nous nous accomplissons. L'anthropologie sous jacente, c'est Hegel : 
\begin{quote}
    "la personne ne devient elle même qu'en se plongeant dans l'autre"
\end{quote}
L'\textit{hybris}, l'excès ne mène à rien. C'est le décentrement, le dessaisissement qui laisse Dieu accomplir en nous notre pleine réalisation.

\subparagraph{repenser l'union hypostatique}
\begin{quote}
    « Puisque l’homme n’est que dans la mesure où il s’abandonne, l’Incarnation de Dieu se présente donc comme le cas suprême et unique de l’achèvement essentiel de l’humaine réalité » (Rahner, 87).  
\end{quote}

\begin{quote}
    « Le devenir-homme de Dieu (…) est le cas unique et suprême de l’accomplissement essentiel de la réalité humaine, lequel tient en ce que l’homme existe en se perdant dans le secret absolu, que nous nommons Dieu » (Rahner, Traité fondamental, 247).  
\end{quote}
Face à Jean-Paul Sartre, autonomie et proximité radicale atteignent leur maximum de grandeur.

\subparagraph{Pourquoi le cas unique ?} une des faiblesses de Rahner.



\paragraph{Le Christ est l’homme parfait }

\begin{quote}
    « Ainsi et c’est le sens de ce qui suit, la christologie peut-elle être traitée comme \textit{une anthropologie qui se transcende elle-même et l’anthropologie comme une christologie déficiente}. Christologie qui (même si, pour nous, elle est en partie postérieure) est le fondement originel de notre anthropologie et de notre conception de la créature, comme le Christ et le premier né de toute création (Col 1,15) » (Rahner, 135).  
\end{quote}
Finalement notre anthropologie est fondée sur la christologie.

Cette relation au Père est très manifestée dans les Écritures. 

L'incarnation n'est donc par un accident ou un miracle. L'homme est \textit{capax dei}, et donc l'incarnation n'est pas un évènement incompréhensible mais nous permet une compréhension plus profonde de l'humanité. Relation entre les deux. Alors que chez les grecs, il y avait une difficulté à ne pas opposer divinité et humanité, le corruptible et l'incorruptible.


%---------------------------------
\section{La christologie « anthropologique » et la notion de christianisme anonyme} 

\paragraph{une dimension sociale} Le Christ est devenu le \textit{nouvel Adam} et donc une humanité nouvelle. 
\begin{quote}
    Jn 12, 32 "j'attirerai à moi tous les hommes"
\end{quote}
Cette humanité élevée en Christ a un impact sur toute l'humanité. Tout au long de l'histoire, chaque homme doit s'ouvrir à la grâce divine, à entrer dans ce monde qui est déjà en Christ. Rahner va parler de \textit{existential surnaturel}.

\begin{quote}
    « Il existe un christianisme implicite, anonyme […]. Il existe parfaitement et doit exister un rapport, dans une certaine mesure anonyme et cependant réel, de l’homme individuel à la concrétude de l’histoire du salut, et par suite aussi à Jésus Christ, et cela en celui même qui n’a pas encore fait toute l’expérience historique concrète, en même temps qu’explicitement réfléchie, dans la parole et le sacrement, qui le lierait à cette réalité de l’histoire du salut, mais  possède le rapport existentiellement réel de façon simplement implicite, dans l’obéissance face à sa propre référence de grâce au Dieu de l’autocommunication absolue, présente dans l’histoire, en ce que cet homme accueille sa propre existence sans prévention […]. 
    \end{quote}
Pour quelqu'un qui ne vit pas des sacrements, s'il accepte la grâce implicite du Christ (jamais imposé), il est chrétien. Il y a un impact sur toute l'histoire de l'incarnation du Christ. \textit{baptême in petto}


\begin{quote}
A côté de quoi il y a le christianisme plénier, venu explicitement à lui-même dans l’écoute croyante de la parole de l’Evangile, dans le sacrement et dans l’accomplissement explicite de la vie chrétienne, cet accomplissement qui se sait lui-même en rapport à Jésus de Nazareth » (Rahner, Traité fondamental, 342-343) 
\end{quote}



\section{Conclusion}  

Pierre Claverie \sn{ Pierre CLAVERIE, Lettres et messages d’Algérie, Paris 1996 } :
\begin{quote}
   :  « L’Occident s’est aperçu, il y a une trentaine d’années, de l’existence de l’islam.  Jusque-là, dans un attitude de colonisateur, il portait un regard méprisant, ou au mieux indifférent, sur cette religion (…) » (p. 17). « Les aventures coloniales et missionnaires du siècle dernier nous ont appris qu’il y avait une véritable perversion à croire que chacun réalise l’universel et qu’il a donc le droit (divin) de s’imposer à tous comme la perfection absolue » (p. 23).  « J’en veux aux religions, et même souvent à mon Église, de pratiquer plus volontiers un monologue agressif et de cultiver leur particularismes, et je souffre de voir quel lamentable témoignage donnent les croyances dans leur prétention à soumettre et à régir l’humanité en l’asservissant à leurs lois » (p. 24). 
\end{quote}