\chapter{Elements de réflexion de Post-Modernité}

\section{Le pape François laisse entendre que l’attachement au rite tridentin constitue un produit du nihilisme}
\mn{tribune
Grégory Solari
Théologien
  le 05/07/2022 à 15:57 La croix
} 
 \begin{Synthesis}
     Grégory Solari éclaire la lecture de la lettre apostolique du pape sur la liturgie à la lumière de la question du nihilisme. L’attachement formaliste au rite tridentin serait une manière pour certains de recourir à la « tradition », pour compenser le déficit symbolique qui caractérise la postmodernité. En oubliant que le rite n’est rien s’il n’y a pas l’Église derrière.
 \end{Synthesis}



Se pourrait-il que l’attachement au rite tridentin constitue un produit du nihilisme ? C’est ce que laisse indirectement entendre la lettre apostolique du pape François sur la formation liturgique Desiderio desideravi. 

\begin{Def}[Nihilisme]
Par nihilisme, il faut comprendre un phénomène affectant de manière différenciée la question de la valeur. 
\end{Def}

Dans la configuration du nihilisme, explique encore le pape, \begin{quote}
    « l’homme se sent perdu, sans références d’aucune sorte, privé de valeurs parce qu’elles sont devenues indifférentes, orphelin de tout, dans une fragmentation où un horizon de sens semble impossible – (une époque) encore chargée du lourd héritage que nous a laissé l’époque précédente » (n. 28).
\end{quote}  

\paragraph{Clé de lecture de Traditionis Custodes } Il ne faut pas passer trop vite sur cet horizon du nihilisme. C’est lui, je crois, qui constitue a posteriori la clé herméneutique de Traditionis custodes.

 
On sait que l’essence du nihilisme consiste dans le phénomène de la dévalorisation. Après la dissolution du lien entre le Nom de Dieu (révélé) et les attributs projetés sur lui par les hommes (« mort de Dieu »), plus aucun critère ne garantit la valorisation. La dissolution du lien entre Dieu et ses attributs a comme fissuré le rapport de l’homme avec ses propres productions, créant un écart que rien ne vient plus combler – ou plutôt : qu’une seule chose comble : le « rien » précisément. Dès lors, plus rien n’a de valeur réelle, et ce qui se trouve encore valorisé ne l’est jamais qu’à partir de critères extérieurs à l’objet (critères économiques, politiques, esthétiques, sociologiques, etc.), et non sans être affecté d’une teneur d’arbitraire.

\paragraph{Liturgie et nihilisme}
La force de l’analyse de François réside d’abord dans sa lucidité : le pape, comme avant lui le Concile sur lequel il appuie sa réflexion, n’esquive pas la réalité du nihilisme. Le n. 29 inscrit explicitement la question de la liturgie et de sa réforme dans cet horizon de la postmodernité : 
\begin{quote}
    « C’est avec cette réalité du monde moderne que l’Église, réunie en Concile, a voulu se confronter, en réaffirmant sa conscience d’être le sacrement du Christ, (…), et ce n’est pas un hasard si cet immense effort de réflexion du concile œcuménique a commencé par une réflexion sur la Liturgie (Sacrosanctum Concilium). »
\end{quote} 

Ce que ménage la liturgie, c’est la possibilité d’une sortie (momentanée) du nihilisme. Mais pas automatiquement, ni sans que le rapport à une forme rituelle ne subisse l’effet de la perte de tout critère formellement « absolu ». Comme il arrive dans le cas d’un attachement exclusif au missel tridentin.

 
Alors que le \textit{« désert croît »} (Nietzsche), on peut en effet comprendre le réflexe qui recourt à la « tradition » pour compenser le déficit symbolique qui caractérise la postmodernité. Mais ce qu’il faut bien voir, c’est que ce réflexe, parce qu’il confond la tradition avec le passé, ne contient pas le nihilisme ambiant, au contraire, il l’alimente. Faute d’une critériologie authentique, toute valorisation repose sur la « volonté de puissance » (toujours subjective et arbitraire). Si François insiste sur le lien de la lex orandi et de la lex credendi, c’est parce qu’il n’y a pas d’écart entre l’Église et la liturgie – il n’y a pas de vide: l’amour du Christ remplit tout et se découvre dès lors comme l’unique critère de valorisation. Tandis qu’à distance de l’Église qui se reçoit et se constitue dans la célébration du Mystère pascal (cf. n. 24-26), toute forme liturgique tend à se transformer en formalisme. Tout style, en stylisation. Toute réalité, en artifice. Le « néo » devient l’autre nom du néant.
\paragraph{L’unique critère ecclésial}

Ce qui ne signifie pas que le rite tridentin soit privé de « valeur ». Simplement, ce qu’il faut retenir de la lettre apostolique sur ce point, c’est que rien hormis la référence à la vie de la communauté ne garantit qu’une valorisation ne soit pas d’une manière ou d’une autre arbitraire. Pourquoi ? Parce que parmi toutes les \textit{« grandeurs institutionnelles »} (Pascal) seule l’Église se trouve depuis toujours privée de tout pouvoir « constitutionnel » sur elle-même. L’institution ecclésiale n’existe que dans l’acte par lequel elle reçoit son existence du Christ. Non pas une fois, mais continuellement, dans la donation du Corps du Christ qui la constitue comme « corps » à son tour. Bref, de bout en bout et sans reste, c’est sur le désir du Christ, et sur lui seul, que repose l’Église (comme événement et comme institution). Et donc aussi la liturgie, dont il faut comprendre les rites comme des expressions de la réponse que la communauté a données et donne à cette continuelle « donation christique ». Voilà pourquoi, surtout dans le temps du nihilisme, mais pas seulement, il n’existe pas d’autre critère de valorisation de la liturgie sinon l’Église elle-même. Seule l’Église, comme « sacrement » du Mystère pascal, résiste à la corrosion subtile du nihilisme.
 
L’attachement exclusif au rite tridentin a réduit à néant le but du motu proprio de 2007 (« fécondation mutuelle » des deux missels). Comme le disait déjà Abraham Heschel, 
\begin{quote}
    « ce n’est pas le rite qui est malade, c’est l’intentionnalité de notre cœur »
\end{quote}
 – ce que nos frères juifs appellent la kavana, condition de toute prière authentique. Devant cet avortement pastoral, le pape, avec Desiderio desideravi, rejoint et prolonge ce qui fut l’impulsion initiale du mouvement liturgique : recouvrer la kavana de la prière chrétienne. Il s’agit non de « bannir la messe en latin », mais de ménager les conditions qui rendront possible ce qui fut l’intention authentique de Summorum Pontificum. La génération « Ecclesia Dei » aurait pu y contribuer. Pour l’heure, son expérience compte malheureusement pour « rien ».
 
 
 
 \section{William Cavanaugh : « L’Église n’est pas l’arche de salut dans un monde en déluge » }
 \mn{entretien
William Cavanaugh
Théologien, professeur à l’université DePaul à Chicago 
Recueilli par Xavier Le Normand et Dominique Greiner, le 20/06/2022 à 10:47 Modifié le 20/06/2022 à 12:48}

\begin{Synthesis}
    
Théologien américain, William Cavanaugh développe depuis des années une réflexion originale sur l’apport politique des chrétiens dans les sociétés libérales. Ce laïc, actuellement professeur à l’université DePaul à Chicago, porte notamment un regard critique sur la bataille menée par l’Église catholique américaine contre le droit à l’avortement.
\end{Synthesis}
 
 
\textit{ 
La Croix : Ces dernières années semblent n’avoir été qu’une succession de crises : financière, économique, sanitaire, écologique, et maintenant la guerre en Ukraine. Comment encore croire aux deux premiers chapitres de la Genèse, à Dieu dans notre histoire ?
}

\begin{quote}
    William Cavanaugh : Il est certainement possible de penser que tout va mal, que la violence est juste la façon dont les choses se présentent. Et beaucoup de gens ont lu l’histoire de la chute d’Adam et Ève comme signifiant que nous sommes tombés et que c’est la façon dont les choses doivent se passer. Mais il n’y a rien qui dise que tel doit être le dernier mot, et je crois que le récit de la chute peut servir une doctrine pleine d’espoir car il y a du bon qui vient, après.

D’ailleurs, je pense que tout le monde cherche un moyen de sortir des crises actuelles, nous voulons tous que la guerre en Ukraine s’arrête, nous voulons tous que le réchauffement climatique soit contrôlé. Ce que tout le monde veut, c’est simplement serrer ses enfants dans ses bras, se nourrir et profiter du soleil chaque jour. C’est justement ce qui est si horrible à propos de la guerre et de l’Ukraine : cela ne devait pas se passer comme ça. Je crois qu’il y a quelque chose de fondamental dans l’humanité qui fait que nous ne pouvons pas rester dans cet état de violence, quelque chose au fond de nous qui sait que les mots finaux sont l’amour et la bonté. Malgré tout, c’est ce que nous recherchons tous, car cela est profondément ancré dans la vie humaine et dans le cœur humain, et même dans la nature de l’univers : ce Dieu qui nous a créés est un Dieu de bonté et l’amour a le dernier mot. Je crois que même ceux qui ne reconnaissent pas Dieu ressentent aussi cela, à un certain niveau, qu’il y a quelque chose de plus.

\end{quote}


\textit{Pour certains, le retour à ce Paradis perdu passe par un repli sur soi de l’Église, face à un monde corrompu…
}

\begin{quote}
    

W. C. : Dire que l’Église est le corps du Christ ou dire qu’elle est un avant-goût du Royaume de Dieu n’est pas une déclaration à propos de nous, mais à propos de Dieu. C’est une affirmation sur ce que Dieu a fait pour nous, sur ce que le Christ a fait pour nous afin de nous offrir la possibilité d’entrer dans cette communauté du pardon. Mais cela ne doit surtout pas nous conduire à un triomphalisme, à donner à voir l’Église comme parfaite et disposant de toutes les réponses. L’Église n’est pas l’arche de salut dans un monde en déluge. Elle est un sacrement, ce qui signifie une sorte de réalité matérielle très ordinaire du monde qui vient de Dieu et à travers laquelle Dieu a choisi de se manifester.



Il y a beaucoup de péchés à l’intérieur des frontières de l’Église visible. Se retirer dans une enclave que nous pensons parfaite est donc impossible. Nous sommes appelés à ouvrir de nouvelles possibilités pour vivre une belle vie et à nous joindre aux non-chrétiens, en essayant d’imaginer ce que pourrait être une vie joyeuse, une vie rachetée. L’une des façons d’y parvenir est de reconnaître et de pratiquer la pénitence et le pardon, en reconnaissant notre propre péché.


Certains considèrent qu’autrefois, il y a quelques années ou quelques décennies, tout allait bien et que nous vivons désormais un enfer, à cause de forces extérieures, comme le sécularisme et l’athéisme. Le pape François essaie de nous éloigner de cette nostalgie. Il nous appelle à voir que ce n’est pas quelque chose qui nous a été fait depuis l’extérieur, mais quelque chose que nous nous sommes fait à nous-mêmes.

Il nous appelle à revenir à l’essentiel, à la joie comme il le dit explicitement dans son exhortation programmatique \textit{Evangelii gaudium}. Il y rappelle notamment qu’il s’agit d’une sorte de joie que l’on ne peut avoir que lorsque l’on renonce aux prétentions au pouvoir. Je crois qu’il suffit d’y renoncer pour accéder à une certaine forme de liberté.
\end{quote}

\textit{Pourtant, dans bon nombre de pays les épiscopats semblent plutôt vouloir agir sur le pouvoir, comme aux États-Unis où l’Église a mené la fronde contre l’arrêt « Roe vs Wade » garantissant le droit à l’avortement et qui vient d’être renversé.}
\begin{quote}
    

W. C. : En effet, et je ne crois pas que cela soit un bon combat. Récemment, a été publiée une recherche très intéressante sur l’affiliation religieuse aux États-Unis. Alors que l’affiliation religieuse des jeunes à l’Église catholique est stable entre 1972 et 1991, elle chute soudainement entre 1991 et 1998. L’une des théories expliquant ce phénomène est que c’est à ce moment-là que les chrétiens aux États-Unis ont commencé à s’impliquer dans les partis politiques. C’est l’ère de la majorité morale, l’époque où les évêques catholiques font pression sur la question de l’avortement. Corrélation n’est pas causalité, mais ces données sont intéressantes.

 
L’Église reçoit ce qu’elle mérite. Lorsque vous pensez la nature politique, sociale et communautaire de l’Église comme quelque chose qui n’est canalisé que par les partis politiques pour s’accrocher à une hégémonie culturelle, c’est que vous savez que vous avez déjà perdu. Et le résultat est forcément désastreux, car c’est un contre-témoignage, l’inverse du chemin de croix. C’est malheureusement une habitude catholique depuis de nombreux siècles, mais ce n’est pas l’Évangile. Sur la question de l’avortement, par exemple, il y a des façons d’essayer de lier la justice sociale et le souci des enfants à naître qui ne sont pas imposées par des politiques coercitives.
\end{quote}

\textit{
L’Église au Chili n’est-elle pas un exemple de cette Église en perdition après s’être trop liée au pouvoir ?
}
\begin{quote}
    

W. C. : Lorsque j’ai quitté le Chili à la fin 1989, le prestige de l’Église était immense, car elle s’était dressée contre le régime militaire et elle avait défendu du mieux qu’elle pouvait les pauvres, ceux qui étaient torturés. Cette réputation s’est effondrée parce qu’à partir de 1978, un autre type d’évêque a commencé à être nommé au Chili et dans toute l’Amérique latine.

Le pape Jean-Paul II était formidable à bien des égards, mais ce qui est arrivé à l’Église en Amérique latine sous sa direction a été tragique sur bien des aspects. À partir de son élection, les évêques nommés étaient souvent alignés avec les régimes militaires franchement de droite, ou du moins n’y étaient pas hostiles. Par exemple, l’archevêque de San Salvador qui a succédé à Mgr Oscar Romero a été nommé général de brigade honoraire de l’armée salvadorienne. La situation au Chili résulte de cette dynamique, et cela me brise le cœur.
\end{quote}

\textit{
Et aux États-Unis ?
}
\begin{quote}
    
W. C. : Aux États-Unis, le changement des évêques a rendu la situation actuelle inévitable. Dans les années 1980, il y avait toujours des lettres pastorales des évêques sur le défi de la paix ou une critique du capitalisme. Depuis, seuls l’avortement et la liberté religieuse semblent préoccuper, et encore pour la seconde, dans une visée d’intérêts personnels, de protection des propres prérogatives de l’Église. Je ne vois pas comment cela pourrait « évangéliser » qui que ce soit. Certes, d’une certaine manière, les plaintes des évêques sont justifiées. Mais quand cela devient la question principale avec l’avortement, alors les gens finissent par considérer les évêques comme alignés avec le Parti républicain, qui est désormais le parti de Donald Trump. Et cela n’est pas une bonne chose.


\end{quote}
\textit{Cela a toutefois permis à l’Église d’obtenir le renversement de « Roe vs Wade »…
}

\begin{quote}
    

W. C. : Je crois que c’est un mauvais débat, avec des deux côtés un individualisme : d’un côté mon choix, et de l’autre vous avez votre bébé, votre problème. Nous devrions avoir un soutien plus cohérent pour les femmes en situation de grossesse indésirée, et ce n’est clairement pas là-dessus que l’accent a été mis. Maintenant, il est un peu tard pour plaider en faveur du soutien aux femmes et aux enfants, ce qui aurait probablement dû être la priorité depuis le début.

De toute façon, je ne suis pas très optimiste quant au fait que telle sera désormais la priorité. Nous n’avons pas développé les habitudes et les vertus qui seraient nécessaires pour le moment présent, à savoir soutenir les femmes et les enfants.
 
En tout état de cause, l’Église officielle a un problème de crédibilité avec les femmes, pas seulement parce qu’elles ne peuvent pas être ordonnées mais plus généralement parce que les femmes n’ont pas eu droit à une voix. Si vous voulez convaincre les gens que l’ordination réservée aux hommes n’est pas une question de pouvoir mais un commandement du Christ, alors il faut que les femmes puissent accéder par d’autres moyens au pouvoir au sein de l’Église. Ce serait le seul argument qui pourrait rendre acceptable la non-ordination des femmes.
\end{quote}

\textit{
À vous entendre, l’avenir est sombre… Êtes-vous tout de même optimiste pour l’Église ?
}

\begin{quote}
    

W. C. : Je suis plein d’espoir, mais pas nécessairement optimiste. Plein d’espoir, car je pense vraiment que le pape François est exactement le pape dont nous avons besoin en ce moment. Il y a aussi des lieux qui me donnent de l’espoir. Par exemple dans certains pays d’Afrique, il y a un dynamisme réel, puissant.

Cela dit, nous avons parfois dans les pays occidentaux comme un désir vampirique vis-à-vis de l’Église africaine, comme si nous étions déjà morts et que nous allions obtenir la vie par elle. C’est une attitude à fuir, à mon sens, notamment car là aussi des scandales d’abus vont exploser, mais surtout car nous ne sommes pas morts. Il y a beaucoup de belles choses qui se passent et nous pouvons utiliser ce contexte de la perte du pouvoir culturel comme une occasion de nous rapprocher de Dieu et comme le temps de briser certaines des frontières entre les soi-disant croyants et les soi-disant non-croyants.

Je dis dans mes articles que l’Église est au bord de l’échec, mais c’est une chance. Vous savez, être sur le point d’échouer, c’est être sur le chemin de la croix. Nous devons apprendre à mourir, comme le grain de blé meurt en terre pour donner vie. L’Église ne va pas mourir, j’en ai l’espoir, mais elle va devoir changer de visage.
\end{quote}
\mn{William Cavanaugh a publié en français Idolâtrie ou liberté, Le défi de l’Eglise au XXIème siècle (Salvator, 2022)}