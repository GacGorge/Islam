\section{Si les catholiques ne font que refléter l’image de la société, ils sont insignifiants}

\mn{La Croix, Monique Baujard, membre du comité de pilotage de Promesses d’Eglise 30 mai 2022
 }

Le résultat de l’élection présidentielle a été longuement analysé et commenté. Le constat est celui d’une France fracturée. Pour les uns, elle est divisée en trois, avec un centre, une extrême gauche et une extrême droite. D’autres renvoient les extrêmes dos à dos et estiment qu’elle est séparée en deux : pour ou contre une démocratie libérale, une économie sociale de marché, l’Europe et l’Otan. Une fracture donc entre ceux qui pensent qu’il est possible de réformer la société dans le système existant et ceux qui prônent une rupture plus radicale.

Les votes n’ont pas toujours été motivés par une adhésion au programme des candidats mais ont permis d’exprimer aussi un vrai mécontentement et une angoisse du lendemain. Ces peurs qui travaillent nos sociétés ont été largement exploitées par certains candidats. La transition écologique, la situation socio-économique, la transformation culturelle ou l’instabilité internationale peuvent nous inquiéter légitimement. Mais les priorités, l’analyse des causes et la proposition des remèdes diffèrent, et ces questions ne sont pas ressenties par tous avec la même acuité.

Une partie importante des catholiques semble particulièrement inquiète devant la transformation culturelle de notre pays. Ils constatent que le christianisme a perdu en Europe le statut privilégié qu’il a eu pendant des siècles, au cœur de la culture. De fait, la sécularisation a entraîné la disparition progressive de cette figure historique d’un christianisme qui était matrice de la culture. Cela ne veut pas dire que le christianisme en tant que tel va disparaître. Le cardinal Jozef De Kesel, archevêque de Malines-Bruxelles, relève que l’Évangile n’a pas forcément besoin d’une culture chrétienne pour rejoindre nos contemporains.

En revanche, l’Église va changer et sera, selon lui, demain plus humble, plus petite, plus confessante et plus ouverte\sn{(1) Foi \& religion dans une société moderne, Salvator, 2021.}. Que cela perturbe des catholiques, c’est une chose. Qu’ils en déduisent le déclin inéluctable de la foi chrétienne et se laissent séduire par les propositions simplistes d’un discours populiste, c’est tout autre chose. Le vote d’extrême droite d’une partie de l’électorat catholique suscite l’incompréhension de l’autre partie, qui le juge incompatible avec l’Évangile (Mt 25,31-40). À l’inverse, la frange plus conservatrice des catholiques pense que l’attitude conciliaire conduit l’Église à sa perte.

En l’état, les catholiques sont donc tout aussi divisés que la société française dont ils font partie. Le dialogue entre les tenants des deux visions semble inexistant. L’épiscopat lui-même est divisé et de toute façon les lieux et la culture de dialogue font défaut dans l’Église catholique. Les catholiques se contentent ainsi aujourd’hui de tenir un miroir à la société française. Mais peuvent-ils se limiter à cela sans trahir leur baptême ?

Lorsque les catholiques ne font plus que refléter l’image de la société, ils sont devenus insignifiants. C’est là un risque contre lequel le pape François les a déjà mis en garde : « Le problème n’est pas d’être peu nombreux mais d’être insignifiants. » Accepter de devenir insignifiants, c’est cesser de prendre la croix au sérieux. Car la croix, elle, n’est jamais insignifiante. Dans un monde déchiré par les conflits, défiguré par les souffrances humaines et l’épuisement de la terre, la croix du Christ reste solidement plantée, comme notre seul phare au cœur des heures les plus sombres. Qui peut alors, au pied de la croix, en contemplant le Crucifié, affirmer que l’histoire est écrite d’avance selon un scénario simpliste ? Que devient l’imprévu de Dieu qui retourne si souvent les situations humaines inextricables en occasions de salut ? La croix reste pour tous les chrétiens le signe par excellence que Dieu peut ouvrir des chemins de vie, là où humainement parlant nous ne voyons que la mort.

C’est ce message d’espérance dont les catholiques ont à témoigner aujourd’hui dans une société pétrie d’anxiété et d’incertitudes. C’est autour de ce témoignage d’espérance qu’ils auront à se rassembler s’ils ne veulent pas être insignifiants dans la société.

