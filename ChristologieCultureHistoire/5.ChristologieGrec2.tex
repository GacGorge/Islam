

\chapter{Le monde grec (IVe-Ve s.) :
La christologie du Verbe incarné}


\mn{Christologies et cultures dans l'histoire
5}


\subsection{Eléments bibliographiques}

DORÉ, J., « Les christologies patristiques et conciliaires » dans
LAURET, B. -- REFOULÉ, F. (dir.), \emph{Initiation à la pratique de la
théologie ** Dogmatique} I, Paris 1982, 185-262.

FÉDOU, M., \emph{La voie du Christ II. Développements de la christologie
dans le contexte religieux de l'Orient ancien. D'Eusèbe de Césarée à
Jean Damascène (IV\textsuperscript{e}-VIII\textsuperscript{e} siècle)},
Paris 2013.

GRILLMEIER, A., \emph{Le Christ dans la tradition chrétienne I. De l'âge
apostolique à Chalcédoine (451)}, Paris 2003\textsuperscript{2}.

GRILLMEIER, A., \emph{Le Christ dans la tradition chrétienne, II-1. Le
Concile de Chalcédoine (451). Réception et opposition (451-513)}, Paris
1990\emph{.}

GRILLMEIER, A., \emph{Le Christ dans la tradition chrétienne, II-2.
L'Église de Constantinople au VIe siècle}, Paris 1993.

SESBOÜÉ, B., \emph{Jésus-Christ dans la tradition de l'Église}, Paris
1982.

SESBOÜÉ, B. -- WOLINSKI, J., \emph{Le Dieu du salut. Histoire de dogmes}
I, Paris 1994.


\hypertarget{introduction}{%
\section{Introduction}\label{introduction}}




  \section{Jésus-Christ est « consubstantiel au Père »}

  
  
  
    \subsection{Le conception d'Arius : le Verbe divin est une créature}
  
    \subsection{Le concile de Nicée (325) : une nouvelle « définition » du
    Christ, vrai Dieu}

    
    
    
      Consubstantiel au Père
    
      L'argument sotériologique
    
      Conclusions sur le plan christologique
    

 ~
  \hypertarget{lincarnation-du-logos-divin}{%
  \section{L'incarnation du Logos
  divin}\label{lincarnation-du-logos-divin}}

  
  
  
    \subsection{L'origine biblique d'un concept formulé par les Pères}
  
    \subsection{Les présupposés grecs à l'idée d'incarnation}

    
    
    
      Le premier présupposé : l'opposition entre le monde divin et le
      monde des hommes
    
      Deuxième présupposé : l'idée de l'homme dans la culture grecque
    
      Troisième présupposé : la descente du Sauveur divin
    
  
 ~
  \hypertarget{les-christologies-de-lincarnation}{%
  \section{Les christologies de
  l'incarnation}\label{les-christologies-de-lincarnation}}

  
  
  
    Les fondements des christologies de l'incarnation
  
    Le modèle antiochien
  
    Le modèle alexandrin
  
 ~
  \hypertarget{la-duxe9finition-chalcuxe9donienne-du-christ-une-personne-en-deux-natures}{%
  \section{La définition chalcédonienne du Christ : « une personne en
  deux natures
  »}\label{la-duxe9finition-chalcuxe9donienne-du-christ-une-personne-en-deux-natures}}

  
  
  
    Au-delà de la confusion et de la division
  
    Le rééquilibrage de Chalcédoine
  
    La formule de Chalcédoine : une définition ou un chemin
    d'intégration ?
  


\hypertarget{arius}{%
\section{Arius}\label{arius}}


{[}Nous disons{]} Que le Fils n'est ni inengendré, ni une partie
d'inengendré, et qu'il ne provient non plus d'aucun substrat : c'est par
volonté et conseil qu'il a eu l'existence avant les temps et les siècles
; il est plein de grâce et de vérité, il est Dieu, Fils unique, immuable
; et avant d'avoir été engendré (voir Pr 8, 25), ou créé (voir Pr 8,
22), ou établi (voir Rm 1, 4), ou fondé (voir Pr 8, 23), il n'était pas
; car il n'était pas inengendré.

Or nous sommes persécutés pour avoir dit : « le Fils a un commencement,
mais Dieu est sans commencement. ». Voilà pourquoi nous sommes
persécutés, et aussi parce que nous avons dit : « Il est à partir du
néant. » (\emph{Lettre d'Arius à Eusèbe de Nicomédie (318 ou 321-322)}
cité dans B. Sesboüé, \emph{Dieu peut-il avoir un Fils}, 32).


\hypertarget{iruxe9nuxe9e-de-lyon}{%
\section{Irénée de Lyon}\label{iruxe9nuxe9e-de-lyon}}


« Si l'homme n'avait pas été uni à Dieu, il n'aurait pu recevoir en
participation l'incorruptibilité. Car il fallait que le `Médiateur de
Dieu et des hommes', par sa parenté avec chacune des deux parties, les
ramenât l'une à l'autre à l'amitié et à la concorde, en sorte que tout à
la fois Dieu accueillît l'homme et que l'homme s'offrît à Dieu. Comment
aurions-nous pu en effet avoir part à la filiation adoptive à l'égard de
Dieu, si nous n'avions pas reçu, par le Fils, la communion avec Dieu ?
Et comment aurions- nous reçu cette communion avec Dieu, son Verbe
n'était pas entré en communion avec nous en se faisant chair » (IRENEE,
\emph{Contre les hérésies}, III 187).


\hypertarget{celse}{%
\section{Celse}\label{celse}}


« Nul Dieu, écrivait celui-ci, nul Fils de Dieu n'est descendu ni ne
saurait descendre (\ldots) Dieu est bon, beau, bienheureux, au plus haut
degré de la beauté et de l'excellence. Dès lors, s'il descend vers les
hommes, il doit subir un changement : changement du bien au mal, de la
beauté à la laideur, de la félicité à l'infortune, de l'état le meilleur
au pire (\ldots) Il est vrai certes que pour un mortel la nature est de
se changer et de se transformer, mais pour un immortel, c'est d'être
identique et immuable. Dieu ne saurait donc non plus admettre un tel
changement. » (ORIGENE, \emph{Contre Celse}, V, 2 et IV , 14)


\hypertarget{origuxe8ne}{%
\section{Origène}\label{origuxe8ne}}


« La fragilité d'un entendement mortel ne voit pas comment elle pourrait
penser et comprendre que cette Puissance si grande de la majesté divine,
cette Parole du Père lui-même, cette Sagesse de Dieu dans laquelle ont
été créés tout le visible et tout l'invisible, ait pu, comme il faut le
croire, exister dans les étroites limites d'un homme qui s'est montré en
Judée, et aussi que la Sagesse de Dieu ait pénétré dans la matrice d'une
femme, qu'elle soit née comme un petit enfant, qu'elle ait émis des
vagissements à la manière des nourrissons qui pleurent ; et ensuite
qu'elle ait été troublée à l'heure de la mort, comme on le rapporte et
comme Jésus le reconnaît lui-même (\ldots) ; et enfin qu'elle ait été
conduite à la mort (\ldots). Je pense que cela dépasse même la capacité
des saints apôtres : bien mieux l'explication d'un tel mystère est peut
être au-dessus des puissances célestes de toute la création » (ORIGENE,
\emph{Des principes}, II, 6,2).


\hypertarget{epuxeetre-apocryphe-de-jacques-835-1030}{%
\section{Epître apocryphe de Jacques, 8,35 --
10,30}\label{epuxeetre-apocryphe-de-jacques-835-1030}}


« Voyez : Je suis descendu, j'ai parlé, j'ai été maltraité, j'ai porté
ma couronne, afin de vous sauver. Je suis descendu, en effet, pour
habiter avec vous, afin que, vous aussi vous demeuriez avec moi. Et
ayant trouvé vos maisons sans toit, j'ai demeuré dans les maisons qui
pourraient me recevoir au moment où je descendrais (\ldots). C'est pour
vous que je suis descendu. C'est vous les bien-aimés. C'est vous qui
allez devenir cause de la Vie en beaucoup ».


\hypertarget{la-formule-christologique-du-concile-de-chalcuxe9doine-451}{%
\subsection{La formule christologique du concile de Chalcédoine
(451)}\label{la-formule-christologique-du-concile-de-chalcuxe9doine-451}}






\begin{table}[h!]
    \centering
    \footnotesize
        \sidecaption{\emph{Source :} SESBOÜÉ, B. -- WOLINSKI, J., \emph{Le Dieu du salut.
Histoire de dogmes} I, Paris 1994, 409\emph{.}}
 

 % Please add the following required packages to your document preamble:
% \usepackage[normalem]{ulem}
% \useunder{\uline}{\ul}{}
 
\begin{tabular}{p{.3\textwidth}p{.3\textwidth}p{.3\textwidth}}
\multicolumn{3}{p{\textwidth}}{Suivant donc les saints Pères, nous enseignons tous unanimement que nous confessons}                                                                                                                                  \\
\multicolumn{3}{c}{Un seul et même Fils, Notre Seigneur Jésus-Christ Le même}                                                                                                                                                            \\
\begin{tabular}[c]{@{}c@{}}
B.\\    \\ Parfait en divinité\end{tabular} & le même                                                                                  & parfait en humanité                                                  \\
                                                                       & le même                                                                                  &                                                                      \\
Vraiment Dieu                                                          &                                                                                          & et vraiment homme   d’une âme raisonnable et d’un corps              \\
consubstantiel au Père                                                 & et le même                                                                               & consubstantiel à nous                                                \\
selon la divinité                                                      &                                                                                          & selon l’humanité                                                     \\
                                                                       &                                                                                          & en tout semblable à nous sauf le péché,                              \\
avant les siècles   engendré                                           &                                                                                          & et au derniers jours (engendré)                                      \\
du Père selon la divinité                                              & le même                                                                                  & pour nous et pour notre salut de la Vierge Marie, Mère de Dieu selon \\
                                                                       &                                                                                          & L’humanité                                                           \\
\multicolumn{1}{l}{}                                                   & \multicolumn{1}{p{.3\textwidth}}{Un seul et même Christ, Fils, Seigneur, l’unique engendré}            & \multicolumn{1}{l}{}                                                 \\
Reconnu en deux natures,                                               &                                                                                          &                                                                      \\
Sans confusion, sans changement                                        &                                                                                          & sans division, sans séparation,                                      \\
                                                                       & La différence des natures n’étant nullement supprimée                                    &                                                                      \\
                                                                       & A cause de l’union                                                                       &                                                                      \\
                                                                       \\
                                                                       & La propriété de l’une et l’autre nature étant bien plutôt sauvegardée                    &                                                                      \\
                                                                       & Et concourant à une seule personne Et une seule hypostase                                &                                                                      \\
                                                                       & Un Christ ne se fractionnant ni se divisant en deux personnes Mais un seul et même Fils, &                                                                      \\
                                                                       & Unique engendré, Dieu Verbe, Seigneur, Jésus-Christ,                                     &                                                                      \\
\multicolumn{3}{p{\textwidth}}{Selon que depuis longtemps les prophètes l’ont enseigné de lui, que Jésus-Christ lui-même nous l’a enseigné et que le Symbole des Pères nous l’a transmis.}                                                          
\end{tabular}
\end{table}



