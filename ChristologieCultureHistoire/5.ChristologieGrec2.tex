

\chapter{Le monde grec (IVe-Ve s.) :
La christologie du Verbe incarné}


\mn{Christologies et cultures dans l'histoire
5}


\subsection{Eléments bibliographiques}

DORÉ, J., « Les christologies patristiques et conciliaires » dans
LAURET, B. -- REFOULÉ, F. (dir.), \emph{Initiation à la pratique de la
théologie ** Dogmatique} I, Paris 1982, 185-262.

FÉDOU, M., \emph{La voie du Christ II. Développements de la christologie
dans le contexte religieux de l'Orient ancien. D'Eusèbe de Césarée à
Jean Damascène (IV\textsuperscript{e}-VIII\textsuperscript{e} siècle)},
Paris 2013.

GRILLMEIER, A., \emph{Le Christ dans la tradition chrétienne I. De l'âge
apostolique à Chalcédoine (451)}, Paris 2003\textsuperscript{2}.

GRILLMEIER, A., \emph{Le Christ dans la tradition chrétienne, II-1. Le
Concile de Chalcédoine (451). Réception et opposition (451-513)}, Paris
1990\emph{.}

GRILLMEIER, A., \emph{Le Christ dans la tradition chrétienne, II-2.
L'Église de Constantinople au VIe siècle}, Paris 1993.

SESBOÜÉ, B., \emph{Jésus-Christ dans la tradition de l'Église}, Paris
1982.

SESBOÜÉ, B. -- WOLINSKI, J., \emph{Le Dieu du salut. Histoire de dogmes}
I, Paris 1994.


\hypertarget{introduction}{%
\section{Introduction}\label{introduction}}

Dans un dialogue, on est aussi transformé par l'autre. On assiste une christologisation du Logos.
On va affirmer la consubstantialité du Logos au Père et l'incarnation/l'humanisation du Logos :
\begin{itemize}
    \item Unité du Christ avec Dieu
    \item Unité du Christ avec l'homme
\end{itemize}

Le problème, c'est que quand on adopte des termes étrangers à notre culture, on risque de mal les comprendre. S'adresser avec des termes reçus mais il faut aussi transformer ces termes pour coller avec la réalité de la Foi.

%-------------------------------------------------------------
\section{Jésus-Christ est « consubstantiel au Père »}
  
Il fallait avant tout garantir l'unité réelle entre le Père et le Fils.
\begin{quote}
    Le Père et moi, nous sommes Uns.
\end{quote}
On peut l'interprêter de différentes manières.
La théologie du \emph{Logos} avait diminué le lien entre le Père et le Fils.
Surtout dans la culture grecque, les attributs de Dieu ne sont pas Dieu mais \textit{dégradés}.


  
  
\subsection{Le conception d'Arius : le Verbe divin est une créature}

\paragraph{Arius} Prêtre d'Alexandrie, qui a obligé l'Eglise pour mieux exprimer son rapport entre le Père et le Fils. 270-338. Le Christ comme \textit{divin} mais \textit{Créature} s'appuyant sur ; 
\begin{itemize}
    \item défend l'\textit{immuabilité de Dieu}. Or, si le Christ a souffert, il ne peut pas être le vrai Dieu car le propre vrai Dieu est impassible. Or la créature est mortelle, corruptible : évolution. 
    \item Il va radicaliser le subordinatianisme au delà du fait que le Père engendre le Fils. il va donc remonter la frontière de la créature / Créateur au niveau du Logos et non entre Logos / hommes.
    \begin{Def}[subordinatianisme]
    Conception théologique liée à l'arianisme, qui suppose dans la Trinité une subordination ontologique du Fils et de l'Esprit par rapport au Père.
    \end{Def}
\end{itemize}
Il va donc considérer que \textit{créer} et \textit{engendrer} son synonyme. Tout ce qui est hors de Dieu est créature.
Il y a une hierarchie dans la créature : le Christ n'est pas juste un homme, il est \textit{médiateur} entre Dieu et les hommes. Sous l'influence du néo-platonisme, il définit Dieu comme inengendré sans commencement.

\begin{quote}
    {[}Nous disons{]} Que le Fils n'est ni inengendré\sn{\emph{Lettre d'Arius à Eusèbe de Nicomédie (318 ou 321-322)}
cité dans B. Sesboüé, \emph{Dieu peut-il avoir un Fils}, 32).}, ni une partie
d'inengendré, et qu'il ne provient non plus d'aucun substrat : c'est par
volonté et conseil qu'il a eu l'existence avant les temps et les siècles
; il est plein de grâce et de vérité, il est Dieu, Fils unique, immuable
; et avant d'avoir été engendré (voir Pr 8, 25), ou créé (voir Pr 8,
22), ou établi (voir Rm 1, 4), ou fondé (voir Pr 8, 23), il n'était pas
; car il n'était pas inengendré.

Or nous sommes persécutés pour avoir dit : « le Fils a un commencement,
mais Dieu est sans commencement. ». Voilà pourquoi nous sommes
persécutés, et aussi parce que nous avons dit : « Il est à partir du
néant. » 
\end{quote}

Le verbe n'a pas le même niveau d'Etre que le Père, il s'appuie sur la philosophie grecque.
Arius raisonne bien d'une certaine façon à partir des catégories grecques. Mais il remet le salut en cause : les affirmations qu'on a sur Dieu ont une conséquence sur notre salut. 






    \subsection{Le concile de Nicée (325) : une nouvelle «définition» du
    Christ, vrai Dieu}

Les débats suscités par Arius vont obliger Constantin à appeler un concile, de Nicée : \textit{nous confessons que}
C'est la première fois qu'un concept philosophique et non biblique va rentrer dans la profession de Foi. 
    
    
    
      \paragraph{Consubstantiel au Père}
      Engendré
       \textit{Engendré} ne suffit pas car Arius parle aussi d'engendrement. Du coup, \textit{non pas créé}. 
       On introduit un mot mais qui est ambigu : 
       \emph{Homo}-\emph{ousie} : de la même nature, semblable. Il faudra écarter deux choses : 
       \begin{itemize}
           \item Jésus n'est pas le Père
           \item mais il n'est pas créé
       \end{itemize}
       
       
      \paragraph{L'argument sotériologique}
      
      L'argument essentiel concerne le Salut : si le Christ n'est pas vraiment Dieu, il ne peut pas nous diviniser. Seul Dieu peut le faire :    D\textit{ieu s'est fait homme pour que nous soyons divinisés} 
      \begin{quote}
       
          
          « Si l'homme n'avait pas été uni à Dieu, il n'aurait pu recevoir en
participation l'incorruptibilité. Car il fallait que le `Médiateur de
Dieu et des hommes', par sa parenté avec chacune des deux parties, les
ramenât l'une à l'autre à l'amitié et à la concorde, en sorte que tout à
la fois Dieu accueillît l'homme et que l'homme s'offrît à Dieu. Comment
aurions-nous pu en effet avoir part à la filiation adoptive à l'égard de
Dieu, si nous n'avions pas reçu, par le Fils, la communion avec Dieu ?
Et comment aurions- nous reçu cette communion avec Dieu, son Verbe
n'était pas entré en communion avec nous en se faisant chair » \sn{IRENEE,
\emph{Contre les hérésies}, III 187}
      \end{quote}
      Irénée avait utilisé cet argument face à la gnose.
      
      Si on affirme cela, c'est dire a contrario l'opposition de nature entre l'homme et Dieu, entre le corruptible et l'incorruptible,...
      
      Dans la vision biblique, l'homme est créé à l'image de Dieu. Pas chez les grecs.
      
      
      
      \paragraph{Conclusions sur le plan christologique}  
      
      \begin{Synthesis}
           Il y a une victoire de la tradition biblique sur la Tradition grecque. Christologisation d'un concept grec et non l'inverse. C'est la Révélation en terme grec. 
      \end{Synthesis}
  
  
  Le problème, ce n'est pas uniquement l'unité avec Dieu, mais aussi de penser l'unité du Logos avec l'homme : \emph{la question de l'Incarnation}. Les grands débats théologiques partent de cela.

 ~
  \hypertarget{lincarnation-du-logos-divin}{%
  \section{L'incarnation du Logos
  divin}\label{lincarnation-du-logos-divin}}

  
  
  
    \subsection{L'origine biblique d'un concept formulé par les Pères}
  
  \begin{Def}[incarnation]
  Du grec, \emph{en-sarkosis},  chair, 
  \end{Def}
  \paragraph{première fois utilisé par Irénée. } mais s'appuie sur
  \begin{quote}
  Le verbe s'est fait chair (Jn 1, 14)    
  \end{quote}
   
   On le retrouve dans le concile de Constantinople : 
   \begin{quote}
       Il a pris chair (incarné) de la vierge Marie et s'est fait homme.
   \end{quote}
  
  On a un redoublement de "il a pris chair" et "il s'est fait homme". D'un côté, l'idée d'incarnation et d'humanisation. Un peu comme dans les psaumes.
  
  \subparagraph{"Il a campé parmi nous"} plutôt qu'il a habité parmi nous. Idée de la Tente, Dieu habite avec son Peuple, Ex 25, 8-9 \sn{\textit{Ils me feront un sanctuaire, et j'habiterai au milieu d'eux.
9 Vous ferez le tabernacle et tous ses ustensiles d'après le modèle que je vais te montrer.}
Autres références : Ap 7, 15}
Et on passe ensuite de la tente au Temple de Jérusalem. 

\subparagraph{L'hympne au Philippiens} dépouillement.  
\begin{quote}
    il s'est abaissé (Kenose)
\end{quote}
On a finit par comprendre ce passage comme une \textit{descente} du Fils.   



% -------------------------------------------
    \subsection{Les présupposés grecs à l'idée d'incarnation}

    \paragraph{Le premier présupposé : l'opposition entre le monde divin et le
      monde des hommes}
      Il s'est transformé en chair. Or pour des hommes grecs, cette affirmation est insupportable. Comme l'infini peut devenir fini.  Comment le concevoir ? Dans une pensée philosophique grecque, ce n'est pas pensable.
      Ils refusent de concevoir le passage de l'Esprit en chair et de la Chair en Esprit. 

Ainsi Celse \sn{cité par ORIGENE, \emph{Contre Celse}, V, 2 et IV , 14} : 
\begin{quote}
    
« Nul Dieu, écrivait celui-ci[Celse], nul Fils de Dieu n'est descendu ni ne
saurait descendre (\ldots) Dieu est bon, beau, bienheureux, au plus haut
degré de la beauté et de l'excellence. Dès lors, s'il descend vers les
hommes, il doit subir un changement : changement du bien au mal, de la
beauté à la laideur, de la félicité à l'infortune, de l'état le meilleur
au pire (\ldots) Il est vrai certes que pour un mortel la nature est de
se changer et de se transformer, mais pour un immortel, c'est d'être
identique et immuable. Dieu ne saurait donc non plus admettre un tel
changement. » 
\end{quote}

Il est bien difficile d'expliciter ce qu'est l'Incarnation pour Origène\sn{ORIGENE,
\emph{Des principes}, II, 6,2.} : 
\begin{quote}
    « La fragilité d'un entendement mortel ne voit pas comment elle pourrait
penser et comprendre que cette Puissance si grande de la majesté divine,
cette Parole du Père lui-même, cette Sagesse de Dieu dans laquelle ont
été créés tout le visible et tout l'invisible, ait pu, comme il faut le
croire, exister dans les étroites limites d'un homme qui s'est montré en
Judée, et aussi que la Sagesse de Dieu ait pénétré dans la matrice d'une
femme, qu'elle soit née comme un petit enfant, qu'elle ait émis des
vagissements à la manière des nourrissons qui pleurent ; et ensuite
qu'elle ait été troublée à l'heure de la mort, comme on le rapporte et
comme Jésus le reconnaît lui-même (\ldots) ; et enfin qu'elle ait été
conduite à la mort (\ldots). Je pense que cela dépasse même la capacité
des saints apôtres : bien mieux l'explication d'un tel mystère est peut
être au-dessus des puissances célestes de toute la création » 
\end{quote}

\begin{Synthesis}
     Pour Origène, c'est impossible à comprendre, il faut le croire. Mais comment en rendre compte et en témoigner
\end{Synthesis}




    \paragraph{Deuxième présupposé : l'idée de l'homme dans la culture grecque}

Ce n'est pas tout à fait l'anthropologie biblique. \emph{le verbe s'est fait chair} a été mal compris par les grecs. Au lieu de comprendre le \emph{SarKh}, \textit{bashar}, la chair, on va en faire le \textit{corps}. Et selon les gnostiques, c'est la prison de l'âme et de l'esprit. Il faut se libérer de la matière, de la \textit{viande}. 
Dans l'anthropologie grecque, il y avait un composé corps / âme : \textit{pneuma-psuche} et il y avait le \textit{noûs}, l'esprit, part de l'âme, qui dirigeait la psuche, qui façonnait le corps.\sn{cf 1 Th 5, 23 \textit{Que le Dieu de paix vous sanctifie lui-même tout entiers, et que tout votre être, l'esprit, l'âme et le corps, soit conservé irrépréhensible, lors de l'avènement de notre Seigneur Jésus-Christ!}}


    \paragraph{Troisième présupposé : la descente du Sauveur divin}

Dans un écrit gnostique, le thème de la \textsc{descente} : 
\begin{quote}
    \sn{Epître apocryphe de Jacques, 8,35 --
10,30}

« Voyez : Je suis descendu, j'ai parlé, j'ai été maltraité, j'ai porté
ma couronne, afin de vous sauver. Je suis descendu, en effet, pour
habiter avec vous, afin que, vous aussi vous demeuriez avec moi. Et
ayant trouvé vos maisons sans toit, j'ai demeuré dans les maisons qui
pourraient me recevoir au moment où je descendrais (\ldots). C'est pour
vous que je suis descendu. C'est vous les bien-aimés. C'est vous qui
allez devenir cause de la Vie en beaucoup ».

\end{quote}
\subparagraph{Descente dans le contexte de la Parousie}
Dans le NT, il y a deux termes pour dire descente : Katebainô et Katerkomai. La descente de l'Esprit sur Jesus (Mt 3, 16), la descente de l'ange du Seigneur, la descente du feu (Lc 9, 54), la descente de la Jérusalem Celeste (Ap), la descente du diable, la descente du Fils de l'homme ou du Christ (\textit{mais à la fin des Temps}).
Seule exception, Jn 6 : \textit{je suis le pain descendu du Ciel} mais c'est une image, celle de la manne. Il faut s'interroger sur les termes choisis même si descente n'est pas faux

\begin{Prop}
En quoi c'est important d'utiliser ou non le terme de descente. Chez les gnostiques, la descente d'un sauveur qui traverse les 7 cieux. 
Chez les néoplatoniciens, dégradation du 1 en multiple via la descente.
\end{Prop}
    
      
    
      
    
      
    
  
 ~
  \hypertarget{les-christologies-de-lincarnation}{%
  \section{Les christologies de
  l'incarnation}\label{les-christologies-de-lincarnation}}

Notion de l'Alliance définitive

\paragraph{Les fondements des christologies de l'incarnation}

\begin{Synthesis}
     La christologie est au service du Salut. Il faut "penser" le Christ pour que nous puissions penser notre salut.
\end{Synthesis}

\begin{Ex}
Annoncer le Christ en Inde via des \textit{Avatars}. Mais le problème, c'est qu'il peut y avoir plusieurs avatars
\end{Ex}

On va arriver à avoir des affirmations contradictoires qui obligent à un choix, entre deux modèles, Antiochien, et Alexandrin.

\paragraph{Le modèle antiochien} Antioche : centre important du Christianisme. On part de l'homme concret Jésus et on reconnait en lui le verbe de Dieu. il est \textit{l'épiphanie de Dieu}. On peut dire que cette église va préférer parler d'humanisation du Logos que d'incarnation. 

Derrière ce modèle, on a l'image de la descente de l'Esprit au Jourdain : le \textit{logos} a assumé un homme concret.

L'avantage de ce modèle c'est que cette unité ne se fait pas au détriment de l'humanité. Il n'y a pas d'amputation de l'humanité. On ne mélange pas ce qui est divin et humain, pour préserver l'intégrité de l'homme.
\subparagraph{Nestorius} refusait de considérer Marie comme \textit{theotokos} mais comme \textit{Christotokos}. On ne mélange pas.

\subparagraph{L'inconvénient} d'une telle approche, c'est de penser l'unité : est-elle indissoluble. Pas une unité ontologique mais \textit{une conjonction} des volontés. Il interprète à partir de Gn 2, 24 \sn{24 À cause de cela, l’homme quittera son père et sa mère, il s’attachera à sa femme, et tous deux ne feront plus qu’un.}.
Cette Alliance peut ne pas être définitive car il peut y avoir séparation. \textit{prosoppon} du Logos (personne). Le Logos se sert de l'humanité du Christ pour se rendre présent au monde.

Il y a une dualité qui est latente. Rien n'empêche finalement le Logos de se réincarner en un autre homme.


Enfin, on peut lire l'Evangile de façon dualiste : on attribue le miracle au Logos, et quand il a soif au Jésus. 
Nestorius refuse que le Verbe s'approprie : "il a soif", "il est né". Il y a toujours l'ombre de deux sujets. 

La question de l'Union entre deux natures \sn{Nestorius parle de Conjonction} est trop extrinsèque.



\paragraph{Le modèle alexandrin}
  
L'incarnation va être exprimée d'une autre façon pour exprimer l'unité de façon plus forte. L'école d'Alexandrie, depuis Philon,... est très importante.

Descente du logos dans le monde et son union avec la nature humaine et le résultat est \textit{Jésus de Nazareth}. Théologie \textit{descendante}, conception virginale.

\subparagraph{l'avantage} est de penser l'unité entre le logos divin et l'homme en Christ. Cette conviction vient de la Bible. Au nom du principe soteriologique de l'homme sauvé, on a une conception ontologique de Jésus.
On va donc insister sur \textit{une seule chair}.

\subparagraph{Inconvénients} sachant qu'aucun modèle théologique n'est parfait. L'union du verbe avec l'homme est tellement forte que l'homme jésus est il encore un homme ? On ne peut séparer l'âme du Corps. Dans cette unité, le \textit{logos} pourrait remplacer l'âme humaine. Et l'homme se réduisant au corps. Le sarx est traduit par \textit{corps} et non par \textit{homme}. 

\subparagraph{Apollinaire de Laodicée 390} s'appuie sur une anthropologie tri-partite, et voit dans le Christ un \textit{noûs} divin et lui ont conféré l'impeccabilité. Et a dirigé sa psuche et son corps. Pas de \textit{noûs} de l'homme Jésus. 
\begin{quote}
    Une seule nature incarnée du Verbe
\end{quote}
Cette affirmation sera ensuite remise en cause. Il ne peut y avoir d'unité dans le Christ que si un des éléments qui le compensent ne sont complètes. \textit{on ne peut pas mettre deux choses complètes ensemble}. Il faut donc amputer l'homme Jésus, pas d'âme rationnel, le \textit{noûs}.
Cyrille d'Alexandrie : 
\begin{quote}
    l'unique nature du verbe incarné (Veme siècle)
\end{quote}

C'est surtout le moins \textit{Euthychès} qui défendra le monophysisme, une seule nature, au détriment de la nature humaine.

  
  \begin{Synthesis}
       Gregoire de Nysse : 
       \begin{quote}
           ce qui n'a pas été assumé, n'est pas sauvé. Si l'âme spirituelle du Christ n'est pas humaine, il ne peut pas sauver la notre.
       \end{quote}
       Même si le monophysisme est nourri d'Ecriture, on est toujours dans l'impassibilité de Dieu comme fond culturel.
  \end{Synthesis}
  
  Hilaire de Poitiers : le Christ, souffrait mais sans douleur (pas d'alteration).
    
  
    
  
    
  
 ~
  \hypertarget{la-duxe9finition-chalcuxe9donienne-du-christ-une-personne-en-deux-natures}{%
  \section{La définition chalcédonienne du Christ : « une personne en
  deux natures
  »}\label{la-duxe9finition-chalcuxe9donienne-du-christ-une-personne-en-deux-natures}}

  \sn{cours du 15/3/22}
  
  Percevoir que la manière et la culture de l'époque ont influencé la façon dont on a exprimé la christologie. \textit{consubstantiel au Père} est situé culturellement. On témoigne donc de la culture de l'époque. 
  
  \paragraph{Au-delà de la confusion et de la division}  
  
  Une théologie qui ne sépare pas qui est le Christ et la manière dont on est sauvé. Quand on dit quelque chose sur le salut, on dit quelque chose sur le Christ (et inversement).
  Trois chemins se présentent : 
  \begin{itemize}
      \item \textit{Ecole d'Antioche} : une unité mais qui laisse demeurer une certaine distinction : Dieu est le moyen extérieur pour sauver l'homme, mais notre salut n'est pas pleinement assurée car l'union avec Dieu n'est pas complète. \textit{Nestorius}. dérive : Face au risque d'avoir deux sujets (unité Dieu et Jésus pas assurée). 
      \item \textit{Ecole d'Alexandrie} : Transformation en Christ. Le Verbe le transforme et le divinise. La divinité absorbe l'humanité. L'homme disparait dans la divinité. On a une unité beaucoup plus forte entre le Christ et Dieu mais on a un problème de salut, si tout l'homme n'est pas assumé par le Christ, alors est ce que l'homme est sauvé. dérive : \textit{Est ce que Jésus est encore un homme ?}
      \item \textit{Chalcédoine } : Chemin de compromis entre Alexandrie et Antioche. A la fois une transformation en Christ mais qui accomplit l'homme et ne dissout pas l'homme. Cette situation n'est pas grec. Car pour les Grecs, il y a quelque chose qui s'oppose entre la corruption de l'homme et l'incorruptibilité du spirituel. Comment on pense notre humanité d'un passage de la corruption à l'incorruption. \textit{Transformation qui respecte l'homme}
      
  \end{itemize}
    
  \paragraph{Le rééquilibrage de Chalcédoine}
   
   En 451, un concile "chemin de crête"\sn{Ephèse et Chalcédoine furent les deux grands conciles christologiques}, compromis entre Cyrille d'Alexandrie et Leon le Grand. Il rejete le monophysisme (deux natures en Christ), "sans mélange ni confusion" (on respecte l'humanité). On rejete aussi le Nestorianisme en affirmant l'unité entre la nature divine et humaine en Christ.
   A noter qu'on n'affirme pas mais on utilise une double négation. On ne rationalise pas ce que c'est, on dit ce que cela n'est pas. 
   
   \paragraph{Un débat entre le Pape Léon le Grand et le patriarche d'Alexandrie} Le Christ est il d'une ou de deux natures ? Union à partir de deux natures ou \textit{en deux natures, l'humanité et la divinité}. On va choisir : 
   \begin{quote}
       le Christ est en deux natures (et non à partir de deux natures)
   \end{quote}
   
   \paragraph{et l'unité ?} on va le faire avec un autre concept, \textit{l'hypostase}, en latin, \textit{une personne}. On parle donc de communication des \href{https://fr.wikipedia.org/wiki/Communicatio_idiomatum}{idiomes} , les propriétés propres à telle nature. 
   \begin{quote}
       Quand on dit "en Christ, Dieu est mort", on parle
   \end{quote}
    
    
  
 






\begin{table}[h!]
    \centering
    \footnotesize
        \sidecaption{La formule christologique du concile de Chalcédoine
(451) \emph{Source :} SESBOÜÉ, B. -- WOLINSKI, J., \emph{Le Dieu du salut.
Histoire de dogmes} I, Paris 1994, 409\emph{.}}
 

 % Please add the following required packages to your document preamble:
% \usepackage[normalem]{ulem}
% \useunder{\uline}{\ul}{}
 
\begin{tabular}{p{.3\textwidth}p{.3\textwidth}p{.3\textwidth}}
\multicolumn{3}{p{\textwidth}}{Suivant donc les saints Pères, nous enseignons tous unanimement que nous confessons}                                                                                                                                  \\
\multicolumn{3}{c}{Un seul et même Fils, Notre Seigneur Jésus-Christ Le même}                                                                                                                                                            \\
\begin{tabular}[c]{@{}c@{}}
B.\\    \\ Parfait en divinité\end{tabular} & le même                                                                                  & parfait en humanité                                                  \\
                                                                       & le même                                                                                  &                                                                      \\
Vraiment Dieu                                                          &                                                                                          & et vraiment homme   d’une âme raisonnable et d’un corps              \\
consubstantiel au Père                                                 & et le même                                                                               & consubstantiel à nous                                                \\
selon la divinité                                                      &                                                                                          & selon l’humanité                                                     \\
                                                                       &                                                                                          & en tout semblable à nous sauf le péché,                              \\
avant les siècles   engendré                                           &                                                                                          & et au derniers jours (engendré)                                      \\
du Père selon la divinité                                              & le même                                                                                  & pour nous et pour notre salut de la Vierge Marie, Mère de Dieu selon \\
                                                                       &                                                                                          & L’humanité                                                           \\
\multicolumn{1}{l}{}                                                   & \multicolumn{1}{p{.3\textwidth}}{Un seul et même Christ, Fils, Seigneur, l’unique engendré}            & \multicolumn{1}{l}{}                                                 \\
Reconnu en deux natures,                                               &                                                                                          &                                                                      \\
Sans confusion, sans changement                                        &                                                                                          & sans division, sans séparation,                                      \\
                                                                       & La différence des natures n’étant nullement supprimée                                    &                                                                      \\
                                                                       & A cause de l’union                                                                       &                                                                      \\
                                                                       \\
                                                                       & La propriété de l’une et l’autre nature étant bien plutôt sauvegardée                    &                                                                      \\
                                                                       & Et concourant à une seule personne Et une seule hypostase                                &                                                                      \\
                                                                       & Un Christ ne se fractionnant ni se divisant en deux personnes Mais un seul et même Fils, &                                                                      \\
                                                                       & Unique engendré, Dieu Verbe, Seigneur, Jésus-Christ,                                     &                                                                      \\
\multicolumn{3}{p{\textwidth}}{Selon que depuis longtemps les prophètes l’ont enseigné de lui, que Jésus-Christ lui-même nous l’a enseigné et que le Symbole des Pères nous l’a transmis.}                                                          
\end{tabular}
\end{table}

  
  \paragraph{La formule de Chalcédoine : une définition ou un chemin
    d'intégration ?
  }
 
  Le problème, c'est le terme de nature, qui a fait que plusieurs Eglises se sont éloignées (pre-chalcédoniennes). Dans les années 1970, plusieurs accords entre ces Eglises en contournant le mot nature (\href{https://www.cairn.info/revue-etudes-theologiques-et-religieuses-2006-1-page-53.htm}{Rencontre de Vienne de 1971} ).
  
  
  \paragraph{Grâce à la pensée Grecque, sortir de la confusion} 
  \paragraph{limites de la formule de Chalcédoine} La dimension humaine du Christ, son corps. On a réduit un peu le Christ à une personne concrete mais le lien avec la reflexion du \textit{nouvel Adam}, Christ mort et ressuscité. On ne voit pas le lien entre la nature humaine et la nature divine.  Les grecs n'arrivent pas à penser l'homme à l'image de Dieu. D'où quelque chose de \textit{statique} dans la formule. Elle est bonne mais ne dit pas tout. 





