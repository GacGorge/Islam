\chapter{La modernité humaniste détermine la christologie  Une christologie « anthropologique »}

\section{Eléments bibliographiques} BRAGUE, R., Le règne de l’homme. Genèse et échec du projet moderne, Gallimard, Paris, 2015. CLAVERIE, P.,  Lettres et messages d’Algérie, Paris 1996. GILLEPSIE, M. A., « Humanism and the Apotheosis of Man » dans ID., The Theological Origins of Modernity, The University of Chicago Press, Chicago – Londres, 2008, 69-100. LUBAC (de), H., Le drame de l’humanisme athée, Paris 1944. MAURICE, E., La christologie de Karl Rahner, Paris 1995. RAHNER, K., « Problèmes actuels de christologie » dans Ecrits théologiques I, Paris 1959, 113-181. RAHNER, K., « Réflexions théologiques sur l’incarnation » dans Ecrits théologiques III, Paris 1963,81-101. RAHNER, K., Traité fondamental de la foi, Paris 1983. SESBOÜÉ, B., Hors de l’Église pas de salut. Histoire d’une formule et problèmes d’interprétation, Paris 2004. 




\section{Introduction }





\paragraph{Au centre l'Homme : tournant anthropologique} Face à un moyen-âge qui est théocentrique, la théologie est la règne des sciences, avec l'anthropocentrie, on va passer aux sciences humaines. 

\paragraph{Une Christologie Anthropologique}

\section{L’humanisme comme le mouvement culturel de la modernité } 

\subsection{L’humanisme italien}   

\paragraph{François Pétrarque 1304 - 1374} Les Papes sont à Avignon, Grande Peste. Début de la Renaissance Italienne. 1336 : il monte le mont Ventoux près d'Avignon et fait une expérience spirituelle. Il médite les \textit{confessions} de Saint Augustin (Chapitre 10) et il se rend compte du monde intérieur.

\paragraph{le projet de Pétrarque} Vie humaine pleine de sens commence par son intériorité, voyage intérieure. Approche individualiste. 
\paragraph{Des figures } En particulier l'Antiquité. Non seulement des saints mais aussi Cicéron, Socrate,... en reintégrant les figures greco-romaines.
Cette affirmation était extraordinaire.

\paragraph{Approche néo-platonicienne du Christianisme} On s'éloigne de l'image d'Adam pécheur pour se rapprocher de l'Homme à l'image de Dieu (Gn 1).  Cela a longtemps été vu comme anti-Chrétien. Cf Etienne Gilson (XX):
\begin{quote}
    Renaissance : pas le Moyen Age plus l'homme, mais le Moyen âge sans Dieu et donc sans l'homme.
\end{quote}

\paragraph{Lien entre humanisme et Christianisme} Humanisme : on voulait mettre ensemble l'Antiquité Paienne, avec la vie monastique, la charité Chrétienne. Synthétiser les deux mondes. Pétrarque admirait les deux mondes, au service de l'Homme et de son individualité. Ils n'étaient pas anti-religieux ou anti-Chrétien. 

\begin{Synthesis}
L'homme n'est pas irrémédiablement déchu mais à l'image de Dieu : un chemin pour l'individualité.
\end{Synthesis}
Mais en devenant plus optimiste sur l'homme, on devient pélagianiste, en oubliant la grâce qui n'est plus centrale. Pélage : le Christ est un exemple et je vais l'imiter. La grâce, c'est de pouvoir imiter le Christ.

\paragraph{Salutati} un autre italien qui va insister sur la volonté. Pour défendre la dignité humaine, on pense que les grands héros de l'antiquité peuvent être sauvés. Alors que Dante mettait Socrate en enfer. Si Socrate était damné, alors injustice de Dieu. Mais si Socrate est sauvé, comment penser le sacrifice du Christ ?

\paragraph{Lorenzo Valla +1447} Un être voulant plus que rationnel. On peut de la contemplation à l'action. Les hommes veulent non seulement contempler le monde mais le transformer. Donner \textit{une forme au monde}. Tout \textit{logos} est une forme de \textit{poesis}, une création. L'homme participe à la Création du monde.



\subsection{L’humanisme d’Erasme } 

\subsection{L’humanisme et la question du salut } 

\section{L’humanisme chrétien et le Christ }
2. 2.1 Le drame de l’humanisme athée 2.2 L’humanisme proposé par Paul VI 2.3 La christologie de Gaudium et spes 



\section{}3. La christologie de K. Rahner en réponse à une société sécularisée 3.1 La remise en cause du mythe de l’incarnation 3.2 Réinterpréter : « le Verbe s’est fait chair » a) Considérations sur la méthode b) L’homme est un mystère c) Le processus d’humanisation  d) Le Christ est l’homme parfait 4. La christologie « anthropologique » et la notion de christianisme anonyme Conclusion  