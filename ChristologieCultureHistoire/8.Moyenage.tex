C\chapter{La chrétienté médiévale et son idée implicite du Christ : Le Christ roi} 

\mn{Christologies et cultures dans l’histoire 8}

\section{Eléments bibliographiques }


BOESFLUG, F., Dieu et ses  images. Une histoire de l’Eternel dans l’art, Paris 2008. 

CLAVERIE, P., Lettres et messages d’Algérie, Paris 1996, 99-100. 

CERFAUX, L., La théologie de l’Église suivant saint Paul, Paris 1942. 

CONGAR, Y., L’Église de saint Augustin à l’époque moderne, Paris 1970. 

DESOUCHE, M-T., « Pie XI, le Christ Roi et les totalitarismes » dans NRT 130 (2008) 741759. 

DESOUCHE, M-T., Le Christ dans l’histoire selon le pape Pie XI, Paris 2008. LAFONT, G., Histoire théologique de l'Église catholique : itinéraire et formes de la théologie, Paris 1994. 

LUBAC (de), H., Corpus Mysticum. L’eucharistie et l’Église au Moyen Age, Paris 1944.  

MOULE, C. F. D., « The corporate Christ » dans The Origin of Christology, Cambridge 1977, 47-96. 

PACAUT, M., La théocratie. L’Église et le pouvoir au Moyen Age, Paris 19892. 

PIE XI, Quas primas, Lettre encyclique du 11 décembre 1925. PIE XII, Mystici Corporis Christi, Lettre encyclique du 29 juin 1943. 

RAHNER, K., « L’appartenance à l’Église, d’après la doctrine de l’encyclique de Pie XII Mystici Corporis Christi » dans Ecrits théologiques II, Paris 1960, 9-112. 

SESBOÜÉ, B., Hors de l’Église pas de salut. Histoire d’une formule et problèmes d’interprétation, Paris 2004. 

SLOTERDIJK, P., « Le découronnement de l’Europe. Anecdote sur la tiare » dans Globes. Sphères II, Paris 2010, 701-711. 

VAUCHEZ, A., « Les chrétiens face aux non-chrétiens » dans Histoire du christianisme 5. Apogée de la papauté et expansion de la chrétienté (1054-1274), Paris 1993, 701-733. 

VAUCHEZ, A., « Le tournant pastoral de l’Église en Occident » dans Histoire du christianisme 5. Apogée de la papauté et expansion de la chrétienté (1054-1274), Paris 1993, 737766.



\section{Le passage à une nouvelle culture : la chrétienté }

Dialogue fécond avec la culture juive, grecque et romaine et confrontation. Mais en assumant la \textit{religio}, la donne change

\subsection{Une nouvelle culture modelée sur l’image du corps du Christ}

Remplacement de la religion romaine par la religion chrétienne. A la fin de l'empire, une nouvelle culture émerge. Le Corps du Christ est appliqué à l'ensemble de la société, et non simplement \textit{l'Eglise}. 

\paragraph{la société se considère comme corps du Christ} et donc, reprenant l'image paulinienne, le Christ devient la tête, le Roi. 


\subsection{Un « corps du Christ » défiguré ? }
Est ce encore le corps du Christ ? 
En identifiant l'Eglise, corps du Christ physique avec le corps du Christ sacramentel (corps mystique) avant le XIIe siècle).

Au XIIe siècle, on renverse les choses : le corps réel devient l'eucharistie et le corps mystique devient l'Eglise. \mn{Cf de Lubac, \textit{corpus mysticum}}

Distendre le lien entre le Christ physique et sacramentel 
Le corps du Christ converge avec le corps politique, on veut unifier le corps du Christ avec les princes, en Occident. Théocratie : les princes sous la responsabilité de l'Eglise. 

On veut intensifier Christocratie interne et externe. Vivre cette culture du Christ et réellement.

Le Christ sera représenté comme Roi, avec le modèle des rois de ce monde. 

\subsection{Plan du chapitre }



\section{La convergence de l’Église, corps du Christ, avec une unité politique }


\paragraph{qu'est ce qu'on appelle théocratie} 
\begin{Def}[Théocratie]
mode de gouvernement où l'autorité souveraine est chargée d'appliquer la loi édictée par Dieu 
 [Dans un État théocratique pur, la loi civile et la loi religieuse se confondent.]
\end{Def}
Les princes sont d'accord mais ils gèrent le temporel.



\subsection{L’élaboration d’une théocratie chrétienne (Ve s. au XIIIe s.) }

\paragraph{ Gélase et la doctrine dualiste (Ve s-VIe s.)} Elle s'appuie sur le dualisme corps-âme. Comme l'âme gouverne le corps, le rôle du pape est premier.

Roi / Pape

Corps / âme

Les Rois ont une autonomie mais ne peuvent pas faire n'importe quoi. Un certain équilibre. Coopération entre les deux.



\paragraph{La réforme grégorienne} XI. Grégoire VII a eu un rôle important (1073-85). La situation évolue avec une reprise en main du pouvoir spirituel. Trop de laics dans les affaires de l'Eglise. Stratégie pour réformer l'Eglise. 
\begin{itemize}
    \item Interdiction des laics dans les affaires religieuses. Aujourd'hui, on est en train de réinterroger la réforme grégorienne. \textit{Querelle des investitures} avec le dépôt de l'empereur Henri. Concretement le pape est au dessus.  
    \item réforme du clergé
    \item pas de frontière entre peuple et peuple de Dieu. 
\end{itemize}

Une entité territoriale, l'occident, qui fait unité. Le droit naturelle est incorporée dans le droit surnaturel, le droit civique dans le droit canonique.


\subsection{Une nouvelle image du Christ : le pape}

Boniface VIII, début XIV : sommet de cette vision. Il rappelle que seule l'Eglise est chargée de mener le peuple au bien, ... pouvoir illimité tant spirituel que temporel. Symboliquement très fort (dans la pratique pas de changement).

\paragraph{Le pape, tenant-lieu du Christ-tête} Cela marque les esprits. L'image du Christ Roi, c'est le pape. Tenant lieu ou lieutenant du Pape. Il y avait des racines anciennes, comme par exemple Grégoire VII : 
\begin{quote}
    le pape est la tête visible du corps du Christ. Avec le pouvoir des \textit{clés}, il pouvait intégrer ou excommunier. 
\end{quote}

    Il était presque le sacrement visible du Christ; il joue le rôle de l'Eglise. La christologie s'est fait ecclésiologie et l'ecclésiologie s'est faite théocratie.


\paragraph{Le pape : de l’autorité sur les chrétiens à l’autorité sur tous les hommes}


\begin{quote}
Et l'on ne peut soutenir, pour nier cette vérité, que par un primat de juridiction établi dans l'Eglise, ce Corps mystique serait pourvu d'une double tête. Car Pierre, par la vertu du primat, n'est que le Vicaire du Christ, et il n'y a par conséquent qu'une seule Tête principale de ce Corps, à savoir le Christ; c'est lui qui sans cesser de gouverner mystérieusement l'Eglise par lui-même, la dirige pourtant visiblement par celui qui tient sa place sur terre, car depuis sa glorieuse Ascension dans le ciel, elle ne repose plus seulement sur lui, mais aussi sur Pierre comme sur un fondement visible pour tous. Que le Christ et son Vicaire ne forment ensemble qu'une seule Tête, Notre immortel Prédécesseur, Boniface VIII, l'a officiellement enseigné dans sa Lettre apostolique Unam sanctam (61) et ses successeurs n'ont jamais cessé de le répéter après lui, Ceux-là se trompent donc dangereusement qui croient pouvoir s'attacher au Christ Tête de l'Eglise sans adhérer fidèlement à son Vicaire sur la terre. Car en supprimant ce Chef visible et en brisant les liens lumineux de l'unité, ils obscurcissent et déforment le Corps mystique du Rédempteur au point qu'il ne puisse plus être reconnu ni trouvé par les hommes en quête du port du salut éternel. \textit{
    Pie XII, Mystici Corporis
}\end{quote}
 
 
Le Christ étant le Roi de toute la Création, il y a une extension du rôle du pape au delà des chrétiens. De droit mais pas de fait, en particulier sur les juifs qui s'écarteraient de la loi naturelle ou de la Loi de Moise. (pas une vision conciliaire mais de canonistes).
\mn{
Anecdotiquement, En France, le curé est le curé de tous les habitants de sa paroisse. 
}
\paragraph{le but, c'est l'unité} On représente le Père, avec une tiare. indirectement, cela représente que le Pape a tout pouvoir sur la terre.

\paragraph{Resistance des princes} qui fait que dans la pratique ne sera pas mis en pratique.

\section{L’intensification de l’unité et de la cohésion du corps social du Christ} 

Comment le pouvoir des Évêques a été de renforcer le pouvoir et la cohérence de l'Église.

\paragraph{Hérésie cathares et Réforme} on va insister sur le fait qu'il faut être incorporé à l'Eglise pour être sauvé. 

\subsection{Le salut se reçoit dans l’Église, le corps du Christ }

\paragraph{Question du baptême au Jourdain} Saint Augustin a interprété ce baptême comme une \textit{préfiguration} de l'Eglise. 
\begin{quote}
    « Et sans doute cette onction du Christ par l’Esprit Saint ne date pas du jour de son Baptême,
où l’Esprit descendit sur lui sous forme de colombe (Mt 3, 16) : ce jour-là il a voulu
préfigurer son corps, c’est-à-dire son Eglise, en laquelle on reçoit le Saint-Esprit,
spécialement au moment du baptême » (Augustin, De Trinitate, XV, 46).
\end{quote}


Thomas d'Aquin : ce qu'a mérité le Christ, il en fait bénéficier tout le corps : 
\begin{quote}
    « Le Christ ne possédait pas seulement la grâce à titre individuel, mais aussi comme tête de
toute l’Église, à qui tous sont unis comme les membres à leur tête, pour constituer avec lui
\textit{une seule personne mystique}. Aussi le mérite du Christ s’étend-il aux autres hommes en tant
qu’ils sont ses membres ; ainsi, dans un individu, l’action de la tête appartient de quelque
manière à tous ses membres, car ce n’est pas seulement pour elle que ses sens agissent, mais
pour tous ses membres » (Thomas d’Aquin, ST 3a q. 19 a.4).
\end{quote}

Il reprend l'idée d'Augustin, d'être attachée au Christ et par ce Corps, on est attaché à sa grâce.

\begin{Synthesis}
on va insiter sur le fait que si on n'est pas en lien avec la tête (\textit{gratia capitis}), on ne reçoit pas la grâce.
\end{Synthesis}

\paragraph{humanité nouvelle}
\begin{quote}
« Adam ayant été constitué par Dieu principe de toute la nature humaine, son péché se
transmet aux autres par la propagation de la vie charnelle. Et pareillement, le Christ ayant été
constitué par Dieu tête de tous les hommes à l’égard de la grâce, son mérite s’étend à tous ses
membres. Les autres reçoivent de la plénitude du Christ (..) une grâce individuelle. De même
que le péché d’Adam ne se transmet aux autres hommes que par voie de génération charnelle,
de même le mérite du Christ ne leur est communiqué que par une régénération spirituelle qui
se réalise dans le baptême et par laquelle ils sont incorporés au Christ, selon l’épître aux
Galates (2,27) : ‘Vous tous, qui avez été baptisés dans le Christ, vous avez revêtu le Christ’.
Qu’il soit à l’homme d’être régénéré dans le Christ, cela même est l’oeuvre de la grâce. Et
c’est ainsi que le salut de l’homme vient de la grâce » (Thomas d’Aquin, ST 3a q. 19 a.4
solutions)
\end{quote}

Limite d'un salut dans le Corps social chez Thomas alors que chez Paul, Adam est le principe du péché \textit{pour tous les hommes}\sn{péché originel} et Jésus le \textit{salut pour tous les hommes}. Quand on parle du \textit{Corps du Christ visible}, on n'honore pas complètement cette idée de salut universel.

\paragraph{Saint Thomas pense dans une chrétienté} Comment repenser cela quand l'Eglise est minoritaire ?


\subsection{Le renforcement du corps du social du Christ du XIIe s. au XIXe s}

Le lien avec le Christ tete se manifeste par : 
\begin{itemize}
    \item les sacrements
    \item les prêtres et Eveques
\end{itemize}

Les réformes visent à s'assurer qu'il y ait un maximum de prêtres pour transférer la grâce à un maximum de personnes.

\paragraph{La réforme des XIIe et XIIIe siècles : un organisme de grâce}
Les hérésies touchent toutes les couches de la populations, montrant une évangélisation superficielle. Les conciles de Latran III (1179) et Latran IV (1215) visent à mettre la pratique des chrétiens plus conforme.

\paragraph{Qu'est ce qu'un bon chrétien} plus seulement un baptisé, qui va à la messe et paye la dime. EN 1215, il doit se confesser et communier une fois par an. On développe la contrition. On essaye d'avoir une foi plus profonde. 
On essaye que la foi soit mieux vécue mais on confond parfois la relation à Dieu à une pratique.
\begin{Ex}
Confession : pureté de conscience, code ecclesiastique. 

\end{Ex}

Une vision qui a survecu longtemps : 
\begin{quote}
    « Comme le corps humain se trouve muni de moyens propres pour pourvoir à sa vie, à sa
santé, au développement de chacun de ses membres, de même le Sauveur du genre humain
(…) a pourvu son Corps mystique de moyens merveilleux en l’enrichissant de sacrements qui
doivent soutenir les membres, comme par des degrés de grâce ininterrompu, depuis le berceau
jusqu’au dernier soupir, et subvenir de même abondamment aux nécessités sociales de tout le
Corps » (Pie XII, Mystici Corporis, 768).
\end{quote}

\paragraph{Le concile de Trente} 

Le corps du Christ devient une société parfaite (vs société nation), pouvoir du pape sur l'Eglise elle-même.

\begin{quote}
    « L’ecclésiologie qui traduit le système est celle d’une société organisée comme un Etat,
ayant, au sommet de la pyramide le pape assisté par les Congrégations romaines, faites de
cardinaux et de bureaux. L’idée que la monarchie est la meilleure forme de gouvernement se
retrouve chez presque tous les auteurs, en preuve des prérogatives du pape (…). Au long d’un
effort qui ne triomphera que vers le milieu du XIXe siècle, la Rome papale tend à être
effectivement la norme de toute la vie ecclésiale : des liturgies, qui doivent s’aligner sur le
calendrier et le rite romain (…) ; du droit ; de l’histoire ; de la théologie elle-même (…).
L’Église est vue et définie, non comme un organisme animé par le Saint-Esprit mais comme
une société ou plutôt même une organisation où le Christ intervient à l’origine, comme
fondateur, et le Saint-Esprit comme garantie de l’autorité (…) ‘L’Église’ désigne alors le
gouvernement de l’Église, surtout son gouvernement ou son magistère romain. Cette est une
‘\textit{societas perfecta}’ : elle a en soi tout ce qu’il faut pour procurer la fin pour laquelle elle est
faite, elle possède elle-même, sans avoir besoin d’être complétée par une intervention de
l’Etat, le pouvoir législatif, judiciaire et coercitif » (Y. Congar, L’Église de saint Augustin,
382-384).
\end{quote}


le pape peut dire : "Eglise c'est moi".

Autoréférentiel.
Certes au XIX, XX, on réintroduira une dimension spirituelle au début du XX.


\begin{Synthesis}
Une christologie où on essaye que les chrétiens intensifient leur foi dans tous les domaines. Le Concile de Trente, après la réforme chrétienne ont façonné de façon durable ce qu'on a appelé la chrétienté. 
\end{Synthesis}

\section{Une christologie implicite à la chrétienté } 

\mn{le 5/4/22}



\subsection{ La chrétienté cultive en elle une image du Christ}
\paragraph{
Le Corps du Christ devient visible } par la pratique de la vie chrétienne. Mais on transfère sur la vie morale ce qui est de la vie spirituelle. 

\paragraph{Image implicite du Christ} le Christ vivant et reignant, Seigneur. Avant le XX, une théologie implicite.




\subsection{Le Christ pantocrator}

\paragraph{Panta-Krator} celui qui tient tout, des représentations du pouvoirs. Elle apparaît dans Ap 15 : 
\begin{quote}
     Tes oeuvres sont grandes et admirables, Seigneur Dieu tout puissant! Tes voies sont justes et véritables, roi des nations!
     Ap 15, 3
\end{quote}

\paragraph{Transposition de l'autorité du Père au Fils} Le Christ devient Pantocrator. 

\paragraph{Cesaro Papisme }Empereur Justinien (527-565). Style hierartique qui va s'éloigner de l'humanité et qui veulent souligner la divinité.  On va le mettre dans une mandorle. On va le mettre loin des fidèles pour bien montrer qu'on est dans un autre espace.

\paragraph{Plusieurs causes à cette évolution} Une lutte contre l'arianisme.  Au V, l'Eglise a battu l'arianisme\sn{Christ première créature par qui tout a été créé}.






\subsection{La Majestas Domini}
 

\paragraph{En occident} Saint Martin de Tours, avec les 4 vivants, vision de l'Apocalypse. A l'époque Carolingienne, se développe avec l'image de David (onction de Reims pour les rois de France).

\section{La « christologie royale » de Pie XI }  


\paragraph{1925} Pie XI va développer une christologie du Christ Roi au moment où l'on fête le Christ Roi. Pour justifier cette solennité, va présenter le Christ Roi.

\subsection{Le contexte } 

\paragraph{Effondrement de la Chrétienté} les Etats deviennent démocratiques, c'est à ce moment là qu'on parle du Christ roi de la terre. \textit{on veut sauver les meubles ?}

\paragraph{Le pape n'est plus chef d'Etat} Depuis le 20 septembre 1860, il n'y a plus de terre du pape. il faudra attendre 1929 pour les accords de Latran qui restaurent le Vatican.

\paragraph{Restaurer la souveraineté du Christ} Retour de l'humanité à son sauveur. Comme s'il y avait une civilisation chrétienne, et que le Christ régnait à cette époque. Nostalgie de la Chrétienté. Assez dur face à l'apostasie de ce temps.

\paragraph{l'apostasie à l'origine de la première guerre mondiale} Soit on refuse le Christ souverain, et les catastrophes arrivent.

\paragraph{la fête du Christ Roi comme rappel à la Chrétienté} néanmoins quelque chose d'idéalisé dans la Chrétienté.

\begin{itemize}
    \item le Christ est souverain
\item le Christ exerce cette souveraineté.
\end{itemize}

Non seulement dans un sens métaphorique.

\subsection{Les sources scripturaires de la christologie royale }

\paragraph{la figure de David - Espérance messianique} Ps 2,7 :
\begin{quote}
    je te donne les Nations
\end{quote}
Ps 72
\begin{quote}
    la justice à ce fils de Roi 
\end{quote}
Jr 23,5

Tradition dans l'AT qui est lié à l'espérance messianique.

\paragraph{Une autre source : le tradition apocalytique} La tradition apocalytique souligne le terme de l'histoire : Dn 7, fils de l'homme.

\paragraph{une fusion des deux visions dans le NT} Ces versets vont être accomplis par Jésus. Utiliser ces figures pour la royauté, c'est risquer de perdre la dimension eschatologique.

Mt 25, jugement dernier, le fils de l'homme qui règne.

Mt 28, 18 Tout pouvoir m'a été donné

Jn 18 Ma royauté n'est pas de ce monde. 

Ap 1, 5 le prince des rois de la Terre

Ap 19, 16 Roi des Rois et Seigneur des Seigneurs

Dans Saint Paul, 1 Co 15, 25, ... tous ces ennemis sous ses pieds.




\subsection{Le fondement théologique de la christologie royale }

Après avoir indiqué ces références, le pape va élaborer sa théologie, en passant directement à l'union hypostatique (Calcédoine). Comme si les choses allaient de soi. 

\paragraph{Union hypostatique} 
\begin{Def}[Union hypostatique]
Le Christ est vraiment homme et Dieu selon son unique personne (Calcedoine).
\end{Def}

\begin{quote}
    « Son pouvoir royal repose sur cette admirable union qu’on nomme union hypostatique. Il en résulte que les anges et les hommes ne doivent pas seulement adorer le Christ comme Dieu, mais aussi obéir et être soumis à l'autorité qu'il possède \textit{comme homme}; car, au seul titre de l'union hypostatique, le Christ a pouvoir sur toutes les créatures. » (Pie XI, Quas primas,§ 8). 
\end{quote}

Le raisonnement passe par la communication des idiomes.
\begin{Def}[Communication des idiomes]
Ce qu'on l'attribue à une nature du Christ, on peut l'attribuer à l'autre. Ex : Mère du homme donne Mère de Dieu.
\end{Def}

Et du coup, par son humanité, il a le pouvoir sur l'humanité.


\paragraph{L’action rédemptrice du Christ} Approche juridique car Pie XI est juriste : 

\begin{quote}
    « Le Christ, en outre, règne sur nous non seulement par droit de nature, mais encore par droit acquis, puisqu'il nous a rachetés? Ah! puissent tous les hommes qui l'oublient se souvenir du prix que nous avons coûté à notre Sauveur : Vous n'avez pas été rachetés avec de l'or ou de l'argent corruptibles, mais par le sang précieux du Christ, le sang d'un agneau sans tache et sans défaut (23). Le Christ nous a achetés à grand prix (24) ; nous ne nous appartenons plus. Nos corps eux-mêmes sont des membres du Christ (25). » (Pie XI, § 9). 
\end{quote}

Dans la tradition biblique, le Goël, qui rachète le bien (1 Co 6, 21). Donc il a un droit acquis par la rédemption : il nous a rachetés.




\subsection{La dimension « totalitaire » de la royauté du Christ }

Le règne spirituel ne nous dérange pas trop mais ce qui nous intéresse, c'est le pouvoir temporel du Christ : 

\begin{quote}
    Pie XI, § 12 : « D'autre part, ce serait une erreur grossière de refuser au Christ-Homme la souveraineté sur les choses temporelles, quelles qu'elles soient: il tient du Père sur les créatures un droit absolu, lui permettant de disposer à son gré de toutes ces créatures. ».  
\end{quote}

Pourtant ce droit absolu va être convertir et montrer pourquoi le pape doit laisser les hommes gérer les affaires humaines.
\begin{quote}
    « Tant qu'il vécut sur terre, [Le Christ] s'est totalement abstenu d'exercer cette domination terrestre, il a dédaigné la possession et l'administration des choses humaines, abandonnant ce soin à leurs possesseurs. Ce qu'il a fait alors, il le continue aujourd'hui » (Pie XI, QP) 
\end{quote}

\begin{Synthesis}
Chez Pie XI, la dimension eschatologique est peu présente. C'est un retour car la visibilité du Christ dans la société était plus forte dans la Chrétienté pour lui. 
"déjà là", "pas encore". il rajoute "était plus présent".
\end{Synthesis}

\subsection{Les ambiguïtés de la christologie royale} 

\paragraph{Un roi sévère et surplombant} dans les expressions (cf Paragraphe 10), un vrai pouvoir.

\begin{quote}
    10. II est presque inutile de rappeler qu'elle comporte les trois pouvoirs, sans lesquels on saurait à peine concevoir l'autorité royale. \ldots C'est, d'ailleurs, un dogme de foi catholique que le Christ Jésus a été donné aux hommes à la fois comme Rédempteur, de qui ils doivent attendre leur salut, et comme Législateur, à qui ils sont tenus d'obéir (26). \ldots

A tous ceux qui observent ses préceptes, le divin Maître déclare, en diverses occasions et de diverses manières, qu'ils prouveront ainsi leur amour envers lui et qu'ils demeureront en son amour (27).

Quant au pouvoir judiciaire, Jésus en personne affirme l'avoir reçu du Père, dans une réponse aux Juifs qui l'accusaient d'avoir violé le Sabbat en guérissant miraculeusement un malade durant ce jour de repos: " Le Père, leur dit-il, ne juge personne, mais il a donné au Fils tout jugement (28). Dans ce pouvoir judiciaire est également compris - car il en est inséparable - le droit de récompenser ou de châtier les hommes, même durant leur vie.

Il faut encore attribuer au Christ le pouvoir exécutif : car tous inéluctablement doivent être soumis à son empire; personne ne pourra éviter, s'il est rebelle, la condamnation et les supplices que Jésus a annoncés.
Pie XI, 10
\end{quote}

On ne parle pas des droits de l'homme, vus comme des droits contre Dieu. Il faudra attendre après la seconde guerre mondiale pour en parler.

\subparagraph{Une vision surplombante}
\begin{quote}
    « Les Etats, à leur tour, apprendront par la célébration annuelle de cette fête que les gouvernants et les magistrats ont l'obligation, aussi bien que les particuliers, de rendre au Christ un culte public et d'obéir à ses lois. » (Pie XI, QP) 
\end{quote}

\begin{quote}
    « Les chefs de la société civile se rappelleront [grâce à la fête du Christ-roi], de leur côté, le dernier jugement, où le Christ accusera ceux qui l'ont expulsé de la vie publique, mais aussi ceux qui l'ont dédaigneusement mis de côté ou ignoré, et punira de pareils outrages par les châtiments les plus terribles; car sa dignité royale exige que l'État tout entier se règle sur les commandements de Dieu et les principes chrétiens dans l'établissement des lois, dans l'administration de la justice, dans la formation intellectuelle et morale de la jeunesse, qui doit respecter la saine doctrine et la pureté des mœurs. » (Pie XI, QP) 
\end{quote}

\paragraph{Quelle image du Christ ?}

\paragraph{Un roi, chef de guerre ?}
Image de la conquête

\begin{quote}
    « On [..] a vu les nombreux pays que de vaillants et invincibles missionnaires ont conquis au catholicisme au prix de leurs sueurs et de leur sang; on […] a vu enfin les immenses territoires qui sont encore à soumettre à la douce et salutaire domination de notre Roi. » (Pie XI, QP, § 2). 
\end{quote}

Est-ce que le Christ a besoin d'être défendu ?

\begin{quote}
    « Mais du jour où l'ensemble des fidèles comprendront qu'il leur faut combattre, vaillamment et sans relâche, sous les étendards du Christ-Roi, le feu de l'apostolat enflammera les cœurs, tous travailleront à réconcilier avec leur Seigneur les âmes qui l'ignorent ou qui l'ont abandonné, tous s'efforceront de maintenir inviolés ses droits. » (Pie XI, § 19)  
\end{quote}

On est dans un registre combattant. Saint Paul avait parlé de l'armure de la foi. Mais attention aux images, qui ne sont jamais anodines. 

\begin{quote}
    « Il faut qu'il règne sur nos corps et sur nos membres : nous devons les faire servir d'instruments ou, pour emprunter le langage de l'Apôtre saint Paul, d'armes de justice offertes à Dieu pour entretenir la sainteté intérieure de nos âmes. » (Pie XI, § 22). 
\end{quote}

\paragraph{Un roi, malgré tout, bon et doux}
Il cite Mt 11,30, il le présente comme un roi pacifique. Il parle de victime. Par contre, le mot \textit{serviteur} n'apparait pas.

\subsection{L’ambiguïté de la christologie royale }
 
 \paragraph{Sa royauté n'est pas pensé à partir de son ministère de service} Il y a une polémique sous-jacente. Comme si le Christ était la solution à tous les problèmes. Pie XI conçoit le Christ selon un modèle monarchique mais pas forcément suffisamment évangélique.
 
 \paragraph{1870 : infaillibilité pontificale} Idée monarchique du pape soulignée et qu'on applique au Christ, au lieu de partir de son service, mourant sur la croix.
 
 \paragraph{Attenuation par Pie XI} Pie XI ajoute trois compléments théologiques :
 \begin{itemize}
     \item le Fils est celui qui fait la volonté de son Père, à partir de la grâce et non à partir de la Chair. L'esprit l'inspire
     \item la Croix est l'évènement central où il porte par son attitude et ses paroles la figure du Roi : une Royauté qui n'est pas de ce monde.
     \item Pie XI met une distance entre le règne de droit et le règne de fait : distance eschatologique. Même la chrétienté n'était pas adéquate par rapport à ce règne.
 \end{itemize}
 

\section{Conclusion } 

\paragraph{né dans une christologie chrétienne} Corps du Christ / monarchique

\paragraph{conception de royauté encore à christologiser}

\paragraph{Face au contexte} de nationalisme et de fascisme, il veut barrer par sa christologie.