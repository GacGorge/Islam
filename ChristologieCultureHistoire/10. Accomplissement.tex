\chapter{La conscience moderne de l’histoire détermine une christologie de l’accomplissement}

\mn{Christologies et cultures dans l’histoire 10 La conscience moderne de l’histoire détermine une christologie de l’accomplissement Le Christ comme principe et accomplissement de l’histoire}




\section{Eléments bibliographiques} 

CHOLVY, B., « L’accomplissement, clé de lecture de Vatican II selon Henri de Lubac », Nouvelle revue théologique 135 (2013) 178-195. 

CLAVERIE, P.,  Lettres et messages d’Algérie, Paris 1996. 

DANIELOU, J., Le Mystère du salut des nations, Paris 1948. 

DANIELOU, J., Essai sur le mystère de l’histoire, Paris 1953. 

DANIÉLOU, J., « Christianisme et religions non chrétiennes » dans Théologie d’aujourd’hui et de demain, Paris 1967, 65-79. 

DUPUIS, J. Vers une théologie chrétienne du pluralisme religieux, tr. par O. PARACHINI, Paris 1997.  

FÉDOU, M., « Le concile Vatican II : un enjeu d’interprétation » dans 
Christoph THEOBALD (dir.), Vatican II sous le regard des historiens. Colloque du 23 septembre 2005 des facultés jésuites de Paris, Paris 2006, 37-157. 

GUGGENHEIM, A., « La théologie de l’accomplissement de Jean Daniélou », Nouvelle revue théologique 128 (2006), 240-257. 

LUBAC (de), H., Le fondement théologique des missions, Paris 1946. 

LUBAC (de), H., Catholicisme. Les aspects sociaux du dogme, Paris 1952. 

SESBOÜÉ, B., Hors de l’Église pas de salut. Histoire d’une formule et problèmes d’interprétation, Paris 2004. 


\section{Introduction }

Toujours le passage de la Chrétienté à la culture moderne. Il ne s'agit pas d'annoncer à une nouvelle culture mais de montrer que malgré les apparences, le Christ n'a pas quitté la culture.




\section{L’évolution du contexte et les changements historiques de la modernité}
\paragraph{Comment Jésus est présent} Dans une culture anthropocentrique, on va annoncer que le Christ accomplit l'humanité
\begin{Synthesis}
La culture est chrétienne mais différemment et l'enjeu est que les non-chrétiens perçoivent comment le Christ peut être présent
\end{Synthesis}

\subsection{La découverte du monde et de nouvelles cultures et religions }



\paragraph{l'humanisme caractérise évidemment la modernité mais il y a aussi la prise de conscience que l'homme est protagoniste de l'histoire.} Il s'agit pour l'homme de construire l'histoire. Cette conscience de cette responsabilité vient de l'élargissement de l'horizon avec les \textit{découverte}.

\subsection{Expansion et suprématie de l’Europe }
\paragraph{L'Europe dépasse la Chine au XVI} Grâce à l'Amérique, l'Europe va s'enrichir.

\subsection{Les Guerres de religion et le début de la « sécularisation » de l’Europe }

\paragraph{La guerre de 30 ans} va décimer l'Europe entre 1618-1648 (1/3 disparait). Certains historiens pensent que c'est à cette période que la première industrialisation se développe du fait du manque de main d'oeuvre.

\paragraph{le christianisme n'étant plus le garant de la paix} on va construire la société sur un \textsc{droit naturel} : \textbf{sécularisation}, avec la construction des Etats sécularisés.


\subsection{La conséquence : émergence d’une nouvelle culture, la culture moderne} 

\paragraph{L'idée de l'homme est celui qui fait son histoire} La réussite des nations européennes renforce cette idée.

\paragraph{Philosophe : Vico} \textit{la science nouvelle} : la seule science dont ont est sûr de comprendre la logique est l'histoire puisque l'homme est moteur de l'histoire. Alors que traditionnellement, c'est Dieu qui fait l'histoire. 

\paragraph{La religion est du côté de la conservation} et non du progrès. Le Christianisme va être associé au \textit{fixisme}. Il faut donc s'émanciper de la Religion, au nom du progrès, de l'émancipation. Un philosophe allemand, Karl Löwith, la philosophie de l'histoire, a été sécularisée : on part d'une histoire globale du progrès inspirée par le christianisme mais sans Dieu (Hegel,...). 


\section{La christologie de l’accomplissement de Daniélou : du Christ dans l’histoire à l’histoire en Christ }


\paragraph{Rechristologiser l'histoire : le programme des théologiens} On étudie Danielou, sj, patristicien car il est clair, mais on aurait pu prendre \textit{de Lubac}. Il fallait répondre au marxisme. Montrer que le Christianisme n'est pas contre l'histoire mais pour l'histoire d'une certaine manière.

\paragraph{Si le Christ devait justifier la cohérence de la société au moyen-âge} il s'agissait pour Danielou de répondre aux enjeux de la modernité.

\paragraph{On est aujourd'hui dans la Post modernité ce qui n'était pas le cadre de Danielou} L'homme n'est plus au centre. On ne croit plus au fait qu'on peut avoir un discours unifié du monde moderne, bercé par le progrès.

\paragraph{
Essai sur le mystère de l’histoire, Paris 1953. 
} Un essai fait à partir de conférences dans les années 40/50. Le Christ n'est pas un homme particulier mais il est la Loi de l'histoire.
Dans sa préface de 1982 (Rondeau, de Lubac), question des civilisations non-occidentales, l'histoire permet de penser et accueillir les autres. L'histoire est en dynamique et peut intégrer de façon plus forte.

\subsection{La nature « théandrique » de l’histoire du salut }

\paragraph{Théandrique} le Christ, Dieu et homme, le dogme de Chalcédoine est insurpassable. Il souligne les deux actions, de Dieu et des hommes, dès l'ancien testament. Il se décale d'une philosophie de l'histoire qui ne serait que des hommes. \textit{Elle est faite de façon synergique}.



\paragraph{synergisme} car action entre la liberté de l'homme et la grâce divine\sn{décret sur la justification du Concile de Trente}. C'est un \textit{patristicien}, donc il est habitué à parler des \textit{figures}.

\paragraph{\textit{magnalia Dei}} les grandes oeuvres de Dieu. Dieu agit dans l'histoire et pour l'homme comme le vigneron dans Is 5 : 
\begin{quote}
    Is 5,1-6 : « Que je chante pour mon amis, le chant du bien-aimé et de sa vigne : Mon bien-aimé avait une vigne sur un coteau plantureux. Il y retourna la terre, enleva les pierres, et installa un plant de choix. Au milieu, il bâtit une tour et il creusa aussi un pressoir. Il en attendait de beaux raisins,  il n’en eut que de mauvais. Et maintenant, habitants de Jérusalem et gens de Juda, soyez donc juges entre moi et ma vigne. Pouvais-je faire pour ma vigne plus que je n’ai fait ? J’en attendais de beaux raisins, pour pourquoi en a-t-elle produit de mauvais ? Eh bien, je vais vous apprendre ce que je vais faire à ma vigne : enlever la haie pour qu’elle soit dévorée, faire une brèche dans le mur pour qu’elle soit piétinée.  J’en ferai une pente désolée, elle ne sera ni taillée ni sarclée, il y poussera des épines et des ronces et j’interdirai aux nuages d’y faire tomber la pluie ». 
\end{quote}

Tout est fait pour que l'homme réponde à cette sollicitation de Dieu.

\begin{quote}
    « Dans l’AT, l’histoire du salut présentait une double ligne : d’une part, elle nous est présentée comme une suite d’interventions divines, comme une histoire des actions de Dieu ; de l’autre nous y avons vu la suite des réponses de l’homme. Ces réponses nous sont apparues comme surtout négatives (…) Dans l’AT, ces deux lignes constituent comme des domaines séparés (…) Cette double histoire ne devenait pleinement intelligible et en même temps n’atteignait son plein accomplissement que dans la personne du Verbe Incarné » (Daniélou, \textit{Essai}, 181). 
\end{quote}

\paragraph{Mystère de l'action de Dieu} Dieu agit mais on ne le voit pas : acte de Foi.

\paragraph{Face à la fidélité de Dieu et sa patience, l'homme est infidèle et impuissant} L'homme ne répond pas toujours aux \textit{magnalia Dei}. Israel, et au delà toute l'humanité, est la vigne du Seigneur. Il attend du raisin mais l'histoire est dramatique parce que la vigne ne donne pas de raisin.\sn{Mt 21, Vignerons homicides}.  Une image allégorique :
\begin{itemize}
    \item initiative divine
    \item réponse de l'homme souvent négative
\end{itemize}


\begin{quote}
    « L’AT annonce que Iahweh accomplira à la fin des temps des actions admirables dont l’éclat fera pâlir celles qu’il a accomplies dans le passé pour Israël (…). C’est Iahweh qui créera des cieux nouveaux et une terre nouvelle (voir Is 65,17) (…) C’est lui qui règnera sur toutes les nations (Is 2,3) » (Daniélou, 185).  
\end{quote}
Cette double histoire va atteindre son accomplissement dans le Verbe incarné. La réponse, auparavant séparée est maintenant liée en Christ, l'initiative et la réponse.

\subsection{Le Christ accomplit l’histoire du salut : la convergence de ces deux lignes en Christ }

\paragraph{Le Christ centre et clef de l’histoire } Le Christ apparaît comme l'oeuvre et l'initiative définitive de Dieu mais aussi la réponse définitive à Dieu (en faisant la volonté de Dieu). Le Christ devient la vraie vigne, qui donne le fruit \mn{La Croix féconde de Saint Clément à Rome avec tous ces rameaux}. En cette humanité que la réponse à Dieu est parfaite.
\begin{quote}
    « En Jésus-Christ seul l’œuvre de Dieu est parfaitement réussie.  (…) En lui la grâce de Dieu donne tous ses fruits ; en Lui, Dieu peut parfaitement se reposer dans une humanité qui porte des fruits de sainteté incomparables. Ce que le peuple d’Israël n’avait pas été capable d’accomplir, Dieu lui-même, dans l’humanité de Jésus-Christ (…) l’accomplit parfaitement. » (Daniélou, 178). 
\end{quote}

\begin{quote}
    « Dans l’AT, l’histoire du salut présentait une double ligne : d’une part, elle nous est présentée comme une suite d’interventions divines, comme une histoire des actions de Dieu ; de l’autre nous y avons vu la suite des réponses de l’homme. Ces réponses nous sont apparues comme surtout négatives (…) Dans l’AT, ces deux lignes constituent comme des domaines séparés (…) Cette double histoire ne devenait pleinement intelligible et en même temps n’atteignait son plein accomplissement que dans la personne du Verbe Incarné » (Daniélou, Essai, 181). 
\end{quote}

\paragraph{L’argument en trois temps de Daniélou } 

En Jésus-Christ, fin de l'histoire, événement eschatologique : 
    \begin{quote}
        He 1,1 À BIEN DES REPRISES et de bien des manières, Dieu, dans le passé, a parlé à nos pères par les prophètes ;
        
        
        Ep 1, 10 10 pour mener les temps à leur plénitude, récapituler toutes choses dans le Christ, celles du ciel et celles de la terre.
    \end{quote}
L'AT nous annonce l'arrivée d'un messie à la fin des temps. 



\paragraph{Apocalyptique et messianisme } On a une intelligence de l'histoire parce qu'on est au bout
\begin{itemize}
 
    
    \item ligne apocalyptique : à la fin des temps. \textit{les Rois}
    \item ligne messianique :  Le messie est celui qui fait la volonté de Dieu mais il est représentant et solidaire du Peuple. \textit{Les Prophètes}
\end{itemize}


\begin{Def}[Le Christ kurios et Christ]
Kurios : Seigneur (apocalyptique) 
Christos : Messie, 

\end{Def}

Abraham, David, Adam sont dans la généalogie du Christ
\begin{quote}
    
\begin{quote}
    « En Jésus-Christ seul l’œuvre de Dieu est parfaitement réussie.  (…) En lui la grâce de Dieu donne tous ses fruits ; en Lui, Dieu peut parfaitement se reposer dans une humanité qui porte des fruits de sainteté incomparables. Ce que le peuple d’Israël n’avait pas été capable d’accomplir, Dieu lui-même, dans l’humanité de Jésus-Christ (…) l’accomplit parfaitement. » (Daniélou, 178). 
\end{quote}
\end{quote}

\subsection{Le dogme de Chalcédoine comme clé herméneutique de l’histoire du salut }

Danielou passe de l'AT au dogme de Chalcedoine, plutôt statique. Mais pour Danielou, cela exprime l'histoire du Salut.


\begin{quote}
    « C’est en effet l’union des deux natures qui permet de montrer comment le Christ est l’aboutissement de l’AT, comment il est la fin, le \textit{telos}, du dessein entier du salut, comment en dernier lieu le retour du Christ consommera  ce dessein. C’est le dogme de Chalcédoine qui permet donc une vraie théologie de l’histoire (…) C’est le dogme de Chalcédoine qui donne consistance au temps et le transforme en histoire » (Daniélou, 183). 
\end{quote}

\begin{quote}
    « En montrant comment s’accomplissait l’union des deux natures dans la personne du Verbe, en montrant comment elle comportait la double intégrité des natures et l’unité de la personne, la doctrine de Chalcédoine donne leur vrai sens à des formules qui autrement resteraient incertaines. (…) ‘Parmi les témoignages des prophètes, relatifs au Messie, les uns l’annonçaient comme Dieu et d’autres comme homme : l’union hypostatique réalise leur conciliation » (Daniélou,  188). 
\end{quote}

La formule de Chalcédoine permet de montrer comment l'action de Dieu intervient dans le monde "sans confusion". Car on ne peut plus dire que l'homme n'a pas sa part dans l'histoire (sans liberté). Inversement, un historien Chrétien peut il penser que Dieu n'intervient pas dans l'histoire.\mn{Signes des Temps, cf aussi Luther et l'action extra-ordinaire de la Création de Dieu}


\subsection{De l’histoire du salut à l’histoire totale}
On peut toujours rester dans l'histoire du Salut.
\begin{itemize}
    \item Sol 1 : dépassé ("c'était les temps anciens, révolus")
    \item Solution 2 : à côté (une histoire du salut à côté)
\end{itemize}
Danielou propose une approche dialectique : 
\begin{itemize}
    \item le christianisme est dans l'histoire
    \item MAIS l'histoire est dans le Christianisme. \textit{Elle constitue une préparation} à l'histoire Sainte. Et l'histoire sainte va vers un achèvement. 
\end{itemize}


\begin{quote}
    « D’une part, le christianisme est dans l’histoire. Il apparaît à un moment donné dans le développement des événements historiques. Il fait partie de la trame de l’histoire totale. En ce sens, il est objet de connaissance pour l’historien qui le décrit en tant qu’il affleure dans la série des faits historiques observables. Mais, par ailleurs, l’histoire est dans le christianisme ; l’histoire profane rentre dans l’histoire sainte, car c’est elle à son tour qui est une partie dans un tout où elle constitue une préparation. Mais le christianisme est précisément le siècle futur, déjà présent en mystère. En ce sens, dans sa réalité profonde, il est un au-delà non seulement d’un moment, mais la totalité de l’histoire. Il est vraiment ‘\textit{novissimus}’, le dernier ; avec lui la ‘fin’ est déjà là » (Daniélou, 30). 
\end{quote}



\paragraph{L’histoire à l’intérieur du Christ total} Dieu s'implique dans l'histoire, \textit{s'incarne} dans l'histoire mais il est toujours libre : on ne peut pas l'enfermer. Des paroissiens, des gens de passage.

\begin{quote}
    « Il y a dans le christianisme une exigence perpétuelle à la fois d’incarnation et de dégagement. L’incarnation est un devoir. Et ceux qui voudraient d’un christianisme étranger à l’histoire, d’une pureté intemporelle, se trompent sur son essence (…). A côté de ces incarnations, il y a un égal devoir de dégagement. Le christianisme ne s’identifie à aucune des formes particulières de culture où il s’incarne » (Daniélou, 31-32). 
\end{quote}

\paragraph{Le Verbe rédempteur est le Verbe créateur} c'est le même. Danielou cale sa théologie sur celle de S. Irénée, qui a fait une théologie de l'histoire\sn{A noter que S. Augustin et S. Irénée sont les deux Pères de Vatican II, tous les deux des théologiens de l'histoire (S. Augustin avec la cité de Dieu)}

\begin{quote}
    « L’histoire sainte constitue en réalité l’histoire totale (…). Irénée reprendra ce thème contre la gnose. C’est non seulement l’histoire humaine, mais la totalité de l’histoire cosmique que l’histoire du salut embrasse. Elle ne se situe pas à l’intérieur d’un  monde de la nature et d’une histoire naturelle, dans lesquels elle ferait irruption. Mais elle embrasse cette histoire même dont elle est constitutive. Le Verbe rédempteur est le même que le Verbe créateur » (Daniélou, 33-34). 
\end{quote}

Le Christ est l'alpha et l'oméga : sur le cierge, on note la croix et l'année, une théologie de l'histoire.

\subsection{Le Christ dans son rapport au monde}

\paragraph{Salut par l'histoire} Finalement le Salut, c'est découvrir le sens de l'histoire. Ratzinger : l'histoire donne un sens et permet de sortir de l'immédiateté : dimension salvifique de l'histoire. Cela nous permet de survivre et d'affirmer que l'homme ne tombe pas accidentellement dans le monde. En terme philosophique, l'histoire nous sauve car nous sommes des chercheurs de sens et que l'histoire permet de mettre en perspective les évènements.


\paragraph{La périodisation} Pas seulement diachronique mais synchronique, on peut avoir plusieurs phases au même moment : 
L'histoire est comme : 
\begin{itemize}
    \item Préparation (ou \textit{préhistoire du salut} car elle prépare le salut) : \textsc{L'homme recherche Dieu}
    \item \textsc{présence cachée} / mystique. Dans l'Ancien Testament, le Christ est préfiguré. 
    \item \textsc{l'accomplissement dans le NT}, 
et anticipation du siècle futur, quand tout sera récapitulé en Christ. \textsc{Le Christ Sacrement}, présence réelle en signe
    \item \textsc{siècle futur }: "cieux nouveaux et terre nouvelle", récapitulation en Christ.
\end{itemize}
L'histoire sainte se situe d'une certaine façon après l'histoire profane. Les autres traditions religieuses sont dans la préparation et donc dans la préparation.

Il voit la croissance de l'Eglise mais en parallèle, il veut garantir de la volonté de Dieu et donc il développe aussi l'image d'un Dieu inattendu (et pas uniquement récapitalisation).

On ne peut calculer la date de la venue du Christ. 
Il y a dans le temps de l'Eglise l'action de Dieu et l'action de l'homme. 




On pourrait en discuter.


\paragraph{La loi « christique » de l’histoire comme accomplissement }

En même temps, accomplissement et dépassement. 
\begin{quote}
    « Les religions naturelles – et c’est ce qui en elles est valable – attestent le mouvement de l’homme vers Dieu ; le christianisme est le mouvement de Dieu vers l’homme qui en Jésus-Christ vient le saisir, pour le conduire à Lui » (Daniélou, 116). 
\end{quote}

\begin{quote}
    « Les religions non-chrétiennes ont pu connaître ce que la raison humaine laissée à elle-même peut atteindre, à savoir l’extérieur de Dieu, son existence et ses perfections telles qu’elles se manifestent par son action dans le monde » (Daniélou, 115).  
\end{quote}

Les traditions non chrétiennes sont du côté de la raison, naturelle. 
Le Christ accomplit la Loi sans l'accomplir. 


\paragraph{Le second aspect de la loi christique de l’histoire : l’accomplissement par le jugement et le renouvellement/élévation}  On est aussi dans la crise et pas dans une évolution continue : \textit{kairoi} ; crise. Il écarte continu et l'aspect cyclique des grandes civilisations; crises successives. L'Eglise est aussi dans ce chemin de purification. 

\begin{quote}
    « L’histoire n’est pas constituée par un progrès continu comme le veut l’évolutionnisme, ni par une suite de civilisations discontinues et hétérogènes, comme le croit Spengler, mais dans une suite de \textit{kairoi}, de crises décisives, qui sont chaque fois l’éclatement et le jugement d’une civilisation qui a péché par excès d’hybris et le renouvellement de l’Église par cette purification. Ces \textit{kairoi} sont à la fois la reprise du Kairos par excellence qui est la passion et la résurrection de Jésus et l’anticipation du \textit{kairos} final qui est le jugement dernier (…) Il faut (…) que toutes les réalités du monde connaissent cette crise qui à la fois les condamne et les sauve » (Daniélou, 36-37). 
\end{quote}
Le mystère pascal est le modèle de la loi de l'histoire, des crises, qui permettent d'aller plus loin.




\paragraph{La loi christique selon les différentes périodes historiques }

Il reprend dans les périodes historiques qu'il a déterminé, comment elles s'appliquent.
\begin{itemize}
    \item {De la pre-histoire} Mystique \textsc{naturelle} (ascèse et effort pour aller vers Dieu)
    \item le versant \textsc{surnaturel} de la mystique : la grâce qui nous purifie. 
\end{itemize}
Passer de l'ascèse à la grâce, de Pelage à S. Augustin. A certains moments, on voit qu'on ne peut être sauvé par nous-mêmes et on accepte la grâce de Dieu. Il voit dans l'hindouisme, une mystique purement naturelle.

\begin{quote}
    « C’est dans cette lumière qu’apparaît la différence des mystiques non chrétiennes et de la mystique chrétienne. Pour les premières l’union à Dieu est le terme d’une ascèse par laquelle l’âme se dépouillant de ce qui lui est étranger, retrouve sa pure essence, qui est Dieu même (…). Le Dieu chrétien est (…)  un Dieu vivant et transcendant qu’aucune technique ne saurait capter. Il se communique librement, quand et comme il veut » (Daniélou, 112). 
\end{quote}

Dans le néoplatonisme, on fait beaucoup d'ascèses.
Alors que dans la mystique surnaturelle, la réponse de l'homme est d'accueillir le don de Dieu.
\begin{quote}
    « Le christianisme n’est pas un effort de l’homme vers Dieu. Il est une puissance divine accomplissant dans l’homme ce qui est au-dessus de l’homme : et à quoi l’effort de l’homme sera seulement une réponse » (Daniélou, 113).  
\end{quote}


\begin{quote}
    « Ceci nous explique qu’une spiritualité missionnaire complète et à la fois une spiritualité d’incarnation (…) et en même temps une spiritualité et un mystère de rédemption en ce sens qu’il y a quelque chose qui doit tout de même être détruit et mourir » (Daniélou, le mystère du salut des nations, 47). 
\end{quote}

\paragraph{Comment on passe de la préhistoire au salut} de la préparation, il faut intégrer, purifier et élever: 
\begin{quote}
    Le christianisme « les purifie de toute erreur, c’est-à-dire qu’il détruit la corruption et surtout l’idolâtrie (…) Ensuite le christianisme achève et accomplit les vérités imparfaites qui subsistent dans les religions païennes par la sagesse chrétienne. Il reprend les valeurs naturelles de l’homme religieux, il les ressaisit pour les consacrer. C’est ainsi que nous voyons le christianisme ancien intégrer, après les avoir purifiées, les valeurs de la philosophie grecque. C’est ainsi que nous pourrons voir demain le christianisme reprendre, après les avoir purifiées, toutes les valeurs que contiennent l’ascèse des Hindous ou la sagesse de Confucius. La mission chrétienne (…) n’est pas destruction, mais libération et transfiguration des valeurs religieuses du paganisme. Le Christ n’est pas venu détruire, mais accomplir » (Daniélou, 118-119). 
\end{quote}

\subsection{Quelques critiques}

\paragraph{vis à vis du judaïsme} Le judaïsme n'est pas uniquement le passé du Christianisme car il perdure.

\paragraph{Contexte européo-centrisme de l'histoire} Il n'est pas indemne d'une pensée du progrès de l'histoire. Même synchronique, on est dans une notion de progrès de l'histoire.
Peut être faudrait il développer l'aspect synchronique et laisser l'histoire du salut et le Christ dans les autres religions.


\section{Le concile Vatican II }

Très marqué la vision de personnes comme Danielou.

\subsection{Lumen Gentium}  
On reprend cette grille d'analyse de Danielou : 
\begin{quote}
    « Tout ce qui, chez eux, peut se trouver de bon et de vrai, l’Église le considère comme une préparation évangélique et comme un don de Celui qui illumine tout homme pour que, finalement, il ait la vie. Bien souvent, malheureusement, les hommes, trompés par le malin, se sont égarés dans leurs raisonnements, ils ont échangé la vérité de Dieu contre le mensonge, en servant la créature de préférence au Créateur » (LG 16). 
\end{quote}

\begin{quote}
    « (L’activité de l’Église) n’a qu’un but : tout ce qu’il y a de germes de bien dans le cœur et la pensée des hommes ou dans leurs rites propres et leur culture, non seulement ne pas le laisser perdre, mais le guérir, l’élever, l’achever pour la gloire de Dieu, la confusion du démon et le bonheur de l’homme » (LG 17). 
\end{quote}

On est dans la même dialectique de purifier et Elever.

\subsection{Le christocentrisme de Gaudium et spes } 

Très marqué dans GS. Christocentrisme inclusive : 
\begin{quote}
    « Puisque le Christ est mort pour tous et que la vocation dernière de l’homme est réellement unique, à savoir divine, nous devons tenir que l’Esprit-Saint offre à tous, d’une façon que Dieu connaît, la possibilité d’être associé au mystère pascal » (GS 22,5).
\end{quote}


\paragraph{Le Christ est coextensif à toute l’histoire }

l'esprit du seigneur qui remplit l'univers.. On parle des signes des temps.




\paragraph{La préparation : la création orientée}
GS 22, 
Verbe créateur $\rightarrow$ Création $\rightarrow$  Christ qui accomplit $\rightarrow$  

La loi de l'histoire, comme chez Theillard de Chardin, c'est le Christ.

\paragraph{L’histoire comme purification et restauration}
\mn{Voici quelques notes prises à partir de : CHOLVY, B., « L’accomplissement, clé de lecture de Vatican II selon Henri de Lubac », Nouvelle revue théologique 135 (2013) 178-195. }
 
\begin{quote}
    Le texte souligne le péché et la figure du Seigneur qui est « venu pour restaurer l’homme dans la liberté et sa force, le rénovant intérieurement ». En effet, GS parle d’une intelligence marquée encore par une part d’obscurité et de faiblesse venant d’un péché (GS 15), d’une conscience qui s’égare, d’une liberté qui dérive vers « une licence de faire n’importe quoi » ou d’une liberté blessée (GS 16). GS décrit l’athéisme : « beaucoup de nos contemporains (…) rejettent explicitement le rapport intime et vital qui unit l’homme à Dieu » (GS 19). L’homme serait « le seul artisan et le démiurge de sa propre histoire » (GS 20). Autrement dit, certains pensent que croire en Dieu et lui obéir seraient contre la dignité de l’homme. La purification et le renouvelant incombent avant tout à l’Église (voir GS 21,5).
\end{quote}
\paragraph{L’histoire comme élévation et purification }
 
\begin{quote}
    Dans cette situation, l’Église, rappelle que « la reconnaissance de Dieu ne s’oppose en aucune façon à la dignité de l’homme, puisque cette dignité trouve en Dieu lui-même ce qui la fonde et ce qui l’achève » (GS 21). On retrouve ici l’englobant historique entre création et accomplissement. En GS 22 le Christ est le point culminant. « Il est l’Homme parfait qui a restauré (…) la ressemblance divine. Parce qu’en lui la nature humaine a été assumée, non absorbée, par le fait même, cette nature a été élevée en nous aussi à une dignité sans égale. Car, par son incarnation, le Fils de Dieu s’est en quelque sorte uni lui-même à tout homme » (GS 22,2). Le Christ a donné l’Esprit : « Par cet Esprit (…), c’est tout l’homme qui est intérieurement renouvelé, dans l’attente de ‘la rédemption des corps’ » (GS 22,4).
\end{quote}

\paragraph{Le Christ comme le sens ultime de l’histoire des hommes }
Le Christ comme le sens ultime de l’histoire des hommes
\begin{quote}
     Le concile relit l’histoire de l’humanité comme orientée et configurée par le Christ :  L’histoire de l’homme créé orientée vers le Christ, comme Adam préfigurait le Christ, l’histoire de l’homme comprise selon l’incarnation : purification (de ce qui est mauvais), élévation et accomplissement  (de ce qui est bon) : je ne suis pas venu abolir mais accomplir la Loi. Unité de l’histoire en tant que la première histoire est « préparation » en vue de et la seconde accomplissement. La seconde histoire intègre la première, elle domine maintenant en raison du mystère pascal (voir GS 22,5 : l’Esprit peut associer tout homme au mystère pascal). A la fin de la première partie, GS résume : « Le Verbe de Dieu, par qui tout a été fait, s’est luimême fait chair, afin que, homme parfait, il sauve tous les hommes et récapitule toutes choses en lui. Le Seigneur est le terme de l’histoire humaine, le point vers lequel convergent les désirs de l’histoire et de la civilisation, le centre du genre humain, la joie de tous les cœurs et la plénitude de leurs aspirations » (GS 45,2). Ap 22,13 : « Je suis l’alpha et l’oméga, le premier et le dernier, le commencement et la fin ». Le Christ englobe toute l’histoire. « Avec GS, le Concile invite à penser le monde et l’ensemble du créé à partir de leur fin, et non d’abord à partir de leur autonomie, toute en affirmant fermement cette autonomie » 
\end{quote}


\section{Conclusion}

\paragraph{L'histoire ne fait plus révé} Mais risque de desespérance. Repli identitaire, globalisation. 