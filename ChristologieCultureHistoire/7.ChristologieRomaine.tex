\chapter{La christologie : de la culture romaine à la chrétienté }
Christologies et cultures dans l’histoire 7 Transition
\section{ Bibliographie}

JEAN-PAUL II, Lettre encyclique Redemptoris missio, Rome, 1991. 

MALANOWSKI, G. E., « Emile Mersch, s. j. (1890-1940). Un christocentrisme unifié », Nouvelle revue théologique 112 (1990) 44-66.  

MERSCH, E., Le corps mystique du Christ. Etudes de théologie historique, I et II, Louvain 1933. 

MERSCH, E., La théologie du corps mystique, Paris 1944. 

MESLIN, M., Le christianisme dans l’empire romain, Paris, PUF, 1970. 

MOULE, C. F. D., « The corporate Christ » dans The Origin of Christology, Cambridge 1977, 47-96. 

QUESNEL, Michel, La première épître aux Corinthiens, Cerf, Paris, 2018. 

QUESNEL, M., « Du Christ préexistant au Seigneur entraînant les humains dans sa victoire. Trajectoire christologique de 1 Corinthiens 7-16 » dans ACFEB, Paul et son Seigneur. Trajectoires christologiques des épîtres pauliniennes, Lectio divina, Cerf, Paris, 2018, 103114. 

SACHOT, M., L’invention du Christ. Genèse d’une religion, Paris, Odile Jacob, 20112. 

SCHEID, J., La religion des Romains, Paris, Armand Colin, 20194. 

SCHLOSSER, J., « Le corps en 1 Co 12,12-31 », dans Le corps et le corps du Christ dans la 1 Co, V. 

GUENEL (éd.), Paris 1983, 97-110. 

VEYNE, P., Quand notre monde est devenu chrétien (312-394), Paris, Albin Michel, 2007. 

\section{Rappel de la logique et du but du cours}

Comment dans la christologie de Saint Augustin, on arrive à introduire sacrifice et médiation, questions de son temps. Comment cela va rentrer dans la grande tradition de l'Eglise.
Christologie : de la culture romaine à la chrétienté, chapitre de transition.

\paragraph{Rappel du Cours} Messie Juif, logos grec, religio médiateur. Augustin a pris l'idée de médiateur et de religion, c'est le Christ. il a pris la matrice romaine de pensée. Il faut prendre la matrice de la culture dans lequel on vit.

\paragraph{Christologie} non pas un concept abstrait mais il rejoint l'universalité dans sa capacité à entrer dans la matrice de chaque culture. Mais à chaque fois, la vision du Christ change. La Christologie se met au service de la parole de Dieu.

Or, l'annonce du Christ n'est pas uniquement l'annonce de sa dimension salvifique mais il a un \textit{corps } qui socialise. Pas seulement des paroles, un \textit{kerygme} mais elle se fait par son corps.
Jésus est un corps et suscite un \textit{corps social}. La visibilité du Christ se fait par son message mais aussi par son corps, \textit{social}.

\paragraph{l'annonce du Christ socialise}
Quels sont les deux effets de la Ressurection ? 
\begin{itemize}
    \item Le kerygme, des personnes commencent à annoncer la bonne nouvelle
    \item des personnes se rassemblent
\end{itemize}

Et ce rassemblement, corps dit quelque chose du Christ.
On peut voir une certaine \textit{anticipation eschatologique}, du Royaume dans le corps du Christ. Avec Saint Augustin et \textit{Religio}, on avait déjà cette intuition sociale.

\section{Le Christ est corporel et suscite un corps social }


 Il y a une risque d'amputer la foi à ignorer la dimension sociale du Corps. Le Christ est présent d'une certaine manière dans son Corps. Il n'est pas seulement celui qui nous fait face, il est celui en qui nous voyons. Il n'est pas seulement le \textit{monogène}, unique engendré, il est aussi le \textit{proto-tokos}, le premier né.
 
\subsection{Le Christ et le corps du Christ }
 
\paragraph{Le corps réel du Christ, crucifié}. Face au gnosticisme.
 
 \begin{quote}
     Lc 24, 39 palpez moi je ne suis pas un fantôme. 
 \end{quote}
\begin{quote}
    On est semé corps psychique et on renait corps spirituel 1Co15, 44
\end{quote}

\paragraph{Mais aussi Corps eucharistique, donné}

 
 \paragraph{Dimension sociale}Enfin, il y a le \textit{corps du Christ} identifié à l'Eglise qui nous pose probablement des difficultés aujourd'hui ?
 
 
 
 \subsection{Les différentes interprétation du « corps » du Christ dans le NT}
 
 \paragraph{organisme sociale qui vit du Christ}
\begin{quote}
     1 Co 10,16-17 : « Le pain que nous rompons n’est-il pas une communion au corps du Christ ? Puisqu’il y a un seul pain, nous sommes tous un seul corps : car tous nous participons à cet unique pain ».  
 \end{quote}
 
 Communion, communauté qui vit du Christ et qui fait corps social et solidaire.
 
 Repris dans : 

 
 \begin{quote}
     

Rm 12,4-5 : « Comme nous avons plusieurs membres en un seul corps et que ces membres n’ont pas tous la même fonction, ainsi, à plusieurs, nous sommes un seul corps en Christ, étant tous membres les uns des autres, chacun pour sa part » (Rm 12,4-5).  
 \end{quote}
 
 Idée d'un organisme organisé. \sn{On a aussi dans 1 Co 12.}
 L'idée d'analogie du corps social est assez courante (Cicéron,...). 
 
 \paragraph{Image de la tête qui est le Christ} une autre image, qui montre le lien entre Christ et Eglise.
 
 \begin{quote}
     Ep 1,22-23 : « [Dieu] a tout mis sous ses pieds, et il l’a donné, au sommet de tout, pour tête à l’Église qui est son corps, la plénitude de Celui que Dieu remplit lui-même totalement ». 
 \end{quote}
 \begin{quote}
     Eph 4,15-16 : « (…) Nous grandirons à tous égards vers celui qui est la tête, Christ. Et c’est de lui que le corps tout entier, coordonné et bien uni grâce à toutes les articulations qui le desservent (…) réalise sa propre croissance pour se construire lui-même dans l’amour ». 
 \end{quote}
 L'idée que l'Eglise est unie au Christ, la tête.
 
 \begin{quote}
     Col 2,19 : « Ils ne tiennent pas à la tête, de qui le corps tout entier, pourvu et bien uni grâce aux articulations et ligaments, tire la croissance que Dieu lui donne ».  » 
 \end{quote}
 Image que saint Augustin reprendra dans son image de Christ total.
 
 \paragraph{Fondation du sacrement du mariage} On parle de Chair(homme) car référence à Gn. 
 
 \begin{quote}
     Ep 5, 31-32 : « C’est pourquoi l’homme quittera son père et sa mère, il s’attachera à sa femme, et tous deux ne seront plus qu’une seule chair. Ce mystère est grand : moi, je déclare qu’il concerne le Christ et l’Église
 \end{quote}
 
 \paragraph{Quel est le type d'union entre le Corps social qu'est l'Eglise et le Christ} Plusieurs expressions mais qu'on ne peut synthétiser aisément.
 
 %---------------------------------------------------
 \subsection{L’interprétation de l’expression  « en Christ » }
 \paragraph{in-existence chrétienne}
 Romano Guardini a introduit le concept d\textit{in-existence} Chrétienne. Inclusion mystique ou pneumatique en Christ. 
 
 \paragraph{\textit{en Christ}, une expression étonnante chez Paul} On peut l'interpréter "en" comme un clause instrumentale (\textit{par}) ou \textit{dans}, dans un sens littéral ou figuré .
  
 \begin{quote}
      l'Eglise de Dieu qui est à Corinthe, à ceux qui ont été sanctifiés \textit{en} Jésus-Christ, appelés à être saints, et à tous ceux qui invoquent en quelque lieu que ce soit le nom de notre Seigneur Jésus-Christ, leur Seigneur et le nôtre: (1 Co 1,2)
 \end{quote}

 
 \begin{quote}
     « Croire, être rénové et scellé par le baptême, indique une opération, par laquelle l’homme entre dans un mode d’existence mystique qui le greffe sur un autre, avec lequel il a des échanges vitaux. Cet autre, c’est le Rédempteur éternel et réel par lequel il reçoit comme forme et contenu d’une nouvelle existence, la figure, l’œuvre, la passion, la mort et la résurrection du Sauveur » (Guardini, 61)\sn{L’essence du christianisme}
 \end{quote}

Emile Mersch : 
\begin{quote}
    "que le Christ est la cause... la chose en Christ... prototype,... sans expliquer comment. D'où la traduction : cause,..."
    Mais à changer à chaque fois le terme, on risque de perdre le sens primitif de l'expression, dans : à l'intérieur. 
\end{quote}
 
 164 utilisations de \textit{en Christ}.
 
 \paragraph{en Christ, ce n'est pas quelque chose de faible}
 
  
  Nouvelle manière d'être. Croire qu'être rénové par le baptême, c'est croire qu'il se greffe à un principe vital. \begin{quote}
      « La foi et le baptême fondent l’ ‘inexistence’ qui caractérise la vie chrétienne ; l’eucharistie la nourrit et l’épanouit » (Guardini, 66)
 \end{quote}
  
  Nous sommes dans le Christ et le Christ est en nous. Ce qui nous intéresse ici, c'est la dimension collective.  Guardini va exprimer cela : 
  \begin{quote}
     « Le Christ spirituel et réel est dans un ‘état’ tel, qu’il devient une sphère vivante dans laquelle l’homme peut exister par la foi ; qu’il devient vis-à-vis de toute être créé une puissance ‘intérieure’, capable de pénétrer en lui, sans toucher à son autonomie et à sa dignité ; capable de subsister, d’agir, de vivre dans l’homme » (Guardini, 60)
 \end{quote} 
 
 
 \begin{quote}
     « Tout l’être et toute la vie de Jésus forment une réalité mystique éternelle (…) Le Christ se tient dans un acte éternel, qui contient toute son œuvre et toute sa destinée rédemptrice (…)» (Guardini, 59)
 \end{quote}
 Notre relation en Christ, elle n'est pas seulement extérieure, nous sommes plongés dans le christ mort et ressuscité.
 
 \paragraph{surtout dans les épitres deuteropauliennes} sa forme intérieure et ce qui l'organise. Inclusion de toute l'humanité en Christ, enfermée en lui. Sens technique. ëtre en Christ, être physiquement dans le Christ social. 
 
 
 \subsection{Le corps du Christ est-il le Christ lui-même ?}
 
 \paragraph{En fait, on pense Eglise = Corps du Christ} 
 La difficulté de penser le corps du Christ dans l'Eglise. 

 
 \begin{quote}
     « Même s’il n’est pas lui-même appelé ‘le corps’, le Christ, comme une personne vivante, transcendante, inclusive et plus qu’individuelle, précède l’Église » (Moule, 70).
 \end{quote}
 
  \paragraph{partir du corps du Christ mort et ressuscité, corps social}
 Dans 1 Co 12,12 : 
 \begin{quote}
     Car, comme le corps est un et a plusieurs membres, et comme tous les membres du corps, malgré leur nombre, ne forment qu'un seul corps, ainsi en est-il de Christ.
 \end{quote}
 le texte sacré va plus loin : il parle du Christ. l'ensemble des chrétiens n'est pas en le christ, il est le christ lui-même. 
 \paragraph{Un Christ corporatiste}
 Selon Moule, 1 Co 12,12 serait le point d’appui en faveur de cette interprétation. 
 \begin{quote}
     « Prenons une comparaison : le corps est un, et pourtant il a plusieurs membres  mais tous les membres du corps, malgré leur nombre, ne forment qu’un seul corps : il en est de même du Christ ».
 \end{quote}
 
 \begin{quote}
     «  Paul ne dit pas ‘il en est de même de l’Église » ou même ‘du corps du Christ’ mais simplement ‘du Christ’. » (Moule, 71)
 \end{quote}
 Autrement dit, ce n'est pas le Christ, c'est le corps du Christ qui fonderait la communauté.
 Ce corps précède l'Eglise. Ce corps ressuscité du Christ n'est pas seulement un corps individuel mais un corps inclusif qui \textit{précède l'Eglise}. 
 \paragraph{le Corps du Christ existe indépendemment de l'Eglise}


 L'Eglise n'est pas elle-même le Corps du Christ, et elle n'est Eglise que si elle est en relation avec ce Corps du Christ.
 
\section{Distinguer le corps du Christ d’avec la culture/civilisation chrétienne }

\sn{le 29/3/22}
cette incorporation au Christ signifie-t-elle que l'on doit sortir de sa culture ? 
Il faut remarquer d'abord qu'il y a des chrétiens dans des cultures non chrétiennes. 

\begin{quote}
"Les chrétiens ne se distinguent des autres hommes ni par le pays, ni par le langage, ni par les coutumes. Car ils n’habitent pas de villes qui leur soient propres, ils n’emploient pas quelque dialecte extraordinaire, leur genre de vie n’a rien de singulier. Leur doctrine n’a pas été découverte par l’imagination ou par les rêveries d’esprits inquiets; ils ne se font pas, comme tant d’autres, les champions d’une doctrine d’origine humaine. 

Ils habitent les cités grecques et les cités barbares suivant le destin de chacun ; ils se conforment aux usages locaux pour les vêtements, la nourriture et le reste de l’existence, tout en manifestant les lois extraordinaires et vraiment paradoxales de leur manière de vivre. Ils résident chacun dans sa propre patrie, mais comme des étrangers domiciliés. Ils s’acquittent de tous leurs devoirs de citoyens, et supportent toutes les charges comme des étrangers. Toute terre étrangère leur est une patrie, et toute patrie leur est une terre étrangère. Ils se marient comme tout le monde, ils ont des enfants, mais ils n’abandonnent pas leurs nouveau-nés\sn{Peu de différences mais tout de même des points qui resistent}. Ils prennent place à une table commune, mais qui n’est pas une table ordinaire.

Ils sont dans la chair, mais ils ne vivent pas selon la chair. Ils passent leur vie sur la terre, mais ils sont citoyens du ciel. Ils obéissent aux lois établies, et leur manière de vivre est plus parfaite que les lois. Ils aiment tout le monde, et tout le monde les persécute. On ne les connaît pas, mais on les condamne ; on les tue et c’est ainsi qu’ils trouvent la vie. Ils sont pauvres et font beaucoup de riches. Ils manquent de tout et ils tout en abondance. On les méprise et, dans ce mépris, ils trouvent leur gloire. On les calomnie, et ils y trouvent leur justification. On les insulte, et ils bénissent. On les outrage, et ils honorent. Alors qu’ils font le bien, on les punit comme des malfaiteurs. Tandis qu’on les châtie, ils se réjouissent comme s’ils naissaient à la vie. Les Juifs leur font la guerre comme à des étrangers, et les Grecs les persécutent ; ceux qui les détestent ne peuvent pas dire la cause de leur hostilité.

En un mot, ce que l’âme est dans le corps, les chrétiens le sont dans le monde. L’âme est répandue dans membres du corps comme les chrétiens dans les cités du monde. L’âme habite dans le corps, et pourtant elle n’appartient pas au corps, comme les chrétiens habitent dans le monde, mais n’appartiennent pas au monde. L’âme invisible est retenue prisonnière dans le corps visible; ainsi les chrétiens : on les voit vivre dans le monde, mais le culte qu’ils rendent à Dieu demeure invisible. La chair déteste l’âme et lui fait la guerre, sans que celle-ci lui ai fait de tort, mais parce qu’elle l’empêche de jouir des plaisirs ; de même que le monde déteste les chrétiens, sans que ceux-ci lui aient fait de tort, mais parce qu’ils s’opposent à ses plaisirs.

L’âme aime cette chair qui la déteste, ainsi que ses membres, comme les chrétiens aiment ceux qui les déteste. L’âme est enfermée dans le corps, mais c’est elle qui maintient le corps; et les chrétiens sont comme détenus dans la prison du monde, mais c’est eux qui maintiennent le monde. L’âme immortelle campe dans une tente mortelle: ainsi les chrétiens campent-ils dans le monde corruptible, en attendant l’incorruptibilité du ciel. L’âme devient meilleure en se mortifiant par la faim et la soif; et les chrétiens, persécutés, se multiplient de jour en jour. Le poste que Dieu leur a fixé est si beau qu’il ne leur est pas permis de le déserter."

De la Lettre à Diognète, nn. 5-6 (Funk, 1, 317-321)

\end{quote}

On ne sent pas ici une idée de culture chrétienne. Cependant, en même temps, les chrétiens transforment la culture. Encore, faut il que les chrétiens deviennent majoritaires et la transforment et cela devient \textsc{une culture Chrétienne}. Même si le lien entre Corps du Christ et corps social n'est pas une nécessité, il est arrivé au moyen âge.

\paragraph{contre-exemple : la monasphère} \href{https://monasphere.fr/}{monasphère} Implantés pour l’Essentiel.
Monasphère vous accompagne de A à Z dans votre projet immobilier à proximité de lieux spirituels chrétiens en France.


\begin{quote}
    « Dire aux destinataires ‘vous êtes le corps de Christ’ n’est alors pas seulement leur dire : ‘Vous êtes Église, ou ‘Vous êtes peuple de Dieu’ ; mais bien plutôt : ‘Vous êtes un certain type d’Église’, un corps s’imposant un mode de vie nourri d’Evangile : corps crucifié devenant, par l’action de Dieu, corps ressuscité. La solidarité ecclésiale s’enracine dans le Christ lui-même ; les chrétiens sont incorporés dans un corps qui existe déjà et qui n’attend pas qu’ils en fassent partie pour exister » (Quesnel, Commentaire de 1 Co, 305-306) » 
\end{quote}
\begin{quote}
     « L’expression \emph{sôma Christou} signifie à la fois que l’Église est un corps (point de vue parénétique de la solidarité) et qu’elle est le corps du Christ (point de vue théologique et ontologique). La communauté est constituée, une et diverse, par intégration à une réalité qui la précède, à savoir le corps personnel du Christ, ou, pour le dire avec les mots de W. G. Kümmel : ‘le corps = le corps du Christ = le Christ ne naît pas…par l’union des chrétiens entre eux, mais les chrétiens sont incorporés dans le corps qui existe déjà » (Schlosser, 109). 
\end{quote}
\section{Le cas de la  « Chrétienté »}

\subsection{Des communautés chrétiennes à la religion d’Etat }
  
\paragraph{Principales dates} (d’après J. Scheid , p. 193-199).
\begin{itemize}
    \item  64 ap JC : Incendie de Rome. Persécution contre les chrétiens La première persecution des chrétiens a lieu entre 64.
    \item 70 : Destruction de la ville et du temple de Jérusalem par Titus 
    \item 197 : Apologétique de Tertullien 
    \item 202 : Edit interdisant le prosélytisme juif et chrétien. Persécutions des chrétiens. 
    \item 250 : Edit de Dèce obligeant les citoyens à sacrifier aux dieux. Persécution des chrétiens 
    \item 257 : Edit de Valérien interdisant le culte chrétien. 
    \item 303-305 : Grande persécution sous Dioclétien 
    \item 311 : Edit de tolérance de Galère reconnaissant le christianisme comme religion licite 
    \item 313 : Edit de Milan. Liberté de culte accordé aux chrétiens, restitution des églises. 
    \item 337 : Baptême de Constantin. 
    \item \textbf{391 : Lois de Théodose }interdisant définitivement le culte païen, les temples sont fermés et détruits. 
    \item 494 : Le pape Gélase Ier interdit aux chrétiens de Rome de participer aux Lupercales (équivalent du Carnaval). 
\end{itemize}
 C'est finalement assez rapide. Entre les persécutions du III et le IV, un basculement total politique.
 
 \paragraph{Nombre de chrétiens au début du IV} Pour Paul Veyne, il y avait peu de chrétiens mais c'est Constantin qui a fait que la religion chrétienne est devenue majoritaire. Pour Baslez (\cite{Baslez:MondeDevnuChretien}, c'était déjà une proportion importante. 
 
 \begin{Synthesis}
 Refus d'être une secte (avec le partage des biens,...) pour être une religion ouverte, acceptation des autorité. Dans les villes mais aussi à la campagne (cf Smyrne). Rôle des évèques, notables en lien avec les autres notables et les autres évèques des autres villes.
 POur ces autorités, relative indifférence mais quand en 250, tous les hommes libres sont devenus chrétiens, regain de religions de la cité pour marquer la participation à la citoyenneté, et du coup, grosses persécutions.
 Mais un courant neoplatonicien, anti sacrifice, regain végétarien, pythagoricien.
 \end{Synthesis}
 


\subsection{L’influence de l’idée de « religio »} \label{ChristologieCultureReligio}
 
 Avec Lactance et Augustin, on a christianisé la culture religieuse gréco-romaine, tout en acceptant la matrice de cette culture.

Une vision du monde, \emph{Weltanschauung}.




\section{La « religio » comme culture }

L'Eglise n'a pas seulement remplacé la médiation de la religio mais a assumé le sens de Religio. L'idée de médiation n'est pas tout à fait celle des Romains sur la \textit{religio}.




\subsection{La religion civile}
\paragraph{Religion Civile} pour les romains, il s'agit d'une attitude respectueuse. Cf Ciceron qui rattache \textit{religare} à \textit{attacher, relier}. Mais pour Ciceron, cela vient de \textit{relegere,} qui a donné scrupule, une hésitation qui empêche et non une action qui lie. \textit{Disposition intérieure}. Tout est perçu par le prisme de religio et non d'une réalité objective extérieure.

Augustin était sensible à ce sens : 
\begin{quote}
     « Toutefois comme en latin courant – non celui des gens incultes, mais des plus savants – on dit qu’il faut avoir la religion de la famille, de l’amitié, de toutes les relations sociales, ce mot n’évite pas l’équivoque quand se pose le problème du culte de la déité » (Augustin, Cité de Dieu X,1,3 BA).
\end{quote}
\begin{Ex}
En Français, on a gardé ce sens avec : 
"\textit{il a la religion de la famille}", on est plutot dans le respect, la révérence.

\end{Ex}


\begin{quote}
     «  A partir de ce fondement personnel, religio en vient, avec le temps, à englober la cité tout entière. A l’époque classique, la religio Romana désigne d’abord une attitude, faite de respect scrupuleux envers l’institué… aussi devient-elle ce qui donne force aux institutions et en garantit la durée, par ce lien, par cet attachement du citoyen à respecter les institutions de sa cité » (172-173).
\end{quote}

En étant \textit{scrupuleux, respectueux}, j'en viens à respecter l'institution qui est maintenu. Respect par rapport à la cité, qui se manifeste dans des rites. 

La religion est donc nécessaire à la cohésion de la cité. 
\begin{quote}
    « C’est à la piété collective et institutionnelle (selon Turcan) aux \textit{religiones} de la cité que les Romains attribuaient le succès de leur politique et leur hégémonie universelle » (Sachot173). 
\end{quote}
Le romain était d'abord un citoyen, et le romain était religieux dans le sens qu'il respectait la cité.


\paragraph{Une orthopraxie et non une orthodoxie} il s'agit de faire correctement et non une foi correcte.
\begin{quote}
     « C’est une religion sans révélation, sans livres révélés, sans dogmes et sans orthodoxie. L’exigence centrale est plutôt celle de l’orthopraxie, de l’exécution correcte des rites prescrits » (J. Scheid,  La religion des Romains, Paris, 2019, p. 25). 
\end{quote}


C'est un autre monde, comme la politesse de lever son chapeau. Très éloigné de notre vision de la religion.

\begin{quote}
     « C’est une religion sans initiation ni enseignement. Les devoirs religieux sont imposés à l’individu par la naissance, par l’adoption, l’affranchissement ou la naturalisation, bref ils sont liés au statut social des individus et non à une décision personnelle d’ordre spirituel » (Scheid, 26).
\end{quote}

Ainsi un citoyen est lié intimement à la religion. 
\begin{quote}
    « Tout acte communautaire comporte donc un aspect religieux, et tout acte religieux possède un aspect communautaire » (Scheid, 28).
\end{quote}

\begin{Synthesis}
Pour les Romains, la religio n'a pas pour but d'être en lien avec les Dieux (sauf les neoplatoniciens), mais de souder la cité.
\end{Synthesis}
Dans les actes des martyrs : à la différence de \textit{sumus Civis}, on devient \textit{sumus christiani} ?

\subsection{Le christianisme devient la religion civile}

C'est cela le problème. Implicitement, progressivement, 
Tertullien, \textit{Apologétique}, il va parler de \textit{religio} juive, 

\begin{quote}
    XXXV. Les Chrétiens sont donc les ennemis de l'Etat, parce qu'ils ne rendent point à l'empereur des honneurs illusoires, mensongers, sacrilèges; parce que, disciples de la religion véritable, ils célèbrent les jours de fêtes de l'empereur par une joie tout intérieure, et non par la débauche. Grande preuve de zèle, en effet, que d'allumer des feux et de dresser des tables dans les rues, d'étaler des festins par les places publiques, de transformer Rome en vaste taverne, de faire couler des ruisseaux de vin, de courir çà et là en bandes tumultueuses, l'insulte à la bouche, l'impudence sur le front, la luxure dans le regard! La joie publique ne se manifeste-t-elle que par la honte publique? Ce qui viole les bienséances tout autre jour, deviendra-t-il légitime aux fêtes de l'empereur? Ces mêmes lois, qu'en d'autres temps on observe par respect pour César, faudra-t-il les fouler aux pieds pour l'honorer aujourd'hui! La licence et le dérèglement s'appelleront-ils piété? De scandaleuses orgies passeront-elles pour une fête religieuse? Oh! que nous méritons bien la mort, d'acquitter les vœux pour les empereurs, et de participer à l'allégresse générale sans nous départir de la sobriété, de la chasteté, de la modestie! Quel crime, dans un jour consacré au plaisir, de ne pas ombrager nos portes de lauriers, de ne pas allumer des flambeaux en plein midi! La joie |314 populaire a sanctifié le désordre: rien de plus honnête alors que de décorer sa maison de toutes les apparences d'un lieu de prostitution nouvellement ouvert.
\end{quote}
 
 L'Eglise en dénonçant la religion romaine, va être obligé de la remplacer. 

\begin{quote}
     « Investi par le christianisme, \textit{religio} va recouvrir tout le donné chrétien (Eglises, foi, épiscopat, clergé, corpus scripturaire, messianisme, etc..) ; inversement, investi par la catégorie de \textit{religio}, le christianisme va s’inscrire dans l’espace que ce terme recouvre en latin et être redéfini par lui : la foi du chrétien va être ressaisie dans l’attitude personnelle du citoyen à l’égard de la cité (civitas) et l’institution ecclésiale (ecclesia) va être reprise dans les catégories qui structurent la société en cités » (Sachot,192).
\end{quote}


\section{Annoncer le Christ dans la Chrétienté}

Dans une culture chrétienne, il s'agit moins de continuer à annoncer le Christ que de \textit{justifier la culture par le Christ}. 
Mais la difficulté, c'est d'annoncer la culture et non le Christ, ce qui s'est passé. 


\paragraph{Inculturation} Echange avec la culture.  
\begin{quote}
     « L’inculturation signifie ‘une intime transformation des authentiques valeurs culturelles par leur intégration dans le christianisme \sn{et non pas le Christ !}, et l’enracinement du christianisme dans les diverses cultures humaines’ » (Redemptoris missio\sn{\href{https://www.vatican.va/content/john-paul-ii/fr/encyclicals/documents/hf_jp-ii_enc_07121990_redemptoris-missio.html}{Redemptoris missio} : Grande encyclique missionnaire de Jean-Paul II, \textit{incarner l'Evangile dans la culture des peuples}}, n. 52). 
\end{quote}

On passe peut être un peu trop facilement d'un concept à l'autre.

\paragraph{limites}
\begin{quote}
    54. A ce propos, certaines précisions restent fondamentales. L'inculturation correctement menée doit être guidée par deux principes: «La compatibilité avec l'Evangile et la communion avec l'Eglise universelle»94. Gardiens du «dépôt de la foi», les évêques veilleront à la fidélité et, surtout, au discernement95, ce qui requiert un profond équilibre; car on risque de passer sans analyse critique d'une sorte d'aliénation par rapport à la culture à une surévaluation de la culture, qui est une production de l'homme, et qui est donc marquée par le péché. La culture a besoin, elle aussi, d'être «purifiée, élevée et perfectionnée».
\end{quote}
une culture est toujours limitée et difficile d'enlever la religion de cette culture. Une vraie question.

\section{Les problèmes missionnaires de l’Église enracinée dans une « culture chrétienne » }
\begin{quote}
      « Par l’inculturation, l’Église (…) transmet (aux cultures) ses valeurs, en assumant ce qu’il y a de bon dans ces cultures et en les renouvelant de l’intérieur » (Redemptoris missio, n.52).  
\end{quote}
 


 

 