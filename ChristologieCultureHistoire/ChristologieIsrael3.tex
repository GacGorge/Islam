


 
\chapter{La christologie façonnée dans la tradition d'Israël I}
 
 
 \mn{Christologies et cultures dans l'histoire
3 le 1/2/22 Les christologies du NT}
  % --------------------------------- 
\section{Eléments bibliographiques :}

\emph{Le Nouveau Testament est-il anti-juif ?}, CE 108, Paris 1999.

BROWN, R., \emph{Jésus dans les quatre évangiles, introduction à la
christologie du Nouveau Testament}, Paris, Cerf, coll. Lire la Bible,
1996, 311 p.

CULLMANN, O., \emph{Christologie du Nouveau Testament}, Neuchâtel 1966.
DUPONT, J., \emph{Etudes sur les Actes des Apôtres}, Paris 1967.

ID., \emph{Nouvelles études sur les Actes des Apôtres}, Paris 1984.

EHRMAN, B. D., \emph{Jésus avant les évangiles. Comment les premiers
chrétiens se sont rappelé, ont transformé et inventé leurs histoires du
Sauveur}, tr. par J.-P. PRÊVOST, Paris 2017.

HURTADO, L. W., \emph{Le Seigneur Jésus Christ. La dévotion envers Jésus
aux premiers temps du christianisme}, Paris 2009.

MARGUERAT, D. -- NORELLI, E. -- POFFET, J-M., \emph{Jésus de Nazareth.
Nouvelles approches d'une énigme}, Genève 1998.

RATZINGER, J., \emph{La foi chrétienne hier et aujourd'hui}, Paris
1996\textsuperscript{2}. RUNACHER, C., \emph{Saint Marc. La Bible tout
simplement}, Paris 2001.

SCHNACKENBURG, R. -- SMULDERS, P., \emph{La christologie dans le Nouveau
Testament et le dogme. Mysterium Salutis. Dogmatique de l'histoire du
salut 10}, Paris 1974, 9-226.

SCHRÖTER, J., \emph{Jésus de Nazareth. A la recherche de l'homme de
Galilée}, Genève, 2018. SILLY, R. (dir.), \emph{L'Ecole biblique de
Jérusalem. Dictionnaire Jésus}, Paris 2021.

THEISSEN, G., \emph{La religion des premiers chrétiens}, tr. par J.
HOFFMANN, Paris -- Genève 2002. TRIMAILLE, M., \emph{La christologie de
saint Marc}, Paris 2001.

TROCMÉ, É., \emph{L'évangile selon saint Marc}, Genève 2000.
 
 % ---------------------------------
\hypertarget{introduction}{%
\section{Introduction}\label{introduction}}

 Comment le mystère du Christ s'est exprimé dans l'ambiance culturel du judaisme ?
 Le mystère de Jésus a été annoncée fut donc la culture biblique le judaïsme du 2nd temple et avec aussi sa dimension aussi l'apocalyptique juive tardive. On a retrouvé un certain nombre de textes intertestamentaires  : les manuscrits de la mer morte par exemple pour apporter un certain nombre de connaissances là-dessus sur l'environnement et les mouvements juifs à l'époque. 
 Le concept clé a été celui de Christ Messie :  Jésus est le Christ. Ainsi pour exprimer le mystère de Jésus les christologie du Nouveau Testament ont largement assumé l'idée biblique de Christ messie tout en transformant cette idée à partir de l'histoire et du destin de Jésus Christ.
 \begin{itemize}
     \item les premiers témoins 
     \item comment les disciples de la 2nde génération la 3e ont-ils fait mémoire de l'histoire et du destin de Jésus à partir de leurs traditions religieuses et de la situation dans laquelle ils se trouvaient
 \end{itemize}

   % ------------------------------------------------------
  \section{L'interprétation que les premiers disciples font de Jésus  après Pâques}
   La première génération dit \textbf{de ce que le Christ est devenu} après la Resurrection et non pas sur sa vie même.
 \begin{Ex}
 Si on enlève les Evangiles, on ne garde que quelques fragments historiques : descendant de David (Rm 1, 3), mariage (1 Co 10), la Cène (1 Co) et qu'il est mort Crucifié et qu'il est apparu vivant. 
 \end{Ex}  
 C'est donc à la seconde génération, progressivement, qu'on a reconstitué les récits que nous trouvons dans l'Evangile
   
 On appelle cette théologie, théologie de \textsc{l'exaltation}. Mais il y a un problème de méthode car on a peu de textes de la première génération. Il faut travailler les \textit{matériaux de réemploi}, les hymnes,... qui sont des couches plus anciennes. 
 
 On pense que certaines traditions ont été reprises et intégrées.
   
     
    \subsection{Jésus devient le Christ, le Fils de Dieu}
     
   
     
    \subsection{Les traits constitutifs d'une christologie de l'exaltation}
     
     Selon Schnackenburg, à la Resurrection, Dieu confie une dignité et un pouvoir au Christ : 
     \begin{itemize}
         \item Ps 2, 7 Intronisation royale
         \begin{quote}
             Je publierai le décret; L'Eternel m'a dit: Tu es mon fils! Je t'ai engendré aujourd'hui. 
         \end{quote}
         \item Ps 110
         \item Philippiens : \emph{Kurios}
     \end{itemize}
     
      \paragraph{Intronisation royale} On a donc intronisation et exhaltation. On pense par exemple que Ac 2 est archaique car l'Esprit Saint normalement, c'est au baptème. Tension : 
\begin{quote}
    Ac 2,32-33 : « Ce Jésus, Dieu l'a ressuscité, nous tous en sommes
témoins. Exalté par la droite de Dieu, il a donc reçu du Père l'Esprit
Saint promis et il l'a répandu comme vous le voyez et l'entendez ».
\end{quote}
A partir du Ps 110,1, on va montrer que c'est fondé. cf Ac 2 : 
\begin{quote}
    
Ac 2,34-36 : « David, qui n'est certes pas monté au ciel, a pourtant dit
: Le Seigneur a dit à mon Seigneur : assieds-toi à ma droite jusqu'à ce
que j'aie fait de tes adversaires un escabeau sous tes pieds. Que toute
la maison d'Israël le sache donc avec certitude : Dieu l'a fait et
Seigneur et Christ, ce Jésus que vous, vous aviez crucifié ».
\end{quote}
L'intronisation royale est fondée sur ce Psaume, il était Seigneur et Christ. On identifie Jésus à David, ou plutôt au Messie Royal annoncé à Israel.


\paragraph{Jésus-Christ est \emph{kurios}} Que cela veut dire ? Dieu l'a fait Seigneur et Christ. Dieu est mentionné d'abord car cela découle du Ps 110. Puis, on découle du fait qu'il est Seigneur, le fait qu'il est Messie. 

\paragraph{Engendrement du Christ}  une autre interprétation, celle de l'engendrement du Christ :  
\begin{quote}
Ac 13,33 : « Il a ressuscité Jésus, comme il est écrit au psaume second
: Tu es mon fils, Moi aujourd'hui, je t'ai engendré ».
\end{quote}
On a rapproché la parole à Nathan (2 Sa) : le Roi devient fils de Dieu. le \emph{syntagme} \textit{Fils de Dieu} dans le sens de \textit{Roi}, de \textit{Messie}. \mn{Pour les Juifs, Israel est fils de Dieu, car Dieu a choisi Israel comme son Peuple. Le Roi est une figure individuelle mais qui représente le Peuple. }
Ici, tout se joue au moment de la Resurrection.



 
     
    \subsection{L'originalité de ce Messie}

On a tout de suite rattaché le Christ à l'Ecriture de l'AT. Mais il y a des aspects inattendus : 
\begin{itemize}
    \item La façon dont Jésus atteint la Royauté, non pas par la force mais par la mort infamante sur la Croix. Comment pouvoir expliquer un tel messie. Certes, dans Is 53, on a une image du Christ souffrant. Mais cela reste déroutant.
    \item le salut est specifique en Ac 5, 31. 
    \begin{quote}
        30 Le Dieu de nos pères a ressuscité Jésus, que vous avez tué, en le pendant au bois.
31 Dieu l'a élevé par sa droite comme Prince et Sauveur, pour donner à Israël la repentance et le pardon des péchés.
    \end{quote}
    \item ce Messie Roi est un récit eschatologique. Son élevation correspond à sa resurrection.
    Jacques Dupont : l'Eglise affirme le côté définitif. Mais dans certains passages, la victoire définitive n'est pas tout à fait acquise. Mais dimension eschatologique qui n'était pas dans la Tradition Juive : seul Dieu intervient pour mettre fin à l'histoire. 
    "Fils de l'homme"; Dn 7, 14 Figure de Daniel,  figure divine. 
\end{itemize}   
Les traditions s'agglutinent en Christ et dépassent ce qui était attendu. 


    \subsection{La christologie entre judaïsme et dépassement du judaïsme}
     
   le qualificatif de \textit{fils de Dieu} chez Paul n'a pas le sens de Dieu, mais Kurios a un sens plus fort.  On passe du \textit{théocentrisme} au Christocentrisme. 
   
   \paragraph{Culte binitaire}
   \begin{Def}[binitaire]
   Le Binitarisme est une théologie chrétienne de deux personnes, personas ou aspects dans une substance/ Divinité (ou Dieu).
   
   \end{Def}
   Teischen.
   \paragraph{L'universalisme religieux} : Jésus et ses disciples étaient imprégnés de Judaisme mais très tôt, les disciples vont ouvrir le judaisme dans un sens universel.
   
   \paragraph{Développement des souvenirs du Christ} 

Selon Theissen, on assiste à un processus de théologisation de \textit{Jésus}. On va raconter l'Evangile pour développer la théologie sous jacente. On va reconnaitre que dans son activité terrestre, Jésus est le Messie.   
   % ------------------------------------------------------
  \
  \section{L'interprétation de Jésus de la deuxième génération
  chrétienne : la christologie de
  Marc}


     
    \subsection{Le souvenir de Jésus dans Marc}
     
   Un souvenir est marqué par le présent. Ecrit entre 60 et 70 après Jésus Christ. Le premier texte évangélique. Ce n'est plus une théologie \textit{exaltée} mais un évangile : récit et parole de Jésus. Ce qu'il a fait dans son ministère terrestre. Il est bien le Christ, le Fils de Dieu, pas seulement par sa Résurrection mais \textit{dès son Baptême}. 
     
    \subsection{Le dévoilement de Jésus comme « Fils de Dieu »}
    \paragraph{Comme Columbo} Le lecteur connait des le début la nature du Christ mais pour les disciples, c'est révélé progressivement : 
   
     \begin{quote}
         Mc 1,1 : « Commencement de l'Evangile de Jésus Christ Fils de Dieu ». Mc
1,11 : « Tu es mon Fils bien-aimé, il m'a plu de te choisir ».
     \end{quote}
On insiste aussi sur le Baptême.
\paragraph{la foi des démons} Les démons connaissent la nature de Jésus (une foi pas féconde) : 
\begin{quote}
    Mc 3,11 : « Les esprits impurs, quand ils le voyaient, se jetaient à ses
pieds et criaient : `Tu es le Fils de Dieu'. » 

// Mc 5,7 : « D'une voix
forte il crie : `Que me veux-tu, Jésus, Fils du Dieu Très-Haut ? ».
\end{quote}
\paragraph{Confession de Pierre à Césarée} Important mais il faut garder le secret car ce n'est pas tout à fait la vraie foi pour déconstruire nos représentations.
\begin{quote}
   Mc
8,29 : « Tu es le Christ » (dans Mt 16,16 : « Tu es le Christ, le Fils
du Dieu vivant »).  
\end{quote}
\paragraph{la nuée} Cela vient de Dieu
\begin{quote}
    Mc 9,7 : « Celui-ci est mon Fils bien-aimé. Ecoutez-le ».

\end{quote}

\paragraph{devant le Grand Prêtre}, le représentant d'Israël. il peut le dire car \textit{les jeux sont faits}, on ne peut plus se tromper (la Croix)
\begin{quote}
    Mc 14,61-62 : « De nouveau le Grand Prêtre l'interrogeait ; il lui dit :
`Es-tu le Messie le Fils du Dieu béni ?' Jésus dit : `Je le suis' ».
\end{quote}
\paragraph{Centurion} Face à Jésus, le centurion le confesse comme fils de Dieu, non pas \textit{en dépit} mais \textit{à cause} de sa mort : 
\begin{quote}
    Mc 15,39 : « Vraiment, cet homme était le Fils de Dieu ».
\end{quote}
Il ne faut pas penser ici à Dieu, mais au Messie. Curieux du passé "était". 



\begin{Synthesis}
L'expression \textit{Fils de Dieu} n'est donc pas défini à priori mais par ce que le Christ a vécu dans sa vie : identité narrative. C'est Jésus lui-même qui va définir le sens ultime du Christ.
\end{Synthesis}
A la fois inculturé dans la Foi Juive mais va la dépasser dans sa vie et sa mort. Mais les Juifs ont une pré-compréhension du Messie et peuvent se méprendre.

Ici, on est peut être dans une communauté dans la persécution. Pour reconnaître le Christ comme Christ, faut il aussi être éprouvé ? \mn{Intéressant}
La connaissance de Jésus ce n'est pas uniquement une connaissance livresque, c'est en le suivant, en l'expérimentant dans notre vie et notre chair. 
\begin{Synthesis}
Faire de sa vie une christologie
\end{Synthesis}
      On peut montrer que Messie, on est élu par Dieu, oint par l'onction, et intronisé par les épreuves (comme David). Ici, en Marc, on a cette même démarche. 

    Jésus par sa vie et sa mort transforme la vision du Messie.
    
    
    
    
 
    \subsection{La signification de « Fils de Dieu » et de « Christ » dans Mc}
     
     
     \paragraph{Fils de Dieu / Christ } Que ce signifie Fils de Dieu,  Messie pour les disciples, qui accomplit la mission de Dieu. Expressions synonymes. 
     
     \paragraph{Mais un Messie serviteur et prophète} On va aller un peu plus loin. Le secret messianique (Mc 8,30) s'explique par la manière qu'incarne Jésus du Christ, ne correspond pas aux attentes du Peuple. Mais avec la notion de pouvoir que Jésus ne veut pas. D'où la dimension du Serviteur.
     Le Baptême se lit avec l'arrière plan d'Is 42 : 
     \begin{quote}
         Voici mon serviteur, que je soutiendrai, Mon élu, en qui mon âme prend plaisir. J'ai mis mon esprit sur lui; Il annoncera la justice aux nations. 2Il ne criera point, il n'élèvera point la voix, Et ne la fera point entendre dans les rues.…
     \end{quote}
     En Is 53, le serviteur devient souffrant.
     
     Dans la Transfiguration,  ne répète pas le baptême : "Ecoutez le", dimension prophétique de Jésus (cf Dt 18,15-18).
     
     \begin{Synthesis}
     Jésus apprait comme Obeissant à Dieu, comme serviteur et prophète et non comme un Roi politique
     \end{Synthesis}
     Pourtant, d'autres éléments sont en faveur d'une interprétation transcendante de \textit{fils de Dieu}
     
    \subsection{« Fils de Dieu » dans sa signification eschatologique : le
    Fils de l'homme}
     Selon certains exégètes, il faut penser \textit{fils de Dieu} comme une notion \textit{inclusive}, qui dit quelque chose de la relation à Dieu. Or, dans Mc, la dimension messianique est lié à la dimension eschatologique, qu'on ne trouve pas dans les récits de l'AT. 
     Par exemple, dans le récit du Baptême, les "cieux se déchirent" : Is 63, 19. Dimension eschatologique : Dieu descend et se fait proche de l'homme. 
     
     \begin{quote}
         qui est donc celui là pour que même les eaux lui obeissent ?
         marche sur les eaux
         Ego Eimi. je suis (cf Ex 3, 14)
         Guérison
     \end{quote}
\begin{Synthesis}
    Toutes ces indications soulignent la souveraineté, transcendante de Jésus. Dès le départ, Jésus a le pouvoir divin par une \textit{retroprojection} de la mort au resurrection de Jésus au baptême : une certaine divinisation du Christ. 
\end{Synthesis}

   % ------------------------------------------------------
 
  
   % ---------------------------------
  \section{L'interprétation de Jean : une christologie qui se sépare du
  judaïsme} 
   
   \begin{Synthesis}
       Peu à peu, un mouvement qui prend ses distances avec le judaïsme rabbinique.
       Le mythe et l'histoire deviennent un : \textit{Il est Un avec Dieu}
   \end{Synthesis}
 
 
 \begin{quote}
     Chez Jean, Jésus n'est pas seulement une personne humaine... on se souvient de lui comme beaucoup plus que cela; un être divin, descendu du ciel, ... qui a partagé... quiconque croit en lui puisse avoir la vie éternelle.
     Hermann
 \end{quote}
 
   Une telle interprétation sort-elle de la tradition biblique d'Israël ?
     
    \subsection{Une certaine fidélité de Jean avec la tradition d'Israël}
    
    Jean affirme le \textit{logos divin}. Pour \textit{Theissen}, il n'y a pas d'incompatibilité avec la foi juive. ainsi, \textit{Philon d'Alexandrie}, le \emph{logos} est la façon dont prend chair Dieu. 
    
    \paragraph{Ange du Seigneur} Ce n'est pas Michel, Gabriel, mais \textit{manifestation de Dieu} pour les juifs. Il se manifeste non par sa transcendance mais sa parole, \ldots
    
    \paragraph{Hypostase divine} Selon Hermann, il y avait déjà cette notion d'hypostase divine, attribut : puissant, sage,... mais s'il est sage, cela veut dire qu'il a la sagesse, qui doit être distinct de Dieu. \textsc{Dialectique de ces attributs divins} à la fois Dieu et un élément distinct de Dieu.
    cf Proverbes, où l'on personnifie la Sagesse.
    \begin{quote}
        22 Le Seigneur m’a faite pour lui, principe de son action, première de ses œuvres, depuis toujours.
 Avant les siècles j’ai été formée, dès le commencement, avant l’apparition de la terre.
 Quand les abîmes n’existaient pas encore, je fus enfantée, quand n’étaient pas les sources jaillissantes.
 Avant que les montagnes ne soient fixées, avant les collines, je fus enfantée, avant que le Seigneur n’ait fait la terre et l’espace, les éléments primitifs du monde.
Quand il établissait les cieux, j’étais là, quand il traçait l’horizon à la surface de l’abîme, qu’il amassait les nuages dans les hauteurs et maîtrisait les sources de l’abîme,
 quand il imposait à la mer ses limites, si bien que les eaux ne peuvent enfreindre son ordre, quand il établissait les fondements de la terre.
 Et moi, je grandissais à ses côtés. Je faisais ses délices jour après jour, jouant devant lui à tout moment,
 jouant dans l’univers, sur sa terre, et trouvant mes délices avec les fils des hommes.
 Et maintenant, fils, écoutez-moi. Heureux ceux qui gardent mes chemins !
Proverbes 8, 22-36
    \end{quote}
     
   
     
    \subsection{L'accomplissement de la logique de la résurrection de Jésus}
     
   \paragraph{En Jn, Jésus semble se faire lui-même Dieu}. on passe de l'élévation de Jésus ressuscité à l'incarnation. \mn{par opposition à Marc, où l'Evangile commence au Baptême, et on a un mouvement de l'Homme vers Dieu}. Ici, on a une perspective qui est dès le départ part de Dieu vers l'\textit{homme-Dieu}.
   En Jn, la gloire de Dieu apparaît là où est Jésus : 
   \begin{quote}
       il manifesta sa gloire (en Cana).
       Jn
   \end{quote}
    
    Zumstein \sn{spécialiste de St Jean} \begin{quote}
        le prologue : Jésus \ldots le dédoublement de Dieu au sein du monde.
        
    \end{quote} 
    
    \paragraph{Un regard retrospectif} : on relit la vie de Jésus à la lumière pascale. Jésus est éclairé d'une lumière qu'il n'avait pas au début. La croix devient exaltation, glorification.
    
    \subsection{Le Jésus de Jean : un dédoublement de Dieu au sein du monde}
     
     \paragraph{5 motifs johanniques}
     
     \begin{itemize}
         \item \textsc{Origine divine}.
         \begin{quote}
Jn 7,28-29 : « Vous me connaissez ! Vous savez d'où je suis ! Et
pourtant je ne suis pas venu de moi- même. Celui qui m'a envoyé est
véridique, lui que vous ne connaissez pas. Moi, je le connais parce que
je viens d'auprès de lui et qu'il m'a envoyé ».

Jn 8,12 : « Je sais d'où je viens et où je vais ; tandis que vous, vous
ne savez ni d'où je viens ni où je vais ».\mn{parallèle avec Jn 3, l'Esprit}
         \end{quote}
         Nous sommes surplombants, on sait dès le début qui il est :
         \begin{quote}
             Jn 1,1.14 : « Au commencement était le Verbe et le Verbe était tourné
vers Dieu, et le Verbe était Dieu {[}\ldots{]} Et le Verbe s'est fait
chair et il a habité parmi nous et nous avons vu sa gloire, cette gloire
que, Fils unique plein de grâce et de vérité, il tient du Père ».
         \end{quote}
         
         \item \textsc{Jésus accusé de blasphème}.
         \begin{quote}
             Jn 10,33 : « Toi qui es un homme tu te fais Dieu ».

Jn 10, 36 : « A celui que le Père a consacré et envoyé dans le monde,
vous dites : `tu blasphèmes', parce que j'ai affirmé que je suis le Fils
de Dieu »

Jn 10,10 : « moi et le Père nous sommes un ».
         \end{quote}
         Il apparait aux yeux des hommes se faire à l'égal de Dieu.
         \item \textsc{Jésus, fils unique, monogène}, \textit{Fils de Dieu} choisi par Dieu, élu par Dieu mais en Jean, \textit{Fils unique}, pas tout à fait le même sens : 
         
         \begin{quote}
           
Jn 1,14 : « Cette gloire que, Fils unique plein de grâce et de vérité,
il tient du Père »

Jn 1,18 : « Personne n'a jamais vu Dieu ; Dieu Fils unique qui est dans
le sein du Père, nous l'a dévoilé ».

Jn 3,16 : « Dieu, en effet, a tant aimé le monde qu'il a donné son Fils,
son unique, pour que tout homme qui croit en lui ne périsse pas mais ait
la vie éternelle »  

Jn 3,18 : « Parce qu'il ne pas cru au nom du Fils unique de Dieu » 1 Jn
4,9 : « Dieu a envoyé son Fils unique dans le monde ».
         \end{quote}
         Terme qui singularise Jésus. Dans la rencontre avec Marie-Madeleine, en Jn 19, "dis leur que je monte... "
         Cela n'exclut pas une relation des hommes à Dieu mais pas tout à fait la même (adoptif).
         \item \textsc{Jésus, le Fils}. L'expression \textit{le Fils} semble phagocyter l'expression \textit{Le Fils de Dieu} et \textit{Le fils unique}. Aux yeux de l'Evangéliste, le messie fils de Dieu, l'est parce que qu'il est fils et c'est pour cela qu'il est envoyé. Pour les juifs, cela pose un problème.  
                                
        \item \textsc{Jésus appelé Dieu} \mn{o theos : le père, alors que theos : peut être attribué à d'autres}
           \begin{quote}
               Jn 1,1.18 et 20,28 : Jésus est appelé Dieu.
           \end{quote}                         
         \end{itemize}
     \paragraph{Autorévélation de Dieu}Jésus est le révélateur du Père. Mais comme Dieu se révèle lui-même, il y a équivalence entre Jésus, le révélateur et Dieu, le révélé. \textit{Autorévélation} de Dieu. C'est Dieu qui nous sauve et non un autre.
   
     
    \subsection{Le passage du théocentrisme juif au christocentrisme chrétien}
     
     Si on prend Jean, on passe d'un théocentrisme juif à un christocentrisme chrétien. 
     
     \paragraph{De l'annonce du Règne à l'annonce du christ : les signes} Les \textit{signes} chez Jn ne sont pas l'anticipation du Royaume, elle révèle l'identité de Jésus. Ce ne sont plus des guérisons et des exorcismes, ce sont des signes qui révèlent l'identité propre de Jésus. 
     

   
     
    \subsection{La christologie johannique s'émancipe du judaïsme}
     
        \paragraph{L'unité avec Dieu : objet de sa prédication} l'expression de Loisy\sn{Jésus annonçait le Royaume, et c'est l'Église qui est venue } est d'une certaine façon déjà présent chez Jn.
    \paragraph{Un changement de rite qui s'éloigne du judaïsme} On va changer la façon de ritualiser autour de Jésus. \textit{Hurtado}, le Culte Binitaire.
    Depuis le début, les disciples savent qui est Jésus. Ce n'est que le premier niveau de connaissance que sont les titres traditionnels : \textit{fils de Dieu}, \textit{fils de David}... il n'accomplit pas seulement les attentes d'Israël.
 
\begin{quote}
    Ego Eimi.
\end{quote}
L'écriture est relativisée chez Jn. Alors que chez les juifs, on ne peut pas relativiser la Torah.
Chez Jn, Jésus dit que ce 
\begin{quote}
31 Si c'est moi qui rends témoignage de moi-même, mon témoignage n'est pas vrai. Il y en a un autre qui rend témoignage de moi, et je sais que le témoignage qu'il rend de moi est vrai.
Vous avez envoyé vers Jean, et il a rendu témoignage à la vérité. Pour moi ce n'est pas d'un homme que je reçois le témoignage; mais je dis ceci, afin que vous soyez sauvés. \ldots
Moi, j'ai un témoignage plus grand que celui de Jean; car les oeuvres que le Père m'a donné d'accomplir, ces oeuvres mêmes que je fais, témoignent de moi que c'est le Père qui m'a envoyé.
Et le Père qui m'a envoyé a rendu lui-même témoignage de moi. Vous n'avez jamais entendu sa voix, vous n'avez point vu sa face, et sa parole ne demeure point en vous, parce que vous ne croyez pas à celui qu'il a envoyé.
Vous sondez les Ecritures, parce que vous pensez avoir en elles la vie éternelle: ce sont elles qui rendent témoignage de moi. \ldots Mais si vous ne croyez pas à ses écrits, comment croirez-vous à mes paroles?
    Jn 5, 31
\end{quote}
 
La parole de Jésus a la même puissance que l'écriture qui doit s'accomplir. Jésus est au dessus des Ecritures et légitime les Ecritures. \textit{Logos} qui s'incarne.  

\paragraph{Exclusion de la synagogue} Témoignage de Jn à la fin du siècle : à cette époque, les chrétiens avaient peur des Juif : 
\begin{quote}
Les parents dirent cela parce qu'ils craignaient les Juifs; car les Juifs étaient déjà convenus que, si quelqu'un reconnaissait Jésus pour le Christ, il serait exclu de la synagogue.
    Jn 9,22 l'aveugle de naissance
\end{quote}

Hermann dit que ce verset est plus tardif : La cosmologie de Jean, le monde de Dieu et le monde d'en bas et une opposition. Alors que dans la vision juive, on est plus dans l'horizontalité, avec le temps comme marqueur.
 
Dieu se révèle et agit dans l'histoire mais les disciples se souviennent de Jésus. L'interprétation de Jésus dans la communauté johannique n'est pas seulement une interprétation méditative, mais elle s'est faite dans la confrontation avec les juifs.

La christologie se fait par introspection (par l'Esprit Saint) mais aussi par le dialogue et la confrontation avec les autres. 

 % ---------------------------------
\hypertarget{conclusion}{%
\section{Conclusion}\label{conclusion}}


\begin{Synthesis}
    La Christologie s'est faite avec les catégories de la tradition d'Israël. Or, la lumière de la Résurrection change la vision de Jésus : il a fallu chercher les figures salvifiques de l'AT :
    \begin{itemize}
        \item Roi d'Israël, ...
        \item combiné avec le serviteur souffrant, ... 
        \item puis on a ajouté des attributs de Dieu : sagesse,...
    \end{itemize}
     mais on rentre alors en confrontation avec les Juifs qui refusent ces attributs de Dieu à Jésus.
   
\end{Synthesis}

 Elles parlent aux Juifs mais comment peuvent elles parler aux Nations ? aux Grecs ? 