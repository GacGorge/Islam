% !TEX TS-program = xelatex
% !TEX encoding = UTF-8

% This is a simple template for a XeLaTeX document using the "article" class,
% with the fontspec package to easily select fonts.

%\documentclass[oneside,10pt]{article} % use larger type; default would be 10pt
\documentclass[oneside,10pt]{article}
% other LaTeX packages.....
%
\usepackage{geometry} % See geometry.pdf to learn the layout options. There are lots.
\geometry{letterpaper} % or letterpaper (US) or a5paper or....
%\usepackage[parfill]{parskip} % Activate to begin paragraphs with an empty line rather than an indent

%\geometry{asymmetric}
\geometry{outer=2.5in}
\geometry{marginparwidth=2in}
\geometry{marginparsep=20pt}

% pour mettre des tableaux au bon endroit avec l'option H
%\usepackage{float}
% grands tableaux... pratiques
\usepackage{longtable}
 % pour faire des beaux tableaux
\usepackage{booktabs}
 
 % format des fonts comme Tufte
 \usepackage{xunicode} % Unicode support for LaTeX character names (accents, European chars, etc)
\usepackage{xltxtra} % Extra customizations for XeLaTeX
\usepackage{amsmath}
\usepackage{amsthm}
 \usepackage{fontspec}
\setmainfont[Renderer=Basic, Numbers=OldStyle, Scale = 1.0]{TeX Gyre Pagella}
\setsansfont[Renderer=Basic, Scale=0.90]{TeX Gyre Heros}
\setmonofont[Renderer=Basic]{TeX Gyre Cursor}


%\usepackage{marginnote}
%\renewcommand*{\raggedleftmarginnote}{}
%\renewcommand*{\raggedrightmarginnote}{}

% Margin Caption (done with sidenotes package)
% UTILISER \sidecaption pour une caption
%\usepackage[margincaption,rightcaption,ragged,wide]{sidecap}
%\usepackage[margincaption,outercaption]{sidecap}
%\sidecaptionvpos{figure}{t} 
%\sidecaptionvpos{table}{t}
% format des captions des figures
%\captionsetup[SCfigure]{format=plain, ...}
%\captionsetup[SCtable]{format=plain, ...}

\usepackage{sidenotes}

% bibliography
%\usepackage{natbib}
%\usepackage[notes,backend=biber]{biblatex-chicago}

%\usepackage[style=reading]{biblatex}
\usepackage[citestyle=reading,bibstyle=authortitle]{biblatex}

\addbibresource{Theo.bib}

%\bibliography{sample}
%\bibliography{siam}

%\newcommand*{\sidecite}[1]{\sidenote{[\cite{#1}].\citeauthor{#1} - \citetitle{#1}}


\usepackage{url}
\setlength\parindent{0pt}
% marginnote mn
\newcommand\mn[1]{\marginpar{\footnotesize #1}}

\newcommand\sn[1]{\sidenote{\footnotesize #1}}

%\usepackage{biblatex} %pour citer des numero de page
\usepackage[english,main=french]{babel}
\babelprovide[import]{arabic}
\babelfont[arabic]{rm}{Amiri}
\usepackage{arabtex}

%%% ToC (table of contents) APPEARANCE
\usepackage[nottoc,notlof,notlot]{tocbibind} % Put the bibliography in the ToC
\usepackage[titles,subfigure]{tocloft} % Alter the style of the Table of Contents
\renewcommand{\cftsecfont}{\rmfamily\mdseries\upshape}
\renewcommand{\cftsecpagefont}{\rmfamily\mdseries\upshape} % No bold!
%\newcommand\TArabe[1]{\foreignlanguage{arabic}{\RL}}
\newcommand\TArabe[1]{\foreignlanguage{arabic}{#1}}
\newcommand{\vide}[1]{}
%Recherche \hypertarget et remplacer par \vide
% \protect\hyperlink par \vide
%\texorpdfstring par RIEN
% \RL : \TArabe
% rechercher \footnote{ et remplacer par \sn{
% rechercher Al Gazali
% package pour faire des réferences à des labels pour le chapitre théologiens
\usepackage{cleveref}

% gros tableau
\usepackage{longtable}

\usepackage{eurosym}  %Euro
\usepackage[super]{nth} %for \nth{1} to give 1st
\usepackage{array} % permet de centrer les tableaux\

% Prints the month name (e.g., January) and the year (e.g., 2008)
\newcommand{\monthyear}{%
  \ifcase\month\or January\or February\or March\or April\or May\or June\or
  July\or August\or September\or October\or November\or
  December\fi\space\number\year
}


% Prints an epigraph and speaker in sans serif, all-caps type.
\newcommand{\openepigraph}[2]{%
  %\sffamily\fontsize{14}{16}\selectfont
  \begin{fullwidth}
  \sffamily\large
  \begin{doublespace}
  \noindent\allcaps{#1}\\% epigraph
  \noindent\allcaps{#2}% author
  \end{doublespace}
  \end{fullwidth}
}

% Inserts a blank page
\newcommand{\blankpage}{\newpage\hbox{}\thispagestyle{empty}\newpage}


%\splittopskip=5cm 

 





\usepackage[framemethod=TikZ]{mdframed}

\usepackage{thmtools}
\usepackage{blindtext} % avoid to cut theorem
% avoid to have theorem or definition in the list of theorm
\makeatletter
\patchcmd\thmt@mklistcmd
  {\thmt@thmname}
  {\check@optarg{\thmt@thmname}}
  {}{}
\patchcmd\thmt@mklistcmd
  {\thmt@thmname\ifx}
  {\check@optarg{\thmt@thmname}\ifx}
  {}{}
\protected\def\check@optarg#1{%
  \@ifnextchar\thmtformatoptarg\@secondoftwo{#1}%
}
\makeatother
% format of theorem
\declaretheoremstyle[
    headfont=\scshape, 
    notebraces={\scshape : }{.},
    bodyfont=\normalfont,
    headpunct={},
    postheadspace=\newline,
%    postheadhook={\textcolor{red}{\rule[.6ex]{\linewidth}{0.4pt}}\\},
    spacebelow=\parsep,
    spaceabove=\parsep,
    preheadhook={\begin{mdframed}[backgroundcolor=white!20, 
        splittopskip = \topskip,
            linecolor=blue!30, 
            linewidth = 2pt,
            innertopmargin=6pt,
            roundcorner=1pt, 
            innerbottommargin=6pt, 
            skipabove=\parsep,     
            skipbelow=\parsep]},
            postfoothook=\end{mdframed}]{Definitionstyle}


% example environment - thmtools


\declaretheorem[ style = Definitionstyle, name = {Definition}]{definition}
\declaretheorem[ style = Definitionstyle, name = {Definition}, sibling=definition]{Def}


\declaretheoremstyle[
    headfont=\scshape, 
    notebraces={\scshape : }{.},
    bodyfont=\normalfont,
    headpunct={},
    postheadspace=\newline,
%    postheadhook={\textcolor{red}{\rule[.6ex]{\linewidth}{0.4pt}}\\},
    spacebelow=\parsep,
    spaceabove=\parsep
]{Exercisestyle}
% example environment - thmtools



\declaretheorem[ style = Exercisestyle, name = Exemple ]{Ex}






% Format Thèse ICP
%-------------------------------------------------
%Pour tous vos devoirs écrits ou dissertations, vous devrez vous reporter à cette fiche comportant les normes de présentation et de police.

%1-	La police de caractère : Times New Roman 12



%3-	Les marges :
%a.	A gauche et à droite : 2,5 cm
%b.	En haut : 1,5 cm
%c.	En bas : 2 cm
%-------------------------------------------------
\usepackage{geometry}
\geometry{a4paper, left=2.5cm, right=2.5cm, top=1.5cm,bottom=1.5cm}
%-------------------------------------------------

%\usepackage{booktabs}
\usepackage{setspace}
 
\setlength\parindent{0pt}

%-------------------------------------------------
% Citation
%-------------------------------------------------
 \usepackage{etoolbox}
% \usepackage{csquotes}
%\AtBeginEnvironment{quote}{\par\singlespacing\small}
% \AtBeginEnvironment{quote}{\singlespace\vspace{-\topsep}\small}
%\AtEndEnvironment{quote}{\vspace{-\topsep}\endsinglespace}

\newenvironment{singlequote}
{  \begin{quote}\begin{singlespace}
  \noindent 
}
{ \end{singlespace}
  \end{quote}
}


%\newenvironment{singlequote}  {\quote\small\singlespacingNoVspace} {\endquote}
  


%2-	L’interligne : 1 ½ 
 \onehalfspacing 
%6-	Les citations d’auteurs de plus de 3 lignes doivent être isolées du corps du texte.
%a.	Times New Roman 11
%b.	Interligne 1
%c.	Retrait 1 cm
%d.	Justifié à droite
%-------------------------------------------------
% Geometry (et sidenotes) : v
%-------------------------------------------------

%\usepackage{sidenotes}

%\usepackage{mwe}

%\usepackage[showframe]{geometry}


% option classique
%-------------------------------------------------
% url
%-------------------------------------------------

\usepackage{blindtext}
\usepackage{hyperref}
\usepackage{url}

%-------------------------------------------------
% tableaux
%-------------------------------------------------
\usepackage{booktabs}
%-------------------------------------------------
% caractère
%-------------------------------------------------




\usepackage[sc]{mathpazo}
 

\usepackage{fontspec} % Font selection for XeLaTeX; see fontspec.pdf for documentation
\defaultfontfeatures{Mapping=tex-text} % to support TeX conventions like ``---''


%\setmainfont{Charis SIL} % set the main body font (\textrm), assumes Charis SIL is installed
%\setsansfont{Deja Vu Sans}
%\setmonofont{Deja Vu Mono}

 % format des fonts comme Tufte
 \usepackage{xunicode} % Unicode support for LaTeX character names (accents, European chars, etc)
\usepackage{xltxtra} % Extra customizations for XeLaTeX
\usepackage{amsmath}
\usepackage{amsthm}
%-------------------------------------------------
% pour le chinois
\usepackage{xeCJK}

%-------------------------------------------------
% caractère
%-------------------------------------------------

%\usepackage{biblatex} %pour citer des numero de page
\usepackage[utf8x]{inputenc}

\usepackage[english,main=french]{babel}



\babelprovide[import]{arabic}
\babelfont[arabic]{rm}{Amiri}
\babelprovide[import]{greek}
\babelfont[greek]{rm}{EB Garamond}
% ex
% \foreignlanguage{greek}{Ἰουδαῖοί τε καὶ προσήλυτο.}
%\babelprovide[import]{greek}
%\babelfont[greek]{rm}[RawFeature=+calt]{SimonciniGaramondPro}
\usepackage{arabtex}

%4-	Pas d’encadré, pas de grisé, pas d’à-plat

%5-	Le texte et les notes doivent être justifiés à droite et chaque alinéa doit être indenté.



%\begin{singlespace}…\end{singlespace}
%\begin{spacing}{2.5}
%\end{spacing}


%7-	Les termes et expressions dans une autre langue que le français doivent être en italique et traduits.

%8-	Pour les citations en grec ou en hébreu, il est recommandé d’utiliser la même police que le corps du texte. Le grec doit comporter accents et esprits. L’hébreu ne doit pas comporter de signes d’accentuation et ne doit pas nécessairement être vocalisé.

%9-	La pagination est en bas de page et centrée.

%10-	Les pages se suivent soit en recto, soit en recto-verso.


%12-	Les cartes, graphiques, tableaux et documents divers doivent être numérotés et reportés dans autant de tables qu’il y a de types de documents.



%-------------------------------------------------
% bibliography
%-------------------------------------------------
% 13-	Les index comportent habituellement les auteurs et les références bibliques. D’autres tables d’index peuvent être créées selon les options à répertorier.

% 
%\usepackage{natbib}
\usepackage[square,numbers]{natbib} % permet de mettre des parenthèses et des nombres
%\usepackage{natbib} %A REMETTRE
%\bibliographystyle{unsrtnat} %A REMETTRE
%\bibliographystyle{kluwer} %A REMETTRE
%\bibliographystyle{natdin-icp} %A REMETTRE FORMAT ICP FAIT MAISON
\bibliographystyle{dinat}

%\bibliographystyle{natdin-icp} %A REMETTRE

%\bibliographystyle{abbrvnat}


%14-	Dans les indications bibliographiques, les titres des livres et des revues doivent être en italique.

%15-	L’écrit ou la dissertation commence par la page de garde et le texte commence avec l’introduction.

%16-	A la fin de l’écrit ou de la dissertation, on doit trouver dans l’ordre :
%a-	Les annexes, 
%b-	La bibliographie, 
%c-	Les tables et index et enfin 
%d-	La table de matières complète avec les numéros de pages.

%17-	Les annexes éventuelles sont présentées dans un volume à part selon leur importance.

 %--------------------------------------------------------------
% Table des matières
%--------------------------------------------------------------
 \usepackage{titletoc}
%%%%% TABLE OF CONTENTS
\setcounter{tocdepth}{1}

\usepackage{etoc}
%%% ToC (table of contents) APPEARANCE
%\usepackage[nottoc,notlof,notlot]{tocbibind} % Put the bibliography in the ToC
%\usepackage[titles,subfigure]{tocloft} % Alter the style of the Table of Contents

\usepackage{cleveref} % referece



\usepackage{eurosym}  %Euro
\usepackage[super]{nth} %for \nth{1} to give 1st
\usepackage{array} % permet de centrer les tableaux\

% Prints the month name (e.g., January) and the year (e.g., 2008)




%\splittopskip=5cm 

 
%-------------------------------------------------
% édition
%-------------------------------------------------
\usepackage{comment}

%-------------------------------------------------
% multi colonnage
%-------------------------------------------------
\usepackage{multicol}
%-------------------------------------------------
% Note de marge
%-------------------------------------------------
%11-	La numérotation des notes de bas de pages se suit sur l’ensemble de l’écrit.

 \newcommand{\mn}[1]{\footnote{\footnotesize #1}}
  \newcommand{\sn}[1]{\footnote{\footnotesize #1}}
 

%--------------------------------------------------------------
% Frame
%--------------------------------------------------------------

\usepackage[framemethod=TikZ]{mdframed}

\usepackage{thmtools}
%\usepackage{amsthm}

\usepackage{blindtext} % avoid to cut theorem
% avoid to have theorem or definition in the list of theorm
\makeatletter
\newcommand{\theosep}{\parsep}
\renewcommand{\theosep}{20pt}


%--------------------------------------------------------------
% Titre des listes de théorèmes
%--------------------------------------------------------------

\renewcommand{\listtheoremname}{List of Important Theorems}

\makeatletter
\def\ll@theorem{%
  \protect\numberline{\csname the\thmt@envname\endcsname}%
  \ifx\@empty\thmt@shortoptarg
    \thmt@thmname
  \else
    \thmt@shortoptarg
  \fi}
\def\l@thmt@theorem{} 
 \makeatother
 

% avoid to have theorem or definition in the list of theorm
\makeatletter
\patchcmd\thmt@mklistcmd
  {\thmt@thmname}
  {\check@optarg{\thmt@thmname}}
  {}{}
\patchcmd\thmt@mklistcmd
  {\thmt@thmname\ifx}
  {\check@optarg{\thmt@thmname}\ifx}
  {}{}
\protected\def\check@optarg#1{%
  \@ifnextchar\thmtformatoptarg\@secondoftwo{#1}%
}

 
\makeatother

% format of theorem


            
\declaretheoremstyle[
    headfont=\scshape, 
    notebraces={\scshape : }{.},
    bodyfont=\normalfont,
    headpunct={},
    postheadspace=\newline,
%    postheadhook={\textcolor{red}{\rule[.6ex]{\linewidth}{0.4pt}}\\},
    spacebelow=\parsep,
    spaceabove=\parsep,
    mdframed={
            backgroundcolor=white!20, 
            splittopskip = \topskip,
            linecolor=blue!30, 
            linewidth = 2pt,
            innertopmargin=\myinnertopmargin,
            roundcorner=1pt, 
            innerbottommargin=6pt, 
            skipabove=\parsep,     
            skipbelow=\parsep} 
    ]{Definitionstyle}
    
\declaretheoremstyle[
    headfont=\scshape, 
    notebraces={\scshape : }{.},
    bodyfont=\normalfont,
    headpunct={},
    postheadspace=\newline,
%    postheadhook={\textcolor{red}{\rule[.6ex]{\linewidth}{0.4pt}}\\},
    spacebelow=\parsep,
    spaceabove=\parsep,
    mdframed={backgroundcolor=white!20, 
            splittopskip = \topskip,
            linecolor=red!30, 
            linewidth = 2pt,
            innertopmargin=\myinnertopmargin,
            roundcorner=1pt, 
            innerbottommargin=6pt, 
            skipabove=\parsep,     
            skipbelow=\parsep} 
    ]{Propertystyle}

%,    postfoothook=
% example environment - thmtools
\declaretheoremstyle[
    headfont=\scshape, 
    notebraces={\scshape : }{.},
    bodyfont=\normalfont,
    headpunct={},
    postheadspace=\newline, 
%    postheadhook={\textcolor{red}{\rule[.6ex]{\linewidth}{0.4pt}}\\},
    spacebelow=\parsep,
    spaceabove=\parsep
]{Exercisestyle}
% example environment - thmtools

\declaretheoremstyle[
    headfont=\scshape, 
    notebraces={\scshape : }{.},
    bodyfont=\normalfont,
    headpunct={},
    postheadspace=\newline,
%    postheadhook={\textcolor{red}{\rule[.6ex]{\linewidth}{0.4pt}}\\},
    spacebelow=\parsep,
    spaceabove=\parsep,
    mdframed={backgroundcolor=gray!30, 
            splittopskip = \topskip,
            linecolor=red!30, 
            linewidth = 2pt,
            innertopmargin=\myinnertopmargin,
            roundcorner=1pt, 
            innerbottommargin=6pt, 
            skipabove=\parsep,     
            skipbelow=\parsep} 
    ]{TobeRetainedstyle}




\declaretheorem[ style = TobeRetainedstyle, name = {Synthèse} ]{Synthesis}


\declaretheorem[ style = Exercisestyle, numbered=no,name = Property]{property}
\declaretheorem[ style = Propertystyle, name = {Property} ]{Prop}
\declaretheorem[ style = Propertystyle, name = Theorem, sibling=Prop]{Theo}
\declaretheorem[ style = Propertystyle, name = Theorem, sibling=Prop]{theorem}
\declaretheorem[ style = Propertystyle, name = Lemma, sibling=Prop]{lemma}
\declaretheorem[ style = Exercisestyle, numbered=no,name = {Remark}]{rem}
\declaretheorem[ style = Definitionstyle, name = {Definition}]{definition}
\declaretheorem[ style = Definitionstyle, name = {Definition}, sibling=definition]{Def}
\declaretheorem[ style = Exercisestyle, name = Exercise]{exercise}
\declaretheorem[ style = Exercisestyle, name = Exercise, sibling=exercise]{Exercise}
\declaretheorem[ style = Exercisestyle, name = Exercise, sibling=exercise]{Exc}
\declaretheorem[ style = Exercisestyle, name = Exercise, sibling=exercise]{Exo}
\declaretheorem[ style = Exercisestyle, name = Problem, sibling=exercise]{problem}
\declaretheorem[ style = Exercisestyle, name = Example]{example}
\declaretheorem[ style = Exercisestyle, name = Example, sibling=example]{Ex}
\makeatother


%--------------------------------------------------------------
% Code
%--------------------------------------------------------------

%% Permet de mettre du code
\usepackage{listings}
\lstdefinestyle{mystyle}{
    basicstyle=\ttfamily\footnotesize,
    breakatwhitespace=false,         
    breaklines=true,                 
    captionpos=b,                    
    keepspaces=true,                 
    numbers=left,                    
    numbersep=5pt,                  
    showspaces=false,                
    showstringspaces=false,
    showtabs=false,                  
    tabsize=2
}
\lstset{%
	aboveskip=\topsep,
	belowskip=\topsep,
	xleftmargin=\parindent}

\lstset{style=mystyle}





\newcommand{\bi}{\begin{itemize}}
 \newcommand{\ei}{\end{itemize}}
  \newcommand{\be}{\begin{Ex}}
 \newcommand{\ee}{\end{Ex}}
 \newcommand{\mn}[1]{\marginnote{\footnotesize #1}}
  \newcommand{\sn}[1]{\sidenote{\footnotesize #1}}

\newcommand{\mzt}{\emph{muʿtazilite}}  
\newcommand{\CD}{\emph{la Cité de Dieu }}  

\newcommand{\CB}{\emph{Cedric Baylocq }} % nom du professeur
\newcommand{\riba}{\emph{Ribâ }}
\newcommand{\gharar}{\emph{Gharar }}


%Recherche \hypertarget et remplacer par \vide
% \protect\hyperlink par \vide
%\texorpdfstring par RIEN
% \RL : \TArabe
% rechercher \footnote{ et remplacer par \sn{
% rechercher Al Gazali
%\newcommand\TArabe[1]{\foreignlanguage{arabic}{\RL}}
\newcommand\TArabe[1]{\foreignlanguage{arabic}{#1}}
\newcommand\TGrec[1]{\foreignlanguage{greek}{#1}}
\newcommand{\vide}[1]{}

\renewcommand{\listtheoremname}{Liste des Definitions}

% Prints the month name (e.g., January) and the year (e.g., 2008)
\newcommand{\monthyear}{%
  \ifcase\month\or January\or February\or March\or April\or May\or June\or
  July\or August\or September\or October\or November\or
  December\fi\space\number\year
}

\newcommand{\tnote}{\textsuperscript}


% Inserts a blank page
\newcommand{\blankpage}{\newpage\hbox{}\thispagestyle{empty}\newpage}


% Prints an epigraph and speaker in sans serif, all-caps type.
\newcommand{\openepigraph}[2]{%
  %\sffamily\fontsize{14}{16}\selectfont
  \begin{fullwidth}
  \sffamily\large
  \begin{doublespace}
  \noindent\allcaps{#1}\\% epigraph
  \noindent\allcaps{#2}% author
  \end{doublespace}
  \end{fullwidth}
}
 


 

% Prints argument within hanging parentheses (i.e., parentheses that take
% up no horizontal space).  Useful in tabular environments.
\newcommand{\hangp}[1]{\makebox[0pt][r]{(}#1\makebox[0pt][l]{)}}
\newcommand{\hangstar}{\makebox[0pt][l]{*}}
%%
% Prints an asterisk that takes up no horizontal space.
% Useful in tabular environments.



% Macros for typesetting the documentation
\newcommand{\hlred}[1]{\textcolor{Maroon}{#1}}% prints in red
\newcommand{\hangleft}[1]{\makebox[0pt][r]{#1}}
\newcommand{\hairsp}{\hspace{1pt}}% hair space
\newcommand{\hquad}{\hskip0.5em\relax}% half quad space

\newcommand{\ie}{\textit{i.\hairsp{}e.}\xspace}
\newcommand{\eg}{\textit{e.\hairsp{}g.}\xspace}
\newcommand{\na}{\quad--}% used in tables for N/A cells

% Prints an epigraph and speaker in sans serif, all-caps type.





%%
\usepackage{graphicx} % support the \includegraphics command and options
\usepackage{comment}
\usepackage[    bibstyle=numeric,    sorting=nyt]{kaobiblio}
\addbibresource{zotero.bib} % Bibliography file
\addbibresource{Theo.bib} % Bibliography file
%\date{} % Activate to display a given date or no date (if empty),
         % otherwise the current date is printed 

%\title{Le dialogue des Religions et \textit{Laudato Si}}
 
%\title{Les \textit{relais} de \LS : L’enjeu de la réception dans le temps \\ Enquête sur la revue \RLimite}
%\title{Le \textit{style} de Laudato Si \\ \large Pour une théologie du dialogue des religions \\ \large }
\title{Urgences pastorales du moment présent \\Christoph Theobald}

%\author{\normalsize ISTR 2023 - Devoir de M1 \\ \normalsize Guillaume Gorge}
\author{Guillaume Gorge}
%\date{12 décembre 2022}
\begin{document}
 
%\citestyle{verbose}


% ENLEVER LES COMMENTAIRES Pour mettre le titre
\maketitle


%-------------------------------------



\pagenumbering{arabic} 
\setcounter{page}{1}
 



%\chapter{Les religions comme des cultures }

\mn{le 9/5/22}

\section{Bibliographie}
CHENO, R., Dieu au pluriel. Penser les religions, Cerf, Paris, 2017. 

CDF, Dominus Iesus, 2000. 

CTI, Le christianisme et les religions, 1997 

DUBUISSON, D., L’invention des religions, Paris, CNRS éditions, 2020. 

DUPUIS, J. Vers une théologie chrétienne du pluralisme religieux, Cerf, Paris, 1997. 

DURAND, M-L., « Le rapport Église/peuple juif comme paradigme du rapport aux autres religions et quasi-religions séculières ? » dans E. PISANI (dir.), Maximum illud. Aux sources d’une nouvelle ère missionnaire, Paris, Cerf, 2020, p. 141-152. 

DURKHEIM, E., Les formes élémentaires de la vie religieuse, PUF, Paris, 20086. 

ELIADE, M., Aspects du mythe, Gallimard, Paris, 1963. 

FEDOU, M., Les religions selon la foi chrétienne, Paris, Cerf, 1996. 

GUÉ, X., « Comment la théologie postlibérale pense-t-elle l’universalité des vérités religieuses ? Une relecture de La nature des doctrines de G. Lindbeck » dans E. PISANI (dir.), Les doctrines religieuses sont-elles condamnées à s’opposer ? Actes du colloque de l’ISTR des 6 et 7 février 2020, Paris, Cerf, 2021, 25-51. 

JEAN-PAUL II, Lettre encyclique Redemptoris missio, 1990. 

LINDBECK, G. A., La nature des doctrines. Religion et théologie à l’âge du postlibéralisme, tr. de M. HEBERT, Paris, Van Dieren éditeur, 2002. 

PHILIPS, l’Église et son mystère au IIe concile du Vatican, t. 1, Desclée, Paris 1967 

SALENSON, C., « La Missio Dei » dans E. PISANI (dir.), Maximum illud. Aux sources d’une nouvelle ère missionnaire, Paris, Cerf, 2020, p. 125-140. 

VILLEMIN, L. et CHEVALLIER, G. , « La distinction « incorporé à / ordonné à » dans Lumen Gentium : quelles conséquences pour la compréhension du rapport Eglise / Royaume ? », in Christoph THEOBALD (dir.), Pourquoi l’Eglise ? La dimension ecclésiale de la foi dans l’horizon du salut, Paris : Bayard, 2014, pp. 165-196. 

WILLAIME, J-P., Sociologie des religions, PUF, Paris, 20125. 



\section{Introduction}
\paragraph{la religion comme socialisation} en répondant aux grandes questions

\begin{quote}

À notre époque où le genre humain devient de jour en jour plus étroitement uni et où les relations entre les divers peuples se multiplient, l’Église examine plus attentivement quelles sont ses relations avec les religions non chrétiennes. Dans sa tâche de promouvoir l’unité et la charité entre les hommes, et aussi entre les peuples, elle examine ici d’abord ce que les hommes ont en commun et qui les pousse à vivre ensemble leur destinée.

Tous les peuples forment, en effet, une seule communauté ; ils ont une seule origine, puisque Dieu a fait habiter tout le genre humain sur toute la face de la terre [1] ; ils ont aussi une seule fin dernière, Dieu, dont la providence, les témoignages de bonté et les desseins de salut s’étendent à tous [2], jusqu’à ce que les élus soient réunis dans la Cité sainte, que la gloire de Dieu illuminera et où tous les peuples marcheront à sa lumière [3].

Les hommes attendent des diverses religions la réponse aux énigmes cachées de la condition humaine, qui, hier comme aujourd’hui, agitent profondément le cœur humain : Qu’est-ce que l’homme? Quel est le sens et le but de la vie? Qu’est-ce que le bien et qu’est-ce que le péché? Quels sont l’origine et le but de la souffrance? Quelle est la voie pour parvenir au vrai bonheur? Qu’est-ce que la mort, le jugement et la rétribution après la mort ? Qu’est-ce enfin que le mystère dernier et ineffable qui embrasse notre existence, d’où nous tirons notre origine et vers lequel nous tendons ?
    Nostra aetate 1
\end{quote}

\subsection{Mircea Eliade et la spécificité du phénomène religieux }

\paragraph{Le phénomène religieux }

\begin{quote}
    « Le mythe raconte une histoire sacrée ; il relate un événement qui a eu lieu dans le temps primordial, le temps fabuleux des ‘commencements’. Autrement dit, le mythe raconte comment, grâce aux exploits des Etres Surnaturels, une réalité est venue à l’existence (…). C’est donc toujours le récit d’une ‘création’ : on rapporte comment quelque chose a été produit, a commencé à être. Le mythe ne parle que de ce qui est arrivé réellement, de ce qui s’est pleinement manifesté (…). En somme, les mythes décrivent les diverses, et parfois dramatiques, irruptions du sacré (ou du ‘sur-naturel’) dans le Monde. C’est cette irruption du sacré qui fonde réellement le Monde et qui le fait tel qu’il est aujourd’hui. Plus encore : c’est à la suite des interventions des Etres Surnaturels que l’homme est ce qu’il est aujourd’hui, un être mortel, sexué et culturel » (Eliade, Aspects du mythe, 16-17). 
\end{quote}

\paragraph{Le mythe à l’origine de la culture et de la société }

\begin{quote}
    « Du fait que le mythe relate les gesta des Etres Surnaturels et la manifestation de leurs puissances sacrées, il devient le modèle exemplaire de toutes les activités humaines significatives » (Eliade, Aspects, 17-18).  
\end{quote}

\begin{quote}
    « Pour l’homme des sociétés archaïques (…) ce qui s’est passé ab origine est susceptible de se répéter par la force des rites. L’essentiel est donc, pour lui, de connaître les mythes. Non seulement parce que les mythes lui offrent une explication du Monde et de son propre mode d’exister dans le Monde, mais surtout parce que, en se les remémorant, en les réactualisant, il est capable de répéter ce que les Dieux, les Héros ou les Ancêtres ont fait a l'origine. Connaître les mythes, c’est apprendre le secret de l’origine des choses. En d’autres termes, on apprend non seulement comment les choses sont venues à l’existence, mais aussi où les trouver et comment les faire réapparaître lorsqu’elles disparaissent » (Eliade, Aspects, 26). 
\end{quote}

\subsection{Durkheim et la « sociogenèse » de la religion }

Thèse sur le totem en Australie, \textit{idéalisation de la société}. 
Les formes élémentaires de la religion

\paragraph{Distinction entre profane et le sacré} On se comporte différemment. La religion génère du sacré : 
\begin{quote}
    « Toutes les croyances religieuses connues (….) supposent une classification  des choses (…) que se représentent les hommes, en deux classes, en deux genres opposés (…) le profane et le sacré. (…) Les croyances, les mythes (…) sont ou des représentations ou des systèmes de représentations qui expriment la nature des choses sacrées, les vertus et les pouvoirs qui leur sont attribués, leur histoire, leurs rapports les unes avec les autres et avec les choses profanes » (Durkheim, 50-51).  
\end{quote}

Ce sont les hommes qui ont séparés ces deux mondes. Le monde séparé entre profane et religieux.

\paragraph{Définition de la religion}
\begin{quote}
« Une religion\sn{Durkheim, Les formes élémentaires
de la vie religieuse, 65} est un système solidaire de croyances et de pratiques relatives à des choses sacrées, c’est-à-dire séparées, inter-
dites, croyances et pratiques qui unissent en une même communauté
morale, appelée Église, tous ceux qui y adhèrent »
\end{quote}

\paragraph{Distinction entre Religion et Magie}
La magie ne fait pas \textit{corps}, \textit{Eglise}. Les clients d'un magicien peuvent n'avoir aucun lien entre eux. 
\begin{quote}
    « Les croyances proprement religieuses sont toujours communes à une collectivité déterminée qui fait profession d’y adhérer et de pratiquer les rites qui en sont solidaires. (…) \textbf{(Les croyances) sont la chose du groupe et elles en font l’unité.} Les  individus qui composent (la collectivité) se sentent liés les uns aux autres, par cela seul qu’ils ont une foi commune. Une société dont les membres sont unis parce qu’ils se représentent de la même manière le monde sacré et ses rapports avec le monde profane, et parce qu’ils traduisent cette représentation commune dans des pratiques identiques, c’est ce qu’on appelle une Église » (Durkheim, 60). 
\end{quote}

La religion génère les groupements, elle est par nature collective.


\paragraph{la société est la source de la religion} Pourquoi ont ils fait la distinction entre profane et sacré ? Alors que le sensible ne permet pas de faire cette distinction ?

\begin{quote}
    « Nous avons montré quelles forces morales (la société) développe et comment elle éveille ce sentiment d’appui, de sauvegarde, de dépendance tutélaire qui attache le fidèle à son culte. C’est elle qui l’élève au-dessus de lui-même : c’est même elle qui le fait. Car ce qui fait l’homme, c’est cet ensemble de bien intellectuels qui constitue la civilisation, et la civilisation est l’œuvre de la société.[…] Pour que les principaux aspects de la vie collective aient commencé par n’être que des aspects variés de la vie religieuse, il faut évidemment que la vie religieuse soit la forme éminente et comme une expression raccourcie de la vie collective tout entière. \textsc{Si la religion a engendré tout ce qu’il y a d’essentiel dans la société, c’est que l’idée de la société est l’âme de la religion} » (Durkheim, 598-599). 
\end{quote}

\begin{Synthesis}
la société élève l'homme, qui permet de se dépasser. 
\end{Synthesis}



\begin{quote}
    « Pour que la société puisse prendre conscience de soi et entretenir, au degré d’intensité nécessaire, le sentiment qu’elle a d’elle-même, il faut qu’elle s’assemble et se concentre. Or, cette concentration détermine une exaltation de la vie morale qui se traduit par un ensemble de conceptions idéales où vient se peindre la vie nouvelle qui s’est ainsi éveillé (…) Une société ne peut ni se créer ni se recréer sans, du même coup, créer l’idéal. (…) La société idéale n’est pas en dehors de la société réelle ; elle en fait partie. (…)\textsc{ Car une société n’est pas simplement constituée par la masse des individus qui la composent (…) mais, avant tout, par l’idée qu’elle se fait d’elle-même.} »(Durkheim, 603-604). 
\end{quote}

\begin{Def}[Religion]
On va sacraliser la société, une hypostase qui a une conscience.  Appartenir à cette société hypostasiée. Elle génère la religion. 
\end{Def}

Et comment fait on dans une société pluraliste ?
\begin{Ex}
Lors de la séparation de l'Eglise et de l'Etat, la société a essayé de singer la religion, avec l'instituteur comme \textit{prêtre}.
\end{Ex}

\paragraph{La religion comme un facteur de cohésion et une force sociale } Force sociale pour supporter les épreuves.
\begin{quote}
    « Le fidèle qui a communié avec son dieu n’est pas seulement un homme qui voit des vérités nouvelles que l’incroyant ignore ; c’est un homme qui peut davantage. Il sent en lui plus de force soit pour supporter les difficultés de l’existence, soit pour les vaincre » (Durkheim, 595).  
\end{quote}

\paragraph{Critique de la théorie sociologique de Durkheim } Première critique, Dieu n'est pas pris en charge dans la religion. Dieu serait une projection de la société. Feuerbach : 
\begin{quote}
    projection d'une... divinisé
\end{quote}

Par ailleurs, la religion est une instance critique de la société. On ne peut pas la réduire à une simple projection idéale de la société. Et comment on fait en cas de société pluri-confessionnelle ?


\begin{quote}
    « Les hommes vivent sur la même planète, mais dans des mondes différents. Chaque culture (…) conçoit en effet le monde à sa manière » (Dubuisson, 211). 
\end{quote}

\begin{quote}
    « Toutes les cultures, toutes les sociétés ont conçu le monde, l’univers à leur manière. Et (…) toujours en ont imaginé la genèse dans un mythe ou une série de mythes cosmogoniques. Chacune de ces conceptions est différente des autres, mais toutes se ressemblent en ce sens où chaque culture vit (…) dans son propre monde et que ces mondes remplissent partout les mêmes fonctions » (Dubuisson, 212). 
\end{quote}

La langue est toujours dans une culture, et la religion est aussi insérée.

\section{L’approche théologique post-libérale des religions }

\subsection{La critique de l’idée d’une essence commune à toutes les religions}
\paragraph{La remise en cause question d’un « lieu tiers } jusqu'à présent, on a essayé de trouver un PPDC : expérience spirituelle,...

\begin{quote}
    « Il suffirait, selon les pluralistes, de débarrasser chaque religion de son appareil dogmatique et symbolique propre pour retrouver ce lieu tiers, cet arrière-plan commun. Le théologien pluraliste, c’est celui qui sait dépasser sa propre tradition pour s’enraciner dans ce lieu tiers d’où il observe toutes les religions et contemple leur convergence » (Chéno, 111-112). 
\end{quote}

Il s'agit de changer cette approche et de se remettre au sein de notre identité, dans notre religion.
De plus, cette mise en aplomb, trouvant un terrain commun, est une version occidentale, mais qui ne respecte pas forcément l'autre.


\paragraph{Le tournant opéré par G. Lindbeck  } Un des initiateurs du \textit{post-libéralisme}, contre le libéralisme (protestant). Le courant libéral dans le catholicisme était attenué par la force du magistère.
Lindbeck a travaillé au concile.  En 1984, il publie la \textit{nature des doctrines}. \textit{Comment on conçoit la vérité en Christianisme ?}

 
 \begin{quote}
     « On considère (…) les religions comme des idiomes différents permettant d’interpréter la réalité, d’exprimer l’expérience et d’organiser la vie (…) Ainsi, les questions qui se posent quand on compare les religions concernent tout d’abord l’adéquation de leurs catégories » (Lindbeck p. 55). 
 \end{quote}
 Mon voisin fait une expérience ineffable mais va le dire de façon différente. Quelle est la vérité noumenale ? Efficacité de leur symbole : mesurera la qualité de la répone. 
  \paragraph{Approche culturo linguisitique}
 

  Le religions sont comme des archipels.
 
 \subsection{Les religions sont incommensurables quant à leur contenu}



 
 \paragraph{Dialoguer entre religion est il pertinent ? }
 
  \begin{quote}
     Qu’une religion soit raisonnable [donc universelle] dépend largement de ses pouvoirs d’assimilation, de sa capacité à fournir dans ses propres termes une interprétation intelligible des diverses situations et réalités que rencontrent ses adhérents. Les religions que nous qualifions de primitives échouent régulièrement à ce test quand elles sont confrontées à des changements importants, tandis que les religions mondiales développent de plus grandes ressources pour faire face aux vicissitudes (Lindbeck, 175). 
 \end{quote}
 
 Toute religion est comme une langue et donc universelle.
 
  \subsection{Les religions sont comparables quant à leur fonction}
  
  Lindbeck ne se met dans une confession particulière : ni chrétienne, ni théologique, mais scientifique et sociologique. Décentrement. 
  Il les définit comme : 
  
   \begin{quote}
     « Les religions […] ressemblent à des langues, ce qui les assimile à des cultures (dans la mesure où ces cultures sont comprises de manière sémiotique comme des systèmes de réalités et de valeurs, c’est-à-dire comme des idiomes qui permettent de construire la réalité et de vivre au quotidien). »  (Lindbeck, p. 16). 
 \end{quote}
  
  \begin{Def}[religion pour Lindbeck]
  Les religions […] ressemblent à des langues.
  \end{Def}
  
  Des mythes, des récits (ex : Gn 1 permet de comprendre le monde, moi-même). Par exemple, pour dire une conversion, on va reprendre le schéma de Charles de Foucauld ou Paul. Jamais pure innovation.
  \paragraph{Image des lunettes} Les religions sont des lunettes à travers lesquelles on voit le monde. 
  
  Ce n'est pas l'idée qui précède le langage, c'est le langage qui nous précède et nous utilisons le langage qui nous est disponible. 
  
  \paragraph{Fécondité performatrice des religions}
  En se référant à Luther, la vérité se vit parce qu'elle est reçue qui y adhère. 
  
  

\begin{quote}
    dire la vérité religieuse \ldots c'est s'engager dans un style de vie.
\end{quote}
 Pas une définition théorique de la vérité, mais une vérité pratique.
  
  \subparagraph{Performatrice} \textit{Je promets de te rester fidèle}. 
  
  \subparagraph{Pour découvrir la tradition de l'autre, rencontrer quelqu'un qui essaye de conformer sa vie à sa tradition} Sinon, on risque d'avoir une mauvaise vision de l'autre.
  
  
  
  Si la religion permet de se comprendre, pourquoi il n'y aurait pas des religions plus universelles que d'autres ? Si on voit que certains sont heureux, il y a une dimension universelle ? Un aspect qui dépasse l'individu, qui serait \textsc{partageable}.
  
  \subparagraph{image de la langue} certaines langues nous ouvrent au monde, qui nous permette d'accueillir de nouvelles experiences du monde, et d'autres moins universelles que d'autres. C'est Lindberg qui dit cela, après avoir dit l'individualité.
  
  \begin{quote}
      Qu’une religion soit raisonnable [donc universelle] dépend largement de ses pouvoirs d’assimilation, de sa capacité à fournir dans ses propres termes une interprétation intelligible des diverses situations et réalités que rencontrent ses adhérents. Les religions que nous qualifions de primitives échouent régulièrement à ce test quand elles sont confrontées à des changements importants, tandis que les religions mondiales développent de plus grandes ressources pour faire face aux vicissitudes (Lindbeck, 175)
  \end{quote}
  
La vérité des religions est testée dans l'histoire. Il y a des traditions religieuses qui ont disparu. On verra à la fin de l'histoire, de façon eschatologique, la vérité des religions du jour. Si cela dure, il y a quelque chose de solide. 

  \paragraph{Fécondité sociale et universalité} 
  
  un autre aspect de la fécondité envisagé par Lindbeck, c'est la fécondité sociale.  Il réaffirme que la religion socialise les hommes, 
  
  \subparagraph{contraste avec l'approche libérale,} où des milliers d'hommes sont contraints à se lancer individuellement dans un supermarché religieux \sn{Petit bambou, retraite dans une abbaye,...}. Il ne s'agit pas de vivre dans son coin sa relation à Dieu mais de vivre socialement. 
  
  \subparagraph{Religion : un tout cohérent} Dans le Christianisme prend cette cohérence quand l'Evangile doit socialiser les personnes, du vivre ensemble, leur permettant d'organiser leur vie, de faciliter la vie ensemble, de lire le monde ensemble. 
  
  \subparagraph{une doctrine n'est vraie que si elle permet de vivre de façon cohérente} des adeptes de telle religion. Les \textit{vérités} des autres traditions religieuses n'ont pas a être méprisées à partir du moment où elles permettent de vivre en cohérence au sein de cette religion.
  
  \subparagraph{une vision communautaire ?} pas forcément.
  
  
  
  
  % ------------------------------------------------
  
  \section{Une théologie post-libérale des religions : penser la différence/l’altérité}
    
 Selon Lindbeck, les religions peuvent être une anticipation voulue et approuvée par Dieu du Royaume à venir.
 
 Du point de vue Chrétien, \paragraph{une préfiguration du Royaume dans les différentes communautés (dans sa dimension sociale)} Des socialisations qui se font à différents niveaux. 
    
\begin{quote}
    Dès lors qu’il est clair qu’une théologie catholique des religions peut affirmer le caractère distinct des fins poursuivies par les autres religions sans préjudice pour une affirmation de l’unique valeur de la communauté chrétienne ou de ses doctrines du salut, alors il devient possible d’affirmer que Dieu veut que les autres religions jouent des fonctions dans son plan pour l’humanité qui ne sont perçues aujourd’hui que de façon confuse et qui seront complètement révélées dans la consommation de l’histoire attendue par les chrétiens non pas parce qu’elles seraient des canaux de la grâce ou des moyens de salut pour leurs adeptes, mais parce qu’elles jouent un rôle réel, même si peut-être pas complètement déterminé, dans le plan divin auquel la communauté rend témoignage. (Joseph DINOIA, The Diversity of Religions : A Christian Perspective, Baltimore, The Catholic University of America Press, 1992, p. 91) 
\end{quote}    
  % ------------------------------------------------
  
  \paragraph{Abu Dhabi}
  le pape François a remis en avant la diversité des hommes.
  
  Dans cette vision des choses, l'Eglise doit socialiser ces membres mais doit être aussi au service du dialogue inter-religieux, le dialogue comme dans le plan de Dieu. 
  
\begin{quote}
    « […]Le Pape Jean-Paul II le disait à Assise, à la fin de la Journée de prière, de jeûne et de pèlerinage pour la paix: 
    \begin{quote}
       «Voyons en ceci une \textsc{anticipation} de ce que Dieu voudrait voir se réaliser dans l’histoire de l’humanité: un cheminement fraternel dans lequel nous nous accompagnons mutuellement vers un objectif transcendant qu’il prépare pour nous »  
    \end{quote}
    (Dialogue et annonce\sn{Document intéressant, en 1991, qui a deux parties, le conseil pontifical pour le dialogue inter religieux et l'autre partie, la propagation de la foi}, § 79). 
\end{quote}  
  
\begin{quote}
« L’Eglise encourage et stimule le dialogue interreligieux non seulement entre elle-même et d’autres traditions religieuses mais aussi entre ces traditions religieuses elles-mêmes. C’est une manière pour elle de remplir son rôle de «sacrement», c’est-à-dire «de signe et instrument de l’union intime avec Dieu et de l’unité de tout le genre humain» (Lumen gentium, 1). L’Esprit l’invite à encourager toutes les institutions et tous les mouvements de caractère religieux à se rencontrer, à collaborer et à se purifier afin de promouvoir la vérité et la vie, la sainteté et la justice, l’amour et la paix, dimensions de ce Règne que le Christ, à la fin des temps, remettra à son Père (cf. 1 Co 15, 24). Par là, le dialogue interreligieux fait vraiment partie du dialogue de salut dont Dieu a pris l’initiative » (Dialogue et annonce, § 80). 
\end{quote}
L'Eglise essaye de faire rencontrer des bouddhistes avec les musulmans. C'est pour elle être \textit{sacrement}. 


Une certaine remise en cause de notre propre socialisation pour l'Eglise. 
Un thème théologique très actuel : comment le dialogue avec les autres peut nous aider à penser le dogme du Règne de Dieu, et dans son lien avec l'Eglise.
Dans VII, des documents.
Peut nous permettre de vivre de façon plus féconde le
  
  
  
 % ------------------------------------------------------------------------------------- 
\section{Église et royaume de Dieu réinterprétés par la théologie des religions}
  
  
  \begin{quote}
      Depuis le temps de Jean-Baptiste jusqu’à présent, le royaume des cieux est forcé, et ce sont les violents qui s’en emparent. Car tous les prophètes et la loi ont prophétisé jusqu’à Jean; et, si vous voulez le comprendre, c’est lui qui est l’Élie qui devait venir. Que celui qui a des oreilles pour entendre entende. Mt 11,12
  \end{quote}
  
\subsection{La signification du Règne de Dieu}

\paragraph{La problématique} 


\begin{quote}
    « L’Église n’a pas été reconnue comme elle l’espérait et elle a souvent réagi avec agressivité et violence pour trouver sa place. Ni attendue  ni désirée ni reconnue, l’Église porte en elle cette expérience douloureuse et longtemps impensée en tant que telle.(…) Le  statut de greffon  ne se gère pas en s’imposant mais en ayant pleinement conscience que l’on n’est chrétien seulement par adoption c’est-à-dire par choix. Etre adopté, ce n’est pas s’imposer » (M.-L. Durand, 151). 
\end{quote}

\begin{quote}
    « À partir des textes bibliques et des témoignages patristiques, comme des documents du Magistère de l'Église, on ne déduit une acception univoque ni pour Royaume des Cieux, Royaume de Dieu et Royaume du Christ ni pour leur rapport avec l'Église, elle-même mystère irréductible à un concept humain. Diverses explications théologiques peuvent donc exister sur ces problèmes. Cependant, aucune de ces explications possibles ne doit refuser ou réduire à néant le lien étroit entre le Christ, le Royaume et l'Église. » (CDF, Dominus Iesus, § 18). 
\end{quote}

\begin{quote}
    « Le christianisme et l’Église s’identifient-ils au Règne de Dieu, pour autant qu’il est présent dans le monde et dans l’histoire ? Ou, au contraire, le Règne de Dieu est-il une réalité universelle qui s’étend au-delà des limites de l’Église chrétienne ? Et, s’il en est ainsi, comment l’Église et les religions sont-elles respectivement reliées au Règne de Dieu ? (…) Et que dire encore du Règne de Dieu dans son achèvement eschatologique au-delà de l’histoire et de son rapport à l’Église et aux ‘autres’ ? » (Dupuis 505). 
\end{quote}
\paragraph{La relation du Règne de Dieu avec l’Église dans Lumen Gentium } 
\begin{quote}
    « C’est pourquoi le Christ, pour accomplir la volonté du Père, inaugura le royaume des cieux sur la terre (…) L’Église, qui est le règne de Dieu déjà mystérieusement présent, opère dans le monde (…) sa croissance visible » (LG 3).  
\end{quote}

\begin{quote}
    « Le texte définit le rapport entre le Royaume et l’Église au moyen de deux formules complémentaires. L’Église est l’exorde [début] sur terre du Royaume des cieux, elle est aussi la révélation du mystère du Christ. Autrement dit, elle est le Royaume du Christ présent in mysterio, de façon mystérieuse, car le mystère est à la fois révélé et caché ; en d’autres termes, la révélation n’éclate pas en pleine lumière mais se déroule sous le couvert des ombres. De plus elle est progressive ; non par ses propres forces, mais par la force de Dieu, l’Église développe sans cesse de façon visible son rôle d’annonciatrice du mystère »  (Philips, l’Église et son mystère au IIe concile du Vatican, t. 1, Desclée, Paris 1967, 86). 
\end{quote}

\begin{quote}
    « L’Église (…) reçoit mission d’annoncer le royaume du Christ et de Dieu [on ajoute ici ‘du Christ’] et l’instaurer dans toutes les nations, formant de ce royaume le germe et le commencement sur la terre » (LG 5). 
\end{quote}

\begin{quote}
    «  La Constitution dogmatique Lumen gentium parle d’une ordination graduelle à l’Église du point de vue de l’appel universel au salut qui inclut l’appel à l’Église. En contrepartie, la Constitution pastorale Gaudium et spes ouvre une perspective christologique, pneumatologique et sotériologique plus vaste. Ce qu’on dit des chrétiens vaut également pour tous les hommes de bonne volonté dans les cœurs desquels la grâce agit de manière invisible. Eux aussi peuvent être associés par le Saint-Esprit au mystère pascal, et ils peuvent par conséquent être conformés à la mort du Christ et marcher à la rencontre de la résurrection. » (CTI 1997, § 71). 
\end{quote}
\paragraph{Le rapport du Règne de Dieu à l’humanité dans Gaudium et spes }



\subsection{L’idée de Royaume de Dieu dans le contexte du pluralisme religieux}
    \paragraph{Le Royaume de Dieu et le salut des non-chrétiens } 
    
    
\begin{quote}
    « Le concile Vatican II reprend à son compte la formule extra Ecclesiam nulla salus. Mais avec elle, il s’adresse explicitement aux catholiques et il limite sa validité à ceux qui connaissent la nécessité de l’Église pour le salut. Le concile considère que l’affirmation est fondée sur la nécessité de la foi et du baptême, affirmée par le Christ. (CTI 1997, § 67). 
\end{quote}

\begin{quote}
    « Certes, l'Eglise n'est pas à elle-même sa propre fin, car elle est ordonnée au Royaume de Dieu dont elle est germe, signe et instrument. Mais, alors qu'elle est distincte du Christ et du Royaume, l'Eglise est unie indissolublement à l'un et à l'autre. Le Christ a doté l'Eglise, son corps, de la plénitude des biens et des moyens de salut; l'Esprit Saint demeure en elle, la vivifie de ses dons et de ses charismes, il la sanctifie, la guide et la renouvelle sans cesse. Il en résulte une relation singulière et unique qui, sans exclure l'action du Christ et de l'Esprit Saint hors des limites visibles de l'Eglise, confère à celle-ci un rôle spécifique et nécessaire. D'où aussi le lien spécial de l'Eglise avec le Royaume de Dieu et du Christ qu'elle a «la mission d'annoncer et d'instaurer dans toutes les nations». » (Redemptoris Missio, § 18).  
\end{quote}

\begin{quote}
    « Il est donc vrai que la réalité commencée du Royaume peut se trouver également au-delà des limites de l'Eglise, dans l'humanité entière, dans la mesure où celle-ci vit les «valeurs évangéliques » et s'ouvre à l'action de l'Esprit qui souffle où il veut et comme il veut (cf. Jn 3, 8); mais il faut ajouter aussitôt que cette dimension temporelle du Royaume est incomplète si elle ne s'articule pas avec le Règne du Christ, présent dans l'Eglise et destiné à la plénitude eschatologique. » (Redemptoris Missio, § 20). 
\end{quote}
    
\paragraph{Le Royaume de Dieu dans le contexte du dialogue interreligieux} 
\begin{quote}
    Une partie [de la mission de l’Église] consiste […] à reconnaître que la réalité de ce Royaume peut se trouver à l’état inchoatif aussi au-delà des frontières de l’Eglise, par exemple dans le cœur des membres d’autres traditions religieuses dans la mesure où ils vivent des valeurs évangéliques et sont ouverts à l’action de l’Esprit. Il faut rappeler cependant que cette réalité est en vérité à l’état inchoatif; elle trouvera son achèvement en étant ordonnée au Royaume du Christ qui est déjà présent dans l’Eglise mais qui ne se réalisera pleinement que dans le monde à venir (Dialogue et annonce,  § 35). 
\end{quote}
\paragraph{Le règne de Dieu et les autres religions }

\begin{quote}
    « Quand les non-chrétiens, justifiés par la grâce de Dieu, sont associés au mystère pascal de Jésus-Christ, ils le sont aussi au mystère de son Corps qui est l’Église. Le mystère de l’Église dans le Christ est une réalité dynamique dans le Saint-Esprit. Bien qu’à cette union spirituelle il manque l’expression visible de l’appartenance à l’Église, les non-chrétiens justifiés sont inclus dans l’Église « Corps mystique du Christ » et « communauté spirituelle ». C’est en ce sens que les Pères de l’Église peuvent dire que les non-chrétiens justifiés font partie de l’Ecclesia ab Abel. Tandis que ces derniers sont réunis dans l’Église universelle avec le Père, ceux qui appartiennent certes « de corps » à l’Église mais non pas « de cœur » ne seront pas sauvés, parce qu’ils n’ont pas persévéré dans la charité. » (CTI 1997, § 72). 
\end{quote}

\begin{quote}
    « Ainsi, on peut parler non seulement en général d’une ordination des non-chrétiens justifiés à l’Église, mais aussi d’un lien avec le mystère du Christ et de son Corps, l’Église. On ne devrait cependant pas parler d’appartenance à l’Église, ni d’appartenance graduelle à l’Église, ni d’une communion imparfaite avec l’Église, expression réservée aux chrétiens non catholiques ; car l’Église est par essence une réalité complexe, constituée de l’union visible et de la communion spirituelle. Bien évidemment, par la mise en pratique de l’amour envers Dieu et le prochain, les non-chrétiens qui ne sont pas coupables de ne pas appartenir à l’Église entrent dans la communion de ceux qui sont appelés au Royaume de Dieu ; cette communion se révélera comme Ecclesia universalis lors de la consommation du Royaume de Dieu et du Christ. » (CTI 1997, § 73). 
\end{quote}
\subsection{Questions ecclésiologiques} 




\section{Autres textes}

%%{\Large\textbf{Quel l’enjeu du thème du Royaume de Dieu dans la théologie des religions ? }}


\chapter{Quel l’enjeu du thème du Royaume de Dieu dans la théologie des religions ? }

Guillaume Gorge - Théologie chrétienne des religions - 16 mai 2022

\section{Introduction}

\paragraph{La Théologie des Religions}
Dans son livre \textit{pour une théologie des Religions}, Robert Schlette présente en 1963 le programme que se donne cette nouvelle discipline : 
\begin{quote}
   [La théologie, "ainsi libérée"
pourra] "{ hardiment réfléchir à ce problème, maintenant plus important,
du sens à donner aux religions en tant que phénomènes objectifs, historiques
et sociaux de l'humanité en tant qu'elle est reliée au mystère transcendant}" ,
et,  "{à partir de là, en tant que moyens de salut" (p. 71). } .\sn{souligné par H. Clavier. {Revue des livres dans Revue d'histoire et de Philosophie religieuses}}
\end{quote}

Il y a d'abord un constat objectif : dans un monde mondialisé, d'autres religions continuent à exister et à se développer, apportant sens, chemin vers la transcendance et lien communautaire à une grande partie de l'humanité. Si elles durent, c'est qu'elles portent quelque chose de solide. 
Du point de vue de la théologie chrétienne des religions, cet apport des religions s'enracine dans la signification de l'action salvifique de Jésus-Christ et il s'agit donc de rendre compte de ce lien.


\paragraph{La dimension sociale du Salut : \textit{le Règne de Dieu}} Or, l'une des dimensions essentielles du salut annoncé par le Christ est l'avènement du \textit{Règne de Dieu}. Cette dimension, redécouverte en théologie par les théologiens Wolfhart Pannenberg et Jürgen Moltmann, questionne aussi la théologie des religions, en ce sens que le salut n'est pas seulement pensé de façon individuelle mais dans sa dimension collective. Suivant la pensée de Pannenberg (\cite{Pannenberg:RoyaumeDieu}), 

\begin{quote}
    Toute Église chrétienne qui veut rester fidèle au message de Jésus doit se comprendre comme une communauté en relation avec le royaume de Dieu annoncé par Jésus. Qui dit royaume de Dieu dit futur du monde et de l’humanité tout entière.  \cite[pp. 74]{Pannenberg:RoyaumeDieu}. 
\end{quote}



Comme toute religion a une dimension sociale - on peut mettre les définir à partir de cette dimension 
 \sn{cf Schleiermacher : la Religion comme \textit{passage à l'expression collective d'une expérience individuelle} ou Durkheim : \textit{ Une religion est un système solidaire de croyances et de pratiques relatives à des choses sacrées, c’est-à-dire séparées, interdites, croyances et pratiques qui unissent en une même communauté
morale, appelée Église, tous ceux qui y adhèrent} } - , le Règne de Dieu questionne particulièrement la théologie des religions. Nous proposons de présenter l’enjeu du thème du Royaume de Dieu dans la théologie des religions, en partant de la réflexion de Pannenberg : 

\begin{itemize}
    \item Prendre au sérieux l'annonce du le Règne de Dieu. Après avoir étudié les conséquences pour la Christologie et l'Ecclésiologie selon la pensée de Pannenberg, nous esquisserons les enjeux que nous pouvons en déduire pour la théologie des Religions.
    \item Quelle actualisation de ces enjeux dans le contexte actuel de la post-modernité et du développement de la théologie depuis 50 ans
\end{itemize}



% ------------------------------------------------------------------------------------------------------------------------------------------------------------------------------------
\section{Prendre au sérieux l'annonce du le Règne de Dieu}
% ------------------------------------------------------------------------------------------------------------------------------------------------------------------------------------

Pannenberg souhaite dire en quel sens Jésus-Christ est interprété comme le fondement de l’Église et de son unité.(\cite[p. 82]{Pannenberg:RoyaumeDieu}). Or la mission de Jésus est déterminée par l'annonce du Règne de Dieu (on connaît l'aphorisme de Loisy : \textit{Jésus annonçait le Royaume, et c'est l'Église qui est venue.}). Le titre même de \textit{Christ}, fondement du kerygme chrétien et référence au Roi d'Israël, renvoie au règne de Dieu. 
Il y a donc un triangle  aux liens étroits entre Jésus-Christ, l'Eglise et son unité, et le Règne de Dieu, chacune se nourrissant de notre compréhension plus fine de l'autre.
 
% - ----------------------------------------
\subsection{Lien et conséquence pour l'Ecclésiologie}
\paragraph{le Règne qui vient : penser l'Eglise théologiquement à partir du Royaume}
Pannenberg commence par les conséquences pour l'Ecclésiologie, se comprenant à l'époque comme \textit{Peuple de Dieu} : 
\begin{quote}
   Si l’Église s’est comprise dès le christianisme primitif comme le nouvel Israël, comme le nouveau peuple de Dieu du temps final, cette compréhension de soi ne concerne pas seulement le lien conscient avec le peuple de Dieu de l’ancienne alliance : elle implique aussi l’idée de la royauté de Dieu, en lien avec le règne définitif attendu de Dieu sur le monde. Comme le règne de Dieu est l’avenir auquel fait face l’humanité tout entière, \textit{l’autocompréhension de l’Église comme peuple de Dieu est seulement à justifier par sa relation totale au monde et à l’humanité. }\cite[pp. 75]{Pannenberg:RoyaumeDieu}. C'est nous qui soulignons). 
\end{quote}
 Pannenberg ne s'oppose pas à l'expression d'\textit{Eglise comme Peuple de Dieu} mais ne doit pas être pensé comme le \textit{petit reste d'Israël}, ou tout du moins dans une articulation avec le monde.

 L'Eglise doit se nourrir de l’idée du règne de Dieu qui vient \cite[pp. 76]{Pannenberg:RoyaumeDieu} à travers un double mouvement lié à l'attente du futur propre au Royaume de Dieu. 
 
 \paragraph{Penser l'Eglise dans le monde où elle est}
 Dans un premier mouvement, toute interprétation de l'Eglise doit prendre en compte sa relation au contexte de vie englobant du monde social. En reprenant les terminologies de Lindbeck, nous pourrions dire que la \textit{langue} dans laquelle s'exprime l'Eglise reprend la langue du monde dans laquelle elle vit.
    
\paragraph{L'Eglise comme \textit{Anticipation de la nouvelle humanité}}
Dans un second mouvement, penser le Royaume de Dieu, c'est  penser son \textit{entremêlement au monde comme un moment essentiel de l'Eglise}\cite[pp. 76]{Pannenberg:RoyaumeDieu}). L'Eglise ne doit pas se retirer du monde mais l'aider à se conformer toujours plus au dessein de Dieu et se penser comme anticipation [Vorwegnahme] de la nouvelle humanité, une humanité régie par Dieu et son Esprit \cite[pp. 76-77]{Pannenberg:RoyaumeDieu}.  


% - ----------------------------------------
\subsection{le Règne de Dieu nourri par la christologie}

Quels critères pour juger de la fidélité de l'Eglise au message du Règne de Dieu ? Comment aussi discerner dans tout corps social, et donc dans d'autres religions le Règne qui vient ? 

\begin{quote}
    Attribuer à Jésus le titre et la dignité de Christ, de Messie, c’est exprimer la manière dont le royaume de Dieu qui vient s’est montré comme une force déterminant le présent, à travers la vie et la mort de Jésus pour toute l’humanité, et comme la transformation du présent par l’amour créateur. Toutes les immenses paroles et formules de la christologie disent vrai, dans la mesure où elles expriment la manière dont le futur du règne de Dieu est la force transformatrice de la vie de Jésus, dans le présent, et, à travers lui, de l’histoire de l’humanité. Dans l’engagement de Jésus pour le futur du règne de Dieu, ce dernier devint présent et, par lui, présent à tous les êtres humains. (\cite[pp. 94-95]{Pannenberg:RoyaumeDieu}). 
\end{quote}
Pour saisir le Règne de Dieu, il nous faut donc revenir à Jésus mais dans une approche renouvelée, transformée par l'Esprit Saint. 

\paragraph{Adéquation au Règne de Dieu}
Une réalité sociale ne préfigure le Règne de Dieu que dans la mesure où elle correspond à l'annonce de Jésus. Pratiquement, Pannenberg propose un certain nombre de pistes permettant de mesurer cette adéquation: 
\begin{itemize}
    \item Tout d'abord, un style, celui d'être sensible à l'approche dynamique de l'Esprit, à la mort et la Resurrection. Cela doit se traduire par l'acceptation du dynamisme des organisations, d'une attention aux signes des temps. 
    \item Une aspiration à la plus grande unité de l'humanité, à l'image de l'unité de Dieu, mais qui ne peut se faire par la violence,
    \item Préférer la justice et amour au droit, dans une conception qui ne soit pas individualiste.
    \item L'importance de la guérison \cite[pp. 87]{Pannenberg:RoyaumeDieu}, comme action de l'Esprit vivifiant.
\end{itemize}

\paragraph{Une vocation spécifique de l'Eglise ?} Pannenberg insiste sur l'aspect eschatologique du Royaume et le risque de toute société humaine d'oublier qu'elle est faillible et de devenir totalisante. 
 L'Eglise, parce qu'elle vit déjà maintenant de la certitude du futur de Dieu sans se laisser tromper par l'expérience du provisoire, est \textit{nécessaire} en tant que signe que nous ne sommes pas près du Royaume. Cet \textit{écart} avec l'idéal, c'est aussi laisser l'espace au travail de l'Esprit Saint dans notre monde.

\paragraph{Enjeu pour la théologie des religions}
Il s'agit donc de faire place des religions en respectant le lien étroit entre le Christ, le Royaume et l'Eglise (CDF, Dominus Iesus, § 18). In fine, du fait de ce lien étroit, on ne pourra pas articuler notre rapport aux autres religions uniquement vis à vis du Christ, ou de l'Eglise mais bien dans les trois dimensions de façon simultanée.

% ------------------------------------------------------------------------------------------------------------------------------------------------------------------------------------
\subsection{quels enjeux du thème du Règne de Dieu pour la théologie des Religions}  

L'objet de Pannenberg est d'honorer la place du \textit{Règne de Dieu} dans la mission du Christ et ses conséquences tant en Christologie qu'en Ecclésiologie. Mais par ce travail, il ouvre un certain nombre de questions qui peuvent nourrir une théologie Chrétienne des Religions.

Pannenberg insiste sur la dimension du Règne de Dieu dépassant l'Eglise mais surtout qui oblige l'Eglise à prendre en compte la dimension universelle du salut : 
\begin{quote}
L’opposition au monde ne peut pas être le motif premier et fondamental quand il est question d’exprimer le rapport de l’Église à la société et à l’humanité. Cela n’est pas seulement dû à une contradiction formelle, érigeant l’opposition, comme élément de la relation, en élément constitutif de l’Église. Il est plus important encore de souligner que la simple opposition de l’Église au monde néglige la tendance universaliste inhérente à l’idée du règne de Dieu. Le règne de Dieu est plus grand que l’Église, et celle-ci ne trouve sa fonction et sa signification spécifiques que dans son orientation vers le Règne de Dieu.(\cite[pp. 75-76]{Pannenberg:RoyaumeDieu} )
 \end{quote}

 Cela \textit{oblige} à prendre en compte ce que les autres religions ont d'irréductibles, car cette irréductibilité fait partie de l'humanité.  
\paragraph{Les religions comme \textit{institutions honnêtes}}
Les religions, et au premier titre, l'Église, sont des institutions et à ce titre se doivent être des institutions honnêtes, que Pannenberg définit comme
\begin{quote}
      une institution qui démasque les limites des formes présentes de la vie sociale et politique et renvoie les êtres humains à la réalité ultime englobant leur destination dernière. La relation de l’être humain à sa destination ultime ne peut devenir consciente, dans sa concrétude mondaine, qu’à la condition de prendre honnêtement en compte la vie présente dans ses limites et dans ses échecs, et qu’on n’essaie pas d’échapper à ses aspects négatifs.\cite[pp. 98]{Pannenberg:RoyaumeDieu}
\end{quote}
Parce que les religions proposent un \textit{ sens véritable de l'existence} et \textit{une pratique liturgique, rituelle et une éthique de vie qui permettent d'y accéder} \sn{DiNoia cité par \cite{Cheno:DieuPluriel}}, elles sont des institutions honnêtes particulièrement structurantes pour l'humanité, mais uniquement dans la mesure où elles démasquent les limites des formes présentes de la vie sociale.

\paragraph{Participation au Règne de Dieu }
Le règne du Christ est actif, partout où les êtres humains prennent conscience de la venue du Règne de Dieu.\cite[p. 85]{Pannenberg:RoyaumeDieu} : faire advenir la justice, l'amour, s'éloigner d'un juridisme froid, lire les signes de l'Esprit, guérir, viser à l'unité de l'humanité dans la paix, ce sont des signes que les religions non chrétiennes participent à la venue du Règne de Dieu.



% ------------------------------------------------------------------------------------------------------------------------------------------------------------------------------------    
\section{enjeux contemporains pour la théologie des religions}    
% ------------------------------------------------------------------------------------------------------------------------------------------------------------------------------------

Le livre de Pannenberg a été écrit à partir de conférences données dans les années 1960. Or, la vision du monde a beaucoup changé, avec la vision marxiste remplacée par celle  \textit{post-moderne} . De plus, la compréhension du Christ, de l'Eglise et du Règne de Dieu ont aussi évolué, et nous souhaiterions aborder comment l’enjeu du thème du Royaume de Dieu peut être pensé dans la théologie des religions aujourd'hui.

% ------------------------------------------------------------------------------------------------------------------------------------------------------------------------------------    
\subsection{Dans une société post-moderne}

Avec 9 références au marxisme dans le chapitre 2, la grille de lecture marxiste de Pannenberg est explicite. Or, le marxisme, s'inscrit dans le courant moderne dans le sens où il ne renonce pas à une clé de lecture globalisante. Cette prétention à la totalité explique d'ailleurs la critique que lui fait Pannenberg de penser que le Règne de Dieu puisse se réaliser ici bas. Cette clé de lecture est moins pertinente aujourd'hui, dans une société \textit{post-moderne} qui a renoncé à toute vérité autre que partielle.


\paragraph{ Quel corps social pour un monde post-moderne ?} Dans un monde post-moderne, le risque est en effet moins le discours totalisant qu'une dilution de tout corps social parallèlement à la recherche d'un salut individualiste. Face à cette évolution, l'annonce du Règne de Dieu est probablement moins audible mais d'autant plus pertinente : il est important de rappeler que le salut Chrétien ne peut se penser de façon individuelle, mais doit intégrer la dimension sociale et universelle du \textit{Règne de Dieu}.


\paragraph{Penser le rôle des autres religions dans ce contexte}
Lindbeck propose de regarder la prétention d'une religion à être raisonnable et universelle à sa capacité à fournir dans ses propres termes une interprétation intelligible des diverses situations et réalités que rencontrent ses adhérents (Lindbeck, 175). Dans un monde marqué par le post-modernisme et l'hyper-individualisation, une attention renouvelée sera portée à la capacité des religion à "relier", à créer un corps social et à faire l'unité de l'humanité sans violence, éléments soulignés par Pannenberg comme des critères d'authenticité de l'annonce du \textit{Règne de Dieu}. 
Dans un contexte post-moderne, certains théologiens \sn{DiNoia, cité par \cite[p.133]{Cheno:DieuPluriel}} proposent d'abandonner le principe sotériologique car c'est appliquer à d'autres religions un cadre de pensée pensé à partir du Christianisme : plutot que le \textit{salut}, chrétien, penser une réalité plus large de \textit{but ultime de la vie}.  Sans entrer dans la discussion des implications d'un tel abandon, il nous semble \textit{a minima} important de rappeler l'importance que ce but ultime de la vie intègre une dimension sociale explicite. 



% ------------------------------------------------------------------------------------------------------------------------------------------------------------------------------------    
\paragraph{Le Règne de Dieu d'après Pagola}
La théologie a bien sûr beaucoup évolué en 50 ans. Ainsi, le terme de \textit{Peuple de Dieu} pour penser l'Eglise, très présent chez Pannenberg, est moins central aujourd'hui même s'il n'a pas perdu sa pertinence pour expliciter la dimension collective du Salut. Il n'est pas question de développer ici toutes les pistes explorées par la théologie  mais de remarquer qu'en insistant aujourd'hui sur certaines dimensions du \textit{règne de Dieu}, le théologien éclairera la dimension sociale du salut de nouveaux éléments.
Ainsi Pagola propose une vision du Royaume de Dieu éclairée par le jugement et la mort du Christ : 
\begin{quote}
    Le Royaume de Dieu défendu par Jésus remet en question à la fois l'édifice romain et le système du Temple. Les autorités juives, fidèles au Dieu du Temple, se sentent tenues de réagir: Jésus est un gêneur. Il invoque Dieu pour défendre la vie des exclus. Caïphe et les siens l'invoquent pour défendre les intérêts du Temple. ils condamnent Jésus au nom de Dieu mais, ce faisant, ils condamnent le Dieu du Royaume, le seul Dieu vivant en qui Jésus croie. Il en va de même avec L'Empire. Jésus ne voit pas dans le système défendu par Pilate un monde organisé selon le coeur de Dieu. Il défend, lui, les intérêts des oubliés de l'Empire. Pilate protège les intérêts de Rome. Le Dieu de Jésus pense aux plus démunis. Les dieux de l'empire protègent la \textit{pax romana}.\cite[p. 402]{Pagola:Jesus} 
\end{quote}

Cet éclairage de Pagola nous incite à chercher les traces du Règne de Dieu dans la défense des plus pauvres et des oubliés, dans le refus de mettre en priorité les intérêts religieux et de l'ordre établi. Par cet éclairage, on pourra répondre à nouveaux frais aux questions suivantes :  comment les religions oeuvrent par rapport au défi écologique face à l'ordre établi ? les migrants ? les personnes fragiles ou handicapées ? Comment les religions acceptent de \textit{gêner}, d'\textit{empêcher le monde de tourner en rond} ? Comment aussi elles \textit{acceptent la critique} de leurs propres institutions ? 


L'Église, dans son rôle d'anticipation du Règne de Dieu, veillera à s'appliquer d'abord cette grille exigeante puis pourra la proposer aux autres religions dans un dialogue sans fard mais fécond.

% ------------------------------------------------------------------------------------------------------------------------------------------------------------------------------------    
\section{Conclusion}    
% ------------------------------------------------------------------------------------------------------------------------------------------------------------------------------------

Le Règne de Dieu permet à l'Eglise de se décentrer : ce qui doit advenir, c'est le Règne de Dieu, qui a vocation à toucher la totalité de l'humanité. Notre vision de l'universel a évolué, probablement plus humble, reconnaissant une partie irréductible dans chaque religion. 
Cette dimension sociale des religions est particulièrement importante à souligner dans un monde \textit{post-moderne} où le salut peut être pensé de façon strictement individuelle. Contre la tention du repli, du confort de l'individu \cite[p.107]{Pannenberg:RoyaumeDieu}, l'Eglise a vocation a reconnaître en elle mais aussi dans tout corps social et toute religion ce qui est de l'annonce du Règne et de la venue du Règne dès ce monde, à partir du message et de la vie de Jésus, toujours relu et actualisé. 


% ------------------------------------------------------------------------------------------------------------------------------------------------------------------------------------
% ------------------------------------------------------------------------------------------------------------------------------------------------------------------------------------


%\chapter{Urgences pastorales du moment présent
Christoph Theobald
 }

\mn{Urgences pastorales du moment présent
- Comprendre, partager, réformer Christoph Theobald}






\section{ETAPES D'UNE ECCLESIOGENÈSE}

Ce dernier chapitre s'encha1ne immédiatement au précédent,.clôturant notre tentative de rassembler et d'articuler, de manière cohérente, les différents paramètres d'une pédagogie de la réforme, tant appelée de ses vœux par le pape François dansson exhortation Evangelii gaudium. Une telle pédagogie est nécessairement anticipatrice : contre tout passéisme ou pragmatisme pastoral-« ona toujours fuit ainsi» (EG, 33)-,  elle doit d'abord avoir le courage d'expliciter la vision d' Iise qui est en train de s'esquisser au sein de nos sociétés européennes, car sans une telle vision d'avenir, au moins embryonnaire, il est impossible d'envisager une« ré-forme» missionnaire de l'Eglise et de former des acteurs capables d'œuvrer en sa faveur (cf chap. 7). Sur la base d'une telle vision, respectueuse des quatre principes du pape François1, en particulier du troisième -  « La réalité est
\mn{l. Rappelons les quatre principes ; l. Le temps est supérieur à l'espace ;
2. L'unité prévaut sur Je conflit; 3. La réalité est plus importante que l'idée;
4. Le tout est supérieur à la partie (EG, 221-237).}
plus importante que l'idée}} (EG, 231-233) -, il faut ensuite développer une manière de procéder (ce que nous venons de faire) et tracer un chemin qui nous rapproche de ce que now, avons perçu; c'est ce que nous allons faire maintenant. L'aspcl-t
« utopique » de la vision se trouve ainsi « contrebalancé» par les moyens concrets mis à notre disposition pour avancer à partir du point exact où se trouve l'Église d'aujourd'hui.
Venu de l'Amérique latine (Léonardo Boff), le conœpt d' « ecdésiogenèse » désigne précisément cet aspect du devenir : un devenir ou une germination possible quand l'existence ecclésiale se réduit à nn « presque rien». Il se situe donc à l'endroit même de la «crise», évoquée dès les premières pages de cet ouvrage1, et au carrefour qui se présente ici : crise fatale 0 1 point de départ d'une transformation et d'une maturation. A ce titre, il exerce la fonction d'un antidote ; ou, pour s'inspirer d'une formule de Dietrich Bonhoeffer (en mai 1944), « il nous renvoie aux débuts de l'Église2 ».

\mn{1.	Cf plus haut, p. 26-29.	.
2.	"Aujourd'hui tu reçois le baptême chrétien. On prononcera sur 101 toutes les grandes paroles anciennes de l'annonce chrétienne et on accom plira sur toi l'ordre de baptiser, donné par le Christ sans que tu n'y com prennes rien. Nous aussi nous sommes renvoyés aux déb ts du compren_dre._Ce. que signifient réconciliation et rédemptiou, nouvelle na1ssauce et Espnt samt. amour des ennemis, croix et résurrection, vie en Christ et imitation de fésus Christ, tout cela est devenu si difficile et si lointain que c'est à peine si nuu osons encore en parler. Nous soupçonnons un souffle uouveau et boulever• sant dans les paroles et les actions qui nous ont été transmises, sans pouvoir encore le saisir et l'exprimer. C'est notre propre faute. Notre Église, 11111 n'a lutté, pendant ces années, que pour se maintenir en vie, comme si elle étt11t}



Cette ecdésiogenèse ne peut cependant être que missionnaire, celle d'une Église « décentrée », non tournée vers elle-même. Vatican Il et l'expérience ecclésiale des cinquante dernières années nous l'auront appris, du moins on peut l'espérer. C'est en tout cas la conviction fondamentale qui anime cet ouvrage. Nous avons déjà montré, dans le chap. 3, que la distinction classique, encore en vigueur au dernier concile, entre <( pays christianisés » et« pays de mission » ne fonctionne plus aujourd'hui, la France, voire l'Europe tout entière, étant devenue un vaste ensemble de pays de mission. Ce qui signifie qu'il faut impérativement sortir le décret conciliaire sur l'activité missionnaire de l'Église Ad gentes (1965) de son statut marginal (ou seulement pertinent sur d'autres continents) et lui donner la fonction de dé d'interprétation des deux grandes constitutions dogmatique et pastorale sur l'Église et l'Église dans le monde de ce temps, diptyque ecclésio-anthropologique qui domine jusqu'à nos jours la réception de Vatican 111.
Or, Ad gentes est le premier document officiel qui développe une « ecclésiogenèse », précisément dans une perspective

son propre but, est incapable d'être la porteuse de la parole réconciliatrice et rédemptrice pour les êtres humains et le monde. C'est pourquoi les paroles antérieures doivent perdre leur force et céder au silence; notre être chrétien ne peut aujourd'hui consister qu'en deux choses: ln prière et faire ce qui est juste parmi les humains. Toute peusée, toute parole et toute organisation, dans le domaine du christianisme, doivent renaitre à partir de cette prière et de
cette action » (D. Bonhoeffer, « Pensées pour le jour du baptême de D.W.R., Mai 1944 », Résistance er.soumission. Lettre et notes de captivité, Genève, Labor et Fides, 2006, p. 353; c'est nous qui mettons en relief certains passages).
\mn{1.	Cf plus haut, p. 104-106, 130 sqq., 204 sqq. et 222.431}

«décentrée» et missionnaire. Nous commencerons donc par expliciter quelque peu cette référence conciliaire trop peu connue, avant de reporter ce parcours à sa matrice scripturaire, essentiellement les Actes des Apôtres qui nous ont accompagnés tout au long de cette troisième partie, et de terminer par le tracé d'un chemin type pour aujourd'hui.


LE DÉCRET AD GENTES :
UNE CONCEPTION GÉNÉTIQ.UE DE L'ÉGLISE·

Avant d'aborder le devenir de l'Église proprement dit, le décret missionnaire Ad gentes traite, dans son premier chapitre, de quelques « principes doctrinaux1 ».

DE SA NATURE L'ÉGLISE EST MISSIONNAIRE

L'exposé de ces principes - soit dit au préalable - est très intéressant ; car il ressaisit, voire développe, en 1965 ce qui, dans la constitution sur l'Église, promulguée l'année précédente, reste encore au stade inchoatif, en particulier l'identification entre la « nature même » de l'Église et son « activité missionnaire », considérées comme différentes dans Lumen gentium (cf. LG, 1). Cette identification est décisive, comme nous l'avons montré au chap. 4 de cet ouvrage, car elle représente le point aveugle de la réception de Vatican II en Europe:


l. Pour ce qui suit, cf C. Theobald, Le concile Vatican TI. Quel avenir?,
Paris, Le Cerf, 2015, p. 205-226.

432
 
ÉTAPES D'UNE ECCLÉSIOGENÈSE

« De sa nature, l'Église, durant son pèlerinage sur terre, est missionnaire, puisqu'elle-même tire son origine de la mission du Fils et de la mission du Saint Esprit, selon le dessein de Dieu le Père» (AG, 2). Les n°' 2, 3 et 4 du décret explicitent alors cette affirmation trinitaire (le dessein du Père, la mission du Fils et la mission de }'Esprit), selon le même ordre que celui qui fut adopté par les n°5 2, 3 et 4 de la constitution Lumen gentium. Mais les différences sont notables et concernent non seulement la perspective missionnaire qui, en 1965, devient dominante, mais aussi l'enracinement biblique
du texte, désormais plus construit et plus nettement appuyé sur l'œuvre lucanienne.
L'approche génétique et historique du décret apparaît dans toute sa clarté au n° 6. L'activité missionnaire de l'Église se différencie, premièrement, selon les « circonstances » (condiciones), c'est-à-dire selon« les peuples, les groupes humains ou les hommes à qui s'adresse la mission» (AG, 6 § 2) et conduit de là, deuxièmement, à un choix d'activités et de (<moyens» (nous en avons développé certains dans le chapitre précédent) ; ce qui fait que telle Église particulière « connaît des commencements et des degrés dans son action [... ] ; de surcroît, elle est parfois contrainte, après des débuts heureux, de déplorer
de nouveau un recul, ou tout au moins de demeurer dans un état de semi-complétude et d'insuffisance» (ibid1.). Dans le
quatrième paragraphe, cette vue est encore plus clairement articulée, après quelques ultimes corrections : les différentes phases de croissance de l'Église ne sont jamais « dépassées »

1.	C'est nous qui soulignons.


433
 
URGENCES PASTORALES DU MOMENT PRÉSENT

(superatis), comme l'affirmait au contraire l'avant-dernière version du texte; elles ne peuvent être qu'« accomplies)> (expletis). Et les Églises particulières ne sont jamais « parfaitement constituées)) (perfecte constitutis) - c'est ainsi que le formulait l'avant-dernière version -, elles sont simplement « déjà constituées» (iam constitutis) ; elles ont dès lors le devoir de continuer l'activité missionnaire.

LA GENÈSE DE L'ÉGLISE SELON AD GENTES

Le but de la mission, celui d'annoncer l'Évangile et d'implanter l'Église, est nettement souligné dès le début du décret missionnaire Ad gentes (AG, 6 § 3), mais sa mise en œuvre effective n'est articulée que dans le deuxième chapitre en trois étapes distinctes d'une genèse <l'Église. Le premier article (I) décrit le point de départ : le « témoignage » des chrétiens dans un environnement non chrétien ; leur << présence » (praesentia, AG, 12 § 1), « afin que les autres, considérant leurs bonnes œuvres, glorifient le Père et - ajout d'importance - perçoivent plus pleinement le sens authentique de la vie humaine et le lien universel de communion entre les hommes» (AG, 11 § 1). On retrouve ici la racine de certaines insistances de notre chap. 6 sur un« nouvel humanisme». Notons en particulier l'intérêt du texte pour la « conversation » des chrétiens avec leurs contemporains, dans l'esprit même des dialogues du Christ Jésus dans les récits évangéliques1•

1.	Cf C. Theobald, « La conversation spirituelle aujourd'hui. Une expérience pastorale)>, Paroles humaines, parole de Dieu, op. cit., p. 159-186.

434
 
ÉTAPES D'UNE ECCLÉSIOGENÈSE

Suit, dans un deuxième article (Il), la prédication, rendue possible par Dieu lui-même. La formule inaugurale du n° 13 est le meilleur antidote contre tout volontarisme missionnaire :
« Partout où Dieu ouvre la porte à la prédication... ». Le texte
analyse alors le processus de conversion et sa structure spirituelle et. sacramentelle, le but étant le rassemblement du peuple de Dieu (AG, 13 et 14). Ce n'est que dans le troisième article (III) qu'est retracé, pas à pas et pour ainsi dire d'en bas, le « façonnement» de la communauté chrétienne, les services qui y sont nécessaires étant mentionnés dans toute leur ampleur (AG, 15
§ 7), avant qu'il ne soit question de la fonction presbytérale et d'autres vocations spécifiques (AG, 16 et 18).
.  es trois étapes sont évidemment à entendre comme des types ideaux; le texte renvoie à plusieurs reprises à leur interconnexion et à leur conditionnement historique. Cela vaut particulièrement aujourd'hui où les conditions de vie postmodernes rendent difficile des trajectoires de vie continues sur le long terme et où la carte du chemin à emprunter s'est quelque peu perdue. Il s'agit dès lors de se rendre attentif à un autre aspect du texte, qui a été trop peu remarqué : sa référence à ['Écriture. Sans pouvoir en présenter ici une analyse détaillée, je voudrais souligner la place des deux volets de l'œuvre lucanienne dans la structure du décret missionnaire: elle apparaît pour la première fois au n° 4 qui, en prolongeant de manière fort originale le n°4 de Lumen gentium, décrit l'envoi de l'Esprit; et, à partir de là, elle est présente tout au long du texte :
« À travers toutes les époques, c'est le Saint-Esprit qui "unifie l'Église tout entière dans la communion et le ministère, qui la munit des divers dons hiérarchiques et charismatiques" (LG, 4),

435
 
URGENCES PASTORALES DU MOMENT PRÉSENT

vivifiant à la façon d'une âme les institutions ecclésiastiques et insinuant dans le cœur des fidèles le 'même esprit missionnaire qui avait poussé le Christ lui-même. Parfois même il prévient visiblement l'action apostolique (Ac 10,44-47; 11,15; 15,8), tout comme il ne cesse de l'accompagner et de la diriger de diverses manières (Ac 4,8; 5,32; 8,26.29.39; 9,31 ; 10 ; 11,24.28;
13,2.4.9; 16,6-7; 20,22-23; 21,11, etc.)» (AG, 4).
On peut négliger ces citations et renvois, les considérant comme de simples appuis. Mais on peut aussi, à l'inverse - et selon les consignes de la constitution sur la révélation Dei verhum (DV, 24) -, comprendre l'œuvre lucanienne, en référence critique à d'autres théologies néotestamentaires, comme matrice d'un processus générateur <l'Église et ensuite, à partir d'Ad gentes, lire la constitution sur l'Église sous ce nouvel angle
de vue.


LIRE LES ACTES DES APÔTRES
COMME RÉCIT D'UNE GENÈSE D'ÉGLISE

Suite à ces quelques indications, il peut être utile, et faire partie des « manières de procéder» dont il a été question dans le chapitre précédent, de lire à plusieurs les Actes des Apôtres, surtout en période synodale1• Ce texte nous propose en effet w1e

1.	On peut se référer ici au commentaire de D. Marguerat, Les Actes des apôtres (1-12) et Les Actes des Apôtres (13-28),Genève, Labor et Fides, 2007 et 2015. Cf aussi C. Theobald, Présences d'Évangile II, op.cit., p. 61-145; on peul lire cet ouvrage comme un petit « manuel» de mission sur nos territoires.

436
 
ÉTAPES D'UNE ECCLÉSIOGENÈSE

véritable vision ecclésiogénétique de l'Église et nous autorise à imaginer aujourd'hui des itinéraires analogues ; ce qui, comme nous venons de le voir, nous est également suggéré par le décret missionnaire Ad gentes.

L'ÉGLISE DU NOUVEAU TESTAMENT

Quand nous parlons de «l'Église» nous nous référons spontanément à une entité parfaitement constituée, sans imaginer l'époque où elle n'existait pas encore ou était en train de naître. Certes, le tout premier écrivain du Nouveau Testament, l'apôtre Paul, dispose déjà d'un vocabulaire ecclésial très précis et d'une véritable « ecclésiologie » comme on dit aujourd'hui. Pensons simplement aux adresses de ses lettres : « Paul, Silvain et Timothée à l'Église des Thessaloniciens qui est en Dieu le Père et dans le Seigneur Jésus Christ» (1 Th 1,1) ou encore: « Paul, appelé apôtre du Christ Jésus, par la volonté de Dieu, et Sosthène le frère, à l'Église de Dieu qui est à Corinthe» (1 Co 1-2). Et l'un des derniers écrits de nos Écritures, !'Apocalypse, débute avec sept lettres, chacune adressée à l'une des sept Églises d'Asie mineure.
Mais a-t-on réalisé que ni l'Évangile de Marc, ni celui de Luc, ni encore celui de Jean ne connaissent le mot « Église » ? Matthieu se contente de deux occurrences, devenues d'ailleurs célèbres : la première dans l'échange de Jésus avec Simon
- « Tu es Pierre et sur cette pierre je bâtirai (au futur) mon Église» (Mt 16,18) - ; la seconde dans la règle de correction fraternelle - « S'il refuse d'écouter (une ou deux autres personnes que toi), dis-le à l'Église, et s'il refuse d'écouter l'Église,

437
 
URGENCES PASTORALES DU MOMENT PRÉSENT

qu'il soit pour toi comme le païen et le collecteur d'impôts »
(Mt 18,17).
Peut-être ce quasi-silence des Évangiles1 signifie-t-il que le
« site » ecclésial où se vit concrètement la vie chrétienne naissante se constitue d'abord sur des chemins, dans des maisons et dans un pays, comme réseau de relations entre personnes (disciples, apôtres, sympathisants, etc.), avant de représenter aussi une réalité sociale, un« corps)> dira Paul, le « corps même du Christ» (1 Co 12). Nos historiens ont montré que cette genèse a grandement bénéficié de structures « associatives », florissantes dans l'Empire romain2, le signe distinctif étant sans aucun doute une mixité sociale, fondée sur le baptême, inexistante par ailleurs :
« Oui, vous tous, lisons-nous chez Paul, qui avez été baptisés en Christ, vous avez revêtu Christ. Il n'y a plus ni Juif, ni Grec; il n'y a plus ni esclave, ni homme libre ; il n'y a plus l'homme et la femme; car tous, vous n'êtes qu'un en Jésus Christ. Et si vous appartenez au Christ, c'est donc que vous êtes la descendance d'Abraham; selon la promesse, vous êtes héritiers» (Ga 3,27-29).




1.	Nous parlons de quasi-silence car l'analyse historique parvient à détecter av une plus ou moins grande probabilité derrière certains épisodes évangehques (par exemple Le 10,38-42) des situations de l':E.glise primitive.
2.	Cf M.-F. Baslez, « La diffusion du christianisme aux r"-m• siècles.
L':E.glise des réseaux>>, RSR 101/4 (2013), p. 549-576.

438
 
ÉTAPES D'UNE ECCLÉSIOGENÈSE


UN EXERCICE DE LECTURE
L'œuvre de Luc confirme cette vision et la pousse jusqu'au bout : absent dans l'Évangile, le mot « Église » intervient avec force dans les Actes des Apôtres, avec pas moins de vingt-trois occurrences; ce qui, nous allons le voir, confirme la perspective génétique qui est en train de se dessiner. Je n'alignerai pas ici toutes les citations où figure le terme « Église » ; les groupes de lecteurs intéressés pourront, pour cet exercice, se référer à un tableau qui se trouve dans un volume antérieur1•
Avant qu'il ne soit question, pour la première fois,
d'« Église)) (Ac 5,11), les Actes mettent les lecteurs en présence de quelques-uns, cent vingt personnes environ avec les Onze, quelques femmes, dont Marie, et les frères de Jésus (Ac 1,13-15), qui se réunissent dans des «maisons» et fréquentent le temple (Ac 2,46). La désignation « Église>) n'apparaît donc que très progressivement et se charge de connotations toujours plus riches, au fur et à mesure que l'expérience ecclésiale se précise et se propage. Par deux fois, Luc rappelle incidemment l'origine biblique et politique du mot: « l'assemblée >> du désert révoltée contre Moïse (Ac 7,38) et « l'assemblée >> d'Ëphèse réunie au théâtre. Il accompagne cette dernière mention d'une note pleine d'humour, tout en soulignant la forme légale de ce genre d' « assemblée » (Ac 19,39.40) : « Chacun bien sûr criait autre chose que son voisin, et la confusion régnait dans l'assemblée où la plupart ignoraient même les motifs de la réunion>> (Ac 19,32) ;

1.	Présences d'Évangile II, op. cit., p. 71-73.

439
 
URGENCES PASTORALES DU MOMENT PRÉSENT

comme s'il voulait nous rappeler que le mot « Église >> est d'abord à prendre en son sens le plus élémentaire et avec ce!> connotations critiques.
Retenons les moments les plus significatifs de la mise en
série du vocabulaire ecclésial. Nommée pour la première fois en lien avec un «événement» -	une affaire d'argent (Ananiai. et Saphira) qui concerne toute la communauté et a des répercussions en dehors de celle-ci (Ac 5,11) -	et en lien avec la persécution lancée par « Saul qui pénètre dans les maisons» (Ac 8,1.3), «l'Église» émerge comme « sujet» qui grandit à l'image d'un enfant : « L'Église, sur toute l'étendue de la Judée, de la Galilée, et de la Samarie, vivait donc en paix, elle s'édifiait et marchait dans )a crainte du Seigneur et, grâce à l'appui du Saint-Esprit, elle s'accroissait» (Ac 9,31).
Or, cette E.glise est liée à un lieu. et à une terre ou une région:
ayant d'abord parlé des maisons et de l'Église de manière générale, Luc introduit rapidement l'expression « Église de Jérusalem >1 (Ac 8,1), désignation qu'il universalise ensuite quand, en parlant d'Antioche, il utilise le terme technique « E.glise du lieu>> (Ac 13,1). Progressivement on voit apparaître l'autonomie de ces :Bglises, pourvues d' « anciens » (presbuteroi) désignés par leur fondateur (Ac 14,23) ou d'autres référents comme des « prophètes » et des « hommes chargés del'enseignement » (Ac 13,l). Le lieu est décisif, parce qu'il marque de ses conditions culturelles et économiques la vie quotidienne des Églises : gestion de la pluralité des langues (Ac 2,5-13) et de la différence financiere entre membres (Ac 2,44 sqq.). On voit réapparaître ici le versant messianique de la tradition chrétienne, dont il a été question

440
 
ÉTAPES D'UNE ECCLÉSIOGENÈSE

à plusieurs reprises'. Des conflits ne manqueront donc pas de surgir, portant sur le partage des biens et surtout sur l'interprétation de certains événements - viennent-ils de Dieu ? - et sur l'interprétation des Écritures. L':Bglise de Jérusalem a déjà développé des oreilles : « La nouvelle de cet événement (il s'agit de la conversion de grecs à Antioche) parvint aux oreilles de l'Église qui était à Jérusalem» (Ac 11,22). Des visites mutuelles et l'envoi de délégués deviennent nécessaires pour régler le différend: « l'Église d'Antioche pourvoit à leur voyage» (Ac 15,3). Le récit est alors le moyen privilégié pour communiquer ce que Dieu fait de neuf parmi nous : « À leur arrivée, ils réunirent l'Église et raèontaient tout ce que Dieu avait réalisé avec eux et surtout comment il avait ouvert aux païens la porte de la foi» (Ac 14,27). Le conflit d'interprétation - faut-il, oui ou non, circoncire les païens convertis ?- se résout dans une assemblée qui passe du récit à la délibération et de la délibération à la décision:« D'accord avec toute l'Église, les apôtres et les anciens
décidèrent alors de choisir dans leurs rangs des délégués qu'ils enverraient à Antioche avec Pau·let Barnabas» (Ac 15,222).
Cette phase de maturation se termine quand Luc rapporte brièvement que Paul parcourt la Syrie et la Cilicie pour « affermir les Églises » : « Les Églises devenaient plus fortes et croissaient en nombre de jour en jour» (Ac 15,41 et 5,15). C'est Paul, dans son testament aux anciens d'Éphèse, qui conduit le devenir de l'Église jusqu'au bout en y introduisant le vocabulaire pastoral et en rapportant à Dieu toute la genèse ecclésiale, et cela en fin
 	
1.	Cf. plus haut, p. 345 sqq., 353 sqq. et 389-392.
2.	Cf pl.us haut, p. 326, 407 sqq. et 412 sqq.

4Al
 
URGENCES PASTORALES DU MOMENT PRÉSENT

de parcours:« Prenez soin de vous-mêmes, dit-il a  ancien , et de toutle troupeau dont l'Esprit Saint vous a é blis_les gard1_ens (episcopoi)>soyez les bergers (poimainein) de 1 f.gltse deDieu, qu'il s'est acquise par son propre sang» (Ac 20,28).

QUE RETENIR DE CE PÉRIPLE ?
1.	D'abord l'expérience d'une Égl:ise en genèse, expé ie ce ,qui peut encore devenir la nôtre : le récit de Luc ab?ut1t 1 ou la constitution sur l'Église Lumen gentium du concile Vat:I an I débute, à savoir au « mystère de l'Église » dans 1  d ssern tri-
m.ta1•red e D'1eu (LG, J... chap1·tre • Le mystère de l Eghse). Il est
donc possible de lire la constitution en s ns inverse-  p ur ainsi dire« d'en bas»-,précisément à partir dela« perspective de fondation», telle qu'on la trouve dans les Actes et dans le
décret missionnaire.	.  . . Notons cependant que l'introduction d'une dimension his-
torique dans Lumen gentium en 1963 - avec le chap.2 sur l.e peuple de Dieu et déjà dans le n°S du premier chapitre-. rédwt la distance entre ces textes. Dans ce no 5, la « perspect ve fo dationnelle » (fundatio) ou génétique est po r- l.a prennè e f 1s explicitée : elle re\ie le «commencement» (mitiu.m) de1 Égli e
dans le message de Jésus sur le Royaume de Dieu (LG, ,s§.  )
au «	commencement >> du	Royaume de Dieu dans l Église (LG, 5 § 2). Cette même pers ective apparaît é a ement d s un paragraphe du n°26 snr les Eglises locales et, sil on poursuit cette piste, aussi dans le n° 19 sur le collège des apôtres et le no24 sur la « diaconie » des apôtres et de lems successeurs. Ce sontsurtout les n""24 et 26 de Lumen gentium qui adoptent la
 
ÉTAPES D'UNE ECCLÉSIOGENÈSE

perspecti.ve narrati.ve des Actes des Apôtres, explicitement documentée dans les renvois scripturaires. Heureusement certains Pères conciliaires n'ont pas oublié la petitesse des débuts et en ont laissé quelques traces dans le texte ; aussi lit-on maintenant dans la constitution cette petite phrase, introduite à la dernière ute:« Dans ces communautés, si petites et pauvres qu'elles puisse-nt être souvent ou dispersées, le Christ est présent par la vert duquel se constitue(consociatur) l'Église une, sainte, catholique et apostolique» (LG, 26). C'est précisément ce que nous pouvons percevoir aujourd'hui dans beaucoup de régions, en France et ei:i Europe, et ce qui représente le point de départ de
notre lecture des Actes avec l'œil exercé par la lecture du décret missionnaire de Vatican II.
2.	L'approche de type lucanien des f:glises locales soulève cependant la question de savoir comment relier la perspective lucanienne et synoptique défendue dans Lumen gentium, 5, qui situe le « commencement» de l'Église dans le message de Jésus sur le Royaume de Dieu, et l'ecdésiologie paulinienne et deutéro-paulinienne, et surtout la doctrine sur les charismes introduite aux n"' 4 et 7 de Lumen gentiunt1. Comme nous l'avons déjà évoqué, le n<>4 d'Ad gentes, influencé par Luc, se réfère simultanément à la doctrine paulinienne des charismes de
Lumen gentium 4 et 7 et établit ainsi une connexion, du moins sous forme d'une esquisse2.
Or, le message de Jésus sur le Royaume de Dieu ne peut être séparé de ses signes et gestes messianiques dont l'effet libérateur

l. Cf plus haut, p. 312 sqq.
2. Cf. plus haut, p. 435 sqq.

443
 
URGENCES PASTORALES DU MOMENT PRÉSENT

est explicitement signalé en Lumen gentium, 5 (Le 11,20). Da_ns la ligne d'Isaïe, ces signes du commencement du emps messianique sont des personnes vivantes : les pauvres qm entendent la Bonne Nouvelle; les prisonniers qui sont libérés ; les aveugles qui voient (Le 4,17-21-7,21-23), etc.; Luc en établit une liste pour la compléter ensuite dans les Actes des Apôtres1. a?s la doctrine des charismes de Paul, plutôt axée sur la vie mtérieure de l'Église, il s'agit également de perso_nnes vi; ntes, e seulement dans un deuxième temps de fonctions prec1ses qm peuvent être comprises et reçues comme des modalités d'apparition de la grâce; cette importante nuance est retenue dans Lumen gentium, 7 à l'aide du concept charismaticus2• Cependant, selon Paul, les « membres du corps que nous tenons pour les plus faibles» et« ceux que nous tenons pour_ les moi?s ?norables » (1 Co 12,22 sqq.) participent de mamère part1cuhere à la construction du Corps du Christ.
Ici apparaît le point de convergence messianique t pneu­
matologique décisif entre la vision synoptique luc menne et l'approche paulinienne. L'idée fondamentale del 1Éghse comme
« sacrement universel du salut », bien présente dans Lumen gentium (LG, 1, 9 et 45) et reprise à nouveau dans Ad e es (AG, 1 et 5), acquiert en cet endroit n n uvell pla s_1 1hté critique3 : les signes messianiques, qm s a;ere t intfrevisibles, et les charismes, donnés hic et mmc par 1Espnt, depassent la

I. Cf- plus haut, p. 345 sqq.
2.	Cf. plus haut, p. 312, et la distinction de P. Congar entre « sacrements- choses » et« sacrements-personnes», p. 352 sqq.
3.	Cf plus haut, p. 343 sqq.

444
 
ÉTAPES D'UNE ECCLÉSIOGENÈSE

sphère classique des sept sacrements et dépassent même en un certai? s ns, l'Église. Signes messianiques et charismes euvent ê!re reums dans le concept biblique de mysterion dont la dirnens10n corporelle et symbolique est ressaisie de manière précise dans sa traduction par sacramentum. Cela nous conduità faire valoir, avec Ad gentes (4), la dimension événementielle et historique du mystère, face à une ritualisation unilatérale, età faire passer notre attention première des gestes significatifs aux personnes elles-mêmes et à leur sollicitude mutuelle (1 Co 12,24 sqq.) comme étant les signes messianiques par excellence.
3. Ce e insistance sur des « personnes significatives» nous recondmt vers le récit lucanien et vers ce qu'il nous apprend, dans une même perspective génétique ou inductive, de ces personnes et de leur relation avec l'Église naissante. Selon l'ordre génétique, elles précèdent l'émergence des groupes ou com, una tés sur place; ce qui est d'ailleurs enregistré par la des1gnat1, n ?u de_uxième livre de Luc qui ne s'intitule pas
« Actes del Eghse naissante » mais, depuis saint Irénée« Actes des_ foAtres ». C mme il a été signalé à plusieurs rep' rises, le
trots1eme Évangile nous offre la phase constitutive de ce jeu
re ationnel entre Jésus et quelques-uns, avec l'appel des premiers disciples, l'institution des Douze et le début d'une étonnante prolifération au moment de l'envoi des Soixante-douze. Cette prolifération continue à œuvrer dans le second livre où s'ajouten_t le  ept don l'un, _l'itinér nt Philippe, jouera un rô e particulier. On v01t enswte paraitre, au sein des Églises naissantes, des« prophètes »  personnes qui, pour faire bref,

l. Sur ce principe dei< prolifération», cf. plus haut, p. 31I-316.

445
 
URGENCES PASTORALES DU MOMENT PRÉSENT

parlent sous l'influence de l'Esprit - et des « hommes chargés de l'enseignement» (Ac 13,1 et 1 Co 12,28) jusqu'à ce qu'émergent « les anciens» (presbuteroi), établis dans chacune des E.glises fondées par Paul et Barnabas. S'il y a, au point de départ, une structure ferme, la relation entre Jésus et les Douze, d'autres figures s'inventent au gré des circonstances et au fur et à mesure que le nombre des croyants augmente.
Ces développements résonnent évidemment avec notre actualité pastorale, surtout aux endroits où le tissu communautaire est très distendu, voire inexistant. Il faut que nous nous en rappelions au moment où nous entreprenons de tracer un chemin type de genèse ecclésiale supposant à la fois la vision esquissée dans le chap. 7 de notre parcours et la manière de procéder explicitée dans le chap. 8.


ET AUJOURD'HUI ?

Derrière notre diagnostic d'une « exculturation >> toujours plus grave de l'Église d'Europe (surtout occidentale), qui a transformé la France et d'autres régions en pays de mission, se tient la conviction théologique selon laquelle l'avenir ne peut être abordé qu'à partir d'un rapport créatif avec les « origines>> du christianisme. D'où notre tentative de souligner la« perspective fondationnelle » et « génétique » dans les textes conciliaires et la fonction à la fois critique et inspiratrice des Écritures. Il nous reste, pour finir, à distinguer quelques «seuils» ou étapes d'une telle genèse dans notre situation actuelle, sans reprendre tout ce qui a été déjà apporté et fondé dans le chap. 7. Nous
 
ÉTAPES D'UNE ECCLÉSIOGENÈSE

visons plutôt la simplicité et l'utilité, car notre but est d'aider telle communauté concrète à discerner le ii point » où elle en est sur le hemin qui pourrait la conduire d'une << pastorale de reproduction » vers une « pastorale missionnaire ».

1.	LA GENÈSE D'ÉGLISE
DANS UN << ESPACE HOSPITALIER »

L'Église naît et renaît là où la fois'engendre. Entendons-nous bien :« foi»  ne désigne pas immédiatement foi en Dieu ou
e? C rist, m s d' b r et vant tout la capacité mystérieuse d un etre à fazre credtt a la vie, à rester debout, même dans les moments les pl s difficiles, en espérant que la vie tient sa promes e, perspective longuement développée dans la deuxième
p rt1e de cet ouvrage. Personne ne peut poser cet acte à la place d	autre. Pourtant cette foi s'engendre; si fragile et cachée so1t.-elle, elle peut être ranimée par ceux qui la perçoivent eyt cr en . ela e pe t se faire que dans un espace hospitalier, qu ils aglsse d un heu ecclésial, d'une maison privée ou d'un
d s multiples espaces de fortune que nous habitons à l'improVIste, par exemple lors d'une conversation avec autrui, etc. Le deuxième chapitre d'Ad gentes parle du «témoignage)> ou de la
« présence» des chrétiens dans un milieu non chrétien comme point de départ d'une genèse <l'Église sur place et utilise ici les co cepts de« conversation » (conversatio),d'«entretien» (colloquium) et de «dialogue» (dialogus) pour décrire ces situations de« présence» où de l'inattendu peut se produire.
Il arrive alors que ceux qui ont bénéficié d'une telle présence mettent ces« croyants-chrétiens >> en position de {{ témoins».
 

446	447
 
URGENCES PASTORALES DU MOMENT PRÉSENT	ÉTAPES D'UNE ECCLÉSIOG ENÈSE

 
Ils les interrogent et leur donnent ainsi l'occasion de révéler comment ils ont été eux-mêmes engendrés à la foi par d'autres et comment ces « relais » les ont mis en relation avec le Christ Jésus. L'Église naît en ces rencontres significatives où l'intérêt gratuit pour la« foi» d'autrui ouvre en même temps un espace où celui-ci peut découvrir le Christ. C'est sur ce «seuil» fondamental qu'est située« l'annonce de l'Évangile»,« partout où Dieu ouvre une porte à la prédication », selon la belle formule du décret missionnaire (AG, 13, 1).
Une communauté chrétienne (ou son conseil pastoral) peut
alors s'interroger sur son esprit d'hospitalité, non seulement sur un plan collectif mais aussi du côté de ses membres, non seulement en termes d'accueil mais aussi sur l'intérêt désintéressé qu'elle porte à son environnement social, au mal-être de telle personne ou de tel groupe, bref aux lieux où la (( foi » en la vie est en jeu. Cette interrogation peut être un tout premier pas vers une « pastorale missionnaire».

2.	LA RELATION À L'ÉCRITURE SAINTE
Un nouveau « seuil» est franchi quand !'Ecriture entre dans le champ visuel de ceux qui se côtoient déjà par ailleurs ; ce qui ne va nullement de soi dans notre société médiatique. Il faut d'ailleurs distinguer ici très clairement entre la Bible comme classique de-la culture européenne et expression d'une certaine humanité et l'Écriture sainte comme livre de l'Église. Cette différenciation se répand de plus en plus dans nos sociétés laïques
 
entrer en jeu le critère distinctif de la pratique religieuse1. La lecture et l'étude communes de ce livre permettent aux chrétiens d'identifier, au contact avec d'autres chrétiens, la genèse de leur propre foi spécifiquement messianique, de faire peut-être l'expérience du lien intime entre l'écoute de l'Évangile et de son annonce et de comprendre comment l'Église a trouvé, pas à pas, sa forme et continue à la trouver aujourd'hui.
Parfois des événements, vécus par les uns ou les autres entrent dansl' hange, ouvrent vers d'autres dimensions psychologiques ou politiques de la lecture en commun et conduisent à tisser des lien plus approfondis _entre les membres d'un groupe. Il se peut aussi quela confrontation aux textes bibliques et à leur humanité crée subitement un climat de recueillement où le silence vécu en commun se fait méditation pour les uns et prière pour les autres.
Là enco e, une commun uté chrétienne peut s'interroger sur la place quelle donne aux Ecritures en son sein, non seulement dans sa liturgie, mais aussi à d'autres occasions, en d'autres lieux ou da s ses groupes et petites communautés plus restreintes. Elle doit se e ander si,,dans les faits, elle croit réellement que c: sont es Ecn u.res (quelle confesse comme saintes ou inspiees) q 1 vont 1 aider à trouver son propre chemin pastoral en mteract1on avec ce qui arrive dans nos sociétés.
 
et permet de réunir autour d'une même table des chrétiens, des		
 
sympathisants et des non-chrétiens, sans immédiatement faire
 
l. Cf. plus haut, p. 380-383.
 

448	449
 
URGENCES PASTORALES DU MOMENT PRÉSENT


3.	LA DÉCOUVERTE DE PERSONNES NOUVELLES ET DE LEURS CHARISMES
Un autre« seuil» est passé quand, dans cet ensemble (< multîtudinariste » de chrétiens, sympathisants et autres, se profilent des personnes, parfois des personnes nouvelles, orteus_es d.e
<<charismes» spécifiques. C'est une double attention qm ,d01t alors émerger ici : une attention aux: personnes, aux. dons qu elles ont reçus et au don qu'elles représent nt pour la c mmunauté et une attention renouvelée aux besoins de celle-ci n ter es defonctions élémentaires à assurer. Ces deux types d attention sont difficiles à articuler, car une focalisation trop forte sur les besoins risque d'occulter la perception de ce qui est eff ct vement donné par Dieu à telle communauté. On p urra1t .etre tenté de se servir immédiatement du schéma des trms fonctions classiques du Christ et de l'Église -	les fonctions sacerdotale, prophétique et royale -	comme critère de discernement • Cela risque d'être prématuré et on passerait sans doute à c,ôté d s charismes les plus importants au cours de cette étape, a savoir celui de la« visite» et celui appelé plus haut« charisme de sourcier ou détecteur de chercheurs de sens2 ». Or, si ces char smes sont déjà en place, on trouvera rapidement leur pend nt mtraecclésial, si je puis m'exprimer ainsi: à sav ir le <, cha:1sme» d: l'accompagnateur de catéchumènes et celm de catéchete c arge de l'initiation chrétienne, comme il est dit dans le demaème article du décret missionnaire Ad gentes (AG, 14).

1.	Cf. plus haut, p. 189-191.
2.	Cf. plus haut, p. 317 sqq.

4S0
 
ÉTAPES D'UNE ECCLÉSIOGENÈSE

On perçoit bien que cette troisième étape représente un véritable carrefour, délicat à passer, sur le chemin d'une pastorale de reproduction vers une pastorale missionnaire. Elle exige en effet une sensibilité pastorale plus affinée par rapport à ce qui se passe en profondeur à la « porte baptismale» de la communauté chrétienne : nous l'avons bien vu quand il a été question des demandes de baptêmes'. Conditionné par le rayonnement hospitalier de l'Église (1re étape), l'intérêt que d'autres lui portent (mouvement centripète) dépend aussi de sa «présence» auprès d'eux (mouvement centrifuge) et de sa capacité de prendre au sérieux celles et ceux qui frappent effectivement à sa porte. Mais cette sensibilité missionnaire ne peut pas devenir réalité si la communauté n'apprend pas en même temps à mettre en lien, sans aucun forcing, les personnes (éventuellement nouvelles) que Dieu lui donne et ses propres besoins en termes missionnaires. Ce n'est pas en effet le prêtre seul qui peut être porteur d'une telle sensibilité pastorale et du discernement que celle-ci exige. D'autres doivent prendre le relais.

4.	UNE COMMUNAUTÉ QUI SE MET À DÉLIBÉRER

C'est précisément à ce moment que peut s'engager une quatrième étape de l'ecclésiogenèse missionnaire traitée dans le troisième article du décret Ad gentes sous le titre de « la formation de la communauté chrétienne (De communitate efformanda) » (AG, 15 et la suite), appelée aussi plus haut « devenir sujet

1. Cf. plus haut, p. 345 et 348 sqq.

4S1
 
URGENCES PASTORALES OU MOMENT PRÉSENT

missionnaire » de l'Église'. La difficulté des trois premières étapes réside en effet dans l'énorme différence des prises de conscience au sein de nos communautés. Parfois seules certaines personnes
-	parmi elles, le prêtre -	ont expérimenté le lien entre l'écoute de l'Évangile et la nécessité intérieure de l'annoncer2 et sont donc conscientes des enjeux missionnaires du christianisme et de leurs conditions de réalisation; parfois c'est le conseil pastoral qui porte cette conscience, plus rarement oute, une communa :é. La perception de ces différences et de bien d autres ncore (he:s à des options culturelles, politiques, anthropologiques,, cclesiales, etc.) conduit vers le désir qu'une communauté chretlenne soit d'abord un lieu où on écoute l'autre, mais aussi où l'on attend sa « participation active» (SC, 11), non seulement dans la liturgie, mais également et plus fondamentalement dans 1 domaine du témoignage et de la simple présence auprès d'autrm, chacun vivant cette exigence missionnaire selon sa mesure. Et puisque rien ne peut être forcé, tout dépendant du travail de l'Esprit saint dans les consciences et entre elles, seuls l'art de la délibération et la synodalité dont il a été longuement question-'\ sont susceptibles de créer un« sentir » commun et une véritable conscience communautaire qui transforme une Église locale en véritable « sujet».
On peut espérer qu'une communauté qui a commencé à compter sur tel ou tel charisme et sur des personn:s ayant pris conscience de ce qui se passe à la « porte baptismale »

1.	Cf plus haut, p. 318-328.
2.	Cf plus haut, chap. 4.
3.	Cf plus haut, p. 325-327 et p. 406-416.

4S2
 
ÉTAPES D'UNE ECCLÉSIOGENÈSE

de l'Église (3e étape) passe aussi ce nouveau seuil qui consiste à faire participer un maximum de ses membres à ses orientations d'avenir, celles-ci ne reposant pas sur la présence du prêtre (toujours accueilli comme nécessaire garant apostolique de l'annonce évangélique), mais sur une véritable conscience partagée de la communauté dans son incarnation locale. Ce
point mérite, comme les précédents et ceux qui suivent, un examen régulier.

5.	LA PERCEPTION
DES DIMENSIONS CORPORELLES DE LA FOI

Un« seuil » d'un autre ordre est passé encore, quand 1a dimension corporelle de la foi est davantage perçue : c'est ici qu'interviennent la sacramentalité de l'Église au sens défini plus haut' et les signes sacramentels, le baptême en premier lieu, le repas du Seigneur, etc., les services pastoraux et le pastorat.
Si l'on a fait l'expérience de la naissance de l'Église dans nos rencontres et relations quotidiennes (1re étape), alors on comprend aussi le caractère relationnel des sacrements2• Ce sont toujours des personnes qui posent des signes et sont ou deviennent ainsi des signes : celui qui est baptisé, mais aussi celui qui est chargé d'un ministère. Le Repas du Seigneur mène à son accomplissement cette « transition » du « poser un signe » à « être un signe» et implique les croyants dans le processus relationnel initié par Jésus au sein de la société galiléenne, processus qui trouve son ultime crédibilité en son don de soi à« quiconque».

1.	Cf. plus haut, p. 343 sqq.
2.	Cf plus haut, p. 352 sqq.


453
 
URGENCES PASTORALES DU MOMENT PRÉSENT

C'est par cette voie aussi que le« souci pastoral» pour l'avenir de l'Évangile et de l'Église peut naître ez  quelques-uns et que l'appel à donner forme à ce « souci» dans sa pro?re existence peut être entendu. La vocation, l'institution collégiale et l'envoi en mission des Douze ainsi que de leurs successeurs peuvent tout à fait être compris dans le cadre_ de la _structure relationnelle et événementielle du Règne de Dieu qm f nde la sacramentalité de l'Église. La question, qui tour ente ?.1en _des chrétiens aujourd'hui à propos du «pourquoi» de 1mstl tion ecclésiale, ne peut plus trouver de réponse p r le b1a s d'une argumentation purement sociologique ou juridique, mais nécessite une justification théologique « simple), que seul pe t offrir le concept de « mission ,, déjà impliqué dans l«' Évangile
de Dieu1 ».
Or, beaucoup de chrétiens ne perçoivent plus que le v,ers t
rituel ou festif du régime liturgique et sacramentel de 1Église dont il a été longuement question au chap. 72 . C' est alors qu'il faut rendre réellement - ce qui veut dire corporellemen-t
perceptibles le lien de nos rituels avec _les.gestes év n?éliqu_es "'.-1 Christ Jésus et leur signification mess1amque et m1sswnmur,e il en va du rayonnement même de l'Évangil: qui f t appelà	os sens et à l'émotion. Certes, la beauté des celébrat1ons et de 1 espace liturgique y est pour beaucoup ; mais l'art et le senti11:ent esthétique qu'il est capable de susciter restent des phénomenes fugitifs tant qu'ils ne parviennent pas à toucher le fond es cœurs humains, cet endroit mystérieux où chacun peut farre
 
ÉTAPES D'UNE ECCLÉSIOGENÈSE

l'heureuse expérience de l'intimité de Dieu et de la proximité du prochain, et peut puiser l'énergie de la sortie de soi.
Il ne suffit pas de se décharger du soin de la corporéité de la foi sur des équipes liturgiques capables d'appliquer quelques bonnes recettes. C'est un signe de maturité que pose une communauté, quand, tout entière, elle prend conscience de cet enjeu et parvient à examiner paisiblement sa situation sur ce point, sans absolutiser des différences légitimes de sensibilité. Cela ne va pas de soi, reconnaissons-le, car même sur un plan plus large, l'fglise de France n'a pas réussi jusqu'à maintenant à engager une délibération collective sur cette pratique hautement sensible1•

6.	LA PRISE EN COMPTE DE LA DIMENSION TOUJOURS PLUS UNIVERSELLE DE L'ÉGLISE
Un autre « seuil» est encore à franchir dès lors qu'une communauté, « si petite et pauvre soit-elle» (LG, 26), perçoit que la « fraternité » chrétienne dépasse toutes les frontières d'espace et de temps et qu'elle éprouve alors le désir d'un échange plus intense avec d'autres groupes et communautés. L'hospitalité prend alors figure, des visites mutuelles ont lieu, les engagements dans la société s'affermissent: la communauté devient sujet de ses actes et signe sacramentel d'une unité toujours plus ample
-		« catholique ». Simultanément,on voit naître un rapport nouveau à la tradition, capable de dépasser des oppositions stériles et sans cesse rejouées entre progressisme et traditionalisme : la
 

1.	Cf. plus haut, p. 327 sqq.
2.	Cf plus haut, p. 351-367,	1. Cf plus haut, p. 189 sqq.

454	455
 
URGENCES PASTORALES OU MOMENT PRÉSENT

gratitude envers les anciens qui nous ont communiqué leur foi à travers des écrits, des monuments et des institutions de toutes sortes va de pair avec la liberté à leur égard et le souci de s'inscrire de manière créatrice dans la trame qu'ils nous ont laissée.

7.	CONTEMPLATION
La genèse de l':Sglise s'achève -	provisoirement-	quand une communauté passe le « seuil » dela contemplation.A vrai dire, elle rejoint alors consciemment l'impulsion initiale du parcours que nousvenons d'accomplir. La << moisson >} est abondante pour ceux qui savent la voir : elle comprend non seulement la fécondité de
la foi des chrétiens, mais surtout le « simple faire crédit à la vie» que perçoivent et ravivent ceux qui sont proches d'autrui. Or, être
« témoin >> de ce qui se passe en quelqu,un ou dans les profondeurs de nos sociétés peut susciter l'action de grâce et la supplication, parfois seulement un gémissement ou l'adoration••• Dans ces actes de prière, l'Église se dessaisit de ce qu'elle reçoit et découvre qu'au sein de l'humanité l'Esprit est en train de construire un
«temple» qui n'est pas fait de mains d'hommes; en admirant ce travail de l'Esprit elle devient « corps du Christ » et reconnaît que Dieu est l'origine abyssale d'un « peuple>>aux dimensions mystérieuses et en attente d'une paix universelle (LG, 17). Notre récit rejoint ici le début du texte de la Constitution.

REGARD SUR LE CHEMIN PARCOURU
On aura compris que le «chemin» retracé ici permet une multiplicité de variantes et ne doit donc pas être figé en un schéma linéaire. Il s'agit plutôt d'une« carte routière» qui peut

456
 
ÊTAPES D'UNE ECCLÉSIOGENÈSE

s'avérer utile quand la pratique pastorale est menacée par l'illisibilité et la discontinuité postmodernes et quand il faut aider les communautés à déchiffrer leurs propres itinéraires. Bien d'autres étapes peuvent devenir importantes et s'inscrire dans les interstices du schéma élémentaire qu'on vient de présenter. N'oublions pas non plus que telle communauté peut connaître des phases de régression, voire disparaître, tandis qu'à d'autres endroits des foyers de vie chrétienne se développent et prennent une figure communautaire.
Le décret missionnaire Ad gentes et les Actes des Apôtres partent évidemment de la situation plus radicale de pays, voire de tout un Empire, qui ne connaissent encore ni l'Évangile de Dieu, ni t>Église. Dans nos régions devenues des pays de mission, ce n'est pas le cas. La conception génétique de l'Église est alors une manière de sortir de la simple reproduction - qui serait devenue inconsciente des enjeux de l'Évangile - pour reparcourir patiemment le chemin du « devenir ecclésial», en y intégrant dès le départ l'élément essentiel qu'est la mission, et donc le devenir « sujet missionnaire » de ]'Église. Le fruit d'un tel parcours est alors une« ré-forme» qui s'enracine dans une expérience évangélique et s'appuie sur des prises de conscience successives, susceptibles de donner progressivement à notre vécu ecclésial une forme véritablement spirituelle ou expérientielle. Le langage du<<seuil» et del'« étape», utilisé dans le parcours
«narratif» qu'on vient d'effectuer, marque à l'évidence que
l'essentiel de l'attitude pastorale consiste à se rendre et à rester sensible aux « événements qui se produisent parmi nous » (cf. Le 1,1), la plupart du temps à l'improviste.

4S7
 
URGENCES PASTORALES DU MOMENT PRÉSENT



CONCLUSION

Rassemblons, une fois encore, quelques acquis.
1.	On peut s'étonner, et m_ême exprimer des doute , devant la perspective génétique de l'Eglise que nous veno s d e;4'lorer dans ce dernier chapitre. Or, il faut se demander s1 cet etonn, ment ne provient pas du simple fait que nous n_ou somme deJa tellement habitués à la situation de crise des Eglises au sem de nos sociétés européennes, qu'on la perçoit comme fatalement irréversible. Dès que la conscience missionnaire émerge en s force évangélique, une conception ecclésio-génétique -	c ll -ci ou une autre -	s'impose : on est inévitablement co_ndu ta la question du chemin à emprunter pour sortir de cet,te s1tua 10_n et de l'horizon vers lequel se mettre en route ; le decret m1ss1onnaire Ad gentes comme les Actes des Apôtres confirment cette
hypothèse.	,
Avancer sur ce chemin implique le respect de ce qu on pour
rait appeler une« normativité ecclésiologique », respect g_arant dans l'ensemble de l'ouvrage et surtout dans ce chap1tre- 1 par un va-et-vient entre les Écritures et les documen:s conci­
liaires, en particulier entre Ad gentes et Lumen fent1 1:7· O_n repérera difficilement des éléments importants de1eccles1ol g1e catholique qui n'ont pas trouvé leur place (ou ne pourraient pas la trouver aisément) dans ce qui vient d'êtr: prop é. _En revanche, il nous faudra bien apprendre, au sem del  Eg_hse, à mieux distinguer tel schématisme doctrinal et normatif et l'itinéraire concret de telle Église ou communauté locale. Ce n'est pas parce que l'accent est mis, dans ce livre, sur ce vernuit

4S8
 
ÉTAPES D'UNE ECCLÉSIOGENÈSE

historique et génétique qu'on aboutit à une vision réductrice de l'Église, loin s'en faut.
Cette insistance sur le processus historique au cours duquel l'Église prend forme en s'adaptant au moment et au lieu où elle s'incarne est le fruit de la vision tripolaire qui nous a accompagnés depuis le début de notre parcours1 et qui relie inséparablement le référent ultime de la foi chrétienne, l'Évangile du Règne de Dieu, la situation historique de la société qui est son espace d'accueil éventuel et l'actuelle figure de l'Église en sa forme multiple et polyédrique (enregistrée par toutes sortes d'enquêtes). Prendre réellement en compte cette historicité de l'Église et, surtout, accepter de voir qu'il y ait des situations en demi-teinte, voire de «crise» - ce que le décret missionnaire du Concile reconnaît comme tout à fait possible (AG, 6 § 2) -, cela nous conduit inévitablement à nous interroger sur les carences et surtout sur les ressorts d'une réforme pour y remédier. Aucun doute ne subsiste sur la réponse : ces ressources se trouvent dans l'expérience missionnaire, telle qu'elle a été présentée au chap. 4 et explicitée ensuite dans les deux chapitres suivants. La question plus précise est alors celle-ci : comment et par quel chemin laisser advenir cette expérience fondamentale au sein et à partir de nos communautés existantes.
2.	A cet endroit et si la question est réellement entendue, le concile Vatican II peut nous donner de précieuses indications, à condition de dégager de son corpus textuel, non pas une vision statique, mais un ensemble processuel complexe, constitué d'une vision attirante, d'une série de méthodes ou de manières de

l. Cf plus haut, p. 54 sqq.

459
 
URGENCES PASTORALES OU MOMENT PRÉSENT	ÉTAPES D'UNE ECCLÉSIOGENÈSE

 
procéder, articulées entre elles, et d'une esquisse des étapes à parcourir en direction de ce qui est perçu à l'horizon. Cet ensemble très souple (que nous venons de solliciter dans les trois derniers chapitres de cet ouvrage) est parfaitement adapté à notre situation spirituelle où nous devons à la fois compter sur les surprises de l'Esprit de Dieu et développer en même temps des dispositifs qui nous permettent de nous disposer à sa manifestation, intérieurement et collectivement.
Les sept étapes d'un itinéraire ecclésiogénétique, que nous venons de parcourir, ont été découpées, sur la base du décret missionnaire Ad gentes et des Actes des Apôtres, de façon à respecter et à rendre même déterminant la double dynamique qui vient d'être énoncée. La naissance d'une expérience missionnaire suppose dans un premier temps qu'une communauté se laisse surprendre par ce qui se passe en telle personne ou en tel groupe, non seulement en tel catéchumène, mais bien plus largement en «quiconque», qu'elle éprouve donc concrètement d'être sans cesse « précédée » par !'Esprit. Ce n'est pas quelque chose que l'on commande ou que l'on peut produire par soi-même, mais pour percevoir cette « précédence », des dispositifs peuvent être mis en place : des espaces hospitaliers (1re étape), la lecture commune des Écritures qui ne cessent de nous confronter à la diversité quasi infinie des présences de l'Esprit dans l'histoire (2eétape) et l'attention à ce que l'Esprit donne aux personnes (3• étape) - et d'autres dispositifs qui peuvent encore être inventés. A partir d'un certa.in moment, ce qui est éprouvé par des individus doit devenir une expérience collective et un bien commun. C'est ce qui peut se passer entre la troisième et la quatrième étape et
 
déterminer toute la suite de l'itinéraire. De nouveau, des dispositifs doivent aider à percevoir la communauté tout entière, et non seulement tel individu, comme « sujet missionnaire » : la délibération et la découverte de l'œuvre de l'Esprit dans l'inattendue entente mutuelle (4c étape), l'expérience d'une surprise spirituelle qui devient possible, quand on prend soin de la corporéité de nos célébrations et de l'Église (Se étape) et la découverte du travail de l'Esprit de Pentecôte dans une Église qui ne cesse de nous surprendre par ses dimensions à la fois universelles et polyédriques. C'est la contemplation sous ses multiples formes (7e étape) qui achève l'itinéraire. Elle rend aussi intérieurement évident que cet itinéraire ne s'achève jamais, tant que l'histoire dure, et qu'il faut reparcourir sans cesse ces différentes étapes - ou d'autres - avec toujours plus de profondeur.
3.	Ce qui vient d'être proposé est certes bien exigeant pour nos communautés chrétiennes. Mais l'expérience montre que celles-ci sont en attente d'une certaine exigence et de rigueur. Ce qui, dans beaucoup de cas -	paroissiaux ou diocésains - pose plutôt problème, c'est l'absence de culture de la concertation, soit par absence de volonté commune soit parce que les changements sont imposés de l'extérieur par un petit groupe. Or la pédagogie de la réforme proposée dans ce livre tente de faire vivre aux communautés une expérience ecclésiale véritable. Cela suppose que les communautés acceptent d'établir avec leur pasteur un « état de santé >> de leur vie et examinent leur vitalité, non pas d'abord au nombre d'actions réalisées ou d'événements produits (ce qui n'est nullement négligeable), mais selon les critères spirituels et missionnaires proposés plus haut, à
 

46D	461
 
URGENCES PASTORALES DU MOMENT PRÉSENT
 
savoir la perception qu'elles ont du travail de l'Esprit saint dans leur environnement et en leur sein. Des exercices de relect :e, adaptés à chaque situation, sont alors à inventer. La der_mere partie de ce chapitre - nos sept é apes - y aura apporte une contribution et, espérons-le, une aide.
 





CONCLUSION GÉNÉRALE



L'ouvrage qu'on vient de lire se veut être un témoignage d'espérance réaliste, ou de réalisme qui donne à espérer. Pendant les vingt-cinq dernières années tous les grands indicateurs statistiques concernant la pratique et les croyances chrétiennes ont été divisés par deux; d'aucuns parlent d'un véritable déclin du christianisme en France et dans de larges régions d'Europe. S'il semble subsister un« archipel» catholique en France (avec 53 % de la population qui se disent catholiques, dont 23 % «engagés» qui se sentent rattachés à l'Église par leurs dons, leur vie familiale, parfois leurs engagements), il ressemble plutôt, au regard de leur pratique, à une pyramide avec, à la base, une immense majorité de très faibles pratiquants ou de non-pratiquants, et, au sommet, une fine pointe de personnes ayant une « pratique régulière» (peut-être 5 %), et par ailleurs assez divisées entre elles quant à leurs options de fond. Il est clair aussi - et les statistiques le montrent également depuis longtemps - que cette mutation gigantesque s'accompagne d'une érosion des représentations de la foi, réduites dans la plupart des cas à un système de valeurs

463

\chapter{devoir}

\cite{chen_histoire_2015}

\cite{chen_economic_1956}

\section{validation}
Mode de validation du cours   Compte rendu d’un livre ou de quelques chapitres d’un livre de votre choix parmi la liste suivante :  
\begin{itemize}
 
\item  Kenneth CH’EN, Histoire du Bouddhisme en Chine, traduit de l’anglais par Dominique Kych, Paris : Les Belles Lettres, 2015. Chapitres 8 – 13 inclus.   

\end{itemize}

Ce travail devra comporter 5 ou 6 pages en police Times New Roman 12, interligne 1,5.   Date de la remise du travail : au plus tard le 9 décembre 2023. 



 

\section{B. vermander critique du livre}
\cite{vermander_kenneth_2016}
 

1 Ce livre est la traduction d’un ouvrage de référence sur le bouddhisme chinois paru en 1964. Le travail de traduction réalisé est excellent, très soigné, avec retour aux textes chinois cités, et un utile lexique des noms et titres chinois. L’ouvrage lui-même est de facture très classique, présentant une histoire ordonnée par dynasties avec chapitres complémentaires sur la doctrine, la traduction du Canon bouddhiste, les temples, ou les moines éminents. L’auteur est particulièrement à son aise dans les résumés doctrinaux et la présentation des textes canoniques, mais on trouvera là d’excellentes synthèses sur bien d’autres sujets, par exemple sur l’organisation et l’économie des grands monastères.

2P rès de 380 pages sont consacrés à la naissance, l’arrivée en Chine, la croissance et l’apogée du bouddhisme, avant que la période du « déclin » commence avec les Song – une ligne narrative poursuivie presque inexorablement. Les dynasties Ming et Qing sont traitées très rapidement, et la description donnée de l’aggiornamento bouddhiste – de la période qui suit l’écrasement de la rébellion Taiping jusqu’à l’établissement de la République populaire de Chine – est clairement insuffisante. La date de publication de la version originale de l’ouvrage explique le ton très sombre des dernières pages (même si l’auteur semble parier en finale sur la résilience et la capacité d’adaptation du bouddhisme chinois, pari que les évolutions intervenues après 1980 ont justifié au-delà de ses espérances).

3 La publication en français d’une pareille synthèse est en soi une excellente nouvelle, même si l’on se demande un peu quel est le public visé. Pédagogique et de lecture aisée, l’ouvrage reste d’écriture assez conventionnelle, parfois monotone, si bien qu’il ne s’agit sans doute pas d’une synthèse idéale pour le grand public cultivé, auquel on recommanderait de préférence des ouvrages plus courts et plus vivants portant sur tel ou tel aspect du sujet couvert. En revanche, étudiants en histoire chinoise ou en sciences religieuses trouveront là un ouvrage de consultation fiable et commode, qui pourra orienter leurs lectures ultérieures. Le rôle de consultation et d’orientation joué par l’ouvrage est encore renforcé par l’excellente bibliographie critique rédigée par l’auteur, comme par la bibliographie additionnelle établie par Sylvie Hureau. Cette dernière annexe est extrêmement complète, très à jour, et ajoute grandement à l’intérêt de la publication.

4 Dans l’idéal, la bibliographie additionnelle aurait pu être accompagnée d’une synthèse relatant les principales évolutions intervenues dans l’approche de l’histoire du bouddhisme chinois depuis la publication de l’ouvrage. Elles sont nombreuses, qu’il s’agisse de l’écriture de l’histoire de l’école Chan, de la relation entre bouddhisme et religion populaire, des origines et de l’affirmation du « bouddhisme humaniste », ou des codes rhétoriques gouvernant les écrits hagiographiques. Le livre de Kenneth Ch’en a marqué un moment de la connaissance, et une évaluation historiographique d’ensemble aurait donné une valeur accrue et un horizon critique aux additions bibliographiques. Par ailleurs, puisque ce sont plus de cinquante ans de l’histoire du bouddhisme chinois qui ne sont pas couverts par l’ouvrage (davantage en fait, toute la partie consacrée à l’après 1949 n’étant guère utilisable), une courte synthèse sur cette période, et notamment sur le renouveau en cours, aurait également été bienvenue. Il est vrai que, tel quel, l’ouvrage est déjà fort volumineux et qu’il constitue une ressource des plus appréciables malgré son caractère quelque peu daté.

\section{Plan - Troisième partie
MATURITÉ ET ACCEPTATION}


\textbf{Chapitre VIII. Apogée du bouddhisme. La dynastie Tang.}

    \begin{itemize}
        \item Le mémoire de Fu Yi contre le bouddhisme.
        \item Taizong et le bouddhisme.
        \item Gaozong et le bouddhisme
        \item Le soutien de l'impératrice Wu Zhao :\textit{Soutien état, floraison exceptionnelle 691 : décret  priorité du bouddhisme sur le taoisme}
        \item Xuanzong . \textit{égalité taoisme et bouddhisme  officiel confucianisme mais en privé bouddhisme pour trouver soutien et consolation.}
        \item Le mémoire de Han Yu contre le bouddhisme
        \item La proscription de 845 ou la persécution de l'ère huichang.
        \item Les pèlerins chinois à l'étranger.
        \item Xuanzhao
        \item Xuanzang. .
        \item Yijing et la route maritime
        \item L'apport des pèlerins à la culture mondiale
    \end{itemize}

Chapitre IX. La communauté monastique

    \begin{itemize}
        \item Les différentes catégories de moines. \textit{p. 247. Normalement décision personnelle. mais sous les Tang, contrôle de l'Etat}. \textit{trois catégories : moines officiels, dans les temples d'état, moins privés, moines du peuple : 47 monastères d'état, 839 aristocratiques, 30 000 du peuple - Wei du Nord}
        \item L'ordination privée ou officielle des moines \textit{Sous la première moitié des Tang, ordination privée (sidu). Mais scandale de personnes devenant moine pour éviter les corvée. Institution en 747 d'un système d'ordination étatique. Puis vente de certificat d'ordination lors de la crise en 775. }
        \item Le registre des moines . \textit{enregistrement minutieux}
        \item La composition de la communauté monastique en Chine. \textit{période de préparation préliminaire au noviciat. accord parentale, ne pas fuir ses engagements temporels,  Etude des textes sacrés, service des hôtes, corvée, en tant que postulant, garde les cheveux. }
        \item L'examen d'ordination \textit{ pour devenir novice, réciter des feuilles du Sutra comme le sutra du lotus, explication de texte. On devient novice (tonsure) et la plupart ne deviennent pas moine}
        \item Les ordinations par faveur impériale et par achat de certificat
        \item L'origine sociale des moines. \textit{Les moines officiels famille aisée. }
        \item L'entretien des moines : nourriture et habillement. \textit{critique des confucianistes : dépendent des fidèles (ordre mendiants) Cout des 200 000 moines en 778, 6 millions dde ligatures, soit l'équivalent de 50\% du budget de l'état}
        \item La propriété des moines. \textit{Depuis les Tang, dot de 30 mu viager (Koufen), pour soutenir le Taoisme (et de façon incidente, le bouddisme)}
        \item Les fonctionnaires du \textit{samgha} et l'administration des monastères.\textit{Faguo et le bureau de supervision ders mérites. En 841-847, sous Wuzong, placé comme rite étranger (bureau des hôtes) }
    \end{itemize}

\textbf{Chapitre X. Les temples bouddhiques et le bouddhisme populaire.}

    \begin{itemize}
        \item Le nombre et le coût des monastères. \textit{les temples de nos jours semblent concus pour surpasser les palais impériaux (Ximingsi, Ximingsi. (西明寺). In Chinese, “Luminosity of the West Monastery,” located in the Tang capital of Chang’an (present-day Xi’an). 656 . Cout estimé d'un pavillon : 22500 ligatures Coup estimé très élevé par de nombreux fonctionnaires. Problème des monnaies de cuivre : entraine l'interdiction d'objet en cuivre d'où la confiscation en 845 de tout le cuivres des monastères.   }
        \item Les activités commerciales : \textit{moulins à grain et pressoirs, don des fidèles. Rôle d'hotellerie}
        \item Les moulins hydrauliques à céréales
        \item Les pressoirs à huile .
        \item Les hôtelleries
        \item Les Trésors inépuisables \textit{wujin zang : en cas d'excédents de dons ou de domaines agricoles. vente pour construction ou réfection de temples. taux d'intéret pouvait s'élever à 50\% ! p. 270 }
        \item Les terres de monastères.
        \item Les paysans des terres de monastères. Des serfs, des graciés, des orphelins. 
        \item L'exemption de taxes accordée aux monastères. Question pas facile. il semble que certains monastères bénéficiaient de réelles exemptions fiscales sur leur terre mais les autres. Sous les Tang, une réelle exemption pour les monastères reconnus P 275
        \item Les cours de mérite : \textit{pour obtenir l'exemption, les famille riche établissaient une cour de mérite. Une partie du domaine, un temple mais exemption demandée sur la totalité du domaine.  On se dispensait même d'édifier le temple. (sous les Tang puis les Song}
        \item Les différentes catégories de temples.
        \item L'administration intérieure des temples.  Un triumvirat (sansang), élus par les moines. Des postes secondaires (le contrôleur (\textit{dianzuo}), le comptable (\textit{zhisui}) et l'intendant (\textit{kuzi})
        \item Les fêtes célébrées dans les monastères. Mission religieuse toujours essentielle, Mahayana : caractère universel du salut : entraîne une vraie mixité sociale, ignorée par le taoïsme et le confucianisme. Des fêtes : célébration de l'anniversaire de l'empereur ou commémoration des souverains défunts  : célébrations officielles qui n'ont rien à voir avec le Bouddhisme. 
        \item La fête des lanternes : En parallèle, des fêtes religieuses tout au long de l'année avec la population locale : fête des lanternes.  début de l'année.
        \item La célébration de l'anniversaire de la naissance du Bouddha.
        \item La fête en l'honneur des reliques du Bouddha : à Xi'an, os de bouddhas, des offrandes, les plus pauvres vivants dans la crainte de ne pas réunir l'argent pour cela. "soldat offre son bras gauche". "brulés".  
        \item La fête des morts. fête le 15 du 7ème mois (6 mois après les lanternes). Maudgalyayana qui sauva sa mère du fond des enfers.
        \item Les assemblées de jeûne : importance. "zhai" jeune à partir de midi. stricte égalité de traitement.
        \item L'éducation religieuse. 
        \item Les bianwen ou récits bouddhiques merveilleux
        \item Les sociétés religieuses.des sociétés. difficile de les quitter; 
        \item Les activités charitables "le monastère joueait un rôle dans presque tous les domaines de la vie du fidèle bouddhiste sous les Tang". P 296
    \end{itemize}

Chapitre XI. Les écoles bouddhiques en Chine.
\begin{itemize}
    \item La secte des Trois degrés (\textit{sanjie jiao}). contexte social en Chine pendant la dynastie Sui. Xinxing. trois périodes : la période de la Loi correcte (les enseignements de Bouddha bien observés), la periode de la Loi contrefaite (simulacre), la période du déclin de la Loi (Loi sur le point d'être anéantie) dans lequel Xinxing vivait. Adaptation face à ce déclin (humiliations,...). Souvenrains Tang s'opposèrent à cette secte. disparut en 845.
    \item L'école du Vinaya.
    \item L'école Kosa
    \item L'école Tiantai. Zhiyi (538-597). Sakyamuni, le Bouddha historique n'est qu'une manifestation terreste du Bouddha éternel. Il a structuré les textes bouddhiques, nombreux. 
    \begin{itemize}
           \item La classification des sutras et des enseignements. classement selon la vie de Bouddha, d'abord très abstrait, puis agama plus pratique, puis grand véhicule (blame des arhat) , puis abstrait négativement, puis positvement.  certaines doctrines plus adaptées que d'autres (critère d'efficacité). Valorisation du sutra du Lotus.Mais ouvetture à tous les courants. 
    \item La triple vérité. pas de nomène en dehors des phénomènes (= noumènes). Vérité de vacuité, impérmanence et médiane. 
    \item L 'esprit absolu.
    \item Concentration et vue pénétrante. méditation permet de ne pas tomber dans l'illusion de la réalité. concentration (zhi) et vue pénétrante (guan). zhi : dharmas n'a pas de nature propre et donc pas de réelle existence (seules dans nos fantasmes).  
    Guan : dharma : créé par l'esprit
    \end{itemize}

    \item L'école Huayan.
    \item Les maîtres de l'école Huayan
    \item La doctrine de l'école Huayan
    \item L'école Faxiang.
    \item La doctrine de l'école Faxiang .
    \item L'école tantrique .
\end{itemize}

Chapitre XII. Les écoles bouddhiques en Chine (suite)

    \begin{itemize}
        \item L'école de la Terre pure
        \item Contenu du Satra de la Terre pure
        \item Avalokitesvara.
        \item L'école et les maîtres de la Terre pure en Chine
        \item L'école du Chan en Chine.
        \item Bodhidharma.
        \item Shenxiu, le sixième patriarche
        \item Shenhui remet en cause la légitimité de Shenxiu
        \item Huineng.
        \item Huineng et le nouveau Chan Qu'est-ce que le Chan?
        \item Les facteurs intellectuels qui favorisèrent l'essor du Chan
        \item Chan et taoisme.
        \item Le Chan est-il bouddhiste ?.
        \item La survie du Chan après la proscription de 845
    \end{itemize}

Chapitre XIII. Le Tripitaka chinois

    \begin{itemize}
        \item Les techniques de traduction.
        \item Problèmes de traduction
        \item Les catalogues de sutras
        \item Les éditions du Tripitaka chinois
        \item Les éditions modernes
        \item Le Sũtra du lotus.
    \end{itemize}



\section{Introduction}

Le livre \textit{Histoire du Bouddhisme en Chine} \cite{chen_histoire_2015} est la traduction d’un ouvrage de référence sur le bouddhisme chinois paru en 1964. Il couvre  la naissance, l’arrivée en Chine, la croissance et l’apogée du bouddhisme, avant que la période du « déclin » commence avec les Song. 
La date de publication de la version originale de l’ouvrage explique le ton très sombre des dernières pages \cite{vermander_kenneth_2016}.
Nous proposons d'étudier l'arrivée et la croissance en Chine (chapitre 8 à 13).






Plan
Apogée du bouddhisme. La dynastie TangS'appuyant sur la réunification accomplie par les Sui, la dynastie Tang(618-907) édifia un immense empire, qui couvrait non seulement le teritoire propre de la Chine, mais étendait également en Asie centrale. Bienque le dan impérial, se targuant de descendre de Laozi, fit favorable au taoisme, le gouvernement mena une politique de tolérance religieuse, qui laissa a chaque religion la possibilité de se développer. Le nestorianisme, l'islam et le manichéisme furent introduits en Chine durant cette dynastie et firent tous trois des adeptes parmi les Chinois. Si cet idéal cosmopolite fut adopté par les Tang, c'est qu'ils ne se considéraient pas uniquement comme empereurs des Chinois, mais également comme souverains de peuples dits barbares.Le bouddhisme, qui s'était déjà largement répandu en Chine, se développa comme jamais auparavant sous le patronage de certains empereurs Tang, si bien que sa puissance et son influence finirent par dépasser de loin celles du taoïsme. On pourrait dire que, sous les Tang, le bouddhisme atteignit en Chine l'âge de la maturité. Toutes les couches de la société, la famille impériale, la noblesse, les familles riches et puissantes, et jusqu'au petit peuple, lui accordèrent leur soutien. Tout en profitant de la position et de l'influence des grandes familles du royaume, le bouddhisme entretint des relations privilégiées avec le commun des fidèles par le truchement de diverses activités sociales et religieuses. Les cours spacieuses des temples bouddhiques servirent de jardins d'agrément aux foules des villes, les fêtes bouddhiques procurent distraction et divertissement aux masses villageoises et urbaines, les assemblées de jeûne et les prêches furent suivis par un très grand nombre de fidèles laïcs. Ce fut de cette capacité à répondre aux besoins de toutes lesL'important corpus de textes qui avait été traduit aux siècles précédents était désormais assimilé par les Chinois, prêt à être soumis aux diverses interprétations qui allaient servir de fondement aux différentes écoles.Certaines, comme celle du Tiantai et du Chan, dégagées de l'influence indienne, développèrent des traits propres à l'expression singulière du génie chinois. Dans les temples prospères des villes comme dans les monastères reculés de montagne, des moines à l'esprit audacieux débattirent des sutras sur lesquels étaient fondées leurs écoles. Pour traiter du bouddhisme sous les Tang, nous examinerons d'abord l'attitude du pouvoir impérial envers cette religion, puis nous étudierons le rôle du samgha dans la société des Tang et enfin, nous décrirons les différentes écoles qui virent le jour à cette époque.Il semble en général justifié de dire que le bouddhisme sous les Tang coqui un caractère plus spécifiquement chinois et s'identifia de manière plus orte, et sous son contrôle, à l'État.



\section{la proscription de 845 ou la persécution de l'ère Huichang}
\begin{itemize}

\item Dans l'état actuel de nos connaissances, aucun lien direct ne peut établi entre le mémoire de Han Yu et la proscription décrétée par l'emper Wuzong en 845. Nous ne savons même pas s'il avait lu ce document. Il avait quoiqu'il en soit, bien d'autres raisons de vouloir éliminer le bouddhisme.
\item  Certains historiens semblent faire de la proscription de 845 le fruit d'une  brusque décision, un épisode supplémentaire de la longue bataille idéologique qui opposa le taoïsme au bouddhisme. S'il est vrai que les taoïstes incitèrent l'empereur à prendre des mesures répressives, on ne peut réduire cet événenement à l'expression d'un affrontement idéologique. En revanche, les luttes entre factions qui sévissaient au sein de la cour semblent avoir joué un certain dans la décision impériale, les lettrés fonctionnaires s'alliant à l'empereur contre la religion étrangère et les eunuques la soutenant.
    \item 
\end{itemize}
Des considérations économiques telles que le désir de la cour de s'emparer et d'utiliser les immenses richesses détenues par les monastères entrèrent aussi en ligne de compte. Ce dessein apparaît clairement dans la déclaration du bureau du Secrétariat impérial, datée du septième mois de l'année 845 :
\begin{singlequote}
    Les images de bronze [des divinités bouddhiques] et les cloches devront être remises au commissaire du Bureau du sel et du fer, afin d'être fondues comme pièces de monnaie. Les statues de fer seront remises aux fonctionnaires locaux pour être transformées en outils agricoles. Les images en or, argent, jade ou autre matériau précieux devront être remises au bureau du Trésor public. Toutes les images en or, argent, bronze et fer détenues par les riches et les nobles devront être remises au gouvernement avant la fin du mois qui suivra la publication de ce décret [...] Les images d'argile, de bois et de pierre sont autorisées à demeurer dans les temples où elles sont placées'.
\end{singlequote}
The last sentence reveals that the motive in the minds of the
proponents was not complete suppression of the Buddhist religion,
but confiscation for state use of the vast economic wealth held by
the temples and monasteries
La perte de revenu que représentait l'exemption de taxes accordées aux 260 000 moines et nonnes, ainsi qu'aux 100 000 esclaves et aux innombrables laïcs employés au service de la communauté bouddhique (dont le n équivalait à celui des moines et nonnes réunis), ajoutée à celle dont béné une grande partie des terres possédées par les monastères, fut une des ra qui poussèrent l'empereur à prendre des mesures drastiques.Les histoires officielles des Tang donnent peu d'informations si divers événements qui ont conduit à la proscription de 845. Il se tr fort heureusement qu'un moine japonais très observateur, nommé E voyagea en Chine à cette époque et qu'il consigna dans son journal renseignements très précieux sur cet épisode : il y apparaît en partic très clairement que les mesures prises par Wuzong contre le bouddhisme sont étalées sur plusieurs années.
\item Ennin note, dès 841, un premier signe de défaveur impériale : à l'occas de son anniversaire, l'empereur octroya aux taoïstes, en remerciement cadeaux qu'ils avaient apportés, la permission de porter la robe pourp privilège qu'il n'accorda pas aux bouddhistes. Il en fut de même lors l'anniversaire de l'empereur en 842. Dans le courant de la même année, moine bouddhiste appelé Xuanxuan se targua de pouvoir défaire les Ouïgol détestés (un peuple d'Asie centrale) grâce à une épée magique, mais dès qu fut mis à l'épreuve, son imposture fut dévoilée.\textit{K. Ch'en, « The Economic Background of the Hui-ch'ang Suppression of Buddhism», Harvard Journal of Asiatic Studies, 19 (1956), p. 68}.L'empereur prit, au dixième mois, un décret ordonnant que les moines et nonnes qui pratiquaient l'alchimie ou la magie, avaient fui l'armée, portaient des traces de flagellation sur le corps, entretenaient des épouses ou violaient les règles monastiques, fussent réduits à l'état laïc. L'argent et les biens tels que domaines et jardins possédés par ces religieux devaient être de plus restitués au gouvernement. En conséquence de ce décret, 1 232 moines et nonnes résidant à l'est de la capitale, et 2 259 (on cite parfois également le chiffre de 2 219) résidant à l'ouest, retournèrent à l'état laïc au premier mois de 843. Au deuxième mois de cette même année, les mesures répressives touchèrent les temples manichéens. Elles ordonnaient que tous les biens, propriétés, champs, commerces, résidences, monnaies et marchandises en leur possession soient rendus à l'État, et interdisaient la présence de tout élément étranger sur leurs domaines. Cette dernière mesure visait particulièrement les Ouïgours, qui étaient alors un puissant soutien du manichéisme en Chine.Revenons un peu un peu en arrière pour expliquer leur présence sur le territoire chinois à cette époque. Afin de mater la révolte d'An Lushan au milieu du vie siècle, les empereurs Tang avaient fait appel à des mercenaires Perses, Arabes et Ouigours, ces derniers constituant le plus gros des troupes.Lorsque la rébellion fut jugulée, les Perses et les Arabes quittèrent le territoire, mais les Ouïgours restèrent en Chine, se comportant en maîtres à Luoyang et Chang'an, pillant et rançonnant, au grand dam des Chinois. Les autoritésimmiccantes devant ce problème car ces étrangers étaientsur l'Asie centrale. Lamais les Ulgous resteren en cune, se comportant en maitres a Luoyang et Chang'an, pillant et rançonnant, au grand dam des Chinois. Les autorités demeuraient impuissantes devant ce problème, car ces étrangers étaient soutenus par le puissant empire ouïgour qui régnait sur l'Asie centrale. La position avantageuse dont ils jouissaient incita les populations étrangères des deux capitales à confier leur richesse aux temples manichéens qui bénéficiaient de cette protection, un peu comme les marchands étrangers installés dans les ports chinois concédés par les « traités inégaux » du XIx® siècle placèrent plus tard leur argent dans les banques britanniques. Les autorités chinoises profitèrent cependant de l'affaiblissement de l'empire ouïgour, à partir de 832, pour reprendre l'initiative et mettre fin à la situation privilégiée de l'église manichéenne, en édictant, en 843, un décret qui visait principalement l'argent et les richesses déposés dans ses temples.À partir de 845, la répression contre le bouddhisme, provoquée en grande partie par les machinations du taoïste Zhao Guizhen et de ses deux acolytes, Deng Yuanchao et Liu Xuanjing, prit une nouvelle ampleur. Au troisième mois de cette même année, deux fonctionnaires confucéens recommandèrent la suppression des assemblées de jeûne. La tradition bouddhique avait en effet institué plusieurs périodes de jeûne, annuel ou mensuel. Les périodes de jeûne annuel avaient lieu les premier, cinquième et neuvième mois. Les périodes de jeûne mensuel étaient de six ou dix jours. Les premiers avaient lieu les 8°, 14°, 15°, 23°, 29° et 30° jours du mois, auxquels s'ajoutaient quatre jours (les 1°, 18°, 24° et 28° jours du mois) pour les jeûnes mensuels de dix jours.Gaozu avait émis, dès 619, un décret qui suspendait les exécutions capitales et l'abattage d'animaux durant ces périodes. Ce décret, qui fut maintenu par l'impératrice Wu Zhao en 692, semble être resté en vigueur durant toute la période des Tang, jusqu'à son abolition en 844. Les bouddhistes furent, cette même année, exclus pour la première fois des cérémonies qui accompagnaient le jour anniversaire de la naissance de l'empereur.Il existait depuis longtemps, au sein même du palais impérial, dans le Pavillon de la longévité, une chapelle où étaient conservées des images du Bouddha et des textes sacrés. Des moines, détachés à son service, pouvaient y circuler librement et y célébrer des cérémonies pour la protection et la prospérité de l'empire, ou fêter l'anniversaire de l'empereur. Écoutant les conseils de son maître taoïste, le fanatique Zhao Guizhen, l'empereur Wuzong fit détruire ces images, brûler les textes et renvoya les moines dans leurs monastères d'origine. Il fit installer dans cette chapelle des images de Laozi et d'autres divinités taoïstes, et poussa les lettrés de la cour à embrasser la religion taoïste. Ennin rapporte qu'aucun d'entre eux ne suivit cette injonction.En 844, Wuzong interdit le culte rendu à la dent du Bouddha et décréta que tous ceux qui feraient un don pour organiser cette cérémonie seraient passibles couns de canne de bambou. À l'occasion de la fête donnée enmonasteres a orgaet d'autres divinités taoïstes, et poussa les lettrés de la cour à embrasser la religion taoïste. Ennin rapporte qu'aucun d'entre eux ne suivit cette injonction.En 844, Wuzong interdit le culte rendu à la dent du Bouddha et décréta que tous ceux qui feraient un don pour organiser cette cérémonie seraient passibles de vingt coups de canne de bambou. À l'occasion de la fête donnée en 844 pour le repos de toutes les âmes, les fidèles firent aux temples bouddhiques des dons d'une exceptionnelle générosité, mais l'empereur les fit saisiret les offrit à un temple taoïste, ce qui provoqua la colère du peuple.Dès le septième mois de 844, la répression se durcit, les mesures visant cette fois la destruction progressive de la communauté bouddhique. Il y eut d'abord les décrets ordonnant la fermeture de tous les temples et sanctuaires mineurs, et la réduction à l'état laïc de tous les religieux dont le nom n'apparaissait pas sur les registres des monastères. Les images et manuscrits que contenaient les établissements mineurs furent transférés aux temples de plus grande importance, et les cloches et autres ornements qui s'y trouvaient, offerts aux temples taoïstes. Selon le moine japonais Ennin, trois cents établissements furent alors détruits dans la seule ville de Chang'an. Le troisième mois de l'année suivante (845), le gouvernement interdit aux monastères le droit d'établir des domaines agricoles et ordonna de dresser le recensement des esclaves employés à leur service, ainsi qu'un inventaire des richesses en argent, marchandises ou grain, en leur possession.wees prises conne le sumaha s'étaient limitées à rédure rde la comminauté qui n'abuervaient pas les réglementoles établissements mineurs furent transférés aux temples de plus grande importance, et les cloches et autres ornements qui s'y trouvaient, offerts aux temples taoïstes. Selon le moine japonais Ennin, trois cents établissements furent alors détruits dans la seule ville de Chang'an. Le troisième mois de l'année suivante (845), le gouvernement interdit aux monastères le droit d'établir des domaines agricoles et ordonna de dresser le recensement des esclaves employés à leur service, ainsi qu'un inventaire des richesses en argent, marchandises ou grain, en leur possession.Jusqu'alors, les mesures prises contre le samgha s'étaient limitées à réduire à l'état laïc les membres de la communauté qui n'observaient pas les règlements ou se livraient à des pratiques peu orthodoxes, telles que la magie noire ou lesincantations, mais la nouvelle politique mise en place par le gouvernement touchait désormais tous les moines, sans tenir compte de leur moralité, de leurs connaissances ou de leur rang. On réduisit d'abord à l'état laïc les moines de moins de quarante ans, mesure qui s'étendit bientôt aux religieux de moins de cinquante. Les moines âgés de plus de cinquante ans qui ne possédaient pas de certificat furent également rapidement contraints de quitter les ordres. Ces mesures s'appliquèrent aussi aux moines étrangers qui résidaient en Chine : tous ceux qui ne possédaient pas le certificat délivré par le Bureau des rites furent réduits à l'état laic et renvoyés dans leur pays d'origine.Afin de procéder à l'éradication totale du bouddhisme, les autorités ordonnèrent, au quatrième mois de l'année 845, un recensement des membres de la communauté monastique et des monastères. On dénombra 260 000 moines et nonnes, 4 600 temples et 40 000 sanctuaires. Muni de ces chiffres, le gouvernement ordonna au septième mois de la même année, la destruction de la presque totalité des établissements bouddhiques du pays. On ne conserva qu'un temple par préfecture principale, généralement le plus beau et le plus orné, et quatre monastères dans chacune des deux capitales, l'effectif étant réduit à trente moines par établissement. Le Secrétariat impérial rédigea ensuite plusieurs mémoires sur la confiscation des biens monastiques. Lestransformées en pièces derac danapale, generalement le plus beau et le plusorné, et quatre monastères dans chacune des deux capitales, l'effectif étant réduit à trente moines par établissement. Le Secrétariat impérial rédigea ensuite plusieurs mémoires sur la confiscation des biens monastiques. Les images de bronze et les cloches devaient être transformées en pièces de monnaie, les statues de fer en outils agricoles et les images en or, argent ou jade remises au Trésor public. Au huitième mois, le gouvernement impérial fit paraître un édit résumant les résultats obtenus par sa politique de répression du bouddhisme :Nous savons que, durant les trois premières grandes dynasties (Xia, Shang et Zhou), il ne fut jamais question du bouddhisme. Cette religion idolâtre n'a commencé à s'épanouir qu'à partir des Han et des Wei. Depuis peu, ses pratiques étranges, devenues familières, se sont imposées au point de corrompre lentement et inconsciemment nos mœurs. Beaucoup ayant été victimes de sa séduction, les masses n'en ont été que plus égarées. Partout sur notre territoire, des régions reculées jusqu'aux murs des palais de nos capitales, des moines surgissent en nombre toujours plus grand, et leurs monastères chaque jour gagnent en prospérité. Ils ont épuisé dans des travaux de construction les forces vives du pays, ont extorqué le peuple pour fabriquer à leur propre usage des ornements d'or et de pierreries ; ils ont incité les hommes à délaisser leur souverain et leurs proches pour suivre des maîtres religieux, à abandonner leurs conjoints pour embrasser la vie monastique, à violer la loi et à causer du tort au peuple. Rien n'est plus exécrable que cette religion.Lorsqu'un homme ne cultive pas la terre, d'autres souffrent de la faim, lorsqu'une femme ne tisse pas, d'autres souffrent du froid. Les moines et nonnes de ce pays sont innombrables, mais ils comptent sur la culture de la terre pour se nourrir et la production de la soie pour se vêtir. Leurs monastères et leurs temples, qui sont innombrables, sont des bâtiments altiers, magnifiquement ornés, qui osent rivaliser avec les plus riches palais.Il n'est pas besoin de chercher ailleurs les causes du déclin matériel et du relâchement moral qui a affecté les dynasties Jin, Song, Qi et Liang.Gaozu et Taizong ont mis fin aux désordres par les armes et régné sur ce beau pays par les arts et les lettres. Ces deux méthodes suffisent à gouverner.Comment cette insignifiante religion venue de l'Ouest pourrait-elle nous en imposer ? Dès les ères zhenguan (627-650) et kaiyuan (713-742) des réformes furent engagées, mais elles n'ont pas suffi à déraciner le mal, qui a continué à se répandre et prospérer.Après avoir pris ample connaissance des rapports précédents et largement consulté l'avis général, nous sommes arrivés à la conviction qu'il fallait agir contre ce mal. Nos ministres à la cour et nos administrateurs en province s'accordent avec nous pour dire qu'il est de la plus grande urgence de réformer [l'église bouddhique]. Nous devons exaucer leurs souhaits sur ce point. Personne ne sera plus déterminé que nous à tarir cette source de corruption, à nous conformer aux lois des cent rois [qui nous ont précédés] et à porter secours à notre navonGaozu et Taizong ont mis fin aux desorates par tes autes et re beau pays par les arts et les lettres. Ces deux méthodes suffisent à Comment cette insignifiante religion venue de l'Ouest pourrait en imposer ? Dès les ères zhenguan (627-650) et kaiyuan (713 réformes furent engagées, mais elles n'ont pas suffi à déracin qui a continué à se répandre et prospérer.Après avoir pris ample connaissance des rapports précédents et consulté l'avis général, nous sommes arrivés à la conviction qu'il contre ce mal. Nos ministres à la cour et nos administrateurs er s'accordent avec nous pour dire qu'il est de la plus grande u réformer [l'église bouddhique]. Nous devons exaucer leurs so ce point. Personne ne sera plus déterminé que nous à tarir cette corruption, à nous conformer aux lois des cent rois [qui nous ont et à porter secours à notre pays pour le bien des masses.Plus de 4600 monastères sont en ce moment détruits dans toutPlus de 260 500 moines et nonnes sont réduits à l'état laïc, a la corvée et à l'impôt. Plus de 40 000 temples et sanctuaires Plusieurs dizaines de millions d'hectares (ging) de champs de terres arables sont confisqués. Nous avons recouvré la poss 150 000 esclaves, qui sont désormais soumis au double impôt. L et les nonnes seront placés sous le contrôle du Bureau des hôtes, m ainsi clairement que le bouddhisme est une religion étrang réduisons [également] à l'état laïc plus de 3000 nestoriens et zoroastriens afin qu'ils ne corrompent pas les coutumes de la Chine.

La persécution de 845 fut incontestablement la plus étendue. Les proscriptions de 446 et 574-577 avaient été très largement limitées au nor de la Chine, dans les régions gouvernées par les Wei et les Zhou du Non.Elles n'avaient pas eu d'effets délétères à long terme sur le bouddhiste et avaient épargné tout à fait les communautés du Sud. La proscriptos décrétée sous les Tang, en revanche, s'appliqua à tout le territoire et caus in8. E. O. Reischauer, op. cit., avec quelques modifications.

P238

dommage permanent à la communauté bouddhique, raison pour laquelle elle est considérée comme un des événements majeurs de l'histoire du bouddhisme en Chine. Le bouddhisme souffrait déjà d'un certain affaiblissement, ainsi qu'en témoignent le relâchement de la foi et l'affaissement de la vigueur intellectuelle en certains endroits, mais ce fut la persécution de 845 qui lui donna le coup de grâce. Cette date, qui marque un point de rupture, indique la fin de son apogée et le début de son déclin.La proscription elle-même fut de courte durée. Un an ne s'était pas écoulé que Wuzong mourait, au troisième mois de 846, probablement affaibli par les élixirs de longue vie qu'il avait consommés. Xuanzong monta sur le trône impérial et s'employa immédiatement à mettre un terme à la répression décrétée par son prédécesseur. Zhao Guizhen, Liu Xuanjing, ainsi que onze autres taoïstes, furent exécutés pour avoir incité l'empereur à prendre des mesures extrêmes contre le bouddhisme. L'empereur Xuanzong autorisa la présence de douze temples au lieu de quatre dans la capitale, de deux temples dans chaque préfecture et de trois dans chaque commanderie. Enfin, les moines de plus de cinquante ans qui avaient été contraints de défroquer l'année précédente furent autorisés à revêtir de nouveau l'habit monastique.L'année suivante, l'empereur donna, pour ainsi dire, à la communauté monastique le feu vert qui l'autorisait à retrouver son fonctionnement habituel.L'édit qu'il fit paraître définissait certes le bouddhisme comme une religionL'édit qu'il fit paraître définissait certes le bouddhisme comme une religionnroclamait également aue. puisau'il n'avait pas portéavait pas être proscrit,unement affaibli parone impérial et s'employa immédiatement à mettre un terme à la répression trône impérial et s' employa immédiat consommés. Xuanzong motil ra trône impérial et s'employa immédiatement à mettre un terme à la répression décrétée par son prédécesseur. Zhao Guizhen, Liu Xuanjing, ainsi que onze autres taoïstes, furent exécutés pour avoir incité l'empereur à prendre des mesures extrêmes contre le bouddhisme. L'empereur Xuanzong autorisa la présence de douze temples au lieu de quatre dans la capitale, de deux temples dans chaque préfecture et de trois dans chaque commanderie. Enfin, les moines de plus de cinquante ans qui avaient été contraints de défroquer l'année précédente furent autorisés à revêtir de nouveau l'habit monastique.L'année suivante, l'empereur donna, pour ainsi dire, à la communauté monastique le feu vert qui l'autorisait à retrouver son fonctionnement habituel.L'édit qu'il fit paraître définissait certes le bouddhisme comme une religion étrangère, mais il proclamait également que, puisqu'il n'avait pas porté atteinte aux principes fondamentaux de l'empire, il ne devait pas être proscrit, particulièrement au vu du fait que les Chinois le pratiquaient depuis longtemps; en conséquence, si les moines désiraient réparer les temples qui avaient été endommagés durant la persécution, ils ne devaient pas en être empêchés par les autorités. Nous pouvons remarquer en conclusion que la persécution de 845 suit le schéma des précédentes proscriptions: d'abord des mesures violentes et drastiques, suivies, peu après la mort du souverain, de l'instauration par son successeur d'une politique de restauration.

s pèlerins chinois à l'étrangernombre de pèlerins partant pour l'Inde en quête de la Loi s'accrut lérablement durant la période d'expansion et de rayonnement du hisme en Chine. Comme la puissance de la dynastie Tang se faisait jusqu'en Asie centrale, de nombreux petits États de cette zone, soucieux lir de bonnes relations diplomatiques avec ce grand empire, s'ingénièrent iter le passage des moines chinois sur leur territoire. Le bouddhisme,
\cite[p. 233-239]{chen_histoire_2015}

\subsection{Impact économique}


\paragraph{le cout de la religion - l'exemple romain}
\begin{singlequote}
    Le dessein des sanctuaires et des cites parait assez clair, il s'agit
avant tout d'assurer Ia perennite des ietes et des sacrifices; de rendre
le sanctuaire le plus riche possible, c'est a dire de le doter de
monuments aussi somptueux que le permettent - mal - les finances
publiques et sacrees. Nous ne relevons Ia aucun souci de chrematistique:
certes les sanctuaires participent de fa~on active a Ia vie
economique, ils en sont meme l'un des moteurs par les besoins specifiques
qui sont les leurs ( materiaux de construction, denrees de
luxe); mais ils n'ontj amais cherche a Ia developper en soi; l'activite
bancaire repond a des considerations sociales plus qu' economiques
et il s'agit avant tout de tenter d'assurer des revenus reguliers au
culte273 ; l'activite monetaire est un phenomene local et trop particulier
pour qu'on puisse lui preter une signification generale: il
convient dans la plupart des cas de l'interpreter dans le cadre de Ia
panegyrie. En bref, ce n'est certainement pas par un mouvement
conscient et raisonne que les sanctuaires ont developpe et diversifie
leurs activites economiques et financieres, cependant en fonction
meme d'un certain volume de richesse, les sanctuaires au moins les
plus importants, ont pese d'un poids indeniable sur les rouages de
l'economie grecque. II ne faut pas d'autre part envisager l'accroissement
de Ia richesse des sanctuaires comme un phenomene
lineaire. Cette accumulation de metal precieux a suscite bien des
convoitises et cela d'autant plus que le sentiment du sacre s'etait
bien transforme a partir de Ia fin de l'epoque classique.
Aspects sociaux et économiques de la vie religieuse dans l'Anatolie gréco-romaine \cite{debord_aspects_1982}
\end{singlequote}
 

\subsection{David A. Palmer importance du temple - Goosaert}

\begin{singlequote}
    
Le troisieme chapitre relate brievement l'histoire du temple en tant qu'institution religieuse en Chine. Les premiers « temples » étaient des mausolées et des autels destinés au culte des ancêtres, qui se pratiquait souvent en plein air. Puis il y eut les sanctuaires du culte impérial de la dynastie Han. Mais c'est le bouddhisme qui popularisa la notion du temple abritant des icônes de divinités, lieu ouvert à tous et consacré au culte religieux.
Le monachisme bouddhique eut un immense impact social et politique à l'époque du Moyen  Age chinois (111"-VI· siecles). La construction de monastères opulents transforma le paysage rural et urbain. Le temple fut alors adopté aussi bien par le tao'isme institué que par les cultes populaires. Ces derniers étaient souvent dédiés à des divinités de la nature ou des héros locaux, et servirent de centres de résistance locale aux fonctionnaires du gouvernement central ainsi qu'au clergé bouddhique et tao'iste. Sous la dynastie des Tang (VIl"  IXº siecles), l'Etat impérial inaugura une politique de contrôle de toutes les institutions religieuses. 11 établit un « concordat » qui garantissait l'unité et l'égalité des trois traditions établies, placées sous sa protection. Les empereurs Tang ont aussi commencé la pratique des canonisations des dieux populaires, en leur assignant une place dans la hiérarchie céleste. Cette pratique favorisa la cooptation de ces cultes, qui devaient demander une
autorisation officielle pour la construction de temples. C'est à cette époque que les temples devinrent l'institution principale de la vie communautaire en Chine. Sous les Song, cependant, l'harmonie entre les « trois religions » fut détruite sous la pression des ambitions hégémoniques du confucianisme. Les cultes locaux réagirent en se libé  rant progressivement de la tutelle de l'Etat ; signe de leur plus grande indépendance, ils constitue  rent de vastes réseaux transrégionaux de temples. A l'époque des Ming et des Qing (XIVe-XIX siecles), l'écart entre la religion d'élite et la religion populaire se creusa. Vers la fin du XIX siecle, alors que le tissu social se fragilisait, les temples et les cultes se multiplierent, au point ou ils consti  tuerent souvent l'institution principale de l'organisation et de la défense villageoise. Au même moment, des mouvements sectaires, tels les Taiping, détruisaient tous les temples des régions qui étaient sous leur contrôle. Et les convertis au christianisme, en refusant de contribuer au financement des temples, contribuerent à briser l'unité de communautés qui avaient traditionnellement considéré la construction et l'entretien des temples comme une responsabilité collective.
Les réformes de Kang Youwei, promulguées en 1898, ont inauguré un changement radical de politique à l'égard des temples, qu'on voulut convertir en éléments d'infrastructure d'un Etat modeme. Cette politique fut systématiquement mise en oeuvre durant tout le XX: siecle : les temples furent ainsi transformés en écoles, en bureaux de police et des impôts, etc. Des milliers de temples furent tout simplement détruits: « Leur rôle d'articulation dans un systeme traditionnel, fragmenté en petites unités et en particularismes, était aux yeux [des réformateurs] impardonnable » (p. 99). Seuls les grands monasteres bouddhiques, isolés géographiquement et relativement à l'écart du systeme social traditionnel, furent épargnés. La Révolution culturelle n'a fait que continuer une histoire de destruction qui a traversé tout le vingtieme siecle. Attjourd'hui, ce sont les urbanistes et les promoteurs immobiliers qui démolissent des temples pour construire des immeubles modemes. On estime qu'cn 1900, il y avait environ un million de temples en Chine : un temple pour
cent familles. De ceux-ci, il ne reste maintenant que quelques milliers. « De 1898 à aujourd'hui s'est écoulé un siecle de destruction continue, par tous les moyens, et qui restera sans doute dans l'histoire de l'humanité comme l'un des plus grands anéantissements du patrimoine » (p. 101). Malgré cela, les temples continuent à prospérer à Taiwan et dans les communautés chinoises d'outre-mer. On assiste aussi à une résurgence de la construction de temples en Chine populaire, financés par le gouvernement, par les Chinois de la diaspora et par les fideles locaux.
Dans le quatrieme chapitre, Goossaert propose quatre modes d'appréhension de l'espace sacré. Tout d'abord, le temple peut être considéré comme un mémorial voué aux ancêtres : un lieu ou l'on honore les morts comme s'ils étaient présents, sans toutefois chercher abusivement leur intercession. Selon ce mode, l'appartenance à une communauté religieuse implique la filiation à une lignée présidée par une divinité. En deuxieme lieu, le temple peut être vu comme une cour de justice. Dans ce cas, le dieu n'est pas un ancêtre mais un fonctionnaire céleste, investi en tant que tel d'une autorité judiciaire. En tant que juges, les dieux peuvent convoquer des témoins de l'au-delà. Ils peuvent aussi être eux-même témoins : d'importants contrats étaient souvent scellés devant les dieux, qui punissaient ceux qui ne se tenaient pas à leurs engagements. En troisieme lieu, le temple peut être considéré comme une maison, un lieu de loisirs et de récréation, un lieu de vie. Il n'y a pas de distinction radicale entre l'architecture d'un temple et celle d'une maison : les temples se différencient par leur hauteur, leur taille, leur ornementation. Les temples sont les résidences impériales des dieux, qui offrent nourriture et logement aux passants. Le quatrieme mode est celui du temple comme montagne : la métaphore de la montagne est souvent utilisée pour décrire le temple en partie Oa pagode, le toit) ou dans son ensemble. I:ascension des montagnes pour arriver aux monasteres perchés au sommet est un acte de dévotion, rapprochant le pelerin des hauteurs étranges et sauvages de la transcendance spirituelle.
Le cinquieme chapitre s'attache à la fondation
des temples, le plus souvent le résultat d'une initiative individuelle, et aux modalités du financement de leur construction et de leur entretien.
Dans le sixieme chapitre, nous découvrons les acteurs de la vie du temple : le clergé, les devins et les mediums. La plupart des temples sont administrés par des comités laies qui emploient et supervisent les officiants qui y résident. Ce sont ces comités de dévots qui, à travers leurs oeuvres charitables et sociales, constituaient la trame même de la vie culturelle et religieuse chinoises. Enfin, le septieme chapitre nous présente la vie religieuse proprement dite des temples : le culte quotidien, les offrandes d'encens, de papier-monnaie et de sacrifices ; les festivals, rituels et processions ; enfin la musique et les banquets qui colorent la vie du temple.
Dans sa conclusion, Goossaert revient au theme du premier chapitre : la relation entre les temples, l'Etat et la société. \begin{singlequote}
    « Le temple chinois est une institution politique: l'Etat s'en sert pour gouverner, et le peuple y fonde son organisation » (p. 33).
\end{singlequote} Les temples sont des lieux d'articulation de la culture institutionnelle et populaire. À l'intérieur, la liturgie de l'élite d'Etat ou monastique apporte légitimité politique et cosmologique, alors qu'à l'extérieur, les fêtes communautaires et les associations de temple apportent le soutien et le financement du peuple, sans lequel l'ensemble de la liturgie officielle ne pourrait survivre. Le temple, donc, est un « lieu de négociation religieuse ». En lui se retrouvent tous les éléments de la religion chinoise. Bien que la coexistence des tendances élitistes et populaires ne soit pas toujours facile, tous les acteurs comprennent que le compromis est essentiel. « Le mélange des éléments, en des proportions toujours variées, rend compte de l'unicité d'une religion chinoise tres étendue, mais dont aucune partie ne veut se séparer radicalement des autres » (p. 204). Le temple est un espace privilégié ou se forme et s'exprime le contenu religieux, et qui attire vers lui toutes les connaissances et les richesses: les dieux parlent à travers les médiums et les orades, les artisans et les jardiniers façonnent la beauté des lieux, les prêtres célebrent les rites, les troupes d'opéra jouent des histoires saintes, les steles et les peintures racontent les faits des dieux et des adeptes, les maitres des arts du corps enseignent 
les secrets du combat ou de la longévité, et les les temples, ont investi les pares et les espaces philanthropes font leurs bonnes oeuvres. Le publics dans les années 80 et 90. La religion hors temple n'est donc pas un immeuble figé, mais un des temples n'étant pas le propos de ce livre, lieu ouvert, à l'intérieur duquel les formes Goossaert ne s'attarde pas sur cette tendance.  

\end{singlequote}
%\chapter{Evolution de l'interdiction du \riba et Gharar}


a. Un dossier sur un thème lié au cours A partir de trois articles universitaires minimum. -  Introduction : intérêt du thème au regard de l’actualité et/ou de vos propres orientations personnelles ; présentation des articles et de leurs auteurs. Présentation du plan de votre travail. - Synthèse (ne pas présenter les articles séparément mais faire une synthèse à partir des différentes thématiques rencontrées) - Contextualisation de ce thème dans l’islam contemporain, éclairage par le contenu du cours (enjeux, dynamiques, etc). 

\section{Introduction}

En tant que membre du jury de l'Institut des Actuaires, j'ai eu à me prononcer sur certains mémoires présentés par des étudiants sur la \textit{finance Islamique} et \textit{l'assurance Takaful}. Ces mémoires ne couvrent pas les présupposés théologiques et commencent généralement par une présentation des interdits de la loi musulmane, comme par exemple ce \href{http://www.ressources-actuarielles.net/EXT/ISFA/1226-02.nsf/0/8c814ff5f2bae57ec1257e1a004407b6/\%24FILE/Memoire_ISFA_Tontines_et_Takaful_Bendimerad_Version_avec_Couverture.pdf}{mémoire d'actuariat} : 
\begin{quote}
     [pour faire face à ses  engagements, l'assureur],peut, par exemple, inclure des actifs obligataires dont les taux de rendement font apparaître des taux d'intérêt. [...] l'interdiction de l'usage du taux d'intérêt, appelé \textit{\riba} en arabe, est l'un des fondements du droit des affaires musulman.
\end{quote}
Dans la même veine, sont généralement définis les autres interdits, comme le \emph{Gharar}, défini dans le même mémoire comme : 
\begin{quote}
    Le contrat d'assurance peut constituer une perte disproportionnée en faveur de l'un des participants aux dépens de l'autre. Ce caractère d'incertitude, appelé \textit{Gharar} en arabe, est un caractère prohibé par l'Islam surtout lorsque le transfert du risque vers un tiers est total, ce qui est notamment le cas pour l'assureur qui porte entièrement la charge du risque cédé par l'assuré.
\end{quote}
Les mémoires d'actuariat consultés essayent de dépasser ces interdictions, qui soulignent une tension avec les \textit{Hadiths} : 

\begin{quote}
     
A l'issue de l'analyse précédente, l'idée même d'une assurance islamique semble être un contresens. Pourtant, la religion musulmane encourage l'individu à prendre des mesures pour réduire l'ampleur des désastres qui pourraient l'affecter. D'après un Hadith authentifié, le prophète conseille à un croyant de placer sa confiance en Dieu et d'attacher son chameau plutôt que de se limiter uniquement à placer sa confiance en Dieu en offrant la possibilité au chameau libre de s'échapper. L'Islam ne s'oppose donc pas à l'idée de vouloir minimiser les risques et par conséquent elle ne s'oppose pas à faire usage de la loi des grands nombres. Elle exclut certes la spéculation et l'incertitude ainsi que le taux d'intérêt. En revanche, elle compte parmi ses principes la coopération et l'entre-aide mutuelle ainsi que le partage équitable des risques et des bénéfices. Toutes ces bases ont permis de concevoir un modèle alternatif à celui de l'assurance conventionnelle.
\end{quote}
De cette tension naît des solutions techniquement assez complexes.


Intrigué par cette apparente clarté vis à vis de l'interdiction de l'\textit{intérêt} et de l'\textit{l'incertitude} à la base de l'économie moderne, je me propose d'étudier l'évolution de l'opposition de la \emph{\riba} et \emph{gharar} dans les différents courants de l'Islam contemporain, depuis la pensée d'Abduh jusqu'aux frères musulmans et le salafisme. En particulier, nous essayerons d'identifier la voie médiane, entre des lectures libérales (\riba comme usure) et à l'opposé, des musulmans socialistes (\riba comme profit).

L'interdiction des deux notions n'a pas la même importance économique et la \textit{\riba} a donné lieu à plus de développement. En partant de ces études, j'essayerais néanmoins de transposer ces travaux à l'assurance.
Après avoir regardé  à travers l'étude de ces deux notions comment l'économie moderne, pensée au XVIIIè et XIXè siècle questionne l'Islam, je poserai quelques pistes sur le propre questionnement du Christianisme. Après tout, le prêt à intérêt était aussi interdit au Moyen-äge en Europe (Concile de Tours de 1163). Ces pistes seront en particulier nourries par l'étude de la pensée de Saint Thomas d'Aquin sur les taux d'intérêt et comment ces pistes peuvent nourrir la réflexion de la théologie musulmane (et chrétienne) contemporaine. 
 

%---------------------------------------------------------------------------------------------------------------
\section{Vision dans l'Islam Classique de l'interdiction de \riba}

\subsection{une interdiction de la \riba et Gharar}

Pour commencer, il est utile de se référer aux textes à l'origine de ces interdits.
\paragraph{Versets du Coran et Hadiths interdisant la \riba}
Le verset 130 de la Sourate 3, Al-Imran mentionne : 
\vocalize % switch diacritics for short vowels on
\transtrue % display the transliteration
\arabtrue % print arabic text (on by default)
\begin{quote}
 
\TArabe{
 يَا أَيُّهَا الَّذِينَ آمَنُوا لَا تَأْكُلُوا الرِّبَا أَضْعَافًا مُّضَاعَفَةً وَاتَّقُوا اللَّهَ لَعَلَّكُمْ تُفْلِحُونَ
  } 
   [3,125] O vous qui croyez !, ne vivez pas de la \textit{\riba} [produisant le] double deux fois ! Soyez pieux envers Allah ! Peut-être serez-vous bienheureux. 

\end{quote}
\begin{quote}
\TArabe{278
 يَا أَيُّهَا الَّذِينَ آمَنُوا اتَّقُوا اللَّهَ وَذَرُوا مَا بَقِيَ مِنَ الرِّبَا إِن كُنتُم مُّؤْمِنِينَ
279 فَإِن لَّمْ تَفْعَلُوا فَأْذَنُوا بِحَرْبٍ مِّنَ اللَّهِ وَرَسُولِهِ وَإِن تُبْتُمْ فَلَكُمْ رُءُوسُ أَمْوَالِكُمْ لَا تَظْلِمُونَ وَلَا تُظْلَمُونَ
} 
 [2, 278] O vous qui croyez !, soyez pieux envers Allah ! Faites abandon de ce qui [vous] reste [à toucher provenant] de la \textit{\riba}, si vous êtes croyants !
[2, 279] Si vous ne le faites point, attendez-vous à une guerre de la part d’Allah et de Son Apôtre ! si vous vous repentez, alors vous récolterez votre capital sans infliger ou être victime d’une injustice. \sn{Traduction de Blachière, sauf la dernière phrase \cite{ElGamal:BanqueFinanceIslamique}}

\end{quote}

De même, un hadith de Abu Hurayra met au même niveau la \textit{\riba} et le meurtre : 
\begin{quote}
    "Evitez les sept turpitudes !"

- "Quelles sont-elles, ô Envoyé d'Allah?", demandèrent les fidèles.


- "Ce sont, répondit-il :

 

-le polythéisme,

-la magie,

-le meurtre qu'Allah a interdit sauf à bon droit

-l'usurpation des biens de l'orphelin,

-la \textit{\riba},

-la fuite du front au jour du djihad et

-la fausse accusation (de fornication) des femmes vertueuses, chastes et croyantes".
\end{quote}
 
  Nous avons évité à ce stade de traduire le terme de \textit{\riba}, par \textit{usure} ou \textit{intérêt}. Le Hadith rapport par at-Tirmidhî a été utilisé pour appliquer le sens d'intérêt à la \textit{\riba}  : 
 \begin{quote}
     Vendez de l'or contre de l'argent (les quantités échangées étant) comme vous voulez, à condition que ce soit main à main. Vendez du blé contre des dattes sèches (les quantités échangées étant) comme vous voulez, à condition que ce soit main à main. Vendez de l'orge contre des dattes sèches (les quantités échangées étant) comme vous voulez, à condition que ce soit main à main (n° 1240)
 \end{quote}  
 On peut effectivement y lire le refus de l'intérêt mais d'autres lectures sont possibles, en particulier en relevant l'insistance du hadith à une transaction \textit{main à main}, qui permet une transaction claire. Par ailleurs, la valeur du temps n'est pas explicitement mentionnée.
 

 
\subsection{un nouveau contexte avec la naissance du Capitalisme}
Du point de vue du prêteur, le prêt à un prix du fait du risque  que l'emprunteur repaye le prêt, la valeur temps (qui correspond à l'inflation et à la perte d'opportunité d'un investissement aujourd'hui qui rapporte plus plus tard). La pertinence économique de l'intérêt et de la prise de risque est donc justifiée même si cela ne veut pas dire qu'elle l'est d'un point de vue religieux.
Avec l'accumulation des richesses dans les villes italiennes au moyen-âge naît le capitalisme et la finance : comment financer et assurer les bateaux partant de Gênes et remplis de richesse ? Cette irruption de la finance pose de nouvelles questions.  Face à cette accumulation, une réponse théologique chrétienne sera la création de l'ordre mineur par Saint François d'Assise. Et, nous reviendrons sur ses développements, la réflexion de l'université de Paris et de Saint Thomas d'Aquin sur la notion d'usure et de juste prix.  
Ces innovations touchèrent tardivement le monde musulman, au XIXème, avec l'arrivée simultanée de l'industrialisation, qui nécessite des capitaux importants et les débuts de la colonisation. Par l'industrialisation, la capacité productive est fortement augmentée par l'investissement, mais aussi la concentration des risques qui ne peuvent être portée par la solidarité traditionnelle.
Face à cette nouvelle question, quelles sont les réponses proposées par les théologiens musulmans depuis l'arrivée de cette question ?


%---------------------------------------------------------------------------------------------------------------
\section{Elements théologiques apportés}

\subsection{Premières réponses face à la nécessité du prêt}
\paragraph{Le développement du \riba et de l'assurance par les Occidentaux} Les prêts à intérêt ont toujours existé au sein de l'Empire Ottoman\cite{Gilbar:Qadi}, et proposés par des familles grecques, arméniennes ou Juives (et en Iran par les arméniens et zoroastriens). A ce premier Groupe, s'ajouta les banques commerciales européennes ou les banques "étrangères-locales" comme la banque impériale ottomane au XIX.
De la même façon pour les assurances, Ibn Abidin, représentant de l'école officielle de droit Hanadi dans l'empire ottoman, suggère le compromis suivant : il est licite d'établir ds contrats d'assurance portant sur les risques encourus à l'intérieur du royaume islamique - le Dar al Islam -à condition que ces contrats soient conclus avec une compagnie d'assurance ayant son siège hors de pays de l'Islam. 

\paragraph{Utilisation de prêt à intérêt par les musulmans malgré l'interdiction du \textit{\riba}.} La \textit{\riba} a pu poser des problèmes dans son implémentation, en particulier pour les grands marchands (tujjār).  
Pour l'école Hanafite, le terme de \riba peut être traduit par usure, dans le sens d'un taux d'intérêt exhorbitant, à l'opposé de l'intérêt raisonnable, le \emph{ribh}. L'autre interprétation dominante, attribuée à 'Abdahallah Ibn 'Abbas, permet d'expliquer la notion du Coran de double capital par la pratique du \textit{\riba al jâhiliyya} (préislamique), avec des taux d'intérêt très élevés avec des intérêts pouvant dépasser le capital en cas de non-paiement. A l'opposé de ces analyses plus ouvertes, la position de l'école Hanbalite fut l'interdiction pure de l'intérêt. Pour contourner l'interdit du \riba face à la nécessité, les marchands musulmans mirent en place des montages complexes où l'intérêt était déguisé en une \textit{double vente} (\emph{mukhâtra}) avec deux ventes fictives successives permises par la \textit{sha'ia}: 
\begin{itemize}
    \item la première, le préteur vent un objet à l'emprunteur pour une somme équivalente au capital et à l'intérêt. l'emprunteur s'engage à payer la valeur de l'objet à la fin de la période (techniquement, une vente avec paiement différé). 
    \item A la fin de cette première transaction, l'emprunteur revend le même objet pour la valeur du principal (techniquement, une vente à terme).
\end{itemize}
 Cependant, l'étude des jugements des cours religieuses au moyen-orient entre le XVI et le XVIII montre que malgré ces montages, des exemptions partielles ou totales de paiement de l'intérêt du fait de son incompatibilité avec l'Islam \sn{GILBAR, GAD G. “The Qadi, the Big Merchant and Forbidden Interst (Ribā).” British Journal of Middle Eastern Studies, vol. 39, no. 1, 2012, pp. 115–36, http://www.jstor.org/stable/23264404. Accessed 3 May 2022.}
 

\section{la pensée d'Abduh et Rida}
\subsection{position d'Abduh} Le Cheikh {Muhammad Abduh} (1849 - 1905) Egyptien, successeur d'Al Afghani, est la figure marquante du réformisme islamique. Il s'oppose au colonialisme atteignant alors l'Egypte. Son séjour  parisien le convainc de la nécessité de réforme de l'Islam et l'initie aux réflexions intellectuelles occidentales et en particulier François Guizot. A sa suite, il il pense l'Islam comme civilisation et pas uniquement comme Religion, avec la notion de progrès lié à la civilisation. Il n'y a pas d'opposition pour lui entre Foi et Raison, la religion musulmane étant \textit{raisonnable}.    Dans son livre \citet{Abdou:Rissalat}, il part de la Raison pour montrer le besoin que l'homme ne se fixe pas sa propre loi.

Il montre tout d'abord la naissance des besoins et des liens entre les hommes. Idéalement ces liens seraient ceux de l'amour : 
\begin{quote}
    Nul ne met en doute que chaque membre d'une société a besoin
des autres membres; et chaque fois que l'individu accroître ses exigences
dans la vie il ressent plus fortement le besoin de recourir au concours
de ses semblables. Ainsi se développent les besoins et à leur suite s'étendent
les relations de la famille à la tribu, de celle-ci à la nation, et finalement
au genre humain tout entier, comme le montre notre époque . Ces
besoins qui créent dans le sein de chaque nation (surtout dans le sein de
celles qui méritent vraiment ce nom) des relations et des rapports
spéciaux la distinguant des autres nations, sont le besoin de se procurer
sa subsistance, celui de profiter des biens de la vie, celui d' acquérir
les choses désirables et d'éloigner de soi celles qui déplaisent. 
Si la vie de l'homme se déroulait selon les lois de la nature, telles
que nous les voyons appliquées aux autres êtres vivants, les besoins
que nous venons de citer auraient été parmi les facteurs les plus puissants
de l'amour entre les individus[...].  

[\ldots]
Si par contre, l'intérêt se mêle aux relations amicales, et si chacun des amis exige un prix pour son amour, celui-ce se change en esprit d'exploitation, il se reporte sur l'effet utilitaire et se transforme chez l'un des amis en un abus de la force, et chez l'autre, en peur avilissante, en dissimulation et en hypocrisie.
\cite[p.66]{Abdou:Rissalat}
\end{quote}
Mais l'homme ne vient pas selon les lois de la nature car il est inconstant : 
\begin{quote}
 l 'homme
a été créé avec un caractère inconstant; quand le malheur l'atteint
il est abattu, et quand il acquiert quelque bien il devient insolent.
(C. ch. 70, v. 19 à 21). [...]
Chaque fois que la mémoire et l'imagination les poussent à
éviter quelque chose qui leur inspire de la crainte ou à atteindre un
objet qui leur fait envie, leur intelligence leur ouvre une porte de la
ruse ou leur découvre une voie de la violence ; alors le rapt remplace
l'échange pacifique, la dispute prend la place de l'union et la conduite
de l'homme riche s 'appuie plus que sur l'astuce et la violence.
\cite[p 68]{Abdou:Rissalat}
\end{quote}
il montre qu'on peut accéder à l'équité par des voies naturelles mais qu'à la foi pour les masses et pour éviter la corruption des élites, il est nécessaire de s'appuyer sur loi externe.
\begin{quote}
[...] ] le genre humain
a surtout besoin, pour conserver son existence, de l'amour ou d'un
sentiment qui le remplace.
A différentes époques il y a eu des penseurs qui ont fait appel à
l'équité ; ils ont pensé, et même quelques mystiques ont exprimé cette
pensée par de nobles paroles, que l'équité remplace l'amour. Cette
assertion ne manque pas de sagesse, mais qui est-ce qui peut établir les
règles de l'équité et amener la totalité des hommes à se soumettre ?
\cite[p 69]{Abdou:Rissalat}
\end{quote}

Suivre la Loi de Dieu, ordonner le bien est alors supérieur à la foi : 

\begin{quote}
  le Coran indique
qu'il n'y a pas de condition meilleure que celle des hommes qui ordonnent
le bien et défendent le mal: \begin{quote}
    Vous êtes les meilleurs parmi les hommes,
vous ordonnez le bien, vous défendez le mal et vous croyez en Dieu. (Cor. ch. 3, v. 106.) 
\end{quote} 
 
Dans ce verset le fait d'ordonner le bien et de
défendre le mal est mentionné avant la foi en Dieu, bien que la foi soit la base même sur laquelle s'appuient les bonnes oeuvres. 
\cite[p 121]{Abdou:Rissalat}
\end{quote}

\paragraph{Conséquence pratique pour la \riba}
'Abduh liste alors une liste de maux que l'Islam permet d'éviter : 
\begin{quote}

[l'Islam] nous invite par dessus tout à dépenser notre bien pour les oeuvres
charitables et souvent il en fait l'expression de la foi et la manifestation
d'une bonne conduite; il déracina par là, du coeur des pauvres, la rancune
et la haine contre ceux que Dieu a favorisé des biens terrestres ; il leur
inspira l'amour pour les riches, tout comme il fît naître dans le coeur de
ceux-ci la pitié pour les malheureux ; ainsi il développa la confiance dans
le coeur de tous les hommes. Quel remède plus efficace contre les maux
dont souffre la société: « C'est une faveur que Dieu accorde à qui il veut,
car Dieu est d'une bienfaisance sans bornes. » (Cor. ch. 57, v. 21.)
\textbf{L' Islam a fermé les deux portes du mal, il a bouché les deux sources
qui minent l'intelligence et détruisent la richesse, en frappant les boissons
enivrantes, les Jeux de hasard et l'usure, d'une interdiction absolue qui
n'admet pas d'infraction.}
\cite[p 122]{Abdou:Rissalat}
\end{quote}
Ainsi, parmi ces maux figure la \riba traduit ici par \textit{usure}.

\paragraph{Pratiquement : l'avis d'Abduh sur l'ouverture de la Caisse d'Epargne en Egypte}
L'ouverture de la Caisse d'Epargne et la répugnance d'un certain nombre de musulmans à toucher les intérêts poussèrent la caisse d'épargne à demander à 1903 une Fatwa à Abduh \cite{Jomier:AdbouCaisseEpargne}. Cette fatwa semble n'avoir jamais existé. En revanche, Ce fut à la requête d'un particulier mais en fait pour la
Compagnie d'Assurances sur la vie Gresham. La Compagnie d'Assurances al-Chark,
au Caire,  conservait une copie d'une fatwa d'Abduh \textit{écrite initialement pour un particulier} afin de la montrer à ses clients, le cas échéant. Le
client effectue des versements réguliers et périodiques par tranches successives
prévues d'avance ; la compagnie les encaisse et se charge de faire fructifier cet
argent. Finalement, à l'échéance, la société remboursera le total des versements
effectués augmenté des bénéfices résultant de la fructification. La fatwa admet
la licéité d'une telle opération en des termes qui, au fond, s'appliqueraient à toute
société en commandite. Il faut noter que la revue Al-Manâr de \textit{Rashid Rida} ne nia pas l'authenticité de cette fatwa après la mort d'Abduh mais protesta contre l'extension faite à toute assurance vie.

\paragraph{la note de la revue Al-Manâr de 1903}


Face aux scrupules de 3000 musulmans à toucher les intérêts (fa'ida), un échange eu lieu entre le directeur de la Caise d'Epargne et Abduh, échange repris par la revue \textit{Al-Manâr}. Abduh y maintenait fermement le principe de
l'interdiction absolue de l'usure (al-ribâ) ; mais il ne classait pas le cas de la Caisse
d'Epargne avec ceux de l'usure. Il l'assimilait à celui d'une société en commandite
(chirkat al-mod'âraba). A la suite de cette discussion, la loi fut modifiée en 1904 : 
 

\begin{quote}
 Il est prévu dans l'article premier que le client devra signer un
formulaire imprimé dans lequel il déclarera donner au Directeur Général de la
Poste tout pouvoir pour faire fructifier les sommes déposées, d'une façon licite et
en excluant toute opération usuraire. Il déclare permettre au Directeur de la Poste
de joindre ses dépôts à ceux des autres clients pour les faire fructifier en commun
à condition de recevoir une part de bénéfices (al-ribh') proportionnelle à ses
versements. On notera que dans cet article comme dans tout le reste de la loi le
mot de \textit{fâ'ida}, intérêt, est totalement absent. \cite{Jomier:AdbouCaisseEpargne}
\end{quote}

\begin{quote}
    L'article deux stipule entre autres ce qui suit. Il évite de parler de tant pour
cent, formule qui rappelle trop les intérêts. Il note que la part des bénéfices ne
dépassera pas un pour quarante du capital et le surplus, s'il y en a, reviendra à
l'administration postale à titre de compensation pour les services rendus et les
frais que ceux-ci comportent. L'assimilation à la société en commandite est ainsi
mise en accord avec le fait que la proportion des bénéfices touchés est fixe. On
notera que la proportion de un pour quarante est familière aux oreilles des
musulmans pieux puisque dans un certain nombre de cas le montant de la Zakat
atteint ce chiffre.
\end{quote}


Abduh développe la raison de cette position dans la fatwa et la note du manâr du 18 mars 1904: 
\begin{quote}
    La raison de l'interdiction de l'usure est de faire cesser l'injustice et de conserver les vertus de solidarité et de bonté mutuelle
\end{quote}
On voit donc la position d'Abduh comme ferme dans les principes et ouvert dans l'application pour vivre de l'Esprit pourrait-on dire, même si cela se traduit par une subtilité juridique.


% - ----------------------------------------
\subsection{L'approche de Rashid Rida sur la \riba} 

Rida est l'un des successeurs spirituels de Abduh. Il fonda avec Abduh la revue néo-réformiste al-Manâr en   1898. Elle définit sa politique éditoriale à l’égard « de la civilisation occidentale » sous forme de deux impératifs : 
\begin{quote}
    \item  1/ Il faut que la terre musulmane puisse rattraper l’Europe sur le plan des sciences modernes, de l’industrie et de l’innovation technique. 
    \textbf{ }2/ En contrepartie, il faut déclarer une guerre sans merci à tout ce qui a accompagné l’entrée des Européens en terre musulmane comme décadence morale et mauvaises mœurs
\end{quote}

\paragraph{Une lecture de la \riba originale à la base de l'économie Islamique } Si la position de Rida vis à vis de la liceité de l'intérêt est dans les faits la même que celle d'Abduh, Rida pose son raisonnement par une analyse détaillée des versets du Coran et des Hadiths parlant de \riba  \cite{Siddique:DemystifyingRiba}. 
Rida part de l'analyse du mot \riba comme \textit{excès}, et en particulier reprend l'analyse de Co 3,125 présentée au début d'un excès lié à la renégociation de la dette. De là, il conclut que ce qui est interdit par le Coran sont les intérêts ajoutés à la fin de la période de prêt (\riba ’l-jahiliyyah). D'une certaine façon, ce sont les intérêts composés ("intérêt d'intérêt") qui sont interdits, ce qui arrive en cas de non paiement à la fin de la période définie. Cette distinction permet à Rida d'autoriser les intérêts bancaires :
\begin{itemize}
    \item ils ne doublent pas les taux
    \item l'ajout d'un intérêt fait partie du principal de la même façon que le prix d'une vente à crédit (\emph{\riba al-buyu}) qui ne sépare pas principal et intérêt et qui est autorisé par le Coran.
\end{itemize}

 La distinction de Rida entre la \riba interdite dans le Coran et celle des Hadiths est présentée dans le graphique \ref{fig:MinorityRiba}.

 \begin{figure}[h!]
     \centering
     \sidecaption{\cite{Siddique:DemystifyingRiba}, une séparation entre Coran et Hadith}
      \includegraphics[width=0.5\textwidth]{CourantsIslamContemporain/ImagesCourantsIslamContemporain/RibaRida.png}
      \caption{La présentation du Riba selon Rida, dite minoritaire}
     \label{fig:MinorityRiba}
 \end{figure}
 Cette distinction entre \riba du Coran et \riba du Hadith permettait à Rida de légitimer la notion d'intérêt mais il ne sera pas suivi par une majorités de jurisconsultes sur ce point qui étendront l'interdiction de la \riba du Coran à tout intérêt, simple ou composé (intérêt d'intérêt). En revanche, ils adopteront sa distinction entre les différents \riba du Coran et du Hadith. Mais cette distinction, loin de simplifier le sujet, rend les discussion assez techniques, en particulier parce que les différentes écoles appliquent aux hadiths des méthodes d'analyse différentes que celles appliquées sur le Coran. \cite{Siddique:DemystifyingRiba} montre que la distinction de Rida est basée sur une argumentation circulaire. Des discussions casuistes, on
 

 
\paragraph{Les frères musulmans}
Dans les perspectives du réformisme musulman,
cette question d'ailleurs est loin d'être insoluble car une société qui émet des
actions peut être assimilée à une société en commandite. Les dividendes
représentent alors la part des bénéfices proportionnelle au capital engagé... 
al-Banna, fondateur des Frères Musulmans en Egypte, avait admis la licéité de ce
genre d'opérations et des Frères avaient, peu avant 1 948, fondé quelques ateliers
ou sociétés financées par ce procédé (2). La difficulté principale vient de ce que la
langue arabe n'a pas de mot spécial pour désigner les dividendes et que le terme
fâ'ida a des relents de ribâ.
\begin{quote}
    III. 2 interdiction de pratiquer l'usure, orienter les banques vers cette interdiction, le gouvernement doit donner l'exemple en abandonnant l'\textit{intérêt} fixé par les banque du prêt et du prêt industriel, etc. \textit{programme des frères Musulmans, 1936}
\end{quote}

\paragraph{les nouveaux penseurs}

\cite{Siddique:DemystifyingRiba}


%---------------------------------------------------------------------------------------------------------------
\section{Naissance de la finance et de l'assurance Islamique}

\subsection{Ce que l'on voit : Malaisie,...} 

\paragraph{une pratique supposant une \textit{shari'ah} qu'on ne peut interroger}

\subsection{Quelle est la théologie sous-jacente}

\paragraph{Une évolution de la shari'a par les principes} Mohammed Talbi 
\begin{quote}
    Il existe
trois principes en islam permettant de faire évoluer le droit et de
l'adapter
à la réalité, 

la \emph{maslaha} c'est-à-dire l'utilité publique, un
concept qui date du II\textsuperscript{e} siècle de l'hégire, 

la
\emph{zharoura}, la nécessité, c'est un principe fort puisqu'il est dit
que "la nécessité rend permis l'interdit" ; 

et les \emph{maqassid}, les
finalités de la loi. 

\begin{Synthesis}
On part des principes contre les principes. 
\end{Synthesis}
\sn{voir p. \pageref{TroisPrincipesEvolutionsShari}}
\end{quote}


\section{Conclusion}

\paragraph{une variété importante de vision mais toujours basée sur la sharia}

\paragraph{une réflexion sur la base de la Sharia'}
voir ce que Candiard dit de ce théologien qui dit de ne pas faire de Kalam mais du droit.

\paragraph{quel pourrait être les pistes de Kalam, quelques pistes de l'expérience occidentale ?}

\paragraph{St Thomas et l'usure : une réflexion autonome de la théologie}

\paragraph{extension du juste prix dans les risques : la dimension actuarielle}
\paragraph{Au XIXème, réflexion neo-thomiste et patristique pour sortir du carcan}
Après avoir regardé comment l'économie moderne, pensée au XVIIIème et XIXème questionne l'Islam à travers l'étude des différents courants de l'Islam contemporain, je poserai quelques pistes sur le propre questionnement du Christianisme. Après tout, le prêt à intérêt était aussi interdit au Moyen-äge en Europe. Ces pistes seront en particulier nourrie par l'étude de la pensée de Saint Thomas d'Aquin sur les taux d'intérêt et comment ces pistes peuvent nourrir la réflexion de la théologie musulmane. 

Nous catholiques au XIX par rapport à la modernité: renouveau thomiste et patristique. Liberté de pensée par rapport aux questions de l’époque
« A quoi on tient en vrai »
Le raccourci de certains théologiens musulmans : faire le court circuit pour repenser les choses : obsession juridique. Comment on sort du droit ? En faisant de la théologie. Et pour cela retour à la tradition
Voile
Le coran la dit donc il faut se voiler. Voile : hijab n’est pas dans le coran. Rideau
Zina : qu’il faut cacher
Jilbab pour les filles et femmes du prophète 
Ibn el jahouzi : voile non mentionné. Dimension non religieuse

Des écrits au XX. Deux raisons : les femmes vont occuper plus de rôle dans l’espace publique.
Et renouveau de l’anti théologie. Le voile est une question de théologie. Question de la foi
	⁃	Saint Paul : la foi et les œuvres
	⁃	On comprend croyant non pratiquant
	
%\chapter{Matériaux bruts pour validation}

\section{A lire}
\href{https://www.persee.fr/doc/ridc_0035-3337\_1955\_num\_7\_3\_9521}{Le prêt à intérêt et l'usure au regard des législations antiques, de la morale catholique, du droit moderne et de la loi islamique }

\section{Refus de l'intérêt en milieu chrétien}

\subsection{1163 : le concile de Tours condamne le prêt à intérêt}


Le 01 septembre 2014

La condamnation du prêt à intérêt par l'Eglise n'a pas empêché, au travers de divers subterfuges, le développement du crédit, moteur de l'activité économique.

Dans l’Occident chrétien médiéval, le rôle de prêteur sur gages, souvent pour de faibles montants, sera longtemps assuré par des juifs, que leur religion autorise à prêter avec intérêt à des non-juifs. Mais dès le XIe siècle, de nombreux bourgeois chrétiens enrichis se livrent au prêt à une autre échelle. Déjà au XIIe siècle, des bourgeois d’Arras, de Cahors, des Lombards d’Asti, des Génois prospèrent par ces opérations.

Certes, l’Eglise et les princes qui se conforment à sa loi bannissent le prêt à intérêt en s’appuyant - entre autres - sur l’Evangile de Luc : "Prêtez-vous l’un à l’autre sans rien en attendre." En France, le concile de Tours de 1163 condamne une des formes du prêt à intérêt dans le cadre de la moralisation de l’Eglise : nul clerc (évêque, abbé) ne peut désormais prêter de l’argent et recevoir en gage un

\section{Extrait site internet}
\begin{quote}
En Islam, le Riba est interdit : il s’agit de l’un des plus grands péchés, qui tire son origine du Coran. Mais il est également l’un des plus banalisés. Le Riba, appelé « usure », désigne l’intérêt perçu sur de l’argent prêté. De nos jours, le Riba est partout présent dans le domaine financier, à savoir, dans les comptes d’épargne, dans les prêts à intérêts, dans les agios… Alors, qu’est-ce que le Riba exactement ? Pourquoi est-il interdit ? Comment financer vos projets sans Riba 
\paragraph{Définition du Riba}

Le Riba peut être défini par intérêt - usure. Si le terme « usure » est la traduction la plus fréquemment donnée à cette interdiction de l’intérêt usuraire, il est important de préciser que le mot Ribâ vient cependant du verbe rabâ \& arbâ qui signifie augmenter et faire accroître une chose à partir d’elle-même. 

Dans le contexte financier, il est donc interdit de réaliser des transactions financières basées sur du Riba. Or, lorsque vous possédez un compte courant conventionnel, l’argent qui y transite alimente un circuit basé essentiellement sur l’usure. En effet, dans ce contexte, la banque dispose de vos fonds afin de spéculer. Même principe lorsque vous êtes à découvert : les agios que vous devez payer vous impliquent dans la pratique de l’usure.

Mais ce n’est pas tout : lorsque vous épargnez votre argent sur des comptes de type livret Jeune, Livret A, ou PEL, les intérêts que vous touchez sont également basés sur la pratique du Riba, et cela que vous fassiez don ou non des intérêts.

Enfin, lorsque vous réalisez un prêt bancaire, vous participez au développement de l’usure en payant des intérêts. 

\paragraph{De nombreux versets du Coran interdisent la pratique de l’usure.}     En voici quelques-uns, qui sont particulièrement explicites :

Verset 130 de la Sourate 3, Al-Imran : « Ô les croyants ! Ne pratiquez pas l’usure en multipliant démesurément votre capital. »
Les versets 278 et 279 de la Sourate 2, Al-Baqarah : (V-278) « Ô les croyants, craignez Allah ; et renoncez au reliquat de l’intérêt usuraire, si vous êtes croyants. (V-279) Et si vous ne le faites pas, alors vous recevrez l’annonce d’une guerre de la part d’Allah et de son prophète. Et si vous vous repentez, vous aurez vos capitaux. Vous ne lèserez personne, et vous ne serez point lésés. »
D’après Abu Hurayra, le Prophète (SAWS) a dit : « Evitez les 7 turpitudes ». Lorsque les compagnons demandèrent : « Quelles sont-elles, Ô envoyé d’Allah ? », il mentionna l’usure aux côtés notamment du polythéisme, du meurtre…
D’après Sahih Muslim : Ubâdah Ibn As-Sâmit rapporte que le Messager d’Allah (SAWS) a dit : « Or pour Or, argent pour argent, blé pour blé, orge pour orge, datte pour datte, sel pour sel, de manière égale, de main en main. Si la nature des produits diffère, vendez comme vous le voulez, si c’est de main en main ». \sn{\href{https://firstunion.fr/quest-ce-que-le-riba/}{First Union} ? }
\end{quote}

 
 \paragraph{Abd ar-Rahman ibn Nasir as-Sadi}\href{https://fr.wikipedia.org/wiki/Abd_ar-Rahman_ibn_Nasir_as-Sadi}{Abd ar-Rahman ibn Nasir as-Sadi} Oulema Hanbalite saoudien 1889. influencé par 	
Ibn Taymiyya
Mohammed ibn Abdel Wahhab
 
 \paragraph{ Le ribâ (l’usure) dans l’Islam}
 
 \href{https://www.ajib.fr/le-riba-lusure-dans-lislam/}{Le riba est l'usure dans l'Islam}
 
 \begin{quote}
    


Pour comprendre le sens réel du ribâ, pour connaître son statut légal dans l’islâm, houkm, qui est bien sûr, l’interdiction, ainsi que les méfaits individuels et collectifs qu’il engendre, il faut comprendre quelques points importants.

Parcourons ensemble ces extraite du livre « Le résumé des vertus de la religion musulmane » du grand savant \textit{Abdarrahman As-Sacdî} qui nous éclaireront sur la vue de la législation islamique, \emph{shariah}, sur les transactions, licites et illicites, ainsi que la sagesse du Législateur, Allah, soubhânah, en établissant cette auguste législation.

Nous verrons en particulier, les problèmes qui concernent le ribâ.

\begin{quote}
    
Que signifie le ribâ ?

Le ribâ signifie l’usure. Si l’on entend par « usure » le prêt à taux supérieur à zéro.

Car la définition contemporaine de « l’usure » est « l’intérêt excessif » ou « l’intérêt supérieur au taux légal », tandis que le ribâ est le prêt à intérêt, aussi minime soit-il.

Le prêt islamique légal est donc le prêt à zéro intérêt. Et le ribâ est l’usure dont l’intérêt est supérieur à zéro.

Il y a plusieurs types de Ribâ :
\begin{itemize}
    \item 1/Le prêt à intérêt : le surplus, aussi minime soit-il, perçu, pour le prêt, par le créancier en échange du délai accordé.

 \item 2/Ribâ al-fadhl : le surplus perçu lors de l’échange d’un bien ribawî (relatif au ribâ) tel que l’échange de l’or avec l’or, de l’argent avec l’argent, du blé avec le blé… etc.

 \item 3/Ribâ an-nasî’ah : différer l’encaissement lors d’un échange d’un bien ribawî de même nature, tel que l’or avec l’or, ou de nature différente, mais dont la cause ribawique est commune, tel que l’or avec l’argent.
\end{itemize}


\paragraph{Quelles sont les transactions dites légales dans le \emph{shariah}, halal ?}

La vente rendue licite par la législation islamique ainsi que le louage[1], les sociétés et les différentes transactions ; celles où s’échangent les marchandises entre les gens, les dettes, les utilités… etc.

\paragraph{Pourquoi une transaction est légale dans le \emph{shariah}, dite : halal ?}

La législation parfaite a permis ce type de transaction et l’a autorisé aux hommes, car cela assure leurs intérêts – nécessaires, indispensables ou complémentaires -.

Elle a élargi ainsi considérablement la liberté des hommes afin que leurs affaires et leurs conditions se réforment et que leurs vies s’ordonnent.

Quelles sont les conditions de la légalité d’une transaction ?

\textbf{La législation, \emph{shariah}, posa comme conditions pour que les transactions soient licites :}

1-La satisfaction des deux parties ;

2-la clarté du contrat ;

3-la connaissance de l’objet du contrat, de la durée du contrat et des conditions qui en résultent.



Quelles sont les transactions interdites dans la \emph{shariah}, dites : haram ?

La législation, \emph{shariah}, a interdit tout ce qui comporte un préjudice ou une injustice tel que :

1-Les différents types de jeu de hasard,

2-le Ribâ ;

3- l’ignorance[2], la jahâlah.
\end{quote}
 

Cette division, licite et illicite, halal et haram, prescrite par Allah et révélée à Son Prophète Mouhammed, ainsi qu’à tous les Prophètes avant lui, a pour objectif essentiel de préserver le but premier escompté par ces opérations d’échange entre les humains : l’intérêt mutuel, l’avantage commun à toutes les parties de l’échange et le bénéfice pour tous.

Inversement, tout ce qui n’apporte pas d’intérêt commun et ne fait pas bénéficier toutes les parties qui opèrent la transaction est illicite, haram et interdit.

Car cela va à l’encontre de la raison de cette opération d’échange, la transaction, qui est l’intérêt commun et l’absence de préjudice.

\textbf{Le ribâ fait partie de cette catégorie car son intérêt va dans un sens unique, qui est celui de l’enrichissement du créancier, et il porte un préjudice considérable au débiteur.}

Ceci s’appelle : injustice, iniquité et celui qui en profite ne fait que manger l’argent et les biens des autres injustement.
 \end{quote}
 
 
 \paragraph{Dictionnaire du Qoran}
 

 \section{Demystifying Riba through the Methodology of Muslim Jurists}
 \href{https://www.proquest.com/docview/2352353188?accountid=143046&parentSessionId=Sg7rHgjzHS0M9aF4pv1Nx2pxOOyR5LkKvILKrOVqG1o\%3D&pq-origsite=summon}{Demystifying Riba through the Methodology of Muslim Jurists}\sn{MUHAMMAD ZAHID SIDDIQUE; MUHAMMAD MUSHTAQ AHMAD.
Islamic Studies; Islamabad Vol. 58, N° 2,  (Jun 30, 2019): 169.}
 \cite{Siddique:DemystifyingRiba}



 \subparagraph{Summary}
  \begin{quote}
  In the post-colonial world when Muslims tried to restructure their public life in accordance with the shari‘ah, they developed a new discipline known as Islamic economics one of the central constructs of which is prohibition of riba. Unfortunately, the discussion among modern academic circles assumed a wrong methodology, which resulted in mystification of this concept and, hence, in a number of unsettling questions. This paper explains the nature of the mistake committed by modern Muslim scholars and economists. It also outlines the structure of correct methodology, which was laid down by premodern Muslim jurists for understanding the concept of riba and all other legal terms. The paper develops a consistent analytical framework for addressing majority of the questions on the subject of riba and attempts to rectify the mystification created around this concept.
  
  \end{quote}
 \subparagraph{Texte intégral}
 \begin{quote}

Keywords: riba, interest, loan, bay‘, Islamic economics, Islamic legal theory, financial regulations of Islam.

1. Introduction

After losing their political rule to the imperial powers, Muslim societies faced the widespread dominance of interest-based banking system. According to the majority of Muslim scholars and jurists, bank interest (riba) was not allowed, but Muslim societies got engaged in it due to growing spread of interest-based banking in modern societies and the non-availability of interest-free banking. \mn{noter aussi que le développement de la banque s'est fait en parallèle dans les sociétés occidentale. }

Muslim scholars and economists demanded its alternative soon after Muslims got independence from their foreign masters. Commitment to follow religious teachings in the public affairs of life and liberty from the colonial oppressors provided the required room, which resulted in what is now known as Islamic economics in general and Islamic finance/banking in particular.

One of the central concepts of Islamic banking is prohibition of riba, which unfortunately and surprisingly remained controversial among Muslim economists and scholars. Different perspectives about the meaning of riba prevailed in the twentieth century. Majority view holds that both usury and bank interest are equally impermissible in Islam while business profit is allowed.\footnote{1 For  detailed  arguments  of  this  position,  see  Ab┴ ’l-A‘la  Maud┴di,  S┴d  (Lahore:  Islamic Publications,  2000),  110–12;  M.  Umer  Chapra,  “The  Nature  of  Riba  in  Islam,”  Hamdard Islamicus  7, no.  1 (1984):  3–24;  Muhammad  Shafi‘, Mas’alah-i  S┴d  (Karachi: Idarat  al-Ma‘arif, 1996), 43–47; Muhammad Ayub, “What  is Riba? A  Rejoinder” Journal of Islamic  Banking and Finance  13,  no.  1 (1996):  7–24;  Muhammad  Taqi Usmani,  The  Historic  Judgment  on Interest Delivered in the Supreme Court of Pakistan (Karachi: Idarat al-Ma‘arif, 1999), 12–16; Mohammad Nejatullah  Siddiqi,  Riba,  Bank  Interest  and  the  Rationale  of  Its  Prohibition  (Jeddah:  Islamic Research  and  Training  Institute, 2004),  45–48;  and  Mahmoud A.  El-Gamal,  Islamic  Finance: Law,  Economics,  and Practice  (Cambridge:  Cambridge University  Press, 2006),  46–52. Within this  category,  there  are  further  two  approaches.  One  approach  that  represents  traditional ‘ulama’ emphasises the resurgence  of only those business contracts that were approved by  the early  Muslim  jurists.  It proposes  profit-and-loss  sharing (PLS)  as an  ideal alternative  to riba. Though it does not deny the permissibility of other than PLS-based financing instruments such as murabahah and ijarah, yet it affirms that equity-based financing method is the primary means of achieving desirable economic objectives. The second approach is pragmatic one. It justifies a more liberal and flexible stance on structuring shari‘ah-compatible transaction forms that looks for financial engineering to meet all demands of modern banking customer.  }

Contrary to the majority view, some modern Muslim scholars dispute that the Qur'anic term riba includes interest paid and charged in the banking system.\footnote{ Muhammad Rashid Rida (d. 1935) was among the foremost proponents of this theory. See his al-Riba  wa ’l-Mu‘amalat  fi ’l-Islam  (Cairo: Dar  al-Manar, 2007).  Also see  Sayyid  Yaqub Shah, “Islam  and Productive  Credit,” The  Islamic  Review  47,  no.  3 (1959):  34–37; Fazlur  Rahman, “Riba  and  Interest,”  Islamic Studies  3, no.  1 (1964):  1–43; Timur  Kuran, “On  the Notion  of Economic  Justice  in  Contemporary  Islamic  Thought,”  International  Journal  of  Middle  East Studies  21,  no. 2 (1989): 171–91;  Izzud-Din Pal,  “Pakistan and  the Question  of Riba,”  Middle Eastern Studies 30, no. 1 (1994): 64–78; and ‘Abd al-Karim Athari, S┴d Kiya Hay? (Mandi Baha’ al-Din: Anjuman-i Isha‘at-i Islam, 2008), 8–12}
To them, replacing bank interest with anything else is tantamount to obstructing natural operation of economy and creating inefficiencies because interest is the just reward of capital reflecting its marginal productivity. \sn{ Constant  J.  Mews  and  Ibrahim  Abraham,  “Usury  and  Just  Compensation:  Religious  and Financial Ethics in Historical Perspective,” Journal of Business Ethics 72, no. 1 (2007): 1–15}

3 According to this perspective, there is no need to have anything distinct like “Islamic banking” to begin with because the existing system is already Islamic.


Finally, on the other extreme are \textit{Muslim socialists}\sn{A l'autre extrême ?} who develop their version of Islamic economics based on socialist policy package.\sn{See Ghul┐m A╒mad Parvaiz, Ni╘┐m-i Rub┴biyyat (Lahore: Id┐ra-i ║ul┴‘-i Isl┐m, 1978). }

 Since socialism considers wage as the only legitimate reward of a factor input, the scope of riba is much wider than usury and bank interest according to these scholars. 

It is believed by some\sn{Raf┘‘ All┐h Shih┐b, Kir┐yah-i Mak┐n┐t k┘ Shar‘┘ ╓aithiyyat (Lahore: Kit┐b Ghar, 1981)} that rental earnings on an asset is also included in riba because rent is similar to interest earnings as both are the prices of capital determined by similar market forces. Others are of the view that not only bank interest but also trade or merchant profit is banned under the category of riba.


6 They argue that as lender is forbidden the right to charge interest from poor borrower, so should be the rich industrialists and landlords from appropriating lion's share of value-added on the name of profits.

They assert that loaning riba (riba 'l-qard) covers money lenders and hoarders who charge against time while riba of excess (riba 'l-fadl) is the domain of landlords, merchants, and middlemen who exploit poor workers and make unequal exchanges. 

These differing perspectives are shown in figure 1. Because this last perspective about riba has gained very little popularity among Muslim scholars and masses as compared to the first two, we exclude its analysis from the scope of this paper, though it would be analysed indirectly.


Spectrum of Riba and Its Scope Liberals Mainstream Muslim Socialists Riba = Usury Riba = Usury, Riba = Usury, (exploitative Interest (all returns Interest, Rent, return on loan) on loan) Profiteering


\includegraphics[width=\textwidth]{CourantsIslamContemporain/ImagesCourantsIslamContemporain/Riba.png}



The above differences have left scholars divided on several important questions that demand straightforward answers. Those questions include the following ones:
\begin{itemize}
    \item 
1. Is bank interest prohibited in the light of the Qur'an and the sunnah? If yes, how?
    \item 
2. Whether the Qur'anic term riba includes all kinds of interest rates or it relates only to the excessive interest rates?
    \item 
3. Whether the scope of riba extends to the interest charged and paid on business transactions in the banking system or is restricted to the interest charged on consumption loans only?
    \item 
4. Does Islam allow loan transactions? If yes, how and in what form?
    \item 
5. Is paying interest a lesser evil as compared to charging it?
    \item 
6. Is borrower always \textit{mazlum} (a losing party) in an interest bearing loan transaction?
    \item 
7. Does Islam allow indexation of loans on the grounds of inflation?
    \item 
8. Is credit-sale with higher deferred price as compared to the spot price allowed?
    \item 
9. Does Islam approve of “time value of money,” especially when charging higher deferred price is allowed in a credit sale?
    \item 
10. Are future currency contracts permissible in Islam?
    \item 
11. How and to what extent is salam transaction permissible?
\end{itemize}


These are but a few questions.
\begin{Synthesis}
We show in this paper that whatever confusion prevails among contemporary scholars on this subject is the outcome of following an inadequate methodology for determining the meaning and scope of riba.
\end{Synthesis}
 In fact, this methodology has mystified the nature of riba, which is otherwise clear when viewed from the methodological view point of the eminent Muslim jurists of the past. The mystification is such that not only it results in confusing answers to these questions but it also begets confusing questions. Unfortunately, the confusion has built up to the extent that the Federal Shariat Court of Pakistan has been struggling to come up with a definition of riba. It is in this background that this paper attempts to explain:

(1) the contemporary Islamic economists' methodology of interpreting and classifying riba; (2) why this methodology is wrong and insufficient; (3) the methodology of understanding riba on the pattern of Muslim jurists of the past; (4) that the methodology given by the Muslim jurists is coherent and compact.

The reader will encounter a number of arguments in this paper that are advanced by those who justify bank interest. Since the paper deals with the legal substance and not with the economic merits of arguments, hence we will restrict ourselves to the legal analysis of those arguments and leave aside their economic analysis and rationale, which require an altogether different methodology. Any legal system has three aspects: (1) what: the legal rulings (i.e., ahkam); (2) how: the rules of deriving those legal rulings (i.e., usul al-fiqh); and (3) why: the underlying rationale(s) and wisdom behind the legal rulings (i.e., hikmah)

It is important not to mix these aspects. The present study deals with the first two aspects of the issue of riba. Moreover, the classification of riba discussed in this paper is primarily based on the methodology of Hanafi jurists for ensuring analytical consistency. We presume that a school of law represents an internally coherent system of interpretation and that mixing up the views of the various schools results in inconsistencies.7 However, views of the other schools have been briefly mentioned in the footnotes wherever required. Finally, the paper does not attempt to show that the Hanafi jurists' approach is superior to all others, rather it explains that the classical jurists' approach (whether Hanafi, Maliki, Shafi‘i or Hanbali) to understanding riba is superior to that of the modern scholars. The methodology of these jurists share several common results that are important in order to answer the above questions.

Following section outlines the method adopted by modern Muslim scholars and economists. The next section discusses problems in this methodology and develops the skeleton for the methodology that is then applied in the coming section, which details out the general rules of riba alongside their resulting implications. The last section concludes the paper by giving a comprehensive definition of riba based on discussions in sections three and four.


\newpage
\subsection{Outline of the Mystifying Methodology}

Imran Ahsan Khan Nyazee\sn{\href{https://en.wikipedia.org/wiki/Imran_Ahsan_Khan_Nyazee}{Pakistan} Nyazee's academic career was inspired by the work of Abdur Rahim. Nyazee argues firstly, that due to its unique set of principles of interpretation, each school of Islamic law represents a theory of law unto itself. Secondly, he points out that Istiḥsān cannot be understood without understanding of the workings of qiyās. It is, therefore, difficult to accept that there was no system of interpretation before al-Shāfi‘ī's time. Thirdly, he concludes that the uṣūl al-fiqh never existed. Furthermore, Nyazee describes beyond the individual fikh of each school of law, another theory of interpretation called maqāṣid al-sharī‘ah (theory for the purpose of the sharī‘ah) which was developed by al-Ghazālī. Nyazee has written and self-published on a number of aspects of Islamic law. He agrees with most Muslim scholars that strictly speaking, selling money (taking interest) is prohibited, according to Islamic law. Some point out a difference between the treatment of riba in the Qur'an versus the Sunnah but Nyazee the two approaches are actually one and the same.Nyazee also proposes that all loans (except those of a charitable nature without a fixed period of repayment) and therefore all banking is prohibited and unIslamic. Nyazee is equally intolerant of murabaha, the Islamic system of business where in-put costs and mark-ups are made transparent between vendor and buyer. He argues riba will inevitably enter such transactions.[10] He extends the prohibition to the creation of wealth on the basis of debt and the fractional reserve banking system. These elements along with zakat (the system of alms-giving) he says, are the differences between Islam and capitalism. He advocates the use of the gold and silver dinars and dirhams as the currency of the Muslim community. Nyazee would also prohibit the corporation or 'legal personality' under Islamic law.} explains that the methodology adopted by modern scholars for determining the meaning of riba is the same, though they disagree in their conclusion regarding whether or not bank interest is riba.8 The fundamental problem of their methodology lies in overlooking the inherent link between the Qur'an and sunnah. This methodology of interpreting riba was initiated by Muhammad Rashid Rida (d. 1935) \sn{voir p. \pageref{Theol:Rida} Frère musulman pas le voyage en Europe. Plue el manar un commentaire coranique, sensé être l'héritage d'abdu}, which goes as follows:9

\paragraph{Lecture de Rida El Manar}
Riba is classified into two categories, riba of the Qur'an (also equated with riba 'l-nasi'ah, i.e., interest on loan transaction) and riba of hadith (equated with riba 'l-fadl, i.e., interest on exchange transaction).

Rida begins with literal meaning of the word riba (excess) and then traces some riba-based transactions practiced by Arabs during the time of Prophet (peace be on him). Rida, relying on some commentators of the Qur'an, asserts that the Qur'anic verse regarding riba deals with a specific practice of Arabs known as credit-sale where the payment of price is deferred to a future period while delivery of goods takes place on spot. Because a seller is allowed to charge whatever price he wants in a sale transaction, no riba is involved in the original price negotiated between the two parties-any excess in future price becomes part of the price. However, they used to increase the price excessively whenever the debtor would be unable to settle his debt obligations at the end of payment period. The debtor was given the option, “Will you pay the debt or increase the amount in lieu of delay?”

For Rida, it was this excessive rate (doubling and multiplying) of interest in debt-based transactions added to the original sum at the end of payment period which was prohibited by the Qur'an (he called it riba 'l-jahiliyyah).10 From this, he concluded that the bank interest is not the same riba that was deemed impermissible by the Qur'an because (a) it is neither doubling and redoubling of rates (b) nor the excess is stipulated in the initial period of the banking transaction-he assumes that the initially added interest is part of the principal or original sum just like the original sum in case of credit-sale. 
\begin{Synthesis}
Hence, for Rida, only compound interest is prohibited.
\end{Synthesis} 

Other scholars, supporting Rida's view, added that business loans were not common among Arabs as theirs was a subsistence economy; loans were largely taken by poor people for consumption purposes on interest and whenever they were unable to repay them at due time, excessive interests were added to the original sum. Hence, it was this type of interest that was declared prohibited by the Qur'an and it has nothing to do with the modern commercial loans, \textbf{which are mutually beneficial for both parties.}11

Having ascribed this meaning to the Qur'anic word riba on the basis of some historical traces, Rida then explains the form of riba declared impermissible in the sunnah as a distinct prohibition from that of the Qur'an.
\begin{Def}[Riba] Usure, profit ou gain réalisé sur un prêt.
\end{Def}

\begin{Def}
[Riba al-buyua] : Usure. Opération de vente dans laquelle une matière première est échangée contre la même matière première mais en quantité différente et la livraison d’une des matières premières est postposée. Pour éviter le riba al-buyu, les matières premières échangées par les deux parties devraient être en quantités égales et l’échange devrait être instantané. Riba al-buyua a été condamné par le Prophète Muhammad afin d’éviter que le riba (intérêt) n’affecte insidieusement l’économie.
\end{Def}
\begin{Def}[Riba al-duyun] : Usure d’une dette.
\end{Def}
 
\begin{Def}[Riba al-fadl] : La différence de quantité entre deux biens échangés et comportant du riba.
\end{Def}
\begin{Def}[Riba al-nasiah] : La différence de paiement liée au report de deux biens comportant du riba
\end{Def}\mn{cf Glossaire des termes financiers  Islamiques \href{https://www.cairn.info/la-banque-et-la-finance-islamiques--9782804167042.htm}{Banque et finance islamique} } He calls it \textit{riba 'l-fadl} which emerges in the exchange of two counter values of the same or different species and hence also called \textit{riba 'l-buyu‘}.12 The position of Rida, which may be termed as minority view, is summarised in figure 2.


\includegraphics[width=\textwidth]{CourantsIslamContemporain/ImagesCourantsIslamContemporain/RibaRida.png}

Thus, Rida dichotomised the two concepts of riba, one attributed to the Qur'an and another to the sunnah. He finally declared the first one as real or explicit riba while latter as lighter or implicit riba.
\textbf{
Though the majority of contemporary scholars did not agree with the conclusion drawn by Rida about legitimacy of bank interest, however they adopted his methodology of classifying riba. The only difference in their opinion is that riba of the Qur'an includes all rates of return on loan and it is not merely restricted to the compound interest of jahiliyyah}\sn{la période pre islamique}


To them, business loans were a part of Arab's economy and any contractual return to lender is unfair because this is tantamount to refusing to share business risk with the borrower. We can depict their views in figure three.
\begin{Synthesis}
On a donc un nouveau problème, le pret commercial lié avec l'industrie et face à ce nouveau problème, une lecture de Rida qui est assez positive (Riba = Excess) mais qui note une différence entre Coran et Sunna) et fait une distinction entre les deux, reprises ensuite par les autres légistes, mais en repartant d'une lecture stricte du Riba comme intérêt. Il conveint d'articulier Coran et Sunna de façon non parallèle mais l'un par l'autre.
Il convient de montrer l'évolution du prêt commercial au XIX
\end{Synthesis}
Because the sunnah is not linked with the Qur'an in this methodology, both the minority and majority Muslim economists have struggled to explain as to why someone would engage in exchange transactions of the forms mentioned in hadith. Some opined that these transactions are declared impermissible because they may open the path for the “real riba” (i.e., riba of the Qur'an).


14 Others assumed that it was meant to discourage the practice of barter exchange and promote market exchange through a medium of exchange.15 Yet another view argues that it eliminates the possibility of benefiting from asymmetric information of the contracting parties.16 The truth is that none of these explanations makes the point.

2.1. The Nature of Debate within Minority and Majority Schools

The debate that has taken place within the followers of this mystifying methodology on the issue of why or why not bank interest is riba may briefly be summarised here. As explained above, Rida asserted that bank interest was not included in the Qur'anic concept of riba of debt because it was different from the riba that was charged by Arabs on credit-sale transaction by doubling and multiplying the price whenever the debtor was unable to settle his debt at due time and asked for relaxation in payment period.18 Rida explained that the Qur'anic verse “Allah has permitted bay‘ and prohibited riba”19 referred to this riba. To strengthen his case, he argued from the verse: “O Believers! Do not devour riba doubled and multiplied and fear God so that you may prosper.”20

This verse complements the former verse in the sense that what was implicit in the first verse was made explicit in the latter-both verses referred to the practice of doubling and multiplying of interest and none of them forbad the bank interest.

How do the majority of scholars respond to this argument? For example Usmani notes the Qur'anic verse:

O you believers! Fear God and give up riba that remains outstanding if you are true believers. Behold! If you do not obey this commandment, then God declares war against you from Himself and from His Prophet. But, if you repent (from riba), then you are entitled to only your principal amounts. Neither should you inflict harm to others, nor others should do harm to you.21

The argument is based on the emphasised words ‘you are entitled to only your principal amounts (ra's al-mal)'. He infers from these words that the rightful entitlement of lenders is the original sum advanced; he cannot charge any increase whether small or large (doubled and tripled). To him, the verse (3:130) forbids a severe form of riba where interest is multiplied, but it does not restrict riba to this specific form. Hence, bank interest falls within the purview of the Qur'anic verse “Allah has permitted bay‘ and prohibited riba.”22 They are also of the view that charging interest on commercial loans was also practiced by Arabs.23

Does the above analysis of mainstream scholars guarantee the prohibition of bank interest? We are afraid it does not. Their arguments rest on two assumptions:

(1) The verses (2:278-79) address the issue of loan-transaction.

(2) Ra's al-mal (principal amount) can only refer to the original principal advanced in loan.

Both of these assumptions are problematic. Following submissions can be made against them:

(a) If the meaning of the verse is to be determined with reference to historical practices, one can equally claim, just like Rida, that the verse is not about loan transaction but about credit sale. In that case, ra's al-mal is not referring to the principal amount lent; rather, it is the deferred future price of the goods sold. On which legal grounds or facts can this claim be dismissed?

(b) Further, this deferred price might include increase over and above spot price. Hence, the future price could consist of two components: spot price plus some additional profit. The sum of these two would constitute ra's al-mal (principal amount) in this transaction (i.e., principal amount in credit sale (ra's al-mal) = spot price + extra profit)

Whenever a debtor was unable to repay full amount, further multiplied increase was added to this original sum, Rida called it interest. This would increase the due amount to: total amount after increase added due to delay, which is interest in addition to ra's al-mal.

Using this structure, one can then argue that the initially added interest in a loan transaction is equivalent to initially added “extra profit,” which becomes part of ra's al-mal. Therefore, entitlement to the ra's al-mal means entitlement to the simple interest, as claimed by Rida.

(a) The only legal justification for ascertaining that the verse is about loan transaction is based on the words “ra's al-mal” (principal amount). But how can it be settled that ra's al-mal here means ra's al-mal of a loan transaction? This question is important because a number of transactions constitute a component of ra's al-mal. For example, there is ra's al-mal both in mudarabah and musharakah contracts. How to exclude these forms of ra's al-mal from the purview of the Qur'anic verse? If someone says, “This verse is about loan, so ra's al-mal refers to that of loan contract and not of mudarabah and musharakah,” he is clearly arguing in circularity. The argument goes like this:

Q: How do we know that the verse is about loan contract?

A: Because the verse talks about ra's al-mal.

Q: How do we know that ra's al-mal here refers to that of loan?

A: Because the verse is about loan contract!

A circular argument is no argument.

(b) Muslim Socialists could maintain that ra's al-mal means principal amount of all business contracts. Therefore, it is not legitimate to charge any excess over and above principal amount, no matter it is mudarabah, musharakah or ijarah.

Not only that the analysis of mainstream scholars does not necessarily imply the prohibition of bank interest, it leads to a set of unsettling arguments that have left Islamic economists bewildering about some basic issues. For example,

(1) Even if it is agreed that ra's al-mal means principal amount of a loan transaction, does it mean “nominal” amount or the “real” (inflation adjusted) amount? Again, what are the legal grounds to settle this issue? Because there are no clear-cut legal grounds available in this methodology, we see scholars are divided on this subject matter-some allow indexation of loan against inflations while others do not.

(2) What about the question of “time value of money?” This question poses challenge for Islamic economists because they, as a rule, approve the practice of charging higher price in credit-sale and murabahah.

(3) The Lawgiver has allowed salam, what are the legal grounds for not extending this permission to currency salam (future currency contracts)?

Undoubtedly, majority view has addressed these issues, but the answers do not seem to be stemming out of a coherent analytical legal system. This approach is often found mixing up legal analysis with economic analysis. This missing coherent analytical legal system is the root cause of most of the mystification that has prevailed all over. It is an unfortunate state of affairs and it is high time to demystify things.

3. Methodological Assumptions of Premodern Muslim Jurists (Fuqaha') for Understanding Riba

To understand the method used by the eminent premodern Muslim jurists for understanding riba, three methodological issues (MI) need to be clarified. They are explained by Nyazee in detail.24

1) Link between the Qur'an and Sunnah

The methodology adopted by the modern Muslim scholars and economists is misleading because it delinks the Qur'an and sunnah. It assumes that the meaning of riba is different in the Qur'an and sunnah, which is not the case. To explain the nature of error made by both the groups, it should be noted that Muslim jurists (fuqaha') classified riba in the category known as mujmal (unelaborated)25 whose meaning and scope cannot be determined without explanation (bayan) of the Lawgiver (Shari‘). The famous hadith (as given in footnote 1) that explains different usurious transactions actually does not add something to the Qur'anic word riba. Rather, it defines its meaning and scope.

Thus, while to the contemporary scholars the meaning of riba is known independent of hadith and they see hadith as adding some more cases to the Qur'anic concept of riba, the jurists say that hadith is the definition of the term riba used by the Qur'an. Thus, riba 'l-nasi'ah and riba 'l-fadl both are included in the Qur'anic concept of riba.

2) Relationship between Loan and Bay‘ (Exchange)

Loan is also classified as a form of exchange transaction (bay‘)26 by Muslim jurists. The scope of this paper does not allow detailed analysis of this assertion.27 For descriptive purposes, it can be seen that a loan of Rs X is an exchange of Rs X today with Rs X after time deferment (and with Rs X + Y if interest payment of Rs Y is included). Figure 4 depicts this nature of loan transaction by illustrating a loan transaction between Mr. A and B:

3) Skeleton of a Coherent Legal System

A coherent shari‘ah-based legal system consists of a set of general rules, called

‘azimah by the jurists, supplemented by some exemptions to these laws, called rukhsah. In the words of Nyazee, ‘Azimah (lit. determination, resolution) is applied to mean a rule that is applied initially and for itself. Such rules form the backbone of the law. As against this, there may be a rule that goes contrary to the requirements of the initial rule, but is permitted by the law. This rule is considered to be a rukhsah (exemption) from the initial rule.28

This classification of ‘azimah (the higher or first order rules) and rukhsah (the lower or second order rules) is important for several reasons.

First, it explains the order in which the rules have to be applied.

Second, it explains why sometimes two opposing cases may be allowed within a given skeleton of law.

Third, the order of rules implies that an exception cannot be extended using any method of argument, whether analytical or analogical. On the other hand, extension of first order rules is legitimate by these methods. In other words, it is not allowed to build a sub-legal system based on exemptions because otherwise it starts negating the primary provisions and objectives of the law-an exemption from the general rule must remain an exemption.

Fourth, because of the logical hierarchy in the operations of ‘azimah and rukhsah, it is clear that an exemption from a rule cannot be used to nullify or change the shari‘ah status (hukm) of any other case that is derived from the general rules. Alternatively put, an exemption (a lower order rule) cannot prevail over the higher order rules.

Fifth, because all rules and exemptions are derived from nusus (the Qur'an and sunnah), hence the only justifiable exemptions are the ones, which are given in nusus (i.e., stated by the Lawgiver Himself).

We call these nusus the “facts” of the shari‘ah-based legal system in this paper. Given these “legal facts,” the task of a jurist is to derive those general rules (‘azimah) from the facts, which render these facts internally consistent and extendible on the one hand and highlight the exemptions (rukhsah), if any, on the other.29 Finally, the general rules and exemptions generate some implications, called ahkam. This skeleton of a shari‘ah-based legal system is illustrated in Figure 5. We apply this skeleton in this paper to elaborate riba.

The relevant “legal facts” used by premodern Muslim jurists to derive general rules and exemptions are quoted at the relevant places in this article. We are now in a position to take on the issue of derivation of the general rules and the implications from those “facts.”

4. Underlying Rules behind the System of Bay‘ in Jurists' Methodology

Our intention in this paper is to reveal that the apparently large and complicated system of legal injunctions (ahkam) is reducible to a few set of rules derived from fewer legal facts. We propose that a majority of ahkam (legal injunctions or provisions) governing economic transactions (buyu‘) can be derived from three broad rules:

1. Rules of riba mentioned in the sunnah. This is not a single rule, rather a set of rules as explained below.

2. Rule about the sale of goods not possessed by a person.

3. Rule about exemption that an exemption is to be treated as exemption.

Before explaining these rules, we first explain the context of the Qur'anic verses that underlies the jurists' methodology of riba to clarify the misconception that the relevant verses of Surat al-Baqarah are about loan transaction and not exchange (bay‘).

4.1. The Context of the Verses of Surat al-Baqarah

The Qur'an states that the disbelievers said, “Verily, bay‘ (sale) is just like riba.” In response to this, it was said, “Allah has permitted bay‘ and prohibited riba.” To understand why the disbelievers said this, consider these three transactions:

(a) A gives B 100 grams of gold in exchange of 110 grams of gold to be paid after one year. This is primarily a sale contract as explained previously (i.e., exchange of 100 grams gold with 110 grams gold with time lag) and involves riba (how, this will be explained in the next section but take it for granted for the moment).

(b) A asks B for 100 grams of gold in exchange of, say, 500 kg wheat at spot. This is a legitimate regular sale contract.

(c) A demands from B 110 grams of gold in exchange of 500 kg wheat for payment of price after one year: this is credit sale contract with higher deferred price as compared to spot price and is also legitimate (this is explained in section 4.3).

The credit sale was a common practice among Arabs and, therefore, they were confused as to why the transaction (a) is impermissible and (c) is permissible while the two are quite similar in nature (i.e., both are credit sales and both involve access payment). In (a), 10 grams of additional gold are paid as counter-value for 100 grams of gold for a delay of one year and similarly 10 grams of gold are paid for a delay of one year in transaction (c). It is for this reason that disbelievers said, “Verily sale is just like riba!” That is, transaction (c) (i.e., the credit sale) is similar to the transaction (a). The technical reason for allowing transaction (c) and forbidding (a) is the similarity of genus which is explained by the sunnah. This is explained in the next sections in detail, but the important point to note here is that the assumption that the Qur'anic term riba is not about sale contract, rather it is about debt, is not implied by these verses.

Thus, the verse says that Allah has approved all forms of buyu‘ (exchange transactions) except those which involve riba.30 The natural question then arises: what is this thing called riba? Has the Qur'an given any definitive description of riba?

One may make one of the two assumptions here. First, the concept of riba was largely a sort of common knowledge for everyone and, hence, it required no legal description by the Qur'an. That common knowledge is traceable by an examination of historical record of Arabs which provides sufficient legal foundations for determining the meaning of riba. As far as the details of riba in the sunnah are concerned, they were additions over and above to that common knowledge of riba and most of these additions were unknown to the Arabs. The liberals and mainstream scholars share this assumption and we believe that this assumption constitutes what we called the “mystifying methodology.”

Second, some forms of riba may be or actually known to the Arabs but these do not set the legal standard against which the Qur'anic concept of riba is to be determined. As it is a legal term, its meaning has to be sought from the Lawgiver. In technical sense, the jurists call it mujmal (unelaborated) for which elaboration (bayan) is sought from the Lawgiver. This elaboration of the legal meaning of the Qur'anic term riba is given by the sunnah. After this elaboration by the Lawgiver, its meaning is determined definitively and it becomes mufassar (elaborated). This is the methodological assumption that the jurists use not only for defining riba but also for other legal terms of the Qur'an, such as salah, zakah, and so on.31 Thus, according to this second assumption, the practices and concepts of Arabs may be referred by the Qur'anic concept riba but it is not the benchmark against which we assign legal meaning to the Qur'anic terms.

For example, the Arabs had some concepts about how to offer salah (prayer), but this information does not define the legal meaning of the Qur'anic term salah nor is this concept limited to this information set. Similar is the case with riba. The Arabs might have been aware of some forms and practices of riba but that does not constitute the legal definition of riba. When the jurists classify a term as mujmal, they mean that this term is a technical legal term and its meaning should be determined with reference to the words of Lawgiver Himself, neither by the linguistics (dictionary) nor by the historically known social concepts and practices that hover around that technical term. It should be emphasised here that considering riba as mujmal does not mean that the Arabs did not know the meaning of this word at all. Nor does it mean that the pre-Islam Arabs did not identify certain transactions as riba-in fact they did and the jurists did consider it part of riba.32

It only means that the meaning of riba in Islamic law is not limited to, and is not based on its usage in the pre-Islam Arabia. The Qur'an and the sunnah added several shades of meaning to this concept. That is why, it became a “technical term” of Islamic law. Hence, its meaning and scope cannot be determined by its dictionary meaning or its practice and understanding by the pre-Islam Arabs. Rather, it must be determined by the Qur'an and the sunnah, like any other legal term such as salah and zakah. Just as we cannot classify concept salah as salah of the Qur'an and salah of hadith, similarly we cannot dichotomise riba. Once it is established that the meaning of riba must not be gathered from pre-Islamic usage and practices but from the Qur'an and the sunnah, the next question is: how to explain the various usages of riba in the Qur'an and the sunnah? The answer, as per the well-established methodology of the jurists, is to consider the sunnah as the elaboration of the mujmal verses of the Qur'an.

This methodology is employed by the jurists for determining the meaning and scope of salah and zakah as well as riba. Let's follow through the path of righteous ones here and have its blessings.

4.2. General Rules of Riba When Transacted Species are Same

Keeping these in mind, one has to understand the classification of riba in the system of Muslim jurists. Because the sunnah defines riba, note the words of hadith,

When you exchange gold for gold, silver for silver, wheat for wheat, rice for rice, dates for dates, and barely for barely, then exchange like for like (in equal measure) and exchange them hand to hand (at spot), else it will be riba.33

To understand what it says, consider these transactions:

1) exchange of 1 gram gold for 1 gram gold on spot;

2) exchange of 1 gram gold for 2 grams gold on spot;

3) exchange of 1 gram gold at spot for 1 gram gold with delay;

4) exchange of 1 gram gold at spot for 2 grams gold with delay.34

As per the hadith, the first transaction is allowed; the second one is disallowed because it involves excess in measurement/quantity (called riba 'l-fadl); the third transaction is also impermissible because the hadith says that the exchange of homogeneous goods is allowed in equal measurement provided it is on spot; therefore, this transaction involves the riba 'l-nasi'ah (i.e., riba of delaying); finally, the fourth transaction involves both types of riba. These transactions provide two guiding rules (R):

R 1.1) Goods of the same species cannot be exchanged immediately unless their measurement (in terms of weight or volume) is same.

R 1.2) Goods of the same species cannot be exchanged with time lag, even with same measurement.

4.2.1. Implications

Five important implications (I) should be noted.

I. 1) Impermissibility of Market for Loanable Funds

Application of rule 1.2 gives the important implication that loan, with or without interest, is prohibited in Islam because, as explained above, a loan is an exchange of homogeneous goods with time lag. Does it mean that loaning is not allowed in Islam under any circumstances? Of course, this implication of the general rule is at odd with a number of legal facts (nusus), which promise reward for offering loan to the needy ones. How to reconcile these apparently contradictory legal facts now? This is where the concept of rukhsah (exemption) is activated by the jurists. Though loaning is against the general rule (‘azimah) given by the Lawgiver, yet it is allowed by Him as an exemption from this prohibition if it takes the form of benevolent giving (tabarru‘ or sadaqah).35 Loan is classified as tabarru‘ if:

(a) it is out of the intention of benevolence to the other person (i.e., the lender consciously bestows upon the borrower the benefits associated with his asset);36

(b) no increase in its value is stipulated, else it would cease to be benevolent and would involve riba 'l-fadl; and

(c) no contractual time limit is stipulated, the lender can ask for his asset anytime he wants.37 Stipulating (legal) time constraint in loaning activity makes it a business transaction as per the application of general rules of shari‘ah and, hence, unlawful because in that case it is simply the exchange of homogeneous goods with time delay, which is not allowed, whether or not interest factor is included. Moreover, making the time period binding would imply that the lender is forced to do, or to continue with, an act of charity. This is against the very nature of charity.

In short, this principle implies that Islamic law does not permit the “market for loanable funds.” It sees loaning as an act of benevolence, especially in favour of one's relatives.38 Stated alternatively, loan is purely a social transaction (a means of tying and strengthening social bonds) in Islam and not a business. It was in this social transaction capacity that the institution of loan prevailed for thousands of centuries not only in Muslim societies but also in other civilisations of the world until the emergence of capitalism in the fifteenth century.39 Note that this important result (impermissibility of the market for loanable funds) does not follow directly from the classification of modern scholars of Islamic economics, as the majority view allows interest-free non-benevolent loans as a general rule and not as an exemption.

This implication answers one of the important arguments in favour of bank interest given by some economists. The argument says that interest should be allowed in shari‘ah because interest is the price of capital and without interest the market for loanable funds cannot be equilibrated. Because we are not dealing with the economic merit of arguments in this paper, we ignore its economic substance and comment on its legal merit only. It is clear from the above implication now that this argument has no shari‘ah basis because shari‘ah does not allow market for loanable funds to begin with, let alone equilibrating it from shari‘ah perspective.

Before moving on to the next implication, the important implication and exemption regarding loan transactions be noted:

I 1.1) A loan transaction is prohibited, whether or not interest factor is added to it.

I 1.2) A benevolent interest-free loan is recommended as an exemption to the general rules of riba by the Lawgiver.

I. 2) Impermissibility of Bank Interest

All forms of bank interest, whether simple or compound, are prohibited by Islam as per Rules 1.1 and 1.2. Similarly, the fact whether loan is made for business or consumption purposes makes no difference to this result. There remains no confusion about these conclusions if the shari‘ah rules are applied with consistency. In fact, the practice of charging interest by the bank includes both kinds of riba and it, therefore, may be stated that it is the most comprehensive form of riba! This can be verified from the figure 6, which depicts detailed structure of riba-based transactions in case of homogenous goods (leaves aside heterogeneous goods for the moment).

I. 3) False Dichotomy between “Giving and Taking” Riba

The recipient of riba is not always the lending party as is usually perceived. It can be seen from above examples that in case of transaction (2) the lender is the beneficiary of riba, but in transaction (3) riba is received by the borrower, and finally both are its recipients in transaction (4). Hence, opinions such as “taking riba is a greater evil than giving it and, hence, paying interest to the bank is a lesser evil” are based on the fallacious assumption that it is only the bank that receives interest in a typical interest-bearing loan transaction. This wrong assumption is the outcome of using the wrong methodology outlined in section two.40

I. 4) Mutually Beneficial Riba is Prohibited

The view that bank interest realised in transaction (4) is or should be permitted (as claimed by liberal Muslim scholars) is implicitly based on the assumption that “two wrongs make one right”-that is, it assumes that mutually enjoyed riba of the lender and borrower can make this transaction acceptable while the matter of fact is that each of them is separately prohibited to begin with.

I. 5) Irrelevance of Time Value of Money

Following the wrong methodology has resulted in another confusing argument that the bank interest should be allowed because of “time value of money.” This argument is based on the presumption that Rs. 1 today is worthier than Rs. 1 tomorrow. Why? Economists believe that this is due to the subjective time preferences of an individual. A rational (i.e., self-interested utility maximising) economic agent is said to have positive time preferences in the sense that consumption today is preferred to consumption tomorrow because the latter is uncertain, which makes him impatient, thus he wants to have it today than tomorrow.

Another reason for having this positive time preference emerges from the institutional arrangements: if I have the option of earning some interest (say Rs. Y) on Rs. 1 by putting it in a bank account today, why should I lend it to someone for free? Putting Rs. 1 in a bank account will make it “Rs. 1 + Rs. Y” for sure (assuming away bank insolvency), say, after one year while lending it to someone will leave it worth Rs. 1. Hence, Rs. Y (which may be expressed in percentage) is the price that should be paid to the lender for a loan of one year, else it would be unfair with him. This argument is more of economic than legal in its substance, however, some comments can be made here to evaluate its legal substance in the light of preceding discussion.

The relevant part of the proposed argument is the second one (the institutional arrangement) because the first one is merely a subjective feeling, which may differ from person to person (as a matter of fact, not everyone prefers to consume more today than tomorrow). The argument presumes that there exists and should exist a well-established legally functional market for loan, which coordinates interest-based loan transactions. But just recall “I.1” that Islam does not approve of the market for loan to begin with. Eliminate this institution of market for loan, and the argument disappears. The point is that the concept of “time value of money” conceived in this economic sense is alien to the discussion of riba. Its validity presumes that there exists a legal institutional market for loanable funds where money is growing continually and, therefore, an individual always has the option of putting his money in that market.

Not only that this assumption is invalid from the point of general rules of shari‘ah as explained, it is also in contradiction with the ontological structure of the universe and economic facts.

The above is not the only format of this argument, it is phrased in some other shades as well. For example, it is stated that money could buy benefits and had the lender not lent it he could have benefitted himself. This implies that lending is an act of sacrificing the benefits associated with money. Therefore, the lender should be compensated for this sacrifice and interest payment is exactly that reward. This reward makes sense given that the borrower takes benefit out of money. The argument is valid to the point that money is beneficial to the lender and that if he makes the choice of not lending it, he can benefit from it. Moreover, it is also true that the borrower enjoys the benefits associated with the money. None of these facts is denied by the shari‘ah rules. But these facts alone cannot formulate the required case for this argument; it requires a moral statement in its premise to derive the desired conclusion.

To see this, note that the argument does not end here, after quoting these facts it then makes a moral assertion: “it is morally (and hence legally) right if money is lent for reciprocal benefits.” Addition of this moral statement is necessary for validating the conclusion that “interest is the just reward for lending.” But this moral assumption contradicts the general rules of the shari‘ah, which are laid down above. Seeking reciprocity in loan is exactly what that changes its status from tabarru‘ to loan as a business transaction and, hence, it becomes nothing but riba. The argument here is quite straightforward:

The owner of money is granted the right of benefitting from his money by shari‘ah rules; he is given the option of making a conscious choice of transferring the benefits associated with his money to another person as an exception to the general rules by the shari‘ah, but there is neither any general rule nor any exemption from the Lawgiver that assigns him the right of lending money in the name of the so-called “mutual benefits” (refer to I. 4 above). Legally speaking, this involves both riba 'l-fadl (because the homogeneous goods are exchanged at different rates) and riba 'l-nasi'ah (because time stipulation is invoked-the lender asks for the excess of measurement for parting with the benefits of his money for a specified time).

Another variant of this argument comes with the heading of “effects of inflation on money.” We deal with it in the next section.

4.3. General Rules of Riba when Transacted Species are Different

What about the exchange of heterogeneous goods? The last words of the hadith are as follows: “If these species differ, then exchange as you like as long as it is from hands to hand.”

They give an immediate rule:

R 1.3) Goods of the different species can be exchanged with difference in measurement.

This rule says that such goods can be exchanged at different rates as far as measurement is concerned. In other words, riba 'l-fadl does not apply in case of heterogeneous goods. Is riba 'l-nasi'ah (prohibition of time delay in payment) also not applicable in this case? Apparently, it seems that it is not because of the words of hadith, “exchange should be on spot.” This has an odd implication that credit sale (sale of goods against money where payment is deferred to future time period) is not permissible under the shari‘ah rules. This is so because credit-sale is an exchange of heterogeneous goods with time lag. But the legal facts reveal that the Lawgiver has allowed credit-sale.41 How to explain this? Is credit sale also an exemption to the general rule, like a loan transaction? The answer is: “No, it falls within the general rules.”

To see how credit-sale is permissible within general rules, one needs to dig deep into the issue of the underlying cause (‘illah) that the Muslim jurists derived from the sunnah to understand the system of riba. The relevant question facing jurists was: Is prohibition of riba restricted only to the six goods named in the hadith or is it extendible to other goods? The answer of the jurists is, yes, it is extendible and for this extension they derived the underlying cause due to which riba was declared prohibited by the Lawgiver. Keeping aside the technical details and arguments, it should be noted that some of the goods are measured in terms of weight while others are measured in terms of volume. In the hadith under discussion, gold and silver were weighable while the other four items were volumeable at the time of Prophet (peace be on him).42 Based on this classification, the jurists derived two further rules:

R 1.4) when species are different but their method of estimation is the same (such as gold vs silver or wheat vs rice), unequal quantities can be exchanged, provided that the exchange is immediate;

R 1.5) when species are different and their method of estimation is also different (such as gold vs wheat), unequal quantities can be exchanged with time delay.43

Thus, the credit sale is allowed due to the application of Rule 1.5. To see this, consider these combinations of transactions:

1) Exchange of 1 gram gold at spot for 2 gram silver on spot (method of estimation same)

2) Exchange of 1 gram gold at spot for 2 gram silver in future (method of estimation same)

3) Exchange of 2 kg wheat at spot for 1 gram gold/silver on spot (method of estimation different)

4) Exchange of 2 kg wheat at spot for 1 gram gold/silver in future (method of estimation different)

The first transaction is allowed but the second is not because when species are measured by same method (i.e., “weight” in this case), then difference in the measurement (fadl) is allowed but deferment (nasi'ah) is not permissible. The third and the fourth transactions are allowed because here not only the transacted species are different but also their method of measurement (one was measured in “weight” while the other in “volume”).

In short, when both of the similarity factors (i.e., species and method of measurement) are found, then both fadl (excess of measurement) as well as nasi'ah (excess of time delay or time deferment) are prohibited. When similarity of measurement is found alone, then fadl is allowed but nasi'ah is prohibited. Finally, when none is found, both fadl and nasi'ah are allowed. Figure 7 depicts all of these rules completely (discussion about the last layer of boxes on the right-hand side of this figure is coming next).

The preceding discussion shows that the hadith explaining the nature of riba was not about the actual practices of Arabs that begged some economic explanations with which Muslim scholars have been struggling. Rather, it stipulated the rules of exchange. It says, “If at all you make exchange transactions, here are the governing rules.” Thus, all transactions that correspond to these general rules are allowed while those in contradiction with them are prohibited (however, some are exempted by the Lawgiver).

4.3.1. Implications

Following implications are derived from the above rules. It is important to note that the first two transactions mentioned in sub-section 4.2 belong to the case when method of estimation of the heterogeneous goods is same while the latter two cover the cases when their method of estimation is different.

I. 6) Placement of Regular and Credit-Sale

Transaction (3) is categorised as regular sale transaction (usually termed bay‘) by the jurists. On the other hand, transaction (4) covers credit sale, which may take two forms: with or without extra profit margin as compared to the spot sale. Because both measurement as well as payment time differential are allowed in this case, hence credit sale of both forms is allowed.

I. 7) Placement of Currency Exchange

The remaining two boxes are relating to the exchange of currencies (termed as bay‘ al-sarf by the jurists). A detailed description of these requires an appreciation of some more technical classifications44 made by the Muslim jurists. However, they are beyond the scope of this paper. Suffice to say that the jurists divided all tradeable species into two: (a) currency items, which are used as means of exchange; they included gold and silver (though other goods may also be treated as currency in this system) and (b) non-currency items, (goods that are exchanged, and are not medium of exchange). They roughly included all but gold and silver.45 Given this division, the jurists broadly mention four types of transactions (buyu‘):

(1) Non-currency item in exchange of non-currency item-called barter exchange.

(2) Spot or delayed currency (say gold) in exchange of spot non-currency (say wheat) item.

(a) If both of them (gold and wheat) are exchanged on spot, it is called regular sale of goods, and

(b) if the currency price (gold) is delayed, this is called credit sale.

(3) Delayed non-currency item (say rice) in exchange of spot currency item (say gold). Here, the price of the good is paid on spot while its delivery is delayed. This is called bay‘ al-salam (advance payment) by the jurists.

(4) One currency (gold) in exchange of another currency (silver)-known as bay‘ al-sarf.

Rules regarding the first two have been discussed above. Here, we have to make some submissions regarding this fourth type of transaction. Because this transaction comes under the umbrella of “different species with same method of measurement,” it is clear from figure 7 that the excess of measurement is allowed in this transaction while time deferment is not. This gives two further rules under rule (1.4):

1.4a) If different currency items (such as gold and silver) are exchanged, then it is allowed to exchange them at any rate;

1.4b) if different currency items (such as gold and silver) are exchanged, then it is not allowed to exchange them with time deferment.

If it is accepted that modern currencies are just substitutes of gold and silver, then two further important results emerge from this discussion:

I 7.1) Future Currency Contracts are Prohibited

Rules (1.4a) and (1.4b) imply that the spot currency transactions are allowed while their future contracts (known as currency salam in Islamic finance literature) are prohibited in Islam as they come under the purview of riba 'l-nasi'ah.

I 7.2) Indexing of Loans is Prohibited

Indexing the value of the currency loans against some underlying assets (say gold) on the ground of inflationary pressures is not allowed. It is often argued that since the value of currency decreases over time due to the presence of inflation, hence an extra-payment equal to the rate of inflation, over and above the original sum given in loan, should be allowed in favour of the lender to keep his purchasing power. Again, because the economic merit of this argument is beyond the scope of discussion in this paper, we restrict only to its legal merit. If it is accepted that one rupee is legally nothing but equivalent of 1 unit of gold or silver (whatever that unit be), then Rules 1.1 and 1.2 (governing the loaning contract in gold or silver currencies) should automatically become operational.

Those rules imply that (a) loaning in the form of currency item is allowed if and only if equal measurement (whatever the unit of measurement) is returned; else it would be riba 'l-fadl; and (b) it is a loan made out of benevolence and not business intention (having time stipulation); else it would be riba 'l-nasi'ah. Hence, adding an extra amount to loan transaction in the name of “indexation” is but both, riba 'l-fadl (because of the excess of measurement) and riba 'l-nasi'ah (because the increase is time bound). 46 Again, let simplicity and sanity prevail.

I. 8) Placement of Salam

To see how the jurists accommodated salam in this scheme, note that there is nothing in the set of rules 1 (from 1.1. to 1.5) which forbids it. However, according to rule 2 (given at the start of this section), selling what one does not possess is not permissible and this is exactly what a salam transaction involves. Thus, a salam transaction should not be allowed as per the general rules of shari‘ah. We are once again faced with the same issue: salam is permitted in the “legal facts;” how and where to place it in the legal skeleton of the shari‘ah? Is there another general rule, which governs its permission as we saw in case of credit sale or is it an exemption from the general rule just like loan? The jurists' answer is the following: Salam is permitted as rukhsah-exemption from the general rules-by the Lawgiver.47

Because it is an exception, as per rule 3, it would be allowed only as “one of its kind” (sui generis) and cannot be used as justificatory mode for deriving more comparable transaction forms (e.g., currency salam). An exception to the general rule remains exception and does not turn into a rule for other cases because then it ceases to be an exception and creates a situation of self-contradictory general rules, which is not acceptable in any legal system. Thus, salam transaction is allowed as an exception for those transactions where (a) a currency item is exchanged against a non-currency item and (b) non-currency item is deferred while the currency-item has been paid at spot.48 This is what the exception is all about; one cannot extend this exception to the transaction types where currency items are exchanged with each other because that would violate condition (a) of the exception case.49

Figure 8 shows a map of interplay among legal facts (nusus), general rules, exemptions, and the derived implications related to riba and bay‘ that are discussed in this paper. This diagram shows that a rather complex looking system of ahkam (implications) showing up at the ending layer boxes of figure 8 emerge out of a set of general rules, which are derived to make underlying legal facts compatible with each other.

5. Conclusion: The Definition of Riba

We conclude this paper by elaborating a comprehensive definition of riba that can be inferred from the discussions in this paper. Let's quote it from al-Sarakhsi:50
\begin{Synthesis}
Riba in its literal meaning is excess... and in the technical sense (in the shari‘ah), riba is the stipulated excess without a counter-value in bay‘ (sale).51
\end{Synthesis}


Let's explain it noting several points about this definition:

(1) Muslim jurists do not introduce the word loan in the definition of riba because they categorise loan transaction under exchange (bay‘). Not appreciating this point resulted in the misconception that since the fiqh conception of riba does not deal with the subject of bank loans, it needs to be inferred directly from the Qur'an.

(2) Riba is excess, either in the form of quantity (qadr) or in the form of benefits of delay (nasa'). The first is called riba 'l-fadl while the latter is called riba 'l-nasi'ah.

(3) This excess is without any counter-value permitted by the shari‘ah. Thus, the excess of quantity paid in lieu of time delay in case of interest-bearing loan is not allowed because these two cannot be the legitimate counter-values (see I. 4).52 For a substance to be counted as counter-value, it must be recognised by the general rules of the shari‘ah to begin with.53

(4) The excess is stipulated in exchange. If the excess is granted voluntarily, it would not be riba.

We started off with specific questions in the introduction. The appendix lists down the answers to these questions in the light of the above definition of riba. It can be seen that once the discussion about riba is placed on the right track, right and clear cut answers start emerging automatically.

Appendix: Questions and their Answers that Follow from the above Analysis

Notes

1 For detailed arguments of this position, see Abu 'l-A‘la Maududi, Sud (Lahore: Islamic Publications, 2000), 110-12; M. Umer Chapra, “The Nature of Riba in Islam,” Hamdard Islamicus 7, no. 1 (1984): 3-24; Muhammad Shafi‘, Mas'alah-i Sud (Karachi: Idarat al-Ma‘arif, 1996), 43-47; Muhammad Ayub, “What is Riba? A Rejoinder” Journal of Islamic Banking and Finance 13, no. 1 (1996): 7-24; Muhammad Taqi Usmani, The Historic Judgment on Interest Delivered in the Supreme Court of Pakistan (Karachi: Idarat al-Ma‘arif, 1999), 12-16; Mohammad Nejatullah Siddiqi, Riba, Bank Interest and the Rationale of Its Prohibition (Jeddah: Islamic Research and Training Institute, 2004), 45-48; and Mahmoud A. El-Gamal, Islamic Finance: Law, Economics, and Practice (Cambridge: Cambridge University Press, 2006), 46-52. Within this category, there are further two approaches.

One approach that represents traditional ‘ulama' emphasises the resurgence of only those business contracts that were approved by the early Muslim jurists. It proposes profit-and-loss sharing (PLS) as an ideal alternative to riba. Though it does not deny the permissibility of other than PLS-based financing instruments such as murabahah and ijarah, yet it affirms that equity-based financing method is the primary means of achieving desirable economic objectives. The second approach is pragmatic one. It justifies a more liberal and flexible stance on structuring shari‘ah-compatible transaction forms that looks for financial engineering to meet all demands of modern banking customer.

2 Muhammad Rashid Rida (d. 1935) was among the foremost proponents of this theory. See his al-Riba wa 'l-Mu‘amalat fi 'l-Islam (Cairo: Dar al-Manar, 2007). Also see Sayyid Yaqub Shah, “Islam and Productive Credit,” The Islamic Review 47, no. 3 (1959): 34-37; Fazlur Rahman, “Riba and Interest,” Islamic Studies 3, no. 1 (1964): 1-43; Timur Kuran, “On the Notion of Economic Justice in Contemporary Islamic Thought,” International Journal of Middle East Studies 21, no. 2 (1989): 171-91; Izzud-Din Pal, “Pakistan and the Question of Riba,” Middle Eastern Studies 30, no. 1 (1994): 64-78; and ‘Abd al-Karim Athari, Sud Kiya Hay? (Mandi Baha' al-Din: Anjuman-i Isha‘at-i Islam, 2008), 8-12 3 Constant J. Mews and Ibrahim Abraham, “Usury and Just Compensation: Religious and Financial Ethics in Historical Perspective,” Journal of Business Ethics 72, no. 1 (2007): 1-15.

4 See Ghulam Ahmad Parvaiz, Nizam-i Rububiyyat (Lahore: Idara-i Tulu‘-i Islam, 1978).

5 Rafi‘ Allah Shihab, Kirayah-i Makanat ki Shar‘i Haithiyyat (Lahore: Kitab Ghar, 1981).

6 Ziaul Haque, “The Nature and Significance of the Midieval and Modern Interpretations of Riba,” The Pakistan Development Review 32, no. 4 (1993): 933-46.

7 For details, see Imran Ahsan Khan Nyazee, Theories of Islamic Law: The Methodology of Ijtihad (Islamabad: Islamic Research Institute, 1994), 9-12.

8 Nyazee, The Concept of Riba and Islamic Banking (Islamabad: Institute of Advanced Legal Studies, 1995), 11-19. Imran Ahsan Khan Nyazee (b. 1945) is a well-known scholar and a prolific writer on the subject of Islamic law and is a former Professor of law in International Islamic University, Islamabad. His major works include Theories of Islamic Law; Islamic Jurisprudence; Islamic Law of Business Organization; and The Concept of Riba and Islamic Banking. He also translated some of the classical texts on Islamic law and jurisprudence, including: Hidayah of Marghinani; Bidayat al-Mujtahid of Ibn Rushd; Amwal of Abu ‘Ubayd; and first two volumes of Muwafaqat of Shatibi.

9 Rida, al-Riba wa 'l-Mu‘amalat fi 'l-Islam, 69ff.

10 Period before the advent of the Prophet (peace be on him) is referred to as jahiliyyah (i.e., the period of uncivilised state of affairs).

11 Fazlur Rahman, “Riba and Interest,” 7-8.

12 In this regard, a hadith reads, “The Prophet said, ‘While exchanging gold for gold, silver for silver, wheat for wheat, barley for barley, dates for dates, and salt for salt, exchange like for like, in equal measure, and exchange from hand to hand. If these species differ, then sell as you like as long as it is from hand to hand.'” Muslim b. al-Hajjaj, Sahih, Kitab al-musaqah, Bab al-sarf wa bay‘ al-dhahab bi 'l-wariq naqdan.

13 Adopted from Nyazee, Concept of Riba.

14 Maududi, Sud, 118-19.

15 Chapra, “Nature of Riba in Islam,” 3.

16 Siddiqi, Riba, Bank Interest and the Rationale of Its Prohibition, 49-50.

17 Adopted from Nyazee, Concept of Riba.

18 Rida, al-Riba wa 'l-Mu‘amalat fi 'l-Islam, 69-70.

19 Qur'an 2:275.

20 Ibid., 3:130.

21 Ibid., 2:278-79.

22 Ibid., 2:275.

23 For details, see Shafi‘, Mas'alah-i Sud, 106-120 and Siddiqi, Riba, Bank Interest and the Rationale of Its Prohibition, 38-40.

24 Nyazee, Concept of Riba, 35-36.

25 Mujmal is a term used by Muslim jurists to refer to a Qur'anic term that begs its explanation through the words of Lawgiver (i.e., God and His Prophet [peace be on him]). One cannot interpret mujmal either by looking its meaning in the dictionary nor can its meaning be determined through historical practices at the time of revelation of the Qur'an. Mujmal can be elaborated only by the Lawgiver. Another example of mujmal is the Qur'anic term salah (prayer) which cannot be interpreted literally.

26 Bay‘ means exchange of counter values, and is not restricted to sale of goods/services. Abu Bakr b. Mas‘ud al-Kasani (d. 587/1191), the illustrious Hanafi jurist, defines bay‘ as “exchange of property with property” and then elaborates that the concept includes not only ordinary sale but also barter, exchange of currencies, advance payment and many other forms of exchange. Bada'i‘ al-Sana'i‘ fi Tartib al-Shara'i‘, ed. ‘Ali al-Mu‘awwad and ‘Adil ‘Abd al-Mawjud (Beirut: Dar al-Kutub al-‘Ilmiyyah, 1997), 6:532-33. Abu 'l-Hasan ‘Ali b. Abi Bakr al-Marghinani (d. 593/1197), author of the authoritative Hanafi manual al-Hidayah, also explicitly asserts that qard (loan) begins as an act of charity but becomes an exchange transaction in the end. al-Hidayah fi Sharh Bidayat al-Mubtadi (Beirut: Dar Ihya' al-Turath al-‘Arabi, n.d.), 3:60

27 See Nyazee, Concept of Riba, 45-46.

28 Ibid., 49.

29 The Hanafis use the methodology of istihsan (juristic preference) for ensuring harmony and analytical consistency within the law when general rules and legal facts seem to contradict. If something appears prohibited in the light of the general principles of law, but has been explicitly permitted by one of the texts (i.e., legal facts), the Hanafi jurists take the position that it is permissible as an exception to the general principle. They use the rule, “prohibited under qiyas but permissible under istihsan” for this purpose. Exceptions to the general principles are made on the basis of the text, consensus, necessity or some other “covered principle” (qiyas khafi), which needs to be uncovered. Muhammad b. Abi Sahl al-Sarakhsi is worth quoting here: “This [istihsan] is the evidence coming in conflict with that apparent principle (qiyas zahiri), which comes into view without one's having looked deep into the matter.

Upon a closer inspection of the rule and the resembling principles, it becomes clear that the evidence that is conflicting with this apparent principle is stronger and it is obligatory to follow it. The one who chooses the stronger of the two evidences cannot be said to be following his own personal caprices.” Muhammad b. Abi Sahl al-Sarakhsi, Tamhid al-Fusul fi 'l-Usul, ed. Abu 'l-Wafa' al-Afghani (Beirut: Dar al-Kutub al-‘Ilmiyyah, 1993), 2:200-202. Another important point made by al-Sarakhsi is that when the jurist uses istihsan and prefers the stronger rule, he abandons the weaker one and as such it is not permissible for him or his followers to follow the latter. He goes on explaining that when istihsan is carried out on the basis of a concealed or covered principle (qiyas khafi), the established rule does not amount to be an exception but becomes a general principle in itself.

Interestingly, not only the Hanafi jurists but also the Maliki jurists explicitly employ the principle of istihsan for resolving the apparent anomaly found in the legal facts where one set of nusus prohibits a loan transaction and another set of nusus allows it. They hold that it is prohibited as an exchange transaction but allowed as an act of charity.

30 Al-Sarakhsi interprets this verse as the following: “Trade is of two kinds: permitted (halal), which is called bay‘ in the law; and prohibited (haram), which is called riba. Both are types of trade. Allah informs us, through the denial of the disbelievers, about the rational difference between sale (bay‘) and riba, and says, ‘That is because they said, “Sale is like riba.”' He, then, distinguishes between prohibition and permission by saying, ‘And Allah has permitted sale and prohibited riba.' Through this, we came to know that each one of these is trade, but only one form is permitted.” Al-Sarakhsi, al-Mabsut, ed. Hasan Isma‘il al-Shafi‘i (Beirut: Dar al-Kutub al-‘Ilmiyyah, 1997), 12:1-2.

31 The famous Hanafi jurist Abu Bakr al-Jassas al-Razi (d. 370/980) says, “In the law (shari‘ah), it (riba) is applied to meanings in which it was not used in the language. This is indicated by the fact that the Prophet (peace be on him) termed nasa' as riba in the tradition of Usamah b. Zayd (God be pleased with him). He said, ‘Verily, riba is in nasi'ah.' ‘Umar b. al-Khattab, (God be pleased with him) said that riba had different forms and out of these salam in teeth, that is, in animals, is not concealed. ‘Umar also said that the verse of riba was one of the last to be revealed, and the Prophet (peace be on him) was taken away before he could elaborate the details for us. Therefore, give up riba and the suspicion of riba. It is established from this that riba became a technical term, for had it been governed by its original meaning in the language, it would not have been obscure for ‘Umar, who was fully aware of the names used in the language, being a native speaker.

This (the conversion of the word into a technical meaning) is also indicated by the fact that the Arabs were not aware of the sale of gold for gold and silver for silver with a delay (nasa') as riba, but this is riba in the technical meaning. If this (meaning of riba) is as we have explained it, then, it became like all the other unelaborated (mujmal) words that are in need of an elaboration (bayan). These are terms that have been transferred from the language to the law and assigned meanings to which the word was not originally applied in the language, like salah, sawm, and zakah. Such words are in need of a bayan and it is not proper to employ them in legal reasoning for the prohibition of any of the contracts, unless an evidence has been adduced to show that such a meaning is employed by the law.

The Prophet (peace be on him) elaborated on many occasions the intention of Allah in a verse, by way of an explicit statement or in response to a query (tawqif), and through these he has indicated the evidence (dalil). The (legal) meanings are, therefore, not lost to those who have knowledge when they employ legal reasoning.... In the technical sense, the word riba is assigned several meanings. The first is the one that was prevalent among the people of the jahiliyyah. The second is excess in the same species out of things measured and weighed, according to the view expressed by our (Hanafi) jurists.... The third is nasa' (delay), which is of several types.” Ahmad b. ‘Ali al-Razi al-Jassas, ed. Muhammad al-Sadiq al-Qamhawi, Ahkam al Qur'an (Beirut: Dar Ihya' al-Turath al-‘Arabi, 1992), 2:183-84. Al-Sarakhsi is also worth quoting here:

“Mujmal is the word the meaning of which is not understandable except by asking the one who used this word.... An example of mujmal is the saying of the Almighty: “He prohibited riba” as riba literally means excess but we know that this is not meant here because sale has been permitted for the purpose of excess. Rather, riba here means prohibition of a sale due to an excess without a counter-value stipulated in the contract; and this excess is either in the form of increase in measure or by way of delay.... It is obvious that this elaboration is not known by literal analysis. Rather, it needs a separate source. Hence, it is mujmal with respect to its intended meaning. The same is the case of salah and zakah. They are also mujmal because their original literal meaning is prayer and growth, but because of their use in specific legal acts, their intended meaning cannot be gathered from their literal analysis.” al-Sarakhsi, Tamhid al-Fusul fi 'l-Usul, 1:168-69.

32 See al-Jassas, Ahkam al-Qur'an, 2:183-84.

33 Muslim, Sahih, Kitab al-buyu‘, Bab bay‘ al-ta‘am bi Mithlih.

34 One can simply substitute “Rs.” for “gram gold” in these transactions if Rs. (currency) is treated as substitute of gold and silver currency.

35 The famous Hanafi manual Hidayah explains the position of a loan transaction in the following words: “It is an act of charity in the beginning and that is why it is not valid from a person who does not have the capacity to do charity, such as a minor or a guardian (of a minor). However, at the end, it becomes a contract of exchange because it turns into exchange of dirhams with dirhams with delay, and that is riba.” See al-Marghinani, 3:60. The commentators explain, “This necessitates invalidity of loan but the shari‘ah has recommended it and the whole ummah agrees on its validity; hence, we hold that it is valid but not binding (and can be terminated at will by any party).” Akmal al-Din Muhammad b. Mahmud al-Babarti, al-‘Inayah sharh ‘ala al-hidayah (Bulaq: al-Matba‘ah al-Kubra al-Amiriyyah, 1316 AH), 5:273. The same position is upheld by Maliki jurists.

Thus, the famous Andalusian Maliki jurist Abu Ishaq al-Shatibi (d. 790/1388) says, “There are many examples of istihsan in the law, such as loan, which is riba in reality because it is exchange of dirham with dirham with delay; but it has been permitted because it benefits and facilitates the needy.” Ibrahim b. Musa al-Shatibi, al-Muwafaqat fi Usul al-Shari‘ah, ed. Abu ‘Ubaydah Mashhur b. Hasan (al-Khobar: Dar Ibn ‘Affan, 1997), 5:194-95.

36 Jurists apply the rules of ‘ariyah (commodate-loan) on these transactions because it is the nearest match for qard and the only way to legally justify a qard transaction. Al-Kasani, 10:600.

37 In much the same way as time period cannot be stipulated in a contract of ‘ariyah because no one can be compelled to do or continue with an act of charity (tabarru‘). Al-Marghinani, al-Hidayah, 3:60. In other words, making the condition of time-period binding changes the nature of the transaction and it no longer remains tabarru‘.

38 This has some income distributional as well as social consequences.

39 For an analysis of the idea of “debt as a social construct” and the transformation of this social construct to the impersonal market form, see David Graeber, Debt: The First 5,000 Years (New York, NY: Melville House Publishing, 2011, 308-60.

40 This false dichotomy is also not consistent with a number of “legal facts” (nusus). For example, in a hadith the Prophet (peace be on him), after explaining the rule of exchange among six goods, said, “Whosoever paid more or demanded more, indulged in riba.” Muslim, Sahih, Kitab al-musaqa, Bab al-sarf wa bay‘ al-dhahab bi 'l-wariq naqdan. Both are treated equally because both are the participants of “market for loan” which is not allowed.

41 The validity of credit-sale is inferred from many facts. These include the general permissibility of sale transactions such as the words of the Exalted, “Allah has permitted sale” (2:275). The jurists hold that all sales are permitted except those which have been prohibited specifically, such as sales involving uncertainty (gharar) or which stand prohibited by the operation of other principles of law, such as the prohibition of riba. The analysis in text explains that credit sale does not fall under the prohibition of riba.

42 This is the Hanafi position. The other schools classify these six items in different ways, but interestingly all classify them into two categories. The below table summarises their positions:

School Position on Gold and Silver Position on other Four Items Hanafi weighable (mawzunat) volumeable (makilt) Hanbali weighable (mawzunat) volumeable (makilt) and countable (ma'dudt) Shfi'i currency (thaman) edibles (mat'umat) Maliki currency (thaman) storable edible items (mat'umat)

The net result is that all the four schools agree on the applicability of the rules of riba on gold and silver (though for different reasons) and they come up with the impermissibility of loan transaction. For the Hanafis, they are also applicable on all items that are either weighed or volumeable (whether they are food items or not, does not matter); the Hanbalis agree with the Hanafis but add a third category of the counted items; for the Shafi‘is, the rules of riba are applicable on food items (whether they are weighed, measured or counted does not matter); the Malikis agree with the Shafi‘is but add a proviso that these food items must be such that people generally prefer to store them. These differences have interesting implications for extending the rules of riba to cases other than the six items specifically mentioned in the traditions. For details, see Nyazee, Concept of Riba, 83-88.

43 Interestingly, although the four schools have determined different ‘illah (cause) for the operation of riba on gold and silver, yet a loan transaction even if interest-free remains prohibited for all the four schools. Thus, for the Hanafis and the Hanbalis gold and silver must be exchanged on spot because they are weighable items, the Malikis and the Shafi‘is deem it necessary because gold and silver are currency items. Resultantly, despite disagreement on the ‘illah of riba, all the four schools agree that a loan transaction is prohibited as an exchange transaction and permitted only as an act of charity.

44 These include the terms ‘ayn, dayn, and thaman. For an elaboration of the meaning of ‘ayn and dayn, see Nyazee, Concept of Riba, 54-57.

45 The jurists treat gold and silver as thaman (price/currency) in exchange with all other items. Even when they are exchanged with each other (as in the contract of sarf), both of them are treated as thaman. That is why they are called thaman mutlaq (absolute thaman). Fungible items (mithliyyat) are deemed thaman if they are exchanged with non-fungible items (qimiyyat). When a fungible item is exchanged with another fungible item, such as when wheat is exchanged with barley, the parties are at liberty to consider any one of them as thaman but they have to specify it in the contract. For details, see al-Kasani, Bada'i‘ al-Sana'i‘, 7:216-17.

46 The last two implications are based on the widely accepted assumption that modern currencies are just like gold and silver currencies and should be treated as their substitutes. See Ghulam Rasul Sa‘idi, Sharh Sahih Muslim (Lahore: Farid Book Stall, 1998), 4:350-361; and Muhammad Taqi Usmani, Islam aur Jadid Ma‘ishat-o Tijarat (Karachi: Ma‘arif-i Islami, 1999). Changing this assumption can change the implications. The alternative to this view is to accept that modern money is a promise of payment, which implies that it is an acknowledgement of debt. In that case, Islamic rules of hawalah (endorsement) transaction will be applicable. Accepting this position can allow the indexation of loans since money is now treaded as value of something which it promises and, therefore, a loan can be linked to the underlying promised asset.

However, accepting the premise that “modern money is debt” leads to the result that exchange of currencies is not allowed even on spot because of another general rule of the shari‘ah, namely, “prohibition of exchanging debt for debt” (bay‘ al-kali' bi 'l-kali'). The prohibition is reported in by many scholars of hadith. For instance, see ‘Ali b. ‘Umar al-Daraqutni, Sunan, Kitab al-buyu‘, Bab nahy ‘an bay‘ al-kali' bi 'l-kali'; Muhammad b. ‘Abd Allah al-Hakim, al-Mustadrak ‘ala 'l-Sahihayn, Kitab al-buyu‘, Bab nahy ‘an bay‘ al-kali' bi 'l-kali'. Thus, one cannot maintain both of these positions simultaneously; either he has to allow indexation of loans or he has to allow exchange of currencies. For details, see Nyazee, Concept of Riba, 96-114.

47 The jurists cite traditions of various Companions who report that the Prophet (peace be on him) prohibited them from selling what they did not possess but gave exemption for salam. Al-Kasani, Bada'i‘ al-Sana'i‘, 7:101-02.

48 Some other conditions are also applicable for the validity of this transaction but they do not relate to our subject matter here. For their details, see al-Kasani, Bada'i‘ al-Sana'i‘, 7:103ff.

49 A misconception prevails regarding the nature of riba due to a tradition, “there is no riba except in nasi'ah (deferred payment transactions).” These words of Ibn ‘Abbas constitute reason that can explain the adoption of wrong methodology by the contemporary Muslim scholars. It is inferred from this tradition that the primary form of riba deals with loan transaction, which is riba 'l-Qur'an. However, several points invalidate this inference as indicated by al-Sarakhsi. See al-Sarakhsi, al-Mabsut, 12:11-12. First, the words of the hadith are quoted from Ibn ‘Abbas who initially had this opinion but he reverted from this position later on when the hadith of riba was brought to his knowledge by Abu Sa‘id al-Khudri. Second, the ahadith of riba are quoted by several Companions of the Prophet (peace be on him) through several sources. Therefore, they cannot be ignored out rightly in favour of this isolated narration.

Third, hence, it is necessary to place these words of Ibn ‘Abbas appropriately within the legal structure of the shari‘ah. Thus, al-Sarakhsi points that the words relate to the exchange of heterogeneous goods measured similarly, because in that case there is no riba except in deferment.

50 Apçna²'

51 Al-Sarakhsi, al-Mabsut, 12:109.

52 That is why, the definition of riba in al-Durr al-Mukhtar, a later Hanafi text, is given as follows: “Riba is an excess without any counter-value recognised by shari‘ah, in favour of one of the parties in a transaction.” Muhammad Amin b. ‘Abidin, Radd al-Muhtar ‘ala 'l-Durr al-Mukhtar Sharh Tanwir al-Absar, ed. ‘Adil Ahmad ‘Abd al-Mawjud and ‘Ali Muhammad Mu‘awwad (Riyadh: Dar ‘Alam al-Kutub, 2003), 7:398-401.

53 For example, if a female sells her body in exchange of mangoes, this would not be legitimate. Nor will it be legitimate if A lends Rs 1,000 to B on the condition that B will repay Rs 1,000 plus a swine.

Appendix: Questions and their Answers that Follow from the above Analysis No Questions Answers 1 Is bank interest prohibited in the light Yes, it is prohibited, because it is of the Qur'n and the sunnah? violation of rules 1.1 and 1.2. 2 Whether the Qur'nic term rib It includes all forms of interests. includes all kinds of interest rates or it This is a necessary implication of relates only to the excessive interest rules 1.1 and 1.2. rates? 3 Whether the scope of rib extends to It extends to all kinds of loans, the interest charged and paid on commercial or consumption, as business transactions in the banking shown by the application of rules 1.1 system or it is restricted to the interest and 1.2. charged on consumption loans only? 4 Does Islam allow loan transactions? If Loan is against the general rules of yes, how and in what form? Islam. However, it is permitted by the Lawgiver as an exemption to the rule if it takes the form of tabarru'. 5 Is paying interest a lesser evil as No, it is not. The assumed compared to charging interest? dichotomy is wrong, as it has been clarified by I.3. 6 Is borrower always malm (a losing No, the borrower can also be the party) in an interest-bearing loan receiver of rib as per rules 1.1 and transaction? 1.2 (see I.3). 7 Does Islam allow indexation of loans No, it does not. The demand for on the grounds of inflation? loan indexation is invalidated by rules 1.4a and 1.4b. 8 Is credit sale with higher deferred price Yes, it is validated by the application as compared to the spot price allowed? of rules 1.4 and 1.5. 9 Does Islam approve of "time value of No, it does not. In fact, the concept money," especially when charging is alien to the subject matter of rib, higher deferred price is allowed in a provided both the concept of time credit sale? value of money and rules of rib are used appropriately. 10 Are future currency contracts No, they are not. It is violation of permissible in Islam? rule 1.4b. 11 How and to what extent is salam Rule 2 implies that salam is against transaction permissible? the general rules of the shar'ah but allowed as an exemption by the Lawgiver, hence, should remain exemption as per rule 3.
 \end{quote}
 
 
 \paragraph{Chapitre 8. Islam et assurances}
 LES CAPITAUX DE L’ISLAM  | Gilbert Beaugé
 \href{https://books.openedition.org/editionscnrs/871?lang=fr}{ Islam et assurances}
\begin{quote}
    
L’assurance est-elle légale du point de vue de la Shari’a ? Ce débat aujourd’hui encore est très vif dans les pays musulmans\sn{1 En fait, quatre éléments ont permis de classer les contrats d’assurance parmi les contrats interdits par le clergé islamique : a) l’imprécision (gharar), b) le jeu de hasard (gimar), c) l’usure (riba) et d) l’échange de choses équivalentes (bai al sain bil-dain). Le travail le plus complet à ce sujet est celui de Muhammad Baltagi, Uqud al-ta’ min wigha al-figh al islami Koweit, 1982. K. Krüger donne une bonne vue d’ensemble du droit privé des États régis par le droit égyptien. Certains problèmes comme celui du riba sont analysés de façon plus détaillée, cf. Recht van de islam, 1987.}. La fonction sociale de l’assurance est tout particulièrement perçue par la population musulmane, dans la mesure où les catégories classiques du droit musulman intègrent cette valeur. Cette fonction suppose l’utilisation du principe d’assurance et, même dans un pays aussi conservateur que l’Arabie Saoudite, on ne rencontre aucune réserve à l’encontre de l’assurance sociale\sn{2 Introduit par décret royal, pp. 98 sv., n° 746 du 5.09.1969.}. En revanche, on continue à y critiquer l’assurance privée et, de temps à autre, cette critique réapparaît dans d’autres pays islamiques.

 
2 L’assurance peut se définir comme les précautions terrestres que l’on prend à l’encontre des coups du destin et des pertes matérielles, autant d’épreuves envoyées par Allah comme le sait tout musulman pieux. De ce point de vue, l’assurance peut apparaître comme le renoncement à la croyance en une prédestination, l’abandon de l’espérance en une miséricorde ou, pour le moins, comme leur limitation. A ces problèmes de nature religieuse ou morale qui, dans des périodes difficiles, comportent également un élément politique, s’ajoute, pour le musulman, la difficulté à concevoir la notion de risque et à la démarquer de celles de jeu ou de pari, afin de pouvoir en saisir la signification fonctionnelle, en rupture avec les contrats spéculatifs qu’interdit la Shari’a\sn{3 Cf. à ce sujet Al Amin Al Darir, Al Gharar wa-athauhu fil-uqud fil figh al islami.}. La discussion sur ce thème, qui se fonde sur des textes classiques de la Shari’a remontant au Moyen Age, s’est développée à la suite de l’implantation des compagnies d’assurance européennes dans l’empire ottoman, au cours de la deuxième moitié du xixe siècle, et des tentatives d’innovation des réformateurs de l’Islam, au début de ce siècle. Une opposition de type nationaliste et pan-islamique se manifesta alors à l’encontre de ces institutions capitalistes et occidentales, qui jouera un rôle non négligeable.
 
\subparagraph{HISTORIQUE DE LA NOTION D’ASSURANCE DANS LES PAYS DU MONDE ARABE}

3 La première réflexion approfondie sur la légalité de l’assurance en pays islamique tint compte de façon singulière des exigences musulmanes. Dans son ouvrage Radd al Muhttar, Ibn Abidin, représentant de l’école officielle de droit Hanafi dans l’empire ottoman, suggère en effet le compromis suivant : il serait licite d’établir des contrats d’assurance portant sur les risques encourus à l’intérieur du royaume islamique – le Dar al Islam – à condition que ces contrats soient conclus avec une compagnie d’assurance ayant son siège hors des pays de l’Islam, dans le pays des infidèles. La prise en charge du risque devrait se faire au siège de la compagnie\sn{4 Cf. l’étude complète de C.A. Nallino, « Belle assicurazioni in diritto musulmano banafita », in Oriente moderne, vol. VII, Rome, 1947. pp. 446 sv.}.

4 On estimait ainsi satisfaire suffisamment les besoins pratiques en assurances nécessités par le trafic international des marchandises, à une époque où Constantinople, Beyrouth et Alexandrie, ports par où transitait le coton, étaient les centres essentiels du commerce britannique du Levant, alors que le commerce français était davantage orienté vers les côtes d’Afrique du Nord.
 
5 Ce n’est que relativement tard, vers 1890, bien après la fondation de banques ou la création de filiales bancaires (1850), que fut créée une compagnie d’assurance à Alexandrie5. L’initiative en fut prise par trois Libanais qui reprirent la filiale d’une petite société anglaise s’occupant d’une affaire d’assurance-vie en Égypte, pour l’étendre, à partir de Beyrouth, à l’ensemble de la Syrie. Au début du siècle, au moment même où des sociétés françaises tentaient de s’établir en Afrique du Nord6, d’autres sociétés anglaises et françaises suivirent le mouvement, en Égypte puis dans les pays du Levant, tandis que la péninsulte arabique demeurait pratiquement à l’écart jusqu’au lendemain de la Seconde Guerre mondiale.

6 Le marché était peu porteur et étroit. Il s’agissait d’opérations d’assurance-vie, d’assurance-incendie et d’assurance-transport pour quelques biens importants. Ces assurances étaient souscrites par des banques étrangères ayant un intérêt particulier à le faire : par exemple, la Banque Ottomane travailla avec Eagle Star dans le domaine du crédit et de l’assurance hypothécaire. En raison de l’inflation qui suivit la Première Guerre mondiale, les polices d’assurance-vie perdirent leur valeur et il fut difficile de remplir de nouveaux portefeuilles dans un pays ruiné par la guerre. Les compagnies d’assurance ainsi que les courtiers s’orientèrent dès lors vers l’assurance des dommages et des transports à court terme. L’activité de ces entreprises était plus importante dans les pays sous influence britannique comme l’Égypte, le Soudan, la Palestine et l’Iran, alors que l’influence française dominait en Afrique de Nord ainsi que dans les pays du nord Levant, Syrie et Liban.


7 Si l’on prend l’exemple de l’Égypte, on constate qu’un système national d’assurance se mit en place, après la Première Guerre mondiale, avec la participation de trusts européens. Son évolution fut lente car le marché était étroit7 et il était nécessaire de former un personnel local. Le Conseil de surveillance introduit en 1939 maintint dans les limites du raisonnable une évolution qui, entre temps, s’était renforcée. Les pays sous influence française connurent également ce processus, sur les détails duquel nous reviendrons. Pour la Lybie, le marché italien eut une influence déterminante sur le système de l’assurance. Il connut une évolution moins marquée, mais parallèle.

 
8 La notion d’assurance privée s’imposa timidement comme instrument de la vie économique moderne dans les pays musulmans arabes. Les réserves de nature religieuse ne disparurent que progressivement : dans le secteur bancaire tout comme dans le secteur de l’assurance, les modernistes cherchèrent d’abord à obtenir des compromis. Ils tentèrent de prouver que les institutions et les pratiques que générait la vie moderne étaient compatibles avec les règlements de la doctrine islamique, mais ne tinrent pas suffisamment compte des réalités politiques et économiques8. Curieusement, le système d’assurance sociale, alors faiblement développé, fut tenu à l’écart de cette discussion. Il était, et reste toujours appréhendé, moins comme une assurance que comme une aide de l’État. Par ailleurs, jusqu’au tout début de la motorisation, les bases manquaient pour que soit mis en place un important volume d’affaires, aussi bien dans le domaine de l’assurance des personnes que dans celui de l’assurance des dommages. Dans les décennies qui suivirent la Seconde Guerre mondiale, les défenseurs du patrimoine islamique national s’opposèrent aux partisans d’une économie occidentalisée dirigée, ou tout au moins influencée, par des pays occidentaux. Sans se situer au cœur des débats, le système d’assurance fut tout de même soumis à examen critique : l’enjeu était d’établir s’il était ou non compatible avec les prescriptions de la Shari’a.
 
9 Ces discussions se développèrent à la fin des années cinquante, stimulées par un mouvement de création de compagnies nationales d’assurance et culminèrent lors d’une semaine de discussions à Damas\sn{Une telle manifestation avait eu lieu à Paris en 1951. Un des principaux orateurs, le professeur syrien Mustapha Ahmed al Zarga, démontra la compatibilité totale de l’assurance sous toutes ses formes avec la Shari’a, tandis que le professeur égyptien Abu Zahra n’envisageait que des associations mutualistes. L’ensemble des références figure dans un rapport complet de Zahra paru à Damas en 1962 : Aqd at-ta’min wa-mauqifal-shari’a al-islamiyya minhu.} : d’éminents juristes prirent position sur la question de l’assurance et l’on s’interrogea également sur la notion de risque et la manière de le définir et de le préciser dans un contexte économique. Le droit islamique, vers lequel il s’agissait de jeter des ponts, n’offrait sur ce terrain que peu de possibilités. Il fut très difficile, entre autre, de vaincre l’opposition unanime à tous les contrats sur les risques, assimilés aux jeux et aux paris\textsuperscript{10}. L’idée moderne d’une communauté organisée pour faire face à son destin\textsuperscript{11} – dont les bateaux et caravanes pouvaient fournir un exemple – et au sein de laquelle chacun participe de manière appropriée à la réparation des dommages, ne figure pas dans le droit islamique. 

\newpage
Sous cette forme, l’idée de risque demeure étrangère et incertaine\sn{ Le terme général est gharar, l’incertitude. \begin{Def}[gharar]
Gharar al-uqud signifie contrat à risque. Ils ont un rôle dogmatique. On range aussi sous cette dénomination les contrats portant sur des prestations que l’on peut préciser, mais qui ne l’ont pas été lors de la signature.\end{Def} Cf. art. « gharar », ainsi que le commentaire de l’article 484f « äegyptisches Zivilgesetzbuch von Sanhuri » (cf. supra, note 14). voir \href{https://fr.wikipedia.org/wiki/Abd_el-Razz\%C3\%A2q_el-Sanhour\%C3\%AE}{Sanhouri et code civil Egyptien}} au musulman, qui aimerait voir codifier non seulement ses devoirs religieux, mais également ses relations vis-à-vis du monde qui l’entoure. La seule chose sensiblement comparable à une assurance personnelle est le financement d’une rente à vie, sorte de rente viagère (umra) prélevée sur les bénéfices retirés de la terre ou de tout autre bien. Toutefois, une telle institution a suscité de nombreuses réserves de la part des érudits musulmans, dans la mesure où la durée de vie du destinataire restait incertaine13.

LES TENDANCES ACTUELLES
10 Actuellement, le problème des dommages est au cœur des discussions, sans qu’il y ait toutefois de différenciation structurelle entre l’assurance des personnes et celle des dommages. L’assurance trouve sa justification première dans l’idée mutualiste d’une aide réciproque contre ce qu’il est convenu d’appeler, de nos jours, les aléas de l’existence.


11 Dans son commentaire sur le nouveau Code civil égyptien, entré en vigueur en 1956, Sanhuri14, père spirituel de cet ouvrage de lois, s’interroge une fois de plus sur la légitimité des opérations d’assurance. Il se refuse à les justifier par analogie avec des contrats comportant des éléments identiques (caution et garantie), mais il fait du contrat d’assurance un contrat autonome de type nouveau. Dans la mesure où, dans le système du droit islamique, il n’existe pas de numerus clausus et étant donné que ses éléments constitutifs sont déjà reconnus, rien ne s’oppose à ce que l’on admette ce type de contrat. Quant à l’imprécision, motif d’annulation pour un tribunal suprême de la Shari’a en Égypte, il la justifie par la nécessité économique d’une telle institution et par la légitimité statistique. Il résulte de cet examen que l’assurance n’est pas une opération de type spéculatif. Pour ce qui est de l’interdiction du riba et de la perception d’un intérêt, il soutient la thèse que le riba Nasi, c’est-à-dire le report de paiement (crédit) devrait être autorisé dans la mesure où il est nécessaire. Quant à la deuxième forme du riba – l’échange injustifié de prestations différentes –, comme par exemple l’assurance-vie lors d’un décès prématuré après paiement de quelques primes seulement, il soutient qu’il s’agit d’une opération qui pourrait être autorisée pour des raisons d’intérêt social supérieur, si cela s’avérait nécessaire.
\end{quote}
\begin{Synthesis}
Assurance : au début, légal car pas depuis Dar al Islam : peut être repris ?
Ensuite, la principale question est l'imprecision juridique, mais elle ne peut être facilement réduite, même par tontine ou autre mécanismes mutualisant (d'ailleurs caravane pas bon non plus) ?
Sanhuri : c'est nouveau donc licite. Imprécision : corrigé par actuariat ! + raison sociale supérieure. A
\end{Synthesis}

\begin{quote}
    


12 L’argument mutualiste – ta awuni ou tabaduli\sn{What does \TArabe{ تبادلي }(tabaduli) mean in Arabic?

reciprocal : 96\% of use and mutual in 4\%} – fut l’argument essentiel avancé en faveur de l’autorisation de l’assurance et, en tout premier lieu, de l’assurance-dommage. Sur ce point, on peut parler d’un accord de tous les érudits musulmans : le système traditionnel de la société d’assurance mutuelle s’offre alors pour résoudre ces problèmes. Il apporte également une solution à la question des intérêts résultant d’une accumulation de capital dans la mesure où les fruits du capital sont répartis entre les adhérents. L’interdiction de l’usure perd ainsi, sur un plan moral, de sa force explosive. Certains docteurs surmontèrent les réticences à l’égard des contrats sur les risques en faisant prévaloir que l’élément dominant dans les sociétés d’assurance mutuelle est positif, puisqu’il s’agit d’un soutien dans la détresse et le malheur. Il serait donc convenable d’accepter une part d’incertitude, élément secondaire dans un contrat d’assurance par rapport aux notions d’aide et de soutien, et qui ne nuit pas à l’efficacité du contrat. En revanche, on tend à éliminer les autres formes d’assurance parce que la mentalité de profit, associée à la structure des sociétés par actions, vise à procurer des bénéfices aux actionnaires et ne saurait être justifiée par sa fonction d’entraide15.
 
 \begin{Synthesis}
 Il est intéressant de voir le même développement des mutuelles en France dans un cadre \textit{éthique}
 \end{Synthesis}
13 Après la Seconde Guerre mondiale, les marchés nationaux reçurent une impulsion décisive du développement rapide de la motorisation, occasion pour les compagnies de recettes provenant des primes obligatoires de responsabilité civile dans l’assurance des véhicules. Dans certains pays comme ceux de la péninsule arabique, l’industrie pétrolière, en pleine expansion, avait besoin d’assurer non seulement des investissements croissants sur les lieux de production mais aussi son environnement économique et social. On s’efforça donc de mettre en place une participation appropriée aux assurances pour les transports pétroliers. Quant aux compagnies aériennes nationales, elles s’assurent pour les risques inhérents au transport aérien auprès de compagnies nationales dans la mesure où celles-ci sont aptes à répondre à leur demande16.
 
14 En raison de la faible capacité caractéristique de tous les marchés arabes, la plus grande partie des risques sont couverts par une réassurance qui, il y a encore quelques dizaines d’années, était le domaine réservé des compagnies européennes. Depuis, les marchés nationaux arabes ont tenté, avec quelque succès, de mettre en place un marché commun arabe de la réassurance. Cependant, celui-ci ne dispose que d’une capacité limitée17. Ces efforts sont soutenus efficacement par l’Association des Assurances Arabes, fondée en 1963, dont les membres sont recrutés auprès des entreprises de l’ensemble des États arabes, depuis la Mauritanie jusqu’au Golfe d’Oman. Notons pour mémoire qu’un autre groupe important a été créé quelque temps après : il s’agit de l’assurance afro-asiatique, compagnie de réassurance dont le siège est au Caire. Par des colloques internationaux, mais également par des cours de formation pour cadres, on s’efforce désormais d’atteindre un haut niveau de technicité dans le domaine de l’assurance et de promouvoir une coopération dans le travail.
 
15 Cette coopération interétatique – expression du panarabisme formulée de manière la plus expresse au sein de la Ligue Arabe – avait été précédée par des aménagements internes sur les marchés arabes qui visaient à la mise en place d’une législation relative aux contrats, assortie d’un contrôle étatique. Cela fut le cas dans les États qui connaissaient une économie libérale, mais se réduisit à quelques fonctions de surveillance dans les pays où l’assurance était un monopole d’État18.

16 Depuis qu’en 1987 a été créée en Arabie Saoudite une compagnie nationale d’assurance19, il n’existe pratiquement plus un seul pays arabe qui n’ait son propre système d’assurance, fondé essentiellement sur l’assurance-véhicule. Par contre, l’assurance-vie n’a pas suivi l’évolution : les suites d’une souscription sont toujours remises en cause par l’inflation ; de plus, l’utilisation d’une devise forte n’est pas possible dans la mesure où – dans la plupart des pays – la loi prévoit expressément que les réserves constituées par les primes seront placées dans le pays20.

\begin{Synthesis}
Nécessité fait force de loi : l'assurance Auto en Arabie Saoudite
le développement du marché est comme en France lié à l'opacité du contrat (cf assurance vie et jeu au début)
\end{Synthesis}
17 Le mouvement qui milite en faveur d’une suppression de la clause d’intérêt sur les marchés arabes, parti d’Arabie Saoudite, a également gagné le terrain des assurances. Dans les pays qui interdisent l’intérêt, comme l’Arabie Saoudite, les difficultés proviennent du traitement des dépôts des réassureurs – même dans le cas de l’assurance-dommage – dans la mesure où ceux-ci n’entrent pas dans le cadre des exceptions pour capitaux étrangers. Dans le domaine de l’assurance-vie, une solution reste envisageable par la coopération avec un fond d’investissement islamique.

\begin{Synthesis}
Voir le hanbalo-wahhabisme et son influence et la gestion de l'assurance auto dans ce cadre
\end{Synthesis}

UN EXEMPLE D’ASSURANCE ISLAMIQUE
18 Un modèle intéressant a été développé par les banques islamiques qui travaillent sans prélever d’intérêt. Leur modèle principal pour le placement de leurs capitaux est la forme bien connue de la mudaraba de participation. De par sa structure, on peut la comparer à une société en participation, à un fond de placement où le déposant a une participation aux bénéfices proportionnelle au montant de son placement et où les pertes se font au détriment de celui-ci. Si on intègre la mudaraba dans le système d’une compagnie d’assurance, on obtient la combinaison suivante :

19 Il s’agit d’une caisse de décès ou, plus exactement, d’associations d’épargnants garantissant l’épargne, même en cas de décès prématuré. Les compagnies, filiales de la banque islamique, se nomment sharikat al-takaful al-islamiyya, leur slogan publicitaire est la participation réciproque garantie (mudaraba al tadamum), c’est-à-dire l’épargne garantie entre les musulmans. Les buts de la société sont présentés comme suit :

Organisation d’un soutien et d’une solidarité réciproques au sein de la communauté islamique.

Suppression dans la communauté musulmane de l’usure (riba), qui est un fléau ainsi que l’atteste le verset du Coran 2,278. « Fidèles ayez confiance en Dieu et laissez s’éloigner de vous ce qui demeure encore du riba, si vous êtes croyants. Si vous ne le faites pas, Dieu et le Prophète vous déclareront la guerre. »

20 Toute personne âgée de 20 à 50 ans peut devenir sociétaire de la compagnie. Chaque année, celle-ci verse une prime définie, dont la plus grande partie est placée à la banque islamique. Quant aux 20 \% restants, ils sont versés librement (!) par le sociétaire à un deuxième fond destiné à soutenir les familles de ceux que le destin a frappés avant qu’ils n’achèvent leur programme d’épargne. Quand l’affiliation cesse et en cas de survie, on verse au sociétaire la quote-part sur ses parts d’investissement ainsi que sa quote-part sur les bénéfices des parts fixes et, éventuellement, sur les bénéfices de ses parts au fond de soutien de l’assurance décès. Les deux fonds opèrent donc d’après le principe de la mudaraba, sans intérêts ni participation aux bénéfices des sommes investies21. Le plus grave problème posé par cette forme de placement est la garantie d’une liquidité suffisante nécessaire pour un règlement rapide des dommages importants. Les parts de l’assuré, tout comme les bénéfices réalisés par le fonds, sont soumis au Zakat ainsi qu’à l’impôt prélevé à la source (respectivement 2.5 et 5-10 \%).

21 Les banques islamiques opèrent dans les pays arabes depuis un bon nombre d’années. Sur le volume effectif d’affaires réalisé par une société observant rigoureusement les principes de la Shari’a, on connaît très peu de choses. On sait par contre que les difficultés dans le domaine de la réassurance internationale, mais également les questions de liquidité pour remboursement de dommages importants, n’ont pas été résolues de manière satisfaisante. Une sérieuse tentative dans ce sens a été la création d’une compagnie d’assurance saoudienne caractérisée par la garantie mutualiste et gérée en conformité avec les règles de la Shari’a. Son champ d’activité se limite à un secteur ne comprenant pas l’assurance-vie, dans la mesure où cet aspect est pris en charge par les filiales des banques islamiques. Seule l’expérience montrera si cette société peut, dans le cadre de ses missions internationales, ne pas se conformer aux exceptions d’usage faites pour la circulation des capitaux internationaux.
 

22 Mentionnons en conclusion que la création des instituts financiers islamiques correspondait également à l’intention de stimuler la circulation de l’argent et des capitaux dans la masse de la population. En tant que facteur économique, l’importance de l’assurance est cependant relativement faible. Le volume global des primes est estimé à moins de 1.5 \% du volume des primes dans le monde, alors que le nombre des entreprises est évalué à 10 \% des entreprises mondiales, pourcentage d’autant plus remarquable que de nombreux États ont nationalisé l’assurance ce qui réduit d’autant le nombre d’entreprises. La majeure partie des primes provient de l’assurance-automobile et de l’assurance du commerce extérieur, auxquelles il faut ajouter l’assurance de quelques grands projets industriels. L’absence d’une véritable critique de l’assurance de la part des milieux fondamentalistes est sans doute liée à son manque d’extension. La publicité pour les produits de l’assurance est faible : la notion de prévoyance n’ayant pas été encouragée chez l’individu, la part de l’assurance-vie dans le volume global des primes est largement inférieur à la moyenne. Ce serait cependant méconnaître les faits que d’en rendre l’islam responsable : l’assurance-automobile obligatoire, introduite par la plupart des pays, a été acceptée par la population, sans réserves et sans critiques. Dans quelques pays, mais surtout dans les pays de la péninsule arabique, le montant de l’assurance est pris en considération pour le paiement des indemnités : en Arabie Saoudite, il est de l’ordre de 60 000 RS et s’élève à 100 000 RS pendant la période de ramadan.

23 De nombreux signes permettent donc de penser que la notion d’assurance, à partir des bases actuelles, s’imposera progressivement. S’interroger sur l’étendue des réalisations, c’est se poser une question d’ordre économique. Les réserves de nature religieuse peuvent concerner l’application pratique de l’assurance, mais non son institutionnalisation en tant que phénomène.

\subparagraph{NOTES}
 



5 Certaines agences de sociétés britanniques travaillaient en Égypte dès les années 1850. Cependant, la poussée véritable se situe après l’occupation anglaise de 1882. Pour plus de détails, cf. Basim A. Paris, Insurance and Reinsurance in the arab world, Lon-don, 1983, pp. 43 et sv.

6 La première société à s’installer dans la région fut bien La Espanola, à partir de 1879, au Maroc. La première vague de création se situe après la Première Guerre mondiale, une deuxième suivra au début des années 60.

7 Après la Seconde Guerre mondiale, il n’y avait que cinq compagnies d’assurance en Égypte. En 1967, s’y ajouta une compagnie de réassurance et plus tard vinrent deux compagnies en joint-venture dans la « zone libre ».

8 Cf. à ce sujet I. Goldziher, Vorlesungen über den Islam, 2e éd, Heidelberg 1925, p. 259, ainsi que les Fetwas citées dans la revue moderniste islamique. Pour l’assurance vie, cf. Al Manar, Le Caire, vol. VI, p. 938, pour l’assurance-marchandise, vol. VIII, pp. 588 sq. Ainsi que vol. VI p. 717 et vol. X p. 330 et sv. Pour la question de l’intérêt et son autorisation dans les caisses d’épargne, cf. Ali Cikri et les conférences de Rida. Pour l’évolution d’ensemble actuelle, cf. E. Klingmüller, « Betrachtungen zur Reislamisie-rung im Recht », in FS H. Hübner, Cologne 1984, pp. 84 et sv.



10 Le muqamara et le rihan, déjà prohibés au début de l’islam, sont expréssement interdits par le Coran (Sourate 2,219 et 5,90). Les exégètes n’ont eu de cesse de déterminer quelles étaient les comportements de la vie quotidienne que pouvaient recouvrir ces concepts. Cf. sur ce point article « qimar », in L. Milliot, Introduction à l’étude du droit musulman, Paris, 1953, p. 653.

11 La protection lors des accidents de voyage (daman khatar al tariq) comme cas particulier d’une caution (mafala) utilisée pour la justification de l’assurance et, par voie analogique (giyas) l’institution du bai-am wafa qui ménage au vendeur une possibilité de rachat, l’acquéreur ayant dans l’intervalle l’usufruit du bien, mais acceptant le risque d’un dommage. Sur tout ceci, cf. Milliot, op. cit., p. 953.



13 Pour plus de détail, cf. Al Dharir, op. cit. p. 633, avec citations complètes de la littérature figh classique. Curieusement, y est mentionnée une prise de position hostile de la confrérie musulmane parue dans son propre journal, en 1941.

14 Abd al Razzag Ahmed Al Sanhuri, dans son commentaire « al-wasit » 1re éd, Le Caire 1964, vol. 7, p. 1088 sq. Cf. également son oeuvre systématique dans laquelle il tente de justifier le maintien des valeurs de l’islam et de trouver un compromis avec les exigences de la vie moderne, Masadir al-haqq fil-fighal-islami. 3e éd., Le Caire, 1967, vol. 3 pp. 32 et sv.

15 Cf. Isa Abduh, in Al-uqud al shari’a, Études pour le congrés sur le droit islamique (Riyadh novembre 1978), publié sur place. Par ailleurs, pour justifier certaines formes nouvelles de contrat, on faisait appel à la clause, déjà répandue au Moyen Age, consistant à se référer à l’utilité publique et à la nécessité d’une utilisation fondée sur l’intérêt général.

16 Lors du dernier congrés de la General Arab Insurance Federation (Damas, mai 1988), fut décidée la création d’un pool aérien ainsi qu’une collaboration plus étroite dans le domaine des risques lourds.

17 La franchise absolue dans une société arabe peut atteindre de 1 à 2 \% de la somme assurée. L’intervention d’une coassurance ou d’une réassurance dans le pays n’augmente pas la capacité du marché de la réassurance. La majeure partie des risques lourds est couverte par une réassurance facultative sur le marché international, ce qui n’est possible que parce que les réassurances apportent leurs connaissances techniques et leur expérience pour juger des différents risques. Cf. également Abdul Zahra Abdul-lah Ali, Insurance development in the arab world, London, 1985, chap. III. Dans ce chapitre sont publiés des chiffres qui ne se font pas encore l’écho de la guerre Iran-Irak.

18 Les marchés nationalisés sont l’Algérie, l’Irak, la Libye, la Mauritanie, la Somalie, la Syrie et le Sud Yémen, alors que les marchés du Maroc, de Tunisie, du Nord Yémen et du Soudan sont dominés par des sociétés contrôlées par l’État, mais fonctionnant comme des sociétés privées. Actuellement, on trouve un marché relativement libre en Égypte, au Qatar, dans les Émirats ainsi que dans le Sultanat d’Oman. Récemment a été fondée en Arabie Saoudite une société d’assurance-dommage soutenue par l’État.

19 Sur un plan formel, la société d’assurance est une société par action qui porte le nom de société nationale reposant sur la réciprocité (ta’min ta’awun). Dans ses statuts, il est expréssement mentionné qu’elle ne doit pas enfreindre les préceptes de l’islam et qu’elle doit tout particulièrement respecter l’interdiction du riba. De manière active ou passive, la société ne doit donc pas travailler avec intérêts. Ceci vaut aussi bien pour les mouvements de capitaux à l’intérieur du pays, que pour les transferts hors du pays (art. 4, alinéa 3 des statuts). Le capital action entièrement versé se monte à 500 millions de RS. Les statuts sont reproduits dans Umm al-Qurra du 12.04.1986.

20 Le clergé musulman a toujours quelques réticences quant à la création d’une société d’assurance-vie. Même dans le cas d’une compagnie assurant les biens matériels, il convient de tenir compte des préceptes du Coran, d’une part pour ce qui est de la structure en société mutuelle, d’autre part pour ce qui est du respect de l’interdiction de l’intérêt qui rend pratiquement impossibles les affaires de réassurance avec dépôts. Les placements de capitaux ne peuvent se faire qu’auprès des banques islamiques, des concessions au marché international des capitaux n’ont, à l’exception de la SAMA, jamais été faites jusqu’à ce jour.

21 Cf. à ce sujet H.P. Kindt, « Das islamische Versicherungswesen », in Versicherungswirtschaft, Karlsruhe, 1985, pp. 585 sq.
\end{quote}

AUTEUR
Ernst Klingmüller



\paragraph{les mutuelles en France}
\href{https://www.cairn.info/revue-les-tribunes-de-la-sante1-2016-3-page-61.htm}{Les mutuelles, un acteur puissant mais peu écouté
François Charpentier}
\begin{quote}
    Dans un pays catholique où, pour des raisons religieuses, le principe de l’assurance était encore prohibé au XVIIIe siècle et où l’individu devait s’en remettre à la providence divine, la mutualité est apparue très tôt comme une alternative possible au besoin de protection contre la maladie et la vieillesse. Les historiens font remonter les premières mutuelles au XIVe siècle. Nous retiendrons pour notre part que l’essor des mutuelles au sens moderne du terme – des collectivités d’hommes et de femmes qui s’organisent entre eux sur des bases professionnelles ou locales pour se protéger – est lié à l’urbanisation galopante de la société, elle-même liée à la révolution industrielle.

La concurrence des sociétés d’assurance
3Pour autant, l’apparition des mutuelles dans le paysage social français n’a pas été un long fleuve tranquille. On rappellera d’abord que l’essor des mutuelles s’est opéré au moment où les compagnies d’assurance obtenaient un droit légal à l’existence avec la création, en 1787 par Louis XVI, sous la pression de banquiers genevois, donc protestants, d’une Compagnie royale d’assurance sur la vie. Bien sûr, il s’agissait d’une institution publique, mais fonctionnant sur le modèle d’une société privée collectant de l’épargne pour faire face aux besoins de ses membres quand se réalisait l’aléa. Quant à la recherche du profit, elle constituait l’élément moteur de cette innovation.

4On retiendra que cette royale initiative a eu pour mérite de rendre à l’homme la liberté de maîtriser les événements de son existence. Elle a sécularisé la société, qui passe alors « du moral au légal, du religieux au laïc [2]
[2]
L. Lautrette, Le Droit de la retraite, PUF, coll. Que sais-je ?… ». Elle a ouvert la voie à la pratique de l’actuariat. Elle a inscrit la prévoyance dans le champ d’intervention de l’État en favorisant la mise en place d’un projet à finalité sociale, puisque cette « prévoyance sociale » doit se développer au profit des professions les plus exposées : marins, pêcheurs, maçons, charpentiers, couvreurs. Elle a anticipé, d’une certaine façon, « le passage des assurances commerciales aux assurances sociales professionnelles [3]
[3]
B. Gibaud, Mutualité, assurances (1850-1914), Economica, 1988. ».

5D’emblée la concurrence a donc été vive entre, d’une part, les assurances développant la prévoyance libre et, d’autre part, des mutuelles prônant la fraternité et la solidarité et, de ce fait, gérées bénévolement par leurs membres qui ne faisaient pas du profit leur raison d’exister. Au contraire même puisque la règle voulait que les surplus accumulés et non utilisés soient rétrocédés aux adhérents sous forme de baisse de cotisation ou de majoration des remboursements.

6En résumé, comme le formule Patricia Toucas-Truyen dans son Histoire de la Mutualité et des assurances : « Les débats éthiques autour du développement de l’assurance sur la vie font clairement apparaître des différences […] de finalités entre caisses de secours fraternelles ou corporatives et assurances : les assurances permettent aux classes aisées d’accroître leurs biens et de les transmettre à leurs héritiers ; les caisses de secours offrent à leurs membres la possibilité d’atténuer les conséquences de la maladie et de la vieillesse. Pour les uns, il s’agit d’obtenir l’assurance d’un capital confortable jusqu’à la fin de leur vie et, au-delà, transmissible à leur postérité. Pour les autres, l’épargne réalisée n’assure guère que la survie [4]
[4]
P. Toucas- Truyen, Histoire de la Mutualité et des assurances,…. »
\end{quote}
 \section{Bibliographie}


« Ad-Dourra Al Moukhtasara » : Le résumé des vertus de la religion musulmane. Par : Abdarrahman As-Sacdî. Traduit par Tamime Khemmar.

Jomier Jacques. L'imam Mohammad 'Abdoh et la Caisse d'Epargne (1903-1904). In: Revue de l'Occident musulman et de la
Méditerranée, n°15-16, 1973. Mélanges Le Tourneau. II. pp. 99-107;

\cite{Chapra:Riba}

\chapter{Plan pour Riba}

\section{Introuction}
\href{https://icp.summon.serialssolutions.com/2.0.0/link/0/eLvHCXMwrV1db9MwFL0a6wNICMYAUQYoT3w8tHPt3Lh5TNukYWoH2jpN4sWyHRehiTK1Rdoj_4F_yC_BN3UjOvGEeIuU2FJ0fK_PTc65BhC8yzq3coKbpzLlzm8QLkYjetazftrqcK6lZ7TkG87H6cciPj8Tkz0ot9aYTbuI5vsbBUqdvinetVkd_yHMIQuRr7FYkGtxlrCuJ5ct-tGA-9AajM4-FU2S5lifnecrECQTPu6KfP46V5O0WwTAzQ4lbYitCIm2eAhXjb_HXne_VF8bf_Wtho__4z0P4EEgsFG2WXGPYM8tDuHu1t-8OoSDIKnzD4XE8RiOZ2UehTMTz2fR9EOZTaf5KHo_OI2ywehiMsnK6C1K9uvHz0Twd0_goshnw7ITjmrofOZIBaidG4Pk16skam4SURkrekJTzWn8LdnrSVt5coGoNTIzN8xJak7vHClNxVO4r0nSv1jX1r_qGUSVpx-JdTK2lsVWCy37fmDKEysSNMy24Q3hoSgS10ttdTAUfFs46mmlMk7NAf0OnLbhdQ2Zut508FB6eUWCNonq8nSsyuH05PKkHKukDUdbTFWI5ZXqI9JZPv24DbJGpplmp4DqK0JFESqqRkXdqHyYFXT5_J9HHsG9zVdsUr29gP318rt7CXf8knoVlvRvpSXyjQ}{The Economist  MOHAMMED IBN ABDULLAH (570-632)}
The absence of a dynamic market economy in many Islamic societies has encouraged the inference that the values of Islam are not compatible with capitalism. However, an examination of the biography and commercial record of Islam's founder, the Prophet Mohammed, refutes this presumption. Mohammed ibn Abdullah was a scion of an elite dynasty of religious, civic and commercial leaders in Mecca. He abandoned his successful business career in Mecca and fled to Medina at the age of 52, where he realised his vision of an Islamic society. In Medina, Mohammed implemented policies for competition, consumer protection and market regulation. Mohammed's approach to fair trading explains his ban on usury, as distinct from a proscription on borrowing. Mohammed's achievements as an economist and market reformer earn him a place in the history of economic thought.

\href{https://icp.summon.serialssolutions.com/2.0.0/link/0/eLvHCXMwnV1Lj9MwEB61e0BIqwXKrggsyJe9rEhJGidO9rI81MChhwroiUPk-KEt0CrbhxD_nhknaWhRLxwcR8pETjLO-LP9zQxANBoG_oFNMDYT2cjgAGF4XEahQtRPQ11spUBES37D44_ZNOdfPkeTHqStawyxLB1N0G3qI14qf5o3OGojyI75bXXvU_oo2mZtcmmQLUZI4Hr3-84i78f45u3uZu1Ch5jBp3l1FtLJ3vhUUxSxXpnKqP3NUtGY1PwRfNt58qhqONeLnSf1QWjH_3mjx3DWQFP2ru5LT6BnlgN40DLjB9CfyF9PYTZboxqYXGpGmcAYGRScCjsF37CG4bxdO4G8DefBalo9my9ZF5eETTtHz3OY5eOvHz75TW4GnwDIxk-1sFGiuI6VUok0QnKtlOCpDU2cIajUJeWcyLhKAi04XrUmQ6gQSWtNmdroAk4lcfiXG-frp58B04g3EmUEVyrgSkZSpHFQZqNERUlcBsqD61Y1RVXH4ii6qMukx4JYeqTHIvDgwn3mnWT7jT1467S5u_BDVt_L7drgv11gmyM8_MZCi7hYzbGEWCpXh3Fxt1l48LrtCH89CLVPCzRFo6f6QSptPbj6R9wJ1vdgi6GTPSInDuWeH3u1F_CwXnomYtElnGxWW_MS-tgrX7kf4g9iMhFq}{
Usury and Just Compensation: Religious and Financial Ethics in Historical Perspective}
Usury is a concept often associated more with religiously based financial ethics, whether Christian or Islamic, than with the secular world of contemporary finance. The problem is compounded by a tendency to interpret riba, prohibited within Islam, as both usury and interest, without adequately distinguishing these concepts. This paper argues that in Christian tradition usury has always evoked the notion of money demanded in excess of what is owed on a loan, disrupting a relationship of equality between people, whereas interest was seen as referring to just compensation to the lender. Although it is often claimed that hostility towards 'usury' has been in retreat in the West since the protestant Reformation, we would argue that the crucial break came not with Calvin, but with Jeremy Bentham, whose critique of the arguments of Adam Smith, upholding the reasonableness of the laws against usury, led to the abolition of the usury laws in England in 1854. There has to be a role for law, whether Islamic or secular, in regulating financial relationships. We argue that by retrieving the necessary distinction between demanding usury as illegitimate predatory lending and interest as legitimate compensation, we can discover common ground behind the driving principles of financial ethics within both Islamic and Christian tradition that may still be of relevance today. By re-examining past ethical discussions of the distinction between usury and just compensation, we argue that the world's religious traditions can make significant contributions to contemporary debate.
\section{Contexte Coranique}

\paragraph{difficulté d'application de la Riba} \href{https://www-jstor-org.icp.idm.oclc.org/stable/23264404}{The Qadi, the Big Merchant and Forbidden Interst (ribā)} While the imposition of interest on loans is expressly forbidden by the Quran, over the generations Muslims in fact found it difficult to observe this prohibition and lent to and borrowed from one another. The Muslim big merchants (tujjār), who held large amounts of liquid capital, were prominent among the lenders. The religious prohibition on the one hand, and everyday constraints on the other, caused a certain cognitive dissonance among many ulema, as manifested in the religious (shar'i) courts. The records of these courts in various regions in the Middle East from the sixteenth to the beginning of the twentieth century include cases in which judges (qadis) exempted borrowers from full or partial repayment of interest on the grounds that a demand for payment of interest violates the precepts of Islam. The present article provides examples of such cases and discusses the possible effects of the courts' retroactive annulment or amendment of contracts on the development of capitalist economies in Muslim countries during the nineteenth century when Middle Eastern economies began integrating into the global economic system. Inter alia, the discussion of this question sheds new light on the historians' critiques of Max Weber's observation regarding 'Kadijustiz' (qadi's justice) and its effect on the development of modern capitalism.
\section{Un premier mouvement, modernisme}

\section{une réaction islamique : Takaful}

\section{Une difficulté à appliquer}


\paragraph{Insurance} \href{https://www-jstor-org.icp.idm.oclc.org/stable/27650571?pq-origsite=summon}{Islamic Insurance: National Features and Legal Regulation} The present paper studies Islamic insurance (takaful) as opposed to conventional one. The first part of the paper covers, among other things, such issues as nature and historic roots of Islamic insurance, early forms of Islamic insurance and narrates the disputes among Muslim scholars concerning the compatibility of insurance with Islamic Shariah. The second part deals with history and emergence of Islamic insurance in the modern financial market, as well as the practice of Islamic insurance in different coutries. The third part discusses the feasibility of Islamic insurance in Russia in the current legal framework. The paper contains comprehensive glossary of related terms.
\paragraph{Assurance Auto}

\paragraph{Tawarruq} \href{https://www-jstor-org.icp.idm.oclc.org/stable/43294670?pq-origsite=summon&seq=1}{Debate on "Tawarruq": Historical Discourse and Current Rulings}

One of the controversial products used by the Islamic financial sector is organized tawarruq. As the substance of this product resembles that of an interest-based loan, there is a debate on its permissibility from a Shari'ah perspective. While discussions on tawarruq have arisen due to the emergence of the practice in Islamic finance, there have been deliberations on this transaction in the past starting immediately after the emergence of Islam. The aim of the article is to provide an overview of the historical discourse on tawarruq and examine the rulings on it by two contemporary jurisprudential bodies to assess the practice of the transaction in Islamic finance. The discussions show that the current rulings concur with the majority view of the past scholars. The practice of organized tawarruq by the Islamic financial industry, however, appears to be inconsistent with both the contemporary and historical rulings.

\section{Rissalat}

\cite{Abdou:Rissalat}

\paragraph{manière de démontrer le besoin qu'ont les hommes de la prophétie}
\begin{quote}
   On a dit : Pourquoi Dieu n •a -t-il pas déposé dans la conscience même
de l'homme le savoir dont celui-ci a besoin ? Pourquoi ne l'a-t-il pas fait
son propre guide qui se dirigerait soi-même vers l'action juste et vers le
chemin droit conduisant au but assigné dans l'autre vie? Quelle est
cette miséricorde étrange qui prend une voie détournée pour nous
guider et nous instruire ? Celui qui parle ainsi exagère les attributions
de la raison et n • apprécie pas comme il convient le sujet dont nous parlons,
c'est-à-dire le genre humain, tel qu'il est, avec les différents éléments
spirituels et intellectuels qui le composent, et qui ont pour conséquence
des degrés divers dans la formation intellectuelle des individus. Il
néglige le fait que chaque individu n'est pas apte à acquérir spontanément
tous les états de la connaissance, mais qu'au contraire son progrès
est basé sur la recherche et sur l'étude. Car s'il pouvait satisfaire
ses besoins par l'instinct, comme le font les autres animaux, il ne serait,
plus de son espèce mais bien un animal, comme l'abeille et la fourmi,
ou un ange, habitant des cieux.
\end{quote}

\begin{quote}
    Nul ne met en doute que chaque membre d'une société a besoin
des autres membres; et chaque fois que l'individu accroître ses exigences
dans la vie il ressent plus fortement le besoin de recourir au concours
de ses semblables. Ainsi se développent les besoins et à leur suite s'étendent
les relations de la famille à la tribu, de celle-ci à la nation, et finalement
au genre humain tout entier, comme le montre notre époque . Ces
besoins qui créent dans le sein de chaque nation (surtout dans le sein de
celles qui méritent vraiment ce nom) des relations et des rapports
spéciaux la distinguant des autres nations, sont le besoin de se procurer
sa subsistance, celui de profiter des biens de la vie, celui d' acquérir
les choses désirables et d'éloigner de soi celles qui déplaisent. 
Si la vie de l'homme se déroulait selon les lois de la nature, telles
que nous les voyons appliquées aux autres êtres vivants, les besoins
que nous venons de citer auraient été parmi les facteurs les plus puissants
de l'amour entre les individus, le facteur qui aurait fait sentir à chacun
que son existence dépend de celle de son groupe, que ce groupe est pour lui comme une faculté nouvelle dont il dispose pour acquérir les
choses utiles et éloigner de soi les choses nuisibles. L'amour est la source
de la paix et de la quiétude dans les coeurs ; quand deux êtres s'aiment,
c'est lui qui pousse chacun d'eux à agir dans l'intérêt de l'autre, à le défendre en cas de danger.  C'est l'amour qui main tient l'harmonie au
sein des peuples, qui est l'essence même de leur existence ; les lois naturelles qui nous régissent ont créé un lien étroit entre l'amour et le besoin,
car l'amour n'est que le besoin que nous éprouvons de nous rapprocher de la personne ou de la chose aimée, et, en croissant, il se transforme en passion.

\ldots
Si par contre, l'intérêt se mêle aux relations amicales, et si chacun des amis exige un prix pour son amour, celui-ce se change en esprit d'exploitation, il se reporte sur l'effet utilitaire et se transforme chez l'un des amis en un abus de la force, et chez l'autre, en peur avilissante, en dissimulation et en hypocrisie.
p.66
\end{quote}

\begin{quote}
    
    « l 'homme
a été créé avec u11 caractère inconstant; quand le malheur l'atteint
il est abattu, et quand il acquiert quelque bien il devient insolent. >;
(C. ch. 70, v. 19 à 21). Les individus sont diversement doués au point
de vue de l'intelligence, de l'activité, de l'assiduité et de la volonté. 11 y
en a qui sont inférieurs à la moyenne, soit _par faiblesse d'esprit, soit
par paresse, tout en la dépassant d ans l'intensité de leurs désirs qu'ils
cherchent à satisfaire avec passion et avidité ; ils considèrent leurs
semblables comme un moyen pour subvenir à leurs besoins, mais ils
se représentent la jouissance que leur procurerait l'usage exclusif de
tout ce qui est dans la main d'autrui et ne se contentent pas d'utiliser
les fruits de leur propre travail en les échangeant contre ce qu'ils désirent ;
et comme ils veulent vivre sans travailler, le meilleur moyen, d 'après
eux, est de s'appliquer à toute espèce de ruses afin de profiter des autres
sans être eux-mêmes d'aucune utilité. Ces sentiments s'emparent d'eux
au point qu'ils s'imaginent qu'il n'y a pas de mal à priver de l'existence
celui qui s 'oppose à leurs désirs et à le supprimer après l'avoir
dépouillé. Chaque fois que la mémoire et l'imagination les poussent à
éviter quelque chose qui leur inspire de la crainte ou à atteindre un
objet qui leur fait envie, leur intelligence leur ouvre une porte de la
ruse ou leur découvre une voie de la violence ; alors le rapt remplace
l'échange pacifique, la dispute prend la place de l'union et la conduite
de l'homme rie s 'appuie plus que sur l'astuce et la violence.
p 68
\end{quote}

\begin{quote}
   Est-il possible que les sociétés humaines, dont la bonne marche et
l'existence même, dépendent de la collaboration et de l'aide que les
hommes s'accordent entre eux, puissent subsister dans ces conditions?
Est-ce que les facteurs que nous venons de faire ressortir ne seront pas
la cause de leur disparition? Certes oui, etc 'est pourquoi le genre humain
a surtout besoin, pour conserver son existence, de l'amour ou d'un
sentiment qui le remplace.
A différentes époques il y a eu des penseurs qui ont fait appel à
l'équité ; ils ont pensé, et même quelques mystiques ont exprimé cette
pensée par de nobles paroles, que l'équité remplace l'amour. Cette
assertion ne manque pas de sagesse, mais qui est-ce qui peut établir les
règles de l'équité et amener la totalité des hommes à se soumettre ?
p 69
\end{quote}

\begin{Synthesis}
Abdou montre qu'on peut accéder à l'équité par des voies naturelles mais que pour les masses, et que par ailleurs la corruption des élites, c'est mieux d'avoir une loi externe, celle qui lutte contre la plus grande détresse, celle qui lui inspire l'inconnu.
\end{Synthesis}

\begin{quote}
    cieux et à
Lui retournera toute chose. " (Cor. ch. 3, v. 100 à 105.) Après ces
exhortations, qui jettent le trouble dans le coeur de ceux qui transgressent
les commandements de Dieu et qui confirment les châtiments pour ceux
qui s'en écartent ou les pratiquent imparfaitement, le Coran indique
qu'il n'y a pas de condition meilleure que celle des hommes qui ordonnent
le bien et défendent le mal: • Vous êtes les meilleurs parmi les hommes,
vous ordonnez le bien, vous défendez le mal et vous croyez en Dieu.
 
(Cor. ch. 3, v. 106.) Dans ce verset le fait d'ordonner le bien et de
défendre le mal est mentionné avant la foi en Dieu, bien que la foi soit la base même sur laquelle s'appuient les bonne oeuvres. 
p121
\end{quote}

\begin{quote}
L'Islam impose aux riches de consacrer une partie déterminée de leurs biens en faveur des pauvres, par ce sacrifice les riches élèvent une digue contre l'indigence, ils adoucissent la détresse des endettés, ils émancipent ceux qui plient sous le joug de la servitude, ils viennent en
aide aux voyageurs \sn{Payer les dettes des débiteurs malheureux. affranchir les esclaves et construire des
caravansérails constituaient dans les pays musulmans, des actes de générosité particulièrement
recommandes par la religion.}.
 
Il nous invite par-de,sus tout à dépenser notre bien pour les oeuvres
charitables et souvent il en fait l'expression de la foi et la manifestation
d'une bonne conduite; il déracina par là, du coeur des pauvres, la rancune
et la haine contre ceux que Dieu a favorisé des biens terrestres ; il leur
inspira l'amour pour les riches, tout comme il fît naître dans le coeur de
ceux-ci la pitié pour les malheureux ; ainsi il développa la confiance dans
le coeur de tous les hommes. Quel remède plus efficace contre les maux
dont souffre la société: « C'est une faveur que Dieu accorde à qui il veut,
car Dieu est d'une bienfaisance sans bornes. » (Cor. ch. 57, v. 21.)
\textbf{L' Islam a fermé les deux portes du mal, il a bouché les deux sources
qui minent l'intelligence et détruisent la richesse, en frappant les boissons
enivrantes, les Jeux de hasard et l'usure, d'une interdiction absolue qui
n'admet pas d'infraction.}

Enfin l'Islam n'a pas laissé une seule des vertus principales sans
en parler, une seule source de bonnes oeuvres sans la vivifier une seule
loi de. l'ordre sans la préciser. Il prépara, pour l'homme arrivé à sa
maturité, l'émancipation de l'esprit, l'indépendance de la raison dans
ses recherches, et, comme conséquence de cette émancipation et de
cette indépendance, l'épanouissement de ses facultés naturelles, le réveil
de sa volonté, son élan sur la voie de l'effort. Celui qui lit le Coran,comme
1I_ doit être lu, y trouve, sous ce rapport, des trésors inépuisables et des
richesses sans fin.
p 122

\end{quote}


\section{L'imam Mohammad 'Abdoh et la Caisse d'Epargne (1903-1904)}

\mn{Jomier Jacques. L'imam Mohammad 'Abdoh et la Caisse d'Epargne (1903-1904). In: Revue de l'Occident musulman et de la
Méditerranée, n°15-16, 1973. Mélanges Le Tourneau. II. pp. 99-107;}
\cite{Jomier:AdbouCaisseEpargne}
\begin{quote}
    L'on pense couramment dans les milieux d'orientalistes que l'Iman
Mohammad *Abdoh aurait approuvé certaines opérations financières au sujet des
quelles l'Islam traditionnel restait réticent. L'on s'appuie sur l'autorité d'Ignace
Goldziher qui, dans ses Vorlesungen ùber den Islam publiées en 1910, écrivait :
"Le mufti égyptien Cheikh Muhammad 'Abduh, mort en 1905, a trouvé le
moyen, dans une savante fatwa sur la matière, de présenter comme licite pour la
société musulmane, au point de vue de la loi religieuse, la Caisse d'Epargne et le
gain des dividendes" (1).
Par contre les études parues en Proche Orient ne parlent guère d'une telle
fatwa alors que d'autres consultations juridiques de Mohammad *Abdoh ont eu un
retentissement considérable. Pourquoi cette différence d'attitudes ?
\end{quote}

\begin{quote}
    1 ) En ce qui concerne les dividendes, je n'ai jusqu'ici rien rencontré dans les
oeuvres de Mohammad 'Abdoh qui en proclame formellement la licéité. Par contre
les principes invoqués à deux reprises, dans une fatwa qu'il a donnée à
l'instigation de la Compagnie d'assurance-vie Gresham (1903) et dans la
modification de la loi sur la Caisse d'Epargne, soumise à son approbation (1904), vont
dans le sens de la légitimation. Dans les perspectives du réformisme musulman,
cette question d'ailleurs est loin d'être insoluble car une société qui émet des
actions peut être assimilée à une société en commandite. Les dividendes
représentent alors la part des bénéfices proportionnelle au capital engagé... 
al-Banna, fondateur des Frères Musulmans en Egypte, avait admis la licéité de ce
genre d'opérations et des Frères avaient, peu avant 1 948, fondé quelques ateliers
ou sociétés financées par ce procédé (2). La difficulté principale vient de ce que la
langue arabe n'a pas de mot spécial pour désigner les dividendes et que le terme
fâ'ida a des relents de ribâ.
\end{quote}

\begin{quote}
    2) En second lieu, de quelle fatwa s'agit-il ? Jusqu'ici les efforts pour
retrouver le texte précis d'une fatwa de l'Iman concernant les Caisses d'Epargne et
les dividendes sont restés vains. Dans sa traduction arabe des Vorlesungen de
Goldziher, le Dr Mohammad Youssef Mousa a une note pour dire qu'il n'a pas vu
personnellement cette fatwa. Il en ignore même le contenu. Il suppose que le
demandeur de la fatwa n'a pas touché à la question des intérêts déterminés
(tah'dîd al-fâ'ida). "Ce que nous savons, ajoute-t-il, c'est que le prêt à intérêt
(fâ'ida) est de l'usure (ribâ) interdite et que personne n'a le pouvoir de le rendre
licite" (3). La traduction du passage de Goldziher cité ici plus haut est d'ailleurs
infléchie dans le sens de la Caisse d'Epargne ; la mention des dividendes a disparu.
Ceux-ci sont examinés dans la phrase suivante et désignés par une périphrase.
\end{quote}

\begin{quote}
    Il existe bien une fatwa donnée par l'Iman Moh'ammad 'Abdoh, en tant que
mufti d'Egypte, en 1903. Ce fut à la requête d'un particulier mais en fait pour la
Compagnie d'Assurances sur la vie Gresham. La Compagnie d'Assurances al-Chark,
au Caire, en conservait une copie afin de la montrer à ses clients, le cas échéant.
La traduction française de cette copie se trouve dans mon étude Le Commentaire
Coranique du Manor (4). Le texte arabe en avait d'ailleurs été publié dans la revue
al-Manâr à l'instigation de lecteurs tunisiens alertés par les journaux al-Wat'an et
al-Maghrib. Ces deux feuilles en avaient parlé, avaient donné le texte ; peu après,
la fatwa avait été l'objet d'un article dans al-Zahra de Tunis. La revue al-Manâr
disait tenir le texte de ce que les deux premiers journaux avaient publié (5).
Il s'agit dans la fatwa d'un contrat passé entre un client et une société. Le
client effectue des versements réguliers et périodiques par tranches successives
prévues d'avance ; la compagnie les encaisse et se charge de faire fructifier cet
argent. Finalement, à l'échéance, la société remboursera le total des versements
effectués augmenté des bénéfices résultant de la fructification. La réponse admet
la licéité d'une telle opération en des termes qui, au fond, s'appliqueraient à toute
société en commandite.
\end{quote}

\begin{quote}
    La revue al-Manâr affirme que la Compagnie d'Assurances Gresham a rajouté
entre parenthèse son propre nom à côté du terme général de société. Pourtant
dans l'extrait de fatwa montré par la compagnie al-Chark et provenant des
Tribunaux Charéis, cette parenthèse figure nettement. Le ton sur lequel la revue
parle de cette fatwa est plus que réservé. Elle n'en nie pas l'authenticité alors que
riman Moh'ammad 'Abdoh était encore vivant et tout proche. On peut tenir la
fatwa pour authentique car l'Iman aurait aussitôt inspiré une protestation s'il y
avait eu un faux (6). Il y a eu probablement dans l'utilisation de ce texte quelque
chose de tendancieux qui a indisposé les musulmans. La revue proteste en effet,
signalant que si la réponse correspond bien à la question du demandeur, cette
question est loin de s'appliquer au cas des assurances sur la vie. Aurait-on profité
de la fatwa pour en tirer des conclusions indues? La question reste toujours
mystérieuse.
\end{quote}

\begin{quote}
    Quant à une fatwa concernant" directement la Caisse d'Epargne, nous en
ignorons l'existence. Le 5 décembre 1903, la revue al-Manâr (p. 717) affirme
qu'une telle fatwa (objet de commentaires de la part du public) n'existait
absolument pas. L'histoire des débuts de la Caisse d'Epargne montre seulement
que Moh'ammad *Abdoh a joué un rôle lors de la rédaction d'un nouveau texte de
loi, à ce sujet, texte remaniant complètement la rédaction de la loi primitive. Il fit
partie d'une commission d'Azhariens chargés d'élaborer le projet et la loi lui fut
soumise, en tant que mufti, avant sa promulgation. Et ceci nous conduit au
troisième élément de la phrase de Goldziher : la Caisse d'Epargne.
3) C'est en effet la Caisse d'Epargne elle-même qui offre la piste de
recherches la plus intéressante. Les textes officiels de lois, leur histoire, les
principes mis en jeu méritent en effet que nous nous y arrêtions.
\end{quote}

C'est très intéressant parce que face aux scrupules de 3000 musulmans de toucher les intérêts (fa'ida), il a fallu une note du Manâr.
\begin{quote}
    La note du Manâr (5 décembre 1903)
Cette note fournit un certain nombre d'informations sur la pensée de
Moh'ammad 'Abdoh. La modification de la loi sur la Caisse d'Epargne partit d'une conversation privée entre lui et le directeur des Postes d'alors (Çaba Pacha
probablement). Ils parlèrent des trois mille usagers de la Caisse d'Epargne qui ne
retiraient pas les intérêts auxquels ils avaient droit alors que les autres usagers le
faisaient (11). Dans la conversation, le mufti maintenait fermement le principe de
l'interdiction absolue de l'usure (al-ribâ) ; mais il ne classait pas le cas de la Caisse
d'Epargne avec ceux d'usure. Il l'assimilait à celui d'une société en commandite
(chirkat al-mod'âraba).
A la suite de cette conversation, deux commissions d'azhariens furent
formées pour étudier la question. L'une d'elles était présidée par le mufti.
\end{quote}

\begin{quote}
    La modification de la loi (loi du 14 février 1904)
La difficulté principale venait de ce que dans une société en commandite la
part des bénéfices varie suivant l'état des affaires tandis que dans le cas de la
caisse d'épargne cette part est fixée une fois pour toutes. Voici comment la
formulation de la loi tourna les difficultés.
L'article premier est rédigé de façon à souligner le caractère de "société créée
pour faire fructifier en commun les dépôts de ses clients" que l'on veut donner à
la caisse d'épargne. Il est prévu dans l'article que le client devra signer un
formulaire imprimé dans lequel il déclarera donner au Directeur Général de la
Poste tout pouvoir pour faire fructifier les sommes déposées, d'une façon licite et
en excluant toute opération usuraire. Il déclare permettre au Directeur de la Poste
de joindre ses dépôts à ceux des autres clients pour les faire fructifier en commun
à condition de recevoir une part de bénéfices (al-ribh') proportionnelle à ses
versements. On notera que dans cet article comme dans tout le reste de la loi le
mot de \textit{fâ'ida}, intérêt, est totalement absent.
\end{quote}

\begin{quote}
    L'article deux stipule entre autres ce qui suit. Il évite de parler de tant pour
cent, formule qui rappelle trop les intérêts. Il note que la part des bénéfices ne
dépassera pas un pour quarante du capital et le surplus, s'il y en a, reviendra à
l'administration postale à titre de compensation pour les services rendus et les
frais que ceux-ci comportent. L'assimilation à la société en commandite est ainsi
mise en accord avec le fait que la proportion des bénéfices touchés est fixe. On
notera que la proportion de un pour quarante est familière aux oreilles des
musulmans pieux puisque dans un certain nombre de cas le montant de la Zaka
atteint ce chiffre.
\end{quote}

Voir la Zaka ?

\begin{Def}[Zakat]
La zakat est le troisième pilier de l'islam et son essence même révèle l'importance de la participation sociale dans l'univers musulman. La zakât est clairement un impôt sur l'avoir et la propriété qu'il faut comprendre, d'abord, comme une obligation devant Dieu. Ce prélèvement purifie sur le plan religieux, sacré et moral le bien de celui qui le possède.
Les différents types de biens soumis à la Zakat
Sont soumis à la zakat quatre types de biens 3 :

Avoirs/biens et fortune (espèces, métaux précieux, dépôts ou titres bancaires) ou zakat al maal
Les récoltes
Fonds de commerce (sur tout bien destiné à la vente)
Les bestiaux (ovins, bovins, ou encore camélidés)
Non soumis à la zakat :

Terrain, immeuble, bâtiment (non destinés à la vente)
Mobiliers, vêtements, voitures, etc.
Hypothèque
Bijoux personnels (pour les femmes suivant l'école Chafiite leurs parures en or sont exemptés de zakat, dans les trois autres écoles, tous les bijoux sont exemptés de zakat sauf l'or et l'argent)
\end{Def}

\paragraph{un pour quarante}
 La zakat constitue 2.5 \% du chiffre annuel épargné. (Elle ne s'applique pas sur les bijoux personnels en or pour les femmes chafiites).

La zakat est obligatoire sur l'argent économisé et qui a été immobilisé un an durant  
 Tous les ans les musulmans doivent se renseigner sur le prix du gramme d'or du pays où ils résident et le multiplier par 85 pour connaître le seuil d'imposition de la zakat sur leur argent personnel. Le minimum imposable est estimé en dollars et en d’autres billets de banque à l’équivalent de 20 mithqal d’or ou 1400 mithqual d’argent selon leur prix du moment où vous devez acquitter la zakat en dollar ou en d’autres monnaies.
 
 \begin{quote}
     La seconde note du Manâr (18 mars 1904)
II s'agit d'une question posée par un lecteur (réel ou fictif ? ) et qui concerne
cet amendement législatif du 14 février 1904. Il est noté dans la réponse que les
nouveaux textes législatifs ont été approuvés par le Mufti (c'est à dire Moh'ammad
'Abdoh). La note explique que les opérations prévues par la loi sont entièrement
différentes des opérations usuraires condamnées par le Coran. L'Administration
des Postes n'emprunte pas parce qu'elle serait dans le besoin : son but est
uniquement d'être au service de tous. Cette façon d'agir est à ranger dans la
catégorie de la vente et non pas dans celle de l'exploitation d'un homme dans le
besoin et tous en profitent. Le dépôt des sommes à la Caisse d'Epargne rentre
donc dans la catégorie juridique des contrats.
 \end{quote}
 
 \paragraph{Texte de la fatwa et de la note du manâr du 18 mars 1904}
 \begin{quote}
     Réponse. L'amendement législatif auquel vous faites allusion a été élaboré
d'après les vues d'une commission d'Oulémas d'al-Azhar. Celle-ci avait été formée
par le Khédive afin que les opérations de la Caisse d'Epargne soient conformes
aux principes du droit religieux. Ces Oulémas donnèrent leurs conclusions et les
envoyèrent au gouvernement. Celui-ci les soumit au Mufti et lorsqu'elles eurent
été approuvées par lui, le gouvernement les fit appliquer. Voilà ce qui est de
notoriété publique. Quant à nous, nous n'avons pas à donner de fatwa sur ce
qu'ils ont écrit, pour manifester notre opinion sur la question. Mais en tout cas,
nous ne pensons pas qu'il soit mal de mettre en pratique ces conclusions. Car nos
critiques visent uniquement celles des ruses juridiques propres aux Oulémas
casuistes et formalistes (aux Oulémas al-rusûm\sn{Intelligent / rusé ?} comme dit al-Ghazali) qui vont
contre les véritables buts de la loi religieuse tels qu'ils sont solidement établis par
le Coran et la tradition. Ainsi par exemple les ruses pour échapper au paiement de
la Zaka, ou les ruses concernant l'usure véritable, celle dont le Coran justifie
l'interdiction par le fait qu'il demande : "vous ne lésez point et vous n'êtes pas
lésés" (Coran, 2,279). Et ailleurs la différence entre l'usure et le commerce est
expliquée par ces mots du Coran : "Dieu vous a permis la vente et il vous interdit
l'usure" (Coran 2,275). Car passer un contrat pour une opération qui profite à
celui qui cède et à celui qui acquiert est une vente ou une opération de commerce.
Et de même ce texte qui explique la raison de l'interdiction par ces mots : "O
vous qui croyez ! Ne vivez pas de l'usure multipliant de double en double"
(Coran 3,130).
Et tout cela parce qu'il y avait à Médine et ailleurs des Juifs et des païens
qui prêtaient aux gens dans le besoin à des conditions honteuses d'usure, comme
nous y ont habitués en Egypte les Juifs et les messieurs étrangers (al-khawâgât),
avec la ruine des maisons qui en est la conséquence. La raison de cette
interdiction de l'usure est de faire cesser l'injustice et de conserver les vertus de
solidarité et de bonté mutuelle. Ou si vous préférez, pour que le riche n'exploite
pas le pauvre qui a besoin de lui (comme le dit l'Iman Mohammad 'Abdoh). C'est
en effet le sens du verset coranique : "Vous aurez vos capitaux, ne lésant point et
n'étant pas lésés" (Coran 2,279).

Il ne vous échappera pas que cette façon d'agir sert à la fois l'intérêt de celui
qui cède et de celui qui acquiert : elle est miséricordieuse pour eux et son absence
signifierait un manque de profit pour tous les deux. Elle n'est donc pas concernée
par les raisons données dans ce verset "ne lésant point et n'étant pas lésés". Elle
en est tout le contraire. Car une façon d'agir qui a pour but la vente et l'achat,
non pas le prêt à des créanciers dans le besoin, est à ranger dans la catégorie de la
vente et non pas dans celle de l'exploitation d'un homme dans le besoin. Il ne
vous échappera pas que l'Administration des Postes est un service gouvernemental
riche et qu'elle fait fructifier les sommes qui sont déposées dans les caisses
d'épargne (p. 29). Le déposant en profite aussi bien que les employés de
l'administration et du gouvernement. Aucun d'eux ne lèse l'autre. Aussi le plus probable
est-il que les conclusions de la commission ne sont pas une ruse légale : il s'agit
seulement d'une véritable association entre ceux qui apportent l'argent et d'autres
qui le font fructifier. Aussi je ne vois aucun empêchement à ce qu'on mette en
pratique l'amendement qu'elle propose en sorte qu'au point de vue du fiqh, l'on
considère l'affaire comme un contrat [. . .]".N.B. Nous ne traduisons pas les
dernières lignes qui ne font que se répéter.
(Revue al-Manâr, tome 7, pp. 28-29)
Jacques JOMIER
 \end{quote}*
%\chapter{Les preuves de l'existence de Dieu de al-Ǧuwaynī et Avicenne}
%{\Large\textbf{Les preuves de l'existence de Dieu de al-Ǧuwaynī et Avicenne }}
\mn{Guillaume Gorge - ISTR 2022 - Introduction à la théologie musulmane à la période classique - 16 mai 2022}

 

% -------------------------------------------------------------------------------------------------------
\section{Introduction}

La métaphysique en Islam (\emph{Kalām}) est aujourd'hui assez délaissée. Pourtant, cette métaphysique a connu une développement important, en particulier au 9è et 10è siècle, autour de deux questions : 
\begin{itemize}
    \item la transcendance de Dieu : si Dieu est transcendant, comment est-il possible d'en dire quelque chose ?
    \item la possibilité du libre-arbitre de l'homme face à la toute puissance de Dieu
\end{itemize}
Une troisième question anime le \emph{Kalām} : peut-on démontrer l'existence de Dieu ?  Nous voudrions étudier cette dernière question à travers deux démonstrations : 
\begin{itemize}
    \item la démonstration de \emph{al-Ǧuwaynī} , maître de d'Al-Ghazâlî et intermédiare entre les "anciens" et les "modernes" ash'arites.
    \item la démonstration d'Avicenne 
\end{itemize}
En montrant les présupposés de ces démonstrations, nous essayerons de montrer qu'elles éclairent finalement les deux premières questions, à la fois la transcendance de Dieu et la possibilité pour l'homme de libre-arbitre.


 

 
% -------------------------------------------------------------------------------------------------------
\section{Les démonstrations de al-Ǧuwaynī}



\paragraph{al-Ǧuwaynī.} Grand théologien ash'arite, al-Ǧuwaynī (1028-1085) dit \textit{l'imam des deux sanctuaires}  suite son séjour en Arabie, propose une approche plus rationnelle que les premiers ash'arites. Ainsi, alors que la démonstration de l'existence de Dieu de Al-Ashari proposait une démonstration basée sur la connaissance scientifique de l'époque et le Coran, la démonstration de al-Ǧuwaynī est plus conceptuelle.


\paragraph{la raison comme commandement divin Ash'arite.} Pour les Ash'arites, la réflexion constitue un devoir religieux (shar'i), à la différence des Mu'tazilites qui présentaient la réflexion comme un devoir mais qui s'applique à tout homme et pas seulement découlant d'une prescription religieuse (\cite{Cambridge:ClassicalIslamicTheology}). La place de la raison est l'un des points de désaccord avec les Hanbalites, courant qui s'interdit de faire de la raison une source du droit religieux. 

Ainsi, les théologiens traditionalistes comme Ibn Taymiyya  (pour qui la connaissance de Dieu est intuitive et immédiate par la vertu de la nature de l'homme \emph{fitra})   jugent vaines les preuves de l'existence de Dieu mais, de façon plus étonnante nous trouvons aussi des théologiens classiques comme Al Ghazali, pourtant l'élève de al-Ǧuwaynī, qui affirme que l'homme connaît Dieu à travers la \textit{fitra} sans besoin de raisonnement discursif. La preuve de Dieu peut néanmoins être nécessaire pour l'homme assailli par le doute \cite[p.197]{Cambridge:ClassicalIslamicTheology}


\paragraph{Classification des preuves de l'Existence de Dieu} 
Suivant A. Shihadeh \cite[198]{Cambridge:ClassicalIslamicTheology}, nous reprenons la catégorisation des preuves fournies par Fakhr al-Dīn al-Rāzī (m. 1210) , philosophe et \textit{mutakallim} (savant du Kalām, ou théologien), qui distingue les catégories suivantes : 
\begin{itemize}
    \item les preuves par la création des attributs des choses  ;   
    \item les preuves par la création des choses ;    
    \item les preuves par de la contingence des attributs des choses   ;  
    \item les preuves tirées de la contingence des choses (argument d'Avicenne).   
\end{itemize}
Les preuves de la première catégorie, historiquement la plus anciennes,  sont basées sur l'observation du monde et du refus d'y voir le fruit du hasard ou d'une entière contingence. Ils y décèlent au contraire son action et son \textit{dessein}. Ces preuves s'inspirent de nombreux versets du Coran (par exemple Co 2, 164). On trouve par exemple la démonstration d'Al-Ashari. Généralement ces preuves, mêmes si elles sont populaires parmi les musulmans, du fait de leur souffle coranique, sont délaissées par les théologiens du \textit{Kalām} car elles ne prouvent que l'existence d'un \textit{concepteur du dessein} (en anglais \textit{designer}) et non d'une création ex-nihilo (\cite[p.204]{Cambridge:ClassicalIslamicTheology}). Les théologiens musulmans comme Al-Ğuwaynī privilégient donc un autre type de démonstration, de type cosmologique, qui regroupent les preuves des catégories 2 et 3 de al-Rāzī.

\paragraph{\emph{Kitāb al-irshād} utilisant les outils de la logique} C'est à cette démonstration cosmologique que s'attelle  al-Ǧuwaynī dans son livre \textit{Kitāb al-irshād}, dont le plan reprend la structure des livres des   mu'tazilites. Il y établi tout d'abord que le raisonnement fait partie des obligations de tout homme. Puis il présente la démonstration de l'existence de Dieu, en utilisant la logique comme outil dans sa démonstration. Il est en cela à l'intersection entre les ash'arites \textit{anciens} qui répugnaient à utiliser les outils philosophiques comme la logique, et les mutakalimins \textit{modernes} qui se basent sur la logique d'Aristote.

\subsection{la démonstration proprement dite}

\begin{quote}
    Maintenant\sn{ (Al-Ğuwaynī, \emph{Kitāb
al-Iršād}, 4 ; trad. Luciani)} qu'il est prouvé que le monde est contingent et qu'il a eu un
commencement, il s'ensuit que le contingent peut exister ou ne pas
exister, et que quel soit le moment où il se produit, il aurait pu se
produire à un moment antérieur ; que l'existence du contingent aurait pu
être retardée de plusieurs heures au-delà de ce moment.

Si donc l'existence possible se produit, au lieu d'une prolongation
également possible de la non-existence, l'esprit saisit, comme une chose
évidente, que (pour se produire) l'existence a eu besoin d'une principe
déterminant (\emph{muḫaṣṣiṣ}) qui détermine sa réalisation. C'est là une
chose qui apparaît nécessairement, sans qu'il y ait besoin de faire des
distinctions ou d'employer le raisonnement.
\end{quote}

\paragraph{la démonstration Ex nihilo} La démonstration part de l'expérience du \textit{temps} qui préexiste au monde : le monde n'a pas été créé de toute éternité mais dans le temps (cf \textit{l'existence du contingent aurait pu être retardée de plusieurs heures au delà de ce moment}). 

\paragraph{Une impossibilité à penser le hasard} La première partie de la démonstration, avancée initialement par le théologien Ash'arite al-Bāqillānī (d. 1013) a comme présupposé une conclusion du \emph{Kalām} classique que le hasard est inconcevable et que tout fait doit être expliqué même celui qui parait le fruit du hasard. Pour al- Bāqillānī,  on observe des choses identiques se produisant à des temps différents. Si la survenance d'une chose est due à une de ses qualités intrinsèques, toutes les choses similaires doivent survenir au même moment. Il faut donc qu'il y ait une cause externe libre qui cause ces choses particulières de survenir à un moment spécifique. \cite[p.209]{Cambridge:ClassicalIslamicTheology} Cette vision, étonnante à nos yeux modernes, vient de la vision atomiste du monde : toute chose consiste en des atomes identiques et en des accidents différents présents en eux, atomes qui viennent à l'existence à chaque moment. Comme les atomes ont des possibilités infinies, il faut bien un facteur externe, \textit{principe déterminant}, car il détermine ou spécifie (\emph{takhīs}), les propriétés et les accidents des choses.

Al-Ğuwaynī reprend cette démonstration en l'étendant au monde puis continue sa démonstration : 

\begin{quote}
Une fois admis le principe général que le contingent exige un principe
déterminant, il faut nécessairement que ce principe soit : ou bien une
cause nécessitant la réalisation de la contingence, comme la cause
nécessite son effet ; ou bien une force physique, comme le pensent les
naturalistes ; ou bien enfin un agent libre. Or il est faux que ce
principe déterminant agisse à la manière d'une cause. La cause en effet
nécessite son effet d'une façon simultanée. Si on supposait que le
principe déterminant fût une cause, celle-ci serait forcément ou
éternelle, ou contingent. Dans le premier cas, elle aurait dû
nécessairement provoquer l'existence du monde de toute éternité, ce qui
conduirait à admettre l'éternité du monde. Or nous avons fourni les
preuves de sa contingence. Si le principe était contingent, il aurait
lui-même besoin d'un principe déterminant, et ainsi de suite à l'infini.
    
\end{quote}


 \paragraph{Une solution possible : la série infinie de principes contingents} Al-Ğuwaynī ne saute pas directement à la conclusion de l'existence d'un principe déterminant. Car, il est possible d'imaginer un monde existant de toute éternité en postulant une série infinie de \textit{principes contingents}, une opposition que mentionnera Averroes. Al-Ğuwaynī, conscient de la faiblesse de la démonstration classique du monde contingent, mentionne ici cette possibilité pour l'écarter.
Un monde qui n'existe pas de toute éternité entraîne donc l'existence d'un \textit{agent qui agit librement} sur les choses contingentes, en leur assignant certains attributs et certains moments :

\begin{quote}
Quant à ceux qui prétendent que le principe déterminant est une force
physique, leur théorie
est inadmissible. {[}\ldots{]}

S'il est faux, par conséquent, que le principe déterminant du contingent
soit une cause nécessitante, ou une force physique qui lui donne par
elle-même l'existence, mais involontairement, il s'ensuit d'une manière
certaine que le principe déterminant des choses
contingentes est un agent qui agit sur elles librement, qui leur assigne
spécialement, en les produisant, certains attributs et certains moments.
\end{quote}
 
 
\subsection{Conséquence de cette démonstration}

\paragraph{Conséquence sur la question du libre Arbitre}

La position d'Al-Ash'ari est fataliste - toute action, y compris les actes humains, sont déterminés par la volonté divine. En particulier, Al-Ashari déduit de la toute-puissance divine, l'attribution du mal et de l'injustice à la volonté de Dieu, mais se pose alors le problème de la responsabilité humaine et de la justice de Dieu à la fin des temps : si l'homme n'est pas responsable, il est injuste de le condamner. Or Dieu est juste. Face aux mu'tazilites qui reconnaissent le libre arbitre humain, Al-Ğuwaynī ne peut se satisfaire de la position d'Al-Ashari et cherche un compromis qui préserve l'idée de puissance de Dieu, tout en faisant davantage de place à Sa justice. Sans rentrer dans l'approche qu'il retient, le fait que le principe déterminant soit appliqué au monde et non directement à toute chose, permet de laisser un espace pour la volonté de l'homme et donc pour sa responsabilité et in fine la justice de son jugement.

\paragraph{Conséquence sur la question des attributs divins}
La position de Al-Ash'ari sur les attributs divins, se veut intermédiaire entre l'approche des mu'tazilites qui purifient (\emph{tanzih}) les attributs indignes de Dieu et les Hanbalites qui se tiennent à la lettre du Coran, mais acceptent l'équivocité des termes, quand ils sont appliqués à Dieu et à l'homme. Al-Ash'ari prend au sérieux les termes et fait le choix de leur unicité (si Dieu est \textit{voyant}, c'est qu'il a la vue), mais refuse l'approche des mu'tazilites qui considèrent que ces attributs font partie de l'essence de Dieu. Mais l'approche d'Al-Ash'ari laisse un peu sur sa faim.
Al-Juwaynī reprend la distinction grammatical  al-Jubbā' qui distingue l'essence de l'attribut du \textit{hal} ou \textit{accusatif d'état}. Cela  permet de distinguer les attributs de Dieu liés à son essence (l'existence, l'unicité, le savoir, l'éternité) et ceux qui sont possibles (la vue, l'ouïe, la parole,..). Al-Ğuwaynī reprend alors la définition de la propriété, ce qui est commun à des choses pourtant distinctes. La propriété est distincte et indépendante de l'entité qu'elle caractérise. Elle ne peut être dite ni existante, ni non-existante. 

% -------------------------------------------------------------------------------------------------------
\section{Les démonstrations d'Avicenne}

\paragraph{Avicenne.} Ibn Sina (980-1037) est probablement la figure la plus marquante de le \textit{falsafa}, ce courant théologique musulman nourri de la philosophie grecque, en particulier d'Aristote et, pour la partie métaphysique, de textes du philosophe néoplatonicien Plotin, faussement attribués à Aristote. 
Sa démonstration de l'existence de Dieu aura une postérité considérable, non seulement en Islam mais aussi sur la théologie Occidentale.

\paragraph{le courant de la \textit{Falsafia}} A la différence de l'Occident où la philosophie est une faculté d'université, certes ancillaire à la théologie, mais autonome, la \textit{falsafia} est une école musulmane concurrente des autres, ce qui fera rejeter leurs outils quand leurs conclusions ne seront plus jugées recevables par la plupart des théologiens musulmans.  

\paragraph{Le livre des guérisons}
    Avicenne développe sa métaphysique dans son  \textit{livre la la guérison}, \emph{Kitâb al-Shifâ'}. Ce livre permet d'apprécier l'érudition d'Avicenne que ce soit en médecine ou en théologie. La partie sur la métaphysique est connue en occident sous son nom latin \textit{Liber de philosophia prima sive scientia divina}.  Avicenne intègre donc la métaphysique dans un ensemble plus large, sur la guérison. Il manifeste ainsi sa vision de la philosophie comme une thérapeutique : la guérison ne concerne pas uniquement le corps mais aussi de l'esprit.


\subsection{Démonstration d'Avicenne}

\paragraph{Une démonstration \textit{géométrique}} Avicenne cherche à faire une démonstration de l'existence de Dieu dans la logique de la \textit{géométrie} axiomatique d'Euclide, et en se basant sur la définition de la notion de l'existence en tant existence sans recours à l'expérience.
. 
\begin{quote}
    [même] « s’il est presque évident (manifestum) par soi pour l’intelligence que tout ce qui commence a un principe, ce n’est pas pour cela que [cette proposition] doit être évidente par soi à la manière dont beaucoup de réalités géométriques prouvent les autres dans le livre d’Euclide » \sn{Avicenne, Philosophia prima I, 1 (I, 8, 33-36)}.
\end{quote} 
 

Mais il doit reconnaître la difficulté de l'approche : 
 
\begin{quote}
     « Mais nous, en raison de la faiblesse de notre âme, nous ne pouvons pas commencer par la voie démonstrative elle-même, qui procède des principes aux conséquents, et de la cause au causé, sauf dans certains ordres d’universalité au sein de ce qui est, sans descendre dans le particulier (sine praecisione). » \sn{Philosophia prima I, 3 (I, 23, 37-41).} 
\end{quote}

\paragraph{La méthode géométrique} L'idée d'Avicenne est de partir de notions disjonctives qui permettent d'appliquer le principe logique "ou bien ou bien". Ces principes disjonctifs seront par exemple : \textit{possible} et \textit{nécessaire}, \textit{interne} et \textit{externe}. \sn{Philosophia prima I, 1 (I, 6, 12-17) : « Inquirit enim universale et particulare, potentiam et effectum, possibile et necesse, et cetera. »} 

La catégorie la plus importante étant  le nécessaire (necesse), car il garde des affinités avec l’être transcendantal, pensé comme le « vehementiam essendi », la détermination à exister, l’affirmation de l’être (\cite{Boulnois:EtreRepresentationAvicenne}).

En utilisant les catégories de la géométrie euclidienne, Avicenne part de l'axiome de \textit{l'étant} : \textit{quelque chose existe}. 
Au sein de l’étant, Avicenne ajoute une disjonction : certains sont par eux-mêmes possibles, non nécessaires, tandis que les autres existent par eux-mêmes nécessairement (\textit{necesse esse per se}). Si cette chose qui existe est nécessaire, alors il y a un existant nécessaire, CQFD. 
\paragraph{Tout ce qui est possible a une cause} : le possible n'existe que si par rapport à sa cause il est nécessaire \sn{Philosophia prima I,6}. Le possible considéré en lui-même n'est déterminé ni à l'existence ni à la non-existence. Il faut cependant qu'il y ait un facteur qui le détermine dans un sens ou dans l'autre : un facteur ne peut être que quelque chose de distinct de la nature du possible, une cause extérieure.
    C'est donc cette cause qui déterminera le possible à l'une des deux alternatives, car si elle n'était pas capable de le faire, il faudrait faire appel à une troisième cause et ainsi à l'infini. 
    La thèse d'Avicenne signifie donc que le possible existera si sa cause le fait exister et qu'il n'existera pas si sa cause le fait pas exister.
    
Donc si cette chose qui existe est possible, alors elle a une cause.  A noter que l'existant possible, quant à son essence, restera toujours ce qu'il est, à savoir un existant possible : le fait d'être un existant possible n'est pas un caractère accidentel et passager \sn{Philosophia prima 1, 7}. l'existant possible ne peut donc devenir un être de soi nécessaire \cite[p. 52]{Avicenne:latinus}.
 
\paragraph{Passage à la totalité des choses possibles} Puis Avicenne passe à la totalité des choses possibles; cette totalité est soit nécessaire en soi, soit possible en soi. Or, en logique, la totalité ne peut être nécessaire en elle-même  puisqu'elle n'existe que par l'existence de ses membres, sans existence propre. Ainsi, la totalité des choses possibles a une cause.
Cette cause est soit interne à la totalité, soit externe à celle-ci.
Si elle est interne à la totalité, alors elle est soit nécessaire, soit possible.
Mais elle ne peut dans ce cas être nécessaire, car la totalité est constituée de choses possibles.
Et elle ne peut pas non plus dans ce cas être possible, puisqu'en tant que cause de toutes les choses possibles, elle serait dans ce cas sa propre cause, ce qui la rendrait nécessaire et non possible après tout, ce qui est une contradiction.
Ainsi, la cause de la totalité des choses possibles n'est pas interne à cette totalité, mais externe à elle.
Mais si elle est en dehors de la totalité des choses possibles, alors elle est nécessaire.
Il y a donc un existant nécessaire ou un \textit{nécessairement-être par soi}, qui est sans cause,  nécessaire par lui-même, incomparable, sans égal \sn{Philosophia prima I, 6 (I, 43, 9-18).}. 
 
 
\paragraph{Unicité du nécessairement-être.} De plus, le concept de nécessairement-être permet de démontrer l’unicité de son objet par l'absurde \sn{Philosophia prima I, 7 (I, 54, 33-35)}. 

% ----------------------------------------------
\subsection{Conséquence de cette démonstration}


\paragraph{Conséquence sur le libre Arbitre}

   Cette position ne conduit-elle pas à un déterminisme universelle ? En effet \sn{\cite[p.55]{Avicenne:latinus}},
si l'existence ou la non existence dépendent entièrement de la cause extérieure, ne faut il pas dire que tout est fixé et qu'il n'y a plus aucune marge laissée à la contingence ? En fait, Avicenne propose plusieurs limites.

Tout d'abord, d'un point de vue cosmologique, Avicenne reprend les 10 sphères célestes de Ptolémée, qui permettent de passer de la toute puissance du \textit{nécessairement-être} à l'homme : dans le monde sublunaire dans lequel nous vivons, la Dixième Intelligence  issue de l'Intelligence du 9° ciel (la Lune), mais sans fonction astronomique, revêt une importance singulière. Elle est aussi appelée \textit{intellect agent} et associée à Gabriel dans le Coran. Mais elle se situe si loin du Principe que son émanation éclate en une multitude de fragments. A chaque sphère, la part du principe diminue, ce qui permet le hasard et la liberté de l'homme.

\paragraph{Attributs de Dieu}
La critique de la démonstration par Al-Ghazali insiste sur l’incompatibilité avec Dieu tel qu'on le connait dans la religion musulmane. Comme Dieu est le \textit{nécessairement-être}, il ne peut avoir de propriétés ou relations contingentes : la causalité de l'univers doit être nécessaire. Al Ghazali réfute ceci comme étant incompatible avec le concept de volonté sans limite de Dieu tel que l'enseigne la théologie ash'arite, volonté sans limite qui aurait permis à Dieu ne pas créer l’univers.

% -------------------------------------------------------------------------------------------------------
\section{Conclusion}

Nous avons voulu montrer que ces démonstrations sont des clés de voûte de la métaphysique de ces théologiens, qui synthétisent les \textit{axiomes} de leurs pensées : « A quoi on tient en vrai ». Or, cette métaphysique, ce \emph{Kalām} est à son tour la source qui permet d’articuler les différentes sources du droit et in fine l’action du musulman.
La sophistication de ces démonstrations est là pour inclure certains principes au cœur de la Foi et non pour convaincre \textit{principalement} de l’existence de Dieu. 

Pour sortir d'un Islam pensé autour du droit, beaucoup de musulmans appellent de leur voeux un renouveau du \emph{Kalām}. Les \textit{preuves de l'existence de Dieu}, débarrassées de leurs visées apologétiques, peuvent être des \textit{synthèses utiles} des axiomes et fondements du \emph{Kalām}, permettant confrontations et discussions théologiques pour ensuite irriguer les autres questions métaphysiques, et in fine vivifier la foi musulmane et lui permettre de répondre aux nouvelles questions qu'elle doit affronter. 
	

%\chapter{Matériaux pour validation}

 

\begin{quote}
Il vous est naturellement possible d'en utiliser d'autres !
\end{quote}

\hypertarget{bibliographie-indicative}{%
\section{Bibliographie indicative}\label{bibliographie-indicative}}

\begin{quote}
H. Davidson, \emph{Proofs for Eternity, Creation and the Existence of
God in Medieval Islamic and Jewish Philosophy}, Oxford University Press,
1987.

W. Hallaq, ``Ibn Taymiyya on the Existence of God'', in \emph{Acta
Orientalia} 52 (1991), pp. 49-69

A. Shihadeh, ``The Existence of God'', in T. Winter, \emph{The Cambridge
Companion to Classical Islamic Theology}, pp. 197-217.
\end{quote}
 

% -------------------------------------------------------------
\section{Al Guwayni} 

\subsection{la preuve}
\begin{quote}
\textbf{PREUVE DE L'EXISTENCE DE DIEU} (Al-Ğuwaynī, \emph{Kitāb
al-Iršād}, 4 ; trad. Luciani)

Maintenant qu'il est prouvé que le monde est contingent et qu'il a eu un
commencement, il s'ensuit que le contingent peut exister ou ne pas
exister, et que quel soit le moment où il se produit, il aurait pu se
produire à un moment antérieur ; que l'existence du contingent aurait pu
être retardée de plusieurs heures au-delà de ce moment.

Si donc l'existence possible se produit, au lieu d'une prolongation
également possible de la non-existence, l'esprit saisit, comme une chose
évidente, que (pour se produire) l'existence a eu besoin d'une principe
déterminant (\emph{muḫaṣṣiṣ}) qui détermine sa réalisation. C'est là une
chose qui apparaît nécessairement, sans qu'il y ait besoin de faire des
distinctions ou d'employer le raisonnement.

Une fois admis le principe général que le contingent exige un principe
déterminant, il faut nécessairement que ce principe soit : ou bien une
cause nécessitant la réalisation de la contingence, comme la cause
nécessite son effet ; ou bien une force physique, comme le pensent les
naturalistes ; ou bien enfin un agent libre. Or il est faux que ce
principe déterminant agisse à la manière d'une cause. La cause en effet
nécessite son effet d'une façon simultanée. Si on supposait que le
principe déterminant fût une cause, celle-ci serait forcément ou
éternelle, ou contingent. Dans le premier cas, elle aurait dû
nécessairement provoquer l'existence du monde de toute éternité, ce qui
conduirait à admettre l'éternité du monde. Or nous avons fourni les
preuves de sa contingence. Si le principe était contingent, il aurait
lui-même besoin d'un principe déterminant, et ainsi de suite à l'infini.

Quant à ceux qui prétendent que le principe déterminant est une force
physique, leur théorie
est inadmissible. {[}\ldots{]}

S'il est faux, par conséquent, que le principe déterminant du contingent
soit une cause nécessitante, ou une force physique qui lui donne par
elle-même l'existence, mais involontairement, il s'ensuit d'une manière
certaine que le principe déterminant des choses
contingentes est un agent qui agit sur elles librement, qui leur assigne
spécialement, en les produisant, certains attributs et certains moments.
\end{quote}
 
 
 \begin{Synthesis}
 -Dieu comme principe déterminant. Il y a besoin d'un principe, démo par récurrence. Ressemble à la première démo d'Avicenne
 Comment Al gazali juge il cette démo ? 
 \end{Synthesis}



\subsection{ce qui wiki en dit}
Par exemple, il tente de démontrer l'existence de Dieu par la raison. Ce fait, à lui seul, montre qu'il a utilisé les outils de la philosophie. En outre, son raisonnement indique qu'il connaissait les concepts néo-platoniciens. Il commence par le constat de la contingence du monde : les choses auraient pu aussi bien ne pas être, ou être autrement. 
\begin{Synthesis}
 est ce la même chose d'Avicenne
\end{Synthesis}

Il faut donc supposer une intervention arbitraire, celle d'un créateur qui a fait un choix, celui de faire exister le monde, à un moment et sous une forme plutôt qu'une autre. L'idée d'un Dieu artisan rappelle le démiurge platonicien ; le concept de contingence a été introduit par Avicenne dans sa propre preuve de l'existence de Dieu. 


Pour Frank Griffel, l'influence directe d'Avicenne ne fait pas de doute : 
\begin{quote}
    « Al-Juwaynî fut le premier théologien musulman à étudier sérieusement les livres d'Avicenne »45(p. 47).
\end{quote} 
Mais selon Jan Thiele, al-Juwayni a pu aussi être inspiré par le mu'tazilite al-Basrī12. Paul Heck tranche ce débat en soulignant que l'atmosphère intellectuelle dans laquelle baignait al-Juwaynî était déjà imprégnée de l'influence d'Avicenne3. 
\begin{Synthesis}
Et Al-Juwaynī, s'il n'a pas inventé cette preuve, est du moins le premier des ash'arites à y avoir recours35.
\end{Synthesis}


Averroès en reconnaît l'originalité dans Al-Kashf ʿan manāhij al-adilla fī ʿaqāʾid al-milla. Il la réfute cependant, en mettant en doute la prémisse sur laquelle toute l'argumentation repose : l'ide que le monde est contingent. Dieu, en effet, dans sa sagesse, a pu choisir de créer le seul monde qui comportait le plus de perfection, et qui par conséquent s'imposait35.
\begin{Synthesis}
A la différence d'Averroes, ce n'est pas le meilleur des mondes. Explique le besoin de recourir au Coran pour éclairer certains comportements, en particulier des élites
\end{Synthesis}



\begin{Synthesis}
Al As'ari
- L'homme est changeant
- référence explicite au Coran sur ce que c'est l'homme
- après cette démonstration, ce qu'on sait de l'homme : c'est le Coran qui nous le donne mais aussi la raison.
\end{Synthesis}


\section{Démonstration d'avicenne}
voir aussi \href{https://books.google.fr/books?id=6jcTAQAAMAAJ&pg=RA1-PA6&lpg=RA1-PA6&dq=Inquirit+enim+universale+et+particulare,+potentiam+et+effectum,+possibile+et+necesse,+et+cetera&source=bl&ots=caRb6_MMwQ&sig=ACfU3U0jQ0xx5Be712aqaVYgR5qLnQa5Ng&hl=fr&sa=X&ved=2ahUKEwiGq8qvyNT3AhWvy4UKHXIHB5cQ6AF6BAgNEAM#v=onepage&q=vehementia&f=false}{Texte latin}

\begin{quote}
    la métaphysique d'avicenne fait partie du \textit{livre la la guérison}, Kitâb al-Shifâ'). La philosophie est une thérapeutique : la médecine ne vise que la guérison des corps. N'a-t-on pas besoin aussi d'une science capable de guérir l'esprit des hommes, l'esprit de ceux qui vivent dans le doute, l'incertitude et l'erreur ? 
\end{quote}
A la différence des intentions premières de la métaphysique, la logique utilise les intentions secondes. 

Suppose la physique : génération et corruption, altérité, lieu , temps sont justifiés en physique. 

\begin{quote}
    D'ailleurs, comment arrive t on à saisir les causes ? On ne peut y arriver qu'en partant des êtres causés données dans l'expérience. on aboutit ainsi à la même impasse que pour l'existence de Dieu : comment les causes au sens absolu pourraient elles être l'objet de la métaphysique puisque leur existence n'est pas donnée comme une évidence initiale, mais doit être démontrée.
    \ldots ce qu'on constat par la perception sensible, c'est la concomitance de certains phénomènes. On ne peut en conclure directement que l'un des deux est la cause de l'autre; surtout dans le cas où des phénomènes sont fréquemment associés, on sera porté à les croire unis par un lien de causalité. Pourtant la fréquence de cette union ne nous fournit concernant ce lien causal qu'une certaine probabilité, elle ne conduit pas à une connaissance certaine. 
\end{quote}

\begin{quote}
   qu'on ne s'y méprenne pas toutefois, cette doctrine d'Avicenne ne signifie pas que de façon absolue la puissance serait antérieure à l'acte. p 51
   
   L'existant possible, quant à son essence, restera toujours ce qu'il est, à savoir un existant possible : le fait d'être un existant possible n'est pas un caractère accidentel et passager (meta 1, 7 quicquid enim est possibile esse, respectu sui, semper est possibile esse.). Il est donc exclu que l'existant possible devienne un jour un être de soi nécessaire (p 52)
   il faut une cause extérieure pour devenir nécessaire de façon permanente. Durant une période limité, s(sans cause extérieure) à la condition qu'il y ait en lui un principe matériel. 53
\end{quote}

\paragraph{comment peut on concevoir le rapport entre le possible et sa cause ?}
\begin{quote}
    Avicenne répond que le possible n'existe que si par rapport à sa cause il est nécessaire (Meta I,6 possibile esse per se habet causam). Le possible considéré en lui-même n'est déterminé à l'existence ni à la non-existence. il faut cependant qu'il y ait un facteur qui le détermine dans un sens ou dans l'autre : ne facteur ne peut être que quelque chose de distinct de la nature du possible, une cause extérieure. (54)
    C'est donc cette cause qui déterminera le possible à l'une des deux alternatives, car si elle n'était pas capable de le faire, il faudrait faire appel à une troisième cause et ainsi à l'infini. 
    La thèse d'Avicenne signifie donc que le possible existera si sa cause le fait exister et qu'il n'existera pas si sa cause le fait pas exister.

\end{quote}
\paragraph{    Cette position ne conduit elle pas à un déterminisme universelle ?}
\begin{quote}
Si l'exisence ou la non existence dépendent entièrement de la cause extérieure, ne faut il pas dire que tout est fixé et qu'il n'y a plus aucune marge laissée à la contingence ? nous ne le croyons pas : lorsque notre auteur parle de différentes espèces de puissance il fait une distinction entre les oeuvres de la nature, ce qui est acquis par habitude, les produits de l'art et les résultats du hasard; cependant l'acquisition d'une habitude et l'apprentissage d'une capacité technique se font de la même manière. Si le possible est amené à l'existence sous l'action d'une cause, il reste à préciser de quelle nature est cette action : activité déterminée ou initiative autonome et libre.  ... ceci n'exclut pas cependant que l'intervention de cette cause puisse être une initiative autonome.55
    
    Le possible ne trouve pas sa raison d'être en lui-même, alors que le nécessaire la possède dans sa constitution même. En outre, ce qui est de soi nécessaire ne peut être égale ou équivalente à une autre existence (55 ( 1,6)
    Simplicité de l'existant nécessaire (il s'oppose au possible qui inclut toujours une certaine composition que ce soit celle de nbature et d'existence ou celle de matière et de forme. ). 
    Avicenne en conclut que l'existant nécessaire n'est pas relatif, pas changeant (il ne peut passer de puissance à acte puisque sous aucun aspect il n'est en puissance), il n'est pas multiple  et il ne partage l'existence qui lui est propre avec aucun autre être. 56
    
    Tous ces arguments d'avicenne se réduisent en somme à une considération fondamentale : \textsc{l'être nécessaire dont il est question est un être qui est de soi nécessaire}. par son essence, il est ce qu'il est à savoir existant nécessaire. 62
    C'est pourquoi Aristote admet un acte pur qui est à la base du devenir qui se produit dans le monde. Avicenne a voulu approfondir cette théorie d'Aristote.
    

\end{quote}

\paragraph{la question de l'être}

\begin{quote}
    la notion d'être générique ou spécifique, c'est à dire une notion capable d'être déterminée par l'addition de certains caractères ou qualificatifs.
    être ; sens univoque ? caractère analogique de l'être (Thomas d'Aquinà). Le problème est important car la validité et le sens du langage métaphysique sur Dieu en dépendent (65)
    Pour avicenne, sens générique et univoque.
\end{quote}

% ----------------------------------------------------------------------------

\subsection{Une généalogie de la métaphysique moderne à l'époque de Duns Scot (XIIIe - XIVe siècle)}
\url{https://www.cairn.info/etre-et-representation--9782130504566-page-327.htm}

Être et représentation
Une généalogie de la métaphysique moderne à l'époque de Duns Scot (XIIIe - XIVe siècle)
Par Olivier Boulnois
Année : 1999
Pages : 544
Collection : Épiméthée
Éditeur : Presses Universitaires de France



40Pour Avicenne, l’idéal de la connaissance de Dieu est celui de la plus haute science, qui procéderait de manière géométrique, par une déduction à partir de propositions primitives évidentes par elles-mêmes, suivant le modèle des Éléments d’Euclide. Avicenne forme le projet d’une métaphysique en soi, qui atteindrait l’évidence de Dieu à la mesure de son intelligibilité, en apercevant la nécessité de son existence dans la contemplation de son essence. Comme la démonstration géométrique, elle atteindrait ses conclusions à partir de la définition de l’essence. Mais la réalité de la preuve ne peut se maintenir au niveau de cet air raréfié. Elle doit être obtenue à partir de l’expérience. Même 
\begin{quote}
    « s’il est presque évident (manifestum) par soi pour l’intelligence que tout ce qui commence a un principe, ce n’est pas pour cela que [cette proposition] doit être évidente par soi à la manière dont beaucoup de réalités géométriques prouvent les autres dans le livre d’Euclide » \sn{Avicenne, Philosophia prima I, 1 (I, 8, 33-36) : « Si paene fuerit manifestum per se apud intelligentiam quod quidquid coepit habet principium aliquod, ideo debet esse manifestum per se, sicut multa ex rebus geometricis per quae probantur cetera in libre Euclidis… »}.
\end{quote} 


\begin{Synthesis}
Une démonstration géométrique. 
\end{Synthesis}
41 Malheureusement, cette métaphysique more geometrico, qui irait a priori, des principes aux conséquences, ne nous est pas accessible.
\begin{quote}
     « Mais nous, en raison de la faiblesse de notre âme, nous ne pouvons pas commencer par la voie démonstrative elle-même, qui procède des principes aux conséquents, et de la cause au causé, sauf dans certains ordres d’universalité au sein de ce qui est, sans descendre dans le particulier (sine praecisione). » \sn{Philosophia prima I, 3 (I, 23, 37-41).} A défaut, nous pouvons passer de l’expérience à ses conditions de possibilité rationnelles, en discernant sous le sensible des ordres d’universalité \sn{In Met. I, q. 1, [6] 22 (III, 22) : « Deum esse desperatum cognosci non est, nec quaesitum in alia scientia, nec in ista secundum se, quamvis quoad nos fait notum ex effectibus, sicut procedit ratio. Potest enim aliquid secundum se notius, fieri nobis notum ex aliis nobis notioribus. »}.
\end{quote}


42En effet, nous avons des notions élémentaires, inscrites d’emblée dans l’intellect : chose, étant, nécessaire (necesse) \sn{Philosophia prima I, 5 (I, 31-32, 2-4).} . Avicenne préconise d’utiliser des couples de propriétés relatives et disjonctives : 
\begin{quote}
    « Cette science étudie les notions (intentiones) qui ne proviennent pas des accidents propres de ces causes en tant que causes. Elle étudie en effet l’universel et le particulier, la puissance et l’acte (effectum), le possible et le nécessaire, etc. » \sn{Philosophia prima I, 1 (I, 6, 12-17) : « Inquirit enim universale et particulare, potentiam et effectum, possibile et necesse, et cetera. »}
\end{quote}
 A chaque fois, le terme aperçu dans la créature permet d’inférer l’existence de son corrélat, qui est Dieu. Selon ce schéma, on constituera a priori des différences d’ordre au sein de l’étant. Dans chaque cas, il y aura un antérieur et un postérieur, Dieu et la créature prise sous un certain « ordre d’universalité », selon le vocabulaire même d’Avicenne. Cette dualité peut se dire en une série de couples transcendantaux, qui permettent tous d’inférer l’existence de Dieu. 
 \begin{Synthesis}
 Discussion sur les attributs de Dieu ? On aperçoit un terme dans la créature et on regarde le corrélat divin.
 Necessaire : le plus important; "la véhemence / l'intensité à exister"
 \end{Synthesis}
 
 – Or le plus haut objet de pensée est le nécessaire (necesse), car l’objet de la science est par excellence ce qui ne peut pas être autrement qu’il n’est ; et le nécessaire garde des affinités avec l’être transcendantal, car il signifie la « vehementiam essendi », la détermination à exister, l’affirmation de l’être. Il est plus connaissable que le possible et l’impossible, car l’être est plus connu que le non-être, l’être étant connu par soi-même et le non-être par lui \sn{Philosophia prima I, 5 (I, 41, 79-82) ; cf. Aristote, Seconds analytiques I, 2, 71 b 15.}. – Le necesse, comme nécessairement être, est convertible avec l’actualité d’être. Mais à cette continuité il faut ajouter une disjonction : au sein de l’étant, certains sont par eux-mêmes possibles, non nécessaires, tandis que les autres existent par eux-mêmes nécessairement (necesse esse per se). Le nécessairement-être par soi est sans cause, il est nécessaire par lui-même, incomparable, sans égal \sn{Philosophia prima I, 6 (I, 43, 9-18).}. 
 
 – De plus, le concept de nécessairement-être permet de démontrer l’unicité de son objet \sn{Cela peut être prouvé par l’absurde : s’il y a plusieurs nécessairement êtres distincts, ils sont distincts par un ajout fait à leur essence. Cet élément est-il nécessaire à leur nécessité propre ? Si oui, ils n’en diffèrent donc pas. Sinon, il faut distinguer deux aspects : la nécessité additive qui caractérise l’un d’eux est ce qui l’affecte et le rend nécessaire, alors qu’ils ont en eux-mêmes un fond commun non-nécessaire. Ce qui veut dire que l’autre n’est pas nécessaire, hypothèse contraire aux prémisses.} : celui qui est nécessairement, comme détermination à exister par lui-même, est unique, et réciproquement, tout ce qui est nécessairement par soi est lui-même \sn{Philosophia prima I, 7 (I, 54, 33-35) : « Si autem fuerit per se ipsum quod ipsum est necesse esse, erit idem ipsum tunc quicquid est necesse esse. »}. Le nécessairement-être est un étant dont l’essence est la nécessité ; il est donc numériquement unique ; son mode d’être n’est communicable à aucun autre \sn{Philosophia prima I, 7 (I, 53, 20). Comme le dira justement Schelling, « Dieu n’est pas simplement l’étant nécessaire, mais il est nécessairement l’étant nécessaire » ; Philosophie der Offenbarung, I, 8, SW, t. XIII, p. 159 ; tr. fr. Philosophie de la Révélation (I, 185).}. La voie part donc de notre premier concept, celui de l’étant, et aboutit nécessairement à l’unique nécessairement-être.



43Peut-on inventer une démonstration plus haute que celle d’Avicenne, conforme à son projet de science métaphysique en soi ? La science métaphysique, qui démontre l’existence de Dieu, peut-elle procéder a priori, comme la géométrie ? Bonaventure a entrevu la possibilité d’une démonstration a priori de Dieu, mais sans lui accorder un privilège particulier dans ses nombreux arguments en faveur de l’existence de Dieu \sn{QD De mysterio Trinitatis q. 1, a. 1 (V, 45 a) : trois voies permettent de démontrer l’existence de Dieu ; 1 / l’innéité de Dieu dans la pensée : il y a dans toutes les âmes rationnelles une certitude de l’existence de Dieu ; 2 / la relation du postérieur à l’antérieur ; 3 / l’argument anselmien : l’existence de Dieu est une vérité telle qu’on ne peut pas penser Dieu sans penser qu’il existe.}. Selon la seconde voie vers Dieu, fondée sur la relation de l’antérieur et du postérieur, toute créature implique l’existence de son principe et proclame son existence comme une vérité indubitable. 
\begin{quote}
    « S’il y a un étant postérieur, il y aussi un étant antérieur, car le postérieur n’existe que par l’antérieur (a priori) : si donc l’universalité des postérieurs existe, il est nécessaire qu’un premier existe. Si donc il est nécessaire de poser qu’il y a quelque chose d’antérieur et de postérieur dans les créatures, il est nécessaire que l’universalité des créatures implique et proclame le premier principe. » [75]
\end{quote}
 Ce texte joue sur deux sens de l’apriorité : être au préalable, venir d’un antérieur. Retourner vers l’intelligible, c’est pour Bonaventure remonter à la fois dans l’ordre de l’être et dans celui de l’être, se hausser vers l’universel et retrouver le divin, puisque l’intelligible pur est à la fois la condition de ma pensée (illumination) et celle de l’être (émanation). L’ambiguïté de la voie a priorivient du fait que Bonaventure admet aussi l’innéité du concept de Dieu dans l’âme : si l’a priori est atteint par inférence à partir du postérieur, il est aussi d’avance (a priori) connu comme son corrélat. En donnant une interprétation ontologique de l’a priori, Bonaventure identifie l’être premier et l’être universel [76] : l’être par excellence (Dieu) et l’être comme premier objet de la pensée coïncident tangentiellement, dans ce qui constitue littéralement le principe de l’analogie.

44Mais Bonaventure ne s’attarde pas sur les questions de méthode impliquées par une telle démarche, ni sur la relation entre cet argument et les suivants : il les énumère et les juxtapose sans s’interroger sur le fait que le premier donne une structure générale alors que les autres en sont des réalisations particulières. L’a priori est donc entrevu en un moment décisif de la seconde voie, mais il n’est pas pensé comme son principe constitutif [77]. Or cette voie repose sur des couples de concepts a priori tirés d’un aspect du réel, et permettant d’inférer l’existence de leur terme premier : celui-ci est donc connu, non comme cause, à la manière d’Aristote ou de Thomas d’Aquin, mais comme principe, en tant que premier dans un couple de différences construit a priori, par une relation d’antéropostériorité. La « voie de la raison » remonte de l’être imparfait à l’être parfait, mais elle peut prendre pour point de départ de multiples concepts : les créatures sont par essence déficientes, mais par essence elles impliquent le concept de la perfection correspondante [78]. L’intellect pense partir de données purement sensibles, l’être muable, relatif, composé, contingent, mais ces imperfections apparaissent comme imparfaites parce qu’il possédait déjà, antérieurement, a priori, le concept des perfections qui les mesurent. Lorsqu’il entreprend de démontrer l’existence de Dieu, l’intellect prend conscience du fait qu’il la connaissait déjà.

 

[75]
De mysterio Trinitatis q. 1, a. 1, arg. 11 (V, 46b) : « Si est ens posterius, est et ens prius, quia posterius non est nisi a priori : si ergo est universitas posteriorum, necesse est, esse ens primum. Si ergo necesse est ponere aliquid esse prius et posterius in creaturis ; necesse est universitatem creaturarum inferre et clamare primum principium. »
[76]
L’Itinéraire de l’esprit vers Dieu V, 3, identifie l’être pur (esse purum) et l’être divin (esse divinum) : « Esse nominat ipsum purum actum entis : esse igitur est quod primo cadit in intellectu […]. Sed hoc non est esse particulare, […] nec esse analogum […] restat igitur, quod illud esse est esse divinum » (Duméry, p. 84). Collationes in Hexaemeron X, 10< | > : « Primum speculabile est Deum esse. Primum nomen Dei est esse, quod est manifestissimum et perfectissimum, ideo primum ; unde nihil manifestius » (tr. fr. p. 269). Bonaventure joue sur deux sens de « esse » (être/exister) : rien n’est plus manifeste que l’existence de Dieu, parce que l’être est le nom le plus propre de son essence, et sur deux sens de « premier » : le plus manifeste (manifestissimum) et le plus excellent (perfectissimum).



Avicenne
- démonstration géometrique : pas besoin du coran pour démontrer Dieu. 
- Dieu est accessible à l'homme sans le Coran
- mais pas très accessible (la démonstration de Al Guwayni est trop simple ?) voir si il lui répond vraiment

- parler de l'intellect 



\hypertarget{annexe}{%
\section{Annexe}\label{annexe}}

\section{Les différentes preuves de Dieu EU}
Au long de l'histoire de la philosophie, les preuves de l'existence de Dieu varient selon le type d'argument choisi pour les fonder. Le philosophe peut partir de l'expérience qu'il fait de la contingence du monde, et en inférer, se plaçant à différents points de vue, l'existence nécessaire d'un Dieu soutenant dans l'être et expliquant à la pensée la contingence de l'expérimenté. C'est ainsi que tout mouvement (entendons tout devenir) impliquerait, d'objet mû en moteur et de moteur en objet mû, la nécessité d'un premier moteur immobile, suprême cause. Il en va pareillement pour l'existence même de ce qui se meut, qui impliquerait l'agir absolu d'une pure existence subsistante, seule capable de faire passer tout être du simple et indigent pouvoir d'exister à l'existence de fait. Ainsi raisonne Aristote (Physique, VIII ; Métaphysique, Λ), repris, à travers Avicenne et Maimonide, par Thomas d'Aquin (ce sont les trois premières voies du Contra Gentiles et de la Somme théologique). Tel est l'esprit des preuves dites cosmologiques ou a contingentia mundi.

\begin{Synthesis}
Preuve cosmologique : de Aristote à Avicenne : il faut une cause.
\end{Synthesis}


Le philosophe peut aussi considérer l'ordre du monde, la finalité qu'il y discerne, la beauté qu'il y contemple, etc., et, se refusant à y voir l'effet du hasard, affirmer l'action suprêmement intelligente d'un Dieu organisateur ultime du cosmos. Cet argument, dit téléologique ou physico-théologique, s'enracine chez Platon et Aristote. Thomas d'Aquin l'utilise (cinquième voie de la Somme théologique) ; Leibniz ne le néglige point (Nouveaux Essais) ; Voltaire y acquiesce. Le philosophe peut encore être sensible au fait que les perfections qu'il constate dans le monde s'y manifestent selon des degrés et, de là, inférer la nécessaire existence d'un absolu divin de perfection. On trouverait des amorces de cet argument chez Platon et Aristote, chez les néo-platoniciens et ceux qui, comme Augustin (De libero arbitrio), ont subi leur influence. Anselme de Cantorbéry (Monologion), Thomas d'Aquin (quatrième voie de la Somme théologique), Descartes, Bossuet l'ont utilisé ou s'en sont inspirés. Ces différentes preuves n'en font qu'une, dans la mesure où elles ont en commun d'aller de l'expérience prise comme conséquence à son principe ; elles procèdent a posteriori. 
\begin{Synthesis}
Partir du monde et refuser d'y voir le hasard : Al Ashari (la boue ne se transforme pas en brique)
\end{Synthesis}
Mais certains penseurs inversent le processus et, considérant la seule idée de Dieu et ses notes constitutives, en infèrent l'existence nécessaire de ce Dieu sans qui, selon eux, il ne saurait y avoir d'idée de Dieu. Anselme de Cantorbéry (Proslogion), le premier, utilisa cet argument, qui fut retrouvé plus tard, sans doute au niveau de la critique qu'en fit le thomisme, par Descartes, lequel le présenta sous des formes originales (Discours de la méthode, 4 ; Méditation cinquième). 
\begin{Synthesis}
Idée de Dieu ne peut être sans Dieu : preuve ontologique pour Kant et en fait caché dans toutes les autres.
\end{Synthesis}
Leibniz l'utilisa selon ses propres perspectives (Nouveaux Essais, Monadologie). Procédant de l'idée à l'existence, cet argument a priori a été dit par Kant « ontologique ». À côté de ces preuves classiques, de nature logique, il faut mentionner la preuve dite morale, où la postulation d'un Dieu apparaît comme seule capable d'accomplir les requêtes de la conscience morale : c'est la position de Kant (Critique de la raison pratique), ainsi que le chemin philosophique vers Dieu que trace M. Nédoncelle à partir de l'existence, inexplicable autrement, selon lui, de l'ordre des personnes humaines.

Les diverses preuves de l'existence de Dieu ont toujours rencontré faveur et défaveur, soit en particulier, soit en bloc. 
\begin{quote}
    « La critique la plus cohérente et la plus ferme qui ait été jamais opposée aux preuves traditionnelles [...] est celle que Kant développe dans la Critique de la raison pure, et dont Hegel lui-même reconnaît qu'elle est la seule à avoir écarté de façon « scientifique » ces preuves » (D. Dubarle)
\end{quote}
. Kant, en effet, fait apparaître sous les preuves cosmologique et téléologique, apparemment autonomes, un recours subreptice et obligé à « cette malheureuse preuve ontologique » (Critique de la raison pure), elle-même dépourvue de toute valeur, dès lors que l'existence réelle de Dieu, qu'on est censé découvrir impliquée dans le concept de Dieu, n'est point, en fait, une perfection analytiquement déductible, mais bien une détermination extérieure au concept analysé, et d'un autre ordre que lui. Du concept à l'existence, la conséquence ne vaut pas. Thomas d'Aquin, après Gaunilon, avait d'ailleurs opposé à l'argument d'Anselme une objection analogue. Lui, du moins, accordait une valeur probante aux preuves a posteriori, où l'affirmation de Dieu venait tout naturellement à sa place au sein d'une représentation prégaliléenne de l'Univers, aujourd'hui sans rapport avec les modernes cosmologies.
\begin{Synthesis}
 Anselme et Avicenne : preuves a posteriori : représentation pregaliléenne de l'univers.
\end{Synthesis}

D'un point de vue plus général, le propos même de prouver l'existence de Dieu se voit opposer des objections de principe. Ce peut être en raison de la transcendance de l'objet, qui le fait échapper par définition à toute insertion dans un plan purement intellectuel. Le croyant Pascal (dont le pari n'entendait rien prouver) n'accordait aux preuves traditionnelles aucune valeur probante : « Une heure après, ils craignent de s'être trompés » (éd. Brunschvicg, 543) ; Kierkegaard, pas davantage. L'incroyant, enfin, peut-il être convaincu par une démarche rationnelle qu'il voit, chez le croyant, se déroulant tout entière portée par une foi préalable qu'il ne partage point ? Au croyant la foi donne non une image du monde, mais le monde tel qu'il est, avec Dieu en son centre ; cela, l'incroyant le sait ; il ne peut dès lors regarder la vision du monde engendrée par la foi autrement que sous l'angle d'une axiomatique.
\begin{Synthesis}
Objection de principe aux preuves de l'existence de Dieu qui ne respectent pas la transcendance de Dieu. Cf Levinas et homme Dieu
\end{Synthesis}

POUR CITER L’ARTICLE
Lucien JERPHAGNON, « DIEU PREUVES DE L'EXISTENCE DE », Encyclopædia Universalis [en ligne], consulté le 7 mai 2022. URL : http://www.universalis-edu.com.icp.idm.oclc.org/encyclopedie/preuves-de-l-existence-de-dieu/


\section{Preuve de l'existence de Dieu Avicenne Wikipedia}

Proof of the Truthful
From Wikipedia, the free encyclopedia



The Proof of the Truthful[1] (Arabic: \TArabe{ برهان الصديقين,} romanized: burhān al-ṣiddīqīn,[2] also translated Demonstration of the Truthful[2] or Proof of the Veracious,[3] among others) is a formal argument for proving the existence of God introduced by the Islamic philosopher Avicenna (also known as Ibn Sina, 980–1037). Avicenna argued that there must be a "necessary existent" (Arabic:\TArabe{ واجب الوجود,} romanized: wājib al-wujūd), an entity that cannot not exist.[4] The argument says that the entire set of contingent things must have a cause that is not contingent because otherwise it would be included in the set. Furthermore, through a series of arguments, he derived that the necessary existent must have attributes that he identified with God in Islam, including unity, simplicity, immateriality, intellect, power, generosity, and goodness.[5]

\begin{Synthesis}
La démonstration d'un Dieu par géométrie, puis les principales propriétés de Dieu.
A noter Levinas, l'homme Dieu, sur l'unité sans entrer dans l'ordre.
\end{Synthesis}
 
Critics of the argument include Averroes, who objected to its methodology, Al-Ghazali, who disagreed with its characterization of God, and modern critics who state that its piecemeal derivation of God's attributes allows people to accept parts of the argument but still reject God's existence. There is no consensus among modern scholars on the classification of the argument; some say that it is ontological while others say it is cosmological.[6]



Origin
The argument is outlined in Avicenna's various works. The most concise and influential form is found in the fourth "class" of his Remarks and Admonitions (Al-isharat wa al-tanbihat).[7] It is also present in Book II, Chapter 12 of the Book of Salvation (Kitab al-najat) and throughout the Metaphysics section of the Book of Healing (al-Shifa).[8] The passages in Remarks and Admonitions draw a distinction between two types of proof for the existence of God: the first is derived from reflection on nothing but existence itself; the second requires reflection on things such as God's creations or God's acts.[1][9] Avicenna says that the first type is the proof for "the truthful", which is more solid and nobler than the second one, which is proof for a certain "group of people".[10][11] According to the professor of Islamic philosophy Shams C. Inati, by "the truthful" Avicenna means the philosophers, and the "group of people" means the theologians and others who seek to demonstrate God's existence through his creations.[10] The proof then became known in the Arabic tradition as the "Proof of the Truthful" (burhan al-siddiqin).[2]

Argument
The necessary existent
Avicenna distinguishes between a thing that needs an external cause in order to exist – a contingent thing – and a thing that is guaranteed to exist by its essence or intrinsic nature – a necessary existent.[12] The argument tries to prove that there is indeed a necessary existent.[12] It does this by first considering whether the opposite could be true: that everything that exists is contingent. Each contingent thing will need something other than itself to bring it into existence, which will in turn need another cause to bring it into existence, and so on.[12] Because this seemed to lead to an infinite regress, cosmological arguments before Avicenna concluded that some necessary cause (such as God) is needed to end the infinite chain.[13] However, Avicenna's argument does not preclude the possibility of an infinite regress.[12][13]

Instead, the argument considers the entire collection (jumla) of contingent things, the sum total of every contingent thing that exists, has existed, or will exist.[12][13] Avicenna argues that this aggregate, too, must obey the rule that applies to a single contingent thing; in other words, it must have something outside itself that causes it to exist.[12] This cause has to be either contingent or necessary. It cannot be contingent, though, because if it were, it would already be included within the aggregate. Thus the only remaining possibility is that an external cause is necessary, and that cause must be a necessary existent.[12]

Avicenna anticipates that one could reject the argument by saying that the collection of contingent things may not be contingent. A whole does not automatically share the features of its parts; for example, in mathematics a set of numbers is not a number.[14] Therefore, the objection goes, the step in the argument that assumes that the collection of contingent things is also contingent, is wrong.[14] However, Avicenna dismisses this counter-argument as a capitulation, and not an objection at all. If the entire collection of contingent things is not contingent, then it must be necessary. This also leads to the conclusion that there is a necessary existent, the very thing Avicenna is trying to prove. Avicenna remarks, "in a certain way, this is the very thing that is sought".[14]

From the necessary existent to God
Bismillahir Rahmanir Rahim
Part of a series on
God in Islam
"Allah" in Arabic calligraphy
Allah Jalla Jalālah
in Arabic calligraphy
List
Allah
Names
Phrases and expressions
Theology
Oneness
Islamic creed
Transcendence
Denial of Divine attributes
Anthropomorphism
Allah-green.svg Islam portal • Category
vte
The limitation of the argument so far is that it only shows the existence of a necessary existent, and that is different from showing the existence of God as worshipped in Islam.[5] An atheist might agree that a necessary existent exists, but it could be the universe itself, or there could be many necessary existents, none of which is God.[5] Avicenna is aware of this limitation, and his works contain numerous arguments to show the necessary existent must have the attributes associated with God identified in Islam.[14]

For example, Avicenna gives a philosophical justification for the Islamic doctrine of tawhid (oneness of God) by showing the uniqueness and simplicity of the necessary existent.[15] He argues that the necessary existent must be unique, using a proof by contradiction, or reductio, showing that a contradiction would follow if one supposes that there were more than one necessary existent. If one postulates two necessary existents, A and B, a simplified version of the argument considers two possibilities: if A is distinct from B as a result of something implied from necessity of existence, then B would share it, too (being a necessary existent itself), and the two are not distinct after all. If, on the other hand, the distinction resulted from something not implied by necessity of existence, then this individuating factor will be a cause for A, and this means that A has a cause and is not a necessary existent after all. Either way, the opposite proposition resulted in contradiction, which to Avicenna proves the correctness of the argument.[16] Avicenna argued that the necessary existent must be simple (not a composite) by a similar reductio strategy. If it were a composite, its internal parts would need a feature that distinguishes each from the others. The distinguishing feature cannot be solely derived from the parts' necessity of existence, because then they would both have the same feature and not be distinct: a contradiction. But it also cannot be accidental, or requiring an outside cause, because this would contradict its necessity of existence.[17]

Avicenna derives other attributes of the necessary existent in multiple texts in order to justify its identification with God.[5] He shows that the necessary existent must also be immaterial,[5] intellective,[18] powerful,[5] generous,[5] of pure good (khayr mahd),[19] willful (irada),[20] "wealthy" or "sufficient" (ghani),[21] and self-subsistent (qayyum),[22] among other qualities. These attributes often correspond to the epithets of God found in the Quran.[21][22] In discussing some of the attributes' derivations, Adamson commented that "a complete consideration of Avicenna's derivation of all the attributes ... would need a book-length study".[23] In general, Avicenna derives the attributes based on two aspects of the necessary existent: (1) its necessity, which can be shown to imply its sheer existence and a range of negations (e.g. not being caused, not being multiple), and (2) its status as a cause of other existents, which can be shown to imply a range of positive relations (e.g. knowing and powerful).[24]

Reaction
Reception
Present-day historian of philosophy Peter Adamson called this argument one of the most influential medieval arguments for God's existence, and Avicenna's biggest contribution to the history of philosophy.[4] Generations of Muslim philosophers and theologians took up the proof and its conception of God as a necessary existent with approval and sometimes with modifications.[4] The phrase wajib al-wujud (necessary existent) became widely used to refer to God, even in the works of Avicenna's staunch critics, a sign of the proof's influence.[2] Outside the Muslim tradition, it is also "enthusiastically"[2] received, repeated, and modified by later philosophers such as Thomas Aquinas (1225–1274) and Duns Scotus (1266–1308) of the Western Christian tradition, as well by Jewish philosophers such as Maimonides (d. 1204).[2][4]

Adamson said that one reason for its popularity is that it matches "an underlying rationale for many people's belief in God",[2] which he contrasted with Anselm's ontological argument, formulated a few years later, which read more like a "clever trick" than a philosophical justification of one's faith.[2] Professor of medieval philosophy Jon McGinnis said that the argument requires only a few premises, namely, the distinction between the necessary and the contingent, that "something exists", and that a set subsists through their members (an assumption McGinnis said to be "almost true by definition").[25]

Criticism
The Islamic Andalusi philosopher Averroes or Ibn Rushd (1126–1198) criticized the argument on its methodology. Averroes, an avid Aristotelian, argued that God's existence has to be shown on the basis of the natural world, as Aristotle had done. According to Averroes, a proof should be based on physics, and not on metaphysical reflections as in the Proof of the Truthful.[26] Other Muslim philosophers such as Al-Ghazali (1058–1111) attacked the argument over its implications that seemed incompatible with God as known through the Islamic revelation. For example, according to Avicenna, God can have no features or relations that are contingent, so his causing of the universe must be necessary.[26] Al-Ghazali disputed this as incompatible with the concept of God's untrammelled free will as taught in Al-Ghazali's Asharite theology.[27] He further argued that God's free choice can be shown by the arbitrary nature of the exact size of the universe or the time of its creation.[27]

Peter Adamson offered several more possible lines of criticism. He pointed out that Avicenna adopts a piecemeal approach to prove the necessary existent, and then derives God's traditional attribute from it one at a time. This makes each of the arguments subject to separate assessments. Some might accept the proof for the necessary existent while rejecting the other arguments; such a critic could still reject the existence of God.[15] Another type of criticism might attack the proof of the necessary existent itself. Such a critic might reject Avicenna's conception of contingency, a starting point in the original proof, by saying that the universe could just happen to exist without being necessary or contingent on an external cause.[26]

Classification
German philosopher Immanuel Kant (1724–1804) divided arguments for the existence of God into three groups: ontological, cosmological, or teleological.[28] Scholars disagree on whether Avicenna's Proof of the Truthful is ontological, that is, derived through sheer conceptual analysis, or cosmological, that is, derived by invoking empirical premises (e.g. "a contingent thing exists").[5][25][28] Scholars Herbert A. Davidson, Lenn E. Goodman, Michael E. Marmura, M. Saeed Sheikh, and Soheil Afnan argued that it was cosmological.[29] Davidson said that Avicenna did not regard "the analysis of the concept necessary existent by virtue of itself as sufficient to establish the actual existence of anything in the external world" and that he had offered a new form of cosmological argument.[29] Others, including Parviz Morewedge, Gary Legenhausen, Abdel Rahman Badawi, Miguel Cruz Hernández, and M. M. Sharif, argued that Avicenna's argument was ontological.[28] Morewedge referred to the argument as "Ibn Sina's ontological argument for the existence of God", and said that it was purely based on his analytic specification of this concept [the Necessary Existent]."[28] Steve A. Johnson and Toby Mayer said the argument was a hybrid of the two.[25][28]

References

\section{avicenne wikipedia}

Avicenne reprend la théorie aristotélicienne des quatre causes. Mais il est le premier à concevoir la causalité efficiente de Dieu (c'est-à-dire une causalité créatrice), par opposition à la causalité motrice aristotélicienne (qui était seulement un principe de mouvement, mais non une cause d'existence ex nihilo)55,56.

Ibn-Sina distingue ainsi la philosophie naturelle, ou la physique, et la théologie, ou la métaphysique. Le métaphysicien tient un discours sur la cause très différent du naturaliste :

« Par “agent”, le métaphysicien ne veut pas seulement dire le principe du mouvement, comme le naturaliste veut le dire, mais le principe et l'origine de l'existence, comme dans le cas de Dieu à l'égard du monde57. »

Spécialiste de sa pensée, Kara Richardson donne une définition importante et contextualisée : « In his Metaphysics, Ibn Sīnā defines each of the four causes in relation to the subject studied in metaphysics : the existent qua existent. He defines the efficient cause or agent as that which gives or bestows the existence of something distinct from it58. »

C'est en ce sens qu'Avicenne écrit : « La cause est pour l’existence seulement » (Kitāb al-Shifā, ou Livre de la guérison, Livre VI, chap. 1).

Les théologiens chrétiens, tels que Albert le Grand et Thomas d'Aquin, le citent dans leurs œuvres et lui sont redevables de cette invention majeure59.

L'essence, pour Avicenne, est non-contingente, ne dépendant que d'elle-même. Possible est chaque essence dans son potentiel à être. Pour qu'une essence soit actualisée dans une instance (une existence), il faut un accident nécessaire. Cette relation de cause à effet, toujours parce que l'essence n'est pas contingente, est inhérente à l'essence elle-même. Ainsi il doit exister une essence nécessaire en elle-même pour que l'existence puisse être possible : l'Être nécessaire, ou encore Dieu60.

L'Être nécessaire est Un - c'est à la fois la conception qu'en a Plotin (το ου), mais aussi le dogme musulman (tawhīd, unicité et unité). La difficulté est alors d'expliquer l'origine de la pluralité des êtres. Comment, de l'unité, peut naître la multiplicité54,61 ?

L'Être nécessaire crée la Première Intelligence par émanation (ou « procession »), notion typiquement plotinienne54. Cette définition altère profondément la conception de la création : il ne s'agit plus d'une divinité créant par caprice, mais d'une pensée divine qui se pense elle-même ; le passage de ce premier être à l'existant est une nécessité et non plus une volonté. Le monde émane alors de Dieu par surabondance de Son Intelligence, suivant ce que les néoplatoniciens ont nommé émanation : une causalité immatérielle. La venue au monde par procession ou émanation heurte la théologie asharite qui souligne le volontarisme divin : la Création est l'effet de la volonté libre et arbitraire d'Allah54,62. C'est pourquoi Fakhr al-Din Al-Razi, qui introduit certaines thèses d'Avicenne dans la théologie asharite, ne suit pas Ibn Sina sur ce point : l'idée d'une Création nécessaire, et non volontaire, ne convient pas à sa représentation de la Toute-puissance divine63. Avicenne s'inspire des travaux d'al-Farabi, mais à cette différence que c'est l'Être nécessaire qui est à l'origine de tout (voir plus bas les Dix intelligences)60. Cette perspective serait donc plus compatible avec le Coran.

« Chaque Intelligence, à l'exception de la dernière de la série, engendre en premier lieu l'Intelligence qui lui est immédiatement inférieure à travers l'acte par lequel elle connaît le Premier Être, puis l'âme de sa sphère à travers l'acte par lequel elle se connaît comme nécessaire en vertu du Premier Être, et en troisième lieu le corps de cette sphère à travers l'acte par lequel elle se connaît comme possible en elle-même64. »

La création de la pluralité va procéder de cette Première Intelligence.

La Première Intelligence, en contemplant le principe qui la fait exister nécessairement (c'est-à-dire Dieu), donne lieu à la Deuxième Intelligence.
La Première Intelligence, en se contemplant comme émanation de ce principe, donne lieu à la Première Âme, qui anime la sphère des sphères (celle qui contient toutes les autres).
La Première Intelligence, en contemplant sa nature d'essence rendue possible par elle-même, c'est-à-dire la possibilité de son existence, crée la matière qui emplit la sphère des sphères, c'est la sphère des fixes.
L'existence de Dieu: l'argument par la contingence
Dans "Le livre de la délivrance" (un condensé du "Livre de la guérison"), Avicenne développe un argument original pour l'existence de Dieu. L'arrière-plan de l'argument est le point de vue d'Avicenne selon lequel l'existence, la nécessité et la possibilité sont mieux connues de nous que tout ce que nous pourrions dire pour les élucider. En particulier, l'affirmation de l'existence d'une chose ou d'une autre est plus manifestement correcte que ne le serait tout argument que nous pourrions donner pour justifier cette affirmation. Et les notions de nécessité et de possibilité sont plus fondamentales que toute autre notion à laquelle nous pourrions faire appel pour tenter de les définir. Néanmoins, Avicenne pense que nous pouvons dire quelque chose pour décrire les notions de nécessité et de possibilité, même si nous ne pouvons pas les définir strictement65.

Le philosophe thomiste Edward Feser résume l'argument d’Avicenne comme suit65:

Quelque chose existe.
Tout ce qui existe est soit possible, soit nécessaire.
Si cette chose qui existe est nécessaire, alors il y a un existant nécessaire.
Tout ce qui est possible a une cause.
Donc, si cette chose qui existe est possible, alors elle a une cause.
La totalité des choses possibles est soit nécessaire en soi, soit possible en soi.
La totalité ne peut être nécessaire en elle-même puisqu'elle n'existe que par l'existence de ses membres.
Ainsi, la totalité des choses possibles est possible en elle-même.
Donc la totalité des choses possibles a une cause.
Cette cause est soit interne à la totalité, soit externe à celle-ci.
Si elle est interne à la totalité, alors elle est soit nécessaire, soit possible.
Mais elle ne peut dans ce cas être nécessaire, car la totalité est constituée de choses possibles.
Et elle ne peut pas non plus dans ce cas être possible, puisqu'en tant que cause de toutes les choses possibles, elle serait dans ce cas sa propre cause, ce qui la rendrait nécessaire et non possible après tout, ce qui est une contradiction.
Ainsi, la cause de la totalité des choses possibles n'est pas interne à cette totalité, mais externe à elle.
Mais si elle est en dehors de la totalité des choses possibles, alors elle est nécessaire.
Il y a donc un existant nécessaire.
Comme le note Jon McGinnis66, parmi les caractéristiques distinctives de cet argument, il y a le fait que non seulement il n'exige pas une prémisse à l'effet qu'un infini réel est impossible comme le font souvent les arguments cosmologiques, mais qu'il ne repose pas non plus sur une prémisse à l'effet que le monde des choses possibles est ordonné (comme le fait un argument téléologique), ou qu'il est en mouvement (comme le fait un argument aristotélicien du mouvement), ou qu'il est multiple par opposition à unifié (comme le pourrait un argument néoplatonicien). Son but est de montrer que si quelque chose existe, il doit alors y avoir un être nécessaire65.

L’Être Nécessaire, selon Avicenne, doit être unique. Car supposons qu'il y ait deux ou plusieurs Êtres Nécessaires. Il faudrait alors que chacune ait un aspect qui la différencie de l'autre - quelque chose que cet Être Nécessaire a et que l'autre n'a pas. Dans ce cas, ils devraient avoir des parties. Mais une chose qui a des parties n'est pas nécessaire en elle-même, puisqu'elle existe par ses parties et ne serait donc nécessaire que par elles. Puisque l'Être Nécessaire est nécessaire en lui- même, il n'a pas de parties, et n'a donc rien par lequel un Être Nécessaire pourrait même en principe différer d'un autre. Il ne peut donc y en avoir plus d'un65.

De toute évidence, il s'ensuit également que l'Être Nécessaire, étant sans parties, est simple ou non-composé. L'Être Nécessaire doit aussi être immatériel, et donc incorporel. Car la matière n'existe que dans la mesure où elle a une forme, et ce qui est composé de forme et de matière n'est pas simple mais composite65.

Aussi, La bonté, pour l'aristotélicien, doit être définie en termes de la fin vers laquelle une chose pointe comme une cause finale. Or, une partie de la métaphysique plus générale d'Avicenne est la thèse selon laquelle toute chose existante "désire" ou vise à s'approcher de l'existence nécessaire autant qu'elle le peut. Mais alors ce qu'elle désire ou vise est de se rapprocher de l'Être Nécessaire, qui en tant qu'objet de ce désir ou de ce but est le bien le plus élevé. L'Être nécessaire doit également être parfait dans la mesure où pour Avicenne, la perfection est une question de ce qui complète une chose par rapport à son existence. Un gland est d'autant plus parfait qu'il est proche d'être un chêne, la Vénus de Milo serait plus parfaite si elle avait ses bras, et ainsi de suite. Mais l'Être Nécessaire, étant absolument nécessaire en lui-même, ne manque de rien en ce qui concerne son existence65.

Influence d'Avicenne sur le kalām
L'influence de la philosophie d'Avicenne dans la théologie rationnelle asharite a été croissante. C'est surtout avec Al-Juwayni que les concepts avicenniens commencent à pénétrer dans le kalām67. La preuve de l'existence de Dieu par la contingence du monde témoigne de cette influence68. Si Al-Ghazālī condamne certaines des thèses d'Avicenne, cela ne l'empêche pas de lui en emprunter d'autres - sans le nommer69. Faḫr ad-Dīn ar-Rāzī n'aura pour sa part aucune réticence à se référer explicitement à Ibn Sinā70. Aux yeux de Louis Gardet, c'est cette place des concepts et méthodes des falāsifa, en particulier Avicenne, qui distingue, parmi les deux grandes périodes de la théologie acharite, celles des modernes71. Preuve de la progression des idées d'Avicenne, le théologien mutazilite al-Malāhimī, au xiie siècle, voit cette influence grandir avec inquiétude, car il reproche au philosophe de dénaturer l'islam72.

Philosophie de l'être
Selon Marie-Dominique Philippe, Avicenne était un croyant au Dieu-Créateur dans l’Islam. La foi d’Avicenne ne l’empêche pas d'utiliser la métaphysique d’Aristote. Mais au contraire, il s’en sert. Il ajoute qu’Avicenne ne fait pas la distinction entre la théologie et métaphysique. Chez Avicenne, il y a un passage de la métaphysique à la théologie comme une sorte d’enveloppement73.

Angélologie
Hiérarchie des dix sphères

Hiérarchie de dix sphères célestes d'après le système de Ptolémée, illustration de Cosmographia, Anvers, 1524, de Petrus Apianus.
Avicenne s'inspire plus particulièrement de l'angélologie d'al-Farabi. L'univers est constitué d'une hiérarchie de mondes sphériques, animés par des Âmes célestes (anges et archanges) procédant du principe divin, et motrices des cieux.

La triple contemplation de la Première Intelligence instaure les premiers degrés de l'être. Elle se répète, donnant naissance à la double hiérarchie :

hiérarchie supérieure, qu'Avicenne désigne comme les Chérubins (Kerubim) ;
hiérarchie inférieure, qu'Avicenne désigne comme les Anges de la magnificence ;
Ces âmes animent les cieux, mais elles sont dépourvues de la perception du sensible ; elles se situent entre pur intelligible et sensible, et elles se caractérisent par leur imagination, qui leur permet de désirer l'intelligence dont elles procèdent. Le mouvement éternel qu'elles impriment aux cieux résulte de leur recherche toujours inassouvie de cette intelligence qu'elles désirent atteindre.

Elles sont à l'origine des visions des prophètes, par exemple. « Il y a donc, dit Avicenne, pour chaque sphère céleste une âme motrice qui intellige [saisit par son intelligence] le bien et qui, à cause de son corps, est douée d'imagination, c'est-à-dire des représentations des particuliers et une volonté des particuliers »74. Le point de départ, ici, était la cosmologie d'Aristote : Dieu est une substance immobile, un premier moteur unique, immobile, qui meut en tant qu'objet de désir et d'intellection du premier ciel, qui est la substance de la circonférence la plus extérieure de l’Univers, à savoir la sphère des étoiles fixes75.

Cette hiérarchie correspond aux Dix Sphères englobantes (Sphère des Sphères, Sphère des Fixes, sept Sphères planétaires, Sphère sublunaire).

Dixième intelligence et intellect
La Dixième Intelligence76, issue de l'Intelligence du 9° ciel (la Lune), mais sans fonction astronomique, revêt une importance singulière: aussi appelée Intellect agent ou l'Ange, et associée à Gabriel dans le Coran, elle se situe si loin du Principe que son émanation éclate en une multitude de fragments. En effet, de la contemplation de l'Ange par lui-même, en tant qu'émanation de la neuvième Intelligence, n'émane pas une âme céleste, mais les âmes humaines. Alors que les Anges de la Magnificence sont dépourvus de sens, les âmes humaines ont une imagination sensuelle, sensible, qui leur confère le pouvoir de mouvoir les corps matériels60.

Pour Avicenne, l'intellect humain n'est pas forgé pour l'abstraction des formes et des idées. L'homme est pourtant intelligent en puissance, mais seule l'illumination par l'Ange leur confère le pouvoir de passer de la connaissance en puissance à la connaissance en acte. Toutefois, la force avec laquelle l'Ange illumine l'intellect humain varie :

les prophètes, inondés de l'influx au point qu'il irradie non plus seulement l'intellect rationnel mais aussi l'imagination, réémettent à destination des autres hommes cette surabondance ;
d'autres reçoivent tant d'influx, quoique moins que les prophètes, qu'ils écrivent, enseignent, légifèrent, participant aussi à la redistribution vers les autres ;
d'autres encore en reçoivent assez pour leur perfection personnelle ;
et d'autres, enfin, si peu qu'ils ne passent jamais à l'acte.
Selon cette conception, l'humanité partage un et un seul intellect agent, c'est-à-dire une conscience collective. Le stade ultime de la vie humaine, donc, est l'union avec l'émanation angélique. Ainsi, cette âme immortelle confère, à tous ceux qui ont fait de la perception de l'influx angélique une habitude, la capacité de surexistence, c'est-à-dire l'immortalité.

Pour les néo-platoniciens, dont Avicenne fait partie, l'immortalité de l'âme est une conséquence de sa nature, et pas une finalité77.

Pour sa part, à la différence d'Avicenne, Averroès va dégager l'aristotélisme des ajouts platoniciens qui s'étaient greffés sur lui : point d'émanatisme chez lui.


\section{Avicenne}
\cite{PolDroit:voyage}

V. Dans les têtes des philo­sophes arabo-musulmans
Surlignement (bleu) - La lecture minutieuse des textes grecs conduit plusieurs générations de philo­sophes arabes à les prolonger, en les interprétant en relation avec la révélation coranique. 
On lui doit en effet , dans ce domaine , de grandes innovations . D’abord , parmi les divisions de l’être , la distinction cruciale entre « être possible » et « être nécessaire » . En réexaminant les catégories d’Aristote , Avicenne montre comment elles se trouvent en quelque sorte traversées , ou surpassées , par cette division entre les êtres qui ne portent pas
 
en eux leur cause et l’être nécessaire par lui - même , de par sa propre essence . La réflexion du philosophe chemine du couple conceptuel « possible - nécessaire » au couple « essence - existence » , qui lui doit également son émergence dans la tradition philosophique , entamant ainsi une très longue histoire , dont l’époque contemporaine porte toujours les marques . On voit aussi le génie d’Avicenne approcher une théorie de la conscience et du Je antérieure au « cogito » de Descartes , à travers son hypothèse de « l’homme volant » . Il imagine un homme créé d’un coup , dans le vide , sans perception provenant d’un monde extérieur ni de son propre corps . Cet homme dépourvu de toute sensation et comme privé de corps n’en aurait pas moins , soutient Avicenne , la conscience d’exister , d’être lui - même , et aucun autre . Mort relativement jeune , à cinquante - sept ans , au cours d’une expédition militaire qu’il accompagnait , le philosophe repose à Hamadan , dans l’actuel Iran , à mi - chemin entre Téhéran et Bagdad , où un mausolée monumental a été édifié en 1952 . Car la gloire d’Avicenne ne s’est pas ternie , et il continue d’être célébré comme esprit universel , même si , dans l’histoire des philosophes arabo - musulmans , il a rencontré des adversaires farouches .


\section{La preuve développée par Abū al-Ḥasan al-Ašʿarī}

  
  \subsection{L'existence de Dieu}
  
 

C'est par une preuve de l'existence de Dieu qu'al-Ašʿarī ouvre un de ses
ouvrages majeurs d'exposition systématique de sa théologie, le
\emph{Kitāb al-Lumaʿ}. La démarche nous
semble naturelle, mais elle n'est alors pas si commune. Al-Ašʿarī n'est
pas le premier à proposer une preuve de l'existence de Dieu (il semble
que ce premier soit, en islam, le théologien zaydite Qāsim ibn Ibrāhīm),
la démarche n'est pas fréquente : beaucoup de théologiens considèrent
que l'existence de Dieu est le fondement même du savoir, une donnée
évidente par elle-même.




\begin{quote}
   Question : Quelle est la preuve que la création a un auteur qui l'a
créée, et un organisateur qui l'a organisée ?\sn{PREUVE DE L'EXISTENCE DE DIEU (al-Ašʿarī, \emph{Kitāb al-Lumaʿ},
§ 3).}

Réponse : La preuve est la suivante. L'être humain, même quand il est au
sommet de sa perfection, a d'abord été successivement du sperme, un
caillot, un petit amas, et enfin de la chair et du sang. Nous savons que
ce n'est pas lui-même qui s'est fait passer d'un état à un autre. En
effet, nous voyons que, alors même qu'il est haut plus au degré de sa
force et de son intelligence, il est incapable de se fabriquer des yeux
pour voir ou des oreilles pour entendre, ni de se créer un membre
quelconque. C'est bien la preuve qu'il était encore plus incapable de le
faire avant même d'avoir acquis sa force et son intelligence : ce qu'il
est incapable de faire dans son état de perfection, à plus forte raison
en sera-t-il incapable dans un état de faiblesse.

De plus, nous voyons que l'homme est d'abord un enfant, puis un jeune,
puis un adulte, enfin un vieillard, et nous savons qu'il ne se fait pas
passer lui-même de l'état de jeunesse à celui de maturité ou de grand
âge, puisque s'il s'efforçait de quitter la maturité ou le grand âge
pour revenir à la jeunesse, il ne le pourrait pas. Ce qui prouve bien
qu'il ne se fait pas passer de lui-même d'un état à un autre, et qu'il y
a quelqu'un qui le fait passer d'un état à un autre et qui l'a organisé
comme il est, car il est impossible qu'il passe d'un état à un autre
sans l'aide de quelqu'un qui le change et l'organise.

On peut prendre un exemple pour l'expliquer : le coton ne peut pas
devenir du fil, et le fil du tissu, sans l'aide d'un fileur et d'un
tisserand. L'homme qui achète du coton en espérant le voir se changer en
fil puis en tissu sans l'aide d'un fileur et d'un tisserand est un fou,
tout comme celui qui, dans un désert, s'attend à voir la boue se changer
en briques qui s'empileraient d'elles-mêmes, sans l'aide d'un briquetier
et d'un maçon, est un imbécile. 
\end{quote}

\section{Avicenne - Intellect agent}

Alfarabi, Avicenna, and Averroes, on Intellect: Their Cosmologies...
par Davidson, Herbert A
1992
A study of problems revolving around the subject of intellect in the philosophies of Alfarabi, Avicenna, and Averroes, this book pays particular attention to the way in which these...

Intellect agent : explique le monde sublunaire, depuis la première cause. 

shifa' : le problème d'expliquer comment un univers pluriel peut dériver d'une unique cause a été posé par Plotin. 
\begin{quote}
    from the one, insofar as it is one, only one can come into existence (yuhad)
\end{quote}
Les émanations successives permettent d'expliquer comment, de ce principe. La plurarité entre parce que les êtres incorporels suivant la première cuase ont plusieurs pensées.

\begin{Def}[Intellect actif]
\begin{itemize}
    \item emanating cause of the matter of the sublunar world
    \item the emanating cause of natural forms appearing in matter, inlcuding the souls of plants, animals, and man
    \item the cause of the actualization of the human intellect.
\end{itemize}
\end{Def}

\section{Avicenna’s Proof of the Existence of God: Problem 7}

In: Doubts on Avicenna
Author: Ayman Shihadeh
Type: Chapter
Pages: 143–155
 
 \begin{quote}
     al Mas'udi's central objection -- that an infinite series cannot be treated as a self contained whole, at least not in Avicenna's matter of facter manner - seems quite compelling, there is much less mileage in his analogical ad hominem argument, as it fails to fulfuil the principal argument of this type of argument, namely, that it should start from the adversary's own view.
 \end{quote}
 
 
 \section{The Cambridge Companion to Classical Islamic Theology (Cambridge Companions to Religion)}
 
 

Tim Winter
Part II Themes
 \begin{quote} > Page 198 · Emplacement 4943
A convenient starting - point will be a categorisation of proofs provided by Fakhr al - Dīn al - Rāzī ( d . 1210 ) , an outstanding philosopher and mutakallim , who surveyed and assessed the previous philosophical and theological dialectic more systematically and insightfully than did his predecessors . He distinguishes between four categories : ( 1 ) arguments from the creation of the attributes of things ( a subspecies of the argument from design ) ; ( 2 ) arguments from the creation of things ; ( 3 ) arguments from the contingency of the attributes of things ( a subspecies of the argument from particularisation ) ; and ( 4 ) arguments from the contingency of things . 4 The first type will be discussed below under “ Common teleological arguments ” ; the second and third under “ Kalām cosmological arguments ” ; and the fourth under “ Avicenna’s argument from contingency ” .
COMMON TELEOLOGICAL ARGUMENTS An argument from design , or a so - called teleological argument , is one which argues from manifestations
\end{quote} 
\subsection{KALĀM COSMOLOGICAL ARGUMENTS}
\begin{quote} > Page 204 · Emplacement 5095
 The early mutakallimūn developed characteristic doctrines and methods of argument ( some of which we will encounter below ) , which formed the speculative frameworks in which they expounded their proofs for the existence of God . Generally , arguments from design were either omitted or accorded secondary importance in kalām works , since they proved only the existence of a “ designer ” , but not the generation of matter and hence creation ex nihilo , and because they were often seen to lack methodological rigour . Instead , the kalām argument par excellence became the argument from creation ex nihilo , or temporal generation (  udūth ) , 31 and the closely related argument from particularisation – both cosmological arguments , since they prove the existence of God starting from the existence of other beings .
\end{quote} 
\paragraph{Arguments from creation ex nihilo}
 \begin{quote} > Page 205 · Emplacement 5105
The basic argument from creation goes as follows . The world is temporally originated (  ādith ) . All that is temporally originated requires a separate originator . Therefore , the world requires a separate originator . This originator must be pre - eternal . Otherwise , if it too is generated , then , by the same reasoning , it will require another originator ; and ultimately the existence of a pre - eternal originator has to be admitted . Both premises in the argument were surrounded by complex discussions , both among theologians , and between them and the philosophers . In what follows , some of the discussions that appeared among the mutakallimūn surrounding the two premises in this proof are examined . That the world is temporally originated Several arguments were advanced in support of this doctrine ( the minor premise in the above proof ) mostly on the basis of the early kalām physical theory that , apart from God , all beings are bodies consisting of both atoms and accidents present
in them . 32 The most commonly used is the so - called argument from accidents ( a‘rā  ) , apparently developed by the Mu‘tazilite Abū Hāshim al - Jubbā’ī , which establishes the generation of atoms on the basis of four principles , as follows : ( a ) Accidents exist in bodies . ( b ) Accidents are generated . ( c ) Bodies cannot be devoid of , or precede , accidents . ( d ) What cannot be devoid of , or precede , what is generated is likewise generated . 33 Earlier mutakallimūn seem to hold that the generation of the world follows from these contentions directly . Yet , as Averroes points out , this line of reasoning involves an equivocation : what is found to be generated in the fourth principle is the single body that necessarily has a particular accident known to be generated , rather than bodies as such , and consequently the world as a whole , as in the conclusion . 34 Indeed , he points out , it will still be conceivable for the world to be pre - eternal , involving infinitely regressing series of temporally originated things (  awādith lā awwala lahā ) .
Later mutakallimūn , as Averroes notes , became more aware of this gap in the proof , and attempted , apparently starting from Juwaynī , to address it by arguing that a pre - eternal series of accidents is inconceivable . 35 Several arguments are found in later works of kalām that support this contention ; the following two are recorded in a later Mu‘tazilite source . For instance , it is argued , rather opaquely , that the whole must be characterised by the same attributes that necessarily characterise each of its individual parts ; for instance , if something consists entirely of black parts , it too must be black . Therefore , since each part of the world is generated and has a beginning , the whole world too must be generated and have a beginning . The infinite regress of accidents is also refuted using proofs from the impossibility of an infinite number , some of which were apparently adopted from John Philoponus ( d . c . 570 ) . 36 For instance , it is argued : When today’s events are combined with past events , these will increase ; without today’s events , they will
diminish . Increase and diminution in what is infinite are inconceivable . This indicates that [ the series of past events ] is finite with respect to its beginning . This is the proof also for the finiteness of the magnitude of the earth and other bodies ; for it is possible to conceive of increase and diminution in them . 37 Many later Ash‘arites adopted Juwaynī’s modified version of the argument for creation ex nihilo , which most theologians treated as an article of faith . Yet this doctrine soon became the centre of conflict between the theologians and most philosophers , who defended the pre - eternity of the world , as the interaction between the two traditions increased . Doubts were raised around the arguments for creation , to the extent that in one of his latest works Rāzī examines all the relevant arguments and counterarguments and admits that no rational or revealed evidence proves either the creation or pre - eternity of the world . 38 Under his influence , it seems , Ibn Taymiyya asserts that no rational or revealed evidence proves the inconceivability of the infinite regress of accidents , apparently suspending judgement on the subject .
without hesitation or reflection ” .

\subsection{Note}
Note - 10 The existence of God > Page 207 · Emplacement 5153
Kahnemann sysstemmem 1
\end{quote} \begin{quote} > Page 207 · Emplacement 5163
this archetypical kalām analogy ( an instance of inferring the “ unobservable ” from the “ observable ” ,
\end{quote} \begin{quote} > Page 207 · Emplacement 5171
My act requires me ( its originator ) , we are told , because it occurs according to my motives ; this connection affirms the judgement in the original case . But in what respect exactly does my act depend on me ? Does it depend on me because it is temporally originated , or for some other ground ?
Therefore , my act depends on me in this respect only , and the ground will thus be affirmed in the original case . It may seem strange to argue for the existence of God from human acts , rather than from the need of natural events generally for causes . Yet this oblique way is forced on those Mu‘tazilites who employ this argument by
physics : many of them reject natural causality , and affirm that God creates all generated things , except accidents produced by the power of living creatures . Hence , when I move my pen , my power will generate the accident of motion in it ; however , when running water moves a pebble , the accident of motion in the pebble will be generated by God’s power , not by the water . Our acts , therefore , provide the only case where we can observe both the originated thing and its originator and conclude that the former is generated by the latter . The existence of the creator will then be the only explanation for the generation of the existence of other accidents and all atoms , as ‘ Abd al - Jabbār writes : “ Everything that is [ beyond the capacity of created beings ] is evidence for Him . ” 43 Mu‘tazilites criticised Ash‘arites on account of their contention that human acts are generated by divine , rather than human , power : since they cannot affirm that power generates things in the “ observable ” realm , they cannot affirm the
They will be unable to accept the causal premise in the argument , and will thus fail both to explain the world as a divine act and to prove the existence of God . Juwaynī retorts that Ash‘arites use the closely related particularisation argument , which does not resort to the above analogy . 44 Ash‘arites indeed rarely use this basic argument from creation , involving the major premise , “ What is originated requires an originator ” , except in an informal and non - technical manner . Rāzī attacks each step in the above analogical argument , arguing at length that “ coming into being ” cannot be the ground for a thing’s requiring a cause . 45
Arguments from particularisation This is the main form of argument used by early Ash‘arites , and is often used by Mu‘tazilites and later Ash‘arites . It turns on the notion of particularisation ( takh  ī  ) , which has its background in a trend distinctly characteristic
classical kalām , stemming from the sense that randomness of any kind , in either quantity or quality , is inconceivable .
\end{quote} 
\begin{Synthesis}
Note - 10 The existence of God > Page 209 · Emplacement 5201
Aleatoire impossible ?
\end{Synthesis}
\begin{quote} > Page 209 · Emplacement 5201
Every seemingly random fact about the world or things therein thus calls for explanation . Different instances of this type of proof cite different facts . The earliest arguments were relatively simple and departed from the atomist framework of classical kalām , as in the following two arguments advanced by the Ash‘arite theologian al - Bāqillānī ( d . 1013 ) .
Note - 10 The existence of God > Page 209 · Emplacement 5203
Vision atomiste
\end{quote} \begin{quote} > Page 209 · Emplacement 5204
He argues that we observe identical things coming into being at different times . If the occurrence of one thing at a particular moment is due to an intrinsic quality thereof , all similar things should occur at the same time . It thus appears that nothing intrinsic to the thing itself could make it more likely to occur at a particular moment rather than at another moment , or more likely to occur at a given moment than another , similar thing . Therefore , there must be an external voluntary effecter , who causes particular things to occur at particular moments .
\end{quote} 

\begin{quote} > Page 209 · Emplacement 5219
There is no natural necessity determining the way things actually are . All things , rather , consist of identical atoms and of different accidents present in them , which come in and out of existence at every moment .
\end{quote} \begin{quote}> Page 210 · Emplacement 5222
As mentioned , the general particularisation argument can take different types of facts as its point of departure . The foregoing examples focus on the when and how with respect to the generation of things . In later , more sophisticated , arguments advanced by Juwaynī , the same lines of reasoning are applied to the world as a whole , which allows
\end{quote} \begin{quote}> Page 210 · Emplacement 5225
him to transcend the occasionalistic bias of earlier particularisation arguments . He argues , first , that since the world is generated , it must have come into being at a particular point in time . This implies that a separate particularisation agent must exist to select this particular moment for creating the world out of other possible moments . Such selection can only be made by a voluntary agent . An unchanging , non - voluntary pre - eternal cause will necessitate its effect and will thus produce a pre - eternal world ; yet the world , Juwaynī argues , has been shown to be temporally originated . 47 This argument faces the problem that it implies that time existed before creation , a doctrine that was subject to much debate . 48
\end{quote}
Note - 10 The existence of God > Page 210 · Emplacement 5226
Ce que jee comprend de la parti cularisatiokn temps deux choses qrriivrnn tq des moments differents. Donx ce nest pas dqns leurr essence majis cekkkee de dieu
Note - 10 The existence of God > Page 210 · Emplacement 5231
A partir de la particukariisatiion dieu eset xelui qui chosit ce momeenet parrixxuxukier




 \begin{quote}> Page 210 · Emplacement 5232
Elsewhere , Juwaynī also argues that if we observe the world , we find that it consists of things that have great variety in their attributes , composition and circumstances . None of these , however , is necessary , as the mind can imagine all things being otherwise . It becomes evident , he continues , that since the world is possible , “ it will require a determinant [ muqta  ī ] , which determines it in the way it
actually is ” . What could exist in different possible ways cannot exist randomly ( ittifāqan ) , without a determinant , in one particular way . 49 Again , the determinant has to be a voluntary agent ; for a non - voluntary factor will necessitate a uniform , undifferentiated effect , whereas this world consists of highly complex parts , which do not behave in simple , uniform ways . 50 Ghazālī writes with reference to the notion of particularisation : “ The world came into existence whence it did , having the description with which it came to exist , and in the place in which it came to exist , through will , will being an attribute whose function is to differentiate a thing from its similar . ” 51 Such particularisation arguments , which refer to characteristics of the world or things therein differ crucially from arguments from design . The latter focus on aspects of perfection , masterly production , or providence in the world . Particularisation arguments , by contrast , depart from the mere fact that existents in this world , regardless of their perfection , imperfection , goodness or badness , are possible , since they exist in one particular way
rather than another , and thus require an external factor to select this possibility over all other possibilities . Such arguments aim only at proving that the world has a voluntary producer , whereas arguments from design seek to prove that the world must have a wise , powerful and good producer . Finally , Juwaynī goes further to develop a third argument by applying the particularisation principle to the fact that the world exists . In this crucial modification to the particularisation argument , he frees it completely from the constraints of atomist physics . He first demonstrates that the world is temporally originated , then writes : What is temporally originated is a possible existent ( jā’iz al - wujūd ) ; for it is possible to conceive its existence rather than its non - existence , and it is possible to conceive its non - existence rather than its existence . Thus , since it is characterised by possible existence rather than possible non - existence , it will require a particularising factor ( mukha   i  ) , viz . the Creator , be He exalted . 52 The argument departs from the fact that the world exists , regardless of what it consists of and the way
in which it exists . Since it is equally possible that the world did not exist , the fact that it does exist points to an external factor which effected one of the two possibilities . In this argument , Juwaynī marries the argument from creation ex nihilo to the particularisation argument , which allows him , as an Ash‘arite , to argue that the world requires an originator because it is temporally originated , without resorting to the Mu‘tazilite analogy from human action . More crucially , Juwaynī’s modified argument brings the particularisation argument close to Avicenna’s argument from contingency , paving the way for a synthesis of the two arguments in later kalām .
\end{quote} 

\begin{Synthesis}
Note - 10 The existence of God > Page 210 · Emplacement 5233
Dieu comme celui quu fait advenirr ce qui doitlp
\end{Synthesis}
\subsection{AVICENNA’S ARGUMENT FROM CONTINGENCY }
\begin{quote}> Page 211 · Emplacement 5262
AVICENNA’S ARGUMENT FROM CONTINGENCY The central proof for the existence of God that Avicenna puts forth is the proof from contingency ( imkān ) . In line with the Neoplatonic tradition , he attempts to prove an ultimate efficient cause for bringing the world into being , rather than a cause for motion in the world , as Aristotle does . Unlike
most other proofs , this proof depicts God as a non - voluntary First Cause , which produces the world from pre - eternity by Its essence . Thus , despite its great influence on later Muslim thought , the proof had to be adjusted to conform to more orthodox conceptions of God . Avicenna claims to advance a purely metaphysical proof ( as opposed to a physical proof ) , one that rests purely on an analysis of the notion of existence qua existence , without consideration of any attributes of the physical world . 53 He writes : Reflect on how our proof for the existence and oneness of the First and His being free from attributes did not require reflection on anything except existence itself and how it did not require any consideration of His creation and acting even though the latter [ provide ] evidential proof for Him . This mode , however , is more reliable and noble , that is , where when we consider the state of existence , we find that existence inasmuch as it is existence bears witness to Him , while He thereafter bears witness to all that comes after Him in existence . 54
If true , this characterisation would set the proof apart from all contemporaneous , cosmological and teleological proofs . In contemporary terminology , it would qualify it to be an ontological proof , that is to say , a proof which argues for the existence of God entirely from a priori premises and makes no use of any premises that derive from our observation of the world . Recent studies of Avicenna’s proof , however , differ on whether the argument is cosmological or indeed ontological . 55 As we will see , doubt with regard to the purported fundamental novelty of Avicenna’s proof was expressed centuries ago . The proof rests on conceptions that , Avicenna contends , are primary in the mind , intuited without need of sensory perception and mental cogitation , namely “ the existent ” and “ the necessary ” . The conception “ the possible ” , being what is neither necessary nor impossible , is either equally primary , or derived directly from the conception “ the necessary ” . An existent , by virtue of itself , is either possibly existent , or necessarily existent . If we posit an existent that is necessary in itself , then , Avicenna argues ,
it will have to be uncaused , absolutely simple , one and unique . If we posit an existent that is possible in itself , it will have to depend for its existence on another existent . The latter will be its cause , not in the sense of being an antecedent accidental cause for its temporal generation , but as a coexistent essential cause for its continuous existence . If this cause is itself a possible existent , it will have to exist by virtue of another . The series of actual existents , Avicenna argues , cannot continue ad infinitum , but must terminate in an uncaused existent that is necessary in itself . But why does a possible existent require a cause to exist ? Avicenna proves this using the argument from particularisation , apparently borrowed from kalām . A possible existent can exist or not exist . It will exist only once “ the scale is tipped ” by an external cause such that its existence becomes preponderant over its non - existence . When this occurs , its existence will be “ necessitated ” by its cause . Now , the proof for the existence of God runs as follows . There is no doubt that there is existence . Every existent , by virtue of itself , is either possible
or necessary . If necessary , then this is the existent being sought , namely God . If possible , then it will ultimately require the necessary existent in order to exist . In either case , God must exist . 56 Apparently based entirely on an analysis of a priori conceptions and premises , the proof will appear ontological . However , other considerations suggest that the proof is fundamentally cosmological . For instance , the deliberately abstract and unexplained premise , “ There is no doubt that there is existence ” , appears to derive from our knowledge that “ there is no doubt that something exists ” , or it may even mean the same as the latter statement . 57 When the proof then goes on to appeal to the dichotomy of possible existence and necessary existence , it branches into two hypothetical directions : that this indubitable existence is either possible or necessary . But this then begs the following question : if our indubitable knowledge that there actually is existence is examined , will this existence turn out to be possible or necessary ? In other words , will this knowledge derive from our awareness ( no matter how primitive and abstract ) of possible
existents or necessary ones ? Of course , we cannot be aware of necessary existents ; therefore , our indubitable knowledge of existence must relate to our awareness of possible existence . Inevitably , it seems , the proof reasons on the basis of possible existence using the causal premise , which explains the existence of possible existents by reference to a necessary existent . It hence appears to hinge on the existence of things other than God to prove His existence . Indeed , eight centuries ago , Rāzī wrote that all proofs for the existence of God depart from facts about the world , except that Avicenna had claimed to have advanced a fundamentally new proof purportedly based on a consideration of existence qua existence , without consideration of things other than God . He quotes Avicenna’s above statement to this effect . This claim , however , invites two objections from Rāzī . First , this proof depends on a causal premise : the proof in fact “ infers the existence of the necessary [ existent ] from the [ actual ] existence of the contingent ” . Second , even if it proves a necessary existent , one will still need

to demonstrate that it is other than the physical things perceptible in this world ( this recalls the series of proofs , already referred to , which Avicenna advances for the simplicity , oneness and uniqueness of the necessary existent ) . 58 In other words , the argument presupposes these different considerations about the world : one should prove that the world is not necessarily existent , but contingent , and that a contingent requires a necessary existent to exist , before concluding that God , therefore , exists . A good proof indeed , Rāzī would add , but not an ontological one . Nevertheless , even if such criticisms are accepted , Avicenna should nonetheless be credited with the first attempt ever to advance such a proof . 59
\end{quote}

%\chapter{Saint Augustin}
\mn{Note de cours M. Neusch Marcel NEUSCH
Institut Catholique de Paris 1997-1998
Pro manuscripto
}


 
[354 - 430]






 
\begin{quote}
    «Pour en revenir aux trois points qui comprennent le culte dû à Dieu, savoir la foi, l'espérance et la charité, il est aisé de dire ce qu'il faut croire, ce
qu'il faut espérer, ce qu'il faut aimer. Mais défendre cet ensemble contre les objections de ceux qui pensent autrement exige un exposé doctrinal plus difficile et plus étendu. Pour le fournir, ce n'est pas un bref Manuel qu'il faut prendre en mains; c'est
un grand zèle qu'il faut allumer dans le coeur." (Enchiridion, 1, 6, BA 9)
\end{quote}









 


\section{AUGUSTIN	(354-430)  FACE AUX DEFIS DE SON TEMPS}

.	.	d	,	fi	· ns sur les
ous ce titre, je voudrais regrouper un certain nombre e	re e io	'est
problemes doctrinaux auxquels Augustin fut confronté au cours de son existence. C
une bonne manière de le connaître, non la seule. Avant d'être le polémiste engagé sur tous les fronts au service de la foi chrétienne, Augustin est un pasteur, souc,e x de prêcher l'Evangile au petit peuple d'Hippone.	S'il est considéré com e le"maitre penser" de l'O cident, c'est en raison de l'influence qu'a exercée s theolog'.e. Ia	u
forger sa pensee progressivement, au contact de ses adversaires. Il n a Ja'!1ais occupé une "chaire universitaire" mais il a toujours réfléchi à partir des questions
qui se posaient concrètement  d ns les débats qui agitèrent l'Eglsi de son temps.
)	Naturellement,	élaborée	1	le feu de l'actualité, sa pensée n'a pas	 
dans	toujours
l'équilibre que pourrait	requérir une réflexion "universitaire", comme celle d'un saint Thomas d'Aquin. Elle gagne en vie, et sans doute aussi en virtuosité!

C'est surtout à partir de son épiscopat qu'Augustin sera amené à occuper une place tout à fait capitale dans les débats de l'époque. Même s'il n'est pas le premier évêque de l'Afrique du Nord - le primat revient à l'évêque de Carthage - , il est sur la brèche dès qu'un débat surgit. Son collègue de Carthage, Aurelius, le laissait volontiers monter en première ligne quand il s'agissait de défendre la foi dans les débats publics, notamment contre les Donatistes. Le souci d'Augustin sera essentiellement de présenter la "doctrine chrétienne", et très vite, il connut le succès même au-delà des cercles chrétiens, si l'on en croit son biographe, Possidius, qui écrit:
\begin{quote}
    " Ses livres et traités qui, par un effet admirable de la grâce de Dieu, se suivaient avec rapidité, étaient appuyés sur de nombreux raisonnements et sur l'autorité des Ecritures; et les hérétiques eux-mêmes accouraient pour en entendre la
lecture avec le même empressement que les catholiques, et tous ceux qui le voulaient
et le pouvaient avaient recours à des secrétaires pour recueillir ses propos. Aussi	)
vit-on bientôt se répandre et se manifester dans toute l'Afrique l'éclatante doctrine et la bonne odeur du Christ; et cette nouvelle remplit aussi de joie l'Eglise de Dieu outre-mer'"
\end{quote}








' Possidlus, VIe de saint Augustin, c. 7. Parmi les vies aujourd'hui disponibles, signalons: John J. O'MEARA, La jeunesse de saint Augustin, Introduction à la lecture des Confessions, Pion, 1958, rééd. Cerf, 1988.   Peter BROWN, La vie de saint Augustin, Seuil, Paris, 1967.   Vera PAAONETTO, Augustin, le message d'une vie, Le Centurion, 1981.· Agostino TRAPE,Salnt Augustin, l'homme, le pasteur, le mystique, Fayard, 1988.   Marcel NEUSCH, Augustin, un chemin de conversion, DDB, 1986. - Id.
Initiation à saint Augustin. Un maitre spirituel. Cerf, 1996.
1
)	)
 
)	)
 
p	.	.	1
our situer correctement l'activité d'Augustin,
 
. d'é oquer ce qui conviendrait	v	.	ne
de  la  liturgie, u
 
l'essentiel de sa vie de pasteur . la prédication au cours Ir ·taient dans son corresPondance avec le monde2  les· multiples affaires qui le 50 ici  sans aborder d" è	'	· d s pauvres etc.	.
ioc se, notamment l'exercice de la justice, le souci e  x  dont' il eut à traiter, Je
tous les problèmes, notamment ces problèmes pa5tor u	t retouver à travers voudrais donc me limiter à évoquer les grands débats de I époque,e		ces débats à eux quelques t,hemes essentiels de la pensée d,Augustin· on peut ram.eneer n soulignant cinq, en  fonction  des adversaires  qu'il  eut  à  combattre,  mais
Positivement les enjeux théologiques de ces débats:

1. Adversaire des Manichéens : le champion de la liberté..
Il. Adversaire des Platoniciens : témoin de l'humilité de Dieu. Ill. Adversaire des Donatistes: défenseur de l'unité chrétienne.
IV.	Adversaire des Païens: le citoyen de la cité de Dieu.
V.	Adversaire de Pélage : le champion de la grâce.


1.	ADVERSAIRE DES MANICHEENS
Le chamption de la liberté	,,


Pendant neuf ans, Augustin a été l'auditeur de cette secte, de 372 à 382. Il s'est "laissé séduire", et il s'est fait "séducteur" (IV, 1, 1 ), déployant tout son talen au service de la secte. Devenu adepte des manichéens en 372 (Ill, 4, 10), il a 'pns progressivement ses distances, la rupture devenant effective à partir de son arrivée à Rome (383). Il ne saurait être question de donner un exposé exhaustif de Mani ni du manichéisme3   Né le 14 avril 216 en Perse, M an i vécut au contact de chrétiens. 11 reçut à l'âge de douze ans une série de visions qui firent de lui le fondateur d'une nouvelle religion. Il prêche sa doctrine qui se répand bientôt dans tout l'empire romain. Persécuté, il subit la prison et un horrible matyre qui le conduisit à la mort le 26 février 277. Dès 300, sa doctrine se répand en Asie Mineure et même en Afrique où elle est attestée à Carthage dès cette date.

La vie d'Augustin dans la secte nous est peu connue. li donne une description assez
)	précise de la doctrine	et	des exigences qu'imposaient les Manichéens à leurs adeptes. Dans une lettre qui nous renseigne en particulier sur la distinction entre les
auditeurs et les élus, lui-même étant toujours resté au rang d'auditeur. Augustin regardera le manichéisme comme la pesti/entissima haeresis, l'hérrésie par excellence" . D'un façon générale, on peut relever avec Puech les trois aspects principaux :




2 Cf. André MANDOUZE, Sa/nt Augustin. L'aventure de la raison et de la grâce. Et. Aug. 1968. pages
539 SV.
3 HenPUECH, Le manichéisme, son fondateur, sa doctrine, Musée Guimet, 1949	JuUen
RIES, Les études manichéennes. Des controverses de la Réforme aux découvertes du XXe
Louvaln·l.a N ve, 1988.   Surtout François DECRET, L'Afrique manichéenne(IVe. Ve siècle).2 tomes Etudes augustiniennes, 1978.	· ·
 	Pour une vue rapide, voir DECRET, Le christianisme en Afrique du Nord ancienne
1996. p. 201 sv.	 	u	1
2
)	)
 
de connu, de la C, h·ne à
'	i s'étend à tout le mon	ue les autres 1, - C est une religion universelle, qu  évélation plénière, alo,:Sq  Mani est le E pagne, et elle se présente comme uner e des révélations partielles. d  qualifie
ehg1ons, comme le christianisme, ne sont d'être appliquée à Mu a ,:na ' avec la sceau des prophètes", une formule qui, ava .	t	le contact du man1cheisme
Mani5 , mais sans doute pas avant la pénétratione	.	d
religion musulmane.	.	.	d	nne comme objectif la rédem t  n u
- C'est une religion missionna,re qui se  O	ù il se présente d ailleurs
monde entier. On en trouve très tôt les trac s en Egypt vrai christianisme», alors
non pas en opposition au christianisme, m isd  co '!'ec'est-à-dire comme une secte
qu'il considère les catholiques	comme des JU aisan s,
du judaïsme.	t	t	resque pas de rituel. Tout est
-	C'est une religion du  livre , sans sacreme .e  p	i	forment  la  bibliothèque
centré sur la prédication du con enu des sept iv es qu fonction	de ces livres manichéenne, et les livres chrétiens sont expurge	en
canoniques.
 	séduire	Augustin, on peut
 
Si l'on regarde plus précisément ce. qui a	pu .	.. deviner certains attraits bien que, dès qu'il en parle lu1-meme,
 
soit plus prompt à
.	.	 	.
 
accabler ses coreligionn'aires d'autref.ois
 
d'"inJ·ures que, d. e rendre Justbice ahcaemeqlusi ea t
 
suscité son attrait. Il décrit les Manichéens comme «dehrants de super e,.c.	trois
bavards à l'excès » (Ill 6 1O) : trois traits dans lesquels on reconnait	l sh,
vices qui forment ··e·n nou' s 'selon Augusti.n, une ,ven·table ant1· -tn·m"t'e. Les Ma,mc e.ens
 
sont affli,ges de tous les tro' ·is : l'orgue·il,. la sensua1,1· te, 1a cun· os·1te' 1·
 
C. e ton . polem1ttqiurée
 
ne doit pas nous tromper. Parmi les traits essentiels de cette doctnne qui ont a Augustin, relevons les suivants :

1.	Un  christianisme  rationnel,  qui  dispense  de croire.

Si Augustin s'est rallié au manichéisme plutôt qu'au catholicisme, c'est qu'en Afrique, on ne faisait guère la différence. «On pouvait passer du Christianisme au Manichéisme et vice-versa sans attirer autant l'attention qu'en passant du Catholicisme au Donatisme.1 » Les Manichéens vénéraient en particulier saint Paul. Mani «se voulait, comme Paul, plus que lui, puisque son Jésus à lui était Yésu Ziwa, le Jésus-de-Gloire, «apostolos Jésou Christou»8   On peut cependant trouver plusieurs motifs au choix d'Augustin.

a)	Si, après sa lecture de !'Hortensius qui venait d'éveiller en lui l'enthousiasme de la sagesse, il a manifesté une préférence pour le manichéisme, c'est sans doute parce que le manichéisme passait pour le «vrai» christianisme. Dès leur pénétration en Egypte, où il s'implanta du vivant de Mani, grâce aux marchands, les manichéens présentèrent leur message comme plus pur que les traditions «enjuivées» des catholiques. Ils considèrent ces derniers comme  des «semichristiani» alors q 'e,ux-mêmes s présen ent cor:nme des « eri christiani»9  On voit	qu'en penetrant en Afrique, moins de vingt ans apres la mort de Mani, cette doctrine a
g Cf. Gedaliahu Guy STROUMSA, Savoir et salut, Cerf, 1992, p. 276 sv.
41 lb. pp.315 SV.
7 O'MEARA, oc pp. 79 sv.
1 STROUMSA, oc p. 301.
9 Cf. Augustin, Contra faustum 1, 2-3. Cf. STROUMSA, oc p. 301 sv.
3
 
)
intégré	un	.	rejetant certaines, en
critiquant d'ensemble de données du christianisme,mais. e	«vrai» christianisme, déb	,	autres, en tous les cas en prétendant offrir e		,	.	Voici
arrasse de l'Ancien Testament et rendu totalement transparent a la raison.
comment Augustin le présente:	'

«Ceux que chez eux on appelle auditeurs mangent de la viande, cultivent la terre et prennent femme s'ils le désirent. Mais ceux qu'on appelle élus ne se permettent nen de cela. Les auditeurs ploient le genoux devant les élus implorant d'eux tous, et nondes seuls évêques, prêtres ou diacres, l'imposition des mains. Avec les élus, l s a d,teurs prient et adorent le soleil. Avec eux ils jeûnent le dimanche. Avec eux ils aJ?utent foi à tous ces blasphèmes qui rendent détestable l'hérésie manich nne :. negat,on de la naissance virginale du Christ· refus de croire véritable sa chatr mais affirmation d'une chair factice et par suit; simulacre de la passion et rejet de la résurrection; blasphème envers les patriarches et les prophètes. La loi donnée par le serviteur de Dieu, Moïse, ils disent qu'elle provient du prince des ténèbres. l!s affirment que sont de la substance de Dieu l'âme humaine et l'âme animale : OUI, tls
font de ces âmes une portion de Dieu. Ils font combattre le Dieu bon et vrai avec la nation des ténèbres; se mêler une part de lui-même aux principes des ténèbres et cette part ficelée et polluée dans le monde entier, voilà qu'elle se purifie en nourrissant les élus ou en aboutissant dans le lune et le soleil. A défaut de cette purification, cette parcelle divine sera à la fin des temps à tout jamais dans les fers. C'est dire qu'ils se figurent muable, corruptible et contaminable ce Dieu dont une partie a pu être amenée à un si grand mal et qui ne peut se purifier tout entier, même à la fin du monde, de cette lamentable et impure promiscuité.'0 »

b)	Si les manichéens attiraient surtout les gens cultivés, comme Augustin, c'est qu'ils avaient la prétention d'offrir une vérité rationnelle. Ce rationalisme était une forme de gnose. A la différence du catholicisme, qu'ils considèrent comme une secte juive (c'est-à-dire païenne), ils faisaient appel dans leurs argumentation non pas à l'autorité et à la foi, mais à la raison et à la recherche personnelle. Faustus, leur évêque, considère qu'en tant que gentil, lui-même est né non pas sous la loi juive, mais sous la loi de la nature, c'est-à-dire de la raison. Le grief qu'il fait aux chrétiens, c'est de préférer l'autorité de !'Ecriture à celle de la raison : « ... Vous qui recevez toute chose sans la soumettre à l'examen, condamnant l'usage de la raison, qui
)	est une prérogative de la nature humaine, et considérant comme impie de distinguer entre vérité et mensonge... » (Contra Faustum XVIII, 3)11  Son épistémologie rationnelle
implique que toute chose peut être expliquée. Les miracles, la foi elle-même sont inutiles. On comprend par là qu'un homme comme Augustin, désireux de tout comprendre, ait été séduit.

c)	C'est sur ce point du rapport  entre  foi  et  raison que, après sa conversion, il mènera la bataille contre les manichéens. 11 soulignera toujours leur étroite connexion, en faisant observer que la foi est première, mais qu'elle appelle la raison dans son sillage : il faut commencer par croire si l'on veut comprendre. Il dira à Faustus :  «L'esprit chrétien doit d'abord être nourri à la foi simple, de sorte à pouvoir devenir capable de comprendre les choses célestes et éternelles.» (Contra



10 Lettre 236, in B. A. 17, p. 12-13.
11 STROUMSA, oc p. 345 sv.
4
)
 
 





)
F	E	·tures et constitue
aus um XII, 46)12. La foi implique de croire à la tradition des  lera sa pensée en u e etapenécessaire sur le chemin de l'intelligence.AuguStm f I f	ut	placer la foi à sant qu' I faut croire pour comprendre. Ou, plus exacte?'ent,v r:e (fides ex auditu)
1 intersection de deux actes d'intelligence : il faut recevo  le	f	.	r	progresser
par l'intelligence, et il faut que l'intelligence prenne appui sur la 0'	verà croire, dans les mystèes de Dieu et de l'homme : «Comprends ma parole pour	,,
et crois à la parole de Dieu pour arriver à comprendre.» (Sermon 43, 9)
2.	Une Bible expurgée, sans l'Ancien Testament.

Conséquence de ce rationalisme les manichéens offrent une Bible exp r ée,àqu i
 
n'en retient que ce qui est compatib' le avec leur propre certain nombre de textes du Nouveau Tetament.
 
d octn· ne. EeII se hmtte	un
 

a)	D'abord, elle est amputée de l'Ancien Testament. Faustus éten nt n'avoir rien à voir avec les Juifs. S'il rejette l'Ancien Testament, c'est qu 11 le JU foncièrement immoral (une critique dont Augustin se souviendra plus tard et qu t1
essaiera toujours de réfuter). Faustus est on ne peut plus clair  : «Vous me demandez si je crois à l'Ancien Testament. Bien sûr que non parce que je n'observe pas ses préceptes. Ni vous, j'imagine... » (Contra Faustum VI, 1 ). «Tout ce que nous recherchons chez les prophètes est prudence et vertu, et un bon exemple, ce que vous savez bien qu'on ne saurait trouver chez les prophètes,» (ib. XII, 1) «Je le dis encore une fois, l'Eglise chrétienne, qui comprend plus de gentils que de juifs, est en mesure de ne rien devoir aux témoignages hébrat·ques.» (ib. XIII, 1 ).

b)	Quant au Nouveau Testament, ils n'en retiennent pas grand chose, sinon en le réinterprétant. Dans le texte cité plus haut, on voit que les principales vérités chrétiennes, comme l'incarnation, la passion et la résurrection, sont ruinées.	Pour fonder leur prétention à expurger le Nouveau Testament, les manichéens font valoir qu'il a été écrit par des disciples tardifs, encore entaché de judaïsme, qui n'ont pas vraiment compris ce que le Christ a voulu dire, si bien qu'ils ne méritent qu'une confiance limitée. A preuve toutes les contradictions qui s'y trouvent. Les manichéens présentent donc une Ecriture expurgée. C'est une théologie libérale avant la lettre, à la manière de Renan. Ils avaient eu l'intuition de garder !'Ecriture, ce qui pouvait séduire les chrétiens, mais en même temps de la rationaliser, ce qui devait séduire un intellectuel comme Augustin toujours réticent devant une Ecriture qui n'était pas digne de Cicéron, et dont le langage anthropomorphique n'était pas digne non plus de Dieu.

c)	C'est Ambroise qui le réconciliera avec !'Ecriture. A la lecture manichéenne il opposera une autre manière de lire, qui fera fond sur le	sens	spirituel. Augus in indiquera ainsi un certain nombre de principes de lecture : 1 ) il distinguera à lasuite, d'Ambroise les différents sens de !'Ecriture; et 2) il dira que, comme pour n'importe quel auteur,	11 faut l'aborder avec des dispositions bienveillantes et à l'aide d'un maître compétent (BA 8, p. 196). Il reproche par exemple aux manichéens, en ce qui concerne l'Ancien Testament, de ne pas distinguer entre préceptes moraux et préceptes
12 Sur le couple auctorltas I ratio, on peut consulter le De vera religlone 28 52 BA ap	494 et 570
«D'une façon générale, on peut définir l'autorltascomme l'ensemble des garanties obj.	ecti:ves·	· · 1- ·
 
les hommes à
 
r confiance à un objet de connaissance, tandis que la ratio est caracté · é
 
l'Idée de réflexion.....	ris e par
13 Voir aussi Enar. in Ps 118; Sermo 18, 3.
5
 

s mboliques : «Par exemple, "Tu ne convoiteras point" est un précepte moral; «Tu circonciras tout mâle au huitième jour» est un précepte symbolique... » . C'est grâce à l'exégèse allégorique qu'Augustin surmontera un certain nombre de difficultés de lecture, sans rejeter la lettre : «En ne faisant pas cette distinction, les manichéens, et tous ceux qui trouvent à redire aux écrits de l'Ancien Testament, ne voyant pas que toute observance fixée par Dieu dans le cadre de l'ordre antérieur était une ombre des choses à venir..., ils les condamnent... » (VI, 2)14
3.	Une réponse simple à la question du mal.
Le problème du mal a été l'un des plus angoissants qu'Augustin ait eu à affronter. Il ne cessera d'en être hanté jusqu'à la fin de ses jours. Les manichéens lui ont procuré une explication élégante par leur dualisme.

a)	Que disent-ils ? Que le mal résulte d'un combat archétypal, originel, entre deux  principes  co-éternels, l'un bon et l'autre mauvais. C'est de ce combat qu'est issu la race humaine, l'homme étant constitué par le mélange des deux, le corps qui est le mal et l'âme parcelle divine en nous. La dualité de la volonté en particulier s'expliquerait par la présence en l'homme de deux âmes, l'une bonne, l'autre mauvaise (VIII, 10, 22). Il s'agit donc d'un dualisme radical. A vrai dire, la doctrine est parfois plus subtile, cherchant à sauvegarder le monothéisme. Voici comment s'exprime Fautus:
\begin{quote}
« Croyons-nous en un Dieu ou en deux ? En un seul, bien entendu. Il est vrai que nous croyons en deux principes; mais l'un, nous l'appelons Dieu, et l'autre hu I è, oo pour utiliser le vocabulaire populaire commun, le diable... Pensez-vous que nous devions les appeler tous deux des dieux parce que nous attribuons, comme il se doit, toute la puissance du mal à la hu I è, et toute la puissance du bien à Dieu ? » ( Contra Faustum XXI, 1 )15     
\end{quote}


b)	Cette explication dualiste, qu'il s'agisse de deux dieux ou de deux principes co­ éternels, devait aussi s'avérer à la longue insuffisante, d'abord parce qu'elle dépouille l'homme de sa liberté, ensuite parce qu'elle se trompe sur la nature du mal, et sur sa véritable origine. Ce qui fera faire à Augustin un pas décisif, c'est donc une double découverte, l'une sur la nature du mal, et l'autre sur son origine :

-	d'abord la nature du mal. "Qu'est-ce que le mal ?" ( De moribus, 2, 2). Non pas une substance, mais un manque, une déficience, une absence. A proprement parler, il n'est "rien". Ou, si l'on veut, il est l'absence d'un bien.  Or, les manichéens font de l'homme une chose manipulée par une puissance étrangère, une substance "aliena". C'est grâce au platonisme qu'il surmontera cette première difficulté.

-	ensuite,  l'origine  du  mal. "D'où vient le mal ?" (VII, 5, 7). Il est le fait, non d'une substance, mais d'une volonté libre, "sive in origine sive in opere" (B.A. 17, p. 47).  C'est Ambroise qui attirera l'attention sur ce fait et l'amènera à la découverte de la liberté. "Il n'y a nulle part de péché sinon dans la volonté"16 , dira Augustin, que ce soit la volonté personnelle (in opere), ou la volonté antécédante
1  Cf. STROUMSA, oc pp. 348 sv.
15 Cf. STROUMSA, oc pp. 346·247 .
,e Les Révisions 1, 15
6
(	C
 
 




l
d'Adam (in origine). Toute autre explication est de mauvaise foi, puisqu'elle aboutit à
nier la respansabilité et à excuser le péché. La découverte du péché originel permettra à Augustin de surmonter les difficultés d'un mal en nous sans nous, tout en évitant les impasses d'une nature "étrangère".

4.	Le salut par la gnose.

Ce qui intéresse finalement Augustin, c'est au-delà de la conception cu mal, la question du salut. Là encore, les manichéens ont une explication séduisante.

a)	Ils conçoivent l'homme comme un "divin prisonnier de la matière" (B.A.1 7 p. 15).  Le  salut  consiste  dès  lors  â  libérer  en  nous  la  parcelle  divine enfouie dans la matière. Augustin devait être tout disposé à accepter cette gnose prêchant le salut par la connaissance. Ce salut doit être conquis par chacun. Si Jésus souffre  (Jesus patibilis peut être dit «la vie et le salut des hommes», cf Contra
Faustum XX, 2), ce n'est pas pour nous sauver, mais il s'agit d'une sorte d'évasion personnelle de la matière. D'ailleurs, il n'a pas réellement souffert, puisque son corps
était pure apparence11  	Ici encore, on ne fait pas appel à la foi, mais à l'explication	(
rationnelle.

b)	Ce salut est lié très étroitement à une morale  exigeante. Se libérer suppose en effet la soumission à une ascèse stricte. Mais les Manichéens, soucieux de s'adapter à la situation de chacun, distinguaient entre deux catégories de membres, les élus qui, en particulier par leur refus du mariage, s'interdisent de perpétrer une création qui est mauvaise par essence, et les auditeurs pour lesquels étaient prévus des accommodements. En restant au stade d'auditeur, Augustin montre qu'il était loin d'accepter toutes les exigences.

« Le "sceau du sein" enfin, cherche à empêcher la propagation du mal, c'est-à­ dire la reproduction de l'espèce. Toutes les relations sexuelles sont interdites, et spécialement l'institution du Diable qu'est le mariage. Nos parents nous ont rendu le plus mauvaisservice en nous donnant la vie. C'est pourquoi il nous faut les quitter pour suivre le Christ...Un homme pèche moins gravement avec une concubine qu'avec sa femme car il y a plus de mal à vouloir propager l'espèce humaine qu'à rechercher seulement le plaisir. Le Manichéen doit donc à tout prix éviter la paternité : pour cela tous les moyens sont bons, mais la continence totale est le meilleur'8 »

c)	C'est la découverte du Christ médiateur qui lui permettra de surmonter l'impasse manichéenne : l'homme est malade dans sa volonté, par sa faute, et c'es le Christ médecin qui vient le guérir. De ce point de vue, Augustin s'apercevra assez vite que les manichéens ne pouvaient pas tenir la promesse de vérité qu'ils faisaient à leurs adeptes. Il les accusera de lui raconter des "fables", c'est-à-dire des mythes. C'est en tous les cas ce point qui éveillera le soupçon et la déception . «Ils disaient : "Vérité, vérité!" Et ils me parlaient beaucoup d'elle, et elle n'était nulle part en eux..." (VI, 7 0) "Chez vous, ...il n'y a rien d'autre qu'une retentissante promesse de la


'1 Cf. De diversis quaest. 14. Cf. STROUMSA, oc p. 347, surtout les notes.
19 O' MEARA, oc p. 97. Cf. Peter BROWN, Le renoncement à la chair. Virignité, célibat et continence dans le christianisme primitif. Gallimard, 1995, p. 252 sv.

7
(
 


vérité...; vous ne faites que promettre la vérité sans la montrer...» 19	Ce qu'il retirera de cet échec, c'est trois vérités fondamentales :

-	Il faut croire pour comprendre. Augustin est convaincu que, pour comprendre et connaitre en vérité, il est "nécessaire de croire à une autorité" (B. A. 17, 787). Concrètement, cela signifie : accueillir le Christ, notre science et notre sagesse. C'est seulement à partir de ce point de départ que l'on fera des progrès dans l'intelligence de la vérité. Non seulement les manichéens ne tiennent pas leur promesse de lui expliquer la vérité rationnellement, mais ils lui "enjoignent de croire" cela
même que par ailleurs la science, l'astrologie en l'occurrence, explique depuis longtemps (V, 3, 4).

-	La vérité n'est pas réservée à une élite. Tout homme peut y accéder, grâce à la foi.  A la "vivacitas intelligendi" des intellectuels, une vivacité qui s'est bien souvent égarée dans l'erreur, Augustin oppose la "simplicitas credendi", la simplicité dans la foi. C'est grâce à l'autorité de la foi que le genre humain dans sa totalité, pas seulement l'élite, a eu accès à la vérité et que tous possèdent une même
(	certitude du salut.
Pour lire /'Ecriture, il faut une clef. Nous l'avons déjà dit. Il n'est pas nécessaire d'éliminer l'Ancien Testament, comme le font le manichéens, mais il faut lire le texte en dépassant la lettre. C'est Ambroise, un virtuose de l'exégèse allégorique, qui ouvrira son intelligence à dépasser la lettre et à s'attacher au "sens spirituel" (VI, 5, 8;  VI, 4, 6; V, 14, 24).

Conclusion. Le zèle d'Augustin pour la secte est désormais "bloqué", même s'il y reste attaché comme à une "solution provisoire" (V, 7, 1). Avant de rompre complètement, il rencontre le maître de la secte, Faustus (V, 6, 10). A la fin du livre V, il nous décrit ses hésitations : le catholicisme "ne me semblait plus un vaincu, il n'apparaissait pas encore comme un vainqueur" (V, 14, 24). Il prend alors la décision de s'inscrire comme "catéchumène dans l'Eglise catholique, qui se recommandait de mes parents, aussi longtemps qu'une certitude ne me montrerait pas dans sa lumière où diriger ma course" (V, 14, 25).
On s'est souvent demandé si Augustin n'est pas, malgré tout, resté marqué par son	(
passage dans le manichéisme. De fait, son insistance sur la "massa damnata", son
dualisme des deux cités qui se combattent, peuvent apparaître comme des survivances du manichéisme. Le moins que l'on puisse dire est qu'il en a écarté toute interpréation dualiste, telle que nous la trouvons dans le manichéisme. Surtout, le dépassement du manichéisme s'est fait sur un point essentiel, décisif, qui marquera toute sa théologie : la liberté de l'homme. Le noeud du manichéisme était là. Toutes ses explications convergeaient vers ce noeud gordien : la liberté devenait une illusion. C'est sur ce point qu'Augustin ne cesse de le combattre. En toute circonstance, il en appelle à la liberté de l'homme.






,g  Ep. du fondement, 4, 5, BA 17 p. 399.
8
(
 
(
Il.	ADVERSAIRE	DES PLATONICIENS
Témoin  de l'humilité  de Dieu

Augustin découvrit le platonisme à Milan, au contact des chrétiens qui avait déjà
fait la synthèse entre le christianisme et les idées platoniciennes. On sait, par les travaux de Courcelles, qu' Ambroise s'inspirait, dans sa prédication, des Ennéades œ
Plotin, dont il recopiait des passages entiers dans ses sermonsm. Si Ambroise lisait couramment le grec, c'est surtout avec les traductions  des  oeuvres  de  Plotin par Marius Victorinus que !"influence néoplatonicienne s'affirmera dans le christianisme. Il est certain qu'Augustin a eu entre les mains des «Libri Platonicorum» (Conf. VII, 1O, 16), sans qu'il soit possible de dire s'il s'agit de Plotin ou de Porphyre, ou des deux. Augustin dit qu'ils lui ont été confié par «un homme gonflé d'un monstrueux orgueil» (VII, 9, 13) . Quand il se convertira, il deviendra "l'adepte d'un néoplatonisme chrétien déjà fortement élaboré21"

Est-il légitime de parler d'un	"cercle	milanais" de chrétiens, comme il le fait, cercle dans lequel les écrits de Plotin "semblent avoir joué le rôle d'un centre d'intérêt autour duquel des hommes de conviction diverse pouvaient sympathiser avec cette discrète tolérance qui convient aux hommes distingués" ?	L'expression n'est sans doute pas la plus heureuse, car de cercle, il n'y en eut point.«// y avait incontestablement à Milan, au moment de la conversion d'Augustin,	une certaine actualité du néoplatonisme; on peut mettre en doute l'existence d'un 'cercle' à proprement parler; car, dans l'état actuel de nos connaissances, on ne voit se dessiner que des rapports d'Augustin avec tel et tel, sans qu'on sache rien d'éventuels rapports néoplatoniciens entre les autres membres22 »

1.	Un incroyable  éblouissement

Il est certain qu'Augustin fut conquis par cette lecture des livres platoniciens. Ce fut  une  véritable  délivrance  de  l'esprit  dans la mesure où il dissipa son
«matérialisme  manichéen». L'attitude d'Augustin à l'égard du platonisme fut d'abord une indéniable séduction par cet effet libérateur, avant de se transformer en une sévère contestation en raison du risque grave qu'il représente comme concurrent du christianisme, enfin une appréciation plus mesurée des ces philosophes «les plus proches de nous».  La lecture des "libri platonicorum" ( VII, 10, 16), eut d'abord t un effet  libérateur : il fut réellement ébloui : "Averti par ces livres, je rentrai en moi-même...". Nous avons, dans le Contra Academicos (Il, 2, 5) les traces de l'enthousiasme que suscita cette lecture des livres de philosophie,  enthousiasme
comparable à celui qu'il éprouva autrefois à la lecture de Cicéron et qui sera tempréré par la suite	·

« Mais vo1c1 que certains livres bien remplis, comme dit Ce/sinus, répandirent sur nous les parfums de l'Arabie et distillèrent sur cette petite flamme quelques gouttes de leur précieuse essence : ce fut une chose incroyable, Romanianus, incroyable que l'incendie qui en résultat, et bien au-dessus de tout ce que tu peux penser...Les honneurs, la pompe humaine, le désir de vaine gloire, enfin les douces
20 Recherches sur les Confessions, Ed. de Broccard, 1968. p. 136. Voir aussi Goulven MADEC, Saint Augustin et la philosophie. Notes critiques. Association André Robert, 1992, pp. 26 sv.
2' Recherches sur les Confessions, Ed. de Broccard, 1968. p. 136.
22 G. MADEC, Platonisme des Pères, in Catholicisme, XI, 491-508.
9
 
::ches de :ette vie mortelle, rien de tout cela pouvait-il alors me toucher ? Sans	(
emparer, Je rentra, en moi tout entier... » (Il, 2, S)

,. L  le ture ne l'orienta pas d'emblée vers le christianisme. Au contraire, il semble qu il ait decouvert dans le néo-platonisme assez de satisfactions pour le combler au point qu'il n'eut aucune envie de chercher ailleurs plus loin. Il avoue clairement dans le passage que nous venons de lire :«Je ne réfléchissais qu'incidemment, je l'avoue, à cette religion qui m'avait été enseignée dès l'enfance et comme enfoncée jusqu'à la moelle : mais c'est bien elle qui à mon insu m'attirait à elle. Aussi, titubant, plein de hâte et d'hésitation, je saisis l'apôtre Paul...» (lb.) Qu'a-t-il découvert dans ces livres ? S'il est difficile de faire un bilan exact du bénéfice qu'il en retira , on a pourtant quelques points de repères assez sûrs. Il   apprit  d'eux   plusieurs
choses  décisives  pour  sa  propre  pensée, et qui marqueront son christianisme
de façon indélébile :

a.	le  chemin de l'intériorité. Ces livres lui permirent de faire retour sur lui-même : «Averti par ces livres,· je rentrais en moi-même... ». L'itinéraire augustinien vers Dieu est désormais tracé : il va de l'extérieur (ab exterioribus) vers l'intérieur (ad interiora) , de l'intérieur vers le supérieur (ad superiora). Mais ce chemin d'intelligence (ou d'orgueil, comme dira Augustin) devra être corrigé, car ce n'est pas l'esprit qui connaît, mais la charité (VII, 10, 16).

b.	la découverte de la nature spirituelle de Dieu, identifié avec l'lpsum Esse de l'Exode (VII, 10, 16), et même le Dieu chrétien, un et trine23   Son ontologie n'a donc rien d'abstrait. A-t-il, dès cette époque, découvert le Verbe, comme le laisse entendre Conf. VII, 9, 13 ? Il est probable qu'il n'ai fait le rapprochement que plus tard.

c.	La question du mal. C'est aussi au contact du platonisme que s'est clarifiée la question du mal (VII, 13, 19), comme nous l'avons vu. Alors que les manichéens s'interrogeaient sur l'origine du mal, les Platoniciens font porter le débat sur sa nature, et ils ont compris d'emblée que le mal n'a pas de nature propre, mais qu'il est la dégradation ou l'absence d'un bien.

d.	L'inquiétude  du salut.  S'il rejette d'emblée «l'idolâtrie égyptienne» (VII, 9, 1 S) de Porphyre, provenant sans doute de la contamination du platonisme par le paganisme, Augustin semble partager son inquiétude du salut, inquiétude qui amena Porphyre, cet adversaire déclaré du christianisme, à faire néanmoins une place au Christ : «Ce philosophe dit aussi du bien du Christ, comme s'il avait oublié ses paroles injurieuses ... » (De civ. Dei XIX, 23, 2).

2.-  Critique  sévère  du Platonisme

Il est indéniable qu'Augustin a trouvé dans le platonisme une doctrine libératrice qui l'a aidé à se dégager définitivement du manichéisme, mais en même temps, revenant sur cette expérience philosophique, Augustin se montre d'une extrême sévérité. Dans les Confessions, à la différence du Contra Academicos, il donne d'emblée une vue négative  du  platonisme, sans doute parce qu'il aurait pu le retenir définitivement
23 Cf. Olivier DU ROY, L'intelligence de la foi en la Trinité selon saint Augustin. Genèse de sa théologie trinitaire jusqu'en 391. Etudes augustiniennes, 1966.
10
(
 
sur le chemin de la conversion au christianisme. Dès qu'il aborde cette période de sa	( vie, il  écrit :"Tu m'as procuré, par l'entremise  d'un homme  gonflé  d'un orgueil monstrueux, certains livres des Platoniciens traduits du grec en latin." (VII, 9, 13)
On peut au moins relever le grief fondamental: l'ignorance du médiateur.

a.	Les Platoniciens ont entrevu la patrie, mais sans  la voie. Les livres des Platoniciens représentent un piège dans la mesure où ils indiquent la "patrie" vers laquelle l'homme est tendu, sans faire connaître la "voie" qui y conduit, cette voie n'étant autre que le Verbe fait chair, voie d'humilité qui s'oppose à la présomption des philosophes (VII, 9, 13). Augustin a bien compris que la philosophie peut devenir dangereuse en faisant croire qu'elle est capable de conduire aux mêmes résultats que la foi, et ainsi en détourner (VII, 20, 26). Elle pêche par suffisance, présomption.

b.	Les Platoniciens ont ignoré le Verbe fait chair. C'est là qu'apparaît le fond de la question: les philosophes ont méconnu la nécessaire médiation du Verbe fait chair. On ne va à Dieu que par Dieu. "Je cherchais la voie...et je ne trouvais pas tant
que je n'avais pas embrassé le médiateur, l'Homme Jésus-Christ" (VII, 18, 24; X,	(
43, 68). Augustin dira : "Deus Christus patria quo imus; homo Christus via qua imus" (Sermo 124, 3, 3) Le Dieu Christ : la patrie vers laquelle nous allons; l'homme Christ : le chemin par lequel nous allons. Dans un tableau contrasté entre ce qu'il a découvert chez les Platoniciens (ibi leg1), et ce qu'il n'y a pas trouvé (ibi non legl), Augustin met l'accent sur le Verbe fait chair. Les Platoniciens n'ont découvert que la moitié de la vérité puisqu'ils ont reconnu le Verbe éternel, sans confesser le Verbe fait chair, venu habiter parmi les hommes. «Quant à ceci: il est venu dans son propre domaine... , dans ces livres je ne l'ai pas lu.» (VII, 9, 13). Le Christ n'était qu'un homme d'une éminente sagesse» (VII, 19, 25). Augustin interpellera Porphyre en disant:

« Oh ! si tu avais connu la grâce de Dieu par Jésus-Christ Notre-Seigneur ! Si tu avais pu voir dans l'incarnation où il a pris une âme et un corps d'homme, le plus beau chef-d'oeuvre de la grâce ! Mais que faire ? C'est en vain, je le sais, que je parle à un mort, du moins pour ce qui te regarde.» (De civ. Dei X, 29, 1 ).

c.	Le platonisme n'a eu aucune  emprise sur les foules. Enfin, dernier	(
grief, le Platonisme, à la différence du christianisme,	n'a aucune autorité sur les
foules. C'est un élitisme qui n'a pas réussi à diffuser sa doctrine au-delà d'un cercle d'initiés. Il a toujours tenu la foule en mépris, l'estimant incapable de comprendre sa doctrine et de la pratiquer. Il a donc échouer dans son entreprise d'éducation de la foule: «Ensuite vint Platon, écrivain plus agréable que persuasif. Ces philosophes n'étaient point faits pour ramener la croyance de leurs compatriotes, du culte superstituieux des idoles et du mensonge de ce monde, au vrai culte du vrai Dieu. » (De vera religione Il, 2).

Si les anciens philosophes revenaient aujourd'hui, estime Augustin, ils opèreraient eux-mêmes le dépassement de leur philosophie vers le christianisme. En voyant qu' "on s'engage dans cette voie (chrétienne) en si grand nombre", ils diraient sans doute : "Voilà l'idéal que nous n'avons pas osé prêcher aux foules." (De vera religione  4, 6). "Si donc ces grands hommes pouvaient revivre notre vie... , ils deviendraient chrétiens, comme la plupart des platoniciens des

11
 
(
dernières générations et  de la nôtre." (ib. 4, 7). En tous les cas, il n'y a pasde
comparaison possible entre les philosophes et le christianisme. Augustin écrit dans un de ses sermons. Ainsi, à propos du verset : «leurs juges seront absorbés par le rocher», il expliquer que le rocher, c'est le Christ, tandis que les juges que le Christ anéantit sont les philosophes :

« Prends Aristote, place-le-près de ce Rocher, il s'enfonce dans le néant. Qui donc est Aristote ? S'il entend ces mots : "Voici les paroles du Christ", il tremble dans l'enfer. "Pythagore a dit ceci", "Platon a dit cela" : place-les-près de ce Rocher, compare leur autorité avec celle de l'Evangile, compare ces orgueilleux avec le Crucifié !a. »

3.	Les plus proches de nous, donc aussi les plus dangereux.

On peut dire qu'Augustin lui-même resta partagé dans son appréciation du platonisme. Tout d'abord, il faut faire fi d'une accusation souvent entendue, à savoir qu'il serait responsable de l'hellénisation du christianisme. La "synthèse" entre le christianisme et Platon était déjà faite à Milan, comme nous l'avons souligné, et quand il se convertit, il entre dans un christianisme déjà platonisé, ainsi que l'a montré Courcelle. Par ailleurs, s'il raisonnne dans les catégories platoniciennes, il n'accueille pas la doctrine sans nuance. Il fait un tri. Il est plutôt porté à dénoncer le danger qu'elle représente, signe qu'il garde un impact sur les esprits.

-	En positif, Augustin reconnaît les mérites du platonisme. Quand il n'est pas préoccupé de polémique ou quand il veut mettre les Platoniciens en opposition entre eux, il concède que leur philosophie peut être une preparatio  evangelica. Au livre VIII de la Cité de Dieu, il reconnaît au platonisme le mérite d'avoir porté la philosophie à sa perfection, et de s'être rapproché au plus près du christianisme.
«Platon, philosophe si supérieur à tous les autres parmi les Gentils... » a découvert
que Dieu est

« la cause de l'existence, la raison de l'intelligence, la règle de la vie : trois aspects dont le premier se rapporte à la partie naturelle de la philosophie, le second à la partie rationnelle, le troisième à la partie morale. ...Si donc, pour Platon, le sage est celui qui imite, qui connaît, qui aime ce Dieu et trouve son bonheur à participer à sa vie, quel besoin y a-t-il d'examiner les autres philosophies ? Aucun d'eux n'est plus proche de nous que les platoniciens.» (De civ. Dei VIII, 4-5)

- En négatif, il reste que, entre le christianisme et la platonisme, il existe un gouffre : l'incarnation, l'humilité de Dieu. Là est la pierre angulaire de la foi chrétienne, la pierre d'achoppement de la pensée philosophique des Platoniciens. Augustin écrit contre Porphyre qui refuse l'incarnation :

« Mais l'incarnation du Fils immuable de Dieu par laquelle nous sommes sauvés et qui nous permet d'atteindre ce que nous croyons ou ce que nous comprenons si peu que ce soit, vous vous refusez à l'admettre. Ainsi découvrez-vous de quelque façon mais de loin, et avec des yeux troubles, la patrie (patria) où nous devons demeurer;
 	En. in Ps 140, 19. Van der Meer, 11, 446.


12
(	l
 
(
et  pourtant  le chemin  (via) qu'il  faut  suivre,  vous  ne  le  tenez pas  ."Pourqu i...refusez-vous d'être chrétiens, inon. parce que le Christ est venu humblement et que vous êtes orgueilleux 1 » (De civ. De, X, l 9)



111.	ADVERSAIRE	DES DONATISTES Le défenseur  de l'unité de l'Eglise


Le donatisme sévit en Afrique depuis presque un siècle. Il ne semble pas que ce schisme ait préoccupé Augustin avant son ordination sacerdotale. Au mo nt de son retour en Afrique, en 388, il se retira à Thagaste, sur les terres héntees de ses parents, et il y fonda une communauté de moines. Ordonné prêtre en 391, da s les circonstances que l'on sait il entra en lice presque aussitôt puisqu'on a de lui une lettre à Maximius, évêque donatiste de Sinitum, qui date de 392, lequel d'ailleurs se rallia à l'Eglise catholique (De civ. Dei XXII, 8, 7)25  11 écrivit aussi un psaume abécédaire contre les Donatistes qui date de 393. On le voit ensuite prendre à plusieurs reprises l'initiative d'écrire ou de rencontrer des évêques donatistes. 11 prend aussi part au concile de Carthage, en octobre 393, et peu à peu intensifie son combat pour l'unité de l'Eglise, surtout à partir de son ordination épiscopale, en 395, jusqu'à l'extinction du schisme.

1.	A  l'origine  de l'Eglise  donatiste

La division de l'Eglise d'Afrique date de l'époque des persécutions, au début du quatrième siècle. En 303, Dioclétien avait exigé des chrétiens de Numidie qu'ils livrent les Ecritures (traditio) et sacrifient aux dieux protecteurs de l'Empire (thurificatio). Il y eut un large mouvement d'abandon. Même le primat de Carthage composa avec la persécution. Mais en même temps, il y eut des résistants et des martyrs. Quand le calme fut revenu, en 305, les résistants, qui avaient tenu dans la tempête, se transformèrent en accusateurs. Ils s'en prirent aux "traditores" qui avaient sacrifié aux divinités païennes et livré les Ecritures. Parmi les personnages mis en accusation, il y avait un certain Cécilien, diacre de Carthage, qui aurait, pendant la persécution, empêché les chrétiens de secourir leurs frères en prison. Or, ce Cécilien, accusé par les résistants d'avoir trahi la cause des martyrs, devait bientôt être élu au siège épiscopal de Carthage.

Ce fut cette élection de 31 2 qui provoqua le schisme.	Aussitôt élu, Cécilien fut cons cré par les évêques présents, sans attendre les évêques	de Numidie. Or, ces derrn rs, au.nombrede 70, arrivèrent en force,	et ils		trouvèrent à Carthage un appui de po1?s parmi les opposants de Cécilien, un homme intransigeant,- il y avait s rto t la riche Lucilia, qui s'était fait rabrouer par Cécilien en raison de ses d:v?tions aux reliques des martyrs -. Ils déclarèrent	nulle la consécration	de Cec   en, notam ent parce que, parmi les consécrateurs, s'était trouvé un évêque
trad tor. Il au: Jouter qu'en la Numidie intérieure, montagneuse, et la côte, d'autres motifs de nvahte subsistaient.


.  25Cf .  es de 1  conférence de Carthage en 411 , vol 1. Sources chrétiennes 194  Cerf 1972	1o
Voir aussiI introduction de Yves Congar, dans BA 28, pour l'histoire du schisme.	'	'	'p.  ·
13
 
 

!
. 1
 	1




A la place de Cécilien, on mit Majorin, qui n'avait pas hésité  à ache er les voix "à raison de 400 pièces d'or chacune" (cf. B.A. 28 p.. 13).  A an que la  tuat,on ne fut dénouée, Majorin mourut, et ce fut  Donat qui fu des1gne pour lui succeder. Celui-ci allait occuper le siège de carthage pendant pres de 40 ans et fortement
organiser l'Eglise donatiste.

 
Cécilien (312)
l
Récusé par
les évêques de Numidie
 
Majorin



évêque pendant
ans (313-347)
 

 
Aur!lius

Eglist catholique
 
Parménien

Eglise schismatique
 

L'administration impériale devait reconnaître la validité de l'ordination de Cécilien. Il y eut des répressions contre les donatistes. On devait organiser des conférences contradictoires. Finalement, un synode romain devait déclarer sa consécration valide. On accusa le pape d'avoir été un traditor...Finalement l'empereur rendit son jugement en 316 en faveur de Cécilien. Malgré la répression dont ils furent l'objet, les donatistes s'implantèrent solidement en Afrique du Nord. En 321 intervint un édit de tolérance. Donat , qui se considérait comme le chef du parti des Purs, aura le temps d'organiser la dissidence, avant d'être exilé en 347, sans pourtant réussir à exporter le schisme au-delà de l'Afrique.

2.	Le combat théologique d'Augustin

Quand Augustin entre en scène, il mène d'abord un combat essentiellement théologique, déployant tout son génie en faveur de l'unité de l'Eglise. Il n'ignore pas que certains évêques furent effectivement des collaborateurs. Mais là n'est pas la véritable question. L'enjeu était théologique. Il s'agit de savoir où est l'Eglise, mais aussi de mesurer les conséquences qui résultent d'un abandon de la véritable Eglise.

a.	Où est la	véritable Eglise	? Dans ce combat qui oppose Donatistes et Catholiques, ce sont deux conceptions de l'Eglise qui s'affrontent. La question était celle
de l'unité de l'Eglise, car il est exclu, pour les uns et les autres, qu'il puisse y en avoir deux :

"Notre discussion avec les Donatistes porte non pas sur le Chef, mais sur le Corps, non pas sur le Sauveur Jésus-Christ lui-même, mais sur son Eglise. A ce Chef, q e no s nous accordons à reconnaitre, de montrer où est son Corps, objet de notre dissentiment, afin que ses paroles mettent désormais fin au désaccord." (De
unitate ecclesiae IV, 7) -
" Voici, n'est-ce pas, la question discutée entre nous : Où est l'Eglise ? Chez eux ou chez nous? De toute façon, il n'y en a qu'une..." (De unitate ecclesiae Il, 2)


(	14
l
 
 	Les Donatistes ont une conception qualitative : l'Eglise est une arche de Noé,	(( un rasse blement de purs.  "Ils se considéraient comme l'unique Eglise, le petit reste,
les Pau c 1, les purs, les seuls sauvés, les seuls vrais serviteurs du Christ."26 Celle-ci est une arche de Noé, un rassemblement de purs. Cette idée, ils la poussaient si loin que, lorsqu'ils s'emparaient d'une communauté, ils lavaient les murs de l'église, aspergeait les parquets d'eau salée, brisaient les autels, etc. Surtout, ils avaient soumis à un nouveau baptême tous ceux qui venaient vers eux. Possidius dit que le parti de Donat était "occupé à rebaptiser la moitié de l'Afrique"27

 	Augustin au contraire insiste sur l'universalité de la véritable Eglise. Le critère qu'il fait intervenir est celui de la catholicité : «N'est-il pas manifeste que, depuis que ce parti (donatistes) s'est retranché de l'unité, de nouvelles nations ont cru, que d'autres sont là, qui n'ont pas encore cru et auxquelles l'évangile ne cesse d'être annoncé chaque jour ?» (Le combat chrétien 29, 31). Or, les Donatistes, confiné en Afrique, n'ont aucun titre à la catholicité. Ils sont un parti, non une Eglise. "Parti" s'oppose à "catholique" et souligne l'idée de séparation. L'Eglise catholique a la
mission de rassembler toute la race humaine. Elle doit  être coextensive à la société.
L'Eglise est là où subsiste l'unité, et celle-ci est assurée par la charité.	(
b.	Situation  tragique au regard du salut.  Si quelqu'un abandonne la véritable Eglise, il faut savoir en tirer les conséquences théologiques. Sur ce point, Augustin est assez radical. Il partage l'opinion selon laquelle : Hors de l'Eglise, pas de salut ! «Hors de la communion, de l'Eglise, Dieu n'a aucun des siens... » (De baptismo IV, 9, 13). «Hors de l'Eglise catholique, on peut tout avoir, sauf le salut !» (Sermo ad caes. eccl. plebem 6). Quiconque abandonne l'Eglise se prive de la source qui le vivifie et donc du salut :

"Si quelqu'un est séparé du corps du Christ, il n'est pas membre du Christ et s'il n'est pas membre du Christ, /'Esprit du Christ ne le nourrit pas.n  (Tract. in Jo· Ev. 27, 6)-"Qu'ils s'agrègent au corps du Christ, s'ils veulent vivre de /'Esprit du Christ. Il n'y a que le corps du Christ qui vive de /'Esprit du Christ..." (ib. 26, 13)

A la la différence des Donatistes, qui considéraient les sacrements donnés par les traditores comme invalides, Augustin au contraire devait défendre leur validité, car ce n'est pas le ministre qui baptise, c'est le Christ, mais aussi leur inutilité au regard du salut. Il reconnaît donc qu'il y a chez les Donatistes des "biens" d'Eglise, des vestiges d'Eglise, même si cette possession est injuste et inutile. En effet, bien qu'ils aient ces "biens", les Donatistes ne peuvent s'en servir que pour leur perte. Car il leur manque ce dont tout procède, la charité. La charité est le don "qui compense pour l'absence de certains autres", mais sans ce don, tout le reste "est possédé en vain" (De baptismo 11,). Autrement dit, quiconque vit hors de l'unité n'a pas réellement la charité et il ne reçoit donc pas réellement l'Esprit. Tous les dons que les Donatistes peuvent posséder
ne valent donc rien28



28 Emilien LAMIRANDE, La situation ecclésiologique des Donatistes d'après saint Augustin, Université d'Ottawa, 1972, p. 71.
21 André MANDOUZE, Saint Augustin, l'aventure de la raison et de la grâce, Etudes augustiniennes, 1968, p. 346.
28 Cf. Lamirande, oc pp. 38 sv.- Voir aussi : Itinéraires augustiniens, n° 8 : L'Eglise.
15
(
 
 





(
3    Le  combat  politique  d'Augustin

Interdits depuis 392,	l'année après l'interdiction du	paganisme (391), les Donatistes ne continuèrent pas moins leurs activités. Augustin reconnaît lui-même que les "lois contre eux ne manquaient pas, mais c'est comme si elles avaient manqué."211  Le combat théologique devait donc s'avérer insuffisant, d'autant plus que les donatistes s'étaient alliés avec les circoncellions , bandits des grands chemins qui se battaient contre les propriétaires romains, et qui n'hésitaient pas à recourir à la force. Le schisme se doublait en effet de revendications sociales. La paix était de plus en plus précaire, et Augustin lui-même	faillit tomber, un jour, entre leurs mains. 11 n'échappa à leur piège que parce qu'il s'était trompé de chemin. C'est au concile de Carthage, en 411, que devait être donné l'assaut final contre le Donatisme.

a. Le concile de 411. Ce n'est pas le premier concile à s'intéresser aux donatistes. En 401, le concile de Carthage avait décidé de discuter «avec douceur et dans un esprit de paix». Un autre concile projeté à Carthage, en 403, fut récusé par les donatistes qui considéraient comme indignes une rencontre entre les fils des martyrs, et les descendants des traditores. En 405, un édit de l'empereur rétablit l'unité en faveur des catholiques, et les biens des donatistes devaient être confisqués. Ces derniers firent un certain nombre de démarches auprès du pouvoir impérial . li y eut un nouvel édit de tolérance en 41O. Après la chute de Rome, la convocation d'un concile à Carthage devait règler définitivement la question.

Accepté par les deux partis, le concile de Carthage qui se tint en 411 réunit une assemblée considérable, près de 600 évêques pour chacun des partis. Les Donatistes ne mirent pas beaucoup de bonne volonté. Ils engagèrent une procédure de vérification des mandats qui retarda l'ouverture jusqu'au 8 juin. Le commissaire de l'empereur, Marcellin - impliqué dans un complot, il sera exécuté... et proclamé saint -, fit preuve d'une grande patience, mais devait au terme des débats reconnaître l'Eglise catholique comme la véritable Eglise. Désormais, le donatisme était hors la loi. On devait transférer leurs biens et leurs églises aux catholiques. Leurs clercs sont exilés et les fidèles soumis à l'amende.

b . Recours légitime	à  la "coercitio".	Fallait-il recourir à la coercitio
pour obtenir le retour des Donatistes ? On remarquera d'abord que Augustin n'a jamais	(
eu recours à la coercition à l'égard des Juifs, qu'il juge pourtant sévèrement, mais
dont il respecte la liberté de culte, ni à l'égard des païens, tout en approuvant les lois interdisant le culte des idoles, ni à l'égard des manichéens, avec lesquels il prévère engager un débat d'idées. Dans le cas des donatistes, son attitude a changé. On sait que, sur ce point du recours à la coercition, sa position a varié. D'abord hostile à l'usage de la coercition, il devait s'y rallier dès 400, comme l'ensemble de l'épioscopat, à la fois comme mesure défensive contre les exactions des Donatistes, et comme moyen de les faire revenir à l'unité catholique. Lui-même s'explique dès avant le concile sur ce changement d'attitude :



29 Cf. Mandouze, oc p. 387.
30 Cf. François DECRET, Le christianisme en Afrique du Nord ancienne. Seuil, 1996, p. 145. Le terme viendrait de circum cellas, ceux qui rôdent autour des granges (cella, celliers) ou des chapelles (cellae, chappelles des martyrs).
16
(
 
« Primitivement, en effet, mon avis se ramenait à ceci : personne ne devait être contraint à l'unité du Christ; c'est par la parole qu'on devait agir, par la discussion qu'on devait combattre, par la raison qu'on devait vaincre : je craingnais qu'autrement nous n'eussions comme faux catholiques ceux que nous avions connus comme francs hérétiques. Mais cette opinion, qui était mienne, devait cèder, non devant des mots, mais devant des exemples. Pour commencer, on m'opposait ma propre cité qui, jadis tout entière acquise au parti de Donat, se convertit à l'unité
catholique par crainte des lois impériales...Et il en était de même pour beaucoup d'autres cités dont les noms m'étaient énumérés. Ainsi la force même des choses m'obligea qu'en ce domaine aussi pouvait bien se comprendre la vérité de cette phrase de /'Ecriture : "Donne au sage l'occasion et il sera plus sage encore."

Combien en effet en connaissons-nous dont on peut affirmer qu'en eux se manifestait déjà le désir d'être catholiques, bouleversés qu'ils étaient par l'évidence aveuglante de la vérité, mais que la crainte d'une violente réaction de la part des leurs poussait chaque jour à différer ! Combien d'entre vous étaient retenus, non par la
vérité, qui n'a jamais été	votre fort, mais par la lourde chaine d'une habitude invétérée!..." (Epist. ad Vincentium 93, 5, in Mandouze, oc p. 371-372)	(
On peut donc expliquer le changement d'attitude d'Augustin par la situation particulière créée par le donatisme. Les raisons qu'il invoque dans la lettre citée (qui date sans doute de 408) sont à la fois d'ordre pratique et théologique:

-	d'ordre  pratique: il fallait créer les conditions de la liberté en libérant de la terreur. Il ne faut pas oublier le climat d'insécurité que faisaient règner les donatistes, alliés aux circoncellions.

-	d'ordre théologique : Dès 405, il alla plus loin en cherchant à justifier le recours à la contrainte. C'est là sans doute que nous avons le plus de mal à suivre ses explications. Je ne les retiens pas toutes (Cf. Mandouze, oc pp. 380-387). Quand Augustin nous dit : "Il ne faut pas considérer la contrainte en soi, mais considérer ce à quoi vise la contrainte, si c'est au bien ou au niai" (Ep 1OS, 2), il y a naturellement lieu de s'inquiéter, car qui est habilité à désigner le bien et le mal. Ou quand il distingue entre une "persécution injuste"  - celle qui est faite à l'Eglise du Christ - et
une "persécution juste" - celle que "les églises du Christ font aux impies" -, on peut	( encore s'inquiéter, car il n'y a pas de juge impartial en la question. (Ep. 1 85, 2).

c.	Justification malaisée d'Augustin.  Même s'il la justifie, Augustin n'accepte la coercition qu'avec réserve. On sent qu'il n'est pas à l'aise avec la violence qui est faite aux Donatistes et qu'il essaie de justifier I' injustifiable. Pour la comprendre , il faut regarder à la fois les explications théologiques et son attitude pratique :

-	D'abord, le terme de "persecutio" qu'il utilise ne désigne pas la "persécution". Dans le latin d'Augustin, le terme a encore son sens initial de "poursuite" , de quête du frère égaré. Il s'agit plus, chez Augustin de la poursuite inspirée par l'amour de celui qui s'égare. "Pourquoi donc l'Eglise ne forcerait-elle pas ses fils perdus à (lui) revenir, si les fils perdus en ont forcé d'autres à se perdre?" (Ep. 185).  A l'arrière-plan, il y a la parabole des serviteurs invités à "faire entrer

17
 
de force" dans la maison ceux qu'ils rencontreront "le long des routes et des haies"31

-	Ensuite, il faut éviter l'anachronisme.  Le «forcer à entrer» a une triste r putation. Il a servi au Moyen-Age à justifier l'inquisition, et celle-ci s'est réclamée A gustm.	Or, même s'il en donne parfois une version très intransigeante, Augustin hsa1t non pas "compelle intrare", mais "cogite intrare", le "cogere" mettant l'accent
sur la convergence, le rassemblement, la conviviabilité, plus que sur la contrainte.

- Enfin, il convient de regarder l'attitude  d'Augustin. Sans vouloir absolument le justifier - il faudrait aussi se souvenir que l'époque fut autre -, il ne faut jamais oublier qu'Augustin, en pratique, interviendra souvent pour que les lois soient atténuées et pour que la peine de mort soit écartée absolument. Il était avant tout animé d'une intention évangélique indéniable, intention qui se traduisait par la passion du frère. Ecoutons-le plutôt :

« J'entends en effet /'Apôtre dire : "Insiste à temps et à contre-temps"...Oui, je suis l'homme du contre-temps, j'ose le dire.Tu veux t'égarer, tu veux te perdre : moi, je ne le veux pas. C'est que ne le veux pas, en fin de compte, Celui qui me fait trembler...Pour autant que le Seigneur qui me fait trembler me donnera la force, je parcourraitout : je rappellerai qui s'égare, j'irai quérir qui se perd. Si tu ne veux pas avoir à me subir, veuille ne pas t'égarer ni rechercher ta perte...» (Sermo 46, 7)32


IV.	ADVERSAIRE	DU PAGANISME
Citoyen des cieux

Parmi les adversaires qu'Augustin eut à combattre, les païens occupent une place importante, surtout à partir de 410. Il faut ici rappeler les circonstances avant de regarder comment Augustin organisa la riposte.

1 . La	chute de	Rome

Le 24 août 410, Rome, la ville éternelle, tombe devant l'invasion des hordes barbares d'Alaric. C'est le scandale, même parmi les chrétiens. Depuis un certain temps déjà, on pouvait s'attendre à une telle issue. Mais on ne pouvait pas y croire. Quand la ville tombe, c'est un cri de douleur à travers tout l'empire. Saint Jérôme, un romain de souche, qui vit à Bethléem, est muet de stupeur:

" Quand la lumière la plus éclatante de toute la terre fut éteint, quand l'empire romain fut coupé de la Capitale, quand, pour parler plus exactement, la terre entière périt avec cette seule ville, je suis resté muet et je me suis humilié..." (B.A. 33 pp. 10-11).

Chez les païens, mais aussi chez beaucoup de chrétiens, ce trouble se transforme en accusation. Voici comment Augustin rapporte dans un sermon les accusations que l'on pouvait entendre parmi les païens :



3' cf. Mandouze, oc p. 387.
32 Cf. Coercitio in Augustinus Lexikon.
18
(
 
« C'est au temps du christianisme que Rome est dévastée,que le fer et le feu ont dévasté Rome...Tant que nous avons pu offrir des sacrifies à nos dieux, Rome se tenait de ut; Rome était florissante. Aujourd'hui que ce sont vos sacrifices à vous qui ont pns le dessus et que, partout, ils sont offerts à votre Dieu, alors qu'il ne nous est plus permis de sacrifier à nos dieux, voilà ce qui arrive à Rome...» (Sermo 296, 7)

Quant aux chrétiens, dont certains interprétaient l'événement comme l'annonce de la fin des temps, le trouble n'était pas moindre. Dans le même sermon, Augustin fait écho à leurs récriminations en ces termes :

"Le corps de Pierre repose à Rome, disait-on; le corps de Paul repose à Rome...Et Rome est malheureuse, et Rome est investie; partout l'affliction, le massacre, l'incendie...A quoi servent les tombeaux des Apôtres ?" (Sermon 296, 6)

Pour s'opposer à ces débats tumultueux, Augustin, qui est à Carthage au moment où les premiers réfugiés arrivent en Afrique du Nord, commence par prêcher, et il en parlait si souvent que les fidèles se disaient en allant à l'église : "Pourvu qu'il ne nous parle pas une fois de plus de la chute de Rome" (S. l OS, 12).Augustin entreprend surtout de répliquer par une riposte en règle dans un "grand ouvrage difficile", la Cité de Dieu, dont la rédaction sera quelque peu retardée en raison de l'affaire donatiste qui continue à l'occuper, et sa rédaction ne sea pas menée d'une seule traite. Elle ne sera achevée qu'en 425. Dans les Révisions, Augustin indique quel fut son but en rédigeant cette somme. Un but nettement apologétique :

« Les païens s'efforcèrent de faire retomber ce désastre sur la religion chrétienne ...C'est pourquoi, brûlé par le zèle de la maison de Dieu, je décidai d'écrire contre leurs blasphèmes et leurs erreurs la cité de Dieu...» (Retr. Il, 43, 1)

Le livre devait d'ailleurs très vite dépasser le point de vue particulier qui lui donna naissance, d'autant plus que la vieille cité se remit de son choc et que la vie y continua comme avant. Il ne saurait être question d'aborder tous les thèmes qui s'enchevêtrent. Il faut d'abord remarquer qu'Augustin ne se situe pas au plan politique. La chute de Rome est surtout l'occasion pour lui de célébrer l'unique cité qui ne périra pas, la cité de Dieu.

2.	La réaction d' Augustin33  

La réplique d'Augustin n'est pas d'abord d'ordre politique, bien qu'il y ait des formules assez péremptoires sur les empires terrestres. Son regard n'est pas particulièrement optimiste. Quand il regarde les empires, y compris l'empire romain, il y dénonce d'emblée l'injustice dont ils se sont rendus coupables . Or, sans la justrice, les royaumes sont des sociétés de brigands. "Otez· la justice, et que sont les gouvernements sinon du brigandage à grande échelle ?" (Cité de Dieu 4, 4). On peut relever dans la Cité de Dieu trois thèmes, de connotation spirituelle,  à travers lesquels s'esquisse la réplique d'Augustin :






33 Sermons 81 et 296, in Humeau.
19
 
	L





a.	Aucune cité humaine n'est éternelle. La civilisation romaine est tout	  aussi mortelle que les autres. C'est donc un pur mensonge quand on leur promet une
destinée éternelle. Dans un sermon, Augustin s'insurge contre ce mensonge dont s'est rendu coupable Virgile :

« Ceux qui ont promis (l'éternité) aux royaumes de la terre n'ont pas été conduits par  la vérité, mais la flatterie leur a inspiré ce mensonge...Le ciel et la terre passeront: combien plus vite passera la fondation de Romulus...» (Sermon 1OS, 7)

Il revient sur ce thème dans un autre sermon où s'esquisse déjà le thème des deux cités, mais pour souligner que seule la cité de Dieu est promise à durer, tandis que le monde doit périr:

« Le monde est comme l'homme : il naît, il grandit, il vieillit...Le Christ arrive à l'heure où tout vieillit pour te renouveler toi-même. Le monde créé, le monde fondé, le monde destiné à périr, incline vers le couchant...Ne t'attache pas à ce vieillard qu'est le monde. Ne refuse pas de te rajeunir dans le Christ qui te dit: ...Le monde est travaillé par l'asthme de la vieillesse. Ne crains rien : ta jeunesse se renouvellera comme celle de l'aigle!» (Sermon 81, 8 in BA 33, p. 13).

Il n'y a donc pas lieu de s'affliger quand sombre une cité : c'est le lot de toute civilisation. Et l'occasion de se souvenir que notre cité est ailleurs. Certes, ce qui vient d'arriver à Rome est une épreuve. Mais l'épreuve fait partie de la vie. Certes, elle atteint le juste comme l'injuste. Mais c'est l'occasion de se souvenir que la vie ici­ bas est une mise à l'épreuve où le juste ne fait que suivre les traces de l'unique Juste, le Christ :

« De la main de Dieu qui cherche à amender, la cité a donc bien plutôt reçu une correction qu'elle n'a trouvé sa perte...Ne soyons donc pas troublés en voyant soufflr les justes : c'est là une épreuve, non une damnation...Car la souffrance de cette cité tout entière a été assumée par lui (le Juste des justes) tout seul..» (VIII, 9).

b.	Une nouvelle phase de l'histoire du salut : Elargissant le regard , Augustin médite alors sur l'avenir. Au lieu de s'afflliger sur un monde qui disparaît, Augustin discerne dans ce qui arrive une nouvelle phase de l'histoire du salut. Apparemment, Rome est envahie, mais en réalité, ce sont de nouveaux chrétiens qui
accourent vers elle. Dans son commentaire du psaume 64, 5, Augustin lit le verset : l	"Ad te omnis caro veniet", et il l'applique aux barbares qui affluent vers la foi. Ce qui arrive est un signe avant-coureur de l'universalité de la foi. La cité de Dieu est en
marche parmi les cités humaines. Il commente :

« Ce n'est pas en effet les Romains, mais toutes les nations qui ont fait l'objet d'un véritable serment par lequel le Seigneur les a promises à la race d'Abraham. Et, de cette promesse, il est d'ores et déjà résulté que certaines nations qui ne sont pas soumies à la puissance romaine reçoivent l'Evangile et s'agrègent à l'Evangile qui fructifie et croit dans le monde entier ..." (Ep. 199, 12, 47)34



3◄Cf. Mandouze oc p. 328.

20
 
l
quand on a cette vision d'avenir, on regarde autrement les ennemis qui viennent enva 1r Rome. D'un point de vue humain, ce sont des ennemis dont il faut redouter qu'il détruisent la cité terrestre, mais d'un point de vue chrétien, ce sont de futurs disciples du Christ. C'est donc du point de vue de la cité de Dieu qu'il faut juger toutes choses :

« La cité de Dieu doit se souvenir que parmi ses ennemis mêmes se cachent plusieurs de ses futurs citoyens...De fait, les deux cités sont mêlées l'une dans l'autre en ce siècle, jusqu'au jour où le jugement dernier les séparera. Je vais donc, dans la mesure où la grâce divine m'y aidera, exposer ce que j'estime devoir dire sur leur origine, leur développement, la fin qui les attend...» (1, 35)

c.	Au fondement des deux cités : Du point de vue de la cité de Dieu, le Romain n'est pas meilleur que le Goth, et inversement. Ce n'est pas là un discours très "patriote". Mais pour Augustin, tous les empires sont transitoires, du fait même qu'ils vivent dans le temps. On se plaint de la rigueur des temps. Mais Augustin fait remarquer que chaque époque maudit le temps dans lequel elle vit. Mais, précise Augustin, le "fait que l'époque soit bonne ou mauvaise dépend de la qualtié morale de la vie intellectuelle et sociale, et cela est en notre pouvoir" (S. 80, 8). C'est de ce point de vue qu'il juge les cités. Or, à cet égard, les deux cités sont antinomiques au possible. Il faut lire ici le passage le plus célèbre de la cité de Dieu sur les deux amours qui ont fait deux cités (XIV, 28).

3.	La Cité de Dieu en pèlerinage ici-bas35

Avec la critique des païens, il y avait un défi à relever : «Je décidai d'écrire contre leurs blasphèmes ou leurs erreurs les livres de La Cité de Dieu.» Commencé en 412, ce «grand ouvrage» ne fut achevé que vers 425. Le plan de l'ouvrage donne une première indication sur son projet. Au total, il comporte vingt-deux livres, dont certains circulaient dans le public avant que l'ensemble ne fut achevé.

a)	Plan de la Cité de Dieu. Sa composition se réalisa selon un plan dont Augustin rappelle à plusieurs occasions les articulations36 
- Dans les  dix  premiers  livres, il s'attache à réfuter les païens qui prétendaient lier le bonheur de l'homme au culte polythéiste. Ces dix livres se divisent eux-mêmes en deux ensembles de cinq livres chacun. Tandis que les livres I à V s'en prennent à ceux qui attendent du culte rendu aux divinités païennes la réussite dans les affaires d'ici-bas, et attribuent à son interdiction les maux actuels de la cité, les livres VI à X sont dirigés contre ceux qui, «tout en avouant que de tels maux n'ont jamais manqué et ne manqueront jamais aux mortels..., déclarent pourtant utile le culte des faux dieux, avec les sacrifices qu'on leur offre, à cause de la vie qui doit suivre le mort». On reconnaît sans peine dans ces derniers les platoniciens : ils placent en effet le bonheur non pas dans le monde sensible, mais au contraire dans la participation à la vie des dieux. «Philosopher, disait Platon, c'est aimer Dieu dont la nature est incorporelle. 'P » Mais Augustin ne critique pas moins les platoniciens, car

35 Cité de Dieu, 1, 1. BA 33, p. 191. En XIX, 17, il dit qu'elle est en exil. BA 37 p. 129. Pour le plan de l'oeuvre, voir Jean-Claude GUY, Unité et structure de la "Cité de Dieu" de saint Augustin. Etudes Augustiniennes, 1961.
35 Cf. Les Révisions, BA 12, p. 525. Voir aussi l'introduction à La Cité de Dieu, BA 33, p. 35 s.
37 La Cité de Dieu VIII, 8. BA 34, p. 261.
21
 
s'ils ont vu «où il faut aller», la «patrie»,	ils n'ont pas vu «par où» y aller, faute	  de croire en «celui qui est la voie conduisant non seulement à la vue, mais encore à l'habitation de la patrie bienheureuse»38  

- La seconde partie de l'ouvrage (XI à XXII) s'attache à présenter la foi chrétienne sous un jour positif. Augustin comprit très vite qu'il ne suffisait pas de réfuter les païens, mais que la crédibilité de son projet dépendait de sa capacité à justifier sa propre religion chrétienne. Cette seconde partie comporte douze livres subdivisés en trois parties. «Des douze derniers livres donc, les quatre premiers traitent de l'origine des deux cités, dont l'une est la cité de Dieu, l'autre la cité de ce monde. Les quatre suivants décrivent leurs progrès, leurs développements. Les autres, qui sont aussi les derniers, montrent les fins qui leur sont dues. Ainsi, tous ces vingt­ deux livres ont pour thème l'une et l'autre cités; mais ils ont emprunté leur titre à la meilleure et ils sont appelés de préférence De la cité de Dieu.5 » Ayant ainsi précisé le plan, Augustin donne des indications pour la reliure de l'ouvrage, souhaitant que l'on respecte soit la division en deux volumes, soit une division en cinq volumes : «Si tu
préfères qu'il y ait plus de deux volumes, il faut en faire cinq», les deux premiers	  (cinq livres chacun) étant consacrés à la réfutation du paganisme, les trois suivants
(quatre livres chacun) à la justification de la foi chrétienne«>.

b)	Evaluation du projet d'Augustin. Ces indications formelles sur le plan de l'ouvrage seraient négligeables si elles se limitaient à des «consignes de reliure» 1. En réalité, elles dévoilent l'intention d'Augustin : La Cité de Dieu est avant tout une défense et illustration  de  la «vraie religion», c'est-à-dire de la foi chrétienne. Comme l'a montré Goulven Madec, Augustin poursuit ici un projet comparable à celui, déjà ancien, du De vera religione, qui date de 390. Dans l'un comme dans l'autre, il développe la thématique du bonheur, ainsi que celle de l'accès à la transcendance : ce sont des thèmes communs à la philosophie et au christianisme. Ces deux ouvrages tracent l'un et l'autre une nette ligne de partage entre platonisme et christianisme, la principale déficience du premier étant de n'avoir pas su reconnaître le Christ fait chair, unique médiateur entre Dieu et les hommes. Ce point de rupture, les Confessions l'avaient déjà souligné avec toute la netteté désirable : «Là, j'ai lu...là je n'ai pas lu» ! Là, chez les platoniciens, Augustin a «lu» que le Verbe est éternel, mais non pas «lu» qu'il s'est fait chair. Si les platoniciens «connaissent Dieu», «ce n'est pas comme Dieu qu'ils le glorifient ou lui rendent grâces»4Z, puisqu'ils ne le
reconnaissent pas dans la chair, et ne le confessent pas comme Médiateur unique entre	1
Dieu et les hommes.

Si tel est le projet d'Augustin, il est clair qu'en opposant les deux cités, i 1 n'oppose pas deux  réalités  identifiables  dans l'ordre  empirique, mais il s'en prend à un type très précis de la cité où les deux réalités, politique et religieuse, n'en faisaient qu'une. Ce qu'il conteste, ce n'est pas la cité humaine en tant que telle - il en justifie la nécessité et en rappelle les devoirs  - mais la collusion qu'elle

38 Confessions VII, 20, 26.
39 Les Révisions Il, 43, 2 BA 12, p. 527.
40 Lettre 1 A' à Firmus. BA 46 B, p. 57. C'est cette division en cinq volumes qu'a adoptée la Bibliothèque augustinienne. BA 33 à 37.
41 Cf. Goulven MADEC, Le De civitate Dei comme De vera religione, dans : «Peti1es études augustiniennes». Institut d'Etudes Augustiniennes, 1994, p. 194.
42 Confessions VII, 9, 13-15.
22
(
 
entretenait entre le politique et le religieux, deux aspects indissociables dans la cité romaine, l'unité politico-religieuse, écrit Ratzinger, étant constitutive de son essence. li faut garder à l'esprit cette conception religieuse de la cité si l'on veut comprendre les griefs émanant des païens à l'encontre du Dieu chrétien lors de la chute de Rome en 41 O. Les païens étaient persuadés qu'en ayant abandonné ses dieux, Rome avait perdu ses protecteurs et ne pouvait que succomber sous les hordes barbares. Les coupables sont donc les chrétiens, puisqu'ils furent les grands démolisseurs du polythéisme. C'est cette accusation qui a provoqué la réaction d'Augustin. Son projet n'est pas de condamner l'organisation politique de la cité - elle est nécessaire - , mais de dénoncer la collusion de la cité avec les dieux païens. fi veut montrer au contraire qu'il n'existe aucun lien entre la suppression de ce culte et la chute de Rome.

Les contours de ces deux cités ne sont pas visibles à l'oeil nu, car elles sont à entendre en un sens allégorique. Ce sont deux réalités spirituelles, deux catégories d'hommes, «ceux qui vivent selon l'homme, ceux qui vivent selon Dieu»43  Elles sont symbolisées par Jérusalem et Babylone, une opposition qui est familière à Augustin, puisqu'on la trouve déjà vers 400, dans sa «Première catéchèse». «Deux Cités, écrit-il, celle des impies et celle des saints, s'avancent depuis l'origine du genre humain jusqu'à la fin du monde; à présent mêlées quant à leurs corps, mais séparées par leurs volontés, au jour du jugement elles seront aussi séparées de corps...». Puis il ajoute : «Or, de même que Jérusalem symbolise la cité et la communauté des saints, de même Babylone symbolise la cité et la communauté des injustes, car le mot signifie, dit-on, confusion....Ces deux cités, depuis l'origine du genre humain jusqu'à la fin du monde, parcourent mêlées l'une à l'autre les diverses époques et qui doivent être séparées au jugement dernier... »44  Mais allégorie ne signifie pas utopie. Cette réalité spirituelle est inscrite dans l'histoire, en négatif comme en positif, bien qu'aucune cité terrestre, pas même l'Eglise, ne réalise pleinement l'idéal de la cité de Dieu.

Si, au regard de leur essence, les deux cités sont diamétralement antagonistes, l'une ayant  comme principe «l'amour de soi jusqu'au mépris de Dieu», l'autre
«l'amour de Dieu jusqu'au mépris de soi»45, au regard de leur inscription dans le temps, elles forment un  mélange  inextricable <46. «De fait, les deux cités sont mêlées et enchevêtrées l'une dans l'autre en ce siècle, jusqu'au jour où le jugement dernier les séparera. Je vais donc, dans la mesure où la grâce divine m'y aidera, exposer ce que j'estime devoir dire de leur origine, leur développement, la fin qui les attend. Je servirai par là la gloire de la Cité de Dieu qui, comparée ainsi à l'autre, se détachera par opposition avec un plus vif éclat.47» Cet enchevêtrement interdit toute séparation prématurée. A Parménien qui excitait à la révolte contre les pervers, anticipant sur le jugement de Dieu, Augustin répond : «La même assemblée les unit en vérité jusqu'au vannage final... »48  Les principes antagonistes, constitutifs des deux cités, s'affrontent aussi bien dans le coeur de l'homme que sur la scène politique du monde. La traduction empirique de la cité de Dieu sera toujours imparfaite et souvent trompeuse. Loin de

43 La Cité de Dieu XV, 1, 1 : où il affirme qu'il parle «en langage allégorique»
44 La Première catéchèse, 19, 31, et 21, 37. BA11/1, p. 157 et 175.
45 La Cité de Dieu XIV, 28 BA 35, p. 465.
 e Cf. Joseph RATZINGER, Herkunft und Sinn des Civffas-Lehre Augustinus, dans Augustinus Magister. Congrès international augustinien, Paris, 21-24 septembre 1954. Et. aug. , p. 965 s.
47 La Cité de Dieu 1, 35. BA 33 p. 301.
48 Contre la Lettre de Parménien, Ill, 3, 19. BA 28, p. 443.
23
 
dénigrer la cité t
et il escom t	er e
 

str ,. Augustin sait reconnaître la vertu des hommes politiques,
 
les retombpe aussi dès 1c1- as au moins une ébauche de la cité de Dieu dont il souligne élimi	ées au plan social. Mais il ne croit pas à la possibilité d'instaurer l'une en
nant totalement l'autre. Le mélange durera jusqu'à la fin des temps.

Tout en accordant une valeur absolue à la Cité de Dieu Augustin ne reste nullement à l'écart des réalités qui font la vie de la ité humaine. Le seul arcou de la table analytique<111 , à la fin de l'ouvrage,  fournit une liste
. tmpress1onnante, non seulement des questions religieuses comme le culte ou le sacnfice, ou éthiques comme la peine de mort, le suicide ou la légitime défense, mais encore des principaux aspects de la vie politique. Augustin envisage même une redistribution des impôts plus équitable qui serait à l'avantage des pauvressi. La clef de voûte de la cité humaine est en tout état de cause la justice : «Supprimée la justice, que sont les royaumes sinon de vastes brigandages ? Car les brigandages eux-mêmes, que sont-ils, sinon de petits royaumes?51 » Son objectif doit être la paix, un bien auquel tous aspirent, même ceux qui font la guerre, même les brigands, bien qu' entre la paix des cités terrestres, si parfaite soit-elle, et la paix de la cité céleste, il n'y ait pas de commune mesure : «La paix de la cité, c'est le concorde bien ordonnée des citoyens dans le commandement et l'obéissance; la paix de la cité céleste, c'est la communauté parfaitement ordonnée et parfaitement harmonieuse dans la jouissance de Dieu et dans la jouissance mutuelle en Dieu52    » Quantité d'autres thèmes affleurent, qu'une «philosophie politique» aura toujours intérêt à méditer : le gouvernement de la cité, le sens de l'autorité et ses fonctions, l'usage de guerre, etc.53

S'il énonce des principes politiques,  Augustin n'ébauche à aucun moment un programme politique. C'est sa limite, qui tient en réalité à sa conception de l'histoire, une conception qu'il partage en fait avec les grands historiens de la Rome antique54 et qui se caractérise par deux traits. D'une part, elle est essentiellement morale. Quand il s'agit de comprendre la grandeur ou la décadence de Rome, l'historiographie romaine fait appel à des motifs non pas politiques ou sociaux, mais moraux : la grandeur est liée à la vertu, la décadence au vice. D'autre part, et en voie de conséquence, lorsque, dans une situation de crise, il s'agit de trouver une solution, les hommes politiques s'appuient moins sur l'analyse de la situation que sur le passé,
en s'inspirant des exemples laissés par les hommes illustres d'autrefois. Cette
conception de l'histoire est aussi celle d'Augustin, bien qu'il n'accepte pas sans nuance	(
le lien causal entre décadence et vice, entre prospérité et vertu, du moins dans le temps, tandis que le recours aux exemples des anciens lui est familier. C'est à partir de ces figures exemplaires qu'il argumente assez fréquemment, figures dont les noms sont dans la mémoire de tous les Romains 55   Il les donne même en exemple aux

◄0 La Cité de Dieu BA 37, p. 898 s.
50 La Cité de Dieu 1, 17, 1. Voir aussi note BA 33, p. 717.
51 La Cité de Dieu IV, 4. BA 33, p. 541.
52 La Cité de Dieu XIX, 23. BA 37 p. 113.
53 Il faudrait compléter le tableau en consultant d'autres écrits d'Augustin, par exemple les lettres 91 et 138, adressées à Nectorius et à Marcellinus, sur la patriotisme et la citoyenneté.
5  Cf. Viktor POSCHL, Augustinus und die rômische Geschichtsauffassung, dans Augustinus Magister, oc p. 957 s.
55 La Cité de Dieu V, 21 : «Certes Dieu réserve exclusivement aux bons le bonheur dans le royaume du ciel; mais il accorde le royaume de la terre aux pieux et aux impies comme il lui plait... Si ses raisons sont cachées, sont-elles injustes?» BA 33, p. 739 et 743.
24	(
 
chrétiens, afin de susciter parmi eux une saine émulation : «Qu'ils fi ent sur ces exemples un regard attentif et sage et voient quel amo r est dû à la patne.céleste en vue de la vie éternelle, quand la cité terrestre est tant a1mee de ses citoyens en we de
la gloire humaine.56 »
Mais on ferait fausse route si l'on cherchait dans la Cité de Dieu un livre de philosophie politique. De ce point de we, on pou ait même y  v ir,  une, contre­ hi st o ire», en ce sens que, à la différence des h1stonens, Augustm n apprec1e pas les événements selon les critères habituels : la grandeur, les victoires, la splendeur, etc. S'il s'appuie sur les mêmes événements que les adversaires, il les interpréte à l'encontre de leurs intentions. C'est ainsi que, à la différence de Cicéron qui écrit sa
«République» pour prouver que l'histoire romaine est en progrès constant vers la justice, Augustin y dénonce une vaste «entreprise de pillage»57  11 apprécie l'histoire de Rome non à sa gloire, mais à ses progrès spirituels. La justice et l'amour peuvent être en progrès au sein même d'une cité qui s'effondre. Alors que la Rome politique est menacée de décadence, la cité de Dieu y progresse de jour en jour car le progrès se mesure à l'avancée de la charité. «Voilà l'homme nouveau, l'homme intérieur, l'homme céleste. li a, lui aussi, analogiquement, ses âges spirituels, que l'on distingue non à ses années, mais à ses progrès.58» Cette «contre-histoire» qu'ébauche Augustin cherche non pas à «sauver les phénomènes», mais veut inciter l'âme à se tracer son itinéraire vers Dieu sans se laisser troubler par les événements.

Tout comme les Confessions traçaient cet itinéraire pour l'âme, la Cité de Dieu le fait  à l'échelle d'un peuple. Augustin ne demande pas à ce peuple de mettre un trait sur les vertus naturelles qui ont fait sa grandeur, mais d'accepter que ces vertus soient purifiées au feu de la foi. Voici comment il interpelle Rome : «Si brille en toi un don naturel estimable, seule la vraie piété peut le purifier et le perfectionner; l'impiété le fait périr et consomme sa ruine. Choisis maintenant ta route pour obtenir d'être loué sans erreur, non en toi-même, mais dans le vrai Dieu. Tu fus autrefois en grand renom parmi les peuples, mais par un secret jugement de la Providence, la vraie religion a manqué à ton choix. Réveille-toi, c'est l'heure...58 » Rome a encore un avenir, à condition de se convertir, c'est-à-dire de délaisser son paganisme et de suivre la trace des martyrs. Comme eux, les Romains doivent s'emparer de la «patrie céleste» au lieu de cultiver la nostalgie du passé : «A cette patrie, nous te convions : viens, nous t'en prions, t'adjoindre à ses concitoyens... » La religion chrétienne est l'unique chance, car elle est seule à offrir la voie qui conduit à la Cité de Dieu.

Conclusion:  Chadwick écrit: "La cité de Dieu ne saurait être considérée comme
n-  tr ité	d	théorie	politique,	ou	comme	l'expression	d'une	philosophie	de
1
1 h,stoire...L h1sto1re du monde voit l'essor et la chute des grandes puissances et la a,son, de c phén mène est loin d'être claire... Mais le croyant sait que ce ui est mcoherent a l'esprit de l'homme est cohérent pour Dieu. Les désastres peuvent nous arracher des larmes, mais ne devraient nous étonner en aucun cas (E 111, 2). Augu tin offre beaucoup plus d'espoir à l'individu qu'aux institutions de la société humaine, particulièrement sujettes à transmettre un égoïsme collectif.m "
59 La Cité de Dieu V, 16. BA 33, p. 715.
57 La Cité de Dieu Il, 21. BA 33, p. 369 s.
58
Non annis, sed provectibus. De vera religione 26, 49. BA 8, p. 92 s.
5 La Cité de Dieu Il, 29, 1. BA 33, p. 405.
eo Henry CHADWICK, Augustin, cerf, 1987, p. 143
25
 
-  --  ..,-, .....  uuu,o ,c 1.n.,e ut: uucleur oe ta grace. l. es1. te comoal cunut: r-t:1c1yt:
qui l'amènera à cetrtaines dérives théologiques en particulier sur le péché originel61   Avant d'aborder l'enjeu théologique de la question, il faut rappeler les principales étapes de l'histoire qui a entrainée la condamnation de Pélage. Puis nous pourrons évoquer les conséquences.

1 . Première  vague : les pélagiens  en Afrique.

Originaire des lies britanniques, où il naquit vers 350, Pélage était établi à Rome dès 382. Il y enseigne donc depuis de longues années au moment de la chute de la ville en 41O. On l'a décrit comme une «nature froide», imbu d'une «estime exagérée de son propre mérite». Mais c'est là sans doute une lecture rétrospective. D'autres soulignent au contraire son «esprit religieux». Paul Orose le caricature en le décrivant comme un Goliath, allusion sans doute à son physique, et Jérôme le traite de
«gros chien de montagne» et même de «grand abruti». En réalité, sa vie, sa personnalité, sa spiritualité ont été déformées par ses adversaires. Autant avouer qu'il nous est mal connu6'Z . C'est un laïc qui entraîne dans les voies spirituelles un certain nombre de chrétiens et de chrétiennes en leur prêchant l'effort et l'ascèse.

Au moment de l'invasion des barbares, Pélage se réfugie comme beaucoup d'autres en Afrique du Nord, puis en Palestine. Ses idées ne retiennent pas particulièrement l'attention jusqu'au moment à ses disciples les répandent en Afrique du Nord, en particulier un certain Caelestius63   On peut retenir différentes phases au terme desquelles Pélage fut définitivement condamné :

-	Dès 411, CéIest i us fut condamné par un concile à Carthage auquel Augustin ne participa pas. Pélage, qui se désolidarisa de Célestius, échappe à une première condamnation, lors du concile de Diospolis, en 41 S. Entre temps, Augustin avait pris connaissance des écrits de Pélage. Il souligna d'emblée les équivoques du concile de Diospolis, et mit en évidence les conséquences de la doctrine de Pélage pour la foi catholique. En 416, les Africains, réunis à Carthage et entraînés par Augustin, renouvellent leur condamnation de la doctrine pélagienne.

-	La condamnation est tranmise au Pape Innocent 1er afin d'obtenir confirmation . Le 27 janvier 417, Innocent 1er leur répondit par plusieurs lettres, les félicitant d'avoir consulté le Saint-Siège et faisant l'éloge de leur vigilance. La condamnation de Pélage leur paraissait sans appel. S'adressant au peuple de Carthage,
61Cf. La synthèse de cette question dans «L'homme et son salut», in Histoire des dogmes, sous la direction de Bernard Sesboüé. Declée, 1995, p. 150 s.
e2 Cf G. de Plinval, Pélage, ses écrits, sa vier et sa réforme. Lausanne, 1943. Voir aussi Goulven Madec, dans Itinéraires Augustiniens n° 13 (janvier 1995).
e3 Augustin n'écrira pas moins de 17 traités contre eux : 8 contre Pélage et Caelestius, 4 contre Julien d'Eclane, 5 contre les semi-pélagiens, sans compter les sermons et les Lettres.
 
Aug stin ne cache pas sa joie, estimant que la question est réglée. «Déjà deux conciles (Carthage et Milev, été 416) ont envoyé leurs décisions au Siège Apostolique dont les rescrits sont parvenus. La cause est finie, plaise à Dieu que l'erreur finisse de même (Sermo 131, 1O). causa finita est...

-	En réalité,. elle était déjà en train de prendre une nouvelle tournure, car le successeur d'innocent, Zosime, qui connaissait mal le dossier, et qui se laissa fléchir par les déclarations d'intention de Célestius et Pélage. Zosime annula la condamnation en 41 7. Il convoqua un synode romain qui «eut peine à retenir ses larmes en constatant qu'on avait pu diffamer de la sorte des hommes d'une foi aussi parfaite» ! Accusés d'avoir agi inconsidérement, les Africains réagirent à leur tour très violemment.

-	Un nouveau concile est convoqué à Carthage en 418. Entre temps, à Rome, du fait de Célestius, les désordres se multiplient, si bien que les pélagiens finissent que se rendre odieux. Le pape accepta que fut constituée une commission, dont fit partie Augustin, qui eut la mission d'examiner la question de fond, et il se rallia à sa conclusion.

-	Après 418, Pélage disparaît de la scène . Mais pour Augustin, le combat allait se poursuivre, notamment contre Julien  d'Eclane - un disciple de Pélage, né en 386 et élu en 416 comme évêque d'Eclane - qui refusa de se soumettre et continua la lutte. Augustin lui répliquera dans différents traités, soutenant inlassablement la même thèse : la nécessité de la grâce même pour les enfants, en raison du péché originel.

2.	Deux  théologies  incompatibleses

Pour saisir l'enjeu du débat, nous considérerons d'abor la thèse de Pélage, puis celle d'Augustin. Pour éclairer le débat, il faudra s'interroger sur les raisons de leur divergences. Commençons par la thèse de Pélage, une thèse qui nous est surtout connue par ses adversaires. Cette thèse présente deux faces.

-	L'homme est libre. A la création, Dieu a donné à l'homme la liberté et la raison, donc la possibilité d'exécuter volontairement le dessein de Dieu et ainsi de mériter par lui-même le salut. Pélage exalte la liberté humaine, plus exactement le libre arbitre, cette capacité dont dispose l'homme de choisir entre le bien et le mal. Fondamentalement, il insiste sur la valeur de l'homme. La perfection n'est pas seulement une possibilité de la nature humaine, mais une obligation morale. Pélage prêche une religion volontariste. En 413, il écrit à Oémétriade, décidée à se faire religieuse : «Chaque fois que je dois donner des règles de conduite et tracer la voie d'une vie sainte, je mets toujours en premier fieu l'accent sur la puissance et la valeur de la nature humaine et sur ce qu'elle est capable de réaliser de peur que ne serve à rien d'exhorter les gens à entreprendre une tâche qui leur apparaitrait impossible à accomplir. » C'est une théologie de l'autonomie humaine, marquée au coin d'un certain naturalisme et d'un certain rationalisme.



e◄Ad Demetriadem 2 (PL 30, 17 B), cité in Peter BROWN, La vie de saint Augustin, Seuil, 1971, p.
406.
27
 
- Dieu est	juste. C'est l'autre aspect de la thèse. Etant juste, Dieu ne peut que	e
récompenser les bons et punir les méchants. Admettre l'idée d'un péché héréditaire
serait contraire à la morale. Adam n'est pas cet être mythique qui nous aurait tous entraîné dans le péché, mais c'est chacun de nous. Pélage n'ignore pas le Christ, mais il le considère seulement comme un modèle à imiter. La grace du Christ n'est pas une rédemption, mais un «exemple» à imiter,  un secours donc purement extérieur.
«Dieu est à côté de l'homme, non en lui.fi5 »

 
GRACE
[salut	gratuit]


Liberté

HOMME
LIBRE	ARBITRE
[ capacité de choisir]





Salut	mérité [un droit de l'homme]
 



REDEMPTION





CREATION
 


A ce naturalisme, Augustin défend la thèse de la grâce. Dès 414, il entreprend de réfuter le livre de Pélage sur la «nature humaine». Il insiste sur deux aspects:

-	Une		nature	blessée.	Alors que Pélage insiste sur la bonté de la nature humaine, pensant ainsi rendre hommage et justice à son «créateur», Augustin lui reproche d'oublier que cette nature est blessée par le péché, ce qui la rend incapable de faire le bien sans l'intervention du «rédempteur».		Voici comment Augustin s'exprime :	«Mais, quand il (Pélage) croit servir la cause de Dieu en défendant la nature, notre auteur ne remarque pas qu'en déclarant cette nature saine, il évince la miséricorde du médecin...Or, nous ne devons pas louer le créateur de façon à nous trouver réduits à avouer que le sauveur est inutile.66»	Augustin n'ignore pas le libre arbitre, mais le pouvoir de choix ne devient efficace que s'il est d'abord libéré.

-	La nécessité du rédempteur. Alors que Pélage développe une théologie de la création, Augustin se fait le champion d'une théologie de la rédemption. Alors que Pélage met l'accent sur la liberté humaine, capable de choisir entre le bien et le mal, et donc capable de mériter son salut, Augustin estime que la liberté étant aliénée, elle ne devient capable de faire le bien que si d'abord elle a été restaurée dans son pouvoir
es Isabelle BOCHET, Saint Augustin et le désir de Dieu. Et. aug. , 1982, p. 320.
ee De natura et gratia 34, 39, in BA 24. Ppour une présentation d'ensemble, voir Paul Agaësse, L'anthropologie chrétienne selon saint Augustin. Image, liberté, péché et grâce. Centre Sèvres, Paris, 1980.
28	(
 
de,.1e faire. Or, étant donné le péché originel, seul Dieu peut redonner à l'homme ce qu ''· a perdu. If n'est donc pas légitime de s'attribuer à soi-même un quelconque ménte. Toute notre justification nous vient de Dieu et le salut est un don gratuit que nul n'a mérité par soi-même. "Tous ont péché" (Rm 5, 18), ce qui signifie que "personne n'est justifié si le Christ ne le justifie"67   Seule la foi dans le Christ rédempteur peut sauver l'homme, «la foi en Jésus Christ fait homme, la foi en son sang, la foi en sa croix, la foi en sa mort et en sa résurrection !88 » Il n'y a donc pas à hésiter sur la nécessité absolue de la grâce dans la condition de l'homme pécheur :
«Reconnaissons que la grâce est nécessaire; crions : Malheureux homme que je suis! qui me délivrera de ce corps qui me voue à la mort? Et que l'on réponde: la grâce de Dieu, par Jésus-Christ notre Seigneur.18 »

Quel est l'enjeu du débat ? Regardons d'abord la thèse	de Pélage. En gros, Pélage insiste fondamentalement sur la valeur de l'homme. A la création, Dieu a donné à l'homme la liberté et la raison, donc la possibilité d'exécuter volontairement le dessein de Dieu et donc de mériter par lui-même le salut. C'est une théologie de l'autonomie humaine, marquée au coin d'un certain naturalisme et d'un certain rationalisme. Pélage vantait les capactiés de la nature humaine. Cet optimisme au sujet des capacités de l'homme devait le conduire	à précher un volontarisme moral . L'homme pélagien est un être "sans reproche" :

« Ce n'est pas grand-chose, disait-il d'être un exemple pour les païens...Ce qui est beaucoup mieux, c'est d'être tel que les saints eux-memes soient édifiés.» (Cf. E.U.)

En face de ce naturalisme, qui menaçait de dévaloriser la grâce, Augustin réagit vivement en défendant la thèse de la grâce. Alors que Pélage insiste sur la bonté de la nature humaine, pensant ainsi rendre hommage et justice au "créateur" de cette nature, Augustin lui reproche d'oublier que cette nature a été blessée par le péché, ce qui la rend incapable de faire le bien et nécessite l'intervention du "rédempteur" et de la grâce. Voici comment Augustin s'exprime:

« Mais, quand il croit servir la cause de Dieu en défendant la nature, notre auteur ne remarque pas qu'en déclarant cette nature saine, il évince la miséricorde du médecin...Or, nous ne devons pas louer le créateur de façon à nous trouver réduits à avouer que le sauveur est inutile.» (De natura et gratia 34, 39, BA 24).

A l'arrière-plan de ce débat, on distingue à la fois deux  expériences différentes.

-	L'expérience d'Augustin est celle d'un converti qui s'est heurté à ses propres limites et pour qui sa conversion est l'expérience fondatrice de sa propre pensée. Le mouvement premier d'Augustin est toujours l'action de grâce pour ce que Dieu a fait de sa vie. Comme l'a bien noté G. Greshake : «Il est impossible, dans la prière, de se présenter à Dieu et de lui dire : Je dois mon salut à toi et à moi, à ce que tu as fait et à ce que j'ai fait. Dans la confessio, on peut dire seulement : ce que j'ai, je



67  De natura et gratia 41, 48.
88 lb. op. cit. 44, 51.
av Rom 8, 24-25, ib. op. cit. 53, 61.
29
 
te ledois.7»0	li dira encore : «Tu es miséricorde, je suis misère.» (Conf. X, 28, 39).	(

-  Pélage nous est  trop peu  connu pour savoir à quelle expérience pers n elle    e réfère. Ce qui est sûr, c'est qu'il représente un type de chnst,anisme qui fait appel en priorité non pas à la grâce, comme Augustin, mais aux capacités de l'homme.

3.	Nouvelle vague : les moines d' Adrumète et Julien d'Eclane

Le jeu n'était donc pas calmé avec la condamnation de Pélage. Une nouvelle phase	C
allait se produire,	provoquée soit par certains disciples trop dociles d'Augustin, tels
 
les	moines du couvent africain d'Adrumète (Sousse, au Sud de Carthage), ou au
 
'il rt
 
contraire par des disciples de Pélage, tel Julien d'Eclane.

- Les moines d' Adrumète répandirent, vers 425-426, l'idée que, puisque Dieu fait tout, la liberté n'est rien et n'a rien à faire. Prenant trop à la lettre certaines affirmations d'Augustin sur la grâce, ils en tiraient que la liberé est inexistante. Alors que les Pélagiens niaient la grâce au nom de la liberté, à l'inverse, ces moines défendaient «la grâce de Dieu jusqu'à nier le libre arbitre de l'homme». Augustin leur
 

rbit
· il
la
 
fit savoir que «confesser la grâce comme il l'avait fait, ce n'était pas nier le libre
 
'at
 
arbitre ni le mérite...» Il écrit  à  Valentin, abbé du monastère :	ma
 
« Certains parmi vous exalteraient la grâce au point de nier le libre arbitre de
 

./IYi
 
l'homme, et, ce qui est plus grave, soutiendraient qu'au jour du jugement Dieu n'aura
 
liJ
 
pas à rendre à chacun selon ses oeuvres...» (p. 53)	t
«S'il n'y a pas de grâce divine, comment Dieu sauve-t-il le monde ? Et s'il n'y a	à
 
pas de libre arbitre, comment juge-t-il le monde ?...Ne niez pas la grâce de Dieu, et
 
mp1
 
ne défendez pas le libre arbitre jusqu'à le rendre indépendant de la grâce de Dieu,
comme	si nous pouvions sans elle concevoir	ou	accomplir quelque chose	selon	m
Dieu 11     » (p.  SS)

 
Augustin invite à tenir ensemble les deux vérités : le libre arbitre	et
 

er1
 
la grâce. Il énumère alors une série d'affirmations tirées de !'Ecriture, les unes en
faveur de la grâce, les autres en faveur du libre arbitre. Si l'on ne peut pas	fr
comprendre comment les deux s'articulent, il faut s'en tenir à !'Ecriture qui .les
affirment simultanément : "Elle nous enseigne à la fois la réalité du libre arbitré · humain et la réalité de la grâce divine; grâce sans laquelle le libre arbitre ne peut ni se tourner vers Dieu ni progresser en Dieu.» (p. 59). Y a-t- il une coordination possible des deux termes ? Augustin donne à la grâce une extension telle qu'elle englobe la liberté. La liberté est déjà grâce. La formule la plus parfaite pour exprimer le lien
entre les deux se trouve dans le "De correptione et gratia"n :	rE
nne
les
10 G. GRESHAKE, Geschenkte Freiheit. Einführung in die Gnadenlehre, Freiburg, 1977, p.
47.	Cité par Goulven MADEC, Pélage et Augustin. Le débat sur la liberté et la grâce.
Itinéraires augustiniens 13 (janvier 1995).
" De gratia et libero arbitrio. B.A. 24, p. 55. "Il en est qui défendent la grâce de Dieu
jusqu'à nier le libre arbitre de l'homme".	t,
12 B.A 24 p. 276 cf. aussi B.A. 73 B : page 480, la grâce et la liberté, et p. 477 ,
préscience divine et liberté humaine.
30
 
«Aguntur  enim  ut	agant,  non  ut	ipsi nihil agantl» (11,  4)
«Ils sont en effet	agis pour agir, non pour ne rien fairel»

-  Avec  Julien d'Eclane, le défenseur  de la liberté, le combat allait prendre une tournure plus dure. Julien invoquait à l'encontre d'Augustin un certain nombre de textes  sur la liberté - en fait sur le libre arbitre - que celui-ci avait écrits autrefois. Augustin sera obligé de rétablir ainsi l'équilibre, mais en mettant de plus en plus l'accent sur la grâce, si bien qu'on peut parfois se demander s'il réussit encore à reconnaître la place de la liberté. Il tire de !'Ecriture une série d'affirmations, les unes en faveur de la grâce, les autres en faveur du libre arbitre. Si l'on ne peut pas comprendre comment les deux s'articulent, répond Augustin, il faut s'en tenir à !'Ecriture qui les affirment simultanément : «Elle nous enseigne à la fois la réalité du libre arbitre humain et la réalité de la grlce divine; grlce sans laquelle le libre arbitre ne peut ni se tourner vers Dieu ni progresser en Dieu.»

L'affrontement entre Augustin et Pélage se joue sur l'idée que l'un et l'autre se font  de la liberté. Pour Augustin, le libre arbitre n'est pas la liberté, mais il est  la simple capacité de choisir,  de disposer librement de sa propre volonté.
« Notre volonté ne serait même plus volonté si elle n'était en notre pouvoir. Mais puisqu'elle est en notre pouvoir, elle est libre pour nous; car ce qui n'est pas libre pour nous, c'est ce qui n'est pas en notre pouvoir; et ce qui l'est ne peut pas ne pas être libre.73» Pélage n'a aucune peine à accepter cette définition. li la tirait même à lui pour démonter que Augustin n'était pas toujours Augustin, c'est-à-dire le champion intransigeant de la grâce. Seulement, alors que pour Pélage, le libre arbitre, confondu avec la liberté, est la capacité de l'homme à choisir entre le bien et le mal, pour Augustin, il se restreint  à la responsabilité de l'homme dans le mal : Il est la
«facuitas peccandi» (1, 16, 35).  Il est rare qu'il fasse appel au libre arbitre dans le choix du bien74   L'orientation vers le bien relève de· la liberté, la  «vraie»  liberté se manifestant dans la soumission à Dieu et n'existant que par la grâce. Or, la liberté qui s'est perdue en raison du péché, est incapable de se réenraciner dans le bien. Pour agir de nouveau selon Dieu, il faut une restauration de la liberté par la grâce. Pélage qui ne connaissait que le libre arbitre ne réalisait pas que, sans la grâce, donc sans la vraie liberté,  le libre arbitre était une serf-arbitre.

Si l'on voulait entrer plus avant dans ce débat, il faudrait évidemment évoquer le problème  de  la  prédestination, qui vient se greffer sur celui de la liberté. Puisque la liberté de l'homme n'est rien sans la grâce, n'est-elle pas une pure illusion, totalement régie de l'extérieur par une autre liberté, celle de Dieu, qui donne ou refuse sa grâce avant tout mérite de l'homme? C'est cette thèse que soutiendront les jansénistes, dont on reparlera. Augustin est moins radical. Mais ses propos sont souvent ambigus, sinon choquants 75   La polémique l'a parfois entraîné dans certains

73 Sur le libre arbitre 111, 3, 8. Cf. G. Madec, Itinéraires augustiniens 13 (janvier 1995.
H Cf. Enchiridion XXVIII, 106. «Car si le péché dépendait du libre arbitre seul, cependant, pour conserver la justice, lelibre arbitre ne suffisait pas si la la participation du bien immuable ne lui assurait le secours divine.»
H Cf. Enchiridion XXIII, 93. BA 91 p. 269 : «Bénigne entre toutes,à ne pas douter, sera la peine de ceux qui au péché qu'ils tenaient de leur origine se gardèrent d'en ajouter aucun.» Cette thèse a été exacerbée par les jansénistes et les calvinistes, qui ont introduit l'idée d'une prédestination négative, Dieu décidant arbitrairement, d'avance et de façon absolue, de
la destinée éternelle de chaque être.
31
 
dé pages, par exemple lorsqu'il écrit, à propos des vertus des païens, qu'elles ne sont	(
qu enflure et orgueil et, on doit, à ce titre, les regarder, non comme des vertus, mais
comme des vices»76 ; ou à propos des enfants non baptisés, lié au problème de la Prédestination, qui sont à jamais exclus de la béatitude, même si leur peine est
atténuée. La prédestination est un mystère qui renvoie à l'insondable volonté de Dieu. Si Augustin n'a pas toujours la formule équilibrée de ce mystère, puisque la volonté de Dieu est, d'une part «que tous les hommes soient sauvés», mais que d'autre part
«beaucoup plus grand est le nombre de ceux qui ne le sont pas», il refuse cependant toujours les excès auxquels certaines de ses formules se prêtaient77  L'exemple de la prédestination est le plus significatif. Augustin s'interdit de l'enseigner comme s'il s'agissait d'un fatalisme, réduisant à néant la liberté78 

S. Plaidoyer  en faveur de la grâce71

Augustin ne prétend pas avoir une théologie personnelle. Son unique livre de théologie est !'Ecriture. C'est pourquoi, il faut voir sur quels textes il appuie l'absolu de la grâce dont il se fait le champion face aux Pélagiens . Je voudrais citer quelques textes clefs qui lui servent à appuyer ses affirmations sur la grâce«>

a.	"Qu'as-tu que tu n'aies reçu... Et si tu l'as reçu, pourquoi t'en faire gloire comme si tu ne l'avais pas reçu ?" (1 Co 4, 7). Et voici le commentaire :

« C'est principalement ce témoignage de /'Apôtre qui m'a convaincu moi-même, quand j'étais dans une erreur semblable à celle de nos frères - il s'agit des moines de Provence - et je m'imaginais que la foi, par laquelle nous croyons en Dieu n'est pas un don de Dieu, mais que nous l'avons de nous-mêmes, et que nous obtenons ainsi par elle les dons divins qui nous permettent de 'vivre en ce monde avec modération' (Tit 2, 12). Je ne voyais pas que la foi est précédée en nous par la grâce de Dieu, pour que par elle nous soit ensuite donné ce que nous demandons utilement...»(De praedestinatione sanctorum 3, 7, B.A. 24 p. 479)
«La foi elle-même figure parmi les dons divins qui nous sont dispensés en ce même Esprit. Ces deux choses : croire et agir, sont nôtres en raison de notre libre arbitre, et cependant l'une et l'autre sont données par /'Esprit de foi et de charité. Car ce n'est pas la charité seulement, mais, comme il est écrit, "la charité avec la foi qui descend de Dieu le Père et du Seigneur Jésus-Christ" (Eph 6, 23).(ib. 3, 7, p. 481-483)





18 Cf. Cité de Dieu XIX, 25, . BA 37, p. 165-167, et p. 761, avec d'autres références p. 957. J. WANG TCH'ANG-TCHE, Saint Augustin et les vertus des païens. Beauchesne, 1938.
11 Enchiridion XXIV, 97 s.
78 Voir par exemple : De dono perseverantiae, 22, 57, BA 24, p. 741, et autres référence page 859. Augustin précise «la manière dont il convient d'enseigner la prédestination», insisitant sur le fait qu'on doit éviter de la présenter «de façon à la rendre odieuse».
79 Voir le dossier scripturaire dans Sesboüé, op. cit. p. 167.
8°Ces textes ont été sélectionnés par Jacques Pintard, Le docteur de la grâce, in Augustin, le
message de la foi, DDB, 1987, pp. 119 sv.:


32
(
 
b.	"Personne ne peut venir à moi si le Père qui m'a envoyé ne l'attire" (Jo 6, 44) "Magnifique éloge de la grâce! Nul ne vient s'il n'est tiré..." (26, 2, BA. 72, p. 487). Aussi est-ce juste de glorifier Dieu pou les mérites : "Lo!"5que tu couronn s leurs mérite tu couronnes tes propres dons. (Lettre 194, à Sixte, 19). Augustm illustre un u plus loin sa pensée par une comparaison cet attrait qu'exerce le Père
pour l'attacher au Christ :
«Donne-moi quelqu'un qui aime, et il sentira la vérité de ce que dis. Donne-moi un homme tourmenté par le désir, donne-moi un homme en marche dans ce désert et qui a soif, qui soupire après la source de l'éternelle patrie, donne-moi un tel homme, il saura ce que je veux dire.» (lb. 26, 4, pp. 491-493).
«Tu montres  un rameau vert à une brebis, tu l'attires. On présente des noix à
un enfant, il est attiré...Si donc ce qui est révélé des délices et des voluptés terrestres à ceux qui les aime les attire, ...comment le Christ révélé par le Père n'attirerait-il pas ? » (ib. 26, s, p. 497).
c.	"Dieu résiste aux orgueilleux et donne sa grâce aux humbles" (Proverbes 3, 34, et cité en I P 5, 5 et Jac 4, 6). Augustin cite SS fois cette sentence dans son oeuvre, selon la comptabilité de La Bonnardière. Elle indique qu'il n'y a pas d'autre voie vers Dieu que l'humilité du Verbe.

Augustin, on l'a vu, fut profondément marqué par cette découverte du Christus humilis. Le grand obstacle à la rencontre de Dieu est l'orgueil: "Dieu s'est fait humble, et l'homme est orgueilleux" (Sermon 142, 6)

d.	"Donne ce que tu commandes et commande ce que tu veux" Da quod jubes et jube quod vis ! (Conf. X, 29, 40; 31, 45; 37, 60). Augustin lui-même rappelle combien cette sentence avait irrité Pélage. Il écrit dans le De dono perseverantiae :
\begin{quote}
    « Parmi mes ouvrages, en est-il un qui ait été plus répandu et plus goûté que mes Confessions ? Or, dans cet ouvrage, publié lui ausi avant que n'eût paru l'hérésie pélagienne, je dis, et plusieurs fois à notre Dieu :"Donnez ce que vous commandez et commandez ce que vous voudrez." Ce sont ces paroles de moi que Pélage, à Rome, entendit un jour citer par un de mes frères et collègues dans l'épiscopat; il ne put les supporter, et dans l'émotion vive qu'il mit à les contredire, peu s'en fallut qu'il se prit de querelle avec celui qui les avait rappelées. Mais qu'est-ce que Dieu nous commande d'abord et par dessus tout, sinon de croire en lui ? Donc, c'est lui qui nous donne de croire..." (B.A. 24, 20, 53)
\end{quote}


\begin{quote}
    L'espespérance ne déçoit point, parce que l'amour de Dieu a été répandu dans nos co ur par le Saint-Esprit qui nous fut donné" (Rom 5, 5). 
\end{quote}
Un verset qui est cité pas moins de 200 fois. Augustin précise:
\begin{quote}
    « · Quand nous disons que Dieu aide à accomplir toute justice et opère en nous le vou o r et le faire (cf. Ph 2, 13) ... ce n'est pas parce qu'il fait ressentir en nos sens xt ieurs les préceptes  de la justice, mais parce qu'il donne l'accroissement mt neur   I Co 3, 7), en répandant la charité dans nos coeurs par /'Esprit-Saint
qw nous a ete donne» (L'esprit et la lettre 25, 42)

\end{quote}

 
 
Loin de contredire le libre arbitre, la présence de !'Esprit le dégage au contraire	( pour le rendre libre : "Partout où est !'Esprit du Seigneur, là est la liberté (2 Co 3,
17, là, le coeur humain est dilaté, ce qui fait dire à Augustin : « Cet Esprit de Dieu dont
la présence nous justifie, nos inspire la haine du péché et nous donne la liberté spirituelle» (ib. 16, 28).

6. Nouveaux	rebondissements sur liberté et	grâce

La thèse de la prédestination rebondit plusieurs fois au cours de l'histoire. Qu'il suffise de donner quelques points de repères. Trois moments, de plus ou moins d'ampleurs, on marqué ce débat.

a)	La question rebondit une première fois au IXe siècle, avec les thèses d'un moine, Gottschalk81 , fils d'un prince saxon, qui vivait au monastère d'Orbais (diocèse de Soissons). D'un esprit étroit, il enseignait que Dieu prédestine aussi bien les damnés à l'enfer que les élus au ciel. L'idée même de prédestination absolue entraîne aussitôt chez Gottschalk la négation de la liberté. De plus, introduisant I' idée d'une prédestination négative, il transforme Dieu en monstre, puisqu'il devient l'auteur du pire mal qui puisse arriver à l'homme. Inutile d'ajouter que, dans ces conditions, la rédemption laisse totalement hors de son champ ceux qui sont d'avance exclus du salut.

Cette théorie monstrueuse, qui se prétend l'expression de la pensée d'Augustin, sera condamnée : le concile de Valence (855) réaffirmera l'universalité du salut et la rédemption de tous, en même temps qu'il rcconna1t .; tout homme la liberté et le pouvoir de se sauver. Durant le Moyen Age, c'est cet augustinisme tempéré qui va s'imposer. Sans rien enlever à l'absolu de la grâce, les scolastiques partiront de l'idée de l'universalité du salut, une doctrine qui était passée quelque peu sous silence, et le reste, c'est-à-dire grâce et liberté, y est subordonné. Pour sauver les hommes, Dieu n'est pas tenu par les sacrements, dira saint Thomas. Il dispose de bien d'autres moyens.

b)	Avec  la  Réforme,  la  question  deviendra  beaucoup  plus  aiguë. Luther et Calvin se réclament tous les deux d'Augustin. De tous les auteurs anciens, Augustin est à leurs yeux le meilleur interprète de !'Ecriture. «Augustinus  meus totus est», déclare Luther dans sa polémique avec Erasme : «De ton côté sont les théologiens modernes et tant d'universités, de conciles, d'évêques, de papes! ...De mon côté, au contraire, seulement Wyclif et Laurent Valla, mais aussi Augustin, que tu passes sous silence, m'appartient tout entier.» De même, Calvin se place sous l'autorité  d'Augustin,  qu'il  cite  341  fois  dans  la  seule  Institution  chrétienne.
«Augustinus totus noster est», écrit-il en écho à Luther : «Quant à saint Augustin, il s'accorde si bien en tout et partout avec nous, il est tellement nôtre (adeo totus noster est) que s'il me fallait écrire une confession sur cette matière, il me suffirait de la composer des témoignages extraits de ses livres... Je ne dis rien qu'il n'ait dit devant moi mot à mot.» Si l'on considère ce qui a séduits les Réformateurs, c'est, on s'en doute, la doctrine augustinienne de la grâce et de la justification. Chez Calvin, c'est toujours dans le contexte de sa défense de la prédestination qu'il cite Augustin, sans

81 Cf. Kurt FLASCH, Introduction à la philosophie médiévale, Cerf, Ed. universitaires de Fribourg, 1992, p. 29 s., en particulier p. 33 s. sur Godescalc et la prédestination divine.
34
 
s a r.cevoir que l'idée d'une double predestination, le salut pour les uns et la J>erdation Pour les autres, dont il se fait le défenseur, est totalement étrangère à Augustin82 

Quand en effet on y regarde de plus près, on s'aperçoit que, au-delà de l'appel explicite à son autorité, Augustin ne sert chez Luther et Calvin qu'à cautionner des thèses qui sont les leurs plus que les siennes. Cet a priori de lecture est évident chez Luther. Alors même qu'il se croit dans la ligne d'Augustin, il oublie des pans entiers de sa pensée. En particulier, sa thèse sur la justification ne saurait être tirée d'Augustin sans autre forme de procès. Augustin ne dissocie jamais foi et charité. Alors que Luther soutient que la foi seule justifie, Augustin dit que c'est la charité dans son obéissance à la loi. Luther n'a d'ailleurs pas été sans lui-même relever cet écart. Mais au lieu de rectifier sa propre position, il accuse Augustin d'inconséquence ou d'infidélité à lui-même. Melanchton, avec l'accord de Luther, avouera explicitement à Brenz : Augustin, qui avait d'abord nié que l'homme puisse être justifié par lui­ même devant Dieu, «imagine ensuite que nous sommes réputés justes en raison de l'accomplissement de la loi que le Saint-Esprit opère en nous... Augustin n'est pas en accord complet avec la doctrine de saint Paul, bien qu'il en approche davantage que les Scolastiques... » Il ajoute que, pour le public, il faut continuer à invoquer l'appui d'Augustin, «bien qu'il n'explique pas assez la justice par la foi»83 

c)	Le	jansemsme	représente	le	troisième	rebondissement.	Sa préhistoire commence avec Michel Baius (1513-1589), chancelier de l'université à Louvain, qui avait fait d'Augustin son cheval de bataille contre les protestants. Il avait lu, nous dit-on, neuf fois tout saint Augustin et soixante-dix fois les écrits sur la grâce f Ce qui ne l'a pas empêché de se fourvoyer sérieusement. Le point névralgique de sa position tourne autour du rapport entre nature et grâce. Dans l'état initial d'Adam, la «nature» de l'homme était dotée d'une puissance autonome en face de Dieu, c'est-à­ dire d'une liberté capable par elle-même de faire son salut, si bien que si l'homme était fidèle à sa nature, il pouvait en droit prétendre à la récompense. «Chez l'homme d'avant la chute, la vie éternelle... n'aurait pas été un don de la grâce, mais un salaire.» C'est la thèse de Pélage : «Dieu me fait homme, mais c'est moi qui me rend juste», sauf que Baius parle de la condition de l'homme	d'avant la chute, tandis que . Pélage parle de sa condition actuelle .

On a dit de Baius qu'il était un «Pélage  du  paradis  terrestre». Depuis la chute, en effet, la grâce est devenue indispensable à l'homme, comme un ajout extérieur à une nature blessée et déficiente, alors qu'elle ne l'était pas avant. En se voulant fidèle à Augustin, Baius en fait le trahit, car pour Augustin,aussi bien dans l'état primitif que dans l'état actuel, sans la grâce, l'homme est incapable de parvenir au salut. Adam, pas plus que l'homme déchu, ne peut s'attribuer à lui-même un quelconque mérite04  	Au fond, Baius n'a sauvé Augustin qu'au prix d'une double déformation, à la fois du concept de nature et du concept de la grâce. Au regard de la pensée d'Augustin, l'extrinsécisme de Baius est une aberration. Mais la question se pose aussitôt : Baius, qui a été condamné, n'aurait-il pas entraîné dans sa chute Augustin lui-même? C'est ce que prétendaient certains jésuites.

82 Cf. Augustinus Magister. Etudes augustiniennes, 1954. Il, 1029 s. pour Luther et saint Augustiin (Léon Cristiani), et 11, 1839 s. pour Calvin et saint Augustin (Jean Cadier).
63 Léon CRISTIANI, Luther et saint Augustin, dans Augustinus Magister Il, 1954, p. 1037 -1038.
84 Cf. Henri de LUBAC, Augustinisme et théologie moderne. Aubier, 1965, p. 15 s.
35
 
(
C'est pour réfuter cette prétention que Jansénius (1585-1638), fu ur évê ue d'Ypres, rédige un ouvrage de synthèse sur la pensée d'Augustin : I' Augustmus qui ne sera publié qu'après sa mort, en 1640, et dont les jansénistes eront leur Bible. De Jansénius aussi, on nous dit qu'il avait lu «dix fois saint Augustm, et plus d? tren e fois les ouvrages de la grâce contre les Pélagiens», ce qui ne l'empêche pasd aboutir, en fin de compte, à une «consciencieuse méprise». En réalité, il ne fait que prolo,nger Baius, mais en l'inversant. Alors que Baius part d'une nature qu1, au moms dans I état d'Adam, peut prétendre au salut, la grâce devenant inutile, Jansénius part de l'absolu de la grâce, expression de la puissance arbitraire de Dieu, qui la donne ou la refuse sans tenir compte de l'homme, pour refuser ensuite à la nature humaine, d nc à la liberté, toute consistance propre. «Si l'un et l'autre tendent également à dissoudre l'union entre Dieu et l'homme en quoi consiste essentiellement le Mystère du Christ, écrit le P. de Lubac, n'est-ce pas que l'un dresse l'homme en face de Dieu dans la réclamation de ses droits, tandis que l'autre l'anéantit ?fJS »
La thèse de Jansénius sera à l'origine de tous les débats modernes autour de la grâce, dont Port-Royal,  avec l'abbé de Saint-Cyran à sa tête, sera le foyer en France. Délaissant les spéculations de Baius sur l'état primitif de l'homme, tout comme celles de Jansénius sur la «pure nature», Saint-Cyran s'intéresse à l'homme concret, pour le placer d'emblée devant le mystère redoutable de la grâce, une grâce rare, qui n'est donnée qu'à un petit nombre d'élus. «Comme le soleil fait des jours sacrés et des jours profanes, des jours de fête et des jours civils, des jours d'été et des jours d'hiver, ainsi Dieu a fait certains hommes pour être sauvés et saints, et d'autres profanes pour être damnés... ». «li ne tombe pas une goûte de grâce pour les païens !» Saint-Cyran, sans lequel le jansénisme n'aurait sans été qu'une «hérésie de professeurs», lui assurera le succès. Mais c'est bien à tort qu'il invoque saint Augustin. A aucun moment, Augustin n'a pensé que le destin de l'homme pouvait être scellé de l'extérieur par Dieu, sans que la responsabilité de l'homme y soit engagée. Alors que les débats sur la grâce et la prédestination, tels qu'ils se sont développés dans le jansénisme, se placent au point de vue de l'éternité, où tout semble joué d'avance, Augustin considère l'homme dans son devenir temporel, où l'alternative entre la lumière et les ténèbres reste toujours ouverte86 

\section{Conclusion}
\begin{Synthesis}
 Augustin maintiendra contre vents et marées le primat de l'action de Dieu. Il n'a pas réussi à concilier par des concepts l'action de Dieu et l'agir de l'homme. Le champion de la liberté contre les manichéens s'est battu avec la même i tran igeance en s faisant le champion de la grâce contre les Pélagiens. Augustin s'en tient a la parole de l'Ecriture, faisant siennes ce texte de l'Apôtre: .
\begin{quote}
    «  Qui es- u pour discuter avec Dieu ? (Rm 9, 20)  Impénétrables  sont ses J gem nts et imcompréhensibles ses voies (Rm 11, 33) Et il ajoute : "Ne cherche pas a attemdre ce qw est trop haut pour toi, ni à scruter ce qui te dépasse"» (Ecc; 3, 22) (De Dono perseverantiae XII, 30, B. A. 24 p. 669)
\end{quote}
\end{Synthesis}




Au cours des siècles, cette doctrine de la grâce fut tirée vers une doctrine mtolerable, celle de la pré?estination, et en particulier de la prédestination négative, notamment chez les Jansernstes et les calvinistes. Dieu est alors un Dieu arbitraire,
85lb., p. 51.
ee lb., p. 81.
 
rvers, qui décide d'avance, de façon absolue, qui va au ciel et qui va en enfer, indépendamment des mérites :

« Par décret de Dieu, et pour la manifestation de sa gloire, tels hommes sont prédestinés à la vie éternelle, tels autres voués à la mort éternelle!» (Confession d Westminster, 1647).

Une telle doctrine n'est pas augustinienne, même si certaines formules peuvent être tirées en ce sens. Mais il faut bien constater que ce sont ces excès modernes qui furent à l'origine d'une certaine forme d'athéisme, l'athéisme humaniste, qui oppose Dieu et l'homme, la grâce et la liberté, et qui rejette Dieu au nom de l'homme et de la liberté. "M'en coOtât-il d'être expédié en Enfer, disait Milton, jamais un tel Dieu ne m'imposera le respect!" (Cf. Max WEBER, L'éthique protestante et l'esprit du capitalisme, Pion, p. 117.)

Augustin était trop conscient de la liberté de l'homme pour croire qu'elle pouvait être mise en échec par la grâce. Celle-ce n'est pas contre la liberté, ni au-dessus de la liberté, mais avec la liberté qu'elle rachète et rend efficace dans l'ordre du salut. L'acte le plus libre est celui par lequel l'homme accepte de reconnaître que tout lui est donné, y compris sa liberté.

Marcel NEUSCH Institut catholique de Paris
1997-1998

























37

%\include{SeminaireMission/RiteChinois}
%M\chapter{Charles de Foucauld et l'évangélisation}

\section{Synthèse}

\subsection{Charles de Foucauld : Biographie et éléments de contexte}

\paragraph{Une carrière d'officier} Né en 1858, Charles de Foucauld commence une carrière d'officier. Rétif à la hiérarchie et noceur invétéré, il quitte l'armée. Il décide alors d'explorer le Maroc, exploration qui lui vaut un succès d'estime dans la société savante et la publication de son livre \textit{Reconnaissance au Maroc}.

\paragraph{Conversion en 1886 et vie cachée du Christ} Il se convertit en Octobre 1886 dans l'Eglise Saint Augustin, en se confessant à l'Abbé Huvelin qui devient son directeur spirituel. Là, il commence à suivre le Christ dans sa vie cachée, à l'abbaye Notre-Dame-des-Neiges, puis à Nazareth puis à Béni-Abbès.

\paragraph{Tamanrasset} En 1905, il commence une vie d'ermite à Tamanrasset puis à l'Assekrem. Il meurt assassiné en 1916.

\subsection{Spiritualité et mission}

\paragraph{Documents étudiés} En préalable, un mot sur les trois documents étudiés : 
\begin{itemize}
    \item deux articles de Pierre Sourisseau, archiviste de la cause de canonisation de Charles de Foucauld
    \item une anthologie de textes de Charles de Foucauld compilée par Michel Lafon par thème. A ce propos, une attention méthodologique est de mise quand on a ce type de sources, avec la recommandation d'aller autant que possible aux textes originaux  pour situer les passages dans leur contexte.
\end{itemize}



\paragraph{Une évangélisation liée à une spiritualité profonde} Chez Charles de Foucauld, l'urgence, le soucis de l'évangélisation s'enracine dans une spiritualité forte, celle d'imiter Jésus dans sa vie cachée à Nazareth, dans sa vie humble.  Ce n'est pas uniquement\textit{ la Croix qui sauve} mais l'incarnation du Christ.

\paragraph{Une ouverture à l'universel} Cette imitation de Jésus, à travers un chemin concret est le moyen de devenir \textit{frère universel}. Il s'agit d'une christologie ascendante, partant du Jésus \textit{historique}, enracinée dans l'Evangile, christologie du bas vers le haut : Dieu agit dans les humbles. Dans la rencontre des autres,  le royaume de Dieu se construit.  

\paragraph{Le renouveau de l'Amour à la fin du XIX} Nous notons par rapport aux textes précédents, y compris celui récent du Cardinal Lavigerie, un changement de tonalité : l'amour est fortement présent chez Charles de Foucauld, comme chez Thérèse de l'Enfant Jésus, marque d'un contexte culturel nouveau.

\paragraph{Conséquence en terme de mission} Par l'insistance sur la grâce et non les oeuvres, Charles de Foucauld comprend l'efficacité de la mission non pas en terme de chiffres de convertis. Elle devient alors une mission conversationnelle, respectant la foi de l'autre, juif ou musulman. Charles de Foucauld cite comme exemple missionnaire St François d'Assise et sa rencontre avec le Sultan et la \textit{Regula Non Bullata}. Cette mission qui est d'abord présence à côté de frères humains,  découle de sa mystique. Elle fut peut être renforcée par le texte \textit{Ius Commissionis}, qui établit un monopole missionnaire par région : Charles de Foucauld aura des liens amicaux avec les Pères Blancs mais il expérimente une mission basée sur la présence et la rencontre et non une mission active ”prosélyte” réservée aux Pères Blancs.

\paragraph{Une activité en relation} L'image d'Epinal de l'Ermite de Tamanrasset n'est pas tout à fait juste, tant il est en lien, à la fois avec les populations locales (Tamanrasset est un carrefour et Charles de Foucauld se plaint d'un trop plein d'activités), mais aussi avec un vaste réseau via sa correspondance active. 



\subsection{Fécondité}
De façon étonnante, l'Evangélisation de Charles de Foucauld a été à court terme un échec, sans conversion de son vivant, ni oeuvre directement créé (les Frères et Sœurs du Sacré-Cœur de Jésus seront créés à partir de sa spiritualité mais pas directement par Charles de Foucauld). Néanmoins, au delà des frères et soeurs du Sacré-Coeur de Jésus, sa fécondité est  notable, dans les pratiques missionnaires (nous citons Monchanin en Inde et la Mission de France).

Et de façon plus récente, le pape François le cite longtement dans la conclusion de l'encyclique \textit{Fratelli Tutti} : 

\begin{quote}
    286. Dans ce cadre de réflexion sur la fraternité universelle, je me
suis particulièrement senti stimulé par saint François d’Assise, et
également par d’autres frères qui ne sont pas catholiques : Martin
Luther King, Desmond Tutu, Mahatma Mohandas Gandhi et beau-
coup d’autres encore. Mais je voudrais terminer en rappelant une
autre personne à la foi profonde qui, grâce à son expérience intense
de Dieu, a fait un cheminement de transformation jusqu’à se sen-
tir le frère de tous les hommes et femmes. Il s’agit du bienheureux
Charles de Foucauld.
287. Il a orienté le désir du don total de sa personne à Dieu vers
l’identification avec les derniers, les abandonnés, au fond du désert
africain. Il exprimait dans ce contexte son aspiration de sentir tout
être humain comme un frère ou une sœur,[286] et il demandait à
un ami : « Priez Dieu pour que je sois vraiment le frère de toutes
les âmes [⋯] ».

[287] Il voulait en définitive être « le frère universel
».

[288] Mais c’est seulement en s’identifiant avec les derniers qu’il
est parvenu à devenir le frère de tous. Que Dieu inspire ce rêve à
chacun d’entre nous. Amen !
\end{quote}

%%\chapter{Théologie des religions et \textit{Laudato Si’}}

\begin{comment}
  \paragraph{Instructions - } A rendre avant le 15 juin
30 pages;
bibliographie critique de chaque livre, probalématique et méthode.
Se concentrer sur un auteur : Laudato Si’
  
\end{comment}



Nous utiliserons les raccourcis suivants : 
\begin{itemize}
    \item LS : Laudato Si', encyclique du pape François \cite{francois_laudato_2015}, mai 2015
    \item EG : Evanglii Gaudium, exhortation apostolique du Pape François, novembre 2013
    \item GS : Gaudium \& Spes, constitution pastorale de Vatican II, décembre 1965
\end{itemize}
%-------------------------------------------------------------------------------------------------------
\section{Problématique proposée} 
%-------------------------------------------------------------------------------------------------------

 

Le pape François mentionne 5 fois le mot \textit{religions} dans son encyclique LS, plus que EG (3) et GS (1). Pourquoi une telle importance dans un texte sur la \textit{sauvegarde de la maison commune} ? Notre hypothèse est que ce regard sur les religions, important et positif, est la conséquence du diagnostic théologique de la crise écologique,  \textit{un \emph{style} de vie hégémonique lié à un mode de production} (LS 145). Reprenant ce terme de \textit{style} (19 occurrences dans LS), notre approche s'appuiera sur le \textit{style} chrétien proposé par LS comme hypothèse de la théologie doctrinale sous-tendant le texte : il nous faudra d'abord montrer la proximité de l'approche de style pensée par Merleau-Ponty et repris par Theobald \cite{theobald_christianisme_2007} avec celle du pape. Ensuite, nous étudierons comment ce style s'enracine dans la Bible en particulier autour des figures bibliques de l'idolâtrie, du prophète et de l'Apocalypse (certaines étant traditionnellement attribuées aux autres religions). Enfin, nous nous proposons d'étudier comment les ressources  de ce \textit{style chrétien} sont organisées dans l'encyclique et permettent de répondre au contexte mondiale de la crise écologique. Pour cela, il nous semble nécessaire de faire un détour par l'enjeu d'une éthique universelle pour le christianisme et plus largement du rapport entre LS et la doctrine sociale de l'Eglise. En effet, il convient de comprendre le rapport de l'Eglise à la société et à la vérité et pourquoi le recours des autres religions est pertinent. 


Dans une deuxième partie, nous étudierons les conséquences pour la théologie chrétienne des religions : comment penser le christianisme comme \textit{style} de dialogue avec les autres religions (pour reprendre le terme de l'encyclique) ?  Nous positionnerons cette approche parmi les diverses théologies chrétiennes pensant le pluralisme religieux (exclusiviste, inclusiviste, pluraliste, post-libéral et celle de Theobald), en montrant les points de convergences et les différences. Nous étudierons aussi le déplacement théologique par rapport notamment au texte de la CTI \cite{commission_theologique_internationale_christianisme_1997}.
Enfin, nous appliquerons cette approche en essayant de proposer ce que LS peut dire pour l'Islam, dans l'esprit du travail de Theobald sur le judaisme et l'Islam \cite{theobald_christianisme_2007}. 
 


\section{Proposition de plan}


\subsection{Le Style de Laudato Si'}

\paragraph{Travail sur EG et LS} Travail personnel sur l'utilisation des mots \textit{Religions} et \textit{idolâtrie} dans EG et LS. 

\paragraph{Importance du Style de Laudato Si’} \textit{La première réforme, le style chrétien,} François ; Antonio Spadaro, sj \cite{francois_premiere_2013}; \textit{Le Christianisme comme Style - Laudato Si'} - Christoph Theobald \cite{theobald_courage_2021}, \textit{Laudato Si’ : un changement dans ce que signifie la conversion ?} - Patrick Goujon \cite{goujon_laudato_2022}

\subsection{Les ressources chrétiennes face à la crise écologique}



\paragraph{Une éthique universelle pour répondre à l'enjeu universel de la crise écologique ?} \textit{De quel genre de pensée a-t-on besoin pour aborder la crise environnementale contemporaine ?} Howles, Timothy ; Kremer, Robert \cite{howles_quel_2022};  \cite{thomasset_recherche_2019}, \textit{La Charte de la terre} citée par LS, \textit{Manifeste pour une éthique planétaire}, Kuschel, Karl-Josef ; Küng, Hans ; \cite{kuschel_manifeste_1995} : \textit{L’éthique planétaire, d’un point de vue philosophique}, Küng, Hans \cite{kung_lethique_2009} \textit{La recherche d’une éthique universelle dans la tradition catholique. La méthode de Laudato Si’ }- A. Thomasset 

\paragraph{Ressources bibliques pour une réponse chrétienne}
\textit{Quelques mots avant l’Apocalypse - lire l’Evangile en temps de crise} - A. Candiard \cite{candiard_quelques_2022}
\textit{Between Exile and the New Jerusalem : Prophetic Mourning, Lament and the Ecological Crisis} - Daniel Castillo \cite{cavanaugh_between_2018}; \textit{Création à l’âge de l’anthropocène} - Christoph Theobald \cite{theobald_repenser_2019},  \textit{Parler de la création après Laudato si’}, Lasida, Elena \cite{lasida_parler_2020}

\paragraph{LS et la doctrine Sociale de l’Eglise} \textit{La doctrine sociale de l’Eglise selon François} - Ch. Theobald \cite{theobald_lenseignement_2016}, \textit{Economie, idolâtrie et sécularisation depuis Gaudium et Spes} - W. Cavanaugh \cite{cavanaugh_fragile_2018}


\subsection{Pour une théologie du dialogue comme théologie chrétienne des religions} 

\paragraph{Positionner LS dans les différences approches de la théologie des religions} au delà des références au cours \textit{Christologies au défi de la culture pluraliste} et \textit{Théologie Chrétienne des Religions} : \textit{Le christianisme et les religions}, Commission Théologique Internationale\cite{commission_theologique_internationale_christianisme_1997}, cité par EG, \textit{L’ unique christ : la symphonie différée}, Christian Duquoc \cite{duquoc_unique_2002}; \textit{Dieu au pluriel : penser les religions ,} Rémi Cheno \cite{cheno_dieu_2017}; \textit{La nature des doctrines. Religion et théologie à l’âge du postlibéralisme. }, George Lindbeck \cite{lindbeck_nature_2002}

\paragraph{Quelles ressources pour les autres religions } \textit{l’Unique et ses témoins - Jalons pour une
théologie de la rencontre entre juifs, chrétiens et musulmans}, Christoph Theobald \cite{theobald_christianisme_2007},  

\paragraph{Un exemple pratique : ce que les chrétiens peuvent dire à l'Islam sur ses ressources à mobiliser face à la crise écologique} \textit{Laudato Si’ : Engaging Islamic Tradition and Implications for Legal Thought}, Powell, Russell \cite{powell_laudato_2017}, : \textit{Ecologie et Religions -
Colloque IDEO 2022,} Pisani, Emmanuel ; Candiard, Adrien ; Hilal, Aziz ; Revol, Fabien \cite{pisani_ecologie_2022}, \textit{Écologie en islam et dialogue interreligieux,} Pisani, Emmanuel \cite{pisani_ecologie_2016}, \textit{The religious vision of nature in the light of Laudato
Si’ : An interreligious reading between Islam and Christianity}. Puglisi, Antonino ; Buitendag, Johan \cite{puglisi_religious_2020}, \textit{Anthologie du soufisme}, Hamès Constant. \cite{hames_vitray-meyerovitch_1988}, cité par LS (Alî al-Khawwâç)

\newpage

%-------------------------------------------------------------------------------------------------------
\subsection{L'Ecologie est un enjeu majeur pour les religions}

%-------------------------------------------------------------------------------------------------------
\paragraph{Question de la pertinence des religions}
\begin{singlequote}
        Qu’une religion soit raisonnable [donc universelle] dépend largement de ses
pouvoirs d’assimilation, de sa capacité à fournir dans ses propres termes une
interprétation intelligible des diverses situations et réalités que rencontrent
ses adhérents. Les religions que nous qualifions de primitives échouent régulièrement
à ce test quand elles sont confrontées à des changements importants,
tandis que les religions mondiales développent de plus grandes ressources pour
faire face aux vicissitudes \cite[ p. 175]{lindbeck_nature_2002}.
\end{singlequote}
%-------------------------------------------------------------------------------------------------------
\paragraph{Enjeu majeur pour les religions : pertinence par rapport au changement climatique}

 Un enjeu de pertinence pour les religions : \textit{mondiale}, \textit{existentielle}, \textit{ne se joue pas à l'échelle individuelle mais d'une transformation collective}

\paragraph{penser un salut collectif mais à travers une démarche qui entraîne tout le monde } d'une certaine façon nous oblige à définir ce qu'est le \textit{salut écologique} 
\label{Comment:MemoireISTR1}






%-------------------------------------------------------------------------------------------------------
\subsection{Le Style de Laudato Si’}  

Style; approche de Theobald; donc regarder construction; derrière le style "pastorale", une vraie théologie

%-------------------------------------------------------------------------------------------------------
\paragraph{La doctrine sociale de l'Eglise selon François} \cite{theobald_lenseignement_2016}

\cite{theobald_repenser_2019}

%-------------------------------------------------------------------------------------------------------
\paragraph{Première lecture de Laudato Si’ : six occurrences du mot Religions} et en particulier la section V du Chapitre V : Les religions dans le dialogue avec les sciences. \cite{francois_laudato_2015}
\begin{singlequote}
     nous ne pouvons pas ignorer qu’outre l’Église catholique, d’autres Églises et communautés chrétiennes – comme aussi d’autres religions – ont nourri une grande préoccupation et une précieuse réflexion sur ces thèmes qui nous préoccupent tous » [LS 7)
        Dans le sillage du concile Vatican II, l’encyclique insiste sur la contribution des religions en tant que vecteur d’une vision et d’une relation à la nature qui permet de répondre aux défis environnementaux et de proposer une alternative ancrée dans une sagesse séculaire pour éviter « l’indifférence, la résignation facile ou la confiance aveugle dans les solutions techniques » [LS 14]. Elles constituent une richesse « pour une écologie intégrale et pour un développement plénier de l’humanité » [LS 62].  « Tous, nous pouvons collaborer comme instruments de Dieu pour la sauvegarde de la création, chacun selon sa culture, son expérience, ses initiatives et ses capacités » [LS 15].  \cite{francois_laudato_2015}
\end{singlequote}
\begin{singlequote}
    201. La majorité des habitants de la planète se déclare croyante, et cela devrait inciter les religions à entrer dans un dialogue en vue de la sauvegarde de
la nature, de la défense des pauvres, de la construction de réseaux de respect
et de fraternité. 
\end{singlequote}
 
 


%-------------------------------------------------------------------------------------------------------
\paragraph{Comment penser Dialogue inter religieux et changement climatique} Notre hypothèse sera que Laudato Si’ n'est pas un simple texte de circonstance, qui doit \textit{cocher des cases} et en particulier le \textit{dialogue inter-religieux}, avec une articulation "Et" : dialogue interreligieux \textit{et} conversion écologique. Mais au contraire, reconnaître le style profondément théologique et construit de l'encyclique. 
 
Dès lors, trouver la pointe du \textit{dialogue inter-religieux} ne peut faire l'économie d'un travail théologique. 

%-------------------------------------------------------------------------------------------------------
\paragraph{Approche retenue : lecture de Laudato Si’ à la lumière de Theobald} Style. Proximité jésuite : dogmatique dans la pastorale. "Tout se tient". 


\paragraph{Apocalytique} p. \pageref{theob:apocalytique}
Discussions sur la place de la dimension apocalytique ? Espérance.


\begin{comment}
    

%-------------------------------------------------------------------------------------------------------
\subsection{LS et la doctrine sociale de l'Eglise selon François}  

%-------------------------------------------------------------------------------------------------------
\paragraph{Idée : changement de Style qu'il faudra qualifier et que Style c'est dogmatique}


%-------------------------------------------------------------------------------------------------------
\subsection{Laudato Si’, entre Prophétisme et Dialogue}  

%-------------------------------------------------------------------------------------------------------
\paragraph{La réponse traditionnelle de l'Eglise}  lien avec la doctrine sociale de l'Église, articulation de la justice nécessaire et de l'action individuelle et collective.  Alors que GS, marquée par la sécularisation, en restait aux principes et laissait l'autonomie à l'économie \cite{cavanaugh_idolatrie_2022}.   François : pas d'autonomie de l'économie (thomiste/sécularisation) qui en s'autonomisant, a pris comme religion l'argent. Permet de voir la nouveauté. 

%-------------------------------------------------------------------------------------------------------
\paragraph{En dialogue} Le titre de l'encyclique \textit{Maison Commune - Oikos} en est une première piste. une logique sapientielle "dialogue". Un nouveau rapport à la vérité. ouverture aux autres religions qui sont appelées à relever ensemble ce défi.

%-------------------------------------------------------------------------------------------------------
\paragraph{Conversion, Prophétisme et dénonçant l'idôlatrie} "Conversion", du registre prophétique
{Laudato Si’} prend un \textit{kairos} et en ce sens, devient prophétique, car parle en situation au sein d'une crise. Mal qui perturbe le peuple et Dieu. \textit{epistrophe}, retour vers Dieu, en ayant horreur de son comportement (metanoia). 


%-------------------------------------------------------------------------------------------------------
\paragraph{Comment articuler les deux}   A la différence Dt, qui critiquait fortement les religions extérieures et toutes les compromissions, ici, il semble que nous ayons un paradoxe : positivité des religions non chrétienne et négativité des "compromissions" mais par rapport à une "religion non nommée".


 %-------------------------------------------------------------------------------------------------------
\subsection{Théologie des religions}
 
\paragraph{Limites des approches pluralistes}

\paragraph{Approche retenue : la théologie de Theobald}

 %-------------------------------------------------------------------------------------------------------
\subsection{Ce que les autres religions peuvent dire de la crise écologique ? }
 
\end{comment}


%-------------------------------------------------------------------------------------------------------
\section{Le Style de Laudato Si’ - Bibliographie}
%-------------------------------------------------------------------------------------------------------

%-------------------------------------------------------------------------------------------------------
\subsection{Lecture de Laudato Si’ - mentions de \textit{Religions}}
%-------------------------------------------------------------------------------------------------------

\begin{singlequote}
        207. La Charte de la Terre nous invitait tous à tourner le dos à une étape d’autodestruction et à prendre un nouveau départ, mais nous n’avons pas encore développé une conscience universelle qui le rende possible. Voilà pourquoi j’ose proposer de nouveau ce beau défi : “Comme jamais auparavant dans l’histoire, notre destin commun nous invite à chercher un nouveau commencement [...] Faisons en sorte que notre époque soit reconnue dans l’histoire comme celle de l’éveil d’une nouvelle forme d’hommage à la vie, d’une ferme résolution d’atteindre la durabilité, de l’accélération de la lutte pour la justice et la paix et de l’heureuse célébration de la vie”.[148]
\end{singlequote}
       
\paragraph{Miser sur un autre style de vie}
\begin{singlequote}
        203. Étant donné que le marché tend à créer un mécanisme consumériste compulsif pour placer ses produits, les personnes finissent par être submergées, dans une spirale d’achats et de dépenses inutiles. Le consumérisme obsessif est le reflet subjectif du paradigme techno-économique. Il arrive ce que Romano Guardini signalait déjà : l’être humain « accepte les choses usuelles et les formes de la vie telles qu’elles lui sont imposées par les plans rationnels et les produits normalisés de la machine et, dans l’ensemble, il le fait avec l’impression que tout cela est raisonnable et juste ».[144] Ce paradigme fait croire à tous qu’ils sont libres, tant qu’ils ont une soi-disant liberté pour consommer, alors que ceux qui ont en réalité la liberté, ce sont ceux qui constituent la minorité en possession du pouvoir économique et financier. Dans cette équivoque, l’humanité postmoderne n’a pas trouvé une nouvelle conception d’elle-même qui puisse l’orienter, et ce manque d’identité est vécu avec angoisse. Nous possédons trop de moyens pour des fins limitées et rachitiques.

\end{singlequote}
\paragraph{les religions dans LS}    
 \begin{singlequote}
        « nous ne pouvons pas ignorer qu’outre l’Église catholique, d’autres Églises et communautés chrétiennes – comme aussi d’autres religions – ont nourri une grande préoccupation et une précieuse réflexion sur ces thèmes qui nous préoccupent tous » [LS 7]
        Dans le sillage du concile Vatican II, l’encyclique insiste sur la contribution des religions en tant que vecteur d’une vision et d’une relation à la nature qui permet de répondre aux défis environnementaux et de proposer une alternative ancrée dans une sagesse séculaire pour éviter « l’indifférence, la résignation facile ou la confiance aveugle dans les solutions techniques » [LS 14]. Elles constituent une richesse « pour une écologie intégrale et pour un développement plénier de l’humanité » [LS 62]. Il s’agit donc pour toutes les religions de puiser dans « leur propre héritage éthique et spirituel », de revenir « à leurs sources » pour « mieux répondre aux nécessités actuelles » [LS 200]. « Tous, nous pouvons collaborer comme instruments de Dieu pour la sauvegarde de la création, chacun selon sa culture, son expérience, ses initiatives et ses capacités » [LS 15]. Cette crise, source de migrations violentes et contenant en elle la possibilité prochaine des guerres, peut aussi être un lieu de rencontre, de dialogue et d’action [LS 15] entre tous les hommes. Dans une perspective dont on a souligné les accents blondéliens.
    \end{singlequote}       

 
        %Juan Carlos Scannone, « La filosofia dell’azione di Blondel   le pape y voit la possibilité de susciter une communion d’action afin d’ouvrir à une « nouvelle solidarité universelle » [LS 14).
 

    
\paragraph{les religions dans le dialogue avec les sciences}    
\begin{singlequote}
        199. On ne peut pas soutenir que les sciences empiriques expliquent complètement la vie, la structure de toutes les créatures et la réalité dans son ensemble. Cela serait outrepasser de façon indue leurs frontières méthodologiques limitées. Si on réfléchit dans ce cadre fermé, la sensibilité esthétique, la poésie, et même la capacité de la raison à percevoir le sens et la finalité des choses disparaissent.[141] Je veux rappeler que « les textes religieux classiques peuvent offrir une signification pour toutes les époques, et ont une force de motivation qui ouvre toujours de nouveaux horizons [...] Est-il raisonnable et intelligent de les reléguer dans l’obscurité, seulement du fait qu’ils proviennent d’un contexte de croyance religieuse ? ».[142] En réalité, il est naïf de penser que les principes éthiques puissent se présenter de manière purement abstraite, détachés de tout contexte, et le fait qu’ils apparaissent dans un langage religieux ne les prive pas de toute valeur dans le débat public. Les principes éthiques que la raison est capable de percevoir peuvent réapparaître toujours de manière différente et être exprimés dans des langages divers, y compris religieux.

        200. D’autre part, toute solution technique que les sciences prétendent apporter sera incapable de résoudre les graves problèmes du monde si l’humanité perd le cap, si l’on oublie les grandes motivations qui rendent possibles la cohabitation, le sacrifice, la bonté. De toute façon, il faudra inviter les croyants à être cohérents avec leur propre foi et à ne pas la contredire par leurs actions ; il faudra leur demander de s’ouvrir de nouveau à la grâce de Dieu et de puiser au plus profond de leurs propres convictions sur l’amour, la justice et la paix. Si une mauvaise compréhension de nos propres principes nous a parfois conduits à justifier le mauvais traitement de la nature, la domination despotique de l’être humain sur la création, ou les guerres, l’injustice et la violence, nous, les croyants, nous pouvons reconnaître que nous avons alors été infidèles au trésor de sagesse que nous devions garder. Souvent les limites culturelles des diverses époques ont conditionné cette conscience de leur propre héritage éthique et spirituel, mais c’est précisément le retour à leurs sources qui permet aux religions de mieux répondre aux nécessités actuelles.

        201. La majorité des habitants de la planète se déclare croyante, et cela devrait inciter les religions à entrer dans un dialogue en vue de la sauvegarde de la nature, de la défense des pauvres, de la construction de réseaux de respect et de fraternité. Un dialogue entre les sciences elles-mêmes est aussi nécessaire parce que chacune a l’habitude de s’enfermer dans les limites de son propre langage, et la spécialisation a tendance à devenir isolement et absolutisation du savoir de chacun. Cela empêche d’affronter convenablement les problèmes de l’environnement. Un dialogue ouvert et respectueux devient aussi nécessaire entre les différents mouvements écologistes, où les luttes idéologiques ne manquent pas. La gravité de la crise écologique exige que tous nous pensions au bien commun et avancions sur un chemin de dialogue qui demande patience, ascèse et générosité, nous souvenant toujours que « la réalité est supérieure à l’idée ».[143]
\end{singlequote}

\paragraph{champ lexical biblique}
\begin{itemize}
    \item  7 citations du livre de la Sagesse, \textit{poumon d'Israël pour respirer l'air commun} (Paul Beauchamp) \cite{goujon_laudato_2022} 
    \item 18 occurrences du vocabulaire stylistique \cite{theobald_courage_2021}. 
    \item 8 occurrences du mot \textit{religion}
\end{itemize}
\begin{comment}
    https://rpubs.com/datadataguy13/1011253
    https://pypi.org/project/lexicalrichness/#example-use-cases
    https://newscatcherapi.com/blog/ultimate-guide-to-text-similarity-with-python
\end{comment}
%-------------------------------------------------------------------------------------------------------
\subsection{Importance du Style de Laudato Si’ - Bibliographie}
%-------------------------------------------------------------------------------------------------------

%-------------------------------------------------------------------------------------------------------
\paragraph{Le Christianisme comme Style - Christoph Théobald} -  \textit{in }\textit{le Courage de penser l'avenir} \cite[p 169-196]{theobald_courage_2021}. Le pape François aime le vocabulaire stylistique. Ainsi, le terme \textit{style} est-il utilisé vingt-deux fois dans EG et dix-huit dans LS. Malheureusement, le terme n'est pas défini précisément. S'agit-t-il de la même définition que celle qu'en donne l'A. dans son livre éponyme \cite{theobald_christianisme_2007} \textit{Le Christianisme comme style}  ? 
\begin{comment}
    Nous nous proposons de lire ce chapitre soulignant la notion de \textit{recours} des religions et la distinction   entre \textit{ressources et sources} déjà entrevu dans \cite{theobald_lenseignement_2016}. 
    notion de \textit{recours}. Distinction entre \textit{ressources et sources}.  Style prophétique et contemplatif de \textit{Laudato Si’}. Nous n
\end{comment}
Reprenant l'analyse que LS fait du monde (présentée p. \pageref{theo:diagnosticLS}), la réponse ne peut être de remédier aux symptômes :
\begin{singlequote}
     il faut adopter résolument un «regard différent» et différencié sur le réel: « une pensée, une politique, un programme éducatif, un style de vie et une spiritualité qui constitueraient une résistance face à l'avancée du paradigme technocratique » [LS, 111]. Ce nouveau style de vie est donc le noyau de ce que l'Encyclique appelle une « \textit{écologie intégrale} »
\end{singlequote}

Ce style comprend vertus solides, apprentissage de "petites actions quotidiennes" ancrées par l'exercice et la répétition [LS 211], style de vie prophétique et contemplatif, qui n'est pas obsédé par la consommation [LS 222], avec un horizon global (\textit{la maison commune}) et local.
L'A. postule donc que le concept de \textit{style} sous-jacent à l'encyclique est celui de la phénoménologie de Merleau-Ponty :

\begin{singlequote}
« Tout style est la mise en forme des éléments du monde qui permettent d'orienter celui-ci vers une de ses parts essentielles ». Il y a signification lorsque les données du monde sont par nous soumises à une « déformation cohérente » \cite[p. 55, cité par l'A.]{merleau-ponty_signes_1960}

\end{singlequote}
L'opération stylistique consiste à envisager la métamorphose du monde. En ce sens, notre action est, dans le vocabulaire judeo-chrétien, \textit{messianique} avec une visée eschatologique et critique du monde actuel, très présente dans LS et EG. L'A. souligne l'articulation très précise entre une conscience aigue de la complexité du monde de la vie et la critique simultanée de ce monde dans ses ressorts d'aliénation de l'autre, comme le montre l'exemple de la conception des villes et des espaces publiques [LS 150 ss] ou la description très précise et scientifiquement fondée des enjeux écologiques dans le premier chapitre de LS.

Dans un premier temps, le style même de l'encyclique LS est performatif : il réalise ce qu'il annonce. L'introduction par la citation du Cantique de François d'Assise s'adresse à l'affectivité du lecteur et l'introduit dans le  mystère joyeux que nous  contemplons dans la joie [LS 12].

Il s'agit ensuite de penser l'articulation entre doctrine et pastorale : 

\begin{singlequote}
    François, quand il introduit le vocabulaire stylistique et parle du «style de vie de l'Évangile» et d'un «style évangélisateur» se sert-il d'une terminologie secondaire, en quelque sorte conventionnelle et conforme à une mentalité post-moderne, mais qui n'ajoute rien de neuf [\ldots] ? Ou bien ce vocabulaire véhicule-t-il une spécificité théologique, voire doctrinale qui ne peut être exprimée autrement, une spécificité dont les textes eux-mêmes seraient conscients?  \cite[p. 180]{theobald_courage_2021}
\end{singlequote}
L'A. fait l'hypothèse de cette spécificité doctrinale, en s'appuyant sur les nombreuses références aux \textit{spirituels} et \textit{mystiques}. Face au "rêve d'une doctrine monolithique, défendue par tous sans nuances" (EG 40), il nous faut apprendre l'art apostolique de circuler entre la cohérence du mystère chrétien (EC 39) et la simplicité du cœur de l'Évangile (EG, 34-45) d'une part et la diversité culturelle et personnelle de ses destinataire d'autre par, ultimement assurée par l'Esprit Saint et incarnée dans des \textit{charismes ou styles religieux}. L'A. reprend ici la présentation des 4 principes vus p. \pageref{theo:principesLS} pour renforcer l'hypothèse d'un choix conscient du pape de placer le \textit{doctrinal} au service de la \textit{pastorale}.
\begin{singlequote}
    «le christianisme n'a pas un modèle culturel unique, mais tout en restant pleinement lui-même, dans l'absolue fidélité à l'annonce évangélique et à la tradition ecclésiale, il revêtira aussi le visage des innombrables cultures et des innombrables peuples où il est accueilli et enraciné ».  [EG ]
\end{singlequote}
L'A. étudie ensuite la nouveauté des styles proposés par EG 18 (\textit{"Style évangélisateur déterminé"}) et LS 222 (\textit{"style de vie prophétique et contemplatif" }), en montrant que la seconde expression peut se comprendre comme une détermination de la première.
Dans LS, la foi chrétienne est présentée de manière \textit{étonnamment neuve}, non pas sous l'angle de la vérité et du registre de l'apologétique, mais comme une \textit{ressource} dont dispose l'humanité pour répondre à la crise écologique. François propose une démarche de conversion, à oser transformer en souffrance personnelle ce qui se passe dans le monde et ainsi à reconnaître la contribution que chacun peut apporter [LS 19]. \label{theo:mourning}

Face à la complexité de la crise, à l'insuffisance des réponses actuelles [LS 2], il est nécessaire d'avoir recours aux diverses richesses culturelles des peuples, à l'art et à la poésie, à la vie intérieure et à la spiritualité \cite[p. 188]{theobald_courage_2021}.
La spécificité chrétienne se traduit par une double argumentation :
\begin{singlequote}
La première, d'ordre épistémologique, consiste à articuler les différents niveaux du réel et à faire dialoguer les différentes disciplines: les sciences et la technique, l'économie et la politique, toujours traités - sur le fond - de manière positive, avant de critiquer ce qui risque de fausser leur apport: l'instinct de puissance et de domination et, surtout, leur spécialisation, apparemment innocente (n° 110), qui occulte cependant de plus en plus les grandes questions et, en particulier,
    l'horizon éthique de l'agir humain avec les principes fondamentaux du « bien commun » et de la « justice entre générations » (traités à la fin du chap. IV). Nous retrouvons ici le combat de François contre le paradigme homogène et unidimensionnel de la technocratie et pour une approche différenciée du réel. Ainsi est ménagée une ouverture où peuvent et doivent intervenir des «ressources» d'énergie autres, nécessaires pour affronter les mutations des mentalités et les conversions collectives qui nous attendent: «ressources » d'énergie intérieure, proprement spirituelle, qui ne relèvent ni de la science et des techniques, ni de la diplomatie et de la gestion économique de nos biens, mais de la culture et de la sagesse religieuse \cite[p. 189]{theobald_courage_2021}.
\end{singlequote}
L'autre argument relève de la théologie de la création, la sagesse faisant le lien entre les deux : 
\begin{singlequote}

    Peu fréquent dans l'Encyclique mais suffisamment précis et significatif, le vocabulaire de la «sagesse» s'inscrit, d'un côté, dans celui de la «culture » prise dans toute sa complexité; il comprend, de l'autre côté, l'ensemble des traditions religieuses, tout en spécifiant le récit biblique (selon le titre qui introduit les n° 65 à 75).Dès le premier chapitre, le texte insiste sur les conditions d'accès à la « vraie sagesse », « fruit de la réflexion, du dialogue et de la rencontre généreuse entre les personnes » (LS, 47; voir aussi 63), honorant ainsi les exigences d'une approche différenciée et non homogène du réel. La sagesse désigne donc une « manière de vivre » ou un « style» spécifique qui forme un « héritage » ou un «trésor», mais sans cesse à redécouvrir au sein de nos cultures, conditionnées par leurs limites et leurs préjugés (LS, 200). Dans le chap. II, cette manière de vivre est rapportée à Dieu et à sa propre sagesse, au travail au sein même de la création et de l'histoire (LS, 69) où elle fonde les traits caractéristiques du style biblique et évangélique, déjà évoqués au début de ce parcours, en particulier le rapport intrinsèque entre nos relations fraternelles avec les autres et avec la nature (LS, 70).  \cite[p. 189]{theobald_courage_2021}.
\end{singlequote}
La nouveauté de LS est donc de présenter la tradition chrétienne comme \textit{ressource} dont la spécificité est un style \textit{prophétique et contemplatif}, les deux dimensions étant liés. \textit{Prophétique} d'abord par la critique d'un anthropocentrisme despotique (LS 68,69, 119,122) et contre une compréhension erronée d'une théologie de la création comme mandat donné à l'homme de dominer la terre.  Il s'agit aussi de réinvestir la présence de l'Esprit Saint (LS 80) et du mystère du Christ dans l'ensemble de la réalité naturelle, sans pour autant en affecter l'autonomie (LS 99). La dénonciation prophétique de LS n'édulcore par l'insoutenable mais ne revêt aucune dimension catastrophique, équilibrée par l'acte d'espérance, marque du style contemplatif \cite[p. 192]{theobald_courage_2021}.  \label{theob:apocalytique} Etre contemplatif, c'est \textit{capable d'apprécier profondément les choses sans être obsédé par la consommation} (LS 222). François suit la théologie de Saint Bonaventure (LS 66 et 233) 
\begin{singlequote}
    « La contemplation est d'autant plus éminente que l'homme sent en lui-même l'effet de la grâce divine et qu'il sait trouver Dieu dans les créatures extérieures »
\end{singlequote}
Cette contemplation permet d'entrevoir le lien mystérieux entre la fraternité avec les plus fragiles et la relation aux mondes et aux créatures [LS 84-92, 221-232]. C'est éclairé par l'approche stylistique [LS 121] qu'il faut comprendre le terme d'\textit{écologie intégrale} utilisé par l'encyclique.
L'A. termine en s'interrogeant sur les ressources de l'approche stylistique de la tradition chrétienne par rapport à la question du mal ou de la violence, tout en soulignant que l'"incomplétude" de la réponse est un élément essentiel de la pensée théologique de François \cite{francois_premiere_2013}.







%-------------------------------------------------------------------------------------------------------
\paragraph{Laudato Si’ : un changement dans ce que signifie la conversion ? - Patrick Goujon} \cite{goujon_laudato_2022} L'A. est jésuite, directeurs des RSR, et intervient ici dans le cadre d'une conférence tenue en 2022 au Centre Sèvres sur la conversion écologique. En quoi consiste l'appel à la conversion écologique de LS ? que devient Dieu dans cette conversion ? Cependant, cette conversion est moins en un sens prophétique que selon une démarche de sagesse, dans le sens où elle dessine un avenir commun dans le \textit{dialogue}. 
Reprenant l'article conversion \cite{lacoste_conversion_2007} de A. Wénin, 
\begin{singlequote}
    Dans un temps où menace la guerre, pèse sur le peuple le risque de l’idolâtrie, de son péché, « d’un mal qui perturbe la relation entre Israël et Dieu ». L’appel presse à revenir vers Dieu (conversio, epistrophè) en abandonnant les idoles et modifiant son action, en prenant en horreur son comportement passé (aversio, metanoia). Centrée sur Dieu, dont le prophète rappelle la miséricorde (Osée) et le jugement (Isaïe), la prédication prophétique insiste sur la responsabilité personnelle et collective, tout en révélant que la conversion est un don de Dieu (Ez 26,25-32) : « Je vous donnerai un cœur nouveau, je mettrai en vous un esprit nouveau ». Aversion, retour et alliance avec Dieu sont les termes clés du mouvement que veut provoquer le prophète par sa parole. \cite[p. 392]{goujon_laudato_2022}
\end{singlequote}

On retrouve ces éléments prophétiques de l'aversion, de la conversion et de l'alliance avec Dieu dans LS : 
\begin{singlequote}
     Invitant à une transformation du rapport aux biens et aux autres, en liant « clameur de la terre et clameur des pauvres », François reste dans la lignée morale qui unit appel à la sainteté, personnelle, et à la justice sociale, selon des traits propres aux invocations des prophètes. Une forme d’ascèse s’impose, une limitation volontaire de notre consommation en réponse à l’avidité et au gaspillage.\cite[p. 393]{goujon_laudato_2022}
\end{singlequote}

Cependant, une autre démarche sous-tend LS, la sagesse, marquée par la \textit{rencontre} de l'autre : 
\begin{singlequote}
     La vraie sagesse, fruit de la réflexion, du dialogue et de la rencontre généreuse entre les personnes, ne s’obtient pas par une pure accumulation de données qui finissent par saturer et obnubiler.  [LS 47]
\end{singlequote}




\begin{singlequote}
 [l'encyclique] tend à l’extrême [les opinions], entre la réponse néolibérale à la crise par la confiance au progrès technique qui trouvera de nouvelles solutions et à l’autre extrémité, l’affirmation de la nécessaire réduction de la présence humaine comme seule possibilité de la sauvegarde de la planète. Et de conclure : « Entre ces deux extrêmes, la réflexion devrait identifier des scénarios futurs, parce qu’il n’y a pas qu’une seule issue » [ES 60]. \cite[par. 5]{goujon_laudato_2022}
 \end{singlequote}
 En affirmant l’existence d’issues multiples, la théologie fondamentale est mobilisée en invitant au dialogue entre science et religion, idée forte de LS :
 \begin{singlequote}
« il est nécessaire d’avoir aussi recours aux diverses richesses culturelles des peuples, à l’art, à la poésie, à la vie intérieure, à la spiritualité » [ES 63]
\end{singlequote}


Pour analyser la situation, “\textit{la réalité est supérieure à l’idée}” : partir de la réalité est par ailleurs un impératif pour les chrétiens car il est lié à l'incarnation de la Parole  [LS 233]. Pour l’analyse et pour l’action, on peut rappeler un quatrième principe avancé par le pape François : “le tout est supérieur à la partie”. Dans EG 235, il précise qu’il faut toujours élargir le regard […] sans pour autant se déraciner”.  
Puis l'A. souligne la spécificité du christianisme dans cette démarche de dialogue [LS 121]. Le discours spirituel doit reconnaître le caractère unique de ce monde, autrement dit qu’il ne suggère en rien un autre monde, qui serait « spirituel », double du monde matériel, mais bien un même monde que l’on peut habiter spirituellement. B. Latour \cite{latour_jubiler_2002} s'interrogeait sur la possibilité d'un discours spirituel à une époque où la communication se réduit à l'information instantanée. Une parole spirituelle demeure pourtant possible.

\begin{singlequote}
Il y a peut-être une façon spirituelle de parler dans ce monde, qui diffère en effet radicalement du transport d’information double clic, mais il n’y a pas de “monde spirituel” en supplément de l’autre »\cite[p.40]{latour_jubiler_2002}
\end{singlequote}
La parole spirituelle doit alors être transformative, à travers la conversation
\begin{singlequote}
transformant quelqu’un qui était lointain en proche – la conversion \cite[p.45]{latour_jubiler_2002}
\end{singlequote}
C'est précisément l'approche de LS d'un langage performatif, langage dans toutes ses formes y compris poésie.
 
La pratique du dialogue se déduit d’une théologie de la création, ce que Laudato Si’ appelle l’harmonie de toute la création. 
La sagesse, la capacité de « saisir la variété des choses dans leurs relations multiples » [LS 86] selon la définition de Saint Thomas d'Aquin, est \textit{pédagogie} de la diversité du monde commun, tout en nous faisant pressentir l’unité de ce qui anime nos existences.
\begin{singlequote}
    le fruit de ce dialogue est de faire naître de l’intérieur même des interlocuteurs ce souci commun de la terre, de l’ensemble de la création, et des êtres, souci de la terre (et non pas « sauvegarde ») et souci d’autrui. \cite[par. 7]{goujon_laudato_2022}
    \end{singlequote}

LS s'adresse à tous et pas uniquement les chrétiens. 
Cependant, cela n'empêche pas de proposer ce que peut être une spiritualité écologique chrétienne [ES 216] non par prosélytisme mais pour dire aux chrétiens comme leur relation au Christ  affermit leur conversion écologique [LS 217]. 
Le pape présente alors la figure de François d'Assise [LS 218], modèle de la joie  [LS 10] qui signe la prophétie chrétienne illuminée de l'expérience pascale : 



\begin{singlequote}
La figure de François d’Assise fait transition entre le propos sapientiel et prophétique de l’appel à la conversion et la conversion à laquelle appelle Jésus-Christ. Le lecteur a déjà été initié à la force de transformation que recèle la figure de François d’Assise, comme étant celui qui par excellence prend \textit{soin} (\textit{cura}, en italien, \textit{cuidado} en espagnol, \textit{care} en anglais) dans la joie et la paix (LS 10, puis 222 et suivants). François d’Assise occupe une position de sage. Il est comme un point de passage entre chrétiens et non-chrétiens. Il joue maintenant comme figure de sainteté, conduisant au Christ et à ce qu’il peut inspirer dans la vie des croyants, attitudes que l’on retrouve chez l’un comme chez l’autre. Ces attitudes de «\textit{fraternité sublime avec la création}» [LS 221], de joie, de paix, d’amour civil et politique (comme le détaillent les n° 222 à 232) vont permettre de franchir le seuil de la foi chrétienne. \cite[par. 10]{goujon_laudato_2022}\end{singlequote}

La rencontre de l'autre crée une tension : si on va vers l'autre, la rencontre, la vraie sagesse, comme revenir, se convertir à Dieu seul ? 
A côté de la sobriété heureuse et de la paix, vertus écologiques permettant le vivre ensemble, la conversion implique Dieu : 
\begin{singlequote}
    Le critère théologal montre comment la conversion écologique commune se joue pour le chrétien comme retour à Dieu, en tant que donateur du monde, comme reconnaissance de sa présence cachée dans le monde unique, faisant naître ainsi du sein même de cette reconnaissance l’attitude nécessaire du soin à prendre d’autrui et de la terre, mais aussi une attitude religieuse de reconnaissance, la louange de Dieu, à même la contemplation du monde. \cite[par. 11]{goujon_laudato_2022}
    \end{singlequote}
Ainsi, nous répondons à notre vocation commune à être rassemblé \textit{tout en tous} (1 Co 15,28 selon LS 20)
    \begin{singlequote}
    
L’itinéraire du Christ présente aux chrétiens engagés dans la militance écologique la figure de la Croix, comme mise en échec, renoncement à la violence et espérance qui a de quoi dérouter nos impatiences à vouloir convertir les chrétiens, ou tout homme, à l’écologie. Le second critère, théologal, qui reconduit à la reconnaissance de l’œuvre de Dieu, ne s’ajoute pas de l’extérieur à l’expérience que nous faisons de ce monde : il surgit comme un chant d’émerveillement et une supplication, termes sur lesquels se conclut Laudato Si’. \cite[par. 12]{goujon_laudato_2022}
    \end{singlequote}
La forme sapientielle n'affaiblit pas le caractère prophétique ni ne nie le combat et le cri de la terre et des pauvres. 
       \begin{singlequote}
Cette figure nouvelle de l’humanité transformée apparaît à la toute fin de l’encyclique, dans un passage auquel j’ai failli ne pas prêter attention parce qu’il m’avait semblé d’abord annexe, comme une finale pieuse, reprenant la figure imposée d’une invocation mariale. « Marie, la Mère qui a pris soin de Jésus, lit-on dans les derniers numéros (241), prend soin désormais de ce monde blessé, avec affection et douleur maternelles ». Marie est convoquée comme la figure de l’humanité convertie, pour avoir médité dans son cœur toute la vie de Jésus, poursuit l’encyclique. Mais ce qui apparaît dans cette conversion, ce n’est pas la disparition de la souffrance du monde, mais la compassion. Marie « compatit à la souffrance des pauvres crucifiés et des créatures de ce monde saccagées par le pouvoir humain ». La conversion ne nous fait pas entrer dans un monde spirituel, délivré du mal, mais nous donne d’entendre et de prendre soin de celles et ceux, humains et non-humains, qui souffrent du même sort que Jésus le crucifié, et d’œuvrer dans le temps de ce monde, avec compassion, temps défini comme un « entre-temps » (244), entre le temps présent et celui de la fin où l’on entendra d’une voix forte « Voici, je fais l’univers nouveau » (Ap 21,5, cité n° 243). Cette espérance, qui vient de l’avenir vers nous, voix prophétique, soutient nos luttes (244) et le soin que nous prenons avec compassion pour la planète et ses pauvres. \cite[par. 12]{goujon_laudato_2022}
\end{singlequote}


 


%-------------------------------------------------------------------------------------------------------
\section{Les ressources chrétiennes face à la crise écologique - Bibliographie}
%-------------------------------------------------------------------------------------------------------

%-------------------------------------------------------------------------------------------------------
\subsection{Une éthique universelle pour répondre à l'enjeu universel de la crise écologique ?}
%-------------------------------------------------------------------------------------------------------
\paragraph{Une éthique universelle pour répondre à l'enjeu universel de la crise écologique ?} \textit{De quel genre de pensée a-t-on besoin pour aborder la crise environnementale contemporaine ?} Howles, Timothy ; Kremer, Robert \cite{howles_quel_2022};  \cite{thomasset_recherche_2019}, \textit{La Charte de la terre} citée par LS, \textit{Manifeste pour une éthique planétaire}, Kuschel, Karl-Josef ; Küng, Hans ; \cite{kuschel_manifeste_1995} : \textit{L’éthique planétaire, d’un point de vue philosophique}, Küng, Hans \cite{kung_lethique_2009} \textit{La recherche d’une éthique universelle dans la tradition catholique. La méthode de Laudato Si’ }- A. Thomasset 



%-------------------------------------------------------------------------------------------------------
\paragraph{La Charte de la terre citée par le pape}


\begin{singlequote}
  La Charte de la Terre nous invitait tous à tourner le dos à une étape d’autodestruction et à prendre un nouveau départ, mais nous n’avons pas encore développé une conscience universelle qui le rende possible. [LS 207]
\end{singlequote}


%    vérifier que charte de la terre et Manifeste =

%-------------------------------------------------------------------------------------------------------
\paragraph{Manifeste pour une éthique planétaire} \cite{kuschel_manifeste_1995} 
 
\begin{singlequote}
    une approche pluraliste :
éthique (ou ethos) planétaire, c’est à dire un accord fondamental en matière d’axiologie, de critères indiscutables et de choix essentiels. A défaut d’un consensus éthique fondamental, toute communauté court tôt ou tard le risque du chaos ou de la dictature.  Un ordre mondial meilleur ne peut se concevoir sans éthos planétaire  \cite{kuschel_manifeste_1995}
(préface. p6) 
{...}
“ethique planétaire ne signifie ni idéologie planétaire, ni religion mondiale unitaire à côté des religions existantes, ni quelque forme syncrétique de toutes les autres religions. Notre humanité est lasse des idéologies unitaires et les diverses religions du monde sont de toute manière si différentes dans l’expression de leurs croyance et dans leurs dogmes, dans leur symbolique et leurs rites, que tout effort d’”unification” est dénué de sens. P 6
\end{singlequote}

\begin{singlequote}
    Un principe se retrouve depuis des milliers d’années dans beaucoup de traditions religieuses et éthiques de l’humanité qui l’ont conservé, c’est la “règle d’or”; ce que tu ne veux pas qu’on fasse à ton endroit, ne le fais pas à l’endroit d’aucun autre. \cite{kuschel_manifeste_1995} P 23
\end{singlequote}


\label{Comment:MemoireISTR2}




%-------------------------------------------------------------------------------------------------------
 \paragraph{Kung : éthique planétaire}
 \cite{kung_lethique_2009}

\begin{singlequote}
Le projet d’éthique planétaire se situe dans la foulée de l’éthique de la responsabilité de Max Weber. Il propose une fondation rationnelle de l’éthique (voir K.-O. Apel et J. Habermas). L’être humain jouit d’une autonomie intramondaine mais ne peut fonder seul l’universalité de l’obligation éthique. Onze thèses fondatrices sont alors énoncées comme, par exemple : le jeu a besoin de règles; le fair-play suppose l’observation de normes; éthique n’équivaut pas à doctrine sociale mais à conscience, conviction et attitudes morales ; les règles éthiques peuvent être fondées à partir de la raison sans référence transcendante, etc.
\end{singlequote}

\label{Comment:MemoireISTR3}



\begin{singlequote}
            Il faut plutôt chercher à atténuer, par une solution pragmatique de problèmes urgents, les oppositions entre visions du monde, sans tenir compte des différences idéologiques : cela pourrait à long terme établir des points communs, y compris justement un \textit{éthos} commun. Le conflit des visions du monde ou des idéologies devrait être apaisé de cette façon. \cite{kung_lethique_2009}
\end{singlequote}

%-------------------------------------------------------------------------------------------------------
\paragraph{Critique de l'approche} 

\begin{singlequote}
  [les théologiens contemporains] ont, pour la plupart, sous des formes diverses, essayé de découvrir un horizon commun, impliquant à la fois le mouvement globale de l'histoire et le devenir des religions; Pour ma part, je ne pense pas que l'on puisse lui donner un contenu défini ou concret, on peut à la rigueur le viser avec des formulations formelles empruntées à ce que H. Küng appelle une "éthique planétaire". Cette perspective, il est vrai, demeure floue et ne suscite pas l’enthousiasme soulevé par les grandes utopies du XX siècle. Elles ont démontré par leur réalisation politique leur caractère illusoire et trop souvent dérisoire et cruel.Il me paraît donc inutile de désigner un horizon commun qui serait supposé favoriser le dialogue en proposant une base minimale d'accord.   \cite[p 243]{duquoc_unique_2002}
\end{singlequote}
\begin{singlequote}
 la principale critique adressée aux théologies pluralistes, c’est leur prétention à disposer d’un lieu tiers, d’un arrière-plan qui se situerait au delà des religions particulières et à partir duquel on pourrait les embrasser toutes : le plan nouménal de la Réalité ultime pour John Hick, une même expérience mystique pour Raimon Pannikar \textit{ou encore un même projet éthique pour la justice et la gestion écologique des ressources de notre planète.}   p. 111  \cite{cheno_dieu_2017}
\end{singlequote}

%-------------------------------------------------------------------------------------------------------
\paragraph{La recherche d'une éthique universelle dans la tradition catholique. La méthode de Laudato Si’ - A. Thomasset} \cite{thomasset_recherche_2020}
\begin{singlequote}
    La tradition catholique s’est toujours souciée de garder une dimension universelle à son discours éthique. Historiquement, cette réflexion théologique qui souhaite s’adresser à tous s’est d’abord fondée sur la notion de loi naturelle, un concept susceptible de deux interprétations. L’encyclique du pape François\textit{ Laudato Si’} se présente comme une nouvelle manière de faire et un laboratoire de la recherche visée : elle allie une perception adéquate de la crise, une vision de ce qui est recherché, un dialogue à tous les niveaux et la proposition de ressources spirituelles. Un tel processus dynamique insiste sur la double nécessité de convertir nos attitudes et de rechercher un consensus sur nos manières d’être, nos styles de vie. Il souligne le rôle des religions et des sagesses pour fournir une « mystique qui nous anime ».\cite{thomasset_recherche_2020}
\end{singlequote}

La théologie morale a d'abord cherché une éthique universelle sous la forme d'une loi naturelle, d'inspiration stoïcienne à partir de normes tiées de l'observation ou d'une réflexion sur l'ordre naturel créé par Dieu. 

\begin{singlequote}
    Elle le fait en particulier à partir des inclinations naturelles que l’homme découvre comme partie intégrante de sa nature, inclinations qui sont considérées comme menant à leur accomplissement les personnes et la société tout entière. La loi naturelle est donc une notion théologique (elle est fondée sur une théologie de la création) et une notion métaphysique, puisqu’elle est enracinée dans une capacité rationnelle commune à tous les hommes et une vision de l’être (l’être des personnes humaines, l’être de l’ordre social et politique). \cite{thomasset_recherche_2020}
\end{singlequote} 

Elle a été remise en avant par Jean-Paul II après avoir été délaissée par Vatican II. L’objectif de ce retour était de lutter contre les tendances au relativisme et au subjectivisme pour fonder une éthique sur des bases objectives solides. Il demande à la Commission Théologique Internationale de travailler ce sujet. 
 Le document qu'elle produit interroge quelques-unes des traditions de sagesse et des religions du monde afin d’y mettre en lumière l’existence d’un patrimoine moral commun, que certaines sagesses font même découler des exigences inscrites dans la nature en général et la nature humaine en particulier (n° 12). Ainsi sont envisagés l’hindouisme (n° 13), le bouddhisme (n° 14), les sagesses de la Chine (n° 15), les traditions africaines (n° 16) et enfin l’islam (n° 17).\cite{bonino_questions_2011}
Par ailleurs, cette éthique universelle est seule « susceptible de fonder un ordre juste et pacifique dans les relations entre les personnes et les communautés » \cite[Commission Théologique Internationale]{bonino_questions_2011}. 
\begin{singlequote}
     la dimension désormais planétaire de la responsabilité écologique (le réchauffement climatique), le phénomène économique de la mondialisation, dont la récente crise financière a rappelé la face obscure, ou encore l’explosion des biotechnologies, qui touchent aux sources mêmes de l’humain \cite{bonino_questions_2011}
\end{singlequote}

L'A. propose d'explorer la démarche de LS dans  une telle recherche éthique universelle à l’heure de la crise écologique. 
Tout d'abord, l'A. note la différence de traitement par l'application dans l'articulation entre la loi naturelle et le discernement, entre d'une part la morale sexuelle (peu d'espace au discernement individuel) et d'autre part la morale sociale (insistance sur l'ordre de la raison). LS se situe dans cette lignée en insistant sur les ressources de la raison humaine. Pour une recherche d’une éthique universelle en vue de la transition écologique, LS est un exemple et un laboratoire de la mise en œuvre du processus que nous tentons de dévoiler et qui met en interaction des convictions, des attitudes, des manières de procéder et des expérimentations.
Il propose le plan suivant : 
\begin{singlequote}
    Le dialogue nécessaire à cette recherche d’une éthique universelle suppose en effet un diagnostic (les premiers chapitres) et une visée façonnée par des convictions (c’est ce qui apparaît dans le chapitre 4) ; elle exige une aptitude et une ouverture au dialogue avec d’autres (chapitre 5) ; elle nécessite des attitudes et des ressources spirituelles pour être mises en œuvre et expérimentées (chap. 6). \cite{thomasset_recherche_2019}
\end{singlequote}
LS commence cette recherche d'une éthique universelle par un diagnostic (les premiers chapitres) : Ce qui met en mouvement l'éthique, c'est la perception de ce qui est intolérable -dans LS, la fragilité de la maison commune et face à cela la faiblesse des réactions. François insiste sur le caractère inédit de cette crise (LS 17) mais aussi sur sa dimension systémique et planétaire qui exige un effort commun. Mais dès cette phase de diagnostic, il reconnaît l'existence de divergence [LS 60].
\begin{singlequote}
    La manière de voir la réalité est en jeu dans ce processus et suppose dès l’origine une conversion des manières de penser et d’agir [\ldots]. Ce point indique bien la dimension circulaire de la démarche d’une éthique universelle. Ce sont les expériences \cite{spohn_jesus_2010} , les actions concrètes ou les réflexions qui ouvrent le regard amenant les personnes à changer leur mode de penser et de vivre. D’une façon ou d’une autre, il faut oser plonger dans le cercle. \cite{thomasset_recherche_2019}
\end{singlequote} 
Pour le pape, les racines du mal sont humaines et l’enjeu éthique : un anthropocentrisme déviant, qui met au-dessus de tout la raison technique et la domination sur la nature, allié à un relativisme pratique, centré sur le bien-être personnel immédiat, ont fait perdre de vue le bien de l’humanité, notamment des plus fragiles, et la relation vitale avec la terre et les autres vivants. 

Le chapitre 4 de LS synthétise les convictions de l'encyclique et propose une visée  pour la recherche et l’action, la notion d’« écologie intégrale ». Elle manifeste le souci d’une prise en compte de toutes les relations qui conditionnent notre manière d’habiter le monde : relation à la nature et aux autres créatures, relation aux autres humains, relation à nous- mêmes et à Dieu (LS 10, 237) : « tout est lié ». Cette vision conditionne la manière de rechercher une éthique universelle. 

La visée éthique ne sera pas seulement de l’ordre d’une recherche de normes communes (recherche procédurale à la Rawls) mais suppose des convictions sur ce qu’il serait bon de vivre \cite[p.335]{ricoeur_soi-meme_1990} : pour LS, la solution implique une conscience de mutuelle appartenance qui implique de « nouvelles convictions, attitudes et formes de vie » (LS 202).

Pour ces convictions soient partagées au niveau universel, il faut développer un \textit{style de dialogue} à tous les niveaux, international (LS 164-175), national et local (LS 176-175) tout en visant la visée bonne: c'est l'objet du chapitre 5.  
\begin{singlequote}
    Deux raisons justifient cette corrélation entre les divers niveaux de décision : le souci d’une égalité entre les partenaires (compte tenu du fait de l’inégalité actuelle entre nations ou au sein des nations) et le souci de transparence et de contrôle dans les processus de décision (compte tenu du fait que les conséquences néfastes des modes actuels de production et de consommation affectent tout le monde, en particulier les plus fragiles). \cite{thomasset_recherche_2019}
\end{singlequote}
Le dialogue nécessaire doit aussi concerner l’économie et la politique. Il doit encore concerner les religions avec les sciences (LS 199-200), ainsi que les religions entre elles (LS 201). 
\begin{singlequote}
    « En réalité, il est naïf de penser que les principes éthiques puissent se présenter de manière purement abstraite, détachés de tout contexte, et le fait qu’ils apparaissent dans un langage religieux ne les prive pas de toute valeur dans le débat public » (LS 199).
\end{singlequote}
 Les principes du pape François (« le temps est supérieur à l’espace » (LS 178), « l’unité est supérieure au conflit » (LS 198), « la réalité est supérieure à l’idée » (LS 201)) sont convoquées dans ce chapitre comme méthode. Ce qui demande patience, ascèse et générosité.

Enfin, l'Encyclique étudie dans son chapitre 6 les attitudes et les ressources spirituelles qui peuvent être mises en oeuvre et expérimentées. Les religions fournissent des ressources spirituelles nécessaires à cette visée universelle. 

\begin{comment}
    religion universelle ou non ? 
\end{comment}

\begin{singlequote}
Il peut sembler paradoxal de faire appel à la particularité des religions et sagesses (qui peuvent être en rivalité) pour bâtir une vision et un engagement communs pour sauvegarder la maison commune. \cite{thomasset_recherche_2019}
\end{singlequote}
Certains philosophes sont en effet sceptiques sur la valeur des religions dans le débat éthique. L'A. cite néanmoins Rawls et Habermas
\begin{singlequote}
    « Il serait ainsi déraisonnable d’écarter a priori d’un revers de main l’idée selon laquelle les religions universelles (…) conservent cependant leur place dans l’espace différencié de la modernité, parce que leur contenu cognitif n’est toujours pas tari. On ne peut en tout cas pas exclure qu’elles soient porteuses de potentiels sémantiques qui, si on en libère les contenus profanes de vérité, pourraient dégager une force d’inspiration valant pour la société dans son entier. » (	\cite[ p. 204.]{habermas_entre_2008})
\end{singlequote} 

LS 199 rappelle la valeur des traditions religieuses et leur capacité à apporter du sens dans des situations et des temps différents. Puis il développe des éléments pour « \textsc{une mystique de l’action} » (LS 216), selon le voir-juger-agir propre à la doctrine sociale de l’Église  : 
\begin{itemize}
    \item \textit{voir et discerner } : prise  de conscience écologique (LS 7),
    \item \textit{cultiver les vertus} : C’est seulement en cultivant de solides vertus écologiques et sociales que le don de soi dans un engagement écologique est possible (LS 211), permettant une prise de conscience collective. Nous sommes dans le domaine de l'éthique et non des normes.
    \item \textit{Nouvelles habitudes} :  éducation pour une « citoyenneté écologique » qui passe par l’apprentissage d’attitudes, des gestes simples de la vie quotidienne (LS 230), par l’ouverture des cœurs à la beauté (LS 226),par une autre manière de consommer. L'A. souligne l'insistance du pape sur le point face aux addictions (consumérisme, individualisme, foi en la technique) .
\end{itemize}
C'est un processus dynamique, les différents éléments sont en lien les uns les autres.
 Pour le pape, « l’attitude fondamentale de se transcender, en rompant avec l’isolement de la conscience et l’autoréférentialité, est la racine qui permet toute attention aux autres et à l’environnement, et qui fait naître la réaction morale de prendre en compte l’impact que chaque action et chaque décision personnelle provoquent hors de soi-même » (LS 208). 
\begin{comment}
    est ce le christianisme ou de toute religion hétéronome par essence ? 
\end{comment}
 

Une éthique mondiale nécessite d'alimenter les règles internationales de l'intérieur par certaines manières d'être : les religions peuvent apporter humilité et sobriété, la gratitude, le sens d'être connecté aux autres créatures (LS 220).
Les religions peuvent aussi permettre de faire naitre des acteurs de la transition écologiques avec de solides conviction et engagés pour le bien commun et nourris par des pratiques et rites.
Pour les chrétiens, Le pape insiste sur l'eucharistie  (LS 236) et l'exemple des Saints comme témoins de ces vertus (saint François ou Thérèse de Lisieux) 

\begin{singlequote}
    La manière de faire dans la recherche éthique – qui inclut le dialogue, la transparence, l’inclusion des sagesses et religions du monde, la participation de tous à divers niveaux –, suppose l’acquisition d’attitudes de respect, d’ouverture, de bienveillance, de dépassement de soi, de renoncement à imposer sa propre opinion. À leur tour, ces attitudes ne seront pas acquises sans une éducation qui les promeut et sans des ressources spirituelles qui les renforcent. Par ailleurs, il s’agit aussi de penser un cercle vertueux entre les petits gestes quotidiens, les expérimentations collectives de toute sorte, les réflexions éthiques, et les engagements politiques.\cite{thomasset_recherche_2019}
\end{singlequote}

  \begin{singlequote}
      Ceci nous amène à la troisième leçon de ce parcours : les religions et les sagesses du monde ont un grand rôle à jouer dans ce processus, comme l’ont montré les discussions de la COP 21 et la mobilisation des Églises et des religions pour le climat. Leur vision du monde, qui est à la fois universelle, sociale et cosmique, est à la hauteur des enjeux contemporains. Elles sont également en mesure de fournir les motivations d’actions nécessaires pour bousculer les habitudes présentes. La conversion « des mentalités et des structures », dont parle Vatican II dans Gaudium et Spes17, suppose une « mystique qui nous anime » (LS 216). Elles sont enfin capables d’alimenter l’espérance indispensable pour entamer un tel chemin.\cite{thomasset_recherche_2019}
  \end{singlequote}


%-------------------------------------------------------------------------------------------------------
\subsection{Ressources bibliques pour une réponse chrétienne}
%-------------------------------------------------------------------------------------------------------

\paragraph{Ressources bibliques pour une réponse chrétienne}
\textit{Quelques mots avant l’Apocalypse - lire l’Evangile en temps de crise} - A. Candiard \cite{candiard_quelques_2022}
\textit{Between Exile and the New Jerusalem : Prophetic Mourning, Lament and the Ecological Crisis} - Daniel Castillo \cite{cavanaugh_between_2018}; \textit{Création à l’âge de l’anthropocène} - Christoph Theobald \cite{theobald_repenser_2019},  \textit{Parler de la création après Laudato si’}, Lasida, Elena \cite{lasida_parler_2020}

%-------------------------------------------------------------------------------------------------------
\paragraph{Quelques mots avant l'Apocalypse -  lire l’Evangile en temps de crise - A. Candiard}\cite{candiard_quelques_2022} L'Auteur, dominicain, islamologue à l'IDEO - Caire, réfléchit aux ressources de l'Evangile face à la multiplicité des crises et en particulier la crise écologie. Il commence par une critique du mythe du progrès, ancré dans notre vision du monde : 



\begin{singlequote}
        Le progrès scientifique et technique exceptionnel que nous avons connu ces derniers siècles, en particulier les réussites incontestables de la médecine, confirmait cette vision des choses, comme la remarquable expansion économique qui l’a accompagné, dont nous commençons tout juste à comprendre qu’elle comporte aussi des effets délétères. nous savions naturellement que tout n’allait pas bien , mais nous pouvions croire cependant que les choses s’amélioraient.

        Ces schémas de pensée sont si naturellement ancrés, ils informent si puissamment notre appréhension du réel que nous avons du mal à y renoncer tout à fait devant les démentis flagrants que nous offre l’actualité. Nous restons, volontairement ou non, consciemment ou non, orphelins des mythes du progrès, et nous serions bien contents de leur trouver un substitut chrétien, une garantie divine que, malgré quelques péripéties, tout ira pour le mieux.

        Ne serait-ce pas un juste retour des choses puisque de l’avis général, ces philosophies de l’histoire auraient simplement transposé sur terre une espérance chrétienne, laicisé la foi au salut et au paradis ? \cite[pp 89-91]{candiard_quelques_2022}
\end{singlequote}



Les ressources que propose la foi chrétienne ne sont pas uniquement de l'ordre de la prière même s'il s'agir pour le chrétien de reconnaître que ce ne sont pas ses efforts qui sauvent : 

 
\begin{singlequote}
    Il serait naïf, évidemment, de prétendre lutter contre les désastres climatiques en s'en remettant à la seule prière, mais il ne serait pas moins naïf d'imaginer vaincre le mal sans s'attaquer à ses causes, et d'oublier que le premier lieu où je peux envisager de le déraciner, c'est dans ma propre vie.
En moi, le combat eschatologique est déjà engagé, avec sa violence et ses incertitudes :
c'est lui qui est à l'œuvre dans mes crampes d’égoïsme et dans mes envies de bien faire, dans mes fidélités et mes impatiences.Et dans ce combat, l'emporter, c'est d'abord accepter que la victoire a déjà été acquise, non pas par mes efforts, mais par l'amour infini qui se donne à voir dans la croix de Jésus et qu'il me faut, peu à peu, laisser entrer dans ma propre vie. \cite[p.91]{candiard_quelques_2022} 
\end{singlequote}

L'A. se propose de réhabiliter la vision apocalytique de la foi chrétienne : 
\begin{singlequote}
Dans notre cas à nous, les conséquences apocalyptiques du péché ne sont pas justes, car elles ne frappent pas spécialement les pécheurs et, moins encore, à proportion du péché. [\ldots] Impossible de dire à un paysan philippin qui a dû quitter sa terre à cause d'inondations dramatiques que c'est après tout de sa faute, et qu'il aurait dû moins polluer, car il fait face, en réalité, à une véritable injustice immanente: il assume les conséquences terribles d'actions dont il n'est nullement responsable.

[\ldots] Si Jésus tient un discours d'apocalypse, de révélation, ce n'est pas pour nous terrifier plus ou moins utilement, mais bien pour nous faire comprendre ce qui se joue sous nos yeux: non la punition divine des fautes de l'homme, mais le déploiement du mal et de ses conséquences destructrices; autrement dit, la fin des temps à l'œuvre, non comme événement inquiétant dont on attendrait la proximité, mais comme cette réalité présente dans l'histoire depuis son début, véritable trame sous-jacente aux événements du monde. Nous avons besoin de ce dévoilement car, tant que la nature du mal restera inconnue, on pourra croire béatement à l'efficacité de solutions purement techniques aux mena ces qui pèsent sur nos existences. Il est sans doute nécessaire, dans bien des domaines, d'améliorer la législation, de modifier nos modes d'organisation, de négocier la réduction des arsenaux nucléaires ou celle des émissions de gaz polluants, de faire évoluer les opinions publiques; l'engagement politique ou l'action associative peuvent être  inexorable, loin de toute volonté consciente initiale: nous ne maîtrisons pas nos propres catastrophes. \cite[p.xx]{candiard_quelques_2022} 
\end{singlequote}


%-------------------------------------------------------------------------------------------------------
\paragraph{Between Exile and the New Jerusalem : Prophetic Mourning, Lament and
the Ecological Crisis - Daniel Castillo} Daniel Castillo est un théologien catholique américain travaillant sur l'interaction entre la théologie de la libération et la théologie "écologique". 

Dans ce chapitre du livre collectif \textit{Fragile World : ecology and the Church} édité par W. Cavanaugh, l'A. explore l'importance du deuil et de la lamentation des Prophètes (\textit{prophetic mourning}) comme réponse de l'Eglise à la réalité de la crise écologique.  Il les lit aux prières de plainte (\textit{lament and protest to God}).  Pour cela,il étudie le livre de l'Apocalypse pour souligner le rôle de ces prières dans la vie chrétienne, non seulement sur les effets de la crise écologiques mais aussi sur la manière dont le péché est à l'origine de ce contexte \cite[p. 152]{cavanaugh_between_2018}
Il commence par une critique du capitalisme global qui identifie l'homme comme consommateur (\textit{person-as-consumer} selon l'expression de Leslie Sklair). Il s'inscrit dans la filiation de Jean-Baptiste Metz et de sa critique du progrès qui a longtemps irrigué la métaphysique occidentale.

Puis, il étudie le rôle du Prophète, en particulier lors de l'Exil. Dans une société marquée par la figure  royale, établie par ordre divin, le prophète doit lutter contre l'engourdissement face à la mort (\textit{numbness about death}) qui tétanise le peuple face à la crise que connaît Israël. La question, de l'A. est la suivante : pourquoi le chagrin ? Quel est le but d'adopter ou même de cultiver une pratique du deuil ? Brueggemann suggère que l'engourdissement, le déni et le désespoir nourris par la figure royale ne peuvent être brisés que "\textit{par l'acceptation de la négativité et par l'expression publique de notre peur et de notre honte pour l'avenir que nous avons choisi}". La tâche du deuil, par conséquent, est le travail qui permet à la personne ou à la communauté de voir la réalité dans toute sa sévérité et de ne pas être submergée par les ténèbres. Le deuil est la pratique qui permet de briser le cycle du déni et du désespoir, permettant ainsi aux possibilités de conversion, de consolation et d'espoir d'émerger. A partir des travaux de la psychanalyste Joanna Macy, il constate que le déni et le désespoir  empêche la communauté d'agir face à une menace extérieure. Nous avons tendance à sombrer davantage dans le déni ou à être encore plus accablés par un sentiment d'inutilité. En d'autres termes, les informations seules sont souvent inefficaces ou contre-productives. Cependant, dans sa pratique, Macy a constaté que grâce au deuil honnête et douloureux - ce qu'elle décrit comme un "travail de désespoir" - la personne ou la communauté peut non seulement trouver le pouvoir de confronter les réalités de la dévastation écologique et sociale, mais aussi imaginer un nouvel avenir.
La tâche prophétique du deuil consiste alors \cite[p. 159, cité par l'A. ]{brueggemann_prophetic_1978} à : \begin{enumerate}
    \item  Offrir des symboles à la hauteur de l'horreur et de l'expérience qui nous submerge et  qui suscite l'engourdissement et  déni.
    \item Exprimer publiquement ces peurs et terreurs même qui ont été niées si longtemps et si profondément refoulées que nous ne savons pas qu'elles sont là.
    \item Parler métaphoriquement mais concrètement de la réalité mortifère qui plane sur nous et qui nous ronge de l'intérieur, et en parler non pas avec une grâce bon marché mais avec une franchise née de l'angoisse et de la passion.
\end{enumerate}

Mais il ne s'agit pas uniquement des effets bienfaisants de la catharsis émotionnelle, car face à une menace qui reste présente, son effet serait de courte durée. L'A. propose comme réponse de cultiver l'espoir en quelque chose au-delà de soi-même et de son propre pouvoir.
L'A. cite le roman \textit{The Good Apprentice}, d'Iris Murdoch : un jeune homme nommé Stuart tente de réconforter son demi-frère Edward,  accablé de désespoir face à sa complicité dans la mort d'un ami. Trouvant Edward désespéré, Stuart résiste à la tentation du déni et implore :
\begin{singlequote}
    "Essaie de prier d'une certaine manière, dis 'délivre-moi du mal', dis que tu es désolé, demande de l'aide, trouve une lumière, quelque chose que les ténèbres ne peuvent obscurcir. Il doit y avoir des choses que tu as, des choses auxquelles tu peux accéder, de la poésie, quelque chose de la Bible, du Christ, s'il signifie encore quelque chose pour toi. Laisse la douleur continuer, mais laisse quelque chose d'autre la toucher comme un rayon venant de l'extérieur, de cet endroit à l'extérieur" (Murdoch, The Good Apprentice, 47).
\end{singlequote}
 

Tout espoir chrétien est fondé ultimement sur la fidélité de Dieu. L'A. s'appuie sur le livre de l'Apocalypse : le Royaume de Dieu est "ce lieu extérieur que les ténèbres ne peuvent obscurcir" où toutes les larmes seront essuyées (Ap 21:4), où la justice et la paix s'embrasseront (Ps 85:10). Du point de vue théologal, le dépassement du désespoir est réalisé de manière appropriée en se tournant - voire en criant - vers Dieu. L'acte de lamentation est une prière qui rassemble la colère, la souffrance, le deuil et l'espoir. 
Les lamentations bibliques sont variées : souffrance innocente (Job),  persécution subie par les saints pour leur fidélité au Christ (Apocalypse),  et l'exil, (expérience collective de culpabilité). Les Églises du Sud - ayant contribué relativement peu au changement climatique - se trouvent par exemple dans une position proche de celle de Job. L'exil reflète l'expérience des églises occidentales. 

Castillo souligne que dans l'Apocalypse, il n'y a pas d'extase (\textit{rapture}) : 
\begin{singlequote}
     "Au contraire, Dieu est 'emporté' vers la terre pour y prendre résidence et 'demeurer (skene, skenoo) avec nous \cite{rossing_rapture_2005}" Il est important de noter que cette "extase" descendante ne se produit qu'après la chute de la "grande ville de Babylone", symbole dans le texte de l'Empire romain. L'espoir apocalyptique raconté dans le livre de l'Apocalypse ne désire donc pas tant la fin du monde que la fin du système mondial du pouvoir impérial romain. Comme le dit l'Apocalypse à ses lecteurs, ce sont les forces qui détruisent la terre qui seront détruites (Ap 11,18). \cite[p. 161] {cavanaugh_between_2018}
\end{singlequote}

Les prières de lamentations sont donc adaptées à la crise écologique : la lamentation \textit{engendre la fureur} face aux injustices et aux abus du monde et pousse à \textit{résister}. C'est l'esprit de la religion pour Jean-Baptiste Metz, pour qui la religion est simplement l'\textit{arrêt}. La lamentation s'apparente au cri de "stop !". Mais Castillo ne s'arrête pas là et souligne l'espérance du règne de Dieu, de \textit{la nouvelle Jérusalem} qui anime le chrétien. 



 


%-------------------------------------------------------------------------------------------------------
\subsection{LS et la doctrine Sociale de l'Eglise - Bibliographie}
%-------------------------------------------------------------------------------------------------------


%-------------------------------------------------------------------------------------------------------
\paragraph{La doctrine sociale de l'Eglise selon François - Ch. Theobald} \cite{theobald_lenseignement_2016} Les deux textes de François qui parlent de la doctrine sociale, \textit{Evangelii gaudium} (EG) et \textit{Laudato Si’} [LS]  se complètent mutuellement et présentent un style différent et signifiant des textes traditionnels du magistère sur la doctrine sociale de l'Eglise, qui nous implique dans un parcours de conversion \cite[par. 4]{theobald_lenseignement_2016}. Cela exige tout d'abord une \textit{unification} théologique de l'enseignement sociale de l'Eglise,  opérée grâce à son inscription dans la globalité de nos Écritures qui lie foi en Christ et Règne de Dieu (Lc 4,43 repris par EG 180) . Cet enseignement doit être reçu de façon concrète, dans un monde pluriel, nouveauté par rapport à Gaudium et Spes, texte référent de l'enseignement social de l'Eglise.  C'est pour l'A. ici qu'intervient le vocabulaire stylistique.  Dans EG, 

\begin{singlequote}
 L’analyse s’affine ici, en particulier dans le chapitre 3 de \textit{Laudato Si’} qui examine « la racine humaine de la crise écologique ». S’inspirant largement du philosophe et théologien allemand Romano Guardini et sans doute (sans les nommer) d’un Ivan Illich et de l’école de Francfort, le pape démonte « la manière dont l’humanité a, de fait, assumé la technologie et son développement  \textit{avec un paradigme homogène et unidimensionnel} » [LS 106]. Le réductionnisme qu’il dénonce s’enracine dans le rapport que nous entretenons avec nos objets : « Il faut reconnaître que les objets produits par la technique ne sont pas neutres, parce qu’ils créent un cadre qui finit par conditionner les \textit{styles de vie}, et orientent les possibilités sociales dans la ligne des intérêts de groupes de pouvoir déterminés » (\textit{LS} 107). Il en résulte « l’imposition (par toute une culture) d’un \textit{style de vie} hégémonique lié à un mode de production » (\textit{LS} 145), la domination technocratique globale, y compris sur l’économie et la politique, par quelques-uns (\textit{LS} 109), faisant que « c’est devenu une contre-culture de choisir un \textit{style de vie }avec des objectifs qui peuvent être, au moins en partie, indépendants de la technique, de ses coûts, comme de son pouvoir de globalisation et de massification » (\textit{LS} 108). \cite[par. 12]{theobald_lenseignement_2016} \label{theo:diagnosticLS}
\end{singlequote}
L'encyclique présente donc un cadre conflictuel et le pape présente ensuite le style de vie alternatif \textit{spécifiquement chrétien}, illustré par le Cantique de François d'Assise au début de \textit{LS}, qui  introduit à l'expérience d'écoute [LS 1-2], des pauvres et de la terre.
La réflexion part de l'écriture et le Règne de Dieu est sout-tendu par une lecture de la Génèse, avec tout le deuxième chapitre de LS consacré à l' \textit{"Evangile de la Création"} et une réflexion sur les récits de Cain et Abel et de Noé :
\begin{singlequote}
    Quand toutes ces relations sont négligées, quand la justice n'habite plus la terre, la Bible nous dit que toute la vie est en danger [LS 70].
\end{singlequote}
Quelle guérison pour cette rupture de relation, ce "péché" ? Il s'agit de quitter les styles de vie unidimensionnelles qui sont aujourd'hui hégémoniques pour s'inscrire dans le \textit{style de vie de l'Evangile}. L'A. identifie 4 principes :
\begin{enumerate}
    \item le temps est supérieur à l'espace
    \item l'unité prévaut sur le conflit
    \item la réalité est plus importante que l'idée
    \item le tout est supérieur à la partie \label{theo:principesLS}

\end{enumerate}
Ce dernier élément est développé par l'A. car il présente une réelle nouveauté par rapport à  Vatican II (GS) , où
 \begin{singlequote}
     Le \textit{singulier} (\textit{tel} individu, \textit{telle} culture ou langue, \textit{tel} peuple), n’y a pas de place ou, disons plutôt, n’y est pas [\ldots] objet d’intérêt. Nous sommes plutôt dans un univers homogène et unidimensionnel, selon le vocabulaire de Laudato Si’ [\ldots]. 
     
     Par contre, la vision du monde d’Evangelii gaudium se comprend selon le modèle du polyèdre (EG 234-237). Le discours doctrinal qui insiste sur les principes n’y perd pas sa nécessaire fonction régulatrice, mais il ne parviendra jamais à rejoindre \textit{« chaque chrétien}, en quelque lieu ou situation où il se trouve » (EG 3), voire « \textit{chaque personne} qui habite cette planète » [LS 3] selon leur singularité en relation, intégrée dans des ensembles sociaux et environnementaux toujours plus larges, mais maintenant leur « \textit{originalité} », selon l’expression du texte. Seule une\textit{ approche stylistique} le permet, car elle est sensible à la confluence de tous les éléments partiels dans une donnée singulière où ces éléments conservent leur originalité tout en étant habités par le tout qu’est la « plénitude de la richesse de l’Évangile ».\cite[par. 29-30]{theobald_lenseignement_2016}
\end{singlequote}
Mais il s'agit aussi de quitter un \textit{anthropocentrisme despotique} [LS 68, 69, 118, 119 et 122], et de quitter le mythe du progrès, perspective absente de GS. LS ne propose pas un scénario mais recommande d'identifier les \textit{possibles scenarios futurs} [LS 60] en refusant les positions extrêmes d'un anthropocentrisme dévié et d'un biocentrisme qui refuse l'intervention de l'homme. Il s'agit d'assurer une recevabilité universelle de l'enseignement social de l'Eglise : 
\begin{singlequote}
    les deux textes du pape François ne se contentent nullement d’une argumentation biblique mais développent un vrai enseignement social qui est particulièrement attentif à sa recevabilité universelle. [\ldots] La différence [avec GS] porte sur la manière de donner droit de cité à l’altérité et à ce qui est divers et pluriel – signifié par la métaphore du polyèdre – et donc au dialogue social qui, s’il est mené en vérité, ne peut qu’introduire la foi chrétienne comme « ressource » vitale ou comme style de vie, fondé sur le principe de « gratuité ». \cite[par. 37]{theobald_lenseignement_2016}
\end{singlequote}
Particulièrement par rapport au thème qui nous intéresse, 
\begin{singlequote}
    Le pape reconnaît parfaitement que « \textit{certains relèguent la richesse que les religions peuvent offrir dans le domaine de l’irrationnel} » [LS 62] ; mais il montre également que la complexité de la crise exige une pluralité d’interprétations et d’apports : « \textit{Il est nécessaire, écrit-il, d’avoir aussi recours aux diverses richesses culturelles des peuples, à l’art et à la poésie, à la vie intérieure et à la spiritualité} » [LS 63]. Et il ajoute que, pour ce qui est des chrétiens et d’autres croyants, « {les convictions de la foi [leur] offrent de \textit{grandes motivations pour la protection de la nature et des frères et sœurs les plus fragiles }}» [LS 64]. \cite[par. 33]{theobald_lenseignement_2016}
\end{singlequote}
Le terme même de "ressource" indique un statut décentré des religions, dans une \textit{vision multidimensionnelle de l'homme au sein de la création}. C'est le rôle de la \textit{"sagesse"} de faire
\begin{singlequote}
    [\ldots] le lien entre ce qui anime chacun et les « ressources » apportées par les religions et la tradition chrétienne ; car la « sagesse » s’inscrit, d’un côté, dans la « culture » prise dans toute sa complexité et comprend, de l’autre côté, l’\textit{ensemble} des traditions religieuses, tout en spécifiant le récit biblique.\cite[par. 34]{theobald_lenseignement_2016}
\end{singlequote}
Cette approche permet de reconnaître théologiquement la valeur du mouvement écologique comme \textit{sagesse}.
\begin{singlequote}
     Le « spirituel » n’est donc nullement réservé aux chrétiens mais s’avère déjà être le fruit du travail de la sagesse au sein de l’humanité. Elle s’exprime aussi à travers les textes cités par l’Encyclique, non seulement ceux de différentes conférences épiscopales nationales et continentales, mais aussi et surtout la Déclaration de Rio sur l’environnement et le développement, reconnue comme « prophétique » par Laudato Si’ [LS 167 et 186], et la Charte de la terre de La Haye [LS 207).\cite[par. 35]{theobald_lenseignement_2016}
\end{singlequote}

La posture de l'Eglise ainsi proposée est donc humble et exigeante, non en aplomb, avec le monopole de l'interprétation du fait sociale [EG 184] mais en écoutant pour \textit{promouvoir le débat} [LS 46 et 188).  Il s'agit d'une vraie mutation du concept même d’enseignement social de l’Église par sa capacité déjà avérée d’initier des « processus» de conversion sociale, au lieu de proposer une synthèse achevée (cf.LS 121), privilégiant le travail à long terme que la possession d'espace. Pratiquement, l'A. repère trois traits de ce style :
\begin{enumerate}
    \item chemin de conversion vers la \textit{sortie de soi}, dépassant nos individualismes [LS 208]. Pour réussir cette conversion, l'Eglise ne doit pas s'adresser simplement à la raison mais adopter un langage direct pour rejoindre le coeur de nos interlocuteurs. 
    \begin{singlequote}
        [\ldots] en étant conscient qu’ils sont divers, situés dans une diversité de cultures et de situations concrètes ; il lui faut donc adopter une forme ou un style « polyédrique » pour s’adresser à eux. \cite[par. 44]{theobald_lenseignement_2016}
    \end{singlequote}
    \item un style \textit{prophétique et contemplatif}, jamais l'un sans l'autre dans l'Evangile. 
    \begin{singlequote}
        L’Évangile est d’abord une « ressource » de bonté radicale déjà à l’œuvre dans les sagesses humaines. C’est pour cela que le prophétisme mis en œuvre par Laudato Si’ ne revêt aucun accent catastrophiste ou « apocalyptique ». Certes l’insoutenable n’est jamais nié ou édulcoré ; mais la dénonciation est d’emblée mise au service d’une espérance inaliénable qui repose sur la création comme « don » [LS 76] : « capables de se dégrader à l’extrême, les êtres humains peuvent aussi se surmonter, opter de nouveau pour le bien et se régénérer, au-delà de tous les conditionnements mentaux et sociaux qu’on leur impose » [LS 205 et 61].\cite[par. 45]{theobald_lenseignement_2016}
    \end{singlequote}
    \item un regard contemplatif "capable d'apprécier profondément les choses sans être obsédé par la consommation" [LS 222]
\end{enumerate}




%-------------------------------------------------------------------------------------------------------
\paragraph{Conversion dans la doctrine sociale de l'Eglise}
 

%-------------------------------------------------------------------------------------------------------
\subsection{Eco-théologie de la libération - Bibliographie}
%-------------------------------------------------------------------------------------------------------
%-------------------------------------------------------------------------------------------------------   
\paragraph{De quel genre de pensée a-t-on besoin pour aborder la crise environnementale contemporaine ? }
        \cite{howles_quel_2022}


\label{Comment:MemoireISTR4}        
   	
    
        Doctrine sociale de l’Eglise : positif sur le role des acteurs

        ecomodernisme : progrès
\begin{singlequote}

        Cependant, pour la nouvelle écologie politique, cette sorte d’écomodernisme est une simple réaffirmation du dualisme de l’être humain face à un monde de la nature passif, non animé et inerte, dans l’attente que grâce à son ingéniosité il soit en mesure de dominer et de maîtriser ce monde. Malgré de bonnes intentions individuelles dans des situations particulières, cela mène invariablement à la perpétuation du paradigme technocratique moderniste et ne réussit pas à prendre en compte ce que le pape François appelle « les racines humaines de la crise écologique » [11]. C’est pourquoi la nouvelle écologie politique rejette entièrement ce paradigme.
\end{singlequote}

        {une critique du judeochristianisme, responsable du modernisme.}
\begin{singlequote}

        Fredric Jameson fait remarquer avec humour que « de nos jours il semble plus aisé d’imaginer la fin du monde que celle du capitalisme »[14]. Bruno Latour considère que cette attitude s’appuie sur des idées religieuses de providence et d’achèvement eschatologique, où le but et point final de l’histoire est décrété d’avance et où les croyants sont invités à structurer en conséquence leurs choix personnels et leurs décisions. Cela a, selon lui, un effet démobilisateur sur les énergies politiques qu’il considère nécessaires pour une action environnementale radicale, révolutionnaire et efficace aujourd’hui.
\end{singlequote}

\label{Comment:MemoireISTR5}  
%-------------------------------------------------------------------------------------------------------
\paragraph{Théologie de la libération} Léonard Boff
{Cri de la terre et clameur des pauvres, quels chantier théologique et quelle pratique ecclesiale ?} \cite{thomasset_recherche_2020}
 


%-------------------------------------------------------------------------------------------------------
\paragraph{Une eco-théologie de la libération} 
\href{https://www.loyola.edu/academics/theology/faculty/castillo}{Daniel Castillo} expose comment une éco-théologie  de la libération : rapport à Dieu, aux hommes à la terre (et de tout ce qui provient de la terre - homme animaux).   il s'appuie sur le livre de la parole de Dieu. 


Il s'appuie sur Gustavo Gutierrez, une libération intégrale : socioéconomie, économique. Des structures, des valeurs et des imaginations et un combat entre le péché et la grâce. Une Influence sur LS 70.
 
\begin{singlequote}
    Dans ces récits si anciens, emprunts de profond symbolisme, une conviction actuelle était déjà présente : tout est lié, et la protection authentique de notre propre vie comme de nos relations avec la nature est inséparable de la fraternité, de la justice ainsi que de la fidélité aux autres. LS 70
\end{singlequote}
 

\textbf{L'écologie intégrale comme libération intégrale - Castillo} 
 Dieu est lui même le jardinier. Joseph est l'anti-adam et préfiguration du Christ.  L'exode, c'est le chemin d'apprentissage où le peuple est amené à retrouver sa vocation. 

\textbf{Lecture politico-écologique de la parole de Dieu} ne pas accumuler la manne. Glaneur donc faible. confiance en Dieu, limitation de l'accaparation.
Sabbat. Repos de la terre et du travail aussi des servantes \textit{et des animaux}. Lv : année sabbatique et année jubilaire. Pdt l’année sabbatique, on retrouve sa position de glaneur. Jubilaire : Quadruple restauration (terre, esclave,...) Jésus, Lc 4 : une année de bienfaits. Option préférentielle
pour les pauvres. cf LS 237 (Dimanche et sabbat).



\paragraph{Theobald - Création à l'âge de l'anthropocène}
\cite{theobald_repenser_2019}



%-------------------------------------------------------------------------------------------------------
\section{Rapport aux autres religions - Bibliographie}
%------------------------------------------------------------------------------------------------------- 


%-------------------------------------------------------------------------------------------------------
\subsection{Visions traditionnelles du Rapport aux religions - Bibliographie}
%-------------------------------------------------------------------------------------------------------

%-------------------------------------------------------------------------------------------------------
\section{Bibliographie - Théologie des religions}
%-------------------------------------------------------------------------------------------------------
\paragraph{Sous-Problématique : Rôle des religions et de l'Eglise dans la conversion écologique } Revenir à la source théologique



%-------------------------------------------------------------------------------------------------------
\paragraph{Duquoc}    
\begin{singlequote}
    L'asymétrie pousse les Églises à la conscience ferme d'une ignorance parce que les desseins de Dieu sur ce monde intermédiaire ne lui sont révélés que sous des espérances aux contenus indicibles ou aux contours flous et des impératifs éthiques non étrangers à la marche plus ou moins chaotique de chaque fragment. En ce sens, l'Église est sacrement du salut dans la mesure où elle renonce à être le tout, c'est-à-dire à étre unique lieu de l'Esprit et la déléguée de son Seigneur. Une analogie se dessine ici entre le parcours historique de Jésus et le chemin terrestre de l'Église. \cite[p 241]{duquoc_unique_2002}
\end{singlequote}


\begin{singlequote}
En ce sens, l'Église est sacrement du salut dans la mesure où elle renonce à être le tout, c'est-à-dire à étre unique lieu de l'Esprit et la déléguée de son Seigneur. Une analogie se dessine ici entre le parcours historique de Jésus et le chemin terrestre de l'Église.    \cite[p. 241]{duquoc_unique_2002}
\end{singlequote}

 %   On pourra voir comment François dessine le rôle de l'Eglise



%-------------------------------------------------------------------------------------------------------
\paragraph{Dieu au Pluriel -  l'approche culturo-linguistique}    
        \cite{cheno_dieu_2017}

\begin{singlequote}
        
        la principale critique adressée aux théologies pluralistes, c’est leur prétention à disposer d’un lieu tiers, d’un arrière-plan qui se situerait au delà des religions particulières et à partir duquel on pourrait les embrasser toutes : le plan nouménal de la Réalité ultime pour John Hick, une même expérience mystique pour Raimon Pannikar ou encore un même projet éthique pour la justice et la gestion écologique des ressources de notre planète. [critique du Manifeste pour une éthique planétaire de Kung]  \cite[p. 111]{cheno_dieu_2017}
\end{singlequote}

        
\begin{singlequote}
       [impossible car] Nous sommes des humains, insérés dans une culture et des pratiques qui nous façonnent. \cite[p. x]{cheno_dieu_2017}

\end{singlequote}


\begin{singlequote}
    Chéno montre que contrairement à l’approche libérale centrée sur l’expérience du sujet et qui serait le lieu où les croyants se retrouvent, le post-libéralisme appelle à ne pas minimiser les différences, bien au contraire ; dans une approche utilitariste, elles sont mêmes précieuses car ce sont ces différences qui donnent à chaque religion une valeur singulière et vitale.  \cite{pisani_cheno_2018}
\end{singlequote}


  


%-------------------------------------------------------------------------------------------------------
\paragraph{The Nature of Doctrine} \cite{lindbeck_nature_2002}
\begin{singlequote}
    Une religion contribuera probablement davantage au futur de l’humanité si elle préserve ses propres caractéristiques et son intégrité que si elle cède aux tendances homogénéisantes qui vont avec l’expressivisme expérientiel libéral \cite[ p. 115]{lindbeck_nature_2002}
\end{singlequote}

 


%-------------------------------------------------------------------------------------------------------
\subsection{Prophétisme et Idolatrie - Bibliographie}
%-------------------------------------------------------------------------------------------------------
\paragraph{Problématique}
est ce que Idolâtrie nous permet de creuser comment le dialogue inter religieux doit être pensé. 
Une critique de l'idolâtrie, \textit{nouveauté} du pape François

Le penser dans le cadre plus large du rapport au monde et à la sécularisation. 




%-------------------------------------------------------------------------------------------------------

\paragraph{Economie, idolâtrie et sécularisation depuis \textit{Gaudium et Spes} - W. Cavanaugh} 
W. Cavanaugh est théologien américain et explore les relations entre l'Eglise et le monde :
\begin{singlequote}
    L'Eglise est un corps social d'un genre particulier et unique en ce monde (\textit{sui generis}) qui porte la "politique" de Dieu afin de transformer ce monde  en vue du royaume des cieux.\cite[p. 10]{cavanaugh_idolatrie_2022}
\end{singlequote}
Dans son article \textit{Economie, idolâtrie et sécularisation depuis \textit{Gaudium et Spes}} paru en anglais en décembre 2015, de façon concommitante à l'encyclique LS, l'A. explore la continuité et les différences entre Vatican II (GS) et la pensée sociale du pape François. 

 
François ne parle pratiquement jamais de l’économie contemporaine sans adresser une accusation d’idolâtrie, accusation absente dans GS, et presque entièrement absente de Vatican II dans son ensemble.

Comment expliquer cette différence de traitement des questions économiques dans GS et chez le pape François ?

 
\begin{singlequote}
    Ma thèse est que François représente une opportunité pour changer le discours catholique sur la sécularisation, une opportunité qui a des implications dans la manière de considérer non seulement l'économie mais aussi d'autres phénomènes séculiers. 
    La pensée catholique progressiste dans la période du Concile Vatican II avait tendance à considérer le monde séculier comme désenchanté. François suggère au contraire, que nous ne sommes pas tant confronté à une perte de foi qu’à une nouvelle religion et un foi idolâtre. 
    \cite[p. 126]{cavanaugh_idolatrie_2022}
\end{singlequote}
Pour cela, l'A. souligne que la sécularisation de certains phénomènes comme l'économie considérée comme acquises par Vatican II, est aujourd'hui remise en cause. 
\begin{singlequote}
    Je soutiendrai que les attaques de François envers l'idolâtrie de l'argent peuvent être comprises dans l'esprit de Vatican II, non comme un jugement négatif sur le monde, mais comme une reconnaissance des aspirations profondes du monde. \cite[p. 127]{cavanaugh_idolatrie_2022}
\end{singlequote}

Cavanaugh prend tout d'abord l'exemple de ce que dit GS de l'athéisme : considéré comme un manque de foi et non une croyance, et souvent motivé par l'hypocrisie des croyants. GS semble s'adresser à un public européen instruit.
\begin{singlequote}
    Lors du débat sur le schéma XIII (qui deviendra GS) au cours de la troisième session du Concile, les évêques du tiers-monde se plaignirent que le document était trop axé sur le contexte des pay industrialisés, trop centré sur la sécualirsation et sur le communisme - cette plainte a également été formulée lors de la version finale. Des passages, comme celui du n° 61 dans le document final qui recommande d'utiliser son temps libre pour le tourisme et les activités sportives, ne font que renforcer l'impression que le texte était écrit par et pour des bourgeois européens. \cite[p. 130]{cavanaugh_idolatrie_2022}
\end{singlequote}
De même, GS 63 montre un certain optimisme sur la possibilité de la raison de corriger les déséquilibres économiques. La liste des principes économiques nécessaires que liste GS dessine \textit{in fine} ce qui ressemble à une économie saine. Mais, en restant au principe, le texte acte l'autonomie de la science économique, sans exclure néanmoins Dieu car "elle correspond à la volonté du Créateur. C'est en vertu de la création même que toutes choses sont établies selon leur ordonnance et leurs lois et leurs valeurs propres, que l'homme doit peu à peu apprendre à connaître, à utiliser et à organiser  [GS 36]. Le domaine de l'économie reste celui des hommes sobres et rationnels qui tentent d'aménager correctement le monde matériel.

Or, malgré son  importance,  premier des Dix commandements, l'idolâtrie n'est mentionné que trois fois dans les documents du Concile : l'Eglise doit arracher les gens \textit{à l'esclavage de l'erreur (\textit{slavery of error and of idols}} (LG 17),"ceux qui se fiant plus que de raison aux progrès de la science et de la technique, sont enclins à une sorte d'idolâtrie des choses temporelles : ils en deviennent les esclaves plutôt que les maîtres" (\textit{Apostolicam actuositatem} 7), et dans GS 41, nous sommes assurés que grâce à notre foi en l'incarnation, la croix et la résurrection du Christ, \begin{singlequote}
    "l'Eglise peut soustraire la dignité de la nature humaine à toutes les fluctuations des opinions qui , par exemple, rabaissent exagérément le corps humain, ou au contraire l'exaltent (\textit{idolize) }sans mesure".[GS 41]
\end{singlequote}

Des son premier texte (\textit{lumen fidei}), François cite quatorze fois le mot idole. 

A la différence de la vision de l'athéisme de GS, quand on cesse de croire en Dieu, on ne cesse pas de croire mais on croit à toute autre sorte de choses d'où la personification de l'argent en Mammon (Mt6,24).
\begin{singlequote}
    Pour cela, l'idolâtrie est toujours un polythéisme, un mouvement sans but qui va d'un seigneur à l'autre. [\ldots] Celui qui ne veut pas faire confiance à Dieu doit écouter les voix des nombreuses idoles qui lui crient : "Fais-moi confiance !" (\textit{Lumen Fidei, 2013, 13]}
\end{singlequote}

Dans \textit{EG}, la critique de l'économie mondiale contemporaine se fait encore plus vive :  
\begin{singlequote}
    Une telle économie tue. Il n'est pas possible que le fait qu'une personne âgée réduite à vivre dans rue, meure de froid ne soit pas une nouvelle, tandis que la baisse de deu points en bourse en soit une (EG 53)
\end{singlequote}

A la différence de la vision neutre de GS, le marché mondialisé s'est absolutisé, de sorte que le vrai Dieu ne peut plus apparaître que comme une menace "incontrôlable" (EG 57).
l'idolâtrie est associée à l'oubli, par opposition à la fidélité de Dieu : 
\begin{singlequote}
    Libère-nous de l'idolâtrie du présent à laquelle se condamne celui qui oublie. [François, 23 mai 2013]
\end{singlequote}
Face à ces idoles, il nous résister et non nous résigner et user de ruse \cite[p. 138]{cavanaugh_idolatrie_2022}. 
En soi, la richesse n'est pas le problème, c'est le sentiment d'auto-suffisance, le refus de reconnaître sa dépendance par rapport à Dieu qui marque l'idolâtrie.


La différence entre GS et la pensée du pape François ne peut être comprise avec les dichotomies qui ont marqué le Concile et sa réception (progressiste vs conservateur, thomiste vs Augustinien, herméneutique de la rupture ou de la continuité). 
La pensée du pape est à la fois réaliste et joyeuse : 
\begin{singlequote}
    Le chrétien est joyeux, il n'est jamais triste. Dieu nous accompagne. [\ldots] Le péché et la mort ont été vaincus. Le chrétien ne peut pas être pessimiste. ! [François, 24 juillet 2013, JMJ de Rio]
\end{singlequote}

La thèse de l'A., c'est que la différence de François avec GS vient moins de sa perspective sud-américaine que de la profonde évolution du regard sur la sécularisation depuis GS. Lors de Vatican II, la \textit{thèse de la sécularisation} est à son apogée, présentant la sécularisation comme un désenchantement du monde. 
\begin{singlequote}
    Mais plutôt que de considérer la sphère séculière comme un domaine neutre et dépourvu de transcendance, je pense, au contraire, qu'il y a une une migration du sacré de l'Eglise vers le monde, de sorte que le capitalisme, par exemple, est mieux compris non comme dépourvu de dieux, mais comme un nouveau type de religion, souvent idolâtre. \cite[p.143]{cavanaugh_idolatrie_2022}
\end{singlequote}
Cette thèse a entraîné un large consensus : en utilisant les termes de Max Weber, il y a \textit{désenchantement / dé-magication} du monde (\textit{Entzauberung}). Ce consensus a limité la possibilité même d'un discours chrétien sur le monde.
La sécularisation est considéré par certains théologiens progressistes comme l'aboutissement de la tradition judéo-chrétienne elle-même, luttant contre les forces animistes. 

L'A. montre comment le regard sur la sécularisation évolue depuis Chenu, Geffré et Schillebeekx, qui pense la sécularisation dans un contexte marqué par l'existentialisme. Depuis la chute de Berlin, la thèse de la sécularisation a connu des temps difficiles.
Peter Berger qui écrivait en 1968 qu'au XIIè siècle, on ne trouvera les croyants religieux probablement que dans de petites sectes, regroupées pour résister à une culture séculière mondiale" a depuis reconnu :
\begin{singlequote}
    Le monde d'aujourd'hui est aussi furieusement religieux qu'il ne l'a jamais été, et dans certains endroits, plus que jamais. Cela signifie que toute une littérature scientifique produite par les historiens et les spécialistes des sciences sociales, vaguement appelée "théorie de la sécularisation" est fondamentalement erronée". \cite{berger_desecularization_1999}
\end{singlequote}

L'A. reprend la thèse de Talal Asad (1993) que le clivage religieux / séculier est une invention de l'Occident moderne. Il ne s'agit pas de dire que la sécularisation n'existe pas mais que la distinction contemporaine entre religieux et séculier ne correspond pas à la distinction entre sacré et profane : deux processus sont à l'oeuvre dan la société contemporaine : la sécularisation de la religion et la sacralisation du séculier. \cite[p. 155]{cavanaugh_idolatrie_2022} En particulier, dans le domaine de l'économie, l'hypothèse de Max Weber qu'une plus grande rationalisation entraîne un désenchantement du monde est fausse, symbolique en sont les cathédrales de la consommation (Ritzer, 2010) ou l'influence des marques comme substitut à la religion traditionnelle (Shachar, Erdem et all, 2011), avec ces milliers de personnes qui communient à l'achat du nouvel iphone en attendant l'ouverture des magasins à minuit. La nouveauté depuis les années 1960 et Vatican II est donc l'historicisation du clivage religieux / séculier et le brouillage qui en découle. 
\begin{singlequote}
    La religion est notoirement difficile à définir. Cependant, si nous adoptons un point de vue fonctionaliste [à la Durkheim], et que nous comprenons la religion comme ce qui nous construit, en nous les enseignant, ce qu'\textit{est le } monde et quel est notre \textit{rôle} dans le monde, il devient évident que les religions traditionnelles remplissent de moins en moins ce rôle parce que cette fonction est supplantée - ou écrasée- par d'autres systèmes de croyances et de valeur. Aujourd'hui, pour expliquer le monde, l'alternative la plus puissante est la science, et le système de valeurs le plus attrayant est devenu le consumérisme. Leur progéniture académique est l'économie, probablement la plus influente des "sciences sociales".En réponse, cet article soutient que notre système économique actuel devrait également être compris comme notre religion, car il en est venu à remplir une fonction religieuse pour nous. David R. 
 \textsc{Loy}, \textit{ The religion of the Market, \textit{Journal of the American Academy of Religion}, 65, 1997; p. 275}
\end{singlequote}
Le domaine de l'économie n'est pas autonome et vide de sacré. L'Eglise elle-même est affectée par l'"attrait de l'argent", la critique de l'idolâtrie par François étant avant tout une auto-critique \cite[p.160]{cavanaugh_idolatrie_2022}. Par ailleurs, face au dérèglement économique, François prône l'espérance et non l'optimisme.  Enfin, il est possible de voir dans l'idolâtrie de l'argent une aspiration plus profonde de l'homme à rechercher Dieu, certes de façon bien mal ordonnée. En Ac 17, 16, Saint Paul est affligé par la ville d'Athènes pleine d'idoles et il appelle les Athéniens à se repentir de leur idolâtrie, mais il le fait avec une forme de sympathie, car il voit dans leur idolâtrie un tâtonnement inchoatif vis-à-vis du vrai Dieu. L'A. conclut par ce double regard sur l'idolâtrie comme celui de l'esprit de Vatican II.

 

\begin{comment}
\begin{singlequote}
    les fonctionnalistes préfèrent définir la "religion", non pas en termes de \textit{ce que} croient les hommes religieux, mais en termes de la \textit{manière dont} ils croient (c'est-à-dire en fonction du rôle que la croyance joue dans la vie des gens". Clrke / byrne 1993
\end{singlequote}
    à la suite de Durkheim.

    voir règne de Dieu
 

  si sécularisation : nouvel idolatrie, on comprend l'ouverture aux autre religions. 

 
    lire Joseph Komochak : Chenu  inclinaison aristotelicienne et thomiste à l'autonomie vs Ratzinger Augustin séparation de nature et Grace. La valuatazioni sulla Gaudium et spes : Cheni, Dossetti and Ratzinger". in Joseph Doré et Alberto Melloni : volti di fine concilio. 2000

    Alors que le père Chenu voyait une chance dans le processus de sécularisation et dans l'autonomisation d'un nombre grandissant d'actions de l'homme, Claude Geffré était déjà plus critique  tout en reconnaissant le caractère positif de la désacralisation, inscrit au coeur même de la religion judéo-chrétienne. Annoncer la Parole de Dieu à un homme qui a conquis son autonomie et qui a démystifié un certain nombre d'aliénations, permet pour Geffré de lui donner le sens qui est absent désormais du monde sécularisé.

\end{comment}
 



%-------------------------------------------------------------------------------------------------------
\paragraph{Conclusion Intermédiaire : un paradoxe} 
A la différence Dt, qui critiquait fortement les religions extérieures et toutes les compromissions, ici, il semble que nous ayons un paradoxe : positivité des religions non chrétienne et négativité des "compromissions" mais par rapport à une "religion non nommée".

%-------------------------------------------------------------------------------------------------------
\subsection{théologie chrétienne du pluralisme religieux - Bibliographie}
%-------------------------------------------------------------------------------------------------------
\paragraph{Introduction à cette partie}
\begin{singlequote}
    Il s’agit donc pour toutes les religions de puiser dans « leur propre héritage éthique et spirituel », de revenir « à leurs sources » pour « mieux répondre aux nécessités actuelles » [LS 200].
\end{singlequote}

%-------------------------------------------------------------------------------------------------------
\paragraph{Dieu au pluriel, Rémi Cheno} \cite{cheno_dieu_2017}   
 

{dominicain de la province de France} Remi Cheno est né en 1959. Après l'école polytechnique, l'ENSG, il entre au noviciat dominicain en 1994 et continue ainsi ses études, passant de bac +6 à bac +13 après sa thèse de document en théologie dogmatique. Spécialisé en ecclésiologie, il s’est intéressé par la suite plus spécialement à la pneumatologie et à l’eschatologie. 

 


 

{Dieu au pluriel, penser les religions}
{Problématique du livre} Dans ce nouvel environnement que le premier chapitre essaye de qualifier, la question du livre est : \textit{Le dialogue interreligieux a-t-il encore un sens et sous quelles formes ? À quelles conditions ?} 

Rémi Chéno le fait de façon structurée mais la plus accessible possible. 

{Les différentes approches chrétiennes} Le premier chapitre s'applique à définir le monde dans lequel nous sommes, marqué par la condition postmoderne. Puis Rémi Chéno, dans son deuxième chapitre reprend et complète la typologie d'Alan Race des différentes théologies chrétiennes du pluralisme religieux, tout en ne retenant pas la notion de \textit{progrès} qui était présente chez Alan Race. Pour cela, il prend un grand théologien pour chacun des courants : l'exclusivisme avec Karl Barth, l'inclusivisme avec Karl Rahner, abandonnant l'ecclesiocentrisme, les théologies pluralistes, abandonnant le christocentrisme avec John Hick et Paul Knitter (regnocentrisme), une approche post-libérale dont il annonce dès l'introduction qu'il la développera particulièrement. Il suit particulièrement deux théologiens anglo-saxons,  George Lindbeck \cite{lindbeck_nature_2002}   et DiNoia. Cette approche ne cherche pas à trouver un plus petit dénominateur commun mais ce qui \textit{cristallise dans chaque religion} et la rend pertinente au futur de l'humanité, avec une analogie linguistique (religion comme une langue). 

Il termine par l'hypothèse qu'au sein d'une personne, on puisse faire l'expérience d'être \textit{bilingue} en différentes religions, avec une religion maternelle mais la possibilité de comprendre le champ culturel de l'autre.  Il ne s’agit pas de justifier de la double
appartenance : le théologien avertit de l’absolue nécessité de la non confusion ;
mais il s’agit de pouvoir goûter l’autre tradition, au point de se sentir comme
l’autre croyant. 

L'A. commence par présenter la condition post-moderne : l'hypothèse d'une \textit{culture techno-scientifique commune et universelle}, qui pouvait sous tendre la modernité (avec sa foi en la raison et dans le progrès) et de récits englobant, n'est plus tenable. La mondialisation fait cohabiter, parfois paisiblement, souvent de façon chaotique, les religions , les systèmes ou les modèles dans un même voisinage \cite[p. 10]{cheno_dieu_2017}. Les récits de la "modernité" croient en l'émancipation du sujet rationnel et celui de l'histoire de l'esprit universel. Face à la disparition de récits unifiants, On se fait sa petite sauce entre différentes religions et pensées philosophiques en une sorte de   « bricolage religieux », juxtaposition des croyances et de pratiques multiples et contrastées, quête d’identités fortes balisant le quotidien, Le religieux connaît un redéploiement de ses manifestations dans un contexte sociétal caractérisé par la généralisation de « l’individualisme narcissique » (Lasch, Sennet, Lipovetsky).  « Bricolage religieux », juxtaposition des croyances et de pratiques multiples et contrastées, quête d’identités fortes balisant le quotidien, conversions fragmentées, perte des références aux grands récits mythologiques, métissage… le religieux connaît un redéploiement de ses manifestations dans un contexte sociétal caractérisé par la généralisation de « l’individualisme narcissique ».  


Quelles réponses individuelles possibles ? Tout d'abord, R. Chéno présente la réponse intégriste/intégrale, qui essaye de reconstituer un récit unifiant de sa vie à partir d'anciens grands récits à mettre à jour \cite[p. 16]{cheno_dieu_2017}. Elle ne doit pas être confondue avec la réponse identitaire, dont le but n'est plus de donner du sens mais de définir non ce que nous sommes mais en prenant comme définition le groupe lui-même ("ceux qui sont pieux" par opposition "qu'est ce que la piété"). "ils sont chrétiens" "français".
L'A. souligne au contraire la chance de \textit{vivre aux éclats} :
\begin{singlequote}
  J'aime ce monde. La condition post-moderne n'est pas un fardeau, [\ldots]  
    Elle est une invitation à la rencontre, à l'échange et, peut être, au dialogue. \cite[p. 20]{cheno_dieu_2017}
\end{singlequote}

Comment penser théologiquement l'existence des autres religions en ce monde post-moderne ? Où est Dieu et qu’est-Il ? Le dialogue interreligieux a-t-il encore un sens et sous quelles formes ? À quelles conditions ?
Si la modernité n'a pas été "tendre" avec les religions, l'A. indique bien que les dangers pour la religion sont maintenant différents. Relégué par la modernité, le problème n'est plus le grand récit de l'athéisme moderne qui réfuterait les religions, mais la juxtaposition de  rationalités. L'A. convoque tout d'abord K. Barth pour la vision exclusiviste ("hors de l'Eglise, point de salut") et K. Rahner pour la vision inclusiviste ("hors du Christ, point de salut"). Il s'appuie à démontrer la pertinence des approches qui ne sont pas rendues obsolètes par les suivantes.
Pour les visions pluralistes, théologies libérales, il convoque  tour à tour Hick, Knitter, Panikkar, Amaladoss et Pieris. Il met en lumière le paradoxe que ces théologies qui invitent à la reconnaissance de la faillibilité de chaque religion – à commencer par la sienne – en vue de promouvoir une approche mutualiste, mais qui glissent « vers un impérialisme de la pensée ». Car nos théologiens pluralistes, 
\begin{singlequote}
    « à vouloir entrer dans un dialogue mutuel, tendent à réduire la diversité, voire à la rejeter »  \cite[p. 104]{cheno_dieu_2017}
\end{singlequote}

Quant aux conditions du dialogue, ils imposent la aussi une vision impérialiste puisqu’il faudrait abandonner tout ce qui n’est pas négociable. Le cadre dans lequel est pensé le pluralisme devrait s’imposer à toutes les religions. Or, ce cadre ne reflète-t-il pas des conceptions propres aux traditions de leurs auteurs ? Par ailleurs, la prétention à s’extraire de sa propre tradition religieuse pour contempler les convergences ne revient-elle pas à adopter le point de vue de Dieu lui-même ? \cite{pisani_cheno_2018}

La dernière approche présentée est celle du théologien luthérien George Lindbeck \cite{lindbeck_nature_2002}et l'approche post-libérale. il s'agit de ne pas minimiser les différences entre religions  car ce sont elles  qui donnent à chaque religion une valeur singulière et vitale : 
\begin{singlequote}
     Une religion contribuera probablement davantage au futur de l’humanité si elle préserve ses propres caractéristiques et son intégrité que si elle cède aux tendances homogénéisantes qui vont avec l’expressivisme expérientiel libéral \cite[p. 115]{cheno_dieu_2017} 
\end{singlequote}
Lindbeck propose une analogie avec l'approche culturo-linguistique : chaque religion a son langage, son territoire et produit une vision du monde propre. 
Mais il s’ensuit une question : la vérité d’une doctrine est-elle liée exclusivement à la communauté qui la produit, à sa cohérence interne ? Plus encore, le caractère incommensurable des religions peut-il être postulé indépendamment de la culture qui la produit ? Peut on même encore parlé de religions universelles ? 

Et qu’en est-il du dialogue interreligieux ? Dans la perspective post-libérale, le dialogue interreligieux se fonde sur la reconnaissance de la cohérence interne à chaque religion et à sa prétention à l’exclusivité (insurpassabilité) \cite[p. 129]{cheno_dieu_2017}. Si l'approche post-libérale permet de défendre le caractère providentielle de la diversité des religions, le dialogue interreligieux connaît une nouvelle acuité avec la pluralité des territoires dans lesquels habite l’individu.


\paragraph{Quelques aspects critiques}
 R. Cheno soutient la possibilité « d’habiter plusieurs religions à la fois » sans confusion\cite[p. 147]{cheno_dieu_2017}. Il s’agit de pouvoir goûter l’autre tradition, au point de se sentir comme l’autre croyant. On peut se demander si Remi Cheno, par sa capacité à évoluer dans ce nouveau monde comme un \textit{anywhere}, selon la définition qu'en donne David Goodhart, ne décrit pas une vision de la religion que ne serait accessible qu'aux \textit{happy fews} de la mondialisation : 
\begin{singlequote}
    La mentalité des \textit{Anywhere} [\ldots]lui semble révélatrice d'un « individualisme progressiste ». «Elle accorde beaucoup de valeur à l'autonomie, à la mobilité et à l'innovation, et nettement moins à l'identité de groupe, à la tradition et aux pactes nationaux (Église, patrie, famille). La plupart des Anywhere voient d'un bon oeil l'immigration, l'intégration européenne et la diffusion des droits humains, autant d'éléments qui ont tendance à diluer les revendications nationales » (p. 19). ... Un groupe social représentant 20 à 25 \% de la population de nos démocraties, qui «  prédomine parmi les décideurs et les faiseurs d'opinion » (p. 48), et comporte un sous-groupe plus radical de 5 \% qu'il appelle les « villageois planétaires » et qui, lui, se recrute principalement « dans l'enseignement supérieur et dans les milieux de la création» (p. 61). \cite{christophe_boutin_anywhere_2022}
\end{singlequote}

\newpage



 
 
 
%-------------------------------------------------------------------------------------------------------
\paragraph{l'Unique et ses témoins - Ch. Theobald} \cite{theobald_christianisme_2007}   

Dans son livre \textit{Le Christianisme comme style, une manière de faire de la théologie en post-modernité} paru en 2007 \cite{theobald_christianisme_2007}, Christoph Theobald prolonge sa théologie vers le dialogue avec les religions monothéistes dans un chapitre que nous nous proposons d'étudier, intitulé : \textit{l'Unique et ses témoins, Jalons pour une théologie de la rencontre entre juifs, chrétiens et musulmans}.  
 Le chapitre que nous étudions se trouve dans la partie IV ainsi introduite : 

 \begin{singlequote}
     Après avoir "ausculté" notre présent et désigné le \textit{kairos} qu'il représente (I) et après avoir réfléchi longuement à l'enracinement spirituel (II) et scripturaire (III) de la théologie chrétienne, le moment est venu d'aborder directement ce que celle-ci doit donner à penser aujourd'hui : le christianisme comme style qui ouvre à une intelligence de lui-même, libre et accessible à tous, petits et grands. 
     \cite[p 699]{theobald_christianisme_2007}
 \end{singlequote}
 
l'A. organise cette partie à partir du \textit{Credo}, en appliquant une \textit{herméneutique dogmatique} \cite[p. 700]{theobald_christianisme_2007}  qui fait le va et vient entre le dogme, versant normatif du mystère chrétien, les textes canoniques, l'histoire et la pratique actuelle de l'Eglise. La première partie présente la foi en Dieu et de la Trinité en post-modernité, en lien avec l'\textit{éthos} chrétien  : 
 

\begin{singlequote}
    On pourrait certes envisager une relation "binaire" entre l'homme et le tout Autre, relation d'alliance avec le "Dieu unique" dont Abraham reste le prototype, mais notre entrée - à égalité\;- dans une relation de familiarité avec Dieu, telle que Jésus de Nazareth l'a risquée avec son "Père", resterait inconcevable; or, c'est cet accès, surprenant et apparemment excessif pour nos possibilités humaines, que nous attribuons à l'"Esprit de Sainteté". \cite[p. 705]{theobald_christianisme_2007}
 \end{singlequote}
 
 Puis Theobald explore le lien entre foi trinitaire des chrétiens et lien social, pour enfin couvrir la question de la multiplicité des témoins et religions monothéistes, le chapitre qui nous intéresse : \textit{ l'Unique et ses témoins, Jalons pour une théologie de la rencontre entre juifs, chrétiens et musulmans.}

 
La problématique se déplace sensiblement entre l'article du colloque et le chapitre éponyme : le colloque tente d'éclairer la question de l'unicité de Dieu à travers plusieurs religions ou \textit{témoins} et pour cela, fait \textit{jouer} le système \textit{Dieu unique, témoin} (prophète,...) et \textit{tiers} (Egypte ou Nations pour Israel, juifs et grecs pour les Chrétiens,...). 
Le chapitre  du livre questionne quant à lui  \textit{l'énigme de la violence entre les trois témoins} et de la difficulté à communiquer entre religions \cite[p.780]{theobald_christianisme_2007}. Cette violence discrédite les religions monothéistes pour nos contemporains. Comme il n'est pas possible de répondre à la place de l'autre, la réponse viendra ici d'une réflexion proprement chrétienne sur la rencontre entre religions.

 

 
 Méditant la figure de Melchisédech, son hypothèse est que le mystère de l'Incarnation et de la Trinité - différence fondamentale du christianisme par rapport au judaïsme et à l'islam - ,  est en même temps le lieu  où se définit une\textsc{ théologie de la rencontre} \cite[p. 793]{theobald_christianisme_2007}. 
 
 

\newline
 
L'argumentation générale se structure autour de l'idée de la rencontre comme apprentissage.  Tout d'abord, l'auteur présente une hypothèse de classement du judaïsme, du christianisme et de l'islam, appelant :
\begin{itemize}
    \item le judaïsme, \textsc{monothéisme éthique}, Israël se définissant par l'injonction éthique d\textit{'aimer l'étranger car en Egypte, vous fûtes des étrangers} (Dt 10,17-1).  
    \item le christianisme, \textsc{monotheisme méta-éthique} au sens où il insiste sur la communication de l'amour excessif de Dieu à tout être humain.
    \item l'islam, \textsc{monothéisme pre-éthique} car sa lutte pour l'unicité de Dieu surpasse toute préoccupation éthique. 
\end{itemize}
 A l'issue de cette première comparaison, une première difficulté de la communication entre religions apparaît car elles ne se placent pas aux mêmes plans. 


{POur éviter la violence entre religions, la comparaison entre religions doit se situer au niveau de leur \textit{style}.} Nous l'avons vu, la modernité impose ses règles du jeu dans la communication entre religions, et en particulier elle impose le \textit{comparatisme}, ce qui peut se traduire par une réelle violence. Chaque religion est donc invitée à relire son propre patrimoine et à entrer dans le \textit{jeu difficile d'une communication qui consiste désormais à conjuguer le regard interne à sa foi sur les deux autres traditions et la perspective externe des deux autres sur lui} \cite[p. 788]{theobald_christianisme_2007}.

 Les sociétés modernes imposent que toute rencontre véritable soit un processus d'apprentissage. Mais un tel processus n'est pas extrinsèque aux religions monothéistes, à travers différentes tournures  comme la figure du prophète pour le judaïsme. Pour les chrétiens, Jésus est le grand "apprenant" par sa souffrance et son obéissance (He 5,8), et  aussi par ses rencontres rapportées par les synoptiques où Jésus \textit{apprend} des autres qui il est.  

{Pratiquement cette rencontre est un processus d'apprentissage.} Comme pour tout style, cet apprentissage passe un processus; d'abord une purification de nos préjugés et le refus de la substitution de l'un ou l'autre témoin, de son exclusion ou de l'inclusion de l'autre dans sa propre mission. La \textit{Règle d'or} est alors l'étalon de la justesse de notre attitude dans la communication avec l'autre témoin. Puis, il s'agit de penser positivement nos liens, soit par la mystique, les courants spirituels traversant aisément les frontières entre les religions,  soit en pensant le jugement d'excellence que je porte sur ma propre religion sans qu'il ne produise de la violence. Pour cela, il faut  accepter que la raison de la multiplicité des \textit{témoins} fasse partie du dessein de Dieu et en rendre compte, chacun avec les ressources propres de sa religion. 

 
l'A. propose alors une application pratique proprement chrétienne : il s'agit de penser la multiplicité des témoins, en ne s'arrêtant pas au nombre de trois mais en lien avec le mystère de l'Incarnation et de la Trinité, ce que Theobald appelle son \textit{hypothèse d'une théologie de la rencontre}. 
 


 L'A. explicite alors ce qu'est le style chrétien, à partir de nombreuses références bibliques (Ps 110, 3; Mt 5, 20.44; Mt 7,12, Lc 3, 22) : la sainteté de Dieu à laquelle nous sommes appelés est l'amour démesuré qui n'attend aucune réciprocité. Elle dépasse donc la Règle d'or. Etre\textit{ engendré}, devenir fils de notre Père, c'est finalement reconnaître  cet appel démesuré à être \textit{comme} Dieu, toujours dans telle ou telle situation pratique. L'A. s'appuie sur la lettre aux Hébreux, où Jésus est à la fois \textit{associé} à l'unicité de Dieu (He 7, 2s) mais, par son sacerdoce selon l'ordre du roi Melchisédech, prince de la \textit{paix}, il ouvre à la multitude les chemins de la Sainteté de Dieu et leur permet d'être engendrés fils. 

 
Ayant ainsi défini ce qu'est le style de vie chrétien, on peut alors revenir au processus d'apprentissage et définir le {style proprement chrétien de la rencontre.} . La purification nécessaire sera pour les chrétiens, celle des peurs qui marquent notre \textit{mesure humaine}. Et de façon positive, le chrétien qui s'est affronté dans sa propre vie à la question de la sainteté et de la démesure divine, peut admettre que juifs et musulmans sont eux aussi, aux prises aux mêmes combats. Theobald définit finalement ce \textit{style chrétien de la rencontre}, comme le \textit{style} qui se caractérise par une singulière manière d'espérer la paix en affrontant la violence. Il conclut avec Philon d'Alexandrie (\textit{le mode de la victoire n'est pas le même pour tous mais que tous sont dignes d'estime}) en assignant aux chrétiens la mission de montrer l'unicité de chaque témoin. 



%-------------------------------------------------------------------------------------------------------
\paragraph{Ecologie - Approche de Théobald } Dynamique spirituelle à l’œuvre dans le projet, selon une lecture chrétienne inspirée par les analyses de Christoph Theobald sur la foi élémentaire qui s’exprime chez certains de ceux qui contribuent à de tels projets au service de la justice sociale et écologique, sans référence explicite à Dieu.      
\cite{de_benaze_conversion_2020}








%-------------------------------------------------------------------------------------------------------
\subsection{Théologie des religions et Ecologie - Une lecture Theobaldienne - Bibliographie}
%-------------------------------------------------------------------------------------------------------

%-------------------------------------------------------------------------------------------------------
\paragraph{\textit{conversion} et \textit{dialogue} : comment l'articulation de ces deux pôles éclaire la théologie des religions ?}
 \cite{lasida_parler_2020} \cite{campos_laudato_2017} \cite{puglisi_religious_2020}


 \paragraph{décentrement}
 \begin{singlequote}
   Le changement de style de vie qui vise à s'arracher au consumérisme est fondé sur l'invitation à sortir de soi, à quitter une attitude autoréférentielle, pour faire attention à l'impact de chacune de nos actions sur les autres et sur l'environnement (208).   Le repos et l'eucharistie sont également présentés comme une manière d'inscrire notre agir dans une dimension réceptive et gratuite (237). Un seul et même mouvement rassemble ces différentes dimensions de la conversion écologique : un mouvement de décentrement.   Don, interdépendance et espérance sont également au cœur des deux prières qui clôturent le chapitre et l'encyclique, et qui sont encore un appel à construire un avenir partagé.   \cite[p. 191]{francois_loue_2020}  
 \end{singlequote}


%-------------------------------------------------------------------------------------------------------
\paragraph{Laudato Si’ et religion} \cite{powell_laudato_2017}    \cite{pisani_ecologie_2016}
 

   
  
  
%-------------------------------------------------------------------------------------------------------
\section{Les réponses des différentes religions au défi écologique - Bibliographie}
%-------------------------------------------------------------------------------------------------------
\paragraph{Problématique} ouverture aux autres religions qui sont appelées à relever ensemble ce défi. Le rôle de l'Eglise est de révéler, accueillir les autres religions dans la façon propre. Theobald. Retrouver les points durs des autres religions pour nous aider à ne pas proposer une solution de type "Dieu et l'écologie" mais bien penser comment s'articule l'un et l'autre.  


%-------------------------------------------------------------------------------------------------------  
\paragraph{13 novembre 2022 - en marge de la COP 27}
L’archevêque anglican Rowan Williams conduit des chefs religieux sur la colline du Parlement pour une cérémonie de repentance climatique le 13 novembre à Londres.  
 
%------------------------------------------------------------------------------------------------------- 
\subsection{Une réponse Chrétienne}

%------------------------------------------------------------------------------------------------------- 
\subsection{Une réponse de l'Islam pour la théologie Chrétienne}

%------------------------------------------------------------------------------------------------------- 
\paragraph{Ecologie en Islam et Dialogue Interreligieux} \cite{pisani_ecologie_2016} 
    Toynbee montrait aussi que c’est au contact les unes des autres, dans l’interaction de leurs mythes et de leurs théologies que se créent les conditions de l’avenir.
    Laudato Si’ : l'enjeu nécessite le concours de tous.
   
    importance de la situation de ‘Alī al-Ḫawwāṣ dans LS (reconnaissance de l'héritage écologique de l'Islam).     
    \textit{habitus écologique} : définies par des rites ou s'affirment des attitudes singulières ou s'entremêlent monde spirituel et monde matériel. 
    Une réponse : l’islam est la solution (si crise, c'est qu'on n'est pas assez islam; pas d'ouverture aux autres religions).
    
    fitra : revenir à un état originaire. la nature vrai musulman, ne se rebelle pas.
    


 
%-------------------------------------------------------------------------------------------------------     
\paragraph{L'analogie avec l'injustice du taux à intérêt} Les religions monothéistes ont toujours porté des interdictions fortes car source d'inégalité. 
  
%-------------------------------------------------------------------------------------------------------   
\paragraph{Ecologie et Religions - Colloque IDEO 2022 } \cite{pisani_ecologie_2022}

 
\begin{quote}

S2 : 30 - Lorsque Ton Seigneur confia aux Anges: "Je vais établir sur la terre un vicaire "Khalifa". Ils dirent: "Vas-Tu y désigner un qui y mettra le désordre et répandra le sang, quand nous sommes là à Te sanctifier et à Te glorifier?" - Il dit: "En vérité, Je sais ce que vous ne savez pas!".

\end{quote}
\label{Comment:MemoireISTR6}   





\paragraph{Emmanuel Pisani - Al-Ghazali et Ecologie} \cite{pisani_ecologie_2022} \cite{bouguignat_denys_revue_2023}

\label{theol:AlGazali25}
\label{Comment:MemoireISTR7}    




\paragraph{Ibn-Taymiyya et Ecologie} \cite{pisani_ecologie_2022}
 \label{Comment:MemoireISTR8}    


\paragraph{Ecologie et religions - Fabien Révol} \cite{pisani_ecologie_2022}
 \label{Comment:MemoireISTR9}   



%------------------------------------------------------------------------------------------------------- 
\subsection{quelle regard chrétien d'une religion écologique}
 \label{Comment:MemoireISTR10} 


\newpage
%%\chapter{Pourquoi lit-on \LS ?}



%-----------------------------------------------------------------------------------------------------------------------------

\section{Comment les religions répondent aux  enjeux écologiques}

 



\paragraph{L'enjeu climatique interroge la pertinence des religions universelles.} Max Weber définit la religion comme \textit{une espèce particulière de façon d'agir en communauté dont il s'agit d'étudier les conditions et les effets}. Mais que se passe-t-il quand les conditions changent ? Par exemple, en cas de changement climatique ? Comment les religions s'\textit{adaptent} pour proposer une façon d'\textit{agir en communauté } à la hauteur de l'enjeu, question essentielle pour les religions universelles comme le Christianisme ou l'Islam : 

\begin{singlequote}
        Qu’une religion soit raisonnable [donc universelle] dépend largement de ses
pouvoirs d’assimilation, de sa capacité à fournir dans ses propres termes une
interprétation intelligible des diverses situations et réalités que rencontrent
ses adhérents. \cite[ p. 175]{lindbeck_nature_2002}.
\end{singlequote}


\paragraph{Interroger les religions sur leurs effets collectifs}
La spécificité de l'enjeu écologique est en effet de questionner non seulement les pratiques individuelles mais aussi collectives, et ceci à un niveau mondial : si on arrête de produire en France pour moins polluer et déplacer le problème dans un autre pays, cela n'est d'aucune utilité par rapport à la crise climatique.
Notre sujet est donc, plus que la transformation \textit{individuelle}, celui de la transformation de l'action \textit{collective}.






\paragraph{Une limitation de la littérature au terreau chrétien} Vue l'ambition de ce travail, nous nous concentrerons sur le terreau Chrétien et plus spécifiquement à  la réception de l'encyclique \LS dans le contexte européen francophone. 

\paragraph{Qu'est ce que la \textit{réception} d'un texte ?} On peut dire qu'un texte est reçu  s'il transforme dans le temps les communautés chrétiennes, ici dans leur rapport \textit{à la maison commune }\cite{revol_reception_2017}. Une approche {sociologique}, qui s'intéresse aux pratiques, est donc pertinente pour juger de cette réception. 

\subsection{Introduction rapide à \LS}


\paragraph{\LS n'est pas le début de toute forme d'écologie chrétienne} En effet, une franche de militants chrétiens
 existent depuis plusieurs années et essayent de lier spiritualité et la cause climatique, telle la figure du \textit{Militant-Méditant} étudiée par \cite{monnot_figure_2021}. Ces mouvements touchaient peu les institutions ecclésiales.


\paragraph{\LS a été reçu de façon étonnante}
    L'encyclique Laudato si', publiée en juin 2015, marque un seuil dans la prise de conscience par l'Église catholique d'une nécessaire « conversion écologique » \cite{lang_generations_2020}.  La réception de \LS est étonnante : il s’est vendu  150 000  exemplaires de l'encyclique, figurant dans le top 20 des ventes françaises pendant l’été 2015. 
    
    
Accueilli positivement en dehors de l'Eglise (cf les contributions diverses de \cite{revol_reception_2017}), sa réception semble plus lente parmi les catholiques.
\begin{singlequote}
    Quand l'écologie fait son entrée dans les milieux catholiques, elle révèle à la fois les fragilités et les inerties des Églises, et le travail de renouveau qui s'opère dans les périphéries souvent inattendues. Monastères, lieux nouveaux, création d'associations sont autant de signes de la diversité de l'écologie chrétienne, dite intégrale.\cite[4ème de couverture]{lang_generations_2020}
\end{singlequote}

\paragraph{Le vocabulaire de \LS }
L'encyclique introduit un vocabulaire spécifique : conversion écologique et \textit{écologie intégrale}. Elle reprend en particulier 10 fois ce dernier terme: 

\begin{singlequote}
Il s’agit vraiment d’une nouveauté par rapport à ses prédécesseurs qui mettaient plus en avant la notion d’« écologie humaine ». [\ldots] Les thèmes qui entrent dans la structuration de cette écologie intégrale sont les suivants selon le Saint-Père : L’intime relation entre les pauvres et la fragilité de la planète ; la conviction que tout est lié dans le monde ; la critique du nouveau paradigme et des formes de pouvoir qui dérivent de la technologie ; l’invitation à chercher d’autres façons de comprendre l’économie et le progrès ; la valeur propre de chaque créature ; le sens humain de l’écologie ; la nécessité de débats sincères et honnêtes ; la grave responsabilité de la politique internationale et locale ; la culture du déchet et la proposition d’un nouveau style de vie. \cite{revol_lencyclique_2016}
\end{singlequote}

%-----------------------------------
\subsection{Typologie de la réception individuelle de \LS}  

Nous proposons de reprendre la typologie proposée par \cite[\textit{un an et demi après, que peut on dire de la reception de} \LS]{revol_reception_2017} :

\paragraph{Catégorie des Catholiques qui prennent conscience mais ne changent rien} Cette catégorie semble la plus commune. 
Ces catholiques s'informent (\textit{ils participent à la session organisée par leur paroisse (sic)}) mais pour l'instant, rien ne change vraiment.

\paragraph{Quand la lecture devient action}
Parmi eux, certains ont été particulièrement transformés dans leur coeur mais se sentent impuissants dans l'action. C'est la force des rites, de la méditations de textes, de la psalmodie, de la calligraphie ou la lecture attentive de textes \textit{classiques} d'actionner les \textit{émotions}, comme l'écrit Michel de Certeau et de devenir \textit{action} : 
\begin{singlequote}
    La lecture peut se faire « jardin des affects »  : Les « saveurs », « goût », « ferveurs » qui la ponctuent supposent une lecture faite de mouvements : émotions et motions s’y conjuguent ; l’\textit{affectus} implique et stimule un \textit{motus}. Aussi la \textit{lectio} est-elle considérée comme une \textit{actio}.   Michel de Certeau, La Fable mystique, II, op. cit., p. 208. 
 
\end{singlequote}
Ce sous-groupe est donc intéressant car potentiellement un vivier pour une conversion effective dans le temps. 

\paragraph{Surfer sur la vague verte} Un second type, souvent membres du clergé, analyse l'encyclique comme la réponse à une mode : le coeur et l'action ne sont pas transformés.
 
\paragraph{La catégorie des Légitimistes : de l'écoscepticisme à l'obeissance} Profondément attachés à l'autorité du pape, ces chrétiens prennent au sérieux l'encyclique : 
\begin{singlequote}
 
J'ai été témoin de changements dans des communautés religieuses ou de laïcs qui ont vraiment modifié leur enseignement.  [\ldots]

Leur état d'esprit serait à peu près : changeons des choses dans notre vie, si le pape nous le demande, c'est que ce sera sûrement quelque chose de bon pour nous et pour le salut du monde. \cite{revol_reception_2017}
\end{singlequote}
\paragraph{Opposition} La réception n'est bien sûr pas unanime parmi les catholiques, avec deux formes d'oppositions;
Tout d'abord les \textit{meurtris},  marqués à gauche, engagés dans le progrès humain des plus pauvres et qui ne comprennent pas l'extension de \LS à la Création, ni le refus du progrès. 

 De l'autre, \textit{Ceux qui sont forts dans leur foi},  de droite,   confiants dans le système économique libérale et plutôt climato-sceptiques : \LS leur apparaît comme une idéologie à l'opposé de leurs convictions. Leur état d'esprit est de s'expliquer avec le pape, lutter contre lui, ou attendre qu'il y ait un nouveau pape (sic).
 
\paragraph{Rejoindre les précurseurs} Ce sont les chrétiens écologistes de la première heure, se sentant reconnus par l'encyclique et heureux que l'Eglise bouge enfin. Je proposerai plus loin une typologie plus précise de ce groupe. 

\paragraph{Rejoindre les parvis} Ce sont ceux qui ont quitté l'Eglise à cause de leur conviction écologique et qui grâce à \LS, envisagent d'y revenir. \cite[p. 15]{lang_generations_2020} mentionne ainsi l'intervention de Cécile Duflot en janvier 2016 où elle témoigne de sa joie à lire l'encyclique. 

\paragraph{Rejoindre les non-chrétiens ?} \LS apparaît comme un texte pouvant réconcilier les écologistes non-chrétien avec le christianisme, surtout  ceux à la recherche d'une dimension spirituelle. 


% ----------------------------------------------------------------------
\subsection{Typologie au sein des précurseurs}


\paragraph{Les relais nécessaires de cette adaptation} Par rapport à l'ampleur de la transformation à opérer, l'encyclique doit s'appuyer sur des relais au sein des \textit{précurseurs}. 


\paragraph{Pourquoi se concentrer sur les précurseurs ? } Lors de l'écriture au livre collectif de Cécile Renouard(\cite{Renouard_entreprise_2015}), j'ai pu me rendre compte de l'importance des \textit{précurseurs} dans la conversion individuelle ou collective (par exemple les entreprises) : les \textit{précurseurs} mettent en mouvement. \cite{lang_generations_2020} ne dit pas autrement :
\begin{singlequote}
        Au fil de cet ouvrage, [\ldots] j'évoque de manière très subjective des visages et des rencontres qui m'ont aidé moi-même à cheminer. Car le processus vital de la conversion passe toujours, rappelle le pape François, par la rencontre bienveillante avec tous, et notamment avec ceux qui sont déjà plus avancés sur le chemin. \cite[p. 11]{lang_generations_2020}
\end{singlequote}
Notre sujet sera d'étudier qui sont ces précurseurs, comment ils le deviennent, puis comment ils peuvent devenir relais de l'encyclique, comment ils se pensent en tant que communauté, et comment ils interagissent  avec l'institution ecclésiale.


\paragraph{Quelques profils de \textit{précurseurs}} Nous ne retenons pas la proposition de typologie de \cite{carle_contre-revolutions_2017} basée sur une lecture politique ("les décroissants", "les écologistes chrétiens") et proposons une typologie basée sur l'articulation des deux termes "écologie" et "intégrale" (tout est lié) : 
\begin{itemize}
    \item ceux qui sont insérés dans l'Eglise, avec \textit{une sensibilité écologique forte}, souvent complété d'une fibre sociale : on peut citer les personnes travaillant autour du \textit{Campus de la Transition} qui assume ce statut de précurseur (« Comprendre pour agir, former pour transformer »), \textit{ CCFD-Terre Solidaire}, \textit{les Semaines Sociales},  certains ordres religieux (nous avons noté en particulier franciscains, jésuites, assomptionistes et beaucoup d'ordres monastiques) qui sont en pointe dans la réception de \LS    
    \item Ceux qui viennent de l'écologie et moins insérés dans l'Eglise catholique. Ils peuvent être déstabilisés par la dimension \textit{intégrale} de l'encyclique, comme Michel Maxime Egger, figure du \textit{militant-méditant} - \textit{cf infra}. (\cite{alexandre_grandjean_christophe_monnot_irene_becci_spiritualites_2018})
    \item ceux pour qui \textit{la dimension intégrale} ("tout est liée") est essentielle et qui ont pu venir à l'\textit{écologie intégrale} non par la crise climatique mais par des combats éthiques, comme le  \textit{Courant pour une écologie humaine}, fondé par Tugdual Derville et la revue \RLimite. Par la proximité de ces combats (Manif pour tous, ...), ces groupes se rattachent souvent aux \textit{légitimistes} mentionnés \textit{supra} même si leur sensibilité écologiste est souvent plus ancienne. 
\end{itemize}
 On a donc une tension au sein des précurseurs catholiques, entre une sensibilité écologie-fibre sociale ("catho de gauche") et une sensibilité éthique-\textit{écologie intégrale} ("catho conservateur"), qui se reconnaissent étonnement tous les deux dans la même encyclique.
 
\paragraph{L'exemple de Michel Maxime Egger } Dans un article de la \textit{Vie} du 19/05/2015, Michel Maxime Egger - MME, sociologue de formation et figure du \textit{Méditant-militant} écologique en Suisse (\cite{alexandre_grandjean_christophe_monnot_irene_becci_spiritualites_2018}) souligne qu'il est heureux de l'encyclique mais précise son inconfort sur la dimension \textit{intégrale} de l'encyclique et sur le manque de solutions concrètes proposées (le pape ne s'oppose pas au nucléaire par exemple): 
\begin{singlequote}
 [\ldots]
Par contre, je suis davantage gêné par ce besoin qu'ont les hiérarques catholiques, et le pape ne fait pas ici exception, de toujours revenir sur les questions de bioéthique ; la contraception, l'avortement, etc. Certes, je reçois l'argument selon lequel le respect de la nature implique le respect de l'homme et réciproquement. Reste que nous sommes dans des ordres de réalités assez différents. Je reste également sur ma faim quant aux solutions proposées, qui restent assez diffuses. Passons sur le fait qu'il ne parle pas du tout du nucléaire et qu'il est prudent, voire ambigu, par rapport aux OGM.[\ldots]
\end{singlequote}

La critique est intéressante car elle montre en creux l'originalité de l'encyclique par l'absence assumée de solutions concrètes : 
\begin{singlequote}
    [\ldots] la réflexion devrait identifier de possibles scénarios futurs, \textit{parce qu’il n’y a pas une seule issue}. Cela donnerait lieu à divers apports qui pourraient entrer dans un dialogue en vue de réponses intégrales. Sur beaucoup de questions concrètes, en principe, l’Église n’a pas de raison de proposer une parole définitive et elle comprend qu’elle doit écouter puis promouvoir le débat honnête entre scientifiques, en respectant la diversité d’opinions.  (\LS, 60-61)
\end{singlequote}
On est bien loin de la tonalité d'\textit{Humanae Vitae} et sa condamnation des \textit{solutions contraceptives précises}... C'est même une transformation du rapport à la vérité et au dialogue. 


\paragraph{Les \textit{précurseurs} semblent souvent en Groupe ou en réseau.}  Dans les exemples ci-dessus, on peut noter l'articulation entre \textit{individus} et \textit{Groupes} plus ou moins formalisés. L'action \textit{Collective} permet une impulsion beaucoup plus forte : 

\begin{singlequote}
    En certains diocèses, spécialement en France, des évêques ont procédé à la nomination de ministres ordonnés ou de ministres laïques dédiés aux questions environnementales, là où ces fonctions n'existaient pas déjà, tel qu'en Allemagne, en Suisse ou même au Québec. [\ldots] Des ordres religieux ont réorienté leurs investissements monétaires vers des ressources énergétiques renouvelables après avoir désinvesti des ressources fossiles. 
 
\cite{revol_reception_2017}
\end{singlequote}

Il sera intéressant d'étudier comment le rapport entre les individus et le collectif fonctionne et il est ressenti par les les individus aux différentes phases de la \textit{conversion}.
 

\paragraph{quelques conclusions intermédiaires}

 A l'issue de ce premier état des lieux, nous retenons la thèse de \cite{revol_reception_2017} d'un \textit{temps long} nécessaire à la réception de l'encyclique, temps d'\textit{inculturation} , d'\textit{assimilation} de la foi chrétienne à l'enjeu écologique. 
De multiples facteurs vont donc concourir à la réception du texte : A court terme, il faudra  que l'encyclique parle \textit{au coeur} de ses lecteurs, à la fois par sa forme (dialogue avec le monde scientifique, prière,...) et par son fond (réponse à une question, même diffuse déjà portée par les chrétiens).
Sur un temps plus long, l'encyclique a besoin de \textit{relais} qui entretiennent et développent le feu. 

Et c'est sur eux que nous proposons de poser notre question de départ. 


%-----------------------------------------------------------------------------------------------------------------------------
\section{Définition de la  question de départ}

\paragraph{Enjeu des \textit{relais} et de leur lien avec l'institution et la communauté qui les entourent} Ces relais sont des \textit{précurseurs} (au sens de la typologie présentée \textit{supra}) qui se reconnaissent explicitement dans \LS, dans leur discours et actions. 
Cette interaction évolue en fonction du temps, de la  prise de conscience à l'action :
\begin{itemize}
    \item \textsc{phase initiale}, de \textit{prise de conscience} préalable à \LS : \textit{les précurseurs sont-ils sans prédécesseurs ?}  qui donne l'impulsion de départ ? Sont-ils les héritiers ? et alors qui ont été leurs références et ceux qu'ils reconnaissent comme des précurseurs ? ou bien, ils n'ont pas de référence ni de précurseurs (ou bien au contraire de multiples précurseurs), et ils font oeuvre d'une pensée foncièrement originale. Dans ce cas, est-ce un processus individuel ou bien collectif voire institutionnel ? Rôle de la dimension religieuse ? 
    \item  \textsc{phase de cristallisation} : Est-ce que l'institution ou le collectif a confirmé l'impulsion ou au contraire a été une résistance  ? 
    \item \textsc{Lecture de \LS} : coïncide-t-elle avec la phase de cristallisation ? pourquoi ? comment la lecture de \LS a transformé / ajusté leur vision préalable ? En quoi le fait que ce soit un texte de nature religieuse est-elle importante ? Lecture collective ? 
   \item  \textsc{phase de déploiement et d'action}  : soutien de l'institution ? du réseau ou du cercle de \textit{précurseurs}? , ou au contraire départs, scission, en quoi \LS a pu être une ressource ? ...
\end{itemize}

On peut voir dans ces questions, deux types de problématiques qui nous intéressent particulièrement: 
d'abord, la présence de la dimension religieuse et de sa capacité à assimiler l'écologie dans les termes chrétiens et de donner une interprétation intelligible de la crise écologique aux chrétiens. Le but de \LS est de montrer de que manière l'engagement pour la sauvegarde de la maison commune doit naturellement surgir de la foi des chrétiens dans le Christ ressuscité, \cite{revol_reception_2017}. Mais est-il reçu ainsi ?
Ensuite, la problématique de l'articulation entre les différents niveaux de réflexion et d'action : individus, communauté/réseau et enfin institution ecclésiale. 




De ces questions, nous proposons comme question de départ : 
\begin{singlequote}
     Comment devient-t-on \textit{relais} de \LS ? L'enjeu de la réception à long terme
\end{singlequote}

Notre question de départ, souligne un paradoxe, les \textit{relais} de l'encyclique semblent moins les réseaux historiques de l'institution ecclésiale, c'est à dire les paroisses que des précurseurs de la cause écologique, aux formes diverses et rattachés de façon lâche à cette institution. 



 


   
   

%-----------------------------------------------------------------------------------------------------------------------------
\section{Un OVNI dans le paysage médiatique Français  : \textit{Limite, revue d'écologie intégrale.}}


Nous proposons de nous intéresser à un terrain particulier de \textit{relais} de \LS, la revue \RLimite. 

%-----------------------------------------------------------------------------------------------------------------------------
\subsection{Qu'est ce que la revue \RLimite ?}
\paragraph{\textit{Limite, revue d'écologie intégrale.}}
 \LS n'a pas inventé le terme d’\textit{écologie intégrale}, proposé par exemple en France par l'essayiste \textit{Falk Van Gaver}. 
Rappelons en quelques mots ce que ce terme recouvre : 
\begin{singlequote}
   [l'écologie intégrale]  articule dans une perspective unifiée les différents aspects de la vie humaine en rapport avec son environnement, considère que le rapport à Dieu, le rapport à soi, le rapport aux autres, et le rapport à la nature, sont des relations dont il faut prendre soin dans une mesure similaire afin de ne pas introduire de désordre dans le monde (le désordre écologique en est un). Le déséquilibre de ces rapports est à l’origine anthropologique de la crise écologique.\cite{revol_lencyclique_2016}
\end{singlequote}
On voit que dans une telle définition, le "désordre écologique" est l'un de ces désordres mais pas la seule porte d'entrée, comme le montre le mouvement des \textit{Veilleurs}.

\paragraph{A l'origine en France, le mouvement des "veilleurs".} les \textit{Veilleurs} (LV) sont de  jeunes intellectuels qui se sont engagés dans la Cité  au moment de la « Manif pour tous » (LMPT), en réaction à la loi sur le mariage homosexuel. Ils se sont d'abord manifestés par des regroupements, d'abord assis, bougies allumées, le soir, lisant des poèmes ou des philosophes. Lors d'une soirée, les participants ont ainsi récité des passages de Jean-Jacques Rousseau, Émile Zola, Mahatma Gandhi ou encore Max Weber et Antonio Gramsci (\cite{geva_non_2019}) :
  \begin{singlequote}
      Ils refusent d'enfermer leur cause dans une case, cherchant le bien commun au-delà des faux clivages et des querelles partisanes.  Ils n'ont ni drapeau ni slogan, ni chef ni porte-parole.(\cite[p. 8]{bes_nos_2014})
      \end{singlequote}

\paragraph{Un enjeu de reconnaissance culturelle} Cette volonté de dialogue souligne d'une part un regard positif sur le monde, qu'il est possible de transformer (par opposition à une position extra-mondaine de repli sur soi); d'autre part, elle peut être lu comme un enjeu de reconnaissance culturelle, selon la grille de lecture de Bourdieu.  
 
\cite{geva_non_2019} souligne le haut niveau culturel des responsables et activistes LMPT et LV et l'importance du catholicisme pour eux. Le contenu théologique des débats en jeu doit certes être étudié (\textit{Studies of conservatism should thus not only analyse the theological content of religiously-grounded conservative movements}). Mais elle souligne l'importance de l'\textit{engagement pour la reconnaissance} (\textit{struggle for distinction}) des bourgeois catholiques éduqués, reconnaissance à la fois vis-à-vis de l'élite bourgeoise financière mondialisée plus riche et aussi vis-à-vis de l'élite culturelle sécularisée accusée d'avoir perdu toute référence morale : 
\begin{singlequote}
    Highly educated Catholics therefore struggle for recognition in a field of bourgeois distinction where they cannot convert their moral knowledge into cultural capital in the secular field of distinction. They struggle for distinction against the putative empty morality of financial elites who are often the wealthier members of the bourgeoisie, and against the putative moral paucity of secular cultural elites whose secular knowledge hierarchy they perceive as a form of moral epistemics rivaling their own,especially on issues to do with human nature, family, sexuality, and gender relations. 
    \end{singlequote}
Cette opposition vis-à-vis des producteurs de savoir reprend une opposition plus ancienne repérée par Bourdieu entre \textit{Grandes Ecoles} et \textit{Université} : 
    \begin{singlequote}
Yet, they arguably also reflect increasing opposition to university-based production of knowledge in the experimental, medical, and social sciences.
\end{singlequote}
 

\paragraph{Un livre comme héritage du mouvement des Veilleurs}
A la suite de ce phénomène, un groupe de \textit{Veilleurs} fédéré par Gaultier Bès  décide d'écrire un livre 
      \textit{Nos limites : pour une écologie
intégrale} (\cite{bes_nos_2014}). 
      \begin{singlequote}
      Notre intuition est simple : l'être humain ne saurait s'épanouir, ni même subsister, sans reconnaître humblement sa finitude, c'est à dire sans accepter les limites de sa condition.  Aussi lui faut-il consentir à voir ses désirs circonscrits par la nature ou par la société. ( \cite[p. 9]{bes_nos_2014})
  \end{singlequote}

Notons que la référence à l'\textit{écologie intégrale} ne vient pas directement de l'urgence écologique mais de l'action contre le \textit{mariage pour tous}. 





\paragraph{Du livre \textit{Nos Limites} au journal \RLimite} La revue \RLimite   naît de cette filiation, avec une soirée de lancement (5 septembre 2015) et 27 numéros jusqu'à son extinction en 2022. C'est un objet plus complexe que le livre \textit{Nos Limites} dans le positionnement du champ politique droite / gauche (\cite{flipo_limite_2019}), avec Paul Piccarreta comme directeur de la publication.  
\begin{singlequote}
    Quand nous avons fondé \RLimite  avec une petite bande d'amis, nous pensions mettre au monde un fanzine d'étudiants qui ne durerait pas plus que quelques numéros. Les spécialistes ne s'y trompaient pas, qui nous promettaient la fin prochaine de la revue à chaque nouveau numéro. Mais comme nous avancions chaque trimestre avec plus de détermination [\ldots] les spécialistes ont fini par admettre que \RLimite  et l'écologie n'étaient peut être pas un feu de paille. (Edito du dernier numéro (27)). 
\end{singlequote}
 
\paragraph{Ecologie intégrale : une réception de \LS}

 Ce journal s'était donné pour \textit{manifeste} : 
\begin{singlequote}
La revue promeut une écologie intégrale qui se fonde sur le sens des équilibres et le respect des limites propre à chaque chose.
[\ldots]
Dans cette perspective, \RLimite  est orchestrée par différentes sensibilités qui coexistent dans un projet commun : encourager toutes les alternatives à la société de marché. Refusant l’« alternance sans alternative » du clivage droite/gauche, \RLimite  tend la main à tous ceux qui combattent le double empire de la technique sans âme et du marché sans loi. (\href{https://revuelimite.fr/notre-manifeste}{Manifeste de \RLimite})
\end{singlequote}
Au delà d'un manifeste autour de l'\textit{Ecologie intégrale}, nous retenons aussi la volonté explicite d'accueillir différentes sensibilités et non de s'enfermer dans un cercle étroit.

Il est intéressant de comparer ce manifeste avec le dernier édito du journal \RLimite, qui reconnaît une filiation directe avec \LS qui n'était pas présente dans le manifeste :  
\begin{singlequote}
    Nous nous sommes inscrits dans une tradition, celle de l'écologie politique et de la presse militante, parfois satirique. [\ldots] Nous avons cultivé des modèles : [\ldots]. Beaucoup de morts, et un pape bien vivant, François, qui en publiant \LS au moment où nous lancions notre premier numéro nous a confirmés dans notre intuition. Ainsi, nous voulons rester comme la "génération François", la génération d'un pape romain et altermondialiste, venu d'Amérique du Sud pour réveiller l'Europe, plus écolo que les écolo.
\end{singlequote}
Nous faisons l'hypothèse que  les fondateurs \RLimite ont été influencés par l'encyclique, d'une façon qu'il conviendra d'expliciter. 

\begin{comment}
    Eviter le gloubi boulga; se donner un texte normatif par rapport aux crises, départ, critiques
\end{comment}



\paragraph{Choix de la \RLimite comme terreau spécifique de \textit{relais} de \LS} Nous avons classé \RLimite dans la catégorie peu nombreuses des \textit{précurseurs} venus par les enjeux éthiques et non l'écologie au sens premier du terme. Le dernier édito de \RLimite nous permet de ranger la revue dans la catégorie des \textit{relais} puisqu'ils placent \textit{a posteriori} leur action dans l'élan de \LS. C'est aussi le \textit{relais} le plus cité par le quotidien \textit{Le Monde} (entre 2015 et aujourd'hui, 15 articles mentionnaient \RLimite, 6 le \textit{Campus de la Transition}, 5 les \textit{Semaines Sociales}, seul le CCFD récoltant nettement plus de mentions. A titre de comparaison, la théologienne Elena Lasida, chargée de la mission écologie de la conférence des Évêques de France n'a jamais été citée  par \textit{le Monde}). 




%-----------------------------------------------------------------------------------------------------------------------------
\subsection{\textit{Limite}, un \textit{relais} de \LS ? }
\paragraph{Un objet dépassant largement le nombre de ces lecteurs} La revue à son pic se vendait à 3000 exemplaires par numéro. En 2022, le nombre d'abonnements était limité à 1200, un lectorat faible donc. Pourtant, la revue a déclenché un intérêt médiatique certain. Le Monde, Libération, Basta ! (\cite{Basta_2015_Limite}), \textit{la Revue du crieur} (\cite{carle_contre-revolutions_2017}),  et même des revues comme \textit{Débats} ( \cite{de_boissieu_quest_2016}) ou \textit{Esprit} (\cite{Schlegel_2018_Limite}). Rien que pour \textit{Le Monde}, on dénombre 3 articles entièrement consacrés à la revue. La tonalité est initialement assez négative, si on en juge par l'indicateur de tonalité Europresse.   

\begin{singlequote}
    Différents articles de presse ont, par exemple, été très critiques du travail de la jeune équipe éditoriale de la revue \RLimite. Ainsi, dans Le Monde du 13 avril 2018, l'écologie intégrale défendue par cette revue à travers différents positionnements est assimilée à une opération d'enfumage pour recycler de vieilles thématiques conservatrices. L'originalité de l'approche du pape François, à laquelle \RLimite fait indirectement écho, n'est donc pas aisée à faire comprendre dans certains milieux de la société française. \cite[p. 9]{lang_generations_2020}
\end{singlequote}

 Cette attention a forcément rejailli sur la publication, à la fois comme un élément de motivation supplémentaire, mais aussi de prudence sur les thèmes abordés pour éviter d'être taxé d'extrême droite.  

\paragraph{Une évolution de \RLimite } La revue est difficilement classable, d'abord par ce que ces membres ont évolués, mais aussi, et cela une hypothèse à vérifier lors des entretiens, parce que la vision des membres de la rédaction a elle aussi pu évoluer : 
\begin{singlequote}
   La revue a évolué, souligne Piccarreta ; certains fondateurs
tels Jacques de Guillebon sont partis vers L’Incorrect ; Eugénie Bastié ne dirige plus la rubrique « politique ». Le souci est de porter la parole écologiste et sociale du côté des conservateurs, et
avec une certaine réussite, d’après lui. Soit. Mais comment l’objectiver, comment avoir les preuves ? Combien de convertis ? [\ldots] transfuge de quoi à quoi ?  \cite{flipo_limite_2019}
\end{singlequote}
Une lecture politique en est fait par \textit{le Monde} :
\begin{singlequote}
    Eugénie Bastié a quitté la revue en 2019, n’appréciant pas « le déséquilibre » , selon elle, entre conservateurs et « cathos de gauche » qui composent alors la revue. « Le clivage droite-gauche nous a finalement rattrapés » , juge Gaultier Bès. (Le monde 27/10/22 - La revue « écolo catho » « Limite » cesse de paraître)
\end{singlequote}
On peut néanmoins s'interroger si c'est bien le clivage droite-gauche ou bien une clarification des priorités (écolo vs éthique voire d'une conception non présente dans \LS de la \textit{limite} comme frontière) que pouvait cacher le concept un peu flou d' \textit{écologie intégrale}. 

\paragraph{\textit{Ecologie intégrale}, une aporie difficile à tenir face à la réalité} Au coeur de \RLimite, il y a le triptyque  : l’écologie, la critique de la technique et la défense de « la vie » sous toutes ses formes, avec une tonalité décroissante et  anti-capitaliste (\cite{flipo_limite_2019}). 
\cite{Schlegel_2018_Limite} souligne la pertinence de sa critique contre l'écologie politique  : 
\begin{singlequote}
     Répétons-le, l’aporie (plus que la contradiction) objectée par \RLimite à l’écologie politique reste pertinente : comment l’assentiment, sans réserve ni débat, aux interventions techniciennes sur le corps humain, que des lois récentes présupposent ou induisent, peut-il aller de pair avec le refus sans concession opposé aux organismes génétiquement modifiés, à tout ce qui pollue, empoisonne, détruit la nature extérieure, au nucléaire ? Il faudrait au moins s’en expliquer, mais chez les Verts français, seul José Bové a manifesté, à notre connaissance, ses réserves sur des propositions de loi bioéthiques en cours. \cite{Schlegel_2018_Limite}
 \end{singlequote}

Mais la revue est aussi critiquée pour ses positions abstraites, restant aux principes et non à la réalité de la société telle qu'elle se présente \cite{flipo_limite_2019}. Face à cette réalité  (par exemple le soutien des gilets jaunes, antilibéraux mais pas très écolo), la synthèse est moins facile dans la réalité que sur le papier. De même, la position anti-contraceptive de \RLimite : 

 \begin{singlequote}
 Le catastrophisme de \RLimite est-il vraiment toujours justifié ou éclairé ? En tout cas, le «féminisme intégral» de \RLimite, où, par exemple, des femmes jeunes et parfois sans enfants défendent non sans arrogance le refus de toute contraception non naturelle, mériterait une sérieuse discussion contradictoire.  \cite{Schlegel_2018_Limite}
\end{singlequote}

Une dernière hypothèse de cette évolution, c'est la transformation qu'aurait opérée \LS sur les rédacteurs de la revue. Ou en tout cas,  ils semblent accepter l'étiquette "chrétienne" de leur démarche. Ainsi, dans l'avant dernier numéro de la revue, on trouve une section de 15 pages de \textit{témoignages chrétiens} donnant par exemple une visibilité à Isabelle Priaulet et le livre issue de sa thèse à la Catho de Lyon : \textit{penser les fondements philosophiques de la conversion écologique}.


 

\paragraph{Importance du vocabulaire} Le vocabulaire de \RLimite est marquée par l'orientation de la revue (Ecologie intégrale, Conversion écologique). Ils semblent importants, mais on peut se demander si ce n'est qu'un problème de vocabulaire : 

\begin{singlequote}
 Paul Piccarretta concède qu’il y a parfois des désaccords au sein de la revue : « Les points d’achoppement tiennent beaucoup à la définition des mots. Le Comptoir n’aime pas le mot “conservateur”, Eugénie [Bastié] n’aime pas “révolution”. »  [\ldots] Gautier Bès tente une synthèse : « Je suis conservateur au sens où je pense qu’il faut préserver nos conditions d’existence, rien à voir avec conserver les privilèges, un certain ordre du monde qui favorise certains et en exclut d’autres. » « En tout cas, comme l’affirme Paul Piccaretta, on n’est certainement pas progressistes » 
    \cite{carle_contre-revolutions_2017}
\end{singlequote}




%-----------------------------------------------------------------------------------------------------------------------------

\section{Méthodologie proposée pour l'enquête}



\paragraph{Utiliser d'abord le matériau accessible : les journaux \RLimite} Nous avons noté une évolution de la revue. La première phase sera d'objectiver autant que possible cette évolution, en observant l'évolution des thèmes et du vocabulaire dans chaque numéro dans les quatre dimensions structurant selon nous la revue, les dimensions anti-libérale sur le plan économique, anti-technique sur le plan éthique, écologique et religieuse (en suivant les mentions explicites de \LS, sujet de notre travail). 


Ce travail pourra être complété par une analyse statistique à partir de  Twitter. Twitter est en effet l'agora moderne où communiquent la plupart des journalistes de \RLimite. Par une analyse des champs lexicaux et leur évolution dans le temps, des liens entre les différents membres à travers leurs citations (et éventuelle évolution de ces liens), on peut aborder de façon objective notre enquête. Je n'ai pu mener ce travail ici mais il est aujourd'hui assez standardisé (\cite{m_analyse_2020}).

\paragraph{Puis organiser les entretiens avec les acteurs de \RLimite.} La source principale d'information sera recueillie lors d'entretiens semi-directifs avec les membres de la revue soit un potentiel de 20 à 30 personnes. Idéalement, nous viserons un panachage des différents profils : personnes ayant connues la revue du début jusqu'à la fin, personnes ayant quitté \RLimite au bout d'un certain temps, personnalités médiatiquement visibles et rédacteurs plus épisodiques.

\paragraph{Difficultés escomptées} En dehors du problème classique de \textit{rentrer} dans la \textit{communauté} des acteurs de la revue, l'enquête nous paraît raisonnable par sa taille. 


\paragraph{Effectuer des entretiens exploratoires } Vu le profil des interviewés (avec une concentration importante d'agrégés) et leur expérience d'entretiens journalistiques serrés, nous risquons d'obtenir des entretiens très lisses. La préparation de l'entretien semi-dirigé sera essentielle. Une attitude d'\textit{Oie blanche} - \textit{je ne connais rien au sujet} - peut aussi limiter les jeux de pouvoir et faciliter une parole moins normée. Le fait que la revue soit arrêtée devrait néanmoins faciliter le travail car l'enjeu du le positionnement politique de la revue a disparu. 

  



%-----------------------------------------------------------------------------------------------------------------------------
\section{présentation de la grille d’entretien }
 
 
\paragraph{La préparation de l'entretien : de la difficulté d'interroger des intellectuels} L'entretien sociologique vise normalement à interroger les pratiques et non directement les \textit{normes et valeurs}. Or, ici, nous allons interroger des intellectuels et leurs \textit{croyances}. Une attention particulière sera donc portée sur les pratiques, changements (ex : \textit{j'ai quitté }\RLimite), pour rebondir en cas de mentions de ces pratiques ou actions et inviter la personne interrogée à approfondir  (ex : \textit{comment c'est passé votre départ ?}, \textit{Y-a-t-il eu un fait précédent votre départ ? }). 


\paragraph{Grille préparatoire } Nous présentons ici la grille en mettant en parallèle les normes et valeurs que nous cherchons à interroger et les pratiques que nous pouvons interroger lors de l'entretien. Pour nous aider, et créer un lien personnel avec l'interviewé, nous avons repris des questions posées par Gaultier Bès à Isabelle Priaulet (\RLimite, n° 26, 2022) - question en italique et nous pourrions ainsi adapter les questions à chaque interviewé.  
 
\begin{tabular}{p{.3\textwidth}p{.6\textwidth}}
\toprule
 Questions de normes et valeurs & Quelle pratique questionner \\
 \midrule
 Originalité du parcours & \textit{vous avez un parcours plutôt original ? Comment en êtes vous venue à l'engagement intellectuel pour l'écologie ?} \\
 Comment devient on \textsc{précurseur ? Relais ?}  & Pouvez vous me raconter l'histoire de \RLimite ? comment avez-vous connu les autres membres du journal ? \\
 
 
  Ecologie \textit{vs} Intégrale     &     
    La revue \RLimite  s'appelle \textit{revue d'écologie intégrale}. Que recouvre ce terme ?  \textit{pourquoi commencer par aborder la technique ? plutôt que la contemplation par exemple ? }
        
        \\
     Références et précurseurs.  &  \textit{Au fil des pages de \RLimite, vous dialoguez avec de grands noms, ... Qu'est ce qui rapproche toutes ces figures ?   }  \\     
         Impact de \LS        &   Comment \LS vous a touché ? Une phrase ou un passage particulier ? Cela a-t-il changé l'orientation de la revue ? Votre participation ?     \\
    Rapport à la Foi Chrétienne ? & Comment vous décririez vous d'un point de vue religieux ou spirituel ? Est ce un élément important dans votre engagement ?  \\

         Rapport à l'Eglise Institution & Allez vous à la messe ? Quels lieux d'Eglise fréquentez-vous ? \\
Gestion du Conflit & Comment était géré le conflit dans la rédaction ? Y a t'il eu des départs ? \\
 Evolution de \RLimite  & Quelle est votre histoire avec \RLimite ? pourquoi êtes vous restés ? partis ? La revue a-t-elle évoluée dans le temps ? De façon concrète, comment cela s'est il traduit ? Y-a-t-il une date charnière ?  \\

 Tester le modèle du Converti & Peut-on parler de conversion à l'écologie intégrale ? Est-ce que la conversion  vous donne de la joie ? \\
 
 Succès de \RLimite      & Pourquoi les médias ont été intéressés par  \RLimite ? \\
 




  
 \textsc{Devenir relais} : Quelles sont les convictions que vous souhaiter transmettre ?   & pourquoi êtes-vous devenu journaliste ? \LS a-t-il transformé votre engagement ? comment ? \\

 

 \bottomrule
\end{tabular}
 
 

%-----------------------------------------------------------------------------------------------------------------------------

\section{Eclairage du cours de sociologie des Religions}

\subsection{Sécularisation et modernité}

\RLimite propose une critique \textit{intégrale} et cohérente du paradigme de progrès.

\paragraph{Critique de la modernité et du paradigme techno-scientifique de progrès} Ce paradigme, c'est à dire l'ensemble de convictions, de questions ou de dogmes acceptés 
et partagés par une communauté à un moment donné, d'un progrès sans \textit{limite} est  au coeur de la modernité. \cite{universalis_modernite_nodate}. 

 \paragraph{une conviction écologique que le tout est supérieur à ces parties} Emile Durkheim analyse le rite comme ce qui permet de symboliser que le \textit{nous} est supérieur à la somme des parties qui le compose. Or, l'écologie présente une proximité du religieux,  sorte d'« affinités électives » \cite{hervieu-leger_religion_1993} : 
 [\ldots] en réaction à une anthropologie individualiste libérale [\ldots] les écologies radicales ont au coeur que le tout est supérieur à la somme de ses parties, ce qui implique que les désirs individuels s’effacent devant les intérêts de la communauté. \cite[35]{carle_contre-revolutions_2017}
 




\paragraph{Dimension contestataire de l'écologie intégrale} La religion a, dans son mode usuel, un rôle d'attestation de l'ordre social, c'est dire qu'elle soutient que l'individu peut se changer mais pas le monde. Mais, à travers la lecture de \LS par la revue \RLimite, on voit se dessiner un exemple de religion contestataire (nous avons vu en cours l'exemple de la théologie de la libération). La question du mal est ici claire et liée à la structure de la société :  \textit{les racines les plus profondes des dérèglements actuels [\ldots] sont en rapport avec l’orientation, les fins, le sens et le contexte social de la croissance technologique et économique} (LS 91). Cette définition du mal peut expliquer une réception plus difficile de \LS dans les milieux libéraux (les \textit{anywhere}, \cite{atlantico_anywhere_2022}), qui sont les principaux bénéficiaires de cet ordre actuel.
 



\subsection{\textit{Limite}, une figure d'autorité dans un monde sécularisé }

\paragraph{Effacement de la Paroisse, marque traditionnelle de l'institution Eclesiale} \LS est une encyclique qui s'adresse \textit{chaque personne qui habite cette planète}. Il ne s'agit plus de s'adresser aux évèques ou aux prêtres, et par eux, à leurs paroissiens mais directement à tout homme. Et de fait, comme l'a noté \cite{revol_reception_2017}, les prêtres et les paroisses ne sont pas les principaux lieux de réception de l'Encyclique.

\paragraph{Les théologiens remplacés par les communiquants ?} Nous avons vu en cours l'importance de la médiatisation aujourd'hui, avec la valorisation de la figure de Papes comme Jean-Paul II ou François. En revanche, les théologiens sont dévalorisés, au profit de ceux qui savent communiquer. Or, s'il faut bien reconnaître une qualité à la génération de \RLimite, c'est leur grande agilité à communiquer avec les codes d'aujourd'hui sans se faire enfermer dans des étiquettes : art consommé de l'humour et la \textit{provoc'} (cf \textit{le pense-bête du progressisme, \RLimite vous a concocté le lexique du parfait progressiste libéral. Désormais, vous pourrez postuler chez McKinsey.} N° 26). La \textit{provoc'} est aussi un discours contestataire puissant qui montre les incohérences du monde actuel. 

\paragraph{une figure pertinente pour la conversion écologique : la figure du converti} La figure du converti \cite{hervieu-leger_pelerin_2001} est en recherche de normativité : après l'expérience joyeuse et transformante de la conversion, il a peur de s'éloigner de l'orthopraxie (\textit{si je m'éloigne de la pratique, je risque de quitter cette joie du converti}). Par ailleurs, le converti a une volonté prosélyte. Sans même y ajouter la composante \textit{intégrale}; l'\textit{écologie} propose un système cohérent de vie,  qui fait facilement un matériau  pour les \href{https://www.youtube.com/watch?v=RhFp_K8oTgc}{caricaturistes} (le fameux \textit{khmer vert}) : c'est donc qu'il y a une vraie orthopraxie \textit{écologiste}, même si elle n'est pas formulée dans un décalogue ou un livre du \textit{Lévitique}. Par ailleurs, la volonté prosélyte est assurément présente dans cette génération de \RLimite qui n'hésite pas à s'exposer, y compris sur des sujets polémiques. La figure du \textit{converti} paraît donc pertinente.  


\subsection{l'Encyclique : forme moderne de communication ? }

\paragraph{Figure valorisée du pape} Le Pape François utilise divers moyens de communication, d'abord ses actes (la maison Ste Marthe, les appels téléphoniques à de simples chrétiens,...), marque de son \textit{style} théologique. Il a aussi une parole qui s'exprime à travers ses homélies, ses discussions dans l'avion de retour de voyage et bien sûr ses encycliques ou exhortations apostoliques.

\paragraph{Des encycliques d'une forme nouvelle} Les encycliques du Pape François sont d'une lecture accessible. \LS contient même des prières, y compris à la Sainte Famille. Comme l'a souligné \cite{revol_reception_2017}, elle peut générer de l'émotion de son lecteur, émotion qui est l'un des critères du vrai dans un monde sécularisé : "Si cela me fait du bien dans l'immédiat, c'est que c'est vrai". Autre prise en compte d'un monde sécularisé, refusant toute autorité auto-instituée, le pape se veut en dialogue avec les scientifiques et les autres religions tout en proposant une vision propre de l'enjeu écologique : "tout est lié". 

\paragraph{Renforcement du rôle normatif de l'encyclique} L'encyclique devient un élément majeur de ce qui fait \textit{le catholicisme} dans un monde sécularisé, court-circuitant d'une certaine manière l'institution ecclésiale. 



 

% -------------------------------------------------------------------------------------

\section{Hypothèse de réponses}


Nous proposons ici quelques hypothèses de réponses à notre question de départ,  
    \textit{ Comment devient-t-on \textit{relais} de \LS ? L'enjeu de la réception dans le temps long}, qui pourront être validés ou non lors des entretiens.


\paragraph{\LS a été normatif pour les rédacteurs de \RLimite} Les jeunes intellectuels qui ont formé \RLimite citent de nombreuses références, ne voulant pas tomber dans la caricature des cathos conservateurs qu'on peut trop facilement étiqueter. Notre hypothèse est que l'expérience des \textit{Veilleurs} a révélé un \textit{style}, à la fois contestataire  et en dialogue intellectuel avec la société. Finalement \LS a pu unifier leurs multiples références et ainsi faciliter le dialogue. Si cette hypothèse est juste, cela a eu pour conséquence : 
\begin{itemize}
    \item une référence explicite à \LS de plus en plus présente 
    \item un regard extérieur (le Monde,...) moins négatif. 
    \item des départs des membres de la revue qui, sans rejeter \LS, peuvent ne pas se sentir à l'aise avec l'une ou l'autre de ses dimensions, en particulier sur la question des frontières et de l'immigration (cf \textit{Il n’y a pas de frontières ni de barrières politiques ou sociales qui nous permettent de nous isoler, et pour cela même il n’y a pas non plus de place pour la globalisation de l’indifférence. LS 52}. 
    \item d'une certaine façon, si on donne poids au dernier édito de \RLimite, on peut lire l'arrêt de la revue comme l'aboutissement de la transformation d'une revue de \textit{veilleurs} en une revue de \textit{relais} de \LS.
\end{itemize}
 
 




%-----------------------------------------------------------------------------------------------------------------------------

\section{Conclusion}

A la fin de ce travail, le sujet s'est d'une certaine façon simplifié. Je suis parti de l'idée initiale de \RLimite comme d'un cercle restreint, une \textit{secte} au sens de Max Weber, avec des idées et pratiques très affirmées. Cela m'intéressait de voir comment ce groupe interagit avec l'Eglise-institution à travers la réception de \LS (avec l'idée que l'encyclique venant du Pape venait de l'institution). 

On voit dans la proposition d'hypothèse de réponses énoncée plus haut que ma vision a évolué, sensible d'abord à la typologie de réception de \LS et au besoin de \textit{relais-précurseurs} de cette encyclique.
Après avoir travaillé le sujet, je trouve plus pertinente la figure du \textit{convertis} pour \RL : cela peut expliquer certaines incompréhensions comme la discussion sur la pilule  (\cite{Schlegel_2018_Limite}). Le converti n'est pas dans une logique de pure rationalité mais d'orthopraxie qu'il convient de transmettre.

Au cours de ce travail, j'ai été étonné par la sélection qu'opèrent les media et l'intérêt qu'ils ont porté pour cette revue. 
J'ai été aussi touché par la figure proposée par \cite{revol_reception_2017} du chrétien ayant le coeur transformé à la lecture de \LS. C'est la force d'une religion et de ses symboles, rites, textes \textit{classiques} que de faire naître cette émotion qui nous fait sentir \textit{membre} d'une communauté. En guise de touche personnelle, m'est revenu cette phrase étrange d'un vieux chant carolingien entendue il y a trente ans en la cathédrale de Rouen : \textit{A furore Normannorum libera nos, Domine}. Cette phrase, conservée malgré la disparition de la peur viking, nous rappelle que nous ne sommes pas les premiers à subir des tribulations, et que la foi peut nous donner des armes pour lutter sans désespérance mais avec réalisme contre des enjeux qui peuvent légitiment nous effrayer.
 
 
%-----------------------------------------------------------------------------------------------------------------------------
  




 
%\chapter{Les relations islamo-chrétiennes entre XIe et XVe siècles}
\mn{Charbel Attallah}

\section{Deux cas différents : l'Afrique de Nord et l'Espagne}


% --------------------------
\paragraph{pourquoi disparition des chrétiens en Afrique}
Au Xè, il n'y a plus de chrétiens. 
Quelques raisons possibles : 
\begin{itemize}
    \item Pas de théologiens marquants en Afrique à cette époque.
\end{itemize}
Cette disparition rapide fait penser que le christianisme était un produit colonial. 


\paragraph{En Andalus} un peu différent, la quasi-disparition des chrétiens vient probablement de l'attirance de la culture musulmane. 
Les Evèques sont \textit{collabo}. Pour la sécurité, on peut sacrifier son identité. 
cf Avaro (854) à Cordoue \mn{cf p. \pageref{par:Alvaro}}. On passe de l'admiration de la culture musulmane à la haine.


\paragraph{Résistance des martyrs de Cordoue} en 850-51, des jeunes qui affirment leur foi et sont tués par les qadis. Entraîne d'autres \href{https://fr.wikipedia.org/wiki/Martyrs_de_Cordoue}{martyrs}. Seule l'intervention des Évêques sur demande des califes calmera ces martyrs. Mais au Nord de l'Espagne, ces martyrs seront affirmés.
Il s'agissait de dénoncer le laxisme de certains chrétiens. Qu'est ce qui animaient ces martyrs andalous ? 

\begin{Synthesis}
    L'acceptation et la compromission des chrétiens en Al Andalus, génère une réaction de martyrs. 
\end{Synthesis}

% --------------------------
\section{Dialogue par dessus des frontières : la correspondance entre Umar II et Léon}
\paragraph{Califat Abbasside} pas de persecution chrétienne insistante. AlMansur autour de l'an 1000 est la notable exception, qui détruit St Jacques de Compostelle (997). 
\paragraph{Fatimide en 900} En 996, autre épisode violent, Al Akim, destruction de l'Eglise du St Sepulcre. mais ne qualifie pas l
\paragraph{Umar II, calife Omeyyade} auteur de capitalisation anti chrétienne.
\paragraph{Ecrit autour de 1000} Avant, des écrits apologétique à visée interne. Umar prend l'initiative d'écrire à l'empereur de Byzance. Probablement pseudographique. Dans cette correspondance.
Les arguments ne sont pas très originaux. Cela donne un bon exemple de controverse et laisse penser que c'est pseudographique. 

\begin{itemize}
    \item Trinité
    \item raison : Islam voulu par Dieu, la preuve rapidité de la victoire
\end{itemize}

Réponse de Léon : 
\begin{itemize}
    \item Victoire eschatologique de Jésus face à a Satan
\end{itemize}

\paragraph{première étape du dialogue islamo-chrétien} On est dans la polémique


\paragraph{les thèmes qui reviennent} la trinité, 
les chrétiens attaque l'immoralité de Mohammed (Zeinab,...) . Face à ce mouvement, au IX, un mouvement de sacralité de Mohammed.

\paragraph{Concentre les stéréotypes des siècles qui ont précédé} importance de l'authenticité des Ecritures (thème qui se développera au XIe).

\paragraph{Salut de l'autre}
Hors de l'Eglise point de salut et 
\begin{quote}
    Si vous ne m'aimez pas, Dieu ne vous aimera pas 3,55
\end{quote}
On a du mal à penser le salut pour l'autre.

\begin{Synthesis}
    une lettre epigraphique mais la première lettre de dialogue, qui reprend certes les apologétiques respectives mais les articulent.
\end{Synthesis}
% --------------------------
\section{L'après dialogue des philosophes}


avec Elie de Nisibe, on a une sortie, reflux de la vague philosophique dans le dialogue islamo-chrétien.

\paragraph{Touchant} A travers une expérience existentielle, un début de changement de paradigmes dans la rencontre  : plus partage. 

\paragraph{Elie de Nisibe} Elie de Nisibe (975-1046) Archevèque. 
% --------------------------
\subsection{Elie de Nisibe}
    

\paragraph{Un évêque, un ministre ...
et un miracle}


\begin{quote}
Il advint que le Vizir (que Dieu ait pitié de lui !) entra à Nisibe le vendredi
26 Gumâdâ !, de l'année dernière, à savoir l'année 720. J'entrai chez lui le samedi
suivant, alors que je ne l'avais jamais vu auparavant. Il me fit approcher de lui, m'honora et me fit asseoir à côté de lui.


Après avoir invoqué Dieu en sa faveur et l'avoir félicité de sa venue, je me
levai pour m'en aller. Il me fit alors arrêter et me dit: "Sache que depuis longtemps
Je désire avoir des rencontres avec toi et te demander beaucoup de choses. Je veux
donc que ta venue et ton départ de chez moi aient lieu selon mon désir". Je lui
répondit qu'il serait écouté et obéi, invoquai Dieu en sa faveur et m'assis. Après
m'avoir mis à l'aise et m'avoir tenu compagnie, après s'être informé de mes
nouvelles et de la marche de mes affaires et après avoir mentionné les savants et les
hommes de science, il me dit:


Sache que mon opinion au sujet des Chrétiens était \textit{autrefois} l'opinion de
qui a établi qu'ils sont impies et polythéistes21 . Mais \textit{maintenant}, je doute de leur
impieté et de leur polythéisme, à cause d'un miracle étonnant dont je fus témoin,
venant de leur religion. Et je doute également de leur monothéisme à cause de
certaines choses auxquelles ils croient qui obligent à douter qu'ils soient
monothéistes.
\end{quote}

\textit{une autre forme de monothéisme}. 
\paragraph{une rencontre liée à une expérience existentielle}

\begin{quote}    
Je dis: "De quoi donc le Vizir (que Dieu prolonge ses jours !) a été témoin
et qui l'oblige à douter de leur impiété ? Et qu'est-ce qui. dans leurs croyances. le
contraint de douter de leur monothéisme ?"
\end{quote}

\paragraph{Récit du miracle}

\begin{quote}
    Le Vizir me dit: "Quant à ce dont j'ai été témoin et qui m'a obligé à douter
de leur impiété, c'est ceci: Alors que je me trouvais la première fois à Diarbekir, je
me suis dirigé vers Badlîs, pour régler des affaires survenues alors.
A mon arrivée là, une grave maladie m'assaillit, qui me fit perdre mes
forces. Mon appétit cessa et je désespérai de moi-même. Je sortis de Badlîs pour
retourner à Maâfâriqîn, en sorte que, si Dieu le Très-Haut avait décrété à mon
sujet l'inévitable, ceci m'advint en cette ville, ou du moins près d'elle.
Or mon être n'acceptait rien, en fait de nourriture ou de boisson; et la
fatigue de la chevauchée m'avait causé une énorme peine. Je me mis à parcourir
chaque jour une petite distance, tandis que ma faiblesse augmentait, que mes forces
diminuaient et que la maladie s'aggravait et devenait plus pénible.
Je parvins alors à un couvent sur le chemin, connu sous le nom de Couvent
de Saint Mârî. J'étais plus faible que jamais, tandis que la maladie était plus forte
que jamais.
Sitôt arrivé au couvent, considérant la faiblesse dans laquelle je me
trouvais, je fis demander un peu de boisson et la pris, dans l'espérance que cela
soutint mes forces. Mais à peine était-elle parvenue dans mon estomac, que je la
rejetai. La faiblesse de mon être augmenta, et je désespérai de moi-même. Tous
ceux qui étaient avec moi furent inquiets.
Alors le moine chargé du service du couvent vint à moi et invoqua Dieu en
ma faveur. Il me fit apporter un peu de grenades et demanda aux serviteurs de les
égrener et de m'en présenter pour que j'en prenne un peu. Ils lui firent alors savoir
que j'étais incapable de parler ou d'entendre quelqu'un parler, que mon être
n'acceptait aucun aliment et que mon estomac ne retenait plus de boisson, pour ne
rien dire du reste.
Le moine insista auprès d'eux et leur dit: "J'aimerais que vous lui en
portiez, afin qu'il prenne de ces grenades. ne fut-ce qu'un peu: il en tirera profit,
grâce à la bénédiction de ce lieu.
Je fis donc signe à l'un des serviteurs d'accepter sa proposition, car je
m'accrochais à la santé. Je mangeai un peu de ces grenades, qui restèrent dans mon
estomac et le soutinrent. Du coup, j'en pris régulièrement, un peu à la fois, si bien
que mon être reprit force et que mon appétit se réveilla.
Or le moine avait cuisiné des lentilles pour les serviteurs, et les leur
apporta. Je les vis manger, j'en demandai et mangeai avec appétit. \sn{Elie de Nisibe, Kitab al-Majalis. Le livre des discussions. Jamais avec un ton polémique}
\end{quote}
\begin{quote}
    Voila donc ce qui me contraint de croire que les Chrétiens ne sont ni des impies ni des polytheistes.

    Quant à ce qui me contraint de croire, en ce qui les concerne qu'ils sont polytheistes, c'est le fait qu'ils croient que Dieu est une substance en trois hypostases. Ainsi adorent-ils trois dieux, et confessent ils trois seigneurs. Ils croient aussi que Jésus (qui est, selon eux, l'homme assumé de Marie) est éternel et incrée.

    Je dis les Chrétiens n'adorent pas trois dieux et ne croient pas que l'homme assumé de Marie est éternel et incréé.

    Le vizir dit : ne disent ils pas que Dieu est une substance en trois hypothases : Père, Fils et Esprit-Saint ?

    Je dis : Certes, c'est ainsi qu'il disent !
    Il dit : n'acceptent-ils pas le Credo que les 318 (pères de Nicée) ont décrété et mis par écrit ?

    Je dis : Certes, nous l'acceptions et l'exaltons.

    Il dit : Votre affirmation que "Dieu est en trois hypostases. Père et Fils et Esprit Saint", est impiété 
\end{quote}

\paragraph{Bcp de musulmans dans les sanctuaires chrétiens} bcp de femmes voilées à ND du Liban, ...
Bcp de guérisons de musulmans. Le miracle est un acte gratuit, fruit d'une providence universelle. 

\paragraph{forte symbolique et mystique} Question sur l'historicité du fait des symboles. 


\paragraph{Vizir muta'zilite} Un grand débat pour les savants musulmans, ce sont les attributs divins. \mn{\pageref{par:AttributdivinMutazilite}} Pour les mutazilites, les attributs sont inséparables de l'essence. Et du coup, la trinité pensée comme attributs.

\paragraph{Elie de Nisibe} nestorien ? Humanocentrique. 

\begin{Synthesis}
On passe d'une apologique interne à un échange de lettres.
On a en parallèle le dialogue entre philosophes.
Ici, avec Nisibe, on a un dialogue existentiel sur les miracles.
\end{Synthesis}


\paragraph{C'est le monotheisme qui sauve même si vous n'admettez pas la prophétie de Mohammed} On est l'année 1000. C'est vraiment assez sidérant, et une annonciation de Nostra Aetate.

\paragraph{Les trois grands paradigmes qui avancent pour arriver à ce miracle} Cette conversation paisible (7 soirées), 


% --------------------------
\section{Au début des Croisades}

\subsection{Lettre du Moine de France (vers 1076)}

\mn{Cf. A Turki, « La lettre du "moine de France »à Al-Muqtadir billâh,
roi de Saragosse, et la réponse d'Al-BâJ1. le faqîh andalou"
(présentation, tex1e arabe, traduction), in: Al-Andalus.. T.XXXI (1966).
p.73-15336}
\begin{quote}

Au nom de Dieu, le Compatissant, le Miséricordieux. Dieu bénisse notre
Seigneur Mohammed et sa famille .,
    Lettre du moine de France (\textit{Dieu la détruise !})
à
Al-Muqtadir billâh, seigneur de Sarragosse.


Au cher ami qui, nous l'espérons, deviendra un intime très proche, lui à
qui Dieu a conféré le pouvoir du royaume de ce monde, au noble roi, de la part
du plus humble des moines qui désire la repentance et la foi au Christ Jésus, fils de
Dieu, notre Seigneur.
. quand nous est parvenue la nouvelle de ta haute position dans le monde, ô
prince puissant, et celle de ta perspicacité à trouver un sens aux vicissitudes d'ici-bas,
nous avons décidé d'entrer en communication avec toi et de t'inviter à préférer
le Royaume eternel a celui qui passe et prend fin. Tu as déjà lu la lettre que nous
t'avons adressée précédemment, et à laquelle tu as répondu d'une façon que les gens
du monde trouvent honorable mais qui ne correspondait pas à notre désir d'une
réponse spirituelle.
C'est pour cette raison que nous avons tardé à te répondre tant nous avions
peur d'entreprendre un travail épuisant et stérile. Et en vérité, le Tout-Puissant qui
a choisi ses amis des avant la création du monde  et ne les a pas prédestinés à la
destruction\sn{Ep 1,4} a illuminé ton coeur\sn{1Th 5,9 ?} et lui a fait connaître la foi en Dieu qui te sauve, le Compatissant, le Miséricordieux, qui pardonne et te guide vers Sa connaissance. Il ne nous est donc pas possible de tarder davantage à faire tout notre
possible pour mener cette affaire a son terme, avec son aide. pour que tu en viennes
a communier avec nous à son Royaume, si tel est ton choix.


2. Pour cette raison, nous t'avons envoyé quelques-uns de nos frères qui
t'apporteront une Parole de Dieu selon ce que Dieu leur donnera dans ce but. Ils
expliqueront en ta présence la vérité de la religion du Christ. notre Seigneur, le seul
en qui nous devions croire41 et le seul dont nous attendons notre salut42. Il est
Dieu qui s'est fait un voile de notre forme humaine pour nous délivrer de la
perdition du diable par son sang innocent.


3. Sur ce sujet, ô noble roi, nous pourrions parler longuement si nous n'avions pas peur que tu éprouves de la peine en l'écoutant. Il s'agit là de la preuve
de la religion chrétienne et de la démonstration de son éminence bien qu'il soit au-dessus des capacités de la Raison humaine d'en comprendre l'essence. Le Royaume
de Dieu est trop grand et trop glorieux, pour que l'entendement humain puisse le saisir ou l'atteindre par raisonnement theologique. Et pourtant, c'est un des signes
miraculeux du Dieu Tout-Puissant qu'il dilate le coeur des fils d'Adam et y introduit
un esprit de connaissance pour que la Foi s'établisse dans leurs âmes. Et puisque la
terre était autrefois habitée par l'erreur et le monde souillé par l'idolâtrie, le Tout-Puissant
a Juge bon d instaurer, a la fin des âges, une époque nouvelle et de rendre
au monde la rectitude dont il avait été privé par Adam notre premier père.
C'est vers ce but qu'étaient guidés nos pères. bien avant Abraham , Isaac et
Jacob, et c'est cela qu'ont expliqué les prophètes après eux. Voilà la promesse que
Dieu avait faite et confirmée avant et après la révélation de la Loi (Torah), de
révéler la Sainte Incarnation. Ce plan n'est d'ailleurs pas contenu seulement dans
nos Écritures mais aussi dans celles des Juifs et de nos contradicteurs de la façon la
plus claire et la plus évidente.


4. Satan, le maudit, qui, dans sa jalousie pour Adam. avait exposé les gens
de ce monde à la mort, chercha. à corrompre cette sainte religion après la venue des
Apôtres qui gardèrent les habitants de la terre par leur prédication, et après la
victoire remportée sur le démon par les martyrs qui versèrent leur sang pour Dieu,
partout dans le monde, en obéissance à sa Sainte Loi. Comme il ne parvenait pas à
séduire les gens de ce monde pour les ramener à l'antique erreur de l'idolâtrie, il
entreprit de tromper les Fils d'îsmaël à propos du Prophète dont ils reconnurent la
mission conduisant ainsi de nombreuses âmes au châtiment de l'enfer. Dans le
passé, \textit{lblîs} avait péché et égaré l'humanité méritant ainsi de recevoir un sévère
châtiment au Jour où Dieu, notre Seigneur Jésus-Christ. ressuscitera les morts.
Maintenant il a péché de plus belle en causant la perte de ces grandes nations.
\end{quote}


% --------------------------
\section{Rencontres amicales}



 %\chapter{Validation}

\section{instruction}
 
Rédaction d’un travail de 8 pages
Vous faites au préalable une recherche documentaire afin de choisir un article, un chapitre
d’un ouvrage ou un ouvrage dans lequel la christologie est interrogée par la culture
postmoderne et le pluralisme religieux.
Puis vous rédigez votre travail selon deux grandes parties :
\begin{itemize}
    \item  Présentez le document choisi et la manière dont l’auteur pense la confrontation de la
christologie à la culture actuelle.
    \item  Vous réagissez personnellement au texte en le discutant et le critiquant de manière
argumentée.
\end{itemize}

Le texte choisi doit être au préalable validé par le professeur
Merci de suivre les normes universitaires : « Normes de présentations de mémoires »
Votre travail écrit doit être déposé sur l’ENT (espace dédié) et la date limite pour la
remise de votre travail est le 1er mai 2023

POints d'attention : 
\begin{itemize}
    \item Christologie
    \item Récent (Grieu,...)
    \item question Ecologie ?
\end{itemize}


\section{Thèmes possibles}

Habiter ses questions, entre l'individu Jésus et l'universalité. 






\section{Theobald}

 

\section{bibliographie}

\begin{itemize}
    \item RICOEUR, P., Amour et justice, Paris 20082.
   \item RICOEUR, P., Soi-même comme un autre, Paris 1990.
   \item THEOBALD, C., « Jésus n’est pas seul. Ouvertures » dans P. GIBERT – C. THEOBALD, Le cas
Jésus Christ. Exégètes, historiens et théologiens en confrontation, Paris 2002, 381-462.
   \item THEOBALD, C., Le christianisme comme style * et **. Une manière de faire de la théologie en
postmodernité, Paris 2007. \cite{theobald_christianisme_2007} 
   \item THEOBALD, C., Selon l’Esprit de sainteté. Genèse d’une théologie systématique, Paris 2015.
   \item THEOBALD, C., « L’unique et ses témoins : jalons pour une théologie de la rencontre entre
juifs, chrétiens et musulmans », Chemins de dialogue 7 (1996), p. 183-202.
\end{itemize}

\section{Introduction}


\paragraph{Approche libérale (et pluraliste) : partent de l'expérience religieuse} alors que notre foi chrétienne a une bipolarité : à la fois l'expérience de Jésus mais \textit{aussi l'expérience historique de la Résurrection de Jésus} : action de Dieu. D'une certaine façon, Dieu répond à Jésus à travers la Résurrection.

\paragraph{On ne peut donc pas partir de l'histoire} On ne part pas ici de Jésus historique mais on part des récits évangéliques, éclairés par la Résurrection, travaillés par elle.


\paragraph{Theobald : part des textes évangéliques} il ne s'agit pas de relativiser le Christ mais que le Christ suscite le dialogue. Il ne déconstruit pas les textes. Et il intègre l'expérience pascale parce que 

\paragraph{Théobald} Jésuite, traduction des oeuvres de Rahner en français. Oeuvre originale. Deux volumes : le Christianisme comme Style. Il essaye de penser le christianisme dans le contexte actuel.


%------------------------------------------------------------
\section{le contexte et le problème}

%------------------------------------------------------------

\subsection{Vivre ensemble dans une société plurielle}

\paragraph{Nécessité de fonder le lien social non sur le religieux mais sur une instance neutre} Pluralité des religions. "Neutralité" du lien social. Comment fait on société alors que nous sommes différents et de culture différente ? Une vraie actualité. 
\begin{quote}
    L’Église « doit désormais proposer la foi au Christ au sein de démocraties pluralistes, prenant donc la mesure de la laïcité de l’Etat moderne et de sa propre position de groupe social parmi d’autres. Précisons brièvement que le pluralisme des ‘convictions axiales’, religieuses ou non, et la cohabitation de leurs organisations dans une même société supposent la forme agnostique du  ‘lien social’ » (style**, 807).
\end{quote}
\begin{Def}[Lien social]
la forme « agnostique » : il s’agit de la neutralité de l’Etat (laïcité). L’Etat est
neutre pour permettre aux différentes traditions religieuses de vivre ensemble.
Lien social : la forme « énigmatique » : l’unité n’est pas disponible, mais elle se fera à la fin
(eschatologie).
\end{Def}
 
\begin{Def}[position agnostique]
  pas lié à une tradition religieuse  
\end{Def}

Il ne faut pas partir de la dogmatique mais éthique

%------------------------------------------------------------
\subsection{Le christianisme n’est plus le ciment de la société}

\paragraph{Nous ne sommes plus en chrétienté} JP II : appelle de ses voeux la \textit{civilisation de l'amour}.

\begin{quote}
    « Une telle situation appelle, selon nous, un déplacement éthique du point de départ de la christologie (...). Les chrétiens (…) doivent montrer qu’ils reconnaissent le caractère énigmatique du lien social, condition d’un véritable pluralisme, non par concession à des pressions externes mais par intime conviction. Or, cette reconnaissance intérieure exige d’eux une véritable reprise christologique » (style**, 808).
\end{quote}


La vision chrétienne cohabite avec d'autres traditions, laic,n Islam...
\begin{Ex}
Historiquement, racine chrétienne de l'Europe.  Mais ce n'est plus la matrice pour comprendre le monde dans lequel on vit.  
\end{Ex}

\paragraph{Christ Roi} 1925- Pie XI. "christologie glorieuse". Dogmatique qui pose aujourd'hui problème.
%------------------------------------------------------------
 \subsection{Le dilemme christologique}
 

\paragraph{soit on continue à affirmer l'unicité du Christ} mais alors comment le Christ peut rassembler le monde. Facteur de division de la société.

\paragraph{Soit on admet d'autres figures religieuses } et alors on relativise le Christ.

\paragraph{Défi qui n'est pas mince} Quel chemin ?

\paragraph{Théobald : l'unicité du Christ fonde un universalisme pluriel} On peut comprendre l'unicité du Christ autrement. 


%--------------------------------------------------
\section{Réflexions sur le Fils « Unique »}
%--------------------------------------------------

Penser son unité avec Dieu et son lien avec tous les hommes. Si on n'insiste que sur l'unité avec Dieu, quelle fraternité avec les hommes ?

%------------------------------------------------------------
\subsection{Le Fils unique et ses frères}

\paragraph{la préférence entraîne la violence}

\begin{quote}
   « L’histoire de Joseph et de ses frères raconte ce qui arrive quand un père préfère un de ses fils à tous les autres : ceux-ci prennent ‘l’Unique’ en haine, ne pouvant plus lui parler amicalement (Gn 37,3s.) » (\citep[p. 821]{theobald_christianisme_2007}.) 
\end{quote}
 
\paragraph{Comment être fils unique d'un père et avoir en même temps des frères ?} il faut penser les deux
\begin{quote}
    « Comment en effet aimer sans privilégier tel ou telle ? Mais préférer quelqu’un, c’est en même temps risquer que jalousie et violence se lèvent à ses et à nos côtés et que la fraternité soit mise à rude épreuve  […] Comment peut-on être ‘fils unique’ d’un Père et avoir en même temps des frères » (Syle**, 821).
\end{quote}

\begin{Prop}
En théologie, penser ensemble ET... ET...
\end{Prop}


\begin{quote}
    « Il nous faut (…) penser ensemble l’une en fonction de l’autre, l’\textit{unicité} de Jésus de Nazareth et la \textit{relation} qu’il entretient avec les siens et, par extension, avec tout être humain. Cela ne va pas de soi dans une tradition théologique qui a fréquemment distingué, voire séparé ces deux facettes d’une seule et même réalité » (Style**, 822).
\end{quote}


\paragraph{Le long parcours de Joseph préfigure Jésus} Élimination de Joseph (puits) et réconciliation. 
\begin{quote}
    « On ne peut être unique que ‘pour’ quelqu’un : Joseph est d’entrée de jeu unique pour Jacob et destiné à le devenir pour ses frères, eux devant entrer à leur tour dans l’expérience de la fraternité ; ce qui nécessite un long parcours qui prend l’allures d’un drame. Jésus est le Fils unique du Père et appelé à devenir l’Unique non seulement pour ses disciples qui reçoivent le nom de ‘frères’ ou d’ ‘amis’, mais encore pour une multitude » (Style**, 822).
\end{quote}


\begin{Synthesis}
Théobald ne part pas du dogme mais de l'éthique / Texte. Interroge les Ecritures et interroge le devenir historique de Jésus.
\end{Synthesis}


%------------------------------------------------------------
\subsection{L’ambivalence de l’unique dans le NT}

Ici, Théobald reprend la distinction de Stanislas Breton\sn{professeur ICP} : \textit{unicité de singularité} (chaque personne est unique) et \textit{unicité d'excellence}(attribué à Jésus dans l'Espace et le Temps) et en ce sens, touche tous les hommes.
\begin{Def}[Unicité de singularité]
 L’unicité de singularité renvoie au caractère unique de toute
personne, à sa singularité : « tout être humain est ‘unique’, au sens où toute rencontre d’autrui
doit surmonter le réflexe de comparaison et aboutir au respect de ce qu’il a d’unique »
\end{Def}
\begin{Def}[    Unicité d’excellence]
  « Le croyant ne doit-il pas accorder à Jésus, outre l’unicité de
singularité qu’il partage avec tout être humain, une ‘excellence’ dans l’espace et le temps ? »
\end{Def}

 




\paragraph{une unicité d'excellence en relation avec les autres} Jésus rentre en relation avec les autres. Théobald part de S. Jean. Tout d'abord le fils unique : 
\begin{Def}[Unicité du Christ]
    cette expression vise à souligner que Jésus est l’unique sauveur de tous
les hommes car il est le « Fils unique » de Dieu. Theobald, néanmoins, développe l’idée que
l’unicité du Christ passe par la reconnaissance de cette unicité par les hommes. L’unicité n’est
pas seulement une qualité métaphysique, mais elle implique une relation réciproque.
\end{Def}
 
\textit{monos} signifie en grec unique mais aussi \textit{seul}. 

\paragraph{Fécondité de l'unique} 
\begin{quote}
    Jn 12,24 : « Le grain de blé tombé en terre, s’il ne meurt pas il reste tout seul (\textit{monos}), vous dis-je, mais s’il meurt il donne beaucoup d’épis ». 
\end{quote}

\begin{quote}
    « La solution de l’ambivalence fondamentale, attachée à toute unicité – lieu où se loge l’ultime tentation de tout homme - , est donc le don de soi, mort du grain pour porter du fruit (…). L’unique n’est donc pas vraiment l’unique, au sens d’une unicité d’excellence, que s’il donne sa vie pour une multitude ; s’il donne sa vie pour que chacun puisse accéder à sa propre unicité » (Le cas, 454).
\end{quote}
Idée que l'unicité de Jésus est une unicité relationnelle : il donne sa vie pour que les autres aient aussi la vie. 

On voit ici comment Théobald relit et reinterprête les écritures avec un regard neuf. 

\begin{Prop}
    Quand on pratique la générosité du samaritain, on participe à l'unicité du Christ. 
\end{Prop}


%------------------------------------------------------------
\subsection{La sainteté et unicité}

Théobald appelle cette singularité, \textit{sainteté}.




\paragraph{Figure du Samaritain et de la démesure qui va jusqu'à aimer sur la croix}
\begin{quote}
    « Celui qui met en jeu son existence au profit du blessé, le Samaritain, devient unique, non seulement pour l’homme rencontré par hasard sur le chemin, mais aussi à ses propres yeux : la démesure de son geste qui défie toute obligation légale s’est avérée à sa mesure, mesure incomparable à celle du voisin » (style**, 827).
\end{quote}

Figure de Jésus. Il y a quelque chose d'excessif. 
Cela est marqué au moment du refus et du rejet qu'il revele qu'il est fils unique : \textit{aimez vos ennemis}. Et cela culmine sur la mort sur la croix. 


%------------------------------------------------------------
\subsection{L’unicité de Jésus et sa manière unique de communiquer la sainteté}


 Ce n'est pas la sainteté qui fait la différence car tout homme est appelé à la \textit{vision béatifique.} Ce qui est unique, c'est qu'il communique de façon unique et définitive de la Sainteté. 
 La sainteté est communiquée \textit{une fois pour toute}\sn{Hebreux}

\begin{quote}
    « Ce n’est donc pas la grâce qui fait la différence entre l’Unique et ses frères humains, ni la sainteté (...). La différence entre Lui et nous consiste seulement dans le fait que c’est Lui, le Fils unique, qui nous communique la sainteté et que c’est en Lui que l’unique promesse de Dieu de nous donner ce qu’il est en Lui-même est devenue dans  notre histoire réalité ‘irrévocable’ » (Style**, 829).
\end{quote}

Cf la distinction que faisaient les Pères de l'Eglise de l'Union hypostatique (en Jésus, les personnes divine et humaine) et l'invitation de Dieu dans les Saints.


\paragraph{Irrévocable} Interprétation de la mort du Christ, sceau de la parole du Christ. 

\paragraph{Comment les approches christologiques présentes doivent être complétées}
%------------------------------------------------------------
\section{Les insuffisances des approches christologiques récentes}
%------------------------------------------------------------

Une théologie, c'est aider les chrétiens à vivre dans un certain contexte. 

\begin{itemize}
    \item Isoler Jésus de ses frères. Seconde Quête
    \item les théologies inclusivistes (Danielou, Rahner), qui insistent sur l'aspect culturel.
\end{itemize}
%------------------------------------------------------------
\subsection{L’unicité de Jésus fondée sur sa relation à Dieu}

\paragraph{Seconde Quête} Käsemann, ... Les disciples de Bultmann qui cherchent le Jésus historique : montrer que Jésus était le seul de son espèce. Montrer que le Jésus de l'histoire et le Christ de la Foi, il y a une continuité. C'était une réponse à Reimarus en montrant qu'en étant historien, il y avait une christologie implicite dans le Jésus historique. 

Techniquement, on avait des critères d'historicité \mn{ex : Jésus discutant avec les pécheurs : on ne faisait pas cela avant ni après}. "Mon Père et votre Père" : distingue bien. "on vous a dit, moi je vous dis" : tous ces motifs qui vont isoler Jésus des autres.

\paragraph{Inconciemment} ces théologies ont mis l'unicité théologique de Jésus au détriment de sa relation avec les hommes.



%------------------------------------------------------------
\subsection{L’unicité de Jésus fondée sur sa dimension eschatologique}

\paragraph{Jésus, un avec les hommes} Rejoint le projet de Théobald de montrer le Christ en Relation. On retrouver cela en G\&S 22 (\textit{Le Christ, homme nouveau})
\begin{quote}
1. En réalité, le mystère de l’homme ne s’éclaire vraiment que dans le mystère du Verbe incarné. Adam, en effet, le premier homme, était la figure de celui qui devait venir, le Christ Seigneur. Nouvel Adam, le Christ, dans la révélation même du mystère du Père et de son amour, manifeste pleinement l’homme à lui-même et lui découvre la sublimité de sa vocation. Il n’est donc pas surprenant que les vérités ci-dessus trouvent en lui leur source et atteignent en lui leur point culminant.
2. « Image du Dieu invisible » (Col 1, 15) , il est l’Homme parfait qui a restauré dans la descendance d’Adam la ressemblance divine, altérée dès le premier péché. Parce qu’en lui la nature humaine a été assumée, non absorbée , par le fait même, cette nature a été élevée en nous aussi à une dignité sans égale. Car, par son incarnation, le Fils de Dieu s’est en quelque sorte uni lui-même à tout homme. Il a travaillé avec des mains d’homme, il a pensé avec une intelligence d’homme, il a agi avec une volonté d’homme, il a aimé avec un cœur d’homme. Né de la Vierge Marie, il est vraiment devenu l’un de nous, en tout semblable à nous, hormis le péché.
3. Agneau innocent, par son sang librement répandu, il nous a mérité la vie ; et, en lui, Dieu nous a réconciliés avec lui-même et entre nous, nous arrachant à l’esclavage du diable et du péché. En sorte que chacun de nous peut dire avec l’Apôtre : le Fils de Dieu « m’a aimé et il s’est livré lui-même pour moi » (Ga 2, 20). En souffrant pour nous, il ne nous a pas simplement donné l’exemple, afin que nous marchions sur ses pas, mais il a ouvert une route nouvelle : si nous la suivons, la vie et la mort deviennent saintes et acquièrent un sens nouveau.
4. Devenu conforme à l’image du Fils, premier-né d’une multitude de frères, le chrétien reçoit « les prémices de l’Esprit » (Rm 8, 23), qui le rendent capable d’accomplir la loi nouvelle de l’amour. Par cet Esprit, « gage de l’héritage » (Ep 1, 14), c’est tout l’homme qui est intérieurement renouvelé, dans l’attente de « la rédemption du corps » (Rm 8, 23) : « Si l’Esprit de celui qui a ressuscité Jésus d’entre les morts demeure en vous, celui qui a ressuscité Jésus Christ d’entre les morts donnera aussi la vie à vos corps mortels, par son Esprit qui habite en vous (Rm 8, 11) [36]. Certes, pour un chrétien, c’est une nécessité et un devoir de combattre le mal au prix de nombreuses tribulations et de subir la mort. Mais, associé au mystère pascal, devenant conforme au Christ dans la mort, fortifié par l’espérance, il va au-devant de la résurrection.
5. Et cela ne vaut pas seulement pour ceux qui croient au Christ, mais bien pour tous les hommes de bonne volonté, dans le cœur desquels, invisiblement, agit la grâce. En effet, puisque le Christ est mort pour tous [39] et que la vocation dernière de l’homme est réellement unique, à savoir divine, nous devons tenir que l’Esprit Saint offre à tous, d’une façon que Dieu connaît, la possibilité d’être associé au mystère pascal.
6. Telle est la qualité et la grandeur du mystère de l’homme, ce mystère que la Révélation chrétienne fait briller aux yeux des croyants. C’est donc par le Christ et dans le Christ que s’éclaire l’énigme de la douleur et de la mort qui, hors de son Évangile, nous écrase. Le Christ est ressuscité ; par sa mort, il a vaincu la mort, et il nous a abondamment donné la vie pour que, devenus fils dans le Fils, nous clamions dans l’Esprit : Abba, Père!
\end{quote}

\paragraph{Une vision une de l'anthropologie} Cela suppose que la vision du monde des Chinois est la même que la vision des Européens. Reste tributaire de la vision universelle des lumières. Il faut parler de plusieurs \textit{universalismes} ou visions du monde : donner un but, une vision du monde. Cela ne veut pas dire que le Concile Vatican II est dépassé : il applique le Concile non pas à la lettre mais comme méthode : \textit{signe des temps } à lire dans la culture actuelle.

Tout en reconnaissant l'intérêt de ces théologies, elles ne peuvent servir de \textit{point de départ}. Partir de l'éthique, pratique, \textit{plus modeste}.

\paragraph{Théologie pratique} qui n'exclue pas la vision théologique ou anthropologique. 


%------------------------------------------------------------
\section{Commencer par une christologie pratique ou éthique}
%------------------------------------------------------------

%------------------------------------------------------------
\subsection{Une nouvelle approche de la christologie}

\paragraph{Normalement, on part de la figure divine de Jésus} Lui propose une autre approche. Ne prennent pas en compte le \textit{chemin } des disciples pour arriver à la Foi, l'unicité relationnelle de Jésus, comme les frères de Joseph. \textit{je suis le Chemin}. Intégrer la genèse de la Foi dans la Christologie même. 
\paragraph{Evangile}
Il s'appuie sur les Evangiles, qui font partie de la Révélation.
Les récits ne nous disent pas qui est Jésus mais ont une pédagogie pour nous faire entrer dans la Foi. \mn{Théobald est Jésuite et les Evangiles ont un goût, histoire}

Il s'agit de faire la même expérience que les disciples :
    
\begin{quote}
    « Le ‘Jésus des historiens’ n’a pas l’actualité et l’autorité nécessaire pour réclamer la foi, c’est-à-dire pour que soient recrées les conditions de la décision jadis appelée par l’évangéliste sur la vérité de ce qu’il appelait la vie ou le royaume » (Le cas 442). Il y a donc une « impossibilité et plus encore une illégitimité de toute communication de la foi par l’histoire » (le Cas 445).
\end{quote}

En lisant Renan, on ne découvre pas la Foi. L'histoire n'est pas là pour accéder à la Foi. Cela peut nous aider à un moment donné. 
\begin{quote}
     Mais ceux-là ont été écrits pour que vous croyiez que Jésus est le Christ, le Fils de Dieu, et pour qu’en croyant, vous ayez la vie en son nom.
    Jn 20,31
\end{quote}


\paragraph{Rencontre de Jésus}
les textes bibliques, pas des textes, mais la parole de Dieu qui nous féconde.
\begin{quote}
    « [La nouveauté du christianisme] ne se réduit pas à l’identité christologique du Nazaréen mais se concentre dans le type de relation qu’il entretient avec ceux qui croisent sa route » (Style*, 57). 
\end{quote}

\paragraph{configuration du récit } Le texte a un effet sur le lecteur : 
\begin{Def}[Configuration]
« Dynamisme intégrateur qui tire une histoire une et complète d’un divers
d’incidents, autant dire transforme ce divers en une histoire une et complète » (Ricoeur,
Temps et Récit II, p. 18). Ici il s’agit de la mise en intrigue, laquelle appartient à la
configuration.
\end{Def}

\begin{Def}[Refiguration]
 
    « J’appelle refiguration l’effet de découverte et de transformation exercé par
le discours sur son auditeur ou son lecteur dans le processus de réception du texte » (Ricoeur,
Amour et justice, 50).
 
\end{Def}
\begin{quote}
    « Comment le soi se comprend-il en se contemplant dans le miroir que lui tend le livre ? » (Ricoeur, Amour et justice, 51). 
\end{quote}
En méditant les Evangiles, on rencontre le Christ. \mn{Arrière fond de Théobald} 



%------------------------------------------------------------
\subsection{Jésus a « impressionné » ses disciples}

\paragraph{Rencontre du Christ : impact, impressionne} La Théologie libérale a insisté sur cette partie. Jésus les a impressionné fortement car il vivait intensément cette relation \mn{Schleiermacher p. XX}

\paragraph{salut lié à la christologie} c'est parce que Jésus est en lien avec tous les hommes qu'il sauve. Et je suis poussé à agir comme lui. Théobald reprend Schleiermacher mais reprend l'Evangile.

%------------------------------------------------------------
\subsection{Le style de vie messianique de Jésus}


\begin{quote}
    Comment caractériser « l’accès à la foi au Christ, compte tenu du point de départ éthique de la christologie, appelé par nos sociétés néo-libérales toujours tentées d’oublier les menaces qui continuent à peser sur le lien social ? On peut, dans la perspective d’une christologie pratique, le définir comme ‘mue d’identité’ qui s’exprime par le passage à un\textit{ style de vie messianique}, à une manière spécifique de se situer dans la société globale » (style**, 813).
\end{quote}

\begin{Def}[Style de vie messianique]
Ce style de vie consiste à vivre l’hospitalité qui suscite la mue
d’identité des personnes rencontrées (par Jésus). « La foi chrétienne n’est pas une doctrine
(…) mais un ‘style de vie’ ou une manière de vivre de la sainteté même de Dieu : seule
l’expérience effective de l’Esprit de sainteté nous permet de confesser et de comprendre un
jour l’indépassable excellence du Fils unique du Père » (L’unique et ses témoins, 199).
\end{Def}

\begin{Def}[Mue d’identité]
Le fait que l’autre puisse se découvrir et accéder à son identité singulière.

\end{Def}

\begin{quote}
    « Les évangiles sont en fait des récits de conversion qui ne mettent pas seulement en scène l’itinéraire de Jésus mais aussi et surtout ce qu’il devient en et pour ceux et celles dont l’itinéraire croise le sien : l’accès au Christ se vit comme véritable \textit{mue d’identité} » (style**, 805). 
\end{quote}



\begin{quote}
    Elle engendre l’autre à son identité propre (conversion ?). L’identité d’une personne coïncide
avec l’émergence de la foi (« Ma fille, ta foi t’a sauvée »Mc 5,34), l’autre vit à partir de cette
foi (foi en Dieu), il devient lui-même en se décentrant. Cette conversion consiste à « suivre
Jésus », c’est-à-dire à adopter chacun à sa manière le style de vie messianique\sn{Le regard de Jésus n’est pas évident. « L’adopter effectivement est de l’ordre d’une véritable conversion ou
nécessite une inversion. L’oeuvre messianique consiste précisément dans la victoire sur cet aveuglement » (Style
73). L’incompréhension des disciples montrent bien la difficulté d’adopter l’hospitalité de Jésus (posture
d’apprentissage et dessaisissement de soi). On doit « pouvoir atteindre effectivement l’absence de mensonge ou
la concordance absolue entre pensées, paroles et actes, entre la ‘forme de vie’ de quiconque et son ‘fond’,
concordance, qui, par principe, est chaque fois unique et incomparable » (style 75).}. Jésus devient
l’unique à leurs yeux : c’est-à-dire celui qui leur a communiqué la sainteté. 
\end{quote}

\paragraph{6 étapes}
\begin{itemize}
    \item Apprentissage de Jésus (He 5, 8) dans ses rencontres, il vit la sainteté
    \begin{Def}[Sainteté]
 Capacité d’apprentissage ou dessaisissement de soi au profit d’une présence à
quiconque, ici et maintenant (voir Mc 8,35).
\end{Def}
\item Jésus ne s'impose pas mais permet à l'autre d'accéder à lui -même. Il va libérer l'autre de ses craintes et se dire. 
\item l'émergence de la Foi : quand on accède à son identité, elle commence à être sauvé. Foi en Dieu. cf la Femme adultère, \textit{Ta Foi t'a sauvé}. de l'hospitalité de Jésus arrive la Foi. Il prend des exemples dans l'Evangile : elle libère les personnes rencontrées et les ouvre à une vie nouvelle
\item refiguration, mue d'identité : conversion des personnes qui croisent Jésus, ces gens qui ont changé de vie en rencontrant Jésus. Chacun à adopter le style de vie messianique. Nous avons à devenir hospitalier et leur permettre d'advenir à leur identité.
\item les personnes qui accèdent à cette vie nouvelle \sn{le possédé à Gerasa reste et ne devient pas disciple de Jésus} ne sont pas forcément tous disciples. Mais ces rencontres ne sont pas forcément un changement de vie. Refus. Comment alors maintenir le lien social quand il y a refus de la rencontre ? 
Théobald introduit la règle d'Or.
\end{itemize}
 
\begin{Def}[Concept d’hospitalité ]
Quand l’hospitalité se produit, c’est l’accomplissement des temps
messianiques. Aussi, dans cette expérience de la rencontre avec Jésus, de l’hospitalité de
Jésus est engendrée la foi, le royaume de Dieu advient. « Les aveugles voient ». C’est
pourquoi on peut qualifier de « messianique » le style de vie hospitalier de Jésus car il libère
les gens rencontrés et les engendre à la foi pour une vie nouvelle.
\end{Def}

\begin{Def}[Engendrement]
Il s’agit du devenir « fils de votre Père ». Il a lieu quand on « réalise tout
d’un coup que l’appel démesuré à être connu comme Dieu, dans telle ou telle situation, est
toujours ‘à la mesure’ de chacun » (L’Unique et ses témoins, 200).
Intercommunication : une autre manière de dire le dialogue avec cette nuance où l’on
communique à l’autre sa sainteté.
\end{Def}
%------------------------------------------------------------
\subsection{Le style de vie de Jésus à l’épreuve du refus}

Jésus suscite l'opposition qui l'amènera à la croix. La notion de réciprocité se métamorphose en règle d'amour, de la \textit{mesure} à la \textit{démesure}. \mn{Jn 10, 20 : il déraisonne. }
\begin{quote}
    « Sans le correctif du commandement d’amour (…) la Règle d’Or serait sans cesse tirée dans le sens d’une maxime utilitaire dont la formule serait do ut des, je donne pour que tu donnes. La règle : donne parce qu’il t’a été donné, corrige le afin que de la maxime utilitaire et sauve la Règle d’Or d’une interprétation perverse toujours possible » (Ricoeur, Amour et justice, 39). 
\end{quote}


\begin{Def}[Règle d’or]
 \begin{quote}
     « Tout ce que vous désirez que les autres fassent pour vous, faites-le vous-mêmes
pour eux : voilà la Loi et les Prophètes » (Mt 7,12). 
 \end{quote} Avec Jésus, on passe de la justice
(relation de réciprocité) à l’amour (démesure : l’amour des ennemis) et de l’amour à l’amour
définitif. Jésus accomplit donc la Loi et les Prophètes en accomplissant la règle d’or : il la
réalise définitivement en aimant ses ennemis.
\end{Def}
Mt 7,12 : règle d'or
\begin{quote}
    
\end{quote}

La règle d'or existe dans d'autres cultures \mn{cf  \cite{kung_lethique_2009}}


Paul Ricoeur : sans 
\begin{quote}
    Do ut des : je donne pour que tu donnes. La règle \textit{donne parce qu'il t'a été donné} Sauve la règle d'or de l'utilitarisme (Ricoeur)
\end{quote}


%------------------------------------------------------------
\subsection{La dimension eschatologique du style de vie de Jésus}

\begin{Def}[Style de vie eschatologique]
« Sa sainteté hospitalière [va] jusque dans son ultime
dessaisissement de soi » (style*, 91). L’absence de mesure ou la démesure fait entrer
l’incomparable de Dieu dans l’histoire (voir style*, 95). « Vous serez parfaits comme votre
Père céleste est parfait » (Mt 5,48). En donnant sa vie, son style de vie est donc définitif,
eschatologique. Sa mort ne dit pas seulement l’une fois pour toutes, mais aussi qu’il
transcende le temps et l’espace en s’identifiant à toutes les victimes comme Mt 25 les
présente.
\end{Def}

 En quoi une dimension eschatologique : 


 \begin{quote}
     « Comment la manière d’être du Nazaréen – son type d’hospitalité absolument unique – a pu engendrer, non seulement la confession messianique des premiers chrétiens, mais encore leur perception du caractère définitif et ultime de ce qui est advenu dans leur rencontre avec lui » (Style 86).
 \end{quote}

C'est la mort de Jésus manifestant la perfection de sa sainteté, dessaisi de lui-même, en partant, pour laisser la place même à l'ennemi.
\begin{quote}
    domine jusqu'au coeur des ennemis
\end{quote}

 


\begin{quote}
    « Sa sainteté hospitalière [va] jusque dans son ultime dessaisissement de soi » (style*, 91). L’absence de mesure ou la démesure fait entrer l’incomparable de Dieu dans l’histoire (voir style*, 95). « Vous serez parfaits comme votre Père céleste est parfait » (Mt 5,48).
\end{quote}

\paragraph{il est lui-même le saint} il brise le refus, Temps et Espace. Il donne à celui qui ne peut pas rendre. Quand on donne à un pauvre, il ne peut pas rendre.


\section{Contexte}



%------------------------------------------------------------
\section{Conclusion}


Jésus n'empêche pas le dialogue mais le favorise. 
Style de vie hospitalier, se dessaisissant pour laisser l'autre advenir à lui-même. 


\begin{Synthesis}
    Une approche éthique, pragmatique partant de Jésus. Approche originale et adaptée à notre monde.
\end{Synthesis}


 


\section{L'unique et ses témoins }

\cite{theobald_christianisme_2007} Jalons pour une théologie de la rencontre entre juifs, chrétiens et musulmans

\paragraph{unicité de Dieu et le témoignage qui lui erst dû auprès des humains et à leur service : foi partagée}

Jésus : Mt 22,34-40
Jn 18,37
Ap 3,14 monothéisme trinitaire des chrétiens


\paragraph{différences entre les attestations} apparition successive dans l'histoire 

\paragraph{violence entre les trois témoins} \textit{liens inextricables} peut être l'origine de la violence (mimétique au sens de René Girard). 
\begin{quote}
    il nous faudra affronter l'énigme de la violence pour voir où se trouve la véritable difficulté à communiquer entre témpoins. 
\end{quote}
\paragraph{Possibilités et limites des rencontres } \textit{qu'as tu appris des deux autres et de la société sur toi-même ? } Aucun des témloins ne peut répondre à la place des deux autres.
Réflexion proprement chrétienne sur la \textit{rencontre} en méditant sutr la figure de Melchisedech, \textit{roi de Gloire} (Ps 23).

\subsection{Difficile communication}

\paragraph{pour éviter la violence, séparation dans le privé} Lumières sépare privé et le public.  \textit{Armistice. } 

\paragraph{la société moderne impose ses règles du jeu à toute rencontre} cf Habermas et Ratzinger à Munich 19 /01/2004.  Esprit 2004/07

\subsubsection{regards croisées dans la famille d'Abraham}

\paragraph{judaisme monotheisme éthique} ne reconnaît pas christianisme et islam comme ses héritiers (cf levinas). Identité propre :
\begin{quote}
    Aimez l'étranger, car au pays d'Egypte, vous fûtes des étrangers Dt 10,17-19
\end{quote}
imitant son Dieu par la justice faite à autrui.


pas besoin du Christianisme ou Islam mais confrontation a un impact. Indépendance mais aussi fragilité vis à vis d'eux. "mystère" du chant du serviteur.  

\paragraph{Christianisme, réalité méta-éthique du don de Soi  } \sn{besoin du Judaisme pour se comprendre "accomplissement de la Loi et des Prophètes" en JC.  Rm 11,18 réalité inouie et excecessive : l'unique Dieu est censé communiquer à la multitude la sainteté qui le constitue en lui-même, et tous peuvent désormais}
2. Le christianisme se situe entre l'aîné et le dernier. Il a d'abord besoin du judaïsme pour se comprendre lui-même parce qu'il ne peut ni raconter ni vivre « l'accomplissement de la Loi et des Prophètes » en Jésus le Christ sans se référer continuellement à la racine sur laquelle il a été greffé (Rm 11, 18) Selon la tradition chrétienne, la foi s'ouvre à une réalité inouie et excessive: l'unique Dieu est censé communiquer à la multitude la sainteté qui le constitue en lui-même, et tous peuvent désormais découvrir par la foi à quel point cette sainteté le habite déjà:
\begin{quote}
    « Vous serez parfaits comme votre Père céleste est parfait » (Mt 5, 48). 
\end{quote}
C'est cela « l'accomplissement de Loi et des Prophètes», vécu dans les gestes les plus quotidien Si on se réfère au vocabulaire de l'éthique, on peut donc appel le christianisme un monothéisme méta-éthique .
\begin{Def}[monothéisme méta-éthique]
    au sens il insiste sur la communication de l'agapé divine, de l'amour surabondant de Dieu à tout être humain
\end{Def}
 
Mais comment se réclamer de cette réalité méta-éthique du don de soi qui dépasse toutes nos mesures humaines ses reconnaître d'abord l'entière autonomie ou consistance de l'ordre éthique de la justice et du respect d'autrui? La fragilité du christianisme ne vient donc pas de sa dépendance par rapport à un autre qui le précède et qui existe à ses côtés: mais elle vient de la tentation d'indépendance qui l'a amené à se substituer à Israël, à se considérer comme le « véritable Israël »; et cela en dépit de l'avertissement paulinien dans l'épître aux Romains qui prévient l'Église contre l'orgueil: les chrétiens risquent d'oublier qu'ils tiennent, grâce à la foi sur une racine qui les porte (Rm 11, 20).

\sn{785}
Peut-être cet oubli n'a-t-il pas été sans influence sur la naissance de l'islam, comme le pensent certains théologiens chrétiens!. Par rapport aux musulmans, l'Église se trouve en tout cas dans une position analogue à celle que le judaïsme occupe par rapport à elle: elle n'a pas besoin de comprendre l'islam pour se comprendre elle-même. Elle a subi l'insensibilité du nouveau venu sur la scène religieuse qu'elle-même a montrée vis-à-vis de l'aîné qui la précède et qui existe à côté d'elle; affrontement d'autant plus violent qu'il oppose deux manières de concevoir « l'accomplissement ». Il faudra attendre le xie siècle2 pour que le christianisme commence à abandonner la répartition traditionnelle de l'humanité entre juifs, païens et chrétiens qui l'avait amené à identifier la foi musulmane à un vague paganisme monothéiste.

\paragraph{Islam, monotheisme pre-éthique}
3. L'islam enfin, le dernier-né des trois, a besoin du judaisme et du christianisme, des « gens du Livre » comme il dit, pour se comprendre lui-même. C'est sur leur trace qu'il affirme le caractère ultime de sa révélation; après les quatre grands prophètes « doués de constance », Noé, Abraham, Moise et Jésus, Muhammad est le dernier, le « sceau des prophètes » (sourate 33, 40), qui met fin aux intervalles entre les époques prophétiques et fixe la communauté des fidèles dans l'attente de l'Heure dernière. Cette dépendance avouée est la force de l'islam et en même temps le lieu de sa fragilité propre.
Force d'abord, parce que sa critique des juifs et des chrétiens se met plutôt en dépendance par rapport à une alliance (mithãq) dite de « pré-éternité », qui précède toute division historique entre judaïsme, christianisme et islam. Il n'y a donc pas de progrès historique dans la révélation, mais rappel ultime et définitif de ce qui a été oublié ou déformé: l'unicité absolue de Dieu, menacée par le polythéisme et tout ce qui lui ressemble, comme l'association du Christ à Dien Ce retour en deçà de l'histoire et de ses divisions, vers l'origine adamique du « pacte » de « pré-éternité », qui d'emblée fait de tout homme un «croyant», fonde l'universalité de l'islam. Celle-ci n'est plus fondée, comme dans le prophétisme juif, sur la qualité éthique de la « relation » de l'Unique avec son témoin et avec l'étranger; \textbf{elle s'appuie sur l'idée que tout homme porte à sa naissance, sceau imprimé par Dieu en son cœur, la proclamation de foi de la pré-éternité}: cette « religion naturelle », liée à la création comme première révélation d'en deçà des temps, est une prédisposition à recevoir l'islam.
Mais cette force qui permet de contourner sans cesse les divisions historiques en les relativisant par la référence obligée à une origine immémoriale constitue en même temps la \textit{fragilité} propre de l'islam: 
\begin{Def}[monothéisme pré-éthique]
\textit{sa lutte pour l'unicité de Dieu précède toute préoccupation éthique}. 
\end{Def}
C'est en ce sens qu'on peut appeler son monothéisme pré-éthique. Il reste, de ce fait, soumis aux interrogations éthiques qui ne peuvent pas ne pas émerger de la rencontre effective entre les trois témoins.
Cette rencontre s'avère donc comme extraordinairement difficile à cause de l'asymétrie entre les trois traditions: si les deux premiers, juifs et chrétiens, se retrouvent, au moins selon la perspective chrétienne, dans une proximité privilé
 
Cette rencontre s'avère donc comme extraordinairement difficile à cause de l'asymétrie entre les trois traditions si les deux premiers, juifs et chrétiens, se retrouvent, au moins selon la perspective chrétienne, dans une proximité privilégiée, ils ne peuvent pas pour autant évacuer l'énigmatique présence du troisième : Ismaël, fils d'Abraham (Ibrähim) et de Hâgar, ancêtre du peuple arabe, qui réclame sa « parenté» avec le premier, prototype de la foi en l'Unique.

\subsection{Rencontre et comparaison.}

\paragraph{pas de facilité à communiquer entre les 3 témoins par la modernité du fait du \textit{comparatisme} et une distance critique qui atteint chacun des trois religions}
La modernité occidentale a-t-elle facilité la communication entre les trois témoins? Ce n'est pas du tout sûr, El nous a appris cependant à « comparer » les figures du Dieu unique, en nous permettant ainsi de prendre une certaine distance par rapport à l'ensemble des trois traditions.
Nous savons qu'il est impossible de rencontrer l'as sans se comparer à lui; et puisque des rencontres entre juit chrétiens et musulmans ont existé depuis les débuts, les compe raisons entre différents interlocuteurs n'ont pas non pis manqué. Mais la communication généralisée qui caractérise as sociétés modernes transforme la comparaison en principe intellectuel. L'homme contemporain ne cesse de comparer ce qui lui est proposé, et cela jusque dans le domaine religieux. Le «comparatisme » systématique des sciences de la religion  constitue donc la face intellectuelle d'une société démocratique qui, pour éviter toute violence religieuse en son sein, n'accorde plus de privilèges à aucun des trois témoins mais se fonde désormais sur une conception a-religieuse ou a-gnostique du «lien social » qui la constitue. La distinction entre le public et le privé dans nos États laïcs s'appuie sur cette prise de distance critique par rapport à chacun des trois; ce qui explique pourquoi elle les atteint dans leur propre identité.
\paragraph{comparatisme pour éviter les violences ? pas sûr pour l'islam}
Il est sûr que la crainte de la violence religieuse a conduit tout au long du xixe et du xxe siècle vers un «comparatisme» critique de toute religion, critique en particulier du mono-théisme: dans leur commune obsession de «l'Unique», juifs, chrétiens et musulmans n'auraient cessé de lutter contre tout ce qui est pluriel, poursuivant, chacun à sa façon, le mirage d'une culture unifiée qui exclut ce qui est différent. Cette critique fait peu de cas des identités propres de chaque figure. 
Et comme elle avait produit déjà au siècle dernier des réactions très violentes de la part du christianisme et du judaïsme officiels, elle suscite aujourd'hui des nouvelles violences et des soubresauts identitaires de la part de l'islam. \sn{cf Ratisbonne}
\paragraph{identité propre entre les 3 témoins : on ne peut simplifier en disant qu'ils cherchent à éliminer le pluriel}Nous sommes là devant une alternative intellectuelle et a arque en particulier du monothéisme: dans leur commune obsession de « l'Unique», juifs, chrétiens et musulmans n'auraient cessé de lutter contre tout ce qui est pluriel, poursuivant, chacun à sa façon, le mirage d'une culture unifiée qui exclut ce qui est différent. Cette critique fait peu de cas des identités propres de chaque figure.
Et comme elle avait produit déjà au siècle dernier des réactions très violentes de la part du christianisme et du judaïsme officiels, elle suscite aujourd'hui des nouvelles violences et des soubresauts identitaires de la part de l'islam.
Nous sommes là devant une alternative intellectuelle et spirituelle tout à fait décisive pour nos sociétés modernes.
Comparer les trois « monothéismes » comme je viens de le faire (sans pouvoir d'ailleurs entrer dans les détails), cela conduit-il nécessairement à aplatir les différences et à produire de nouvelles violences ? Ou peut-on espérer que le «comparatisme » réussisse à mettre en valeur le mystère de nos identités? 

\paragraph{comparatisme pour mettre en valeur nos identités, diversité et pour se situer dans la société avec nos propres styles mais en fouillant dans nos ressources. }
\begin{Synthesis}
Mon pari est qu'il peut faire paraître avec une acuité toujours plus grande la diversité extraordinaire de nos manières de nous situer dans la vie commune et par rapport au « lien social ».     
\end{Synthesis}
À condition cependant que la communication entre ces différents \textbf{« styles »}, inévitable dans la société moderne, provoque non pas une fermeture définitive de certains mais suscite en chacun une véritable auto-interrogation2, un retour \sn{788} réflexif sur soi pour découvrir dans son propre patrimoine des ressources jusqu'alors inaperçues, permettant d'affronter le nouveau pluralisme radical.
Ainsi compris le « comparatisme » renvoie chaque tradition au niveau le plus profond de sa propre identité, à sa foi et à l'exercice (parfois autocritique) d'un retour sur soi. Chacun des trois est convié au jeu difficile d'une communication qui consiste désormais à conjuguer le regard interne à sa foi sur les deux autres traditions et la perspective externe des deux autres sur lui; exigence de communication déjà présente dans la célèbre Règle d'or: \begin{quote}
    «Tout ce que vous voulez que les autres fassent pour vous, faites-le pour eux ! » 
\end{quote}Certes, cette règle de réciprocité est au cœur de la préoccupation éthique du judaïsme; mais elle traverse aisément les frontières entre traditions parce qu'elle existe en toute culture. À ce titre, elle se trouve aujourd'hui au fondement de nos sociétés démocratiques\sn{Ricoeur règle d'or}.

\subsection{Savoir apprendre de l'autre}

\paragraph{différence avec la rencontre de la société : la société poblige à des règles de communication, la rencontre oblige à un processus d'apprentissage}
Quand nos sociétés démocratiques imposent aux trois témoins certaines règles de communication, la rencontre des deux autres invite chacun à entrer dans un long processus d'apprentissage. Peut-être l'histoire de la modernité leur fait elle-même d'abord comprendre que, loin de les éloigner de leur propre tradition, cette capacité d'apprendre constitue depuis toujours l'identité la plus profonde du témoin.

\subsubsection{L'interrogation prophétique.}

\paragraph{prophétisme, figure de l'apprentissage "à se mettre à la place d'autrui"}
En effet, cette disponibilité à se laisser enseigner par autrui n'a pas été simplement imposée de l'extérieur aux traditions monothéistes; elle est née en leur sein sous la figure du prophétisme. Dire que Dieu est unique suppose déjà une prise de conscience de haut niveau. Mais que la tradition juive découvre un jour que son Dieu ne fait acception de personne et qu'il désire avoir un témoin comme lui généreux envers tout homme, cela suppose un apprentissage éthique d'un tout autre ordre encore : la capacité du juste à « se mettre à la place d'autrui», qui lui vient du souvenir d'avoir déjà occupé cette position: « Aimez l'étranger, car au pays d'Égypte vous fûtes des étrangers » (Dt 10, 17-19).
\paragraph{Jésus, Grand apprenant (He) de la rencontre avec l'autre, de ses souffrance vers l'accomplissement}
Jésus a été, lui aussi, un grand « apprenant», selon les dires de l'épître aux Hébreux : \begin{quote}
    «Tout Fils qu'il était, il apprit par ses souffrances l'obéissance, et, conduit jusqu'à son propre accomplissement, il devint pour tous ceux qui lui obéissent cause de salut éternel » (He 5, 8).
\end{quote} 
Ce que l'épître aux Hébreux affirme avec vigueur, les évangiles synoptiques le racontent en montrant Jésus apprenant des autres qui il est : du lépreux (Mc 1, 40), de la femme hémorroisse (Mc 5, 30), de la Syro-Phénicienne (Mc 7, 29), de Pierre et de bien d'autres encore.
Cet apprentissage le conduit vers l'accomplissement, vers l'expérience méta-éthique du don de soi pour la multitude.

\paragraph{Musulman, interrogation du pacte avec Dieu }
Le musulman, enfin, réitère l'intransigeant jugement initial qui exclut tout pluriel de l'Unique. Lui aussi vit donc la «foi » dans une sorte d'interrogation constante, qui s'exerce, comme on l'a déjà noté, dans le champ pré-éthique du « pacte » avec Dieu, imprimé en tout homme avant sa naissance. C'est en ce sens qu'il faut entendre la réserve du Coran quand il s'adresse directement au prophète Muhammad: 
\begin{quote}
    « Dis: je ne suis qu'un Avertisseur. Il n'est de divinité que Dieu, l'Unique, l'Invincible » (sourate 38, 65).
\end{quote}
Aujourd'hui il faut donc mettre en valeur ce retour critique sur soi qui caractérise le prophétisme biblique et coranique, indépendamment de la tournure précise qu'il prend dans chaque cas: il constitue, au sein des trois traditions, ce lieu mystérieux où des rencontres imprévisibles avec d'autres pourront se nouer et provoquer un véritable apprentissage entre partenaires.
\sn{790}

\subsubsection{Les étapes de l'apprentissage.}

\paragraph{1. purifier nos préjugés. Rester dans la règle d'Or; apprentissage de la société moderne} 
1. Une première étape consisterait alors à nous purifier des préjugés qui ont entraîné la violence. Je les ai déjà nommés: il peut s'agir de schèmes de «substitution » ou d'«exclusion», quand l'un prétend se substituer brutalement à l'autre dans sa mission religieuse au sein de l'humanité; de façon plus subtile il peut s'agir aussi d'« inclure » l'autre dans sa propre mission, de l'enfermer par exemple dans un rôle de préparation. Une autre manière encore de sortir de la « Règle d'or» de nos rencontres, symétriquement opposée à la précédente, serait de dénier à l'un des interlocuteurs ou à tous les trois toute capacité d'apprentissage dans la société moderne\sn{important pôur l'écologie}. Nous avons vu que le véritable « comparatisme » permet de décrypter cette capacité spécifique d'un retour sur soi au cœur de la foi de chacun des trois.

\paragraph{2. repenser positivement nos liens}
2. La deuxième étape de l'apprentissage est plus difficile.
Il s'agit de repenser positivement nos liens. Or, nous touchons 1 aux limites structurelles de la communication entre les trois, aux limites aussi de notre capacité d'apprentissage.
À nouveau plusieurs possibilités se présentent.

\paragraph{mystiques, facile de traverser les religions} On a toujours enseigné, dans chacun des monothéismes. que l'Unique ne fait pas nombre avec ceux qui témoignent de lui. Ce type d'argument, si familier aux mystiques, consiste à critiquer nos représentations de Dieu, à développer toute une série de procédures à la fois intellectuelles et affectives pour approcher corporellement le mystère du Dieu tout autre. Les spirituels de tradition différente se rencontrent sur ces voies à la fois ascétiques et mystiques; ils traversent aisément les frontières entre les religions parce que, pour eux, un espace infini de communication avec autrui s'est ouvert dans la différence indépassable entre Dieu et ce qu'on peut dire de lui.

\paragraph{mystique pas facile pour christianisme du fait de l'incarnation qui nous dévoile le père, qui n'est pas ineffable}
Relativement bien ajusté au monothéisme juif et musulman, ce type d'expérience mystique se heurte, en christianisme, au mystère de l'Incarnation: \begin{quote}
    « Personne n'a jamais vu Dieu; le Fils unique, qui est dans le sein du Père, nous l'a dévoilé » (Jn 1, 18).
\end{quote} 

\paragraph{risque du mystique d'éluder la question - car elle ne l'intéresse pas ? }Peut-être doit-on ajouter que le spirituel, qui relativise les différences entre les trois monothéismes. risque de ne plus se laisser interroger par la discordance des témoignages qui discrédite toujours la cause elle-même.
\sn{791}

\paragraph{accepter que l'autre n'abandonne pas sa prétention à la prééminence mais accepter de penser cette prééminence sans produire la violence}Faut-il alors préserver la prééminence ou l'excellence de l'un des trois ? D'un simple point de vue anthropologique, je ne vois pas comment éviter le jugement d'excellence sur la tradition à laquelle j'appartiens. D'un point de vue théolo-gique, je ne vois pas non plus comment l'un des trois pourrait renoncer à ses prérogatives d'excellence sans renoncer à sa propre identité. La question se transforme alors en exigence de penser aujourd'hui « l'ultime sceau de la lignée prophétique », « l'accomplissement des Écritures » ou la « mission d'être lumière des nations » de telle manière que ces prérogatives irrépressibles ne produisent pas de violence.

\paragraph{3. \textit{pourquoi} trois témoins}
3. Je reviendrai sur cette tâche proprement théologique.
Cependant, ce traitement du « comment » de l'excellence ne doit pas nous détourner prématurément de la question du «pourquoi ». C'est la troisième étape sur le chemin de l'apprentissage: « Mais pourquoi finalement trois témoins ? » Certes, nos traditions ont quelques réponses à leur disposition pour «expliquer » la venue d'un deuxième et même d'un troisième témoin ou pour rendre raison de leur propre existence après l'arrivée d'un premier et d'un deuxième envoyé: on évoque la dissidence ou l'hérésie de l'héritier, l'aveuglement ou le péché de l'aîné, l'infidélité ou l'exagération des deux prédé-cesseurs.   Mais c'est une chose de raconter la venue progressive d'un premier, d'un deuxième et même d'un troisième, et c'est une autre de penser le « mystère » de leur cohabita-ton dans nos sociétés. N'est-ce qu'un accident de l'histoire, une contingence fortuite ? Ou faut-il y découvrir la main de Dieu, son «dessein » ? Et ce « dessein » est-il même concerné par l'avènement d'une société dont le retrait fondamental par rapport aux trois monothéismes suscite la question de leur «pourquoi» ? Le terme « mystère» n'est sûrement pas trop fort pour désigner la «chose», puisqu'il a déjà servi à Paul pour penser l'énigmatique présence d'Israël aux côtés des chrétiens (Rm 11, 25). Ce « mystère » n'est-il pas devenu plus insondable encore et plus impénétrable (Rm 11, 33-36) depuis que nous sommes «trois témoins » ? La présence des deux autres, que nous apprend-elle sur nous-mêmes et sur l'Unique \sn{792}
que nous ne puissions pas savoir par nous-mêmes ? Question adressée à chacun des trois et à laquelle aucun des trois ne peut répondre à la place des deux autres. 
\paragraph{quelle réponse en tant que Chrétien ?}Nous sommes donc reconduits vers la perspective interne qui ne peut être que chrétienne et théologique pour nous.

\subsection{Entrer dans le règne de l'incomparable}

\paragraph{ne pas compter, ce n'est pas le nombre trois qui est important, mais entrer dans la comparaison (pour y voir in fine un plan de Dieu ?)} Il n'est pas sûr qu'il y ait une réponse à la question «pourquoi trois ?» Peut-être faut-il même renoncer à compter.
On se souvient ici de l'embarras de saint Augustin quand il réfléchit dans son \textit{De Trinitate }sur la différence trinitaire en Dieu: \begin{quote}
    «Le Père, le Fils, le Saint-Esprit sont trois, nous cherchons donc: trois quoi? (tres quid) »
\end{quote}
L'extraordinaire difficulté vient de ce que le terme « personne » (trois personnes) est déjà un générique qui ne peut désigner l'absolu singularité de «chacun», comme le générique « monothéisme » n'arrive jamais à dire ce qu'est l'islam, le christianisme et le judaïsme.
Face à cette limite du nombre dans un domaine où on ne peut «co-numérer», saint Basile conseille de renoncer à compter.
Ce qu'il dit de l'Unique, il faut l'appliquer, me semble-t-il aux trois témoins: \begin{quote}
    « Et s'il faut tout de même compter, du moins que la vérité ne soit point falsifiée: ou bien qu'on honore en silence les choses ineffables, ou bien qu'on compte avec piété et respect. »
\end{quote}  Mais pour pouvoir renoncer à compter il faut bien d'abord compter jusqu'à trois, et pour entrer avec respect dans le règne de l'incomparable il faut bien commencer par comparer. C'est ce que nous avons fait jusqu'ici. Mais dans cette dernière partie je voudrais brièvement tracer un chemin de rencontre qui nous conduit de la comparaison à la découverte de l'incomparable singularité de chacun des trois témoins. \sn{793}
\begin{Synthesis}
Mon hypothèse est que la différence fondamentale du christianisme par rapport au judaïsme et à l'islam, le mystère de l'Incarnation et de la Trinité, est en même temps le lieu  
où se définit une\textsc{ théologie de la rencontre} à la hauteur des enjeux développés dans les deux premières parties. 
\end{Synthesis}

Commençons donc par prendre au sérieux cette différence.
\subsubsection{La contestation de l'unicité du Christ.}

\paragraph{partir du scandale de l'association de Jésus à Dieu pour l'Islam}
Considérer le «témoin » par excellence du christianisme, Jésus, comme un de la Trinité et l'associer à l'unique Dieu, voilà une manière bien scandaleuse, aux yeux des juifs et des musulmans, d'introduire le nombre en Dieu et de diviniser un parmi d'autres sur terre. L'islam a porté sa contestation au cœur même de cette foi: \begin{quote}
    «Ô gens du Livre! N'allez pas au-delà du bon sens dans votre religion. Ne proclamez que la vérité sur Dieu. Le Messie, Jésus fils de Marie, n'est qu'un envoyé de Dieu... Croyez donc en Dieu et en ses prophètes.
Ne dites jamais "trois". Arrêtez cette imposture... Dieu est un Dieu unique. Le Messie ne se sent pas indigne d'être serviteur de Dieu, comme le sont les anges les plus proches de Dieu» (sourate 4, 171s.)

« Il n'est pas concevable que Dieu se donne un fils» couriste 19 35). « Il n'a pas engendré; n'est Dieu» (sourate 4, l/Is.). « Ce n'est pas concevable que Dieu se donne un fils » (sourate 19,35). « Il n'a pas engendré; n'est égal à lui, personne » (sourate 112).
\end{quote}  
\paragraph{des psaumes messianiques comme réponse chrétienne}
Ces quelques versets du Coran nous renvoient, par contraste, à un autre ensemble, les textes messianiques des Écritures juives, et en particulier les psaumes 2 et 110:
\begin{quote}
    «Je proclame le décret du Seigneur. Il m'a dit: "Tu es mon fils; moi, aujourd'hui je t'ai engendré. Demande et je te donne en héritage les nations, pour domaine la terre entière" » (Ps 2, 7). 
    
    «Oracle du Seigneur à mon Seigneur: "Siège à ma droite !... Domine jusqu'au cœur de l'ennemi!" Le jour où paraît ta puissance, tu es prince, éblouissant de sainteté : "Comme la rosée qui naît de l'aurore, je t'ai engendré." Le Seigneur l'a juré dans un serment irrévocable: "Tu es prêtre à jamais selon l'ordre du roi Melchisédech" » (Ps 110, 1-4). 
\end{quote}
\paragraph{Et pour les juifs, laisser ouvert l'avenir du peuple messianique}
Si l'islam conteste l'idée même d'un « engendrement» en Dieu, et cela au nom de sa critique prééthique de toute association d'un pluriel à l'Unique, le judaïsme, lui, ne peut pas reconnaître l'accomplissement définitif de ces psaumes en l'itinéraire de Jésus parce qu'il doit laisser ouvert l'avenir du peuple messianique: au nom même de sa mission éthique il se méfie de tout ce qui occulte la situation d'exil de l'humanité qui durera jusqu'à la fin.

\paragraph{face à cette double contestation, quelle ressort meta éthique ?}
La rencontre de cette double contestation ne nous oblige pas à renoncer aux prérogatives du Christ; je l'ai déjà dit Mais alors en quoi consiste la « purification » de nos schème d'excellence (1re étape de toute rencontre)? Comment pouvons nous aborder positivement la communication avec les deux autres témoins (2e étape de rencontre)? Il est probable que l'enjeu de la rencontre n'est plus d'abord la réaffirmation d'une différence doctrinale mais la mise en œuvre discrète du caractère méta-éthique du christianisme dans l'histoire d la communication humaine : autrement dit, il s'agit pour le chrétiens de vivre de la sainteté même de Dieu, rien de plus e rien de moins. C'est seulement alors que peut leur paraître avec une acuité nouvelle, la capacité de l'Unique à engendre non seulement un «Fils unique » mais aussi avec lui «un multitude de fils ».
\paragraph{Foi Chrétienne comme style de vie, de vivre la sainteté}
On ne soulignera jamais assez le renversement de perspective qui vient d'être produit. La foi chrétienne n'est pas d'abord une doctrine (comme toute une tradition l'a prétendu) mais un « style de vie » ou une manière de vivre de la sainteté même de Dieu: seule l'expérience effective de l'Esprit de sainteté nous permet de confesser et de comprendre un jour l'indépassable excellence du Fils unique du Père. N'est-ce pas cela que la rencontre des autres témoins nous apprend?
Essayons donc de comprendre.
\subsubsection{Unique et la « multitude des fils» (He 2, 10).}
\paragraph{psaumes messianiques : communication de la sainteté de Dieu au roi et à son peuple}
L'enjeu des psaumes messianiques, cités à l'instant, est la communication de la sainteté de Dieu au roi et à son peuple:
\begin{quote}
    « Le jour où paraît ta puissance, tu es prince, éblouissant de sainteté» (Ps 110, 3). 
\end{quote}
S'adressant à la multitude, le Nouveau Testament l'a bien compris quand il appelle tous à être «parfaits comme votre Père céleste est parfait » (Mt 5, 48).
Ce vin nouveau de la perfection divine qui n'est rien d'autre que l'accomplissement inouï et excessif de la Loi (« la justice qui surpasse la justice » selon Mt 5, 20). le Sermon sur la montagne l'introduit dans les «outres» de la Règle d'or:
\begin{quote}
    «Tout ce que vous voulez que les hommes fassent pour vous, faites-le vous-mêmes pour eux : c'est la Loi et les Prophètes » (Mt 7, 12). 
\end{quote}
\paragraph{Sainteté de Dieu, démesurer au delà de la règle de réciprocité}
Cette règle de réciprocité qui se trouve au fondement de nos sociétés modernes peut en effet nous réserve quelques surprises, individuelles et collectives, quand on doit affronter l'antipathie d'autrui ou quand on entend subitement l'invitation à renoncer à toute réciprocité et à prendre sur soi la violence et la faute d'autrui: \begin{quote}
    « Aimez vos ennemis, et priez pour ceux qui vous persécutent, afin d'être vraiment les fils de votre Père aux cieux, car il fait lever son soleil sur les méchants et les bons, et tomber la pluie sur les justes et les injustes » (Mt 5, 44 s.).
\end{quote}
Qu'il s'agisse, dans la communication de la sainteté de Dieu à l'homme, d'un véritable \textit{engendrement} \sn{1. Cette communication de la sainteté même de Dieu fixe définitivement le sens du terme « engendrement », jusque dans la haute christologie de Nicée (« engendré non pas créé, de même nature que le Père »). La forme évangélique ou « narrative » de cet engendrement a été abordée dans le chapitre Il de la deuxième partie, p. 477 s. Je reviendrai aux chapitres v, vi et xin de cette partie à son enjeu christologique et eschatologique; voir plus loin, p. 826, p. 849 s. et p. 1036 s.}
(« devenir fils de votre Père»), on ne le découvre que progressivement: quand on réalise tout d'un coup que l'appel démesuré à être \textit{comme} Dieu, toujours dans telle ou telle situation, s'avère «à la mesure» de celui qui l'entend. 
\paragraph{unification traverse les forces de mort, de peur de l'autre et de nous mêmes. Nous nous "comparons", violence entre témoins}
Mais loin d'être d'emblée acquise, cette gracieuse unification affronte et traverse des violences et des forces de mort qui se déclarent précisément à la « limite » infiniment mobile entre la « démesure divine » et nos multiples « mesures humaines ». La peur face à l'inconnu
- celui de l'autre, de nous-mêmes et finalement de la mort - nous pousse à fixer nos frontières, à garder nos terrains et à entrer dans un jeu de comparaison, voire une lutte sans merci entre « semblables ». Jamais pourtant ce que l'un accueille de la «démesure», \textit{inscrite en tout être humain}, ne se laissera mesurer en fonction de ce qu'en conscience l'autre jugera «à sa mesure». Qui peut faire sortir l'homme de ses mortelles comparaisons, sinon celui qui atteint vraiment en lui la peur d'être soi-même, peur qui est sans doute la racine ultime des mystérieuses violences entre « témoins » ? Il faut entendre la voix d'un autre pour lâcher cette peur; voix qui certes doit émaner de quelqu'un devenu « proche » mais voix qui doit venir en même temps de plus loin: de Celui qui, ici et maintenant, se montre « à la mesure» d'un tel ou d'une telle, se faisant entendre à lui et en lui, avec la douceur et la discrétion
de l'Esprit: «Aujourd'hui, je t'ai engendré» (Lc 3, 22 et Ac 13, 33).

\paragraph{Expérience de l'engendrement : en Christologie, permet le pluriel (He). L'association  de Jésus à l'unicité de Dieu ne peut être séparé de berger de Paix et prêtre de Melchisédech}
Cette expérience inouïe de « l'engendrement » nous fait comprendre les deux versants intimement liés d'une théologie chrétienne de la rencontre. D'abord le versant christologique: la passion de certains auteurs du Nouveau Testament pour le « pluriel ». À ce propos l'épître aux Hébreux, citée tout au début, est tout à fait exemplaire parce qu'elle déplace la «filiation divine » vers la condition commune de tous: \begin{quote}
    «Il convenait, en effet, écrit-il, à celui pour qui et par qui tout existe et qui voulait conduire à la gloire une multitude de fils, de mener à l'accomplissement par des souffrances l'initiateur de leur salut. Car le sanctificateur et les sanctifiés ont tous une même origine; aussi ne rougit-il pas de les appeler frères» (He 2, 10 s.)
\end{quote}
Certes, l'unicité ou l'excellence de Jésus n'est pas niée dans ce texte étonnant: il reste pour l'épître aux Hébreux celui qui ouvre, au cœur de nos mesures humaines, l'accès à la sainteté démesurée de Dieu. Mais Jésus n'est «associé » à l'unicité de Dieu (He 7, 2 s.) que parce qu'il est en même temps berger de paix (roi de Salem) et prêtre selon l'ordre du roi Melchisédech, l'homme unique donc qui est entièrement façonné par le don de soi et appelé à s'effacer pour ouvrir dans l'humanité \textbf{les chemins multiformes de la sainteté de Dieu}.
\paragraph{Expérience de l'engendrement : spirituellement, pas de comparaison entre Fils et témoin }
Ensuite, le versant spirituel ou pneumatologique de la rencontre. La découverte que Dieu engendre une « multitude de fils » sur les chemins de sa sainteté amène à renoncer à toute comparaison entre fils et témoins. Dieu n'est-il pas à la mesure de tant et de tant de mesures humaines, devenues toutes, de ce fait, incomparables? Le Dieu unique des chrétiens est mystère du lien entre des incomparables \sn{voir p 77}. Mais il faut s'être affronté, dans sa propre  vie, à la question de la sainteté, de la démesure divine à ma mesure, pour pouvoir admettre que juifs et musulmans sont, eux aussi, aux prises avec un même combat. On sera alors conduit non seulement au respect dans la rencontre des deux autres témoins mais encore au risque d'y « laisser sa peau » : \begin{quote}
    « En Christ, je dis la vérité, je ne mens  pas, par l'Esprit saint ma conscience m'en rend témoignage », écrit Paul dans l'épître aux Romains. « Oui, je souhaiterais être anathème, être moi-même séparé du Christ pour mes frères..., eux qui sont les Israélites » (Rm 9, 1-5).
\end{quote}
Voilà ce que j'entends par « style chrétien de rencontre»; 
\begin{Def}[style chrétien de rencontre]
    «style» qui se caractérise par une singulière manière d'espérer la paix en affrontant la violence. 
\end{Def}
Il se « définit » au lieu même de la différence fondamentale du christianisme par rapport au judaïsme et à l'islam, dans le mystère de l'Incarnation et de la Trinité. Ce qui a été mon hypothèse.

\subsection{En guise de Conclusion : un jeu de compétition}

\paragraph{« compétition » autour de la sainteté de Dieu, le mode de victoire n'est pas le même pour tous}
La rencontre des trois monothéismes est en dernière instance une « compétition » autour de la sainteté de Dieu.
\begin{quote}
    «Les coureurs, dans le stade, courent tous mais un seul gagne le prix» (1 Co 9, 24),
\end{quote}
 écrit saint Paul qui affectionne la métaphore grecque des Jeux olympiques!. Le juif Philon d'Alexandrie l'a précédé quand il loue dans le \textit{De agricultura} 
 \begin{quote}
 «le seul concours olympique » qui « pourrait être appelé sacré à juste titre: non pas celui que célèbrent les gens d'Élide, mais celui qui vise à acquérir les vertus divines et vraiment olympiennes», en ajoutant, avec une rare finesse que « le mode de la victoire n'est pas le même pour tous mais que tous sont dignes d'estime2».    
 \end{quote}
  On connaît par ailleurs ce verset du Coran. \begin{quote}
      « Si Dieu avait voulu, il aurait fait de vous une seule communauté. Mais il a voulu vous éprouver par le don qu'il vous a fait. Cherchez à vous surpasser les uns les autres dans vos bonnes actions » (sourate 5, 48).
  \end{quote}

\paragraph{rôle des chrétiens : montrer l'unicité de chaque témoin}
Le rôle spécifique des chrétiens dans ce singulier « jeu de compétition », aux allures parfois dramatiques, ne serait-il pas de renoncer à compter et à comparer pour mettre en valeur l'unicité incomparable de chaque partenaire? Tâche difficile qui peut les conduire aujourd'hui encore dans l'expérience du don de soi.\sn{798}
\paragraph{mystère de la diversité des témoins : penser la pluralité des monothéismes}
Mais revenons, pour finir, à notre question: pourquoi des témoins (dernière étape de la rencontre)? Renoncer un jour à poser cette question et abandonner le « comptage », c'est certes accepter avec l'épître aux Romains que le « dessein » de Dieu est insondable et impénétrable. Mais il ne s'agit absolument pas d'un scepticisme qui renoncerait à penser vraiment pluralité des monothéismes. Penser le mystère des trois témoins c'est baliser, au sein d'une histoire faite de compétitions chemin de rencontre qui nous amène à vivre dans le règne trinitaire de l'incomparable.
\paragraph{Plan : on avait pensé la communauté ecclésiale, il faut penser le sans religion}
Ce qui précède nous a déjà conduit vers le deuxième lien de l'expérience trinitaire de la foi: la référence de la communauté ecclésiale à la sainteté messianique de Jésus de Nazareth.
Les deux dernières études de cette première étape entreront davantage dans la question christologique en faisant d'abord état de la transformation de la christologie catholique au xxe siècle et en revenant ensuite au rapport du Christ aux hommes religieux et à ceux qui sont « sans religion ».

 
\section{devoir}

\section{L'unique et ses témoins, le christianisme comme style}

\subsection{Christoph Theobald}

\paragraph{Théologien allemand vivant en France} Chrstoph Théobald est un jésuite d'origine allemande et vivant en France. Véritable pont entre les deux cultures théologiques, il est en particulier connu pour sa traduction des oeuvres de Karl Rahner. Il est aussi très marqué par la philosophie Française, Merleau-Ponty, Lévinas et Ricoeur.

\paragraph{Curieux de la théologie contemporaine} Il a été longtemps le directeur des RSR, la \textit{Revue des Sciences Religieuses}, présentant les dernières recherches de théologie fondamentale au public français. Il porte aussi les questions de son temps comme le montre l'un de ses derniers ouvrages, \cite{theobald_urgences_2017}. De même, le chapitre que nous étudions a d'abord fait partie d'un colloque \cite{centre_sevres_paris_unique_1996}

\paragraph{Un théologien Jésuite} Christoph Théobald est jésuite. Outre son intérêt pour les théologiens Karl Rahner et Urs van Balthazar, il est marqué par l'esprit \textit{Exercices Spirituels} et de suivre Jésus concret. Certains passages de \cite{theobald_christianisme_2007} nous ont rappelé les méditations de l'appel du roi temporel (ES 91)  qui propose \textit{l'imitation de Jésus en endurant tous les outrages, tout blâme et toute pauvreté (ES 98)}



\subsection{Bibliographie}
\sn{Eventuellement, notez l’ensemble des textes et références auxquels vous avez eu recours pour préparer l’exposé, y compris vos sources pour la biographie et les sites internet visités. }





\subsection{Nature du texte}


\paragraph{Un colloque} Dans sa \href{https://centresevres.com/content/uploads/2017/07/bibliographie-complete-de-christoph-theobald-sj-2021.pdf}{bibliographie}, on note que le texte a eu au moins trois moutures différentes : en 1995, lors du colloque de rentrée au Centre Sevres (numéro 60 de la bibliographie), publié en 1996 \cite{centre_sevres_paris_unique_1996} dans  une version différente du texte précédent.  

\paragraph{Puis un chapitre du \textit{Christianisme comme Style}} Théobald a publié le \textit{Christianisme comme style, une manière de faire de la théologie en postmodernité}\cite{theobald_christianisme_2007} en 2007. Le livre comment par une \textit{ouverture} sur le Christianisme comme \textit{Style}, reprenant le terme de Merleau-Ponty (verif). 

 \begin{quote}
     Après avoir "ausculté" notre présent et désigné le \textit{kairos} qu'il représente (I) et après avoir réfléchi longuement à l'enracinement spirituel (II) et scripturaire (III) de la théologie chrétienne, le moment est venu d'aborder directement ce que celle-ci doit donner à penser aujourd'hui : le christianisme comme style qui ouvre à une intelligence de lui-même, libre et accessible à tous, petits et grands. 
     \cite[p 699]{theobald_christianisme_2007}
 \end{quote}
\begin{quote}
    

... herméneutique dogmatique qui tente de penser le versant normatif du mystère chrétien en relation constante avec l'histoire et la pratique actuelle de l'Eglise.  700
\end{quote}
Suivant le le symbole de Nicée, il commence par une réflexion sur Croire en Dieu  :
\begin{quote}
    le christianisme est il un monotheisme ?" 703
\end{quote}
Mais n'y répond pas directement.


Le chapitre 3 de cette partie est le chapitre qui nous intéresse, l'Unique et ses témoins, Jalons pour une théologie de la rencontre entre juifs, chrétiens et musulmans.
Il est précédé par un premier chapitre sur Dieu en postmodernité, une manière de désigner la sainteté comme mystère du monde, et le chapitre 2 sur la foi trinitaire des chrétiens et l'énigme du lien social.
Il précède une reflexion sur la christologie. 




\subsection{Le contexte historique et textuel}

La publication dans le contexte de la polémique de la conférence donnée à Ratisboonne sur \textit{Foi, Raison et Université} du pape Benoit XVI du 12 septembre 2006, qui cite le dialogue publié par le professeur Khoury (de Münster) entre l’empereur by-
zantin lettré Manuel II Paléologue et un savant persan sur le lien entre raison et religion \sn{vérifier que les premières versions ne contiennent pas de référence à Benoit XVI}. Cette polémique entraîna une vague de violence dans le monde musulman. 
Situez la production du texte dans son contexte historique (date de la publication). A quelle occasion a-t-il été rédigé (suite à quel événement), dans quel contexte culturel et social, etc. ? A qui est-il adressé ?

Situez le texte dans son environnement littéraire, s’il est extrait d’un ouvrage ou d’un corpus. Présentez rapidement l’ouvrage, indiquez ce qui précède et ce qui suit le texte choisi, etc. Est-ce une traduction ? \mn{s'il y a 700 pages, dire les chapitres avant et après}



\section{Présentation}


	\subsection{5.1 La problématique }

Déterminer la problématique en vous inspirant des questions suivantes :
Après la phase de lecture pas à pas, vous construisez la question à laquelle l’auteur s’affronte :
-	pourquoi l’auteur se bat-il ?
-	quel problème essaie-t-il de régler, d’éclairer ? En général, il n’est pas difficile de trouver exposée la problématique en toute lettre dans le texte lui-même – parfois même de manière redondante ;
-	en fonction de quel contexte culturel, social, culturel, économique, politique l’auteur construit-il sa problématique ? Notez les événements déterminants auxquels il se réfère et les auteurs qu’il évoque – plus ou moins explicitement – comme ses alliés ou ses adversaires – et le cas échéant, renseignez-vous sur eux. 

\subsection{5.2	 La thèse ou plutôt hypothèse}

En fonction de la problématique de l’auteur, vous établissez la thèse (ou hypothèse) de l’article ou du texte étudié : quelle solution l’auteur apporte-t-il à sa question ? quelle perspective établit-il ? Là encore, il n’est pas difficile de trouver cette thèse exposée de manière explicite dans le texte lui-même. 

\subsection{5.3	 L’argumentation }


Présentez la logique argumentative en fonction de laquelle l’auteur établit sa thèse (passe de l’hypothèse à la thèse vérifiée).



\section{Reaction personnelle en discutant et critiquant de manière argumentée}

\paragraph{partir de Jésus}


\section{vision aux He} pertinent pour le Judaisme. pour l'islam ?
souffrance ? peut on penser la souffrance comme apprentissage ? peut être 
 

\section{Règle d'Or : dignité et perfection ? }

 \subsection{Medine et la Mecques} juridique vs anti juridique : on ne peut rester à l'ante juridique



 







 %Frères,





(et soeurs),



Ce dimanche, nous écoutons l'Evangile du Bon Pasteur : 

\begin{singlequote}
En ce temps-là, Jésus déclara :
\begin{quote}
      « Amen, amen, je vous le dis :
celui qui entre dans l’enclos des brebis
sans passer par la porte,
mais qui escalade par un autre endroit,
celui-là est un voleur et un bandit.
    Celui qui entre par la porte,
c’est le pasteur, le berger des brebis.
    Le portier lui ouvre,
et les brebis écoutent sa voix.
Ses brebis à lui, il les appelle chacune par son nom,
et il les fait sortir.
    Quand il a poussé dehors toutes les siennes,
il marche à leur tête,
et les brebis le suivent,
car elles connaissent sa voix.
    Jamais elles ne suivront un étranger,
mais elles s’enfuiront loin de lui,
car elles ne connaissent pas la voix des étrangers. »
\end{quote}
  

    Jésus employa cette image pour s’adresser aux pharisiens,
mais eux ne comprirent pas de quoi il leur parlait.
C’est pourquoi Jésus reprit la parole :
\begin{quote}
    « Amen, amen, je vous le dis :
Moi, je suis la porte des brebis.
    Tous ceux qui sont venus avant moi
sont des voleurs et des bandits ;
mais les brebis ne les ont pas écoutés.
    Moi, je suis la porte.
Si quelqu’un entre en passant par moi,
il sera sauvé ;
il pourra entrer ; il pourra sortir et trouver un pâturage.
Le voleur ne vient que pour voler, égorger, faire périr.
Moi, je suis venu pour que les brebis aient la vie,
la vie en abondance. »
\end{quote}
 Jn 10,10
\end{singlequote}

Ce passage est aussi fameux car il a été repris par Nietzsche dans \textit{Also sprach Zarathustra} :

\begin{singlequote}
Mes yeux se sont ouverts : J’ai besoin de compagnons, de compagnons vivants, 
– non point de compagnons morts et de cadavres que je porte avec moi où je veux. Mais j’ai besoin de compagnons vivants qui me suivent, parce qu’ils veulent se suivre eux-mêmes partout où je vais. 

Mes yeux se sont ouverts : Ce n’est pas à la foule que doit parler Zarathoustra, mais à des compagnons ! 

Zarathoustra ne doit pas être le berger et le chien d’un troupeau ! 
C’est pour enlever beaucoup de brebis du troupeau que je suis venu. Le peuple et le troupeau s’irriteront contre moi : 

Zarathoustra veut être traité de brigand par les bergers. Je dis bergers, mais ils s’appellent les bons et les justes. Je dis bergers, mais ils s’appellent les fidèles de la vraie croyance. Voyez les bons et les justes ! Qui haïssent-ils le plus ? Celui qui brise leurs tables des valeurs, le destructeur, le criminel : 
– mais c’est celui-là le créateur. 

Voyez les fidèles de toutes les croyances ! Qui haïssent-ils le plus ? Celui qui brise leurs tables des valeurs, le destructeur, le criminel : 
– mais c’est celui-là le créateur. 
Des compagnons, voilà ce que cherche le créateur et non des cadavres, des troupeaux ou des croyants. Des créateurs comme lui, voilà ce que cherche le créateur, de ceux qui inscrivent des valeurs nouvelles sur des tables nouvelles. 
\end{singlequote}


\paragraph{Prendre Nietzsche au sérieux} Il convient tout d'abord de prendre Nietzsche au sérieux. Il y a ici une double critique qui s'adresse à tous les croyants mais en particulier aux chrétiens (et aux juifs) :
\begin{itemize}
    \item les chrétiens sont des brebis et non des compagnons, ou est leur liberté ?
    \item les chrétiens attaquent ceux qui brisent l'ordre établi.
\end{itemize}
Cette critique \textit{touche} car la religion a trop souvent été du côté des puissants et \textit{du parti de l'ordre}.
Qui voudrait d'un tel sauveur, qui préfère l'esclavage à la liberté ? 


\paragraph{Mais de quel Jésus parle Nietzsche} De la même façon que les zoroastres perses ne reconnaitraient pas la figure de Zoroastre présentée par Nietzche, est ce que le Jésus de Nietzsche ressemble à celui que nous connaissons ? Un philosophe contemporain a écrit : 

\begin{singlequote}
    Nietzsche concentre le message tout entier sur l'aujourd'hui, sur la présence vivante de Jésus. Toute attente, toute peine, tout renvoi pénible, toute future apocalypse est effacé de T« aujourd'hui, tu seras comme moi au Paradis » (Le 23,43) —
\end{singlequote}


\paragraph{Le Christianisme comme Style} Deux dangers guettent toute religion. 
\begin{itemize}
    \item Le premier, c'est de mettre la "main" sur Dieu ou notre salut : \textit{je suis la loi ou la "méthode" donc je suis sauvé}. A noter que cela concerne tout homme, le juif qui suit les 613 commandements de la torah, celui de \textit{comment se faire des amis} ou celui qui fait un \textit{ultra-trail} ou suis \textit{petit bambou}. C'est la critique de Nietzsche, de chercher notre salut en abandonnant notre liberté. C'est aussi l'expérience que l'on peut faire que la loi, le processus peut aider mais qu'en particulier sur les choses de l'esprit, elle ne marche pas.
    \item Mais le risque inverse, c'est celui pour être religion, d'être insignifiante, de ne pas proposer un \textit{chemin} vers Dieu. Ou plus subtile, de proposer un chemin de Salut qui passe par la connaissance, \textit{la gnose} : je suis \textit{initié} (parce que j'ai fait une licence de théologie,...), donc je suis \textit{sauvé}. Croire que je suis meilleur que les autres.
\end{itemize}

Face à ces deux risques, le christianisme est \textit{un style } :
\begin{itemize}
    \item c'est à dire avec une partie qui nous permet de reconnaître les chrétiens : non pas des habits, ni une nourriture spéciale, mais un accueil des plus pauvres, des malades, un rassemblement le dimanche,... 
    \item mais ce style se caractérise par son ouverture sur l'imprévu : se laisser surprendre par le Christ.
\end{itemize}

\paragraph{Suivre Jésus de Nazareth} Quelle familiarité avons-nous de Jésus ? La prière, la lecture 

\begin{singlequote}
    Et pour vous, qui suis-je ?
\end{singlequote}
Demande Jésus aux apôtres à Césarée de Philippe. 

Retour à Jésus, la mort du Christ, certes mais d'abord un homme qui a vécu, 


\begin{singlequote}
    La négligence dans la charge de cultiver et de garder une relation adéquate avec le voisin, envers lequel j’ai le devoir d’attention et de protection, détruit ma relation intérieure avec moi-même, avec les autres, avec Dieu et avec la terre LS 70
\end{singlequote}

Décentrement

Mort alors se comprend

1880 : Ste Thérèse Charles de Foucault, Jésus, amour.


\begin{singlequote}
    Pape François, Exhortation apostolique post-synodale « Amoris Laetitia » (2016), n° 267 (cité désormais comme AL). Voir aussi : « Il est nécessaire de développer des habitus. De même, les habitudes acquises depuis l’enfance ont une fonction positive, en aidant à ce que les grandes valeurs intériorisées se traduisent par des comportements extérieurs sains et stables. […] Le renforcement de la volonté et la répétition d’actions déterminées construisent la conduite morale, et sans la répétition consciente, libre et valorisée de certains bons comportements, l’éducation à cette conduite n’est pas achevée. Les motivations, ou bien l’attraction que nous sentons pour une valeur déterminée, ne deviennent pas une vertu sans ces actes adéquatement motivés » (AL, 266).
\end{singlequote}


 %{Sujets de mémoires ISTR}

30 pages;
bibliographie critique de chaque livre, probalématique et méthode.



\paragraph{Partir d'un texte} 
Se concentrer sur un auteur : laudato Si

\paragraph{Reenchantement du monde}
 Djinn, esprits, Apocalypse

\paragraph{Récits}
 : religion : mythe explicite; vision du monde : toucher les relions sur le mythe du progrès. 


\section{Dialogue inter religieux et changement climatique}

\paragraph{Mise en problématique autour de l'idolâtrie}
\begin{itemize}
    \item Un enjeu de pertinence pour les religions : \textit{mondiale}, \textit{existentielle}, \textit{ne se joue pas à l'échelle individuelle mais d'une transformation collective}
    \item la réponse de l'Église catholique : lien avec la doctrine sociale de l'Église, articulation de la justice nécessaire et de l'action individuelle et collective. 
    \item Une \textit{nouveauté} du discours du pape François, une critique de l'idolâtrie dans des \textit{modes de vie} individualiste
    \item Et ouverture aux autres religions qui sont appelées à relever ensemble ce défi
    \item A la différence Dt, qui critiquait fortement les religions extérieures et toutes les compromissions, ici, il semble que nous ayons un paradoxe : positivité des religions non chrétienne et négativité des "compromissions" mais par rapport à une "religion non nommée".
  
\end{itemize}


\paragraph{Quelques pistes à explorer}
\begin{itemize}
    \item François : pas d'autonomie de l'économie (thomiste/sécularisation) qui en s'autonomisant, a pris comme religion l'argent. Alors qur GS, marquée par la sécularisation, en restait aux principes et laissait l'autonomie à l'économie \cite{cavanaugh_idolatrie_2022}. 
    \item  quelle regard chrétien d'une religion écologique ("gaia"),... Peut on être chrétien et Gaia ? ou autre idolâtrie potentielle mais non dénoncée (
\end{itemize}



\section{Le changement climatique, \textit{Enjeu de pertinence} contemporaine pour les religions} 


\paragraph{Quel mode de vie  pour une sobriété heureuse ?}  


 Un enjeu de pertinence pour les religions : \textit{mondiale}, \textit{existentielle}, \textit{ne se joue pas à l'échelle individuelle mais d'une transformation collective}

\paragraph{penser un salut collectif mais à travers une démarche qui entraine tout le monde } d'une certaine façon nous oblige à définir ce qu'est le \textit{salut écologique} 

\paragraph{Ecologie \textit{ou Changement climatique}} \begin{itemize}
    \item Changement climatique : question \item \textit{scientifique} \textit{Ecologie} : rapport au monde, plus large, et porteur d'une vision du monde. Une certaine perméabilité entre les deux terme.

\end{itemize}


% -----------------------------------------------
\section{la réponse de l'Église catholique dans la lignée de la Doctrine Sociale}    

\paragraph{critique de la modernité}
\begin{singlequote}
Ces schémas de pensée [du progrès scientifique et technique] sont si naturellement ancrés, ils informent si puissamment notre appréhension du réel que nous avons du mal à y renoncer tout à fait devant les démentis flagrants que nous offre l’actualité. Nous restons, volontairement ou non, consciemment ou non, orphelins des mythes du progrès, et nous serions bien contents de leur trouver un substitut chrétien, une garantie divine que, malgré quelques péripéties, tout ira pour le mieux. \cite[p.89]{candiard_quelques_2022}

\end{singlequote}

\paragraph{L'analogie avec l'injustice du taux à intérêt} Les religions monothéistes ont toujours porté des interdictions fortes car source d'inégalité. 

 \begin{singlequote}
        « nous ne pouvons pas ignorer qu’outre l’Église catholique, d’autres Églises et communautés chrétiennes – comme aussi d’autres religions – ont nourri une grande préoccupation et une précieuse réflexion sur ces thèmes qui nous préoccupent tous » (LS 7)
        Dans le sillage du concile Vatican II, l’encyclique insiste sur la contribution des religions en tant que vecteur d’une vision et d’une relation à la nature qui permet de répondre aux défis environnementaux et de proposer une alternative ancrée dans une sagesse séculaire pour éviter « l’indifférence, la résignation facile ou la confiance aveugle dans les solutions techniques » (LS 14). Elles constituent une richesse « pour une écologie intégrale et pour un développement plénier de l’humanité » (LS 62). Il s’agit donc pour toutes les religions de puiser dans « leur propre héritage éthique et spirituel », de revenir « à leurs sources » pour « mieux répondre aux nécessités actuelles » (LS 200). « Tous, nous pouvons collaborer comme instruments de Dieu pour la sauvegarde de la création, chacun selon sa culture, son expérience, ses initiatives et ses capacités » (LS 15). Cette crise, source de migrations violentes et contenant en elle la possibilité prochaine des guerres, peut aussi être un lieu de rencontre, de dialogue et d’action (LS 15) entre tous les hommes. Dans une perspective dont on a souligné les accents blondéliens \sn{Juan Carlos Scannone, « La filosofia dell’azione di Blondel e…, le pape y voit la possibilité de susciter une communion d’action afin d’ouvrir à une « nouvelle solidarité universelle » (LS 14).}

    \end{singlequote}


\paragraph{}
% -----------------------------------------------
\section{Une critique de l'idolâtrie, \textit{nouveauté} du pape François}


\paragraph{Idolâtrie, une opportunité invitant à considérer de façon différente l'économie et es autres phénomènes séculiers.}

\begin{singlequote}
François ne parle pratiquement jamais de l’économie contemporaine sans adresser une accusation d’idolâtrie, acccusation absente dans GS, et presque entièrement absente de Vatican II dans son ensemble.

Comment expliquer cette différence de traitement des questions économiques dans GS et chez le pape François ?

Ma thèse est que François représente une opportunité pour changer le discours catholique sur la sécularisation; une opportunité qui a des implications dans la manière de considérer non seulement l’économie, mais aussi d’autres phénomènes séculiers.

La pensée catholique progressiste dans la période du Concile Vatican II avait tendance à considérer le monde séculier comme désenchanté. François suggère au contraire, que nous ne sommes pas tant confronté à une perte de foi qu’à une nouvelle religion et un foi idolâtre. p. 126
\end{singlequote}


% -----------------------------------------------
\section{ouverture aux autres religions qui sont appelées à relever ensemble ce défi}



% -----------------------------------------------
\section{Paradoxe ?}
A la différence Dt, qui critiquait fortement les religions extérieures et toutes les compromissions, ici, il semble que nous ayons un paradoxe : positivité des religions non chrétienne et négativité des "compromissions" mais par rapport à une "religion non nommée".


\paragraph{est ce que Idolâtrie nous permet de creuser comment le dialogue inter religieux doit être pensé}

\paragraph{Dépasser une approche purement éthique} Kung

[critique du Manifeste pour une éthique planétaire de Kung]
\begin{singlequote}
la principale critique adressée aux théologies pluralistes, c’est leur prétention à disposer d’un lieu tiers, d’un arrière-plan qui se situerait au delà des religions particulières et à partir duquel on pourrait les embrasser toutes : le plan nouménal de la Rélité ultime pour John Hick, une même expérience mystique pour Raimon Pannikar ou encore un même projet éthique pour la justice et la gestion écologique des ressources de notre planete. \cite[p. 111]{cheno_dieu_2017} 
\end{singlequote}


\paragraph{Retrouver les points durs des autres religions pour nous aider à ne pas proposer une solution de type "Dieu et l'écologie"} mais bien penser comment s'articule l'un et l'autre. 


\paragraph{Par le thème d'idolâtrie, vocabulaire religieux}


\section{Points non intégrés à ce stade}
\paragraph{Et l'écologie comme religion scientifico ?} corpus de textes + transmission ?

% -----------------------------------------------

\begin{comment}


\paragraph{En quoi cela concerne le dialogue inter religieux} et pas uniquement Christianisme et écologie / ...
D'abord, Règne de Dieu. ensuite, intuition comme pour le dialogue inter-religieux que le Christianisme porte dans sa matrice une universalité et une hospitalité aux questions du temps, qui l'ouvre peut être de façon privilégiée à ces questions. 

\paragraph{Dialogue inter religieux et écologie} hypothèse que les religions peuvent aider à un effort maintenant pour un gain plus long, donner du sens, Règne de Dieu. Voir comment le Christianisme peut être pertinent sur le sujet.
Dialogue inter-religieux dans ce contexte. Sociologie des religions : permettant de valider cette hypothèse.
Règne de Dieu et Ecologie

\end{comment}



%-------------------------------------------------------------------------------------------------------
\section{Lecture de Laudato Si}
%-------------------------------------------------------------------------------------------------------



\paragraph{Résumé} \cite{francois_laudato_2015}
\begin{singlequote}
       La 4e de couverture indique : "Avec cette encyclique, le pape François s'adresse à tous les hommes de bonne volonté. Il les invite tous à un dialogue amical sur la crise écologique et sociale qui menace notre maison commune et il demande de suivre une voie conjointe pour répondre à ce défi mondial. Il ne s'agit pas de considérations théoriques avec quelques objectifs pratiques. Le pape ne veut pas seulement une amélioration dans des détails, mais une conversion fondamentale au vu de l'aggravation critique de la situation générale, qui ne permet plus d'esquive. Il s'agit de prendre conscience que nous habitons la même maison, donnée par Dieu, et que nous sommes les enfants de l'unique Créateur et Père des Cieux."
\end{singlequote}

\begin{singlequote}
        207. La Charte de la Terre nous invitait tous à tourner le dos à une étape d’autodestruction et à prendre un nouveau départ, mais nous n’avons pas encore développé une conscience universelle qui le rende possible. Voilà pourquoi j’ose proposer de nouveau ce beau défi : “Comme jamais auparavant dans l’histoire, notre destin commun nous invite à chercher un nouveau commencement [...] Faisons en sorte que notre époque soit reconnue dans l’histoire comme celle de l’éveil d’une nouvelle forme d’hommage à la vie, d’une ferme résolution d’atteindre la durabilité, de l’accélération de la lutte pour la justice et la paix et de l’heureuse célébration de la vie”.[148]
\end{singlequote}
       
\paragraph{Miser sur un autre style de vie}
\begin{singlequote}
        203. Étant donné que le marché tend à créer un mécanisme consumériste compulsif pour placer ses produits, les personnes finissent par être submergées, dans une spirale d’achats et de dépenses inutiles. Le consumérisme obsessif est le reflet subjectif du paradigme techno-économique. Il arrive ce que Romano Guardini signalait déjà : l’être humain « accepte les choses usuelles et les formes de la vie telles qu’elles lui sont imposées par les plans rationnels et les produits normalisés de la machine et, dans l’ensemble, il le fait avec l’impression que tout cela est raisonnable et juste ».[144] Ce paradigme fait croire à tous qu’ils sont libres, tant qu’ils ont une soi-disant liberté pour consommer, alors que ceux qui ont en réalité la liberté, ce sont ceux qui constituent la minorité en possession du pouvoir économique et financier. Dans cette équivoque, l’humanité postmoderne n’a pas trouvé une nouvelle conception d’elle-même qui puisse l’orienter, et ce manque d’identité est vécu avec angoisse. Nous possédons trop de moyens pour des fins limitées et rachitiques.

\end{singlequote}

        
\paragraph{les religions dans le dialogue avec les sciences}    
\begin{singlequote}
        199. On ne peut pas soutenir que les sciences empiriques expliquent complètement la vie, la structure de toutes les créatures et la réalité dans son ensemble. Cela serait outrepasser de façon indue leurs frontières méthodologiques limitées. Si on réfléchit dans ce cadre fermé, la sensibilité esthétique, la poésie, et même la capacité de la raison à percevoir le sens et la finalité des choses disparaissent.[141] Je veux rappeler que « les textes religieux classiques peuvent offrir une signification pour toutes les époques, et ont une force de motivation qui ouvre toujours de nouveaux horizons [...] Est-il raisonnable et intelligent de les reléguer dans l’obscurité, seulement du fait qu’ils proviennent d’un contexte de croyance religieuse ? ».[142] En réalité, il est naïf de penser que les principes éthiques puissent se présenter de manière purement abstraite, détachés de tout contexte, et le fait qu’ils apparaissent dans un langage religieux ne les prive pas de toute valeur dans le débat public. Les principes éthiques que la raison est capable de percevoir peuvent réapparaître toujours de manière différente et être exprimés dans des langages divers, y compris religieux.

        200. D’autre part, toute solution technique que les sciences prétendent apporter sera incapable de résoudre les graves problèmes du monde si l’humanité perd le cap, si l’on oublie les grandes motivations qui rendent possibles la cohabitation, le sacrifice, la bonté. De toute façon, il faudra inviter les croyants à être cohérents avec leur propre foi et à ne pas la contredire par leurs actions ; il faudra leur demander de s’ouvrir de nouveau à la grâce de Dieu et de puiser au plus profond de leurs propres convictions sur l’amour, la justice et la paix. Si une mauvaise compréhension de nos propres principes nous a parfois conduits à justifier le mauvais traitement de la nature, la domination despotique de l’être humain sur la création, ou les guerres, l’injustice et la violence, nous, les croyants, nous pouvons reconnaître que nous avons alors été infidèles au trésor de sagesse que nous devions garder. Souvent les limites culturelles des diverses époques ont conditionné cette conscience de leur propre héritage éthique et spirituel, mais c’est précisément le retour à leurs sources qui permet aux religions de mieux répondre aux nécessités actuelles.

        201. La majorité des habitants de la planète se déclare croyante, et cela devrait inciter les religions à entrer dans un dialogue en vue de la sauvegarde de la nature, de la défense des pauvres, de la construction de réseaux de respect et de fraternité. Un dialogue entre les sciences elles-mêmes est aussi nécessaire parce que chacune a l’habitude de s’enfermer dans les limites de son propre langage, et la spécialisation a tendance à devenir isolement et absolutisation du savoir de chacun. Cela empêche d’affronter convenablement les problèmes de l’environnement. Un dialogue ouvert et respectueux devient aussi nécessaire entre les différents mouvements écologistes, où les luttes idéologiques ne manquent pas. La gravité de la crise écologique exige que tous nous pensions au bien commun et avancions sur un chemin de dialogue qui demande patience, ascèse et générosité, nous souvenant toujours que « la réalité est supérieure à l’idée ».[143]
\end{singlequote}
       

%-------------------------------------------------------------------------------------------------------
\section{Bibliographie - Laudato Si}
%-------------------------------------------------------------------------------------------------------

%-------------------------------------------------------------------------------------------------------
\paragraph{Colloque RSR Ecologie} \cite{goujon_laudato_2022}



%-------------------------------------------------------------------------------------------------------
\subsection{Bibliographie - Vision Catholique de la crise Ecologie}


%-------------------------------------------------------------------------------------------------------
\paragraph{Quelques mots avant l'Apocalypse}\cite{candiard_quelques_2022}

Risque de l'écologie philosophique qui pense la décroissance mais ne sait plus pour quel but ? 


\begin{singlequote}
        Le progrès scientifique et technique exceptionnel que nous avons connu ces derniers siècles, en particulier les réussites incontestables de la médecine, confirmait cette vision des choses, comme la remarquable expansion économique qui l’a accompagné, dont nous commençons tout juste à comprendre qu’elle comporte aussi des effets délétères. nous savions naturellement que tout n’allait pas bien , mais nous pouvions croire cependant que les choses s’amélioraient.

        Ces schémas de pensée sont si naturellement ancrés, ils informent si puissamment notre appréhension du réel que nous avons du mal à y renoncer tout à fait devant les démentis flagrants que nous offre l’actualité. Nous restons, volontairement ou non, consciemment ou non, orphelins des mythes du progrès, et nous serions bien contents de leur trouver un substitut chrétien, une garantie divine que, malgré quelques péripéties, tout ira pour le mieux.

        Ne serait-ce pas un juste retour des choses puisque de l’avis général, ces philosophies de l’histoire auraient simplement transposé sur terre une espérance chrétienne, laicisé la foi au salut et au paradis ? \cite[pp 89-91]{candiard_quelques_2022}
\end{singlequote}

\begin{singlequote}
du mal qui est censé en supporter les inconvénients, sans quoi il n'y a plus de justice. Si l'accident de voiture que cause mon imprudence blesse ou tue un innocent, impossible de parler de justice. Dans notre cas à nous, les conséquences apocalyptiques du péché ne sont pas justes, car elles ne frappent pas spécialement les pécheurs et, moins encore, à proportion du péché. Un pacifiste n'est pas moins menacé par la destruction nucléaire qu'un dictateur. Impossible de dire à un paysan philippin qui a dû quitter sa terre à cause d'inondations dramatiques que c'est après tout de sa faute, et qu'il aurait dû moins polluer, car il fait face, en réalité, à une véritable injustice immanente: il assume les conséquences terribles d'actions dont il n'est nullement responsable.

Cette responsabilité est d'autant moins personnelle, et la conséquence d'autant moins juste, qu'une conséquence en entraîne une autre, en chaîne, de manière souvent imprévisible et surtout Si Jésus tient un discours d'apocalypse, de révélation, ce n'est pas pour nous terrifier plus ou moins utilement, mais bien pour nous faire comprendre ce qui se joue sous nos yeux: non la punition divine des fautes de l'homme, mais le déploiement du mal et de ses conséquences destructrices; autrement dit, la fin des temps à l'œuvre, non comme événement inquiétant dont on attendrait la proximité, mais comme cette réalité présente dans l'histoire depuis son début, véritable trame sous-jacente aux événements du monde. Nous avons besoin de ce dévoilement car, tant que la nature du mal restera inconnue, on pourra croire béatement à l'efficacité de solutions purement techniques aux mena ces qui pèsent sur nos existences. Il est sans doute nécessaire, dans bien des domaines, d'améliorer la législation, de modifier nos modes d'organisation, de négocier la réduction des arsenaux nucléaires ou celle des émissions de gaz polluants, de faire évoluer les opinions publiques; l'engagement politique ou l'action associative peuvent être  inexorable, loin de toute volonté consciente initiale: nous ne maîtrisons pas nos propres catastrophes.
\end{singlequote}

 
\begin{singlequote}
        
des voies nobles et utiles pour rendre meilleure la vie de tous. Il serait naïf, évidemment, de prétendre lutter contre les désastres climatiques en s'en remettant à la seule prière, mais il ne serait pas moins naïf d'imaginer vaincre le mal sans s'attaquer à ses causes, et d'oublier que le premier lieu où je peux envisager de le déraciner, c'est dans ma propre vie.

En moi, le combat eschatologique est déjà engagé, avec sa violence et ses incertitudes :

c'est lui qui est à l'œuvre dans mes crampes d’égoïsme et dans mes envies de bien faire, dans mes fidélités et mes impatiences. Et dans ce combat, l'emporter, c'est d'abord accepter que la victoire a déjà été acquise, non pas par mes efforts, mais par l'amour infini qui se donne à voir dans la croix de Jésus et qu'il me faut, peu à peu, laisser entrer dans ma propre vie.
\end{singlequote}



%-------------------------------------------------------------------------------------------------------
\section{Bibliographie - Idolatrie}
%-------------------------------------------------------------------------------------------------------


%-------------------------------------------------------------------------------------------------------    
\paragraph{Economie, idolâtrie et sécularisation depuis Gaudium \& Spes}
    \cite{cavanaugh_idolatrie_2022}
    Résumé 	La sacralisation de l'argent, du pouvoir et par là même de l'individu est au fondement des multiples crises que nous traversons aujourd'hui. L'Église a un rôle déterminant à jouer pour parer à toute tentation idolâtrique, y compris en son propre sein. En prenant appui sur l'Évangile, la Tradition et les témoins de la foi, le peuple des baptisés a le devoir de débusquer, dans nos sociétés sécularisées, tout ce qui peut blesser l'image divine inscrite au coeur de chaque être humain. Par une réflexion théologique originale, William Cavanaugh nous aide à retrouver l'adoration véritable qui permet d'accéder à la liberté authentique des enfants de Dieu
    
    Notes :
\begin{singlequote}
François ne parle pratiquement jamais de l’économie contemporaine sans adresser une accusation d’idolâtrie, accusation absente dans GS, et presque entièrement absente de Vatican II dans son ensemble.

Comment expliquer cette différence de traitement des questions économiques dans GS et chez le pape François ?

Ma thèse est que François représente une opportunité pour changer le discours catholique sur la sécularisation; une opportunité qui a des implications dans la manière de considérer non seulement l’économie, mais aussi d’autres phénomènes séculiers.

La pensée catholique progressiste dans la période du Concile Vatican II avait tendance à considérer le monde séculier comme désenchanté. François suggère au contraire, que nous ne sommes pas tant confronté à une perte de foi qu’à une nouvelle religion et un foi idolâtre. p. 126
\end{singlequote}

        pour un nouvel Dt, choisis la vie.

        voir p 127 l’économie dans Gaudium \& Spes

GS ce qui doit changer dans la société p 131        
\begin{singlequote}
Le passage sur l'économie [de GS] a également été critiqué pour son moralisme et son incapacité à requérir l'avis de professionnels dans le domaine de l'économie. Mais les Pères concilaires n'avaient pas l'intention de faire davantage qeu de fournir des "principes de justice et d'équité, demandés par la droite raison" 132


L'autonomie et la rationalité de la sphère économique sont impliquéees, plus en amont dans la constitution, par l'affirmation de l'"autonomie des réalités terrestres" et de l'autonomie des sciences" (n°36), ce qui inclut vraisemblablement la science économique. p132

\end{singlequote}        

\subparagraph{Pape François et l'économie}
lumen fidei : quatorze fois idole. le contraire de la foi n'est pas un manque de foi mais bien l'idolâtrie. Quand on cesse de croire en Dieu, on ne cesse pas simplement de croire, mais on croit à autre chose.  l'idolâtrie et toujours un polytheisme; un mouvement sans but qui va d'un seigneur à l'autre. \textbf{"confiance".}
 

%-------------------------------------------------------------------------------------------------------
\section{Bibliographie - Théologie des religions}
%-------------------------------------------------------------------------------------------------------


%-------------------------------------------------------------------------------------------------------
\section{Bibliographie - Théologie des religions et Ecologie}
%-------------------------------------------------------------------------------------------------------





  
%-------------------------------------------------------------------------------------------------------
\section{Bibliographie - Dialogue Inter-religieux et Ecologie}
%-------------------------------------------------------------------------------------------------------

%------------------------------------------------------------------------------------------------------- 
\paragraph{Ecologie en Islam et Dialogue Interreligieux} \cite{pisani_ecologie_2016} 
    Toynbee montrait aussi que c’est au contact les unes des autres, dans l’interaction de leurs mythes et de leurs théologies que se créent les conditions de l’avenir.
    Laudato Si : l'enjeu nécessite le concours de tous.
   
    importance de la situation de ‘Alī al-Ḫawwāṣ dans LS (reconnaissance de l'héritage écologique de l'Islam).     
    \textit{habitus écologique} : définies par des rites ou s'affirment des attitudes singulières ou s'entremelent monde spirituel et monde matériel. 
    Une réponse : l’islam est la solution (si crise, c'est qu'on n'est pas assez islam; pas d'ouverture aux autres religions).
    
    fitra : revenir à un état originaire. la nature vrai musulman, ne se rebelle pas.
    
 
    

  




\paragraph{Dieu au Pluriel.  Chapitre sur l'approche culturo-linguistique}    
        \cite{cheno_dieu_2017}

    Notes :
\begin{singlequote}
        
        la principale critique adressée aux théologies pluralistes, c’est leur prétention à disposer d’un lieu tiers, d’un arrière-plan qui se situerait au delà des religions particulières et à partir duquel on pourrait les embrasser toutes : le plan nouménal de la Rélité ultime pour John Hick, une même expérience mystique pour Raimon Pannikar ou encore un même projet éthique pour la justice et la gestion écologique des ressources de notre planete. [critique du Manifeste pour une éthique planétaire de Kung] p. 111
\end{singlequote}

        impossible car
\begin{singlequote}
        Nous sommes des humains, insérés dans une culture et des pratiques qui nous façonnent. p

\end{singlequote}

        Le modèle culturo-linguistique
\paragraph{The Nature of Doctrine} \cite{lindbeck_nature_2002}

    
\paragraph{De quel genre de pensée a-t-on besoin pour aborder la crise environnementale contemporaine ? }
        \cite{howles_quel_2022}
Résumé   	
\begin{singlequote}

L'écologie politique contemporaine donne un nouvel infléchissement aux débats environnementaux qui risquent, sinon, de rester bloqués dans un paradigme réducteur et moderniste. Il est intéressant de noter que cette nouvelle écologie politique s'inspire de plus en plus du langage et de concepts théologiques, en particulier dans l'œuvre de Bruno Latour. Le présent article explore les raisons pour lesquelles il en est ainsi et quelle contribution cette approche peut apporter. L'écologie politique assigne un rôle à la religion en ce que celle-ci génère le genre de conversion aux valeurs humaines qui s'avèrent nécessaires pour une véritable transformation sociétale. En procédant ainsi, l'écologie politique pourrait même être considérée comme un partenaire de dialogue (surprenant) pour la théologie catholique et pour des approches de la crise environnementale qui s'appuient plus largement sur la tradition de l'enseignement social catholique.
\end{singlequote}
 
Notes :

\begin{singlequote}

        À première vue, la popularité et la diffusion de ces idées est peut-être surprenante, car dans leur engagement radical à faire exploser la dichotomie moderniste supposée des humains et de la nature, ces théoriciens compliquent la compréhension du rôle de l’action humaine pour faire face à la crise planétaire à laquelle nous sommes confrontés. Ils sont notamment pessimistes quant au potentiel des réponses managériales ou technologiques d’origine humaine et visent fréquemment des partisans de l’« écomodernisme » et ceux qui proposent des stratégies grâce auxquelles les sociétés humaines pourraient envisager d’atteindre un « bon Anthropocène » [9]. 
\end{singlequote}

        Doctrine sociale de l’Eglise : positif sur le role des acteurs

        ecomodernisme : progrès
\begin{singlequote}

        Cependant, pour la nouvelle écologie politique, cette sorte d’écomodernisme est une simple réaffirmation du dualisme de l’être humain face à un monde de la nature passif, non animé et inerte, dans l’attente que grâce à son ingéniosité il soit en mesure de dominer et de maîtriser ce monde. Malgré de bonnes intentions individuelles dans des situations particulières, cela mène invariablement à la perpétuation du paradigme technocratique moderniste et ne réussit pas à prendre en compte ce que le pape François appelle « les racines humaines de la crise écologique » [11]. C’est pourquoi la nouvelle écologie politique rejette entièrement ce paradigme.
\end{singlequote}

        {une critique du judeochristianisme, responsable du modernisme.}
\begin{singlequote}

        Fredric Jameson fait remarquer avec humour que « de nos jours il semble plus aisé d’imaginer la fin du monde que celle du capitalisme » [14]. Bruno Latour considère que cette attitude s’appuie sur des idées religieuses de providence et d’achèvement eschatologique, où le but et point final de l’histoire est décrété d’avance et où les croyants sont invités à structurer en conséquence leurs choix personnels et leurs décisions. Cela a, selon lui, un effet démobilisateur sur les énergies politiques qu’il considère nécessaires pour une action environnementale radicale, révolutionnaire et efficace aujourd’hui.
\end{singlequote}

        Latour : sous jacent du film dystopique sur l’asteroide : quelques personnes font le salut des autres (geo ingénierie” : alpocalypse pour les autres
\begin{singlequote}

        Il existe de très nombreux travaux relatifs à la théorie de Gaïa, des points de vue scientifiques comme non scientifiques, à la fois positifs et critiques vis-à-vis de sa méthodologie et de son potentiel explicatif [20]. Il est certain que nombre des tenants de l’écologie politique en question lisent Lovelock en non-spécialistes et l’abordent « d’une manière enthousiaste mais créative », comme le formule un commentateur [21]. Néanmoins, le concept de Gaïa leur sert d’outil pour décrire un système vraiment immanent qui n’est pas dirigé ou piloté par une force ou un acteur externe.

        La valence écologique de l’idée de Lovelock réside dans la manière dont elle rappelle aux êtres humains leur statut d’acteurs opérant au sein des systèmes écologiques de la planète Terre. Cette approche implique au moins deux avantages pour l’écologie politique. D’abord, elle réfute la logique écomoderniste qui situe l’acteur humain pour ainsi dire à l’extérieur de ce système, avec la fausse assurance qu’il est équipé pour le gouverner lui-même de manière quasi divine. Car la pensée Gaïa stipule qu’un système stable de la Terre est la fonction de processus multiples, intimement imbriqués, mais non régis de l’extérieur. Bruno Latour le formule de la manière suivante : « Je pense que c’est cela que Lovelock laisse sous-entendre en quelque sorte, à savoir que la Terre est connectée. Chaque élément des entités qui la constituent construit son propre environnement, mais il n’y a pas d’“organisateur” ». Il n’y a pas de Dieu, en fin de compte » [23]. Il n’est donc pas étonnant que Latour considère Gaïa comme un outil « séculier » pour réorganiser la politique, loin de l’impasse théologique dans laquelle elle a été entraînée jusqu’ici [24], ce qui fait progresser, en ce sens, « l’intuition […] entièrement séculière de Lovelock » [25]. 

        Le second atout de l’idée de Lovelock, aux yeux des penseurs écologiques, consiste en ce qu’elle invite à un mode d’action adéquatement responsable. Isabelle Stengers souligne que comprendre Gaïa c’est se rendre compte que la Terre elle-même est devenue « chatouilleuse », « susceptible » d’une certaine manière, dans le sens où ses mécanismes homéostatiques finement accordés ne sont pas indépendants de nos actions, et que nous ne pouvons pas non plus présumer qu’ils rebondiront en réponse au stress que nous leur imposons [26]. Nous avons beau être une partie mineure d’un système plus large, il n’en reste pas moins que notre empreinte environnementale particulière importe vraiment, qu’elle soit grande ou petite.
    \end{singlequote}

 \paragraph{Kung : éthique planétaire}
 \cite{kung_lethique_2009}
{Résumé}
\begin{singlequote}
Le projet d’éthique planétaire se situe dans la foulée de l’éthique de la responsabilité de Max Weber. Il propose une fondation rationnelle de l’éthique (voir K.-O. Apel et J. Habermas). L’être humain jouit d’une autonomie intramondaine mais ne peut fonder seul l’universalité de l’obligation éthique. Onze thèses fondatrices sont alors énoncées comme, par exemple : le jeu a besoin de règles; le fair-play suppose l’observation de normes; éthique n’équivaut pas à doctrine sociale mais à conscience, conviction et attitudes morales ; les règles éthiques peuvent être fondées à partir de la raison sans référence transcendante, etc.
        
\end{singlequote}
    
Notes :
\begin{singlequote}

9 – Une argumentation rationnelle abstraite ne parvient que difficilement à convaincre des gens de différentes cultures et de différents milieux

John Rawls déduit des règles éthiques de principes généraux de justice, entendue comme fair-play, abstraite consciemment de tout contexte concret et de toute situation. Mais ce n’est qu’une idée étendue de justice qui lui permet après coup de développer une conception du droit et de la justice qui puisse aussi s’appliquer aux principes et aux normes du droit international et des relations internationales.

L’éthique de la discussion de Karl-Otto Apel et de Jürgen Habermas insiste à juste titre sur l’importance du consensus rationnel et de la discussion. Ils prétendent en cela pouvoir formuler des normes, dans une moindre dépendance à l’égard du contexte, qui vaillent inconditionnellement, et ce à partir de la communauté humaine de communication et d’argumentation. Les principes religieux et les interprétations religieuses de la morale, dévalorisés face à l’espace public, doivent être remplacés par une discussion rationnelle, par un jeu de langage régulé, par la « contrainte de l’argument sans contrainte ». Compte tenu de la réalité concrète de la vie, il est discutable qu’on puisse atteindre un éthos global (pour ainsi dire jusqu’au dernier village indien ou africain), qui soit réellement obligatoire et contraignant, à l’aide d’une discussion rationnelle abstraite.
\end{singlequote}

\begin{singlequote}
        11 – Les traditions religieuses ne doivent pas être objet de mépris, mais de réflexion critique

        L’anthropologie culturelle nous l’enseigne : les normes éthiques concrètes, les valeurs et les intuitions éthiques se sont développées graduellement, selon un processus socio-dynamique extrêmement complexe.

        Selon que des besoins vitaux apparaissaient, que des urgences et des nécessités entre humains se manifestaient, dès le début il a fallu des orientations et des régulations de l’action : des conventions déterminées, des sagesses, des mœurs, bref des critères éthiques, des règles, des normes, qui au cours des siècles et des millénaires ont été éprouvés. En cela, il est frappant que certaines normes éthiques se ressemblent partout dans le monde. Mais c’est un fait historique : à travers des millénaires, ce sont les religions qui fournirent des systèmes d’orientation, qui formèrent les bases d’une certaine morale, qui les légitimèrent, et qui souvent sanctionnèrent les déviations par des peines.

        À l’origine, la philosophie et la religion, la philosophie et la théologie ont plutôt vécu en symbiose ; celle-ci ne peut plus être rétablie. Mais une coopération plus intensive est recommandable, en vue d’une même vision d’espérance : “To make the world a better place”. Pour réaliser cette espérance, il faut au principe un changement de conscience vers un éthos humain, au service d’une culture de la non-violence et d’un respect de toute vie, de la solidarité et d’un ordre économique juste, de la tolérance et de la vie en vérité, d’une égalité des droits et d’un partenariat entre hommes et femmes.
\end{singlequote}

\subparagraph{approche philosophique}        

        eviter le recours au concept de loi naturelle
\begin{singlequote}
        
        Il faut plutôt chercher à atténuer, par une solution pragmatique de problèmes urgents, les oppositions entre visions du monde, sans tenir compte des différences idéologiques :

            Il était clair dès le départ que le projet « Éthique planétaire » se situait dans la ligne de l’éthique de responsabilité de Max Weber, qui ne met pas entre parenthèses l’orientation juste, mais veut concentrer l’attention sur les conséquences raisonnables. Jean-Paul Sartre voyait déjà que le domaine de validité de la responsabilité s’étendait à tout le monde des humains. Hans Jonas l’a étendue à toute la biosphère, appelant à réfléchir aux conséquences dangereuses de l’agir, y compris pour les générations à venir. En cela, le « principe responsabilité » de Jonas et le « principe espérance » d’Ernst Bloch ne s’excluent pas. Emmanuel Levinas a fait ressortir que la situation entre les humains est au centre de cette responsabilité, et qu’il faut prêter attention à l’altérité de l’autre, comme à son caractère étranger, et qu’il faut avoir des égards même pour les ennemis. À la recherche de compétences pour une communication responsable, Hannah Arendt a encouragé en particulier une façon de penser plus large, un imaginaire et un sens commun, et – ultimement – elle a mis avant tout en relief la vertu de vérité, qui s’efforce d’atteindre la vérité des faits, car naturellement, sans elle, il n’y a aucune communication ouverte possible entre les hommes.

            Il faut plutôt chercher à atténuer, par une solution pragmatique de problèmes urgents, les oppositions entre visions du monde, sans tenir compte des différences idéologiques : cela pourrait à long terme établir des points communs, y compris justement un éthos commun. Le conflit des visions du monde ou des idéologies devrait être apaisé de cette façon.
\end{singlequote}

\paragraph{Manifeste pour une éthique planétaire}
    \cite{kuschel_manifeste_1995}
    Notes :

        explication de Hans Kung

        sur la difficulté du genre, pas casuistique (mais rappeler l’importance e la vie), pas un énoncé des droits universels (si on n’a rien à dire,…), pas une dissertation philosophique. Autocritique
\begin{singlequote}
En Allemand, la déclaration portera le titre de déclaration pour un éthos planétaire, non pas pour une éthique planétaire. Ethos désigne la disposition morale fondamentale de l’homme, tandis qu’éthique nome la doctrine (philosophique ou théologique) concernant les dispositions, valeurs et normes morales.P60
\end{singlequote}

        Un texte “moderne” : non européocentré (cf discussion sur le nom de “Dieu”)

        mais le risque de tomber dans un PGCD est réel. Est ce que les religions font tout en passant par cette déclaration ?

        Parti pris que les religions ont chacune à travailler directement le sujet.

        une approche plraliste
\begin{singlequote}
éthique (ou ethos) planétaire, c’est à dire un accord fondamental en matière d’axiologie, de critères indiscutables et de choix essentiels. A défaut d’un consensus éthique fondamental, toute communauté court tôt ou tard le risque du chaos ou de la dictature.  Un ordre mondial meilleur ne peut se concevoir sans éthos planétaire

        (préface. p6)

\end{singlequote}

        MAIS AUSSITOT
\begin{singlequote}
“ethique planétaire ne signifie ni idélogie planétaire, ni religion mondiale unitaire à côté des religions existantes, ni quelque forme syncrétique de toutes les autres religions. Notre humanité est lasse des idéologies unitaires et les diverses religions du monde sont de toute manière si différentes dans l’expression de leurs croyance et dans leurs dogmes, dans leur symbolique et leurs rites, que tout effort d’”unification” est dénué de sens. P 6

\end{singlequote}


“ce manifeste … point de départ”P 7

\begin{singlequote}

Il nous faut tendre à l’instaration d’un ordre social et économique juste, au sein duquel chacun jouise de chances égales, au bénéfice de toutes ses possibilités humaines. P14

        Il est illusoire de vouloir rendre cette planète meilleure, sans changer d’abord la conscience des individus. Nous nous engageons dès lors à élargir notre capacité de perception, en acceptant pour nos esprits la discipline de la méditation, de la prière et de la réflexion. Refuser aucun risque ou aucun sacrifice revient à empêcher toute mutation sensible de notre situation présente. C’est pourquoi nous nus engageons à vivre selon cette éthique planétaire, dans l’intelligence réciproque, et à respecter entre nus des modes d’existence propres à promouvoir la tolérance mutuelle, la paix sociale et internationale et le respect de la nature. P 14

        Un pricnipe se retrouve depuis des milliers d’années dans beaucoup de traditions religieues et éthiques de l’humanité qui l’ont conservé, c’est la “règle d’or”; ce que tu ne veux pas qu’on fasse à ton endroit, ne le fais pas à l’endroit d’aucun autre. P 23
\end{singlequote}
        
        
\subsection{MOOC Bernardin}

 	SEMAINE 6
Habiter la maison commune
 
par Fabien Revol

	Les actualités du MOOC :
o	La séance 5 a montré comment l'écologie de la vie quotidienne correspond à une écologie humaine.
o	Poursuivez avec la séance 6 sur les grands concepts de l'écologie intégrale !

\href{https://www.lecampusdesbernardins.fr/resource/26/?utm_source=sendinblue&utm_campaign=20221107_NL_MOOC_sans_dons&utm_medium=email}{lien avec le mooc}

\section{Actualité}
\paragraph{13 novembre 2022 - en marge de la COP 27}
L’archevêque anglican Rowan Williams conduit des chefs religieux sur la colline du Parlement pour une cérémonie de repentance climatique le 13 novembre à Londres. -
 
\begin{singlequote}
    « La majorité des habitants de la planète se déclare croyante, et cela devrait inciter les religions à entrer dans un dialogue en vue de la sauvegarde de la nature. » (Pape François)
\end{singlequote}


\subsection{Ecouter Texte IDEO}
 %\chapter{coupes divinatoires}


\paragraph{Coupe divinatoire} dit bol \textit{magique} en alliage de cuivre partiellement étamé, de forme circulaire aux bords évasés. La paroi extérieure est gravée d'une fleur de lotus à huit pétales calligraphiés au coeur en forme d'une étoile à cinq branches, et une longue inscription le long du rebord externe. L'intérieur est décoré d'un cartouche inscrit, de deux motifs de tughra, d'un sceau de propriétaire en forme d'amande «sahib Tador (Théodore ?) «et d'un cachet ARMENIEN daté 1875». Le rebord est gravé d'une longue frise épigraphique sur deux lignes. Empire ottoman, datée 1875.
Haut. : 4,9 ; Diam. : 15,1 cm

\includegraphics[width=\textwidth]{GénéralISTR/Image/bolsmagiques.jpeg}

\includegraphics[]{GénéralISTR/Image/Tughra_of_Abdülaziz 1861-76.jpeg}

\paragraph{fleur de lotus} Dans le bouddhisme, le lotus est associé à la pureté, à l’éveil spirituel et à la fidélité. La fleur est considérée comme pure car elle est capable de sortir des eaux troubles le matin et d’être parfaitement propre. Il est également connu pour symboliser la pureté de la parole, du corps et de l’esprit.


\paragraph{Bol talismanique ou bol magique} coupe de forme circulaire à bords évasés, ombiliquée en laiton anciennement étamé incrustation de pâte noire gravé à l'intérieur de mihrabs, inscriptions en écriture naskh\sn{Le naskh, aussi appelé naskhi ou neskhi  est le style d'écriture le plus répandu pour les langues utilisant l'alphabet arabe. C'est ce style que l'on apprend à l'école et que l'on emploie pour la calligraphie et l'écriture usuelle, manuscrite ou imprimée.} et nasta'liq\sn{Le nastaliq    est un des styles de la calligraphie persane, en alphabet persan, dont l'origine est attribuée à Mir Ali Tabrizi, originaire de Tabriz, au xive siècle} d'invocation religieuse et verset coranique, patine d'usage. Iran XIX-XXe.
Haut. : 4 ; Diam. : 13,5 cm




\paragraph{Duquoc} Nous nous trouvons là devant une ferme conviction : « Jésus ne se replie pas sur l’instant, il l’ouvre à sa profondeur », car le présent est « l’habitat de Dieu » (p. 113). L’auteur revient sur ce point avec insistance, comme pour surmonter le caractère paradoxal des affirmations du Christ, puisque si le Règne est là, le mal et l’injustice, eux aussi, semblent continuer de régner, inexorablement. La « tentation » est alors grande de vouloir faire coïncider le Règne présent avec la fin d’une histoire réconciliée qui dénie les réalités de notre histoire souffrante, en faisant appel à la magie, à la puissance messianique, à la conquête instaurant de force l’unité du Règne promis (comme le suggère le tentateur à Jésus au seuil de sa mission). L’Église n’a pas toujours résisté à cette tentation, comme on le voit lorsqu’elle cherche à dominer et à orienter la politique, illusion dénommée chrétienté (l’auteur développe ici sa réflexion en tirant profit du livre de Marcel Gauchet : La religion dans la démocratie. Parcours de la laïcité, Gallimard, Paris, 1998).  

\section{Dictionnaire sociologie religions}

 
 du prophétisme
Du latin divinatio, deviner, le terme « divination » désigne l'action qui se donne pour objectif de deviner, prévoir et/ou influencer
une réalité cachée, à l'aide de la lecture d'éléments, ou présages, selon une technique particulière impliquant leur observation ou leur manipulation. Ainsi définie, la divination apparaît comme une pratique universelle des sociétés  Les procédures divinatoires peuvent être motivées par diverses raisons découvrir les causes d'une maladie ou d'une infortune, retrouver un objet perdu, connaître les déterminations pesant sur l'avenir proche (dans le domaine de l'amour, du travail, etc.), être informé des circonstances propices à la réalisation d'une action ainsi que de ses chances de succès (départ à la guerre ou à la chasse, construction d'une maison, installation sur un territoire, etc.). La divination peut aussi être exécutée de manière quasi automatique, parce qu'elle est requise en telle circonstance par la tradition ou qu'elle fait partie intégrante d'un rituel. La divination est, soit explicative   et renvoie alors à des éléments passés, soit prédictive permettant de
connaître l'avenir de sorte à agir en consé-quence. Elle peut être réalisée pour le compte d'un individu, d'un groupe, ou d'un « bien » (du bétail, par exemple).
Les éléments qui servent de présages sont variés et leur lecture est susceptible d'être effectuée selon des procédés très divers : observations et/ou manipulations d'entrailles d'animaux sacrifiés, de vols d'oiseaux, de craquelures sur une carapace de tortue calcinée, de marc de café, de combinaisons d'objets lancés, de cartes, de présence de taches dans un jaune d'œuf, de l'emplacement des planètes, etc. Cette multiplicité des techniques s'illustre par la diversité des termes formés à partir de la racine grec \textit{mantiké} -divination-pour les désigner : chiromancie, géomancie, cartomancie, etc. La divination peut aussi être opérée sans autre support que le devin lui-même. Elle s'inscrit alors plutôt dans le domaine de la voyance : le devin établit un contact qui est dit direct -via les rêves, la transe ou la possession- avec des forces sur-naturelles. Mais si de multiples classifications des formes de divination ont été proposées formes ou techniques intuitives, inspirées, inductives, raisonnées..), aucune n'apparaît universellement pertinente, et il semble plus utile de se référer aux classifications indigènes
- les modes de divination étant généralement pluriels dans une même société.

En se fondant sur la combinaison, qui
nés à semble résulter du hasard, des éléments qui tion- servent de présages, la procédure divinatoire incie, manipule l'aléatoire pour mieux le dépasser.
si être
Une signification étant attribuée à chacune
 des combinaisons possibles. le hasard est   exclu de la trame des événements. L'incertitude qui motive la divination se trouve alors   intégrée dans un ordre intelligible et
  rassurant.
 
De ce fait, l'analyse des processus divins osées   informe sur les conceptions indigens
 , paraît
de la causalité. Les résultats de la divination
: plus peuvent renvoyer à un système de correspons
 
dances, permettant de penser que les procé
dures divinatoires relèvent d'un processus de rationalisation; elles activent un ordre dast licatoire et les systemes de correspondants


l'implique pas nécessairement, et le cours du destin peut toujours être plus ou moins « forcé ». Les processus divinatoires prédictifs servent d'ailleurs souvent à se prémunir contre un événement futur potentiel, voire à créer un autre futur : il est fréquent de recommencer une divination jusqu'à obtenir le résultat escompté, comme s'il s'agissait de mettre en place les conditions propices à la réussite de l'action que l'on souhaite entre-prendre. De ce fait, la divination intéresse les réflexions sur les processus de prise de décision et les théories de l'action. Entérinant certaines formes d'actions, elle œuvre comme une procédure de validation, ce qui la rapproche des systèmes juridiques.
La divination relève en fin de compte autant de l'individuel que du social. Elle est le résultat d'une interaction, impliquant au minimum le devin et son client, et généralement tout un pan de la société. En inscrivant le destin individuel dans un ordre englobant, les processus divinatoires font de l'individu un élément d'un système socio-cosmique qui le dépasse. Entre l'individu et le social, l'aléatoire et la détermination, l'interprétation et l'action, le religieux et le juridique, les processus divinatoires, variés dans leurs formes et leurs fonctions, touchent à de nombreux aspects de la vie d'une société, qu'ils permettent d'articuler.
• ADLER A. et ZEMPLENI A., Le Bâton de l'aveugle : divination, maladie et pouvoir chez les Moundang du Tchad, Paris, Hermann « Savoir », 1972. - CAQUoT A.
et LEIBOVICI M. (éds), La Divination, Paris, PUF,
1968. - EVANS-PRITCHARD E. E., Sorcellerie, oracles et magie chez les Azandé (1937), traduction ft. L.
\textbf{Evrard, Paris, Gallimard, 1972. - FAHD T., La Divination arabe: études religieuses, sociologiques et folkloriques sur le milieu natif d'Islam, Leyde, Brill, 1966.} \href{https://books.google.fr/books?id=ETsVAAAAIAAJ&lpg=PP1&hl=fr&pg=PA8#v=onepage&q&f=false}{La divination Arabe}
MAUPOIL B., La Géomancie à l'ancienne côte des esclaves, Paris, Institut d'Ethnologie, 1941, - PEEK P.-
M., African Divination Systems: Ways of Knowing.
Bloomington, Indiana University. Press, 1991.
VERNANT J.-P, VANDERMEERSCH L.. GERNET J., BOTTERO J. et al, Divination et rationalité, Paris, Seuil « Recherches Anthropologiques », 1974.
Grégoire SCHLEMMER
316



 
\section{Deux coupes magico-thérapeutiques, biens de fondation pieuse (Nord du Yémen) : transmission du savoir et efficacité}
 
Deux coupes magico-thérapeutiques (sing. \textit{tâsa}) sont dites biens de fondation pieuse ({\textit{waqf}}) d'une mosquée, à Sanaa. Elles appartiennent à une collection d'objets, tous ayant le même statut juridique et tous utilise à des fins thérapeutiques. La famille du responsable de l'intendance à la mosquée, a la garde de ces objets et doit les remettre à quiconque les réclame dans un dessein, bien sûr, thérapeutique. Le statut juridique précis de ces objets, inédit, soulève de nombreuses questions, et, en particulier, pour le domaine ici couvert, celle de savoir ce qui les rend efficaces.
Afin de me situer dans le cadre thématique de cet ouvrage, je me pencherai sur les différentes inscriptions, représentations et figures géométriques gravées sur les parois des deux coupes. Différentes réflexions sur l'usage et la fabrication des coupes en général, à partir des travaux existants, amèneront à se demander dans quelle mesure il est possible de parler de «coupes talismaniques ».
Cette étude a fait l'objet d'un terrain entre les années 1995 et 1998.
Nous ne disposons que de peu de descriptions de coupes se trouvant au Yémen ou ayant un lien avec ce pays, par rapport à la quantité d importante de travaux publiés sur ces objets.
\subsection{Description des coupes}


La première coupe A, hémisphérique, sans pied, à lèvre arrondie, réalisée dans un alliage cuivreux. La surélévation qu'on remarque en so centre semble résulter d'un choc. Ses dimensions, estimées d'après phote sont pour sa hauteur, de 7 à 8 cm environ, et pour son diamètre, de 15 16 cm environ*. Elle est antérieure à 1313/1895-96 (date de mise en \textit{waqf}). Elle porte sur ses parois interne et externe un décor incisé de façon assez sommaire, et a peut-être été rapportée du pèlerinage à La Mecque par son donateur.


Le décor de la paroi interne est organisé à partir du motif central qu'orne le fond de la coupe. Il présente deux carrés entrecroisés dont les hui pointes, prolongées à 45° par des droites, se rattachent à un cercle situé quelques centimètres de la lèvre, formant ainsi un motif structurant analogue à celui d'une roue. Au centre de la coupe, dans l'espace circonscrit par les deux carrés. se trouvent. gravés sur huit lignes, des bâtonnets, isolés les uns des autres et saturant l'espace. L'espace résiduel entre les pointes et les côtés des carrés est occupé par des marques, reprenant sans doute les bâtonnets. Huit médaillons occupent le registre intermédiaire, structuré par les pointes développées des deux carrés. On distinguera quatre médaillons, dans la moitié inférieure desquels figurent des représentations animalières et, dans la moitié supérieure, des bâtonnets, semblables à ceux qui viennent d'être décrits. Ces quatre médaillons alternent avec quatre autres, comportant des textes en arabe. Dans le premier type de médaillon, on parvient à identifier, en ce qui concerne les animaux représentés et dans le sens des aiguilles d'une montre, un scorpion (voir flèche fig. 1, coupe A), un chien, deux dragons affrontés (?), surmontés d'une bande ondulée, et un autre quadrupède (un cheval, un lion ?), avec au-dessus du dos un sceau de Salomon. Les bâtonnets « magiques ». quant à eux, sont chaque fois disposés sur trois lignes. Statistiquement, leur nombre est à peu près le même pour un médaillon donné : partant du scorpion et suivant le même sens, de la ligne supérieure à la ligne inférieure, on obtient le résultat suivant : 18/18/19 ; 19/19/17 ; 16/17/16 et 15/15/15. Quant aux textes des médaillons, en partant de celui situé entre le scorpion et le chien (?) et en allant dans le sens des aiguilles d'un montre, on parvient à déchiffrer :
\begin{itemize}
    \item (1) la \textit{basmala}, suivie de s84v1-4 (sourate « La déchirure », a
\textit{Inshigãq}), le 4° verset s'achève à : \textit{wa algat má fihà}, puis sur la ligne
les lettres \textit{kâf-râ'} (?), et, en dessous, sìn (?)-käf.
\item (2) dans le style d'une invocation (azîma) : « yâ Nüh, Banüh, Kali
(2), Kalükh, Kalkh, \textit{alif}-lâm-mim-ra' [lettres liminaires, sourate 13], alij lâm-mîm [lettres liminaires, sourates 2, 3, 29, 30, 31 et 32], \textit{alif}-lâm-min
ra' [ibid.], \textit{hâ}'-mêm-ayn-sîn-gaf [lettres liminaires, sourate 42], kâf-\textit{hâ}
ya'-ayn-sad [lettres liminaires, sourate 19], ta' (?), tâ'-\textit{hâ} [lettres lim naires, sourate 20], ta'-\textit{hâ} (?) », enfin dernière ligne, des chiffres (?).
\item (3) ???, des lettres séparées sur les deux dernières lignes : sìn-\textit{waw} (?
mim (?), ba', puis \textit{kâf}, ra' (?).
\item (4) ???, des lettres séparées sur les deux dernières lignes : shin-ra
\textit{hâ}'-ra', \textit{alif}, \textit{hâ} (?), puis \textit{alif}, hā'-ra' , \textit{alif}, dal, \textit{alif}.
\end{itemize}:

\includegraphics[width=\textwidth]{HommeetIslam/Images/IMG_2455.JPG}


Ces textes sont écrits pour trois d'entre eux sur sept lignes, un seul occupe huit lignes. Enfin, le registre compris entre la lèvre de la coupe le cercle auquel se rattachent les pointes des carrés, présente, aux extremités du cercle, 4 rectangles (façades de bâtiment, talismans?) alternant avec  4 formes cintrées (mihrabs ou stèles funéraires, amulettes ?). Ils contiennent les mêmes bâtonnets, arrangés sur des lignes, que les médaillons animaliers. Leur nombre est à peu près constant, dans les rectangles, entre 7 et 9 par ligne, dans les formes cintrées, entre 6 et 8 par ligne. L'espace entre ces motifs est occupé en alternance par des écrits magiques et de l'arabe, placés de façon à correspondre au style des écritures contenues dans les médaillons : les écrits magiques, au-dessus des écrits magiques et l'arabe au-dessus de l'arabe. A propos des écrits magiques, on relève systématiquement la présence d'hexagrammes ou sceaux de Salomon, suivis (dans le sens de l'écriture arabe, i. e. de droite à gauche) d'une série relativement stable de signes, de sorte que l'on a sensiblement quatre fois le même ensemble. Ces séries peuvent être assimilées à ce qu'al-Bûni appelle les sept lettres (\textit{al-ahruf al-sabr}), ou bien les
sept sceaux (\textit{al-khawâtim al-sab}°), ou encore \textit{al-tilsam al-Sulaymani}. 

Après le seau de Salomon, on relève successivement trois traits verticaux
surmontés d'un trait horizontal; la lettre \textit{mim} (parfois réduite à un trait sans véritable boucle, mais ce trait apparaît bien distinct des trois précédent reliés par le surlignage); deux traits verticaux, plus longs que les autre comportant deux barres obliques, qui les rendent analogues à des dièses (dans un cas sur quatre, les deux barres obliques n'apparaissent pas); quatre traits; pour conclure, une nouvelle étoile à six branches, mais aussi une sorte de y (?). La fin diffère de la série donnée par al-Bûnì, qui cite, après le quatre traits, les lettres \textit{\textit{hâ}}', puis \textit{wâw}. A moins de considérer, pour la coupe A, que la ligne qui clôt chaque série, et commence sur la ligne d'écriture revient vers le haut en s' incurvant et en direction des « lettres » précédentes ne soit un \textit{wâw} (Rehatsek, 1875b, 301, fig. 2). Chaque ensemble formé par ces « lettres » ou « sceaux » se trouve délimité par le cercle qui définit registre dans sa partie inférieure et par un surlignage plus ou moins contin qui rejoint le cercle en fin de séquence : chaque ensemble apparaît donc ins crit dans un cartouche. Le fait de délimiter un écrit porteur d'une puissanc magique est une pratique courante en islam arabe. Quant aux textes en arabe, on déchiffre : 
\begin{enumerate}
 
    \item   « La basmala, suivie de trois mots (?) ;   \item  \textit{wa ma
yatawakkal'ala Allah fa-huwa} (?) ;   \item  fa-huwa [une seconde fois ?] \textit{hasbuhu inna Alläh baligh amrihi;}
\item \textit{ wa al-salâh wa al-salâm 'alâ sayyidina Muhammad}
\end{enumerate}
Les textes 2 et 3 sont tirés de s65v3 (« La Répudiation », \textit{al-talâq}). La \textit{basmala} est liminaire et la prière adressée à Dieu en faveur du Prophète Muhammad, qui vient souvent clore un écrit, se trouve dans le dernier cartouche, suivant le sens de lecture de gauche à droite. Il peut donc s'agir d'une seule et même formule qui, déroulée sur le pourtour en 4 seg-ments, pourrait être lue perpétuellement. Cette coupe, dans sa composition interne, est, on le constate, largement construite selon le chiffre huit, souvent obtenu par l'utilisation de 4 fois 2 types d'éléments différents.
 
\includegraphics[width=\textwidth]{HommeetIslam/Images/IMG_2456.JPG}
La paroi extérieure comporte trois zones concentriques. Le fond de la coupe est occupé par un sceau de Salomon. Un premier anneau sur lequel s'alignent des bâtonnets est interrompu trois fois par des sceaux de Salomon, disposés en triangle. Un second anneau, situé dans la partie basse de la coupe, reprend le même système de bâtonnets et de sceaux, cette fois au nombre de quatre et disposés en carré. On dénombre ainsi huit sceaux de Salomon. Enfin, entre cet anneau et la lèvre, on relèvera deux lignes de texte. L'une, dont l'incision est plus profonde, se trouve gravée à proximité de la lèvre et s'étend sur tout le pourtour; elle indique :
 \begin{quote}
     « Hädhihi (2)\mn{Soit : « Celle-ci [i. e. cette coupe] [sert] pour la piga-re du serpent et du scorpion, les chiens enragés, à faciliter l'accouche. ment, aux saignements de nez, [pour les douleurs à] l'estomac ... (?), les coliques …. (?), un sceau de Salomon ... (?) »} li-lasat al-hayya wa al-agrab wa al-kalib (sic) al-kalib wo
li-usr al-walad wa al-ruäf wa al-miada ... (?) al-qawlanj ... (?) un sceau de Salomon ... (?) ».
 \end{quote}
 Deux traits verticaux avec une barre médiane séparent le début de la fin de la phrase. L'autre inscription, située entre celle-ci et le second anneau, est incisée plus profondément, le trait en est plus épais que la précédente. Cela suggère un rajout,
le texte le confirme : 
\begin{quote}
    « Waqafa \mn{[legs pieux du Häjj Husayn al-Hawa (?) à la mosquée, la protégée, en 1313/1895-96]} al-Hajj Husayn al-Hawâ (?) \textit{hâ}dhihi al-
tâsa (sic) alâ al-Jâmi al-mahrûs 1313 »
\end{quote}   La lecture du nom du donateur pose problème, car un \textit{alif} semble avoir été tracé à la suite du \textit{hâ}', puis un \textit{waw} gravé en partie sur le \textit{alif} et avec la marque d'un rattachement à une lettre qui le précèderait, mais qui n'est pas donnée; enfin de petites encoches semblent indiquer une volonté de raturer ce \textit{waw}.
Ensuite apparaît un signe ressemblant à deux \textit{waw}-s inversés dont les boucles s'entrelacent : il peut être identifié comme une marque de fin de phrase, à la manière des petits décors que l'on trouve dans les manuscrits.
La seconde coupe B, plus grande et plus ouvragée sur ses parois internes et externes au décor incisé, sans pied, à lèvre arrondie, a été réalisée dans un alliage cuivreux, et son fond, fendillé de manière semi-circulaire par l'usage, a fait l'objet d'une réparation. Il s'agit vraisemblablement d'une soudure. Ses dimensions sont de 6,5 à 6,6 cm de hauteur et de 18,3 cm de diamètre, mesuré de bord à bord extérieurs, la lèvre comptant pour 0,4 cm. Elle est de provenance inconnue. En ce qui concerne le décor de ses parois extérieure et intérieure, elle se rapproche d'autres coupes se référant à Abû al-Muzaffar Yûsuf (habituellement identifié comme Saladin), avec la liste de ses vertus curatives - il en sera question plus bas - mais s'en distingue, par l'absence de représentation de la Kaba en son centre (intérieur) (Savage-Smith, 1997, 73). Elle se rapproche davantage de quatre autres coupes, trois décrites par Rehatsek, et la quatrième, propriété du Science Museum à Londres. Elle n'est pas datée, mais à coup sûr fabriquée postérieurement à l'époque de Saladin,  semblablement entre le VIII-IX'/XIV°-XV° s. et le XII/XVIII, peut-être XIII /XIX s..

 \includegraphics[width=\textwidth]{HommeetIslam/Images/IMG_2457.JPG}

L'ensemble du décor de la paroi intérieure se répartit sur trois registres, chacun délimité par un double cercle. On a ainsi trois doubles cercles concentriques dont le dernier suit les contours de la lèvre. 
Le premier registre, constitué par le fond de la coupe et délimité par un premier cercle concentrique, est occupé par un écrit de type magique qui reste à décrypter'. On distingue autant que la détérioration et la réparation le permettent, des traits de hauteur inégale, des lettres de l'alphabet arabe. isolées, essentiellement \textit{hâ}', la (?), \textit{hâ} et \textit{kâf}, des chiffres, 6, 7, 95. Ils forment neuf lignes horizontales, la dixième suivant la courbure du cercle. Le second registre, situé dans la partie basse de la coupe, a essentiellement pour décor des bandes qui se chevauchent, formant seize pointes, une étoile à seize branches. L'espace résiduel est rempli d'écrits magiques, identiques aux précédents. Certains sont disposés sur le premier double cercle concentrique. Le troisième registre, enfin, occupe la plus grande surface de la coupe. Son décor est structuré par un motif de feuilles ou de pétales s'ouvrant en corolle, répartis sur deux hauteurs et décalés les uns par rapport aux autres. Ces « feuilles » forment autant de médaillons. La première série de médaillons, au nombre de seize, installée sur le second cercle concentrique, comprend exclusivement des écrits magiques identiques aux premiers, gravés sur six lignes, dont l'une est constituée par le second cercle concentrique. Quant à la série supérieure de seize médaillons, elle comporte, alternés, des représentations et des restes en arabe. On parvient à identifier les figures suivantes, dans le sens des aiguilles d'une montre : un soleil à huit rayons enfermé dans un cercle, un chien, un scorpion, un personnage (une femme avec un enfant, allaitant ?), un croissant de lune, un cheval, un serpent, un second personnage (personne mordue par un serpent ? ou possédée par un esprit malin ?). Autour d'eux, les mêmes écrits magiques, qui diffèrent cependant des précédents en ce qu'ils ne sont pas gravés sur des lignes.
L'ouvrage est fait de telle façon que la lune se trouve opposée au soleil, que l'un des personnages est dans l'axe de l'autre et les deux quadrupèdes face à face. En ce qui concerne les textes en arabe, il s'agit d'extraits du Coran, tous écrits sur huit lignes. En partant de la sourate liminaire, placée entre le chien et le scorpion, et en allant dans le sens des aiguilles d'une montre, on déchiffre :
\begin{enumerate}
   \item al-Fâtiha, jusqu'à
" .. ghayr al-maghdûb 'alayhim" ;
  \item précédée de la basmala, une partie de s25v45 (« la Loi ou la
Salvation », al-Furqân), jusqu'à "la-jaalahu sâkinan", enchaînée à s6v13
(« Les troupeaux », al-An'âm), enfin quelques 4 mots non déchiffrés;
  \itemla basmala, puis s84v1-4 (« La Déchirure », al-Inshiqâq), le verset
4 s'achève à : "wa algat má fiha", la fin reste à déchiffrer ;
  \item sans basmala, s24v35 (« La Lumière », al-Nür), jusqu'à « Júgad
min shajara mubâraka zaytâna là sharqiyya wa là [gharbiyya] », quelques lettres isolées (?), puis ra', les chiffres 7 et 2 ou 3
  \item la basmala, puis sur la 2° ligne, les 3 lettres liminaires apparaissant
au début de six sourates, celles de la Vache, d'al-'Imran, de l'Araignte, des Romains, de Lugmân et de la Prosternation (soit 2, 3, 29, 30, 31 er
32), à savoir \textit{alif}-lâm-mim, suivies de 3 mots non déchiffrés, puis sur la 3-ligne, les 5 lettres liminaires de s19 (« Marie », Maryam), à savoir \textit{kâf}.
hã'-ya'-ayn-sâd, suivies de quelques mots non déchiffrés, puis sur la 4. ligne, les 3 lettres liminaires du début de s26 et   28 (« Les poètes » et « le
récit », al-Shu'ara' et al-Qasas), à savoir tâ'-sîn-mim, enfin les autres
lignes n'ont pu être déchiffrées ;
  \item la basmala, suivie de deux lignes et demi non déchiffrées, puis viennent peut-être une partie de s16v69 (« Les Abeilles », al-Nahl),
« yakhruj min butûni\textit{hâ} sharâb mukht\textit{alif} alwânuhu fihi shif@' » (?), et une partie de s17v82 (« Le Voyage nocturne », al-Isrâ'), « wa-nunazzil
min al-Our'ân mâ huwa shifà »" (?) ;
  \item la basmala (?), puis on lit \textit{hâ}-\textit{alif}, lâm-\textit{hâ}' al-rahmân al-rahim), suivi de s8v62-64 (« Le Butin », al-Anf@l) : le dernier mot du v63,
« hakîm », n'est pas très lisible et le v. 64 est déchiffrable jusqu'à « hasbuka Alla », quelques mots restent ensuite à comprendre;
  \item) sans basmala, 2v255 (« La Vache », al-Bagara, « verset du
Trône », ayat al-kursi), jusqu'à : « ... là yuhîtûn »'.
\end{enumerate}

L'espace compris entre les médaillons et l'ultime double cercle est
occupé par les écrits magiques déjà rencontrés plusieurs fois. Sur le cercle supérieur de ce dernier double cercle, enfin, est gravée une rangée
des mêmes écrits. Le décor de la paroi extérieure offre trois registres. Le premier est constitué par le fond de la coupe qui laisse deviner, en dépit de l'usure, le même type de composition magique que sur la paroi intérieure; la forme géométrique qui l'enserrait a disparu. On reconnaît ensuite un cercle concentrique constitué encore des écrits magiques. Le second registre occupe la partie médiane de la coupe. Les écrits magiques sont cette fois inscrits dans cinq cercles et cinq trapèzes alternés. Enfin, le troisième registre présente une ligne d'écriture qui suit le bord de la lèvre et s'étend sur tout le pourtour. Elle est gravée de telle sorte que le début du texte s'enchaîne sans rupture avec la fin. Cette phrase ininterrompue pourrait tenir dans un double cercle concentrique : le graveur s'est appliqué à ne jamais en dépasser les limites invisibles et à emplir l'espace de telle sorte qu'elle donne l'impression d'une bande circulaire continue, très décorative. Elle est chargée d'indiquer les vertus curatives de la coupe, comme c'est le cas pour la coupe A24:
\begin{quote}
    « wa \mn{Soit : « Pour notre Seigneur, le Sultan, al-Malik al-Mujähid, le manda-victorieux Abû al-Muzaffar Yûsuf26. Y [la coupe] sont réunis des bienfaits éprouvés par l'expérience, elle [sert] pour les piqûres de serpent et de scorpion, pour la fièvre, la parturiente? et augmenter le lait, pour les morsures de] chiens atteints de la rage, pour les douleurs stomacales et les liques, la migraine et les élancements (?)?, pour conjurer les sortilèges, ou faire cesser le flux du sang, pour le mauvais œil et le mauvais sort  
pour empêcher la paralysie faciale et pour le rétablissement de la conscience des épileptiques (?), pour la dysurie, pour la réconciliation des adversaires (ou : les proches parents ?), pour les enfants agités. L'ensor-celé et celui qui est atteint, de même que la parturiente en difficulté (?), doivent [en boire le contenu par gorgées (?)].} li-mawlâna al-sultân al-Malik al-Mujähid al-muwakkal al-man.
sûr Abû al-Muzaffar Yasuf wa jumi a fiha manâfi mujarraba wa hiya li-
las'at al-hayya wa al-'agrab wa-li-al-hummâ wa al-mutlaga wa al-magh-
la wa li-al-kalb al-kalib wa li-al-maghass wa al-qawlanj wa al-shagiga
wa al-zarabân (?, sic) li-ibtâl al-sihr wa li-ramì al-dam wa li-al-ayn wa
al-nazra wa li-râd al-lawaga wa li-ifâdat al-masrû (?) wa li-usr al-baw!
wa li-sulh bayn al-aqrân (al-agrâb ?) wa li-nakad al-atfäl wa al- m.r: (?.
ou 'm.r.j?) bi\textit{hâ} al-mashür wa al-musâb wa al-bint (?) al-mu sira" ».
\end{quote}

\paragraph{Titulature de Saladin}
La titulature, la \textit{kunya} (= Abû al-Muzaffar) et le nom (ism = Yûsuf) (à
moins qu'il ne s'agisse que d'une kunya = Abû al-Muzaffar Yüsuf), mentionnés dans l'inscription, peuvent-ils servir à identifier le personnage ?

Et constituent-ils un élément fiable de datation de notre objet ? La même formule (sauf al-muwakkal) se retrouve chez Reinaud, sur les coupes n° 9420, chez Wiet, n° 14, chez Canaan - qui n'identifie pas - et surtout sur les deux coupes, dites de Saladin, étudiées par Zéki Pacha : « Izz li-mawlâna al-sultân al-malik al-mujâhid al-muayyid al-Mansûr Abû al-Muzaffar Yûsuf » et : « 'Izz. li-mawlânâ al-sultân al-malik al-mujâhid Abû al-Muzaffar Yasuf »». Les coupes dédiées à Saladin sont réputées nombreuses   mais ne sont pas nécessairement indicatrices d'un temps et d'un lieu de facture particulier. 
En effet, Wiet (1922, 319-28), dans un article très précis, critique la datation de Zéki Pacha en s'appuyant essentiellement sur des documents épigraphiques, mais aussi sur des chroniques : il montre que les inscriptions de ces deux coupes font entorse à la titulature de Saladin, et donc au protocole habituel, que ces formules sont rares au VI/XI s., et que les dates qui suivent la mention de souverains, sur les coupes, ne sont pas un gage de leur époque de fabrication; il concède toutefois que l'on puisse y voir une allusion au souverain ayyûbide - si l'on se rapporte à d'autres objets sur lesquels se trouve le même type d'anomalies - mais qu'en aucun cas, ces coupes ne peuvent être contemporaines de Saladin. Le style de la coupe B, de même que les conclusions de Wiet, font plutôt penser à une attribution posthume.

  \includegraphics[width=\textwidth]{HommeetIslam/Images/IMG_2458.JPG}
  \paragraph{coupe anti poison}
L'étude des écrits, représentations, signes et figures géométriques des deux coupes fait apparaître un registre commun avec la talismanique. La symbolique de la coupe A, telle qu'elle se dégage à partir des éléments décrits, vient confirmer certaines de ses propriétés curatives. Tout d'abord, il s'agit d'une coupe anti-poison ; al-Bânì cite une longue incantation (azima) versifiée, valable pour toute œuvre magique : elle décrit les sept lettres spéciales, rappelle qu'elles sont le nom suprême de Dieu, puis que moyennant l'ajout de lettres de la Torah, des Evangiles et du Coran, l'on sera protégé des serpents, scorpions et lions?. Il s'agirait donc, dans l'ensemble, des animaux nuisibles ou devenus dangereux dont la coupe guérit de la morsure ou de la piqûre. En outre, elle aide en cas d'accouchement difficile, ce qu'indique la présence de la sourate \textit{al-
Inshiqâq}. Des textes gravés sur d'autres coupes comparent le mouvement de la terre qui rejette ce qui est en elle et se vide, à celui de la femme qui met un enfant au monde dans des conditions favorables. Spontanément, les deux coupes m'ont d'ailleurs été présentées comme des « coupes pour l'accouchement », l'usage ayant probablement consacré cette fonction entre toutes. Selon al-Bûnî (s.d., (a), 91) encore, les trois bâtonnets, sans l'\textit{alif} sur le dessus, désigné simplement alors comme « protubérance » ou pointe de lance, et accompagnés du pentagramme, peuvent être employés pour guérir des maladies affectant l'intérieur du corps, dont les coliques (gawlanj), auxquelles fait référence la liste des maux soignés par la coupe
A. Enfin, les deux carrés entrecroisés gravés à l'intérieur de la coupe forment une étoile à 8 branches - cette étoile finissant dans un double cercle peut aussi représenter le soleil, dardant ses rayons? -. Décrite comme la gravure ou le diamant de l'anneau de Salomon, l'une de ses variantes se trouve sur les rouleaux magiques d'Ethiopie, où elle est appelée « sceau de Salomon »». Son centre, qui est également le centre de la paroi interne, est entièrement occupé de bâtonnets magiques. 
\paragraph{Pouvoir du roi Salomon - de son sceau}
Qu'il soit bague ou talisman, selon la Tradition, il donne à ce Roi son pouvoir sur les démons
-  pouvoir que le Coran lui reconnaît, dans la sourate Sâd, aux versets 34-39 - et donc sur la cause des maladies, ou tout au moins de certaines.
Dans certaines versions, il oblige les démons, qui sont des sortes de gardiens des maladies, à livrer les remèdes (Barkaï, 1996, 193). Dans l'ensemble, les démons - parfois appelés djinns - sont divisés en tribus liés à un habitat, et la maladie est ainsi territorialisée. Salomon n'éradique certes pas la maladie, mais, grâce à la puissance de son sceau, don du ciel, il est celui par lequel vient la guérison. On remarquera que le sceau de Salomon est gravé ici parmi les maux que la coupe est censée soigner.

Leur liste étant incomplète, on ne fera qu'évoquer les maladies spécifiquement provoquées par les djinns ou les démons que Salomon et ses formules peuvent exorciser. Pour ce faire, on doit s'asperger ou se laver avec l'eau versée dans la coupe. L'architecture de la paroi interne est ici toute conçue et rythmée à partir de ce motif. Elle est largement construite selon le chiffre huit ou ses diviseurs. L'étoile à huit branches et les huit étoiles à six branches, structurant les parois interne et externe, indiquent singulièrement qu'elle en appelle à la protection de Salomon par qui vient la guérison, les autres grands pourvoyeurs de guérison étant Dieu et le Prophète (évoqué dans la formule gravée près de la lèvre supérieure de la coupe A).


La coupe B présente également l'image des animaux à piqûre et morsure (scorpion, serpent, chien, cheval) et le début de la \textit{sourate de la Déchirure}. L'un des deux personnages (femme avec un enfant, l'allaitant ?) rappelle son pouvoir de faire venir le lait. La basmala, prononcée avant toute action et notamment avant d'ouvrir la bouche pour manger, de même que les Versets du Siège, appelés âyat al-hars wa l-hirz, protègent contre les mauvais dinns ou démons. C'est le cas également des cercles.
Thème que pourrait reprendre encore l'un des deux personnages (personne possédée par un esprit malin ?). Deux sourates renvoient aux sources de la guérison plutôt qu'elles ne visent une maladie en particulier. Le verset 69 de la sourate des Abeilles fait en effet allusion aux pouvoirs curatifs du miel, qui fait partie des liquides versés dans les coupes, selon mon informateur, et à absorber donc par les malades. Les vertus du miel sont extrêmement nombreuses et constituent un thème de prédilection de la médecine pronhétigue4) Par ailleurs. s17v82 (« Le Voyage nocturne »)
'une manière plus générale, s8v62-oulait, allaitant
) rappelle son pouvoir de faire venir le lait. La basmala, prononcée avant toute action et notamment avant d'ouvrir la bouche pour manger, de nême que les Versets du Siège, appelés ayat al-hars wa l-hirz, protègent contre les mauvais djinns ou démons. C'est le cas également des cercles.
Thème que pourrait reprendre encore l'un des deux personnages (person-ne possédée par un esprit malin ?). Deux sourates renvoient aux sources de la guérison plutôt qu'elles ne visent une maladie en particulier. Le verset 69 de la sourate des Abeilles fait en effet allusion aux pouvoirs curatifs du miel, qui fait partie des liquides versés dans les coupes, selon mon informateur, et à absorber donc par les malades. Les vertus du miel sont extrêmement nombreuses et constituent un thème de prédilection de la médecine prophétique*. Par ailleurs, s17v82 (« Le Voyage nocturne ») rappelle le pouvoir curatif du Coran. D'une manière plus générale, s8v62-64 (« Le Butin ») martèle l'idée que : « Dieu te suffit »*. Quant à la sourate liminaire, ses vertus sont tellement innombrables, qu'elle n'est plus indicative : elle est bonne pour tout. Enfin, les deux Luminaires, le soleil et la lune, sont des symboles de vie, de prospérité et d'abondance, comme le remarque Canaan (1936, 100, 121). De la même manière que les deux Carrés entrecroisés contenant des écritures magiques, à l'intérieur de la
«/iat (1932. 95-96. п° 3906. 121, л° 4431) : Сдатcoupe A, peuvent représenter le soleil sous la forme d'une étoile à huit pointes et qu'elle est structurée selon le chiffre huit, le soleil, dans la coupe B, compte également huit pointes. Par ailleurs, le fond de la coupe B est occupé par une étoile à seize branches, qui soutient sa composition interne en seize médaillons, puis en deux fois huit médaillons (à texte et à figures). Quelques coupes étudiées par ailleurs montrent une corrélation entre la présence des Luminaires et le chiffre 16 (cartouches ou cercles).
Le rapport est donc net entre le chiffre huit et le soleil. On est tenté de mentionner alors le fait que huit correspond à la valeur isopséphique du \textit{hâ}, lui-même clé de la vie (hayât), selon un procédé bien connu en science des lettres. Cependant, les versets des sourates 6, 24 et 25, gravés sur la coupe B, rappellent l'omnipotence et l'omniscience de Dieu, cause de tout et à qui tout doit revenir, au-delà des deux Luminaires : 
\begin{quote}
   « C'est à lui qu'appartient ce qui subsiste dans la nuit et le jour »; « Dieu est la lumière des cieux et de la terre ».
En guise de remarque finale sur les deux coupes, ne peut-on pas dire qu'elles consqu'appartient ce qui subsiste dans la nuit et le jour » ; « Dieu est la lumiè re des cieux et de la terre ». 
\end{quote}

\paragraph{Cosmos}
En guise de remarque finale sur les deux coupes, ne peut-on pas dire qu'elles constituent une tentative de reproduire le monde clos du cosmos ?

\includegraphics[width=\textwidth]{HommeetIslam/Images/IMG_2459.JPG}



En effet, le soleil qui préside à l'architecture interne des deux coupes, les cercles concentriques, les entrelacs de rubans ininterrompus, les textes eux-mêmes parfois écrits de telle manière qu'ils ne s'achèvent ni ne commencent, les motifs alternés qui se répondent, ainsi que la concavité et le caractère hémisphérique des deux objets (Canaan, 1936, 82), sans compter la composition organisée en registres, concourent à le reconstituer, quelques sourates rappelant que Dieu demeure Le plus puissant, Le plus savant et la Cause unique et suprême. Si l'on songe que le mode d'emploi des coupes consiste à passer par un liquide qu'on y verse, celui-ci se trouve donc en contact avec toute chose du monde : ambition holistique des coupes.Le recours aux coupes thérapeutiques est d'un usage bien établi au
Yémen, aussi bien dans la communauté juive que musulmane (Brauer, 1934, 182-3)47, surtout dans les cas d'accouchements difficiles ainsi que, concurremment à d'autres pratiques, contre le venin des serpents. L'acte de soigner différents venins, parmi lesquels celui des scorpions, est particulièrement investi par diverses médecines ou pratiques. C'est sans doute un indice de la fréquence du danger. Selon l'un de ceux qui ont la responsabilité des coupes à la mosquée, leur mode d'emploi consiste principale-328
nent à boire le liquide - de l'eau, du bouillon - ou le miel que l'on y a versé". Le docteur Sarnelli rapporte, pour le Yémen des années 30, qu'il aut boire l'eau à petites gorgées en prononçant des louanges à l'endroit lu Seigneur des mondes, de ses anges et de ses prophètes. Pour les nêmes années, Brauer cite d'autres manières de les utiliser chez les juifs éménites, il en sera question un peu plus loin (Brauer, 1934,182-3). Une éritable description ethnographique, susceptible de nous renseigner sur ensemble des étapes à suivre pour utiliser les coupes, manque cepen-ant, que ce soit pour le Yémen, ou un autre lieu.
Le produit en contact avec l'ensemble des écrits et l'iconographie est hargé de communiquer quelque chose au malade, mais selon un principe ui reste à définir. Le liquide en relation avec les représentations des ani-aux nuisibles transmet-il au patient une partie de leur force, l'immuni-int par assimilation de quelque chose de leur substance, agissant tel un intrepoison ? Ou au contraire, s'agit-il d'une mithridatisation ? Les présentations ont-elles une valeur prophylactique, comme c'est le cas ur des talismans chassant les bestioles et les chiens des maisons (Doutté.
194, 144-46) гdant, que ce soit pour le
Le produit en contact avec l'ensemble des écrits et l'iconographie chargé de communiquer quelque chose au malade, mais selon un princi qui reste à définir. Le liquide en relation avec les représentations des a maux nuisibles transmet-il au patient une partie de leur force, l'immu sant par assimilation de quelque chose de leur substance, agissant tel contrepoison ? Ou au contraire, s'agit-il d'une mithridatisation ? I représentations ont-elles une valeur prophylactique, comme c'est le pour des talismans chassant les bestioles et les chiens des maisons Dou 1994, 144-46) ? La séquence écrits-liquide-boire, enfin, rappelle une p ique présente dans l'ensemble du monde arabe, qui consiste à écrire in papier, à l'encre noire ou de couleur, généralement des versets ce riques, et à le tremper dans un récipient empli d'eau, que l'on demande nalade de boire, notamment dans les cas d'ensorcellement!.
Cependant, dans le contexte yéménite des hauts plateaux, il sem! ue le bouillon versé dans les coupes serve à soutenir plus spécialem s femmes en couches??. Le bouillon de viande (marag) est gras, robotif, et rassemble la substantifique moelle de la viande : il est offert en début de repas et particulièrement aux hôtes, comme un met prisé et recherché, le meilleur en somme. Il est par conséquent indiqué pour soutenir physiquement la femme en travail. Ici, le liquide n'est donc pas seulement conducteur ou à la rigueur stimulateur, dans l'économie de la guérison par les coupes, mais il apporte ses vertus propres, qui viennent en addition de celles des écrits et représentations. En médecine domestique, pour la même région du Yémen, le recours simultané à des médecines et à des médications différentes est courant. Plutôt que le reflet d'un tâtonnement thérapeutique, il faut y voir au contraire un discernement et la volonté d'attaquer le mal, dans son éventuelle pluralité, dans toutes ses facettes. Les symptômes sont en effet soigneusement observés et triés, et, à partir d'eux, il s'agit d'évaluer les causes, qui peuvent être diverses pour un même symptôme. Symptômes et causes ne peuvent être tous traités de la même manière, et entraînent la décision de recourir à tel ou tel praticien. En outre, lorsqu'il y a maladie, elle est toujours soupçonnée d'être le résultat de causes multiples, de nature diverse, déchaînées en même temps. L'utilisation des coupes, dans le cas qui vient d'être évoqué, n'est clairement pas un acte médical unique.
Des traînées de bougie apparaissent au fond de la coupe B. Sont-elles en rapport avec l'utilisation des coupes ? Enfin, on observera que la coupe B, réparée, est toujours considérée comme en service. Pourtant, les détériorations et les réparations ont entraîné, pour partie, la disparition du décor gravé au fond.Institutionnalisation de pratiques magiquesQue ces objets soient wagf et non milk ne fait aucun doute pour leurs gardiens. On trouve d'ailleurs gravé sur la paroi externe de la coupe A, à la suite de la liste de ses propriétés, l'indication suivante: « Wagafa al-
Hajj Husayn al-Hawâ (?) \textit{hâ}dhihi al-tâsa (sic) 'alâ al-Jâmi al-mahrûs
1313 » [legs pieux (wagf) du Häjj Husayn al-Hawa (?) à la mosquée, la protégée, en 1313/1895-96]. Outre l'intérêt immense d'avoir ici une date, certes tardive et à ne pas prendre pour une date de fabrication, on retiendra qu'un tel statut pour un tel objet est bien un fait. Et le cas qui nous occupe n'est peut-être pas singulier. L'une des coupes signalées par
Ahmed Zéki Pacha, provenant du Bimâristân al-Mansär - du nom de son
fondateur al-Malik al-Mansûr Qalâwûn, en 683/1284, au Caire - et encore en service à la fin du XIX s., permet sans doute de ne pas limiter
au Yémen zavdite. L'ensemble des instruments utilisesl'hôpital était en effet considéré comme wagf. Cependant l'établisse. ment du statut juridique des objets qui nous occupent, pose problème.
Tout d'abord, ils ne seraient pas recensés par le registre des wagfs. Il est difficile de croire à une négligence d'autant que leur qualité de bien wagf n'est pas dissimulée. En 1998, dans une ambiance peu favorable il est vrais, la tentative de consulter ces registres du Ministère des wagf-s a tourné court. Mais aux alentours de 1999, les objets semblent avoir été récupérés par ce même Ministère, selon ce qu'a révélé une nouvelle visite au lieu où ils étaient auparavant déposés. Jusque là, ils semblaient échapper au contrôle de l'administration des \textit{waqf}-s et à toute redevance.
Ils m'ont été présentés comme « wagf-s oraux ». La procédure paraît connue et rodée. Le donateur vient et dit au récipiendaire : « Je donne cet
objet en \textit{waqf} à la mosquée [awgaftu \textit{hâ}dha al-shay' alâ Jâmi kad\textit{hâ} ».
Et voilà tout. Elle est adoptée pour des objets isolés, tels les corans, les tapis, offerts à la mosquée par des particuliers, musulmans, et qui ne représentent ni de grosses quantités, ni sans doute une valeur à l'unité très importante : dans ce cas, le principe de l'absence de tout document fixant destinataires et frais de gestion (wagfiyya), de même que l'absence de démarche d'enregistrement au Ministère des wagf-s, sont admis et connus du même Ministères. Certains d'entre ces objets, tels les coupes, le Livre, ou tout ouvrage en général, peuvent porter - mais pas nécessairement - une inscription mentionnant qu'ils sont biens wagf, d'autres non, tels les tapis. Il existe des tampons pour les ouvrages, usant de formules consa-crées, où le donateur peut ajouter son nom%. En somme le message est suffisamment clair : la mise en wagf de ces objets signifie essentiellement qu'il est interdit à quiconque de les utiliser à des fins personnelles et de
les vendra
une inscription mentionnant qu'ils sont biens wagf, d'autres non, tels les tapis. Il existe des tampons pour les ouvrages, usant de formules consa-crées, où le donateur peut ajouter son nom*. En somme le message est suffisamment clair: la mise en wagf de ces objets signifie essentiellement qu'il est interdit à quiconque de les utiliser à des fins personnelles et de les vendre.
Si les avoirs transmis sont clairement ici les objets, dans les faits, on ne voit pas bien en revanche comment se fait la prise en charge de leur entretien et, pour ceux qui nous concernent, des frais entraînés par l'accueil des demandeurs. En effet, ils suscitent des dépenses - même si on peut estimer qu'elles sont infimes - et pas de revenus. Nos quatre
objets sont prêtés sur simple demande, sans contrepartie d'argent, afin que même les plus démunis puissent se soigner. Autrement dit, ils génèrent plutôt du bien ou un service, du fait de leur usage thérapeutique (l-al-khayr). D'un autre côté, ils ont besoin d'être entretenus et réparés : les coupes se perforent par usure - c'est le cas de la coupe B, et il n'est pas rare de trouver des coupes dont le fond est percé - le manche du miroir a été soudé à la partie centrale et cette soudure peut se détériorer, enfin, la pierre risque d'être attaquée par le venin. Quant au destinataire, il s'agit de la mosquée, l'inscription de la coupe A le dit explicitement. Toute personne malade ou envoyée pour un malade et même si elle n'est pas du quartier ou de la Vieille ville, peut emprunter ces objets. Sur ce point, les dispositions légales ne font que reprendre un usage bien établi, en ce qui concerne les coupes tout au moins. Car même en propriété privée, elles ont toujours été prêtées à la demande et elles ne mentionnent dans leurs inscriptions aucun destinataire en particulier pour qui elles auraient été fabriquées, j'aurai à y revenir plus loin. De plus, l'usage collectif ancestral des coupes trouve parmi les catégories connues des fondations pieuses, sa case légale : il relève des fondations « d'orientation publique
ou charitable (wagf khayrt) » (Deguilhem, 1995, 16).
Au regard encore de la définition d'une fondation pieuse, de son principe même, le cas qui nous occupe demeure problématique ou inédit, si
Cas gla dent l°obiet d'un écrit chargé de fixer les modalitéstral des coupes louve parmi les categones connues des fondations pieuses, sa case légale : il relève des fondations « d'orientation publique
ou charitable (wagf khayrt) » (Deguilhem, 1995, 16).
Au regard encore de la définition d'une fondation pieuse, de son principe même, le cas qui nous occupe demeure problématique ou inédit, si tout wagf fait normalement l'objet d'un écrit chargé de fixer les modalités par lesquelles des revenus seront alloués à un (ou plusieurs) destinataire(s) (Deguilhem, id, 15). En ce qui concerne des dispositions particulières, force est de constater que le Yémen a fait l'objet de peu d'études, aussi bien en langue arabe qu'en langue européenne, sur les diverses modalités de fondations pieuses y ayant cours; il reste donc relativement méconnu sur ce points
Il est encore possible de se reporter à la transmission orale, si l'on veut tenter de préciser la provenance et la date à laquelle ces objets ont acquis leur statut. En se plaçant du strict point de vue de la collecte des données, les informations restent imprécises et contradictoires, liées aux souvenirs de chacun et à l'ancienneté de leur présence dans ces lieux. Sur les quatre objets concernés, l'un était propriété du QQ'im actuel : c'est lui qui l'a donnée en wagf à la mosquée. D'après ce dernier, les trois autres objets bénéficient du même statut depuis très longtemps, puisque non seulement il les a toujours vus, mais ils sont là depuis plusieurs générations, si l'on se fie au témoignage de son grand-père. Tandis que son fils déclare que la332
CORAN ET TALISMANS
coupe A n'est devenue wagf de la mosquée que depuis six ans [données de 19951 : elle était auparavant la propriété d'un membre de la famille, habitant hors de Sanaa, sans pourtant qu'il puisse préciser où (la mention de bien waaf, gravée sur la coupe A, n'indique pas plus de lieu). Au terme de la discussion, une solution est finalement proposée, qui intègre les divers épisodes. Elle consiste à dire que la coupe a été volée par les tribus entrées dans Sanaa lors de l'assassinat de l'Imam Yahyâ, en 194858 ; puis qu'elle a été restituée, sans autre détail sur les péripéties. Il est vrai que ces coupes magiques sont considérées comme des objets précieux (tuhfa).
Finalement, les désaccords entre les informateurs ne portent à aucun moment sur le statut légal des objets. L'information substantielle à retirer d'un point de vue ethnologique concerne, semble-t-il, la perception reçue des hommes de tribu, chez les citadins : selon eux, ce sont des pilleurs, susceptibles de s'emparer de biens wagf-s et, au total, ils représentent des responsables idéals, pour ne pas dire miraculeux.
Au total, ces quatre objets sont des biens donnés par des particuliers en wagf khayri à une mosquée, sans wagfiyya, i. e. d'acte fixant en particulier par écrit leur gestion, dans le cadre des modalités de répartition dessusceptibles de s'emparer de biens wagf-s et, au total, ils représentent des responsables idéals, pour ne pas dire miraculeux.
Au total, ces quatre objets sont des biens donnés par des particuliers
en wagf khayri à une mosquée, sans wagfiyya, i. e. d'acte fixant en particulier par écrit leur gestion, dans le cadre des modalités de répartition des revenus; ils n'apparaissent pas non plus dans les registres du Ministère des \textit{waqf}-s. On peut cependant se demander si ce statut légal avéré ne représente pas par ailleurs une « solution » intéressante étant données le relations conflictuelles entre ce Ministère et ses administrés, et si elle n'est pas une modalité de la résistance, par ex., à une mainmise sur les fondations pieuses, sans même supposer de position idéologique. De plus la coupe A a un itinéraire compliqué, dont la traçabilité ne semble par toujours claire, et n'est « sauvée » de ses lacunes que par des solution pour le moins providentielles. Quoi qu'il en soit, l'inscription gravée su la paroi externe de la coupe A atteste que la mise en \textit{waqf} de ce typ d'objet est séculaire. Son statut juridique est de la sorte officialisé, rend public. Depuis plusieurs générations, aucun interdit majeur ne pèse don sur la pratique qui consiste à rendre un tel bien fondation pieuse et àrattacher à une mosquée. C'est pourquoi, on parlera ici d'institutionnalisation d'une pratique magique.Transmission du savoir et efficacitéLa responsabilité de la « gestion » des quatre objets magiques, en tant que biens wagfs, dépend du systeme régissant la succession à la charge suprême, dans la mosquée concernée. C'est, en l'occurrence, aux représentants de la même famille qu'échoit le titre de Mudir al-Jami depuis des générations*. Hériter de cette responsabilité n'est à aucun moment lié au fait de posséder un savoir en magie. Cela laisse planer la possibilité d'un divorce entre le responsable, ses références, ses convictions personnelles d'une part, et la nature des objets, son devoir de les prêter, d'autre part.
C'est précisément le cas à propos d'un autre objet déposé en wagf : l'un de ses « gardiens » doute de ses vertus magiques, même si cela n'engage pas nécessairement son opinion sur les coupes. Ses critiques sont formulées d'un point de vue rationnel. Le mode d'emploi des coupes semble de plus largement connu??, D'autre part, le statut de wagf khayri de ces deux coupes impose un usage non privé et non personnel. La manière d'utiliser les objets qui nous occupent est-elle alors ou non représentative d'une pratique ou d'une tradition et en définitive sur quoi repose l'efficacité des coupes ?
Si l'on se rapporte aux travaux antérieurs sur les coupes, il apparaît qu'elles sont la propriété de particuliers ou de familles dont rien n'indique un rapport particulier avec les pratiques magico-thérapeutiques. Regardées aussi comme des objets précieux, elles appartiennent au mobilier d'illustres familles et, à ce titre encore, elles figurent dans les Musées* : certaines334
sont des pièces d'orfèvrerie, fabriquées à partir de matériaux nobles, et leurs vertus leur confèrent en outre de la valeur. Elles sont habituellement prêtées à ceux qui les réclament - en principe le malade ou une personne depêchée par lui, et non un praticien - sur leur simple demandes. En effet, le mode d'emploi ne semble pas nécessiter la présence d'un prati-cien. Lorsqu'il est gravé et indique les maux soignés par la coupe, seuls le liquide à y verser et la manière de l'utiliser (boire à petites gorgées, asper-ger, etc.) sont spécifiés (Canaan, 1936,125-26). En outre, à défaut cependant de véritable étude ethnographique, les rares relevés ne mentionnent pas davantage l'existence de praticiens®. Néanmoins, l'étude ethnologique d'Erich Brauer sur les juifs yéménites, qui a l'avantage de donner une chaîne un peu plus complète d'opérations, mentionne le rôle de l'accoucheuse : la femme aidant à l'accouchement (mahjeräh) emprunte une coupe, qui a déjà donné de bons résultats, aux riches familles connues pour en posséder une ; elle la pose alors sur le ventre de la parturiente pour faciliter l'accouchement. S'y ajoute, le cas d'une praticienne juive yéménite qui exerce au Nord du Yémen et que je n'ai pu observer.
En dehors de ces exemples notables, on ne relève en général pas d'intervention d'un intermédiaire reconnu pour ses compétences en magie ou, en tout cas, pour soigner avec ce type d'objet.qu'elles ne sont pas réservées à un utilisateur particulier - de même que dans notre cas de wagf khayr? - et, sauf exception, elles ne mentionnent aucun nom de propriétaire ou de destinataire pour lequel elles auraient été spécialement fabriquées, Sur ce point, elles se distinguent des coupes araméennes et de certains bols arabes en poterie, employés dans les années 30, qui portent le nom du malade et celui de sa mère, suivant des pratiques talismaniques encore très répandues en monde arabe et subsaha-rien (Canaan, 1936, 80, 123-4). Le rapprochement avec les talismans qui sont imprimés et vendus dans des boutiques en monde arabo-musulman, tel le fameux Sab "uhād Sulayman, est tentant, mais une fois utilisés par son acheteur, peuvent-ils être empruntés par tout autre ? Les deux coupes, objet de cette étude, sont donc représentatives dans l'ensemble, car d'un usage collectif qui ne suppose, à aucun stade de leur emploi, le concours d'un homme de magie.
L'efficacité des coupes reposerait-elle alors sur leur fabrication par un homme de magie ? Quelques travaux des années 20 et 30 en font une spécialité de Persans, dont les ateliers auraient été soit en Perse, soit à La Mecque". Annette Ittig, s'appuyant sur le fait que le nom du destinataire-propriétaire n'est pas spécifié sur les coupes, avance qu'elles peuvent être fabriquées par des ateliers ayant une forte productivité, et l'existence de copies médiocres vont dans le sens de cette observation (Ittig, 1982, 94 ;
Canaan, 1936, 117). Mais 19 des coupes déjà publiées se réferent à un événement astral sous les auspices duquel elles auraient été gravées".

volt repor-
te sur les coupes, et lorsque c'est le cas, il n'est pas toujours identifiable comme persan ou/et astrologue ou homme de magie". En outre, sur les 19 coupes mentionnant un événement astral, 13 donnent à la suite une (pseudo ?)-date de fabrication, et sur ces 13, 8 sont « datées » du VICXII's.
On sait déjà que la date de l'une de ces coupes est problématique, car supposée être gravée lorsque la lune était en Scorpion, en l'année
570/1174-75, pour al-Mansûr Asad al-Din Shirkuh, l'oncle de Saladin : or, Shirkuh est mort en 564 H'. Enfin, l'événement astral est parfois évoqué par des formules quasiment identiques. A-t-on affaire à un phénomè. ne historiquement situé, qui consistait à vouloir placer l'efficacité des coupes sous le signe ou l'influence d'un événement astral, sans vraiment en maîtriser la connaissance ? Il est clair, en tous les cas, que l'on a affaire pour partie à des copies, même si - c'est très important de le souligner
- aucune des coupes publiées n'est identique, de même qu'aucune de celles que j'ai vues. Les artisans apparaissent donc comme chevronnés, car capables d'introduire des variations tout au moins dans l'architecture du décor, et certains ont eu éventuellement des compétences en talisma-nique; on songe moins à des praticiens de la magie auxquels aurait été confié le soin de graver. L'étendue de leur savoir (religion, astrologie, magie,.) et de leur technique peut être exceptionnel, mais certains imi tent les motifs, en particulier magiques, sans plus en avoir la clé. Cela signifie que même dans le cas où les inscriptions des coupes aiguillent sur l'existence d'hommes de magie au moment de leur fabrication, cela n'est pas toujours le cas. Enfin, dans l'hypothèse où les formules se référant àun événement astral auraient été copiées, doit-on penser qu'il s'agit simplement de faux ? Ne faut-il pas plutôt considérer que l'artisan procédait ainsi parce que la simple évocation de l'événement astral, doublée parfois du nom d'un souverain, ou rapporté à l'époque de Saladin, venant en addition des écrits et motifs gravés sur la coupe, ajoutait à la valeur et à l'efficacité de la coupe (Savage-Smith, 1997, 73) ? Certains textes gravés sur les coupes, ne laissent aucun doute sur le fait qu'être une copie, représente une valeur ajoutée. D'après Canaan, les Palestiniens expliquent ainsi l'origine des coupes : les bons anges en employaient de semblables pour faire leurs ablutions. Ils en oublièrent un jour quelques exemplaires à côté de la source où ils avaient l'habitude de se rassembler pour se laver. Un homme passant par là les trouva et s'en empara. Les propriétés miraculeuses de la coupe furent bientôt découvertes. Des copies en furent réalisées, qui manifestèrent les mêmes propriétés (Canaan, 1923,130;
1936. 127)
 Comme au premier jour, selon ce récit, la coupe abandonnee puis reproduite conserve son pouvoir.
Il viendrait alors principalement des écritures et figures gravées qui le communiquent au malade par le biais d'un liquide absorbé, aspergé, la surface la plus investie par cet ensemble de signes restant l'intérieur des coupes". Il faudrait également compter avec les propriétés des différents métaux utilisés, généralement le bronze et le cuivre jaune. Mais si le liquide versé varie en fonction du mal, il ajoute ses propres vertus à l'opération et l'eau elle-même possède ses propriétés particulières" : à preuve, la recherche d'une eau « plus active ». Canaan mentionne que le en fin d'operation
souvent par l'indication que les coupes sont eprouvées (« \textit{hâ}dhihi al-tasa mujarraba »), plus rarement par une « garantie » de qualité, en tant que copies, sans doute d'originaux réputés. Certaines coupes sont reconnues nettement plus efficaces que d'autres, telle celle dédiée à Salâh alDin (Zéki Pacha, 1916,254), ou comme Brauer l'a relevé. La répétition d'une formule ou la multiplication d'un même type de sentence, des versets par exemple, voire l'ancienneté des coupes y contribue®. Pour finir, la présence de ces objets dans la mosquée leur confère certainement un pouvoir supplémentaire auprès des utilisateurs. Cette remarque va, du reste, dans le sens de certaines inscriptions figurant sur les coupes, qui soulignent qu'elles ont été produites à La Mecque ou pendant le mois de Ramadân.
En résumé, les coupes sont généralement utilisées sans qu'œuvre un homme de magie ou quelqu'un qui, à des degrés divers, aurait des notions de magie. Elles sont d'usage collectif, elles ne sont en principe pas fabriquées pour un destinataire en particulier. De ce point de vue, les deux coupes, objet de notre étude, peuvent donc être ramenées au cas général.
Quant à la fabrication des coupes en général, nous savons encore fort peu sur leurs artisans. Cependant même dans les cas où les inscriptions sur les coupes évoquent le recours à une astro-magie, il peut s'agir de la simple référence à un événement astral, sans rapport certain avec le moment de leur fabrication. On ne peut en induire d'emblée que l'artisan a effectivement eu des notions de magie ou d'astrologie. L'efficacité des coupes guérisseuses serait donc intrinsèque en ce sens qu'elle ne supposerait pas nécessairement le concours d'un homme de magie.
\section{Du magico-thérapeutique}
Le rapprochement entre les écrits, représentations, signes, et figures géométriques apparaissant sur les parois des coupes et les talismans a339
déjà été fait, non seulement par des études, mais il est aussi bien attesté par l'inscription suivante gravée sur certaines coupes : « Hadhihi al-tilas-mât etc. ». L'expression, au pluriel, semble bien désigner en effet ce qui est gravé sur leurs parois%.
À notre connaissance cependant, aucune dénomination vernaculaire ne désigne les coupes elles-mêmes comme talisman. D'autre part, les spécificités que l'on vient de dégager, à savoir que les coupes ne font interve-nir, généralement, aucun praticien dans leur mode d'utilisation, qu'elles ne sont pas fabriquées pour un destinataire nominalement désigné, qui en aurait besoin pour résoudre un problème précis et pour son usage particu-lier, et enfin, que l'on constate largement un phénomène de copies, et de copies de copies, tout ceci en ferait une catégorie très particulière de talis-mans®. Rappelons, au sujet du dernier point, l'analyse très eclairante que fait Constant Hamès à propos de l'usage talismanique du Coran : il y a pratique talismanique à partir du moment où un homme de l'art œuvre en re-travaillant le « matériel » coranique (Hamès, 2001, 95).
Enfin, reste le problème de l'apport du liquide dans l'usage des coupes. A ce titre, l'exemple yéménite est particulièrement intéressant et clair, car il s'inscrit dans le cadre d'une médecine domestique qui « attaque le mal » par addition des thérapies. C'est pourquoi, la dénomination de « coupes magico-thérapeutiques » a paru préférable. 
Les deux coupes étudiées ici appartiennent à la collection de biens de fondation pieuse (wagf) d'une mosquée yéménite ; l'une d'entre elle porte, gravée sur sa paroi externe, sa date de mise en wagf, en 1313/1895.
96. L'étude des écrits, représentations et figures géométriques des deux coupes fait largement apparaître un registre commun avec la talisma-nique. Le motif du soleil, central, structure leur architecture interne. Ne faut-il pas voir dans le décor des coupes l'ambition de représenter le monde clos du cosmos, aux destinées duquel préside Dieu, le seul à pouvoir réellement guérir, le Suffisant ? Leur mode d'emploi consiste à verser un liquide (ou du miel) qui est absorbé par le malade (ou son envoyé).
Une véritable étude ethnographique manque cependant, pour le Yémen, mais même à titre comparatif. Les coupes sont particulièrement utilisées au Yémen pour faciliter l'accouchement. Le bouillon de viande que l'on verse pour ce faire, illustre le fait que le liquide n'est pas seulement conducteur ou à la rigueur stimulateur, dans l'économie de la guérison par les coupes, mais qu'il apporte ses qualités propres, en addition de celles des écrits et représentations. En médecine domestique, sur les hauts plateaux yéménites, le recours simultané à des médecines et à des médications différentes est courant. Ces objets sont présentés comme wagfs oraux, statut a priori paradoxal. Néanmoins, leur statut juridique de biens \textit{waqf}-s d'une mosquée, permet de dire qu'il y a ici institutionnalisation d'une pratique magique, d'une part. D'autre part, qu'ils relèvent des fondations « d'orientation publique ou charitable » (wagf khayri). Ce statut fait qu'elles sont prêtées à toute personne qui les demande à condition bien sûr que ce soit pour un malade. Il a aussi pour conséquence que le mode de transmission de la responsabilité de ces objets est lié aux règles de succession à la charge de l'intendance de la mosquée, et n'a donc pas de rapport avec une quelconque compétence en magie. La question se pose alors de savoir si le mode d'emploi des deux coupes, tel que transmis par la mosquée, s'inscrit dans une tradition, en est le reflet. A terme, on se demande sur quoi repose leur efficacité. Sur la base des études déjà nombreuses sur ces objets, il est possible d'avancer prudemment que l'usage des coupes est généralement collectif, qu'elles ne sont pas fabriquées au nom d'un bénéficiaire particulier, qu'aucun praticien de la magie, enfin, n'intervient dans leur mode d'emploi. Les deux coupes du Yémen entrent alors dans le cas général. En outre, si nous ne disposons encore que de peu d'informations sur leur fabrication et leurs artisans, les événements astraux sous l'égide desquels les coupes sont réputées avoir été fabriquées, comme en font état des inscriptions sur la paroi de certaines d'entre elles, même s'ils suggèrent le recours à une astro-magie, ne laissent pas une totale certitude sur le rapport réel entre la référence à cet événement astral et leur fabrication. Un mythe d'origine des coupes, relevé par Canaan, véhicule l'idée que la copie peut avoir autant d'effet que l'original. L'efficacité des coupes guérisseuses serait donc intrinsèque en ce sens qu'elle ne supposerait pas nécessairement le concours d'un homme de magie: elle reposerait essentiellement sur les écrits, représentations et figures géométriques gravées sur leurs parois. Le lien déjà établi entre ces éléments gravés et les talismans est pertinent, mais il est pru-dent, à notre avis, de désigner les coupes comme magico-thérapeutiques, plutôt que comme talismaniques, sous peine de passer sous silence le rôle important du liquide qui doit y être versé.

\section{Bibliographie}
\begin{itemize}
    \item 

BARKAI, Ron, 1996, « Médecine, astrologie et magie », in A l'ombre d'Avicenne. La médecine au temps des c\textit{alif}es, Institut du Monde
Arabe/Snoeck-Ducaju \& Zoon, 1996, 189-193.
BEY, Dr Ahmed Issa, 1928, Histoire des Bimaristans (Höpitaux) à l'époque islamique. Discours prononcé au Congrès Médical tenu au Caire à l'occasion du centenaire de l'Ecole de médecine et de
l'Hôpital Kasr-el-Aîni, Le Caire, imp. Paul Bamey, 116 p.
AL-BÜN1, Ahmed ben Ali, (a), s. d., Shams al-madrif al-kubra.
Beyrouth, al-Maktaba al-thaqâfiyya, 576 p.
\item (b). s. d., Manbo usúl al-hikma, Beyrouth, al-Maktaba al-haditha
li-al-tibâa wa-al-nashr, 335 p.
AL-DAYRABI, s. d., Mujarrabát al-Dayrabi al-kabir, al-musamma bi-
fath al-malik al-mujid al-mu' allaf li-naf al-abid wa-bi-\textit{hâ}mishiha al-
shaykh Abi Abd Alläh Muhammad b. Yüsuf al-Sanúsi al-Hasani
Beyrouth, Sanaa, al-Maktaba al-haditha, Maktabat al-Sanhani, 160 p.
DEGUILHEM, Randi, 1995, Le Wagf dans l'espace islamique. Outil de pouvoir socio-politique, Damas, IFD, 337 + 100 p.
DOUTTÉ, Edmond, 1994, Magie et religion dans l'Afrique du Nord, Paris, J. Maisonneuve et P. Geuthner, 1994 (Ire éd. 1908), 617 p.
DOZY, R., 1881, Supplément aux dictionnaires arabes, Leyden, E. J.
Brill, 1881,2 t.
DUGAT, Gustave, 1853, « Etudes sur le traité de médecine d'Abou
Djäfar Ah'mad, intitulé : Zad al-Moçafir 'La provision du voya-geur' », Journal Asiatique, I, 288-253.
DUNLOP, D. M., 1960, « Bimâristân », E.1.2, Leyden-Paris, E. J. Brill.
G.-P. Maisonneuve, 1259-1262.
EL-BOK\textit{hâ}RI, 1984, Les traditions islamiques, trad. et notes par O.
Houdas, Paris, A. Maisonneuve, t. 4.
ÉTIENNE, Marc, 2000, Heka. Magie et envoûtement dans l'Egypte ancienne, Paris, Réunion des musées nationaux, « Les dossiers du musée du Louvre », 126 p.
HAMES, Constant, 2001, « L'usage talismanique du Coran », Revue de l'histoire des religions 218, 1, « Les usages du Livre saint dans l'islam et le christianisme », 83-95.
JANSEN, J. J. G., 1996, "al-Shawkâni", E.J.2, Leiden, E. J. Brill, 390.
LANE, E. W., 1877, Arabic-English Lexicon, Londres, William and
Norgate, 2 t.
MERCIER, Jacques, 1979, Rouleaux magiques éthiopiens, Seuil, 37 p. +
40 tab.
MERMIER, Franck, 1988, Les souks de Sanaa et la société citadine (République Arabe du Yémen), thèse de doctorat de l'EHESS (dir.
Gilbert Grandguillaume), 2 tomes, 573 p.
PORTER, Venetia, 1987, "The Art of the Rasulids", in Werner Daum (ed.), Yemen. 3000 Years of Art and Civilisation in Arabia Felix, Innsbruck, Frankfurt/am/Main, 232-253.
PSEUDO-MAJRITI, 1933, Das Ziel des Weisen. 1. Arabischer Text, her-
ausgegeben von Hellmut Ritter, Leipzig, B. G. Teubner, "Studien der
Bibliothek Warburg", VI + 416 p.
RITTER / PLESSNER, 1962, Das Ziel des Weisen von Pseudo-Majriti,
translated into german from the arabic by Hellmut Ritter and Martin
Plessner, London, The Warburg Institute/Univ. of London, LXXVIII
+ 435 р.
SERJEANT aLsAA L Ons les c\textit{alif}es, Paris, Snoeck-Ducaiu \& 7 on cerne. la medecine au
" temps des c\textit{alif}es, Paris, Snoeck-Ducaju \& Zoon/Institut du Monde Arabe.
ANSALDI, Cesare, 1933, Il Yemen nella storia e nella leggenda, Rome.
ARTS D'ORIENT, juin 1999, Catalogue Paris-Drouot Montaigne de la vente du lundi 7 juin, pièces 111à 113, 38.
BRAUER, Erich, 1934, Ethnologie der jemenitischen Juden, Heidelberg,
XIX + 402 p. + 8 tableaux.
CANAAN, Tewfik, 1914, Aberglaube und Volksmedizin im Lande der
Bibel, Hamburg, L. Friederichsen \& co, XI+ 153+6 tables.
\item 1923, "fiâsit er-radifeh (Fear Cup)", Journal of the Palestine Oriental Society [JPOS] 3, 122-131.
\item 1936, "Arabic Magic Bowls", JPOS 16, 79-127.
\item 1937, "The Decipherment of Arabic Talismans", Berytus 4, 69-110,
\item 1938, "The Decipherment of Arabic Talismans", Berytus 5, 1938,
141-151.
CANOVA, Giovanni, 1995a, "Nota su una coppa magica egiziana", in :
Azhàr. Studi arabo-islamici in memoria di Umberto Rizzitano (1913-
1980), a cura di A. Pellitteri e G. Montaina, Palermo, 59-68.
\item 1995b, « La tâsat al-ism : note su alcune coppe magiche yemenite », Quaderni di Studi Arabi 13, « Divination, magie, pouvoirs au
Yémen », 73-92.
CASANOVA, M., 1891, « Notice sur une coupe arabe », Journal asia-tique VIII, 17, 323-330.
\item 1921, « Alphabets magiques arabes », Journal Asiatique XI, 18,
37-55, et 1922, XI, 19, 250-262.
CINGOLANI, Cinzia, 1987-88, Coppe magiche del museo Gayer-Anderson del Cairo, tesi di laurea, Università di Venezia, (Je n'ai pu consulter ce travail).
DAVIS, S., 1955, "Divining Bowls, Their Uses and Origin: Some African Examples and Parallels from the Ancient World", Man 55
(143), 132-5.
FODOR, Alexandre, 1990, "Amulets from the Islamic World", Catalogue of the Exhibition held in Budapest in 1988, The Arabist, Budapest studies in arabic 2, Budapest, 160 sq.
GRASSI, Vincenza, 1987, "Una 'coppa magica' proveniente dall'Egitto",
Studi magrebini XIX, 65-89, 7 pl.
ISMAIL, Muhd, 1922-23, "Two Arabic Medecine-Cups", Journal of the
Bombav P
f the Royal Asiatic Society (JBBRASI XXVI, n° 2.
ologiques
alicmanic Bowl". AnD344
KRISS. Rudolf / KRISS-HEINRICH, Hubert, 1962, Volksglaube im
Bereich des Islam II : Amulette, Zauberformein und Beschwörungen, Wiesbaden, Otto Harrassowitz, 126-137, pl. 100 à 111.
LANE, Edward William, 1923 (?). The Manners and Customs of the
Modern Egyptians, London/Toronto, E. P. Dutton, (réimp.).
XXV+630, 255, 261.
NASSAR, Nahla, 2001, « Notice de la coupe magique au nom de Nûr al-
Din Mahmud ibn Zangì » in : L'Orient de Saladin. L'art des Ayyoubides, Paris, IMA/Gallimard, n° 211, 223.
OMAN, Giovanni, 1981, « Le 'coppe magiche' nella medicina popolare
araba », in : La Bisaccia dello Sheikh. Omaggio ad Alessandro
Bausani islamista nel sessantesimo compleanno, Quaderni del semi-
nario di iranistica, uralo-altaistica e caucasologia dell'Universita'
degli sudi di Venezia 19, 215-219.
\item 1987, « Materiali per lo studio delle coppe magiche nella medicina
popolare araba », Quaderni Ticinesi [di) Numismatica e Antichita classiche 16, 337-358.
"REGOURD, Anne, 2002, « Une coupe thérapeutique du Yémen »,
L'aventure des écritures (CDRom), Paris, BNF, RMN.
\item 2005, « Une coupe magico-thérapeutique au Louvre (inv. MAO
1284) », in Livres de Parole. Torah, Bible, Coran, Catalogue de l'exposition, Paris, B.N.F., 187-9 + illustr.
REHATSEK, E., 1874, Description de deux coupes dans Indian
Antiquary III, 36.
\item 1875a, "Explanations and Facsimiles of eight Arabic Talismanic
Medecine-Cups", Journal of the Bombay Branch of the Royal Asiatic Society [JBBRAS] X(1871-74), 150-162.
\item 1875b, « The Evil Eye, Amulets, Recipes, Exorcisation, \&... », JBBRAS X(1871-74), 299-315.
\item 1880, « Magic », JBBRAS XIV (1878-1880), 199-218.
REICH, S., 1937-38, « Quatre coupes magiques », Bulletin d'Etudes
Orientales 7-8, 159-175, pl. XI-XII.
REINAUD, J. T., 1828, Monumen[t)s arabes, persans et turcs, du Cabinet de M. le Duc de Blacas et d'autres cabinets (ou : Descriptions des monuments musulmans du cabinet de M. le Duc de Blacas), II, Paris.
ROSU, Arion, 1992, « Une coupe magique d'époque moghole au musée
Guimet », Journal asiatique CCLXXX, 3-4, 251-277.
SARNELLI, Tommaso, 1934, « Notizie preliminari sui risultati della mia
missione sanitaria nell'alto Yemen : con particolare riguardo alla
medicina indigena », Archivio italiano di scienze mediche e coloniale
15, 1-43.
SAVAGE-SMITH. Emilie. 1997  
 
Mopping the Universe, The Nasser D. Khal F Colection of Islamie
Ma, vol. XII, London and Oxford The tour Foundation, in associa-
Ain with Azimuth Ed, and Oxford Univers Press 72-99;
SPOER, H., Henri, 1935, "Arabic Magie Medicinal Bowls", Journal of
African and Oriental Society [JAOS] 55, 237-256.
1938, "Arabic Magic Bowls II : an astrological Bowl", JAOS.
58,366-383.
VON GLADISS, Almut, 1999, "Medizinische Schalen. Ein islamisches
Heilverfahren und seine mittelalterlichen Hilfsmittel", Damaszener
Mitteilungen 11, 147-161, pl. 22 à 25.
\item 2001, « Notice de la coupe magique Syrie ? 621H/1224 », in : L'Orient de Saladin. L'art des Ayyoubides, Paris, IMA/Gallimard, n° 211,224.
WALKER, John, 1934, Folk Medecine in Modern Egypt, London, Luzac, 128 p. + 2 pl.
WIET, Gaston, 1932, Catalogue général du Musée arabe du Caire.
Objets en cuivre, Le Caire, IFAO, 77-78, n° 3213 (pl. LXIII) ; 94,
n° 3862 (pl. LXI) ; 95, n° 3897 (pl. LXIII) ; 95-96, n° 3906 (pl. LXII) ;
100, n° 3965 ; 101, n° 3981 ; 121, n° 4431 (pl. LX) ; 150, n° 9364 ;
150-51, n° 9380 ; 151, n° 9420 ; 265-69, n° 530 à 544 ; 286, index « coupe magique ».
\item 1958, « Inscriptions mobilières de l'Egypte musulmane », Journal Asiatique 246, 237-285, pl. 8.
\item et alii, 1937-1954, Répertoire chronologique d'épigraphie arabe, Le Caire, IFAO, 1937, VIII, n° 2952 ; 1937, IX, n° 3319, 3389, 3392-
3394 ; 1939, X, n° 3648; 1944, XIII, n° 5043-5045 : 1954, XIV, n° 5247,5262, 5455.
WÜNSCHE, Michela, 1993, « Traditionelle Heilmethoden im Jemen »,
Jemen Report, 24/2, 31-2
YA \textit{kâf}I, YA SHÄFI, 1999, The Tawfik Canaan Collection of Palestinian
Amulets. An exhibition, October 30, 1998-February 25, Birzeit University, 50 p.
ZEKI PACHA, Ahmed, 1916, « Coupe magique dédiée à Salah ad-Din (Saladin). Titres royaux, tolérance et portrait de Salâh ad-Din », Bulletin de l'Institut Egyptien X, Sè série, 241-289. « Note additionnelle sur une seconde coupe de Salah ad-Din », en fin d'article.
ZWEMER, Samuel, M., 1920, The influence of animism on Islam. An account of popular superstitions, New York, MacMillan, VIII+246
p. + 1 pl. (coupe).
\end{itemize}
 

 %\chapter{Introduction à l'Arabe}

\TArabe{إِنَّ اللَّهَ وَمَلَائِكَتَهُ يُصَلُّونَ عَلَى النَّبِيِّ يَا
أَيُّهَا الَّذِينَ آَمَنُوا صَلُّوا عَلَيْهِ وَسَلِّمُوا تَسْلِيمًا}
 

 %\chapter{Liste des théologiens}

\section{Grands théologiens Musulmans}


\subsection{al-Ġazālī}

le joyau d'al-Ġazālī~: \emph{Le Tabernacle des Lumières}, traduit
par Deladrière, Paris, Seuil, texte très dense et très profond aux
implications nombreuses.
\cpageref{theol:AlGazali1,theol:AlGazali4,theol:AlGazali5,theol:AlGazali6,theol:AlGazali7,theol:AlGazali9,theol:AlGazali10,theol:AlGazali11,theol:AlGazali13,theol:AlGazali14,theol:AlGazali16,theol:AlGazali17,theol:AlGazali18,theol:AlGazali19,theol:AlGazali20,theol:AlGazali21,theol:AlGazali22,theol:AlGazali23,theol:AlGazali24}
\pageref{theol:AlGazali29}
\pageref{theol:AlGazali2}
\pageref{theol:AlGazali3}
\pageref{theol:AlGazali8}
%\pageref{theol:AlGazali31}
\pageref{theol:AlGazali25}
%theol:AlGazali31,theol:AlGazali32,theol:AlGazali33,theol:AlGazali34,theol:AlGazali35,theol:AlGazali36,theol:AlGazali37,theol:AlGazali38} 
%\label{theol:AlGazali1}
\section{Ibn Taymiyya}


 

C'est le très grand penseur (controversé) du 13\textsuperscript{ème}
siècle. Un certain nombre de ses ouvrages ont été traduits (souvent mal,
je sélectionne les meilleures traductions).


\emph{La lettre de Palmyre} traite de deux questions théologiques~: les
attributs divins et la prédestination~!

\includegraphics[width=1.27534in,height=1.63243in]{Images/image26.png}

-~Ibn Taymiyya, \emph{Réponse Raisonnable aux Chretiens ?} édité,
traduit et commenté par Laurent Basanese, sj., Ifpo, 2011.

-~Ibn Taymiyya, \emph{Musique et danse selon Ibn Taymiyya}: Le livre du
\emph{Samâ°} et de la danse (\emph{Kitâb al-Samâ° wa l-Raq.s}), Paris,
Vrin, 2000.

-~Ibn Taymiyya, \emph{Pourquoi les savants divergent,} Al-Hadith
éditions, 2012


Voir p. \pageref{ibn-taymiyya}.

\section{Autres théologiens classiques}
\paragraph{Ibn Hanbal}

\pageref{Theol:IbnHanbal1}

\paragraph{Ibn Salah}
Ibn Salah
\pageref{Ibnsalah1}

\paragraph{Ibn Khaldūn}
Le penseur andalou Ibn Khaldūn \pageref{theol:IbnKhaldun} 

\paragraph{Ibn Qutayba}
Ibn Qutayba -- si ce nom ne vous est pas encore familier, cela devrait
faire `tilt' car nous l'avons rencontré au début de cette leçon. Il a
écrit un Traité sur comment rendre compte et comprendre les divergences
dans le \emph{ḥadīṯ.} A-t-il été traduit en français~? La réponse est en
note 3 --
\pageref{Theol:IbnQutayba1}

\paragraph{Kalābāḏī}
  est un auteur persan, mort aux environs de 990. Cet ouvrage
cherche à réconcilier le soufisme et l'orthodoxie. 
\pageref{theol:Kalabadi}


\paragraph{ʿAlī Ṭanṭāwī} \label{theo:AliAlTawani}
{Ali Al tantawi est originaire d'une
famille de lettrés égyptiens qui a émigré à Damas à la fin du XIXème
siècle.


Il s'est opposé à l'impérialisme occidental dans les pays
arabes et, en particulier, à la présence de la France comme mandataire
en Syrie et celle de l'Angleterre en Irak. Après l'indépendance de la
Syrie, en 1947, ses positions contre le communisme, qu'il considère
incompatible avec l'Islam lui valent d'être menacé dans son propre pays.
En 1963, il quitte la Syrie pour l'Arabie Saoudite et devient
enseignant.
Extrêmement populaire dans son pays d'adoption, il a
présenté des programmes à la radio et à la télévision pendant un quart
de
siècle.}


\subsection{Ibn Toumart}
\label{IbnToumart}
\mn{E.B., « Ibn Toumart », in 23 | Hiempsal – Icosium, Aix-en-Provence, Edisud (« Volumes »,
no 23) , 2000 \href{http://
encyclopedieberbere.revues.org/1629}{revue}}


Ibn Toumart est la personnalité religieuse et politique la plus marquante du Maghreb au
XIIe siècle. Fondateur du mouvement almohade*, il devait préparer la revanche des Sanhadja
montagnards contre l’empire déjà vacillant des Almoravides. Bien que ses disciples aient
manipulé sans vergogne sa généalogie pour le rattacher à la descendance du Prophète et en
faire, donc, un chérif, il est sûr qu’Ibn Toumart était issu d’une tribu du Sous, celle des Hergha,
appartenant au groupe montagnard des Maçmouda.
 L’un de ses premiers disciples, le pieux el Baïdaq, se fit son chroniqueur et grâce à son
récit, souvent dithyrambique, il est possible de saisir l’évolution spirituelle de celui qui
devait mériter le titre de Mahdi Almohade et le qualificatif d’Impeccable. Célèbre dès son
adolescence, pour son zèle religieux et son érudition qui lui avait fait donner le surnom d’asufu
(le tison, le flambeau, en berbère), Ibn Toumart quitta un beau jour son village d’Igliz et
ses montagnes pour compléter, en Orient, sa connaissance de l’islam et jeter les bases d’une
réforme radicale.
 Son séjour en Espagne n’est pas assuré, mais demeurent des concordances étroites entre
les textes d’Ibn Hazm et ses propres propositions. En revanche, sa présence à Bagdad est
pleinement confirmée, alors que son passage à Damas est peut-être légendaire et les entretiens
qu’on lui prête avec Ghazali certainement inventés.
 Dix ans après son départ d’Igliz, Ibn Toumart entreprend un long voyage de retour au Maghreb,
au cours duquel il multiplie les étapes, passant par Alexandrie, Tripoli, Mahdia, Tunis,
Constantine et Béjaia. Sa condamnation des moeurs citadines relâchées provoque souvent des
échauffourées. A Béjaia, ses violences verbales déclenchent la fureur populaire contre lui. Le
sultan hammadite, qui l’avait d’abord bien accueilli, lança ses sicaires à sa poursuite, mais Ibn
Toumart trouva refuge dans la tribu voisine, celle des Beni Urigol, dans le village de Melala.
\paragraph{
La doctrine almohade}
 Ibn Toumart y élabore sa doctrine et réunit ses premiers disciples. Le plus cher à son coeur,
celui qu’il considère comme l’homme providentiel qui doit lui succéder, est Abd el Moumen,
le fils d’un potier de Nédroma (Algérie occidentale). El Baïdaq nous a laissé le récit émouvant
de la désignation du futur calife. Le réformateur proclama un soir en prenant sa main : “La
mission sur laquelle repose la vie de la religion ne triomphera que par Abd el Moumen ben Ali,
le flambeau des Almohades.” Celui-ci, en pleurant, dit avec humilité : “Ô Maître, je n’étais
nullement qualifié pour ce rôle, je ne suis qu’un homme qui cherche ce qui pourra effacer ses
péchés.” – “Ce qui te purifiera de tes péchés, répondit Ibn Toumart, ce sera précisément le rôle
que tu joueras dans la réforme de ce monde.”
 Une conversation avec deux pèlerins de l’Atlas qui passaient par Bougie est l’occasion du
départ des premiers Almohades vers le Maghreb el Aqsa. La petite troupe, d’une dizaine de
personnes, gagne Marrakech non sans avoir semé la bonne parole et causé quelques troubles
dans les villes traversées : Tlemcen, Oujda, Taza, Fès, où Ibn Toumart se fait remarquer par
le saccage des magasins des marchands de musique, contre lesquels il semble avoir eu une
aversion certaine. Il réitère à Marrakech, brisant à coups de bâton instruments de musique
et jarres de vin, pourchassant sous les huées la soeur de l’émir almoravide, qui chevauchait
dévoilée dans les rues de la capitale.
 Après la prise de Tin Mel (1123), il se proclame Mahdi et, de retour dans les tribus Masmouda,
ses frères de race, il organise solidement la communauté almohade avec un soin et une
connaissance des hommes qui font de ce clerc un grand homme d’État. Il crée un véritable
État montagnard, solidement organisé, disposant d’une armée fanatisée chargée de répandre
la doctrine almohade jusqu’en Ifriqiya et en Espagne.

Nous retrouvons dans cette réforme la même tendance innée vers le rigorisme moral et la
simplicité doctrinale que nous ont révélés tous les schismes et hérésies nés en Berbérie à travers
les siècles.
Dans la condamnation absolue des richesses de ce monde et de ses frivolités, c’est la voix
d’Ibn Toumart qui tonne, mais elle fait écho à celle, non moins véhémente, de Tertullien. La
lente marche des Berbères vers le Dieu unique semble ici se parachever dans la proclamation
de l’Unicité absolue de Dieu, dont Ibn Toumart rejette jusqu’aux adjectifs (le Puissant,
le Miséricordieux, le Victorieux) que lui dorment les musulmans, parce qu’ils risquent de
faire apparaître comme divisible la puissance divine. La conséquence inévitable de la toutepuissance
de Dieu ainsi comprise est la prédestination de tous les êtres créés : chacun doit
attendre dans la soumission totale ce qui lui a été assigné de toute éternité.
Cette forme de l’islam ne peut qu’être fanatique, elle ne supporte ni relâchement des moeurs,
ni relativisme dans le dogme, ni présence d’Infidèles.
11 Ces données concordaient trop bien avec l’intransigeance fondamentale des Berbères pour ne
pas aboutir : aussi, sous Abd el Moumen, le raz de marée almohade balaya le Maghreb de
toute impureté. C’est alors, semble-t-il, que disparurent les dernières communautés chrétiennes
autochtones.
\paragraph{L’État almohade}
Respectueux des traditions tribales des Berbères du Haut Atlas, Ibn Toumart organisa son
gouvernement en établissant une hiérarchie entre différents conseils imités des assemblées
tribales. Au sommet siège le Conseil des Dix, qui sont les premiers et les plus fidèles
compagnons (Abd el-Moumen*, Abou Hafs Omar*, El Bachir...). Au-dessous du Conseil
des Dix, le Conseil des Cinquante est composé de contribules d’Ibn Toumart, des Hergha et
d’autres Maçmouda de Tin Mel ou des Hintata. Les différentes tribus de la montagne étaient
ainsi représentées dans ce Conseil dont les pouvoirs étaient restreints.
Toute la société almohade était strictement hiérarchisée. A l’intérieur des groupements
ethniques apparaissait une autre hiérarchie, fondée sur les fonctions exercées, depuis celles
des compagnons les plus proches jusqu’à celles confiées aux abid (serviteurs noirs). Au
sommet de la pyramide, le Mahdi tenait solidement les rênes d’un pouvoir absolu. Cette
domination reposait sur une logique implacable : tout Almohade suspecté de tiédeur risquait
l’élimination : ainsi lors de la “journée du tri” plusieurs milliers d’almohades “infidèles” furent
massacrés. C’est par de telles actions qu’Ibn Toumart réussit à construire l’État almohade et à
assurer la naissance de la nouvelle dynastie moumenide. Seuls le prestige et la volonté d’Ibn
Toumart réussirent à faire admettre Abd el-Moumen comme le successeur désigné du Mahdi.
Encore fut-il nécessaire de cacher la mort de celui-ci pendant plus de deux ans avant de faire
reconnaître le nouveau souverain par les Cheikhs almohades.
\paragraph{références}
voir p. \cpageref{theol:IbnTumart1}

\section{Théologiens modernes}
\subsection{Tarik Ramadan}
 \begin{itemize}
  \item Paradoxe de Tarik Ramadan~: dire que le radicalisme vient de
    l'occident. Et ensuite, valoriser le communitarisme pour refuser les
    coutumes locales et en particulier celles de la laicité.
  \end{itemize}
  
Voir réflexion sur le moratoire.

%-------------------------------------------------
\section{Théologiens pronant un retour à l'Islam pur}
\label{theol:SayyidQutb}
\paragraph{Sayyid Qutb}
Sayyid Qutb (arabe :\TArabe{ سيد قطب,} Sayyid Quṭb), né le 9 octobre 1906, dans le sud de l'Égypte, et exécuté par pendaison le 29 août 1966 au Caire, est un poète, essayiste et critique littéraire égyptien, puis un militant musulman membre des Frères musulmans. Il entrera en rupture avec ces derniers à la suite du développement d'une idéologie islamiste radicale, le \textbf{qutbisme}.


Les idées de Sayyid Qutb se résument schématiquement ainsi :
\begin{itemize}
    \item 
L'islam est en crise. Les millions de gens qui se réclament de l'islam n'en comprennent en réalité pas grand-chose, ils ne sont pas de vrais musulmans. Qutb prononce donc une condamnation très forte de la société égyptienne contemporaine.
  \item 
Un retour aux vraies valeurs de l'islam est nécessaire. Malheureusement les masses populaires manipulées par le nassérisme sont incapables de s’en sortir. Il appartient donc à une élite de guider les masses en jouant le même rôle que celui des compagnons du prophète de l'islam; cette élite qu'il appellera dans plus d'un ouvrage "\textit{annawâte assoulba}" (littéralement "le noyau dur"). Le but est de réislamiser la société.
  \item 
L'islam apporte une solution complète à tous les problèmes, politiques, économiques, sociaux. En revanche, les influences occidentales sont dangereuses et nuisibles. Il dénie le qualificatif de « civilisation » (notamment dans son livre Mushkilât al-hadâra : Problèmes de la civilisation) aux blocs de l'est (socialiste) et de l'ouest (capitaliste), qu'il renvoie dos à dos comme représentant deux faces d'une même entité qu'il qualifie de \textit{Jahiliya} (ignorance). Ce terme, qui renvoie à la période précédant l'islam durant laquelle l'Arabie était polythéiste et ignorante donc du vrai Dieu, a une forte connotation négative dans l'imaginaire musulman.
  \item 
L'idée d'une « lutte contre les Juifs » fut aussi présente dans la pensée de Sayyid Qutb, qui écrivit au début des années 1950 l'opuscule \textit{Notre combat contre les Juifs}. Dans son commentaire de la sourate 5, Sayyid Qutb réaffirmera l’accusation : « Depuis les premiers jours de l’islam, le monde musulman a toujours dû affronter des problèmes issus de complots juifs. (…) Leurs intrigues ont continué jusqu’à aujourd’hui et ils continuent à en ourdir de nouvelles. » 
\end{itemize}

%--------------------------------------------------
\section{Théologiens libéraux}

\paragraph{Mohammed Arkoun}
Savant à la pensée profonde, Mohammed Arkoun (1928-2010) était également un intellectuel engagé. Son analyse serrée des processus à l’œuvre dans l’islam d’hier était indissociable de ses appels répétés à une réforme des sociétés islamiques contemporaines. Il n’a cessé de porter ce message dans les divers colloques où il était convié, y compris là où l’on ne s’attendrait guère à croiser un islamologue : à un congrès de psychanalystes lacaniens, dans des conférences sur la condition féminine…
Il a choisi de consacrer les dernières années de sa vie à retravailler les textes issus de ces rencontres. Traitant de la nécessité de la réforme, voire de la « subversion » de l’islam, de l’ouverture lacanienne à la parole et à la « raison émergente », de la condition féminine en islam ou encore du rapprochement entre sunnites et shî‘ites, ils montrent combien la pensée de Mohammed Arkoun est plus que jamais féconde pour penser notre époque.
Voir  \cpageref{theol:Arkoun1,theol:Arkoun2} 
\label{theol:Arkoun3}

\paragraph{Rachid Benzine}.
Islamologue et historien, Rachid Benzine s’est intéressé à ces questions après sa rencontre avec le prêtre catholique Christian Delorme à Lyon et a beaucoup travaillé avec des théologiens protestants.



\section{Islamologues}

\subsection{Louis Massignon}


L’islamologue Massignon s’est avant tout situé dans la grande lignée des études sur l’Islam orthodoxe, son premier souci étant de démontrer que l’Islam a une dimension mystique et que c’est l’Islam sunnite essentiellement qui se prête à cette dimension. Mais que ce soit à travers son thème de recherche principal, Ḥallāğ, ou dans le reste de son œuvre, dans ses cours au Collège de France et à l’École des Hautes Études, dans ses incessants déplacements en Iran, en Syrie, au Liban, il s’est heurté au šī‘isme à tous les carrefours.

  \paragraph{Mansur al-Ḥallāğ pour Massignon} figure du Christ de l'Islam.

  Dieu demande à toutes les créatures devant l'homme. Une créature
  refuse. Traditionnellement, c'est considéré comme un refus
  d'obeissance de l'ange (c'est de la boue puante~»). Pour al Hallaj
  c'est uniquement devant Dieu et c'est une épreuve de Dieu, c'était
  révélé à l'ange ce qu'est Dieu. Dieu est dans l'homme.
  \url{https://www.youtube.com/watch?v=Let1X-8zsXU\&t=1428s}
  C'est à cause de cette divinisation qu'il sera condamné, hétérodoxe.

\paragraph{Pierre Lory}
\label{Theol:PierreLory}
Un des grands connaisseurs de Qušayrī est
Pierre Lory.

Même en dehors des cercles salafistes, nombreux sont aujourd'hui les
musulmans~

\begin{cite}
« qui voudraient que le Coran soit un discours unique,
et qui se méfient des interprétations différentes les unes des autres
»,~
\end{cite}
\emph{}déplore~{{Pierre
Lory}}. Cet islamologue a contribué au récent site {{Coran 12-21}}
Internet~\url{https://coran12-21.org/fr/} , qui
présente différentes versions du Coran, dans trois langues différentes.

Pour lui, comme pour d'autres spécialistes, considérer le Coran comme un
tout dogmatique et intouchable est non seulement dangereux, mais aussi
erroné.

\paragraph{Abdessamad
Belhaj}
\begin{cite}
« La lecture littérale est en elle-même une
interprétation, puisqu'elle est fondée sur la prémisse que les propos du
Coran sont généralisables et peuvent faire fi des circonstances
particulières »,~
\end{cite}
remarque
l'islamologue~{\underline{Abdessamad
Belhaj}}, chercheur au Centre interdisciplinaire d'études de l'islam
dans le monde contemporain de l'Université catholique de Louvain.

\subsection{Autres Islamologues - IDEO}
\paragraph{Emmanuel Pisani}
Frère dominicain et islamologue, Emmanuel Pisani est directeur de l’Institut de Science et Théologie des religions (ISTR) à l’Institut catholique de Paris.

\paragraph{Adrien Candiard}



\section{Matériel d'Etude}\label{matuxe9riel}

\url{https://www.onelittleangel.com/livres/sacres/le-saint-coran.asp}

\url{https://coran.oumma.com/sourate/20} . Conseillé par Emmanuel
Pisani.

\url{https://www.lexilogos.com/clavier/arabe_latin.htm}

\url{https://referenceworks.brillonline.com/browse/encyclopedie-de-l-islam}

\url{https://www.lexilogos.com/coran.htm} : lexilogos y compris mot à
mot




\section{Notes diverses à repositionner}







Important de connaître un auteur pour avoir un avis objectif.





\begin{itemize}
\
  
\item
  Est-ce que j'utilise la raison, l'analogie, la coutume locale~? c'est
  cela les différences.
\item
  Si la coutume est la laicité, je dois en tenir compte pour mon avis.

  \begin{itemize}
  \item
    Le shavinisme qui intègre le ur, la coutume,
  \item
    le hanafisme, aussi~
  \item
    la où on le ferait moins, c'est le hanbalisme.
  \end{itemize}
\end{itemize}



  Pour quitter l'Islam, la peine est celle de la Loi locale. En France,
  si on définit l'islam comme une loi, on dit son aversion. Tout le
  chapitre sur la loi, «~islam = loi~», est un schématisme redoutable.
  Ce n'est pas qu'un texte législatif. Malheureusement, il y a tout un
  courant dans l'Islam qui encourage cette lecture caricaturale. Dans
  les pays occidentaux, on peut combattre avec les idées l'islam
  radical.
  
  
  \subsection{Islam compliqué}


Rachid Benzime

Islam, compliqué à lire le coran sans clé herméneutique

Passé colonial~en France~:

champ sémantique~: passer d'un champ indigène, à immigré, à ~musulman.
La religion devient le marqueur identitaire.

Pisani~:

Compliqué l'islam~; islam~: complexe

Edgar Morin~: c'est quoi la complexité d'un fait social. Si complexe,
reponse complexe

Ici, les arrières pensées~: on croit connaitre de l'islam alors que ce
n'est qu'une réalité.

Les musulmans comme «~citadelles assiégées~»

Difficile pour les musulmans de voir une certaine réalité car l'islam
quelque chose de beau.

«~un terroriste qui se dit musulman, on n'a pas le droit de lui dire
qu'il n'est pas musulman~»

Derrida~: «~il faut bien séparer l'Islam de l'Islam~».

Accepter que l'islam est pluriel, alors que l'islam est vécu par les
personnes comme unique.

Pisani~:

Macron~: l'islam est en crise.

Ce n'est pas possible pour les musulmans~: «~l'islam ne peut pas être en
crise~» - méta religieux.

Mohammed Arkoun~: le fait islamique. Dieu est absent.

Trop de représentations dans le champ «~Islam~». Mot trop chargé.

What is Islma.

\section{Le Coran peut-il être interprété ?}

 

Considéré par la plupart des musulmans comme un livre « incréé », et
donc parfois intouchable, le Coran a fait l'objet, au Moyen Âge, d'une
tradition de commentaires d'une grande profusion. 

\begin{itemize}
\item
  Mélinée Le Priol,~
\end{itemize}


Plusieurs anecdotes transmises par la tradition islamique montrent un
croyant venant consulter le prophète Mohammed sur le sens de tel ou tel
verset.

\subsection{Pourquoi l'interprétation du Coran est-elle un sujet sensible
?}

Synonyme de « récitation »,~\emph{Al Qur'an}~(en français le Coran)
contient, selon la tradition islamique, la révélation reçue par le
prophète de l'islam Mohammed, entre 610 et 632. L'ange Gabriel lui
aurait dicté les versets tels quels, et ceux-ci auraient été mis par
écrit une vingtaine d'années après la mort du Prophète, qui n'aurait
fait que les réciter à ses compagnons. Malgré son statut bien connu
de~\href{https://www.la-croix.com/Religion/Islam/Comprendre-Coran-2016-06-10-1200767802}{\underline{livre
« incréé »}}, le Coran a été abondamment étudié et commenté.

\emph{« Le courant littéraliste, qui considère que le Coran se suffit à
lui-même, que ses ambiguïtés sont voulues par Dieu et que l'interpréter
est source d'égarements, a toujours existé, mais il a longtemps été très
marginal en islam »,}~rappelle l'historien Mohammad
Ali Amir-Moezzi 
 . C'est depuis une cinquantaine d'années, avec
l'essor du salafisme, que cette conception a gagné du terrain,
valorisant essentiellement l'apprentissage par cœur.
 
 

\subsection{ Quelle est l'histoire du commentaire coranique ?}

Plusieurs anecdotes transmises par la tradition islamique montrent un
croyant venant consulter le prophète Mohammed sur le sens de tel ou tel
verset. Il faut dire que le texte coranique est un corpus
particulièrement ardu, au contenu souvent allusif, parfois
contradictoire. Non content d'entremêler des thèmes et registres
différents, il n'a pas été agencé selon l'ordre chronologique de la
révélation.

 
D'où la nécessité de l'interpréter. Riche de plusieurs milliers de
volumes, le commentaire du Coran a connu son âge d'or du
VIII\textsuperscript{e}~au XII\textsuperscript{e}~siècle.\emph{~« Les
grands courants de pensée islamiques se sont tous développés à partir de
la même interrogation : comment comprendre l'écriture sainte
?,~}explique Mohammad Ali Amir-Moezzi.\emph{~­L'islam a hérité de cette
culture exégétique des milieux bibliques au sein desquels il s'est
développé. »}

Les commentateurs ne pouvaient toutefois pas interpréter le Coran à leur
guise. On dénombre trois méthodes principales : la traditionnelle avait
essentiellement recours aux sources scripturaires (le Coran et les
hadiths) et à des analyses sur la langue arabe et la culture tribale ;
la rationnelle faisait appel à la logique et à la pensée spéculative, la
logique aristotélicienne ; et la mystique reposait sur une~\emph{«
illumination »}.

 

\emph{« Une posture postmoderne veut que le Coran soit dorénavant ouvert
à toute interprétation,~}s'inquiète Abdessamad Belhaj.\emph{~Mais cela
ne doit pas être une excuse pour ne pas explorer l'intelligence interne
du texte lui-même. »}~Le lecteur contemporain du Coran doit tout de même
recouvrer son~\emph{« autonomie »,}~reconnaît-il toutefois, regrettant
que l'abondante littérature produite dans les siècles passés ait pu
être~\emph{« sacralisée »}.

\subsection{► Comment renouveler l'interprétation du Coran au
XXI\textsuperscript{e}~siècle ?}

Si la~\emph{« quasi-totalité »}~des commentaires du Coran se font,
encore aujourd'hui, dans le registre traditionnel,~\emph{« on sent
depuis trois décennies les frémissements d'une nouvelle exégèse, qui
recourt davantage aux méthodes académiques occidentales »,~}observe
Mohammad Ali Amir-Moezzi. Non confessante, cette islamologie née dans le
monde occidental commence à trouver un écho dans des pays musulmans
comme l'Iran, la Tunisie ou la Turquie.

 

Pour ces chercheurs, l'enjeu est de ne plus seulement étudier le Coran à
partir des sources musulmanes datant d'au moins un siècle et demi après
la mort de Mohammed, mais de recourir aussi aux sources non musulmanes
(notamment juives et chrétiennes) du contexte religieux de l'Antiquité
tardive au sein duquel le Coran a émergé. Longtemps restées cloisonnées,
ces deux approches -- confessante et scientifique -- pourraient bientôt
se réconcilier.

Islam, pourquoi cette sévérité avec les autres croyants et les
incroyants ?

\emph{Explication~}

« Mécréants », « infidèles » : les terroristes islamistes s'en sont pris
violemment, ces dernières années, à tous ceux qu'ils jugent hors de
l'islam « authentique ». Une intolérance fondée sur une lecture
littérale du Coran. 

\begin{itemize}
\item
  Mélinée Le Priol,~
\item
  le~28/01/2021 à 13:04~
\item
  Modifié le~28/01/2021 à 13:12
\end{itemize}

Lecture en 3 min.

\includegraphics[width=6.3in,height=4.20208in]{media/image4.jpeg}

Selon la théologie musulmane, l'islam est la religion originelle de
l'humanité.VICTOR MOUSSA - STOCK.ADOBE.COM

\subsection{► Que dit la tradition ?}

Selon la théologie musulmane, l'islam est la religion originelle de
l'humanité.~\emph{« Tout homme est né musulman »,}~dit un hadith
attribué
au~\href{https://www.la-croix.com/sacralite-prophete-lislam-2020-11-06-1101123195}{\underline{prophète
Mohammed}}. L'homme est né pour adorer Dieu : certes, il a reçu une
dignité plus haute que les autres créatures, mais celle-ci est
conditionnée à sa soumission au Dieu unique. Plus un homme applique la
loi divine (\emph{charia}), plus il devient humain. Quant au « mécréant
» (\emph{kâfir}), qui refuse de suivre la charia, il se situe en quelque
sorte à un degré inférieur d'humanité.

Cette sévérité envers les non-musulmans s'appuie sur la lecture du texte
coranique qui s'est imposée à partir du IX\textsuperscript{e}~siècle,
lors de la transformation de l'islam en un empire soucieux de se
légitimer. Confortée par des hadiths rédigés à cette époque, elle
dépeint une vérité unique et non négociable. Elle insiste sur les
versets du Coran particulièrement virulents envers les polythéistes,
païens ou idolâtres, qualifiés d'\emph{« associateurs
»}~(\emph{mouchrikoun}) car ils « associent » à Dieu d'autres divinités.

Quant aux athées,~\emph{« ils appartiennent, selon la théologie
musulmane, à une catégorie de mécréance encore inférieure aux
polythéistes, aux juifs et aux chrétiens »,~}explique l'islamologue
Abdessamad Belhaj, chercheur au Centre interdisciplinaire d'études de
l'islam dans le monde contemporain de l'Université catholique de
Louvain. Même si des institutions comme le Haut Conseil des oulémas du
Maroc ou la Maison de la fatwa en Égypte considèrent que les apostats ne
peuvent plus être condamnés à mort, cette peine reste appliquée dans une
dizaine de pays, comme l'Afghanistan ou
la~\href{https://www.la-croix.com/Monde/Afrique/prisons-Mauritanie-calvaire-dun-apostat-2019-09-30-1201051050}{\underline{Mauritanie}}.

\subsection{ Pourquoi juifs et chrétiens bénéficient-ils d'un statut
spécifique ?}

Selon la tradition musulmane, chrétiens et juifs font l'objet d'un
traitement différent des autres non-musulmans : ils bénéficient dans le
droit islamique d'une protection juridique particulière (\emph{dhimma})
toutefois accompagnée d'injonctions humiliantes, comme l'interdiction de
monter à cheval ou de construire des lieux de culte dépassant ceux des
musulmans.
 

\emph{« Le Coran est très ambivalent au sujet des ``gens du Livre''
»,~}rappelle
l'historien~\href{https://www.la-croix.com/Culture/Livres-et-idees/historiens-decryptent-Coran-avant-lislam-2019-11-27-1201063090}{\underline{Guillaume
Dye}}, professeur à l'Université libre de Bruxelles (1). Selon la
sourate 5, juifs et chrétiens ne doivent pas être pris pour~\emph{«
alliés »~}(5, 51) mais, quelques versets plus loin, on lit qu'ils ne
seront~\emph{« point affligés »~}(5, 69). Les chrétiens se voient
reprocher de nier l'unicité de Dieu mais du respect est exprimé pour les
prêtres et les moines, qui~\emph{« ne s'enflent pas d'orgueil ».}

Selon une théologie dite de la falsification (\emph{tahrif}), les juifs
et les chrétiens ont altéré le message transmis par leurs prophètes
respectifs (Moïse, Jésus), message qui n'était autre que l'islam. Le
Coran, lui, corrige cette déviation en transmettant fidèlement le
message révélé à un ultime prophète, Mohammed. À Médine, celui-ci aurait
signé une~\emph{« Constitution »~}disposant que les juifs, notamment,
pouvaient pratiquer leur religion en sécurité, mais ces relations se
sont rapidement détériorées.

\subsection{► Quelles pistes pour une « théologie du pluralisme » ?}

Les attentats visant des « mécréants » en terrasse à Paris, les
persécutions contre les Yézidis ou les chrétiens en Irak, sont autant de
conséquences d'une lecture littéraliste du Coran encouragée par l'essor
du salafisme saoudien à partir des années 1970. D'autres lectures ont
pourtant existé dès les premiers siècles de l'islam. Contrairement à la
doctrine sunnite traditionnelle, l'exégèse rationaliste a par exemple
conclu très tôt à une~\emph{« égalité entre tous les êtres humains, tous
étant dotés de la même raison les rendant aptes à comprendre la parole
de Dieu »,~}rappelle l'islamologue Pierre Lory, directeur d'études à
l'École pratique des hautes études (EPHE).

Pour Abdessamad Belhaj, tout l'enjeu est aujourd'hui de refonder le
rapport à l'altérité sur la base de l'éthique, et de\emph{~« mettre
l'homme au cœur de la théologie »}. Pour cela, certaines valeurs
présentes dans l'islam gagneraient à être redécouvertes, comme celles du
soin, du don et du service à l'humanité, longtemps éclipsées selon lui
par l'autorité et la loyauté à la communauté musulmane ou à la tribu.

(1) Il a codirigé avec Mohammad Ali Amir-Moezzi, Le Coran des
historiens, 2019, Éd. du Cerf, 3~408~p., 89~€.

Faudra-t-il sauver les salafistes ?

Le gouvernement français a voulu lancer en octobre 2019 une offensive
contre l'islamisme et les courants radicaux, rapidement relayée par un
emballement médiatique qui a échappé à tout contrôle. Or, l'ennemi
désigné n'a nullement été identifié selon des termes juridiques, pas
plus que ses torts. On lui reproche sa piété rigoureuse, son voile, sa
pratique du jeûne de Ramadan, sa barbe fournie, son refus de toucher les
femmes, ce qui le rapproche dangereusement de n'importe quel fidèle
conservateur.

L'offensive vise donc une manière de concevoir la piété musulmane, et
nullement une qualification criminelle ou une atteinte à l'ordre public.
C'est dire que nous sommes confrontés à un « délit de sale gueule »,
lequel échappe à la tradition juridique républicaine, délit qui est
indiscernable, sans limite, extensible, mais politiquement pratique
auprès d'une opinion chauffée à blanc par les attentats et
l'immigration.

\subsection{Un engagement d'abord religieux}

Si l'islamiste ainsi décrit ressemble évidemment
au~\href{https://www.la-croix.com/Religion/Islam/Quest-salafisme-2018-10-14-1200975866}{\underline{salafiste}},
c'est oublier un peu vite que l'écrasante majorité des~\emph{salafi~}--
ceux qui sont attachés au modèle des « anciens » (les~\emph{salaf}),
c'est-à-dire les compagnons du Prophète -- se veulent quiétistes : leur
mode d'action est la prédication et l'action missionnaire
(la~\emph{da`wa}). Le salafiste souhaite d'abord vivre un islam épuré et
intégriste -- au sens d'intégral -- dans le cadre de sa famille et de sa
communauté.

Ce mouvement est distinct d'un engagement politique, de sorte que les
salafistes sont rarement liés aux Frères musulmans, qui eux forment un
mouvement politique. Si la matrice religieuse et idéologique du
salafisme imprègne les mentalités djihadistes, elle ne se confond pas
avec celles-ci, ni dans la pensée, ni dans les faits. La radicalisation
concerne donc à des degrés différents et sous des formes incomparables
les sympathisants du salafisme et les partisans du djihadisme de Daech.
Les premiers ont un engagement d'abord religieux, tandis que les autres
sont mus à la fois par la volonté de puissance, des facteurs politiques,
sociaux et religieux.

\subsection{L'autodidacte de l'islam présente plus de risques que le
salafiste}

L'hostilité des salafistes envers les courants djihadistes a été prouvée
à de nombreuses reprises par des déclarations publiques et surtout en
fournissant du renseignement de qualité auprès des services de police.
Le meilleur ennemi du terroriste est souvent le~\emph{salafi}, et
l'autodidacte de l'islam présente plus de risques que le salafiste.

En outre, le salafisme n'a pas été désavoué par les représentants du
culte musulman pour la simple raison que ce courant n'est pas une
idéologie : il faudrait donc lui enlever son~\emph{isme}~final et
l'appeler, selon la tradition religieuse, la~\emph{salafiya~}; il s'agit
d'un vieux courant légitime de l'islam, qui a fourni des générations
d'imams et de lettrés attachés au sens littéral du Coran et de la Sunna.

\subsection{Un « écosystème » étroit mais rassurant}

Il est évident que le salafisme représente une alternative culturelle et
sociale au modèle français, modèle égalitaire, inclusif, ouvert (au
moins en théorie). Les quelques salafi que j'ai connus -- des convertis
à 25 ou 30 \% d'entre eux -- vivaient dans un étroit triangle
géographique. Parce qu'ils souhaitent faire les cinq prières à leur
heure, sans les décaler, et ce dans une salle de prière, ils sont
contraints de vivre et de travailler non loin d'une mosquée. Ils passent
ainsi de leur habitation au lieu de travail et à la salle de prière,
lesquels se situent nécessairement dans un « écosystème » étroit mais
rassurant. Ils ne peuvent guère être exigeants sur le plan
professionnel.

\includegraphics[width=1.97917in,height=1.40972in]{media/image6.jpeg}

\href{https://www.la-croix.com/Religion/Le-Coran-peut-etre-interprete-2021-01-25-1201136852}{Le
Coran peut-il être interprété ?}

Le salafisme, qui représente au moins 40 000 individus, est socialement
dangereux car il impose l'auto-ségrégation, le refus des contacts avec «
ceux qui n'en sont pas ». C'est la raison pour laquelle les spécialistes
des questions de sécurité se refusent à les impliquer dans la lutte
contre le djihadisme. Salafistes et terroristes participeraient à une
même matrice intellectuelle, celle du bien contre le mal, une sorte de
vision sectaire du monde. La différence vient du rapport à la violence :
assumé chez les djihadistes, rejeté chez les salafistes. Leur
fondamentalisme présente l'avantage d'une certaine forme de morale : à
Sartrouville les quartiers salafisés ont vu s'effondrer la toxicomanie
et la délinquance, avec le soutien de la mairie.

\subsection{Confondre l'approche culturelle avec la lutte contre le
terrorisme}

Ces courants ne peuvent être incriminés sur le plan sécuritaire. On
confond donc l'approche culturelle avec la lutte contre le terrorisme. À
moins de changer tout le droit européen, la première doit être menée par
l'éducation, la philosophie, la raison, le débat ; quant à la seconde
elle doit s'appuyer sur le droit et sur des qualifications pénales, et
non sur de vagues impressions de « radicalisation », notion qui n'a
toujours pas été appréhendée de façon rigoureuse en termes sociologiques
et psychologiques.

Comme la guerre d'Algérie nous l'enseigne, une telle manière de
concevoir l'action politique va aboutir à l'effet inverse de celui
recherché : le renforcement de la méfiance collective, le repli
communautaire du côté musulman, l'action violente du côté des « anti »,
et, finalement, la fragmentation sociale et l'insécurité.

\subsection{Islam : les fumées de la radicalisation}

Olivier Hanne, médiéviste (université de Poitiers), chercheur en
islamologie, estime qu'il est très difficile de définir le parcours type
d'une personne radicalisée. Le dernier de trois articles consacrés à
l'islam en France. 
 

Qui parle d'islam aujourd'hui pense aussitôt à la radicalisation. En
2015, on estimait entre 8 000 et 10 000 le nombre de Français
radicalisés. Leurs profils sont si variés qu'il est difficile de donner
des catégories fixes : les mineurs représentent 25 \% des cas, les
femmes 27 \%, les personnes signalées sont plutôt jeunes (entre 16 et 30
ans), leur niveau scolaire est généralement faible, même si l'on
rencontre des diplômés.

La plupart travaillent. Internet représente pour tous ces individus un
passage obligé, même s'il se concrétise différemment : terrain initial
de la radicalisation, facteur de renforcement ou vecteur unique de
l'expression radicale, le partage des contenus djihadistes sur Internet
n'a pas du tout la même fonction chez une adolescente connectée, un
salafiste convaincu et un combattant expérimenté déjà parti en Syrie.

\subsection{Les autorités font feu de tout bois}

De toute évidence, l'attraction pour la radicalité religieuse n'est pas
nécessairement liée à un phénomène de rupture sociale. Les failles de la
société contemporaine (éclatement des familles, déclin des autorités et
des idéologies, chômage, ghettoïsation) créent un terreau facilitateur,
mais nullement déterminant. La frustration individuelle alimente le
recours à des convictions extrêmes, voire le passage à l'acte
terroriste, mais n'est qu'un facteur parmi tant d'autres.

Les autorités font feu de tout bois pour tenter de faire face à une
radicalisation multiforme. En avril 2015, le premier ministre français,
Manuel Valls, annonçait l'ouverture d'une dizaine de centres de
prévention de la radicalisation, dont la plupart furent un échec. Des
sites Internet officiels sont créés et proposent des fiches techniques
contre la radicalisation et le terrorisme, dont le contenu est souvent
simple, voire binaire. Ainsi sur le site
français~\emph{stop-djihadisme.gouv.fr}, un bandeau intitulé «
Radicalisation djihadiste, les premiers signes qui peuvent alerter »
énonce pêle-mêle : « ils se méfient des anciens amis qu'ils considèrent
maintenant comme des impurs » ; « ils changent brutalement leurs
habitudes alimentaires » ; « ils arrêtent d'écouter de la musique car
elle les détourne de leur mission » ; « ils ne regardent plus la
télévision et ne vont plus au cinéma ». Autant de signes extérieurs qui
se rapprochent de l'adolescente anorexique\ldots{} L'efficacité de ces
dispositifs a d'ailleurs été très contestée dès 2015.

\subsection{L'État, tenté d'être omniprésent}

Toute l'entreprise de déradicalisation définit en creux le modèle
positif occidental : monde de loisirs, de consommation, d'épanouissement
personnel et professionnel. Le vocabulaire de la radicalisation masque
le rejet de ce modèle culturel. Et les pouvoirs publics d'hésiter à
appeler leur objectif par son vrai nom : le reconditionnement mental.

Le danger de la déradicalisation se situe dans l'élargissement des
intrusions de l'État : en voulant réinsérer, l'État pénètre dans
l'intimité des individus afin de redéfinir le religieux et lui redonner
une place acceptable. Or, l'État a-t-il compétence pour définir ce
qu'est l'islam, le « bon » islam ? Ne sachant cerner la menace, l'État
est tenté d'être omniprésent, sans en avoir la capacité légale. La
déradicalisation pourrait relever de la posture intellectuelle.

Le problème vient sans doute des hésitations du vocabulaire. Car,
après-tout, qu'est-ce que la radicalisation ? Au
XIX\textsuperscript{e}~le mot anglais~\emph{radical}~était employé pour
désigner les partis politiques britanniques exigeant une réforme
démocratique libérale. Transféré tel quel en France, on l'appliqua aux
partis de gauche, laïques et libéraux qui voulaient réformer la société.

\subsection{Réactions épidermiques}

Le verbe « radicaliser » fut employé régulièrement dans les années
1960-1970 dans une acception politique avec l'idée de « devenir plus
intransigeant, se durcir » ou « plus extrême ». Le premier sens était
donc politique et pas nécessairement négatif. Se déradicaliser était un
synonyme pour « se compromettre ». Appliqué à l'islamisme, le verbe
impose une redéfinition complète des termes : à partir de quand
juge-t-on l'islam intransigeant ou extrême ? par rapport à quelle norme
? à quelle moyenne ?

Les réactions épidermiques qui ont suivi le meurtre de l'enseignant de
Conflans-Sainte-Honorine en octobre 2020 sont tristement révélatrices :
les imams doivent s'exprimer ! les musulmans doivent désavouer le
terrorisme et faire allégeance à la France ! Mais quand ils le font,
c'est encore insuffisant, déloyal et mensonger. Le gouvernement proposa
même qu'ils prient pour la République au cours de la prière collective
du vendredi. Nos références sur la question religieuse restent
tragiquement celles de la Révolution française : comme il y eut les «
prêtres jureurs », adhérant à la loi, contre les « prêtres réfractaires
», obstinés dans leur obéissance à Rome, de la même façon il nous faut
des « imams jureurs », intimement républicains. L'État se retrouve donc
juge des reins et des cœurs.

%\bibliography{Theo} A REMETTRE
%\bibliography{Theo}
%\printbibliography
%\bibliographystyle{siam}
%\printbibliography

%\listoftheorems[ignoreall,show={Def}]
%Les courants contemporains de l’islam Glossaire général

\mn{Vérifier les termes}

bid‘a : innovation ; pratique « déviante ».

da‘wa : invitation ; prédication – appel à la conversion (dans les deux sens).

fasiq : pécheur ; mauvais musulman.

fiqh : compréhension ; corpus du droit musulman.

fitna: discorde, querelle ; conflit interne au monde musulman.

hadith : récit d’un dire ou faire du Prophète, rapporté par ses compagnons.

hajj : pèlerinage annuel à La Mecque.

hijra (héjire) : « exode » - départ de Mahomet pour Médine (622).

‘ibadat : culte ; partie du droit traitant du culte.

ijma‘ : consensus ; consensus des ulama sur un point de droit.

ijtihad : effort ; effort d’interprétation du Coran.

imam : chef suprême de la communauté musulmane ; successeur du Prophète, utilisé communément par les chiites pour Ali et ses descendants.

islah : réforme.

isnad : chaîne ; chaîne de transmission des hadiths.

jihad : lutte ; soit intérieure, contre ses propres faiblesses ; soit extérieure, contre les ennemis de la communauté musulmane.

ka‘aba : monument cubique noir situé au centre de la grande mosquée de La Mecque ; selon les musulmans, désigne l’emplacement du premier autel élevé par Abraham pour le Dieu unique. Point vers lequel se dirigent les musulmans pour prier.

kafir : infidèle, mécréant

khalifa (calife) : successeur, représentant ; successeur du Prophète et chef de la communauté musulmane (sunnisme).

madrasa : école ; lieu où est assuré la transmission du savoir religieux.

mihrab : niche indiquant la direction de La Mecque dans une mosquée.
 
mu‘amalat : relations ; partie du droit traitant des relations humaines.

qibla : orientation de la prière rituelle (salat), correspondant à la direction de La Mecque.

qiyas : raisonnement par analogie (domaine du droit)

salat : prière rituelle.

seyyed : prince, chef ; descendant du Prophète par Hossein ou Hassan, fils d’Ali.

shari‘a : sentier, voie ; loi divine.

sheykh (cheykh) : vieil homme ; chef d’une tribu ; chef religieux ; personne à la tête d’une congrégation soufie, ayant la capacité de guider ses disciples.

shirk : associationnisme : fait d’adorer d’autres êtres en dehors de Dieu.

shura : principe de consultation soufi : mystique musulman sourate : chapitre du Coran
sunna : coutume ; pratiques du Prophète et de la première communauté musulmane, faisant autorité pour guider le mode de vie des croyants et déterminer la loi religieuse.

tafsir : commentaire du Coran.

tajdid : renouveau (=>mujaddidi : qui renouvelle)

taqlid : imitation ; imitation stérile des anciens (par opposition à l’ijtihad).

tariqa : voie : confrérie soufie.

tawhid : unicité (divine). Dogme fondamental de l’islam.

ulama (oulémas) : terme collectif pour désigner les lettrés musulmans.

umma : peuple ou communauté ; communauté islamique dans son ensemble.

waqf : bien immobilier ou foncier dit « de-main-morte », dépendant des institutions religieuses.

zakat : aumône rituelle, obligatoire pour les croyants.

%\listoftheorems
\begin{singlespace}
 \defbibnote{bibnote}{ } % Prepend this text to the bibliography
\printbibliography[heading=bibintoc, title=Bibliographie, prenote=bibnote] % Add the bibliography heading to the ToC, set the title of the bibliography and output the 
 
%\bibliography{zotero} 
%\bibliography{zotero, Theo,Theo2}
\end{singlespace}
\end{document}







 