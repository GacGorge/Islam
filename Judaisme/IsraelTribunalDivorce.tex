\section{En Israël, un couple pris au piège de la justice des rabbins ultraorthodoxes}

\mn{international, vendredi 5 mai 2023 Louis Imbert}


Le tribunal rabbinique d’Ashkelon refuse de prononcer leur divorce tant que l’épouse ne se sera pas soumise à une enquête prouvant sa judéité, et exige que la Cour suprême reconnaisse sa compétence en la matière.

LETTRE D’ASHKELON

C’était au printemps 2020, devant le tribunal rabbinique d’Ashkelon, dans le sud d’Israël. Masha et Asaf (deux noms d’emprunt) imploraient le juge de prononcer leur divorce. Masha avait quitté le domicile familial un an plus tôt, en mars 2019. Elle avait emménagé dans un village agricole de la région, non loin de l’école de ses deux enfants. Elle avait refait sa vie, mais l’Etat refusait de le reconnaître.

L’Etat, en l’occurrence, était représenté par l’un des douze tribunaux rabbiniques d’Israël, aux mains de rabbins ultraorthodoxes, qui ont autorité sur les mariages et les divorces depuis la fondation de l’Etat, en 1948. Et ce matin de 2020, le juge a des doutes. Il craint que Masha ne soit pas juive. Ce rabbin estime qu’elle a pu se faire reconnaître abusivement comme telle lors de son arrivée en Israël, à 8 ans, en 1990, durant la chute de l’URSS, en même temps qu’un million de juifs originaires de l’ex-espace soviétique. Il refuse de prononcer le divorce tant que Masha ne se sera pas soumise à une enquête approfondie.

Son mari, Asaf, répond qu’un autre rabbin les a mariés religieusement en 2006. D’ordinaire réservé, il s’emporte : « Vous m’avez donné une femme casher il y a quinze ans. Nous avons deux enfants et, maintenant, vous me dites qu’elle n’est pas juive ? Je devrais vous faire un procès ! » Le juge clôt la séance en pointant un doigt accusateur sur Masha : « Nous ferons toute la lumière sur ce cas ! »

Assise sous le porche de sa maison, en ce début de printemps 2023, Masha n’a toujours pas vu la lumière. Elle refuse encore de se soumettre à « l’inquisition » du grand rabbinat. Elle demeure mariée. Son dossier se perd dans un labyrinthe judiciaire. Elle en pleure : « C’est fou. Vous devez prouver ce qui coule dans vos veines. Pourquoi revient-il à cet homme, qui n’a pas la même conception du judaïsme que moi, de dire qui je suis ? Je ne veux pas lui donner ce pouvoir. »

« Le pays vit une crise d’identité »

Masha salue les milliers de manifestants qui ont déferlé, jeudi 4 mai, dans les grandes villes d’Israël pour dénoncer l’emprise des religieux sur le nouveau gouvernement. Depuis décembre 2022, celui-ci allie partis conservateur, ultraorthodoxes et extrême droite messianique. Devant le tribunal rabbinique de Tel-Aviv, des femmes se sont rassemblées en silence, portant la cornette blanche et la robe rouge des « servantes écarlates », ces personnages de roman et de série télévisée, réduites en esclavage par des religieux fondamentalistes.

Ces femmes dénoncent un projet de loi qui vise à faire des cours rabbiniques des instances d’arbitrage compétentes dans tous les domaines de la vie civile : conflits commerciaux ou de cadastre, droit du travail, accidents de la route… Un système légal parallèle régi par la halakha, la loi rabbinique. « Le pays vit une crise d’identité. Nous sommes contraints de nous demander encore ce qu’est un juif, ce que signifie être juif, parce que ces partis veulent former un Etat halakhique », déplore l’avocate de Masha, Susan Weiss. Directrice du Centre pour la justice pour les femmes (CWJ), religieuse et de gauche, elle milite pour mieux séparer la religion et l’Etat.

Masha, quant à elle, a « peur » que ce conflit ne « détruise » son pays. Elle s’inquiète d’un second projet de loi, qui vise à limiter l’immigration juive en Israël en resserrant les critères déterminant une ascendance juive. Cette femme de 40 ans à la peau de blonde, espiègle, rieuse, est une enfant de la méritocratie israélienne. Son père, ingénieur mécanique originaire de Saint-Pétersbourg, a divorcé peu après leur arrivée en Israël et émigré au Canada. Elle a rejoint un pensionnat d’élite à Jérusalem, étudié au célèbre Technion d’Haïfa et passé dix ans dans l’armée, ingénieure en bâtiment au sein de la force aérienne.

D’un épais classeur, elle tire une photo de la tombe en Israël de sa grand-mère, née à Babi Yar, près de Kiev, où les nazis exécutèrent 34 000 juifs en deux jours en 1941. Elle montre aussi une copie du certificat de naissance de sa mère, en 1952, à Osmiakovsky, en Iakoutie, près du goulag où son grand-père fut interné après-guerre. Une écriture liée, tout en courbe, indique « ivrika » : hébraïque, juive.

Nombreux mariages à l’étranger

A 23 ans, Masha la laïque s’est mariée religieusement avec un camarade d’université issu d’une famille juive d’Irak, attachée aux traditions. Elle se souvient à peine de son bref passage devant un rabbin pour préparer son mariage, à Haïfa. Lorsqu’il lui faut déclarer son divorce, elle n’y voit qu’une formalité administrative. « J’avais l’air heureuse, j’étais un peu arrogante », regrette-t-elle aujourd’hui.

Masha refuse le dédommagement financier assigné traditionnellement au mari, et attend sa convocation à une cérémonie de divorce. Puis un courrier lui demande de fournir les preuves de ses origines juives. « J’étais naïve, j’ai envoyé des documents. Ce n’était jamais assez. » Masha finit par mettre un terme à ces échanges, craignant pour ses proches : « S’ils me déclarent non juive, leur soupçon se portera alors sur ma sœur, puis sur ma tante et ses enfants. »

Elle craint aussi pour ses propres enfants, dont la judéité est d’ores et déjà douteuse aux yeux du grand rabbinat : « Cela peut les handicaper lorsqu’ils voudront se marier. » Nombre de couples juifs israéliens s’épargnent de tels tracas en se mariant à l’étranger, à Chypre, par exemple. Mais ils demeurent contraints de faire acter leur divorce par une cour rabbinique. « Quand je rencontre un homme, je mens : je lui dis que je suis divorcée. Mais je ferai quoi, demande-t-elle encore, si mon prochain copain veut des enfants ? Officiellement, ce seront encore ceux de mon premier mari. »

En janvier 2023, Masha a fait remonter son cas jusqu’à la Cour suprême du pays, à Jérusalem. Cette cour honnie des partis ultraorthodoxes et du gouvernement, qui cherche à briser ses pouvoirs, suscitant depuis janvier les plus vastes mouvements de protestation de rue de l’histoire d’Israël. Trop occupée par ses pleurs, Masha n’a rien compris aux paroles des trois juges. Son avocate, Mme Weiss, lui a expliqué qu’ils avaient classé l’affaire sans suite. Le conseiller légal des cours rabbiniques s’est engagé à ce que Masha obtienne la cérémonie religieuse actant son divorce. Pourtant, en avril, le tribunal rabbinique a une nouvelle fois refusé d’organiser cette cérémonie, quand Masha l’y pressait. Il exige que la Cour suprême reconnaisse d’abord sa compétence à réévaluer la judéité des personnes apparaissant devant lui. Il fait de Masha une otage de son combat avec la justice séculière.

