
%-------------------------------------------------
% Geometry (et sidenotes) : format tufte light
%-------------------------------------------------

\usepackage{sidenotes}
%\usepackage{mwe}

%\usepackage[showframe]{geometry}
\usepackage{geometry}

\geometry{letterpaper, left=2.5cm, right=2.6in, top=50pt,bottom=50pt, marginparsep=20pt, marginparwidth=2in,  footskip=40pt}
\renewcommand{\baselinestretch}{1.1} 
\usepackage{placeins} % floatbarrier
\usepackage{fullwidth}
 
\makeatletter
%\renewcommand{\@sidenotes@adjust}{%
% \checkoddpage%
% \ifoddpage%
% %
% \else%
% %\hspace{\@sidenotes@extrawidth}%    %% this was originally there
% \fi}
%%
%% or
%%
\let\@sidenotes@adjust\relax
\makeatother

\usepackage{url}
\setlength\parindent{0pt}
% marginnote mn
\newcommand\mn[1]{\marginpar{\footnotesize #1}}

\newcommand\sn[1]{\sidenote{\footnotesize #1}}

%-------------------------------------------------
% cadre
%-------------------------------------------------

\usepackage{tikz}
\usepackage[framemethod=TikZ]{mdframed}
\usetikzlibrary{positioning}  


\usepackage{xcolor}
%\hypersetup{colorlinks}% uncomment this line if you prefer colored hyperlinks (e.g., for onscreen viewing)
\usepackage{units}
% Typesets the font size, leading, and measure in the form of 10/12x26 pc.


\newcommand{\measure}[3]{#1/#2$\times$\unit[#3]{pc}}
\newcommand{\coefGraph}{1} % Taille du graphe dans les marges; sert uniquement pour Epub, sinon = 1 x textwidth
%\usepackage{multicol} %multicolum for Definition
\newcommand{\largecoefGraph}{1.2} % Taille du graphe dans les marges en proportion de textwidth; sert uniquement pour Epub, sinon = textwidth; remplacer \textwidth par \coefGraph\textwidth
%\usepackage[table]{xcolor}
%\usepackage[xcdraw]{xcolor}
%\usepackage[dvipsnames]{xcolor}
%\usepackage{amsmath,amssymb,amsthm}
%\usepackage{mathtools}
%\usepackage{mathspec}
%\usepackage{xltxtra,xunicode}
\newcommand{\myinnertopmargin}{0pt} % marge qui sert pour les définitions et proprietés



%-------------------------------------------------
% tableau
%-------------------------------------------------

% pour mettre des tableaux au bon endroit avec l'option H
%\usepackage{float}
% grands tableaux... pratiques
\usepackage{longtable}
 % pour faire des beaux tableaux
\usepackage{booktabs}
 
 
%-------------------------------------------------
% caractère
%-------------------------------------------------

\usepackage{fontspec} % Font selection for XeLaTeX; see fontspec.pdf for documentation
\defaultfontfeatures{Mapping=tex-text} % to support TeX conventions like ``---''


%\setmainfont{Charis SIL} % set the main body font (\textrm), assumes Charis SIL is installed
%\setsansfont{Deja Vu Sans}
%\setmonofont{Deja Vu Mono}

 % format des fonts comme Tufte
 \usepackage{xunicode} % Unicode support for LaTeX character names (accents, European chars, etc)
\usepackage{xltxtra} % Extra customizations for XeLaTeX
\usepackage{amsmath}
\usepackage{amsthm}
% \usepackage{fontspec}
%\setmainfont[Renderer=Basic, Numbers=OldStyle, Scale = 1.0]{TeX Gyre Pagella}
%\setsansfont[Renderer=Basic, Scale=0.90]{TeX Gyre Heros}
%\setmonofont[Renderer=Basic]{TeX Gyre Cursor}
% Palatino for main text and math
%\usepackage[osf,sc]{mathpazo}

% Helvetica for sans serif
% (scaled to match size of Palatino)
%\usepackage[scaled=0.90]{helvet}

% Bera Mono for monospaced
% (scaled to match size of Palatino)
%\usepackage[scaled=0.85]{beramono}
 
\setmainfont[Numbers=OldStyle, Scale = 1.0]{TeX Gyre Pagella}
\setsansfont[Scale=0.90]{TeX Gyre Heros}
\setmonofont{TeX Gyre Cursor}
%\usepackage{marginnote}
%\renewcommand*{\raggedleftmarginnote}{}
%\renewcommand*{\raggedrightmarginnote}{}

% Margin Caption (done with sidenotes package)
% UTILISER \sidecaption pour une caption
%\usepackage[margincaption,rightcaption,ragged,wide]{sidecap}
%\usepackage[margincaption,outercaption]{sidecap}
%\sidecaptionvpos{figure}{t} 
%\sidecaptionvpos{table}{t}
% format des captions des figures
%\captionsetup[SCfigure]{format=plain, ...}
%\captionsetup[SCtable]{format=plain, ...}

%-------------------------------------------------
% bibliography
%-------------------------------------------------

% 
%\usepackage{natbib}
%\usepackage[notes,backend=biber]{biblatex-chicago}

%\usepackage[style=reading]{biblatex}
\usepackage[citestyle=reading,bibstyle=authortitle]{biblatex}

\addbibresource{Theo.bib}

%\bibliography{sample}
%\bibliography{siam}

%\newcommand*{\sidecite}[1]{\sidenote{[\cite{#1}].\citeauthor{#1} - \citetitle{#1}}




%-------------------------------------------------
% caractère
%-------------------------------------------------

%\usepackage{biblatex} %pour citer des numero de page
\usepackage[utf8x]{inputenc}
\usepackage[english,main=french]{babel}

\babelprovide[import]{arabic}
\babelfont[arabic]{rm}{Amiri}
\babelprovide[import]{greek}
\babelfont[greek]{rm}{EB Garamond}
% ex
% \foreignlanguage{greek}{Ἰουδαῖοί τε καὶ προσήλυτο.}
%\babelprovide[import]{greek}
%\babelfont[greek]{rm}[RawFeature=+calt]{SimonciniGaramondPro}
\usepackage{arabtex}
%babel-greek

\newcommand\TArabe[1]{\foreignlanguage{arabic}{#1}}
\newcommand{\vide}[1]{}


%Recherche \hypertarget et remplacer par \vide
% \protect\hyperlink par \vide
%\texorpdfstring par RIEN
% \RL : \TArabe
% rechercher \footnote{ et remplacer par \sn{
% rechercher Al Gazali
% package pour faire des réferences à des labels pour le chapitre théologiens
%-------------------------------------------------
% table of contents
%-------------------------------------------------

%%% ToC (table of contents) APPEARANCE
\usepackage[nottoc,notlof,notlot]{tocbibind} % Put the bibliography in the ToC
\usepackage[titles,subfigure]{tocloft} % Alter the style of the Table of Contents
\renewcommand{\cftsecfont}{\rmfamily\mdseries\upshape}
\renewcommand{\cftsecpagefont}{\rmfamily\mdseries\upshape} % No bold!
%\newcommand\TArabe[1]{\foreignlanguage{arabic}{\RL}}

\usepackage{cleveref}

% gros tableau
\usepackage{longtable}

\usepackage{eurosym}  %Euro
\usepackage[super]{nth} %for \nth{1} to give 1st
\usepackage{array} % permet de centrer les tableaux\

% Prints the month name (e.g., January) and the year (e.g., 2008)
\newcommand{\monthyear}{%
  \ifcase\month\or January\or February\or March\or April\or May\or June\or
  July\or August\or September\or October\or November\or
  December\fi\space\number\year
}


% Prints an epigraph and speaker in sans serif, all-caps type.
\newcommand{\openepigraph}[2]{%
  %\sffamily\fontsize{14}{16}\selectfont
  \begin{fullwidth}
  \sffamily\large
  \begin{doublespace}
  \noindent\allcaps{#1}\\% epigraph
  \noindent\allcaps{#2}% author
  \end{doublespace}
  \end{fullwidth}
}

% Inserts a blank page
\newcommand{\blankpage}{\newpage\hbox{}\thispagestyle{empty}\newpage}


%\splittopskip=5cm 

 






