
\usepackage{geometry} % See geometry.pdf to learn the layout options. There are lots.
\geometry{letterpaper} % or letterpaper (US) or a5paper or....
%\usepackage[parfill]{parskip} % Activate to begin paragraphs with an empty line rather than an indent

%\geometry{asymmetric}
\geometry{outer=2.5in}
\geometry{marginparwidth=2in}
\geometry{marginparsep=20pt}

% pour mettre des tableaux au bon endroit avec l'option H
%\usepackage{float}
% grands tableaux... pratiques
\usepackage{longtable}
 % pour faire des beaux tableaux
\usepackage{booktabs}
 
 % format des fonts comme Tufte
 \usepackage{xunicode} % Unicode support for LaTeX character names (accents, European chars, etc)
\usepackage{xltxtra} % Extra customizations for XeLaTeX
\usepackage{amsmath}
\usepackage{amsthm}
 \usepackage{fontspec}
\setmainfont[Renderer=Basic, Numbers=OldStyle, Scale = 1.0]{TeX Gyre Pagella}
\setsansfont[Renderer=Basic, Scale=0.90]{TeX Gyre Heros}
\setmonofont[Renderer=Basic]{TeX Gyre Cursor}


%\usepackage{marginnote}
%\renewcommand*{\raggedleftmarginnote}{}
%\renewcommand*{\raggedrightmarginnote}{}

% Margin Caption (done with sidenotes package)
% UTILISER \sidecaption pour une caption
%\usepackage[margincaption,rightcaption,ragged,wide]{sidecap}
%\usepackage[margincaption,outercaption]{sidecap}
%\sidecaptionvpos{figure}{t} 
%\sidecaptionvpos{table}{t}
% format des captions des figures
%\captionsetup[SCfigure]{format=plain, ...}
%\captionsetup[SCtable]{format=plain, ...}

\usepackage{sidenotes}

% bibliography
%\usepackage{natbib}
%\usepackage[notes,backend=biber]{biblatex-chicago}

%\usepackage[style=reading]{biblatex}
\usepackage[citestyle=reading,bibstyle=authortitle]{biblatex}

\addbibresource{Theo.bib}

%\bibliography{sample}
%\bibliography{siam}

%\newcommand*{\sidecite}[1]{\sidenote{[\cite{#1}].\citeauthor{#1} - \citetitle{#1}}


\usepackage{url}
\setlength\parindent{0pt}
% marginnote mn
\newcommand\mn[1]{\marginpar{\footnotesize #1}}

\newcommand\sn[1]{\sidenote{\footnotesize #1}}

%\usepackage{biblatex} %pour citer des numero de page
\usepackage[english,main=french]{babel}
\babelprovide[import]{arabic}
\babelfont[arabic]{rm}{Amiri}
\usepackage{arabtex}

%%% ToC (table of contents) APPEARANCE
\usepackage[nottoc,notlof,notlot]{tocbibind} % Put the bibliography in the ToC
\usepackage[titles,subfigure]{tocloft} % Alter the style of the Table of Contents
\renewcommand{\cftsecfont}{\rmfamily\mdseries\upshape}
\renewcommand{\cftsecpagefont}{\rmfamily\mdseries\upshape} % No bold!
%\newcommand\TArabe[1]{\foreignlanguage{arabic}{\RL}}
\newcommand\TArabe[1]{\foreignlanguage{arabic}{#1}}
\newcommand{\vide}[1]{}
%Recherche \hypertarget et remplacer par \vide
% \protect\hyperlink par \vide
%\texorpdfstring par RIEN
% \RL : \TArabe
% rechercher \footnote{ et remplacer par \sn{
% rechercher Al Gazali
% package pour faire des réferences à des labels pour le chapitre théologiens
\usepackage{cleveref}

% gros tableau
\usepackage{longtable}

\usepackage{eurosym}  %Euro
\usepackage[super]{nth} %for \nth{1} to give 1st
\usepackage{array} % permet de centrer les tableaux\

% Prints the month name (e.g., January) and the year (e.g., 2008)
\newcommand{\monthyear}{%
  \ifcase\month\or January\or February\or March\or April\or May\or June\or
  July\or August\or September\or October\or November\or
  December\fi\space\number\year
}


% Prints an epigraph and speaker in sans serif, all-caps type.
\newcommand{\openepigraph}[2]{%
  %\sffamily\fontsize{14}{16}\selectfont
  \begin{fullwidth}
  \sffamily\large
  \begin{doublespace}
  \noindent\allcaps{#1}\\% epigraph
  \noindent\allcaps{#2}% author
  \end{doublespace}
  \end{fullwidth}
}

% Inserts a blank page
\newcommand{\blankpage}{\newpage\hbox{}\thispagestyle{empty}\newpage}


%\splittopskip=5cm 

 





\usepackage[framemethod=TikZ]{mdframed}

\usepackage{thmtools}
\usepackage{blindtext} % avoid to cut theorem
% avoid to have theorem or definition in the list of theorm
\makeatletter
\patchcmd\thmt@mklistcmd
  {\thmt@thmname}
  {\check@optarg{\thmt@thmname}}
  {}{}
\patchcmd\thmt@mklistcmd
  {\thmt@thmname\ifx}
  {\check@optarg{\thmt@thmname}\ifx}
  {}{}
\protected\def\check@optarg#1{%
  \@ifnextchar\thmtformatoptarg\@secondoftwo{#1}%
}
\makeatother
% format of theorem
\declaretheoremstyle[
    headfont=\scshape, 
    notebraces={\scshape : }{.},
    bodyfont=\normalfont,
    headpunct={},
    postheadspace=\newline,
%    postheadhook={\textcolor{red}{\rule[.6ex]{\linewidth}{0.4pt}}\\},
    spacebelow=\parsep,
    spaceabove=\parsep,
    preheadhook={\begin{mdframed}[backgroundcolor=white!20, 
        splittopskip = \topskip,
            linecolor=blue!30, 
            linewidth = 2pt,
            innertopmargin=6pt,
            roundcorner=1pt, 
            innerbottommargin=6pt, 
            skipabove=\parsep,     
            skipbelow=\parsep]},
            postfoothook=\end{mdframed}]{Definitionstyle}


% example environment - thmtools


\declaretheorem[ style = Definitionstyle, name = {Definition}]{definition}
\declaretheorem[ style = Definitionstyle, name = {Definition}, sibling=definition]{Def}


\declaretheoremstyle[
    headfont=\scshape, 
    notebraces={\scshape : }{.},
    bodyfont=\normalfont,
    headpunct={},
    postheadspace=\newline,
%    postheadhook={\textcolor{red}{\rule[.6ex]{\linewidth}{0.4pt}}\\},
    spacebelow=\parsep,
    spaceabove=\parsep
]{Exercisestyle}
% example environment - thmtools



\declaretheorem[ style = Exercisestyle, name = Exemple ]{Ex}


