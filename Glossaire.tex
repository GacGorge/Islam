Les courants contemporains de l’islam Glossaire général

\mn{Vérifier les termes}

bid‘a : innovation ; pratique « déviante ».

da‘wa : invitation ; prédication – appel à la conversion (dans les deux sens).

fasiq : pécheur ; mauvais musulman.

fiqh : compréhension ; corpus du droit musulman.

fitna: discorde, querelle ; conflit interne au monde musulman.

hadith : récit d’un dire ou faire du Prophète, rapporté par ses compagnons.

hajj : pèlerinage annuel à La Mecque.

hijra (héjire) : « exode » - départ de Mahomet pour Médine (622).

‘ibadat : culte ; partie du droit traitant du culte.

ijma‘ : consensus ; consensus des ulama sur un point de droit.

ijtihad : effort ; effort d’interprétation du Coran.

imam : chef suprême de la communauté musulmane ; successeur du Prophète, utilisé communément par les chiites pour Ali et ses descendants.

islah : réforme.

isnad : chaîne ; chaîne de transmission des hadiths.

jihad : lutte ; soit intérieure, contre ses propres faiblesses ; soit extérieure, contre les ennemis de la communauté musulmane.

ka‘aba : monument cubique noir situé au centre de la grande mosquée de La Mecque ; selon les musulmans, désigne l’emplacement du premier autel élevé par Abraham pour le Dieu unique. Point vers lequel se dirigent les musulmans pour prier.

kafir : infidèle, mécréant

khalifa (calife) : successeur, représentant ; successeur du Prophète et chef de la communauté musulmane (sunnisme).

madrasa : école ; lieu où est assuré la transmission du savoir religieux.

mihrab : niche indiquant la direction de La Mecque dans une mosquée.
 
mu‘amalat : relations ; partie du droit traitant des relations humaines.

qibla : orientation de la prière rituelle (salat), correspondant à la direction de La Mecque.

qiyas : raisonnement par analogie (domaine du droit)

salat : prière rituelle.

seyyed : prince, chef ; descendant du Prophète par Hossein ou Hassan, fils d’Ali.

shari‘a : sentier, voie ; loi divine.

sheykh (cheykh) : vieil homme ; chef d’une tribu ; chef religieux ; personne à la tête d’une congrégation soufie, ayant la capacité de guider ses disciples.

shirk : associationnisme : fait d’adorer d’autres êtres en dehors de Dieu.

shura : principe de consultation soufi : mystique musulman sourate : chapitre du Coran
sunna : coutume ; pratiques du Prophète et de la première communauté musulmane, faisant autorité pour guider le mode de vie des croyants et déterminer la loi religieuse.

tafsir : commentaire du Coran.

tajdid : renouveau (=>mujaddidi : qui renouvelle)

taqlid : imitation ; imitation stérile des anciens (par opposition à l’ijtihad).

tariqa : voie : confrérie soufie.

tawhid : unicité (divine). Dogme fondamental de l’islam.

ulama (oulémas) : terme collectif pour désigner les lettrés musulmans.

umma : peuple ou communauté ; communauté islamique dans son ensemble.

waqf : bien immobilier ou foncier dit « de-main-morte », dépendant des institutions religieuses.

zakat : aumône rituelle, obligatoire pour les croyants.
