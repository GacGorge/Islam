

\chapter{L'Islam comme religion Universelle ?}


L'Islam semble offrir aujourd'hui deux visages au voyageur candide qui
observe le monde. D'un côté, il peut admirer des mosquées magnifiques en
forme de pagodes, avec des stèles traditionnelles chinoise, à Xi'an,
ancienne capital de la Chine~; de même la Mesquita à Cordoue, l'art
Moghol en Inde, qui s'inspire des influences architecturales locales
tout en créant un univers inédit et reconnaissable. Et au-delà de
l'architecture, il peut rencontrer des musulmans de toutes les Nations.
De l'autre, il peut observer régulièrement une sorte de séparation par
des pratiques ou des tenues, qu'il retrouve à travers le monde, tunique,
barbe, et chaussure Nike, sans parler de la tenue des femmes et du
fameux \emph{voile}.

Au deuxième siècle de notre ère, chrétiens et juifs répondirent de façon
très différente, à la question du rapport au monde, le rabbin Tryphon
faisant la remarque aux chrétiens que \sn{Dialogue de saint Justin
  avec le juif Tryphon. Chapitre 10.«~Mais ce qui nous embarrasse surtout, c'est que vous vous dites
  pieux; que vous estimez différer des autres tout en ne vous en
  séparant pas; et que dans votre vie, vous n'êtes pas différents des
  nations, puisque vous n'observez ni les fêtes, ni les sabbats, que
  vous n'avez pas la circoncision; et encore, tandis que vous mettez
  votre espoir en un homme qui a été crucifié, vous espérez en même
  temps quelque bien de Dieu, sans observer ses commandements.}
  \begin{quote}
       «~vous
vous dites pieux; vous estimez différer des autres tout en ne vous en
séparant pas; et dans votre vie, vous n'êtes pas différents des nations,
puisque vous n'observez ni les fêtes, ni les sabbats, que vous n'avez
pas la circoncision~».
  \end{quote}



Qu'en est-il de l'Islam~? Quel est son rapport à l'universalité~?
Question qui nous semble être un préalable à d'autres questions, comme
celle de la possibilité d'un \emph{Islam de France} ou de son rapport à
la démocratie\emph{.} Nous proposons d'étudier la réponse faite en 1882
par Abraham Kuenen. Cette année-là, ce théologien hollandais de
l'université de Leyde, spécialiste reconnu du Pentateuque, et l'un des
pères de l'approche \emph{historico-critique.} se rend en Angleterre
pour dispenser un cours, intitulé \emph{National Religions et Universal
Religion}. Il s'attache alors, avec rigueur à tester ces deux concepts
de religion nationale et religion universelle à l'Islam~: \emph{l'Islam
offre-t-il les caractères de l'universalisme religieux}~? \sn{Kuenen,
  A. ``\emph{L'Islam offre-t-il les caractères de l'universalisme
  religieux}?'' Revue De L'histoire Des Religions, vol. 6, 1882, pp.
  1--40. JSTOR,
  \url{http://www.jstor.org/stable/23658819}.}
Les 150 ans qui nous sépare de cette conférence, bien loin de constituer
un handicap, permettent de se situer à une période précédent la fin de
l'empire Ottoman - et de son paradigme \emph{l'Islam Impérial ou
classique}\sn{Adrien Candiard, \emph{Comprendre l'islam: ou plutôt
  : pourquoi on n'y comprend rien, 2016,} emp. 427}\emph{,} en nous
offrant un recul utile pour répondre à cette question de
l'universalité\emph{.} Nous proposons dans un premier temps d'étudier
les différentes étapes du raisonnement de Kuenen qui l'amène à ne pas
attribuer le terme de \emph{religion universelle} à l'Islam puis dans un
second temps, de proposer quelques éléments qui nous semble permettre
d'atténuer ce jugement en particulier en testant les présupposés
méthodologiques de Kuenen et en proposant quelques pistes entrevues lors
du cours sur les fondations de l'Islam.

\section{La thèse de Kuenen : L'islam comme religion particulière}
Pour répondre à la question posée par l'Islam, il nous faut d'abord
définir ce qui définit une religion universelle par rapport à une
religion nationale, bornée à un peuple unique ou à une groupe de peuples
de mêmes origines. Kuenen propose de ne pas s'arrêter au fait (l'Islam a
bien été adopté par plusieurs nations différentes et est une religion
internationale), mais de regarder son caractère propre. \sn{car
  «~le universel de fait peut venir de la force au moins en dépit de son
  caractère propre~; que son défaut ou du moins sa faiblesse en fait
  d'éléments vraiment universels ait trouvé sa contrepartie ou son
  dédommagement dans différentes particularités qui ne viennent pas en
  ligne de compte quand on dresse son signalement\emph{~».}} Mais
comment définir le caractère propre d'une religion~? Fidèle à sa méthode
historico-critique\sn{Kuenen a établi quatre règles permettant
  l'interprétation d'un texte~:

  \emph{rigueur philologique et grammaticale}, c'est-à-dire conditionnée
  par le contexte de l'époque et le style de l'auteur.

  \emph{rationnellement cohérente}, un passage doit toujours être
  interprété dans son contexte

  \emph{historique}, c'est-à-dire en correspondance avec les données
  historiques à notre disposition

  \emph{psychologique}, elle doit nous rendre l'état mental préalable au
  texte.} permettant de remonter à l'origine du texte, il essaye de
ressaisir «~l'élément authentiquement universel {[}\ldots{]} qui tient
immédiatement à l'origine~»\sn{Ibid, p. 5 «~élément
  authentiquement universel, n'est pas à mon sens, l'effet d'une
  addition postérieure, mais tient immédiatement à l'origine des
  religions dans lesquelles nous saisissons cet élément, à la nature du
  rapport ou elles se trouvent avec les religions nationales d'où elles
  sont sorties ou sur le terrain desquelles elles se sont développées.~»}
de l'Islam. Nous connaissons relativement bien cette origine, d'une
religion née «~en pleine lumière~» selon l'expression de Renan. Pour
bien capter cette origine, Kuenen étudie les religions nationales qui
précèdent, et en particulier la \emph{Hanîffyya,} la religion d'Abraham
telle qu'elle est mentionnée dans la Sira (I, 222-232)\sn{Ce
  passage de la Sira mentionne 4 \emph{hanifs}, qui anticipent la venue
  de Mohammed. «~Zayd ibn 'Amr ibn Nufayl~{[}\ldots{]} interrogea le
  moine sur la \emph{Hanîfiyya}, la religion d'Abraham. «~Tu recherches
  une religion à laquelle tu ne trouveras personne aujourd'hui pour te
  conduire. Cependant, le temps est proche où un prophète sortira de ton
  pays que tu viens de quitter et prêchera la religion d'Abraham.
  Rejoins-le, car c'est bien la période prévue pour sa mission.~Hichâm,
  Ibn. \emph{La biographie du prophète Mahomet}, Sira I, 222-232}. Mais
il conclut que nous ne savons pas grand-chose de ces \emph{hanifs} et
que la référence à Abraham est probablement un \emph{hysteron proteron,}
une reconstruction ex-post permettant de faire le lien avec les juifs et
chrétiens mais sans se lier à eux\sn{Cf «~Abraham ne fut ni juif
  ni chrétien, mais fut \emph{hanif} et \emph{muslim} {[}à Allah{]}. Et
  il n'était point du nombre des Associateurs » (Coran 3:67)}.

Le contexte n'explique donc par l'Islam et c'est donc vers Mohammed
qu'il faut se tourner~: «~l'islam est bien plus que la plupart des
autres religions, le produit non d'une époque, non d'un peuple, mais de
la personne de son fondateur~»\sn{Ibid, p. 17}. Kuenen le décrit
comme un homme religieux, surtout avant la fuite à Médine, avec une
\emph{pensée sémitique,} en particulier par son eschatologie marquée par
le judaïsme.

L'Islam valorise dès son origine le Coran, glorifié à l'intérieur même
du Coran. Il transmet certes un noyau du Judaïsme transporté sur le sol
d'Arabie. Mais Mohammed avait une vision plus large, comme le soulignent
différentes sourates.~: «~le Coran, est en vérité un avertissement pour
toutes les créatures~»\sn{Ref Coran 8,87~?}, «~nous ne t'avons pas
envoyé aux hommes en général que pour prêcher et menacer, cependant la
plupart des hommes ne comprennent pas~»\sn{Coran 34,27}.~

Mais Kuenen constate en parallèle de ces versets, de nombreux passages
qui valorise les Arabes, ce qui semble limiter cette vocation à
l'universel\sn{« avec quelle extrême insistance est relevé le
  privilège des Arabes qui, dans les signes qui leur ont été montrés,
  c'est-à-dire dans les versets du Qurân, possèdent maintenant la parole
  même d'Allah, {[}qui dépasse{]} de beaucoup (celles qu'ont reçue les
  fondateurs des autres religions) » Ibid, p.18}. De plus, il souligne
des exceptions à l'universalité, qui semblent même s'opposer
formellement à l'essence de l'Islam, comme la Ka'ba et la mise en
exergue de la Mecque, «fragment incompréhensible de paganisme qui est
passé dans l'islam sans être digéré~»\sn{Ibid, p 24}.

Mais alors comment comprendre que l'Islam soit devenu \emph{de fait} une
religion universelle. Pour Kuenen, cela vient de la pauvreté de
l'essence de l'Islam, de sa plasticité dirions-nous aujourd'hui, qui lui
permet de dériver des variétés persane, hindoue ou javanaise. La
simplicité et la concision d'une religion sont pour Kuenen le plus grand
éloge qu'on puisse lui faire à la condition qu'elle permette le
développement spirituel de l'homme, ou qu'au moins elle n'y fasse pas
obstacle. «~A cette condition, mais à cette seule condition, elle peut,
en dépit de son étroitesse, avoir un caractère universel, être une
bénédiction pour l'humanité. »\sn{Ibid, p 26}. Ce n'est pas au
niveau de la vie politique ou juridique qu'est l'enjeu mais bien dans
«~la vie de l'âme et de la conviction religieuse~». Mais Kuenen
considère que de cette pauvreté spirituelle originelle vient de la foi
même de l'Islam~: Allah est certes~\emph{ar-rahmano'r-rahimo,~} le
miséricordieux et le compatissant mais il est surtout «~un dieu de
loin~»\sn{Ibid, p 32} dont il faut suivre les devoirs religieux
prescrits par lui. Pour compenser cet éloignement, les croyants ont
développé par la suite des éléments mystiques, comme le soufisme et la
foi en la \emph{médiation}, soit de Mohamed, soit des saints (dont le
culte est fervent dans l'Islam du XIX\textsuperscript{ème} siècle). Mais
pour le culte des saints, il remarque qu'il est «~plutôt une
protestation contre la religion~»\sn{Ibid, p 31}~: le musulman
recherche ce que sa foi ne lui fournit pas et il le cherche là où,
d'après l'autorité même qu'il reconnait, il ne devrait pas le chercher.
De même pour le soufisme~: le vrai soufi vit une vie intérieure
remarquable mais n'est plus musulman\sn{Ibid, p 31}.

Mais alors qu'est-ce que l'Islam~? Mais «~malheureusement, la foule
n'avait pas tort~» quand elle s'opposait aux rationalistes Mo'tazilites.
En imposant le Coran incréé, la foule a «~barré à leur religion la voie
qui conduit au véritable universalisme. Car l'élément éthique est
l'élément universellement humain~».\sn{Ibid, p 36}

\subsection{Le Wahhabisme comme véritable Islam}
Pour Kuenen, le wahhabisme est le \emph{véritable islam}\sn{Ibid,
  p 37}, formule étonnante quand on pense qu'elle a été écrite il y a
150 ans. Cette doctrine enseignée par Mohammed ben Abdelwahhab en
XVIIIème siècle, propose une forme intransigeante de l'Islam, s'opposant
à tout ce qui ressemble à de l'associationisme, luttant en particulier
contre le culte des saints ou des soufis et prêchant une application
stricte de la loi~: il commande ainsi l'exécution publique par
lapidation d'une femme adultère. En 1880, le wahhabisme est aux franges
géographiques et théologiques du sunnisme, considéré comme hérétique par
l'\emph{Islam Impérial} et réduit aux habitants de l'oasis du
\emph{Nejd,} au centre de l'Arabie. Même si Kuenen n'est pas sûr de son
succès, il reconnaît dans le wahhabisme, la cohérence avec le message et
la pureté originels de l'Islam\sn{Ibid, p 37}. Ses
caractéristiques «~ témoignent d'une façon aussi décisive contre
l'universalisme de l'Islam. Une religion qui peut ainsi être restaurée
en retournant à bon escient à ses origines authentiques, peut répondre
aux besoins des habitants du désert qui l'a vue naître, elle est
incapable de satisfaire des besoins différents et plus
élevés~»\sn{Ibid, p. 38}.

Finalement, l'Islam est une religion nationale, «~un gourmand du
Christianisme, et plutôt encore, disons-nous, du judaïsme~: un extrait,
pour ainsi dire, de la Loi et de l'Evangile fait par un Arabe et pour
les Arabes, calculé d'après leurs capacités et {[}\ldots{]} gâté
devrions-nous dire par des éléments nationaux qui devaient rendre
l'acceptation facile~».\sn{Ibid, p. 39}
\section{Limites de la thèse de Kuenen}
On ne peut réfuter facilement la conclusion de Kuenen, de par la rigueur
de sa démonstration présentée ici rapidement, sa connaissance de
l'Islam, et certaines conclusions qui apparaissent comme prescientes,
comme l'importance théologique du wahhabisme bien avant son
développement numérique. Il nous semble néanmoins que l'avocat de la
défense peut avancer quelques éléments.

Le premier élément, concerne le cadre épistémologique c'est à dire la
\emph{méthode} de Kuenen de chercher à définir le religion telle qu'elle
était \emph{à l'origine}. De nombreux théologiens suivaient la même
approche au XIXème~: ainsi, Adolf Harnack essayait de retrouver le
christianisme originel, que n'aurait pas corrompu l'Hellénisme des
premiers siècles de notre ère. Or, nous savons depuis Albert
Schweitzer\sn{Dans son \emph{Histoire des recherches sur la vie de
  Jésus} (1906) il met en évidence la grande diversité des
  interprétations de Jésus, toutes anachroniques. Il nous faut accepter
  cette distance irréductible, marquée aussi par l'existence de 4
  Evangiles et non d'un seul.} que ce retour à l'origine nous est
impossible~: ce que nous trouvons dans ce passé est largement ce que
nous y projetons, d'où le caractère anachronique et daté de ces
reconstructions des origines. D'une certaine façon, c'est la limite de
l'écriture seule, \emph{scriptura sola}, qui réfute toute légitimité à
la tradition. Or, toute écriture, pour être vivante, a besoin d'un corps
de croyants, qui \emph{confesse} sa foi et transmet le message originel
en l'adaptant à la culture et la langue actuelle\sn{Et comme
  inversement chaque tradition a besoin de se laisser convertir par
  l'Ecriture pour rester authentique}.

Ainsi, par méthode, il est normal que Kuenen identifie le wahhabisme
comme le véritable Islam, avec son projet théologique de retour aux
sources. Mais est-ce le véritable Islam~?

Imiter les \emph{salaf}, les pieux anciens des trois premières
générations musulmanes, encore préservé de la dégradation progressive de
la tradition\sn{Comprendre les crises de l'islam contemporain
  \textgreater{} Emplacement 500}, est-ce l'islam ou bien une projection
des attentes et des frustrations des musulmans des lieux et époques où
wahhabisme et salafisme sont nés\sn{A ce titre, il n'est pas
  anodin que le wahhabisme apparaisse justement au XVIIIème siècle quand
  la décadence de l'Empire Ottoman apparait évident, sous le règne du
  calife Moustafa III~ et le pouvoir des janissaires.}. Seuls les
musulmans d'aujourd'hui peuvent dire ce qu'est le véritable Islam.

Et à ce titre, la critique de Kuenen du wahhabisme est particulièrement
pertinente et mérite d'être entendu~: si les musulmans dans leur
ensemble décidaient que le véritable Islam, c'est le wahhabisme (ou le
salafisme), n'y a-t-il pas un véritable risque de fermeture à
l'universel~? La théologie de l'école hanbalite\sn{Adrien
  Candiard, \emph{Du fanatisme,} 2020, le Cerf. Emplacement 153} met au
centre de son approche l'absolue transcendance de Dieu car « Rien n'est
semblable à Lui » \sn{Coran 42 , 11}. Nous ne pouvons connaître
que ce qu'Il nous fait connaître à travers le Coran. Mais ce qu'il a
révélé d'après la théologie hanbalite, ce n'est pas sa nature,
inaccessible pour l'homme, c'est sa \emph{volonté}. On a pu qualifier
cette théologie de «pieux agnosticisme », c'est-à-dire une théologie qui
pense sa propre inutilité. Seul compte notre action, celle de faire Sa
volonté, et non point notre for intérieur. Adrien Candiard prend
l'exemple d'une fatwa de Ibn Taymiyya. Ce grand théologien rigoriste de
l'école hanbalite est aujourd'hui très influent auprès des milieux
salafistes. Il répond dans cette fatwa à ce qu'il convient de penser des
musulmans qui participent, avec les chrétiens, aux réjouissances qui
entourent le jour de Pâques. Si être musulman, c'est faire ce que Dieu a
commandé (prières,\ldots), de manière symétrique, se réjouir de Pâques,
c'est «~être chrétien~», et donc coupable d'apostat. D'où l'engouement
de l'islam contemporain pour des questions jusque-là marginales, ,
qu'elles soient alimentaires ou vestimentaires~: le voile, manger des
pastèques\ldots{} «~avoir la foi, c'est inscrire dans sa chair même sa
soumission à la loi divine , en portant la barbe ou le voile, en
affirmant visiblement cet amour de la loi. Reprocher à ces croyants de
se perdre dans des détails secondaires et d'en oublier l'essentiel, la
relation à Dieu, c'est entamer un parfait dialogue de sourds
.~»\sn{Candiard, Ibid, Emplacement 206} Le wahhabisme peut
présenter un réel pouvoir d'attraction en dehors du monde arabe du fait
de son caractère très normé, cohérent et peut être même par l'absence de
for intérieur dans un monde changeant et complexe. Est-ce que cela peut
nourrir le besoin spirituel de l'homme~? Nous aimerions, avec Kuenen,
pouvoir répondre par la négative, tant l'image de l'homme qu'elle
véhicule est pessimiste, si éloignée de celle d'un homme \emph{pneumo
spermatikon,} ouvert à la raison et à l'esprit de Dieu.

Kuenen propose l'éthique comme le seul élément universellement humain.
Même si le véritable Islam ne peut se réduire au seul wahhabisme, est-ce
que l'Islam est ouvert à la réflexion éthique, malgré la condamnation de
l'école muza'lite ou bien la pense-t-il dans la limite des devoirs
religieux prescrits par Dieu ? Il faut tout d'abord rappeler
l'importance dans le Coran des limites à mettre à la raison de l'homme
(\emph{Voilà les limites d'Allah. Ne les transgressez donc pas. Et ceux
qui transgressent les ordres d'Allah ceux-là sont les
injustes}.\sn{S.2, 229}). Mais la compréhension de ces limites est
sujet à plusieurs écoles juridiques, hanbalite, mais aussi hanafite,
malikite et le šafi`ite, qui font jouer différemment Coran, hadiths,
coutume et raison ~: il y a donc déjà au sein même de l'Islam une vraie
diversité d'interprétation de la \emph{volonté de Dieu} à suivre,
diversité qui en elle-même permet de dessiner un premier niveau
d'éthique.

Mais plus fondamentalement, au-delà de ces écoles, l'enjeu éthique
concerne aussi la place des devoirs religieux\emph{.} Faut-il tout
formaliser, tout codifier dans la vie du croyant~ou bien plutôt penser
la \emph{šarī`a,} terme qui n'apparaît qu'une fois dans le Coran non pas
une Loi mais une méthode, une voie\sn{Selon le réformateur Rašīd
  Riḍā (m. 1935) \emph{Le Califat}, cité dans le cours.}. Et de même,
penser le \emph{fiqh}, cette formalisation de la loi divine en
prescriptions pratiques, plutôt comme un éveil à la vie spirituelle, en
s'inspirant du grand penseur musulman 'Al-Ġazālī  \sn{« la science de la voie qui mène à la vie de
  l'Au-delà, la connaissance détaillée des maladies de l'âme, de ce qui
  rend les œuvres corrompues, de la puissance avilissante de ce
  bas-monde, de la force d'aspiration des délices du Paradis et de
  l'emprise de la peur sur le cœur » Al-Ġazālī, \emph{Le livre de la
  science}, Paris, La Ruche, p. 58-59.}

\begin{quote}
    

«~C'est le \emph{fiqh} entendu en ce sens qui éveille et avertit et non
les définitions de la répudiation (\emph{talāq}), de l'affranchissement
(`\emph{itāq}), du serment d'anathème (\emph{li`ān}), de la vente à
terme (\emph{salam}), du salaire de location (\emph{ijāra})\ldots{} Tout
cela n'avertit pas et ne fait pas naître la crainte pieuse, bien au
contraire»
\end{quote}
Et il propose ce programme pour les \emph{faqīh,} ces juristes qui
émettent les avis religieux~: «~{[}\ldots{]} ne {[}pas faire{]} douter
les gens de la miséricorde de Dieu, et {[}ne pas les{]} rassurer non
plus au sujet de la ruse divine~; {[}ne pas les{]} désespérer de la
bonté de Dieu, et {[}ne pas{]} négliger le Coran en désirant autre chose
que Dieu~»\sn{Al-Ġazālī, \emph{Le livre de la science}, p. 100.}.

L'Islam parait donc avoir en son sein les ressources pour être une
\emph{vraie bénédiction pour l'humanité}, non en retournant à
l'imitation des pieux anciens, mais proposant la voie (\emph{šarī`a})
originale et fidèle à sa tradition et à sa foi, qui s'attaque et réponde
aux défis de notre monde.
