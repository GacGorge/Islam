

\chapter{Les fondements du droit (uṣūl al-fiqh)}

\vide{introduction-3}{%
\section{Introduction}\label{introduction-3}}

La loi semble être au cœur de la définition de l'identité musulmane.
Certains journalistes ou orientalistes définissent le spécifique de
l'islam comme religion de la Loi, religion de la \emph{šarī`a.} Sur fond
de révolution iranienne, de wahhabisation de l'Afrique noire, d'État
islamique en Syrie, le terme \emph{šāri`a} est assimilé à l'application
de châtiments corporels~: appliquer la \emph{šāri`a}, c'est couper la
main du voleur, lapider la femme adultère, soumettre l'impudique aux
coups de fouet. La \emph{šāri`a} est devenue source de fantasmes et de
peurs, elle est l'expression du retour à une culture obscurantiste. Un
âge que l'on croyait dépassé refait surface et s'il promet l'ordre
social, il est assorti de la restriction des libertés fondamentales les
plus élémentaires et de l'encadrement de la société par des polices de
mœurs.

S'il est certain qu'elle constitue une expression caractéristique de ce
qu'est l'islam, ou de ce qu'il est devenu au cours de l'histoire, à
savoir une \emph{lex divina} au sens plénier du terme, c'est-à-dire une
science, un système juridique, éthique et religieux universel au sein du
monde musulman, qu'en est-il exactement de la \emph{šāri`a~}?

L'étymologie du mot \emph{šarī`a}~est intéressante~: d'après le
\emph{Lisān al-`arab}, c'est le grand dictionnaire arabe, le verbe
\emph{šara`a} signifie s'abreuver. Au sens premier, la šarī`a~désigne le
lieu où les animaux vont s'abreuver. Par extension, c'est la route qui
conduit à ce lieu. D'ailleurs \emph{šāri`} désigne le législateur mais
aussi la rue. Ainsi, la \emph{šarī`a}~a été interprétée comme désignant
la grande route. La voie qui mène là où tous vont s'abreuver. On la
distingue de la \emph{tarīqa}, le sentier, la voie étroite des fidèles.
La \emph{šarī`a} est formée de la loi exotérique religieuse tandis que
la \emph{tarīqa} conduit vers la vérité, au cœur de la \emph{šarī`a}.
Elle comporte des règles, des méthodes qui se superposent à celles de la
\emph{šarī`a}. ~

Le célèbre juriste Mustapha al-Zarka (1904-1999) la définit comme
«~l'ensemble des commandements et prescriptions dogmatiques et pratiques
que l'islam doit appliquer en vue de réaliser ses objectifs tendant à la
réforme de la société~»\sn{~Mustapha al-Zarka, \emph{al-Fiqh fī
  ṯawbihi al-ǧadīd}, Beyrouth, Dār al-Fiqr, 7\textsuperscript{ème} éd.,
  p. 30}. En toute rigueur, il ne peut y avoir de société musulmane sans
l'application de la \emph{šāri`a}. Dans un sens plus restreint, le
professeur Sélim Jahel la définit comme «~un ordre juridique, assuré
d'une réelle positivité, mais aussi comme un ensemble de principes
régulateurs du droit et de la vie sociale, comme peut l'être, par
exemple, dans un Etat occidental le préambule d'une
constitution~»\sn{Salim Jahel, \emph{La place de la chari'a dans
  les systèmes juridiques des pays arabes}, Paris, Panthéon-Assas, 2012,
  p. 8.}.

L'étude des prescriptions propres à la \emph{šāri`a} permet de dégager
des sources et des fondements~: les \emph{uṣūl al-fiqh}. Ils sont
principalement au nombre de quatre~: le Coran, la Sunna, le consensus
des juristes (\emph{iǧma`}) et le raisonnement analogique
(\emph{qiyās}). Il faut distinguer ces principes du \emph{fiqh} en tant
que science du droit musulman qui analyse les actes et les dires de la
personne majeure et responsable et qui vérifie leur conformité aux lois
divines. Ainsi, le juriste (le \emph{faqīh}) va examiner les questions
relatives à la vente, au mariage, à la délégation de pouvoir mais aussi
les pratiques rituelles, la prière, le vol, l'arrestation, l'accusation
d'adultère ou d'homicide. En revanche, les \emph{uṣūl al-fiqh} renvoient
à la théorie du droit. Il s'agit d'étudier les sources du droit à partir
desquelles on définit les prescriptions. 
\begin{Def}[mujtahid - spécialiste du droit]
Le spécialiste de ces
\emph{uṣūl} se nomme un
\href{https://fr.wikipedia.org/wiki/Mujtahid}{\emph{mujtahid}}.
\end{Def} 
Il va
étudier la nature des prescriptions, leur formulation -- s'agit-il d'un
ordre ou d'un interdit --, la formulation est-elle générale ou
particulière~? Ce travail de clarification aboutit à l'élaboration de
catégories juridiques.

\begin{itemize}
\item
  L'ordre~: «~O vous qui croyez, respectez vos engagements~!~» (S. 5, 1). Honorer un contrat est donc une obligation
 
 
\item
  L'interdiction~: «~O vous qui croyez~! ne vous moquez pas les uns des
  autres~» (S. 49, 11). Tourner autrui en dérision est illicite
 
\item
  Prescription générale~: «~Vos mères vous sont interdites~» (S. 4, 23). Aucun homme ne peut épouser sa mère.
\end{itemize}



Sur la base de ces sources, le juriste va donc pouvoir les appliquer à
des cas particuliers et donner des avis (\emph{fatwa}). 

\begin{Def}[fatwa - Avis juridique]
Il s'agit d'une
loi pratique. La \textbf{fatwa} est donc, un avis juridique donné par un
spécialiste de la loi islamique. Il est une réponse à une question
particulière. 
\end{Def}
En règle générale, une \emph{fatwa} est émise à la demande
d'un individu pour régler un problème. Le spécialiste qui émet des
\emph{fatwas} est appelé un \textbf{mufti} ou \emph{faqīh}.
Contrairement à l'opinion souvent répandue, une fatwa n'est pas
forcément une condamnation~: elle est d'abord un avis religieux pouvant
porter sur des domaines variés~: les règles fiscales, les pratiques
rituelles, l'alimentation.

Mais comment la \emph{šāri`a} a-t-elle émergé~? Comment s'est-elle
constituée~? Un bref historique s'impose car il permet d'en saisir la
nature. Dans un second temps, nous présenterons les sources de la
\emph{šarī`a} avant de traiter des principes de la \emph{šarī`a}.

 
\section{Historique de la
\emph{šāri`a}}

Du point de vue historique, l'émergence de la \emph{šāri`a} va
constituer une révolution juridique, économique, sociale, politique.
Jusqu'alors, les principes appliqués dans la péninsule arabe sont ceux
des sociétés tribales. C'est l'appartenance au groupe qui organise les
rapports de solidarité entre les individus. Le sociologue Ibn Khalūn a
décrit le monde de la cohésion sociale au sein d'un univers tribal par
le terme de \emph{ʿaṣabiyya} (\TArabe{عصبية}). Terme que l'on retrouve
parfois à l'époque moderne et qui renvoie à l'idée de communautarisme
voire de nationalisme.

Dans cette perspective, le statut de l'individu n'existe pas en soi~; il
se fond avec celui de la tribu. La vie quotidienne est réglementée par
des coutumes héritées de générations en générations que l'on
appelle\ldots{} \emph{Sunna}. La révolution apportée par l'islam
consiste à supplanter un mode de régulation sociale basé sur la tribu
par un mode de régulation basé sur une foi unique qui transcende les
appartenances tribales.

Il s'ensuit la naissance d'un nouveau lieu de solidarité~: la
\emph{umma} qui vient transcender, dépasser les liens déjà existants.

Il reste que fondamentalement, la nature des liens reste marquée par
l'esprit tribal. On retrouve les fameux trois piliers de l'islam
travaillés par Jacqueline Chabbi. \textbf{L'alliance, le pacte, le
serment d'allégeance} (\emph{bay`a,} \TArabe{بَيْعَة}‎‎) y sont réutilisés
par Muḥammad. On en a un exemple en juin 622, peu de temps avant
l'Hégire dans le pacte d'al-ʿAqaba~où 73 musulmans font acte
d'allégeance à Muḥammad. L'historien tunisien Hicham Djaït y voit l'acte
constitutif du proto-État musulman -- avant même la Constitution de
Médine\sn{Hichem Djaït, \emph{La Grande Discorde}, Paris,
  Gallimard, 2008, p.~44-45.}. Voici ce qu'en dit Ibn Iṣḥāq~:

\begin{quote}
«~Lorsque Dieu~---~Très Haut~---~permit à l'Envoyé d'Allah de faire la
guerre, et après que des gens parmi les Anṣārs lui ont prêté serment en
embrassant l'islam, en le soutenant, lui (Muḥammad) et ceux qui le
suivent et ceux qui sont restés chez eux, parmi les musulmans~---~alors
l'Envoyé d'Allah ordonna à ses compagnons, aussi bien à ceux qui avaient
émigré qu'à ceux qui étaient restés avec lui à La Mecque, d'émigrer à
Médine et de rejoindre leurs frères parmi les Anṣārs~; l'envoyé d'Allah
leur dit~: `Dieu vous a donné des frères et une demeure où vous serez en
sûreté'\sn{~Sīra éditée par Ferdinand Wüstenfeld, 1858-1859, t.1,
  p. 314. Traduction française par Abdurrahmân Badawî~: Ibn Ishaq,
  \emph{Muhammad}, Paris, éditions Al Bouraq, 2001, t.1, p. 374 (revue
  et corrigée par nos soins).} » En principe, vous deviez tout
déchiffrer, y compris le terme de \emph{anṣār}, sinon allez rejeter un
coup d'œil dans le chapitre 5.
\end{quote}

À la suite de l'Hégire en 622, Muḥammad se retrouve à Médine avec autour
de lui à la fois des musulmans mecquois et des musulmans médinois, au
sein d'une ville hétérogène où l'on trouve des communautés juives. C'est
dans ce contexte que se constitue le proto-État avec le document
\emph{al-Ṣaḥīfa,} La Constitution dite de Médine. Elle explicite plus
que le pacte de ʿAqaba la naissance d'une seule communauté en vue de
garantir la paix et la convivence entre les différents groupes
religieux. Il y est clairement explicité, d'après la \emph{Sīra}~que
«~Les émigrés Qoraïchites et ceux de Yathrib, et ceux qui les suivirent
et luttèrent avec eux forment une seule communauté à part et que tous
les musulmans quels que soient leurs tribus ou clans partagent entre eux
le prix du sang, payent la rançon des captifs selon le bon usage et
l'équité~». Nous avons déjà vu le texte de ladite constitution~! La
constitution de la communauté islamique (\emph{al-umma al-islamiyya}) se
renforce de l'union des croyants musulmans contre les tribus juives qui
sont chassées de Médine en 627 en raison de leur mécréance. L'umma est
donc l'association, la communauté des adorateurs du Dieu unique, de ceux
qui suivent son Prophète Muḥammad et respectent les règles divines qu'il
a transmises~:
\begin{quote}
    

S.~9, 71~:

\TArabe{وَالْمُؤْمِنُونَ وَالْمُؤْمِنَاتُ بَعْضُهُمْ أَوْلِيَاءُ بَعْضٍ
يَأْمُرُونَ بِالْمَعْرُوفِ وَيَنْهَوْنَ عَنِ الْمُنكَرِ وَيُقِيمُونَ
الصَّلَاةَ وَيُؤْتُونَ الزَّكَاةَ وَيُطِيعُونَ اللَّهَ وَرَسُولَهُ
أُولَئِكَ سَيَرْحَمُهُمُ اللَّهُ إِنَّ اللَّهَ عَزِيزٌ حَكِيمٌ}

«~Les croyants et les croyantes sont les alliés les uns des autres. Ils
commandent le convenable, interdisent le blâmable, accomplissent la
\emph{ṣalāt}, acquittent la \emph{zakāt} et obéissent à Allah et à Son
messager. Voilà ceux auxquels Allāh fera miséricorde, car Allāh est
puissant et sage.
\end{quote}

\begin{Synthesis}
Il s'ensuit une conséquence politique et juridique considérable~: la
\emph{umma} est gouvernée par Dieu.
\end{Synthesis}


Il s'ensuit une conséquence politique et juridique considérable~: la
\emph{umma} est gouvernée par Dieu. Il est le roi (\emph{al-mālik}), il
détient la souveraineté, le pouvoir sur son peuple. Il reviendra aux
hommes de rendre compte de leurs actes, non devant des tribunaux
humains, mais devant Dieu, Lui qui connaît tout, qui est omniscient.
L'islam vient apporter la révélation de la Loi divine, elle vient offrir
la Loi transmise à Abraham, Moïse, David et Jésus.

Il s'ensuit une révolution dans la conception de l'être humain~:
\begin{Synthesis}
il
n'est plus lié à l'appartenance tribale et familiale, au groupe
parental, mais il acquiert une part d'autonomie en tant que croyant.
\end{Synthesis}
il
n'est plus lié à l'appartenance tribale et familiale, au groupe
parental, mais il acquiert une part d'autonomie en tant que croyant. À
la \emph{`aṣabiyya} où la solidarité est fondée sur l'appartenance au
sang suit une solidarité fondée sur l'islam. Le croyant appartient à une
\emph{umma} et il est tenu d'assister son frère. Mais la révolution
porte aussi sur l'égalité des croyants~: \textbf{devant Dieu, tous sont
égaux, et tous doivent suivre sa Loi.} Non une loi humaine, mais sa Loi.
La distinction entre les hommes n'est possible qu'au niveau de
l'application pieuse à suivre les préceptes divins.

La direction éthique du Coran est celle de la promotion de la justice et
de la lutte contre l'injustice.
\begin{quote}
S.3, 110~:

\TArabe{كُنتُمْ خَيْرَ أُمَّةٍ أُخْرِجَتْ لِلنَّاسِ تَأْمُرُونَ
بِالْمَعْرُوفِ وَتَنْهَوْنَ عَنِ الْمُنكَرِ وَتُؤْمِنُونَ بِاللَّهِ
وَلَوْ آمَنَ أَهْلُ الْكِتَابِ لَكَانَ خَيْرًا لَّهُم مِّنْهُمُ
الْمُؤْمِنُونَ وَأَكْثَرُهُمُ الْفَاسِقُونَ}
\end{quote}
\begin{quote}
«~Vous êtes la meilleure communauté qu'on ait fait surgir pour les
hommes. Vous ordonnez le convenable, interdisez le blâmable et croyez à
Allah. Si les gens du Livre croyaient, ce serait meilleur pour eux, il y
en a qui ont la foi, mais la plupart d'entre eux sont des pervers~».
\end{quote}

\begin{Def}[ḥisba - ordonner le bien]
Le devoir ici est collectif. Il s'agit d'ordonner le bien et d'interdire
le mal (\emph{al ʿamr bi-l maʿrūf wa al-nahy ʿan al munkar}). Ce
principe est particulièrement important dans le wahhabisme. Le terme
utilisé pour désigner ce principe fondamental est la \emph{ḥisba}
(\TArabe{حِسْبة}). 
\end{Def}
Toute la vie sociale, économique, politique, financière
doit tendre à l'application de ce principe. C'est dans ce contexte
qu'émerge la notion de Loi (\emph{šarī`a}) en islam. Il faut bien
comprendre que la Loi est conçue en islam non comme émanant d'un pouvoir
humain (qu'il s'agisse d'un homme, du parlement ou du peuple) issu de
l'intellect de l'homme en vue de définir des normes, mais de Dieu seul.
Dieu est le Grand Législateur. La loi est parole de Dieu. Aussi, le
croyant musulman d'âge adulte et sain d'esprit est responsable de ses
actes~: il est dit \emph{mukallaf} (\TArabe{مُكَلَّف}). Il est de son devoir
de respecter inconditionnellement la Loi divine, de suivre la Voie qu'Il
donne.

De ce point de vue, le terme même de \emph{šarī`a} signifie la voie, le
chemin qui conduit à une source d'eau. Il renvoie donc à l'idée d'un
chemin tracé qui conduit à la vie, au salut -- puisque sans eau, c'est
la mort.

La lecture classique a consisté à voir dans la \emph{šarī`a} une loi
donnée aux seuls musulmans~: elle renvoie à la fois à leur conduite
humaine, leur for externe (\emph{a`mal al-badan}) -- ce sont les actes
accomplis par le corps -- et à leur conscience, leur for interne
(\emph{a`mal al-qalb}). Ici, le sens est plus restreint et la
\emph{šarī`a} acquiert le sens de loi religieuse, révélée aux
musulmans~; elle prend la valeur de droit islamique, \emph{šarī`a
al-islamiyya}, c'est-à-dire la voie donnée par Dieu pour régler et
guider non seulement les comportements humains, les actes de la vie
quotidienne, mais aussi les actes religieux (les piliers de l'islam).

Pour bien saisir le sens de la \emph{šarī`a} il faut entrer dans la
conception musulmane de la \textbf{liberté humaine}. La parole de Dieu est un
acte de bonté, de bienfait, accordé à l'homme. En retour, il est attendu
son obéissance. Il s'agit d'une loi de libération en tant qu'elle
apporte à l'homme un équilibre entre l'excès et la rigueur. 
\begin{Def}[Loi]
La loi
divine n'est pas issue du caprice divin, d'un arbitraire. La loi divine
est donnée à l'homme pour l'extirper de son égoïsme et de sa violence.
\end{Def}


Elle lui est donc utile~; il y a une \emph{maṣlaha} (\TArabe{مصلحة}), un
intérêt public. 
Mais la liberté de l'homme est faillible, faible,
fragile puisqu'il est lui-même limité. Il ne doit user de sa liberté que
pour servir Dieu et non pour se perdre, s'égarer.
\mn{On voit la différence avec la liberté Chrétienne, le fors intérieur. Cf Abraham discutant avec Dieu sur le nombre de personnes justes pour détruire Sodome, ou Jésus qui se laisse convaincre par la syrophénicienne}.

C'est la raison pour laquelle, Allāh (Dieu) met des limites (\emph{ḥudūd
-} \TArabe{حدود}) qui doivent être observées. Une liberté humaine sans
limites pourrait conduire au désordre et à la destruction de la
communauté. Ces frontières qui définissent le périmètre de l'action
représentent le cadre de la loi.

\mn{S.2, 229}


\begin{quote}
\TArabe{تِلْكَ حُدُودُ اللَّهِ فَلَا تَعْتَدُوهَا وَمَن يَتَعَدَّ حُدُودَ
اللَّهِ فَأُولَئِكَ هُمُ الظَّالِمُونَ}

Voilà les limites d'Allah. Ne les transgressez donc pas. Et ceux qui
transgressent les ordres d'Allah ceux-là sont les injustes.
\end{quote}

Ainsi la Loi indique à l'homme le juste comportement, la conduite qu'il
doit mener. 
\begin{Def}[al-aḥkām al-ḫamsa]

Elle classifie les actes selon 5 catégories
juridiques~(\emph{al-aḥkām al-ḫamsa -} \TArabe{الأحكام الخمسة}) :

\begin{enumerate}
\def\labelenumi{\arabic{enumi}.}
\item
  Acte obligatoire (\emph{farḍ} - \TArabe{فرض}) ou \emph{wādǧib} \TArabe{(واجب)
}
\item
  Acte recommandé \emph{Manḏūb} -- \TArabe{مندوب}
\item
  Acte licite, permis, autorisé \emph{mubāh} ou \emph{ḥalāl} - \TArabe{حلال}
\item
  Acte déconseillé \emph{makrūh} \TArabe{مكروه}
\item
  Acte interdit~(\emph{ḥarām} -- \TArabe{حرام})
\end{enumerate}
\end{Def}

\begin{Def}[kāfir]
Celui qui ne respecte pas l'acte obligatoire, est considéré comme un
mécréant, un \emph{kāfir}.
\end{Def}

 Il est puni ici et dans l'au-delà. Tombent
sous le coup de l'obligation les actes relatifs au culte, à la prière
cinq fois par jour, au jeûne du Ramaḍān (S. 2, 182), au paiement de la
\emph{zakāt} (S. 2, 43).

Celui qui commet un acte interdit sera puni ici-bas et dans l'au-delà.
Renvoient à cette catégorie l'interdiction de conclure un contrat de
mariage avec des membres de sa famille, d'épouser une femme répudiée par
trois fois, de forniquer, de consommer de l'alcool, ou des aliments à
base de porc.
\begin{Ex}
Imaginez un homme sous la colère qui répudie sa femme par
trois fois~! C'est fini\ldots{} le mariage est fini~!!! Il ne peut plus
la reprendre pour épouse. Mais pas d'inquiétude, on va s'en sortir,
comme toujours~: la femme va épouser une autre personne qui est dans le
coup avec le mari, et aussitôt le mariage conclu, il va la répudier par
trois fois, et ensuite, elle pourra de nouveau se marier avec son mari,
celui qui s'était mis en colère contre sa pauvre femme parce qu'elle
avait perdu les clefs de la voiture\ldots{} Selon la doctrine malikite,
en principe, le mariage après la répudiation ne doit pas être contracté
avec l'intention de permettre au premier mari de reprendre sa femme.

\end{Ex}

Dans les actes recommandables, on trouve les actes de «~charité~»~:
nourrir un pauvre, pratiquer la circoncision masculine, se recueillir
pour la prière du vendredi, célébrer l'appel à la prière du muezzin, et
surtout se marier -- acte qui est considéré au niveau communautaire
comme obligatoire (`recommandable' au niveau individuel, mais
`obligatoire' au niveau communautaire). 
À propos des actes déconseillés,
s'en abstenir garantit la louange sur terre et après la mort, tandis que
celui qui les accomplit sera montré du doigt. Ainsi, on considère comme
réprouvés les actes qui perturbent l'attention à la prière -- comme
parler pendant l'oraison -- ou pratiquer le commerce le vendredi, ou
bien signer un contrat matrimonial par la volonté unilatérale du mari.

Enfin, est considéré comme \emph{ḥalāl} tout le reste, tout ce qui n'est
pas mentionné dans la Loi. Tout ce que la Loi n'a pas interdit de manger
est donc mangeable.

L'angle qui est le nôtre, à savoir celui des fondations, renvoie à une
période historique performative où la loi va être définie, précisée.
Elle se distingue des autres périodes où elle est surtout appliquée.

Joseph Schacht a montré que la formalisation de ces notions, la
constitution de manuels de \emph{fiqh} s'élaborent jusqu'au milieu du
3\textsuperscript{ème} siècle de l'Hégire. Mais des recherches plus
récentes avancent que la période performative s'étend jusqu'au milieu du
4\textsuperscript{ème} siècle. Cette période réalise 4 dimensions
essentielles de la Loi musulmane\sn{\cite{HallaqLaw}}
\sn{~Wael B. Hallaq, \emph{The
  Origins and Evolution of Islamic Law, Cambridge}, Cambridge University
  Press, 2005, p. 3}~:

\begin{itemize}
\item
  la mise en place d'un système judiciaire à part entière avec ses lois
  et procédures
\item
  la pleine élaboration d'une doctrine juridique
\item
  l'émergence à part entière d'une science de la méthodologie et de
  l'interprétation de la Loi
\item
  la mise en place d'écoles juridiques ce qui présuppose l'existence de
  différents systèmes juridiques.
\end{itemize}

 
\section{Les sources de la šarī`a
} 
\begin{quote}
\TArabe{يَا أَيُّهَا الَّذِينَ آمَنُوا أَطِيعُوا اللَّهَ وَأَطِيعُوا
الرَّسُولَ وَأُولِي الْأَمْرِ مِنكُمْ فَإِن تَنَازَعْتُمْ فِي شَيْءٍ
فَرُدُّوهُ إِلَى اللَّهِ وَالرَّسُولِ إِن كُنتُمْ تُؤْمِنُونَ بِاللَّهِ
وَالْيَوْمِ الْآخِرِ ذَلِكَ خَيْرٌ وَأَحْسَنُ تَأْوِيلًا}
\end{quote}
\begin{quote}
«~O les croyants! Obéissez à Allah, et obéissez au Messager et à ceux
d'entre vous qui détiennent le commandement. Puis, si vous vous disputez
en quoi que ce soit, renvoyez-le à Allah et au Messager, si vous croyez
en Allah et au Jour dernier. Ce sera bien mieux et de meilleure
interprétation (et aboutissement)~». \sn{Sourate 4, 59~}
\end{quote}

L'origine de la \emph{šarī`a} est la volonté de Dieu. La communauté
croyante va donc se référer à toutes les sources dans lesquelles la
volonté divine se manifeste. Les sources du droit sont à l'unanimité des
savants au nombre de quatre. Mais il peut y en avoir d'autres, selon les
écoles de jurisprudence. L'étude des sources, des principes du fiqh
correspond à une discipline spécifique~: \emph{uṣūl al-fiqh}. Parmi ces
grands principes, on distingue le Coran, la Sunna, le consensus,
l'analogie.

\vide{le-coran}{%
\subsection{{Le Coran }}\label{le-coran}}

Bien sûr, sans surprise, le Coran est une source de la Loi. Mais les
choses ne sont pas si simples, puisque comme nous l'avons dit dans les
cours consacrés à l'étude du Coran, il comporte des versets ambigus,
abscons\ldots{} Les juristes vont donc devoir distinguer entre deux
aspects des versets~: l'aspect définitif et l'aspect spéculatif.

 
\paragraph{Aspect définitif
(qaṭ`i)} 

Dans ce cas, il s'agit d'identifier une règle, une loi telle que
l'énonce le Coran dans la mesure où elle est sans ambiguïté. Elle
s'impose par sa clarté et sa spécificité. Elle est définitive. Elle n'a
qu'une signification et n'admet aucune interprétation.

\begin{Ex}
Et à vous la moitié de ce que laissent vos
épouses, si elles n'ont pas d'enfants (S.4, 12)~»

\TArabe{\textbf{وَلَكُمْ} \textbf{نِصْفُ} \textbf{مَا} \textbf{تَرَكَ}
\textbf{أَزْوَاجُكُمْ} \textbf{إِنْ} \textbf{لَمْ} \textbf{يَكُنْ}
\textbf{لَهُنَّ} \textbf{وَلَدٌ}}
\end{Ex}

\begin{Ex}
La fornicatrice et le fornicateur, fouettez-les
chacun de cent coups de fouet~» (S.24, 2)

\TArabe{\textbf{الزَّانِيَةُ} \textbf{وَالزَّانِي} \textbf{فَاجْلِدُوا}
\textbf{كُلَّ} \textbf{وَاحِدٍ} \textbf{مِنْهُمَا} \textbf{مِئَةَ}
\textbf{جَلْدَةٍ}}
\end{Ex}

\begin{Ex}
Et ceux qui lancent des accusations contre des
femmes chastes sans produire par la suite quatre témoins, fouettez-les
de quatre-vingts coups de fouet~» (S.24, 4).

\TArabe{\textbf{وَالَّذِينَ} \textbf{يَرْمُونَ} \textbf{الْمُحْصَنَاتِ}
\textbf{ثُمَّ} \textbf{لَمْ} \textbf{يَأْتُوا} \textbf{بِأَرْبَعَةِ}
\textbf{شُهَدَاءَ} \textbf{فَاجْلِدُوهُمْ} \textbf{ثَمَانِينَ}
\textbf{جَلْدَةً}}
\end{Ex}
 

La dimension quantitative est sans équivoque. Elle doit être suivie par
tous. Elle ne donne pas lieu à une interprétation (\emph{iğtihād}).

 
\paragraph{{Aspect spéculatif
(\emph{ẓannī})}}

Dans ce cas, le Coran conduit à interpréter, à recourir à la raison.
\begin{Ex}
«~Vous sont interdites vos mères et vos sœurs~» (S.4,
23)

\TArabe{\textbf{وَبَنَاتُكُمْ} \textbf{حُرِّمَتْ} \textbf{عَلَيْكُمْ}
\textbf{أُمَّهَاتُكُمْ}}


La question a été de savoir si le mariage avec une sœur qui serait née
d'un adultère est envisageable. Il faut bien voir que nous sommes dans
une mentalité clanique, tribale. Ainsi, cela a été sujet à discussion.
Pour les ḥanafītes, c'est impossible, tandis que les šāfi`ītes en
admettent la possibilité~: il n'est interdit de se marier qu'avec la
sœur née du même mariage.
\end{Ex}
 

La compréhension des versets du Coran mobilise très largement le genre
littéraire appelé les circonstances de la révélation (\emph{asbāb
al-nuzūl}). Pour les connaître, il faut se plonger dans la Sunna. Et
l'on voit comment elle devient aussi une source de loi.

\vide{la-sunna}{%
\subsection{La Sunna}\label{la-sunna}}

Les juristes rappellent que l'usage de la \emph{Sunna} est voulu par
Dieu lui-même. Autrement dit, c'est le Coran qui en justifie l'usage. En
ce sens, ceux qui revendiquent le recours au seul Coran sont infidèles
au Coran.

\vide{la-sunna-preuve-ux1e25ujja-affirmuxe9e-par-le-coran}{%
\subsubsection{{2.1 La Sunna~: preuve (\emph{ḥujja})
affirmée par le
Coran}{2.1 La Sunna~: preuve (ḥujja) affirmée par le Coran}}\label{la-sunna-preuve-ux1e25ujja-affirmuxe9e-par-le-coran}}

Nous allons citer trois versets sur lesquels s'appuie la légitimité
coranique du recours à la Sunna.

\textbf{Exemple~1 :} «~Prenez ce que le Messager vous donne et ce qu'il
vous interdit, abstenez-vous en~» (S.59, 7)

\TArabe{\textbf{الرَّسُولُ} \textbf{فَخُذُوهُ} \textbf{وَمَا}
\textbf{نَهَاكُمْ} \textbf{عَنْهُ} \textbf{فَانْتَهُوا} \textbf{وَمَا}
\textbf{آَتَاكُمُ}}

\textbf{Exemple 2~: «~}Ô les croyants! Obéissez à Allah, et obéissez au
Messager et à ceux d'entre vous qui détiennent le commandement parmi
vous~». (S. 4, 59).

\TArabe{يَا أَيُّهَا الَّذِينَ آَمَنُوا أَطِيعُوا اللَّهَ وَأَطِيعُوا
الرَّسُولَ وَأُولِي الْأَمْرِ مِنْكُمْ}

\textbf{Exemple 3~}: «~Celui qui obéit au messager obéit à Dieu~» (S. 4,
80).

\TArabe{\textbf{مَنْ} \textbf{يُطِعِ} \textbf{الرَّسُولَ} \textbf{فَقَدْ}
\textbf{أَطَاعَ} \textbf{اللَّهَ}}

Si vous vous demandez comment ces versets justifient-ils le recours à la
Sunna, rappelez-vous que la Sunna consigne les paroles de Muḥammad, et
donc aussi ses ordres, ses consignes, ses commandements.

Mais la justification du recours à la Sunna ne se limite pas aux
injonctions coraniques. Il y a aussi l'expérience des compagnons qui se
conformaient à ses jugements, qui respectaient ses interdictions, qui
s'abstenaient de faire ce qu'il avait interdit. Quand Abū Bakr se
trouvait devant un cas où il ignorait l'avis du Prophète il demandait
parmi les Compagnons si l'un d'eux avait entendu le prophète répondre
face à une telle situation.

Mais une fois fondé le recours au Coran et à la Sunna, comment
s'articulent ces deux sources~?

\vide{articulations-entre-le-coran-et-la-sunna}{%
\subsubsection{Articulations entre le Coran et la
Sunna}\label{articulations-entre-le-coran-et-la-sunna}}

\vide{priorituxe9-du-coran-sur-la-sunna}{%
\paragraph{Priorité du Coran sur la
Sunna~?}\label{priorituxe9-du-coran-sur-la-sunna}}

Dans l'ordre des sources, le Coran est d'abord premier. Il semblerait
donc que l'on n'ait à recourir à la Sunna qu'en cas d'absence
d'information dans le Coran.

Par ailleurs, le Coran est sans doute aucun, ce qui n'est pas le cas de
la Sunna. En outre la Sunna est davantage un éclaircissement, une
interprétation, et la source est le Coran et non la Sunna. La source
prime sur son interprétation. Un \emph{ḥadīṯ} établit cette suprématie
du Coran, s'il fallait la prouver par les textes.

Cette suprématie apparaît dans le vocabulaire juridique des
ḥanafītes~dans la distinction entre le \emph{farḍ} et le \emph{wāǧib}
qui renvoient à chaque fois à des obligations, mais le premier est fondé
sur l'autorité définitive du Coran, le second sur la Sunna, il est donc
d'un degré inférieur en raison de la possibilité de doute.

Pour autant, la Sunna constitue aussi une source indépendante.

\vide{la-sunna-source-induxe9pendante}{%
\subsubsection{La Sunna, source
indépendante}\label{la-sunna-source-induxe9pendante}}

\vide{la-sunna-confirme-et-ruxe9affirme-le-coran}{%
\paragraph{La Sunna confirme et réaffirme le
Coran}\label{la-sunna-confirme-et-ruxe9affirme-le-coran}}

\textbf{Exemple~:} «~Il n'est pas permis de prendre le bien d'un
musulman sans son accord explicite~» (\emph{ḥadīṯ})
\begin{quote}
   S. 4, 29~: «~Ô les croyants! Ne dévorez pas les biens des uns et des
autres illicitement -- seules les transactions mutuellement acceptées
sont permises. Vous ne vous suiciderez pas. DIEU est Miséricordieux
envers vous~».
 
\end{quote}

\vide{la-sunna-explique-et-clarifie-le-coran}{%
\paragraph{{La Sunna explique et clarifie le Coran
}{La Sunna explique et clarifie le Coran }}\label{la-sunna-explique-et-clarifie-le-coran}}
\begin{quote}
  \textbf{S. 2, 185~:} «~Que celui d'entre vous qui est présent au mois
(de Ramadan), qu'il jeûne~»

\TArabe{\textbf{فَمَنْ} \textbf{شَهِدَ} \textbf{مِنْكُمُ} \textbf{الشَّهْرَ}
\textbf{فَلْيَصُمْهُ}}
  
\end{quote}

La sunna précise le général qui est formulé dans le Coran.

Le \emph{ḥadīṯ} exonère trois catégories~: les mineurs, les fous, les
personnes endormies.

\vide{la-sunna-donne-des-ruxe8gles-luxe0-ouxf9-le-coran-est-silencieux}{%
\paragraph{La Sunna donne des règles là où le Coran est
silencieux}\label{la-sunna-donne-des-ruxe8gles-luxe0-ouxf9-le-coran-est-silencieux}}

C'est ici que se pose la question de l'indépendance de la Sunna. On peut
donner un certain nombre d'exemples~: ainsi, il est interdit d'épouser à
la fois la tante maternelle et la tante paternelle~; la Sunna précise le
paiement du prix du sang, l'interdiction de la consommation de certaines
viandes ou oiseaux, la peine par lapidation pour l'adultère.

\vide{la-sunna-et-le-coran}{%
\paragraph{{La Sunna et le Coran
}{La Sunna et le Coran }}\label{la-sunna-et-le-coran}}

Selon certains, la Sunna peut même abroger un verset du Coran, mais tout
dépend de la nature du \emph{ḥadīṯ}~: il faut que le \emph{ḥadīṯ} soit
dit \emph{mutawātir}, c'est-à-dire rapporté par tellement de rapporteurs
intègres et justes, qu'il n'y a aucun doute.

Prenons un exemple~: 
\begin{quote}
  S.~2, 180~:~«~\emph{On vous a prescrit, quand la
mort est proche de l'un de vous et s'il laisse des biens, de faire un
testament en règle en faveur de ses (père et mère) et de ses plus
proches. C'est un devoir pour les pieux.}~»  
\end{quote}


~Ce verset montre que toute personne qui possède suffisamment de biens
doit obligatoirement en faire le testament en faveur de ses (père et
mère). Mais cette loi a été abrogée par certains savants par le
\emph{ḥadīṯ}~: 
\begin{quote}
    «\emph{Certes, Allah a attribué à tout ayant-droit ce
qu'il mérite. Point de testament en faveur d'un héritier}.»
\end{quote}


~Ce \emph{Ḥadīṯ} a abrogé la loi contenue dans ce verset, mais la
lecture de ce dernier ne l'a pas été.

Il semble en tous les cas que ce soit une question disputée et certains
rapportent un \emph{ḥadīṯ} de Muḥammad qui dit qu'en cas de
contradiction entre le Coran et sa parole, il faut prendre le Coran. Il
est évident que pour le courant des coranistes, une telle articulation
entre le Coran et la Sunna est une aberration.

On trouve aujourd'hui un cas tragique d'abrogation du Coran par la Sunna
dans l'application de la lapidation de l'adultère. Le Coran restreint la
peine à cent coups de fouets mais le \emph{ḥadīṯ} prévoit la lapidation.
De même, le verset coranique «~Pas de contrainte en religions » est
abrogé par un \emph{ḥadīṯ}~: «~Celui qui quitte sa religion, tuez-le~».

\vide{tous-les-actes-du-prophuxe8te-sont-ils-uxe0-suivre}{%
\paragraph{{Tous les actes du Prophète sont-ils à suivre~?
}{Tous les actes du Prophète sont-ils à suivre~? }}\label{tous-les-actes-du-prophuxe8te-sont-ils-uxe0-suivre}}

Non, certains comportements sont spécifiques au Prophète. Par exemple,
le coran limite le mariage d'un musulman à quatre femmes (S. 4, 3) alors
que Muḥammad en épousa plus de quatre. De même Muḥammad va prononcer la
culpabilité sur un seul témoignage alors que le Coran exige deux
témoignages.

Par ailleurs on distingue, dans les actes du prophète, ce qui relève de
la prophétie et ce qui relève de son expérience humaine. Ainsi, dans une
bataille, alors qu'il avait planifié un plan, on lui demanda si la
stratégie venait de Dieu. Il répondit qu'il n'en était rien, que c'était
seulement une ruse de guerre. Un de ses compagnons lui montra alors un
lieu mieux situé, et c'est celui-ci qui fut suivi. Ou encore, alors
qu'il vit les Médinois pratiquer la pollinisation artificielle de leurs
palmiers, il leur interdit cette pratique. Ils lui obéirent, mais la
récolte fut désastreuse. Il dit alors~: «~dans ce cas, fécondez vos
palmiers. Vous êtes meilleurs connaisseurs que moi dans les affaires de
votre vie profane~».

Au Coran et à la Sunna, il faut ajouter aussi d'autres sources de Loi.

\vide{le-consensus-iux1e7mux101}{%
\subsection{{Le consensus (\emph{iǧmā`})
}}\label{le-consensus-iux1e7mux101}}

\begin{Def}[{iǧmā`}]
Le terme de \emph{iǧmā`} désigne le consensus sur un cas juridique.
\end{Def}


\vide{preuve-du-consensus-dans-le-coran}{%
\subsubsection{ Preuve du consensus dans le
Coran}\label{preuve-du-consensus-dans-le-coran}}
\begin{quote}
    

S. 4. 59~: «~Ô les croyants! Obéissez à Allah, et obéissez au Messager
et à ceux d'entre vous qui détiennent le commandement~».

S. 4, 115~: «~Et quiconque fait scission d'avec le Messager, après que
le droit chemin lui est apparu et suit un sentier autre que celui des
croyants, alors Nous le laisserons comme il s'est détourné, et le
brûlerons dans l'Enfer. Et quelle mauvaise destination!~»

S. 3, 110~: «~Vous êtes la meilleure communauté qu'on ait fait surgir
pour les hommes~; vous ordonnez le convenable, interdisez le blâmable et
croyez à Allah~».
\end{quote}
\vide{preuve-du-consensus-dans-la-sunna}{%
\subsubsection{Preuve du consensus dans la
Sunna}\label{preuve-du-consensus-dans-la-sunna}}
\begin{quote}
    

«~Ma communauté ne sera jamais en accord sur une erreur (\emph{ḫaṭa}) /
un égarement (\emph{ḍalāla})~» (Ibn Māğa, \emph{Sunan}, II, 1303,
\emph{ḥadīṯ} n°3950).
\end{quote}
Sur la base de ces textes, le consensus est la troisième source du
droit.

\vide{discussions}{%
\subsubsection{ Discussions}\label{discussions}}

Certains savants et notamment aussi des ši`ites avancent que les
conditions requises ne sont jamais réalisées~: comment savoir que tous
les savants émérites ont été consultés~? Sur quels critères les
définir~? comment les consulter sur l'ensemble des continents~? Par
ailleurs, il peut y avoir des malentendus, des incompréhensions~; un
savant peut se rendre compte qu'il a commis une erreur. Or, le consensus
doit être attesté au même moment. Ibn Hanbal disait à cet égard qu'il
vaut mieux dire «~à ma connaissance, il n'y a pas eu de conflit
d'opinions sur cette question~».

Toutefois, la plupart le reconnaissent comme une source en soulignant
qu'il s'est réalisé au sujet par exemple de la nomination d'Abū Bakr...
ou de l'interdiction de la graisse de porc.

\vide{lanalogie-qiyux101s}{%
\subsection{{L'analogie
(\emph{qiyās})}}\label{lanalogie-qiyux101s}}

C'est l'application d'un cas de la \emph{sha`ria} à un autre, par
extension. C'est l'extension d'une règle textuelle à un nouveau cas. Son
usage implique d'identifier par la raison une relation entre les deux
cas. 
\begin{Def}[{iğtihād}]
C'est ici que s'effectue l'\emph{iğtihād}, c'est-à-dire l'effort
intellectuel d'interprétation afin de répondre à une question.
\end{Def}
 La
justification de cette source s'appuie sur un \emph{ḥadīṯ} mettant en
scène Mu`āḏ Ibn Ǧabal~:
\begin{quote}
    Le Prophète l'avait envoyé au Yémen pour qu'il y exerce les fonctions de
juge et Muḥammad lui demanda~: «~\emph{Sur quoi fonderas-tu ton
jugement~? Sur le livre de Dieu~! répondit Mu`āḏ. Et si le jugement ne
s'y trouve pas~? Sur la tradition du Prophète. Et si tu n'y trouves
rien~? Alors je mettrai en œuvre toutes mes facultés intellectuelles en
vue de formuler mon jugement. Sur quoi, le Prophète conclut~: Louanges à
Dieu qui a facilité au messager du Messager de Dieu (Mu`āḏ) d'adopter
une position qui coïncide avec l'agrément du Prophète}~».
\end{quote}


 

Il faut toutefois remarquer que ce \emph{ḥadīṯ} est emblématique. D'une
part, il est considéré comme faible (\emph{munkar}) selon al-Bukharī,
d'autre part, les \textbf{savants sunnites d'obédience salafiste} y
voient un danger et une incompatibilité avec les principes du
\emph{fiqh} dans la mesure où il privilégie le Coran sur la Sunna et
laisse sous-entendre que l'analogie est à la Sunna ce que la Sunna est
au Coran. Or, pour le juriste salafiste Muḥammad Nāṣir ad-Dīn al-Albanī
(m.~1999), on ne peut émettre un jugement sur la seule base du Coran. Il
est impératif de rechercher aussi dans la Sunna. Pour lui, le Coran et
la Sunna ne constituent qu'une seule et même source.

\vide{istiux1e25sux101n-uxe9quituxe9-pruxe9fuxe9rentielle-et-maux1e63laux1e25a-mursala-intuxe9ruxeat-collectif.}{%
\subsection{{\emph{Istiḥsān} (équité préférentielle) et
\emph{maṣlaḥa mursala} (intérêt collectif).
}}\label{istiux1e25sux101n-uxe9quituxe9-pruxe9fuxe9rentielle-et-maux1e63laux1e25a-mursala-intuxe9ruxeat-collectif.}}
\begin{Def}[istiḥsān]
L'\emph{istiḥsān} désigne le choix préférentiel, l'application d'une
exception dans un cadre particulier en vue du bien qui peut en sortir.
\end{Def}

Il est appliqué dans le cadre de deux \emph{qiyās} divergents ou de deux
prescriptions religieuses contraires. On va préférer une prescription
exceptionnelle à une prescription générale selon le contexte. Il s'agit
aussi d'éviter une nuisance. Cette source secondaire du droit n'est pas
sans critique. Ainsi, al- Šāfi`ī accuse le juriste recourant à
l'\emph{istiḥsān} de se prendre pour le Législateur. Or, seul Dieu
l'est. Pour autant, les hanafites répondent qu'il s'agit de choisir
entre deux \emph{qiyas} contraires, produits par des liens logiques.

\begin{Ex}
Le Shaykh Muhammad Abu Zahrah donne comme exemple, dans son
livre sur l'école malékite, le cas d'un couple où la femme vient à
mourir, laissant derrière elle un mari, deux enfants du couple et deux
enfants de la femme issus d'un premier mariage.\\
L'application stricte du principe d'analogie reviendrait à donner la
moitié de l'héritage au mari, le sixième à la fille et le tiers au fils'
du couple. Leurs demi-frère et demi-sœur ne recevraient rien. Confronté
à ce problème, Sayyiduna Umar ibn al-Khattab a considéré, à la lumière
de l'\emph{istiḥsān}, qu'ils devaient eux aussi hériter de la même
proportion qui revient aux enfants du couple.
\end{Ex}
\begin{Ex}
en cas de maladie, on va accepter qu'une femme se montre nue
à un docteur homme car il est question de sa vie.
\end{Ex}
\begin{Ex}
on va suspendre l'application des peines légales en temps de
famine, car la famine pousse nécessairement à voler. Ainsi, en temps de
famine, on ne coupera pas la main au voleur.
\end{Ex}


\vide{urf-coutume}{%
\subsection{`Urf (coutume)}\label{urf-coutume}}

C'est la coutume, l'usage qui est fait par les gens. Il se différencie
du consensus qui relève des spécialistes. On distingue les bonnes
habitudes des mauvaises habitudes. Les bonnes habitudes sont celles qui
sont compatibles avec la religion, à l'exemple de la consommation du
mariage seulement après le versement de la dot. Les mauvaises habitudes
sont incompatibles avec la religion~: la pratique de l'usure, des jeux
du hasard, des manifestations excessives d'émotion lors des mariages ou
des funérailles.

Pour bien saisir comment sont comprises ces sources et articulées entre
elles, il importe de préciser les principes fondamentaux de la Loi
musulmane.

\vide{question-disputuxe9e-lhomme-peut-il-uxe9tablir-une-loi}{%
\subsection{{Question disputée~: l'homme peut-il établir
une loi~?
}}\label{question-disputuxe9e-lhomme-peut-il-uxe9tablir-une-loi}}

La loi définissant le bien et le mal, ce qui est autorisé et interdit,
l'homme par sa raison est-il en mesure de parvenir à en donner une
définition juste ou a-t-il besoin d'une révélation~? Sur ce point
plusieurs écoles, plusieurs penseurs se sont affrontés. Nous pouvons
distinguer trois grandes tendances~:

\begin{itemize}
\item
  Pour les \textsc{mu`tazilites}. Wāṣil Ibn `Aṭā' (\TArabe{واصل بن عطاء} )
  (m. 748) considère que la loi ordonne en fonction de la beauté ou de
  la laideur des actions inhérentes à leur nature. La raison définit ce
  qui est beau et ce qui est laid et la Loi confirme ce que la raison a
  défini. Il ne saurait donc y avoir d'opposition entre la loi divine et
  la loi humaine, et les peuples qui n'ont pas reçu de révélation sont
  responsables devant la loi. Il reste que Dieu révèle certains
  commandements qui ne sont pas accessibles à la raison comme le fait
  qu'il y ait cinq prières quotidiennes ou des conditions même de
  validité de ces prières.
\item
  Pour les \textsc{aš`arites}~: Al-Aš`arī (m. 935) souligne
  l'impossibilité pour la raison de connaître et de comprendre les
  prescriptions divines. Non seulement l'homme a besoin d'une
  révélation, mais il a aussi besoin d'aide, de messagers de Dieu, pour
  se faire expliquer la Loi. L'envoi de messagers est donc nécessaire.
  Il va y avoir bien sûr une justification de la Sunna. Pour al-Aš`arī
  la raison est inconstante, elle juge ce qui est bon en fonction des
  passions, des désirs du moment, elle ne permet pas de savoir ce que
  Dieu juge bon. L'être humain doit donc répondre de ses actes non
  devant sa raison mais devant Dieu et sa Loi. Qu'en est-il des peuples
  qui n'ont pas reçu de révélation~? Question débattue, mais pour
  certains aš`arites, ils ne sont pas responsables devant Dieu du bien
  ou du mal qu'ils auraient accompli car selon la parole coranique~:
  «~Nous n'avons tourmenté aucune nation avant de lui avoir envoyé un
  Apôtre~» (S. 17, 15)
\item
  Pour les maturidites~: Abū Manṣūr al Māturīdī (\TArabe{ابو المنصور
  الماتريدي}) (m. 944), propose une voie intermédiaire. La raison permet
  de distinguer le bien et le mal, mais tout dépend des qualifications
  de ceux qui y recourent.
\end{itemize}

 

Le penseur andalou Ibn Khaldūn \label{theol:IbnKhaldun1}  (m. 1406), s'est posé la question. À la
fois philosophe et sociologue, il constate que les sociétés dans
lesquelles Dieu n'avait pas envoyé sa révélation sont les plus
nombreuses et qu'elles ont connu des siècles de grande prospérité. Sur
la base de l'observation, il en conclut que le pouvoir théocratique
n'est pas indispensable. Il fait quand même une exception à propos des
Arabes puisqu'il écrit dans son Discours sur l'histoire universelle~:
«~En raison de leur sauvagerie innée, ils sont parmi tous les peuples,
trop réfractaires pour accepter l'autorité d'autrui, par rudesse,
orgueil, ambition et jalousie. Leurs aspirations tendent rarement vers
un seul but. Il leur faut l'influence de la loi religieuse, par la
prophétie ou la sainteté, pour qu'ils se modèrent d'eux-mêmes et qu'ils
perdent leur caractère hautain et jaloux. Il leur est alors facile de se
soumettre et de s'unir, grâce à leur communauté religieuse. Ainsi,
rudesse et orgueil s'effacent et l'envie et la jalousie sont
freinées~»\sn{~Ibn Khaldun, \emph{Discours sur l'histoire
  universelle}, p. 89.}.

Pour autant si la Loi divine n'est pas indispensable, elle a la
préférence d'Ibn Khaldūn qui y voit une Loi sûre visant à assurer le
bonheur de l'homme dans l'au-delà tandis que les lois humaines sont
toutes conjoncturelles et concernent seulement des intérêts temporels.


\section{{Les 5 principes fondamentaux de la šāri`a
}
\label{iii--les-5-principes-fondamentaux-de-la-ux161ux101ria}}

Nous avions dit que la Loi divine n'est pas arbitraire~: elle a une
utilité. 
\begin{Def}[maqāṣid al-šari`a]
La \emph{šarī`a} doit guider et veiller à la préservation des 5
principes~que sont la religion ~ \TArabe{الدين} , la vie \TArabe{الحياة} ,
l'intelligence, \TArabe{العقل} , les biens \TArabe{المال} et la descendance
\TArabe{النسل}.
\end{Def}


Ce sont les \emph{maqāṣid al-šari`a}, les finalités de la Loi. Sur la
base de ces finalités, il s'agit d'orienter la lecture de la Loi et de
montrer que son intention répond à ces 5 principes. Il s'agira aussi,
par rapport à une question posée ou une difficulté, de la lire à la
lumière de ces cinq principes.

\vide{pruxe9server-la-religion}{%
\subsection{Préserver la religion}\label{pruxe9server-la-religion}}

\textsc{La religion répond au désir de l'homme d'adorer Dieu,} elle aide
l'homme à se souvenir qu'il est fait pour adorer Dieu. C'est la religion
qui apporte la paix, la quiétude, le bonheur. La religion est vitale
pour l'homme. On s'appuie sur la Sourate 30, 30~: 
\begin{quote}
    «~Dirige tout ton
être vers la religion exclusivement {[}pour Allah{]}, telle est la
nature qu'Allah a originellement donnée aux hommes - pas de changement à
la création d'Allah -. Voilà la religion de droiture; mais la plupart
des gens ne savent pas~».
\end{quote}

Pour les juristes, il s'agit de préserver l'islam, mais le Coran parle
de religion, non d'islam.

\vide{pruxe9server-la-vie}{%
\subsection{Préserver la vie}\label{pruxe9server-la-vie}}

On trouve parmi les nécessités que la loi doit préserver, celle de la
vie. Elle est inviolable et la loi consiste à la sauvegarder. Ainsi, le
mariage est prescrit car il contribue à la reproduction des êtres
humains. La relation entre époux est sacrée et est un signe de
l'alliance de Dieu.

Pour vivre, l'homme a besoin de nourriture, de boisson, d'habits et de
logement. Ce sont les besoins primaires. Ainsi, il est interdit au
musulman de se priver de ces «~nécessaires » qui menaceraient sa vie. La
société a le devoir de fournir ces minimums vitaux à ceux qui ne
seraient pas aptes à se les procurer.

\vide{pruxe9server-la-raison}{%
\subsection{Préserver la raison}\label{pruxe9server-la-raison}}

\begin{Def}[Raison - `aql]
Al-Ġazālī \label{theol:AlGazali21} définit le \emph{`aql} comme la faculté de connaître, de
comprendre et de penser. C'est elle qui distingue l'homme de
l'animal
\end{Def}

\mn{\textsuperscript{~}\textsc{Al-Ġazālī },
  \emph{Iḥyā'}, \emph{op.cit.}, K.1 (\emph{Kitāb al-`ilm}), B.7, b.2,
  p.~103 {[}V.1, p.~312-314{]}. Dans \emph{Mīzān al-`amal}, al-Ġazālī
  souligne que l'homme dispose en son âme de caractéristiques
  spécifiques par rapport aux animaux, mais pour le reste, ses facultés
  lui sont communes~: «~il est donc créé à un échelon intermédiaire
  entre la bête et l'ange. Il possède un ensemble de facultés et
  d'attributs~; en tant qu'il se nourrit et se reproduit, il est
  végétal~; animal en tant qu'il sent et se meut (\ldots) C'est
  uniquement la faculté de l'intelligence et de l'entendement des
  essences des choses (\emph{qūwat al-`aql wa darak ḥaqā'iq al-ašyā'})
  qui est le trait caractéristique de l'homme et qui justifie sa
  création~»~: \textsc{Al-Ġazālī}, \emph{Mīzān al-`amal},
  \emph{op.cit.}, p.~210 {[}trad. fr., p.~22{]}.}

Communément traduit
par raison ou intellect, le \emph{`aql} est la propriété qui permet à
l'homme de concevoir et de comprendre la réalité. Il est une lumière par
laquelle l'homme atteint les objets.

En conséquence, l'islam interdit ce qui porte préjudice à la raison,
notamment les drogues, le vin, etc.~: 
\begin{quote}
    «~Ô vous qui avez cru~! le vin, le
jeu, la divinisation par les entrailles des victimes ainsi que le tirage
au sort ne sont qu'un acte impur de ce que fait Satan » (S. 5, 90).

\TArabe{يَا أَيُّهَا الَّذِينَ آَمَنُوا إِنَّمَا الْخَمْرُ وَالْمَيْسِرُ
وَالْأَنْصَابُ وَالْأَزْلَامُ رِجْسٌ مِنْ عَمَلِ الشَّيْطَانِ
فَاجْتَنِبُوهُ لَعَلَّكُمْ تُفْلِحُونَ}

\end{quote}

Il exhorte à recourir à sa raison et non pas à l'imitation sans
s'appuyer sur une preuve.
\begin{quote}
    «~Ils dirent~: `N'entrera jamais au Paradis que celui qui est juif ou
chrétien'. C'est là du moins leurs espoirs. Dis~: `Apportez votre preuve
si vous êtes véridiques'~» (S.2, 111).

\TArabe{وَقَالُوا لَنْ يَدْخُلَ الْجَنَّةَ إِلَّا مَنْ كَانَ هُودًا أَوْ
نَصَارَى تِلْكَ أَمَانِيُّهُمْ قُلْ هَاتُوا بُرْهَانَكُمْ إِنْ كُنْتُمْ
صَادِقِينَ}

\end{quote}

Cela a des conséquences pratiques~: il est préférable de faire la prière
après avoir mangé, car on a plus de force que s'il l'on est affamé.

Dans cet esprit, on valorisera tout ce qui a trait au savoir, aux
études, etc. En cas de conflit d'intérêt, on n'omettra pas de
s'interroger sur ce qui favorise la science.

\vide{pruxe9server-les-biens}{%
\subsection{Préserver les biens}\label{pruxe9server-les-biens}}

\textsc{Ils permettent à l'homme de s'accomplir, de réaliser le bien, de
se préserver du mal. D'où la reconnaissance de la propriété privée et
l'interdiction de toute forme de} monopolisation de la richesse par une
minorité. La \emph{zakāt} a pour objectif d'assurer la redistribution.

Dans ce cadre, on légitime le travail~et on souligne la nécessité d'un
juste salaire : 
\begin{quote}
    «~\emph{Personne n'a jamais consommé une nourriture
meilleure que le fruit de son labeur~; et le Prophète Dâwud mangeait du
fruit de son travail.}~»
\end{quote}

On retrouve toutes les questions relatives à la transmission et à la
nécessité de son équité. Le commerce ne doit pas porter atteinte au bien
d'autrui, léser une partie, etc. Pas de vol, pas d'escroquerie. L'homme
n'est qu'un mandataire. On s'assurera de garantir les biens des mineurs,
de veiller à un accord mutuel pour tout contrat~: il ne peut être valide
sans l'accord des deux parties.

\vide{pruxe9server-la-filiation}{%
\subsection{{Préserver la filiation
}}\label{pruxe9server-la-filiation}}

Cela revient à assurer la perpétuation de l'espèce humaine. D'où la
prescription du mariage~: elle est la voie naturelle. C'est dans cette
optique que le Coran expose une critique de la chasteté des moines.

\mn{S. 57, 27~}
\begin{quote}
 «~Nous avons ensuite envoyé sur leurs traces nos autres prophètes et
nous avons envoyé après eux Jésus, fils de Marie. Nous lui avons donné
l'Évangile. Nous avons établi dans les cœurs de ceux qui le suivent la
mansuétude et la compassion et la vie monastique (\emph{rahbāniyya})
qu'ils ont instaurée -- nous ne leur avons pas prescrite -- uniquement
poussés par la recherche de la satisfaction de Dieu. Mais ils ne l'ont
pas observée comme ils auraient dû le faire. Nous avons donné leur
récompense à ceux d'entre eux qui ont cru, alors que beaucoup d'entre
eux sont pervers~».

\TArabe{ثُمَّ قَفَّيْنَا عَلَى آَثَارِهِمْ بِرُسُلِنَا وَقَفَّيْنَا بِعِيسَى
ابْنِ مَرْيَمَ وَآَتَيْنَاهُ الْإِنْجِيلَ وَجَعَلْنَا فِي قُلُوبِ
الَّذِينَ اتَّبَعُوهُ رَأْفَةً وَرَحْمَةً وَرَهْبَانِيَّةً ابْتَدَعُوهَا
مَا كَتَبْنَاهَا عَلَيْهِمْ إِلَّا ابْتِغَاءَ رِضْوَانِ اللَّهِ فَمَا
رَعَوْهَا حَقَّ رِعَايَتِهَا فَآَتَيْنَا الَّذِينَ آَمَنُوا مِنْهُمْ
أَجْرَهُمْ وَكَثِيرٌ مِنْهُمْ فَاسِقُونَ}
   
\end{quote}

Pour Muqātil, le monachisme de cette époque ne correspond pas à celui du
christianisme à ses commencements\sn{Muqātil, \emph{Tafsīr}, iv,
  p. 246.}. Muqātil commente en soulignant que le nombre des
polythéistes a crû alors que les croyants furent humiliés et défaits.
Les croyants se réfugièrent dans des lieux d'ermitage. Certains
inventèrent le christianisme (\emph{naṣrāniyya})~: c'est pourquoi Dieu a
dit~: `ils inventèrent le christianisme'. Ils se mirent alors à
accomplir des pratiques qui n'étaient pas dans ce que Dieu recommanda,
comme le célibat -- \emph{tabattalū}. D'autres cependant, restèrent
fidèles à la religion de Jésus jusqu'à l'arrivée de Muḥammad.
L'ascétisme (\emph{raḥbāniyya}) n'est donc pas condamné mais le célibat
et les doctrines inventées par les chrétiens.

De ce 5\textsuperscript{ème} principe, il s'ensuit aussi le souci
d'éducation et du maintien des liens familiaux. Les parents doivent
éduquer leurs enfants et subvenir à leurs besoins. Toutes les
prescriptions sur la famille, les liens matrimoniaux doivent répondre à
ce devoir. La question de la pudeur et du voile est souvent reliée à ce
principe. De même pour l'interdiction de l'adultère, de la diffamation.


\section[Les écoles Juridiques]{Les différentes écoles juridiques~: leurs
fondateurs, leurs méthodologies, leurs enseignements
}

Au cours du premier siècle de l'Hégire, les opinions juridiques se
multiplient et deux principales écoles ou méthodologies vont se dégager.
La question est de savoir~s'il y a une école plus progressiste qu'une
autre. Une autre question est~: Est-ce qu'une école favorise une
certaine paralysie de l'islam~?
\begin{itemize}
    \item la première méthodologie, celle des Gens du \emph{ḥadīṯ} , prônait
l'application stricte et rigoureuse du Coran et de la \emph{Sunna},
mettant l'accent sur la lettre et la narration. Cette école eut de
nombreux adeptes et trouva une terre fertile dans le \underline{H}ijâz
en général, et à Médine en particulier. En effet, cette méthodologie
était en harmonie avec la vie à Médine, la ville du Prophète~: une ville
fortement attachée aux enseignements du Prophète et ayant préservé sa
simplicité et son climat sain. Médine se dressa longtemps comme un
rempart devant les idéologies sociales et politiques étrangères issues
des nombreuses conquêtes islamiques et du contact avec de nouveaux
peuples et de nouvelles cultures.
\item  la deuxième méthodologie, l'école de l'Opinion, plus interprétative
que la précédente, prônait également l'attachement, le respect et
l'application du Coran et de la \emph{Sunna}, mais mettait davantage
l'accent sur le rôle de l'intellect dans l'appréhension et
l'interprétation des énoncés ainsi que dans la déduction des jugements
légaux selon les règles de cette discipline. Cette école s'était
fortement répandue en Irak qui était, à cette époque, le foyer
scientifique musulman le plus actif. L'Irak était fort d'une histoire
scientifique riche~; le recours à la recherche et à l'analyse
rationnelle étaient devenus familiers dans l'environnement irakien,
confronté à diverses cultures, notamment la culture persane où
foisonnaient les idéologies et les philosophies.
\end{itemize}

C'est dans ce contexte que vont émerger des écoles de droit. Elles sont
nombreuses et leur spécificité dépend de la méthodologie adoptée. On
distingue généralement quatre écoles, mais c'est réducteur. Il y en a eu
beaucoup plus. Cependant, quatre grandes écoles se sont imposées au sein
de l'islam sunnite. Ce qui est remarquable, c'est qu'il n'y a pas
anathémisation d'un avis divergent donné par une autre école sur un
sujet, mais respect de l'opinion énoncée, même si elle n'est pas
partagée.
\mn{On n'est pas obligé d'être d'une école mais si on fait partie d'une école, il faut suivre la totalité de son enseignement. Approche d'humilité ?}

Je vais rapidement présenter ces écoles et exposer comment elles
articulent les sources du \emph{fiqh}.
%-------------------------------------------------

\vide{le-hanafisme}{%
\subsection{Le hanafisme}\label{le-hanafisme}}

Abū Ḥanīfa (80H-150H / 699-766) est un irakien Sa jurisprudence
s'inscrit dans l'esprit de l'école de Kūfa connue pour~recourir~à~:

\begin{itemize}
\item
  la méthode du \emph{qiyās} (analogie)
\item
  la raison (\emph{al-ra'y})~: mais il s'agit d'une raison qui ordonne
  les textes, non d'une raison autonome.
\end{itemize}

On dit d'elle qu'elle est l'école du Coran pour l'importance qu'elle lui
accorde. C'est aussi l'école de l'utilité publique (\emph{maṣlaḥa}).

Abū Ḥanīfa connaît l'enseignement de `Alī et de ses successeurs~ce qui
témoigne d'une recherche de la profondeur des choses. Il est aussi très
attentif à l'authentification du \emph{ḥadiṯ}

Voici la manière dont il définit sa méthodologie~:

\begin{quote}
« J'examine d'abord le Livre de Dieu si j'y trouve la solution {[}au
problème posé{]}. Si je n'y trouve pas, j'examine alors la sunna
prophétique et ce qui en a été rapporté de manière sûre. Si je n'y
trouve point, j'examine alors les avis des Compagnons. S'il en a été
rapporté plusieurs, alors je choisirai l'un des avis, tout en ne sortant
pas de l'ensemble des avis. Et si la question arrive jusqu'à Ibrâhîm,
Al-Sha`bī, Al-Ḥasan, Ibn Sirīn ou Sa`īd Ibn al-Musayyib, je procèderai
de mon propre effort (\emph{iğtihād}) comme ils l'ont fait. »
\end{quote}

De tous ces principes, on peut déduire certaines spécificités de son
\emph{fiqh.} Ainsi, par exemple~:

\begin{itemize}
\item
  la femme pubère peut se marier sans l'avis de son représentant (waliy
  \TArabe{ﱄﻭ}),
\item
  il refuse les restrictions (\emph{hağr}) accordées à l'idiot ou à
  l'endetté, contre l'avis unanime des autres savants.
\item
  il laisse la liberté au propriétaire d'un bien de l'utiliser comme il
  l'entend.
\item
  il est possible de donner la \emph{zakāt al-fitr} en espèces.
\end{itemize}

\vide{le-malikisme}{%
\subsection{Le malikisme}\label{le-malikisme}}

L'Imām Mālik~(89 -173 / 711-795) est né à Médine.

Sa méthodologie\emph{~}est marquée par la prédominance des avis des
compagnons sur le \emph{ḥadīṯ āḥād}. Vous ne vous souvenez probablement
pas. Il s'agit d'un \emph{ḥadīṯ} rapporté par une seule personne.

Dans sa conception, la \emph{maṣlaha}~consiste à enlever une gêne
(\emph{daf` haraǧ}), à éloigner une nuisance (\emph{daf` madarra}).
Bref, il faut avoir du bon sens et éviter de se compliquer la vie.

Pour Mālik,

\begin{itemize}
\item
  la foi peut augmenter, mais en revanche, il soutient qu'elle ne peut
  diminuer~: aucun texte du Coran ne permet en effet une telle
  affirmation.
\item
  concernant le verset : «~Dis: `Dans ce qui m'a été révélé, je ne
  trouve d'interdit, à aucun mangeur d'en manger, que la bête (trouvée)
  morte, ou le sang qu'on a fait couler, ou la chair de porc - car c'est
  une souillure - ou ce qui, par perversité, a été sacrifié à autre
  qu'Allah'. Quiconque est contraint, sans toutefois abuser ou
  transgresser, ton Seigneur est certes Pardonneur et Miséricordieux~».
  (S. 6. 145), il interprète ce verset dans son sens général et
  considère licite toute autre viande non citée dans ce verset en
  opposition à d'autres savants qui les interdisent à partir d'un
  \emph{ḥadīṯ} rapporté par Bukhârî et Abû Dâwûd, \emph{ḥadīṯ āḥād}.
  Mālik le considère comme non authentique car il s'oppose à un verset
  coranique clair.
\end{itemize}

\vide{le-ux161afiisme}{%
\subsection{Le šafi`isme}\label{le-ux161afiisme}}

Al- Šāfi`ī (150-204 H / 767-819) est né en Palestine de père de
descendance mecquoise. Il a une très grande maîtrise de la langue arabe.
C'est un disciple de l'Imām Mālik~; il a été \emph{qadī} au Yémen. Il
fut accusé de connivence avec les `alides. Il vit 9 ans à La Mecque
avant de revenir à Bagdād. Puis, il part en Égypte où il affirme une
certaine indépendance par rapport aux deux précédentes écoles. C'est lui
qui fonde la science des \emph{uṣūl al-fiqh}.

Du point de vue de sa méthodologie, il excelle dans l'analogie, les
règles de l'abrogeant et de l'abrogé. Il distingue cinq niveaux dans les
sources de la Loi~:

Niveau 1~: le Livre et la Sunna (même niveau)

Niveau 2~: les avis communs des Compagnons (\emph{qawl as- ṣahābī})

Niveau 3.~: les avis divers et multiples des Compagnons sur une même
question.

Niveau 4~: le consensus (\emph{iǧmā`}) si l'on ne trouve ni verset
coranique ni sunna prophétique.

Niveau 5~: l'analogie (\emph{qiyās}).

Vous remarquerez qu'il place les deux sources scripturaires au même
niveau supérieur. C'est un défenseur du \emph{ḥadīṯ} contre ceux qui ne
voulaient reconnaître que le Coran comme source de loi.

\vide{le-ux1e25anbalisme}{%
\subsection{Le ḥanbalisme}\label{le-ux1e25anbalisme}}

Ibn Ḥanbal (164-241H / 780-855) est né à Bagdād. C'est d'abord un
spécialiste du \emph{ḥadīṯ.}

Il a pris part à la querelle du Coran créé ou incréé~contre les
mu`tazilites (les rationalistes de l'islam).

Dans sa méthodologie, il repère cinq sources~:

\begin{itemize}
\item
  les textes scripturaires,
\item
  l'avis des Compagnons, y compris s'il y a divergence entre eux,
\item
  la possibilité d'utiliser le \emph{ḥadīṯ mursal} {[}le compagnon n'est
  pas mentionné dans l'\emph{isnād}{]}
\item
  la possibilité d'user du \emph{ḥadīṯ} faible s'il existe,
\item
  la déduction analogique (\emph{qiyās}).
\end{itemize}

Evidemment, ce recours au \emph{ḥadīṯ} faible peut donner des fatwas
très originales et très ouvertes\ldots{}

A noter que le wahhabisme / salafiste~, comme il n'y avait pas d'école au moment de
  Prophète, ils ne font pas partie d'une école. Néanmoins, l' Arabie Saoudite reconnaissant le hanbalisme, c'est une école qui a du poids pour le wahhabisme.
  
\vide{les-tentatives-dunification}{%
\subsection{{\textbf{Les tentatives
d'unification}}}\label{les-tentatives-dunification}}

Si les écoles juridiques ont pu entrer en concurrence et si l'histoire
ne manque pas d'exemples d'anathémisation (\emph{takfīr}) en raison de
l'appartenance à une école particulière, le politique est aussi
intervenu afin de favoriser une unification. Symptomatique à cet égard
est l'initiative du calife al-Manṣūr (m. 775) qui voulait imposer
\emph{al-Muwatta'} de Mālik, mais celui-ci le lui déconseilla.

Dans l'empire ottoman, le sultan Salim I (1512-1520) imposa l'école
hanafite comme école officielle pour toutes les questions relatives aux
questions juridiques, sauf les questions cultuelles. Mais il faut
attendre le 19\textsuperscript{ème} siècle pour qu'une tentative de
codification voie le jour~avec le code \emph{Maǧallat al-aḥkām
al-`adliyyah}.

L'Arabie saoudite n'a pas de code civil, mais il existe une compilation
de l'enseignement de l'école hanbalite qui est l'école officielle~:
\emph{Maǧallat al-aḥkām al-šari`yya} composée de 2382 articles et
élaboré par le šayḫ Aḥmad al-Qārī (m. 1940).

Dans les états, on trouve aussi des tentatives d'élaborer un code en
puisant dans l'enseignement de différentes écoles tout en s'appuyant
particulièrement sur l'une d'elles. Le code civil des Émirats arabes
unis de 1985 dit~: «~À défaut d'une disposition dans cette loi, le juge
statuera d'après le droit musulman, donnant préférence aux solutions les
plus appropriées de l'école de l'Imam Mālik et de l'école de l'Imam
Aḥmad Ibn Ḥanbal, et à défaut, à celles de l'école de l'Imam al-Šafī`i
et de l'école de l'Imam Abū Hānifa, selon l'intérêt de la question~».

La tentative d'unification se voit plus encore dans le cadre de projets
établis au sein de la Ligue des pays arabes et du Conseil de Coopération
des pays arabes du Golfe.

\vide{v-lapplication-de-la-ux161arux12ba-dans-les-pays-musulmans}{%
\section{{ L'application de la šarī`a dans les pays
musulmans
}}\label{v-lapplication-de-la-ux161arux12ba-dans-les-pays-musulmans}}

Après avoir exposé l'esprit de la šarī`a, sa logique, la question est de
savoir comment, pratiquement, cela se passe dans les pays musulmans.
D'emblée, il faut souligner que sous l'empire ottoman, entre 1869 et
1876, a été promulguée la \emph{Maǧallah} qui codifie les règles de
droit. Il s'agit d'une réponse pratique aux difficultés rencontrées, à
savoir l'absence de personnes compétentes, mais aussi la pluralité des
réponses due à la méthodologie de l'école hanafite, l'évolution des
contextes, l'insuffisance des lois existantes. Mais la grande évolution
vient de l'adoption de codes étrangers ce qui va reléguer le droit
musulman dans l'histoire. À la fin de l'empire ottoman, l'emprunt aux
codes européens est systématique~: la Turquie puise son code pénal dans
le code italien, le code de commerce de l'Allemagne, le code de
procédure civile en Allemagne et en Suisse. Mais il y a une unification
du droit puisque l'abolition des tribunaux religieux musulmans conduit à
l'abolition des dispositions communautaires propres aux non-musulmans en
matière de statut personnel ou familial.

Pour autant, si le droit occidental a largement inspiré le droit
constitutionnel, mais aussi le code civil, le code pénal et le droit
administratif, le droit musulman n'est pas totalement abandonné. Non
seulement il influence les domaines juridiques, mais il sert aussi de
source secondaire pour combler les lacunes. Sami Aldeeb Abu Sahlieh
analysant le droit égyptien conclut~: «~Sur le plan formel le droit
musulman couvre peu de domaines. Mais dans la réalité, il joue un rôle
important dans presque tous les aspects de la vie. Ainsi, il sert de
référence pour déterminer ce qui est licite et ce qui est illicite dans
les domaines de l'éthique sexuelle (mixité entre hommes et femmes,
rapports sexuels hors mariage, etc.) et médicale (avortement,
procréation artificielle, planification familiale, tabagisme, etc.), de
la tenue vestimentaire, des interdits alimentaires, des limites du
sport, des restrictions sur le plan artistique et de la liberté
d'expression, de l'économie (intérêts pour dettes et activités
bancaires, paris et jeux de hasard, assurances, impôt religieux, etc.)
du travail de la femme et de sa participation à la vie politique, de
l'intégrité physique (circoncision masculine et féminine,
etc.~»\sn{~Sami Aldeeb Abu Sahlieh, \emph{Introduction à la
  société musulmane. Fondements, sources et principes}, Paris, Eyrolles,
  2006, p. 315.}.

\vide{ruxe9actions-islamistes}{%
\subsection{Réactions islamistes}\label{ruxe9actions-islamistes}}

Face à cette évolution du droit musulman, les courants islamistes, en se
fondant sur des versets coraniques, rejettent catégoriquement toute
réception du droit étranger. En s'appuyant sur les articles
constitutionnels faisant de l'islam la religion de l'Etat, ils
considèrent que toute loi contraire à l'islam est contraire à la
constitution et que les tribunaux ont le devoir de ne pas les appliquer.
Dans une fatwa demandant si l'on peut comparer le droit musulman avec le
droit positif sans rabaisser le droit musulman, la Commission saoudienne
de la fatwa a répondu qu'il n'y avait rien de mal à opérer une telle
comparaison dès lors qu'elle montrait le caractère complet du droit
musulman.

Concrètement, les islamistes demandent à ce que le code pénal soit celui
du droit musulman et qu'il cesse de correspondre à l'humanisation des
sanctions due aux droits de l'homme imposés par l'Occident. Sur le plan
de la finance, il s'agit d'imposer un mode bancaire islamique. Sur le
plan civil, on trouve l'interdiction du travail des femmes, de la
musique, du cinéma, la démolition des statues, etc. Certains juristes
islamistes vont jusqu'à demander le retour de l'esclavage. Le théologien
pakistanais al-Mawdūdī (m. 1979) est allé dans ce sens.

Au sein même des juges, certains militent pour rejeter les lois
positives issues du monde occidental. À cet égard, le juge Maḥmud `Abd
al-Ḥamīd Ghurab a publié en 1986 un livre intitulé \emph{Aḥkām
islāmiyyah idānatan li-l-qawānīn al-wad`iyya} {[}Jugements islamiques
contredisant les lois positives, Le Caire, Dār al-i`tisām{]} dans lequel
il rapporte 37 jugements qu'il a rendus et qui furent controversés. Il
conclut en exhortant les juges à appliquer le droit musulman.

Dans la perspective des Frères musulmans et notamment d'al-Qarādāwī, la
bataille juridique ne peut se gagner qu'à la condition que les musulmans
vivent leur islam individuellement avec fidélité, ferveur, rigueur.
C'est alors que la société sera contrainte de changer, conformément à un
verset du Coran~ \mn{S.~13, 11~} :


\begin{quote}
    
\TArabe{إِنَّ اللَّهَ لَا يُغَيِّرُ مَا بِقَوْمٍ حَتَّى يُغَيِّرُوا مَا
بِأَنفُسِهِمْ}

«~En vérité, Allah ne modifie point l'état d'un peuple, tant que les
{[}individus qui le composent{]} ne modifient pas ce qui est en
eux-mêmes~».
\end{quote}

\vide{ruxe9actions-libuxe9rales}{%
\subsection{Réactions libérales}\label{ruxe9actions-libuxe9rales}}

Plusieurs juristes libéraux musulmans se sont opposés à la réaction des
islamistes et ont notamment cherché à saper les fondements du droit
musulman. À cet égard, Muḥammad Sa'id al-Ašmawī qui fut président du
tribunal égyptien de la haute sécurité d'Etat, a écrit deux ouvrages
dont un \emph{Uṣūl al-šarī`a}. Il dit la nécessité de distinguer entre
la \emph{šarī`a} et le \emph{fiqh}. La \emph{šarī`a} a été révélée à
Muḥammad tandis que le \emph{fiqh} est un développement des juristes et
une œuvre humaine. Seule la \emph{šarī`a} doit donc être gardée.

À cet égard, il s'inscrit pleinement dans le courant réformiste de la
fin du 19\textsuperscript{ème} siècle du Manār. En effet, en réaction à
un rapport que Lord Cromer, homme d'État britannique, avait rédigé~sur
l'islam, le réformateur Rašīd Riḍā (m. 1935) lui avait écrit cette
lettre~:

\begin{quote}
«~Excellence,

(\ldots) Entendez-vous, par les termes que vous avez employés, dans
votre dernier rapport, sur l'application de la Loi musulmane `qui a été
instituée depuis plus de mille ans', la religion musulmane elle-même qui
n'est autre que le Coran et la Sunna du Prophète~? Ou bien entendez-vous
par là le corps des doctrines élaborées par les juristes~?

Si vous prenez, en effet, ces termes dans la dernière acception, il faut
effectivement voir dans cette Loi une œuvre essentiellement humaine.
Ceux qui l'ont élaborée ont ajouté aux emprunts qu'ils ont faits à la
religion elle-même beaucoup de conceptions toutes personnelles. Aussi
les docteurs ont-ils pu se reprocher leurs erreurs mutuelles et les
hommes d'Etat musulmans ont-ils renoncé à appliquer une grande partie de
ces lois. Les partisans de réformes peuvent, à l'intérieur de chaque
rite, critiquer bon nombre de ces conceptions d'élaboration purement
doctrinale~»\sn{Rašīd Riḍā, \emph{Le Califat}, p. 204-205.}.
\end{quote}

Rašīd Riḍā distingue aussi entre la Loi de Moïse et la Loi de Muḥammad.
Le premier est surnommé le Législateur tandis que le second a pour
mission de transmettre une miséricorde, une morale. Il souligne que la
\emph{šarī`a,} terme qui n'apparaît qu'une fois dans le Coran désigne
non pas une Loi mais une méthode, une voie.

Quant à l'interprétation des versets appelant au jugement de Muḥammad,
il souligne que justement, seul Muḥammad est habilité à juger et non un
homme doué de raison.

Il rappelle qu'aucun verset du Coran ne légitime un régime politique
particulier. Pour lui, le vrai gouvernement musulman est celui choisi et
contrôlé par le peuple en toute liberté, avec le droit de changer sans
que le sang coule et sans être accusé d'athéisme ou de mécréance.

Quant à l'application des peines, il rappelle la nécessité de multiples
conditions~: une communauté de croyants pieux et honorables, une société
caractérisée par la justice sociale, économique, politique~;
l'application des peines ne peut être le fait de gouvernements injustes
ou de tribunaux d'exception sur la base de faux témoignages, ou
d'arbitraires.

\vide{conclusion-revenons-uxe0-al-ux121azux101lux12b}{%
\section{{Conclusion~: Revenons à al-Ġazālī
}}\label{conclusion-revenons-uxe0-al-ux121azux101lux12b}}

Al-Ġazālī \label{theol:AlGazali22} souligne dans \emph{Le Livre de la science} que le terme
\emph{fiqh} désignait dans les premiers temps «~la science de la voie
qui mène à la vie de l'Au-delà, la connaissance détaillée des maladies
de l'âme, de ce qui rend les œuvres corrompues, de la puissance
avilissante de ce bas-monde, de la force d'aspiration des délices du
Paradis et de l'emprise de la peur sur le cœur~». Et de poursuivre~:
«~C'est le \emph{fiqh} entendu en ce sens qui éveille et avertit et non
les définitions de la répudiation (\emph{talāq}), de l'affranchissement
(`\emph{itāq}), du serment d'anathème (\emph{li`ān}), de la vente à
terme (\emph{salam}), du salaire de location (\emph{ijāra})\ldots{} Tout
cela n'avertit pas et ne fait pas naître la crainte pieuse, bien au
contraire, s'en occuper exclusivement durcit le cœur et en fait sortir
la crainte révérencielle de Dieu comme nous pouvons le voir de nos jours
parmi ceux qui en sont devenus les spécialistes »\sn{~Al-Ġazālī,
  \emph{Le livre de la science}, traduit de l'arabe et annoté par Tayeb
  Chouiref, Paris, La Ruche, p. 58-59.}.

Ici, comme on le voit, al-Ġazālī est particulièrement sévère à l'égard
des jurisconsultes qui ont perverti l'intention du \emph{fiqh}. Ils
utilisent les subterfuges du \emph{fiqh} pour se prévaloir de payer ce
qui est dû. Il pose en quelque sorte la question de l'évasion fiscale~:

\begin{quote}
«~On raconte qu'Abū Yūsuf, le juge, donnait ses biens à son épouse, à la
fin de l'année pour éviter de payer l'aumône légale (\emph{zakat}). On
rapporta cela à Abū Ḥanīfa et il répondit~: `cela est dû à sa profonde
connaissance de la jurisprudence'. Abū Ḥanīfa a raison~; cette
jurisprudence est relative à ce bas-monde. Mais agir ainsi est plus
grave pour l'Au-delà que tout crime~; une telle science est le modèle
d'une science néfaste~»\sn{\emph{Kitāb al-`ilm} (cité par M.
  Hogga, p. 160).}.
\end{quote}

Et al-Ġazālī de citer un \emph{ḥadīṯ}~: «~Voulez-vous que je vous dise
qui est réellement le \emph{faqīh} qui possède le fiqh~? -- Oui répondit
l'assistance. Il dit~: C'est celui qui ne fait pas douter les gens de la
miséricorde de Dieu, et qui ne les rassure pas non plus au sujet de la
ruse divine~; celui qui ne les fait pas désespérer de la bonté de Dieu,
et qui ne néglige pas le Coran en désirant autre chose que
Dieu~»\sn{\emph{Le livre de la science}, p. 100.}.

On comprend à la lumière de cette position qu'al-Ġazālī dût subir les
foudres des gens de la Loi. Mais l'on comprend aussi pourquoi les
réformateurs de l'islam s'attachent à cet auteur en qui ils trouvent un
précurseur d'une vision nouvelle du droit musulman devenu hélas une
casuistique étouffante et mortifère.

\vide{annexe-1}{%
\section{Annexe 1}\label{annexe-1}}

\textbf{Le Moratoire de Tariq Ramadan}

\mn{Il faut bien voir que ce châtiment est dans la loi. Tarif Ramadan. Sur la question du moratoire, on
\textit{cesse} d'appliquer les châtiments. Dans la Umma, c'était la seule
position acceptable en dehors d'un islam Européen.}
Je vous propose la lecture de l'Appel international à un moratoire sur
les châtiments corporels, la lapidation et la peine de mort de Tariq
Ramadan. En son temps, cet appel fit couler beaucoup d'encre\ldots{} Le
relire, à la lumière de ce cours, permet de mieux comprendre comment
Ramadan cherche à fonder ce moratoire, d'un point de vue strictement
musulman.


\vide{annexe-2}{%
\section{Annexe 2}\label{annexe-2}}

\textbf{Exemple de fatwa~politique : Est-il licite de se rendre à
Auschwitz~?}

L'histoire est racontée par Farid Abdelkrim. Ce jeune nantais, après
être devenu délinquant, rencontre l'islam puis les Frères musulmans. Il
intègre l'UOIF et devient le Président des Jeunes musulmans de France,
mouvement des Frères musulmans. Il finit par quitter le mouvement et en
explique les raisons dans son livre \emph{Pourquoi j'ai cessé d'être
islamiste.} Parmi ces raisons, on trouve la fatwa concernant l'illicéité
de se rendre à Auschwitz. En effet, en 2003, à l'initiative du curé de
Nazareth, Emile Shoufani, et de Jean Mouttapa, un voyage est organisé à
Auschwitz rassemblant des Français, des Belges, des Israéliens, des
Palestiniens juifs, chrétiens et musulmans afin d'écouter la douleur du
peuple juif. Alors qu'il décide de s'y rendre, l'UOIF le lui déconseille
et l'invite plutôt à se rendre en Palestine pour écouter la souffrance
du peuple palestinien.

Durant le voyage, des responsables des Étudiants musulmans de France (de
l'UOIF) demandent une \emph{fatwa} pour vérifier la licéité de ce
voyage. Dans sa réponse, la fatwa affirme que c'est là une grave erreur.

\subsection{Exemple de fatwa~liée à l'alimentaire~:}

\subsubsection{Peut-on consommer des fraises Tagada~?}

Il s'agit d'une \textbf{fatwa~:} c'est un avis juridique donné par un
spécialiste de la loi islamique. Il est une réponse à une question
particulière. En règle générale, une \emph{fatwa} est émise à la demande
d'un individu pour régler un problème. Le spécialiste qui émet des
\emph{fatwas} est appelé un mufti. Contrairement à l'opinion souvent
répandue, une fatwa n'est pas forcément une condamnation~: elle est
d'abord un avis religieux pouvant porter sur des domaines variés~: les
règles fiscales, les pratiques rituelles, l'alimentation.

Question~: la consommation de bonbons à base de gélatine est-elle
\emph{halāl}~?

Pour les hanafites ou les malékites un élément impur peut devenir pur
après sa transformation. Ainsi, le vinaigre qui résulte de la
transformation du vin n'est pas impur. Mais pour les šafi`ites et les
hanbalites, un élément impur ne devient pas pur après sa transformation.
Cependant, il y a une exception pour les aliments qui se transforment
d'eux-mêmes, sans intervention humaine, comme le vin. Pour certains
savants, la graisse de porc est tellement transformée qu'elle devient
pure. Le savant salafiste contemporain Uthmayn conclut que le bonbon n'a
plus rien à voir avec l'odeur, le goût, l'apparence du porc. On peut
donc manger des fraises Tagada ou des yaourts Taillefine. Mais d'autres
récusent cette position.

La question est donc de savoir à partir de quand la transformation a
perdu son caractère impur~? Pour Uthmayn, il suffit qu'il n'y ait plus
la saveur, l'odeur, la couleur. Pour al-Qaradāwī, il faut une
transformation chimique, pour Usmani il faut une transformation
moléculaire complète.

\subsubsection{Fatwa sur les rites du Sheikh Mustafa al-Zarqa à propos de
Zakāt~al-Fitr}

Nous avons rencontré le nom de cet éminent savant dans l'introduction.
Voici une de ses réponses à une question posée sur la \emph{zakāt}. Il
appartient à l'école hanafite.

Une question a été posée à l'éminent savant hanafite Sheikh Mustafa
al-Zarqa par Sheikh `Abd Al-Rahmân Al-Turqî d'Arabie Saoudite.~
\begin{quote}
    

Quel est le statut de \emph{Zakāt al Fitr} si on l'envoie à nos pays à
cause du besoin des gens là-bas~?\\
Si nous donnons la \emph{Zakāt al Fitr} en graines, selon l'opinion
répandue, nous voyons le vendeur qui nous vend les graines et à côté de
lui les femmes qui prennent ladite \emph{Zakāt}. Nous achetons de lui
les graines puis nous les donnons en tant que \emph{zakāt} aux femmes.
Ainsi, le vendeur garantit le réachat des graines à la moitié du prix
d'achat.~N'est-il pas plus judicieux de donner la \emph{Zakāt al Fitr}
directement en argent liquide conformément à la \emph{maslaha}
(l'intérêt) des pauvres~?\\
Avisez-nous, que Dieu vous avise.

Signature : Sheikh `Abd al-Raḥmān al-Turqī
\end{quote}
Le Sheikh Dr. Mustafa al-Zarqa a répondu en disant :
\begin{quote}
Il n'y a aucune preuve dans les textes législatifs de la Charia qui
invalide la donation de la \emph{Zakāt al Fitr} si elle est donnée
autrement qu'en nourriture, blé, ou autres graines. Les savants de
l'école hanafite considèrent qu'il est préférable de donner la
\emph{Zakāt al Fitr} en monnaie, car le pauvre connait mieux ses besoins
et que si les autres biens comme le blé, l'orge, les dattes, les raisins
secs ont été mentionnés dans le \emph{ḥadīṯ}, c'est en guise d'exemple,
afin de faciliter le donateur qui peut ne pas avoir de monnaie. De plus,
l'intérêt du pauvre, ainsi que celui de ses enfants, peut être d'avoir
de la monnaie car elle peut résoudre son problème. Cela est une opinion
dans l'école de l'imam Ahmad que Dieu l'agrée.

Ainsi, l'obligation de donner la \emph{Zakāt al Fitr} uniquement en
biens précités, a abouti au résultat ridicule mentionné dans la question
et qui est à l'opposé de l'intention législative où le pauvre vend les
biens reçus, parmi les types précités, pour un prix très bas à cause de
son besoin d'argent et non de produits tels le blé, l'orge ou autre.

L'objectif légal de la \emph{Zakāt al Fitr} est de combler le besoin du
pauvre en ce jour de fête et lui est plus connaissant sur ce dont il a
besoin. Il suffit pour le donateur que son acte soit conforme à
{[}l'opinion{]} d'une des écoles admises.

Quant à l'envoi de la \emph{Zakāt al Fitr} à un autre pays, cela n'est
autorisé que si le pays destinataire a plus besoin que le pays de
résidence du responsable.

Ainsi, il est autorisé de donner la \emph{Zaqat al-Fitr} selon la valeur
monétaire d'un de ces biens réels.\\
Et Dieu est plus savant.

écrit par : Mustafa al-Zarqā.
\end{quote}
