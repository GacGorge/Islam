

\vide{introduction}{%
\chapter{Introduction}\label{introduction}}

\marginpar{Emmanuel Pisani}
Ce cours présente les croyances, les pratiques et les différentes
expressions (sunnisme, shi`isme) de l'islam. Pour autant, il n'est pas
un « catéchisme musulman ». Il tient compte des sources musulmanes et
extra-musulmanes, des manuscrits du Coran, des recherches en épigraphie,
de la confrontation des herméneutiques et des lectures historiques au
sein même du monde musulman ainsi que des différents courants
juridiques, dogmatiques et mystiques. Il souligne et introduit à la
pluralité de l'islam.

Compétences à acquérir à l'issue de l'enseignement

- distinguer les principales sources de l'islam

- distinguer les finalités de la loi et la jurisprudence

- acquérir le vocabulaire de base de l'islamologie

- mieux comprendre l'actualité à la lumière de l'histoire des fondations

\paragraph{Sommaire et thèmes}

- le milieu religieux de la péninsule arabe ante-islamique

- définir l'islam : religion, croyance

- la vie du prophète Muḥammad

- le Coran et la Sunna

- la Loi (šarī`a) et ses principes

- les cinq piliers de l'islam

- les articles de la foi musulmane

- le šī`isme

- le soufisme

- l'eschatologie musulmane

\paragraph{Pédagogie et méthodologie}

Chaque cours constitué d'une quinzaine de pages est accompagné de liens
audio et vidéo ainsi que d'images ou d'extraits de textes sources.

Le cours est accompagné de parties « off » signalées en bleu et qui
visent à accompagner le lecteur d'une manière ludique afin de l'aider
dans l'assimilation des concepts, expressions et auteurs propres à
l'islam.

\paragraph{Ouvrages à lire au cours de l'enseignement}

- Quelques sourates du Coran.

- La Sīra d'Ibn Hicham, traduction de Wahib Atallah, Paris, Fayard,
2004.