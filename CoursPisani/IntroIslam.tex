\chapter{Qu'est ce que l'Islam ?}

\section{Calligraphie}

\includegraphics[width=\textwidth]{Images/image011.png}

« Le sens du mot islam est à mettre en relation avec le fait de
rester~''sain et sauf'' et, pour cela, de ne pas avoir d'intention
agressive. Il n'a donc aucun rapport avec le sens de~``soumission''
qu'on prête généralement à ce mot. Le~\emph{muslim}, celui qui fait
montre d'islam, ne fait que se mettre~``sous la sauvegarde'' de Dieu,
car c'est Dieu qui protège et garde en sécurité et en vie. Dans le
Coran, le mot islam est à comparer avec le mot~``imân'', qui a un sens
plus fort car il renvoie au fait d'être fidèle mais aussi d'agir en
conséquence. L'idée de soumission n'émerge que plus tard, lorsque la
société initiale se transforme en empire et que commence à s'imposer le
juridisme du sunnisme, à la fin du IX\textsuperscript{e}~siècle. » Jacqueline Chabbi



« Cette calligraphie est structurée autour d'un cercle, avec un axe au
milieu, comme une colonne vertébrale à l'intérieur d'un être humain. En
haut et en bas, deux courbes, de la même largeur, vont dans des
directions opposées. C'est un mouvement que l'on trouve dans de
nombreuses  oeuvres d'art, chez Michel-Ange comme dans le yin et le
yang\ldots{} Pour moi, il n'y a pas un islam, mais des islams très
différents les uns des autres, selon les époques et les pays. Cette
ouverture est une manière de montrer que l'islam est pluriel. Quant au
bleu, il vient de la mosquée de mon enfance, dans une ville du désert
irakien. Tout y était de couleur ocre, sauf cet édifice recouvert de
céramiques couleur cobalt et turquoise. C'est un bleu qui reste chaud,
comme celui de la mer Méditerranée\ldots{} » Hassan Massoudy


\vide{introduction-1}{%
\section{Introduction}\label{introduction-1}}

Salama~: Islam

Partir de ce que cela veut dire~?

Religion à partir des grands penseurs classiques puis contemporains
(Airn Boubakeur)~: une construction.

Enfin dans une dernière partie, une méthodologie. Comment fait-on pour
entrer dans un autre univers. Penser l'altérité~? Comment fait-on quand
on est musulman quand on veut entrer dans une réalité différente.


\vide{leuxe7on-introductive-aux-fondations-de-lislam}{%
\section{Leçon introductive aux fondations de
l'islam}\label{leuxe7on-introductive-aux-fondations-de-lislam}}
objectifs:
Comment aborder l'islam du point de vue
musulman
Bien voir l'importance de la culture du
désert pour saisir l'islam
L'islam se comprend en référence à la
langue arabe, langue sémitique.
L'islam se décline au
pluriel


\subsection{La Mecque et l'Arabie}

Pascal remarquait que l'islam est né dans 
\begin{quote}
    «~un canton retiré de
l'univers~».
\end{quote} 
Certes, La Mecque est au 6\textsuperscript{e} siècle de
notre ère un village du désert, mais il n'est pas pour autant
déserté\sn{~Ziauddin Sardar, \emph{Histoire de La Mecque. De la
  naissance d'Abraham au XXI\textsuperscript{e} siècle}, Paris, Payot,
  2015.}. Selon l'historiographie musulmane, La Mecque est un lieu de
pèlerinage multiconfessionnel. Il abrite une Ka`ba, un temple sacré, aux
nombreuses statues et icônes. L'historienne Jacqueline Chabbi remarque
dans une étude documentée\sn{ Jacqueline Chabbi, \emph{Le Seigneur
  des tribus. L'islam de Mahomet,} Paris, Noêsis, 1997.} que La Mecque
est répertoriée pour la première fois au IIe siècle par le géographe
Ptolémée dans son ouvrage \emph{Géographie} sous le nom de
Makoraba\sn{Ptolémée, \emph{Géographie}, VI, 7, 31-37.}. Elle y
retrouve la racine \emph{baraka} -- lisez le nom de la ville à l'envers
en ne retenant que les consonnes et en abandonnant le préfixe `ma'
propre aux noms de lieux, ça marche -- ce qui suggère la présence
miraculeuse en ce lieu désertique d'une eau pérenne, signe d'une
bénédiction. Selon Christoph Luxenberg cependant, il faut lire derrière
La Mecque la racine syriaque \emph{makk} désignant une vallée. Ce
rapprochement n'est pas anodin~: s'il atteste, selon l'auteur, du
caractère syro-araméen du lieu {,} il lui permet aussi de
rattacher la naissance de l'islam à la tradition syriaque\sn{~Christoph
  Luxenberg, \emph{The Syro-Aramaic reading of the Koran~: a
  Contribution to the Decoding of the language of the Koran}, éd.
  Schiler, 2007, p. 327.} -- et donc, si l'on veut comprendre l'islam et
le Coran, il va falloir recourir au syriaque, c'est ce qu'il cherche à
montrer dans son livre --.
Plus largement, la perception de La Mecque comme lieu de pèlerinage
pluriconfessionnel permet de rendre compte de l'existence dans le Coran
et la tradition musulmane de thématiques, de récits, de personnages
communs à la tradition judéo-chrétienne. Il apparaît en effet que des
juifs et des chrétiens passaient à La Mecque. S'il est difficile
d'attester de l'existence d'une communauté, des chrétiens semblent y
avoir habité. En tous les cas, la tradition musulmane
l'affirme\sn{~Emmanuel Pisani, «~Waraqa Ibn Nawafal~: un chrétien
  aux origines de l'islam~?~», dans \emph{La Règle herméneutique}, n°36,
  décembre 2014, p. 31-53.} -- cette affirmation n'en fait pas une
vérité historique. Elle peut aussi être véhiculée pour des motifs
apologétiques, nous y reviendrons quand nous aborderons la biographie du
Prophète Muḥammad (qu'on appelle en arabe Sīra) -- . Quel rôle les juifs
et les chrétiens ont-ils pu jouer dans la prédication de l'islam à son
origine~? Quel rôle leur est-il donné dans le Coran~? Autrement dit,
l'islam est-il un abrahamisme, s'inscrivant dans la théologie biblique
ou est-il une religion distincte du judaïsme et du christianisme même
s'il réinvestit certains personnages de la tradition biblique~? On voit
que la connaissance de l'islam, de ses croyances, de ses pratiques, de
sa tradition n'est pas sans conséquence pour la théologie. On ne peut
pas réfléchir en théologie sur le sens de l'islam, sur ce que veut dire
du point de vue chrétien la présence de cette religion et de son
développement, sans aborder ces questions.

Ce cours doit donc servir de base aux théologiens, mais il vise aussi à
donner une connaissance de l'islam de l'intérieur. Il ne s'agit pas pour
autant d'un «~catéchisme musulman~». De la même manière que la foi en
théologie chrétienne est éclairée et épurée par la lumière de la raison,
il s'agira de présenter la foi et les pratiques de l'islam à partir des
penseurs et théologiens musulmans et de leur recours à la raison. En ce
sens, nous montrerons que si les penseurs du wahhabisme -- c'est-à-dire
de l'islam promu en Arabie saoudite -- réduisent le \emph{credo}
musulman à un ensemble de vérités à croire `sans discussion', il s'agit
là d'un affaiblissement considérable de la tradition musulmane et de sa
richesse.

En tenant compte de l'histoire, des sources musulmanes et
extra-musulmanes, des manuscrits du Coran et des différences manuscrites
pouvant exister, des recherches en épigraphie et de leur enseignement
sur l'histoire de la langue arabe, de la confrontation des
herméneutiques et des lectures historiques au sein du monde musulman,
des différences de courants juridiques, dogmatiques, spirituels et
mystiques, nous soulignerons la \emph{diversité}, l'\emph{hétérogénéité}
de l'islam.

L'affirmation selon laquelle l'islam, c'est simple~: il s'agit
d'affirmer que «~Dieu est Un et que Muḥammad est son Prophète~» n'est
peut-être pas si «~simple~». Or, mon «~intime conviction~» est que la
prise de conscience de cette diversité, de cette richesse au sein de
l'islam est aujourd'hui une nécessité incontournable à l'heure où l'on
assiste à son uniformisation, à l'élaboration d'un \emph{credo}
mondialisé, laissant croire qu'il est le seul \emph{credo} autorisé, la
seule définition originelle de ce qu'est l'islam. L'étude des fondations
et des fondements de l'islam ouvre un regard autre, plus que jamais
nécessaire, sachant que les premières victimes d'une vision uniforme de
l'islam sont les musulmans eux-mêmes.


\subsection{Définition musulmane de l'islam
}
\label{duxe9finition-musulmane-de-lislam}

Qu'est-ce que l'islam~? Pour répondre à cette question, il est une
méthodologie particulière suivie par les savants musulmans au cours de
l'histoire que nous nous proposons de suivre. Elle donne d'ores et déjà
un indice sur la manière de procéder pour répondre à une question posée.
Dans un premier temps, on prend soin de partir de la langue arabe, de la
racine du mot et de la constellation sémantique qu'elle désigne. Dans un
deuxième temps, on ouvre le Coran pour voir ce que le Livre sacré dit
lui-même du mot interrogé, puis on ouvre la tradition prophétique,
c'est-à-dire les paroles que Muḥammad a tenues en présence de ses
compagnons depuis le début de la révélation coranique en 610 jusqu'à sa
mort. Enfin, on se réfère à d'éminents penseurs avant de proposer
éventuellement une formulation renouvelée, modernisée. Nous allons donc
suivre leur démarche.

Une première manière de définir l'islam, consiste donc à partir du champ
sémantique de la racine, donc de la langue arabe. Cette langue est
intrinsèquement liée à l'islam. Rašīd Riḍā, penseur égyptien et
réformateur du début du XXème siècle, écrit~:

\begin{quote}
«~Une des grandes réformes religieuses et sociales de l'islam a été de
réaliser l'unité linguistique. Il a fait de l'arabe sa langue, la langue
de toutes les races qui l'avaient adopté. L'islam et la langue arabe se
sont toujours prêté un mutuel appui. Sans l'Islam, la langue arabe se
fût altérée, à l'exemple des autres langues, et comme elle-même l'avait
été auparavant. Sans la langue arabe, l'Islam n'eut pas manqué de voir
les hommes s'éloigner de plus en plus de ses doctrines et de se morceler
lui-même en une infinité de sectes s'excommuniant réciproquement
(\ldots) La langue arabe n'est pas l'apanage des Arabes, fils de Qaḥtān,
mais bien celle de tous les musulmans, des peuples même non arabes --
comme elle est aussi celle des Arabes non musulmans~»\sn{Rašīd
  Riḍā, \emph{Le Califat}, p. 149}.
\end{quote}

\subsubsection{ Le recours à la langue arabe} \mn{Plusieurs dictionnaires sont précieux en la matière. Le
  dictionnaire incontournable pour les arabisants est le \emph{Lisān
  al-`arab} d'Ibn Manẓūr. Il y a 15 volumes édités à Beyrouth. On en
  trouve une version en ligne sur le site
  \href{http://www.baheth.info/}.
  En français, le \emph{Dictionnaire arabe-français} de Kazimirski fait
  référence car il prend en compte l'arabe classique, l'arabe ancien --
  ce que ne fait pas le Larousse qui se limite à l'arabe moderne. Pour
  les textes anciens, le Kazimirski est indispensable.}

En arabe, chaque mot est composé d'une racine, la plupart du temps
structurée par trois consonnes -- nous reviendrons dans la deuxième
leçon sur la spécificité de la langue arabe et de sa dimension
sémitique. Pour l'heure, remarquons que dans le mot
i\textbf{sl}a\textbf{m}, on trouve donc la racine Sa.La.Ma. Quel est son
champ de significations~?

 
\paragraph{Forme I~: al-salāmat (\TArabe{سلامة}) :
absence de défaut, droiture, intégrité, perfection, salut.
}

\emph{Al-salāmat} désigne un état parfait, l'absence de vices ou de
défauts.

\paragraph{ Forme I : \emph{al-salām}} (\textbf{\TArabe{سلام}})~: salut, sécurité,
paix

C'est l'idée de paix absolue, de paix totale, l'assurance de la
sécurité.

Dār al-salām, Demeure de la Paix, est aussi un nom pour désigner le
Paradis. C'est aussi le surnom de Bagdad (\emph{madīnat al-salām}).

On retrouve le mot dans la salutation quotidienne du musulman à un
autre.


\paragraph{ Forme II : taslīm
(\TArabe{تسليم})~}

\emph{Taslīm} est un nom d'action. On y trouve l'idée de conserver
intact, sain et sauf, mais aussi de protéger contre un danger.

Avec la préposition \emph{min}, il signifie sauver quelqu'un, le tirer
d'un danger.

\emph{Taslīm} désigne aussi le salut adressé une personne, le fait de
lui dire «~la paix soit avec toi~». Le \emph{taslīm} est aussi la fin de
la prière musulmane avec l'invocation à droite et à gauche de la
formule~: «~\emph{al-Salāmu ʿalaykum wa-raḥmatu Llāhi wa-barakātuhu~}»~:
idée de demander la paix de Dieu, sa miséricorde, ses
bénédictions.À la fin de la prière, le priant tourne sa tête à
  droite puis à gauche, saluant les anges, et recourant à cette formule.
  Dans certains rites, comme le rite malikite (c'est le rite majoritaire
  dans les pays du Maghreb, on y reviendra), il ne dit que
  \emph{al-Salāmu}. Cette formule est inspirée d'un \emph{ḥadīṯ}
  (prononcer hadiiith), c'est-à-dire une parole de Muḥammad, rapportée
  par Ibn Hajr.

\paragraph{Forme X~: istislām (\TArabe{(استسلام)}:
s'abandonner, se soumettre, se livrer, rendre les armes, obéir sans
opposition.} Le terme renvoie à une obéissance sans condition. Il s'agit de suivre le
chemin droit sans s'en écarter.
Obéissance peut aussi se dire al-ṭā`a~: c'est l'obéissance, la subordination mais ce terme suppose la possibilité, sous certaines conditions, de ne pas obéir.
\sn{ J'obéis à mon patron dans le cadre de mon travail~mais pas en dehors. Le terme recouvre aussi celui d'obéissance sous couvert de résistance intérieure. cf. Bertolt Brecht, Geschichten vom Herrn Keuner~: Massnahmen gegen die Gewalt et la question de l'agent  -- veux-tu me servir -- et lui de ne pas répondre, mais de mettre la couverture le soir, de chasser les mouches, de le servir sept années durant, puis quand il eut crevé de dire~: «~Nein~! ». En islam~: je fais ma prière, par obéissance, mais pas avec le coeur.}



À l'inverse, \emph{al-istislām} est l'obéissance absolue, sans
résistance, sans la moindre contrainte ou le moindre frein
psychologique, spirituel, etc\ldots{} Elle prend deux dimensions~:
\begin{itemize}
    \item l'obéissance à Dieu dans l'acte d'adoration et dans le suivi de sa Loi.
    \item L'obéissance au prophète Muḥammad.
\end{itemize}



\paragraph{ Forme IV~ : islām (\TArabe{إسلام})~:
conserver intact, sain et sauf.
}

L'islam renvoie donc à ce qui sauve, qui est pur, intégral, originel,
préservé d'erreurs ou de scories, ce qui est dans le vrai chemin. Islām
implique l'idée d'obéissance, d'abandon à Dieu. Traduit parfois par
soumission, en référence à sa forme \emph{istislām}, il s'agit d'une
soumission intégrale à Dieu et à Dieu seul. C'est une 
\begin{quote}
    «~remise confiante
de soi à Dieu~»\sn{Mohammed Talbi et Gwendoline Jarczyk,
  \emph{Penseur libre en islam}, Paris, Albin Michel, 2002, p. 158.} ou
encore l'«~adhésion consciente et active à la Paix (\emph{salām}) de
Dieu~». 
\end{quote}Étymologiquement, \emph{islām} renvoie à une attitude religieuse
universelle. Elle est une vertu commune à tous les hommes. Il s'ensuit
que certains \emph{`ulamā'} (\emph{ce sont des savants}) définissent
l'islam non comme une religion mais comme l'attitude spirituelle commune
à toute l'humanité. Dans cette perspective, l'islam se serait mué en
religion, mais il n'aurait pas dû l'être~; il aurait dû rester un rappel
(\emph{ḏikr}) à se souvenir de Dieu, à adhérer à lui avec confiance et
abandon\sn{Soubhi Saleh, \emph{Réponse de l'islam aux défis de
  notre temps}, Beyrouth, Arabelle, 1979, p. 102. Voir aussi Muḥammad
  al-Ġazālī, \emph{Al-ta`assub wa al-tasāmuh bayna al-masīhiyya wa
  al-islām} {[}Fanatisme et tolérance entre le christianisme et
  l'islam{]}, Damas, 2005, p . 77.}.

\vide{le-mot-islux101m-ux625ux633ux644ux627ux645-dans-le-coran}{%
\subsubsection{{Le mot islām \TArabe{ (إسلام)}~: dans le Coran}}\label{le-mot-islux101m-ux625ux633ux644ux627ux645-dans-le-coran}}

Dans le Coran, la racine Sa.La.Ma se retrouve 157 fois. On la trouve
notamment dans le nom \emph{muslim}, celui qui suit l'islam, qui a donné
en français \textbf{«}~musulman~\textbf{»}. En revanche, le mot islam
\TArabe{إِسْلَامِ} lui-même, \emph{masdar} (nom actif) de
4\textsuperscript{ème} forme du verbe \emph{aslama}, le fait de faire
acte d'islam, n'est utilisé que 8 fois.

Nous allons reprendre les versets où le mot apparaît et voir quels sont
les sens qui semblent émerger dans le contexte immédiat de son emploi.
On s'appuiera aussi sur les versets comprenant le verbe \emph{aslama}.
Précisons que cette démarche devrait en toute rigueur être accompagnée
de la lecture des commentaires coraniques traditionnels pour comprendre
comment les musulmans lisent les versets référés et comprennent le mot
\emph{islām}.

\begin{table}[h!]
    \centering
     \footnotesize
   \begin{tabular}{p{0.35\textwidth}p{0.3\textwidth}p{0.28\textwidth}}



S.3, 19~: «~Certes, la religion auprès d'Allah, c'est l'islam. Ceux
auxquels le Livre a été apporté ne se sont disputés, par agressivité
entre eux, qu'après avoir reçu la science. Et quiconque ne croit pas aux
signes d'Allah... alors Allah est prompt à demander compte!~» &
\TArabe{إِنَّ الدِّينَ عِنْدَ اللَّهِ \textbf{الْإِسْلَامُ} وَمَا اخْتَلَفَ
الَّذِينَ أُوتُوا الْكِتَابَ إِلَّا مِنْ بَعْدِ مَا جَاءَهُمُ الْعِلْمُ
بَغْيًا بَيْنَهُمْ وَمَنْ يَكْفُرْ بِآَيَاتِ اللَّهِ فَإِنَّ اللَّهَ
سَرِيعُ الْحِسَابِ} & Inna a\textbf{l}ddeena AAinda All\underline{a}hi
\textbf{alisl\underline{a}mu} wam\underline{a}~ikhtalafa
alla\underline{th}eena ootoo alkit\underline{a}ba ill\underline{a}~min
baAAdi m\underline{a}~j\underline{a}ahumu alAAilmu baghyan baynahum
waman yakfur bi\underline{a}y\underline{a}ti All\underline{a}hi fainna
All\underline{a}ha sareeAAu
al\underline{h}is\underline{a}b\textbf{i}~(⁎) \\


S.3, 20~: S'ils te contredisent, dis leur : "Je me suis entièrement
soumis à Allah, moi et ceux qui m'ont suivi". Et dis à ceux à qui le
Livre a été donné, ainsi qu'aux illettrés : "Avez-vous embrassé l'Islam?
" S'ils embrassent l'Islam, ils seront bien guidés. Mais, s'ils tournent
le dos... Ton devoir n'est que la transmission (du message). Allah, sur
{[}Ses{]} serviteurs est Clairvoyant~». & \TArabe{فَإِنْ حَاجُّوكَ فَقُلْ
أَسْلَمْتُ وَجْهِيَ لِلَّهِ وَمَنِ اتَّبَعَنِ وَقُلْ لِلَّذِينَ أُوتُوا
الْكِتَابَ \textbf{وَالْأُمِّيِّينَ} أَأَسْلَمْتُمْ فَإِنْ أَسْلَمُوا
فَقَدِ اهْتَدَوْا وَإِنْ تَوَلَّوْا فَإِنَّمَا عَلَيْكَ الْبَلَاغُ
وَاللَّهُ بَصِيرٌ بِالْعِبَادِ} & Fain~\underline{ha}jjooka faqul
aslamtu wajhiya lill\underline{a}hi wamani ittabaAAani waqul
lilla\underline{th}eena ootoo alkit\underline{a}ba
wa\textbf{a}lommiyyeena \textbf{aaslamtum} fain \textbf{aslamoo} faqadi
ihtadaw wain tawallaw fainnam\underline{a}~AAalayka
albal\underline{a}ghu wa\textbf{A}ll\underline{a}hu ba\underline{s}eerun
bi\textbf{a}lAAib\underline{a}d\textbf{i} \\
\end{tabular}
\caption{versets comprenant le verbe \emph{aslama}. S 3,19-20}
\end{table}


\begin{table}[h!]
    \centering
    \footnotesize
  \begin{tabular}{p{0.35\textwidth}p{0.3\textwidth}p{0.28\textwidth}}

S. 3, 85~: Et quiconque désire une religion autre que l'Islam, se verra
refuser son choix, et il sera, dans l'au-delà, parmi les perdants~». &
\TArabe{وَمَن يَبْتَغِ غَيْرَ الْإِسْلَامِ دِينًا فَلَن يُقْبَلَ مِنْهُ
وَهُوَ فِي الْآخِرَةِ مِنَ الْخَاسِرِينَ} & Waman yabtaghi ghayra
alisl\underline{a}mi deenan falan yuqbala minhu wahuwa fee
al\underline{a}khirati mina alkh\underline{a}sireen\textbf{a}~( \\
S.5, 3~: «~Et aujourd'hui, j'ai parachevé pour vous votre religion et
j'agrée l'islam comme religion~». & \TArabe{الْيَوْمَ أَكْمَلْتُ لَكُمْ
دِينَكُمْ وَأَتْمَمْتُ عَلَيْكُمْ نِعْمَتِي وَرَضِيتُ لَكُمُ
الْإِسْلَامَ} & AAalaykum niAAmatee wara\underline{d}eetu lakumu
alisl\underline{a}ma deenan (religion) famani i\underline{dt}urra fee
makhma\underline{s}atin ghayra mutaj\underline{a}nifin liithmin fainna
All\underline{a}ha ghafoorun ra\underline{h}eem\textbf{un}~(⁎) \\
6, 125~: «~Et puis, quiconque Allah veut guider, Il lui ouvre la
poitrine à l'Islam. Et quiconque Il veut égarer, Il rend sa poitrine
étroite et gênée, comme s'il s'efforçait de monter au ciel. Ainsi Allah
inflige Sa punition à ceux qui ne croient pas~». & \TArabe{فَمَنْ يُرِدِ
اللَّهُ أَنْ يَهدِيَهُ يَشْرَحْ صَدْرَهُ لِلْإِسْلَامِ وَمَنْ يُرِدْ
أَنْ يُضِلَّهُ يَجْعَلْ صَدْرَهُ ضَيِّقًا حَرَجًا كَأَنَّمَا يَصَّعَّدُ
فِي السَّمَاءِ كَذَلِكَ يَجْعَلُ اللَّهُ الرِّجْسَ عَلَى الَّذِينَ لَا
يُمِنُونَ} & Faman yuridi All\underline{a}hu an yahdiyahu
yashra\underline{h}~\underline{s}adrahu lilisl\underline{a}mi waman
yurid an yu\underline{d}illahu
yajAAal~\underline{s}adrahu~\underline{d}ayyiqan~\underline{h}arajan
kaannam\underline{a}~ya\underline{ss}aAAAAadu fee
a\textbf{l}ssam\underline{a}i ka\underline{tha}lika yajAAalu
All\underline{a}hu a\textbf{l}rrijsa
AAal\underline{a}~alla\underline{th}eena
l\underline{a}~yuminoon\textbf{a}~(⁎) \\
\end{tabular}

\end{table}
\begin{table}[h!]
    \centering
     \footnotesize
 \begin{tabular}{p{0.35\textwidth}p{0.3\textwidth}p{0.28\textwidth}}

9, 74~: «~Ils jurent par Allah qu'ils n'ont pas dit ce qu'ils ont
proféré, alors qu'en vérité ils ont dit la parole de la mécréance et ils
ont été infidèles après leur conversion à l'islam. Ils ont projeté ce
qu'ils n'ont pu accomplir. Mais ils n'ont pas de reproche à faire si ce
n'est qu'Allah - ainsi que Son messager - les a enrichis par Sa grâce.
S'ils se repentaient, ce serait mieux pour eux. Et s'ils tournent le
dos, Allah les châtiera d'un douloureux châtiment, ici-bas et dans
l'au-delà; et ils n'auront sur terre ni allié ni secoureur~». &
\TArabe{يَحْلِفُونَ بِاللَّهِ مَا قَالُوا وَلَقَدْ قَالُوا كَلِمَةَ
الْكُفْرِ وَكَفَرُوا بَعْدَ إِسْلَامِهِمْ وَهَمُّوا بِمَا لَمْ يَنَالُوا
وَمَا نَقَمُوا إِلَّا أَنْ أَغْنَاهُمُ اللَّهُ وَرَسُولُهُ مِنْ فَضْلِهِ
فَإِنْ يَتُوبُوا يَكُ خَيْرًا لَهُمْ وَإِنْ يَتَوَلَّوْا يُعَذِّبْهُمُ
اللَّهُ عَذَابًا أَلِيمًا فِي الدُّنْيَا وَالْآَخِرَةِ وَمَا لَهُمْ فِي
الْأَرْضِ مِنْ وَلِيٍّ وَلَا نَصِيرٍ} & Ya\underline{h}lifoona
bi\textbf{A}ll\underline{a}hi m\underline{a}~q\underline{a}loo walaqad
q\underline{a}loo kalimata alkufri wakafaroo baAAda
isl\underline{a}mihim wahammoo bim\underline{a}~lam yan\underline{a}loo
wam\underline{a}~naqamoo ill\underline{a}~an aghn\underline{a}humu
All\underline{a}hu warasooluhu min fa\underline{d}lihi fain yatooboo
yaku khayran lahum wain yatawallaw yuAAa\underline{thth}ibhumu
All\underline{a}hu AAa\underline{tha}ban aleeman fee
a\textbf{l}dduny\underline{a}~wa\textbf{a}l\underline{a}khirati
wam\underline{a}~lahum fee alar\underline{d}i min waliyyin
wal\underline{a}~na\underline{s}eer\textbf{in}~(⁎) \\
S. 39, 22~: «~Est-ce que celui dont Allah ouvre la poitrine à l'Islam et
qui détient ainsi une lumière venant de Son Seigneur... Malheur donc à
ceux dont les coeurs sont endurcis contre le rappel d'Allah. Ceux-là sont
dans un égarement évident.~» & \TArabe{أَفَمَنْ شَرَحَ اللَّهُ صَدْرَهُ
لِلْإِسْلَامِ فَهُوَ عَلَى نُورٍ مِنْ رَبِّهِ فَوَيْلٌ لِلْقَاسِيَةِ
قُلُوبُهُمْ مِنْ ذِكْرِ اللَّهِ أُولَئِكَ فِي ضَلَالٍ مُبِينٍ} & Afaman
shara\underline{h}a All\underline{a}hu~\underline{s}adrahu
lilisl\underline{a}mi fahuwa AAal\underline{a}~noorin min rabbihi
fawaylun lilq\underline{a}siyati quloobuhum min~\underline{th}ikri
All\underline{a}hi ol\underline{a}ika
fee~\underline{d}al\underline{a}lin mubeen\textbf{in}~(⁎) \\
\end{tabular}

\end{table}
\begin{table}[h!]
    \centering
    \footnotesize
 \begin{tabular}{p{0.35\textwidth}p{0.3\textwidth}p{0.28\textwidth}}

S. 49, 14~:~«~Les Bédouins ont dit: \textbf{`\,`Nous avons la foi'\,'}.
Dis: `\,`Vous n'avez pas encore la foi. Dites plutôt: Nous nous
\textbf{sommes simplement soumis,} car la foi n'a pas encore pénétré
dans vos coeurs. Et si vous obéissez à Allah et à Son messager, Il ne
vous fera rien perdre de vos  oeuvres'\,'. Allah est Pardonneur et
Miséricordieux~». & \TArabe{قَالَتِ الْأَعْرَابُ آَمَنَّا قُلْ لَمْ
تُؤْمِنُوا وَلَكِنْ قُولُوا أَسْلَمْنَا وَلَمَّا يَدْخُلِ الْإِيمَانُ
فِي قُلُوبِكُمْ وَإِنْ تُطِيعُوا اللَّهَ وَرَسُولَهُ لَا يَلِتْكُمْ مِنْ
أَعْمَالِكُمْ شَيْئًا إِنَّ اللَّهَ غَفُورٌ رَحِيمٌ} &
Q\underline{a}lati
alaAAr\underline{a}bu~\textbf{\underline{a}mann\underline{a}}~qul lam
tuminoo wal\underline{a}kin qooloo
\textbf{aslamn\underline{a}}~walamm\underline{a}~yadkhuli
aleem\underline{a}nu fee quloobikum wain tu\underline{t}eeAAoo
All\underline{a}ha warasoolahu l\underline{a}~yalitkum min
aAAm\underline{a}likum shayan inna All\underline{a}ha ghafoorun
ra\underline{h}eem\textbf{un} \\
S. 49, 17~: «~Ils te rappellent leur conversion à l'Islam comme si
c'était une faveur de leur part. Dis: `Ne me rappelez pas votre
conversion à l'Islam comme une faveur. C'est tout au contraire une
faveur dont Allah vous a comblés en vous dirigeant vers la foi, si
toutefois vous êtes véridiques'. & \TArabe{يَمُنُّونَ عَلَيْكَ أَنْ
أَسْلَمُوا قُلْ لَا تَمُنُّوا عَلَيَّ إِسْلَامَكُمْ بَلِ اللَّهُ يَمُنُّ
عَلَيْكُمْ أَنْ هَدَاكُمْ لِلْإِيمَانِ إِنْ كُنْتُمْ صَادِقِينَ} &
Yamunnoona AAalayka an aslamoo qul l\underline{a}~tamunnoo AAalayya
isl\underline{a}makum bali All\underline{a}hu yamunnu AAalaykum an
had\underline{a}kum lileem\underline{a}ni in
kuntum~\underline{sa}diqeen\textbf{a} \\
S. 61, 7~: «~Et qui est plus injuste que celui qui invente un mensonge
contre Allah, alors qu'il est appelé à l'Islam? Et Allah ne guide pas
les gens injustes~». & \TArabe{وَمَنْ أَظْلَمُ مِمَّنِ افْتَرَى عَلَى
اللَّهِ الْكَذِبَ وَهُوَ يُدْعَى إِلَى الْإِسْلَامِ وَاللَّهُ لَا
يَهْدِي الْقَوْمَ الظَّالِمِينَ} & Waman a\emph{\underline{th}}lamu
mimmani iftar\underline{a}~AAal\underline{a}~All\underline{a}hi
alka\underline{th}iba wahuwa
yudAA\underline{a}~il\underline{a}~\textbf{alisl\underline{a}mi}
wa\textbf{A}ll\underline{a}hu l\underline{a}~yahdee alqawma
a\textbf{l}\underline{\emph{thth}a}limeen\textbf{a} \\

\end{tabular}

\end{table}

Un premier constat s'impose, ces références au mot \emph{islām} dans le
Coran ou au verbe \emph{aslama} permettent d'ores et déjà d'en préciser
le sens, ne serait-ce que par jeu d'opposition, d'antinomie ou de
rapprochement lexical.


\begin{itemize}
\item
  Ainsi, une série voit en l'islam un \textbf{appel divin}. Il est
  précisé être une guidance~: S.61, 7.
\item
  D'autres versets associent \emph{islām} à religion~(\emph{dīn}) : 5,
  3~; 3, 19~; 3, 85. L'islam est donné par Dieu dans un processus
  temporel aux arabes. Il est agréé comme religion par Dieu. Notons qu'à
  la lumière de ce verset il n'est pas dit que l'islam se substitue aux
  autres religions. Le parachèvement est celui de l'islam lui-même. Le
  verset 3, 19 semble être exclusiviste~:~«~La religion auprès de Dieu,
  c'est l'islam~», mais tout dépend s'il faut bien comprendre \emph{dīn}
  par religion. Mahmad al-Alūsī (m. 1853) commente les avis en disant
  que le mot islam désigne le principe d'abandon propre aux croyants.
\item
  Un troisième sens émerge dans un jeu d'opposition notamment ici au
  \emph{kufr}, c'est-à-dire à la mécréance, l'infidélité à l'unicité
  divine, et l'appel à la conversion~: S. 9, 74~; 49, 17.
\item
  Enfin, une opposition est établie entre la foi (\emph{imān}) et
  l'\emph{islām}. Il n'y a pas synonymie. Selon le verset indiqué S. 49,
  14~, on peut être musulman sans être croyant. Le contraire est-il
  vrai~? Question très importante pour la théologie musulmane et dans
  ses conséquences pour aujourd'hui par rapport à la question du statut
  de l'autre. Nous y reviendrons.
\end{itemize}

On voit à la lumière de ces versets l'établissement de distinctions,
l'articulation de différents couples sémantiques en rapport avec
l'islam. Ainsi par exemple, l'\emph{islām} est une religion (\emph{dīn})
parmi d'autres~; l'islam est considéré comme un chemin donné par Dieu,
une faveur qui n'est pas donnée à tous. Autre opposition, entre
l'\emph{islām} et le \emph{kufr}. Le Coran souligne l'existence d'hommes
redevenus infidèles après s'être converti à l'islam, il indique aussi la
possibilité du repentir. On retrouve aussi l'idée de vérité et de
justice associée à l'islam.

\vide{din-religion-plus-large-voie-coutume-jugement}{%
\paragraph{Din~: religion~? plus large~: voie, coutume,
jugement}\label{din-religion-plus-large-voie-coutume-jugement}}

Cette étude des mots \emph{islām} et \emph{dīn} a fait l'objet d'une
recherche pour une thèse qui a été soutenue à la fin du mois de
septembre 2016~par Cyrille Moreno, sous la direction d'Éric Geoffroy et
avec dans le jury Moezzi, Denis Gril, Makram Abbès, Jean-Louis Schlegel.
Elle est intitulée~: \emph{Analyse Littérale des termes dîn et islâm
dans le Coran -- Dépassement spirituel du religieux et nouvelles
perspectives exégétiques.}

Elle remet en cause bien des données de l'islamologie classique, à
commencer par la signification étymologique du mot \emph{dīn}. L'auteur
montre qu'il a plusieurs origines~: il dérive selon les versets soit de
\emph{dên} en pehlevi, du concept avestique de daênâ, soit de
l'araméo-hébraïque \emph{dân\textbf{,}} soit de la racine arabe
\emph{dāna}.

Il trouve 15 significations~parmi lesquelles foi, voie, rétribution,
croyance, culte, rituel, tradition, coutume, obéissance, sentence,
jugement, usage. Il est donc réducteur de ne retenir du mot \emph{dīn}
dans le Coran que celui de religion, culte, voire rétribution. Il
remarque que \emph{dīn} n'est pas employé pour désigner une religion ou
une nouvelle religion. Il en conclut que le Coran ne définit pas une
nouvelle religion, mais une nouvelle foi. L'idée que \emph{dīn} renvoie
à une religion est postérieure au Coran. Quant au mot \emph{islām}, il
signifie «~voie spirituelle~». Il relève de l'ordre intérieur,
ésotérique. Il est avant tout spirituel.

\vide{le-recours-uxe0-une-parole-prophuxe9tique}{%
\subsubsection{1.3 Le recours à une parole
prophétique}\label{le-recours-uxe0-une-parole-prophuxe9tique}}

Dans un fameux \emph{\textbf{ḥadīṯ}} ~:

\begin{Def}[ḥadīṯ]

un \emph{ḥadīṯ} (hadith) est une parole
de Muḥammad, l'ensemble des hadiths forme la \emph{Sunna}

\end{Def}

Muslim rapporte un échange entre Muḥammad et Ğibrīl -- l'ange Gabriel.
L'ange interroge le Prophète sur l'\emph{islām}, l'\emph{īmān} et
l'\emph{iḥsān.} Qu'est-ce qu'être musulman~? Qu'est-ce que la foi
musulmane~? Comment devient-on meilleur musulman~? Muḥammad répond et
donne trois définitions fondamentales \sn{Muslim, \emph{Ṣaḥīḥ}, ``Īmān'',
  I, n°5. Voir aussi al-Nawawi, hadith n°2.}. :
\begin{Def}[islām]
«~L'islam (islām), dit-il, consiste à adorer Dieu sans rien Lui
associer, à s'acquitter de la prière prescrite, à verser l'impôt
religieux (zakāt), à jeûner durant le Ramadan~».
\end{Def}

\begin{Def}[īmān]
«~La foi (īmān) consiste à croire en Dieu, à ses Anges, à son Livre, à
sa rencontre dans l'au-delà, à ses prophètes, à la Résurrection~».
\end{Def}

\begin{Def}[iḥsān]

«~La perfection (iḥsān) {[}devenir un parfait musulman{]} consiste à
servir et à adorer Dieu comme s'il était devant vos yeux, car si vous ne
Le voyez pas, Lui vous voit~»
\end{Def}

Dans ce \emph{ḥadīṯ}, l'islam relève de la pratique tandis que la foi
renvoie à l'adhésion à un ensemble de croyances, à un credo
(\emph{`aqīda}). À la lumière de ce \emph{ḥadīṯ}, certains penseurs
musulmans ont soutenu que c'est l'islam qui est premier et non la foi
(\emph{imān})~; l'ordre du questionnement n'est pas anodin~; mais
certains font remarquer qu'il existe un ordre d'intensification de
l'adhésion. Ainsi, on peut pratiquer sans croire. La foi est de l'ordre
de l'intérieur, du coeur, tandis que la pratique relève de l'extérieur~:
celle-ci peut être superficielle et ne pas refléter dans le coeur du
pratiquant ce qu'elle est censée exprimer.

% CULTURE D'ISLAM
 \mn{\url{https://www.franceculture.fr/emissions/cultures-dislam/le-salafisme-en-devenir-0}

    Le wahhabisme : a appauvri l'Islam et la diversité de l'Islam et
    pervertit le salafisme, les pieux ancêtres, les trois premières
    générations de l'Islam.

   8eme siècle critique : question entre l'opinion et l'esprit vs la lettre en
    cas de nouvelles questions

  Ibn tammayyia~: La réfutation des logiciens et Contre Aristote~; pensée profonde même si elle est intransigeante


  Olivier Roy : déculturation car toute culture locale est écrasée ;
    ne reste qu'une Techno-islam, mondialisée, car neutre : des
    tours, des Nikes \ldots{}

   Destruction des mosquées de Mossoul du XIème siècle par DAECH. Ou Abd El Aziz
    qui fonde l'arabie Saoudite et détruit les mausolées des compagnons. Cela est lié à la \textsc{médiation avec Dieu}  : il faut
    empêcher les moyens d'une impiété. 
}
\FloatBarrier
 \paragraph{Ibn Taymiyya},
penseur éminent du treizième siècle soutient que l'on \textsc{peut être musulman
sans être croyant}. Tout le problème sera alors de savoir si l'on peut
être croyant -- avec la profondeur que la foi implique -- sans être
musulman.

Retenez le nom de ce penseur qui reviendra à maintes reprises. Sachez
d'ores et déjà qu'il est une référence incontournable dans les
prédications et enseignements wahhabites. Auteur controversé à
l'enseignement souvent extrémiste (cf. ses \emph{fatwas}), sachez
cependant que la profondeur et la subtilité de son propos ne semblent
pas souvent compris\ldots{} Les grands cheikhs de Médine en caricaturent
la pensée, à commencer par Ibn `Abd al-Wahhab, le fondateur, au
18\textsuperscript{ème} siècle du wahhabisme. Il y a beaucoup de
contresens. Les relever est important, car ils permettent de s'appuyer
sur une autorité reconnue dans le wahhabisme pour dépasser son
enseignement fermé.
 Écoutez ce que dit Abdel Wahhab Meddeb en septembre
  2014 à propos d'Ibn Taymiyya dans l'émission \emph{Culture d'islam}
  consacrée au salafisme sur France culture.
\label{sec:wahhabisme}

Forts de ces définitions, comment les théologiens musulmans ont-ils
défini l'islam~? Comment aujourd'hui est-il défini ?

\vide{quelques-duxe9finitions-de-penseurs-musulmans}{%
\subsection{{Quelques définitions de penseurs musulmans
}}\label{quelques-duxe9finitions-de-penseurs-musulmans}}

\vide{al-ux11furux11fux101nux12b-740-8161339-1413-encyclopuxe9diste-persan}{%
\subsubsection{{Al-Ğurğānī (740-816/1339-1413),
encyclopédiste persan
}}\label{al-ux11furux11fux101nux12b-740-8161339-1413-encyclopuxe9diste-persan}}

Al-Ğurğānī est connu pour être un encyclopédiste. Il a notamment écrit
un livre fameux, le \emph{Kitāb al-Taʿrīfāt} (\TArabe{تعريفات}),
c'est-à-dire \emph{Le Livre des définitions}\sn{~Ouvrage édité par
  G. Flügel, Leipzig, 1845, publié à Constantinople (1837), Cairo (1866,
  etc.), et St Petersburg (1897). Maurice Gloton en a assuré une
  traduction en français~: Al-Jurjânî, \emph{Le Livre des Définitions},
  traduction, introduction et annotations par Maurice Gloton, Préfacé
  par Pierre Lory, Paris, Albouraq, 2005.}. Reprenons la définition
qu'il donne justement de l'islam~\sn{Al-Jurjânî, \emph{Le Livre des Définitions},
  \emph{op. cit}., p. 68.}.:


\begin{Def}[Islam pour Al-Ğurğānī]
«~L'islām est la soumission, la résignation, la préservation intégrale,
la profession d'islam.

C'est la soumission (\emph{ḫudū`}) à ce que le Messager de Dieu a
notifié et la docilité (\emph{inqiyād}) envers lui. C'est, a-t-on dit
dans le Commentaire coranique appelé al-Kaššāf (de Zaramaḫšarī), la
reconnaissance par la langue (\emph{iqrār}) {[}des 5 piliers du culte{]}
sans adhésion (\emph{muwāṭa'a}) du coeur. Quand le coeur est en accord
avec la langue, il s'agit de l'acte de foi (\emph{īmān}). Tel est l'avis
de l'Imām al-Šāfi`ī. Abû Ḥanīfa n'a fait aucune différence entre ces
deux notions~»
\end{Def}


\begin{itemize}
\item
  On retrouve dans sa définition les connotations sémantiques de la
  racine SaLaMa. La préservation intégrale renvoie à l'absence d'erreur,
  la profession renvoie à l'attitude religieuse, à l'adhésion, au
  témoignage.
\item
  Au coeur de l'islam se trouve mentionné le Messager de Dieu. Il n'est
  pas question des messagers, mais du Messager, même si l'islam
  reconnaît les prophètes de la Bible. Il est question d'un attachement,
  d'une docilité à son enseignement. Dans cette perspective, l'islam, ce
  n'est pas seulement suivre le Coran, le Livre donné par Dieu, mais
  aussi suivre les enseignements du Prophète (Messager = Prophète?). On
  voit ici canonisée la Sunna dont nous avons déjà parlé.
\item
  La distinction entre \emph{islām} et \emph{imān}, entre islam et foi,
  est de nouveau mentionnée. L'islam reste du domaine de l'extériorité,
  de l'expression verbale tandis que seule la foi est du côté de
  l'adhésion du coeur, l'adhésion sincère à ce qui est professé.
\item
  Question cependant disputée et al-\textbf{Ğur}ğ\textbf{ānī} ne prend
  pas position en rapportant deux positions opposées, celle de l'Imām
  \textbf{al-Šāfi`ī} et celle d'\textbf{Abū Ḥanīfa}, tous deux
  fondateurs d'une «~école de jurisprudence~» faisant encore aujourd'hui
  autorité. Nous traiterons des «~écoles de jurisprudence~»
  (\emph{fiqh}) dans un chapitre à venir. Retenez déjà deux d'entre
  elles le shafi'isme et le ḥanifisme.
\end{itemize}

Nous allons à présent exposer la définition donnée par deux penseurs
musulmans contemporains. Le premier a joué un rôle notoire dans
l'émergence de l'islam politique, le second a été un des représentants
de la Grande Mosquée de Paris de 1957 à 1982.

\vide{mawdux16bdux12b-le-grand-penseur-de-lislam-du-xxuxe8me-siuxe8cle}{%
\subsubsection{Mawdūdī, le grand penseur de
l'islam du xx\textsuperscript{ème} siècle
}\label{mawdux16bdux12b-le-grand-penseur-de-lislam-du-xxuxe8me-siuxe8cle}}

Dans son livre \emph{Risāla-é-dīniyāt}, le théologien musulman
pakistanais du vingtième siècle, Mawdūdī (1903-1979), considéré par
beaucoup comme le fondateur de l'islamisme contemporain\sn{Avant
  le 20\textsuperscript{ème} siècle, en français, on utilisait le mot
  islamisme pour définir la religion des musulmans. Mais c'est Louis
  Massignon qui a invité à appeler cette religion par le nom utilisé par
  les musulmans eux-mêmes. Depuis les années 80, on appelle islamisme
  l'islam politique.} -- il est le fondateur du \emph{Jamā'at
al-islamī}, parti islamique voulant constituer un État islamique --,
note que l'islam est la seule de toutes les religions qui ne porte pas
le nom de son fondateur\sn{La traduction française de ce livre
  dont l'original est écrit en ourdu~: Abul A'la Maudoudi,
  \emph{Comprendre l'islam,} Islamic Foundation England, Paris, 1973, p.
  15.} (contrairement au christianisme, au bouddhisme, au zoroastrisme,
ou au judaïsme -- de la tribu de Juda). Particularité qui d'emblée est
située dans un signifiant universel~: «~Le mot islam n'est le propre
d'aucune personne, d'aucun peuple ou pays particuliers. Il n'est pas le
produit d'un esprit humain, il ne se limite pas à une communauté
particulière. C'est une religion universelle qui a pour but de susciter
et de cultiver la qualité et l'attitude de l'islam~»\sn{\emph{Ibid}.,
  p. 15.}.

Pour Mawdūdī, entendre l'islam comme une religion au sens traditionnel
du terme est donc une méprise. L'islam revêt une dimension universelle,
totalisante, qui s'inscrit dans un cadre qui dépasse la réalité humaine
et qui n'est comparable à aucune religion jusqu'alors existante. L'islam
est méta-religion, parachèvement de toutes les religions.

Concrètement, cette qualité et attitude de l'islam consiste en la
soumission et l'obéissance absolue à Dieu et à sa Loi.

Mawdūdī remarque que l'univers est soumis à un ordre, des galaxies à
l'électron, tous suivent l'ordre de la matière. De même, l'homme suit
les lois biologiques. Il grandit, il vieillit. Cet ordre qui régit
l'univers est la loi de Dieu. Comme la création obéit à cette loi
cosmique, tout l'univers suit l'islam. Tout est soumis à cet ordre qu'il
s'agisse du soleil ou des animaux. «~Tout dans l'univers est musulman
car tout obéit aux lois qui lui ont été assignées par Dieu~»\sn{\emph{Ibid}.
  p. 17.} si \emph{(maj.)} bien que pour Mawdūdī, la langue de l'athée,
de celui qui professe que Dieu n'existe pas, est musulmane.

Cependant, la réalité humaine n'est pas exclusivement biologique. Doué
de raison, l'homme est doté du libre arbitre~: «~à l'inverse des autres
créatures, il a reçu la liberté de pensée, d'opinion et
d'action~»\sn{\emph{Ibid}., p. 17}. Si l'homme est biologiquement
musulman, il peut choisir de ne pas l'être spirituellement. C'est ce qui
distingue le croyant du non-croyant. «~Celui qui choisit de reconnaître
son créateur, l'accepte pour Maître unique, se soumet scrupuleusement à
Ses commandements, suit la Loi qu'il a révélée à l'homme pour sa vie
individuelle et sociale devient ainsi un parfait musulman~»\sn{\emph{Ibid}.,
  p. 17}. Il est un parfait musulman dans le sens où il y a harmonie
entre ce qu'il est en son être et ce qu'il confesse, adhère et croit.

Pour Mawdūdī, pour être un parfait musulman, il faut d'abord être
croyant. La foi est première. C'est la connaissance de Dieu qui fait que
l'homme se soumet à Lui, Lui obéit, et cherche à acquérir les attributs
qui Lui sont propres~:

«~Sans la foi, personne ne peut être un vrai musulman. C'est un point
essentiel~; ou plutôt c'est le point de départ. Le rapport entre
l'\emph{islām} et l'\emph{imān} est celui d'un arbre avec sa graine. De
même qu'un arbre ne peut croître sans une graine, de même il n'est pas
possible à l'homme qui n'a pas la foi au départ de devenir un musulman.
Cependant, de même qu'on trouve parfois un arbre qui malgré la graine
semée ne pousse pas, et cela pour des quantités de raisons, ou même s'il
pousse, sa croissance est compromise ou retardée, de même on peut
trouver un homme qui a la foi, mais à cause de certaines faiblesses,
peut ne pas devenir un musulman ferme et véritable~»\sn{\textsc{Mawdudi},
  \emph{Connaître l'islam}, \emph{op. cit}., p. 33.}.

Cette distinction entre l'islam, la foi et la perfection de l'islam
reprend la parole prophétique que nous avons rencontrée. Elle témoigne
d'une articulation entre les notions caractéristiques des
traditionnistes (penseurs attachés aux traditions et aux premiers
commentaires coraniques), mais nous avons vu que cette position est
discutée en théologie musulmane.


\paragraph{Le Cheikh Si Hamza Boubakeur}
(1912-1995), Recteur de la Grande Mosquée de Paris de 1957 à 1982. On trouve une notice biographique sur le site de la Grande Mosquée de
Paris.

Il a publié en 1985 un \emph{Traité moderne de théologie islamique}. Dès
les premières pages de ce \emph{Traité}, Si Hamza Boubakeur s'attache à
donner une définition de l'islam. D'emblée, il définit l'islam dans le
cadre propre du langage de la modernité et de la culture française. D'où
un vocabulaire original qui tranche avec les définitions du Coran, de la
Sunna ou des théologiens classiques.

\begin{quote}
    
«~L'islām, révélation divine\textbf{,} est une religion monothéiste de
vérité spirituelle, de lumière intérieure, d'amour, de fraternité
humaine, de justice sociale\textbf{,} ouverte à toutes les races et à
tous les peuples sans aucune distinction, aux hommes et aux femmes de
toutes les contrées et de tous les siècles, quels que soient le degré de
leur savoir et l'importance de leur fortune. Il implique la foi en un
Dieu Unique et Absolu (\emph{Allahu}) et en la mission de Son Envoyé,
notre Seigneur (SAWS) qu'il a choisi pour la transmission de son message
(Coran). Ce message universel et permanent de liberté, d'égalité, de
fraternité, de charité, de paix, de monothéisme sous la forme la plus
pure, exige \emph{a priori} de l'homme sa soumission inconditionnelle à
Dieu et son abandon total à sa volonté. Tel est d'ailleurs le sens
étymologique du mot Islam. Il se résume en peu de mots : il n'y a qu'un
Dieu ! Muḥammad est un Envoyé de Dieu.

Le Coran exclut toute doctrine religieuse ou philosophique basée sur le
polythéisme, la trinité, l'incarnation, la métempsycose, toute
conception d'un Dieu enfanté ou ayant enfanté, dont l'homme serait
l'image (~!?), toute théorie psychométaphysique d'un panthéisme
universel, toute errance philosophique qui ne reconnaît pas à la foi sa
primauté et à la raison la relativité de sa capacité et les limites de
ses dimensions.

En proclamant l'unicité et la transcendance divines, le Coran condamne
toute association à Dieu d'une autre divinité, d'un quelconque
parèdre\sn{Du grec \emph{paredros} (avoir à côté de). Désigne dans
  l'Antiquité une divinité de rang inférieur.}, tout attachement à un
être, à un objet ou à une cause pouvant faire oublier Dieu ou éloigner
de Lui. Une telle condamnation frappe non seulement l'idolâtrie, mais
encore le culte des fausses divinités modernes qui dépouillent l'homme
de la haute signification qui s'attache à sa vie sur terre~: machinisme,
argent, esprit de jouissance, abus de droit sous toutes ses formes,
racisme, scientisme, lesquels risquent d'entraîner peu à peu l'espèce
humaine vers le chaos moral, l'obscurantisme intellectuel, le vice, la
violence et de la faire régresser vers les abîmes de la primitivité. Les
chimères, l'anagogie\sn{Le ravissement de l'âme dans la
  contemplation des choses divines.} et les mystifications sont
contraires à l'enseignement de l'islam qui donne comme fondements à la
vérité, la raison (\emph{`aql}) et la révélation (\emph{waḥy}).

Il peut, à l'analyse de son ensemble, se définir comme un dogme
(\emph{dīn}), une loi (\emph{sharī`a}), une communauté (\emph{umma}) et
une civilisation (\emph{madaniyya})~»\sn{~Si Hamza
  \textsc{Boubakeur}, \emph{Traité moderne de théologie islamique} :
  contenu doctrinal, ramifications, écoles orthodoxes et hétérodoxes,
  soufisme, théologie comparée, concordances et divergences des
  Écritures révélées, avenir de l'islâm dans le monde , Paris,
  Maisonneuve et Larrose, 3\textsuperscript{ème} édition, 2003, p.
  21-22.}.
\end{quote}

\begin{itemize}
\item
  Dans sa définition, l'islam~revêt une dimension universelle.
\item
  Sa définition s'ancre dans le Coran ou le \emph{ḥadīṯ}, mais elle est
  aussi foncièrement ancrée dans un langage contemporain~: plus encore,
  l'auteur fait coller sa définition de l'islam à la modernité et à son
  paradigme sociologique et politique. Il s'agit aussi de montrer la
  compatibilité de l'islam et de la société française. La devise
  républicaine se trouve convoquée dans sa présentation.
\item
  Le propos est ici, comme avec Mawdudi, apologétique. L'islam est
  défini par opposition aux autres religions qui sont en même temps
  vitupérées pour leur irrationalité.
\end{itemize}

Dalil Boubakeur a proposé une définition de l'islam qui reprend très
largement celle de son père et qui se trouve sur le site de la Mosquée
de Paris, mais avec des nuances\sn{\url{http://mosquee-de-paris.net/Conf/Monde/III0110.pdf}consulté
  le 30/06/2014.}. Ainsi, il n'est pas mentionné l'opposition du Coran à
la Trinité ou à l'Incarnation. Le dernier paragraphe sur le culte des
idolâtries modernes est aussi supprimé. Est-ce à dire que Boubakeur fils
envisage la possibilité de soutenir la Trinité et l'Incarnation à partir
du Coran~? Est-ce par souci de dialogue islamo-chrétien et par esprit de
conciliation~?

\vide{approche-muxe9thodologique}{%
\subsection{{Approche méthodologique
}}\label{approche-muxe9thodologique}}

Suivre un cours sur les fondations de l'islam nécessite de prendre
conscience de la nécessité de notre part d'une certaine distanciation
vis-à-vis de tout ce que nous pouvons déjà connaître de l'islam. Il y a
sans doute de nombreuses vérités, mais très probablement sont-elles
aussi mêlées d'\emph{aprioris}, de préjugés ou de stéréotypes. Or,
approfondir l'étude de l'islam revient à dépasser ces idées reçues
souvent stériles et exclusivistes. Comme l'écrit Mohammad Arkoun dans
son \emph{Histoire de l'islam et des musulmans en France}, l'islamologie
a pour fin «~d'affermir chez tous les acteurs d'aujourd'hui une
conscience historique autocritique par-delà les idéologies d'exclusion
réciproques~»\sn{Mohammed Arkoun (dir.), \emph{Histoire de l'Islam
  et des musulmans en France}, Albin Michel, 2006, p. \textsc{xviii}.}.
Le travail entrepris dans ce cours et que vous avez donc commencé à
suivre s'inscrit dans cette perspective.

Du point de vue méthodologique, il convient de se demander comment
approcher la réalité d'une religion, de ses croyances et de ses
pratiques. Les sciences humaines ont élaboré une méthodologie afin de
garantir l'objectivité des résultats et de la présentation. Ce souci
d'objectivité est le nôtre. Dans l'esprit de ce cours -- partir et
s'appuyer sur les auteurs musulmans -- nous voudrions nous mettre à
l'école d'un des plus grands penseurs de l'islam, Abū Ḥāmid al-Ġazālī
(m. 1111). Il a lui-même été amené à mener enquête et scruter les
croyances des non\textbf{-}musulmans afin d'apprendre à les connaître.
Dans son livre \emph{auto}biographique \emph{Al-Munqiḏ min al-ḍalāl},
\emph{(La délivrance de l'erreur),} il développe une méthodologie
originale qui n'est pas sans rappeler la méthodologie des sciences
sociales. En ce sens, il apparaît même comme un précurseur et un modèle
encore à suivre\sn{Emmanuel Pisani, «~Abû Ḥāmid al-Ġazālī. Un
  précurseur musulman de la sociologie des religions~», \emph{Archives
  en Sciences Sociales des Religions}, n°169, janv-mars 2015, p.
  287-305.}.

\begin{itemize}
\item
  Dans l'émission sur le salafisme de septembre 2014 et dont nous avons
  déjà écouté un extrait, Meddeb évoque aussi la figure d'al-Ġazālī \label{theol:AlGazali1}
  qu'il oppose à celle d'Ibn Taymiyya. Il y montre son admiration et la
  grandeur spirituelle de ses écrits. L'ouvrage qu'il mentionne a pour
  nom~: \emph{Iḥyā' `ulūm al-dīn}, \emph{La Revivification des sciences
  de la religion}\sn{La plupart des livres composant cette somme
    ont été traduits en français aux éditions Al-Bouraq. Les traductions
    sont très inégales selon les traducteurs~: elles peuvent contenir
    des contresens (déplorables) ou être de très bonne qualité.}. Nous
  aurons à reparler de cet auteur que l'on compare parfois à Saint
  Thomas d'Aquin. Saint Thomas, son cadet de plus d'un siècle, le
  connaît d'ailleurs, il le cite sous le nom d'Algazel, mais mal, lui
  attribuant des thèses philosophiques qu'al-Ġazālī, en réalité, réfute.
  Ce n'est pas de sa faute. Saint Thomas ne connaissait pas la
  Réfutation, le \emph{Tahafūt al-falāsifa}, et al-Ġazālī a si bien
  présenté les thèses de ses adversaires que Thomas a pensé qu'il y
  adhérait\ldots{} Preuve de la qualité de son exposé\ldots{}
\end{itemize}

Mais quelle est donc cette méthode~?

\vide{lapproche-des-religions-vue-par-al-ux121azux101lux12b-m.1111.}{%
\subsubsection{{L'approche des religions vue par
al-Ġazālī (m.1111).
}}\label{lapproche-des-religions-vue-par-al-ux121azux101lux12b-m.1111.}}

Je vous propose de lire quelques extraits du \emph{Munqiḏ min
al-ḍalāl}\sn{~\textsc{al-Ġazālī}, \emph{Al-Munqiḏ min
  al-ḍalāl}, Erreur et délivrance, trad. Farid Jabre, Beyrouth,
  Commission libanaise pour la traduction des chefs d' oeuvre, 1969.
  {[}Ghazali, \emph{La délivrance de l'erreur}, Paris, Albouraq,
  Boutaleb, 2012{]}.}, ce récit souvent comparé aux \emph{Confessions}
de saint Augustin~:
\begin{quote}
    
Je n'ai jamais cessé, dès ma prime jeunesse, dès avant mes vingt ans
jusqu'à ce jour -- j'en ai plus de cinquante --, de me lancer dans les
profondeurs de cet océan {[}que constituent les religions et les
croyances des hommes (\emph{al-adyān wa-l-milal}){]}\sn{~\textsc{al-Ġazālī},
  \emph{Al-Munqiḏ min al-ḍalāl}, \emph{op.cit.}, (ar. p.10, fr. p.59).}.

Je ne quitte pas un `intérioriste' sans désirer connaître sa doctrine,
ou un `extérioriste', sans chercher à vouloir savoir ce qu'est la
sienne. Je tiens à connaître la réalité de la pensée du `philosophe'. Du
\emph{kalām}, je tâche de comprendre à quoi mènent la `scholastique' et
sa dialectique. Je veux pénétrer (\emph{aḥriṣ `alā}) le secret
(\emph{sirr}) du `mystique' (\emph{sūfī})\sn{\emph{Ibid.}, (ar.
  p.10, fr. p.60).}.

Un de mes amis, qui est devenu des leurs, m'a rapporté ce propos. Il me
dit que la secte en question se moque de ses détracteurs et prétend
qu'ils n'ont rien compris à sa position. C'est alors qu'il m'exposa leur
thèse. Je l'ai reprise, à mon tour, pour ne pas être taxé d'ignorance,
et je l'ai clairement exposée, pour qu'on ne puisse m'accuser de n'y
avoir rien compris\sn{\textsc{al-Ġazālī}, \emph{Al-Munqiḏ
  min al-ḍalāl}, \emph{op.cit.}, (ar. p.29, fr. p.87).}.

Il incombe de rechercher la vérité comme une finalité à atteindre,
qu'elle se manifeste grâce à soi-même ou à celui qui vous assiste. Il
importe de le considérer comme son partenaire (\emph{rafīqahu}) et non
comme un adversaire (\emph{lā ḫaṣman}), en sachant le remercier
(\emph{yaškuruhu}) quand il indique l'erreur ou lorsqu'il met en lumière
la vérité\sn{\textsc{al-Ġazālī}, \emph{Iḥyā'},
  \emph{op.cit.}, K.1 (\emph{Kitāb al-`ilm}), B.4, b.1, 6, p.57 {[}V.1,
  p.164{]}.}.

Je ne me suis jamais mesuré (\emph{mā nāẓartu}) à quelqu'un en
souhaitant qu'il se fourvoie~; je n'ai jamais discouru avec une personne
sans désirer qu'elle réussisse à définir {[}son propos{]} et qu'elle
jouisse de l'aide et de la protection de Dieu. Quand je m'entretiens
avec quelqu'un, j'ai toujours le souci que Dieu manifeste la vérité soit
par ma bouche soit par la sienne\sn{\textsc{al-Ġazālī},
  \emph{Iḥyā'}, \emph{op.cit}., K.1 (\emph{Kitāb al-`ilm}), B.2, b.2,
  p.37 {[}V.1, p.99{]}.}.
\end{quote}

\begin{itemize}
\item
  Ces citations nous livrent une approche du dialogue à la fois
  sociologique, politique et philosophique. Sociologique, car il s'agit
  de connaître et cette connaissance implique une attitude objective et
  la définition d'une méthode multidimensionnelle. Il ne s'agit pas
  seulement de décrire, mais il s'agit de s'assurer de la compréhension,
  de percer la réalité de l'intérieur et pas uniquement de l'extérieur.
\item
  Politique et philosophique car al-Ġazālī reprend l'approche socratique
  des Dialogues. Si nous avons des convictions (adhésion à des dogmes
  définissant l'identité spécifique d'une communauté), il s'agit
  d'aborder l'autre sans le dogmatisme d'attitude.
\end{itemize}

Une démarche éminemment moderne. Relisez ce qu'écrit le philosophe
Jean-Marc Ferry à cet égard\sn{~Jean-Marc Ferry note que~:
  «~Personne n'est obligé de confronter sérieusement ses convictions
  profondes avec celles des autres. Mais, si dans le dialogue, on
  discute vraiment, alors cela n'a pas de sens de poser l'autre comme
  étant constitutivement dans l'ignorance, dans l'erreur. Par cette
  manière d'enténébrer autrui, on s'exclut du schéma symétrique qui
  convient à une discussion véritable. La discussion suppose une
  symétrie entre les personnes, que ce soit dans le dialogue entre les
  religions ou dans le dialogue entre religions et pouvoirs publics,
  entre religions et monde séculier. Là, on doit poser une égalité de
  principe. L'attitude contractuelle est requise~» in Jean-Marc Ferry,
  \emph{Les lumières de la religion}, Paris, Bayard, 2014, p. 121.}.

\vide{louverture-du-regard-voir-avec-ses-deux-yeux}{%
\subsubsection{{L'ouverture du regard~: voir avec ses
deux yeux
}}\label{louverture-du-regard-voir-avec-ses-deux-yeux}}

C'est une exigence méthodologique. L'idéologue est celui qui ne voit que
d'un seul  oeil et si son regard peut être profond, son angle de vue est
étroit. Je voudrais l'illustrer en m'appuyant sur une histoire extraite
de la tradition prophétique musulmane, et plus particulièrement d'un
«~dit~» de Jésus.

Comme nous le verrons, Jésus est considéré comme un prophète de l'islam,
mais paradoxalement, il parle peu dans le Coran. En revanche, la
tradition prophétique lui prête un nombre important de paroles, de
\emph{logia}\sn{On pense à l'évangile apocryphe de Thomas. Nombre
  de dits de Jésus selon la tradition musulmane sont issus de cet
  évangile.}. Ces paroles se trouvent dans les \emph{Histoires des
prophètes}, les enseignements soufis, les  oeuvres de belles-lettres
(\emph{adab}) ou les anthologies de sagesse. Au siècle dernier,
l'orientaliste Miguel Asin y Palacios a relevé l'ensemble de ces dits,
près de trois cents. Plus récemment, Tarif Khalidî en a proposé une
nouvelle recension\sn{Tarif \textsc{Khalidi}, \emph{L'Evangile
  musulman}, Paris, Seuil, 2003, 272 pages.}.

On raconte donc qu'un jour Jésus était accompagné de ses disciples
lorsqu'ils passèrent près d'un chien mort\sn{Nous reprenons cet
  exemple du Père Michel Lagarde lors de son discours prononcé à
  l'UNESCO au cours de la remise du Prix Sharjah le 29 septembre 2005,
  dans \emph{Islamochristina,} PISAI, n° 31, 2005, pp. 210-211. Je
  m'inspire largement de son beau commentaire.}. Les disciples, frappés
par son odeur nauséabonde, s'exclamèrent~: «~comme cette charogne~pue
!~» et Jésus dit~: «~Comme est parfaite la blancheur de ses dents~». Ce
récit, certes simple, presque banal, n'en a pas moins une fonction
remarquable~: nous initier à l'art de voir autre chose et autrement.
Voir autrement, voir autre chose de l'islam que ce que l'on nous montre,
que ce que l'on entend.

Il va de soi que suivre un cours intitulé \emph{Connaissance de l'islam}
s'inscrit pleinement dans cet esprit~: voir et comprendre ce que l'on ne
connaissait pas, ce que l'on ne voyait pas. Pour filer la métaphore, on
ne se limitera pas à la mauvaise odeur du fanatisme islamique, de
l'intégrisme et de l'intolérance de Daesh ou d'al-Qaïda. Comme Jésus,
dans cette histoire, nous voulons prendre le parti de remarquer aussi
l'éclat, la blancheur, la beauté et découvrir la richesse des auteurs
arabo-musulmans, des «~bouches d'or~» qui, dans le monde musulman aussi,
n'ont pas manqué.

On peut dire que les sirènes de l'intégrisme ne sont pas que du côté de
l'islam. On voit même certains laïcs brandir aujourd'hui le Coran et
citer les versets dits «~\underline{du sabre}~» \sn{« Après que les mois sacrés expirent, tuez les associateurs où que vous les trouviez. Capturez-les, assiégez-les et guettez-les dans toute embuscade. Si ensuite ils se repentent, accomplissent la Salat et acquittent la Zakat, alors laissez-leur la voie libre, car Allah est Pardonneur et Miséricordieux. »
— Le Coran (trad. Muhammad Hamidullah), « Le repentir (At-Tawbah) », IX, 5.} avec la même violence et
la même virulence que les «~fous d'Allah~». Cette pseudo-islamologie est
tout aussi dangereuse que le fondamentalisme islamique car elle est un
fondamentalisme entretenant le même rapport au texte coranique. Certes,
elle n'a pas conduit à la mort d'innocents, mais elle peut fort bien
faire le lit d'une violence mortifère.

\vide{la-connaissance-comme-principe-de-la-charituxe9}{%
\subsubsection{La connaissance comme principe de la
charité}\label{la-connaissance-comme-principe-de-la-charituxe9}}

Enfin, il importe de s'interroger sur la finalité de la connaissance.
Pourquoi connaître l'autre~? («~désirer ou vouloir connaître~» serait
plus modeste) Pourquoi connaître sa religion~? Les motivations à suivre
un cours sur l'islam sont sans doute nombreuses. Mais puisque ce cours
s'inscrit au sein de la faculté du \emph{Theologicum} de l'ICP, il me
semble intéressant de rappeler, pour ma part, la tradition augustinienne
où la fin de la connaissance n'est pas la connaissance elle-même~; on ne
cherche pas à connaître pour connaître, mais parce que de cette
connaissance découle l'amour (et en même temps, c'est de l'amour que
découle la connaissance\ldots), principe vital, principe d'ouverture,
d'accueil, d'hospitalité. On ne peut aimer que celui que l'on connaît
vraiment \emph{(et on ne peut connaître l'autre sans d'abord l'aimer)}.
Pour autant, cette démarche d'amitié n'est pas sans épreuves.

On raconte que Sirâj al-Dîn al-Shiblî, un mystique indien du
14\textsuperscript{ème} siècle, dut être hospitalisé au Caire, près de
la Mosquée d'Ibn Tûlun. Dès qu'ils en furent informés, ses amis vinrent
à son chevet. Il leur demanda~: «~Qui êtes-vous~?~». Nous sommes tes
amis lui répondirent-ils. Alors, il se mit à leur jeter des pierres. Et
eux, prenant leurs jambes à leur cou, s'enfuirent à toute hâte. -- «~Si
vous étiez mes amis, leur cria al-Shiblî, vous ne vous enfuiriez pas
pour si peu~!~

De cette jolie boutade, note Michel Lagarde, se dégage une nécessité,
celle de durer en amitié. Toute relation est une épreuve, la relation
islamo-chrétienne ou judéo-musulmane n'échappe pas à cette règle. Pour
réussir, la relation doit assumer le caractère parfois éprouvant et
décapant de l'amitié. Dans le dialogue islamo-chrétien, dans le dialogue
entre la république et l'islam, les épreuves ne manquent pas, mais notre
connaissance de l'islam en faisant naître un lien de charité, un lien de
fraternité, doit nous aider à dépasser ces épreuves et les peurs
qu'elles suscitent. On raconte qu'un bédouin vivait attaché à un arbre,
loin des gens et du monde. Alors qu'on l'interrogeait, il disait de son
arbre~: «~C'est un compagnon qui a trois qualités~: s'il m'entend, il ne
me calomnie pas, si je crache sur lui, il me supporte et si je
l'escalade, il ne s'emporte pas contre moi~»\sn{\textsc{al-Ġazālī},
  \emph{Iḥyā' `ulūm al-dīn}, K.16, B.2, fa.4, p. 678 {[}V.4, p. 292{]}.
  Vous avez remarqué que cette histoire vient justement du livre
  d'al-Ġazālī cité par Meddeb~!}. Ce récit de la tradition musulmane est
une actualisation de ce thème. Connaître pour pouvoir devenir l'arbre de
l'autre. Récit qui a aussi le mérite d'appartenir au fond même de la
tradition musulmane.

\vide{annexe-limmersion}{%
\section{Annexe~: l'immersion}\label{annexe-limmersion}}

La connaissance de l'islam suppose une immersion culturelle et
religieuse. Elle ne peut être que le fait d'efforts patients et
courageux. Il faut savoir s'immerger pour quitter son monde, ses
références, ses jugements \emph{a priori}. Il est vrai que~l'on retrouve
dans l'islam un grand nombre des prophètes mentionnés dans la Bible. Il
est vrai que l'arabe\textbf{,} comme l'hébreu, comme l'araméen\textbf{,}
est une langue sémitique. Mais il y a plus, il y a dans l'islam quelque
chose qui sent le désert et qu'aucun occidental ne peut comprendre s'il
ne fait pas l'effort de s'immerger. Dans le désert, il faut s'adapter.
Un européen qui y garderait les manières d'agir et d'être de l'Europe,
serait certain de mourir. Et si en Europe, tous les chemins mènent à
Rome\ldots{} dans le désert un seul chemin mène à La Mecque~!

Le grand orientaliste français Louis Massignon dans un beau texte
indiquait la profonde dissemblance culturelle et religieuse entre le
monde musulman et le monde chrétien\sn{Louis \textsc{Massignon},
  \emph{Opera Minora}, Textes recueillis, classé et présentés avec une
  bibliographie par Y. Moubarac, sous le Patronage du Centre d'Etudes
  Dar El-Salem, Tome 1, Dar al-Maaref, Liban, 1963, pp. 13-14.}. Je vous
invite à lire ce texte pour saisir le hiatus culturel qui peut exister.
\begin{quote}
 «~Il suffit pour cela de visiter «~les deux oasis que ce désert recèle
pour le repos des yeux et la paix de l'âme~: un jardin et une mosquée.
Les deux idées sont d'ailleurs, en arabe, étroitement associées~: à
Médine, la mosquée où le Prophète est enterré s'appelle la \emph{rawda},
c'est-à-dire le jardin. Qu'il s'agisse du jardin de Généralife à
Grenade, de l'Aguedal à Marrakech, des jardins du Caire, de Damas, de
Bagdad ou d'Ispahan, la conception musulmane du jardin nous frappe par
sa constance, c'est essentiellement un lieu de rêverie qui transporte
hors du monde. Même s'il contient les mêmes arbres et les mêmes fleurs
que les nôtres, ce type de jardin fait un tel contraste avec les jardins
d'Occident. Dans notre jardin classique qui commence avec l'Empire
romain, continue avec les Médicis et Louis XIV, le but est de dominer le
monde d'un point de vue central~: de grandes perspectives conduisent à
l'horizon, de grands bassins d'eau reflètent les lointains, encadrés par
des arbres taillés impeccablement, conduisant l' oeil, petit à petit, à la
conquête de tout le pays environnant. Au lieu de cela, dans le jardin
musulman, la première chose qui importe, c'est une fermeture isolant du
dehors, et, au lieu que l'intérêt soit à la périphérie, il siège au
centre. Ce jardin se fait en prenant un morceau de terrain, en
«~vivifiant~» un carré de désert où l'eau est amenée~; au-dedans d'un
mur d'enceinte très haut, au-dessus duquel la curiosité ne peut plus
passer à l'intérieur, nous trouvons des quinconces d'arbres et de fleurs
qui se pressent de plus en plus à mesure que l'on va de la périphérie
jusqu'au centre et, au centre, se trouve, auprès d'une fontaine
jaillissante, le kiosque. Ce jardin, à l'inverse du jardin classique et
du jardin paysager des Japonais, procure un délassement de la pensée
repliée sur elle-même~;   
\end{quote}

\begin{quote}
    

«~La mosquée, qui est le lieu du culte de la communauté musulmane, nous
frappe également par sa différence d'avec le lieu de culte chrétien,
l'église, même lorsque ses constructeurs ont emprunté à l'église les
matériaux taillés et les motifs décoratifs. La mosquée a commencé par
être à ciel ouvert, et contient généralement une grande cour centrale,
mais les murs extérieurs de sa clôture sont opaques sans ces échappées
digérant la lumière que sont les vitraux des cathédrales et l'on n'y
entre qu'après avoir passé par le bassin des ablutions rituelles. Il y a
bien une chaire de vérité, mais ce meuble mobile est relégué à une place
secondaire, car les croyants massés en rangs parallèles, comme des
soldats, doivent, durant la prière, garder tous les yeux concentrés vers
une seule direction, celle d'une petite niche axiale vide, la qibla, qui
repère la direction de la Mekke, lieu de sacrifice annuel de
propitiation voué au Dieu d'Abraham. Si les portes de bois sont souvent
ornées, et les claveaux des voûtes alternativement sombres et clairs, la
nef reste nue et dépouillée, sans statues faisant des beautés d'ici-bas
l'intermédiaire qui élève l'âme de l'adorateur jusqu'à son Dieu unique,
car la représentation de la figure humaine est proscrite~; seules, des
inscriptions arabes sur les parois, commémorant, de façon rigide et
solennelle, la Loi. Enfin, à la place du clocher, c'est le minaret qui
s'élève où surgit la voix humaine elle-même, remplaçant le bronze
inanimé, pour le quintuple appel quotidien à la prière canonique. Tout
l'intérêt est concentré volontairement, dans ce style dépouillé, vers la
qibla, symbole de l'orientation du coeur vers l'Unique~» (p. 13-14).
\end{quote}
\begin{quote}
    
«~Dans d'autres édifices que la mosquée, dans l'architecture civile des
palais et des maisons privées, le décor des murs est moins strictement
contrôlé, mais les mêmes principes de style dominent~: le décor
artistique musulman ne cherche pas à imiter le Créateur dans ses  oeuvres
par le relief et le volume des formes, mais l'évoque, par son absence
même, dans une présentation fragile, inachevée, périssable comme un
voile, qui souligne simplement, avec une résignation sereine le passage
fugitif de ce qui périt et tout est périssable `\,`excepté son
visage'\,'~» (p. 14).
\end{quote}

\vide{questions-analyses-approfondissement}{%
\subsection{Questions \textasciitilde{} Analyses \textasciitilde{}
Approfondissement}\label{questions-analyses-approfondissement}}

\begin{itemize}
\item
  Retrouver dans l'historiographie musulmane les descriptions du temple
  de la Ka'ba et sa dimension multiconfessionnelle.
\item
  L'interprétation de La Mecque \emph{Makoraba} comme lieu de
  bénédiction (\emph{baraka}) a été rejetée par Patricia Crone. Il
  s'agit d'une orientaliste américaine de grande qualité. Elle est
  décédée en juillet 2015. Retrouver la thèse et l'argument.
\item
  Chercher parmi des auteurs classiques ou contemporains musulmans
  d'autres définitions de l'islam. Comparer avec le Coran, la Sunna, le
  sens de la racine SaLaMa et les définitions exposées dans le cours.
  Analyser.
\end{itemize}

\vide{ibn-taymiyya}{%
\section{Ibn Taymiyya~}\label{ibn-taymiyya}}

\vide{validation-les-fondations-de-lislam}{%
\section{Validation -- Les fondations de
l'islam}\label{validation-les-fondations-de-lislam}}

\textbf{Le but} de la validation d'un cours universitaire n'est pas de
bachoter, mais de s'approprier la matière, de l'intégrer, de
l'assimiler, d'approfondir.

Pour les fondations de l'islam, il me semble opportun de se «~jeter à
l'eau~» en s'ouvrant à la lecture d'un auteur classique. Aussi, au-delà
de la riche bibliographie en travaux d'orientalistes et d'islamologues
de renommée, la validation ne portera pas sur leurs travaux, mais
\textbf{sur un ouvrage ou le chapitre d'un ouvrage} d'un auteur musulman
classique.

Quelques noms qui vous deviendront familiers au cours du semestre et
quelques exemples d'ouvrages d'al-Tirmidhī, al-Ġazālī, Ibn Taymiyya, Ibn
`Arabī\ldots{} Ces auteurs sont cités durant le cours. Il s'agit de
résumer leur propos et de situer l'ouvrage par rapport aux différents
courants de l'islam, d'identifier le statut du texte, la méthode de
l'auteur.

\textbf{La fiche de lecture} situera l'auteur, l'ouvrage (ou son
chapitre au sein de l'ouvrage). Elle rendra compte des motivations de
l'écriture de l'ouvrage, avec qui il entre en dialogue, à qui il
répond\ldots{} Elle prendra soin de veiller à préciser le contexte
d'écriture. Enfin, l'étudiant est invité à faire part de l'intérêt qu'a
pu constituer pour lui la lecture de cet ouvrage, en quoi il peut
renouveler sa réflexion et son appréciation de l'islam.

\textbf{Suggestion d'ouvrages d'auteurs musulmans pour la validation.}
Vous pouvez choisir d'autres ouvrages, d'autres auteurs. Veillez dans ce
cas à «~valider~» votre choix avec l'enseignant du cours.

\vide{question-de-la-culture}{%
\subsubsection{Question de la culture}\label{question-de-la-culture}}

\textbf{1.~Étude d'une sourate et de ses commentaires}

Ouvrage de Tabari, Ibn Kathīr, etc.

\textbf{2. Étude d'un ouvrage d'al-Ġazālī tiré de l'Iḥyā'}



Abou Hamed AL-GHAZALI



\textbf{4. Quelques mystiques}

-~Al-Tirmidhi, \emph{Le Livre des nuances}. Ou de l'impossibilité de la
synonymie de Al-Hakim Al-Tirmidhi, traduit et commenté par G. Gobillot,
Paris, Geuthner, 2006.

-~Sur les femmes soufies, lire Sulamī, \emph{Femmes soufies}, texte du
13\textsuperscript{ème} siècle, traduit en français aux éditions
entrelacs, Collection Hikma, 2011.

-~Les textes d'Ibn `ArabĪ~: ils sont difficiles, mais très profonds. On
pourra rendre compte du livre \emph{La sagesse des prophètes}, Paris,
Coll. Spiritualités vivantes, Paris, Albin Michel, 2010 (en poche) ou
\emph{La profession de foi} traduit de l'arabe et introduit par Roger
Deladrière, Coll. Babel, Paris, Actes Sud, 2010.

\textbf{5. Pour ceux qui ont la fibre historique, sociologique,
économique ou politique}

-~Je suggère de se plonger dans un ou plusieurs chapitres à défaut du
livre entier d'Ibn Khaldun, \emph{Muqqadima}, Discours sur l'histoire
universelle, Paris, Sinbad, 3\textsuperscript{ème} édition, 1997.

-~Tabari, \emph{Histoire des prophètes et des rois}, trad. Zotenberg,
Paris Sindbad.



