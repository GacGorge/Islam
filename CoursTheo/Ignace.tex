\chapter{Saint Ignace de Loyola}


\begin{quote}
    J'aimais d'abord l'étonnante continuité de sa vie, qui n'est pas perceptible au premier regard. On a coutume de parler de la conversion d'Ignace après que le canon de Pampelune lui eut fracassé la jambe,. [...] Mais les signes de Dieu dans sa vie, on peut les discerner après, à Manrèse surtout, ou avant, dans l'abcès au nez qui humilia le jeune page d'Arevalo. La \textit{vconversion} en tant qu'événement, est une invention des hommes. Aussi bien rejette-t-elle dans un oublir réprobabteur la vie passée. Celle d'Ignace n'avait pas été méprisable.[...] Ignace, une sorte de reître, sans doute, mais héroique et plus adonné à la gloire, même vaine, qu'au confort. Ainsipar bien des côtés, l'appel de Dieu n'a pas contredit cette nature, mais l'a poussée, en la puifiant, à son point d'aboutissement. \cite[p. 164]{Sureau:inigo}  
\end{quote}