\part{Sacramentaire}



\chapter{Introduction}
%---------------------------------------------------------------------------------------------------------------

\mn{LM Chauvet}

\hypertarget{pour-se-repuxe9rer-dans-le-vaste-monde-de-la-sacramentalituxe9-lorganisme-sacramentel}{%
\section{Pour se repérer dans le vaste monde de la sacramentalité~:
l'organisme
sacramentel}\label{pour-se-repuxe9rer-dans-le-vaste-monde-de-la-sacramentalituxe9-lorganisme-sacramentel}}

\hypertarget{sacrement-un-concept-analogique}{%
\paragraph{«~Sacrement~»~: un concept
analogique}\label{sacrement-un-concept-analogique}}

\hypertarget{lorganisme-sacramentel-analyse-systuxe9mique}{%
\paragraph{l'organisme sacramentel~: analyse
systémique}\label{lorganisme-sacramentel-analyse-systuxe9mique}}

\hypertarget{une-duxe9finition-des-sacrements}{%
\paragraph{\texorpdfstring{Une définition des
sacrements}{ Une définition des sacrements}}\label{une-duxe9finition-des-sacrements}}
 

%---------------------------------------------------------------------------------------------------------------
\section{Ouverture sur la problématique
d'ensemble} 

la pointe des sacrements~: la communication de Dieu, la grâce.

«~\textbf{pour la Gloire de Dieu et le salut du monde~»}

dimension latreutique dimension sanctificatrice

Nos liturgies exercent de multiples fonctions~: cognitives, philo,
économiques~!, sociales\ldots{} quei ne doivent pas masquer
l'essentiel~: \textbf{la communication de Dieu.}

\begin{Ex}
 «~le Corps du Christ~: Amen~»

\begin{itemize}
\item
  on ne peut pas faire plus bref
\item
  de la foi
\item
  ie~: les mots «~nus
\end{itemize}

Cette nudité des mots qui se visibilise par la nudité de la main tendue,
\textbf{nue.}
\end{Ex}


Le salut de Dieu , dit dans la Parole, vient se déposer sur le corps.

Pb~: Comprendre ce que l'on célèbre~: «~Lex orendi, lex credendi~».

\paragraph{3 modèles de compréhension~:}

\hypertarget{moduxe8le-objectiviste}{%
\paragraph{Modèle objectiviste}\label{moduxe8le-objectiviste}}

St Thomas d'Aquin -\textgreater{} Trente -\textgreater{} Catéchisme de
1947\sn{«~des signes sensibles institués par Notre Seigneur Jésus
  Christ pour produire ou augmenter la grâce dans nos âmes~».}

\textbf{Sacrement comme instrument de production de la grâce~;
insistance sur la dimension de «~cause~» ou de moyen du salut.}

Dieu -\textgreater{} Sacrement -\textgreater{} Homme -\textgreater Dieu

Pb~: l'homme n'intervient pas dans le sacrement

\begin{itemize}
\item
  peu importe que l'homme «~comprenne~», pourvu qu'il soit sanctifié =
  limite)
\end{itemize}

Insistance~: Sacrement comme moyen, cause de salut

Avantage~: \textbf{souligne l'action de Dieu.}

\textbf{Instrument, canal, remède, germe.}

\hypertarget{moduxe8le-subjectiviste}{%
\paragraph{Modèle subjectiviste}\label{moduxe8le-subjectiviste}}

En réaction

Cf K. barth, qui craignait que la Liberté de Dieu soit «~compromise~»
par les sacrements, trop canalisée.

Le Baptême ne fait que \textbf{refléter} la \emph{justification et la
sanctification.} Dieu l'a déjà fait.

Contre une conception~»magique~» des sacrements.

Dieu -\textgreater{} Homme --\textgreater{} Sacrement -\textgreater{}
Dieu

\textbf{Sacrement comme instrument de traduction du déjà-là de la grâce.
Insistance sur la dimension de «~signe~» de salut (déjà donné).}

Avantage~: prend bien en compte le vécu humain. Dieu n'a jamais été lié
par les sacrements dans sa puissance de salut.

Inconvénient~: ne rend pas compte de l'efficacité du sacrement
lui-même.~: Trop exclusif.

Mais ce schéma va contre toute la \textbf{Tradition Chrétienne }(y
compris les pères).

Il est pourtant très à la mode, en réaction contre l'Église et en
insistant sur l'orthopraxie au lieu de l'orthodoxie.

\hypertarget{moduxe8le-vatican-ii}{%
\paragraph{Modèle Vatican II}\label{moduxe8le-vatican-ii}}

Schéma à «~double sens~»

Essai de tenir ensemble les dens sens (moyen de salut et signe du salut)

Dieu $\leftarrow$ $\rightarrow$ Homme $\leftarrow$ $\rightarrow$ Sacrement $\leftarrow$ $\rightarrow$ Dieu

Dieu~: \textbf{sujet opérateur des sacrements (}moyen de
salut-\textgreater{} vie humaine sanctifiée = louange à Dieu)

Difficulté~: tenir bien ensemble les deux sens

«~signe~» et «~cause~» ne sont pas des \textbf{catégories homogènes}


\paragraph{Question d'ouverture}

St Thomas d'Aquin~: II a II ae 89 prologue~; question sur l'éthique

→ on pourrait traiter ici (éthique) des sacrements → \textbf{sommet de
la vie éthique}

il aurait pu dire~: «~cause signifiante~» ou signe «~causant~».

un signe qui ne signifie que par mode de cause

une cause qui ne cause que par mode de signe

Chauvet, «~symbole et sacrement~»

Cf langage

\emph{Alliage homogène entre les deux~; grâce aux outils concernant le
\textbf{langage. }}

«~signe efficace~»~: mais c'est comme le langage~!

%-------------------------------------------------------------------------------------------------------------------------------
\section{Sacrement~: approche historique}

%-------------------------------------------------------------------------------------------------------------------------------
 
La constitution de la sacramentaire

Article LM Chauvet, «~sacrements~» dans Catholicisme

\hypertarget{mysterion-sacramentum}{%
\subsection{Mysterion -- Sacramentum}\label{mysterion-sacramentum}}

méfiance initiale contre le mot~; utilisé par les religions à mystère.

\textbf{Tertullien~;} aspect juridique~: la caution, le serment (cf
renonciation au mal)

Le Baptême comme \emph{pacte (foedus).}

La traduction en latin a fait perdre le lien avec le mysterion biblique,
l'économie divine.

Un vécu liturgique essentiel

\hypertarget{la-contreverse-sur-le-baptuxeame-des-huxe9ruxe9tiques}{%
\subsection{La contreverse sur le Baptême des
hérétiques}\label{la-contreverse-sur-le-baptuxeame-des-huxe9ruxe9tiques}}

Cyprien → Baptême lié à l'Église donc si deux Églises, deux Baptêmes .

→ Donatiste

en face Augustin~:~»le Christ agit «~etiam per malum ministrum~».

mais réception elle est fonction des dispositions personnelles.

\hypertarget{le-haut-moyen-uxe2ge}{%
\subsection{Le Haut moyen-âge}\label{le-haut-moyen-uxe2ge}}

présence réelle~: première question

\hypertarget{la-scholastique}{%
\subsection{la scholastique}\label{la-scholastique}}

la théologie comme science.

D'abord la finalité puis Importance de la causalité (Airstote)

Pratique de la LOI~: signifiait le Foi qui sauve mais ne causait pas le
salut. Signe non causal.

Importance de la pratique

Limites~:

\begin{itemize}
\item
  un schéma fondamentalement «~productionniste~»
\item
  le Cloisonnement des sept sacrements (et non les sacrements liés BEC
  cf 44)
\item
  l'absence d'ecclésiologie entre la christologie et la sacramentaire
\item
  une théologie trop en marhe de l'action liturgique
\item
\end{itemize}

\hypertarget{ouverture}{%
\section{ouverture}\label{ouverture}}

la sacramentaire scolastique est le fruit de la nouvelle
«~\emph{epistemé~»} né au 12° siècle . La fidélité à la grande tradition
théologique nous requiert non pas de reproduire mais d'en faire une
hérméneutique (décodage/recodage) nécessairement nouvelle ou sein de la
nouvelle culture qui est la notre.

 
%-------------------------------------------------------------------------------------------------------------------------------
\chapter{Sacrement de la Réconciliation}

%-------------------------------------------------------------------------------------------------------------------------------

Le «~Sacramentum~» comme langage symbolique et rituel

\hypertarget{introduction}{%
\section{Introduction}\label{introduction}}

Trilogie d'inspiration agustinienne~:

\begin{itemize}
\item
  \textbf{sacramentum -- rite- célébration --} ce qui se voit
\item
  \textbf{res sacramenti -- réalité du sacrement = la grâce}
\item
  \textbf{res et sacramentum~: 1\textsuperscript{er} effet de la grâce,}
  qui n'est pas la grâce finale ou \textbf{caractère~:}
\item
   
  ex~: on peut «~perdre~» la res sacrementi du Baptême tout en
  conservant la \emph{res et sacramentum.} (on reste baptisé).
   
\end{itemize}

Sacramentum → res sacramenti

\emph{res et sacramentum~}comme frontière

Entre le signe humain →

→ \textbf{médiation} ecclésiale ou 1\textsuperscript{ère} effet

Et la grâce de Dieu →

 %---------------------------------------------------------------------------------------
  \section{Problématique du langage} 

\begin{Synthesis}
    \begin{enumerate}
  \item
    {La langage n'est pas un instrument à la disposition de
    l'homme~; il est la médiation d'avènement du sujet c'est à dire le
    milieu dans lequel advient le sujet}
  \item
    {Puisqu'il n'est jamais de sujet hors langage (ou hors
    culture), constamment nous parlons ou «~ça parle~» en nous} \sn{Exemple~: culture où il n'y a qu'un seul mot «~s'occuper~» pour
      dire \emph{travailler} ou \emph{avoir des loisirs.} Le plus
      naturel , chez nous, est souvent le plus culturel.}

 
  \item
    {D'où la révolution épistémologique contemporaine quant à la
    manière de comprendre le rapport entre le sujet et le réel}
  \end{enumerate}
\end{Synthesis}


\textbf{Sujet} $\leftarrow$ $\rightarrow$ \textbf{Réel}

→ \emph{perception du}

\emph{← image mentale~; concept}

\begin{itemize}
\item
  \emph{restitution}
\end{itemize}

Langage

Pb de ce schéma~: il pose le langage comme instrument~: sujet =
antérieur (logiquement) au langage


\begin{Prop}
    {le concept d'~"expression" dans cette perspective~: le
  langage est simultanément «~révélateur~» et «~opérateur~»}
\end{Prop}



\textbf{Sujet} $\leftarrow$ $\rightarrow$ \textbf{Réel}

\textbf{Langage - culture}

{$\rightarrow$ \textit{le rapport au réel est toujours déjà aménagé~;} il est toujours
construit (le découpage sémantique n'est pas le même selon les cultures)
\textbf{le réel construit comme monde}}

\emph{$\leftarrow$le sujet se construit ainsi comme sujet}

Transition~:

Fonction du langage~: pas seulement utilitaire (information) mais
dimension symbolique (communication)

\hypertarget{probluxe9matique-du-symbole}{%
\section{Problématique du symbole}\label{probluxe9matique-du-symbole}}

\hypertarget{le-symbole}{%
\subsection{Le symbole}\label{le-symbole}}

Le symbole est médiateur de reconnaissance et d'alliance entre les
sujets.\sn{Cf Tertullien~; \emph{militia Christi}}

 
\paragraph{Le symbole antique}
 

Cf Tobie, 5 «~quel signe de reconnaissance donnerai-je~?~»

Triple aspect~:

\begin{itemize}
\item
  Systémique~: 1 élément n'est symbole que dans son rapport avec les
  autres éléments du même ensemble
\item
  Identitiaire~: se reconnaître comme partenaire
\item
  Juridique~: il faut être d'accor sur les \textbf{règles du jeu~.}
  Instance tierce de la Tradition, de la Loi~?
\end{itemize}
\mn{Cours du 4/2/03

Cours du //03}


Echange symbolique~: hors valeur

Intérêt théologique, grâce et échange symbolique

\hypertarget{luxe9change-symbolique}{%
\subsection{2.3 L'échange symbolique}\label{luxe9change-symbolique}}

\hypertarget{intuxe9ruxeat-thuxe9ologique}{%
\paragraph{2.32 Intérêt
théologique}\label{intuxe9ruxeat-thuxe9ologique}}

\hypertarget{b.-parole}{%
\paragraph{b. Parole}\label{b.-parole}}

Il n'y a rien de plus efficace que la parole

La parole ce n'est pas que des mots

Il suffit d'un «~je t'aime~» pour que la vie revienne

Mais aussi à travers le corps

\hypertarget{ruxe9flexion-sur-leucharistie}{%
\subparagraph{Réflexion sur
l'eucharistie}\label{ruxe9flexion-sur-leucharistie}}

Jn 6~: Pain de vie

Jésus est le pain de vie descendu du ciel

Rapport à la \emph{manne.}

\begin{itemize}
\item
  Mannu~: qu'est ce que c'est~?
\item
  On ne peut pas la stocker (ce n'est pas de la valeur)
\item
  Elle est donné quotidiennement
\item
  Sg 16~: «~elle était comme le pain des anges et elle s'adaptait au
  goût de chacun) Sg C'est vraiment du miel qu'est la sagesse
\item
  Beauchamp~: \emph{c'est le pur signe non chose}
\end{itemize}

La manne se prétait particulièrement à une analogie de la PAROLE.

1\textsuperscript{ère} tentation de la faim~: Mt 4,4 l'homme ne vit pas
seulement de pain mais de la parole qui sort de la bouche de Dieu.

\begin{itemize}
\item
  Dieu donne la manne pour que l'homme comprenne que la parole fait
  vivre (ref à Dt 8~?)
\item
  Pour manger l'eucharistie, il faut avoir ruminer la parole (cf
  élévation de la Bible et élévation de l'eucharistie)
\item
  Christ se présente comme la parole de vie envoyé par Dieu. Jn comprend
  la manducation d'eucharistie est une rumination de la parole douce et
  amère d'un Christ qui se livre pour la vie du monde.
\end{itemize}

L'expérience anthropologique de manger physiquement la parole pour
qu'elle soit complétement entendu~: d'où l'eucharistie~: parole donnée
sous le mode sacramentel

La Parole, c'est le Christ

\begin{itemize}
\item
  2 Co 10~: Le Christ est le OUI de Dieu au monde.
\end{itemize}

\hypertarget{eviter-le-piuxe8ge-de-la-chosification}{%
\subparagraph{Eviter le piège de la
chosification}\label{eviter-le-piuxe8ge-de-la-chosification}}

L'échange symbolique permet de ne pas chosifier ni magnifier (S Thomas
n'avait pas cet outil conceptuel~: dommage~!)

\hypertarget{c.-gratuituxe9}{%
\paragraph{Gratuité}\label{c.-gratuituxe9}}

\textbf{Charis}~: la grâce, ce qui est gratuit

Deux sens~:

\begin{itemize}
\item
  gratis Data~: donné gratuitement
\item
  \emph{gratiam gratum~:} gracieuseté. celle qui nous rend gracieux aux
  yeux de Dieu. Père de l'Eglise.
\end{itemize}

D'une part la grâce est gratuite~: elle ne dépend pas de nos mérites.

MAIS cela ne veut pas dire qu'il n'y ait pas de CONTRE DON, car sinon,
Dieu nous traiterait comme un OBJET (ALIENATION~: nous serions alors
écrasés). Or, il nous traite comme un SUJET. \emph{Un DON oblige.} Ce
n'est pas la \emph{conséquence} mais la \emph{marque} du don, car sinon
ce n'est pas un don car il n'est pas reconnu par la personne qui le
reçoit.

DON RECEPTION CONTRE-DON

Quand nous rendons grâce à Dieu pour les dons, nous ne sommes pas dans
le marchandage.

\hypertarget{baptuxeame-des-petits-enfants}{%
\subparagraph{Baptême des petits
enfants}\label{baptuxeame-des-petits-enfants}}

Le plus bel exemple de la gratuité de la grâce~: il ne tient pas compte
des mérites et démérites pour nous sauver. Mais il renforce un schème de
concurrence entre Dieu et l'Homme. Car le contre-don n'est pas marqué.
D'où l'importance de penser le baptême des petits enfants comme un
dérivé des adultes~: ACTE LIBRE~; êtres responsables.

\hypertarget{d.-le-thuxe9ologal-dans-lanthropologique}{%
\paragraph{le théologal dans
l'anthropologique}\label{d.-le-thuxe9ologal-dans-lanthropologique}}

échange symbolique pour approcher le mystère de la rencontre entre Dieu
et l'homme dans les Sacrements (leur efficacité)

\begin{itemize}
\item
  pas une simple analogie~: c'est la STRUCTURE dans laquelle s'effectue
  cette échange
\item
  le théologal se joue dans l'anthropologal, dans la constitution de
  l'homme
\end{itemize}

\hypertarget{lacte-de-symbolisation-analyse-et-application-uxe0-leucharistie}{%
\subsection{L'acte de symbolisation~: analyse et application à
l'eucharistie}\label{lacte-de-symbolisation-analyse-et-application-uxe0-leucharistie}}

2 agents secrets, avec billet de banque~déchiré; acte de symbolisation
pour se reconnaître partenaire

caractéristique

\begin{itemize}
\item
  systémique
\item
  symbolique
\item
  juridique (il faut avoir défini préalablement que le billet était le
  code de rencontre)
\end{itemize}

 

\begin{table}[h!]
    \centering
    \sidecaption{articuler  }
 
\begin{tabular}{p{.2\textwidth}p{.2\textwidth}p{.2\textwidth}}
\toprule
Christ &$\leftarrow$ $\rightarrow$ & Eglise \\
 
\\
\bottomrule
\end{tabular}
\label{tab:my_label}
\end{table}

 

 
\begin{table}[h!]
    \centering
\sidecaption{Des éléments du même ensemble mais qui sont
différents  }
 
\begin{tabular}{p{.2\textwidth}p{.2\textwidth}p{.2\textwidth}}
\toprule
Christ &($\leftarrow$ $\rightarrow$) & Eglise \\
 
\\
\bottomrule
\end{tabular}
\label{tab:my_label}
\end{table}


 

Ex~: un signe de Croix chez les papous~: ils ne peuvent pas le rattacher
aux autres éléments~: FONCTIONNEMENT imaginaire

\begin{table}[h!]
    \centering
\sidecaption{Chaque élément ne tient sa pertinence symbolique que de
par le rapport qu'il entretient avec les autres éléments de
l'ensemble
Le Christ n'existe que par l'Eglise dans le sens que sans Eglise,
personne ne connaîtrait le Christ }
\begin{tabular}{p{.2\textwidth}p{.2\textwidth}p{.2\textwidth}}
\toprule
Christ &$\leftarrow$& Eglise \\
        &   $\rightarrow$ \\
\\
\bottomrule
\end{tabular}
\label{tab:my_label}
\end{table}


\hypertarget{eucharistie}{%
\subparagraph{Eucharistie}\label{eucharistie}}

1Co 11~: Paul reproche au Co de prendre le repas du Seigneur

\begin{enumerate}
\def\labelenumi{\arabic{enumi}.}
\item
  niveau \emph{ecclésiologique et Ethique}~: vous prétendez prendre le
  repas alors que votrer pratique est en contradiction
\item
  niveau sacramentaire~ ou plutôt liturgique : Paul raconte le récit de
  la cène~: iatus entre la réponse et le problème posé
\item
  donc «~discernez le corps~». Le problème des Co, ce n'est pas la
  transsubstantiation\sn{Question apparue au IXème\ldots{}} (pas
  un problème sur la présence réélle). Mais mettre en rapport les
  membres du corps du Christ avec le corps Sacramentel (1 et 2)
\end{enumerate}

\hypertarget{rite-de-communion}{%
\subparagraph{Rite de Communion}\label{rite-de-communion}}

Lex Orandi, Lex Credendi

Regarder comment se pratique la célébration

\begin{table}[h!]
    \centering
      \sidecaption{ }
 
\begin{tabular}{p{.2\textwidth}p{.4\textwidth}}
\toprule
Geste de paix & «~Donnez-vous la paix~»~: ce qui est premier, c'est le
rapport horizontal avec mes frères \\
Fraction de paix & Rapport vertical (le Christ) pour faire l'unité des
membres de son corps (horizontal) \\
Démarche de Communion & Rapport vertical \\
\\
\bottomrule
\end{tabular}
\label{tab:my_label}
\end{table}
 

On ne peut pas séparer l'Eglise du Christ

\hypertarget{huxe9ruxe9sie-de-buxe9renger-de-tours}{%
\subparagraph{Hérésie de Bérenger de
Tours}\label{huxe9ruxe9sie-de-buxe9renger-de-tours}}

XIème 1053

Hérésie sur la présence du Christ dans l'eucharistie

Pb~: on n'a pas les outils conceptuels pour penser cette présence du
Christ. On pensait une adhérence du Christ aux espèces, qu'on aurait dû
le voir. Seulement par un deuxième miracle, Dieu étant un voile qui nous
le cache, afin que nous fassions un acte de foi (Isidore de Seville).

Il a fallu attendre le XII, pour que l'on trouve le concept adéquat,
celui de SUBSTANCE aristotélicien (on ne voit que les Accidents).

Les trois corps sont liés~:

\begin{enumerate}
\def\labelenumi{\arabic{enumi}.}
\item
  Corps historique et glorieux
\item
  Corps mystique sacramentel
\item
  Corps écclésial
\end{enumerate}

2---3~: Soyez ce que vous voyez~; recevez ce que vous êtes. Aug.
\sn{A lire sur S. Augustin~: Chair\ldots, Chair de l'Eglise Tillard}

Corpus Verum~était devenu le Corps ecclésial (H. de Lubac). Mais
berenger diminue le lien entre 1 et 2. En réaction, on a renforcé 1 et 2
au détriment de 2 et 3. L'adoration du Christ dans l'Eucharistie est
devenue plus importante que la Communion.\sn{\textbf{Historique}~:

  Pratique de la communion annuelle

  La réserve eucharistique (pour les malades) se formalise

  Evolution d'une conception de l'eucharistie action vers l'eucharistie
  chose (ex~: Transubstanciation en réaction contre l'hérésie de
  bérenger de Tours, qui utilise le concept aristotélicien de substance
  XII).

  Développement des processions, XIII «~le peuple n'a plus faim du pain
  eucharistique, rassasié des processions~».

  Elévation~: telle importance que l'on en attend de merveilleux effets
  deprotection contre les malheurs ou de vision miraculeuse du Christ
  lui même
  Exposition du Saint Sacrement dans une monstance lors de la
  contre-réforme}

\hypertarget{quimporte-la-valeur-de-luxe9luxe9ment-symbolique}{%
\paragraph{Qu'importe la valeur de l'élément
symbolique}\label{quimporte-la-valeur-de-luxe9luxe9ment-symbolique}}

Le symbole ne fonctionne pas à la valeur~: Crucifix~: bout de bois~;
bague de fiançailles

Musique~: ky-ri-e suffit -- pas besoin de 10 Mn de kyrie de
Mozart\ldots{}

\hypertarget{le-laboureur-et-ses-enfants-de-la-fontaine}{%
\subparagraph{«~Le laboureur et ses enfants~» de la
Fontaine}\label{le-laboureur-et-ses-enfants-de-la-fontaine}}

\begin{itemize}
\item
  Gardez-vous de vendre l'héritage, il y a un grand trésor dedans
\item
  ils labourent le champ
\item
  le travail est le trésor
\end{itemize}

Analogie intéressante avec le sacrement~:

\begin{itemize}
\item
  le sacrement n'est pas un trésor dans le champ
\item
  J'en suis obligé Grammaticalement~: la grâce est posé comme Objet
\item
  SUJET -- VERBE -- OBJET~: Dieu Donne la Grâce
\item
  Penser en théologie comme en philosophie, c'est apprendre à se défaire
  de son discours.
\item
  Heiddegger casse le langage~; Lévinas. Contre réduction
  phénoménologique. Risque certes de préciosité. Mais il faut casser les
  représentations.
\item
  La grâce n'est pas un objet valeur, même spirituel
\item
\item
  nous sommes la terre labourée. DurchArbeitung de Freud~: nous sommes
  travaillés
\item
  le socle de la charrue, c'est la parole de Dieu qui nous advient sous
  le mode rituel
\item
  il faut une force de traction, c'est l'ES
\item
  Nous allons nous transformer en acteur d'Alliance, alors que nous
  sommes tous héritier de Cain.
\item
  Cet engendrement permanent de nous même, nous devenons un peu plus
  fils de Dieu.
\end{itemize}

\begin{Def}[Grâce]
 un «~se recevoir~» toujours en cours
\end{Def}
  

\hypertarget{lacte-de-symbolisation-est-simultanuxe9ment-ruxe9vuxe9lateur-et-opuxe9rateur}{%
\paragraph{L'acte de symbolisation est simultanément révélateur et
«~opérateur~»}\label{lacte-de-symbolisation-est-simultanuxe9ment-ruxe9vuxe9lateur-et-opuxe9rateur}}

Le billet est à la fois «~révélateur~»~: ils se reconnaissent équipiers
mais aussi «~opérateur~»~: ils sont liés jusqu'à la mort dans leur
projet.

\hypertarget{doxologie}{%
\subparagraph{Doxologie}\label{doxologie}}

Par lui, avec lui et en lui

On présente à dieu , conjoint la terre au Ciel.

La voix monte aussi vers Dieu ainsi que les bras. Mais ce que nous
montons, c'est le Christ.

Puissance incroyable

Le Chrétien~: se reconnaît à Dieu, pour Dieu, en Christ.

C'est à force de faire des gestes comme cela que l'on devient Chrétien
et de les faire en Eglise. Attention à ne pas le psychologiser.

Une Foi ne vit qu'en se MANIFESTANT. Si on ne le dit jamais, notre foi
est quasiment morte.

\hypertarget{quelques-lois-de-la-ritualituxe9}{%
\section{Quelques lois de la
ritualité}\label{quelques-lois-de-la-ritualituxe9}}

\hypertarget{un-jeu-de-langage-original}{%
\subsection{Un «~jeu de langage~»
original}\label{un-jeu-de-langage-original}}

Ex~: génocide du Rwanda

\begin{itemize}
\item
  type \textbf{exactitude}~: discours du politologue
\item
  type témoignage~: discours du témoin~: reconstruit l'évenement pour ce
  faire comprendre. \textbf{Vérité narrative ou herméneutique}
\item
  type poétique~: vérité \textbf{poétique.} On ouvre un monde possible
  (Ricoeur)
\item
  Les trois types de langages visent la vérité de différentes voies.
\end{itemize}

Différents langages de ritualité~:

\begin{itemize}
\item
  théologique
\item
  mystique~: métaphore, affects (S. jean de La Croix)
\item
  témoignage~: MCC. On se raconte.
\item
  rituel
\end{itemize}

Attention, à ne pas se tromper de registres~: respecter les jeux de
langage différents. Ex~: à la messe, on ne parle pas de la théologie du
péché originel.

La liturgie croule sur les demandes contradictoires. Elle est faite pour
que l'on puisse recevoir le don de Dieu. Elle ne permet pas tout.

\hypertarget{lecture-thuxe9ologique-de-quelques-caractuxe9ristiques-majeures-de-lexpression-rituelle}{%
\subsection{Lecture théologique de quelques caractéristiques
majeures de l'expression
rituelle}\label{lecture-thuxe9ologique-de-quelques-caractuxe9ristiques-majeures-de-lexpression-rituelle}}

\hypertarget{un-langage-pragmatique-langage-action}{%
\paragraph{3.21 Un langage pragmatique~: langage
--action}\label{un-langage-pragmatique-langage-action}}

c'est une «~URGIE~» - ergon~: action\sn{Siderurgie,
  chirurgie,\ldots{}}

de l'ordre de la communication symbolique. Ne s'indresse pas à
l'intellect (non pas un LOGOS\sn{théologie,}) mais à tout le
corps. Lieu de jouissance~: on a une rencontre avec le Dieu Bon (et non
pas le Dieu Vrai\ldots).

\begin{itemize}
\item
  mon âme a soif de Dieu
\end{itemize}

\hypertarget{cuxe9luxe9bration-pour-les-enfants}{%
\subparagraph{Célébration pour les
enfants}\label{cuxe9luxe9bration-pour-les-enfants}}

But~: leur laisser 5 mn avec Dieu

\hypertarget{cest-la-muxe9moire}{%
\subparagraph{C'est la mémoire}\label{cest-la-muxe9moire}}

La liturgie, c'est une mémoire~: les places sont en attente\ldots{} Même
les gestes,

La faiblesse de Croire, Michel de Certeau

\textbf{Faites ce que vous dites~}: ex~: soyez dans la joie = chanter.
Ex~: onction au Baptême~: pas besoin d'expliquer~: cela sent bon,
imprègne le front.

\mn{Cours du 7/03/02}

\hypertarget{petite-lecture-thuxe9ologique}{%
\paragraph{Petite lecture
théologique}\label{petite-lecture-thuxe9ologique}}

Notion de \textbf{parole}~: sens d'Action en hebreu

\begin{itemize}
\item
  pas le logos grec
\item
  le DABAR (traduire par  dans la LXX)
\item
  SACREMENT, déploiement de ce qu'est le DABAR, la parole
\end{itemize}



    \paragraph{Un langage symbolique, donc économique~: sobriété et
    «~réserve~» du
    rite}

Symbole~: il rend PRESENT la chose (cf le drapeau brulé)~: il EST et il
n'EST pas le réel qu'il représente.

Les symboles représentent des mondes

Le symbole est \textbf{sobre.} Un peu suffit. Ex~: \emph{donnons-nous un
signe de paix~: ainsi nous nous engageons à nous réconcilier~; pas
besoin de faire le tour de l'Eglise.} Un peu de vin, un peu d'eau. Il ne
faut jamais substituer le réel au symbole.

Donner la chance au symbole~: ne pas tout expliquer~: on s'est trompé de
registre.

Goupillon~: moyen age~: préférer du buis.

Attention à la dérive \textbf{festive.} Injonction~: «~quand il
\emph{faut} faire la fête, ce n'est pas drôle\ldots~». Elle peut surtout
faire obstacle au passage de la mort vers la vie à faire avec le Christ.

Geneviève Hébert, Eloge \emph{de la pudeur en matière de dévotion et
ailleurs}, MD 215-225

LM Chauvet, \emph{Eschatologie et Sacrement}, MD 220\\
nos symboles liturgiques sont sobres. Elle est la médiation de la
condition eschatologique dans laquelle nous sommes~: le salut est déjà
donné mais pas encore achevé. Nous chantons «~\emph{Alleluia}~» car déjà
là mais nous ne nous laissons pas entraîner car tout n'est pas déjà
acquis~: cela peut être une insulte à tous les gens qui souffrent depuis
2000 ans~: RESERVE eschatologique.

J. CAILLOT, \emph{Liturgie et eschatologie,} MD 220

Grande originalité. Puisque le salut est déjà venu, il faut nous donner
sans réserve dans les tâches éthiques. Mais puisqu'il n'est pas complet,
toujours donné à Dieu le dernier mot.

JB METZ~: RESERVE~:tout pouvoir politique est réservé à Dieu.

Ici différent, «~PAS de JUBILATIO sans MODERATIO~» Augustin.

Expression de \textbf{Sainte Réserve.} Ce morceau de pain est un peu
dérisoire mais nous faisons une genouflexion~: nous prenons acte du
salut déjà donné mais nous le faisons devant un symbole tellement
dérisoire, qu'il indique que ce salut n'est pas achevé.

VERTUS DE LA PHENOMENOLOGIE en théologie.

Isabelle XXXX, \emph{Les actes de langage dans la prière} LMD 196

S'inspire de PEIRCE, entre un représentant (prêtre) et un représenté (le
Christ), il y a quatre relations possibles~:

\begin{itemize}
\item
  ICONE, de type photographique\sn{cf Critique de Theissen de la
    lecture iconique des sacrements, vision protestante, \emph{La
    religion des premiers chrétiens}}
\item
  Relation de DIAGRAMME (MODELE)
\item
  Relation de l'INDICE, de la GIROUETTE par rapport au VENT~: il y a un
  contact mais pas de ressemblance.
\item
  Rapport SYMBOLIQUE\sn{attention, pas le sens donné précédemment},
  entre un signifiant et signifié dans une langue (ex~: HORSE = cheval).
\end{itemize}

Ex~: Centurion, «~je ne suis pas digne~»~:

\begin{itemize}
\item
  je suis le centurion (ICONE)
\item
  je dis cela parce que l'on m'a demandé (SYMBOLIQUE)
\end{itemize}

un jeu possible entre les différents possibles~: chacun peut faire jouer
différemment le rituel, qui respecte chacun. Parfois, nous serons le
centurion. En liturgie, \textbf{Le moi est en état de recomposition.}

P°108

Rapport du prêtre au Christ~: ICONE, DIAGRAMME, INDICE, les trois sont
possibles en théologie catholique mais probablement pas SYMBOLIQUE
(vision protestante)~:

\begin{itemize}
\item
  la représentation ICONE est inadaptée à l'ordination de femme mais les
  deux autres.
\item
  A creuser
\end{itemize}

\hypertarget{un-langage-huxe9tuxe9rotopique-qui-reluxe8ve-dun-autre-lieu-que-le-quotidien-lordinaire-lutilitaire}{%
\paragraph{un langage «~hétérotopique~», qui relève d'un autre
lieu que le quotidien, l'ordinaire,
l'utilitaire}\label{un-langage-huxe9tuxe9rotopique-qui-reluxe8ve-dun-autre-lieu-que-le-quotidien-lordinaire-lutilitaire}}

kyrie 3 fois, étole, position du corps

il reste donc une distance entre la langue du rituel et la langue
vernaculaire.

\begin{itemize}
\item
  cela permet de comprendre pourquoi le Latin a aussi duré aussi
  longtemps
\item
  «~Cela est juste et Bon~»~: plutôt qu'on le «~ratifie~».
\end{itemize}

Fonctionnalité esthétique

Deux dérives possibles~: il faut que ce soit simple à prendre (verre) et
une dérive esthétisante (le bel objet)

TOUJOURS une rupture SYMBOLIQUE dans une célébration liturgique.

\begin{itemize}
\item
  rompt avec l'utilitarisme
\item
  crée un espace de gratuité dans lequel Dieu va pouvoir advenir
\end{itemize}

\hypertarget{un-langage-programmuxe9-et-donc-ruxe9ituxe9rable}{%
\paragraph{Un langage programmé (et donc
réitérable)}\label{un-langage-programmuxe9-et-donc-ruxe9ituxe9rable}}

On reçoit un rite. Le rite est codé (ex~: comment rentrons nous dans une
salle de cours). Cela permet d'économiser de l'énergie.

Il est reçu de générations antérieures, avec un Ancêtre. Le langage
courant le dit~: «~est rituel, ce qui est habituel~». Programme qui est
reçu.

Cette programmation n'est pas sans ambiguité~: une routine.

En particulier, les jeunes s'ennuient très vite. NEW -- LIVE -- SHOW.
Or, la liturgie , c'est exactement l'inverse.

Pourtant, cette programmation est protectrice de liberté. Ex~:
\emph{quand nous prenons congé après un diner, nous executons le rite.
Mais ce rite peut être chaleureux ou non.}

Il faut accepter de jouer le jeu du code mais il y a une manière
d'habiter le code.

facteur de liberté. Cela m'empêche de me demander ce que je dois faire.

Possiblement libérateur.

\hypertarget{valeur-thuxe9ologique}{%
\paragraph{Valeur théologique}\label{valeur-thuxe9ologique}}

Double valeur théologique~:

\begin{itemize}
\item
  \textbf{Christologique}~: institution de l'eucharistie~: rien n'est
  plus programmé~: les 4 verbes de cette institution (prendre, bénir,
  rompre, donner) servent à structurer l'eucharistie. Confession de foi
  de l'Eglise par mode de faire symbolique, que le Christ est SEIGNEUR~:
  le faire parce qu'il a dit de le faire. (approche phénoménologique~:
  je regarde la chose et je pense ce qui ce passe)~;
  Métaphore\sn{renvoie}
\item
  \textbf{Ecclésiologique~:} Manifester que toute célébration est
  manifestation de l'Eglise. Métonimie\sn{la partie pour le tout}.
  \emph{Quand on célébre en Afrique, on manifeste l'Eglise entière.}
  Elle n'appartient à personne. Certes particularité de la célébration
  mais pas particularisme. C'est ce que nous rappelle la programmation
  rituelle.
\end{itemize}

\hypertarget{evanguxe9liser-la-ritualituxe9}{%
\subsection{Evangéliser la
ritualité}\label{evanguxe9liser-la-ritualituxe9}}

Rite meilleur et pire.

Pire~:

\begin{itemize}
\item
  Routine, compulsion de répétition
\item
  sacralisation du politique (cf te deum)
\item
  ambiguité spirituelle~: substituer la magie du rite à la question
  existentielle de Dieu~: les protestants y sont très sensibles~:
  PRIORITE de la PAROLE.
\end{itemize}

La ritualité est toujours à evangéliser~: le rite ne devient SACREMENT
qu'avec la PAROLE et l'ESPRIT SAINT. Mais cette parole ne nous parvient
que sous mode rituelle.

%-------------------------------------------------------------------------------------------------------------------------------
\chapter{Place et fonction des sacrements dans la structure de l'Identité
Chrétienne}

%-------------------------------------------------------------------------------------------------------------------------------
 


\hypertarget{i.-introduction}{%
\section{Introduction}\label{i.-introduction}}

Se rappeler que les sacrements ne sont pas le TOUT de l'existence
chrétienne

\hypertarget{lc-24-les-disciples-demmaus}{%
\subsection{Lc 24, les disciples
d'Emmaus}\label{lc-24-les-disciples-demmaus}}

Catéchèse du passage de la non foi à la foi, de la méconnaissance à la
reconnaissance. La question qui guide ce texte, s'il est vrai que le Ch
est ressuscité, pourquoi ne pouvons-nous pas le VOIR~? Immédiatement

Comment se fait il que nous ne puissions pas le PROUVER~? question des
destinataires de Lc

Lc répond~: Renoncer à VOIR, TOUCHER, TROUVER, à leur immédiaté. Ces
verbes ne sont appliqués que sur le Cadavre du Christ. Le verbe
«~TOUCHER~» dans la 3\textsuperscript{ème} séquence~: toucher les
marques de sa mort. Si l'on veut reconnaître le Christ vivant, il faut
renoncer à cette vision de Christ Cadavre.

La transformation des deux disciples est rendue possible par~3 éléments
successifs~:

\begin{itemize}
\item
  \textbf{le rapport aux écritures,} le déblocage commence quand le
  Christ prend la parole~: «~il leur expliqua tout ce qui le concernait
  dans les Ecritures. Kerygme. \emph{Dihermeneusein}. En filigrane de ce
  que fait Jésus, il y a l'Eglise~: lecture de la thora et des
  prophètes.\\
  Technique rabbinique~: on rapprochait un élément des textes lus d'un
  autre texte, d'un élément de la tradition orale pour l'actualiser.
  Plus d'importance chez les Chrétiens pour les prophètes que pour
  Moïse.\sn{une des raisons de la rupture avec les juifs}
  Désormais, toutes les Ecritures seront interprétées par la mort et la
  résurrection du Christ . Il y a 50 ans de pratique ecclésiale. A
  chaque fois que nous relisons les Ecritures et les interprétons sous
  la lumière de Pâque, alors le Christ est présent au milieu de nous
  (\emph{Christ vivant}). \textbf{Apprendre à reconnaître le Christ dans
  l'Eglise.} «~Acclamons la parole de Dieu -- Louange à toi, Seigneur
  Jésus~».
\item
  \textbf{la fraction du pain}, au repos, à table. Il rompit le pain. 4
  verbes du récit de l'institution. il y a 50 ans de pratique
  eucharistique. Renoncer à Voir Jésus directement \textbf{mais
  apprendre à le reconnaître dans l'Eucharistie}. C'est lui qui bénit le
  pain, le rompt pour nous et nous le donne. Alors, il disparaît~: les
  yeux des disciples s'~ouvrent sur du vide, mais sur du vide plein.
  \emph{Anastantes~}: ils se lèvent (résurrection). On ne peut pas
  reconnaître Jésus sans être nous même relevés. Ils retournent alors à
  Jérusalem~: ils commencent par accueillir le témoignage des 11. C'est
  après qu'ils joignent leur témoignage. (synectode~: la partie pour le
  tout~: la bénédiction qui ouvrait le repas signifiant la totalité du
  repas).
\item
  \textbf{conduite éthique}, moins évident. Je déborde le texte pour
  aller voir les sommaires (Ac 2~; 4).Lc croque le portrait de la
  première communauté de Jérusalem~: fraction du pain, communion,
  partage fraternel, didascalie. Il insiste sur le \textbf{partage
  fraternel.} Grâce à ce partage, il n'y a plus de frères démunis. De ce
  fait, c'est le signe de la communauté eschatologique~: témoignage
  rendu à la résurrection~: «~voyez comme ils s'aiment~».\sn{J.
    DUPONT} Rappelle la théologie du 4\textsuperscript{ème} évangile.
  Comparaison en COMME~: «~la table est jaune comme le mur «~ (OS) et
  «~je suis chauve comme mon père~» (KATHOS)~: il y a participation. Cf
  lavement des pieds~: Kathos. Participation à ce qu'il fait. Quand les
  frères «~se lavent les pieds~», Jésus continue son œuvre à travers
  eux. Tertullien~: «~Le sacrement du frère~». «~tu as vu ton frère, tu
  as vu ton Dieu~» Clément d'Alexandrie.
\end{itemize}

\hypertarget{la-muxe9diation-de-leglise}{%
\subsection{La médiation de
l'Eglise}\label{la-muxe9diation-de-leglise}}

\hypertarget{schuxe9ma}{%
\paragraph{Schéma~:}\label{schuxe9ma}}

Pas de rapport direct entre nous et le Christ~: il faut passer par
l'Eglise.

Christ

Le cercle,

L'Eglise

Il n'y a pas de Chrétiens en dehors de l'Eglise. En dehors de l`Eglise,
point de salut confessé.

P. Lieger, op, \emph{l'être ensemble des Chrétiens}

L'Evangile est source d'un a priori communautaire. E\^{}tre chrétien,
c'est être d'emblée mis ensemble. Nous sommes sous sa Seigneurie tous.

Paul

\begin{itemize}
\item
   
  Col 3,9-11,
   
\item
   
  1 Co 12, 13~;
   
\item
   
  Ga 3, 26-28
   
\end{itemize}

Le baptême, c'est la réalisation du salut eschatologique~: «~il n'y a ni
juif ni grec, ni homme ni femme~».

Ep 2, 14 ss Christ est mort pour abolir le mur de séparation~; il a tué
la haine. Le juif et le grec

Ep 3~: juifs et paiens héritiers du même héritage

Yves de Montcheuil, mort en 1944 dans le Maquis, \emph{Aspects de
l'Eglise}

\emph{«~ce ne sont pas les chrétiens qui en se réunissant forment
l'Eglise, c'est l'Eglise qui fait les Chrétiens~».}

Devenir Chrétien, c'est apprendre à aimer l'Eglise peu à peu.

Cette priorité de l'Eglise.

\hypertarget{la-priorituxe9-du-nous-eccluxe9sial.-remarque-sur-la-pruxe9sidence-par-un-ministuxe8re-ordonnuxe9}{%
\paragraph{La priorité du «~nous~» ecclésial. Remarque sur la
présidence par un ministère
ordonné}\label{la-priorituxe9-du-nous-eccluxe9sial.-remarque-sur-la-pruxe9sidence-par-un-ministuxe8re-ordonnuxe9}}



\hypertarget{eglise-et-assembluxe9e-cuxe9luxe9brante}{%
\paragraph{Eglise et assemblée
célébrante}\label{eglise-et-assembluxe9e-cuxe9luxe9brante}}

\hypertarget{lassembluxe9e-dominicale}{%
\paragraph{L'assemblée
dominicale}\label{lassembluxe9e-dominicale}}

\hypertarget{ouverture-pastorale}{%
\paragraph{Ouverture pastorale}\label{ouverture-pastorale}}

\hypertarget{laccuxe8s-uxe0-la-foi-comme-consentement-uxe0-une-perte}{%
\subsection{L'accès à la foi comme consentement à une
perte}\label{laccuxe8s-uxe0-la-foi-comme-consentement-uxe0-une-perte}}

\hypertarget{triple-tentation}{%
\paragraph{Triple tentation}\label{triple-tentation}}

\hypertarget{la-bonne-santuxe9-e-la-foi}{%
\paragraph{La bonne santé de la
Foi}\label{la-bonne-santuxe9-e-la-foi}}

%---------------------------------------------------------------------------------------------------
\hypertarget{ii.-fonctionnement-de-la-structure}{%
\section{ Fonctionnement de la
structure}\label{ii.-fonctionnement-de-la-structure}}

\hypertarget{la-priuxe8re-eucharistique}{%
\subsection{1. la prière
eucharistique}\label{la-priuxe8re-eucharistique}}
 
  \paragraph{Analyse narrative}\label{analyse-narrative} 
  \paragraph{Statut du récit de l'institution~: récit de
  l'Eglise} 
  
  \paragraph{le discours
  d'anamnèse} 
 
  \paragraph{Le discours d'Epiclèse  } 
  \paragraph{Schéma du procès
  d~«~eucharisticité~»} 

\hypertarget{remarque-sur-ce-procuxe8s}{%
\paragraph{Remarque sur ce
procès}\label{remarque-sur-ce-procuxe8s}}

\hypertarget{a.-un-moduxe8le-et-non-pas-du-pruxeat-uxe0-porter}{%
\subparagraph{un modèle et non pas du prêt à
porter}\label{a.-un-moduxe8le-et-non-pas-du-pruxeat-uxe0-porter}}

La structure mentionnée la fois dernière est un patron. Le mouvement ne
commence pas nécessairement par l'Ecriture (don), cela peut être une
célébration (récéption) ou un témoignage éthique. (contre don).

Si c'est l'Evangile qui vous fait agir\ldots~? qu'y a t-il donc dans
l'Evangile~?

Le déclencheur peut être l'un des éléments, l'histoire commence
lorsqu'on ouvre l'Ecriture.

Il n'est pas d'itinéraire chrétien hors du processus Ecriture Sacrement
Ethique, vu pour la prière eucharistique.

\hypertarget{b.-le-don-duxe9pend-de-dieu-seul-mais-la-ruxe9ception-de-ce-don-comme-tel-duxe9pend-du-contre-don.}{%
\subparagraph{Le don dépend de Dieu seul~; mais la réception de ce don
comme tel dépend du contre
don.}\label{b.-le-don-duxe9pend-de-dieu-seul-mais-la-ruxe9ception-de-ce-don-comme-tel-duxe9pend-du-contre-don.}}

Dieu a l'initiative du don~! (hors dépendance de l'attitude éthique de
l'être humain) Mais il n'est de don reçu qu'avec le contre don éthique
(conversion du cœur, foi, agapè).

Validité et fécondité~:

Validité~: dépend de l'action de Dieu (et pas du sujet)

En revanche la fécondité dépend de notre cœur~: si notre vie ne s'ouvre
pas à la grâce

Un sacrement peut être valide et être reçu pour notre condamnation~! (1
Co 11).

Cf. St Augustin (controverse avec les donatistes~: leurs sacrements sont
valides «~vrais~» mais inféconds, car ils sont hors de l'Eglise.

Ne dites pas que la grâce de Dieu dépend de notre foi à l'oral~!

\hypertarget{c.-toute-cuxe9luxe9bration-sacramentelle-fonctionne-selon-ce-moduxe8le}{%
\subparagraph{toute célébration sacramentelle fonctionne selon ce
modèle}\label{c.-toute-cuxe9luxe9bration-sacramentelle-fonctionne-selon-ce-moduxe8le}}

les exemples ne manquent pas. (Ecriture et Sacrement OK, éthique =
envoi~! Peu développé à la messe, mais davantage pour un mariage)

Les sacrements ne sont pas autre chose que la cristallisation de la
Parole de Dieu.

\subparagraph{le moment sacrement~: ni point de départ,
ni point d'arrivée. Ce n'est qu'un point de passage. Mais il
{est} point de passage obligé et
obligeant} 

le moment réception n'est pas départ (ce sont les \textbf{Ecritures}) ni
arrivée (\textbf{vie chrétienne éthique})

point de passage obligeant~: car toute réception d'un don comme don
oblige~! (reconnaissance, foi, éthique de réponse)

point de passage obligé~: pour que l'éthique soit proprement chrétienne,
il faut qu'elle soit vécue comme un éthique de réponse au don premier de
Dieu (éthique théologale). Ainsi, la sanctification par le don de Dieu
est en ce sens «~obligé~». L'éthique devient elle même une liturgie, un
sacrifice spirituel. (cf prochain cours)

\hypertarget{identituxe9-juive-et-identituxe9-chruxe9tienne}{%
\paragraph{Identité juive et identité
chrétienne}\label{identituxe9-juive-et-identituxe9-chruxe9tienne}}

\hypertarget{place-du-moment-sacrement-par-rapport-au-moment-ecriture}{%
\subsection{Place du moment Sacrement par rapport au moment
Ecriture}\label{place-du-moment-sacrement-par-rapport-au-moment-ecriture}}

\hypertarget{les-ecritures-sacrement-de-la-parole}{%
\paragraph{Les Ecritures sacrement de la
Parole}\label{les-ecritures-sacrement-de-la-parole}}

Le premier «~tabernacle~» de la Parole de Dieu, les premiers sacrements
(mysteria) ce sont les Ecritures, pour les pères~!

Augustin~: sacramentum, c'est à dire n'importe quelle parole des saintes
lettres~!

N'importe quel épisode de l'Ecriture est traité comme un sacramentum~:
révélation en figure du dessein de salut de Dieu.

Pas étonnant que le livre des Ecritures ait fait l'objet d'une véritable
vénération dans la liturgie (ex~: évangéliaire)

Cf. Origène, SC7 p. 211 homélie sur la genèse. (écriture traitée avec
autant de poids que l'eucharistie). Voir aussi SC16 p. 263 homélie sur
exode

Ainsi l'Evangéliaire est orné, on l'encense, on chante alléluia\ldots{}
Voire procession~: Dei Verbum~21 : une seule table aussi bien de la
Parole que du Corps du Christ, et une seule vénération.

L'expression de «~pain de vie~» est aussi bien employée pour la parole
de Dieu que pour l'Eucharistie~! Il faut retrouver le réflexe de penser
au pain de vie comme étant d'emblée la Parole~!

(si on avait le temps, la sacramentaire devrait inclure les icônes)

Livre tabernacle -- sacrement de la Parole de Dieu

Il existe une tentation de dire que la Parole de Dieu est l'Ecriture
interprétée. NON C'est l'Ecriture dans sa lettre~! (cf. Parler
d'Ecriture Saintes~: prise en compte de la lettre dans la positivité
historique, son altérité culturelle, comme sacramentum). L'esprit n'est
trouvé que si la lettre n'est pas esquivée. C'est pareil pour la
sacramentaire.

On peut craindre qu'à force d'encenser le livre, on idôlatre la lettre~:
il faut alors rappeler que la bible n'est pas le Coran

\begin{enumerate}
\setcounter{enumi}{1}
\item
  La Parole de Dieu (PDD) c'est le Christ, pas le livre
\end{enumerate}

\begin{enumerate}
\setcounter{enumi}{1}
\item
  La lettre peut être hébraique ou grecque ou latine, \ldots{} Certes
  les textes fondateurs ont un privilège. Mais cessons de courir après
  le mythe du texte originaire~! La PDD n'est pas liée à une langue. Il
  existe un jeu symbolique entre la lettre et la Parole de Dieu.
  \emph{(Acclamons la Parole de Dieu Louange à toi Seigneur Jésus}).
\item
  La lettre se dédouble~: il est écrit qu'autre chose est à écrire
  (\emph{la création annonce une seconde création~: la manne annonce une
  autre manne, l'exode annonce un autre exode}) La lettre est alors non
  idole, qui arrête le regard, mais icône, qui demande qu'on la traverse
  en direction d'un autre qu'elle-même.
\end{enumerate}

\hypertarget{le-sacrement-pruxe9cipituxe9-des-ecritures}{%
\paragraph{\texorpdfstring{2.2. Le Sacrement, «~précipité~» des
Ecritures
}{2.2. Le Sacrement, «~précipité~» des Ecritures }}\label{le-sacrement-pruxe9cipituxe9-des-ecritures}}

Chaque célébration le montre. La Parole vient nous rejoindre dans le
sacrement comme dans son prolongement, dans son déploiement.

Dans cette perspective il faut comprendre les paroles sacramentelles (ex
\emph{je te baptise au nom du Père et du Fils et du Saint Esprit}) comme
des \underline{synthèses} de la révélation de Dieu dans les Ecritures.

\hypertarget{a-accedit-verbum-ad-elementum-et-fit-sacramentum}{%
\paragraph{«~Accedit verbum ad elementum et fit
sacramentum~»}\label{a-accedit-verbum-ad-elementum-et-fit-sacramentum}}

deux traductions possibles~:

 
\emph{La Parole vient sur l'élément et ainsi se fait le sacrement} ou

\emph{Le Verbe vient sur l'élément et devient sacrement.}
 

Le Verbe, c'est d'abord le~Christ~; deuxième niveau~: le Christ tel
qu'il est annoncé dans la fête ou le temps liturgique du jour~; enfin la
Parole sacramentelle.

\hypertarget{b-manducation-de-la-parole-jn-6-manne-parole-pain}{%
\paragraph{Manducation de la Parole (Jn 6 Manne Parole
Pain)}\label{b-manducation-de-la-parole-jn-6-manne-parole-pain}}

cf Grelot (introduction au nouveau testament tome~??) c'est une homélie
sur Ex 16, à la manière rabbinique.

La manne, (question) se prête déjà à être figure de la Parole~: cf. Dt
8,3 cité lors du récit des tentations. Mt 4,4.

L'objet du discours du pain de vie n'est pas l'Eucharistie (sauf après
le v. 51), mais le mystère du Christ, Dieu crucifié pour la vie du
monde, envoyé de Dieu qui donne sa chair pour la vie du monde.

Double scandale~: sur l'identité de Jésus

v. 42~: comment peut-il se prétendre descendu du ciel~?

v51~: comment peut-il se prétendre donner sa chair à manger~?

Espace symbolique ambiant = manducation de la Parole (cf Ez 2, 3~:
manger le rouleau de la Torah, cf. aussi Apocalypse), et aussi
manducation eucharistique (bien sûr~!)

Devenir chrétien, c'est mâcher, ruminer la Parole de Dieu, le scandale à
la fois doux et amer d'un messie crucifié pour la vie du monde~: c'est
ça que nous faisons dans l'eucharistie~! Jusqu'à ce que ça fasse corps
avec notre corps~: et c'est beaucoup plus fort~: on parle du mystère du
Christ plutôt que de mentionner directement l'eucharistie.

Pour manger avec fruit l'eucharistie, il faut donc avoir d'abord ruminé
la Parole. Cf. St Ambroise «~cette nourriture (parole) mange la d'abord,
enfin de pouvoir en venir ensuite à la table du Seigneur~». (citation
non littérale)

Cf traités 26 et 27 d'Augustin sur Jn. (manger jusqu'à avoir part à
l'Esprit Saint).

Le rite ne constitue pas le sacrement, c'est la PAROLE DE DIEU qui nous
advient en forme rituelle

\hypertarget{evanguxe9lisation-et-sacramentalisation}{%
\paragraph{2.3. Evangélisation et
sacramentalisation}\label{evanguxe9lisation-et-sacramentalisation}}

Avant l'accès au sacrement, il y a l'annonce de l'Evangile.

Priorité à l'évangélisation~: mais elle doit être structurée
sacramentellement. (cf ce que les évêques disent de la catéchèse).~

Sur le fond~: l'évangélisation est annonce d'un Christ sacrement, et non
pas d'un Christ exemple~! Ne pas annoncer un modèle à imiter (derrière
qui on devrait courir~! décourageant) mais quelqu'un qui vient nous
sauver, il est notre passeur.

Sur la forme~: dans toute catéchèse, il faut faire référence au
sacrements. Ne pas se contenter de se référer aux deux pôles biblique et
éthique. La dimension liturgique et sacramentelle est trop souvent ce
dont on parle quand l'occasion se présente~: elle mériterait d'être
intégrée à la catéchèse.

Ex~: l'appel des disciples. ordination~? envoi des catéchistes à la
messe de rentrée~?

Texte de guérison des malades (onction~: même si c'est pas au
programme~!)

Esprit Saint~: liturgie de la Pentecôte. L'expérience liturgique est un
lieu théologique de première importance.

\hypertarget{place-du-moment-sacrement-par-rapport-au-moment-ethique-.}{%
\subsection{Place du moment «~Sacrement~» par rapport au moment
«~Ethique~».}\label{place-du-moment-sacrement-par-rapport-au-moment-ethique-.}}

\hypertarget{le-statut-du-culte-en-judauxefsme-historico-prophuxe9tique}{%
\paragraph{Le statut du culte en judaïsme~: historico
prophétique}\label{le-statut-du-culte-en-judauxefsme-historico-prophuxe9tique}}

\hypertarget{a-la-foi-en-un-dieu-intervenant-en-histoire}{%
\subparagraph{La foi en un Dieu intervenant en
histoire}\label{a-la-foi-en-un-dieu-intervenant-en-histoire}}

Dans les cultes païens, le temps est cyclique ou en forme de spirale~:
calendrier de type cosmique.

Dans le judaïsme, on brise cette circularité. Le temps est vécu de
manière différente, car la Bible, au lieu de privilégier le retour du
même, privilégie les évènements comme avènement de l'inédit, du nouveau.
Le temps devient plus linéaire. La clef de lecture de la Bible, c'est
l'eschatologie, c'est l'Omega qui donne à lire l'Alpha.

Le premier lieu de révélation biblique est l'histoire, avant le cosmos
(la création du monde est l'œuvre du Dieu d'Israël, son peuple).

Bereshit (premier mot de la bible) André Néher \emph{Les culture et le
temps}, a noté~: non pas «~au commencement~», mais «~en un
commencement~» Dieu créa. Ce qui est primordial, c'est qu'il y ait eu un
début~: le primordial c'est que la création inaugure le temps.

Le 26/3/03

Le temps de la Bible, ce n'est pas le temps scientifique, mais c'est le
temps d'un \textbf{peut-être, temps de la liberté humaine, arraché à} la
fatalité.

Cf Ecart par rapport au rite mésopotamien~: ici , un risque, un mariage
entre Dieu et Israêl, souvent malheureux entre le projet de Dieu et
l'homme libre, représenté par Israël.

\hypertarget{b.-douxf9-le-culte-muxe9morial-de-ce-dieu-renvoie-uxe0-la-prise-en-charge-de-lhistoire}{%
\subparagraph{d'où le culte, mémorial de ce Dieu, renvoie à la prise en
charge de
l'histoire}\label{b.-douxf9-le-culte-muxe9morial-de-ce-dieu-renvoie-uxe0-la-prise-en-charge-de-lhistoire}}

Le culte va renvoyé Israël à la prise en charge de l'histoire.

Liturgie~: liturgie du prochain.

\hypertarget{muxe9morial}{%
\subparagraph{Mémorial}\label{muxe9morial}}

C'est PRECISEMENT dans la mesure où son identité est fondée sur la
relation à un Dieu entré en histoire qu'Israel, en son culte, est
renvoyé à sa responsabilité dans l'histoire, et plus précisément à
autrui.

Théologiquement, c'est le concept de \emph{mémorial} : essence
historique et prophétique de ce culte.

Zikkaron (ZKR~: se souvenir) LXX~: mnemosunon, anamnesis

\textbf{Paradigme}~: la fête de Pâque.

Ex 12,1-20

«~tu transmettras cet enseignement~: «~tu haggaderas à ton fils ce jour
là~»

La mishna commente~: «~de génération en génération chacun doit se
reconnaître comme étant sorti lui-même d'Egypte~» (Pes 10,5)

S~. Augustin, \emph{lettre 98 à l'Evèque Boniface},

Sur le baptême des petits enfants

Il faut que nous défaisions l'image naive du temps (passé, présent,
futur). Cf critique Heidegger.

L'être humain est de nature commémorative, mais il ne l'est qu'en tant
que nature projective.

JB. METZ~: \emph{la Foi dans l'histoire et la société}, Cogitatio Fidei

Un classique dans le domaine politique

Récit, mémoire

En particulier, mémoire dangereuse et mémoire de la souffrance

Il y a une mémoire et mémoire. Il y a une mémoire vive, et une mémoire
aliénante. La mémoire aliénante sort les photos jaunis des bons moments.
Une mémoire maternante, anecdotique qui idéalise, qui fait régresser. La
mémoire devient vive quand elle fait bouger~: «~plus jamais cela~»
d'Auschwitz, de la guerre. Elle ne doit pas oublier les souffrances du
passé car elle mobilise les énergies pour faire que cela n'arrive plus.
Une mémoire n'est vive que si elle s'ouvre sur un avenir.

La mémoire du passé fait bouger le présent, elle remet debout, en vue
d'un nouveau recommencement, ceux qui sont prostrés dans le silence et
l'oppression de l'EXIL.

Il ne s'agit pas de faire simplement mémorisation, simple souvenir
auquel on a volé son avenir (JB Metz). Acte vif de COM-MEMORATION
(commune) Tout projet d'avenir semble s'enraciner dans le réveil d'une
telle tradition~: l'homme n'a d'avenir que parce qu'il a de la mémoire.

relecture recueillement du passé anticipant le futur. Il est sous le
régime du futur antérieur.

La mémoire chrétienne est une mémoire de la Passion du Christ et de sa
résurrection. \emph{Memoria Passioni.} «~tenons en éveil la mémoire du
Seigneur~».

Futur antérieur~: Dt 26,1-11 les prémices~; La terre, qu'Israel a déjà
est toujours à recevoir.

\textbf{Chiasme~}:

Histoire à Vivre FUTUR

Rite à Accomplir PRESENT

Mémorial Confession de foi «~mon père était un araméen errant~» PASSE

Rite à Accomplir PRESENT

Histoire à Vivre (levite, Immigré) FUTUR


\begin{table}[h!]
    \centering
    \sidecaption{  }
 \footnotesize
\begin{tabular}{p{.15\textwidth}p{.15\textwidth}p{.15\textwidth}p{.15\textwidth}p{.15\textwidth}}
\toprule
TU + Futur & & & & TU + Futur \\
Prescription rituelle & JE + Présent & & JE + Présent & Prescriptions
éthiques \\
& Parole rituelle & NOUS + Passé & Parole + gestes rutuels & \\
& & Confession de foi/ Mythe de ses origines\sn{Tout le
  Pentateuque + Josué~: le mémorial.} & & \\
 
\\
\bottomrule
\end{tabular}
\label{tab:my_label}
\end{table}


Israël est exhorté à ouvrir la terre à Dieu, mais dans le but de
l'offrir à autrui.

Lorsque nous demandons (rite), l'insérer dans le mémorial et la
pratique. Le rite permet d'actualiser le passé (on refait le rite des
prémices). La dépossession de la terre qui est signifié par le rite doit
être vérifiée par l'acte éthique envers le lévite (sans terre par
vocation au plus intime d'Israël) et l'immigré (sans terre par
nécessité, ext. à Israël).

Les prémices, véritable \emph{sacramentum} ou \emph{visibile verbum}
(Aug.) du mémorial est l' «~expression~» au sens fort où l'identité
d'Israel s'effectue en s'énonçant.

Le passé est encadré par un présent qui en indique l'actualité~;
cependant, il ne s'y arrête pas et doit s'ouvrir sur l'autre. Israël
n'est véritablement prenant la terre de Dieu qu'en l'ouvrant aux autres,
et ceci le rite le rappelle constamment à Israël. On est au cœur de
l'identité JUIVE.

\textbf{Manne}~: \emph{pure signe non chose}, de la non possession, de
la pure attente, Israël pouvait vivre que de la grâce de Dieu ou du seul
ciel~; Dès la prise du pays, la tentation est d'oublier la leçon du
désert, c'est à dire de s'approprier la terre comme \emph{pure chose non
signe}

\hypertarget{une-crise-rituelle}{%
\subparagraph{Une Crise rituelle}\label{une-crise-rituelle}}

Renvoi à la pratique historique de la «~liturgie du prochain~» amène
inévitablement à une crise rituelle. La parole vient en effet sacrifier
la première naiveté rituelle~: Israël ne peut plus être, comme les
autres religions paiennes, en possession tranquille de son culte. Cf
prophète contre le formalisme cultuel. Dieu ne veut pas d'un culte qui
serait en opposition avec l'attitude éthique de son peuple.

Am 5, Is 1, Jr 7, Mic 6 ,Ps 50

Jésus reprendra (cf Mt 9~?~: ce que je veux ce ne sont pas les
sacrifices mais un cœur miséricordieux).

Voie ouverte de substitution des sacrifices par la pratique éthique.

 
Ben Sirac, 34 observer la loi vaut tous les sacrifices

Augustin

Les chrétiens maintiendront tout de même un sacrifice, qui n'a plus rien
de sacrifice pour témoigner du sacrifice de la vie éthique. Rm 12, 1
\emph{offrez vous vous mêmes en sacrifice saint}
 

\hypertarget{juxe9sus-et-le-culte-juif}{%
\paragraph{Jésus et le culte
juif}\label{juxe9sus-et-le-culte-juif}}

\hypertarget{a.-une-attitude-critique-dans-le-sillage-de-la-spiritualisation-des-sacrifices}{%
\paragraph{une attitude critique, dans le sillage de la
spiritualisation des
sacrifices}\label{a.-une-attitude-critique-dans-le-sillage-de-la-spiritualisation-des-sacrifices}}

Os 9, Jr 7 «~ce temple, caverne de bandit~».

Jésus fortement critique contre le formalisme rituel. Pas innovant, de
la même façon que le commandement de l'amour (cf Lc 10, le scribe~; Mc
12 le scribe renchérit). Il existait un courant bien vivant dans le
judaïsme.

Charles Perrot, \emph{Jésus et l'histoire}

Revoir jésus et le temple et le baptisme. A Kumran, on acceptait
théoriquement des sacrifices à Jérusalem mais on exhaltait le sacrifice
des lèvres~; de même, chez les baptistes et les judéo-chrétiens.

Logike latreia~: rationabilis hostia~: sacrifice raisonnable, droit .

Ce sacrifice, les grecs l'avaient exhalté.

«~le plus beau sacrifice, c'est de montrer l'homme le plus juste~».
IVème avant JC

Sénèque

Double critique du sacrifice, l'une venant du judaïsme, l'autre du
bassin méditerranéen. Au confluent de ces deux mouvements, Philon
d'Alexandrie.

Jean Laporte, \emph{Le vocabulaire eucharistique chez Philon d'Al}.

Philon~: Dieu n'a besoin de rien. Les sacrifices, pour nous exhorter à
la piété. Nous les \emph{a-charistoi}, sans grâce, nous apprenions à
devenir des \emph{eu-charistoi}. Au plus haut de la hiérarchie, le
sacrifice de Todah\sn{Article Maison Dieu 123 de Cazelles

  De vaux, les institutions de l'Ancien Testaments

  CE sur les sacrifices} est traduit par Philon par le sacrifice
d'eucharistie. Sacrifice de paix, les offrants mangent une partie du
sacrifice, sacrifice de communion. Mais il n'a de valeur, que s'il
manifeste les bonnes dispositions intérieures~: «~il convient de
critiquer ces propres motifs d'Action de grâce et de rendre grâce pour
les raisons les meilleures~(son amour)~».

\textbf{Todah}~:

\begin{enumerate}
\def\labelenumi{\arabic{enumi}.}
\item
   
  Victime
   
\item
   
  offrande végétale (galette, eventuellement vin~; rester prudent sur
  comparaison Eucharistie)
   
\item
   
  Psaume Hallel
   
\end{enumerate}

On supprime 1 pour ne garder que 2 et 3.

Dans traduction aquila, le mot Todah est aussi traduit par eucharistie.

X. Leon Dufour, le partage du pain eucharistique dans le NT

La proclamation de la mort du Christ correspond exactement à la Todah
juive.

Perrot\sn{MD 123} parle de substitution de la todah par
eucharistie.

\hypertarget{b.-mais-quelle-attitude-au-juste}{%
\subparagraph{Mais quelle attitude au
juste~?}\label{b.-mais-quelle-attitude-au-juste}}

Synagogue Lc 4 «~selon son habitude'~».

11 mentions de Jésus au temple mais pas présenté en train de prier mais
en train d'enseigner.

J. Dupont (1978, MD Jésus et la prière liturgique)~: Jésus devait prier
mais en aucun cas un sacrifice.

Perrot note que Mc 11~: objet ou vase, matériel du Temple. Jésus arrête
la marche du culte sacrificiel. Mais difficile de se prononcer sur
l'attitude de Jésus par rapport au Temple (cf applique la loi sur les
purs/impurs~; ne dispense pas de l'offrande à l'autel~; sacrifice en cas
de guérison).

\textbf{Cependant,} l'autorité personnelle de jésus, la nouveauté de son
message (pardonne des péchés, fréquente les pécheurs) était telle que sa
condamnation à mort mettait son message en dehors du sillage des
prophètes.

Cette parole au sujet du Temple (3 jours) a dû être l'un des motifs
réels de sa mort. Chrétiens génés~:

\begin{itemize}
\item
  Suppression en Lc cf faux témoignages sur Etienne~: «~le Temple doit
  être supprimé~».
\item
  Reconstruction chez Mt
\item
  Substitution chez Mc . Mc 14 parle d'une substitution d'un Temple fait
  de mains d'homme à un Temple fait par Dieu.
\item
  Spirituel chez Jn
\end{itemize}

Critère de différence~; critère de cohérence. Cazemann

\emph{Différence}~: cette phrase n'a pu être inventé par les juifs ni
par les premiers chrétiens

\emph{Cohérence}~: cohérence de cette parole avec le reste du message du
Christ et son attitude souvent conflictuelle avec l'autorité du Temple.

Pbt une \emph{ipsissima Verba Jesu}.

\textbf{Annonce un débordement de la critique du CULTE.} Cependant,
cette radicalisation ne sera comprise qu'après Pâques.

\hypertarget{un-nouveau-statut-du-culte-en-christianisme-statut-eschatologique}{%
\paragraph{Un nouveau statut du culte en christianisme~: statut
eschatologique}\label{un-nouveau-statut-du-culte-en-christianisme-statut-eschatologique}}

\hypertarget{le-vocabulaire-cultuel-dans-le-n.t.}{%
\paragraph{le vocabulaire cultuel dans le
N.T.}\label{le-vocabulaire-cultuel-dans-le-n.t.}}

temple, naos,

prêtre, sacerdos,

sacrifice, tusia

Ce vocabulaire existe mais est orienté vers le Christ et non pour les
apôtres, l'eucharistie. He ajoute accomplit DONC abolit le système
sacrificiel du Temple.

Deuxièmement, la vie des chrétiens, le sacrifice des lèvres et la
communion entre frères, la koinonia.

\textbf{Cf document vocabulaire cultuel dans le NT}

\hfill\break
INTRODUIRE Document vocabulaire cultuel

Détournement de vocabulaire.

Différence de vocabulaire avec AT (utilisé pour le Christ ou l'activité
quotidienne des chrétiens~: cf kaseman, Essais exégétiques, le culte
dans la vie quotidienne)

Rm 12,1~: offrande du «~Corps~». 2 Co 9~: Collecte , comme leitourgia
(thusia de la part des Philippiens (Ph 4, 18). Mais aussi en Ep 5,2 où
la mort du Christ comme sacrifice. He~: sacerdoce au Christ

Chrétiens comme proserchomenoi (processionnants) qui s'avancent vers
Dieu. La vie de la communauté chrétienne est ainsi présentée comme un
longue liturgie sacerdotale.

\begin{itemize}
\item
   
  Confession de foi (sacrifice des lèvres)
   
\item
   
  Bienfaisance et de l'entraide communautaire (koinonia)
   
\item
   
  Paul~: mon sacerdoce, la mission
   
\end{itemize}

Ministère du Peuple de Dieu, pas en lien premier avec le problème des
ministères dans l'Eglise, mais avec le ministère de l'Eglise dans le
monde, celle ci est chargée d'une fonction médiatrice, substitutrice,
vicaire et son sacrifice spirituelle est «~d'être auprès du monde
présence de Dieu et devant Dieu présence du monde~».

Importance de l'humilité~: par la confession des fautes et le pardon des
frères, l'assemblée dominicale en vue de l'action de grâces par la
fraction du pain est constituée en sacrifice.

\hypertarget{b.-fondement-de-cette-nouveautuxe9}{%
\paragraph{Fondement de cette
nouveauté}\label{b.-fondement-de-cette-nouveautuxe9}}

\hypertarget{eschatologie}{%
\subparagraph{Eschatologie}\label{eschatologie}}

La mémoire du Christianisme est eschatologique~: c'est une mémoire
d'avenir~: nous faisons mémoire du christ mort, ressuscité et qui
reviendra.

Eschatologie~: différence entre christianisme et judaisme. Ne pas le
réduire au «~pas encore~». Car le pas encore, c'est l'ultime
manifestation de la force ressuscitante du Christ transfigurant dès
maintenant l'humanité par le don de l'Esprit. Certes, douleur de
l'enfantement~: le monde continue de s'éprouver comme non encore
racheté~: «~nous avons été sauvés, mais c'est en espérance~» (rm 8,24)
L'eschatologie pense le Christ ressuscitant le monde, et l'histoire se
présente comme le lieu même de sa possibilité.

Le statut eschatologique du culte en christianisme implique ainsi la
reprise du statut historico-prophétique de celui du judaïsme dont il est
l'héritier.

\hypertarget{la-duxe9chirure-pascale}{%
\subparagraph{La déchirure pascale}\label{la-duxe9chirure-pascale}}

JB~: croit à la venue imminente d'un messie fondateur d'un royaume où,
selon la prophétie d'Ezéchiel, l'Esprit de Dieu sera donné à l'homme
dans une lustration d'eau pure qui le rendrait capable de pratiquer les
commandements, pour une justice jusqu'alors constamment transgressée.(
H. Cazelles, naissance de l'Eglise)

Jr 31

Ez 36,24-28

La métaphore de la déchirure

Une des premières métaphores du changement et de la réalisation des
prophéties d'Ezéchiel et de Jérémie

\begin{itemize}
\item
  les cieux, lors du bapteme
\item
  vieilles outres de la Loi, face à l'Evangile
\item
  vêtement neuf
\item
  déchirure des habits du Grand Prêtre
\item
  déchirure du rideau du temple
\item
  le saint des saints est désormais vide, le temple de la présence de
  dieu, c'est le corps du Ressuscité (Jean) ou l'assemblée des fidèles
  (Paul)
\item
  lettre aux hébreux~: le seul temple, le corps glorifié du Christ, lee
  seul autel, la croix, le seul prêtre et sacrifice, la personne même du
  Christ. \emph{Christ est la seule liturgie possible}
\end{itemize}

La déchirure, vraie question que devait se poser tout juif. Si ce que
vous annoncez Chrétiens, que deviennent la \textbf{Loi} et le
\textbf{Temple}.

\begin{itemize}
\item
  «~la Loi demeure bonne comme expression de la volonté de Dieu mais
  n'est plus un moyen de salut~» Paul cf Galates et Hebreux
\item
  Hébreux dit la même chose au niveau du sacerdoce du Temple~: désormais
  caduque pour accèder à Dieu~; il n'y a plus qu'un seul prêtre.
\end{itemize}

\hypertarget{une-diffuxe9rence-thuxe9ologale}{%
\subparagraph{une différence
théologale}\label{une-diffuxe9rence-thuxe9ologale}}

\textbf{Principe de la justification change~: Ce n'est plus par la Loi
qui sauve, mais celui qui a incorporé la Loi~;}. Ce principe
ChristoPneumatique est nouveau donc la modalité est nouvelle. culte
autre ordre que le culte juif. Non au niveau moral.

\emph{C'est la Foi au Christ qui sauve.} On n'accède pas à Dieu à la
force des poignets. Accueillir le don de Dieu en Christ. Notre existence
devient le lieu de l'accueil du don de Dieu.

Tout part de la confession de Jésus comme Christ. Tout repose donc sur
Pâques et Pentecôte. La différence est eschatologique.

Justification par les oeuvres pour les juifs. Même si pureté de cœur est
recherchée. Christianisme~: développement de cette attitude
eucharistique. Mais par Paul, changement théologique~: c'est le Christ
lui même l'action de grâce du Chrétien et non d'abord sa pratique fidèle
de la Loi ou son cœur droit. Christ seul justifie~: principe différent.

Etre chrétien, c'est vivre sous la loi de l'Esprit, synthèse de Jr 31 et
Ez 36

Justification, non plus sur les œuvres de la Loi, mais la foi en Jésus
comme Christ et Seigneur.

Cœur nouveau, esprit nouveau.

Un nouveau statut du culte

Désormais, \textbf{le lieu du théologal, c'est le lieu de
l'anthropologal.}

Esprit de la Loi~: toujours se hisser vers Dieu à la force du poignet

Esprit de la Foi~: accueillir le salut de Dieu lui même

Nous avons à accueillir le salut dans notre vie historique

Véritable déchirement de la Loi

Le culte premier des chrétiens est celui de l'accueil de cette grâce de
Dieu dans leur vie quotidienne par la foi et la charité théologales~;

SCHEMA VETUSTE NOUVEAUTE

Le nouveau schéma est en trois coté Dieu, Culte liturgique, culte
quotidien par foi et charité

Le Culte ne se trouve pas dans une position intermédiaire mais comme
\emph{révélateur symbolique} de ce qui permet à l'existence humaine
d'être vécue comme existence chrétienne. C'est aussi l'\emph{opérateur
symbolique} rendant possible cet acte sacerdotal et sacrificiel
«~agréable à Dieu~» par le Christ et dans l'ES.

Portée théologique

Adieu au sacrifice~: massif dans le NT~: doit être interpréter au même
niveau que la déchirure eschatologique de Pâques -- pentecôte.
\textbf{Subversion anti sacrificielle et anti sacerdotale.}

Le culte c'est la confession de foi vécue dans l'\emph{agapé} du partage
au service des plus pauvres, de la réconciliation et de la miséricorde.

Mémoire rituelle et mémoire existentielle.

Mémoire rituelle à vérifier dans la mémoire existentielle. Ex~: lavement
des pieds . «~c'est un exemple que je vous ai donné afin que, selon ce
(kathos) que j'ai fait pour vous, vous fassiez vous aussi~». Kathos~:
acte d'engendrement plus que d'exemplarité «~en agissant ainsi, je vous
donne d'agir de même~». Valeur de sacramentum, don de la part du christ.
Se laver les pieds les uns aux autres, c'est vivre existentiellement la
mémoire du Christ que l'eucharistie fait vivre rituellement.

C'est ce qui fait les sacrements, lien entre existence et rite.

«~Mémoire dangereuse~» (JB Metz)

Non seulement parce que la \emph{sequela christi} entraîne tout croyant
sur le chemin crucifiant de la libération, mais parce que cette suite du
Christ est sacramentellement lieu du Christ même continuant d'accomplir
libération pour laquelle il a donné sa vie.

D'où la narration rituelle de ce pour quoi Jésus a donné sa vie

Renvoie les chrétiens à leur responsabilité de prise en charge de
l'histoire de son nom. Vivante mémoire.

\hypertarget{retournement-du-sacruxe9}{%
\subparagraph{Retournement du sacré}\label{retournement-du-sacruxe9}}

Sanctification et non sacralisation (mise à part)

Médiation (corps) et non intermédiaire (de la Loi et du sacrifice)

Pas une critique de la sacralité en tant que telle (différent de 68)
mais de son statut

Distance

Eschatologie inaugurée~: crée une solution de continuité avec le
judaïsme. Car importance de la déchirure n'a été que comprise que
lentement~: «~conflit des herméneutiques~».

Cours du 2/4/03

dans le sillage de cette lecture, Paul écrit «~\emph{la lettre du
Christ, c'est vous, écrit sur vos cœurs. Cette lettre est visible par
tous}~». 2 Co 3, 2-3

La \textbf{sacralisation,} c'est à dire la mise à part pour honorer
Dieu, est remplacé par la \textbf{sanctification~du profane :} puisque
c'est sanctifier, c'est interdit de profanation.

Cf 4\textsuperscript{ème} temps de la messe~: l'ENVOI

A la catégorie d'\textbf{intermédiaire} (le levite), est proposé la
catégorie d'\textbf{intermédiation~;}

Realisation d'Ex 19~: tu seras pour les nations comme les Lévites au
milieu de toi, c'est un sacerdoce royale.

Vous voulez être chrétien~: vivez au quotidien~: soit liturgie, soit
sacrifice spirituel.

Rapport très étroit entre éthique et liturgie.

P. Chauvet, Collectif de l'ISTR, le sacrifice dans les religions.

M. Neusch, le sacrifice chez S. Augustin

\hypertarget{sacrifice}{%
\subparagraph{Sacrifice}\label{sacrifice}}

Notion ambiguë dans le christianisme~: je mange Dieu~: forcément, on est
dans le mécanisme de type sacrificiel~: vocabulaire inévitable mais
ambigu. Mais en fait, le vrai sacrifice, c'est le sien propre, pour la
Gloire de Dieu et le salut du monde~. Le sacrifice ne suffit pas.
\textbf{Il renvoie vers l'éthique.}

\begin{table}[h!]
    \centering
    \sidecaption{  }
 
\begin{tabular}{p{.2\textwidth}p{.2\textwidth}p{.2\textwidth}}
\toprule
Voir

\emph{Dieu n'est pas à voir (idole)} & & \\
& Entendre

\emph{judaïsme} & \\
& & Vivre

\emph{Mais en fait il est à rencontrer dans le vivre} \\
\bottomrule
\end{tabular}
\label{tab:my_label}
\end{table}
 

Il ne s'agit pas d'être meilleur que les juifs.

Rite

Mise en scène du corps

\textbf{Transit (Action de Pâques)}

Lettre corps

Ecriture Existence

Ethique

%-------------------------------------------------------------------------------------------------------------------------------
\chapter{Ce que les sacrements nous révèlent de Dieu}

%-------------------------------------------------------------------------------------------------------------------------------
 



\emph{\textbf{Cf cours écrit}}

Tradition~: Dieu agit par ces actes que l'on appelera sacrement, depuis
la plus haute antiquité.

\textbf{Penser Dieu comme «~humain dans sa divinité~»~.}

\hypertarget{notre-point-de-duxe9part-le-mystere-pascal}{%
\section{Notre point de départ~: Le mystère Pascal}\label{notre-point-de-duxe9part-le-mystere-pascal}}

On ne part pas ici de l'union hypostatique mais de Pâques.

On ne partait pas de la résurrection mais la Résurrection était un objet
comme les autres.

\hypertarget{la-sacramentaire-scholastique}{%
\subsection{La sacramentaire
scholastique}\label{la-sacramentaire-scholastique}}

\hypertarget{la-tradition-ancienne}{%
\subsection{La tradition ancienne}\label{la-tradition-ancienne}}

L'histoire de la rédaction des Evangiles~:

\begin{itemize}
\item
  les récits de l'Enfance sont plus tardifs~: cela ne veut pas dire
  qu'ils sont moins intéressants mais qu'ils n'ont pas aidé à la pensée
  primitive (cela dit, vision pascale~: cf les anges «~Seigneur Jésus)
\end{itemize}

\begin{itemize}
\item
  Noel~: vient de NATALE, manifestation (en grec, epiphanie). A noter~:
  n'est pas dans le Credo avant le moyen age (pas dans Nicée)
\item
  Avant~: Pâques seul fête (et encore ne vient qu'au Iième siècle, avant
  tous les dimanches)~: on n'est pas en registre
  \textbf{d'anniversaire}~: on est en régime de
  \textbf{mémorial}\sn{ne pas rompre le pain en disant les paroles
    `il rompit le pain'~: car on est en mémorial}~!
\item
  La «~cinquantaine joyeuse~»~: désignait ces 50 jours~: un grand
  dimanche sur 50 jours. On a cloturé ce «~dimanche~» de Pâques le jour
  de la pentecôte à partir du Ivème siècle.
\item
  Ascension~: \emph{Anastasis.} Cf Jean~: Je relèverai tout à moi.
\end{itemize}

L'année liturgique n'est pas une grande dramaturgie.

Mystère Pascal

Mort Résurrection Parousie

Les sacrements~: le \textbf{gage de la parousie, le commencement de la
vie éternel.} Nous avons à retrouver chez les orthodoxes, ce sens du
sacrement qui vient de la parousie et non de la mort du Christ.

Moltman~: \textbf{humain dans sa divinité~:} être même de Dieu.

\begin{itemize}
\item
   
  Renvoie à la Croix du Christ
   
\item
   
  Renvoie à l'Esprit
   
\end{itemize}

\hypertarget{discours-sacramentaire-et-discours-christologique}{%
\section{Discours sacramentaire et discours
christologique}\label{discours-sacramentaire-et-discours-christologique}}

Geffré, \emph{Croire et interprêter}

Discussion interreligieux~: ne pas abandonner que Jésus nous donne la
plénitude de la révélation de Dieu

\hypertarget{le-cri-de-juxe9sus-en-croix-mon-dieu-mon-dieu-un-maximum-christologique}{%
\paragraph{Le cri de Jésus en croix (`mon Dieu, mon Dieu')~:
«~un maximum
christologique~»}\label{le-cri-de-juxe9sus-en-croix-mon-dieu-mon-dieu-un-maximum-christologique}}

\begin{itemize}
\item
  éviter le psychologisme~: cf Bourdalou et Bossuet
\item
  une parole scandaleuse pour les premières communautés
\item
  3 points fondamentaux~:
\item
   
  refus de l'hypothèse d'effondrement de Jésus dans la Foi (Bultmann,
  Calvin). Comprendre dans la relation à Dieu car PSAUME. Il s'agit de
  l'abandon du juste par Dieu.
   
\item
   
  Abandon du Fils par le Père (déréliction)~: \textbf{Kénose}~:
  \textbf{forme creuse pour la plénitude de Dieu.}
   
\item
   
  La mort de la cause de Dieu~: qui donc est Dieu~?
   
\end{itemize}

\hypertarget{le-fils-et-le-puxe8re}{%
\paragraph{Le fils et le Père}\label{le-fils-et-le-puxe8re}}

\hypertarget{c.-thuxe9ologie-chruxe9tienne-juxe9sus-comme-le-fils-en-humanituxe9}{%
\paragraph{théologie chrétienne~: Jésus comme «~le Fils en
humanité~»}\label{c.-thuxe9ologie-chruxe9tienne-juxe9sus-comme-le-fils-en-humanituxe9}}

\hypertarget{he-2-passage-remarquable}{%
\subparagraph{\texorpdfstring{He 2~: passage remarquable
}{He 2~: passage remarquable }}\label{he-2-passage-remarquable}}

Jésus s'est fait frère en tout.

Cf fils Prodigue~: il y a un troisième fils qui va chercher le Fils
prodigue pour le ramener au Père

\hypertarget{theleiosis}{%
\subparagraph{\texorpdfstring{Theleiosis~:
}{Theleiosis~: }}\label{theleiosis}}

Accomplissement, perfectionnement

Rite de consécration du Grand Prêtre~; rite du remplissage des mains

Lumière para-doxale~: de biais

Le Centurion~: vraiment, celui est fils de Dieu alors que un homme sur
la Croix~: \textbf{Là est le mystère.} C'est dans ce \emph{sous-homme}
que le Centurion voit le Fils de Dieu. Dieu se révèle dans le plus autre
que lui-même

Kénose

La grâce sacramentelle est \textbf{l'expression} de cette humanité de
Dieu.

\textbf{La Kénose comme théologie centrale de la sacramentaire.}

Il manque un troisième terme , l'Esprit.

\hypertarget{discours-sacramentaire-et-discours-pneumatologique}{%
\section{Discours sacramentaire et discours
pneumatologique}\label{discours-sacramentaire-et-discours-pneumatologique}}

Congar,

\hypertarget{lesprit-muxe9diateur-de-la-diffuxe9rence-et-ainsi-de-la-communication-entre-dieu-et-lhomme}{%
\subsection{L'esprit, médiateur de la Différence et ainsi de la
communication entre Dieu et
l'homme}\label{lesprit-muxe9diateur-de-la-diffuxe9rence-et-ainsi-de-la-communication-entre-dieu-et-lhomme}}

Symbole de l'Esprit-Saint non maîtrisable

To pneuma~: neutre~; Concile de Constantinople~: invente un néologisme~:
to kurion. Ne le nomme pas Dieu pour des raisons politiques mais fait
tout comme.

L'Esprit, c'est Dieu au neutre~; \emph{Anti nom}.

\emph{Révélant non révélé}

Grégoire de Naziance~: Temps de l'Eglise, Temps de l'Esprit.
Simplificateur mais intéressant car l'AT et le NT n'en parle pas
beaucoup en dehors des Actes.

L'Esprit-Saint est moins dans la lettre qu'il est support de la lettre.
Il est ce qui rend vivant la liste.

Il est moins OBJET dont on parle que PRINCIPE qui fait parler~:
CONSPIRATION Vers le Christ.

Augustin~:~

\begin{itemize}
\item
  Locutoire~: ce qui est dit
\item
  Perlocutoire~: effets (\emph{témoignage, martyria)}
\item
  Illocutoire~: rapport de place créé par cette parole (\emph{justice~:
  juste, que ce soit dans le discours sur Dieu, théologique que le
  discours à Dieu (cf Rm 8, 24ss~: nous ne gémissons et l'Esprit porte
  notre prière).}
\end{itemize}

L'esprit est moins dans le contenu de ce qui est dit que dans
l'illocutoire et le perlocutoire. C'est le rôle de l'Esprit-Saint
d'ajuster les paroles à notre existence~: que nos paroles deviennent
VIE~: \textbf{témoignage de la martyria. 2co 3}

L'autre absolu, insaisissable, le plus loin

\textbf{Mais le plus proche, car celui qui somatise~: à la création et
la résurrection et entre les deux, il est le souffle de
\emph{sanctification.}}

Il en est de même pour l'Eglise.

L'Eglise est constamment contesté par celui qui l'a institué, l'Esprit
Saint~: l'esprit a besoin de frontière mais est appelé à les dépasser~!

\hypertarget{lesprit-muxe9diateur-de-la-diffuxe9rence-et-ainsi-de-la-communication-entre-les-hommes}{%
\paragraph{3.2 L'esprit, médiateur de la différence et ainsi, de la
communication entre les
hommes}\label{lesprit-muxe9diateur-de-la-diffuxe9rence-et-ainsi-de-la-communication-entre-les-hommes}}

\hypertarget{lesprit-muxe9diateur-de-la-diffuxe9rence-et-ainsi-de-la-communication-entre-dieu-et-dieu}{%
\paragraph{3.3 L'esprit, médiateur de la différence et ainsi, de la
communication entre Dieu et
Dieu}\label{lesprit-muxe9diateur-de-la-diffuxe9rence-et-ainsi-de-la-communication-entre-dieu-et-dieu}}

Ce mystérieux troisième qui accomplit la différence sans la moindre
séparation.

Marc Richir, le \emph{Corps},

Phénoménologie du corps, très court, beau

«~coapparatenance entre le Corps et la parole~»

\hypertarget{rapport-de-ce-discours-pneumatologique-avec-la-christologique-luxe9ccluxe9siologie-et-la-sacramentaire}{%
\paragraph{3.4 Rapport de ce discours pneumatologique avec la
christologique, l'écclésiologie et la
sacramentaire}\label{rapport-de-ce-discours-pneumatologique-avec-la-christologique-luxe9ccluxe9siologie-et-la-sacramentaire}}

\hypertarget{c.-rapport-avec-les-sacrements.-luxe9picluxe8se}{%
\paragraph{c. rapport avec les sacrements.
L'épiclèse}\label{c.-rapport-avec-les-sacrements.-luxe9picluxe8se}}

sacrement et Esprit-Saint sont indissociable~: Epiclèse indispensable.

Sacrement de réconciliation~: on a un peu oublié l'Esprit-Saint alors
qu'elle est présente dans l'Orient.

Sacrement de mariage

%-------------------------------------------------------------------------------------------------------------------------------
\chapter{Les sacrements de l'initiation Chrétienne}

%-------------------------------------------------------------------------------------------------------------------------------

\textbf{Le 23/4/2003}



\hypertarget{initiation-chruxe9tienne}{%
\section{«~Initiation Chrétienne~»}\label{initiation-chruxe9tienne}}

\emph{Rituel de l'initiation Chrétienne des adultes, 1997}

Impossible de travailler le sujet sans ce document

Vient relayer le premier rituel de 1974, cherche moins à innover mais
est bien plus riche

\hypertarget{intuxe9ruxeat-thuxe9ologique-de-lexpression-d-initiation-chruxe9tienne}{%
\subsection{1. Intérêt théologique de l'expression d'~ «~initiation
chrétienne~»}\label{intuxe9ruxeat-thuxe9ologique-de-lexpression-d-initiation-chruxe9tienne}}

Oublié

1946~: redécouverte de l'initiation chrétienne
 3 rituels~:
\begin{itemize}

 
\item
  Bébé, avant cheminement personnel

\item
  Enfants d'âge scolaire (après 7 ans), avec cheminement personnel
 
\item
  Adulte, avec cheminement personnel

\end{itemize}

\hypertarget{un-ensemble-dynamique}{%
\paragraph{un ensemble dynamique}\label{un-ensemble-dynamique}}

«~Baptême -- Confirmation -- Eucharistie~» comme un ensemble~:

Finalement, qu'UN seul SACRAMENTUM, ces trois gestes~: Baptême --
Confirmation -- Eucharistie~

Mais ensemble dynamique~: on va du baptême à l'Eucharistie~: «~Engrangé,
\ldots{} imbibés d'eau pour devenir une seule pate, cuisson du S.
Esprit.\\

 \begin{quote}
      «~Soyez ce que vous voyez, recevez ce que vous êtes\sn{soyez
  (ecclésialement) ce que vous recevez (Eucharistiquement) et recevez
  (Eucharistiquement) ce que vous êtes (ecclésialement)}~». \sn{S. Aug. Sur
le catéchuménat. Sermon 229, 272}  
 \end{quote}
 
 

\begin{quote}
    «~Les petits enfants communient spirituellement le jour de leur
baptême~» S. Thomas
\end{quote}


On ne peut pas être baptisé sans être tendu vers l'Eucharistie~: d'où
Baptême se termine autour de l'autel~: \textbf{indique le déficit de
l'Eucharistie.} Ils ont reçu la \emph{res.}

\textbf{Un peu de souplesse.}

VII , rituel

«~par les sacrements de l'initiation chrétienne, reçoivent l'esprit
d'adoption et célèbrent l'Eucharistie.

Par le Baptême,\ldots{}

Dans la Confirmation,\ldots{}

Enfin, l'Eucharistie, \ldots~»

D'où importance de réviser le rituel de la confirmation, pour le relier
avec toute l'initiation chrétienne~.

\textbf{Attention à ne pas réduire à des étapes psychologiques} (qui
ferait que l'on sépare la communion et le baptême pour les enfants).

\hypertarget{sens-strict-et-sens-large}{%
\paragraph{Sens strict et sens
large}\label{sens-strict-et-sens-large}}

\hypertarget{sens-strict-on-est-initiuxe9-par-les-sacrements}{%
\subparagraph{sens strict~: on est initié par les
sacrements}\label{sens-strict-on-est-initiuxe9-par-les-sacrements}}

C'est le sens des Pères, cf~\emph{mystères mystagogiques}

Lire S. Ambroise~; \emph{des sacrements} et \emph{des mystères}

Lire Cyrille de Jérusalem

Lire Suzanne Poc sur S. Augustin

P. Gy~; \emph{la notion chrétienne d'initiation}. Montre bien que le
moment où l'on passe de non initié à initié, c'est la célébration du
batpême et l'~Eucharistie.

Rituel 41~: B.E.C. est le sacrement de l'initiation

Initiation~: première participation à la mort et résurrection du Christ.

\textbf{Avantage~:} Initiation comme fruit de la grâce de Dieu.

\textbf{S.} Jean Chrysostome, SC 50, p°~: \emph{«~le Christ est là qui
t'initie par l'eau et l'Esprit} ».

\hypertarget{sens-large-on-est-initiuxe9-aux-sacrements}{%
\subparagraph{sens large~: On est initié aux
sacrements}\label{sens-large-on-est-initiuxe9-aux-sacrements}}

Sens contemporain.

On initie aux sacrements.

Rituel P°202~: B.E.C «~ultimus Gradus~» de l'initiation.

\textbf{Avantage~:} montre l'épaisseur temporel~; dimensions diverses
doctrinales, liturgiques, morales~: conversion des mœurs, initiation aux
textes chrétiens.

Ne pas oublier le premier terme.

\hypertarget{initiation-et-mystuxe8res}{%
\subsection{«~initiation~» et
«~mystères~»}\label{initiation-et-mystuxe8res}}

\hypertarget{le-vocabulaire-mystuxe9rique}{%
\paragraph{le vocabulaire
mystérique}\label{le-vocabulaire-mystuxe9rique}}

Vocabulaire initiatique lié au vocabulaire mystérique~: On est
\emph{initié} aux \emph{mystères} sous forme de \emph{sacrement.}

Rappel~: Mystère~: ce dans quoi plus j'avance, plus cela me donne à
vivre~; \emph{inépuisable.}

 vient de la sagesse grec et apocalyptique juive (Daniel)~:
REVELATION à quelques privilégiés~du dessein caché de Dieu, souvent
grâce à des visions.

Dans le monde païen, a trait~:

\begin{itemize}
\item
   
  aux cultes à mystère
   
\item
   
  Initiation à la philosophie
   
\end{itemize}

\hypertarget{dans-le-nt}{%
\paragraph{Dans le NT}\label{dans-le-nt}}

Mt 13,11

Paul~: Ep 3

Mystère~: dans l'AT, mystère de Dieu devient ici mystère du Christ~:
Christ n'est pas seulement le révélateur du mystère de Dieu, il est
aussi le mystère de Dieu.

\textbf{Mystère du Christ~:} il faut s'étonner de cette expression~:
formidable

\textbf{Mystère, c'est Christ au milieu de vous} Co 1, 27

Dieu -\textgreater{} Christ -\textgreater{} Eglise~: amour sponsal du
Christ~: mega musterion -\textgreater{} Ministère~: pas d'autre objet
que l'annonce du mystère de l'Evangile (Ep 6)

Mais mystère pas très employé pour ne pas confondre avec les cultes à
mystère (cf Justin~: mystères utilisés démoniaquement par les religions
à mystère)

A partir du IV~: OK

\emph{Cathéchèses mystagogiques}~: on commente les mystère que l'on a
déjà reçu~: La mystagogie se fait toujours après coup (pas d'explication
a priori mais a posteriori, reconnaître ce qui a été vécu pour en
déployer le sens)

\emph{On se fit beaucoup mieux à la vue qu'à l'ouie}.

Notion qui invite à les vivre d'abord, pour s'appuyer dessus et ensuite
d'essayer de la comprendre.

S. Ambroise.

\textbf{L'expérience précède l'explication.}

\hypertarget{chez-les-puxe8res}{%
\paragraph{Chez les Pères}\label{chez-les-puxe8res}}

L'annonce du mystère de l'Evangile est publique.

Le mot mystère n'invite pas à garder pour soi mais à annoncer.

L'entrée dans le mystère du Christ n'est pas lié à la \emph{personne} du
mystagogue ni à celle du maître. Fruit de la grâce de Dieu (Ep.3,2).

\hypertarget{initiation-pauxefenne-et-initiation-chruxe9tienne}{%
\subsection{Initiation païenne et initiation
chrétienne}\label{initiation-pauxefenne-et-initiation-chruxe9tienne}}

intérêt de ce détour pour la \textbf{pédagogie.}

\hypertarget{linitiation-dans-les-cultures-traditionnelles}{%
\paragraph{3.1 l'initiation dans les cultures
traditionnelles}\label{linitiation-dans-les-cultures-traditionnelles}}

Pas un modèle unique. Mais des points communs~:

\begin{itemize}
\item
  3 étapes~:
\item
   
  Mise à l'écart~: Rupture
   
\item
   
  Retraite dans le bois sacré~: stage
   
\item
   
  Réintégration des initiés avec un nouveau statut, un nouveau nom
   
\end{itemize}

Article de Basselide dans l'Euniversalis

RITE de PASSAGE~:

\begin{itemize}
\item
  passage de l'enfance à l'adulte~: on devient mariable
\item
  la nature (puberté) n'est pas laissée seule et intégrée socialement
\item
  Epreuve de mort et renaissance
\end{itemize}

2.36 initiation comme épreuve de renaissance

Vécu à 3 niveaux~:

\begin{itemize}
\item
  au niveau de la communauté, qui transmet ses valeurs fondatrices
  (atmosphère particulière en cas d'initiation)~: L'initiation est
  d'abord \textbf{transmission de Tradition.}
\item
  Naissance du Groupe d'initiés comme tels~: généralement, ce sont des
  groupes comme tels. Le Groupe franchit un passage et une solidarité de
  Groupe qui joue.
\item
  au niveau de corps de chacun, réenfantement. Risque de dérive de
  sadique. Il s'agit d'un passage par la mort.
\end{itemize}

Le terrain où la parole initiatique est ensemencée.

Intégration de la culture du Groupe.

On y apprend le respect des traditions, en respectant les anciens.

Elle fonde un savoir être. C'est en faisant que l'~on apprend.

Ajustement de soi \textbf{aux autres, aux ancêtres~: on apprend à
trouver sa place en mettant chacun à sa place.} Immense Puzzle.

Initiation~: entrer en humanité~; beaucoup plus performant que l'école.

\hypertarget{intuxe9ruxeat-de-ce-duxe9tour-ethnologique}{%
\paragraph{Intérêt de ce détour
ethnologique}\label{intuxe9ruxeat-de-ce-duxe9tour-ethnologique}}

Pas de regret de l'initiation~: \textbf{fonctionne en monde fermé et non
ouvert.}

La communauté reçoit sa tradition en la transmettant. L'entrée dans le
mystère du Christ. Pas d'autre moyen que de se laisser prendre en lui~:
impossible de prétendre voir sur le mode du savoir. Il faut être
concerné.

Efficace si elle est un processus globale, pas seulement intellect mais
au cœur, au desir, à la mémoire.

Ce qui est transmis, c'est aussi l'Eglise d'hier mais aussi celle
d'aujourd'hui. on en peut initier qu'~en étant adossé à la grande
Eglise.

\hypertarget{les-initiuxe9s}{%
\subparagraph{Les initiés}\label{les-initiuxe9s}}

Ils progressent ensemble, faire corps~; franchir ensemble les
différentes étapes.

\hypertarget{corps-de-chacun}{%
\subparagraph{Corps de chacun}\label{corps-de-chacun}}

Il faut un minimum d'exercice de mémorisation (prières, geste)~: c'est
au niveau du corps que cela se passe.

L'initiation n'est pas au bout d'une démarche intellectuelle où l'on
aurait tout compris.

\hypertarget{il-faut-bien-faire-apparauxeetre-la-diffuxe9rence-chruxe9tienne}{%
\paragraph{Il faut bien faire apparaître la «~différence~»
chrétienne}\label{il-faut-bien-faire-apparauxeetre-la-diffuxe9rence-chruxe9tienne}}

\begin{table}[h!]
    \centering
    \sidecaption{  }
 
\begin{tabular}{p{.4\textwidth}p{.4\textwidth}}
\toprule
\textbf{Pôle attestaire} & \textbf{Pôle Contestataire} \\
 \midrule
Pôle unique de l'initiation traditionnelle.\\
\textbf{Support institutionnel} 
 & \\
Héritage & Liberté antique \\
Marques d'appartenance -- Particularité & Ouverture à l'universel
(chrétiens, pas un ghetto, pas umma musulmane) \\
Temps Marqué (celui qui a tout reçu~: «~il est chrétien~»). & Tu l'es si
tu le deviens sans cesse. Devenir incessant. \textbf{Rôle de l'appel} \\
Initiateurs & Humblement accompagnateurs~: eux mêmes évangélisés par
ceux qu'ils évangélisent. \\
\bottomrule
\end{tabular}
\label{tab:my_label}
\end{table}
 

Force et faiblesse du Christianisme.

Si l'on oublie un pôle, cela ne marche pas.

\emph{\textbf{Le fait de sentir l'inconfort est un bon signe de
santé~!}}

\hypertarget{la-confirmation-vs-baptuxeame}{%
\paragraph{La confirmation vs
baptême}\label{la-confirmation-vs-baptuxeame}}


\begin{table}[h!]
    \centering
    \sidecaption{  }
 \footnotesize
\begin{tabular}{p{.25\textwidth}p{.25\textwidth}p{.25\textwidth}}
\toprule
Avant cheminement & \textbf{Qui~?} & Après cheminement \\
Tradition & & «~mystérique~» (quelques uns) \\
\textbf{Au nom de quoi~?} & & \textbf{Quel type d'initiation~?} \\
Conversion & & Socio-Tribale (tous) \\
Confessante & \textbf{Quelle figure~?} & Multitudiniste \\
\\
\bottomrule
\end{tabular}
\label{tab:my_label}
\end{table}


Théologiquement, le baptême est plus important que la confirmation.
Cependant, à cause de l'importance de la liberté antique, la
confirmation est renforcée (cela devrait être l baptême mais qui est
devenu sacrement de Chrétienté).

\hypertarget{le-baptuxeame}{%
\section{Le Baptême}\label{le-baptuxeame}}

Le baptême d'adultes, figure exemplaire du baptême.

\hypertarget{les-deux-piliers-majeurs-du-baptuxeame}{%
\subsection{Les deux «~piliers~» majeurs du
baptême}\label{les-deux-piliers-majeurs-du-baptuxeame}}

\hypertarget{le-mystuxe8re-pascal-du-christ}{%
\paragraph{Le mystère pascal du
Christ}\label{le-mystuxe8re-pascal-du-christ}}

Baptême, c'est Pâque.

Cf. baptême à Pâques (veillée pascale) ou le dimanche (Pâque, chaque
dimanche cf Mystère Pascal p°36).

\hypertarget{la-communautuxe9-eccluxe9siale}{%
\paragraph{La communauté
ecclésiale}\label{la-communautuxe9-eccluxe9siale}}

Recommandé de regrouper les baptêmes~: entrée dans l'Eglise et donc acte
communautaire.

On demande que ce soit fait dans l'Eglise paroissiale.

\textbf{Le 30/04/03}

\hypertarget{b.-noyau-central-du-baptuxeame-passage}{%
\section{Noyau central du Baptême~:
passage}\label{b.-noyau-central-du-baptuxeame-passage}}

Le Baptême est un passage d'une situation négative à une situation
positive~: renonciation et profession de foi ne font qu'un seul rite.

\hypertarget{renonciation-uxe0-satan}{%
\paragraph{Renonciation à Satan}\label{renonciation-uxe0-satan}}

III-IV siècle

Moment poignant de la démarche des catéchumènes~: s'engager à s'éloigner
solennellement~:

\begin{itemize}
\item
   
  attrait du théâtre
   
\item
   
  superstition
   
\item
   
  jouissance trompeuse
   
\item
   
  doctrine des hérésiarques
   
\item
   
  TRES CONCRET
   
\end{itemize}

R. CABIE Tome III l'Eglise en Prière

A lire

\textbf{Sacrement Serment au plus strict du terme.} Un sacramentum pour
Tertulien, c'est un Serment, on s'engage~: on dépose un sacramento.

Si on rompt le serment, s'engager dans la \emph{militia christi,}

Jean Chrysostome~: suntheke, diatheke

On ne dit pas que l'on renonce au péché~: on demande à lutter contre le
péché, dans le sillage de l'arrachement du péché par le Christ~: d'un
\textbf{Royaume à l'autre.}

Décodeur~:

Le péché règne là où règne l'Egoïsme~; là où règne l'argent roi

Renoncez vous à Satan~?

PASSAGE

\hypertarget{profession-de-foi}{%
\paragraph{Profession de Foi}\label{profession-de-foi}}

IV --V siècle

Rome -- Milan -- Afrique du Nord

 
Crois tu en Dieu le Père-- Plouf

Crois tu en Christ -- Plouf

Crois tu en l'Esprit Saint-- Plouf
 

\emph{On est littéralement plongé dans la Foi de l'Eglise.}

Dommage que cette pratique ne soit plus en cours. Le geste vient
visibilité de la parole

Théodore de mopsueste / chrysostome

«~Est baptisé un tel au nom du père, \emph{plouf}, du fils,
\emph{plouf}, du Esprit-Saint , \emph{plouf}~».

Passif divin~: est non le prêtre «~je te baptise~».

On est baptisé dans la foi de l'Eglise et non dans la foi des
personnes~: d'où le Dialogue~: crois tu~?

C'est cela qui est demandé.

MD 207, LMC \emph{Quelle est la Foi que l'Eglise demande aux parents~?}

\hypertarget{les-eaux}{%
\subsection{Les eaux}\label{les-eaux}}

Gilbert Durand, \emph{Eaux,} Encyclopédia Universalis

5 volets~:

\begin{itemize}
\item
   
  l'eau source de vie
   
\item
   
  l'eau médicale (Lourdes,\ldots)
   
\item
   
  l'eau baptismale (passage)
   
\item
   
  l'eau lustrale (purification, aspertion)
   
\item
   
  l'eau diluviale~: régénération de l'humanité (baptême)
   
\end{itemize}

\textbf{Ambivalence}

Mort
 
    «~Plongeant dans l'eau comme un sépulcre, le vieil homme est enseveli
tout entier et quand nous sortons, l'homme nouveau apparaît~»\sn{Chrysostome}
 



Rm 6

Caractéristique du Judeo-Christianisme

On les réfère à l'histoire~: Exode~; mort et résurrection de Jésus.

Danielou, \emph{Bible et liturgie}

Lire deux chapitres absolument

Pères de l'Eglise~: flanerie patristique~; \textbf{lecture cordiale}

Rite baptismal

- La création

- La mer Rouge

- Le franchissement du Jourdain (JoSue~: mêmes consonnes que JeSu)

\hypertarget{cruxe9ation}{%
\subparagraph{Création}\label{cruxe9ation}}

Très fréquemment chez les Pères, référence à la Création du monde

Esprit Planant/ Eaux fécondent

Chant~: Rimaud~;

Baptise dans l'eau

\hypertarget{la-mer-rouge}{%
\subparagraph{La mer rouge}\label{la-mer-rouge}}

1 Co 10 «~les pères baptisés dans la mer Rouge~»

Egypte = humanité pécheresse

Terre Promise = humanité sauvée en Christ

S. Ambroise, Traité des sacrements, 12

«~extraordinaire, peuple juif à travers la mer~; baptême. Chacun y fait
son passage~; \ldots{} de la souillure à la sainteté~»

\hypertarget{la-buxe9nuxe9diction-de-leau}{%
\paragraph{La bénédiction de
l'eau}\label{la-buxe9nuxe9diction-de-leau}}

Prière romaine~: V siècle

Aspect exorciste

Epiclèse en bonne et due forme

\hypertarget{rapport-eau-esprit}{%
\paragraph{Rapport eau / Esprit}\label{rapport-eau-esprit}}

Tertullien~: v. 200 Traité du baptême 4,4

Impréniation, \emph{Energia, Teiopoinese,} transformation de la matière

Pas de baptême sans eau sanctifié

Ambroise, des Sacrements et des mystères

A lire

L'efficacité vient de l'Esprit-Saint

Meta = Trans (former)

P. Siman, l'expérience de l'Esprit selon Tradition syrienne d'Antioche

Epiclèse sur le pain, l'eau et le S. Chrème (muron)

Comment se fait il que l'on jette l'eau baptismal~?

\hypertarget{baptistuxe8re}{%
\paragraph{Baptistère}\label{baptistuxe8re}}

Baptistère de Poitiers

\hypertarget{rite-du-baptuxeame}{%
\paragraph{Rite du Baptême}\label{rite-du-baptuxeame}}

Immersion d'abord

Dans l'Antiquité

Source Chrétienne 366, porté en note,

Trois cathéchèses baptismales

Grande diversité~; en particulier, en cas de baptême de milliers de
personnes

On descendait d'un coté, on remontait de l'autre.

Immersion ou inondé

On est baptisé par un autre~; qu'importe la dignité du ministre~; qu'il
soit Pierre ou Judas.

\hypertarget{les-rites-post-baptismaux}{%
\subsection{Les rites post baptismaux}\label{les-rites-post-baptismaux}}

Dépouillement du vieil homme et habillement de l'homme nouveau~:
vêtement blanc (lumière).

Lumière~: moins ancien~; fin XI~; les catéchumènes~: les illuminés. Mais
le rite de la lumière postérieur.

\hypertarget{eucharistie-baptismale}{%
\subparagraph{\texorpdfstring{Eucharistie baptismale~:
}{Eucharistie baptismale~: }}\label{eucharistie-baptismale}}

Tant qu'on n'est pas baptisé, on ne participe pas à la Prière des
fidèles~; d'où l'envoi (Missa).

Dès le baptême, on participe à l'Eucharistie~: on apporte les dons.

III -- Cyprien~: reproche à une riche carthaginoise de ne rien apporter
et de partager ce que le pauvre a apporté.

Participe donc à la procession des DONS.

Ps 22~; le Seigneur est mon berger.


\section{C. Réflexion Sur la Confirmation }

\hypertarget{situation-paradoxale}{%
\subsection{Situation paradoxale}\label{situation-paradoxale}}

Gros effort pastoral sur la Confirmation.

Importance de la liberté

Mais d'un point de vue théologique~:

\begin{itemize}
\item
  soit discours très clair
\item
  soit discours plus complet mais alors parle-t-on encore du même
  sacrement~?
\item
   
  Confirmation post Eucharistie pose problème
   
\end{itemize}

Pas de scandale d'une pratique où les théologiens ne proposent pas
d'explication satisfaisante. C'est la même chose pour le mariage et la
réconcialiation~: d'une grande richesse.

\hypertarget{un-uxe9luxe9ment-de-linitiation-chruxe9tienne}{%
\subsection{Un élément de l'Initiation
Chrétienne}\label{un-uxe9luxe9ment-de-linitiation-chruxe9tienne}}

\begin{itemize}
\item
  Important en particulier dans le dialogue œcuménique.
\item
  Un élément du baptême
\item
   
  Le rite rattaché au don de l'Esprit-Saint était lié au baptême~: pas
  d'existence indépendante dans les premiers temps.
   
\item
   
  Il est cependant significatif que ce rite soit commenté de façon
  différente~: lié à l'Esprit-Saint~: à la fois il ne fait qu'un avec le
  baptême, à la fois séparé EAU/ESPRIT.
   
\item
   
  UN PARMI D'AUTRES~; MAIS un RITE avec une importance particulière à
  cause de l'Esprit-Saint.
   
\end{itemize}

- 4.21 Don Bote, Lumière et Vie 54

Baptême

5è siècle, en Gaule, Confirmation~, rite post baptismal, en lien avec le
Don de l'~Esprit-Saint. \textbf{Nommer} veut dire détachement progressif
d'avec le baptême. Même quand le rite est séparé, ce serait un
contre-sens que de chercher autre chose qu'une affermir au sens
d'achever. Cela n'ajouter rien au baptême.

Alquin, «~après la communion au pain du Christ, on confirme à la
communion au sang~».

\begin{itemize}
\item
  3 leçons~:
\item
   
  VII, confirmation, sens plein en lien avec Baptême et Eucharistie.
   
\item
   
  Même si elle est reçue longtemps après le baptême, elle n'a de sens
  qu'immédiatement après le baptême~: reprise du baptême.
   
\item
   
  La confirmation est moins importante que le Baptême théologiquement.
   
\end{itemize}

\hypertarget{un-rite-qui-a-variuxe9}{%
\subsection{Un rite qui a varié}\label{un-rite-qui-a-variuxe9}}

Doublement varié~:

- selon Eglise

- selon époque

\hypertarget{pas-la-muxeame-place-dans-le-rituel}{%
\paragraph{Pas la même place dans le
rituel}\label{pas-la-muxeame-place-dans-le-rituel}}

\begin{itemize}
\item
  Avant~: Corneille reçoit le Baptême après l'Esprit-Saint (Ac 10)
\item
  Pendant~: Ac 2,38~; baptême dans l'Esprit-Saint d'Ac 1,5
\item
  Après~: Ac 8 Samarie~; imposer les mains.~; expression ecclésiale.
\item
   
  Ac 19~: Johanite d'Ephèse (pas entendu parlé de l'Esprit-Saint).
   
\end{itemize}

Cette diversité se retrouve dans les Traditions postérieures~:

\begin{itemize}
\item
  Avant~: Orient Syrien / S. Ephrem~: pour recevoir le Christ, il faut
  avoir reç l'Esprit-Saint
\item
  Pendant~: Chrysostome~: au moment de la plongée baptismale (SC 50),
  par les paroles du prêtre et par ses mains (qui plongent), que
  survient le don de l'Esprit-Saint.
\item
  Après~: la plus commune, en Orient hors Syriaque~; occident. Après le
  Baptême que se fait le don de l'Esprit-Saint.
\end{itemize}

\hypertarget{le-rite-lui-muxeame-pas-partout-le-muxeame}{%
\paragraph{Le rite lui --même Pas partout le
même}\label{le-rite-lui-muxeame-pas-partout-le-muxeame}}

Ac 8~; Ac 19 L'imposition des mains

 
2 Co 1,21-22 L~`onction et le sceau (sphragis)
 

\hypertarget{onction}{%
\subparagraph{Onction}\label{onction}}

P. Cazelle, \emph{le messie de Dieu,}

Le Roi était intronisé par une onction, imprégné de la puissance de Dieu
, qu'il allait devoir représenter devant son Peuple.

Confirmation~: Imprégnation d'une onction d'huile, du Christ.

D'abord une imposition des mains

Puis l'onction avec l'huile

Puis le sceau (le signe de croix sur la terre)

Enfin amalgamage rituel des 3.

\hypertarget{petite-thuxe9ologie-de-la-confirmation-telle-que-donnuxe9e-situxf4t-le-baptuxeame}{%
\subsection{Petite théologie de la confirmation telle que donnée
sitôt le
baptême}\label{petite-thuxe9ologie-de-la-confirmation-telle-que-donnuxe9e-situxf4t-le-baptuxeame}}

\hypertarget{a.-imposition-des-mains}{%
\paragraph{imposition des mains}\label{a.-imposition-des-mains}}

Onction des mains collectives

\hypertarget{b.-s.-chruxeame}{%
\paragraph{s. chrême}\label{b.-s.-chruxeame}}

Rite principal~: onction du S Chrème

\begin{itemize}
\item
  bonne odeur
\item
  imprégnation, symbolique la plus importante
\item
  abondante mais diversifié
\item
   
  «~coulait sur le corps~» Tertullien
   
\item
   
  «~sur la tete~»
   
\item
   
  «~le front, les narines, les oreilles~» Cyrille de Jérusalem
  mystagogique 3,4
   
\end{itemize}

On est alors représentant du CHRIST.

Cf 3\textsuperscript{ème} mystagogique~: «~empreinte de l'Esprit-Saint~;
arrive aux chrétiens en réalité, ce qui arrivait en figure à Aaron ,
prêtre, prophète et Roi~».

\hypertarget{c.-le-sceau}{%
\paragraph{Le sceau}\label{c.-le-sceau}}

grec~: sphragis ,mais aussi charactger~; latin~: signaculum

\textbf{Onction sous forme de sceau~:} reçois le sceau du Don du
Esprit-Saint~; «~sois marqué de l'Esprit-Saint, le don de Dieu~».

\textbf{Sceau~:}

\begin{itemize}
\item
  marque officielle~; seule la personne habilitée peut le faire. Le
  porteur du capital symbolique de l'Eglise~: l'\,'\textbf{évèque}
\item
  appartenance définitive
\end{itemize}

\hypertarget{d.-luxe9vuxe8que}{%
\paragraph{{l'évèque }}\label{d.-luxe9vuxe8que}}

ministre originel ou ordinaire de la confirmation.

Rôle central dans la confirmation

Garant de l'apostolicité de l'Eglise dont il a la charge.

Evèque~: pose le sceau = autonomie

Appartenance de celui qui se joint au troupeau du Christ

L'évèque est le ministre \textbf{originel et ordinaire} de la
confirmation.

↓

Rôle central dans la confirmation

Car ministre de la communion et garant de l'apostolicité de l'Église
dont il a la charge.

\hypertarget{e.-essai-de-synthuxe8se-thuxe9ologique-du-rapport-sceau-uxe9vuxeaque-esprit}{%
\paragraph{Essai de synthèse théologique du rapport sceau/
évêque/
Esprit}\label{e.-essai-de-synthuxe8se-thuxe9ologique-du-rapport-sceau-uxe9vuxeaque-esprit}}

Eglise

Sceau Esprit

Evêque

Juste après le rite de l'eau, les «~baptisés~» passent entre les mains
de l'évêque → achever, parfaire le Baptême, ie scellé officiellement
l'authenticité écclésiale du Baptême.\\
On ne devient Chrétien qu'en passant entre les mains de l'évêque comme
chef de l'Église locale.

Sceau → comme «~tampeau d'identité Chrétienne~» (Dimension juridique)

→ \textbf{Œuvre de l'Esprit Saint} donc pas seulement juridique mais
valeur sacramentelle.

L'Esprit Saint a pour rôle d'\textbf{intégrer dans l'Église} .( cf n°25
du rituel de la confirmation~: la confirmation achève car fait entrer
pleinement dans la communauté de l'Église)

Confirmation = dimension pneumatologique du baptême

Cf Concile d'Arles -- 314

On ne rebaptise pas un baptisé dans une Église schismatique, mais on lui
impose les mains pour qu'il reçoive l'Esprit Saint , soit qu'il intègre
dans la pleine communion de l'Église.

\hypertarget{la-suxe9paration-de-la-confirmation-davec-le-baptuxeame-en-occident}{%
\subsection{la séparation de la confirmation d'avec le Baptême en
occident}\label{la-suxe9paration-de-la-confirmation-davec-le-baptuxeame-en-occident}}

Pour des raisons pratiques

Il arrivait qu'un prêtre, un diacre, même un laïc baptise quelqu'un en
danger de mort.

Si celui-ci retrouve la santé, son Baptême doit être \emph{«~complété
par l'imposition des mains~»} par l'évêque. (concile d'Elvire vers 300
canon 38)

Ce qui était exceptionnel au départ devint plus courant avec
l'Evangélisation des campagnes (accélérée en occident par la disparition
des villes~suite à l'invasion des Barbares !~)~: les évêques envoient
des presbytres gérer les «~paroisses~» naissantes.

Ce forment des communautés rurales, éloignées du siège episcopal.

Les évêques leur demandent de leur envoyer les baptisés pour
confirmation.

 
 
\paragraph{  stratégies possibles~:} 

→ \textbf{orient~:} maintenir l'\textbf{unité} de la célébration
(Baptême -- Eucharistie -- Confirmation). Lien ave l'évêque grâce au S.
Chrême

→ \textbf{occident~:} privilégie le passage entre les mains de l'évêque.
Pb~: on dédouble l'onction au Baptême et à la confirmation

Les deux «~stratégies~» ont leur cohérence

\hypertarget{consuxe9quences-de-cette-suxe9paration}{%
\subsection{Conséquences de cette
séparation}\label{consuxe9quences-de-cette-suxe9paration}}

\hypertarget{a-lordre-des-sacrements-de-linitiation}{%
\paragraph{l'ordre des sacrements de
l'initiation}\label{a-lordre-des-sacrements-de-linitiation}}

Avant le IX°, on donnait l'Eucharistie (y compris aux bébés) le jour du
Baptême (une goutte de vin consacré)

Cf S Augustin~: si vous ne buvez pas le sang du fils de l'homme~:
Eucharistie~: nécessaire au salut

Baptême -- Eucharistie -- Confirmation~(→ quand l'évêque passait (s'il
passait~!))

Ce bouleversement de l'ordre n'était pas théorisé

Théologiquement, toute personne baptisable est confirmable et
eucharistiable~!~!

L'Église latine demande qu'on ait l'âge de raison pour communier mais
c'est seulement disciplinaire (à partir du 9°)

Après séparation Baptême -- Eucharistie -- Confirmation, on a retrouvé
l'ordre initial (au moins dans les traités de sacrements)

1910~: décret de Pie X~: \emph{quam singulam} veut favoriser la
communion dès 7 ans~: mis en valeur contre des pratiques jansénisantes

→ dans un certain nombre de pays, on confirme très jeune (2-3 ans)

→ dans d'autres pays, cela a bouleversé l'ordre traditionnel

\hypertarget{limites-mais-intuxe9ruxeat}{%
\subsection{Limites mais intérêt}\label{limites-mais-intuxe9ruxeat}}

Cours du 14/5/03

→ Evite de faire de la confirmation une simple profession de foi.

Erasme (avant luther) écrit sur la \textbf{confirmation} considérée + ou
-- comme profession de foi (optique de la Renaissance).

Confirmation = vue comme sacrement → vécue comme Don de Dieu

\hypertarget{proposition-pastorale}{%
\subsection{Proposition pastorale}\label{proposition-pastorale}}

\hypertarget{adultes}{%
\paragraph{Adultes}\label{adultes}}

Possible de réstaurer l'ordre ancien comme le demande le rituel~:
exemplaire sinon normatif.

Beaucoup d'évêques en France notamment ont jugé comme raison grave pour
reporter 1 an après le Baptême en raison des \textbf{difficultés des
néophytes à trouver leur place} dans la paroisse\ldots{}

De fait, il y a une difficulté mais faut il utiliser un sacrement pour
tenter de le résoudre~?

Cf LMD 211, 4.37

Interprétation souvent utilisée

Il faut un sas d'un an pour se sentir à l'aise dans l'Église →
\emph{année mystagogique} qui pourrait conduire au sacrement de la
réconciliation (comme replongée dans le Baptême ).

Le vocabulaire est en place~:

\begin{itemize}
\item
  augmentation de grâce
\item
  force pour lutter\ldots{}
\item
  ie sacrement de la \emph{militance}
\item
   
  ascétique (lutte contre les vices)
   
\item
   
  militante~: oser témoigner du Christ \textbf{devant tous} sans rougir
   
\end{itemize}

On alors une théologie \textbf{trop} claire~! et on sacrifie la
tradition ancienne.

Le succès de cette homélie est qu'elle fut attribuée à un pape (fausses
décrétales du 9\textsuperscript{ème} siècle).

\textbf{Christ~:}

Pâques $\leftarrow$ $\rightarrow$ Pentecôte

Déploiement de Pâques pour les disciples~:l \emph{prise de corps de} la
force du ressuscité dans l'histoire → Eglise comme Œuvre de l'Esprit
Saint.

Pour le Christ, la Pentecôte n'ajoute rien à Pâques~!

\textbf{Par analogie,}

\textbf{Chrétien:}

Baptême $\leftarrow$ $\rightarrow$ Confirmation

Déploiement de la portée de ce que le Baptême a fait de nous~: ie
\textbf{membrs du corps du Christ . → dimension ecclésiale et
missionaire}

Pour le Chrétien, n'ajoute pas quelque chose

Ce qui pose question ce n'est pas tant la \textbf{séparation} comme
telle, c'est \textbf{l'ordre~! Eucharistie} avant la
\textbf{confirmation.}

Quand on reçoit le corps Eucharistique du Christ, on est pleinement
intégré à son corps écclesial.

\textbf{Pb~:} si on n'a pas reçu la marque de l'Esprit Saint qui fait
l'Eglise.

\hypertarget{b.-lesprit-saint-au-baptuxeame-et-uxe0-la-confirmation}{%
\subparagraph{L'Esprit Saint au Baptême et à la
Confirmation}\label{b.-lesprit-saint-au-baptuxeame-et-uxe0-la-confirmation}}

La question se pose dès la séparation.

\textbf{Baptême dans l'eau et l'Esprit.} Pourquoi «~bisser~»~?

Cf Jérome contre les lucifériens

Cité dans 4.32 biblio p°45-49

Cf aussi 4.36 (plus synthétique)

Homélie de Faust de Rié (illustre inconnu à l'époque) mais recopiée tout
au long du moyen-âge (fin du 5° s)

Moment clé de la séparation~:

Baptême / Confirmation

Bébé / adulte

Naissance / croissance

L'Esprit Saint donne~:

La plénitude / accroissement

Quant à / quant à

L'innocence / la grâce\sn{pb si on considère comme un + qui
  manquerait (cf dom Botte)}

Lavé / fortifié

\hypertarget{b.-a-des-enfants}{%
\subparagraph{A des enfants}\label{b.-a-des-enfants}}

qu'on ne touche à rien~!

Conserver l'Eucharistie avant la confirmation (car fruit important dans
la vie des jeunes)

On tient Compte de la tradition mais concrètement, on ne peut pas
détruire ce qui se construit à ce moment là.

Concile de Trente~:

Décret~: communion aux deux espèces et communion des petits enfants.
 
    «~l'Église a tout pouvoir sur les sacrements (leur substance étant
sauve), et donc le pouvoir de modifier les rites selon l'utilité
spirituelle de ceux qui les reçoivent~».
 


\hypertarget{d.-baptuxeame-des-petits-enfants}{%
\section{Baptême des petits
enfants}\label{d.-baptuxeame-des-petits-enfants}}

Cf article \emph{Baptême des petits enfants et péché originel -- LM
Chauvet}

\hypertarget{hermuxe9neutique-du-rapport-au-baptuxeame-des-petits-enfants-et-du-puxe9chuxe9-originel}{%
\subsection{Herméneutique du rapport au baptême des petits enfants et du
péché
originel}\label{hermuxe9neutique-du-rapport-au-baptuxeame-des-petits-enfants-et-du-puxe9chuxe9-originel}}

Rapport analogique~avec le péché originel:

\begin{itemize}
\item
  \textbf{similitude}~: péché comme puissance~: «~péché du monde~» de St
  Jean. Tout homme qui nait est lié aux autres . Anthropologie du Corps.
  Dans le plus spirituel que nous sommes, nous sommes en RELATION avec
  autrui, dès le ventre. \textbf{Solidarité de chacun} est plus facile à
  comprendre que dans une anthropologie de St thomas, centré sur l'âme.
  Baptême~des petits enfants : Tu es solidaire mais ce mal, qu'il
  éprouvera lui aussi comme une sorte de chape de plomb, ce mal n'a pas
  le dernier mot sur le monde. \textbf{L'alliance est plus forte que la
  violence.}
\item
  \textbf{Différence~:} existential, oui mais existentiel, non car le
  petit enfant n'a pas encore ratifié le péché. Monstrueux. Comment
  appliquer la notion de péché, qui implique une liberté, à une personne
  sans liberté~?
\item
  \textbf{Chrysostome~:} pas de liberté, pas de péché~: refuse le péché
\item
  \textbf{Augustin~:} baptême des petits enfants~: «~renonce t il au
  péché~?~» c'est donc qu'il n'est pas que victime mais solidaire du
  péché.
\end{itemize}

Ricoeur, le conflit des interprétations. Péché originel

\hypertarget{rapport-entre-etat-et-action}{%
\paragraph{Rapport entre Etat et
action}\label{rapport-entre-etat-et-action}}

Rm 5, 12

Tous ont péché en Adam

Ev O~: dans le sens de epi , \emph{étant donné que}

Note de la TOB~:
 
  notion de solidarité entre la transgression d'Adam et le péché
  personnel de chaque Homme.
 

Idem Cahier Evangile de Grelot

 
Etat Acte
 
Je («~PO~»)

«~Il n'est que dans le péché personnel que le péché d'Adam se réalise
entièrement~».

\textbf{Etat de pécheur~:} pas à comprendre de la même façon que pour un
adulte car n'a pas été encore été ratifié par l'enfant~: il n'a pas
ratifié l'héritage, héritage inévitable.

Anthropologiquement , très important

Rm 7~: je fais le mal que je ne veux pas. \ldots{} le mal vient plus
loin que moi.

Grâce à cela, je ne m'oppose pas à Dieu et l'homme avec son mal ne
s'oppose pas.

Le Serpent permet de libérer l'homme de cette toute puissance du mal.

\textbf{Le péché est acte mais il est aussi Puissance, il est
\emph{Loi}}\emph{.}

Cyprien~: logique de victime~: virus

Augustin~: pecatum alienum/ péché aliéné au sens non approprié.

Par analogie, le petit enfant croit en une Foi aliénée, la Foi de
l'Eglise et de ses parents qu'il s'appropriera. \textbf{Un baptême en
attente d'appropriation.}

Faire jouer l'analogie~: nécessaire car sinon~:
\textbf{déresponsabilisation,} d'une certaine façon un acte de type bouc
émissaire (Girard).

\hypertarget{gratuituxe9-de-la-gruxe2ce}{%
\paragraph{Gratuité de la grâce}\label{gratuituxe9-de-la-gruxe2ce}}

Baptême des petits enfants~: on voit bien la gratuité de la grâce.

Mais attention à l'aliénation~: car la grâce n'est pas plus grande parce
que nous ne pouvons pas répondre. Argument ambigu.

\emph{Pensons toujours que le référent est le baptême des adultes
responsables.}

\hypertarget{salut}{%
\paragraph{salut}\label{salut}}

Méprise symbolique~: des personnes qui pratiquent mais ne comprennent
pas

\hypertarget{question}{%
\paragraph{Question}\label{question}}

Comment faire cela d'un point de vue pastoral



%-------------------------------------------------------------------------------------------------------------------------------
\chapter{Sacrement de la Réconciliation}

%-------------------------------------------------------------------------------------------------------------------------------

6.37 Rituel, offerts au pere Gy \emph{Nova et vetera}

Ego te baptizo~: article de Paul de Clerk

Article Penitence dans Dictionnaire Critique de Théologie

6. 12 Cyrille Vogel~: Amusant, indispensable à lire

6. 14

La réconciliation dans son rapport à l'Eglise

%----------------------------------------------------------------------------------------------
\hypertarget{trois-leuxe7ons-fondamentales}{%
\section{Trois leçons
fondamentales}\label{trois-leuxe7ons-fondamentales}}

\hypertarget{historique}{%
\subsection{Historique}\label{historique}}



\begin{table}[h!]
    \centering
    \sidecaption{Un Sacrement qui a connu des révolutions  }
 \footnotesize
\begin{tabular}{p{.2\textwidth}p{.2\textwidth}p{.2\textwidth}p{.2\textwidth}}
\toprule
150 & 600 & 1100 & Latran IV (1215)

Trente \\
\midrule
Non système~: pas d'institution pénitenciaire que le Baptême

Hermas

Cf inceste~de Corinthe: 1Co 5 & Tertullien~: un autre rituel

\textbf{La pénitence canonique,} publique, non réitérable. L'idée n'est
pas d'humilier les gens mais l'intercession de la communauté. Pendant la
pénitence, mis au ban avec les cathécumènes & \textbf{Pénitence
tarifée,} réitérable, et avec une pénitence par péché &
\textbf{Pénitence moderne,} l'absolution est donnée avant la
satisfaction (l'aveu est devenu tellement humiliant qu'il est devenu la
satisfaction). \\
Pourquoi~? & Eglise grandit et il faut intervenir & Culture germanique~:
objectivité de la faute~; & Abelard, sentences, bourgeoisie,
université~: changement de théologie, on passe à \textbf{l'intention} et
non à l'acte en lui-même. \\
 
\\
\bottomrule
\end{tabular}
\label{tab:my_label}
\end{table}


\hypertarget{la-conversion-du-cux153ur}{%
\subsection{La conversion du cœur}\label{la-conversion-du-cux153ur}}

Evidence pour tous les Pères de l'Eglise~: \textbf{Dieu pardonne dès que
l'homme demande pardon.} L'âge moderne (Trente) a voulu une contrition à
la hauteur de la faute. Mais attention à la dérive psychologique (mise
en garde de St Thomas)

Grégoire le Grand (600) écrit~: \emph{nous devons absoudre par notre
autorité pastorale ceux que nous savons qu ils ont été vivifiés par la
Grâce.}

Pour S Thomas, le pécheur qui vient se confesser pour un péché mortel
(et non pas péché véniel~: cependant il y en avait beaucoup) est déjà
pardonné avant.

\textbf{Alequando}~: \emph{parfois} seulement pendant le sacrement qu'il
est pardonné.

\hypertarget{duxe9cret-sur-la-puxe9nitence-concile-de-trente}{%
\paragraph[Décret sur la pénitence~: Concile de
Trente]{\texorpdfstring{Décret sur la pénitence\sn{Très beau}~:
Concile de
Trente}{Décret sur la pénitence~: Concile de Trente}}\label{duxe9cret-sur-la-puxe9nitence-concile-de-trente}}

1551~: renversement par rapport à St Thomas~: pour que la contrition
permette le pardon avant le sacrement, il faut qu'elle soit si parfaite
que les cas sont rares.

Catéchisme du Concile de Trente~: 1553 accentue le concile de Trente.

\hypertarget{repentir-pierre-contre-remordjudas}{%
\paragraph{Repentir (Pierre) contre
remord(Judas)}\label{repentir-pierre-contre-remordjudas}}

Le repentir ouvre à l'autre

Le vrai repentir inclut le désir du sacrement~:

St Thomas~: Tertia Pars, traité sur la pénitence, pas achevé

\begin{itemize}
\item
   
  4parties intégrales
   
\item
   
  contrition intérieure
   
\item
   
  la satisfaction de bouche
   
\item
   
  la pénitence
   
\item
   
  l'action du prêtre
   
\end{itemize}

\hypertarget{une-conversion-du-cux153ur-sans-acte-extuxe9rieur-est-une-illusion}{%
\subsection{Une conversion du cœur sans acte extérieur est une
illusion}\label{une-conversion-du-cux153ur-sans-acte-extuxe9rieur-est-une-illusion}}

Les actes sont le chemin normal des péchés quotidiens.

Dans l'antiquité, il y avait des listes de péchés graves~: apostasie,
meurtre,

Les autres, \emph{peccata quotidia,} péchés véniels, ce n'est pas le
sacrement, mais la triade de l'\textbf{Aumône, le jeune et la prière
(}Dans la \emph{Secunda Clementi,} dans cet ordre).



\begin{table}[h!]
    \centering
    \sidecaption{  }
 
\begin{tabular}{p{.3\textwidth}p{.3\textwidth}}
\toprule
Aumône

  & Pratiquer la justice \\
 
Jeune

  & Renforcement du corps \\
  
  Prière & \emph{Y compris théologie\ldots{}} \\
\\
\bottomrule
\end{tabular}
\label{tab:my_label}
\end{table}



La Confession nécessaire pour les péchés graves.

Trente, Confession, V, \emph{on peut expier les péchés quotidiens
d'autres façons.}

\textbf{Péché mortel et péché véniel}

Il n'y a pas de différence de niveau mais \emph{de nature.}

Cependant, qui peut le plus peut le moins~: des fruits réels de la
confession fréquence

Mais si on se confesse moins, ce n'est pas forcément un drame.

Le chemin normal~: l'aumône, le jeune et la prière.

\textbf{La vertu de la pénitence,} vertu quotidienne dans lequel
s'inscrit le sacrement de pénitence.

\hypertarget{ce-nest-pas-leglise-mais-dieu-qui-est-uxe0-la-source-du-pardon}{%
\subsection{Ce n'est pas l'Eglise mais Dieu qui est à la source du
pardon}\label{ce-nest-pas-leglise-mais-dieu-qui-est-uxe0-la-source-du-pardon}}

Le rôle du prêtre

\textbf{Forme déprécative} (que Dieu te baptise) / Forme indicative (Je
te baptise)

Absoudre et pardonner

Thomas, De forma absolutionis

Prend la défense de Ego te Absolvo

Va influencer Trente~; dans la ligne de sa théologie

Bonaventure~: théorie occasionaliste

«~Pacte~» de Dieu avec l'homme de pardonner quand on dit~: «~je te
pardonne~»

Pas terrible

6.16 André Duval, \emph{des sacrements au Concile de Trente}

1.102

Le prêtre pas seulement un médecin (vision Orientale) mais agit comme un
\textbf{juge}. Mais l'hermeneutique nous apprend \textbf{que
l'absolution est performative}, comme le juge lorsqu'il fait un jugement
Il ne s'agit pas seulement d'un promesse (Luther).

Dans l'occident, Forme performative mais le Ego ministériel ne se
substitue pas à Dieu.

\emph{Et moi, au nom du père, du Fils et du Esprit-Saint, je te
pardonne.}

\hypertarget{augustin}{%
\paragraph{Augustin}\label{augustin}}

\textbf{Guérison des dix lépreux}

«~Allez vous montrer au prêtre~»~: le prêtre vient déclarer que les
péchés sont pardonnés. Il ne fait que montrer

\textbf{Résurrection de Lazare}

«~déliez le~»

C'est Jésus qui ressuscite mais l'Eglise délie et donc le réintègre dans
l'Eglise.

\hypertarget{duxe9bats-entre-bonaventure-et-thomas}{%
\paragraph{Débats entre Bonaventure et
Thomas}\label{duxe9bats-entre-bonaventure-et-thomas}}

Bonaventure~: L'Eglise ne peut que montrer ce qui est absout.
Distinguait la culpe et la peine. Sur la peine, l'Eglise a une prise
directe.

Thomas~: pour lui, on ne peut séparer la culpe et la peine.

\hypertarget{sept-autres-leuxe7ons-sur-la-penitence}{%
\section{\texorpdfstring{Sept autres leçons sur la penitence
}{Sept autres leçons sur la penitence }}\label{sept-autres-leuxe7ons-sur-la-penitence}}

\hypertarget{le-premier-sacrement-de-ruxe9mission-des-puxe9chuxe9s-cest-le-baptuxeame}{%
\subsection{Le premier sacrement de rémission des péchés, c'est le
baptême}\label{le-premier-sacrement-de-ruxe9mission-des-puxe9chuxe9s-cest-le-baptuxeame}}

Ivème siècle : deux institutions du catéchuménat et de la pénitence se
développent conjointement.

Confession~: baptême à sec (mais tertullien disait baptême dans les
larmes)

Christ est mort une fois pour toute, Donc il n'y a qu'un seul baptême

Puisqu'il n'y a qu'un seul Baptême, il n'y a qu'une seule rémission

\emph{«~C'est maintenant que je souhaiterais être baptisé~: «~}

Sacrement de la réconciliation~: un «~nouveau~» baptême

\hypertarget{ne-nous-uxe9tonnons-pas-quun-nouveau-systuxe8me-puxe9nitentiel-se-cherche}{%
\subsection{Ne nous étonnons pas qu'un nouveau système pénitentiel se
cherche}\label{ne-nous-uxe9tonnons-pas-quun-nouveau-systuxe8me-puxe9nitentiel-se-cherche}}

Chaque système est lié à la culture, en particulier sur la place
minoritaire ou majoritaire de l' Eglise.

Cf histoire

Impossible que cela n'évolue pas.
Evolution et révolution~: à chaque époque, un nouveau système
\mn{Le mercredi 21 mai 2003 LM Chauvet, le sacrement de réconciliation}





\hypertarget{chaque-systuxe8me-puxe9nitentiel-a-mis-un-point-daccent-diffuxe9rent}{%
\subsection{Chaque système pénitentiel a mis un point d'accent
différent}\label{chaque-systuxe8me-puxe9nitentiel-a-mis-un-point-daccent-diffuxe9rent}}

Aujourd'hui, peut être un accent trop important sur l'absolution

A l'époque des Pères, on insistait sur \textbf{la conversion de vie.}

A l'époque de la pénitence tarifée, on insistait sur la
\textbf{satisfaction.}

La confession indique la pénitence. La pénitence entraîne la
\textbf{satisfaction}. La satisfaction procure la rémission des péchés.

Vogel II, 219-220~: pénitentiel~: «~dis au pénitent qu'à la fin du
jeune, il sera pardonné de ses péchés~»

Aux temps moderne, on insiste sur la confession. Au XII, les théologiens
(cf Pierre Lombard) insiste sur le fait que l'absolution \textbf{montre
que} Dieu a déjà pardonné. Au XIII, on a changé de perspective et on
insiste sur la confession qui inclut la satisfaction
(\textbf{synecdoque}~: la partie désigne le tout)~: «~aller chercher
dans les replis de la conscience~».

Aujourd'hui, on insiste sur l'\textbf{absolution,} la partie du
prêtre\textbf{~: si pas d'aboslution,} on ne se sent pas pardonné. La
grâce de la pénitence n'a pourtant jamais détourné de la \textbf{grâce
de la conversion.}

\begin{itemize}
\item
  confession
\item
  absolution
\item
  satisfaction
\end{itemize}

\textbf{Ayons de la perspective historique~:} attention à ne projeter
sur les premiers siècles des élaborations postérieures (tridentine ou
scholastique).

\hypertarget{retrouver-un-juste-uxe9quilibre-entre-pruxeatre-et-communautuxe9}{%
\subsection{Retrouver un juste équilibre entre Prêtre et
communauté}\label{retrouver-un-juste-uxe9quilibre-entre-pruxeatre-et-communautuxe9}}

Antiquité~: Pères~: seul Dieu qui convertit les cœurs, mais ils
rappelaient le rôle de l'Église au même moment~: il n'est de vrai pardon
avec Dieu qu'~:

\hypertarget{en-uxe9glise-en-corps-collectif-puxe9chuxe9-collectif}{%
\paragraph{en Église en corps collectif (péché
collectif)}\label{en-uxe9glise-en-corps-collectif-puxe9chuxe9-collectif}}

\hypertarget{avec-luxe9glise}{%
\paragraph{\texorpdfstring{avec l'Église
}{avec l'Église }}\label{avec-luxe9glise}}

\begin{itemize}
\item
  Karl Rahner, Vérité oubliée sur le sacrement de la pénitence (Ecrits
  théologiques II)
\end{itemize}

cf Prenotanda du rituel

Sacramentum Res et Sacramentum Res Sacramenti

Rite Reconciliation avec l'Église Réconciliation avec Dieu

\hypertarget{par-luxe9glise}{%
\paragraph{\texorpdfstring{Par l'Église
}{Par l'Église }}\label{par-luxe9glise}}

Difficile Aujourd'hui. Car on réduit souvent l'Église au prêtre.

Augustin compare l'Église à la colombe et donc à l'Esprit Saint.

Chez \textbf{Augustin}, Pierre est le symbole de l'Église~: Pierre est
le type du confessant.

«~Vous aussi vous liez, vous aussi vous déliez~»

L'action du prêtre ne donne sens que dans l'action de l'Église .

Caractère publique de la pénitence antique~: surtout pour que la
communauté prie pour la personne pénitente.

\textbf{Intercession de la communauté.} Difficile à faire passer
aujourd'hui. Chacun est pour les autres un ministre et intercesseur.

Cyrille, didascalie~: Evèque, impose la main pendant que toute la
communauté prie pour lui.

Père Siman, expérience de l'Esprit dans l'Église syriaque d'Antioche,
Beauchène

IIIème siècle

La pointe de la prière, c'est la réintroduction dans la communauté

Tertullien, SC 313

Ambroise, SC 319

Les pécheurs demandent le perdon des péchés par l'Église.

Si importante que l'on tombe dans des \emph{exagérations}.

Césaire d'Arles doit mettre en garde les pénitents de reposer un peu
trop facilement sur l'intercession de la communauté.

(Sermon 67~)

 
    {«~lorsque les ministres pardonnent au
nom de Dieu, ils exercent leur action au cœur même d'une action de
l'Église dont ils sont les
serviteurs~».}\sn{Numéro 20 des prénotanda}
 

Lire Hervé Legrand~: «~Il faut penser la persona Christi dans la persona
ecclesia~».

 
\subsection{Equilibre entre fermeté et
souplesse} 

Exemple rapide~: \textbf{émergence de la pénitence non réitérable}.

Comment se fait il que l'Église en soit venue à une telle rigueur~?
L'émergence de la pénitence non réitérable n'est pas rigoriste. Contre
les courants rigoristes qui ne voulaient rien d'autres que le baptême.

Hermas~: 130-160~: déclare avoir vu une vision d'une pénitence non
réitérable. Position anti rigoriste au départ. Une certaine souplesse.

\textbf{Le passage à la pénitence réitérable~:}

Les Pères étaient extrêmement stricts sur le principe de la pénitence
non réitérable. Cf Ambroise, Baptême, 2, 95 «~de même qu'il y a qu'un
seul baptême, il y a une seule pénitence~\ldots{} car le Christ n'est
mort qu'une seule fois pour nos péchés ».

Mais les évêques ont eu une pastorale plus souple. Cf Ambroise~: «~j'ai
rencontré plus de personnes ayant gardé leur innocence baptismal que des
personnes ayant observées leur pénitence~».

Pratiquement, on attendait un âge canonique.

Article de MD 118, la pénitence publique durant les 6 premiers siècles

122-123 Lettre 95 de St Augustin~: ferme sur les principes mais dans la
pratique, on fait ce que l'on peut.

«~Que d'incertitudes dans tout cela~»

Augustin ne pense cependant pas à la pénitence réitérable.

\textbf{Mais dysfonctionnement pratique}. Fin IV~: être chrétien est
devenu un avantage. Mais beaucoup de gens ne sont pas baptisés.

Parmi les pénitents, pas beaucoup de pénitents faisaient une vraie
pénitence~: \emph{Ordo penitencia~: «~}Ce qui devait être le lieu de
l'humilité, est devenu le lieu de l'iniquité~» Augustin

On arrivait à une image avec les happy few et une masse éloignée.

Orléans 538~: les gens mariés ne rentrent dans l'ordre des pénitents
mais qu'avancés en âge.

La hiérarchie elle même a consacré la faillite de la pénitence
canonique. Mais on n'a rien mis à la place.

Césaire prend acte que la plupart demanderont sur leur lit de mort.
Césaire avertit les chrétiens qu'ils doivent se préparer à cette
pénitence tout au long de leur vie.

\textbf{Impasse et même perversion.}

\emph{Au départ,}

Quelques uns en vue d'une conversion de vie au sein de l'Ecclesia

\emph{Au temps de Césaire}

Tous In extremis, en vue du ciel de façon individuelle

Dérive ritualiste

Sacramentaire dit léonien~VI (: S Leon V~;)

Vogel I, 199-200~: prière spéciale pour ceux qui sont morts sans
pénitence

\textbf{Lorsque l'on veut être fidèle au principe mais que la situation
sociologique et culturelle a fortement changé, on arrive à des
retournements complets du sacrement.}

Aujourd'hui, situation comparable.

\hypertarget{laveu-ou-la-confession}{%
\subsection{L'aveu ou la confession}\label{laveu-ou-la-confession}}

Pas la même importance.

Dans la pénitence canonique, l'aveu est peu important et est un
préalable(parfois

Dans la pénitence tarifée, important.

Mais surtout important dans la pénitence moderne.

\begin{itemize}
\item
  Aspect psychologique.
\item
  L'aveu est la principale partie de la satisfaction
\item
  Désormais, partie sacramentaire car élément de pénitence
\end{itemize}

La crise aujourd'hui est pour une grande part une \textbf{crise de
l'aveu.} Chute dans les années 70.

\hypertarget{diffuxe9rence-entre-orient-et-occident}{%
\paragraph{Différence entre Orient et
Occident}\label{diffuxe9rence-entre-orient-et-occident}}

En orient, rôle médicinal que juge~: \emph{pater pharmaticos}~: qualité
du pater qui compte en orient~: cf les staretz , moines~: mais ce ne
sont pas des laïcs (charisme dans l'Église qui n'appartient pas au
ministère, mais un charisme de discernement).

Dans l'occident, on a insité davantage sur l'aspect judiciaire. Le
Concile de Trente a publié un grand décret sur la pénitence~: \emph{Acte
de Jugement} , comme un acte judiciaire.\\
Herméneutique du concile de Trente~: contre les mises en doute des
réformateurs, la parole du prêtre est performative, comme celle du
juge~:

\begin{itemize}
\item
  Cela fait ce que cela dit
\end{itemize}

L'aveu est moins important dans l'Orient.

\textbf{Tradition}

\hypertarget{il-nous-faut-apprendre-uxe0-desserrer-le-lien-entre-sacrement-et-accompagnement-spirituel}{%
\subsection{Il nous faut apprendre à desserrer le lien entre sacrement
et accompagnement
spirituel}\label{il-nous-faut-apprendre-uxe0-desserrer-le-lien-entre-sacrement-et-accompagnement-spirituel}}

\textbf{Deux dimensions~:}

\hypertarget{dimension-sacrementelle-eccluxe9siale}{%
\subparagraph{\texorpdfstring{Dimension sacrementelle, ecclésiale~:
}{Dimension sacrementelle, ecclésiale~: }}\label{dimension-sacrementelle-eccluxe9siale}}

\begin{itemize}
\item
  Dans l'antiquité, lier~: ex-communier~; aspect disciplinaire
\item
  Mais l'acte doit être considéré sacramentel~: on met cela en lien avec
  l'Esprit Saint~:
\item
   
  Ambroise~: Jn 20, 21-22 C'est un droit de l'Esprit Saint de lier et
  délier les crimes
   
\item
   
  Toute intégration dans l'Église ou réintégration dans l'Église est
  considéré dans l'Antiquité comme œuvre de l'Esprit Saint.
   
\end{itemize}

\hypertarget{dimension-personnelle-et-spirituelle}{%
\subparagraph{Dimension personnelle et
spirituelle}\label{dimension-personnelle-et-spirituelle}}

\begin{itemize}
\item
  une pratique différente, pratique sacramentelle pour les fautes non
  graves~: pénitence réitérable aussi souvent que besoin. On est passé à
  la pénitence \textbf{aussi souvent que possible~: confession de
  dévotion.} Dès le XIII, \emph{perfectisimi} (Groupe de labusière~?)~:
  certains prêtres doivent même mettre le holà~en cas de confession plus
  régulière qu'une fois par semaine.
\item
  Devient même un critère de sanctification
\item
  Elle se rattache plus à l'\textbf{aveu thérapeutique~: la cure d'âme~;
  Accompagnement spirituel.} Et non le sacrement proprement dit.
\item
  Attention~: cette confession fréquente a eu des fruits intéressants
  mais pas à intégrer d'un point de vue dogmatique.
\end{itemize}

\hypertarget{assainir-la-situation}{%
\subparagraph{Assainir la situation}\label{assainir-la-situation}}

\begin{itemize}
\item
   
  Jésuite , Robert Taft, prof à la grégorienne~: distinguer la
  confession et la procédure de réconciliation en cas de rupture
  d'écclésialité.
   
\item
   
  Idem~: de Clerck~: ne pas faire jouer le rôle de l'autre.
   
\end{itemize}

\hypertarget{quelques-propositions-pastorales}{%
\section{Quelques propositions
pastorales}\label{quelques-propositions-pastorales}}

\begin{itemize}
\item
  Il manquerait quelque chose à une communauté d'Église (paroise) si
  elle ne célébrait pas régulièrement le sacrement de réconciliation
\item
   
  Certes Baptême mais pénitence seconde
   
\item
  pénitence individuelle ou communautaire
\item
   
  \textbf{individuel}
   
\item
   
  De plus en plus de chrétiens demandent un accompagnement spirituel~:
  être mis au pied du mur de sa conversion. Pb~: prêtres ne peuvent y
  suffire mais pas limités au prêtre~; affaire de charisme.
   
\item
   
  Des laïcs avec charisme reconnu (cf Renouveau)
   
\item
   
  Devoir de l'Église de promouvoir ce charisme
   
\item
   
  Célébration sacramentelle
   
\item
   
  \textbf{Confession de relèvement}~: après un acte grave. Certes Dieu
  pardonne mais démarche.
   
\item
   
  \textbf{Confession de conversion}~: après 10-15 ans, on revient à
  l'Église et on souhaite recommunier~: nécessité de refaire une
  démarche écclesiale. Cela ne veut pas dire que \textbf{subjectivement}
  on a péché~; mais \textbf{objectivement,} on a coupé de l'Église.
   
\item
  Communautaire
\item
   
  \textbf{Communautaire avec confession individuelle}~: Succès très fort
  mais position mixte. Difficulté de reprendre la dimension
  communautaire après la dimension individuelle. Possibilité de faire
  cela rapidement
   
\item
   
  \textbf{Absolution collective~}: Grand succès surtout dans les régions
  à forte pratique. Pb vis à vis de la gène de l'aveu. L'avenir n'est
  pas de ce coté là. Car il faut que le sacrement retrouve de la vérité
   
\item
  L'étalement du sacrement sur un temps liturgique~: Avent ou carême
\item
   
  Il faut de la vérité et une véritable démarche personnelle.
   
\end{itemize}

Pas forcément un sacrement en crise~: il est en \textbf{profonde
mutation.}



%-------------------------------------------------------------------------------------------------------------------------------
\chapter{Sacrement du mariage~: approche historique}

%-------------------------------------------------------------------------------------------------------------------------------


\hypertarget{bibliographie}{%
\section{Bibliographie}\label{bibliographie}}

Lire le \emph{mariage aujourd'hui et demain}

Mariage dispars~: mariage valide donc sacramentel~: mais cependant, pas
un baptisé~: pas un \textbf{sacrement comme analogie}

Ce qui est sacrement, c'est l'Alliance.

Lire Mathon

Attention à la \emph{Doctores Disputant~!} \textbf{Prudence~!~!~!}

\hypertarget{les-difficultuxe9s-thuxe9oriques-uxe0-la-reconnaissance-du-mariage-comme-sacrement}{%
\section{Les difficultés théoriques à la reconnaissance du mariage comme
sacrement}\label{les-difficultuxe9s-thuxe9oriques-uxe0-la-reconnaissance-du-mariage-comme-sacrement}}

Dès le début, sa densité de signification au plan biblique (Ep 5, 22-33
«~ce mystère est grand~»~: rapport entre l'homme et la femme donne à
penser le rapport entre le Christ et l'Église~; «~magnum sacramentum~»),
toute la révélation de Dieu sous la forme nuptiale.

Sa densité de signification au plan anthropologique

\textbf{On aurait pu penser que ce fut le premier sacrement, ce fut le
dernier. «~}Comme c'est le sacrement qui comporte le moins de
spiritualité, on le met à la fin~» S Thomas

Au départ, pas un sacrement.

\hypertarget{lamour-a-souvent-uxe9tuxe9-peu-valorisuxe9-comme-motif-du-mariage}{%
\subsection{L'amour a souvent été peu valorisé comme motif du
mariage}\label{lamour-a-souvent-uxe9tuxe9-peu-valorisuxe9-comme-motif-du-mariage}}

Certes, il existait de vrai mariage d'amour mais la règle générale~: on
vous mariait en espérant que vous arriviez à vous aimer.

Il n'y a que la société occidentale qui refuse que l'on vous marie (mais
on vous marie quand même).

Le Roy-Ladurie~: procès contre les cathares. Le verbe «~adamare~» de
dilection est très peu employé.

Le pb, comment faire le lien avec Ep 5~: l'amour du Christ pour son
épouse.

Il fallait d'abord développer l'amour dans le mariage.

\hypertarget{lantiquituxe9}{%
\paragraph{L'antiquité}\label{lantiquituxe9}}

\hypertarget{moyen-age}{%
\paragraph{Moyen Age}\label{moyen-age}}

Il faut attendre Hugues de Saint Victor et sa dilection

\hypertarget{le-jugement-pessimiste-de-leglise-sur-la-sexualituxe9}{%
\subsection{1.2 Le jugement pessimiste de l'Eglise sur la
sexualité}\label{le-jugement-pessimiste-de-leglise-sur-la-sexualituxe9}}

\hypertarget{le-contexte-uxe0-luxe9poque-patristique}{%
\paragraph{Le contexte à l'époque
patristique}\label{le-contexte-uxe0-luxe9poque-patristique}}

Le mariage est discrédité dans l'empire romain. D'un point de vue
éthique, dévergondage\ldots{}

A accentué le pessimisme sur la sexualité.

\textbf{Principe~:}

\begin{itemize}
\item
  le \textbf{mariage est bon}~: dieu l'a voulu dès le commencement~:
  \emph{sacrement de la Nature}
\item
   
  affirmation contre les courants manichéens, qui le voyaient comme une
  création mauvaise.
   
\item
   
  Mouvements enchratistes~: cf S. Jérome
   
\item
   
  Contre les mouvements laxistes~: on peut faire ce que l'on veut
   
\item
  Mais ils affirment (1 Co 7) la \textbf{supériorité de la virginité}.
\item
   
  Cela dit, un principe qu'il faut modérer.
   
\end{itemize}

On l'affirmait non pas de la Bible mais de la lutte contre l'ambiance
mais surtout des grands courants philosophiques (on visait l'apathéia)
surtout Stoiciens et neo platoniciens~: \textbf{abstention par rapport à
la sexualité.}

Les Pères souhaitaient dépasser les philosophes sur ce point.

Certains écrivains chrétiens ont pu passer du «~mariage~moins bon que
chasteté~» à «~mariage pas bon~».

\hypertarget{augustin-1}{%
\paragraph{Augustin}\label{augustin-1}}

Augustin a beaucoup parlé du mariage~: position typique des Pères.

La sexualité, voulue par Dieu n'est pas un mal. Mais l'usage que nous en
faisons depuis Adam ne peut pas ne pas la confiner au mal. L'acte
conjugal, même accompli comme il se doit (intention de procréer), a
besoin de \emph{venia,} miséricorde et souvent \emph{peccamineux.}

Il ne dit pas que l'acte est bon~: il dit que ce n'est pas un mal.

Mais il y a trois dons du mariage qui sont \emph{venia~,
«~}excuses~»\emph{:}

\begin{itemize}
\item
  \emph{proles}~: prolifique~; \textbf{la procréation.} Pourtant, on en
  parle pas dans la bible. Mais dans les tables nuptia, sur lesquels on
  se mariait à Rome, c'était marqué.
\item
  \emph{Fides~:} non pas fidélité mais \textbf{entraide mutuelle}~: Gn~:
  Adam n'avait pas d'aide.
\item
  \emph{Sacramentum~:} Augustin le tire d'Ep 5\sn{Texte sur la
    fidélité amoureuse.}. C'est ce qui a rapport à
  \textbf{l'indissolubilité}~: l'amour fidèle, jusqu'à la mort. Amour
  sauveur de Dieu pour le Christ. Mais le sacramentum n'est pas le
  sacrement au sens du XII.
\end{itemize}

Augustin a marqué la théorie du mariage jusqu'au XX.

\hypertarget{les-scolastiques}{%
\paragraph{Les scolastiques}\label{les-scolastiques}}

«~le mariage est un remède à la concupiscence~».

Cf Bonaventure~: «~ un fonction~{[}officium{]} mais maintenant un remède
à la concupiscence {[}libinis mordum{]}~»

Mais Thomas d'Aquin, refusant le pessimisme d'Augustin, refuse d'y voir
un péché systématique mais peut être \textbf{méritoire.} Un acte n'est
jamais neutre~:

Il innove face à l'objection (Supplément à la somme théologique)

Question 42, 1~: objection 3~: les sacrements tirent leur qualité de la
conformation au CHRIST à la passion. Le sacrement du mariage n'est pas
une conformation à la passion donc ce n'est pas un sacrement.

\textbf{S. Thomas répond~: c'est un sacrement car il est conforme par
rapport à l'amour.}

Question 42, 4~: le mariage sans union charnelle sanctifie davantage.

Cf Mariage de Joseph et Marie.

\hypertarget{quel-est-luxe9luxe9ment-constitutif-du-lien-matrimonial-consentement-seul-consentement-et-consommation}{%
\subsection{Quel est l'élément constitutif du lien matrimonial~?
Consentement seul~? Consentement et
consommation~?}\label{quel-est-luxe9luxe9ment-constitutif-du-lien-matrimonial-consentement-seul-consentement-et-consommation}}

Droit Germanique~: consommation charnelle nécessaire car signifiant là.

Yves de Chartres~: renoue avec le droit romain

La solution au 13\textsuperscript{ème} siècle

\begin{itemize}
\item
   
  l'indissolubilité commence après la consommation mais l'essence du
  mariage vient du consentement.(Lombard)
   
\end{itemize}

\textbf{la desexualisation probablement nécessaire pour la
sacramentalité.} La bénédiction des prêtres n'est qu'un sacramental et
non un sacrement pour Thomas.

\hypertarget{la-buxe9nuxe9diction-faite-par-les-pruxeatres-nappartient-pas-uxe0-lessence-du-mariage}{%
\subsection{La Bénédiction faite par les prêtres n'appartient pas à
l'essence du
mariage}\label{la-buxe9nuxe9diction-faite-par-les-pruxeatres-nappartient-pas-uxe0-lessence-du-mariage}}

Lettre du pape Nicolas I (866) aux Bulgares, Mathon, p°159

Du coté byzantin, il fallait passer devant le prêtre pour être marié.

Nicolas Ier raconte la cérémonie du mariage~: les époux sont ministres
dans l'Église latine (mais ambiguité).

\hypertarget{orient-et-occident}{%
\paragraph{Orient et occident}\label{orient-et-occident}}

Orient Insistance sur le prêtre

\textbf{On se mariait selon la coutume.}

La cérémonie vient de la présence de l'évêque aux mariages chrétiens, de
sa bénédiction lors de la cérémonie.

Sacrement~: il signifie ce qu'il figure

Au XII, on l'appelle sacrement mais sans très bien comprendre en quoi il
est sacrement (le terme de sacrement est lié à \emph{sacramentum).}

\textbf{Le mariage procure -t-il la grâce~?}

\begin{itemize}
\item
  grâce adjuvante~: pas une grâce sanctifiante mais adjuvante~: elle
  aide (S~. Thomas)
\end{itemize}

\textbf{Les enjeux pratiques sont primordiaux}

\hypertarget{la-pratique-la-prise-de-luxe9glise-sur-le-mariage-aux-plans-juridique-et-liturgique}{%
\section{La pratique~: la prise de l'Église sur le mariage aux plans
juridique et
liturgique}\label{la-pratique-la-prise-de-luxe9glise-sur-le-mariage-aux-plans-juridique-et-liturgique}}

\hypertarget{luxe9glise-sest-toujours-intuxe9ressuxe9e-au-mariage-des-chruxe9tiens}{%
\subsection{L'Église s'est toujours intéressée au mariage des
chrétiens}\label{luxe9glise-sest-toujours-intuxe9ressuxe9e-au-mariage-des-chruxe9tiens}}

Beaucoup d'homélies sur le mariage par les Pères de l'Église

\begin{itemize}
\item
  adultère
\item
   
  Beaucoup occupés au plan disciplinaire, surtout pour l'adultère (homme
  et femme)
   
\item
  rejet du divorce
\item
   
  Grégoire de Nazianze~: deux Christ, deux maris\ldots{} car Ep 5~: le
  Christ et l'Église.
   
\item
  Seconde noce
\item
  Concubinage (cf Augustin et sa première femme)
\item
   
  Concile de Tolède 300~ (cf crouzelle) :
   
\item
   
  mariage + concubine~: écarté de la communion
   
\item
   
  concubine~: intégrée dans la communion
   
\item
   
  certes ce n'était pas leur faute mais n'y a t il pas une analogie avec
  le mariage actuel~et les formes de cohabitation actuelles~?
   
\end{itemize}

\hypertarget{la-lutte-contre-les-mariages-incestueux}{%
\subsection{La lutte contre les mariages
incestueux}\label{la-lutte-contre-les-mariages-incestueux}}

Pourquoi une prohibition de l'inceste jusqu'à une parenté aussi
éloignée~?

Lv 18,6~: très vague

Enquête préalable avant le mariage~ pour éviter les mariages qui ne
tenaient pas (il y avait toujours un lien de parenté)

Duby~: le chevalier, la femme, le prêtre~: contraduction entre inceste
et indussolubilité.

2.3 La lutte contre le poids des intérêts familiaux

Le «~\emph{consensus de futuro}~» (fiançailles) est mis au second plan
pour le consentement au présent, devant le prêtre. Ce faisant, l'Église
a fait œuvre civilisatrice.

Elle promouvait la liberté des époux mais d'autre part, il fallait que
le prêtre soit présent (forte incitation à la présence du prêtre).

Mais alors nouvelle contradiction car cette présence du prêtre n'est pas
indispensable.

D'où la «~plaie~» des mariages clandestins. La présence de cette
autorité n'était pas nécessaire.

Latran IV, 1215~: les mariages clandestins sont déclarés illicites mais
pas invalides.

Le décret \emph{Tam Et Si~:} et pourtant du Concile de Trente (1563)~:

On ne touche pas au consentement lui même mais on va considérer les
personnes inhabiles à un tel consentement si elle ne le font pas devant
curé et deux témoins.

\textbf{Il est donc invalide.}

\hypertarget{le-marieur-nest-pas-le-pruxeatre}{%
\subsection{Le «~marieur~» n'est pas le
prêtre}\label{le-marieur-nest-pas-le-pruxeatre}}

le mariage, c'est le consentement

Rituel anglais de 1125-1135

Avant le XII, rôle du prêtre comme exhorciste~: bénédiction de la
chambre nuptiale

Mais on assiste à un transfert vers l'Église au début du XIIème.

1125~: Lens \emph{in face Ecclesia~:} église matériel mais aussi Église
.

liturgie de mariage~; bénédiction sous le voile

cf Molin

La marieur, c'est toujours le Père de la fille~: «~je te la donne~».

\textbf{Mais le prêtre est devenu le témoin privilégié.}

Tout est en place pour la reconnaissance du mariage comme sacrement.

Au XIV, le prêtre marie (en joignant les mains)

Cf Missel de Rouen~: ego coniungo vos~: je vous marie~; \emph{formule
performative.} Mais jamais l'Église ne l'a considéré comme obligatoire
(\textless\textgreater{} ego te baptizo).

\textbf{Cours du 5 juin}

\hypertarget{est-il-moral-de-proposer-le-mariage-catholique---ph-bordeyne}{%
\paragraph{Est il moral de proposer le mariage catholique - Ph
Bordeyne~?}\label{est-il-moral-de-proposer-le-mariage-catholique---ph-bordeyne}}

\begin{itemize}
\item
  Une nouveauté d'une ampleur insoupçonnée entre la cohabitation et le
  mariage
\item
  Une quête collective de savoir faire
\item
   
  Rousseau~: donner consistance à un amour en leur donnant une véritable
  expression sociale
   
\item
  Le don de soi et pas seulement une fête
\item
   
  Amour requalifié~: cela change tout de se lier d'amour.
   
\item
   
  Ricoeur~: la mariage n'est que «~la meilleure chance de tendresse~»
   
\item
  Importance de célébrer la fragilité dans le mariage
\item
  les ressources morales de la tradition Salésienne~:
\item
   
  bon de désirer beaucoup mais mettre un ordre dans les désirs
   
\item
   
  arbre à ses fruits
   
\item
   
  commencer par de petites choses~: s'ajuster à la différence de l'autre
   
\item
   
  «~ les inspirations se font sentir à nous sans nous, mais elles ne
  nous font pas consentir sans nous~».
   
\end{itemize}

\hypertarget{conclusion}{%
\subsection{Conclusion}\label{conclusion}}

\hypertarget{a.-le-rapport-thuxe9orie--pratique}{%
\paragraph{le rapport théorie
-pratique}\label{a.-le-rapport-thuxe9orie--pratique}}

Bourdieu~: l'effet de théorie

«~y avait il un inconscient avant Freud~? Oui et Non~»

Le fait de faire une théorie change le comportement

\hypertarget{b.-ouverture-de-la-puxe9riode-moderne}{%
\paragraph{Ouverture de la période
moderne}\label{b.-ouverture-de-la-puxe9riode-moderne}}

\hypertarget{c.-lun-des-probluxe8mes-majeurs-aujourdhui-le-rapport-entre-le-mariage-comme-contrat-et-mariage-comme-sacrement}{%
\paragraph{L'un des problèmes majeurs aujourd'hui~: le rapport
entre le mariage comme «~contrat~» et mariage comme
«~sacrement~»}\label{c.-lun-des-probluxe8mes-majeurs-aujourdhui-le-rapport-entre-le-mariage-comme-contrat-et-mariage-comme-sacrement}}

\hypertarget{eluxe9ments-de-thuxe9ologie-du-sacrement-de-mariage}{%
\section{Eléments de théologie du sacrement de
mariage}\label{eluxe9ments-de-thuxe9ologie-du-sacrement-de-mariage}}

\hypertarget{ep-5-21-33}{%
\subsection{Ep 5, 21-33}\label{ep-5-21-33}}
\begin{quote}
    «~vous qui craignez le Christ, soumettez vous les uns aux autres.
Femmes, soumettez vous à votre mari\ldots. Mari, aimez vos femmes comme
le Christ a aimé l'Église.~»
\end{quote}


Tricotage étonnant \textbf{d'Isotopies}~:

\begin{enumerate}

\item
  Chair~: on ne méprise pas son propre corps
\item
  Église
\item
  Genèse
\item
  Christ -- Église~: c'est cela dont il parle depuis le début
\end{enumerate}

Ce texte s'insère dans une série de passages où il est rapport de~:



\begin{table}[h!]
    \centering
    \sidecaption{  }
 
\begin{tabular}{p{.4\textwidth}p{.4\textwidth}}
\toprule
Ep 5,1-6,9 & Col 3, 16-4,1 \\
 {\linewidth}\raggedright
\begin{itemize}
\item
  Chant des hymnes
\item
  femmes -- maris
\item
  parents -- enfants
\item
  esclaves -- maîtres
\end{itemize}
  & \\
\\
\bottomrule
\end{tabular}
\label{tab:my_label}
\end{table}

\textbf{Rapport différent à cause de la nouveauté de la mort et
Résurrection du Christ.}

\begin{table}[h!]
    \centering
    \sidecaption{  }
 
\begin{tabular}{p{.4\textwidth}p{.4\textwidth}}
\toprule
UNS & AUTRES \\
Femme & Maris \\
Eglise & Christ \\
\\
\bottomrule
\end{tabular}
\label{tab:my_label}
\end{table}
 

Soumission~: ne pouvait être appliquée qu'à la femme et non aux maris,
car culturellement et socialement appliquée à la femme. Mais cependant,
il est bien indiqué que la soumission est des \emph{uns aux autres,} et
les maris font partie des uns, par leur appartenance à l'Église.

Pas de révolution sociale~: cf Onésime~: Paul ne cherche pas à libérer
Onésime mais demande à xxxxx de le regarder différemment .

Ce \textbf{regard} va tout changer avec le temps.

Ep 5, 32~: magnum mysterium~: moi, je déclare qu'il concerne le Christ
et l'Église.

Dans l'AT, on parlait du mariage comme révélant Dieu~: l'époux fidèle.

Ici, un renversement~: Paul nous parle de l'amour de Dieu pour
l'humanité, du Christ pour l'Église pour révéler la grandeur de l'amour
et le mariage Chrétien.

Ep 5~: théologie de l'amour \textbf{fidèle.}

Paul Ricoeur~: la distance est nécessaire à l'interprétation.

Mais nous avons cependant à vivre quelque chose de cet amour du Christ
et l'Église.

\hypertarget{rapport-entre-ep-5-et-gn-2}{%
\subsection{Rapport entre Ep 5 et Gn
2}\label{rapport-entre-ep-5-et-gn-2}}

Conséquences théologiques et pastorales importantes

Ne pas se précipiter trop vite sur la figure pascale en oubliant la
figure créationnelle.

Le mariage nous renvoie au Dieu Créateur avant de nous renvoyer au Dieu
Sauveur.

Il ne peut pas être la figure porteuse de l'amour sauveur s'il n'est pas
le sacrement de l'amour créateur.

Familiaris Consortio, 68

De l'amour créateur à l'amour sauveur, Charles Bonnet

Parmi tous les sacrements, particularité d'exister dans la nature~:
CREATION. Si projet divin, implique réellement une profonde obéissance à
sa volonté~: ce qui nécessite sa grâce. \textbf{Cheminement de salut.}

Quand un homme et une femme ont un tel projet, même si ce n'est pas
sacramentel (cf mariage dispars), il y a beaucoup de sacramentalité~:
vivre dans un amour indissoluble et une fidélité sans condition, c'est
vivre quelque chose de l'amour de Dieu pour son Peuple, du Christ pour
son Église.

\hypertarget{ce-qui-est-sanctifiuxe9-cest-lalliance-entre-les-uxe9poux}{%
\subsection{Ce qui est sanctifié, c'est l'alliance entre les
époux}\label{ce-qui-est-sanctifiuxe9-cest-lalliance-entre-les-uxe9poux}}

Ce ne sont pas les époux chacun l'un à côté de l'autre~: C'est
l'alliance entre les deux~.\\
C'est ce qui fait que si l'un n'est pas baptisé, il ne peut y avoir
sacrement.

Khalil Gilbran~: l'autonomie, part de liberté

Comment aucun sacrement ne vient ajouter de la grâce, en dehors du
sacrement de l'initiation.

Spécifier le chemin de vie sur lequel ils sont appelés à développer
cette grâce~: «~devoir d'état~»

 
\subsection{Le ministre du sacrement~: croiser le
rôle des époux et celui du ministre officiel de l'Église
} 

Des laïcs pourraient être assistants extraordinaires (et non pas
célébrant puisque le prêtre assiste au mariage et n'est pas le ministre
explicite).

Tradition byzantine~:~le prêtre est le ministre

Tradition latine~: ce sont plutôt les époux même s'il n'y a pas de
décisions canoniques sur ce point.

Rituel de 91~: traduction bloquée. Dans le rituel latin, l'échange des
consentements sera suivi directement par la bénédiction nuptiale, avec
épiclèse explicite.

\begin{itemize}
\item
  croiser l'échange des consentements (époux)~: Contrat
\item
  et bénédiction nuptiale (ministre)~: alliance
\end{itemize}

Ce n'est pas le contrat qui prime, mais l'Alliance matrimoniale et non
un \emph{Consensus.} Donc rééquilibrer vers l'Alliance.

\hypertarget{le-mariage-comme-vocation-et-mission}{%
\subsection{Le mariage comme vocation et
mission}\label{le-mariage-comme-vocation-et-mission}}

article d'Anne Marie pelletier~: le mariage une vocation~?

Elle rappelle qu'une vocation a 4 dimensions~:

\begin{itemize}
\item
  un appel par un autre
\item
  cet appel vise une mise à part
\item
  cette mise à part vise une mission
\item
  quand Dieu appelle à une mission, il en donne les moyens
\end{itemize}

le mariage~?

\begin{itemize}
\item
  appel~?
\item
  mise à part~? ce sont les prêtres
\item
  la mission~?
\end{itemize}

Mais aujourd'hui, le mariage fait objet d'un \textbf{Choix~:}

\begin{itemize}
\item
  position utopique aux yeux du monde en ce qui concerne la fidélité et
  l'indissolubilité
\item
  Expression de la nouveauté de l'Evangile.
\end{itemize}

Mgr Dannels~: Signe privilégié de la singularité de la vie Chrétienne
(comme les moines au I° siècle)

Autre point~: indépendamment de cela, Mt 19~: \emph{le mariage, une
vocation} . Après le passage sur l'indissolubilité du mariage,
comparaison entre le mariage et ceux qui \emph{se sont rendus eunuque
pour le Royaume des cieux.} Un des éléments clé du texte~: «~comprenne
qui peut comprendre~» répond à l'étonnement des disciples devant la
difficulté du mariage chrétien.

\textbf{Une unique vocation Chrétienne} qui expérimente d'un impossible
à l'homme qui devient possible en Dieu.

\hypertarget{quelle-vocation}{%
\paragraph{Quelle vocation~?}\label{quelle-vocation}}

Pourquoi mise à part~: on ne se marie pas seulement devant Dieu. On est
marié PAR Dieu.

Amour fidèle et exclusif~: ils entrent dans un amour qui est certes le
leur mais un amour dilaté, qu'ils reçoivent de Dieu.

\hypertarget{mission}{%
\paragraph{Mission}\label{mission}}

Idée de l'heureuse illusion narcissique~: une belle étape , qui a sa
valeur spirituelle.

Mais on se rend compte~: aimant l'autre, j'aime l'image que je me fais
de lui. Début d'une histoire.

Absence

Chemin pascal.

\textbf{A lire absolument ce passage.}


\subsection{Tout mariage valide entre baptisés est
\emph{Ipso facto}
sacramentel}

Questions posées par cette position magistérielle

Trente~: décret \emph{Tam EtSi}

Relativement modéré sur ce point. Il n'a pas voulu trancher le rapport
entre «~contrat~» de mariage des chrétiens et «~sacrement~».

André Duval, spécialiste de Trente, «~des sacrements au Moyen Age~».~:
loin d'énoncer explicitement ce rapport , les Pères ont laissé la
question ouverte.

Mais après discussion (cf doc )

\hypertarget{le-probluxe8me-aujourdhui.}{%
\paragraph{Le problème
aujourd'hui.}\label{le-probluxe8me-aujourdhui.}}

\textbf{- Demande de mariage à l'Église par des baptisés non croyants}

beaucoup de mariages dispars~: face au phénomène de la mal-croyance, on
peut se poser de la question d'une telle affirmation purement
juridique~. Cela dit, cela n'est pas le sujet de l'accueil.

\textbf{- ne convient il pas de reconnaître au mariage civil de baptisés
une certaine valeur~?}

ne satisfont pas au décret \emph{tamEtsi.} Ne faudrait il pas
reconnaître une certaine valeur au mariage civil~?

\textbf{Baptisés~: fatalité dont on ne pourrait se défaire.} Mais si
existentiellement, ils se sont défaits de leur baptême, que faire~?

Gaudium et Spes~: l'Église est invitée à renoncer à ses droits acquis
quand ils font obstacles à sa mission.

Le droit canonique est logique mais la vie ne s'enferre pas dans cette
logique.

\hypertarget{des-pasteurs-perplexes-par-michel-scouarnec}{%
\paragraph{Des pasteurs perplexes par Michel
Scouarnec}\label{des-pasteurs-perplexes-par-michel-scouarnec}}

Sur les propositions non sacramentaires

Question~: ne peut on pas considérer les formes de cohabitations comme
des temps non neutres~?

\hypertarget{quand-le-mariage-est-il-sacramentel-par-ph.-toxuxe9}{%
\paragraph{Quand le mariage est il sacramentel~? par Ph.
Toxé}\label{quand-le-mariage-est-il-sacramentel-par-ph.-toxuxe9}}

Mariage sacrement sauf si explicitement refusé ou pas compris.

Cependant sacrement de foi~: comment joue l'articulation~?

Martelet à VII~: sacrement pas de façon juridique mais en passant par le
Christ par lequel ils sont incorporés par le Baptême.

\hypertarget{divorcuxe9s-remariuxe9s}{%
\subsection{Divorcés remariés}\label{divorcuxe9s-remariuxe9s}}

documents ouvrant \emph{familiaris consortio}~: par les trois évêques
allemands.

Réaliste

Droit Canonique -- Le mariage

\hypertarget{la-duxe9finition-canonique-du-mariage}{%
\section{La définition canonique du
mariage}\label{la-duxe9finition-canonique-du-mariage}}

\hypertarget{la-communautuxe9-de-toute-la-vie}{%
\subsection{La communauté de toute la
vie}\label{la-communautuxe9-de-toute-la-vie}}

Canon 1055

\hypertarget{les-fins-du-mariage}{%
\subsection{Les fins du mariage}\label{les-fins-du-mariage}}

biens des conjoints

génération et éducation des enfants

\hypertarget{les-propriuxe9tuxe9s-essentielles-du-mariage}{%
\subsection{Les propriétés essentielles du
mariage}\label{les-propriuxe9tuxe9s-essentielles-du-mariage}}

S. Augustin~sur les propriétés essentielles du mariage~-- droit
naturel\sn{Rappel~: droit divin naturel~: que l'on peut tirer d'un
  raisonnement~; droit divin positif~: que l'on tire des textes
  bibliques. Ne peut être levé.}:

- le \emph{bonum prolis} les enfants

- le \emph{bonum fidei} Unité - la fidélité

- le \emph{bonum sacramenti} l'indissolubilité\sn{Peut induire à
  une confusion~: l'indissolubilité n'est pas lié au sacrement mais vaut
  pour tous les mariages.}

D'où le fait que l'Église ne reconnaît pas le divorce. Le mariage tant
qu'il est valide est indissoluble.

Le sacrement renforce ce droit.

\hypertarget{le-ruxf4le-du-consentement}{%
\subsection{Le rôle du
consentement}\label{le-ruxf4le-du-consentement}}

C'est le consentement entre personnes valides qui fait le mariage. Ne
peut être suppléer par aucune personne humaine.

Consentement~: droit romain

Consommation du mariage~: droit germanique~: dès rapport physique, il y
a mariage

Compromis~: Alexandre III, 1170~: on maintient que c'est le consentement
qui fait le mariage mais il devient indissoluble dès qu'il y a eu
consommation.

\begin{itemize}
\item
  d'où mariage conclu (\emph{ratum)}
\item
  mariage consommé (\emph{consommatum)}
\end{itemize}

D'où un des rares cas où l'Église \textbf{dissout} le mariage (Canon
1061, 2)

\hypertarget{ii.-la-luxe9gislation-applicable-au-mariage}{%
\section{La législation applicable au
mariage}\label{ii.-la-luxe9gislation-applicable-au-mariage}}

\hypertarget{le-mariage-des-catholiques}{%
\subsection{Le mariage des
catholiques}\label{le-mariage-des-catholiques}}

Canon 1059~:

\begin{itemize}
\item
  si l'un des deux au moins est catholique, alors droit divin\sn{Positif
    tel que dans la Bible ou naturel, tel que l'indissolubilité} et
  droit canonique (purement humain).
\item
  Si mariage de deux personnes pas catholiques, alors jugement du
  tribunal ecclésiastique qui doit juger de la nullité du mariage
\item
   
  Nécessaire en cas de remariage avec un catholique
   
\end{itemize}

\hypertarget{le-mariage-des-non-catholiques}{%
\subsection{Le mariage des non
catholiques}\label{le-mariage-des-non-catholiques}}

\begin{itemize}
\item
  droit divin toujours naturel
\item
  droit civil
\item
  droit de son Église si il baptisé
\end{itemize}

\hypertarget{iii.-les-conditions-de-la-validituxe9}{%
\section{Les conditions de la
validité}\label{iii.-les-conditions-de-la-validituxe9}}

\hypertarget{les-empuxeachements-canoniques}{%
\subsection{Les empêchements
canoniques}\label{les-empuxeachements-canoniques}}

Obstacles qui empêchent la personne de contracter un mariage canonique~:

\begin{itemize}
\item
  \textbf{age} (16 ans / 14 ans -- la conférence épiscopale peut
  demander un age plus élevé~: cela ne change pas la validité mais la
  licéité (en particulier le mariage est valide mais pas selon les
  normes de l'Église)
\item
   
  purement de droit humain et donc on peut donner une dispense
   
\item
  \textbf{L'impuissance} (au sens strict), ie qui existe avant le
  mariage et perpétuel~: le mariage est invalide. Mais ce n'est pas la
  stérilité.
\item
   
  Condition minimale mais ce n'est pas la fécondité
   
\item
   
  Empêchement de droit divin
   
\item
  Une des personnes est mariée avec une autre personne (parce que
  l'Église ne l'a pas reconnue comme invalide)
\item
  \textbf{Les mariages dispars} (disparité de culte)
\item
   
  Peut être dispensé car de droit ecclésiastique
   
\item
   
  Car communauté de toute la vie
   
\item
   
  Si protestant ou orthodoxe, alors mariage \textbf{mixte} (pas de
  dispense)
   
\item
  L'une des personnes est \textbf{ordonnée}
\item
   
  Diacre ou prêtre~: l'ordre des sacrements est important.
   
\item
   
  Cf~: les Églises orientales: les futurs diacres se marient (vœux du
  célibat ecclésiastique).
   
\item
   
  Mais aussi les diacres veufs~! cependant, une dispense peut être
  accordée mais uniquement par Rome.
   
\item
   
  Idem pour les vœux de chasteté~: c'est Rome slt qui peut accorder une
  dispense.
   
\item
  L'empêchement de la consanguinité
\item
   
  Parenté biologique~: parenté en ligne directe
   
\item
   
  Cousin germain (4\textsuperscript{ème} degré) inclus est un
  empêchement
   
\item
   
  Droit ecclésiastique
   
\item
   
  Reconnaît l'adoption légale
   
\item
  Parrain et marraine~: parenté spirituelle~: n'est plus un empêchement
\end{itemize}

\hypertarget{le-consentement-matrimonial}{%
\subsection{2. Le consentement
matrimonial}\label{le-consentement-matrimonial}}

\begin{itemize}
\item
  liberté nécessaire
\item
  \textbf{Canon 1095~: le plus souvent utilisé comme chefs de nullité~:}
\item
   
  Usage suffisant de la raison (soit dès leur naissance, soit sénile)
   
\item
   
  Ils ne savent pas exactement à quoi ils s'engagent
   
\item
   
  souffrent d'~un \emph{grave} défaut de discernement concernant les
  droits et devoirs~: appréciation critique~: soit le mariage en tant
  que tel~; soit sur soit même. Qui se vérifie lors de la célébration du
  mariage.
   
\item
   
  On fait abstraction de ce qui est vécu après.
   
\item
   
  Immaturité
   
\item
   
  Nature psychique d'assumer les obligations du mariage
   
\item
   
  Nouveau car l'Église s'ouvre sur les sciences humaines~: évolution de
  la jurisprudence et qui trouve un reflet dans la législation.
   
\item
   
  Mais difficulté car les tribunaux ne sont pas des spécialistes de la
  psychiatrie.
   
\item
   
  Pas de frontières nettes
   
\item
   
  Réaliser ce que l'on a promis (au moins un minimum)
   
\item
   
  Drogué~; homosexualité~;
   
\item
   
  Névrose~; psychose.
   
\item
   
  Appeler sa mère plusieurs fois par jour
   
\item
  Une anthropologie sous jacente~:
\item
   
  On peut séparer chez une personne les différents aspects de la
  personnalité
   
\end{itemize}

\textbf{Interprétations~:} donc on peut qualifier certains tribunaux
comme «~laxistes~».

\begin{itemize}
\item
  D'autres vices de consentement~:
\item
   
  On peut se tromper sur l'identité de la personne
   
\item
   
  Sur la qualité de la personne~: ex~: une personne pense épouser une
  personne catholique et en fait elle ne l'est pas. Virginité de la
  personne.~; stérilité~; Mais il faut que ce soit un élément essentiel
  de la qualité de la personne.
   
\end{itemize}

\hypertarget{la-forme-canonique}{%
\subsection{la forme Canonique}\label{la-forme-canonique}}

Canon 1108

\begin{itemize}
\item
  Curé ou ordinaire du lieu
\item
  prêtre ou diacre délégué par le curé ou l'ordinaire
\end{itemize}

Demander et recevoir au nom de l'Église le consentement des époux

\begin{itemize}
\item
  présence de deux témoins
\end{itemize}

Curé ou ordinaire du lieu~: Il le fait dans la limite de son territoire.

\begin{itemize}
\item
  leurs sujets
\item
  ceux qui ne le sont pas mais sont de rites latins
\end{itemize}

→ \textbf{titre de compétence}

Pour les autres, les vicaires, les diacres, on besoin d'une dispense.

\begin{itemize}
\item
  acte de délégation
\end{itemize}

Ce n'est pas le lieu mais la personne qui célèbre qui fait la
canonicité.

Canon 1117

\begin{itemize}
\item
  doivent se marier à l'Église toute personne de rite catholique.
\item
  Si elles ne se marient pas à l'Église, la forme canonique n'est pas
  reconnue et donc l'Église ne reconnaît pas ce mariage
\item
   
  Alors les personnes peuvent se remarier à l'Église.
   
\item
  on peut demander une dispense de la forme canonique du mariage~: on
  veut se marier dans le temple protestant.
\item
   
  Dès qu'il y a eu dispense, le mariage est valide
   
\item
  Mais si acte formel de ne plus être considéré comme catholique
\item
   
  Lettre à son curé~: On met alors une annotation dans la marge
   
\item
   
  Témoin de jehovah~; orthodoxe.
   
\item
   
  Alors puisqu'ils ont quittés l'Église, le mariage civil est valide et
  même sacramentel car ils sont tout les deux baptisés~!~!~!
   
\end{itemize}

Forme extraordinaire Canon 1116

Devant les seuls témoins, en cas de danger de mort. Une situation qui ne
durera un mois.

\begin{itemize}
\item
  né pour des motifs pastoraux~: dans les pays communistes,.
\end{itemize}

\hypertarget{les-mariages-mixtes}{%
\subsection{Les mariages mixtes}\label{les-mariages-mixtes}}

Ce qui est demandé, ce sont des promesses de la personne catholique~:

\begin{itemize}
\item
  de faire son possible pour éduquer son enfant dans la foi catholique
\item
  de faire son possible pour garder la foi
\end{itemize}

Le conjoint a juste à être informé.

\textbf{Un grand changement car le code de 1917} essayait de convertir
l'autre.

\hypertarget{iv.-la-dissolubilituxe9-de-certains-mariages}{%
\section{La dissolubilité de certains
mariages}\label{iv.-la-dissolubilituxe9-de-certains-mariages}}

 

Questions à creuser~:

\begin{enumerate}
\def\labelenumi{\arabic{enumi}.}
\setcounter{enumi}{5}
\item
  pratique du baptême
\end{enumerate}

\begin{itemize}
\item
   
  faut il des signes minimum d'adhésion des parents pour la pratique du
  baptême~?
   
\end{itemize}
