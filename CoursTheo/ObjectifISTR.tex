\chapter{Objectif de la licence de théologie ISTR}



\begin{quote}

"Bence, me dit ensuite Guillaume, est la victime d'une grande luxure, qui n'est pas celle de Béranger ni celle du cellerier. Comme de nombreux chercheurs, il a la luxure du savoir. Du savoir en soi. 
Exclu d'une parte de ce savoir. il
voulait s'en emparer. Maintenant, il s'en est emparé. Malachie connaissait son homme et il a utilisé le meilleur moyen
pour ravoir le livre et sceller les lèvres de Bence. Tu me
demanderas à quoi bon contrôler une telle réserve de savoir
si on accepte de ne pas le mettre à la disposition de tous les
jures. Mais c'est précisément pour ça que j'ai parlé de
luxure. Elle n'était pas luxure, la soif de connaissance de
Roger Bacon, qui voulait user de la science pour rendre plus
heureux le peuple de Dieu, et ne cherchait donc pas le savoir
pour le savoir. La curiosité de Bence n'est qu'insatiable
orgueil de l'intellect, une façon comme une autre, pour un
moine, de transformer et apaiser les désirs de ses reins, ou
l'ardeur qui fait d'un autre un guerrier de la foi ou de l'hérésie. Il n'y a pas que la luxure de la chair. Luxure, que celle de
Bernard Gui, luxure altérée de justice qui s'identifie à une
luxure de pouvoir. Luxure de richesse, que celle de notre
saint et non plus romain pontife. Luxure de témoignage et de
transformation et de pénitence et de mort que celle du cellérier dans sa jeunesse. 
Et celle de Bence est une luxure de
livres. Comme toutes les luxures, comme celle d'Onan qui
répandait par terre sa propre semence, c'est une luxure stérile, et elle n'a rien à voir avec l'amour, pas même avec
l'amour charnel. \cite{Eco:NomRose} P.425
\end{quote}
