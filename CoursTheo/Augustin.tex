\chapter{Saint Augustin [354 - 430] face aux défis de son temps}
\mn{Note de cours M. Neusch Marcel NEUSCH
Institut Catholique de Paris 1997-1998
Pro manuscripto
}


 







 
\begin{quote}
    «Pour en revenir aux trois points qui comprennent le culte dû à Dieu, savoir la foi, l'espérance et la charité, il est aisé de dire ce qu'il faut croire, ce
qu'il faut espérer, ce qu'il faut aimer. Mais défendre cet ensemble contre les objections de ceux qui pensent autrement exige un exposé doctrinal plus difficile et plus étendu. Pour le fournir, ce n'est pas un bref Manuel qu'il faut prendre en mains; c'est
un grand zèle qu'il faut allumer dans le coeur." (Enchiridion, 1, 6, BA 9)
\end{quote}









 


\section{Augustin	(354-430)  face aux défis de son temps}

Sous ce titre, je voudrais regrouper un certain nombre de
problemes doctrinaux auxquels Augustin fut confronté au cours de son existence. C'est
une bonne manière de le connaître, non la seule. Avant d'être le polémiste engagé sur tous les fronts au service de la foi chrétienne, Augustin est un pasteur, soucieux de prêcher l'Evangile au petit peuple d'Hippone.	S'il est considéré comme le "maitre à penser" de l'Occident, c'est en raison de l'influence qu'a exercée sa theologie. Il a du forger sa pensee progressivement, au contact de ses adversaires. Il n'a jamais occupé une "chaire universitaire" mais il a toujours réfléchi à partir des questions
qui se posaient concrètement  dans les débats qui agitèrent l'Eglise de son temps.
)	Naturellement,	élaborée	dans le feu de l'actualité, sa pensée n'a pas	 toujours l'équilibre que pourrait requérir une réflexion "universitaire", comme celle d'un saint Thomas d'Aquin. Elle gagne en vie, et sans doute aussi en virtuosité!

C'est surtout à partir de son épiscopat qu'Augustin sera amené à occuper une place tout à fait capitale dans les débats de l'époque. Même s'il n'est pas le premier évêque de l'Afrique du Nord - le primat revient à l'évêque de Carthage - , il est sur la brèche dès qu'un débat surgit. Son collègue de Carthage, Aurelius, le laissait volontiers monter en première ligne quand il s'agissait de défendre la foi dans les débats publics, notamment contre les Donatistes. Le souci d'Augustin sera essentiellement de présenter la "doctrine chrétienne", et très vite, il connut le succès même au-delà des cercles chrétiens, si l'on en croit son biographe, Possidius, qui écrit:
\begin{quote}
    " Ses livres et traités qui, par un effet admirable de la grâce de Dieu, se suivaient avec rapidité, étaient appuyés sur de nombreux raisonnements et sur l'autorité des Ecritures; et les hérétiques eux-mêmes accouraient pour en entendre la
lecture avec le même empressement que les catholiques, et tous ceux qui le voulaient
et le pouvaient avaient recours à des secrétaires pour recueillir ses propos. Aussi	)
vit-on bientôt se répandre et se manifester dans toute l'Afrique l'éclatante doctrine et la bonne odeur du Christ; et cette nouvelle remplit aussi de joie l'Eglise de Dieu outre-mer" \sn{Possidius, \textit{vie de saint Augustin}, c. 7. Parmi les vies aujourd'hui disponibles, signalons: John J. O'MEARA, \textit{La jeunesse de saint Augustin}, \textit{Introduction à la lecture des Confessions}, Pion, 1958, rééd. Cerf, 1988.   Peter BROWN, La vie de saint Augustin, Seuil, Paris, 1967.   Vera PAAONETTO, \textit{Augustin, le message d'une vie}, Le Centurion, 1981.· Agostino TRAPE, \textit{Saint Augustin, l'homme, le pasteur, le mystique}, Fayard, 1988.   Marcel NEUSCH, Augustin, un chemin de conversion, DDB, 1986. - Id.
\textit{Initiation à saint Augustin. Un maitre spirituel.} Cerf, 1996.}
\end{quote}


Pour situer correctement l'activité d'Augustin, il conviendrait d'évoquer ce qui est l'essentiel de sa vie de pasteur ; la prédication au cours de la liturgie, une correspondance avec le monde \sn{Cf. André MANDOUZE,\textit{ Saint Augustin. L'aventure de la raison et de la grâce}. Et. Aug. 1968. pages
539 SV.}, les  multiples affaires qui le sollicitaient dans son diocèse, le soucis des pauvres, l'exercice de la justice, etc.  Sans aborder tous les problèmes, notamment ces problèmes pastoraux dont il eut à traiter, je voudrais 
donc me limiter à évoquer les grands débats de l'équipes, et trouver à travers eux quelques themes essentiels de la pensée d'Augustin. On peut  ramener ces débats à cinq, en  fonction  des adversaires  qu'il  eut  à  combattre,  mais en soulignant positivement les enjeux théologiques de ces débats:
\begin{itemize}
    \item 1. Adversaire des Manichéens : le champion de la liberté..
\item II. Adversaire des Platoniciens : témoin de l'humilité de Dieu.
\item III. Adversaire des Donatistes: défenseur de l'unité chrétienne.
\item IV.	Adversaire des Païens: le citoyen de la cité de Dieu.
\item  V.	Adversaire de Pélage : le champion de la grâce.
\end{itemize}


\section{Adversaire des manichéens, \textit{le champion de la liberté}}
 


Pendant neuf ans, Augustin a été l'auditeur de cette secte, de 372 à 382. Il s'est "laissé séduire", et il s'est fait "séducteur" (IV, 1, 1 ), déployant tout son talent au service de la secte. Devenu adepte des manichéens en 372 (Ill, 4, 10), il a pris progressivement ses distances, la rupture devenant effective à partir de son arrivée à Rome (383). Il ne saurait être question de donner un exposé exhaustif de Mani ni du manichéisme. \sn{Henri Charles PUECH, Le manichéisme, son fondateur, sa doctrine, Musée Guimet, 1949	Julien
RIES,\textit{ Les études manichéennes. Des controverses de la Réforme aux découvertes du XXe}
, 1988.   Surtout François DECRET, L'Afrique manichéenne (IVe. Ve siècle).2 tomes Etudes augustiniennes, 1978.}

Né le 14 avril 216 en Perse, Mani vécut au contact de chrétiens. Il reçut à l'âge de douze ans une série de visions qui firent de lui le fondateur d'une nouvelle religion. Il prêche sa doctrine qui se répand bientôt dans tout l'empire romain. Persécuté, il subit la prison et un horrible matyre qui le conduisit à la mort le 26 février 277. Dès 300, sa doctrine se répand en Asie Mineure et même en Afrique où elle est attestée à Carthage dès cette date.

La vie d'Augustin dans la secte nous est peu connue. li donne une description assez
)	précise de la doctrine	et	des exigences qu'imposaient les Manichéens à leurs adeptes. Dans une lettre qui nous renseigne en particulier sur la distinction entre les
auditeurs et les élus, lui-même étant toujours resté au rang d'auditeur. Augustin regardera le manichéisme comme la \textit{pestientissima haeresis}, l'hérésie par excellence\sn{Pour une vue rapide, voir DECRET, Le christianisme en Afrique du Nord ancienne
1996. p. 201 sv.} . 
D'un façon générale, on peut relever avec Puech les trois aspects principaux :
\begin{itemize}
    \item C'est une religion \textit{universelle}, qui s'étend à tout le monde connu, de la Chine à l'Espagne, et se présente comme une révélation plénière, alors que les les autres religions comme le christianisme, ne sont que des révélations partielles. Mani est le \textit{sceau des Prophètes}, une formule qui avant d'être appliquée à Muhammad, qualifie Mani, mais sans doute pas avant la pénétration et le contact du manichéisme avec la religion musulmane. \sn{g Cf. Gedaliahu Guy STROUMSA, Savoir et salut, Cerf, 1992, p. 276 sv.}
    \item c'est une religion \textit{missionnaire} qui se donne comme objectif la rédemption du monde entier; on en trouve très tôt les traces en Egypte ou il se présente d'ailleurs non pas en opposition au christianisme mais comme le \textit{vrai christianisme} alors qu'il considère les catholiques comme des judaisants \sn{lb. pp.315 SV.} c'est à dire comme une secte du judaïsme
    \item c'est une religion du \textit{livre } sans sacrement et presque sans rituel. Tout est centré sur la prédication du contenu des sept livres que forment a bibliothèque manichéenne, et les livres chrétiens sont expurgé en fonction de ces livres canoniques.
\end{itemize}



 
Si l'on regarde plus précisément ce. qui a	pu séduire Augustin, on deviner certains attraits bien que, dès qu'il en parle lui-meme, il soit plus prompt à accabler ses coreligionnaires d'autrefois  d'injures que de rendre Justice ç ce qui a suscité son attrait.
Il décrit les Manichéens comme «délirants de superbe, charnels et bavards à l'excès » (III 6 1O) : trois traits dans lesquels on reconnaît	les trois
vices qui forment en nous selon Augustin, une véritable anti-trinité. Les Manichéens sont affligés de tous les trois : l'orgueil, la sensualité, la curiosité ! 
 
Ce ton polémique ne doit pas nous tromper. Parmi les traits essentiels de cette doctrine qui ont a Augustin, relevons les suivants :

\subsection{Un  christianisme  rationnel,  qui  dispense  de croire.}

Si Augustin s'est rallié au manichéisme plutôt qu'au catholicisme, c'est qu'en Afrique, on ne faisait guère la différence.
\begin{quote}
    «On pouvait passer du Christianisme au Manichéisme et vice-versa sans attirer autant l'attention qu'en passant du Catholicisme au Donatisme.\sn{O'MEARA, oc pp. 79 sv.} »
\end{quote} 

Les Manichéens vénéraient en particulier saint Paul. Mani «se voulait, comme Paul, plus que lui, puisque son Jésus à lui était Yésu Ziwa, le Jésus-de-Gloire, «apostolos Jésou Christou»\sn{STROUMSA, oc p. 301.}   On peut cependant trouver plusieurs motifs au choix d'Augustin.

\paragraph{se présente comme le vrai christianisme}	Si, après sa lecture de l'Hortensius qui venait d'éveiller en lui l'enthousiasme de la sagesse, il a manifesté une préférence pour le manichéisme, c'est sans doute parce que le manichéisme passait pour le «vrai» christianisme. Dès leur pénétration en Egypte, où il s'implanta du vivant de Mani, grâce aux marchands, les manichéens présentèrent leur message comme plus pur que les traditions «enjuivées» des catholiques. Ils considèrent ces derniers comme  des «semichristiani» alors qu'eux-mêmes se présentent comme des « \textit{Veri christiani}» \sn{Cf. Augustin, Contra faustum 1, 2-3. Cf. STROUMSA, oc p. 301 sv.}  On voit	qu'en pénétrant en Afrique, moins de vingt ans après la mort de Mani, cette doctrine a
intégré	un ensemble de données du christianisme, mais en	rejetant certaines, en
critiquant d'autres, en tout cas en prétendant offrir le  «vrai» christianisme, débarrassé dee l'Ancien Testament et rendu totalement transparent a la raison.
Voici comment Augustin le présente:	
\begin{quote}
    «Ceux que chez eux on appelle auditeurs mangent de la viande, cultivent la terre et prennent femme s'ils le désirent. Mais ceux qu'on appelle élus ne se permettent rien de cela. Les auditeurs ploient le genoux devant les élus implorant d'eux tous, et non des seuls évêques, prêtres ou diacres, l'imposition des mains. Avec les élus, les auditeurs prient et adorent le soleil. Avec eux ils jeûnent le dimanche. Avec eux ils ajoutent foi à tous ces blasphèmes qui rendent détestable l'hérésie manichéenne :. negation de la naissance virginale du Christ· refus de croire véritable sa chair mais affirmation d'une chair factice et par suit; simulacre de la passion et rejet de la résurrection; blasphème envers les patriarches et les prophètes. La loi donnée par le serviteur de Dieu, Moïse, ils disent qu'elle provient du prince des ténèbres. lls affirment que sont de la substance de Dieu l'âme humaine et l'âme animale : Oui, ils
font de ces âmes une portion de Dieu. Ils font combattre le Dieu bon et vrai avec la nation des ténèbres; se mêler une part de lui-même aux principes des ténèbres et cette part ficelée et polluée dans le monde entier, voilà qu'elle se purifie en nourrissant les élus ou en aboutissant dans le lune et le soleil. A défaut de cette purification, cette parcelle divine sera à la fin des temps à tout jamais dans les fers. C'est dire qu'ils se figurent muable, corruptible et contaminable ce Dieu dont une partie a pu être amenée à un si grand mal et qui ne peut se purifier tout entier, même à la fin du monde, de cette lamentable et impure promiscuité.\sn{Lettre 236, in B. A. 17, p. 12-13.} »
\end{quote}


\paragraph{rationalité}	Si les manichéens attiraient surtout les gens cultivés, comme Augustin, c'est qu'ils avaient la prétention d'offrir une vérité rationnelle. Ce rationalisme était une forme de gnose. A la différence du catholicisme, qu'ils considèrent comme une secte juive (c'est-à-dire païenne), ils faisaient appel dans leurs argumentation non pas à l'autorité et à la foi, mais à la raison et à la recherche personnelle. Faustus, leur évêque, considère qu'en tant que gentil, lui-même est né non pas sous la loi juive, mais sous la loi de la nature, c'est-à-dire de la raison. Le grief qu'il fait aux chrétiens, c'est de préférer l'autorité de l'Ecriture à celle de la raison : 
\begin{quote}
  « ... Vous qui recevez toute chose sans la soumettre à l'examen, condamnant l'usage de la raison, qui	est une prérogative de la nature humaine, et considérant comme impie de distinguer entre vérité et mensonge... » (Contra Faustum XVIII, 3)\sn{STROUMSA, oc p. 345 sv.}   
\end{quote}
Son épistémologie rationnelle
implique que toute chose peut être expliquée. Les miracles, la foi elle-même sont inutiles. On comprend par là qu'un homme comme Augustin, désireux de tout comprendre, ait été séduit.

\paragraph{Foi et Raison}	C'est sur ce point du rapport  entre  foi  et  raison que, après sa conversion, il mènera la bataille contre les manichéens. Il soulignera toujours leur étroite connexion, en faisant observer que la foi est première, mais qu'elle appelle la raison dans son sillage : il faut commencer par croire si l'on veut comprendre. Il dira à Faustus :  

\begin{quote}
    «L'esprit chrétien doit d'abord être nourri à la foi simple, de sorte à pouvoir devenir capable de comprendre les choses célestes et éternelles.» (Contra Faustum, XII, 46) \sn{}
\end{quote}


%%% PAGE 5
 
F	E	·tures et constitue
aus um XII, 46)12. La foi implique de croire à la tradition des  lera sa pensée en u e etapenécessaire sur le chemin de l'intelligence.AuguStm f I f	ut	placer la foi à sant qu' I faut croire pour comprendre. Ou, plus exacte?'ent,v r:e (fides ex auditu)
1 intersection de deux actes d'intelligence : il faut recevo  le	f	.	r	progresser
par l'intelligence, et il faut que l'intelligence prenne appui sur la 0'	verà croire, dans les mystèes de Dieu et de l'homme : «Comprends ma parole pour	,,
et crois à la parole de Dieu pour arriver à comprendre.» (Sermon 43, 9)
2.	Une Bible expurgée, sans l'Ancien Testament.

Conséquence de ce rationalisme les manichéens offrent une Bible exp r ée,àqu i
 
n'en retient que ce qui est compatib' le avec leur propre certain nombre de textes du Nouveau Tetament.
 
d octn· ne. EeII se hmtte	un
 

a)	D'abord, elle est amputée de l'Ancien Testament. Faustus éten nt n'avoir rien à voir avec les Juifs. S'il rejette l'Ancien Testament, c'est qu 11 le JU foncièrement immoral (une critique dont Augustin se souviendra plus tard et qu t1
essaiera toujours de réfuter). Faustus est on ne peut plus clair  :
\begin{quote}
    «Vous me demandez si je crois à l'Ancien Testament. Bien sûr que non parce que je n'observe pas ses préceptes. Ni vous, j'imagine... » (Contra Faustum VI, 1 ). «Tout ce que nous recherchons chez les prophètes est prudence et vertu, et un bon exemple, ce que vous savez bien qu'on ne saurait trouver chez les prophètes,» (ib. XII, 1) «Je le dis encore une fois, l'Eglise chrétienne, qui comprend plus de gentils que de juifs, est en mesure de ne rien devoir aux témoignages hébraiques.» (ib. XIII, 1 ).

\end{quote}

b)	Quant au Nouveau Testament, ils n'en retiennent pas grand chose, sinon en le réinterprétant. Dans le texte cité plus haut, on voit que les principales vérités chrétiennes, comme l'incarnation, la passion et la résurrection, sont ruinées.	Pour fonder leur prétention à expurger le Nouveau Testament, les manichéens font valoir qu'il a été écrit par des disciples tardifs, encore entaché de judaïsme, qui n'ont pas vraiment compris ce que le Christ a voulu dire, si bien qu'ils ne méritent qu'une confiance limitée. A preuve toutes les contradictions qui s'y trouvent. Les manichéens présentent donc une Ecriture expurgée. C'est une théologie libérale avant la lettre, à la manière de Renan. Ils avaient eu l'intuition de garder !'Ecriture, ce qui pouvait séduire les chrétiens, mais en même temps de la rationaliser, ce qui devait séduire un intellectuel comme Augustin toujours réticent devant une Ecriture qui n'était pas digne de Cicéron, et dont le langage anthropomorphique n'était pas digne non plus de Dieu.

c)	C'est Ambroise qui le réconciliera avec !'Ecriture. A la lecture manichéenne il opposera une autre manière de lire, qui fera fond sur le	sens	spirituel. Augus in indiquera ainsi un certain nombre de principes de lecture : 1 ) il distinguera à lasuite, d'Ambroise les différents sens de !'Ecriture; et 2) il dira que, comme pour n'importe quel auteur,	11 faut l'aborder avec des dispositions bienveillantes et à l'aide d'un maître compétent (BA 8, p. 196). Il reproche par exemple aux manichéens, en ce qui concerne l'Ancien Testament, de ne pas distinguer entre préceptes moraux et préceptes
12 Sur le couple auctorltas I ratio, on peut consulter le De vera religlone 28 52 BA ap	494 et 570
«D'une façon générale, on peut définir l'autorltascomme l'ensemble des garanties obj.	ecti:ves·	· · 1- ·
 
les hommes à
 
r confiance à un objet de connaissance, tandis que la ratio est caracté · é
 
l'Idée de réflexion.....	ris e par
13 Voir aussi Enar. in Ps 118; Sermo 18, 3.
5
 

s mboliques : «Par exemple, "Tu ne convoiteras point" est un précepte moral; «Tu circonciras tout mâle au huitième jour» est un précepte symbolique... » . C'est grâce à l'exégèse allégorique qu'Augustin surmontera un certain nombre de difficultés de lecture, sans rejeter la lettre : «En ne faisant pas cette distinction, les manichéens, et tous ceux qui trouvent à redire aux écrits de l'Ancien Testament, ne voyant pas que toute observance fixée par Dieu dans le cadre de l'ordre antérieur était une ombre des choses à venir..., ils les condamnent... » (VI, 2)14
3.	Une réponse simple à la question du mal.
Le problème du mal a été l'un des plus angoissants qu'Augustin ait eu à affronter. Il ne cessera d'en être hanté jusqu'à la fin de ses jours. Les manichéens lui ont procuré une explication élégante par leur dualisme.

a)	Que disent-ils ? Que le mal résulte d'un combat archétypal, originel, entre deux  principes  co-éternels, l'un bon et l'autre mauvais. C'est de ce combat qu'est issu la race humaine, l'homme étant constitué par le mélange des deux, le corps qui est le mal et l'âme parcelle divine en nous. La dualité de la volonté en particulier s'expliquerait par la présence en l'homme de deux âmes, l'une bonne, l'autre mauvaise (VIII, 10, 22). Il s'agit donc d'un dualisme radical. A vrai dire, la doctrine est parfois plus subtile, cherchant à sauvegarder le monothéisme. Voici comment s'exprime Fautus:
\begin{quote}
« Croyons-nous en un Dieu ou en deux ? En un seul, bien entendu. Il est vrai que nous croyons en deux principes; mais l'un, nous l'appelons Dieu, et l'autre hu I è, oo pour utiliser le vocabulaire populaire commun, le diable... Pensez-vous que nous devions les appeler tous deux des dieux parce que nous attribuons, comme il se doit, toute la puissance du mal à la hu I è, et toute la puissance du bien à Dieu ? » ( Contra Faustum XXI, 1 )15     
\end{quote}


b)	Cette explication dualiste, qu'il s'agisse de deux dieux ou de deux principes co­ éternels, devait aussi s'avérer à la longue insuffisante, d'abord parce qu'elle dépouille l'homme de sa liberté, ensuite parce qu'elle se trompe sur la nature du mal, et sur sa véritable origine. Ce qui fera faire à Augustin un pas décisif, c'est donc une double découverte, l'une sur la nature du mal, et l'autre sur son origine :

-	d'abord la nature du mal. "Qu'est-ce que le mal ?" ( De moribus, 2, 2). Non pas une substance, mais un manque, une déficience, une absence. A proprement parler, il n'est "rien". Ou, si l'on veut, il est l'absence d'un bien.  Or, les manichéens font de l'homme une chose manipulée par une puissance étrangère, une substance "aliena". C'est grâce au platonisme qu'il surmontera cette première difficulté.

-	ensuite,  l'origine  du  mal. "D'où vient le mal ?" (VII, 5, 7). Il est le fait, non d'une substance, mais d'une volonté libre, "sive in origine sive in opere" (B.A. 17, p. 47).  C'est Ambroise qui attirera l'attention sur ce fait et l'amènera à la découverte de la liberté. "Il n'y a nulle part de péché sinon dans la volonté"16 , dira Augustin, que ce soit la volonté personnelle (in opere), ou la volonté antécédante
1  Cf. STROUMSA, oc pp. 348 sv.
15 Cf. STROUMSA, oc pp. 346·247 .
,e Les Révisions 1, 15
6
(	C
 
 




l
d'Adam (in origine). Toute autre explication est de mauvaise foi, puisqu'elle aboutit à
nier la respansabilité et à excuser le péché. La découverte du péché originel permettra à Augustin de surmonter les difficultés d'un mal en nous sans nous, tout en évitant les impasses d'une nature "étrangère".

4.	Le salut par la gnose.

Ce qui intéresse finalement Augustin, c'est au-delà de la conception cu mal, la question du salut. Là encore, les manichéens ont une explication séduisante.

a)	Ils conçoivent l'homme comme un "divin prisonnier de la matière" (B.A.1 7 p. 15).  Le  salut  consiste  dès  lors  â  libérer  en  nous  la  parcelle  divine enfouie dans la matière. Augustin devait être tout disposé à accepter cette gnose prêchant le salut par la connaissance. Ce salut doit être conquis par chacun. Si Jésus souffre  (Jesus patibilis peut être dit «la vie et le salut des hommes», cf Contra
Faustum XX, 2), ce n'est pas pour nous sauver, mais il s'agit d'une sorte d'évasion personnelle de la matière. D'ailleurs, il n'a pas réellement souffert, puisque son corps
était pure apparence11  	Ici encore, on ne fait pas appel à la foi, mais à l'explication	(
rationnelle.

b)	Ce salut est lié très étroitement à une morale  exigeante. Se libérer suppose en effet la soumission à une ascèse stricte. Mais les Manichéens, soucieux de s'adapter à la situation de chacun, distinguaient entre deux catégories de membres, les élus qui, en particulier par leur refus du mariage, s'interdisent de perpétrer une création qui est mauvaise par essence, et les auditeurs pour lesquels étaient prévus des accommodements. En restant au stade d'auditeur, Augustin montre qu'il était loin d'accepter toutes les exigences.

« Le "sceau du sein" enfin, cherche à empêcher la propagation du mal, c'est-à­ dire la reproduction de l'espèce. Toutes les relations sexuelles sont interdites, et spécialement l'institution du Diable qu'est le mariage. Nos parents nous ont rendu le plus mauvaisservice en nous donnant la vie. C'est pourquoi il nous faut les quitter pour suivre le Christ...Un homme pèche moins gravement avec une concubine qu'avec sa femme car il y a plus de mal à vouloir propager l'espèce humaine qu'à rechercher seulement le plaisir. Le Manichéen doit donc à tout prix éviter la paternité : pour cela tous les moyens sont bons, mais la continence totale est le meilleur'8 »

c)	C'est la découverte du Christ médiateur qui lui permettra de surmonter l'impasse manichéenne : l'homme est malade dans sa volonté, par sa faute, et c'es le Christ médecin qui vient le guérir. De ce point de vue, Augustin s'apercevra assez vite que les manichéens ne pouvaient pas tenir la promesse de vérité qu'ils faisaient à leurs adeptes. Il les accusera de lui raconter des "fables", c'est-à-dire des mythes. C'est en tous les cas ce point qui éveillera le soupçon et la déception . «Ils disaient : "Vérité, vérité!" Et ils me parlaient beaucoup d'elle, et elle n'était nulle part en eux..." (VI, 7 0) "Chez vous, ...il n'y a rien d'autre qu'une retentissante promesse de la


'1 Cf. De diversis quaest. 14. Cf. STROUMSA, oc p. 347, surtout les notes.
19 O' MEARA, oc p. 97. Cf. Peter BROWN, Le renoncement à la chair. Virignité, célibat et continence dans le christianisme primitif. Gallimard, 1995, p. 252 sv.

7
(
 


vérité...; vous ne faites que promettre la vérité sans la montrer...» 19	Ce qu'il retirera de cet échec, c'est trois vérités fondamentales :

-	Il faut croire pour comprendre. Augustin est convaincu que, pour comprendre et connaitre en vérité, il est "nécessaire de croire à une autorité" (B. A. 17, 787). Concrètement, cela signifie : accueillir le Christ, notre science et notre sagesse. C'est seulement à partir de ce point de départ que l'on fera des progrès dans l'intelligence de la vérité. Non seulement les manichéens ne tiennent pas leur promesse de lui expliquer la vérité rationnellement, mais ils lui "enjoignent de croire" cela
même que par ailleurs la science, l'astrologie en l'occurrence, explique depuis longtemps (V, 3, 4).

-	La vérité n'est pas réservée à une élite. Tout homme peut y accéder, grâce à la foi.  A la "vivacitas intelligendi" des intellectuels, une vivacité qui s'est bien souvent égarée dans l'erreur, Augustin oppose la "simplicitas credendi", la simplicité dans la foi. C'est grâce à l'autorité de la foi que le genre humain dans sa totalité, pas seulement l'élite, a eu accès à la vérité et que tous possèdent une même
(	certitude du salut.
Pour lire /'Ecriture, il faut une clef. Nous l'avons déjà dit. Il n'est pas nécessaire d'éliminer l'Ancien Testament, comme le font le manichéens, mais il faut lire le texte en dépassant la lettre. C'est Ambroise, un virtuose de l'exégèse allégorique, qui ouvrira son intelligence à dépasser la lettre et à s'attacher au "sens spirituel" (VI, 5, 8;  VI, 4, 6; V, 14, 24).

Conclusion. Le zèle d'Augustin pour la secte est désormais "bloqué", même s'il y reste attaché comme à une "solution provisoire" (V, 7, 1). Avant de rompre complètement, il rencontre le maître de la secte, Faustus (V, 6, 10). A la fin du livre V, il nous décrit ses hésitations : le catholicisme "ne me semblait plus un vaincu, il n'apparaissait pas encore comme un vainqueur" (V, 14, 24). Il prend alors la décision de s'inscrire comme "catéchumène dans l'Eglise catholique, qui se recommandait de mes parents, aussi longtemps qu'une certitude ne me montrerait pas dans sa lumière où diriger ma course" (V, 14, 25).
On s'est souvent demandé si Augustin n'est pas, malgré tout, resté marqué par son	(
passage dans le manichéisme. De fait, son insistance sur la "massa damnata", son
dualisme des deux cités qui se combattent, peuvent apparaître comme des survivances du manichéisme. Le moins que l'on puisse dire est qu'il en a écarté toute interpréation dualiste, telle que nous la trouvons dans le manichéisme. Surtout, le dépassement du manichéisme s'est fait sur un point essentiel, décisif, qui marquera toute sa théologie : la liberté de l'homme. Le noeud du manichéisme était là. Toutes ses explications convergeaient vers ce noeud gordien : la liberté devenait une illusion. C'est sur ce point qu'Augustin ne cesse de le combattre. En toute circonstance, il en appelle à la liberté de l'homme.






,g  Ep. du fondement, 4, 5, BA 17 p. 399.
8
(

\section{Adversaire des platoniciens  - \textit{Témoin de l'humilité de Dieu}} 

Augustin découvrit le platonisme à Milan, au contact des chrétiens qui avait déjà
fait la synthèse entre le christianisme et les idées platoniciennes. On sait, par les travaux de Courcelles, qu' Ambroise s'inspirait, dans sa prédication, des Ennéades de
Plotin, dont il recopiait des passages entiers dans ses sermonsm. Si Ambroise lisait couramment le grec, c'est surtout avec les traductions  des  oeuvres  de  Plotin par Marius Victorinus que !"influence néoplatonicienne s'affirmera dans le christianisme. Il est certain qu'Augustin a eu entre les mains des «Libri Platonicorum» (Conf. VII, 1O, 16), sans qu'il soit possible de dire s'il s'agit de Plotin ou de Porphyre, ou des deux. Augustin dit qu'ils lui ont été confié par «un homme gonflé d'un monstrueux orgueil» (VII, 9, 13) . Quand il se convertira, il deviendra "l'adepte d'un néoplatonisme chrétien déjà fortement élaboré \sn{
Recherches sur les Confessions, Ed. de Broccard, 1968. p. 136.}"

Est-il légitime de parler d'un	"cercle	milanais" de chrétiens, comme il le fait, cercle dans lequel les écrits de Plotin "semblent avoir joué le rôle d'un centre d'intérêt autour duquel des hommes de conviction diverse pouvaient sympathiser avec cette discrète tolérance qui convient aux hommes distingués" ?	L'expression n'est sans doute pas la plus heureuse, car de cercle, il n'y en eut point.
\begin{quote}
  «Il y avait incontestablement à Milan, au moment de la conversion d'Augustin,	une certaine actualité du néoplatonisme; on peut mettre en doute l'existence d'un 'cercle' à proprement parler; car, dans l'état actuel de nos connaissances, on ne voit se dessiner que des rapports d'Augustin avec tel et tel, sans qu'on sache rien d'éventuels rapports néoplatoniciens entre les autres membres \sn{ G. MADEC, Platonisme des Pères, in Catholicisme, XI, 491-508.{}} »  
\end{quote}


\subsection{Un incroyable  éblouissement}

Il est certain qu'Augustin fut conquis par cette lecture des livres platoniciens. Ce fut  une  véritable  délivrance  de  l'esprit  dans la mesure où il dissipa son
«\textit{matérialisme  manichéen}». L'attitude d'Augustin à l'égard du platonisme fut d'abord une indéniable séduction par cet effet libérateur, avant de se transformer en une sévère contestation en raison du risque grave qu'il représente comme concurrent du christianisme, enfin une appréciation plus mesurée des ces philosophes «les plus proches de nous».  La lecture des "libri platonicorum" ( VII, 10, 16), eut d'abord t un effet  libérateur : il fut réellement ébloui : "Averti par ces livres, je rentrai en moi-même...". Nous avons, dans le Contra Academicos (Il, 2, 5) les traces de l'enthousiasme que suscita cette lecture des livres de philosophie,  enthousiasme
comparable à celui qu'il éprouva autrefois à la lecture de Cicéron et qui sera tempréré par la suite	·
\begin{quote}
    « Mais vo1c1 que certains livres bien remplis, comme dit Ce/sinus, répandirent sur nous les parfums de l'Arabie et distillèrent sur cette petite flamme quelques gouttes de leur précieuse essence : ce fut une chose incroyable, Romanianus, incroyable que l'incendie qui en résultat, et bien au-dessus de tout ce que tu peux penser...Les honneurs, la pompe humaine, le désir de vaine gloire, enfin les douces
::ches de :ette vie mortelle, rien de tout cela pouvait-il alors me toucher ? Sans	(
emparer, Je rentra, en moi tout entier... » (Il, 2, S)
\end{quote}


,. L  le ture ne l'orienta pas d'emblée vers le christianisme. Au contraire, il semble qu il ait decouvert dans le néo-platonisme assez de satisfactions pour le combler au point qu'il n'eut aucune envie de chercher ailleurs plus loin. Il avoue clairement dans le passage que nous venons de lire :«Je ne réfléchissais qu'incidemment, je l'avoue, à cette religion qui m'avait été enseignée dès l'enfance et comme enfoncée jusqu'à la moelle : mais c'est bien elle qui à mon insu m'attirait à elle. Aussi, titubant, plein de hâte et d'hésitation, je saisis l'apôtre Paul...» (lb.) Qu'a-t-il découvert dans ces livres ? S'il est difficile de faire un bilan exact du bénéfice qu'il en retira , on a pourtant quelques points de repères assez sûrs. Il   apprit  d'eux   plusieurs
choses  décisives  pour  sa  propre  pensée, et qui marqueront son christianisme
de façon indélébile :

a.	le  chemin de l'intériorité. Ces livres lui permirent de faire retour sur lui-même : «Averti par ces livres,· je rentrais en moi-même... ». L'itinéraire augustinien vers Dieu est désormais tracé : il va de l'extérieur (ab exterioribus) vers l'intérieur (ad interiora) , de l'intérieur vers le supérieur (ad superiora). Mais ce chemin d'intelligence (ou d'orgueil, comme dira Augustin) devra être corrigé, car ce n'est pas l'esprit qui connaît, mais la charité (VII, 10, 16).

b.	la découverte de la nature spirituelle de Dieu, identifié avec l'lpsum Esse de l'Exode (VII, 10, 16), et même le Dieu chrétien, un et trine\sn{23 Cf. Olivier DU ROY, L'intelligence de la foi en la Trinité selon saint Augustin. Genèse de sa théologie trinitaire jusqu'en 391. Etudes augustiniennes, 1966.}  Son ontologie n'a donc rien d'abstrait. A-t-il, dès cette époque, découvert le Verbe, comme le laisse entendre Conf. VII, 9, 13 ? Il est probable qu'il n'ai fait le rapprochement que plus tard.

c.	La question du mal. C'est aussi au contact du platonisme que s'est clarifiée la question du mal (VII, 13, 19), comme nous l'avons vu. Alors que les manichéens s'interrogeaient sur l'origine du mal, les Platoniciens font porter le débat sur sa nature, et ils ont compris d'emblée que le mal n'a pas de nature propre, mais qu'il est la dégradation ou l'absence d'un bien.

d.	L'inquiétude  du salut.  S'il rejette d'emblée «l'idolâtrie égyptienne» (VII, 9, 1 S) de Porphyre, provenant sans doute de la contamination du platonisme par le paganisme, Augustin semble partager son inquiétude du salut, inquiétude qui amena Porphyre, cet adversaire déclaré du christianisme, à faire néanmoins une place au Christ : «Ce philosophe dit aussi du bien du Christ, comme s'il avait oublié ses paroles injurieuses ... » (De civ. Dei XIX, 23, 2).

\subsection{Critique  sévère  du Platonisme}

Il est indéniable qu'Augustin a trouvé dans le platonisme une doctrine libératrice qui l'a aidé à se dégager définitivement du manichéisme, mais en même temps, revenant sur cette expérience philosophique, Augustin se montre d'une extrême sévérité. Dans les Confessions, à la différence du \textit{Contra Academicos}, il donne d'emblée une vue négative  du  platonisme, sans doute parce qu'il aurait pu le retenir définitivement
sur le chemin de la conversion au christianisme. Dès qu'il aborde cette période de sa vie, il  écrit :"Tu m'as procuré, par l'entremise  d'un homme  gonflé  d'un orgueil monstrueux, certains livres des Platoniciens traduits du grec en latin." (VII, 9, 13)
On peut au moins relever le grief fondamental: l'ignorance du médiateur.

a.	Les Platoniciens ont entrevu la patrie, mais sans  la voie. Les livres des Platoniciens représentent un piège dans la mesure où ils indiquent la "patrie" vers laquelle l'homme est tendu, sans faire connaître la "voie" qui y conduit, cette voie n'étant autre que le Verbe fait chair, voie d'humilité qui s'oppose à la présomption des philosophes (VII, 9, 13). Augustin a bien compris que la philosophie peut devenir dangereuse en faisant croire qu'elle est capable de conduire aux mêmes résultats que la foi, et ainsi en détourner (VII, 20, 26). Elle pêche par suffisance, présomption.

b.	Les Platoniciens ont ignoré le Verbe fait chair. C'est là qu'apparaît le fond de la question: les philosophes ont méconnu la nécessaire médiation du Verbe fait chair. On ne va à Dieu que par Dieu. 
\begin{quote}
    Je cherchais la voie...et je ne trouvais pas tant
que je n'avais pas embrassé le médiateur, l'Homme Jésus-Christ" (VII, 18, 24; X,
43, 68).
\end{quote} 
Augustin dira : 
\begin{quote}
    "Deus Christus patria quo imus; homo Christus via qua imus" (Sermo 124, 3, 3) Le Dieu Christ : la patrie vers laquelle nous allons; l'homme Christ : le chemin par lequel nous allons.
\end{quote}
 Dans un tableau contrasté entre ce qu'il a découvert chez les Platoniciens (ibi leg1), et ce qu'il n'y a pas trouvé (ibi non legl), Augustin met l'accent sur le Verbe fait chair. Les Platoniciens n'ont découvert que la moitié de la vérité puisqu'ils ont reconnu le Verbe éternel, sans confesser le Verbe fait chair, venu habiter parmi les hommes. «Quant à ceci: il est venu dans son propre domaine... , dans ces livres je ne l'ai pas lu.» (VII, 9, 13). Le Christ n'était qu'un homme d'une éminente sagesse» (VII, 19, 25). Augustin interpellera Porphyre en disant:
\begin{quote}
    « Oh ! si tu avais connu la grâce de Dieu par Jésus-Christ Notre-Seigneur ! Si tu avais pu voir dans l'incarnation où il a pris une âme et un corps d'homme, le plus beau chef-d'oeuvre de la grâce ! Mais que faire ? C'est en vain, je le sais, que je parle à un mort, du moins pour ce qui te regarde.» (De civ. Dei X, 29, 1 ).
\end{quote}


c.	Le platonisme n'a eu aucune  emprise sur les foules. Enfin, dernier	(
grief, le Platonisme, à la différence du christianisme,	n'a aucune autorité sur les
foules. C'est un élitisme qui n'a pas réussi à diffuser sa doctrine au-delà d'un cercle d'initiés. Il a toujours tenu la foule en mépris, l'estimant incapable de comprendre sa doctrine et de la pratiquer. Il a donc échouer dans son entreprise d'éducation de la foule: «Ensuite vint Platon, écrivain plus agréable que persuasif. Ces philosophes n'étaient point faits pour ramener la croyance de leurs compatriotes, du culte superstituieux des idoles et du mensonge de ce monde, au vrai culte du vrai Dieu. » (De vera religione Il, 2).

Si les anciens philosophes revenaient aujourd'hui, estime Augustin, ils opèreraient eux-mêmes le dépassement de leur philosophie vers le christianisme. En voyant qu' "on s'engage dans cette voie (chrétienne) en si grand nombre", ils diraient sans doute : "Voilà l'idéal que nous n'avons pas osé prêcher aux foules." (De vera religione  4, 6). "Si donc ces grands hommes pouvaient revivre notre vie... , ils deviendraient chrétiens, comme la plupart des platoniciens des
dernières générations et  de la nôtre." (ib. 4, 7). En tous les cas, il n'y a pasde
comparaison possible entre les philosophes et le christianisme. Augustin écrit dans un de ses sermons. Ainsi, à propos du verset : «leurs juges seront absorbés par le rocher», il expliquer que le rocher, c'est le Christ, tandis que les juges que le Christ anéantit sont les philosophes :

« Prends Aristote, place-le-près de ce Rocher, il s'enfonce dans le néant. Qui donc est Aristote ? S'il entend ces mots : "Voici les paroles du Christ", il tremble dans l'enfer. "Pythagore a dit ceci", "Platon a dit cela" : place-les-près de ce Rocher, compare leur autorité avec celle de l'Evangile, compare ces orgueilleux avec le Crucifié !a. »

\subsection{Les plus proches de nous, donc aussi les plus dangereux.}
 

On peut dire qu'Augustin lui-même resta partagé dans son appréciation du platonisme. Tout d'abord, il faut faire fi d'une accusation souvent entendue, à savoir qu'il serait responsable de l'hellénisation du christianisme. La "synthèse" entre le christianisme et Platon était déjà faite à Milan, comme nous l'avons souligné, et quand il se convertit, il entre dans un christianisme déjà platonisé, ainsi que l'a montré Courcelle. Par ailleurs, s'il raisonnne dans les catégories platoniciennes, il n'accueille pas la doctrine sans nuance. Il fait un tri. Il est plutôt porté à dénoncer le danger qu'elle représente, signe qu'il garde un impact sur les esprits.

-	En positif, Augustin reconnaît les mérites du platonisme. Quand il n'est pas préoccupé de polémique ou quand il veut mettre les Platoniciens en opposition entre eux, il concède que leur philosophie peut être une preparatio  evangelica. Au livre VIII de la Cité de Dieu, il reconnaît au platonisme le mérite d'avoir porté la philosophie à sa perfection, et de s'être rapproché au plus près du christianisme.
«Platon, philosophe si supérieur à tous les autres parmi les Gentils... » a découvert
que Dieu est
\begin{quote}
    « la cause de l'existence, la raison de l'intelligence, la règle de la vie : trois aspects dont le premier se rapporte à la partie naturelle de la philosophie, le second à la partie rationnelle, le troisième à la partie morale. ...Si donc, pour Platon, le sage est celui qui imite, qui connaît, qui aime ce Dieu et trouve son bonheur à participer à sa vie, quel besoin y a-t-il d'examiner les autres philosophies ? Aucun d'eux n'est plus proche de nous que les platoniciens.» (De civ. Dei VIII, 4-5)
\end{quote}


- En négatif, il reste que, entre le christianisme et la platonisme, il existe un gouffre : l'incarnation, l'humilité de Dieu. Là est la pierre angulaire de la foi chrétienne, la pierre d'achoppement de la pensée philosophique des Platoniciens. Augustin écrit contre Porphyre qui refuse l'incarnation :
\begin{quote}
    « Mais l'incarnation du Fils immuable de Dieu par laquelle nous sommes sauvés et qui nous permet d'atteindre ce que nous croyons ou ce que nous comprenons si peu que ce soit, vous vous refusez à l'admettre. Ainsi découvrez-vous de quelque façon mais de loin, et avec des yeux troubles, la patrie (patria) où nous devons demeurer;
et  pourtant  le chemin  (via) qu'il  faut  suivre,  vous  ne  le  tenez pas  "Pourquoi...refusez-vous d'être chrétiens, sinon parce que le Christ est venu humblement et que vous êtes orgueilleux 1 » (De civ. De, X, l 9)
\end{quote}


\section{Adversaire des donatistes - \textit{Le défenseur  de l'unité de l'Eglise}}

 
Le donatisme sévit en Afrique depuis presque un siècle. Il ne semble pas que ce schisme ait préoccupé Augustin avant son ordination sacerdotale. Au mo nt de son retour en Afrique, en 388, il se retira à Thagaste, sur les terres héntees de ses parents, et il y fonda une communauté de moines. Ordonné prêtre en 391, da s les circonstances que l'on sait il entra en lice presque aussitôt puisqu'on a de lui une lettre à Maximius, évêque donatiste de Sinitum, qui date de 392, lequel d'ailleurs se rallia à l'Eglise catholique (De civ. Dei XXII, 8, 7)25  11 écrivit aussi un psaume abécédaire contre les Donatistes qui date de 393. On le voit ensuite prendre à plusieurs reprises l'initiative d'écrire ou de rencontrer des évêques donatistes. 11 prend aussi part au concile de Carthage, en octobre 393, et peu à peu intensifie son combat pour l'unité de l'Eglise, surtout à partir de son ordination épiscopale, en 395, jusqu'à l'extinction du schisme.
\subsection{A  l'origine  de l'Eglise  donatiste}
 
La division de l'Eglise d'Afrique date de l'époque des persécutions, au début du quatrième siècle. En 303, Dioclétien avait exigé des chrétiens de Numidie qu'ils livrent les Ecritures (traditio) et sacrifient aux dieux protecteurs de l'Empire (thurificatio). Il y eut un large mouvement d'abandon. Même le primat de Carthage composa avec la persécution. Mais en même temps, il y eut des résistants et des martyrs. Quand le calme fut revenu, en 305, les résistants, qui avaient tenu dans la tempête, se transformèrent en accusateurs. Ils s'en prirent aux "traditores" qui avaient sacrifié aux divinités païennes et livré les Ecritures. Parmi les personnages mis en accusation, il y avait un certain Cécilien, diacre de Carthage, qui aurait, pendant la persécution, empêché les chrétiens de secourir leurs frères en prison. Or, ce Cécilien, accusé par les résistants d'avoir trahi la cause des martyrs, devait bientôt être élu au siège épiscopal de Carthage.

Ce fut cette élection de 31 2 qui provoqua le schisme.	Aussitôt élu, Cécilien fut cons cré par les évêques présents, sans attendre les évêques	de Numidie. Or, ces derrn rs, au.nombrede 70, arrivèrent en force,	et ils		trouvèrent à Carthage un appui de po1?s parmi les opposants de Cécilien, un homme intransigeant,- il y avait s rto t la riche Lucilia, qui s'était fait rabrouer par Cécilien en raison de ses d:v?tions aux reliques des martyrs -. Ils déclarèrent	nulle la consécration	de Cec   en, notam ent parce que, parmi les consécrateurs, s'était trouvé un évêque
trad tor. Il ajouter qu'en la Numidie intérieure, montagneuse, et la côte, d'autres motifs de nvahte subsistaient.


.  25Cf .  es de 1  conférence de Carthage en 411 , vol 1. Sources chrétiennes 194  Cerf 1972	1o
Voir aussiI introduction de Yves Congar, dans BA 28, pour l'histoire du schisme.	'	'	'p.  ·
13
 

A la place de Cécilien, on mit Majorin, qui n'avait pas hésité  à ache er les voix "à raison de 400 pièces d'or chacune" (cf. B.A. 28 p.. 13).  A an que la  tuat,on ne fut dénouée, Majorin mourut, et ce fut  Donat qui fu des1gne pour lui succeder. Celui-ci allait occuper le siège de carthage pendant pres de 40 ans et fortement
organiser l'Eglise donatiste.

 
Cécilien (312)
l
Récusé par
les évêques de Numidie
 
Majorin



évêque pendant
ans (313-347)
 

 
Aur!lius

Eglist catholique
 
Parménien

Eglise schismatique
 

L'administration impériale devait reconnaître la validité de l'ordination de Cécilien. Il y eut des répressions contre les donatistes. On devait organiser des conférences contradictoires. Finalement, un synode romain devait déclarer sa consécration valide. On accusa le pape d'avoir été un traditor...Finalement l'empereur rendit son jugement en 316 en faveur de Cécilien. Malgré la répression dont ils furent l'objet, les donatistes s'implantèrent solidement en Afrique du Nord. En 321 intervint un édit de tolérance. Donat , qui se considérait comme le chef du parti des Purs, aura le temps d'organiser la dissidence, avant d'être exilé en 347, sans pourtant réussir à exporter le schisme au-delà de l'Afrique.
\subsection{Le combat théologique d'Augustin}
 

Quand Augustin entre en scène, il mène d'abord un combat essentiellement théologique, déployant tout son génie en faveur de l'unité de l'Eglise. Il n'ignore pas que certains évêques furent effectivement des collaborateurs. Mais là n'est pas la véritable question. L'enjeu était théologique. Il s'agit de savoir où est l'Eglise, mais aussi de mesurer les conséquences qui résultent d'un abandon de la véritable Eglise.

a.	Où est la	véritable Eglise	? Dans ce combat qui oppose Donatistes et Catholiques, ce sont deux conceptions de l'Eglise qui s'affrontent. La question était celle
de l'unité de l'Eglise, car il est exclu, pour les uns et les autres, qu'il puisse y en avoir deux :
\begin{quote}
    "Notre discussion avec les Donatistes porte non pas sur le Chef, mais sur le Corps, non pas sur le Sauveur Jésus-Christ lui-même, mais sur son Eglise. A ce Chef, q e no s nous accordons à reconnaitre, de montrer où est son Corps, objet de notre dissentiment, afin que ses paroles mettent désormais fin au désaccord." (De
unitate ecclesiae IV, 7) -
" Voici, n'est-ce pas, la question discutée entre nous : Où est l'Eglise ? Chez eux ou chez nous? De toute façon, il n'y en a qu'une..." (De unitate ecclesiae Il, 2)
\end{quote}
 
 
 	Les Donatistes ont une conception qualitative : l'Eglise est une arche de Noé,	(( un rassemblement de purs.  "Ils se considéraient comme l'unique Eglise, le petit reste,
les Pau c 1, les purs, les seuls sauvés, les seuls vrais serviteurs du Christ."26 Celle-ci est une arche de Noé, un rassemblement de purs. Cette idée, ils la poussaient si loin que, lorsqu'ils s'emparaient d'une communauté, ils lavaient les murs de l'église, aspergeait les parquets d'eau salée, brisaient les autels, etc. Surtout, ils avaient soumis à un nouveau baptême tous ceux qui venaient vers eux. Possidius dit que le parti de Donat était "occupé à rebaptiser la moitié de l'Afrique"27

 	Augustin au contraire insiste sur l'universalité de la véritable Eglise. Le critère qu'il fait intervenir est celui de la catholicité : «N'est-il pas manifeste que, depuis que ce parti (donatistes) s'est retranché de l'unité, de nouvelles nations ont cru, que d'autres sont là, qui n'ont pas encore cru et auxquelles l'évangile ne cesse d'être annoncé chaque jour ?» (Le combat chrétien 29, 31). Or, les Donatistes, confiné en Afrique, n'ont aucun titre à la catholicité. Ils sont un parti, non une Eglise. "Parti" s'oppose à "catholique" et souligne l'idée de séparation. L'Eglise catholique a la
mission de rassembler toute la race humaine. Elle doit  être coextensive à la société.
L'Eglise est là où subsiste l'unité, et celle-ci est assurée par la charité.	(
b.	Situation  tragique au regard du salut.  Si quelqu'un abandonne la véritable Eglise, il faut savoir en tirer les conséquences théologiques. Sur ce point, Augustin est assez radical. Il partage l'opinion selon laquelle : Hors de l'Eglise, pas de salut !
\begin{quote}
    «Hors de la communion, de l'Eglise, Dieu n'a aucun des siens... » (De baptismo IV, 9, 13). 
    
    «Hors de l'Eglise catholique, on peut tout avoir, sauf le salut !» (Sermo ad caes. eccl. plebem 6).
\end{quote} 

Quiconque abandonne l'Eglise se prive de la source qui le vivifie et donc du salut :
\begin{quote}
    "Si quelqu'un est séparé du corps du Christ, il n'est pas membre du Christ et s'il n'est pas membre du Christ, /'Esprit du Christ ne le nourrit pas.n  (Tract. in Jo· Ev. 27, 6)
    
    -"Qu'ils s'agrègent au corps du Christ, s'ils veulent vivre de /'Esprit du Christ. Il n'y a que le corps du Christ qui vive de /'Esprit du Christ..." (ib. 26, 13)

\end{quote}

A la la différence des Donatistes, qui considéraient les sacrements donnés par les traditores comme invalides, Augustin au contraire devait défendre leur validité, car ce n'est pas le ministre qui baptise, c'est le Christ, mais aussi leur inutilité au regard du salut. Il reconnaît donc qu'il y a chez les Donatistes des "biens" d'Eglise, des vestiges d'Eglise, même si cette possession est injuste et inutile. En effet, bien qu'ils aient ces "biens", les Donatistes ne peuvent s'en servir que pour leur perte. Car il leur manque ce dont tout procède, la charité. La charité est le don "qui compense pour l'absence de certains autres", mais sans ce don, tout le reste "est possédé en vain" (De baptismo 11,). Autrement dit, quiconque vit hors de l'unité n'a pas réellement la charité et il ne reçoit donc pas réellement l'Esprit. Tous les dons que les Donatistes peuvent posséder
ne valent donc rien\sn{Emilien LAMIRANDE, La situation ecclésiologique des Donatistes d'après saint Augustin, Université d'Ottawa, 1972, p. 71. André MANDOUZE, Saint Augustin, l'aventure de la raison et de la grâce, Etudes augustiniennes, 1968, p. 346. f. Lamirande, oc pp. 38 sv.- Voir aussi : Itinéraires augustiniens, n° 8 : L'Eglise.}



 \subsection{Le  combat  politique  d'Augustin}
 
 
Interdits depuis 392,	l'année après l'interdiction du	paganisme (391), les Donatistes ne continuèrent pas moins leurs activités. Augustin reconnaît lui-même que les "lois contre eux ne manquaient pas, mais c'est comme si elles avaient manqué."\sn{ Cf. Mandouze, oc p. 387.
 Cf. François DECRET, Le christianisme en Afrique du Nord ancienne. Seuil, 1996, p. 145. Le terme viendrait de circum cellas, ceux qui rôdent autour des granges (cella, celliers) ou des chapelles (cellae, chapelles des martyrs).}  Le combat théologique devait donc s'avérer insuffisant, d'autant plus que les donatistes s'étaient alliés avec les circoncellions , bandits des grands chemins qui se battaient contre les propriétaires romains, et qui n'hésitaient pas à recourir à la force. Le schisme se doublait en effet de revendications sociales. La paix était de plus en plus précaire, et Augustin lui-même	faillit tomber, un jour, entre leurs mains. 11 n'échappa à leur piège que parce qu'il s'était trompé de chemin. C'est au concile de Carthage, en 411, que devait être donné l'assaut final contre le Donatisme.

\paragraph{Le concile de 411.} Ce n'est pas le premier concile à s'intéresser aux donatistes. En 401, le concile de Carthage avait décidé de discuter «avec douceur et dans un esprit de paix». Un autre concile projeté à Carthage, en 403, fut récusé par les donatistes qui considéraient comme indignes une rencontre entre les fils des martyrs, et les descendants des traditores. En 405, un édit de l'empereur rétablit l'unité en faveur des catholiques, et les biens des donatistes devaient être confisqués. Ces derniers firent un certain nombre de démarches auprès du pouvoir impérial . li y eut un nouvel édit de tolérance en 41O. Après la chute de Rome, la convocation d'un concile à Carthage devait règler définitivement la question.

Accepté par les deux partis, le concile de Carthage qui se tint en 411 réunit une assemblée considérable, près de 600 évêques pour chacun des partis. Les Donatistes ne mirent pas beaucoup de bonne volonté. Ils engagèrent une procédure de vérification des mandats qui retarda l'ouverture jusqu'au 8 juin. Le commissaire de l'empereur, Marcellin - impliqué dans un complot, il sera exécuté... et proclamé saint -, fit preuve d'une grande patience, mais devait au terme des débats reconnaître l'Eglise catholique comme la véritable Eglise. Désormais, le donatisme était hors la loi. On devait transférer leurs biens et leurs églises aux catholiques. Leurs clercs sont exilés et les fidèles soumis à l'amende.

\paragraph{Recours légitime	à  la "coercitio".} 	Fallait-il recourir à la coercitio
pour obtenir le retour des Donatistes ? On remarquera d'abord que Augustin n'a jamais	(
eu recours à la coercition à l'égard des Juifs, qu'il juge pourtant sévèrement, mais
dont il respecte la liberté de culte, ni à l'égard des païens, tout en approuvant les lois interdisant le culte des idoles, ni à l'égard des manichéens, avec lesquels il prévère engager un débat d'idées. Dans le cas des donatistes, son attitude a changé. On sait que, sur ce point du recours à la coercition, sa position a varié. D'abord hostile à l'usage de la coercition, il devait s'y rallier dès 400, comme l'ensemble de l'épioscopat, à la fois comme mesure défensive contre les exactions des Donatistes, et comme moyen de les faire revenir à l'unité catholique. Lui-même s'explique dès avant le concile sur ce changement d'attitude :


\begin{quote}
    « Primitivement, en effet, mon avis se ramenait à ceci : personne ne devait être contraint à l'unité du Christ; c'est par la parole qu'on devait agir, par la discussion qu'on devait combattre, par la raison qu'on devait vaincre : je craingnais qu'autrement nous n'eussions comme faux catholiques ceux que nous avions connus comme francs hérétiques. Mais cette opinion, qui était mienne, devait cèder, non devant des mots, mais devant des exemples. Pour commencer, on m'opposait ma propre cité qui, jadis tout entière acquise au parti de Donat, se convertit à l'unité
catholique par crainte des lois impériales...Et il en était de même pour beaucoup d'autres cités dont les noms m'étaient énumérés. Ainsi la force même des choses m'obligea qu'en ce domaine aussi pouvait bien se comprendre la vérité de cette phrase de /'Ecriture : "Donne au sage l'occasion et il sera plus sage encore."

Combien en effet en connaissons-nous dont on peut affirmer qu'en eux se manifestait déjà le désir d'être catholiques, bouleversés qu'ils étaient par l'évidence aveuglante de la vérité, mais que la crainte d'une violente réaction de la part des leurs poussait chaque jour à différer ! Combien d'entre vous étaient retenus, non par la
vérité, qui n'a jamais été	votre fort, mais par la lourde chaine d'une habitude invétérée!..." (Epist. ad Vincentium 93, 5, in Mandouze, oc p. 371-372)
\end{quote}

 
On peut donc expliquer le changement d'attitude d'Augustin par la situation particulière créée par le donatisme. Les raisons qu'il invoque dans la lettre citée (qui date sans doute de 408) sont à la fois d'ordre pratique et théologique:

-	d'ordre  pratique: il fallait créer les conditions de la liberté en libérant de la terreur. Il ne faut pas oublier le climat d'insécurité que faisaient règner les donatistes, alliés aux circoncellions.

-	d'ordre théologique : Dès 405, il alla plus loin en cherchant à justifier le recours à la contrainte. C'est là sans doute que nous avons le plus de mal à suivre ses explications. Je ne les retiens pas toutes (Cf. Mandouze, oc pp. 380-387). Quand Augustin nous dit : "Il ne faut pas considérer la contrainte en soi, mais considérer ce à quoi vise la contrainte, si c'est au bien ou au niai" (Ep 1OS, 2), il y a naturellement lieu de s'inquiéter, car qui est habilité à désigner le bien et le mal. Ou quand il distingue entre une "persécution injuste"  - celle qui est faite à l'Eglise du Christ - et
une "persécution juste" - celle que "les églises du Christ font aux impies" -, on peut	( encore s'inquiéter, car il n'y a pas de juge impartial en la question. (Ep. 1 85, 2).

c.	Justification malaisée d'Augustin.  Même s'il la justifie, Augustin n'accepte la coercition qu'avec réserve. On sent qu'il n'est pas à l'aise avec la violence qui est faite aux Donatistes et qu'il essaie de justifier I' injustifiable. Pour la comprendre , il faut regarder à la fois les explications théologiques et son attitude pratique :

-	D'abord, le terme de "persecutio" qu'il utilise ne désigne pas la "persécution". Dans le latin d'Augustin, le terme a encore son sens initial de "poursuite" , de quête du frère égaré. Il s'agit plus, chez Augustin de la poursuite inspirée par l'amour de celui qui s'égare. "Pourquoi donc l'Eglise ne forcerait-elle pas ses fils perdus à (lui) revenir, si les fils perdus en ont forcé d'autres à se perdre?" (Ep. 185).  A l'arrière-plan, il y a la parabole des serviteurs invités à \textit{"faire entrer
de force" dans la maison ceux qu'ils rencontreront "le long des routes et des haies"}\sn{Cf. Mandouze oc p. 328.}

-	Ensuite, il faut éviter l'anachronisme.  Le «forcer à entrer» a une triste r putation. Il a servi au Moyen-Age à justifier l'inquisition, et celle-ci s'est réclamée A gustm.	Or, même s'il en donne parfois une version très intransigeante, Augustin hsa1t non pas "compelle intrare", mais "cogite intrare", le "cogere" mettant l'accent
sur la convergence, le rassemblement, la conviviabilité, plus que sur la contrainte.

- Enfin, il convient de regarder l'attitude  d'Augustin. Sans vouloir absolument le justifier - il faudrait aussi se souvenir que l'époque fut autre -, il ne faut jamais oublier qu'Augustin, en pratique, interviendra souvent pour que les lois soient atténuées et pour que la peine de mort soit écartée absolument. Il était avant tout animé d'une intention évangélique indéniable, intention qui se traduisait par la passion du frère. Ecoutons-le plutôt :
\begin{quote}
    « J'entends en effet /'Apôtre dire : "Insiste à temps et à contre-temps"...Oui, je suis l'homme du contre-temps, j'ose le dire.Tu veux t'égarer, tu veux te perdre : moi, je ne le veux pas. C'est que ne le veux pas, en fin de compte, Celui qui me fait trembler...Pour autant que le Seigneur qui me fait trembler me donnera la force, je parcourraitout : je rappellerai qui s'égare, j'irai quérir qui se perd. Si tu ne veux pas avoir à me subir, veuille ne pas t'égarer ni rechercher ta perte...» (Sermo 46, 7)\sn{ Cf. Coercitio in Augustinus Lexikon.}
\end{quote}


\section{Adversaire du paganisme - \textit{Citoyen des cieux}}
 


Parmi les adversaires qu'Augustin eut à combattre, les païens occupent une place importante, surtout à partir de 410. Il faut ici rappeler les circonstances avant de regarder comment Augustin organisa la riposte.

1 . La	chute de	Rome

Le 24 août 410, Rome, la ville éternelle, tombe devant l'invasion des hordes barbares d'Alaric. C'est le scandale, même parmi les chrétiens. Depuis un certain temps déjà, on pouvait s'attendre à une telle issue. Mais on ne pouvait pas y croire. Quand la ville tombe, c'est un cri de douleur à travers tout l'empire. Saint Jérôme, un romain de souche, qui vit à Bethléem, est muet de stupeur:

" Quand la lumière la plus éclatante de toute la terre fut éteint, quand l'empire romain fut coupé de la Capitale, quand, pour parler plus exactement, la terre entière périt avec cette seule ville, je suis resté muet et je me suis humilié..." (B.A. 33 pp. 10-11).

Chez les païens, mais aussi chez beaucoup de chrétiens, ce trouble se transforme en accusation. Voici comment Augustin rapporte dans un sermon les accusations que l'on pouvait entendre parmi les païens :



3' cf. Mandouze, oc p. 387.

18
(
 
« C'est au temps du christianisme que Rome est dévastée,que le fer et le feu ont dévasté Rome...Tant que nous avons pu offrir des sacrifies à nos dieux, Rome se tenait de ut; Rome était florissante. Aujourd'hui que ce sont vos sacrifices à vous qui ont pns le dessus et que, partout, ils sont offerts à votre Dieu, alors qu'il ne nous est plus permis de sacrifier à nos dieux, voilà ce qui arrive à Rome...» (Sermo 296, 7)

Quant aux chrétiens, dont certains interprétaient l'événement comme l'annonce de la fin des temps, le trouble n'était pas moindre. Dans le même sermon, Augustin fait écho à leurs récriminations en ces termes :

"Le corps de Pierre repose à Rome, disait-on; le corps de Paul repose à Rome...Et Rome est malheureuse, et Rome est investie; partout l'affliction, le massacre, l'incendie...A quoi servent les tombeaux des Apôtres ?" (Sermon 296, 6)

Pour s'opposer à ces débats tumultueux, Augustin, qui est à Carthage au moment où les premiers réfugiés arrivent en Afrique du Nord, commence par prêcher, et il en parlait si souvent que les fidèles se disaient en allant à l'église : "Pourvu qu'il ne nous parle pas une fois de plus de la chute de Rome" (S. l OS, 12).Augustin entreprend surtout de répliquer par une riposte en règle dans un "grand ouvrage difficile", la Cité de Dieu, dont la rédaction sera quelque peu retardée en raison de l'affaire donatiste qui continue à l'occuper, et sa rédaction ne sea pas menée d'une seule traite. Elle ne sera achevée qu'en 425. Dans les Révisions, Augustin indique quel fut son but en rédigeant cette somme. Un but nettement apologétique :

« Les païens s'efforcèrent de faire retomber ce désastre sur la religion chrétienne ...C'est pourquoi, brûlé par le zèle de la maison de Dieu, je décidai d'écrire contre leurs blasphèmes et leurs erreurs la cité de Dieu...» (Retr. Il, 43, 1)

Le livre devait d'ailleurs très vite dépasser le point de vue particulier qui lui donna naissance, d'autant plus que la vieille cité se remit de son choc et que la vie y continua comme avant. Il ne saurait être question d'aborder tous les thèmes qui s'enchevêtrent. Il faut d'abord remarquer qu'Augustin ne se situe pas au plan politique. La chute de Rome est surtout l'occasion pour lui de célébrer l'unique cité qui ne périra pas, la cité de Dieu.

2.	La réaction d' Augustin33  

La réplique d'Augustin n'est pas d'abord d'ordre politique, bien qu'il y ait des formules assez péremptoires sur les empires terrestres. Son regard n'est pas particulièrement optimiste. Quand il regarde les empires, y compris l'empire romain, il y dénonce d'emblée l'injustice dont ils se sont rendus coupables . Or, sans la justrice, les royaumes sont des sociétés de brigands. "Otez· la justice, et que sont les gouvernements sinon du brigandage à grande échelle ?" (Cité de Dieu 4, 4). On peut relever dans la Cité de Dieu trois thèmes, de connotation spirituelle,  à travers lesquels s'esquisse la réplique d'Augustin :






33 Sermons 81 et 296, in Humeau.
19
 
	L





a.	Aucune cité humaine n'est éternelle. La civilisation romaine est tout	  aussi mortelle que les autres. C'est donc un pur mensonge quand on leur promet une
destinée éternelle. Dans un sermon, Augustin s'insurge contre ce mensonge dont s'est rendu coupable Virgile :

« Ceux qui ont promis (l'éternité) aux royaumes de la terre n'ont pas été conduits par  la vérité, mais la flatterie leur a inspiré ce mensonge...Le ciel et la terre passeront: combien plus vite passera la fondation de Romulus...» (Sermon 1OS, 7)

Il revient sur ce thème dans un autre sermon où s'esquisse déjà le thème des deux cités, mais pour souligner que seule la cité de Dieu est promise à durer, tandis que le monde doit périr:

« Le monde est comme l'homme : il naît, il grandit, il vieillit...Le Christ arrive à l'heure où tout vieillit pour te renouveler toi-même. Le monde créé, le monde fondé, le monde destiné à périr, incline vers le couchant...Ne t'attache pas à ce vieillard qu'est le monde. Ne refuse pas de te rajeunir dans le Christ qui te dit: ...Le monde est travaillé par l'asthme de la vieillesse. Ne crains rien : ta jeunesse se renouvellera comme celle de l'aigle!» (Sermon 81, 8 in BA 33, p. 13).

Il n'y a donc pas lieu de s'affliger quand sombre une cité : c'est le lot de toute civilisation. Et l'occasion de se souvenir que notre cité est ailleurs. Certes, ce qui vient d'arriver à Rome est une épreuve. Mais l'épreuve fait partie de la vie. Certes, elle atteint le juste comme l'injuste. Mais c'est l'occasion de se souvenir que la vie ici­ bas est une mise à l'épreuve où le juste ne fait que suivre les traces de l'unique Juste, le Christ :

« De la main de Dieu qui cherche à amender, la cité a donc bien plutôt reçu une correction qu'elle n'a trouvé sa perte...Ne soyons donc pas troublés en voyant soufflr les justes : c'est là une épreuve, non une damnation...Car la souffrance de cette cité tout entière a été assumée par lui (le Juste des justes) tout seul..» (VIII, 9).

b.	Une nouvelle phase de l'histoire du salut : Elargissant le regard , Augustin médite alors sur l'avenir. Au lieu de s'afflliger sur un monde qui disparaît, Augustin discerne dans ce qui arrive une nouvelle phase de l'histoire du salut. Apparemment, Rome est envahie, mais en réalité, ce sont de nouveaux chrétiens qui
accourent vers elle. Dans son commentaire du psaume 64, 5, Augustin lit le verset : l	"Ad te omnis caro veniet", et il l'applique aux barbares qui affluent vers la foi. Ce qui arrive est un signe avant-coureur de l'universalité de la foi. La cité de Dieu est en
marche parmi les cités humaines. Il commente :

« Ce n'est pas en effet les Romains, mais toutes les nations qui ont fait l'objet d'un véritable serment par lequel le Seigneur les a promises à la race d'Abraham. Et, de cette promesse, il est d'ores et déjà résulté que certaines nations qui ne sont pas soumies à la puissance romaine reçoivent l'Evangile et s'agrègent à l'Evangile qui fructifie et croit dans le monde entier ..." (Ep. 199, 12, 47)34





20
 
l
quand on a cette vision d'avenir, on regarde autrement les ennemis qui viennent enva 1r Rome. D'un point de vue humain, ce sont des ennemis dont il faut redouter qu'il détruisent la cité terrestre, mais d'un point de vue chrétien, ce sont de futurs disciples du Christ. C'est donc du point de vue de la cité de Dieu qu'il faut juger toutes choses :

« La cité de Dieu doit se souvenir que parmi ses ennemis mêmes se cachent plusieurs de ses futurs citoyens...De fait, les deux cités sont mêlées l'une dans l'autre en ce siècle, jusqu'au jour où le jugement dernier les séparera. Je vais donc, dans la mesure où la grâce divine m'y aidera, exposer ce que j'estime devoir dire sur leur origine, leur développement, la fin qui les attend...» (1, 35)

c.	Au fondement des deux cités : Du point de vue de la cité de Dieu, le Romain n'est pas meilleur que le Goth, et inversement. Ce n'est pas là un discours très "patriote". Mais pour Augustin, tous les empires sont transitoires, du fait même qu'ils vivent dans le temps. On se plaint de la rigueur des temps. Mais Augustin fait remarquer que chaque époque maudit le temps dans lequel elle vit. Mais, précise Augustin, le "fait que l'époque soit bonne ou mauvaise dépend de la qualtié morale de la vie intellectuelle et sociale, et cela est en notre pouvoir" (S. 80, 8). C'est de ce point de vue qu'il juge les cités. Or, à cet égard, les deux cités sont antinomiques au possible. Il faut lire ici le passage le plus célèbre de la cité de Dieu sur les deux amours qui ont fait deux cités (XIV, 28).

3.	La Cité de Dieu en pèlerinage ici-bas35

Avec la critique des païens, il y avait un défi à relever : «Je décidai d'écrire contre leurs blasphèmes ou leurs erreurs les livres de La Cité de Dieu.» Commencé en 412, ce «grand ouvrage» ne fut achevé que vers 425. Le plan de l'ouvrage donne une première indication sur son projet. Au total, il comporte vingt-deux livres, dont certains circulaient dans le public avant que l'ensemble ne fut achevé.

a)	Plan de la Cité de Dieu. Sa composition se réalisa selon un plan dont Augustin rappelle à plusieurs occasions les articulations36 
- Dans les  dix  premiers  livres, il s'attache à réfuter les païens qui prétendaient lier le bonheur de l'homme au culte polythéiste. Ces dix livres se divisent eux-mêmes en deux ensembles de cinq livres chacun. Tandis que les livres I à V s'en prennent à ceux qui attendent du culte rendu aux divinités païennes la réussite dans les affaires d'ici-bas, et attribuent à son interdiction les maux actuels de la cité, les livres VI à X sont dirigés contre ceux qui, «tout en avouant que de tels maux n'ont jamais manqué et ne manqueront jamais aux mortels..., déclarent pourtant utile le culte des faux dieux, avec les sacrifices qu'on leur offre, à cause de la vie qui doit suivre le mort». On reconnaît sans peine dans ces derniers les platoniciens : ils placent en effet le bonheur non pas dans le monde sensible, mais au contraire dans la participation à la vie des dieux. «Philosopher, disait Platon, c'est aimer Dieu dont la nature est incorporelle. 'P » Mais Augustin ne critique pas moins les platoniciens, car

35 Cité de Dieu, 1, 1. BA 33, p. 191. En XIX, 17, il dit qu'elle est en exil. BA 37 p. 129. Pour le plan de l'oeuvre, voir Jean-Claude GUY, Unité et structure de la "Cité de Dieu" de saint Augustin. Etudes Augustiniennes, 1961.
35 Cf. Les Révisions, BA 12, p. 525. Voir aussi l'introduction à La Cité de Dieu, BA 33, p. 35 s.
37 La Cité de Dieu VIII, 8. BA 34, p. 261.
21
 
s'ils ont vu «où il faut aller», la «patrie»,	ils n'ont pas vu «par où» y aller, faute	  de croire en «celui qui est la voie conduisant non seulement à la vue, mais encore à l'habitation de la patrie bienheureuse»38  

- La seconde partie de l'ouvrage (XI à XXII) s'attache à présenter la foi chrétienne sous un jour positif. Augustin comprit très vite qu'il ne suffisait pas de réfuter les païens, mais que la crédibilité de son projet dépendait de sa capacité à justifier sa propre religion chrétienne. Cette seconde partie comporte douze livres subdivisés en trois parties. «Des douze derniers livres donc, les quatre premiers traitent de l'origine des deux cités, dont l'une est la cité de Dieu, l'autre la cité de ce monde. Les quatre suivants décrivent leurs progrès, leurs développements. Les autres, qui sont aussi les derniers, montrent les fins qui leur sont dues. Ainsi, tous ces vingt­ deux livres ont pour thème l'une et l'autre cités; mais ils ont emprunté leur titre à la meilleure et ils sont appelés de préférence De la cité de Dieu.5 » Ayant ainsi précisé le plan, Augustin donne des indications pour la reliure de l'ouvrage, souhaitant que l'on respecte soit la division en deux volumes, soit une division en cinq volumes : «Si tu
préfères qu'il y ait plus de deux volumes, il faut en faire cinq», les deux premiers	  (cinq livres chacun) étant consacrés à la réfutation du paganisme, les trois suivants
(quatre livres chacun) à la justification de la foi chrétienne«>.

b)	Evaluation du projet d'Augustin. Ces indications formelles sur le plan de l'ouvrage seraient négligeables si elles se limitaient à des «consignes de reliure» 1. En réalité, elles dévoilent l'intention d'Augustin : La Cité de Dieu est avant tout une défense et illustration  de  la «vraie religion», c'est-à-dire de la foi chrétienne. Comme l'a montré Goulven Madec, Augustin poursuit ici un projet comparable à celui, déjà ancien, du De vera religione, qui date de 390. Dans l'un comme dans l'autre, il développe la thématique du bonheur, ainsi que celle de l'accès à la transcendance : ce sont des thèmes communs à la philosophie et au christianisme. Ces deux ouvrages tracent l'un et l'autre une nette ligne de partage entre platonisme et christianisme, la principale déficience du premier étant de n'avoir pas su reconnaître le Christ fait chair, unique médiateur entre Dieu et les hommes. Ce point de rupture, les Confessions l'avaient déjà souligné avec toute la netteté désirable : «Là, j'ai lu...là je n'ai pas lu» ! Là, chez les platoniciens, Augustin a «lu» que le Verbe est éternel, mais non pas «lu» qu'il s'est fait chair. Si les platoniciens «connaissent Dieu», «ce n'est pas comme Dieu qu'ils le glorifient ou lui rendent grâces»4Z, puisqu'ils ne le
reconnaissent pas dans la chair, et ne le confessent pas comme Médiateur unique entre	1
Dieu et les hommes.

Si tel est le projet d'Augustin, il est clair qu'en opposant les deux cités, i 1 n'oppose pas deux  réalités  identifiables  dans l'ordre  empirique, mais il s'en prend à un type très précis de la cité où les deux réalités, politique et religieuse, n'en faisaient qu'une. Ce qu'il conteste, ce n'est pas la cité humaine en tant que telle - il en justifie la nécessité et en rappelle les devoirs  - mais la collusion qu'elle

38 Confessions VII, 20, 26.
39 Les Révisions Il, 43, 2 BA 12, p. 527.
40 Lettre 1 A' à Firmus. BA 46 B, p. 57. C'est cette division en cinq volumes qu'a adoptée la Bibliothèque augustinienne. BA 33 à 37.
41 Cf. Goulven MADEC, Le De civitate Dei comme De vera religione, dans : «Peti1es études augustiniennes». Institut d'Etudes Augustiniennes, 1994, p. 194.
42 Confessions VII, 9, 13-15.
22
(
 
entretenait entre le politique et le religieux, deux aspects indissociables dans la cité romaine, l'unité politico-religieuse, écrit Ratzinger, étant constitutive de son essence. li faut garder à l'esprit cette conception religieuse de la cité si l'on veut comprendre les griefs émanant des païens à l'encontre du Dieu chrétien lors de la chute de Rome en 41 O. Les païens étaient persuadés qu'en ayant abandonné ses dieux, Rome avait perdu ses protecteurs et ne pouvait que succomber sous les hordes barbares. Les coupables sont donc les chrétiens, puisqu'ils furent les grands démolisseurs du polythéisme. C'est cette accusation qui a provoqué la réaction d'Augustin. Son projet n'est pas de condamner l'organisation politique de la cité - elle est nécessaire - , mais de dénoncer la collusion de la cité avec les dieux païens. fi veut montrer au contraire qu'il n'existe aucun lien entre la suppression de ce culte et la chute de Rome.

Les contours de ces deux cités ne sont pas visibles à l'oeil nu, car elles sont à entendre en un sens allégorique. Ce sont deux réalités spirituelles, deux catégories d'hommes, «ceux qui vivent selon l'homme, ceux qui vivent selon Dieu»43  Elles sont symbolisées par Jérusalem et Babylone, une opposition qui est familière à Augustin, puisqu'on la trouve déjà vers 400, dans sa «Première catéchèse». «Deux Cités, écrit-il, celle des impies et celle des saints, s'avancent depuis l'origine du genre humain jusqu'à la fin du monde; à présent mêlées quant à leurs corps, mais séparées par leurs volontés, au jour du jugement elles seront aussi séparées de corps...». Puis il ajoute : «Or, de même que Jérusalem symbolise la cité et la communauté des saints, de même Babylone symbolise la cité et la communauté des injustes, car le mot signifie, dit-on, confusion....Ces deux cités, depuis l'origine du genre humain jusqu'à la fin du monde, parcourent mêlées l'une à l'autre les diverses époques et qui doivent être séparées au jugement dernier... »44  Mais allégorie ne signifie pas utopie. Cette réalité spirituelle est inscrite dans l'histoire, en négatif comme en positif, bien qu'aucune cité terrestre, pas même l'Eglise, ne réalise pleinement l'idéal de la cité de Dieu.

Si, au regard de leur essence, les deux cités sont diamétralement antagonistes, l'une ayant  comme principe «l'amour de soi jusqu'au mépris de Dieu», l'autre
«l'amour de Dieu jusqu'au mépris de soi»45, au regard de leur inscription dans le temps, elles forment un  mélange  inextricable <46. «De fait, les deux cités sont mêlées et enchevêtrées l'une dans l'autre en ce siècle, jusqu'au jour où le jugement dernier les séparera. Je vais donc, dans la mesure où la grâce divine m'y aidera, exposer ce que j'estime devoir dire de leur origine, leur développement, la fin qui les attend. Je servirai par là la gloire de la Cité de Dieu qui, comparée ainsi à l'autre, se détachera par opposition avec un plus vif éclat.47» Cet enchevêtrement interdit toute séparation prématurée. A Parménien qui excitait à la révolte contre les pervers, anticipant sur le jugement de Dieu, Augustin répond : «La même assemblée les unit en vérité jusqu'au vannage final... »48  Les principes antagonistes, constitutifs des deux cités, s'affrontent aussi bien dans le coeur de l'homme que sur la scène politique du monde. La traduction empirique de la cité de Dieu sera toujours imparfaite et souvent trompeuse. Loin de

43 La Cité de Dieu XV, 1, 1 : où il affirme qu'il parle «en langage allégorique»
44 La Première catéchèse, 19, 31, et 21, 37. BA11/1, p. 157 et 175.
45 La Cité de Dieu XIV, 28 BA 35, p. 465.
 e Cf. Joseph RATZINGER, Herkunft und Sinn des Civffas-Lehre Augustinus, dans Augustinus Magister. Congrès international augustinien, Paris, 21-24 septembre 1954. Et. aug. , p. 965 s.
47 La Cité de Dieu 1, 35. BA 33 p. 301.
48 Contre la Lettre de Parménien, Ill, 3, 19. BA 28, p. 443.
23
 
dénigrer la cité t
et il escom t	er e
 

str ,. Augustin sait reconnaître la vertu des hommes politiques,
 
les retombpe aussi dès 1c1- as au moins une ébauche de la cité de Dieu dont il souligne élimi	ées au plan social. Mais il ne croit pas à la possibilité d'instaurer l'une en
nant totalement l'autre. Le mélange durera jusqu'à la fin des temps.

Tout en accordant une valeur absolue à la Cité de Dieu Augustin ne reste nullement à l'écart des réalités qui font la vie de la ité humaine. Le seul arcou de la table analytique<111 , à la fin de l'ouvrage,  fournit une liste
. tmpress1onnante, non seulement des questions religieuses comme le culte ou le sacnfice, ou éthiques comme la peine de mort, le suicide ou la légitime défense, mais encore des principaux aspects de la vie politique. Augustin envisage même une redistribution des impôts plus équitable qui serait à l'avantage des pauvressi. La clef de voûte de la cité humaine est en tout état de cause la justice : «Supprimée la justice, que sont les royaumes sinon de vastes brigandages ? Car les brigandages eux-mêmes, que sont-ils, sinon de petits royaumes?51 » Son objectif doit être la paix, un bien auquel tous aspirent, même ceux qui font la guerre, même les brigands, bien qu' entre la paix des cités terrestres, si parfaite soit-elle, et la paix de la cité céleste, il n'y ait pas de commune mesure : «La paix de la cité, c'est le concorde bien ordonnée des citoyens dans le commandement et l'obéissance; la paix de la cité céleste, c'est la communauté parfaitement ordonnée et parfaitement harmonieuse dans la jouissance de Dieu et dans la jouissance mutuelle en Dieu52    » Quantité d'autres thèmes affleurent, qu'une «philosophie politique» aura toujours intérêt à méditer : le gouvernement de la cité, le sens de l'autorité et ses fonctions, l'usage de guerre, etc.53

S'il énonce des principes politiques,  Augustin n'ébauche à aucun moment un programme politique. C'est sa limite, qui tient en réalité à sa conception de l'histoire, une conception qu'il partage en fait avec les grands historiens de la Rome antique54 et qui se caractérise par deux traits. D'une part, elle est essentiellement morale. Quand il s'agit de comprendre la grandeur ou la décadence de Rome, l'historiographie romaine fait appel à des motifs non pas politiques ou sociaux, mais moraux : la grandeur est liée à la vertu, la décadence au vice. D'autre part, et en voie de conséquence, lorsque, dans une situation de crise, il s'agit de trouver une solution, les hommes politiques s'appuient moins sur l'analyse de la situation que sur le passé,
en s'inspirant des exemples laissés par les hommes illustres d'autrefois. Cette
conception de l'histoire est aussi celle d'Augustin, bien qu'il n'accepte pas sans nuance	(
le lien causal entre décadence et vice, entre prospérité et vertu, du moins dans le temps, tandis que le recours aux exemples des anciens lui est familier. C'est à partir de ces figures exemplaires qu'il argumente assez fréquemment, figures dont les noms sont dans la mémoire de tous les Romains 55   Il les donne même en exemple aux

◄0 La Cité de Dieu BA 37, p. 898 s.
50 La Cité de Dieu 1, 17, 1. Voir aussi note BA 33, p. 717.
51 La Cité de Dieu IV, 4. BA 33, p. 541.
52 La Cité de Dieu XIX, 23. BA 37 p. 113.
53 Il faudrait compléter le tableau en consultant d'autres écrits d'Augustin, par exemple les lettres 91 et 138, adressées à Nectorius et à Marcellinus, sur la patriotisme et la citoyenneté.
5  Cf. Viktor POSCHL, Augustinus und die rômische Geschichtsauffassung, dans Augustinus Magister, oc p. 957 s.
55 La Cité de Dieu V, 21 : «Certes Dieu réserve exclusivement aux bons le bonheur dans le royaume du ciel; mais il accorde le royaume de la terre aux pieux et aux impies comme il lui plait... Si ses raisons sont cachées, sont-elles injustes?» BA 33, p. 739 et 743.
24	(
 
chrétiens, afin de susciter parmi eux une saine émulation : «Qu'ils fi ent sur ces exemples un regard attentif et sage et voient quel amo r est dû à la patne.céleste en vue de la vie éternelle, quand la cité terrestre est tant a1mee de ses citoyens en we de
la gloire humaine.56 »
Mais on ferait fausse route si l'on cherchait dans la Cité de Dieu un livre de philosophie politique. De ce point de we, on pou ait même y  v ir,  une, contre­ hi st o ire», en ce sens que, à la différence des h1stonens, Augustm n apprec1e pas les événements selon les critères habituels : la grandeur, les victoires, la splendeur, etc. S'il s'appuie sur les mêmes événements que les adversaires, il les interpréte à l'encontre de leurs intentions. C'est ainsi que, à la différence de Cicéron qui écrit sa
«République» pour prouver que l'histoire romaine est en progrès constant vers la justice, Augustin y dénonce une vaste «entreprise de pillage»57  Il apprécie l'histoire de Rome non à sa gloire, mais à ses progrès spirituels. La justice et l'amour peuvent être en progrès au sein même d'une cité qui s'effondre. Alors que la Rome politique est menacée de décadence, la cité de Dieu y progresse de jour en jour car le progrès se mesure à l'avancée de la charité. «Voilà l'homme nouveau, l'homme intérieur, l'homme céleste. li a, lui aussi, analogiquement, ses âges spirituels, que l'on distingue non à ses années, mais à ses progrès.58» Cette «contre-histoire» qu'ébauche Augustin cherche non pas à «sauver les phénomènes», mais veut inciter l'âme à se tracer son itinéraire vers Dieu sans se laisser troubler par les événements.

Tout comme les Confessions traçaient cet itinéraire pour l'âme, la Cité de Dieu le fait  à l'échelle d'un peuple. Augustin ne demande pas à ce peuple de mettre un trait sur les vertus naturelles qui ont fait sa grandeur, mais d'accepter que ces vertus soient purifiées au feu de la foi. Voici comment il interpelle Rome : «Si brille en toi un don naturel estimable, seule la vraie piété peut le purifier et le perfectionner; l'impiété le fait périr et consomme sa ruine. Choisis maintenant ta route pour obtenir d'être loué sans erreur, non en toi-même, mais dans le vrai Dieu. Tu fus autrefois en grand renom parmi les peuples, mais par un secret jugement de la Providence, la vraie religion a manqué à ton choix. Réveille-toi, c'est l'heure...58 » Rome a encore un avenir, à condition de se convertir, c'est-à-dire de délaisser son paganisme et de suivre la trace des martyrs. Comme eux, les Romains doivent s'emparer de la «patrie céleste» au lieu de cultiver la nostalgie du passé : «A cette patrie, nous te convions : viens, nous t'en prions, t'adjoindre à ses concitoyens... » La religion chrétienne est l'unique chance, car elle est seule à offrir la voie qui conduit à la Cité de Dieu.

Conclusion:  Chadwick écrit: "La cité de Dieu ne saurait être considérée comme
n-  tr ité	d	théorie	politique,	ou	comme	l'expression	d'une	philosophie	de
1
1 h,stoire...L h1sto1re du monde voit l'essor et la chute des grandes puissances et la a,son, de c phén mène est loin d'être claire... Mais le croyant sait que ce ui est mcoherent a l'esprit de l'homme est cohérent pour Dieu. Les désastres peuvent nous arracher des larmes, mais ne devraient nous étonner en aucun cas (E 111, 2). Augu tin offre beaucoup plus d'espoir à l'individu qu'aux institutions de la société humaine, particulièrement sujettes à transmettre un égoïsme collectif.m "
59 La Cité de Dieu V, 16. BA 33, p. 715.
57 La Cité de Dieu Il, 21. BA 33, p. 369 s.
58
Non annis, sed provectibus. De vera religione 26, 49. BA 8, p. 92 s.
5 La Cité de Dieu Il, 29, 1. BA 33, p. 405.
eo Henry CHADWICK, Augustin, cerf, 1987, p. 143
25
 
-  --  ..,-, .....  uuu,o ,c 1.n.,e ut: uucleur oe ta grace. l. es1. te comoal cunut: r-t:1c1yt:
qui l'amènera à cetrtaines dérives théologiques en particulier sur le péché originel61   Avant d'aborder l'enjeu théologique de la question, il faut rappeler les principales étapes de l'histoire qui a entrainée la condamnation de Pélage. Puis nous pourrons évoquer les conséquences.

1 . Première  vague : les pélagiens  en Afrique.

Originaire des lies britanniques, où il naquit vers 350, Pélage était établi à Rome dès 382. Il y enseigne donc depuis de longues années au moment de la chute de la ville en 41O. On l'a décrit comme une «nature froide», imbu d'une «estime exagérée de son propre mérite». Mais c'est là sans doute une lecture rétrospective. D'autres soulignent au contraire son «esprit religieux». Paul Orose le caricature en le décrivant comme un Goliath, allusion sans doute à son physique, et Jérôme le traite de
«gros chien de montagne» et même de «grand abruti». En réalité, sa vie, sa personnalité, sa spiritualité ont été déformées par ses adversaires. Autant avouer qu'il nous est mal connu6'Z . C'est un laïc qui entraîne dans les voies spirituelles un certain nombre de chrétiens et de chrétiennes en leur prêchant l'effort et l'ascèse.

Au moment de l'invasion des barbares, Pélage se réfugie comme beaucoup d'autres en Afrique du Nord, puis en Palestine. Ses idées ne retiennent pas particulièrement l'attention jusqu'au moment à ses disciples les répandent en Afrique du Nord, en particulier un certain Caelestius63   On peut retenir différentes phases au terme desquelles Pélage fut définitivement condamné :

-	Dès 411, CéIest i us fut condamné par un concile à Carthage auquel Augustin ne participa pas. Pélage, qui se désolidarisa de Célestius, échappe à une première condamnation, lors du concile de Diospolis, en 41 S. Entre temps, Augustin avait pris connaissance des écrits de Pélage. Il souligna d'emblée les équivoques du concile de Diospolis, et mit en évidence les conséquences de la doctrine de Pélage pour la foi catholique. En 416, les Africains, réunis à Carthage et entraînés par Augustin, renouvellent leur condamnation de la doctrine pélagienne.

-	La condamnation est tranmise au Pape Innocent 1er afin d'obtenir confirmation . Le 27 janvier 417, Innocent 1er leur répondit par plusieurs lettres, les félicitant d'avoir consulté le Saint-Siège et faisant l'éloge de leur vigilance. La condamnation de Pélage leur paraissait sans appel. S'adressant au peuple de Carthage,
61Cf. La synthèse de cette question dans «L'homme et son salut», in Histoire des dogmes, sous la direction de Bernard Sesboüé. Declée, 1995, p. 150 s.
e2 Cf G. de Plinval, Pélage, ses écrits, sa vier et sa réforme. Lausanne, 1943. Voir aussi Goulven Madec, dans Itinéraires Augustiniens n° 13 (janvier 1995).
e3 Augustin n'écrira pas moins de 17 traités contre eux : 8 contre Pélage et Caelestius, 4 contre Julien d'Eclane, 5 contre les semi-pélagiens, sans compter les sermons et les Lettres.
 
Aug stin ne cache pas sa joie, estimant que la question est réglée. «Déjà deux conciles (Carthage et Milev, été 416) ont envoyé leurs décisions au Siège Apostolique dont les rescrits sont parvenus. La cause est finie, plaise à Dieu que l'erreur finisse de même (Sermo 131, 1O). causa finita est...

-	En réalité,. elle était déjà en train de prendre une nouvelle tournure, car le successeur d'innocent, Zosime, qui connaissait mal le dossier, et qui se laissa fléchir par les déclarations d'intention de Célestius et Pélage. Zosime annula la condamnation en 41 7. Il convoqua un synode romain qui «eut peine à retenir ses larmes en constatant qu'on avait pu diffamer de la sorte des hommes d'une foi aussi parfaite» ! Accusés d'avoir agi inconsidérement, les Africains réagirent à leur tour très violemment.

-	Un nouveau concile est convoqué à Carthage en 418. Entre temps, à Rome, du fait de Célestius, les désordres se multiplient, si bien que les pélagiens finissent que se rendre odieux. Le pape accepta que fut constituée une commission, dont fit partie Augustin, qui eut la mission d'examiner la question de fond, et il se rallia à sa conclusion.

-	Après 418, Pélage disparaît de la scène . Mais pour Augustin, le combat allait se poursuivre, notamment contre Julien  d'Eclane - un disciple de Pélage, né en 386 et élu en 416 comme évêque d'Eclane - qui refusa de se soumettre et continua la lutte. Augustin lui répliquera dans différents traités, soutenant inlassablement la même thèse : la nécessité de la grâce même pour les enfants, en raison du péché originel.

2.	Deux  théologies  incompatibleses

Pour saisir l'enjeu du débat, nous considérerons d'abor la thèse de Pélage, puis celle d'Augustin. Pour éclairer le débat, il faudra s'interroger sur les raisons de leur divergences. Commençons par la thèse de Pélage, une thèse qui nous est surtout connue par ses adversaires. Cette thèse présente deux faces.

-	L'homme est libre. A la création, Dieu a donné à l'homme la liberté et la raison, donc la possibilité d'exécuter volontairement le dessein de Dieu et ainsi de mériter par lui-même le salut. Pélage exalte la liberté humaine, plus exactement le libre arbitre, cette capacité dont dispose l'homme de choisir entre le bien et le mal. Fondamentalement, il insiste sur la valeur de l'homme. La perfection n'est pas seulement une possibilité de la nature humaine, mais une obligation morale. Pélage prêche une religion volontariste. En 413, il écrit à Oémétriade, décidée à se faire religieuse : «Chaque fois que je dois donner des règles de conduite et tracer la voie d'une vie sainte, je mets toujours en premier fieu l'accent sur la puissance et la valeur de la nature humaine et sur ce qu'elle est capable de réaliser de peur que ne serve à rien d'exhorter les gens à entreprendre une tâche qui leur apparaitrait impossible à accomplir. » C'est une théologie de l'autonomie humaine, marquée au coin d'un certain naturalisme et d'un certain rationalisme.



e◄Ad Demetriadem 2 (PL 30, 17 B), cité in Peter BROWN, La vie de saint Augustin, Seuil, 1971, p.
406.
27
 
- Dieu est	juste. C'est l'autre aspect de la thèse. Etant juste, Dieu ne peut que	e
récompenser les bons et punir les méchants. Admettre l'idée d'un péché héréditaire
serait contraire à la morale. Adam n'est pas cet être mythique qui nous aurait tous entraîné dans le péché, mais c'est chacun de nous. Pélage n'ignore pas le Christ, mais il le considère seulement comme un modèle à imiter. La grace du Christ n'est pas une rédemption, mais un «exemple» à imiter,  un secours donc purement extérieur.
«Dieu est à côté de l'homme, non en lui.fi5 »

\begin{Def}[Grâce]
salut	gratuit
\end{Def} 
 

\begin{Def}[Liberté]
capacité de choisir
\end{Def} 

 \includegraphics[]{Augu}





Salut	mérité [un droit de l'homme]
 



REDEMPTION





CREATION
 


A ce naturalisme, Augustin défend la thèse de la grâce. Dès 414, il entreprend de réfuter le livre de Pélage sur la «nature humaine». Il insiste sur deux aspects:

-	Une		nature	blessée.	Alors que Pélage insiste sur la bonté de la nature humaine, pensant ainsi rendre hommage et justice à son «créateur», Augustin lui reproche d'oublier que cette nature est blessée par le péché, ce qui la rend incapable de faire le bien sans l'intervention du «rédempteur».		Voici comment Augustin s'exprime :	«Mais, quand il (Pélage) croit servir la cause de Dieu en défendant la nature, notre auteur ne remarque pas qu'en déclarant cette nature saine, il évince la miséricorde du médecin...Or, nous ne devons pas louer le créateur de façon à nous trouver réduits à avouer que le sauveur est inutile.66»	Augustin n'ignore pas le libre arbitre, mais le pouvoir de choix ne devient efficace que s'il est d'abord libéré.

-	La nécessité du rédempteur. Alors que Pélage développe une théologie de la création, Augustin se fait le champion d'une théologie de la rédemption. Alors que Pélage met l'accent sur la liberté humaine, capable de choisir entre le bien et le mal, et donc capable de mériter son salut, Augustin estime que la liberté étant aliénée, elle ne devient capable de faire le bien que si d'abord elle a été restaurée dans son pouvoir
es Isabelle BOCHET, Saint Augustin et le désir de Dieu. Et. aug. , 1982, p. 320.
ee De natura et gratia 34, 39, in BA 24. Ppour une présentation d'ensemble, voir Paul Agaësse, L'anthropologie chrétienne selon saint Augustin. Image, liberté, péché et grâce. Centre Sèvres, Paris, 1980.
28	(
 
de,.1e faire. Or, étant donné le péché originel, seul Dieu peut redonner à l'homme ce qu ''· a perdu. If n'est donc pas légitime de s'attribuer à soi-même un quelconque ménte. Toute notre justification nous vient de Dieu et le salut est un don gratuit que nul n'a mérité par soi-même. "Tous ont péché" (Rm 5, 18), ce qui signifie que "personne n'est justifié si le Christ ne le justifie"67   Seule la foi dans le Christ rédempteur peut sauver l'homme, «la foi en Jésus Christ fait homme, la foi en son sang, la foi en sa croix, la foi en sa mort et en sa résurrection !88 » Il n'y a donc pas à hésiter sur la nécessité absolue de la grâce dans la condition de l'homme pécheur :
«Reconnaissons que la grâce est nécessaire; crions : Malheureux homme que je suis! qui me délivrera de ce corps qui me voue à la mort? Et que l'on réponde: la grâce de Dieu, par Jésus-Christ notre Seigneur.18 »

Quel est l'enjeu du débat ? Regardons d'abord la thèse	de Pélage. En gros, Pélage insiste fondamentalement sur la valeur de l'homme. A la création, Dieu a donné à l'homme la liberté et la raison, donc la possibilité d'exécuter volontairement le dessein de Dieu et donc de mériter par lui-même le salut. C'est une théologie de l'autonomie humaine, marquée au coin d'un certain naturalisme et d'un certain rationalisme. Pélage vantait les capactiés de la nature humaine. Cet optimisme au sujet des capacités de l'homme devait le conduire	à précher un volontarisme moral . L'homme pélagien est un être "sans reproche" :

« Ce n'est pas grand-chose, disait-il d'être un exemple pour les païens...Ce qui est beaucoup mieux, c'est d'être tel que les saints eux-memes soient édifiés.» (Cf. E.U.)

En face de ce naturalisme, qui menaçait de dévaloriser la grâce, Augustin réagit vivement en défendant la thèse de la grâce. Alors que Pélage insiste sur la bonté de la nature humaine, pensant ainsi rendre hommage et justice au "créateur" de cette nature, Augustin lui reproche d'oublier que cette nature a été blessée par le péché, ce qui la rend incapable de faire le bien et nécessite l'intervention du "rédempteur" et de la grâce. Voici comment Augustin s'exprime:

« Mais, quand il croit servir la cause de Dieu en défendant la nature, notre auteur ne remarque pas qu'en déclarant cette nature saine, il évince la miséricorde du médecin...Or, nous ne devons pas louer le créateur de façon à nous trouver réduits à avouer que le sauveur est inutile.» (De natura et gratia 34, 39, BA 24).

A l'arrière-plan de ce débat, on distingue à la fois deux  expériences différentes.

-	L'expérience d'Augustin est celle d'un converti qui s'est heurté à ses propres limites et pour qui sa conversion est l'expérience fondatrice de sa propre pensée. Le mouvement premier d'Augustin est toujours l'action de grâce pour ce que Dieu a fait de sa vie. Comme l'a bien noté G. Greshake : «Il est impossible, dans la prière, de se présenter à Dieu et de lui dire : Je dois mon salut à toi et à moi, à ce que tu as fait et à ce que j'ai fait. Dans la confessio, on peut dire seulement : ce que j'ai, je



67  De natura et gratia 41, 48.
88 lb. op. cit. 44, 51.
av Rom 8, 24-25, ib. op. cit. 53, 61.
29
 
te ledois.7»0	li dira encore : «Tu es miséricorde, je suis misère.» (Conf. X, 28, 39).	(

-  Pélage nous est  trop peu  connu pour savoir à quelle expérience pers n elle    e réfère. Ce qui est sûr, c'est qu'il représente un type de chnst,anisme qui fait appel en priorité non pas à la grâce, comme Augustin, mais aux capacités de l'homme.

3.	Nouvelle vague : les moines d' Adrumète et Julien d'Eclane

Le jeu n'était donc pas calmé avec la condamnation de Pélage. Une nouvelle phase	C
allait se produire,	provoquée soit par certains disciples trop dociles d'Augustin, tels
 
les	moines du couvent africain d'Adrumète (Sousse, au Sud de Carthage), ou au
 
'il rt
 
contraire par des disciples de Pélage, tel Julien d'Eclane.

- Les moines d' Adrumète répandirent, vers 425-426, l'idée que, puisque Dieu fait tout, la liberté n'est rien et n'a rien à faire. Prenant trop à la lettre certaines affirmations d'Augustin sur la grâce, ils en tiraient que la liberé est inexistante. Alors que les Pélagiens niaient la grâce au nom de la liberté, à l'inverse, ces moines défendaient «la grâce de Dieu jusqu'à nier le libre arbitre de l'homme». Augustin leur
 

rbit
· il
la
 
fit savoir que «confesser la grâce comme il l'avait fait, ce n'était pas nier le libre
 
'at
 
arbitre ni le mérite...» Il écrit  à  Valentin, abbé du monastère :	ma
 
« Certains parmi vous exalteraient la grâce au point de nier le libre arbitre de
 

./IYi
 
l'homme, et, ce qui est plus grave, soutiendraient qu'au jour du jugement Dieu n'aura
 
liJ
 
pas à rendre à chacun selon ses oeuvres...» (p. 53)	t
«S'il n'y a pas de grâce divine, comment Dieu sauve-t-il le monde ? Et s'il n'y a	à
 
pas de libre arbitre, comment juge-t-il le monde ?...Ne niez pas la grâce de Dieu, et
 
mp1
 
ne défendez pas le libre arbitre jusqu'à le rendre indépendant de la grâce de Dieu,
comme	si nous pouvions sans elle concevoir	ou	accomplir quelque chose	selon	m
Dieu 11     » (p.  SS)

 
Augustin invite à tenir ensemble les deux vérités : le libre arbitre	et
 

er1
 
la grâce. Il énumère alors une série d'affirmations tirées de !'Ecriture, les unes en
faveur de la grâce, les autres en faveur du libre arbitre. Si l'on ne peut pas	fr
comprendre comment les deux s'articulent, il faut s'en tenir à !'Ecriture qui .les
affirment simultanément : "Elle nous enseigne à la fois la réalité du libre arbitré · humain et la réalité de la grâce divine; grâce sans laquelle le libre arbitre ne peut ni se tourner vers Dieu ni progresser en Dieu.» (p. 59). Y a-t- il une coordination possible des deux termes ? Augustin donne à la grâce une extension telle qu'elle englobe la liberté. La liberté est déjà grâce. La formule la plus parfaite pour exprimer le lien
entre les deux se trouve dans le "De correptione et gratia"n :	rE
nne
les
10 G. GRESHAKE, Geschenkte Freiheit. Einführung in die Gnadenlehre, Freiburg, 1977, p.
47.	Cité par Goulven MADEC, Pélage et Augustin. Le débat sur la liberté et la grâce.
Itinéraires augustiniens 13 (janvier 1995).
" De gratia et libero arbitrio. B.A. 24, p. 55. "Il en est qui défendent la grâce de Dieu
jusqu'à nier le libre arbitre de l'homme".	t,
12 B.A 24 p. 276 cf. aussi B.A. 73 B : page 480, la grâce et la liberté, et p. 477 ,
préscience divine et liberté humaine.
30
 
«Aguntur  enim  ut	agant,  non  ut	ipsi nihil agantl» (11,  4)
«Ils sont en effet	agis pour agir, non pour ne rien fairel»

-  Avec  Julien d'Eclane, le défenseur  de la liberté, le combat allait prendre une tournure plus dure. Julien invoquait à l'encontre d'Augustin un certain nombre de textes  sur la liberté - en fait sur le libre arbitre - que celui-ci avait écrits autrefois. Augustin sera obligé de rétablir ainsi l'équilibre, mais en mettant de plus en plus l'accent sur la grâce, si bien qu'on peut parfois se demander s'il réussit encore à reconnaître la place de la liberté. Il tire de !'Ecriture une série d'affirmations, les unes en faveur de la grâce, les autres en faveur du libre arbitre. Si l'on ne peut pas comprendre comment les deux s'articulent, répond Augustin, il faut s'en tenir à !'Ecriture qui les affirment simultanément : «Elle nous enseigne à la fois la réalité du libre arbitre humain et la réalité de la grlce divine; grlce sans laquelle le libre arbitre ne peut ni se tourner vers Dieu ni progresser en Dieu.»

L'affrontement entre Augustin et Pélage se joue sur l'idée que l'un et l'autre se font  de la liberté. Pour Augustin, le libre arbitre n'est pas la liberté, mais il est  la simple capacité de choisir,  de disposer librement de sa propre volonté.
« Notre volonté ne serait même plus volonté si elle n'était en notre pouvoir. Mais puisqu'elle est en notre pouvoir, elle est libre pour nous; car ce qui n'est pas libre pour nous, c'est ce qui n'est pas en notre pouvoir; et ce qui l'est ne peut pas ne pas être libre.73» Pélage n'a aucune peine à accepter cette définition. li la tirait même à lui pour démonter que Augustin n'était pas toujours Augustin, c'est-à-dire le champion intransigeant de la grâce. Seulement, alors que pour Pélage, le libre arbitre, confondu avec la liberté, est la capacité de l'homme à choisir entre le bien et le mal, pour Augustin, il se restreint  à la responsabilité de l'homme dans le mal : Il est la
«facuitas peccandi» (1, 16, 35).  Il est rare qu'il fasse appel au libre arbitre dans le choix du bien74   L'orientation vers le bien relève de· la liberté, la  «vraie»  liberté se manifestant dans la soumission à Dieu et n'existant que par la grâce. Or, la liberté qui s'est perdue en raison du péché, est incapable de se réenraciner dans le bien. Pour agir de nouveau selon Dieu, il faut une restauration de la liberté par la grâce. Pélage qui ne connaissait que le libre arbitre ne réalisait pas que, sans la grâce, donc sans la vraie liberté,  le libre arbitre était une serf-arbitre.

Si l'on voulait entrer plus avant dans ce débat, il faudrait évidemment évoquer le problème  de  la  prédestination, qui vient se greffer sur celui de la liberté. Puisque la liberté de l'homme n'est rien sans la grâce, n'est-elle pas une pure illusion, totalement régie de l'extérieur par une autre liberté, celle de Dieu, qui donne ou refuse sa grâce avant tout mérite de l'homme? C'est cette thèse que soutiendront les jansénistes, dont on reparlera. Augustin est moins radical. Mais ses propos sont souvent ambigus, sinon choquants 75   La polémique l'a parfois entraîné dans certains

73 Sur le libre arbitre 111, 3, 8. Cf. G. Madec, Itinéraires augustiniens 13 (janvier 1995.
H Cf. Enchiridion XXVIII, 106. «Car si le péché dépendait du libre arbitre seul, cependant, pour conserver la justice, lelibre arbitre ne suffisait pas si la la participation du bien immuable ne lui assurait le secours divine.»
H Cf. Enchiridion XXIII, 93. BA 91 p. 269 : «Bénigne entre toutes,à ne pas douter, sera la peine de ceux qui au péché qu'ils tenaient de leur origine se gardèrent d'en ajouter aucun.» Cette thèse a été exacerbée par les jansénistes et les calvinistes, qui ont introduit l'idée d'une prédestination négative, Dieu décidant arbitrairement, d'avance et de façon absolue, de
la destinée éternelle de chaque être.
31
 
dé pages, par exemple lorsqu'il écrit, à propos des vertus des païens, qu'elles ne sont	(
qu enflure et orgueil et, on doit, à ce titre, les regarder, non comme des vertus, mais
comme des vices»76 ; ou à propos des enfants non baptisés, lié au problème de la Prédestination, qui sont à jamais exclus de la béatitude, même si leur peine est
atténuée. La prédestination est un mystère qui renvoie à l'insondable volonté de Dieu. Si Augustin n'a pas toujours la formule équilibrée de ce mystère, puisque la volonté de Dieu est, d'une part «que tous les hommes soient sauvés», mais que d'autre part
«beaucoup plus grand est le nombre de ceux qui ne le sont pas», il refuse cependant toujours les excès auxquels certaines de ses formules se prêtaient77  L'exemple de la prédestination est le plus significatif. Augustin s'interdit de l'enseigner comme s'il s'agissait d'un fatalisme, réduisant à néant la liberté78 

S. Plaidoyer  en faveur de la grâce71

Augustin ne prétend pas avoir une théologie personnelle. Son unique livre de théologie est !'Ecriture. C'est pourquoi, il faut voir sur quels textes il appuie l'absolu de la grâce dont il se fait le champion face aux Pélagiens . Je voudrais citer quelques textes clefs qui lui servent à appuyer ses affirmations sur la grâce«>

a.	"Qu'as-tu que tu n'aies reçu... Et si tu l'as reçu, pourquoi t'en faire gloire comme si tu ne l'avais pas reçu ?" (1 Co 4, 7). Et voici le commentaire :

« C'est principalement ce témoignage de /'Apôtre qui m'a convaincu moi-même, quand j'étais dans une erreur semblable à celle de nos frères - il s'agit des moines de Provence - et je m'imaginais que la foi, par laquelle nous croyons en Dieu n'est pas un don de Dieu, mais que nous l'avons de nous-mêmes, et que nous obtenons ainsi par elle les dons divins qui nous permettent de 'vivre en ce monde avec modération' (Tit 2, 12). Je ne voyais pas que la foi est précédée en nous par la grâce de Dieu, pour que par elle nous soit ensuite donné ce que nous demandons utilement...»(De praedestinatione sanctorum 3, 7, B.A. 24 p. 479)
«La foi elle-même figure parmi les dons divins qui nous sont dispensés en ce même Esprit. Ces deux choses : croire et agir, sont nôtres en raison de notre libre arbitre, et cependant l'une et l'autre sont données par /'Esprit de foi et de charité. Car ce n'est pas la charité seulement, mais, comme il est écrit, "la charité avec la foi qui descend de Dieu le Père et du Seigneur Jésus-Christ" (Eph 6, 23).(ib. 3, 7, p. 481-483)





18 Cf. Cité de Dieu XIX, 25, . BA 37, p. 165-167, et p. 761, avec d'autres références p. 957. J. WANG TCH'ANG-TCHE, Saint Augustin et les vertus des païens. Beauchesne, 1938.
11 Enchiridion XXIV, 97 s.
78 Voir par exemple : De dono perseverantiae, 22, 57, BA 24, p. 741, et autres référence page 859. Augustin précise «la manière dont il convient d'enseigner la prédestination», insisitant sur le fait qu'on doit éviter de la présenter «de façon à la rendre odieuse».
79 Voir le dossier scripturaire dans Sesboüé, op. cit. p. 167.
8°Ces textes ont été sélectionnés par Jacques Pintard, Le docteur de la grâce, in Augustin, le
message de la foi, DDB, 1987, pp. 119 sv.:


32
(
 
b.	"Personne ne peut venir à moi si le Père qui m'a envoyé ne l'attire" (Jo 6, 44) "Magnifique éloge de la grâce! Nul ne vient s'il n'est tiré..." (26, 2, BA. 72, p. 487). Aussi est-ce juste de glorifier Dieu pou les mérites : "Lo!"5que tu couronn s leurs mérite tu couronnes tes propres dons. (Lettre 194, à Sixte, 19). Augustm illustre un u plus loin sa pensée par une comparaison cet attrait qu'exerce le Père
pour l'attacher au Christ :
«Donne-moi quelqu'un qui aime, et il sentira la vérité de ce que dis. Donne-moi un homme tourmenté par le désir, donne-moi un homme en marche dans ce désert et qui a soif, qui soupire après la source de l'éternelle patrie, donne-moi un tel homme, il saura ce que je veux dire.» (lb. 26, 4, pp. 491-493).
«Tu montres  un rameau vert à une brebis, tu l'attires. On présente des noix à
un enfant, il est attiré...Si donc ce qui est révélé des délices et des voluptés terrestres à ceux qui les aime les attire, ...comment le Christ révélé par le Père n'attirerait-il pas ? » (ib. 26, s, p. 497).
c.	"Dieu résiste aux orgueilleux et donne sa grâce aux humbles" (Proverbes 3, 34, et cité en I P 5, 5 et Jac 4, 6). Augustin cite SS fois cette sentence dans son oeuvre, selon la comptabilité de La Bonnardière. Elle indique qu'il n'y a pas d'autre voie vers Dieu que l'humilité du Verbe.

Augustin, on l'a vu, fut profondément marqué par cette découverte du Christus humilis. Le grand obstacle à la rencontre de Dieu est l'orgueil: "Dieu s'est fait humble, et l'homme est orgueilleux" (Sermon 142, 6)

d.	"Donne ce que tu commandes et commande ce que tu veux" Da quod jubes et jube quod vis ! (Conf. X, 29, 40; 31, 45; 37, 60). Augustin lui-même rappelle combien cette sentence avait irrité Pélage. Il écrit dans le De dono perseverantiae :
\begin{quote}
    « Parmi mes ouvrages, en est-il un qui ait été plus répandu et plus goûté que mes Confessions ? Or, dans cet ouvrage, publié lui ausi avant que n'eût paru l'hérésie pélagienne, je dis, et plusieurs fois à notre Dieu :"Donnez ce que vous commandez et commandez ce que vous voudrez." Ce sont ces paroles de moi que Pélage, à Rome, entendit un jour citer par un de mes frères et collègues dans l'épiscopat; il ne put les supporter, et dans l'émotion vive qu'il mit à les contredire, peu s'en fallut qu'il se prit de querelle avec celui qui les avait rappelées. Mais qu'est-ce que Dieu nous commande d'abord et par dessus tout, sinon de croire en lui ? Donc, c'est lui qui nous donne de croire..." (B.A. 24, 20, 53)
\end{quote}


\begin{quote}
    L'espespérance ne déçoit point, parce que l'amour de Dieu a été répandu dans nos co ur par le Saint-Esprit qui nous fut donné" (Rom 5, 5). 
\end{quote}
Un verset qui est cité pas moins de 200 fois. Augustin précise:
\begin{quote}
    « · Quand nous disons que Dieu aide à accomplir toute justice et opère en nous le vou o r et le faire (cf. Ph 2, 13) ... ce n'est pas parce qu'il fait ressentir en nos sens xt ieurs les préceptes  de la justice, mais parce qu'il donne l'accroissement mt neur   I Co 3, 7), en répandant la charité dans nos coeurs par /'Esprit-Saint
qw nous a ete donne» (L'esprit et la lettre 25, 42)

\end{quote}

 
 
Loin de contredire le libre arbitre, la présence de !'Esprit le dégage au contraire	( pour le rendre libre : "Partout où est !'Esprit du Seigneur, là est la liberté (2 Co 3,
17, là, le coeur humain est dilaté, ce qui fait dire à Augustin : « Cet Esprit de Dieu dont
la présence nous justifie, nos inspire la haine du péché et nous donne la liberté spirituelle» (ib. 16, 28).

6. Nouveaux	rebondissements sur liberté et	grâce

La thèse de la prédestination rebondit plusieurs fois au cours de l'histoire. Qu'il suffise de donner quelques points de repères. Trois moments, de plus ou moins d'ampleurs, on marqué ce débat.

a)	La question rebondit une première fois au IXe siècle, avec les thèses d'un moine, Gottschalk81 , fils d'un prince saxon, qui vivait au monastère d'Orbais (diocèse de Soissons). D'un esprit étroit, il enseignait que Dieu prédestine aussi bien les damnés à l'enfer que les élus au ciel. L'idée même de prédestination absolue entraîne aussitôt chez Gottschalk la négation de la liberté. De plus, introduisant I' idée d'une prédestination négative, il transforme Dieu en monstre, puisqu'il devient l'auteur du pire mal qui puisse arriver à l'homme. Inutile d'ajouter que, dans ces conditions, la rédemption laisse totalement hors de son champ ceux qui sont d'avance exclus du salut.

Cette théorie monstrueuse, qui se prétend l'expression de la pensée d'Augustin, sera condamnée : le concile de Valence (855) réaffirmera l'universalité du salut et la rédemption de tous, en même temps qu'il rcconna1t .; tout homme la liberté et le pouvoir de se sauver. Durant le Moyen Age, c'est cet augustinisme tempéré qui va s'imposer. Sans rien enlever à l'absolu de la grâce, les scolastiques partiront de l'idée de l'universalité du salut, une doctrine qui était passée quelque peu sous silence, et le reste, c'est-à-dire grâce et liberté, y est subordonné. Pour sauver les hommes, Dieu n'est pas tenu par les sacrements, dira saint Thomas. Il dispose de bien d'autres moyens.

b)	Avec  la  Réforme,  la  question  deviendra  beaucoup  plus  aiguë. Luther et Calvin se réclament tous les deux d'Augustin. De tous les auteurs anciens, Augustin est à leurs yeux le meilleur interprète de !'Ecriture. «Augustinus  meus totus est», déclare Luther dans sa polémique avec Erasme : «De ton côté sont les théologiens modernes et tant d'universités, de conciles, d'évêques, de papes! ...De mon côté, au contraire, seulement Wyclif et Laurent Valla, mais aussi Augustin, que tu passes sous silence, m'appartient tout entier.» De même, Calvin se place sous l'autorité  d'Augustin,  qu'il  cite  341  fois  dans  la  seule  Institution  chrétienne.
«Augustinus totus noster est», écrit-il en écho à Luther : «Quant à saint Augustin, il s'accorde si bien en tout et partout avec nous, il est tellement nôtre (adeo totus noster est) que s'il me fallait écrire une confession sur cette matière, il me suffirait de la composer des témoignages extraits de ses livres... Je ne dis rien qu'il n'ait dit devant moi mot à mot.» Si l'on considère ce qui a séduits les Réformateurs, c'est, on s'en doute, la doctrine augustinienne de la grâce et de la justification. Chez Calvin, c'est toujours dans le contexte de sa défense de la prédestination qu'il cite Augustin, sans

81 Cf. Kurt FLASCH, Introduction à la philosophie médiévale, Cerf, Ed. universitaires de Fribourg, 1992, p. 29 s., en particulier p. 33 s. sur Godescalc et la prédestination divine.
34
 
s a r.cevoir que l'idée d'une double predestination, le salut pour les uns et la J>erdation Pour les autres, dont il se fait le défenseur, est totalement étrangère à Augustin82 

Quand en effet on y regarde de plus près, on s'aperçoit que, au-delà de l'appel explicite à son autorité, Augustin ne sert chez Luther et Calvin qu'à cautionner des thèses qui sont les leurs plus que les siennes. Cet a priori de lecture est évident chez Luther. Alors même qu'il se croit dans la ligne d'Augustin, il oublie des pans entiers de sa pensée. En particulier, sa thèse sur la justification ne saurait être tirée d'Augustin sans autre forme de procès. Augustin ne dissocie jamais foi et charité. Alors que Luther soutient que la foi seule justifie, Augustin dit que c'est la charité dans son obéissance à la loi. Luther n'a d'ailleurs pas été sans lui-même relever cet écart. Mais au lieu de rectifier sa propre position, il accuse Augustin d'inconséquence ou d'infidélité à lui-même. Melanchton, avec l'accord de Luther, avouera explicitement à Brenz : Augustin, qui avait d'abord nié que l'homme puisse être justifié par lui­ même devant Dieu, «imagine ensuite que nous sommes réputés justes en raison de l'accomplissement de la loi que le Saint-Esprit opère en nous... Augustin n'est pas en accord complet avec la doctrine de saint Paul, bien qu'il en approche davantage que les Scolastiques... » Il ajoute que, pour le public, il faut continuer à invoquer l'appui d'Augustin, «bien qu'il n'explique pas assez la justice par la foi»83 

c)	Le	jansemsme	représente	le	troisième	rebondissement.	Sa préhistoire commence avec Michel Baius (1513-1589), chancelier de l'université à Louvain, qui avait fait d'Augustin son cheval de bataille contre les protestants. Il avait lu, nous dit-on, neuf fois tout saint Augustin et soixante-dix fois les écrits sur la grâce f Ce qui ne l'a pas empêché de se fourvoyer sérieusement. Le point névralgique de sa position tourne autour du rapport entre nature et grâce. Dans l'état initial d'Adam, la «nature» de l'homme était dotée d'une puissance autonome en face de Dieu, c'est-à­ dire d'une liberté capable par elle-même de faire son salut, si bien que si l'homme était fidèle à sa nature, il pouvait en droit prétendre à la récompense. «Chez l'homme d'avant la chute, la vie éternelle... n'aurait pas été un don de la grâce, mais un salaire.» C'est la thèse de Pélage : «Dieu me fait homme, mais c'est moi qui me rend juste», sauf que Baius parle de la condition de l'homme	d'avant la chute, tandis que . Pélage parle de sa condition actuelle .

On a dit de Baius qu'il était un «Pélage  du  paradis  terrestre». Depuis la chute, en effet, la grâce est devenue indispensable à l'homme, comme un ajout extérieur à une nature blessée et déficiente, alors qu'elle ne l'était pas avant. En se voulant fidèle à Augustin, Baius en fait le trahit, car pour Augustin,aussi bien dans l'état primitif que dans l'état actuel, sans la grâce, l'homme est incapable de parvenir au salut. Adam, pas plus que l'homme déchu, ne peut s'attribuer à lui-même un quelconque mérite04  	Au fond, Baius n'a sauvé Augustin qu'au prix d'une double déformation, à la fois du concept de nature et du concept de la grâce. Au regard de la pensée d'Augustin, l'extrinsécisme de Baius est une aberration. Mais la question se pose aussitôt : Baius, qui a été condamné, n'aurait-il pas entraîné dans sa chute Augustin lui-même? C'est ce que prétendaient certains jésuites.

82 Cf. Augustinus Magister. Etudes augustiniennes, 1954. Il, 1029 s. pour Luther et saint Augustiin (Léon Cristiani), et 11, 1839 s. pour Calvin et saint Augustin (Jean Cadier).
63 Léon CRISTIANI, Luther et saint Augustin, dans Augustinus Magister Il, 1954, p. 1037 -1038.
84 Cf. Henri de LUBAC, Augustinisme et théologie moderne. Aubier, 1965, p. 15 s.
35
 
(
C'est pour réfuter cette prétention que Jansénius (1585-1638), fu ur évê ue d'Ypres, rédige un ouvrage de synthèse sur la pensée d'Augustin : I' Augustmus qui ne sera publié qu'après sa mort, en 1640, et dont les jansénistes eront leur Bible. De Jansénius aussi, on nous dit qu'il avait lu «dix fois saint Augustm, et plus d? tren e fois les ouvrages de la grâce contre les Pélagiens», ce qui ne l'empêche pasd aboutir, en fin de compte, à une «consciencieuse méprise». En réalité, il ne fait que prolo,nger Baius, mais en l'inversant. Alors que Baius part d'une nature qu1, au moms dans I état d'Adam, peut prétendre au salut, la grâce devenant inutile, Jansénius part de l'absolu de la grâce, expression de la puissance arbitraire de Dieu, qui la donne ou la refuse sans tenir compte de l'homme, pour refuser ensuite à la nature humaine, d nc à la liberté, toute consistance propre. «Si l'un et l'autre tendent également à dissoudre l'union entre Dieu et l'homme en quoi consiste essentiellement le Mystère du Christ, écrit le P. de Lubac, n'est-ce pas que l'un dresse l'homme en face de Dieu dans la réclamation de ses droits, tandis que l'autre l'anéantit ?fJS »
La thèse de Jansénius sera à l'origine de tous les débats modernes autour de la grâce, dont Port-Royal,  avec l'abbé de Saint-Cyran à sa tête, sera le foyer en France. Délaissant les spéculations de Baius sur l'état primitif de l'homme, tout comme celles de Jansénius sur la «pure nature», Saint-Cyran s'intéresse à l'homme concret, pour le placer d'emblée devant le mystère redoutable de la grâce, une grâce rare, qui n'est donnée qu'à un petit nombre d'élus. «Comme le soleil fait des jours sacrés et des jours profanes, des jours de fête et des jours civils, des jours d'été et des jours d'hiver, ainsi Dieu a fait certains hommes pour être sauvés et saints, et d'autres profanes pour être damnés... ». «li ne tombe pas une goûte de grâce pour les païens !» Saint-Cyran, sans lequel le jansénisme n'aurait sans été qu'une «hérésie de professeurs», lui assurera le succès. Mais c'est bien à tort qu'il invoque saint Augustin. A aucun moment, Augustin n'a pensé que le destin de l'homme pouvait être scellé de l'extérieur par Dieu, sans que la responsabilité de l'homme y soit engagée. Alors que les débats sur la grâce et la prédestination, tels qu'ils se sont développés dans le jansénisme, se placent au point de vue de l'éternité, où tout semble joué d'avance, Augustin considère l'homme dans son devenir temporel, où l'alternative entre la lumière et les ténèbres reste toujours ouverte86 

\section{Conclusion}
\begin{Synthesis}
 Augustin maintiendra contre vents et marées le primat de l'action de Dieu. Il n'a pas réussi à concilier par des concepts l'action de Dieu et l'agir de l'homme. Le champion de la liberté contre les manichéens s'est battu avec la même i tran igeance en s faisant le champion de la grâce contre les Pélagiens. Augustin s'en tient a la parole de l'Ecriture, faisant siennes ce texte de l'Apôtre: .
\begin{quote}
    «  Qui es- u pour discuter avec Dieu ? (Rm 9, 20)  Impénétrables  sont ses J gem nts et imcompréhensibles ses voies (Rm 11, 33) Et il ajoute : "Ne cherche pas a attemdre ce qw est trop haut pour toi, ni à scruter ce qui te dépasse"» (Ecc; 3, 22) (De Dono perseverantiae XII, 30, B. A. 24 p. 669)
\end{quote}
\end{Synthesis}




Au cours des siècles, cette doctrine de la grâce fut tirée vers une doctrine mtolerable, celle de la pré?estination, et en particulier de la prédestination négative, notamment chez les Jansernstes et les calvinistes. Dieu est alors un Dieu arbitraire,
85lb., p. 51.
ee lb., p. 81.
 
rvers, qui décide d'avance, de façon absolue, qui va au ciel et qui va en enfer, indépendamment des mérites :

« Par décret de Dieu, et pour la manifestation de sa gloire, tels hommes sont prédestinés à la vie éternelle, tels autres voués à la mort éternelle!» (Confession d Westminster, 1647).

Une telle doctrine n'est pas augustinienne, même si certaines formules peuvent être tirées en ce sens. Mais il faut bien constater que ce sont ces excès modernes qui furent à l'origine d'une certaine forme d'athéisme, l'athéisme humaniste, qui oppose Dieu et l'homme, la grâce et la liberté, et qui rejette Dieu au nom de l'homme et de la liberté. "M'en coOtât-il d'être expédié en Enfer, disait Milton, jamais un tel Dieu ne m'imposera le respect!" (Cf. Max WEBER, L'éthique protestante et l'esprit du capitalisme, Pion, p. 117.)

Augustin était trop conscient de la liberté de l'homme pour croire qu'elle pouvait être mise en échec par la grâce. Celle-ce n'est pas contre la liberté, ni au-dessus de la liberté, mais avec la liberté qu'elle rachète et rend efficace dans l'ordre du salut. L'acte le plus libre est celui par lequel l'homme accepte de reconnaître que tout lui est donné, y compris sa liberté.

Marcel NEUSCH Institut catholique de Paris
1997-1998

























37
