\part{Confucianisme, Taoïsme et Bouddhisme Chinois}
\chapter{Introduction}

\mn{Jia Bingwei, Paris Cité, Collège de France}

\paragraph{spiritualité Chinoise}

\paragraph{objectif } acquérir une vision générale du religieux chinois dans l'histoire et aujourd'hui


\paragraph{une notion centrale : le xiushen chinois}

\begin{Def}[xiushen]
    Apparu au Vè avant JC, cette notion est :
    \begin{itemize}
        \item \textit{taillé}, \textit{corrigé}
        \item shen : le corps
    \end{itemize}
    idée de se perfectionner sans arrêt
\end{Def}

\paragraph{se dépasser à travers certaines pratiques}
Une vision de l'homme et une vision au monde et à la pratique : un système de pratique. C'est valable pour les 3 religions. \mn{proche du gnosticisme chrétien ?}

\begin{itemize}
    \item 13-20 septembre : introduction générale. La chine, chronologie et évolution territoriale, qu'est ce qu'une religion en Chine ? langue et écriture chinoise : origines et évolution
    \item 27 septembre, 4 et 11 octobre : la tradition confucéenne; Le temps de confucieus (551 -471 av JC). 

    \item Dynastie des Song (960-1279) : carrefour des trois religions

    \item 
\end{itemize}

Mode de validation : compte rendu d'un livre ou quelques chapitres d'un livre de votre choix. 

\paragraph{Transliterration selon Pinyin} en 1955. Beijing / pekin. 
Toaisme : Daoisme.

\begin{singlequote}
    le cantonnais est une langue parlée, qui ne s'écrit pas ? 
\end{singlequote}

\section{introduction à la Chine}

\paragraph{34 provinces} dont Taiwan. 3 paliers, le Tibet, à l'Est, les plaines et au milieu, en diagonale, des moyenne montagne. L'économie suit

\paragraph{Les fleuves} Le fleuve Jaune (Huanghe), car Loess. au sud, le Changjiang, le fleuve long, appelé en français, le fleuve bleu.


\paragraph{chronologie} une vieille histoire mais mouvementée. Relu en fait par les nouveaux envahisseurs pour s'approprier le payes. 

\paragraph{- 221 av JC : l'Empire de Qin} centralisé, avec une bureaucratie.

\paragraph{1911} dernier empire est renvoersé
\paragraph{1949} république populaire de la Chine.


\paragraph{les dynasties} Superficiellement, on peut avoir l'impression d'une stabilité avec un cycle "mauvaise récolte, révolte, changement de dynastie". Mais c'est ignorer ce que chaque dynastie apporte à commencer par la façon de recruter les fonctionnaires.


\paragraph{rapport entre l'homme et l'espace} Comment l'homme arrive à s'installer sur le territoire. 

\paragraph{l'homme de pékin, \textit{homo erectus}}

\paragraph{révolution néolithique 5000-3000 av JC} on peut s'installer grâce au stockage liée à la poterie.
Grâce à Andersson au XIX, les poteries à Stockholm.

\paragraph{néolithique tardif 3000-2000 av JC} des objets en jade. Hierarchisation de la société. première forme d'Etat.


\paragraph{une question : Chine singulière ou plurielle} \textit{Zhong gao} : Empire du Milieu, appelation très ancienne. Pourrait faire penser à un noyau, un centre qui aurait irradié autour. Mais les découvertes archéologiques mettent en question cette interprétation. 


\paragraph{dynastie Shang (1250-1055 av JC} culture dominaire autour de Anyang. premières traces de l'écriture, les \textit{jiaguwen}

\begin{Def}[juaguwen]
    inscriptitions oraculaires sur carapaces de tortue et os.
    Divinatoire
\end{Def}

On pose des questions à Dieu : "est ce que j'aurais une bonne récolte ?"
On fait chauffer la carapace et on interprète la craquelure (orientation et sur quel mot). 
\paragraph{Même structure} préface (quel jour on a posé la question) + quelle question (charge) + oracle proprement dit. 

\subsection{la dynastie des Zhou}
on dit : "djzo"

\paragraph{féodalisme en occident} Le roi octroie des territoires à ces membres de familles.  des Bronzes marquant la différence entre roi, prince,...


\paragraph{lien de suzeraineté} le roi donne un bronze qui marque la zuzeraineté et les remerciements du roi. On les met dans le temple des ancêtres. Souvenir pour les générations suivantes

\paragraph{le mandat céleste} Le roi croit dans le fait qu'ils ont un mandat de Dieu et qu'ils peuvent condamner.

\paragraph{Obligation de diviser le territoire} pour octroyer des terres car plus de marche. Affaiblissement économique des Rois Zhou.
Les territoires des marches arrivent à s'aggrandire (les chu, les qin,...) et deviennent de plus en plus puissants. 

\paragraph{la grande muraille} 

\subsection{L'empire des qin et des Han }
\paragraph{Qin} ne dure pas longtemps '221-206
\paragraph{Plus d'allocation de terres mais bureaucratie} Centralisation. et Standardisation des poids et mesures.Une monnaie unique. 

\paragraph{Standardisation de l'Ecriture} Ce sont majoritairement les scribes qui écrivent. Sur les tablettes et manuscrit, on passe à la variante vulgaire des \textit{Qin. } \mn{comme en grec, le Sigma qui devient C}

\paragraph{période de mélange éthnique très important} d'où le fait que le nom de la dynastie donne le nom à la population \textit{Han}. 

\paragraph{Les Hans} analysent pourquoi les Qin n'ont pas duré. Constat que c'est parce qu'ils gouvernent par les lois et non les vertus.

\paragraph{Arrivée du Bouddhisme} à la fin de la dynastie Han. 
Une véritable création pour trouver dans la langue chinoise des termes pour traduire les termes bouddhiques. 

\paragraph{Vème siècle} On écrit sur du papier et non de la soie, du bambou. On écrit avec le pinceau. 

% ---------------------------------------------------
\subsection{La dynastie des Tang 618-907}

\paragraph{bcp d'échange avec l'Asie centrale - route de la soie}

\paragraph{Ecole Shan bouddhique} en Japonais, Zen. On se concentre sur la méditation.


\subsection{La dynastie des Song}

\paragraph{La dunastie des Song du Nord}

\paragraph{Néo confusianisme}

\subsection{La dynastie Yuan}

\paragraph{fait partie de l'empire Mongol}


\paragraph{dynastie des Ming}

\paragraph{la dynastie des Qing} dynastie Mandchoue. 


\subsection{le cours}

\paragraph{les trois enseignements (jiao)} qu'on utilisait avant à la place du mot \textit{religion}
Qu'est ce qu'un enseignement, avant de parler de religion ?




