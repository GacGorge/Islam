\chapter{devoir}


\section{validation}
Mode de validation du cours   Compte rendu d’un livre ou de quelques chapitres d’un livre de votre choix parmi la liste suivante :  
\begin{itemize}

\item  Vincent GOOSSAERT, Dans les temples de la Chine, Paris : Albin Michel, 2000.  
\item  Herbert FINGARETTE (Charles Le Blanc trad.), Confucius, du profane au sacré, Montréal : Presses de l’Université de Montréal, 2004.  
\item  Adeline HERROU, La vie entre soi : les moines taoïstes aujourd’hui en Chine, Nanterre : Société d’ethnologie, 2005. Chapitres 3 – 8 inclus.  
\item  Kenneth CH’EN, Histoire du Bouddhisme en Chine, traduit de l’anglais par Dominique Kych, Paris : Les Belles Lettres, 2015. Chapitres 8 – 13 inclus.   

\end{itemize}

Ce travail devra comporter 5 ou 6 pages en police Times New Roman 12, interligne 1,5.   Date de la remise du travail : au plus tard le 9 décembre 2023. 

\section{Vincent GOOSSAERT, Dans les temples de la Chine}

\subsection{Hervieu léger}

L’auteur de ce livre se définit lui-même comme un inlassable déchiffreur d’inscriptions. Les milliers de stèles qu’il a examinées lui ont fourni la matière d’un ouvrage dont le grand intérêt est d’offrir une introduction synthétique à la « religion de la Chine ». De cette religion, l’A. a choisi de traiter dans son unité, sans distinguer entre les « grandes » religions (confucianisme, taoïsme, bouddhisme) et les « petites » traditions, sans distinguer non plus entre les formes populaires de la religiosité et la religion des élites. Car l’unité de cette religion découle de la place centrale qu’y occupe le temple – le bâtiment même – considéré comme le lieu par excellence de la pratique religieuse et de la coexistence entre les différentes traditions établies. Très peu de temples relèvent en effet d’une seule tradition, administrée et défendue par son propre clergé. Dans leur immense majorité, les temples chinois sont des espaces où coexistent et cohabitent, selon de savantes hiérarchies, de multiples divinités grandes ou petites, auxquelles les fidèles rendent un culte. Un culte qui prend physiquement la forme d’un « parcours des offrandes ».

2Pour saisir la logique de cette cohabitation -qui prend à revers les conceptions spontanées que nous avons de la subordination des pratiques cultuelles à des « croyances obligatoires » en principe clairement distinguées comme telles par les fidèles – il faut entrer dans la logique des lieux, et c’est à quoi nous entraîne la « visite guidée » d’un temple proposée par V.G. Il faut également saisir les principes paradoxaux de l’administration des temples par l’État : une administration qui s’étend aux « canonisations » (à l’enregistrement) des divinités autorisées et à l’édiction des canons scripturaux des trois grandes traditions mères, mais qui s’exerce en fait dans des limites extrêmement étroites dans la mesure où prime avant tout le respect de l’équilibre des positions. Dans ce contexte formellement mais faiblement administré, prolifèrent depuis des siècles les fondations les plus diverses, étatiques ou privées, immenses ou minuscules. Ces espaces sont chargés de fonctions multiples : lieux du culte des ancêtres, de l’audience entre le fidèle et la divinité et de l’expression d’une communauté locale, ils peuvent également être cour de justice, théâtre et opéra. V.G. retrace les différentes phases historiques au fil desquelles la Chine est devenue « un pays couvert de temples » : des rites anciens à l’égard des ancêtres à l’arrivée du bouddhisme, de la prolifération des cultes populaires à la mise en place de la doctrine de coexistence entre les trois religions sous la dynastie Tang, de la volonté d’emprise confucianiste sur la scène religieuse au fractionnement local de l’organisation communautaire sous les Ming et les Qing et à la prolifération des sectes populaires. Cette vie religieuse conflictuelle et incontrôlable renforça l’hostilité des élites politiques à l’égard des temples : leur densité maximum (environ 1 million de temples en 1900) est atteinte au moment où s’ouvre l’ère de leur destruction systématique. Il ne reste aujourd’hui en Chine que quelques milliers de temples en activité. À partir de cette mise en perspective historique indispensable, V.G. décline les différents aspects de la vie religieuse, économique et culturelle de ces lieux, faisant apparaître, par-delà la diversité infinie des conditions de fondation et des situations locales, la fonction majeure des temples et de leurs clergés dans la définition et de la redéfinition permanente des rapports entre le public et le privé, dans la production du lien social local, dans la légitimation des pouvoirs et dans l’élaboration des normes de la vie collective. Ce livre, à la fois savant et accessible, sera utile à beaucoup.


\subsection{David A. Palmer}

'EST dans les temples qu'on peut comprendre la religion chinoise telle qu'elle est vécue dans le quotidien
- une religion qui semble bien loin des versets mys  tiques d'un Laozi ou des
doctrines métaphysiques d'un Sakyamuni. En Occident (et dans la Chine continentale d'aujour  d'hui), on définit en général la religion chinoise comme l'enseignement des trois grandes traditions que sont le confucianisme, le tao1sme et le boud  dhisme, considérées comme des entités indépen  dantes et intemporelles. Au cours des dernieres décennies, la recherche universitaire a dépassé cette vision et explore une « quatrieme » tradition, celle de la religion « populaire », telle qu'elle est pratiquée par les Chinois ordinaires. Grâce à de nombreuses études de cultes spécifiques de diverses époques et régions, notre connaissance de la richesse et de la diversité de la religion popu  laire chinoise à travers les âges s'est considérable  ment approfondie.
Avec Dans les temples de la Chine, Vincent Goossaert a réussi à présenter une synthese remar  quable d'une grande partie de ces données. Il le fait non pas en comparant les différents cultes, sectes ou traditions, mais en se concentrant sur le lieu central de la vie religieuse et sociale chinoise d'avant 1949: le temple. Tout en nous guidant à tra  vers les bâtiments, les pratiques religieuses, l'organisation et l'histoire des temples chinois, il nous fait découvrir le temple comme point focal vers lequel les divers éléments de la religion chinoise, à la fois différents et contradictoires, convergent et s'interpénetrent pour finalement imprégner la vie sociale de leur rayonnement.
Ce livre est basé sur un postulat controversé : il
n'y a qu'une seule religion chinoise. Selon
Goossaert, il faut partir d'une vision unitaire de la religion chinoise : « Sans chercher à distinguer des formes populaires des formes élitistes, ni des
« grandes » et des« petites» traditions, il s'agit de mettre en évidence l'existence concrete d'un fon  dement commun : le temple, dans ses diverses formes, considéré comme le lieu par excellence d'une pratique religieuse et de la coexistence des diverses traditions établies. Cette approche englo  be tous les temples, qu'ils soient bouddhiques, confucianistes, tao1stes ou consacrés aux cultes populaires » (p. 16). Bien que cette approche risque de mener à des généralisations excessives, ce qui émerge de l'étude de Goossaert, c'est une image du temple comme espace à l'intérieur duquel se déploie toute la diversité de la vie reli  gieuse chinoise.
Le livre commence par une visite guidée d'un temple typique. Vauteur explique le plan, la dispo  sition des cours et des bâtiments selan des formes symboliques et des príncipes de géomancie, la signification des icônes et du mobilier rituel, le rôle central du bn1le-encens, véritable cceur de tout temple chinois, et l'importance des steles, qui constituent la mémoire de la communauté reli  gieuse.
Dans le deuxieme chapitre, la discussion porte sur les différents types de temples et la difficulté de leur classification. Tout d'abord, l'auteur pré  sente la terminologie des temples chinois, expli  quant les différentes significations et étymologies des nombreux termes chinois qui sont communé  ment traduits par le vocable de« temple » (miao, si, guan, an, gong, ci, etc.), mais il conclut qu'il est difficile de classer les temples d'apres leur nom. On peut alors être tenté de distinguer les temples selan leur obédience religieuse (bouddhique, tao1s  te ou confucianiste ), mais la plupart des temples défient une telle catégorisation : le même temple peut par exemple combiner le culte de divinités protectrices bouddhiques, une liturgie tao1ste et des offrandes carnées, permises dans les rites confucéens, mais interdites dans le bouddhisme et
le tao'isme. Les grands monasteres, habités exclu  sivement par un clergé bouddhique ou tao'iste, font exception parmi une masse de temples qui ne peuvent être classés selon leur « obédience », comme on pourrait le faire, par exemple, pour identifier des églises protestantes. Un même temple offre habituellement une douzaine de cultes à des divinités différentes, parfois même jusqu'à une centaine. La déité la plus sollicitée par les fideles est rarement celle qui a le rang hiérar  chique le plus élevé dans le temple : il serait donc erroné de classer les temples en fonction de leur divinité principale. Goossaert souligne ensuite l'omniprésence des différents autels et lieux de culte qui se trouvaient au centre de la plupart des institutions sociales de la Chine précommuniste : académies privées, écoles confucéennes, guildes, association de compatriotes .
Le troisieme chapitre relate brievement l'histoire du temple en tant qu'institution religieuse en Chine. Les premiers « temples » étaient des mau  solées et des autels destinés au culte des ancêtres, qui se pratiquait souvent en plein air. Puis il y eut les sanctuaires du culte impérial de la dynastie Han. Mais c'est le bouddhisme qui popularisa la notion du temple abritant des icônes de divinités, lieu ouvert à tous et consacré au culte religieux.
Le monachisme bouddhique eut un immense impact social et politique à l'époque du Moyen  Age chinois (111"-VI· siecles). La construction de monasteres opulents transforma le paysage rural et urbain. Le temple fut alors adopté aussi bien par le tao'isme institué que par les cultes populaires. Ces derniers étaient souvent dédiés à des divinités de la nature ou des héros locaux, et servirent de centres de résistance locale aux fonctionnaires du gouvemement central ainsi qu'au clergé boud  dhique et tao'iste. Sous la dynastie des Tang (VIl"  IXº siecles), l'Etat impérial inaugura une politique de contrôle de toutes les institutions religieuses. 11 établit un « concordat » qui garantissait l'unité et l'égalité des trois traditions établies, placées sous sa protection. Les empereurs Tang ont aussi com  mencé la pratique des canonisations des dieux populaires, en leur assignant une place dans la hiérarchie céleste. Cette pratique favorisa la coop  tation de ces cultes, qui devaient demander une
 
autorisation officielle pour la construction de temples. C'est à cette époque que les temples devinrent l'institution principale de la vie commu nautaire en Chine. Sous les Song, cependant, l'har  monie entre les « trois religions » fut détruite sous la pression des ambitions hégémoniques du confu  cianisme. Les cultes locaux réagirent en se libé  rant progressivement de la tutelle de l'Etat ; signe de leur plus grande indépendance, ils constitue  rent de vastes réseaux transrégionaux de temples. A l'époque des Ming et des Qing (XIVe-XIX siecles), l'écart entre la religion d'élite et la reli  gion populaire se creusa. Vers la fin du XIX siecle, alors que le tissu social se fragilisait, les temples et les cultes se multiplierent, au point ou ils consti  tuerent souvent l'institution principale de l'organi  sation et de la défense villageoise. Au même moment, des mouvements sectaires, tels les Taiping, détruisaient tous les temples des régions qui étaient sous leur contrôle. Et les convertis au christianisme, en refusant de contribuer au finan  cement des temples, contribuerent à briser l'unité de communautés qui avaient traditionnellement considéré la construction et l'entretien des temples comme une responsabilité collective.
Les réformes de Kang Youwei, promulguées en 1898, ont inauguré un changement radical de poli  tique à l'égard des temples, qu'on voulut convertir en éléments d'infrastructure d'un Etat modeme. Cette politique fut systématiquement mise en oeuvre durant tout le XX: siecle : les temples furent ainsi transformés en écoles, en bureaux de police et des impôts, etc. Des milliers de temples furent tout simplement détruits: « Leur rôle d'ar  ticulation dans un systeme traditionnel, fragmenté en petites unités et en particularismes, était aux yeux [des réformateurs] impardonnable » (p. 99). Seuls les grands monasteres bouddhiques, isolés géographiquement et relativement à l'écart du sys  teme social traditionnel, furent épargnés. La Révolution culturelle n'a fait que continuer une histoire de destruction qui a traversé tout le ving  tieme siecle. Attjourd'hui, ce sont les urbanistes et les promoteurs immobiliers qui démolissent des temples pour construire des immeubles modemes. On estime qu'cn 1900, il y avait environ un million de temples en Chine : un temple pour
cent familles. De ceux-ci, il ne reste maintenant que quelques milliers. « De 1898 à aujourd'hui s'est écoulé un siecle de destruction continue, par tous les moyens, et qui restera sans doute dans l'histoi  re de l'humanité comme l'un des plus grands anéantissements du patrimoine » (p. 101). Malgré cela, les temples continuent à prospérer à Taiwan et dans les communautés chinoises d'outre-mer. On assiste aussi à une résurgence de la construc  tion de temples en Chine populaire, financés par le gouvernement, par les Chinois de la diaspora et par les fideles locaux.
Dans le quatrieme chapitre, Goossaert propose quatre modes d'appréhension de l'espace sacré. Tout d'abord, le temple peut être considéré comme un mémorial voué aux ancêtres : un lieu ou l'on honore les morts comme s'ils étaient présents, sans toutefois chercher abusivement leur intercession. Selon ce mode, l'appartenance à une communauté religieuse implique la filiation à une lignée présidée par une divinité. En deuxieme lieu, le temple peut être vu comme une cour de justice. Dans ce cas, le dieu n'est pas un ancêtre mais un fonctionnaire céleste, investi en tant que tel d'une autorité judi  ciaire. En tant que juges, les dieux peuvent convo  quer des témoins de l'au-delà. Ils peuvent aussi être eux-même témoins : d'importants contrats étaient souvent scellés devant les dieux, qui punissaient ceux qui ne se tenaient pas à leurs engagements. En troisieme lieu, le temple peut être considéré comme une maison, un lieu de loisirs et de récréation, un lieu de vie. Il n'y a pas de distinction radicale entre l'architecture d'un temple et celle d'une maison : les temples se différencient par leur hauteur, leur taille, leur ornementation. Les temples sont les résidences impériales des dieux, qui offrent nourriture et loge  ment aux passants. Le quatrieme mode est celui du temple comme montagne : la métaphore de la mon  tagne est souvent utilisée pour décrire le temple en partie Oa pagode, le toit) ou dans son ensemble. I:ascension des montagnes pour arriver aux monas  teres perchés au sommet est un acte de dévotion, rapprochant le pelerin des hauteurs étranges et sau  vages de la transcendance spirituelle.
Le cinquieme chapitre s'attache à la fondation
des temples, le plus souvent le résultat d'une ini  tiative individuelle, et aux modalités du financement de leur construction et de leur entretien.
Dans le sixieme chapitre, nous découvrons les acteurs de la vie du temple : le clergé, les devins et les mediums. La plupart des temples sont admi  nistrés par des comités laies qui emploient et supervisent les officiants qui y résident. Ce sont ces comités de dévots qui, à travers leurs oeuvres charitables et sociales, constituaient la trame même de la vie culturelle et religieuse chinoises. Enfin, le septieme chapitre nous présente la vie religieuse proprement dite des temples : le culte quotidien, les offrandes d'encens, de papier-mon  naie et de sacrifices ; les festivals, rituels et pro  cessions ; enfin la musique et les banquets qui colorent la vie du temple.
Dans sa conclusion, Goossaert revient au theme du premier chapitre : la relation entre les temples, l'Etat et la société. « Le temple chinois est une institution politique: l'Etat s'en sert pour gouver  ner, et le peuple y fonde son organisation » (p. 33). Les temples sont des lieux d'articulation de la culture institutionnelle et populaire. À l'inté  rieur, la liturgie de l'élite d'Etat ou monastique apporte légitimité politique et cosmologique, alors qu'à l'e:xtérieur, les fêtes communautaires et les associations de temple apportent le soutien et le financement du peuple, sans lequel l'ensemble de la liturgie officielle ne pourrait survivre. Le temple, donc, est un « lieu de négociation reli  gieuse ». En lui se retrouvent tous les éléments de la religion chinoise. Bien que la coexistence des tendances élitistes et populaires ne soit pas tou  jours facile, tous les acteurs comprennent que le compromis est essentiel. « Le mélange des élé  ments, en des proportions toujours variées, rend compte de l'unicité d'une religion chinoise tres étendue, mais dont aucune partie ne veut se sépa  rer radicalement des autres » (p. 204). Le temple est un espace privilégié ou se forme et s'exprime le contenu religieux, et qui attire vers lui toutes les connaissances et les richesses: les dieux par  lent à travers les médiums et les orades, les arti  sans et les jardiniers façonnent la beauté des lieux, les prêtres célebrent les rites, les troupes d'opéra jouent des histoires saintes, les steles et les peintures racontent les faits des dieux et des adeptes, les maitres des arts du corps enseignent
les secrets du combat ou de la longévité, et les les temples, ont investi les pares et les espaces philanthropes font leurs bonnes oeuvres. Le publics dans les années 80 et 90. La religion hors temple n'est donc pas un immeuble figé, mais un des temples n'étant pas le propos de ce livre, lieu ouvert, à l'intérieur duquel les formes Goossaert ne s'attarde pas sur cette tendance. bouillonnantes de la religion chinoise se inêlent Mais sa synthese magistrale nous ayant fait décou  et se développent.	vrir les trésors culturels des temples, leur margi  Au XX siecle, cependant, la vie des temples a en nalisation actuelle ne peut que susciter des inter  grande partie disparu. Les temples qui n'ont pas rogations : est-ce un phénomene temporaire, pro  été détruits sont souvent devenus des lieux touris  duit artificiel de la politique de l'Etat, ou bien tiques, des musées d'une culture qui n'existe plus. s'agit-il d'un changement profond des formes de Dans les villes, la vie religieuse est, le plus sou   religiosité en Chine ? Et si le temple n'est plus le vent, sortie des temples. Les groupes de qigong, centre de la vie religieuse et sociale en Chine, par exemple, se voyant interdire la pratique dans  qu'est-ce qui prendra sa place? G


\section{B. vermander critique du livre}

Kenneth Ch’en, Histoire du bouddhisme en Chine
Traduit de l’anglais par Dominique Kych. Postface de Sylvie Hureau.Paris, Les Belles Lettres, 2015, 592 p., index. Bibliographie.
Benoît Vermander
p. 292
https://doi-org.icp.idm.oclc.org/10.4000/assr.28207
Référence(s) :
Kenneth Ch’en, Histoire du bouddhisme en Chine, Traduit de l’anglais par Dominique Kych, Postface de Sylvie Hureau, Paris, Les Belles Lettres, 2015, 592 p., index, Bibliographie

1 Ce livre est la traduction d’un ouvrage de référence sur le bouddhisme chinois paru en 1964. Le travail de traduction réalisé est excellent, très soigné, avec retour aux textes chinois cités, et un utile lexique des noms et titres chinois. L’ouvrage lui-même est de facture très classique, présentant une histoire ordonnée par dynasties avec chapitres complémentaires sur la doctrine, la traduction du Canon bouddhiste, les temples, ou les moines éminents. L’auteur est particulièrement à son aise dans les résumés doctrinaux et la présentation des textes canoniques, mais on trouvera là d’excellentes synthèses sur bien d’autres sujets, par exemple sur l’organisation et l’économie des grands monastères.

2P rès de 380 pages sont consacrés à la naissance, l’arrivée en Chine, la croissance et l’apogée du bouddhisme, avant que la période du « déclin » commence avec les Song – une ligne narrative poursuivie presque inexorablement. Les dynasties Ming et Qing sont traitées très rapidement, et la description donnée de l’aggiornamento bouddhiste – de la période qui suit l’écrasement de la rébellion Taiping jusqu’à l’établissement de la République populaire de Chine – est clairement insuffisante. La date de publication de la version originale de l’ouvrage explique le ton très sombre des dernières pages (même si l’auteur semble parier en finale sur la résilience et la capacité d’adaptation du bouddhisme chinois, pari que les évolutions intervenues après 1980 ont justifié au-delà de ses espérances).

3 La publication en français d’une pareille synthèse est en soi une excellente nouvelle, même si l’on se demande un peu quel est le public visé. Pédagogique et de lecture aisée, l’ouvrage reste d’écriture assez conventionnelle, parfois monotone, si bien qu’il ne s’agit sans doute pas d’une synthèse idéale pour le grand public cultivé, auquel on recommanderait de préférence des ouvrages plus courts et plus vivants portant sur tel ou tel aspect du sujet couvert. En revanche, étudiants en histoire chinoise ou en sciences religieuses trouveront là un ouvrage de consultation fiable et commode, qui pourra orienter leurs lectures ultérieures. Le rôle de consultation et d’orientation joué par l’ouvrage est encore renforcé par l’excellente bibliographie critique rédigée par l’auteur, comme par la bibliographie additionnelle établie par Sylvie Hureau. Cette dernière annexe est extrêmement complète, très à jour, et ajoute grandement à l’intérêt de la publication.

4 Dans l’idéal, la bibliographie additionnelle aurait pu être accompagnée d’une synthèse relatant les principales évolutions intervenues dans l’approche de l’histoire du bouddhisme chinois depuis la publication de l’ouvrage. Elles sont nombreuses, qu’il s’agisse de l’écriture de l’histoire de l’école Chan, de la relation entre bouddhisme et religion populaire, des origines et de l’affirmation du « bouddhisme humaniste », ou des codes rhétoriques gouvernant les écrits hagiographiques. Le livre de Kenneth Ch’en a marqué un moment de la connaissance, et une évaluation historiographique d’ensemble aurait donné une valeur accrue et un horizon critique aux additions bibliographiques. Par ailleurs, puisque ce sont plus de cinquante ans de l’histoire du bouddhisme chinois qui ne sont pas couverts par l’ouvrage (davantage en fait, toute la partie consacrée à l’après 1949 n’étant guère utilisable), une courte synthèse sur cette période, et notamment sur le renouveau en cours, aurait également été bienvenue. Il est vrai que, tel quel, l’ouvrage est déjà fort volumineux et qu’il constitue une ressource des plus appréciables malgré son caractère quelque peu daté.

\section{Plan}

Troisième partie
MATURITÉ ET ACCEPTATION
\begin{itemize}
    \item \textbf{Chapitre VIII. Apogée du bouddhisme. La dynastie Tang.}
    \begin{itemize}
        \item Le mémoire de Fu Yi contre le bouddhisme.
    \end{itemize}
    \begin{itemize}
        \item Taizong et le bouddhisme.
    \end{itemize}
    \begin{itemize}
        \item Gaozong et le bouddhisme
    \end{itemize}
    \begin{itemize}
        \item Le soutien de l'impératrice Wu Zhao
    \end{itemize}
    \begin{itemize}
        \item Xuanzong .
    \end{itemize}
    \begin{itemize}
        \item Le mémoire de Han Yu contre le bouddhisme
    \end{itemize}
    \begin{itemize}
        \item La proscription de 845 ou la persécution de l'ère huichang.
    \end{itemize}
    \begin{itemize}
        \item Les pèlerins chinois à l'étranger.
    \end{itemize}
    \begin{itemize}
        \item Xuanzhao
    \end{itemize}
    \begin{itemize}
        \item Xuanzang. .
    \end{itemize}
    \begin{itemize}
        \item Yijing et la route maritime
    \end{itemize}
    \begin{itemize}
        \item L'apport des pèlerins à la culture mondiale
    \end{itemize}
    \item Chapitre IX. La communauté monastique
    \begin{itemize}
        \item Les différentes catégories de moines
    \end{itemize}
    \begin{itemize}
        \item L'ordination privée ou officielle des moines
    \end{itemize}
    \begin{itemize}
        \item Le registre des moines .
    \end{itemize}
    \begin{itemize}
        \item La composition de la communauté monastique en Chine
    \end{itemize}
    \begin{itemize}
        \item L'examen d'ordination
    \end{itemize}
    \begin{itemize}
        \item Les ordinations par faveur impériale et par achat de certificat
    \end{itemize}
    \begin{itemize}
        \item L'origine sociale des moines
    \end{itemize}
    \begin{itemize}
        \item L'entretien des moines : nourriture et habillement
    \end{itemize}
    \begin{itemize}
        \item La propriété des moines
    \end{itemize}
    \begin{itemize}
        \item Les fonctionnaires du samgha et l'administration des monastères.
    \end{itemize}
    \item \textbf{Chapitre X. Les temples bouddhiques et le bouddhisme populaire.}
    \begin{itemize}
        \item Le nombre et le coût des monastères
    \end{itemize}
    \begin{itemize}
        \item Les activités commerciales
    \end{itemize}
    \begin{itemize}
        \item Les moulins hydrauliques à céréales
    \end{itemize}
    \begin{itemize}
        \item Les pressoirs à huile .
    \end{itemize}
    \begin{itemize}
        \item Les hôtelleries
    \end{itemize}
    \begin{itemize}
        \item Les Trésors inépuisables
    \end{itemize}
    \begin{itemize}
        \item Les terres de monastères.
    \end{itemize}
    \begin{itemize}
        \item Les paysans des terres de monastères.
    \end{itemize}
    \begin{itemize}
        \item L'exemption de taxes accordée aux monastères
    \end{itemize}
    \begin{itemize}
        \item Les cours de mérite
    \end{itemize}
    \begin{itemize}
        \item Les différentes catégories de temples.
    \end{itemize}
    \begin{itemize}
        \item L'administration intérieure des temples
    \end{itemize}
    \begin{itemize}
        \item Les fêtes célébrées dans les monastères
    \end{itemize}
    \begin{itemize}
        \item La fête des lanternes
    \end{itemize}
    \begin{itemize}
        \item La célébration de l'anniversaire de la naissance du Bouddha.
    \end{itemize}
    \begin{itemize}
        \item La fête en l'honneur des reliques du Bouddha
    \end{itemize}
    \begin{itemize}
        \item La fête des morts.
    \end{itemize}
    \begin{itemize}
        \item Les assemblées de jeûne
    \end{itemize}
    \begin{itemize}
        \item L'éducation religieuse.
    \end{itemize}
    \begin{itemize}
        \item Les bianwen ou récits bouddhiques merveilleux
    \end{itemize}
    \begin{itemize}
        \item Les sociétés religieuses.
    \end{itemize}
    \begin{itemize}
        \item Les activités charitables
    \end{itemize}
    \item Chapitre XI. Les écoles bouddhiques en Chine.
    \begin{itemize}
        \item La secte des Trois degrés
    \end{itemize}
    \begin{itemize}
        \item L'école du Vinaya.
    \end{itemize}
    \begin{itemize}
        \item L'école Kosa
    \end{itemize}
    \begin{itemize}
        \item L'école Tiantai
    \end{itemize}
    \begin{itemize}
        \item La classification des sutras et des enseignements
    \end{itemize}
    \begin{itemize}
        \item La triple vérité
    \end{itemize}
    \begin{itemize}
        \item L 'esprit absolu.
    \end{itemize}
    \begin{itemize}
        \item Concentration et vue pénétrante
    \end{itemize}
    \begin{itemize}
        \item L'école Huayan.
    \end{itemize}
    \begin{itemize}
        \item Les maîtres de l'école Huayan
    \end{itemize}
    \begin{itemize}
        \item La doctrine de l'école Huayan
    \end{itemize}
    \begin{itemize}
        \item L'école Faxiang.
    \end{itemize}
    \begin{itemize}
        \item La doctrine de l'école Faxiang .
    \end{itemize}
    \begin{itemize}
        \item L'école tantrique .
    \end{itemize}
    \item Chapitre XII. Les écoles bouddhiques en Chine (suite)
    \begin{itemize}
        \item L'école de la Terre pure
    \end{itemize}
    \begin{itemize}
        \item Contenu du Satra de la Terre pure
    \end{itemize}
    \begin{itemize}
        \item Avalokitesvara.
    \end{itemize}
    \begin{itemize}
        \item L'école et les maîtres de la Terre pure en Chine
    \end{itemize}
    \begin{itemize}
        \item L'école du Chan en Chine.
    \end{itemize}
    \begin{itemize}
        \item Bodhidharma.
    \end{itemize}
    \begin{itemize}
        \item Shenxiu, le sixième patriarche
    \end{itemize}
    \begin{itemize}
        \item Shenhui remet en cause la légitimité de Shenxiu
    \end{itemize}
    \begin{itemize}
        \item Huineng.
    \end{itemize}
    \begin{itemize}
        \item Huineng et le nouveau Chan Qu'est-ce que le Chan?
    \end{itemize}
    \begin{itemize}
        \item Les facteurs intellectuels qui favorisèrent l'essor du Chan
    \end{itemize}
    \begin{itemize}
        \item Chan et taoisme.
    \end{itemize}
    \begin{itemize}
        \item Le Chan est-il bouddhiste ?.
    \end{itemize}
    \begin{itemize}
        \item La survie du Chan après la proscription de 845
    \end{itemize}
    \item Chapitre XIII. Le Tripitaka chinois
    \begin{itemize}
        \item Les techniques de traduction.
    \end{itemize}
    \begin{itemize}
        \item Problèmes de traduction
    \end{itemize}
    \begin{itemize}
        \item Les catalogues de sutras
    \end{itemize}
    \begin{itemize}
        \item Les éditions du Tripitaka chinois
    \end{itemize}
    \begin{itemize}
        \item Les éditions modernes
    \end{itemize}
    \begin{itemize}
        \item Le Sũtra du lotus.
    \end{itemize}
\end{itemize}


\section{Introduction}

Le livre \textit{Histoire du Bouddhisme en Chine}\cite{chen\_histoire\_2015} est la traduction d’un ouvrage de référence sur le bouddhisme chinois paru en 1964. Il couvre  la naissance, l’arrivée en Chine, la croissance et l’apogée du bouddhisme, avant que la période du « déclin » commence avec les Song. 
La date de publication de la version originale de l’ouvrage explique le ton très sombre des dernières pages \cite{vermander\_kenneth\_2016}.
Nous proposons d'étudier l'arrivée et la croissance en Chine (chapitre 8 à 13).


\subsection{le cout de la religion - l'exemple romain}

Le dessein des sanctuaires et des cites parait assez clair, il s'agit
avant tout d'assurer Ia perennite des ietes et des sacrifices; de rendre
le sanctuaire le plus ccriche·· possible, c'est a dire dele doter de
monuments aussi somptueux que le permettent - mal - les finances
publiques et sacrees. Nous ne relevons Ia aucun souci de chrematistique:
certes les sanctuaires participent de fa~on active a Ia vie
economique, ils en sont meme l'un des moteurs par les besoins specifiques
qui sont les leurs ( materiaux de construction, denrees de
luxe); mais ils n'ontjamais cherche a Ia developper en soi; l'activite
bancaire repond a des considerations sociales plus qu' economiques
et il s'agit avant tout de tenter d'assurer des revenus reguliers au
culte273 ; l'activite monetaire est un phenomene local et trop particulier
pour qu'on puisse lui preter une signification generale: il
convient dans Ia plupart des cas de }'interpreter dans le cadre de Ia
panegyrie. En bref, ce n'est certainement pas par un mouvement
conscient et raisonne que les sanctuaires ont developpe et diversifie
leurs activites cceconomiques)) et financieres, cependant en fonction
meme d'un certain volume de richesse, les sanctuaires au moins les
plus importants, ont pese d'un poids indeniable sur les rouages de
l'economie grecque. II ne faut pas d'autre part envisager l'accroissement
de Ia richesse des sanctuaires comme un phenomene
lineaire. Cette accumulation de metal precieux a suscite bien des
convoitises et cela d'autant plus que le sentiment du sacre s'etait
bien transforme a partir de Ia fin de l'epoque classique.
Aspects sociaux et économiques de la vie religieuse dans l'Anatolie gréco-romaine

Series: 
Études préliminaires aux religions orientales dans l'Empire romain, Volume: 88
Author: Pierre Debord
Copyright Year: 1982
