\chapter{devoir}

\cite{chen_histoire_2015}

\cite{chen_economic_1956}

\section{validation}
Mode de validation du cours   Compte rendu d’un livre ou de quelques chapitres d’un livre de votre choix parmi la liste suivante :  
\begin{itemize}
 
\item  Kenneth CH’EN, Histoire du Bouddhisme en Chine, traduit de l’anglais par Dominique Kych, Paris : Les Belles Lettres, 2015. Chapitres 8 – 13 inclus.   

\end{itemize}

Ce travail devra comporter 5 ou 6 pages en police Times New Roman 12, interligne 1,5.   Date de la remise du travail : au plus tard le 9 décembre 2023. 



 

\section{B. vermander critique du livre}
\cite{vermander_kenneth_2016}
 

1 Ce livre est la traduction d’un ouvrage de référence sur le bouddhisme chinois paru en 1964. Le travail de traduction réalisé est excellent, très soigné, avec retour aux textes chinois cités, et un utile lexique des noms et titres chinois. L’ouvrage lui-même est de facture très classique, présentant une histoire ordonnée par dynasties avec chapitres complémentaires sur la doctrine, la traduction du Canon bouddhiste, les temples, ou les moines éminents. L’auteur est particulièrement à son aise dans les résumés doctrinaux et la présentation des textes canoniques, mais on trouvera là d’excellentes synthèses sur bien d’autres sujets, par exemple sur l’organisation et l’économie des grands monastères.

2P rès de 380 pages sont consacrés à la naissance, l’arrivée en Chine, la croissance et l’apogée du bouddhisme, avant que la période du « déclin » commence avec les Song – une ligne narrative poursuivie presque inexorablement. Les dynasties Ming et Qing sont traitées très rapidement, et la description donnée de l’aggiornamento bouddhiste – de la période qui suit l’écrasement de la rébellion Taiping jusqu’à l’établissement de la République populaire de Chine – est clairement insuffisante. La date de publication de la version originale de l’ouvrage explique le ton très sombre des dernières pages (même si l’auteur semble parier en finale sur la résilience et la capacité d’adaptation du bouddhisme chinois, pari que les évolutions intervenues après 1980 ont justifié au-delà de ses espérances).

3 La publication en français d’une pareille synthèse est en soi une excellente nouvelle, même si l’on se demande un peu quel est le public visé. Pédagogique et de lecture aisée, l’ouvrage reste d’écriture assez conventionnelle, parfois monotone, si bien qu’il ne s’agit sans doute pas d’une synthèse idéale pour le grand public cultivé, auquel on recommanderait de préférence des ouvrages plus courts et plus vivants portant sur tel ou tel aspect du sujet couvert. En revanche, étudiants en histoire chinoise ou en sciences religieuses trouveront là un ouvrage de consultation fiable et commode, qui pourra orienter leurs lectures ultérieures. Le rôle de consultation et d’orientation joué par l’ouvrage est encore renforcé par l’excellente bibliographie critique rédigée par l’auteur, comme par la bibliographie additionnelle établie par Sylvie Hureau. Cette dernière annexe est extrêmement complète, très à jour, et ajoute grandement à l’intérêt de la publication.

4 Dans l’idéal, la bibliographie additionnelle aurait pu être accompagnée d’une synthèse relatant les principales évolutions intervenues dans l’approche de l’histoire du bouddhisme chinois depuis la publication de l’ouvrage. Elles sont nombreuses, qu’il s’agisse de l’écriture de l’histoire de l’école Chan, de la relation entre bouddhisme et religion populaire, des origines et de l’affirmation du « bouddhisme humaniste », ou des codes rhétoriques gouvernant les écrits hagiographiques. Le livre de Kenneth Ch’en a marqué un moment de la connaissance, et une évaluation historiographique d’ensemble aurait donné une valeur accrue et un horizon critique aux additions bibliographiques. Par ailleurs, puisque ce sont plus de cinquante ans de l’histoire du bouddhisme chinois qui ne sont pas couverts par l’ouvrage (davantage en fait, toute la partie consacrée à l’après 1949 n’étant guère utilisable), une courte synthèse sur cette période, et notamment sur le renouveau en cours, aurait également été bienvenue. Il est vrai que, tel quel, l’ouvrage est déjà fort volumineux et qu’il constitue une ressource des plus appréciables malgré son caractère quelque peu daté.

\section{Plan - Troisième partie
MATURITÉ ET ACCEPTATION}


\textbf{Chapitre VIII. Apogée du bouddhisme. La dynastie Tang.}

    \begin{itemize}
        \item Le mémoire de Fu Yi contre le bouddhisme.
        \item Taizong et le bouddhisme.
        \item Gaozong et le bouddhisme
        \item Le soutien de l'impératrice Wu Zhao :\textit{Soutien état, floraison exceptionnelle 691 : décret  priorité du bouddhisme sur le taoisme}
        \item Xuanzong . \textit{égalité taoisme et bouddhisme  officiel confucianisme mais en privé bouddhisme pour trouver soutien et consolation.}
        \item Le mémoire de Han Yu contre le bouddhisme
        \item La proscription de 845 ou la persécution de l'ère huichang.
        \item Les pèlerins chinois à l'étranger.
        \item Xuanzhao
        \item Xuanzang. .
        \item Yijing et la route maritime
        \item L'apport des pèlerins à la culture mondiale
    \end{itemize}

Chapitre IX. La communauté monastique

    \begin{itemize}
        \item Les différentes catégories de moines. \textit{p. 247. Normalement décision personnelle. mais sous les Tang, contrôle de l'Etat}. \textit{trois catégories : moines officiels, dans les temples d'état, moins privés, moines du peuple : 47 monastères d'état, 839 aristocratiques, 30 000 du peuple - Wei du Nord}
        \item L'ordination privée ou officielle des moines \textit{Sous la première moitié des Tang, ordination privée (sidu). Mais scandale de personnes devenant moine pour éviter les corvée. Institution en 747 d'un système d'ordination étatique. Puis vente de certificat d'ordination lors de la crise en 775. }
        \item Le registre des moines . \textit{enregistrement minutieux}
        \item La composition de la communauté monastique en Chine. \textit{période de préparation préliminaire au noviciat. accord parentale, ne pas fuir ses engagements temporels,  Etude des textes sacrés, service des hôtes, corvée, en tant que postulant, garde les cheveux. }
        \item L'examen d'ordination \textit{ pour devenir novice, réciter des feuilles du Sutra comme le sutra du lotus, explication de texte. On devient novice (tonsure) et la plupart ne deviennent pas moine}
        \item Les ordinations par faveur impériale et par achat de certificat
        \item L'origine sociale des moines. \textit{Les moines officiels famille aisée. }
        \item L'entretien des moines : nourriture et habillement. \textit{critique des confucianistes : dépendent des fidèles (ordre mendiants) Cout des 200 000 moines en 778, 6 millions dde ligatures, soit l'équivalent de 50\% du budget de l'état}
        \item La propriété des moines. \textit{Depuis les Tang, dot de 30 mu viager (Koufen), pour soutenir le Taoisme (et de façon incidente, le bouddisme)}
        \item Les fonctionnaires du \textit{samgha} et l'administration des monastères.\textit{Faguo et le bureau de supervision ders mérites. En 841-847, sous Wuzong, placé comme rite étranger (bureau des hôtes) }
    \end{itemize}

\textbf{Chapitre X. Les temples bouddhiques et le bouddhisme populaire.}

    \begin{itemize}
        \item Le nombre et le coût des monastères. \textit{les temples de nos jours semblent concus pour surpasser les palais impériaux (Ximingsi, Ximingsi. (西明寺). In Chinese, “Luminosity of the West Monastery,” located in the Tang capital of Chang’an (present-day Xi’an). 656 . Cout estimé d'un pavillon : 22500 ligatures Coup estimé très élevé par de nombreux fonctionnaires. Problème des monnaies de cuivre : entraine l'interdiction d'objet en cuivre d'où la confiscation en 845 de tout le cuivres des monastères.   }
        \item Les activités commerciales : \textit{moulins à grain et pressoirs, don des fidèles. Rôle d'hotellerie}
        \item Les moulins hydrauliques à céréales
        \item Les pressoirs à huile .
        \item Les hôtelleries
        \item Les Trésors inépuisables \textit{wujin zang : en cas d'excédents de dons ou de domaines agricoles. vente pour construction ou réfection de temples. taux d'intéret pouvait s'élever à 50\% ! p. 270 }
        \item Les terres de monastères.
        \item Les paysans des terres de monastères.
        \item L'exemption de taxes accordée aux monastères
        \item Les cours de mérite : \textit{pour obtenir l'exemption, les famille riche établissaient une cour de mérite. Une partie du domaine, un temple mais exemption demandée sur la totalité du domaine.  On se dispensait même d'édifier le temple. (sous les Tang puis les Song}
        \item Les différentes catégories de temples.
        \item L'administration intérieure des temples
        \item Les fêtes célébrées dans les monastères
        \item La fête des lanternes
        \item La célébration de l'anniversaire de la naissance du Bouddha.
        \item La fête en l'honneur des reliques du Bouddha
        \item La fête des morts.
        \item Les assemblées de jeûne
        \item L'éducation religieuse.
        \item Les bianwen ou récits bouddhiques merveilleux
        \item Les sociétés religieuses.
        \item Les activités charitables
    \end{itemize}

Chapitre XI. Les écoles bouddhiques en Chine.
\begin{itemize}
    \item La secte des Trois degrés
    \item L'école du Vinaya.
    \item L'école Kosa
    \item L'école Tiantai
    \item La classification des sutras et des enseignements
    \item La triple vérité
    \item L 'esprit absolu.
    \item Concentration et vue pénétrante
    \item L'école Huayan.
    \item Les maîtres de l'école Huayan
    \item La doctrine de l'école Huayan
    \item L'école Faxiang.
    \item La doctrine de l'école Faxiang .
    \item L'école tantrique .
\end{itemize}

Chapitre XII. Les écoles bouddhiques en Chine (suite)

    \begin{itemize}
        \item L'école de la Terre pure
        \item Contenu du Satra de la Terre pure
        \item Avalokitesvara.
        \item L'école et les maîtres de la Terre pure en Chine
        \item L'école du Chan en Chine.
        \item Bodhidharma.
        \item Shenxiu, le sixième patriarche
        \item Shenhui remet en cause la légitimité de Shenxiu
        \item Huineng.
        \item Huineng et le nouveau Chan Qu'est-ce que le Chan?
        \item Les facteurs intellectuels qui favorisèrent l'essor du Chan
        \item Chan et taoisme.
        \item Le Chan est-il bouddhiste ?.
        \item La survie du Chan après la proscription de 845
    \end{itemize}

Chapitre XIII. Le Tripitaka chinois

    \begin{itemize}
        \item Les techniques de traduction.
        \item Problèmes de traduction
        \item Les catalogues de sutras
        \item Les éditions du Tripitaka chinois
        \item Les éditions modernes
        \item Le Sũtra du lotus.
    \end{itemize}



\section{Introduction}

Le livre \textit{Histoire du Bouddhisme en Chine} \cite{chen_histoire_2015} est la traduction d’un ouvrage de référence sur le bouddhisme chinois paru en 1964. Il couvre  la naissance, l’arrivée en Chine, la croissance et l’apogée du bouddhisme, avant que la période du « déclin » commence avec les Song. 
La date de publication de la version originale de l’ouvrage explique le ton très sombre des dernières pages \cite{vermander_kenneth_2016}.
Nous proposons d'étudier l'arrivée et la croissance en Chine (chapitre 8 à 13).






Plan
Apogée du bouddhisme. La dynastie TangS'appuyant sur la réunification accomplie par les Sui, la dynastie Tang(618-907) édifia un immense empire, qui couvrait non seulement le teritoire propre de la Chine, mais étendait également en Asie centrale. Bienque le dan impérial, se targuant de descendre de Laozi, fit favorable au taoisme, le gouvernement mena une politique de tolérance religieuse, qui laissa a chaque religion la possibilité de se développer. Le nestorianisme, l'islam et le manichéisme furent introduits en Chine durant cette dynastie et firent tous trois des adeptes parmi les Chinois. Si cet idéal cosmopolite fut adopté par les Tang, c'est qu'ils ne se considéraient pas uniquement comme empereurs des Chinois, mais également comme souverains de peuples dits barbares.Le bouddhisme, qui s'était déjà largement répandu en Chine, se développa comme jamais auparavant sous le patronage de certains empereurs Tang, si bien que sa puissance et son influence finirent par dépasser de loin celles du taoïsme. On pourrait dire que, sous les Tang, le bouddhisme atteignit en Chine l'âge de la maturité. Toutes les couches de la société, la famille impériale, la noblesse, les familles riches et puissantes, et jusqu'au petit peuple, lui accordèrent leur soutien. Tout en profitant de la position et de l'influence des grandes familles du royaume, le bouddhisme entretint des relations privilégiées avec le commun des fidèles par le truchement de diverses activités sociales et religieuses. Les cours spacieuses des temples bouddhiques servirent de jardins d'agrément aux foules des villes, les fêtes bouddhiques procurent distraction et divertissement aux masses villageoises et urbaines, les assemblées de jeûne et les prêches furent suivis par un très grand nombre de fidèles laïcs. Ce fut de cette capacité à répondre aux besoins de toutes lesL'important corpus de textes qui avait été traduit aux siècles précédents était désormais assimilé par les Chinois, prêt à être soumis aux diverses interprétations qui allaient servir de fondement aux différentes écoles.Certaines, comme celle du Tiantai et du Chan, dégagées de l'influence indienne, développèrent des traits propres à l'expression singulière du génie chinois. Dans les temples prospères des villes comme dans les monastères reculés de montagne, des moines à l'esprit audacieux débattirent des sutras sur lesquels étaient fondées leurs écoles. Pour traiter du bouddhisme sous les Tang, nous examinerons d'abord l'attitude du pouvoir impérial envers cette religion, puis nous étudierons le rôle du samgha dans la société des Tang et enfin, nous décrirons les différentes écoles qui virent le jour à cette époque.Il semble en général justifié de dire que le bouddhisme sous les Tang coqui un caractère plus spécifiquement chinois et s'identifia de manière plus orte, et sous son contrôle, à l'État.



\section{la proscription de 845 ou la persécution de l'ère Huichang}
\item Dans l'état actuel de nos connaissances, aucun lien direct ne peut établi entre le mémoire de Han Yu et la proscription décrétée par l'emper Wuzong en 845. Nous ne savons même pas s'il avait lu ce document. Il avait quoiqu'il en soit, bien d'autres raisons de vouloir éliminer le bouddhisme.
\item  Certains historiens semblent faire de la proscription de 845 le fruit d'une  brusque décision, un épisode supplémentaire de la longue bataille idéologique qui opposa le taoïsme au bouddhisme. S'il est vrai que les taoïstes incitèrent l'empereur à prendre des mesures répressives, on ne peut réduire cet événenement à l'expression d'un affrontement idéologique. En revanche, les luttes entre factions qui sévissaient au sein de la cour semblent avoir joué un certain dans la décision impériale, les lettrés fonctionnaires s'alliant à l'empereur contre la religion étrangère et les eunuques la soutenant.

Des considérations économiques telles que le désir de la cour de s'emparer et d'utiliser les immenses richesses détenues par les monastères entrèrent aussi en ligne de compte. Ce dessein apparaît clairement dans la déclaration du bureau du Secrétariat impérial, datée du septième mois de l'année 845 :
\begin{singlequote}
    Les images de bronze [des divinités bouddhiques] et les cloches devront être remises au commissaire du Bureau du sel et du fer, afin d'être fondues comme pièces de monnaie. Les statues de fer seront remises aux fonctionnaires locaux pour être transformées en outils agricoles. Les images en or, argent, jade ou autre matériau précieux devront être remises au bureau du Trésor public. Toutes les images en or, argent, bronze et fer détenues par les riches et les nobles devront être remises au gouvernement avant la fin du mois qui suivra la publication de ce décret [...] Les images d'argile, de bois et de pierre sont autorisées à demeurer dans les temples où elles sont placées'.
\end{singlequote}
The last sentence reveals that the motive in the minds of the
proponents was not complete suppression of the Buddhist religion,
but confiscation for state use of the vast economic wealth held by
the temples and monasteries
La perte de revenu que représentait l'exemption de taxes accordées aux 260 000 moines et nonnes, ainsi qu'aux 100 000 esclaves et aux innombrables laïcs employés au service de la communauté bouddhique (dont le n équivalait à celui des moines et nonnes réunis), ajoutée à celle dont béné une grande partie des terres possédées par les monastères, fut une des ra qui poussèrent l'empereur à prendre des mesures drastiques.Les histoires officielles des Tang donnent peu d'informations si divers événements qui ont conduit à la proscription de 845. Il se tr fort heureusement qu'un moine japonais très observateur, nommé E voyagea en Chine à cette époque et qu'il consigna dans son journal renseignements très précieux sur cet épisode : il y apparaît en partic très clairement que les mesures prises par Wuzong contre le bouddhisme sont étalées sur plusieurs années.
\item Ennin note, dès 841, un premier signe de défaveur impériale : à l'occas de son anniversaire, l'empereur octroya aux taoïstes, en remerciement cadeaux qu'ils avaient apportés, la permission de porter la robe pourp privilège qu'il n'accorda pas aux bouddhistes. Il en fut de même lors l'anniversaire de l'empereur en 842. Dans le courant de la même année, moine bouddhiste appelé Xuanxuan se targua de pouvoir défaire les Ouïgol détestés (un peuple d'Asie centrale) grâce à une épée magique, mais dès qu fut mis à l'épreuve, son imposture fut dévoilée.\textit{K. Ch'en, « The Economic Background of the Hui-ch'ang Suppression of Buddhism», Harvard Journal of Asiatic Studies, 19 (1956), p. 68}.L'empereur prit, au dixième mois, un décret ordonnant que les moines et nonnes qui pratiquaient l'alchimie ou la magie, avaient fui l'armée, portaient des traces de flagellation sur le corps, entretenaient des épouses ou violaient les règles monastiques, fussent réduits à l'état laïc. L'argent et les biens tels que domaines et jardins possédés par ces religieux devaient être de plus restitués au gouvernement. En conséquence de ce décret, 1 232 moines et nonnes résidant à l'est de la capitale, et 2 259 (on cite parfois également le chiffre de 2 219) résidant à l'ouest, retournèrent à l'état laïc au premier mois de 843. Au deuxième mois de cette même année, les mesures répressives touchèrent les temples manichéens. Elles ordonnaient que tous les biens, propriétés, champs, commerces, résidences, monnaies et marchandises en leur possession soient rendus à l'État, et interdisaient la présence de tout élément étranger sur leurs domaines. Cette dernière mesure visait particulièrement les Ouïgours, qui étaient alors un puissant soutien du manichéisme en Chine.Revenons un peu un peu en arrière pour expliquer leur présence sur le territoire chinois à cette époque. Afin de mater la révolte d'An Lushan au milieu du vie siècle, les empereurs Tang avaient fait appel à des mercenaires Perses, Arabes et Ouigours, ces derniers constituant le plus gros des troupes.Lorsque la rébellion fut jugulée, les Perses et les Arabes quittèrent le territoire, mais les Ouïgours restèrent en Chine, se comportant en maîtres à Luoyang et Chang'an, pillant et rançonnant, au grand dam des Chinois. Les autoritésimmiccantes devant ce problème car ces étrangers étaientsur l'Asie centrale. Lamais les Ulgous resteren en cune, se comportant en maitres a Luoyang et Chang'an, pillant et rançonnant, au grand dam des Chinois. Les autorités demeuraient impuissantes devant ce problème, car ces étrangers étaient soutenus par le puissant empire ouïgour qui régnait sur l'Asie centrale. La position avantageuse dont ils jouissaient incita les populations étrangères des deux capitales à confier leur richesse aux temples manichéens qui bénéficiaient de cette protection, un peu comme les marchands étrangers installés dans les ports chinois concédés par les « traités inégaux » du XIx® siècle placèrent plus tard leur argent dans les banques britanniques. Les autorités chinoises profitèrent cependant de l'affaiblissement de l'empire ouïgour, à partir de 832, pour reprendre l'initiative et mettre fin à la situation privilégiée de l'église manichéenne, en édictant, en 843, un décret qui visait principalement l'argent et les richesses déposés dans ses temples.À partir de 845, la répression contre le bouddhisme, provoquée en grande partie par les machinations du taoïste Zhao Guizhen et de ses deux acolytes, Deng Yuanchao et Liu Xuanjing, prit une nouvelle ampleur. Au troisième mois de cette même année, deux fonctionnaires confucéens recommandèrent la suppression des assemblées de jeûne. La tradition bouddhique avait en effet institué plusieurs périodes de jeûne, annuel ou mensuel. Les périodes de jeûne annuel avaient lieu les premier, cinquième et neuvième mois. Les périodes de jeûne mensuel étaient de six ou dix jours. Les premiers avaient lieu les 8°, 14°, 15°, 23°, 29° et 30° jours du mois, auxquels s'ajoutaient quatre jours (les 1°, 18°, 24° et 28° jours du mois) pour les jeûnes mensuels de dix jours.Gaozu avait émis, dès 619, un décret qui suspendait les exécutions capitales et l'abattage d'animaux durant ces périodes. Ce décret, qui fut maintenu par l'impératrice Wu Zhao en 692, semble être resté en vigueur durant toute la période des Tang, jusqu'à son abolition en 844. Les bouddhistes furent, cette même année, exclus pour la première fois des cérémonies qui accompagnaient le jour anniversaire de la naissance de l'empereur.Il existait depuis longtemps, au sein même du palais impérial, dans le Pavillon de la longévité, une chapelle où étaient conservées des images du Bouddha et des textes sacrés. Des moines, détachés à son service, pouvaient y circuler librement et y célébrer des cérémonies pour la protection et la prospérité de l'empire, ou fêter l'anniversaire de l'empereur. Écoutant les conseils de son maître taoïste, le fanatique Zhao Guizhen, l'empereur Wuzong fit détruire ces images, brûler les textes et renvoya les moines dans leurs monastères d'origine. Il fit installer dans cette chapelle des images de Laozi et d'autres divinités taoïstes, et poussa les lettrés de la cour à embrasser la religion taoïste. Ennin rapporte qu'aucun d'entre eux ne suivit cette injonction.En 844, Wuzong interdit le culte rendu à la dent du Bouddha et décréta que tous ceux qui feraient un don pour organiser cette cérémonie seraient passibles couns de canne de bambou. À l'occasion de la fête donnée enmonasteres a orgaet d'autres divinités taoïstes, et poussa les lettrés de la cour à embrasser la religion taoïste. Ennin rapporte qu'aucun d'entre eux ne suivit cette injonction.En 844, Wuzong interdit le culte rendu à la dent du Bouddha et décréta que tous ceux qui feraient un don pour organiser cette cérémonie seraient passibles de vingt coups de canne de bambou. À l'occasion de la fête donnée en 844 pour le repos de toutes les âmes, les fidèles firent aux temples bouddhiques des dons d'une exceptionnelle générosité, mais l'empereur les fit saisiret les offrit à un temple taoïste, ce qui provoqua la colère du peuple.Dès le septième mois de 844, la répression se durcit, les mesures visant cette fois la destruction progressive de la communauté bouddhique. Il y eut d'abord les décrets ordonnant la fermeture de tous les temples et sanctuaires mineurs, et la réduction à l'état laïc de tous les religieux dont le nom n'apparaissait pas sur les registres des monastères. Les images et manuscrits que contenaient les établissements mineurs furent transférés aux temples de plus grande importance, et les cloches et autres ornements qui s'y trouvaient, offerts aux temples taoïstes. Selon le moine japonais Ennin, trois cents établissements furent alors détruits dans la seule ville de Chang'an. Le troisième mois de l'année suivante (845), le gouvernement interdit aux monastères le droit d'établir des domaines agricoles et ordonna de dresser le recensement des esclaves employés à leur service, ainsi qu'un inventaire des richesses en argent, marchandises ou grain, en leur possession.wees prises conne le sumaha s'étaient limitées à rédure rde la comminauté qui n'abuervaient pas les réglementoles établissements mineurs furent transférés aux temples de plus grande importance, et les cloches et autres ornements qui s'y trouvaient, offerts aux temples taoïstes. Selon le moine japonais Ennin, trois cents établissements furent alors détruits dans la seule ville de Chang'an. Le troisième mois de l'année suivante (845), le gouvernement interdit aux monastères le droit d'établir des domaines agricoles et ordonna de dresser le recensement des esclaves employés à leur service, ainsi qu'un inventaire des richesses en argent, marchandises ou grain, en leur possession.Jusqu'alors, les mesures prises contre le samgha s'étaient limitées à réduire à l'état laïc les membres de la communauté qui n'observaient pas les règlements ou se livraient à des pratiques peu orthodoxes, telles que la magie noire ou lesincantations, mais la nouvelle politique mise en place par le gouvernement touchait désormais tous les moines, sans tenir compte de leur moralité, de leurs connaissances ou de leur rang. On réduisit d'abord à l'état laïc les moines de moins de quarante ans, mesure qui s'étendit bientôt aux religieux de moins de cinquante. Les moines âgés de plus de cinquante ans qui ne possédaient pas de certificat furent également rapidement contraints de quitter les ordres. Ces mesures s'appliquèrent aussi aux moines étrangers qui résidaient en Chine : tous ceux qui ne possédaient pas le certificat délivré par le Bureau des rites furent réduits à l'état laic et renvoyés dans leur pays d'origine.Afin de procéder à l'éradication totale du bouddhisme, les autorités ordonnèrent, au quatrième mois de l'année 845, un recensement des membres de la communauté monastique et des monastères. On dénombra 260 000 moines et nonnes, 4 600 temples et 40 000 sanctuaires. Muni de ces chiffres, le gouvernement ordonna au septième mois de la même année, la destruction de la presque totalité des établissements bouddhiques du pays. On ne conserva qu'un temple par préfecture principale, généralement le plus beau et le plus orné, et quatre monastères dans chacune des deux capitales, l'effectif étant réduit à trente moines par établissement. Le Secrétariat impérial rédigea ensuite plusieurs mémoires sur la confiscation des biens monastiques. Lestransformées en pièces derac danapale, generalement le plus beau et le plusorné, et quatre monastères dans chacune des deux capitales, l'effectif étant réduit à trente moines par établissement. Le Secrétariat impérial rédigea ensuite plusieurs mémoires sur la confiscation des biens monastiques. Les images de bronze et les cloches devaient être transformées en pièces de monnaie, les statues de fer en outils agricoles et les images en or, argent ou jade remises au Trésor public. Au huitième mois, le gouvernement impérial fit paraître un édit résumant les résultats obtenus par sa politique de répression du bouddhisme :Nous savons que, durant les trois premières grandes dynasties (Xia, Shang et Zhou), il ne fut jamais question du bouddhisme. Cette religion idolâtre n'a commencé à s'épanouir qu'à partir des Han et des Wei. Depuis peu, ses pratiques étranges, devenues familières, se sont imposées au point de corrompre lentement et inconsciemment nos mœurs. Beaucoup ayant été victimes de sa séduction, les masses n'en ont été que plus égarées. Partout sur notre territoire, des régions reculées jusqu'aux murs des palais de nos capitales, des moines surgissent en nombre toujours plus grand, et leurs monastères chaque jour gagnent en prospérité. Ils ont épuisé dans des travaux de construction les forces vives du pays, ont extorqué le peuple pour fabriquer à leur propre usage des ornements d'or et de pierreries ; ils ont incité les hommes à délaisser leur souverain et leurs proches pour suivre des maîtres religieux, à abandonner leurs conjoints pour embrasser la vie monastique, à violer la loi et à causer du tort au peuple. Rien n'est plus exécrable que cette religion.Lorsqu'un homme ne cultive pas la terre, d'autres souffrent de la faim, lorsqu'une femme ne tisse pas, d'autres souffrent du froid. Les moines et nonnes de ce pays sont innombrables, mais ils comptent sur la culture de la terre pour se nourrir et la production de la soie pour se vêtir. Leurs monastères et leurs temples, qui sont innombrables, sont des bâtiments altiers, magnifiquement ornés, qui osent rivaliser avec les plus riches palais.Il n'est pas besoin de chercher ailleurs les causes du déclin matériel et du relâchement moral qui a affecté les dynasties Jin, Song, Qi et Liang.Gaozu et Taizong ont mis fin aux désordres par les armes et régné sur ce beau pays par les arts et les lettres. Ces deux méthodes suffisent à gouverner.Comment cette insignifiante religion venue de l'Ouest pourrait-elle nous en imposer ? Dès les ères zhenguan (627-650) et kaiyuan (713-742) des réformes furent engagées, mais elles n'ont pas suffi à déraciner le mal, qui a continué à se répandre et prospérer.Après avoir pris ample connaissance des rapports précédents et largement consulté l'avis général, nous sommes arrivés à la conviction qu'il fallait agir contre ce mal. Nos ministres à la cour et nos administrateurs en province s'accordent avec nous pour dire qu'il est de la plus grande urgence de réformer [l'église bouddhique]. Nous devons exaucer leurs souhaits sur ce point. Personne ne sera plus déterminé que nous à tarir cette source de corruption, à nous conformer aux lois des cent rois [qui nous ont précédés] et à porter secours à notre navonGaozu et Taizong ont mis fin aux desorates par tes autes et re beau pays par les arts et les lettres. Ces deux méthodes suffisent à Comment cette insignifiante religion venue de l'Ouest pourrait en imposer ? Dès les ères zhenguan (627-650) et kaiyuan (713 réformes furent engagées, mais elles n'ont pas suffi à déracin qui a continué à se répandre et prospérer.Après avoir pris ample connaissance des rapports précédents et consulté l'avis général, nous sommes arrivés à la conviction qu'il contre ce mal. Nos ministres à la cour et nos administrateurs er s'accordent avec nous pour dire qu'il est de la plus grande u réformer [l'église bouddhique]. Nous devons exaucer leurs so ce point. Personne ne sera plus déterminé que nous à tarir cette corruption, à nous conformer aux lois des cent rois [qui nous ont et à porter secours à notre pays pour le bien des masses.Plus de 4600 monastères sont en ce moment détruits dans toutPlus de 260 500 moines et nonnes sont réduits à l'état laïc, a la corvée et à l'impôt. Plus de 40 000 temples et sanctuaires Plusieurs dizaines de millions d'hectares (ging) de champs de terres arables sont confisqués. Nous avons recouvré la poss 150 000 esclaves, qui sont désormais soumis au double impôt. L et les nonnes seront placés sous le contrôle du Bureau des hôtes, m ainsi clairement que le bouddhisme est une religion étrang réduisons [également] à l'état laïc plus de 3000 nestoriens et zoroastriens afin qu'ils ne corrompent pas les coutumes de la Chine.

La persécution de 845 fut incontestablement la plus étendue. Les proscriptions de 446 et 574-577 avaient été très largement limitées au nor de la Chine, dans les régions gouvernées par les Wei et les Zhou du Non.Elles n'avaient pas eu d'effets délétères à long terme sur le bouddhiste et avaient épargné tout à fait les communautés du Sud. La proscriptos décrétée sous les Tang, en revanche, s'appliqua à tout le territoire et caus in8. E. O. Reischauer, op. cit., avec quelques modifications.

P238

dommage permanent à la communauté bouddhique, raison pour laquelle elle est considérée comme un des événements majeurs de l'histoire du bouddhisme en Chine. Le bouddhisme souffrait déjà d'un certain affaiblissement, ainsi qu'en témoignent le relâchement de la foi et l'affaissement de la vigueur intellectuelle en certains endroits, mais ce fut la persécution de 845 qui lui donna le coup de grâce. Cette date, qui marque un point de rupture, indique la fin de son apogée et le début de son déclin.La proscription elle-même fut de courte durée. Un an ne s'était pas écoulé que Wuzong mourait, au troisième mois de 846, probablement affaibli par les élixirs de longue vie qu'il avait consommés. Xuanzong monta sur le trône impérial et s'employa immédiatement à mettre un terme à la répression décrétée par son prédécesseur. Zhao Guizhen, Liu Xuanjing, ainsi que onze autres taoïstes, furent exécutés pour avoir incité l'empereur à prendre des mesures extrêmes contre le bouddhisme. L'empereur Xuanzong autorisa la présence de douze temples au lieu de quatre dans la capitale, de deux temples dans chaque préfecture et de trois dans chaque commanderie. Enfin, les moines de plus de cinquante ans qui avaient été contraints de défroquer l'année précédente furent autorisés à revêtir de nouveau l'habit monastique.L'année suivante, l'empereur donna, pour ainsi dire, à la communauté monastique le feu vert qui l'autorisait à retrouver son fonctionnement habituel.L'édit qu'il fit paraître définissait certes le bouddhisme comme une religionL'édit qu'il fit paraître définissait certes le bouddhisme comme une religionnroclamait également aue. puisau'il n'avait pas portéavait pas être proscrit,unement affaibli parone impérial et s'employa immédiatement à mettre un terme à la répression trône impérial et s' employa immédiat consommés. Xuanzong motil ra trône impérial et s'employa immédiatement à mettre un terme à la répression décrétée par son prédécesseur. Zhao Guizhen, Liu Xuanjing, ainsi que onze autres taoïstes, furent exécutés pour avoir incité l'empereur à prendre des mesures extrêmes contre le bouddhisme. L'empereur Xuanzong autorisa la présence de douze temples au lieu de quatre dans la capitale, de deux temples dans chaque préfecture et de trois dans chaque commanderie. Enfin, les moines de plus de cinquante ans qui avaient été contraints de défroquer l'année précédente furent autorisés à revêtir de nouveau l'habit monastique.L'année suivante, l'empereur donna, pour ainsi dire, à la communauté monastique le feu vert qui l'autorisait à retrouver son fonctionnement habituel.L'édit qu'il fit paraître définissait certes le bouddhisme comme une religion étrangère, mais il proclamait également que, puisqu'il n'avait pas porté atteinte aux principes fondamentaux de l'empire, il ne devait pas être proscrit, particulièrement au vu du fait que les Chinois le pratiquaient depuis longtemps; en conséquence, si les moines désiraient réparer les temples qui avaient été endommagés durant la persécution, ils ne devaient pas en être empêchés par les autorités. Nous pouvons remarquer en conclusion que la persécution de 845 suit le schéma des précédentes proscriptions: d'abord des mesures violentes et drastiques, suivies, peu après la mort du souverain, de l'instauration par son successeur d'une politique de restauration.

s pèlerins chinois à l'étrangernombre de pèlerins partant pour l'Inde en quête de la Loi s'accrut lérablement durant la période d'expansion et de rayonnement du hisme en Chine. Comme la puissance de la dynastie Tang se faisait jusqu'en Asie centrale, de nombreux petits États de cette zone, soucieux lir de bonnes relations diplomatiques avec ce grand empire, s'ingénièrent iter le passage des moines chinois sur leur territoire. Le bouddhisme,
\cite[p. 233-239]{chen_histoire_2015}

\subsection{Impact économique}


\paragraph{le cout de la religion - l'exemple romain}
\begin{singlequote}
    Le dessein des sanctuaires et des cites parait assez clair, il s'agit
avant tout d'assurer Ia perennite des ietes et des sacrifices; de rendre
le sanctuaire le plus riche possible, c'est a dire de le doter de
monuments aussi somptueux que le permettent - mal - les finances
publiques et sacrees. Nous ne relevons Ia aucun souci de chrematistique:
certes les sanctuaires participent de fa~on active a Ia vie
economique, ils en sont meme l'un des moteurs par les besoins specifiques
qui sont les leurs ( materiaux de construction, denrees de
luxe); mais ils n'ontj amais cherche a Ia developper en soi; l'activite
bancaire repond a des considerations sociales plus qu' economiques
et il s'agit avant tout de tenter d'assurer des revenus reguliers au
culte273 ; l'activite monetaire est un phenomene local et trop particulier
pour qu'on puisse lui preter une signification generale: il
convient dans Ia plupart des cas de l'interpreter dans le cadre de Ia
panegyrie. En bref, ce n'est certainement pas par un mouvement
conscient et raisonne que les sanctuaires ont developpe et diversifie
leurs activites economiques et financieres, cependant en fonction
meme d'un certain volume de richesse, les sanctuaires au moins les
plus importants, ont pese d'un poids indeniable sur les rouages de
l'economie grecque. II ne faut pas d'autre part envisager l'accroissement
de Ia richesse des sanctuaires comme un phenomene
lineaire. Cette accumulation de metal precieux a suscite bien des
convoitises et cela d'autant plus que le sentiment du sacre s'etait
bien transforme a partir de Ia fin de l'epoque classique.
Aspects sociaux et économiques de la vie religieuse dans l'Anatolie gréco-romaine \cite{debord_aspects_1982}
\end{singlequote}
 

\subsection{David A. Palmer importance du temple - Goosaert}

\begin{singlequote}
    
Le troisieme chapitre relate brievement l'histoire du temple en tant qu'institution religieuse en Chine. Les premiers « temples » étaient des mausolées et des autels destinés au culte des ancêtres, qui se pratiquait souvent en plein air. Puis il y eut les sanctuaires du culte impérial de la dynastie Han. Mais c'est le bouddhisme qui popularisa la notion du temple abritant des icônes de divinités, lieu ouvert à tous et consacré au culte religieux.
Le monachisme bouddhique eut un immense impact social et politique à l'époque du Moyen  Age chinois (111"-VI· siecles). La construction de monastères opulents transforma le paysage rural et urbain. Le temple fut alors adopté aussi bien par le tao'isme institué que par les cultes populaires. Ces derniers étaient souvent dédiés à des divinités de la nature ou des héros locaux, et servirent de centres de résistance locale aux fonctionnaires du gouvernement central ainsi qu'au clergé bouddhique et tao'iste. Sous la dynastie des Tang (VIl"  IXº siecles), l'Etat impérial inaugura une politique de contrôle de toutes les institutions religieuses. 11 établit un « concordat » qui garantissait l'unité et l'égalité des trois traditions établies, placées sous sa protection. Les empereurs Tang ont aussi commencé la pratique des canonisations des dieux populaires, en leur assignant une place dans la hiérarchie céleste. Cette pratique favorisa la cooptation de ces cultes, qui devaient demander une
autorisation officielle pour la construction de temples. C'est à cette époque que les temples devinrent l'institution principale de la vie communautaire en Chine. Sous les Song, cependant, l'harmonie entre les « trois religions » fut détruite sous la pression des ambitions hégémoniques du confucianisme. Les cultes locaux réagirent en se libé  rant progressivement de la tutelle de l'Etat ; signe de leur plus grande indépendance, ils constitue  rent de vastes réseaux transrégionaux de temples. A l'époque des Ming et des Qing (XIVe-XIX siecles), l'écart entre la religion d'élite et la religion populaire se creusa. Vers la fin du XIX siecle, alors que le tissu social se fragilisait, les temples et les cultes se multiplierent, au point ou ils consti  tuerent souvent l'institution principale de l'organisation et de la défense villageoise. Au même moment, des mouvements sectaires, tels les Taiping, détruisaient tous les temples des régions qui étaient sous leur contrôle. Et les convertis au christianisme, en refusant de contribuer au financement des temples, contribuerent à briser l'unité de communautés qui avaient traditionnellement considéré la construction et l'entretien des temples comme une responsabilité collective.
Les réformes de Kang Youwei, promulguées en 1898, ont inauguré un changement radical de politique à l'égard des temples, qu'on voulut convertir en éléments d'infrastructure d'un Etat modeme. Cette politique fut systématiquement mise en oeuvre durant tout le XX: siecle : les temples furent ainsi transformés en écoles, en bureaux de police et des impôts, etc. Des milliers de temples furent tout simplement détruits: « Leur rôle d'articulation dans un systeme traditionnel, fragmenté en petites unités et en particularismes, était aux yeux [des réformateurs] impardonnable » (p. 99). Seuls les grands monasteres bouddhiques, isolés géographiquement et relativement à l'écart du systeme social traditionnel, furent épargnés. La Révolution culturelle n'a fait que continuer une histoire de destruction qui a traversé tout le vingtieme siecle. Attjourd'hui, ce sont les urbanistes et les promoteurs immobiliers qui démolissent des temples pour construire des immeubles modemes. On estime qu'cn 1900, il y avait environ un million de temples en Chine : un temple pour
cent familles. De ceux-ci, il ne reste maintenant que quelques milliers. « De 1898 à aujourd'hui s'est écoulé un siecle de destruction continue, par tous les moyens, et qui restera sans doute dans l'histoire de l'humanité comme l'un des plus grands anéantissements du patrimoine » (p. 101). Malgré cela, les temples continuent à prospérer à Taiwan et dans les communautés chinoises d'outre-mer. On assiste aussi à une résurgence de la construction de temples en Chine populaire, financés par le gouvernement, par les Chinois de la diaspora et par les fideles locaux.
Dans le quatrieme chapitre, Goossaert propose quatre modes d'appréhension de l'espace sacré. Tout d'abord, le temple peut être considéré comme un mémorial voué aux ancêtres : un lieu ou l'on honore les morts comme s'ils étaient présents, sans toutefois chercher abusivement leur intercession. Selon ce mode, l'appartenance à une communauté religieuse implique la filiation à une lignée présidée par une divinité. En deuxieme lieu, le temple peut être vu comme une cour de justice. Dans ce cas, le dieu n'est pas un ancêtre mais un fonctionnaire céleste, investi en tant que tel d'une autorité judiciaire. En tant que juges, les dieux peuvent convoquer des témoins de l'au-delà. Ils peuvent aussi être eux-même témoins : d'importants contrats étaient souvent scellés devant les dieux, qui punissaient ceux qui ne se tenaient pas à leurs engagements. En troisieme lieu, le temple peut être considéré comme une maison, un lieu de loisirs et de récréation, un lieu de vie. Il n'y a pas de distinction radicale entre l'architecture d'un temple et celle d'une maison : les temples se différencient par leur hauteur, leur taille, leur ornementation. Les temples sont les résidences impériales des dieux, qui offrent nourriture et logement aux passants. Le quatrieme mode est celui du temple comme montagne : la métaphore de la montagne est souvent utilisée pour décrire le temple en partie Oa pagode, le toit) ou dans son ensemble. I:ascension des montagnes pour arriver aux monasteres perchés au sommet est un acte de dévotion, rapprochant le pelerin des hauteurs étranges et sauvages de la transcendance spirituelle.
Le cinquieme chapitre s'attache à la fondation
des temples, le plus souvent le résultat d'une initiative individuelle, et aux modalités du financement de leur construction et de leur entretien.
Dans le sixieme chapitre, nous découvrons les acteurs de la vie du temple : le clergé, les devins et les mediums. La plupart des temples sont administrés par des comités laies qui emploient et supervisent les officiants qui y résident. Ce sont ces comités de dévots qui, à travers leurs oeuvres charitables et sociales, constituaient la trame même de la vie culturelle et religieuse chinoises. Enfin, le septieme chapitre nous présente la vie religieuse proprement dite des temples : le culte quotidien, les offrandes d'encens, de papier-monnaie et de sacrifices ; les festivals, rituels et processions ; enfin la musique et les banquets qui colorent la vie du temple.
Dans sa conclusion, Goossaert revient au theme du premier chapitre : la relation entre les temples, l'Etat et la société. \begin{singlequote}
    « Le temple chinois est une institution politique: l'Etat s'en sert pour gouverner, et le peuple y fonde son organisation » (p. 33).
\end{singlequote} Les temples sont des lieux d'articulation de la culture institutionnelle et populaire. À l'intérieur, la liturgie de l'élite d'Etat ou monastique apporte légitimité politique et cosmologique, alors qu'à l'extérieur, les fêtes communautaires et les associations de temple apportent le soutien et le financement du peuple, sans lequel l'ensemble de la liturgie officielle ne pourrait survivre. Le temple, donc, est un « lieu de négociation religieuse ». En lui se retrouvent tous les éléments de la religion chinoise. Bien que la coexistence des tendances élitistes et populaires ne soit pas toujours facile, tous les acteurs comprennent que le compromis est essentiel. « Le mélange des éléments, en des proportions toujours variées, rend compte de l'unicité d'une religion chinoise tres étendue, mais dont aucune partie ne veut se séparer radicalement des autres » (p. 204). Le temple est un espace privilégié ou se forme et s'exprime le contenu religieux, et qui attire vers lui toutes les connaissances et les richesses: les dieux parlent à travers les médiums et les orades, les artisans et les jardiniers façonnent la beauté des lieux, les prêtres célebrent les rites, les troupes d'opéra jouent des histoires saintes, les steles et les peintures racontent les faits des dieux et des adeptes, les maitres des arts du corps enseignent 
les secrets du combat ou de la longévité, et les les temples, ont investi les pares et les espaces philanthropes font leurs bonnes oeuvres. Le publics dans les années 80 et 90. La religion hors temple n'est donc pas un immeuble figé, mais un des temples n'étant pas le propos de ce livre, lieu ouvert, à l'intérieur duquel les formes Goossaert ne s'attarde pas sur cette tendance.  

\end{singlequote}