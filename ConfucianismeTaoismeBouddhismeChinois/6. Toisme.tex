\chapter{Laozi et le Daode jing}

\subsection{Laozi}
\paragraph{Disparate et complexe}

\paragraph{Laozi, personnage légendaire} Cité dans un ouvrage du IV av JC. Au II av JC, la première biographie. 



Laozi, un personnage largement légendaire


Éléments principaux de la légende concernant Laozi:
\begin{itemize}
    \item  	Il aurait eu pour nom Li, pour prénom Er
    \item 	Il aurait été archiviste ou annaliste-devin de la dynastie Zhou
    \item 	La visite que Confucius lui aurait rendue
    \item 	Son exil vers l’ouest à dos de buffle
    \item 	Sa rencontre avec Yin Xi, le gardien de la passe, qui lui demanda de coucher par écrit sa doctrine
\end{itemize}


A 50 ans, il arrive sur un buffle noir. Le gardin est content. Il demande à Laozi de coucher son enseignement.

\begin{figure}[!h]
    \centering
        \sidecaption{Laozi sur le buffle Zhang Lu (1490-1563)}
    \includegraphics{}

    \label{fig:enter-label}
\end{figure}



\paragraph{L'Ouest} Théorie qui arrive au IIè siècle selon lequel il voulait aller en Inde devenir un Bouddha.



 \subsection{Le Laozi ou le Daode jing }

\paragraph{Le Laozi ou le Daode jing} : texte attribué à Laozi mais de nature fondamentalement composite dont l’origine reste un mystère
\begin{itemize}
    \item 	La version canonique

    \item 	Les version de Mawangdui

    \item 	Les versions de Guodian
\end{itemize}

 \begin{figure}[!h]
    \centering
        \sidecaption{Un groupe de tombes datant du début des Han occidentaux (202 av. J.-C.– 9 apr. J.-C.) découvertes à Mawangdui, près de la ville de Changsha, capitale de la province du Hunan actuelle, dans les années 70.}
    \includegraphics{}

    \label{fig:enter-label}
\end{figure}


 


 \begin{figure}[!h]
    \centering
        \sidecaption{Fragments de la version A du Laozi trouvés dans la tombe n° 3 de Mawangdui. Cette tombe, datant de 168 av. J.-C., appartenait à Li Xi, fils de la marquise de Dai.}
    \includegraphics{}

    \label{fig:enter-label}
\end{figure}

 
Tombe n°1 de Guodian, datant du début du 4ème siècle av. J.-C., faisait partie d’un cimetière de l’ancienne capitale de la principauté de Chu, dans la province du Hubei actuelle. Elle a été découverte en 1993.

   
 
Les Etats indépendants avant l’unification de l’empire en 221 avant notre ère
  
Manuscrits découverts dans la tombe n° 1 de Guodian

 

\subsection{La bureaucratie céleste}
 \begin{figure}[!h]
    \centering
        \sidecaption{À droite: « Le vénérable céleste du commencement originel »
Vers 1700 Musée Guimet

À gauche: « Le seigneur de Fengdu (les enfers taoïstes) et ses six ministres devenus immortels » Vers 1600
Musée Guimet}
    \includegraphics{}

    \label{fig:enter-label}
\end{figure}





