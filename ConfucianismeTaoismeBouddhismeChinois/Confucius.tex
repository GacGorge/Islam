\chapter{Confucius}  

Versions citées : Anne CHENG (trad.), Les Entretiens de Confucius, Paris : Seuil, 1981.  Jean LEVI (trad.), Les deux arbres de la Voie : Les Entretiens de Confucius, Paris : Belles Lettres, 2018.   Pierre RYCKMANS (trad.), Les Entretiens de Confucius, Paris : Gallimard, 1987.  

\section{La personne de Confucius : }

\begin{itemize}

     \item  1. \newline Le Maître a dit : À quinze ans, j’avais le cœur à l’étude, à trente j’étais indépendant, à quarante je n’hésitais plus, à cinquante je connus mon lot, à soixante mes oreilles s’étaient faites à tout, maintenant que j’en ai soixante-dix, je lâche la bride à mes passions sans jamais dépasser la mesure.   \textit{\small -- Les Entretiens, II. 4, traduction de Jean Levi, p. 8.  } 
\item 2. \newline Le Grand Intendant s’étant étonné auprès de Zigong [disciple de Confucius] que Confucius puisse être un saint avec tous ses talents divers, celui-ci rétorqua :  -- Assurément le Ciel, tout en le destinant à la sainteté, s’est plu à le doter de toutes sortes de capacités. Ces propos étant revenus aux oreilles du Maître, celui-ci déclara :  \newline -- Me connaît-il, ce Grand Intendant ! Ayant connu la pauvreté dans ma jeunesse, il m’a fallu exercer toutes sortes de petits métiers. L’homme noble a-t-il de multiples talents ? Non, il n’en a pas.  \textit{\small -- Les Entretiens, IX. 6, traduction de Jean Levi, p. 58.  }
\item 3. \newline Quelqu’un demande à Confucius : Maître, comment se fait-il que vous n’exerciez aucune fonction officielle ?  À quoi Confucius répond : \newline Il est dit dans le Livre des Documents : « Être bon fils, être simplement bon fils et bon frère, c’est déjà prendre part au gouvernement. » Vous voyez donc que point n’est besoin d’occuper un poste pour remplir une fonction.  \textit{\small -- Les Entretiens, II. 21, traduction d’Anne Cheng, p. 36-37.  
}
\item 4. \newline Escogriffe des Marais et Éminence Engloutie bêchaient et labouraient de conserve. Confucius envoya Zilu leur demander où se trouvait le gué. \newline -- C’est qui celui là-bas qui conduit la voiture ? demanda Escogriffe des Marais. \newline -- C’est Confucius. \newline -- Le Confucius de Lu ? \newline -- C’est ça. \newline -- Il connaît le gué.  \newline Zilu s’adressa alors à Éminence Engloutie, lequel l’apostropha :  \newline -- Toi, tu es qui ? \newline -- Je suis Zilu.\newline  -- Tu ne serais pas, par hasard, un disciple de Confucius ? \newline -- En effet.  \newline -- L’empire est pris dans un tourbillon ; qui peut rien y changer ? Pourquoi, au lieu d’être à la remorque de quelqu’un qui va de l’un à l’autre, ne suivrais-tu pas plutôt des sages qui ont su fuir le monde ? \newline  Sur ce, il se remit à bêcher sans désemparer. Zilu s’en fut rapporter ces propos au Maître. Celui-ci resta un moment perplexe avant de s’exclamer :  \newline  -- Je ne puis tout de même pas vivre en compagnie des oiseaux et des quadrupèdes ! Si je refusais de fréquenter mes semblables, dans la société de qui vivrais-je ? Et puis, il va de soi que si l’ordre régnait dans le monde je ne me mêlerais pas de vouloir le changer.  \textit{\small -- Les Entretiens, XVIII. 6, traduction de Jean Levi, p. 129-130.  }

\item 5. \newline Alors que Zilu s’apprêtait à passer la nuit au gîte de la Porte de Pierre, le gardien de l’octroi lui demanda : \newline-- D’où venez-vous ? \newline -- De chez Confucius, répondit Zilu. \newline-- Ah, fit l’homme, n’est-ce pas celui qui s’obstine dans une tâche qu’il sait impossible ?  \textit{\small -- Les Entretiens, XIV. 36, traduction de Jean Levi, p. 104.  }
\item 6. \newline Le Maître a dit : Je n’ai jamais refusé d’instruire personne pourvu qu’on se présente à moi en y mettant un minimum de forme. \textit{\small -- Les Entretiens, VII. 7, traduction de Jean Levi, p. 44.   
}
\end{itemize}
\section{L’idéal de l’homme chez Confucius :}  
\begin{Def}[Le junzi 君子]
     (homme de bien, honnête homme, gentilhomme, homme noble)   
\end{Def}
\begin{itemize}
    
\item 7. \newline Le Maître dit : « L’honnête homme cherche la vérité, il ne cherche pas un gagne-pain. Labourez, et vous ne mangerez pas nécessairement à votre faim. Étudiez, et vous ferez peut-être carrière. L’honnête homme se soucie de la vérité, il ne se soucie pas de la pauvreté. » \textit{\small -- Les Entretiens, XV. 88, traduction de Pierre Ryckmans, p. 88. } 
\item 8. \newline Le Maître dit : L’homme de bien n’est pas un ustensile destiné à un seul usage. \textit{\small -- Les Entretiens, II. 12, traduction d’Anne Cheng, p. 35.  }
\item 9. \newline Confucius dit : L’homme de bien a respect pour trois choses : les Décrets du Ciel, les hommes éminents et les paroles d’un Sage. L’homme de peu ne craint pas la Volonté Céleste pour la simple raison qu’il ne la connaît pas, prend des libertés avec les grands et tourne en dérision les paroles du Sage. \textit{\small -- Les Entretiens, XVI. 8, traduction d’Anne Cheng, p. 131. }
\item 10. \newline Zilu demanda ce qui constitue un gentilhomme. \newline Le Maître dit : « Il se cultive, et acquiert ainsi de la gravité. – Est-ce tout ? – Il se cultive, et donne ainsi la paix à autrui. – Est-ce tout ? – Il se cultive, et donne ainsi la paix au peuple. Mais là, Yao et Shun eux-mêmes ont peiné. \textit{\small -- Les Entretiens, XIV. 42, traduction de Pierre Ryckmans, p. 83.  }
\item 11. \newline  Sima Niu : Qu’est-ce qu’un homme de bien ? \newline Le Maître dit : L’homme de bien est celui qui n’a ni inquiétudes, ni craintes. Sima Niu : Ni inquiétudes, ni craintes—c’est donc cela l’homme de bien ? \newline Le Maître : Si, regardant en lui-même, il n’y trouve aucune tache, quelles inquiétudes, quelles craintes pourrait-il avoir ? \textit{\small -- Les Entretiens, XII. 4, traduction d’Anne Cheng, p. 96.   3 }
\item 12. \newline Le Maître voulait émigrer chez les Barbares. On lui dit : « Comment pourriez-vous vous accommoder d’une existence sauvage ? » \newline Le Maître répondit : « Là où réside l’honnête homme, il n’y a pas de sauvagerie qui tienne. » \textit{\small -- Les Entretiens, IX. 14, traduction de Pierre Ryckmans, p. 51.  }

\item 13. \newline Le duc Ling de Wei interrogea Confucius sur l’art de manœuvrer les armées. Confucius répondit : \newline « Je sais l’une ou l’autre chose en ce qui regarde l’art de disposer les vases rituels, mais je n’ai jamais étudié celui de disposer les régiments et les bataillons. » \newline Il s’en alla le lendemain.          Au pays de Chen, on lui coupa les vivres. Ses disciples affaiblis ne tenaient plus sur leurs jambes. Indigné, Zilu vint le trouver et dit : \newline « Se peut-il qu’un honnête homme tombe dans la détresse ? » \newline Le Maître dit : « Bien sûr qu’un honnête homme peut tomber dans la détresse. Dans la détresse, seul l’homme vulgaire se laisse démonter. »  \textit{\small -- Les Entretiens, XV. 1-2, traduction de Pierre Ryckmans, p. 84.  }
\paragraph{Le junzi par opposition au xiaoren 小人 (homme de peu, homme vulgaire)  }
\item 14. \newline L’homme de bien exige tout de lui-même, l’homme de peu attend tout des autres.  \textit{\small -- Les Entretiens, XV. 20, traduction d’Anne Cheng, p. 124.  }
\item 15. \newline Le Maître a dit :  L’homme noble conçoit tout en termes de devoir, l’homme vulgaire de profit. \textit{\small -- Les Entretiens, IV. 16, traduction de Jean Levi, p. 23.  }
\item 16. \newline Le Maître dit : L’homme de bien aide à s’accomplir ce que les autres ont de bon, non ce qu’ils ont de mauvais. L’homme de peu fait tout le contraire.  -- \textit{\small Les Entretiens, XII.  16, traduction d’Anne Cheng, p. 99.  }
\paragraph{Apprendre (xue 學) pour s’accomplir   }
\item 17. \newline Le Maître dit : « N’est-ce pas une joie d’étudier, puis, le moment venu, de mettre en pratique ce que l’on a appris ? N’est-ce pas un bonheur d’avoir des amis qui viennent de loin ? Et n’est-il pas un honnête homme celui qui, ignoré du monde, n’en conçoit nul dépit ? » -- \textit{\small Les Entretiens, I. 1, traduction de Pierre Ryckmans, p. 13.  }
\item 18. \newline Le Maître dit : « Autrefois on étudiait pour soi, aujourd’hui on étudie pour impressionner les autres. » -- \textit{\small Les Entretiens, XIV. 24, traduction de Pierre Ryckmans, p. 80.  }
\item 19. Fan Chi prie \newline Le Maître de lui enseigner l’agriculture. \newline Le Maître : Pour cela, adresse-toi plutôt à un vieux paysan. Fan Chi lui demande alors des instructions sur le jardinage. \newline Le Maître : Pour cela, tu ferais mieux d’aller voir un vieux jardinier.    \newline     Le disciple sorti, \newline Le Maître s’exclame : Un homme de bien peu, ce Fan Chi ! Que les gouvernants s’attachent au rituel, et il ne se trouvera personne dans le peuple pour y contrevenir ; qu’ils s’attachent au Juste, et il n’y aura pas un signe d’insoumission ; qu’ils s’attachent à la bonne foi, et personne n’osera parler contre la vérité. C’est ainsi qu’ils verront venir à eux de toutes parts le peuple entier, les mères accourant avec leurs enfants sur le dos. Que leur servirait de savoir cultiver la terre ? \textit{\small -- Les Entretiens, XIII. 4, traduction d’Anne Cheng, p. 103.    4 }

\item 20. \newline Le Maître dit : « Zigong, crois-tu que je sois quelqu’un qui étudie une masse de choses et qui les retient par cœur ? » \newline L’autre répondit : « En effet. N’en est-il pas ainsi ? \newline –Nullement. J’ai un seul fil pour enfiler le tout. » \textit{\small -- Les Entretiens, XV. 3, traduction de Pierre Ryckmans, p. 84.  }
\item 21. \newline Le Maître dit : Étudier sans réfléchir est vain ; méditer sans étudier est périlleux. -- \textit{\small Les Entretiens, II. 15, traduction d’Anne Cheng, p. 35.   }
\subsection{Les cinq vertus   le ren 仁 (le sens de l’humain)  }
\item 22. \newline Fan Chi : Qu’est-ce que le ren ?  \newline Le Maître : C’est aimer les hommes.  […]  \textit{\small -- Les Entretiens, XII. 22, traduction d’Anne Cheng, p. 101.  }
\item 23. \newline Zigong demanda : « Que penseriez-vous d’un homme qui comblerait le peuple de bienfaits et qui viendrait en aide à la multitude ? Peut-on dire qu’il aurait atteint la vertu suprême ? \newline Le Maître dit : « Ceci n’a rien à voir avec la vertu suprême. Pareil homme serait un saint—même Yao et Shun seraient bien en peine de l’égaler. En revanche, pour ce qui est d’atteindre la vertu suprême, la recette est à la portée de la main : prenez pour guide vos propres aspirations—assurez à votre prochain le sort que vous vous souhaiteriez à vous-même, obtenez pour lui ce que vous souhaiteriez obtenir pour vous-même. -- \textit{\small Les Entretiens, VI. 30, traduction de Pierre Ryckmans, p. 38. } 

\item 24. \newline Qu’est-ce que le ren ? Yan Hui pose la question au Maître, qui lui répond : \newline L’homme de ren est celui qui fait effort sur lui-même pour revenir au rituel ; quiconque s’en montrerait capable, ne serait-ce qu’une journée, verrait le peuple entier honorer son ren. N’est-ce pas de soi-même, et non des autres, qu’il faut en attendre l’accomplissement ? \newline Yan Hui : Pourriez-vous m’indiquer la démarche à suivre ? \newline Le Maître : Ce qui est contraire au rituel, ne le regarde pas, ne l’écoute pas ; ce qui est contraire au rituel, n’en parle pas et, à plus forte raison, n’y commets pas tes actions. Yan Hui : Je ne suis guère intelligent, mais je ferai de mon mieux pour appliquer ce précepte. \textit{\small -- Les Entretiens, XII. 1, traduction d’Anne Cheng, p. 95.  }
\item 25.\newline Fan Chi interrogea le Maître sur la vertu suprême. \newline Le Maître dit : « Être digne dans la vie privée ; diligent dans la vie publique ; loyal dans les relations humaines. Ne pas départir de cette attitude, même parmi les barbares. » \textit{\small -- Les Entretiens, XIII. 19, traduction de Pierre Ryckmans, p. 74.  }
\paragraph{Le yi 義 (le sens du juste)}  
\item 26. \newline Le Maître dit : Dans les affaires du monde, l’homme de bien n’a pas une attitude rigide de refus ou d’acceptation. Le Juste est sa règle.  -- Les Entretiens, IV. 10, traduction d’Anne Cheng, p. 45.  

\paragraph{Le li 禮 (le rituel)}   
5 
\item 27. \newline Le Maître dit : « Une politesse qui n’est pas tempérée par le rituel est fastidieuse ; une prudence qui n’est pas tempérée par le rituel est peureuse ; une bravoure qui n’est pas tempérée par le rituel est violente ; une franchise qui n’est pas tempérée par le rituel est blessante. […] » \textit{\small -- Les Entretiens, VIII. 2, traduction de Pierre Ryckmans, p. 45.  }
\item 28. \newline Le Maître dit : « Avec les lettres pour s’ouvrir l’esprit et les rites pour discipliner, on ne saurait s’écarter du droit chemin. » -- \textit{\small  Les Entretiens, XII. 15, traduction de Pierre Ryckmans, p. 68.  }
\item 29. \newline Le Maître dit : « Ils disent : "les rites par-ci, les rites par-là", comme s’il s’agissait seulement d’ornement de jade et de soie ! Ils disent : "la musique par-ci, la musique par-là, comme s’il s’agissait de cloches et de tambours ! » \textit{\small -- Les Entretiens, XVII. 11, traduction de Pierre Ryckmans, p. 96.  }
\item 30. \newline Le Maître dit : Gouvernez à force de lois, maintenez l’ordre à coups de châtiments, le peuple se contentera d’obtempérer, sans éprouver la moindre honte. Gouvernez par la Vertu, harmonisez par les rites, le peuple non seulement connaîtra la honte, mais de lui-même tendra vers le Bien.  -- \textit{\small  Les Entretiens, II. 3, traduction d’Anne Cheng, p. 33.  }
\paragraph{Le zhi 知/智 (Le discernement, la perspicacité)  }
\item 31. \newline Le Maître dit : Le sage est franc de toute incertitude, l’homme de ren de toute inquiétude, le brave de toute crainte.  \textit{\small -- Les Entretiens, IX. 28, traduction d’Anne Cheng, p. 80.  }
\item 32. \newline Le Maître dit : « L’homme sage aime l’eau, l’homme bon aime la montagne. L’homme sage est actif, l’homme bon est tranquille. L’homme sage est joyeux, l’homme bon vit longtemps. » -- Les Entretiens, VI. 23, traduction de Pierre Ryckmans, p. 37.  
\item 33. \newline Le Maître a dit : -- Zilu, veux-tu que je te dise ce que c’est que savoir ? Savoir que l’on sait quand on sait, et l’on ne sait pas quand on ne sait pas, voilà savoir.  -- Les Entretiens, II. 17, traduction de Jean Levi, p. 10.  

\item 34. Maître Zeng dit : Chaque jour je m’examine plusieurs fois : me suis-je fidèlement acquitté de mes engagements ? Me suis-je montré digne de la confiance de mes amis ? Ai-je mis en pratique ce qu’on m’a enseigné ? \textit{\small -- Les Entretiens, I. 4, traduction de Pierre Ryckmans, p. 13.  }

35. \newline Le Maître dit : Ne crains point de rester méconnu des hommes, mais bien plutôt de les méconnaître toi-même. \textit{\small -- Les Entretiens, I. 16, traduction d’Anne Cheng, p. 32.  }
\item 36. (Voir aussi la citation n° 17)  \newline Fan Chi : Qu’est-ce que le ren ?  \newline Le Maître : C’est aimer les hommes. \newline Fan Chi : Et la sagesse ?  \newline Le Maître : C’est connaître les hommes. (Voyant Fan Chi perplexe) C’est savoir choisir les hommes droits et les placer au-dessus des hommes pernicieux afin que ces derniers s’en trouvent redressés.  \newline 6 Fan Chi, en quittant  le Maître, rencontre Zixia : Je viens de voir  le Maître et de l’interroger sur la sagesse. Il m’a répondu : C’est savoir choisir les hommes droits et les placer au-dessus des hommes pernicieux afin que ces derniers s’en trouvent redressés. Qu’en penses-tu ? \newline Zixia : Quelle richesse dans ces quelques mots ! Lorsque Shun, devenu maître du monde, distingua Gao Yao de la multitude en l’élevant au rang de ministre, les hommes privés de ren s’éloignèrent. Lorsque Cheng Tang, devenu maître du monde, distingua Yi Yin en l’élevant au rang de ministre, les hommes privés de ren disparurent.   \textit{\small -- Les Entretiens, XII. 22, traduction d’Anne Cheng, p. 101.  }  
\paragraph{Le xin 信 (fidélité à la parole donnée, fiabilité, sincérité, confiance)}  
\item 37. \newline Le Maître dit : La honte de l’homme de bien, c’est de voir ses paroles excéder ses actions. \textit{\small  -- Les Entretiens, XIV. 29, traduction d’Anne Cheng, p. 116.  }
\item 38. \newline Le Maître dit : « Mine débonnaire et belles paroles » sont rarement signe de vraie vertu.  \textit{\small -- Les Entretiens, I. 4, traduction de Pierre Ryckmans, p. 13.  }
\item 39. \newline Zizhang demanda comment agir. \newline Le Maître dit : « Parlez avec loyauté et bonne foi, agissez avec honnêteté et prudence, et votre action sera efficace, même parmi les Barbares. Si vous parlez sans loyauté ni bonne foi, si vous agissez sans honnêteté ni prudence, comment votre action pourrait-elle être efficace, même dans votre propre village ? Gardez cette maxime constamment devant les yeux, gravez-la sur le timon de votre char, et votre action sera efficace. » \newline Zizhang l’inscrivit sur sa ceinture.  \textit{\small -- Les Entretiens, XV. 6, traduction de Pierre Ryckmans, p. 85.  }
\item 40. \newline Zigong : Qu’est-ce que gouverner ? \newline Le Maître : C’est veiller à ce que le peuple ait assez de vivres, assez d’armes, et s’assurer sa confiance. \newline Zigong : Et s’il fallait se passer d’une de ces trois choses, laquelle serait-ce ? \newline Le Maître : Les armes. \newline Zigong : Et des deux autres, laquelle serait-ce ? \newline Le Maître : Les vivres. De tout temps, les hommes sont sujets à la mort. Mais sans la confiance du peuple, aucun État ne saurait tenir. -- Les Entretiens, XII. 7, traduction d’Anne Cheng, p. 97.  
\item 41. \newline Zixia dit : « Un gentilhomme fait d’abord régner la confiance et ensuite il peut mobiliser ses gens. Sans cette confiance, ceux-ci pourraient se croire brimés. Un gentilhomme fait d’abord régner la confiance, et ensuite il peut critiquer son souverain. Sans cette confiance, celui-ci pourrait se croire insulté. » \textit{\small -- Les Entretiens, XIX. 10, traduction de Pierre Ryckmans, p. 105. }

 
\end{itemize}