





L’introduction du bouddhisme en Chine
 
Le bouddhisme a été introduit en Chine sous la dynastie Han (202 av. J.- C. - 220 apr. J.-C.)

Au moment de l’arrivée du bouddhisme, il existe déjà sous les Han des traditions et des croyances bien enracinées : le confucianisme devenu la doctrine d’État (2ème s. av. J.-C.) et plus tard l’apparition du taoïsme religieux avec la divinisation de Laozi (2ème s. apr. J.-C.)

Le mythe:
le rêve de l’empereur Ming (r. 58-75) et le Sūtra en quarante-deux sections

La réalité:
Le bouddhisme était présent en Chine bien avant le règne de l’empereur Ming.
Selon une source considérée comme fiable par les historiens, en l’an 2 av. J.-C., un étudiant à l’Académie impériale reçut d’un ambassadeur du roi des Kouchans, la transmission orale de la loi bouddhique.
 
Contexte: la communication de la Chine avec les pays de l’Asie centrale

•	Sous les Han, le mot xiyu   désignait toutes les contrées à l’ouest de l’empire, depuis le Xinjiang actuel jusqu’au monde parthe, en passant par l’Asie centrale.

•	Les Xiongnu (connus en Europe sous le nom de Huns): une population de pasteurs qui nomadisaient dans les steppes situées au nord de la Chine. À la fin du 3ème siècle av. J.-C., ils fondèrent un véritable empire qui s’étendit de la Manchourie, à l’est, jusqu’au Xinjiang à l’ouest et au lac Baïkal, au nord.
 









































https://spooksrus.tripod.com/barbarians/image s/xiongnu_empire_126bce_02_filled.gif
 
Contexte: la communication de la Chine avec les pays de l’Asie centrale
•	La politique offensive et expansionniste de l’empereur Wu (r. 141-87 av. J.-C.)
•	La mission diplomatique de Zhang Qian (mort en 113 av. J.-C.) visant à négocier une alliance avec les Yuezhi contre les Xiongnu.
•	Les Yuezhi: population de pasteurs, les Yuezhi nomadisaient, à l’avènement des Han, sur un vaste territoire correspondant au Xinjiang actuel. Chassés par les Xiongnu vers 170-160, ils nomadisèrent dès lors entre le lac Balkash et la vallée de l’Ili avant de descendre vers la vallée de l’Oxus au sud. Vers 130 avant notre ère, ils franchirent l’Oxus et occupèrent tout le territoire de la Bactriane grecque (sud de la Sogdiane, Bactriane et est de la Margiane). Ils y fondèrent au 1er siècle de notre ère l’Empire kuchan, étendant leur puissance sur toute l’Inde du Nord.
 










































https://en.wikipedia.org/wiki/Zhang_Qian#/media/File:Han_Expansion.png
 
Pays décrits dans les rapports de Zhang Qian (mort en 113 av. J.-C.).
Ceux que Zhang a lui-même visités sont soulignés en bleu.


https://en.wikipedia.org/wiki/Zhang_Qian#/media/File:ZhangQianTravel.jpg
 

 


L’empire kuchan (1er – 3ème siècle de notre ère)
 
L’assimilation du bouddhisme par la culture chinoise
Les grandes étapes:
•	Période de préparation (des origines jusqu’au début du 4ème siècle)
Période marquée par une certaine ambiguïté des premières motivations chinoises fondées sur le désir d’immortalité et sur le goût des lettrés pour les discussions ésotériques; confusion avec le taoïsme.

•	Période d’implantation (4ème – 6ème siècles)
Le bouddhisme se diffuse largement dans une Chine divisée en deux parties: le Nord dominé par des régimes étrangers et le Sud où règnent des dynasties chinoises; appropriation du bouddhisme avec des nouvelles traductions de textes bouddhiques permettant de séparer le bouddhisme du taoïsme.
 
L’assimilation du bouddhisme par la culture chinoise



•	Période de sinisation ou d’assimilation véritable (seconde moitié du 6ème siècle – fin du 10ème siècles), sous les dynasties Sui et Tang
Le bouddhisme pénètre toutes les couches de la société et se sinise entièrement. Les différents courants, ou « écoles », bouddhiques se forment au fil du temps.
 
Période de préparation (des origines jusqu’au début du 4ème siècle)
































Kenneth CH’EN, Histoire du Bouddhisme en Chine, traduit de l’anglais par Dominique Kych, Paris : Les Belles Lettres, 2015, p. 58.
 
Premiers moines étrangers arrivés en Chine investis dans la traduction de textes bouddhiques:

•	An Shigao   : Premier grand traducteur de textes bouddhiques en chinois. D’origine parthe arsacide, An Shigao était, semble-t-il, un prince ayant renoncé au trône pour se faire religieux. Il arriva en 148 à Luoyang, alors capitale des Han postérieurs.
Ses traductions relevaient toutes du hīnayāna dont beaucoup exposent des pratiques méditatives proches de celles du taoïsme.

•	Zhilou jiachan     ou Zhi Chan   : Kuchan arrivé à Luoyang vers 167. A la différence d’Anshigao, Zhi Chan s’intéresse d’abord aux textes de la prajñāparamitā (la perfection de sagesse ou connaissance transcendante), un ensemble de textes relevant du mahāyāna.
 
Obstacles rencontrés dans l’assimilation de la doctrine bouddhique par la culture chinoise:

•	Au niveau des langues et des modes de pensée
Méthode de traduction employée dans cette période: « geyi »   (appariement des notions »
Hajime NAKAMURA, Ways of Thinking of Eastern Peoples: India-China-Tibet-Japan, trad. Philip Wiener, Honolulu: The University of Press of Hawaii, 1971.

•	Au niveau des traditions
Lihuo lun (Dialogues pour dissiper la confusion) de Mouzi (Maître Mou)
 
A la fin de la dynastie Han, deux courants s’étaient déjà dégagés au sein du bouddhisme.

•	Le premier, représenté par l’école du dhyāna, mettait l’accent sur le contrôle de l’esprit, la concentration et la suppression des désirs. L’école se fondait principalement sur les textes traduits par An Shigao et était de nature hīnayāna.
•	À l’opposé, l’école de la prajñā, fondée en grande partie sur les textes traduis par Zhi Chan, était plutôt d’inspiration mahāyāna et se consacrait surtout à sonder, derrière le voile des apparences, la nature de Bouddha et la réalité ultime.
 






Période de division après la chute des Han (220-581)
 
Période des Trois royaumes (220-265)
 

Jin occidentaux (265-316)
 




































https://pandaist.com/blog/en/chinese-dynasty-jin-dynasty-266-420-ce-western-and
 
Période des Dynasties du Nord et du Sud (317-589)
Période d’implantation du bouddhisme en Chine (4ème – 6ème siècles)
 
Période des Dynasties du Nord et du Sud (317-589)
 
Dans le Sud:

•	L’« étude du Mystère », aussi dénommée par le terme « néo- taoïsme »
•	« la causerie pure »
 
Dans le Nord:

L’arrivée de Kumārajīva (344 ou 350 – 413) ouvre une nouvelle phase de la traduction des textes bouddhiques en langue chinoise.
Parmi les textes qu’il a traduits:
•	Le Sutra du Lotus; le Sutra de Vimalakīrti
•	les trois traités fondamentaux de l’école Madhyamika fondée par Nāgārjuna: le Traité en 100 versets, le Traité de la voie moyenne et le Traité des douze portes.
 

Période de sinisation ou d’assimilation véritable (seconde moitié du 6ème siècle – fin du 10ème siècles), sous les dynasties Sui et Tang
 




























Dynastie des Sui (581- 618)









https://www.worldhistory.org/trans/fr/1-15273/dynastie-sui/
 







Dynastie des Tang (618-907)
