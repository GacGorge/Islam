\chapter{Zhuangzi}

\section{Zhangzi : vie}


\paragraph{A réellement existé; un ermite}

\paragraph{bcp d'anecdotes} Tortue. préfère vivre libre qu'enchainé.

\paragraph{grande qualité littéraire} 33 chapitres (interne de 1 à 7, attribué à Zhuangzi), de 8 à 22 (chapitres externes, on est pas sur), et de 23 à 33 (\textit{mixte}, encore plus compliqué). Le style est discursive et peu homogène, quelques poèlme/.

Au début des Hans, bcp de textes transmis. Les lettrés ont fait un gros travail d'organisation et regroupement.


\paragraph{lien avec le Laozi} Critique l'éducation et le fondement de la morale. Pronent une vie retirée. Mais il y a aussi des \textit{différences notables}. Le Laozi a eu des conséquences politiques. 


\paragraph{discours sur l'identité des choses} Les choses sont équivalentes car on met tout à plat. \textit{de la mise à plat qui mette les choses équivalentes}. Il faut dépasser le niveau de la relativité de toute chose pour atteindre le Dao.




\section{Texte}
  Version citée :  Jean LEVI (trad.), Les œuvres de Maître Tchouang, Paris : Éditions de l'Encyclopédie des Nuisances, 2006.   


  \subsection{Tout est relatif}
  \begin{singlequote}
      1. Il n’y a pas de chose qui ne soit un « cela » ; mais en même temps il n’est chose qui ne soit un  « ceci ». Mais je ne puis concevoir le point de vue du « cela » ; je ne connais qu’à partir du « ceci ». C’est pourquoi je dis que tout « cela » naît du « ceci » et que tout « ceci » suppose un « cela ». De là découle la théorie de l’engendrement réciproque du « ceci » et du « cela ». […] Toute dénomination juste est en même temps fausse et, réciproquement, toute dénomination fausse est en même temps juste. Si bien que toute qualification est à la fois juste et fausse et fausse et juste. C’est pourquoi le sage ne s’abandonne pas au prestige fallacieux des mots, mais se laissant illuminer par le Ciel, il se conforme aux circonstances. Tout « ceci » est donc aussi un « cela », tout « cela » un « ceci ». Tout « cela » détermine un ensemble d’affirmations et de négations, de même que tout « ceci » détermine son propre ensemble de négations et d’affirmations. \textbf{Le lieu où le « ceci » et le « cela » ne rencontrent plus leur contraire constitue le pivot du Tao.  }
      \textit{\small - Chapitre 2, « Discours sur l’identité des choses », traduction de Jean Levi, p. 22. }
  \end{singlequote}
  

Si on se place au niveau de la relativité, il n'y a pas de vérité suprême, tout est un point de vue. Tout dépend de la perspective qu'on adopte.

    
  \begin{singlequote}
      2. - Où se trouve le Dao ? demanda un jour le Maître du Mur de l’Est à Maître Zhuang. - Partout, répondit Maître Zhuang. - Sois plus précis ! - Alors, dans une fourmi. - Pourquoi dans quelque chose de si bas ? - Dans un brin d’herbe. - Pourquoi encore plus bas ? - Dans cette tuile. - Pourquoi encore plus bas ? - Dans l’excrément et l’urine. Le Maître du Mur de l’Est ne répond plus.  - Chapitre 22, traduction basée sur celle de Jean Levi. 
  \end{singlequote}
  
    Des réponses très choquantes. On a l'habitude de penser le Dao comme sublime. Mais en fait, tout cela est un jugement. Le Dao est omni-présent. 

\subsection{il va jusqu'à confondre le rêve et la réalité}
    
  \begin{singlequote}
      3. Un jour Tchouang Tcheou rêva qu’il était un papillon froufroutant, qui, tout à sa joie, donnait  libre cours à ses désirs, sans savoir qu’il était Tchouang Tcheou ; puis, brusquement, il s’éveilla, retrouvant la lourdeur de son corps ; il se demanda s’il était Tchouang Tcheou qui avait rêvé qu’il était un papillon ou un papillon qui se rêvait Tchouang Tcheou.  \\
      \textit{\small - Chapitre 2, « Discours sur l’identité des choses », traduction de Jean Levi, p. 30.}
  \end{singlequote}

  Confusion totale. Tchouang rêve qu'il est un papillon. mais peut être l'inverse

  Consiste à montrer qu'il y a des choses infinniment plus grande que ce que l'homme peut apercevoir. Le langage ne sert à rien. Il nous empêche de voir la vérité suprême.
    
  \begin{singlequote}
      4. Kong-souen Long, le sophiste, demanda au prince Meou de Wei : - […] Je croyais posséder la plus haute dialectique et voici qu’en écoutant ce tantôt les paroles de votre maître Tchouang, je me suis trouvé complètement désorienté. Je ne sais si c’est dû à quelque faille dans ma doctrine ou à la limitation de mon intelligence, mais devant lui je n’ai pu ouvrir le bec. Pouvez-vous m’en expliquer la raison ?  

      Le prince Meou s’appuya sur son accoudoir, poussa un grand soupir, leva les yeux au ciel et dit en éclatant de rire :  
      - « La grenouille d’un puits effondré dit à la grande tortue de la mer Orientale : « Comme je suis heureuse ! je puis sortir de mon trou et sautiller sur la margelle, je puis me lover dans les anfractuosités de la paroi. Si je nage dans l’eau, celle-ci soutient mes aisselles et caresse mes cuisses ; je puis aussi me vautrer dans la vase, y enfouissant mes chevilles et mes pattes, quand je contemple autour de moi toutes les autres bestioles, larves, crabes, têtards et mollusques, je me dis que je suis vraiment la maître d’un trou d’eau dont je dispose à ma guise, allant et venant librement entre les murs effondrés dont je suis la seule propriétaire. J’aimerais vous en faire les honneurs. »  
      
      Quand la tortue voulut mettre sa patte gauche dans le puits, celle de droite le remplissait déjà. Elle déclara avec hauteur : « On ne peut se représenter l’immensité ni la profondeur des océans. Aucune calamité, inondation ou sécheresse n’affecte le niveau de la mer, parce que sa masse liquide est infinie. Telle est la joie grandiose de la mer Orientale dont j’ai fait mon logis. » La petite grenouille se rencogna dans son trou et demeura tout impressionnée et frissonnante.  
      
      Les pauvres sires de votre espèce, qui, sous prétexte qu’ils jonglent avec la logique en faisant passer le faux pour le vrai, se targuent de disséquer la doctrine d’un Maître Tchouang, me font penser à un pou qui essaierait de porter une montagne sur son dos, ou à une limace qui ferait la course avec le fleuve Jaune. Absurde ! Celui dont l’intelligence est incapable d’embrasser le fond des choses et d’épuiser les subtilités de l’univers, car il ne cherche que de petits succès dans des discussions dialectiques, n’est-il pas comme la chétive pécore du puits ?   - Chapitre 17, « La crue d’automne », traduction de Jean Levi, p. 140.  
  \end{singlequote}
  
    
  \begin{singlequote}
      5. Le duc Huan lisait un livre. Le charron qui travaillait à sa roue en bas des degrés monta le trouver : - Que lisez-vous ? - Les paroles des saints. - Sont-ils vivants ? - Ils sont morts.   - Alors ce que vous lisez, ce sont leurs déjections. - Comment ! un charron ose discuter de ce que lit son seigneur ! éructa le duc. Si tu parviens à justifier ton assertion, je te fais grâce, sinon je te coupe la tête ! - Quand on façonne une roue, trop doux, il y a du jeu, trop fort, les pièces s’imbriquent mal. Ni trop doux ni trop fort, il faut l’avoir dans les doigts. L’esprit se contente d’obéir. Il y a dans mon activité quelque chose qui ne peut s’exprimer par des mots, aussi n’ai-je pu le faire comprendre à mon fils. J’ai soixante-dix ans bien sonnés et je suis encore là à faire des roues en dépit de mon grand âge. Ce que les anciens n’ont pu transmettre est bien mort et les livres que vous lisez ne sont que leurs déjections.   
      
      \textit{\small - Chapitre 13, « L’action du Ciel », traduction de Jean Levi, p. 114.  }
  \end{singlequote}


  \paragraph{roue} On prenait des planches et en les assemblant, il faut trouver la juste mesure. \textit{Ni trop doux ni trop fort}

\paragraph{Ecriture}
  Les gens apprécient les discours et les paroles. Ce qui est écrit a le moins de valeur. Ce qui est le plus important, c'est le vécu, l'immédiat, et on oublie dans la lecture ce qui est important.

  \paragraph{la main trouve et l'esprit le raconte} Sortir l'homme de sa représentation du monde.

  \paragraph{mais que fait on dans la vie} Pour Zhuangzi, la chose qui est du sens, c'est d'essayer de faire union avec le cosmos. Le reste (étude...), n'est pas important. 



    
  \begin{singlequote}
      6. Depuis que Chouen a semé le désordre dans le monde en prônant la charité et la justice,  l’humanité s’est ruée à corps perdu à leur poursuite, n’est-ce donc pas la preuve que ce sont elles et elles seules qui ont perverti sa nature ? Pour être plus précis, depuis que les Trois Dynasties ont pris en main le destin de l’empire il n’est personne dont la nature profonde n’ait été corrompue par un objet extérieur. Le manant est prêt à se sacrifier en vue du profit, le noble du renom, le Grand officier de l’éclat de sa maison, le saint du bien de l’empire ; si chacun poursuit des buts différents, et si des réputations opposées s’attachent à chacun, il n’en reste pas moins qu’ils s’accordent tous à ruiner leur nature et à sacrifier bêtement leur personne.          Il y avait un bon valet et un mauvais valet. Tous deux perdirent les moutons dont ils avaient la garde. Le bon laissa échapper ses bêtes parce qu’il avait le nez plongé dans ses livres ; le mauvais, lui, perdit les siennes parce qu’il jouait aux dés. Toutes différentes qu’elles étaient, leurs occupations eurent un seul et même résultat : la perte des moutons.   
      - Chapitre 8, « Pieds palmés », traduction de Jean Levi, p. 75.  
  \end{singlequote}
      \paragraph{condamnation des confucéens : Chouen} Proposer l'ordre par la Loi, c'est n'apporter que le désordre

      l'homme doit savoir se préserver
    
  \begin{singlequote}
      7. Tseu-kong, l’un des disciples favoris du Maître [Confucius], alors qu’il revenait d’une  mission au Tch’ou, vit sur le bord de la route menant au Tsin un vieil homme arroser son potager au moyen d’une jarre qu’il allait remplir au puits dans lequel il descendait par une galerie. C’était beaucoup d’efforts pour peu de résultats.          Tseu-kong s’arrêta et lui demanda s’il n’aimerait pas utiliser une machine qui lui permettrait d’arroser cent plates-bandes en une journée, donnant beaucoup de résultats avec peu d’efforts. L’homme leva les yeux sur l’intrus et lâcha un : « Dis toujours. »         Tseu-kong expliqua qu’il suffisait de faire un trou dans une perche lestée à l’arrière, dont on se servirait comme d’un levier.  - On tire, et l’eau coule à flots. C’est ce qu’on appelle une bascule, conclut Tseu-kong.         L’homme changea de contenance, manifesta de la colère, émit un ricanement, puis déclara : - J’ai entendu dire ceci par mon maître : « Les machines créent les activités mécaniques.  Les activités mécaniques mécanisent le cœur. Qui a un cœur mécanique dans la poitrine perd sa candeur native ; qui a perdu sa candeur native ne saurait connaître la paix de l’âme. » Je suis parfaitement au courant des avantages de cet instrument mais j’aurais honte de m’en servir !  - Chapitre 12, « Ciel et terre », traduction de Jean Levi, p. 100.
  \end{singlequote}


    Cette activité mécanique mécanise le coeur. Une fois qu'on cherche la vitesse, alors on est entravé par ces machines. Dans quel état est ce qu'on vit. 

\subsection{L'homme authentique}
    
  \begin{singlequote}
      8. Les hommes authentiques ne s’insurgent pas de leurs faiblesses, ne forcent pas le succès et  n’ourdissent jamais de plans.          De tels hommes ne regrettent pas de s’être trompés ni ne se glorifient d’avoir vu juste. De tels hommes gravissent les plus hauts sommets sans trembler, entrent dans l’eau sans se mouiller, traversent les flammes sans se brûler. Ainsi sont ceux dont l’esprit est capable de s’élever dans la nue jusqu’au Tao !         L’homme authentique de jadis avait un sommeil sans rêves, un réveil sans tourments, une nourriture sans saveur. Profonde, profonde était sa respiration ! L’homme authentique respire avec les talons, l’homme ordinaire respire avec le gosier. […]         Les hommes authentiques de jadis ne savaient pas ce que signifiait se réjouir de la vie, pas plus qu’ils ne savaient ce que signifiait avoir peur de la mort, aussi nulle joie en entrant, nulle protestation en sortant. Insouciants ils s’en venaient, insouciants ils s’en allaient. Gardant en mémoire le pourquoi de leur origine, ils ne se tourmentaient pas du pourquoi de leur trépas. Ils étaient heureux de ce qu’ils recevaient en partage et le restituaient sans un mot à leur disparition. Voilà qui s’appelle ne pas forcer le cours naturel des choses par l’intervention de la conscience, ni seconder la part céleste qui est en soi par l’humain. C’est à ça que se reconnaît l’homme authentique. 
      \textit{\small - Chapitre 6, « À l’école du premier ancêtre », traduction de Jean Levi, p. 54.  }
  \end{singlequote}
     Le sage est hors du monde, exempt de toute inquiétude métaphysique. Il a l'esprit libre. \textbf{c'est l'homme authentique}.
    Toutes ces pratiques servent à faire l'homme authentique. 

     
  \begin{singlequote}
      9. Le cuisinier Ting était en train de dépecer un bœuf pour le prince Wen-houei. Wouah ! il  empoignait de la main l’animal, le retenait de l’épaule et, les jambes arc-boutées, l’immobilisait du genou. Wooh ! le couteau frappait en cadence comme s’il eût accompagné la grande pantomime rituelle de la Forêt des mûriers ou l’hymne solennel de la Tête de lynx. - Admirable ! s’exclama le prince en contemplant ce spectacle, j’aurais jamais cru que l’on pût atteindre pareille virtuosité ! - Vous savez, ce qui m’intéresse, ce n’est pas tant l’habileté technique que l’être intime des choses. Lorsque j’ai commencé à exercer j’avais tout le bœuf devant moi. Trois ans plus tard, je ne percevais plus que les éléments essentiels, désormais j’en ai une appréhension intuitive et non pas visuelle. Mes sens n’interviennent plus. L’esprit agit comme il l’entend et suit de lui-même les linéaments du bœuf. Lorsque ma lame tranche et sépare, elle suit les fentes et les interstices qui se présentent, ne touchant ni aux veines ni aux tendons, ni à l’enveloppe des os ni bien entendu à l’os lui-même. Les bons cuisiniers doivent changer de couteau chaque année parce qu’ils taillent dans la chair. Le commun des cuisiniers en change tous les mois parce qu’il charcute au petit bonheur. Moi, après dix-neuf ans de bons et loyaux services, mon couteau est comme neuf. Je sais déceler les interstices et, le fil de ma lame n’ayant pratiquement pas d’épaisseur, j’y trouve l’espace suffisant pour la faire évoluer. Quand je rencontre une articulation, je repère l’endroit difficile, l’action délicate de la lame, les parties se séparent avec un bruissement léger comme de la terre qu’on déposerait sur le sol. Mon couteau à la main, je me redresse, je regarde autour de moi, amusé et satisfait. Après avoir nettoyé la lame, je la remets au fourreau.  - Merveilleux ! s’écria le prince, je viens enfin de saisir l’art de nourrir sa vie !  - Chapitre 3, « Principes pour nourrir sa vie », traduction de Jean Levi, p. 32.
  \end{singlequote}
 
  \paragraph{technique de longévité} maitrise du souffle et \textit{essence}. Son soule est tellement forte qu'il va dans tout le corps.
  