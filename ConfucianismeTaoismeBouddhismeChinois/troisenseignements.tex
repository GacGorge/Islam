\chapter{Introduction : les 3 enseignements}

\mn{Cours du 20/09/23}


\section{présentation du glossaire}
\paragraph{l'opposition (Dieu / Satan) n'est pas chinoise} Opposition complémentaire. 
voir p. \pageref{DefGlossaire} pour voir les opposées : 

\begin{Ex}[Opposition]
    Voir Yi à li. L'homme de bien doit s'interesser aux questions morales, alors que l'homme pauvre ne va s'intéresser qu'à la richesse.
\end{Ex}.

\begin{Def}[Sanjiao 三教]
    les Trois Enseignements (le confucianisme, le taoïsme et le bouddhisme) 
\end{Def}


\section{le confucianisme}

\begin{Def}[rujiao  儒教]
 sous les Zhou occidentaux, relations familiales. Relations ritualisées.    
\end{Def}


\section{Quelques caractéristiques saillantes de la pensée chinoise}

\begin{singlequote}
    Aucun philosophe occidental ne songerait à nier le rôle de la Grèce dans les origines et les orientations dominantes de nos traditions philosophiques. La claire distinction de l’intelligible et du sensible, de la vérité et de l’opinion, de l’être et du devenir, a été décisive et fondamentale. Or les Chinois n’ont pas admis l’idée d’un domaine propre à la pensée abstraite et au raisonnement logique sur des abstractions. Ils ignorent toute théorie et tout système, et les œuvres de leurs penseurs se présentent le plus souvent comme une suite de notes ou comme des commentaires à des œuvres plus anciennes. Leurs réflexions les plus remarquables sont généralement associées à des préoccupations d’ordre pratique: comment, par exemple, se perfectionner soi-même ou parvenir à l’harmonie sociale?
 
À vrai dire, il n’y a pas en Chine de philosophes de profession. On n’y aurait donc affaire qu’à une forme encore balbutiante de philosophie et tout au plus pourrait-on parler de sagesse.
Et, en effet, il y aurait abus à juger des conceptions chinoises en fonction de nos propres critères, sans admettre qu’ils puissent être remis en cause. Or, c’est à cette remise en cause que peut inciter l’examen d’un certain nombre de thèmes fondamentaux de la pensée chinoise qui forment un ensemble dont la cohérence est manifeste.
\textit{Jacques Gernet, « Introduction à la pensée chinoise », dans Sylvain AUROUX (dir.), La pensée chinoise,
Paris : Quadrige, 2017, p. XXIII.
}

\end{singlequote}

\begin{singlequote}
    Si on devait caractériser d’un mot ce qui fait l’originalité de cette pensée par rapport à nos traditions philosophiques– et même, de façon plus générale, par rapport à nos habitudes mentales–, c’est la notion d’inclusion ou de logique de l’inclusion qui conviendrait le mieux. La pensée chinoise répugne à opposer des contradictoires et à exclure. Ce n’est pas qu’elle ignore le principe de contradiction, dont elle fait usage aussi bien que nous, mais elle ne lui a pas accordé le rôle privilégié que nous lui attribuons. Le thème mythique de la lutte entre natures ennemies (dieux et titans, lumière et ténèbres, Dieu et Satan), ce thème n’est pas chinois. Pour la pensée chinoise, ce sont des opposés complémentaires, non exclusifs les uns des autres, qui forment la trame du monde.\textit{-- Jacques Gernet, ibid., p. XXVI.}
\end{singlequote}


\section{confucianisme - rujiao    }
Confucius : latinisation de Maître Kong, Kongfuzi.  551-479 av. J.-C.


\paragraph{Sous les zhou} la guerre est ritualisée et permet de régler les conflits. 
Mais dans la seconde moitié de cette dynastie, les liens de parenté sont distendus et font d'avantage la guerre pour développer leurs territoires. C'est pendant cette période que Confucius vit et voit les traditionnelles normes s'écrouler.




\paragraph{recueil de poèmes qui fait partie de l'éducation de l'élite, un livre sur les rites...} 5 livres. Forme le noyau de la tradition confucéenne. Etudié jusqu'à début du XX.

\paragraph{débat sur confucius} Pour certains, il a vraiment existé avec une fonction politique, en plus de son enseignement. D'autres chercheurs (américains) pensent qu'il n'a pas existé.

\paragraph{université impériale} on n'enseigne que le Confucianisme. Importanced reso. Les Hans veulent gouverner non par les \textit{lois } mais par les \textit{vertus}. Idéal confucianiste. Les fonctionnaires, quand ils arrivent dans leur poste, la première chose qu'ils font, c'est d'établir une école pour transmettre les vertus.

\paragraph{doctrine officielle sous les Han}{Les Hans, légalisme à l'extérieur, confucianisme à l'intérieur}
les Hans veulent établir un système cohérent de la Foi  : rang de chaque dieux.
 Au IIème siècle avant notre ère (début de la dynastie Han), le confucianisme devint la doctrine officielle de l’État. Sous les Han (202 av. J.-C. – 220 apr. J.-C.), l’État commença à puiser dans la tradition confucéenne pour organiser les rites sacrificiels impériaux afin d’assurer la prospérité de l’empire.
Les réformes rituelles furent achevées au milieu de la dynastie.  
l’une des dimensions de la religiosité du confucianisme
\begin{Synthesis}
    On voit la volonté de contrôler le pouvoir des Hans.
    On pourrait parler d'une religion \textit{civique}, insistant sur les rites sociaux et les vertus (voir religion civique Romaine). 
\end{Synthesis}
\mn{voir cours de christologie et culture, la confrontation avec la \textit{religio} romaine p. \pageref{Def:Religio}.}

\paragraph{une religion ou pas ? } Ce qui rattache à une religion : les rites; Ce qui ne rattache pas à une religion : le \textit{ciel} rattache à la nature et non à Dieu. 

\paragraph{gros travail des philosophes Han} pour organiser les savoirs.

\paragraph{renouveau Song. \textit{le neoconfucianisme}} La dynastie Song (960-1279), grand moment de renouveau du confucianisme sur le plan théorique grâce aux emprunts faits au bouddhisme et au taoïsme  le néoconfucianisme devint la nouvelle idéologie d’État et le restera jusqu’au début du 20ème siècle.
 Floraison et systématisation des rites familiaux


\section{Taoisme}

\begin{Def}[daojiao 道教]
    remonte à Laozi, avec deux textes fondamentaux : 
    \begin{itemize}
        \item le \textit{Daode Jing} : le Livre de la Voie et de la Vertu, IV-III s av JC
        \item le Zhuangzi
    \end{itemize}
\end{Def}

\paragraph{divinisation de Laozi} au 2ème siècle de notre ère, avec l'apparition du taoïsme et de la religion. 

\paragraph{Certains membres de l'aristocratie commencent à lui rendre hommage.}
Les transformations du taoïsme à partir du 3ème siècle . religion acceptée par la haute société. Apparition des premiers monastères taoïstes subventionnés par l’aristocratie.

\paragraph{Les pratiques de Longue Vie}   Ge Hong (283-343). Voir plus loin.


\paragraph{Tang et song} La dynastie Tang (618-907), grande période d’épanouissement du taoïsme.
 Sous la dynastie Song (960-1279), développement de l’alchimie intérieure (neidan  ) par opposition à l’alchimie extérieure (waidan   )

\paragraph{Une notion fondamentale : la voie}
\begin{Def}[dao 道]
    Voie, chemin ; Voie à suivre (en morale, en politique); règle (morale); voie morale; principe (métaphysique) ;      Le Dao : la Voie, vérité ultime ou réalité ultime. La Voie qui ne peut être appréhendée par l’esprit discursif, mais est manifestée ds le devenir naturel des Dix mille êtres. Réalité suprême qui transcende les modalités sensibles et non sensibles de l’être, mais que l’on connaît par l’expérience qu’en donne la pratique de la vertu (德 de).
\end{Def}

\begin{Def}[daode 道德]
    vertu, morale, moralité, bonnes mœurs ; (Philos. chin. – Tao.) la Voie et la vertu / la Voie et sa vertu
\end{Def}
Vérité qui se site au delà de toute différentiation.  Dimension absolue. 


\subsection{Daode jing}
\begin{Def}[Daode jing 道德經]
    Livre de la Voie et de la vertu / Classique de la Voie et de sa vertu  
\end{Def}

Le livre parle de ce que c'est la voie mais aussi de \textit{la pratique} pour l'atteindre ou \textit{y retourner}. On est toujours "dispersé" par l'environnement. Pour retrouver la vérité, il faut diriger son esprit vers l'intérieur.

\paragraph{Une idée de \textit{non-agir}}

dans le sens de non gouvernement de soi. Mais aussi pour la politique, l'empereur ne doit pas trop en faire et laisser les gens faire, car ils connaissent la nature.

\subsection{le Zhuangzi}

Il est plus probable que Zhuangzi ait existé, IVème avant notre ère. Il reprend la notion de la voie, le non-agir et la sérenité. Mais il n'aborde plus la dimension politique.

l'homme peut se transformer.

Le Zhuangzi est plus poétique, parle d'un grand sage qui s'intègre dans la nature, alors que le Daode jing est plus sec ("des proverbes" ?).

\paragraph{Les pratiques de Longue Vie immortalité - alchimie} un des objectifs des Taoistes. A la fin des Hans, des méthodes très concrètes pour les taoistes. \textit{Elixir} très sophistiqué pour vivre l'immortalité (Pb, Or, Mercure). \sn{des empereurs Tang ont avalé cet elixir et en sont morts.}

\begin{Def}[jing 精]
    surchoix, élite ; pur ; fin, subtil, délicat, raffiné. (Philos. chin.) Essence : puissance de vie façonnant et maintenant spécifiquement les êtres. Sans forme mais pourvues de qualités, les essences sont la base des formes corporelles et physiques comme des formes mentales et psychologiques. Sperme. (Tao.) 1. Essence : la forme la plus subtile de la vie. 2. Énergie sexuelle ; force vitale. 3. Éléments liquides et yin du corps. 4. Premier stade de l’œuvre alchimique. 
\end{Def}

\paragraph{Ge Hong. } Comment éviter de perdre l'énergie de l'essence afin de retourner à l'origine de la vie. 

\paragraph{des révoltes paysannes qui ont entrainés la disparition des Han} diffusion du Taoisme. 


\paragraph{5ème siècle : monastère} payé par les aristocrates, intéressés par l'elixir de longue vie. Et ainsi, une religion de paysans touchent l'aristocratie. 

\paragraph{XIII : une image forte, alchimie intérieur} On se rend compte du danger de l'elixir \textit{extérieur}. Le nom des différents métaux devient des symboles. Le feu devient l'esprit. L'eau. A travers la méditation, on arrive à diriger notre énergie pour que les différents éléments (Or, Argent, Mc) qui sont symboliquement dans notre corps, fusionnent. Mettre l'énergie vers le nombril.



\section{Le Bouddhisme - fojiao}

\paragraph{Apparition en Chine au Ier siècle ap. JC}


\begin{itemize}
 \item 	Religion apparue en Inde au 6ème ou 5ème siècle av. J.- C. et introduite en Chine au 1er siècle apr. J.-C., dans la seconde moitié de la dynastie Han.
 \item 	Entre le 3ème et le 6ème siècles, propagation rapide du bouddhisme sur tout le territoire chinois.
 \item Deux courants du Mahāyāna (Grand véhicule) introduits en Chine poseront les bases pour les écoles bouddhiques chinoises : le Mādhyamaka (la Voie du milieu) et le Vijñanavāda (Rien que
connaissance)
 \item De la fin du 6ème siècle au 10ème siècle (dynasties Sui et Tang) : formation et développement des écoles bouddhiques proprement chinoises. Le bouddhisme est entièrement sinisé.
écoles les plus importantes: Tiantai (Terrasse céleste), Jingtu (Terre pure), Chan.
\end{itemize}


\section{religions en Chine}
\paragraph{Le problème avec le concept de « religion »:}

\begin{itemize}
    \item  	Le concept occidental moderne de religion:
un système structuré de croyances et de pratiques, séparé de la société, qui organise les fidèles en Église.
    \item  Il n’existe pas en chinois d’équivalent précis du concept occidental moderne de la « religion ».
    \item  	un néologisme adopté du japonais : zongjiao  
    \begin{Def}[zongjiao - 宗教 ]
    Religion au sens moderne, néologisme du japonais  Shūkyō
\end{Def}
    \item  	Ce mot s’impose à l’usage à partir de 1901 dans la langue chinoise.
\end{itemize}

 
\subsection{Les Cultes locaux - Religion Populaire}

\begin{Ex}[temple du dieu des Murs et des Fossés]
    
\end{Ex}

\begin{Ex}
    
\end{Ex}


\begin{Def}[ religion chinoise par Vincent Goossaert :]
    \begin{itemize}
        \item 	Un système cohérent, de nature englobante, non exclusive.
\item 	Elle comprend l’ensemble des formes de pratique religieuse individuelle (méditation, techniques de salut, techniques du corps dont les art martiaux, accès à la connaissance et à la révélation par la transe et l’écriture inspirée) et collective (cultes aux saints locaux, aux ancêtres, rites funéraires)
\item 	Ces pratiques s’inscrivent dans le cadre de la cosmologie chinoise.

    \end{itemize}
\end{Def}

\begin{Prop}
La religion chinoise existe sans avoir de nom propre, parce qu’elle n’a pas de structure ecclésiale ni d’autorité dogmatique globale.


\end{Prop}

Elle rassemble l’ensemble des formes de la vie religieuses en Chine, à l’exception de certaines religions d’origine étrangère qui, parce qu’elles revendiquent une adhésion exclusive et un monopole de la vérité, n’ont pu s’y intégrer : les trois monothéismes, islam, judaïsme, christianisme. \sn{« Le concept de religion en Chine et l’Occident », Diogène, 2004/1, n°205, p. 11-21}

\section{Le 20ème siècle comme tournant historique}
 

Avant cette époque, la religion chinoise était déjà caractérisée par la diversité, mais elle était structurée autour d’un centre de contrôle : l’État politico-religieux.
\begin{itemize}

    \item  	Cette collusion avec le pouvoir conféra au confucianisme un statut privilégié dans le champ religieux.
    \item	Face aux cultes locaux nombreux et divers, l’État essaya de distinguer l’orthodoxe de l’hétérodoxe en établissant le registre des  

  \end{itemize}

\paragraph{Hétérodoxie et orthodoxie} ∗ La frontière entre l’orthodoxe et l’hétérodoxe est pourtant assez floue. Quels sacrifices sont reconnus par l’État? C’est une question en constante négociation entre la population et le pouvoir.

\paragraph{après 1912}Après 1912, fin du régime impérial.
Les rituels et les sacrifices confucianistes ont perdu leur place dans la politique.

\paragraph{statut de confucianisme} Quel statut accorder au confucianisme aujourd’hui? Débat toujours en cours
\paragraph{Distinction entre la religion et la superstition.}
Les cinq religions reconnues: catholicisme, protestantisme, islam, bouddhisme, taoïsme.

\begin{itemize}
    \item  	Aujourd’hui: réhabilitation des cultes locaux : la catégorie de
« religion populaire » vient remplacer celle de « superstition » dans les discours académiques.
   \item 	Dans les faits, l’idée de superstition reste ancrée dans l’esprit des chinois sans que celle-ci puisse être définie de manière rigoureuse.
\end{itemize}

