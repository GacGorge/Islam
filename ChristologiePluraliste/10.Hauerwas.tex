\Chapter{La fécondité sociale du Christ II}{La christologie postlibérale de Hauerwas}

\section{Eléments bibliographiques}
\begin{itemize}
    \item BOSS, M. – EMERY, G. et GISEL, P. (éd.), Postlibéralisme ? La théologie de George Lindbeck. thèse en Français. 
et sa réception, Genève 2004.
    \item CHÉNO, R., Dieu au pluriel. Penser les religions, Paris 2017.
    \item HAUERWAS, S., « Jesus : The Story of the Kingdom » dans A Community of Character.
    \item Toward a Constructive Christian Social Ethic, Notre Dame (Indiana) 1981, 36-52.
    \item HAUERWAS, S. –WILLIMON, W. H., Etrangers dans la cité, Cerf, Paris 2016 (la première
édition en anglaise remonte à 1989). Pas forcément terrible. 
    \item LINDBECK, G. A., La nature des doctrines. Religion et théologie à l’âge du postlibéralisme,
tr. par M. HEBERT, Paris 2002.
    \item MICHENER, Ronald T., « The Community ethics of Stanley Hauerwas » dans Postliberal
Theology. A Guide for the perplexed, Londres – New York 2013, 72-77.
\end{itemize}



\section{Introduction}
\section{1. Qui est Stanley Hauerwas ?}

Stanley Hauerwas, né le 24 juillet 1940 à Dallas (Texas), est un théologien méthodiste et professeur de droit américain, spécialiste des questions d'éthique.
 
Influencé notamment par John Howard Yoder, H. Richard Niebuhr et Alasdair MacIntyre, Stanley Hauerwas est célèbre pour son pacifisme radical ; il a été qualifié en 2001 par la revue TIME de « meilleur théologien des États-Unis » .

 
 
Témoignez !
\begin{quote}
    « Le témoignage est incontournable pour ceux d'entre nous qui se réclament comme chrétiens parce que le Dieu que nous adorons n'est pas une vérité générale qui pourrait être connue en dehors de ceux qui l'adorent et ont été appelés à participer à son royaume. Il n'est pas accidentel que Jésus appelle des disciples pour témoigner de lui. L'être disciple et le témoignage, ensemble, constituent la christologie ; Jésus ne peut pas être connu sans témoin qui le suivent ; l'être disciple et le témoignage nous rappellent ensemble que le Christ que nous suivons et de qui nous portons témoignage défie toute génération. Le témoignage des disciples, de plus, a une structure définie. 
Au chapitre 10 de Matthieu, Jésus donne à ses disciples des instructions précises pour les envoyer aux brebis perdues d'Israël. Ils ont à aller témoigner de Jésus, dans lequel le royaume des cieux s'est approché. Et ils ont un travail à accomplir : soigner les malades, relever les morts, purifier les lépreux et chasser les démons. De plus, ils ne recevront aucune compensation pour leur travail. Ils ne devraient pas être dissuadés par ceux qui n'accueillent pas leur mission, mais plutôt voir le rejet comme une invitation à aller vers d'autres. Leur témoignage est le résultat d'un don qu'ils ont reçu. L'histoire qu'ils racontent est leur histoire dans la mesure où ils racontent ce qui leur est arrivé. Mais ils témoignent à travers eux du Dieu qu'ils croient avoir rencontré en Jésus-Christ. »
\end{quote}


 

En chemin
\begin{quote}
   « Les Évangiles sont limpides quant au fait que les premiers disciples n'avaient pas la moindre idée de ce qui les attendait en choisissant de suivre Jésus.
« Suivez-moi », c'est par cette phrase simple que Jésus a poussé des gens ordinaires à se manifester et à s'engager sur une route tortueuse où l'inattendu les guettait constamment. Ce n'est pas une coïncidence si les auteurs des Évangiles ont choisi d'évoquer la vie de Jésus sous forme d'un voyage : Et Jésus vint, il prit son disciple là où il était. À partir de ce moment, il commença à leur enseigner que, etc. 
\end{quote}


L'Église existe aussi aujourd'hui comme pérégrine, elle est comme une tête de pont dans une société incroyante. La culture occidentale a perdu le sens du voyage et de l'aventure en perdant la foi en un au-delà de la culture, ce qui restreint toujours plus son horizon à l'auto-préservation et à l'affirmation de soi.
Vivre dans la communauté chrétienne n'est pas confortable. À force de voir ce qui nous importe le plus être l'objet d'attaques constantes, de courir le risque de voir nos propres enfants s'éloigner de nous et d'être considérés comme une menace par une culture athée, cette dernière assujettissant tout le monde au nom de la Liberté et de l'Égalité, la communauté chrétienne peut être perçue par ses membres comme constituant un véritable défi. »
 

\subsection{La critique de la christologie libérale au nom de la dimension sociale de la
christologie}



\subsection{Le constat de la séparation entre le Christ et la dimension sociale}

\begin{quote}
    « ‘Quelle est la relation entre la christologie et l’éthique sociale ?’. Que nous puissions nous poser une
telle question indique que quelque chose est faux. La question présuppose que la signification et la
vérité de l’engagement à la suite de Jésus peuvent être déterminées en dehors de sa signification
sociale. En revanche, je veux montrer que ce pour quoi Jésus est digne de foi, n’est explicable
uniquement qu’en termes de signification sociale. » (Hauerwas, 37).
\end{quote}
 \subsection{Origine de cette dichotomie}

\begin{quote}
    « L’assomption humaniste du libéralisme devint l’équivalent de la christologie du Logos, pour Jésus,
maintenant dépouillé de prétentions métaphysiques, il devint le meilleur exemple de l’homme vivant
moralement. Cet exemple se tenait indépendamment de tout connaissance personnelle ou de relation
avec Jésus. » (Hauerwas, 42).
\end{quote}
 
 \paragraph{séparation entre l’expérience religieuse et la morale sociale}

 \begin{quote}
     « L’affirmation que l’histoire de Jésus est une éthique sociale signifie qu’il n’y pas d’essence morale
ou un message qui est séparable de l’histoire de Jésus comme nous le trouvons dans les évangiles. Il
ne peut pas y avoir une figure du Christ car Jésus est le Christ. (...). Il n’y a pas de signification qui est
séparable de sa propre histoire. » (Hauerwas, 42-43).
 \end{quote}
 \paragraph{Séparation entre la moralité abstraite et la vie concrète de Jésus}

\begin{quote}
    « Une christologie qui n’est pas une éthique sociale est déficiente. A partir de cette perspective les
christologies les plus ‘orthodoxes’ sont inadéquates lorsqu’elles échouent à suggérer que la manière de
croire en Jésus implique que nous soient fournies les capacités pour décrire et gérer notre existence
sociale » (Hauerwas, 37).
\end{quote}

\begin{quote}
    « Comment cet homme peut être Dieu et en conséquence se trouver universel, absolu, et d’une
signification insurpassable ? (…) La tentation est d’enraciner l’universalité de Jésus dans une
représentation métaphysique ou anthropologique, abstraite des vicissitudes de l’histoire. Faire cela
sépare Jésus de l’éthique sociale en libérant ceux qui revendiquent être ses disciples d’affronter le fait
que son universalité repose sur leur fidélité aux exigences de son Royaume. (…) Le langage
christologique est signifiant en relation avec la vie actuelle et l’impact de l’homme Jésus » (Hauerwas,
43-44).
\end{quote}
  
 \paragraph{séparation entre fait et signification}


 
\begin{quote}
    « [L’] identité [de Jésus] n’est pas saisie à travers d’autres histoires de sauveur, mais par
l’apprentissage de celui qui le suit, qui est la condition nécessaire pour être citoyen de son
royaume […] Je soutiens que la personne de Jésus ne peut pas être séparée de son oeuvre, l’incarnation
de la rédemption. » (Hauerwas, 43).
\end{quote}
 \subsection{Les christologies héritières de cette dichotomie}
 
  \subsection{Comment penser le lien intrinsèque entre le singulier et l’universel ?}

  lien entre l'individu et le social. Universel extrinsèque. Tentation d'universalité de Jésus dans une représentation métaphysique de Jésus (le Logos, la vérité,...) et non son universalité basée sur les \textit{exigences du Royaume}.
L'universalité chrétienne passe par une universalité concrete, celle du peuple. : 
\begin{quote}
    « Toute christologie doit traiter la manière dont cet individu est présenté comme le sauveur de tout le
peuple. Mais cette forme appropriée de son universalité est perdue si des théories métaphysiques et
anthropologiques sont faites pour remplacer le témoignage indispensable des vies chrétiennes et des
communautés exprimant l’importance de son histoire. Le témoignage présuppose et réclame
l’universalité, mais d’une manière qui rende clair que l’universel ne peut être revendiqué qu’à travers
l’apprentissage de la forme particulière du discipulat requis par cet homme particulier » (Hauerwas,
41)
\end{quote}


  \begin{Prop}
  Cela devient concret.
  \end{Prop}
\begin{Ex}
 En Ecclésiologie catholique, l'évéché est la concretisation de l'Eglise universelle ici.
\end{Ex}


 \begin{quote}
     « Jésus n’avait pas une éthique sociale, mais son (…) histoire est une éthique sociale » (Hauerwas,
37).
 \end{quote}
Devient concrète. 
 
\section{Etre socialisé ou configuré par l’histoire (story) de Jésus} 

Story : Geschichte.
\begin{quote}
    « [L’] identité [de Jésus] n’est pas saisie à travers d’autres histoires de sauveur, mais par
l’apprentissage de celui qui le suit, qui est la condition nécessaire pour être citoyen de son
royaume […] Je soutiens que la personne de Jésus ne peut pas être séparée de son oeuvre, l’incarnation
de la rédemption. » (Hauerwas, 43).
\end{quote}


\begin{quote}
L’histoire de Jésus ne façonne pas seulement l’individu, ne débouche pas uniquement sur une éthique
individuelle mais elle façonne un peuple et donc une éthique sociale. C’est de cette manière que le
Christ peut être dit universel.
    
\end{quote}

\paragraph{peuple qui veut prendre sa croix}
\begin{quote}
    « L’universalité de Jésus est manifestée uniquement par un peuple qui veut prendre sa croix comme
étant son histoire, comme la condition nécessaire pour vivre en vérité dans cette vie. Comme sa croix
était une éthique sociale, ainsi il devient la continuation de cette éthique dans le monde, jusqu’à ce que
tout soit rassemblé dans son royaume » (Hauerwas, 44).
\end{quote}
C'est l'histoire du Royaume. En Jésus, le Royaume est à portée de main.


\paragraph{dimension de narration} L'histoire EST une éthique sociale. 
\begin{quote}
    En redécouvrant la \textit{dimension narrative de la christologie} nous serons capable de voir que Jésus
n’avait pas une éthique sociale, mais que son histoire est une éthique sociale. La validité sociale et
politique d’une communauté provient de sa réalité formée par une histoire véridique, une histoire qui
nous donne les significations pour vivre sans crainte les uns avec les autres. C’est pourquoi il n’y a pas
de séparation de la christologie d’avec l’ecclésiologie, c-à-d de Jésus d’avec l’Église. La véridicité de
Jésus créé et est connue par le type de communauté que son histoire doit former. » (Hauerwas, 37).
\end{quote}

\paragraph{L'Ecclésiologie est une christologie sociale} recrée un lien. 
\section{Jésus comme l’autobasileia}

\paragraph{En suivant Jésus, on comprend Jésus} et non pas de l'extérieur.

\paragraph{Universalité}
L'universel n'est pas un concept abstrait (le logos), c'est le peuple concret. 
Au lieu d'imposer une universalité abstraite, elle se construit. L'universalité est partageable donc elle doit faire des Eglises.

 \subsection{Jésus est une éthique sociale car il est l’autobasileia} 
 
 
\paragraph{Autobasileia} le Royaume lui-même


 Origène parlait déjà de Jésus comme Autobasileia, le Royaume incarné. C'est pour cela qu'il faut s'attacher au Christ. Pas d'autres chemins éthiques que la forme de vie de Jésus pour les chrétiens. Il est le Roayume \textit{inchoatif}. Hauerwas pense \textit{l'histoire du Royaume}, donc une dimension histoire non épuisée par la vie de Jésus.


\begin{quote}
    « Le seul Jésus que nous connaissons est le Jésus de la foi, le Jésus créé par l’Église. […] Il n’y a pas
de vrai Jésus si ce n’est celui qui est connu à travers le genre de vie qu’il a exigé de ses disciples ; que
les évangiles montrent la grammaire d’une telle vie ne devrait pas être une surprise pour nous. Il est
clair que la demande de précision historique est anhistorique dans la mesure où les évangiles montrent
pourquoi l’histoire de cet homme est inséparable de la façon dont cette histoire nous enseigne à le
suivre. Comme les évangiles le montrent, c’est seulement parce que les disciples ont d’abord suivi
Jésus à Jérusalem qu’ils furent capable de comprendre la signification de la résurrection. » (Hauerwas,
41-42).
\end{quote}


 \begin{quote}
     « Le royaume est ‘totalement et exclusivement agir de Dieu. Il ne peut être obtenu par un effort
religieux et moral, imposé par une bataille politique, ou projeté par des calculs. Nous ne pouvons pas
l’organiser, le faire ou le construire. Il est donné » (Hauerwas, 45).
 \end{quote}



\subsection{La configuration du disciple comme citoyen du royaume} 

\paragraph{Confession de Césarée Mc 8,27} la vraie rencontre change l'interlocuteur. Pierre répond avec la réponse exacte mais ne comprend pas le mot, car le mot a besoin d'une histoire. Sinon, le mot est un concept. Le mot, sans l'histoire est abstrait. Il faut suivre le Christ pour comprendre le mot. 
\begin{itemize}
    \item il faut être juif
    \item il faut suivre le Christ jusqu'à la Croix pour comprendre le mot
    \item sinon, sens abstrait.
\end{itemize}
\begin{quote}
    « La bref rencontre entre Pierre et Jésus rend clair que l’histoire de Jésus (…) change l’auditeur. Une
histoire qui revendique d’être la vérité de notre existence requiert que nos vies, comme les vies des
disciples, soient changées en suivant Jésus. » (Hauerwas, 47).
\end{quote}

\paragraph{Servir sans pouvoir}
\begin{Synthesis}
Servir sans pouvoir ! peut nous éclairer pour la synodalité
\end{Synthesis}
\begin{quote}
    « Le pouvoir du Messie est d’un ordre différent et les pouvoirs de ce monde le mettront
nécessairement à mort car ils reconnaissent, mieux que Pierre, ce à quoi ressemble une menace au
pouvoir. Car ici [le Messie] est quelqu’un qui invite les autres à participer à un royaume de l’amour de
Dieu, un royaume qui libère la capacité de donner et de servir. Les pouvoirs de ce monde ne peuvent
pas comprendre un tel royaume. Nous avons ici un homme affirmant qu’il est possible, si le règne de
Dieu est reconnu et crue, de servir sans pouvoir. » (Hauerwas, 48).
\end{quote}

\begin{quote}
    « Sa mort était la fin et l’accomplissement de sa vie. Dans sa mort il a terminé l’oeuvre qui était la
mission qu’il avait à accomplir. En ce sens la croix n’est pas un détour ou un obstacle sur le chemin du
Royaume (…) il est la venue du Royaume. En effet, la croix plus que tout autre événement révèle la
caractère social de la mission de Jésus. Jésus était le porteur d’une nouvelle possibilité de relations
humaines et sociales. C’est pourquoi l’incarnation n’est pas l’affirmation de l’approbation divine de
l’homme (…) mais la manière dont Dieu brise les limites de la définition de l’homme, pou redonner
une nouvelle définition configurante de ce qu’est l’humain en Jésus. » (Hauerwas, 48-49).
\end{quote}
\mn{voir René Girard, le bouc émissaire}


\paragraph{va changer les relations sociales : nouvelle forme politique} suivre le Christ jusqu'à la croix, c'est changer nos relations avec les autres. Cela nous configure plus que cela nous protège. 

\begin{quote}
    « C’est de cette manière que le discipulat chrétien créé une forme politique ; c’est ainsi qu’il est une
forme politique : être un chrétien est une expression de notre obéissance à et dans une communauté
fondée sur la messianité de Jésus. Et c’est ce que Pierre n’avait pas compris ; il pensait que le
Royaume devait ressembler aux royaumes de ce monde. Mais il se trompait : les royaumes du monde
obtiennent leur existence de notre peur les uns des autres, ; le règne de Dieu signifie qu’une
communauté peut exister là où règne la confiance, la confiance rendue possible par la connaissance
que notre existence est liée par la vérité. » (Hauerwas, 49).
\end{quote}

Va un peu plus loin, notre peur des uns des autres (Hobbes ?, Etat maintient la société par la violence). 


\paragraph{Une autre manière d'être avec les autres : la christologie joue un rôle déterminant}


\section{Du Christ autobasileia à l’Église}

\subsection{L’Église comme continuation de l’histoire de Jésus}

Cette histoire est continue et toujours configurante pour nous. 


\begin{quote}
    Si on ne peut pas dire de Jésus qu’il a une éthique sociale ou qu’il implique une éthique sociale, mais
dire qu’il \textsc{est} une éthique sociale, alors la forme de l’Église doit manifester cette éthique. » (Hauerwas,
40)
\end{quote}

Il y a aussi une discontinuité entre Jésus et l'Eglise, il n'en parle pas trop.

\paragraph{Est ce que l'Eglise raconte l'histoire de Jésus} Au delà des individus, des saints, est ce que l'Eglise est sacrement ?

\subsection{Une contre-culture politique et sociale} 

L'Eglise ne doit pas s'inculturer dans une société ultra-libérale mais \textit{être une contre-culture}. Savonarole ? Les évangiles sont des manuels. Les évangiles sont des histoires performatives qui permettent n'advenir le Règne de Dieu dans notre monde. 


\paragraph{Créer les conditions d'une nouvelles cultures}


\paragraph{Liberté} Les chrétiens ne sont plus sur la défensive car ils n'ont plus peur.

\begin{quote}
    « Un peuple libéré de la menace (de la peur) de la mort forme nécessairement un système politique,
car il peut faire face à la vérité de son existence sans peur et sans attitude de défense. Il peut même
prendre le risque d’adopter l’histoire d’un Seigneur crucifié comme sa réalité centrale. Celui-ci est
Seigneur étrange, il apparaît sans pouvoir, mais son impuissance s’avère être le pouvoir de la vérité
contre la violence du mensonge » (Hauerwas, 50).
\end{quote}


Si nous croyons à l'impuissance, au Christ crucifié, on n'a plus peur. On n'est plus dans la défense de nos intérêt. Chemin de la christologie, qui fonde l'éthique sociale et la politique \sn{je ne suis pas de ce monde}.

\paragraph{les communautés sont elles fidèles au Christ}

Risque d'absolutiser l'Eglise qui se croit le Christ alors qu'elle est \textit{simul pecator et salvi}, pécheresse et sauvé par le Christ qui n'est pas pécheur.

\sn{On verra la christologie de la libération}




\section{La christologie sociale et la question du pluralisme religieux et culturel ?}

\subsection{Le contre-pied d’une approche « libérale »}

\begin{quote}
    « Apprendre à suivre Jésus signifie que nous devons apprendre à accepter un tel pardon, et cela n’est
pas facile à accepter, comme acceptation qui requiert la reconnaissance de notre péché et de notre
vulnérabilité. » (Hauerwas, 50).
\end{quote}
\subsection{Consentir à des formes de vie différentes} 
\begin{quote}
    « La communauté chrétienne est formée par une histoire qui rend capables ses membres de faire
confiance dans l’altérité des autres comme le grand signe du caractère pardonnant du Royaume de
Dieu » (Hauerwas, 50).
\end{quote}

\subsection{La vérité est plurielle} 

\begin{quote}
    « Jésus est l’histoire qui forme l’Église. Cela signifie que l’Église sert tout d’abord le monde en
l’aidant à se connaître comme monde. Car sans un « contre-modèle » le monde n’a pas le moyen de
connaître ou de sentir, pour survivre, l’étrangeté de sa dépendance au pouvoir. Grâce à l’Église, le
monde peut sentir l’étrangeté d’essayer de construire une politique qui est intrinsèquement malhonnête
et fausse ; le monde ne dispose pas le fondement pour exiger la vérité à partir de son peuple. Grâce à
une communauté formée par l’histoire du Christ, le monde peut connaître ce que signifie être une
société engagée pour la croissance des dons individuels et des différences. Dans une communauté qui
n’a pas peur de la vérité, l’altérité de l’autre peut être la bienvenue comme un don plutôt que comme
une menace » (Hauerwas, 50-51).
\end{quote}
\begin{quote}
« Un tel regard n’est pas fondé sur des doctrines faciles de tolérance et d’égalité, mais est forgé à partir
de notre expérience commune d’être entraîné à être disciples de Jésus. L’universalité de l’Église est
fondée sur la particularité de l’histoire de Jésus et sur le fait uns et les autres comme peuple de Dieu. » (Hauerwas, 51).
\end{quote}
\subsection{La diversité providentielle des religions}  