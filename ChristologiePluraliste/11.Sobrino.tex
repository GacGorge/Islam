\Chapter{La christologie de la libération de Sobrino}{La fécondité sociale du Christ III}


\section{Bibliographie}

\begin{itemize}
    \item 
Théologies de la libération. Documents et débats, Paris 1985.
    \item BERTEN, I., « La théologie de la libération est-elle encore d’actualité ? », Lumière et Vie 273
(2007) 93-102.
    \item BOFF, C., Théorie et pratique. La méthode des théologies de la libération, Paris 1990
(l’original en Portugais est de 1977).
    \item BOFF, L., Jésus Christ libérateur, Paris 1974.
    \item CHEZA, M. – MARTINEZ SAAVEDRA, L. – SAUVAGE, P. (dir.), Dictionnaire historique de la
théologie de la libération. Les thèmes, les lieux, les acteurs, Namur – Paris 2017.
    \item GUTIERREZ, G., « La libération par la foi », Lumière et Vie 275 (2007) 5-15.
    \item MANZATTO, A. – XAVIER, D.-J., « Medellin, 50 ans après. Qu’en est-il de la théologie latinoaméricaine
? », Revue théologique de Louvain 50 (2019) 58-80.
    \item SCANNONE, J.-C., La théologie du peuple. Racines théologiques du pape François, Namur –
Paris 2017.
    \item SEGUNDO, J.-L., Jésus devant la conscience moderne. L’histoire perdue, Paris 1988.
    \item SOBRINO, J., Jésus Christ libérateur. Lecture historico-théologique de Jésus de Nazareth, tr.
par T. BENITO, Paris 2014. L’original en espagnol est de 1991.
    \item SOBRINO, J., La foi en Jésus Christ. Essai à partir des victimes, tr. par T. BENITO, Paris 2015.

\end{itemize}

% -----------------
\section{Introduction à la théologie de la libération}

\mn{11/4/23}
\begin{Synthesis}
Nous avons besoin d'être en dialogue mais aussi d'être une communauté vivante dans le monde d'aujourd'hui. Avec Hauerwas, nous avons vu comme faire \textit{histoire} et de tenir en partageant autour de nous.
\end{Synthesis}


\paragraph{Introduction}
Passage de la modernité à la post-modernité; \textit{une mondialisation qui produit des déchets}.

Fin des années 1960, deuxième conférence de la CELAM à Medellin en 1968 et la publication de G. Gutierrez, \textit{théologie de la libération, perspectives.}

\begin{quote}
    Comment dire aux pauvres que Dieu vous aime ? Gutierrez
\end{quote}

\paragraph{a) Cette théologie est au confluent de plusieurs événements et mouvements.} 1960, mouvement d'alphabetisation de culture populaire au Brésil. 
Coups d'Etat (1964 Brésil,...).

\begin{Synthesis}
    Un mouvement de développement brisé politiquement par les coups d'Etat.
\end{Synthesis}

La théorie de la \textit{dépendance} : ce ne sont pas des causes internes mais une dépendance du Nord qui ne veut pas notre développement.

Le deuxième évènement, c'est \textit{Vatican II}. Botzmann et Metz développent une théologie politique dans l'élan de VII. 

\textsc{Un Kairos} émerge.

\paragraph{b) La théologie de la libération : méthode et orientations}
Une théologie qui va accompagner les communautés de base, souvent engagées contre les dictatures militaires et le système capitaliste. Fournit des outils intellectuels et spirituels pour la \textit{lutte}. Il y a une dimension collective et engagée.

Ce n'est pas une théologie mais une \textit{manière de faire de la théologie}. Ce n'est pas une théologie déductive partant du dogme, ni historique, partant de l'histoire mais c'est une théologie \textit{inductive} qui développe par rapport aux opprimés. 

Il s'agit d'une compréhension non naive, une \textit{théologie là pour avoir une vision plus profonde}.

Des passages mis en valeur : Isaie, Elisée,
Amos, Magnificat, Mt 25.
Elle est plutôt classique en terme de théologie mais il y a une critique interne envers l'Eglise et son rapport au monde. 

\paragraph{c) Au sujet des pauvres}
Pas slt économique, le pauvre est l'\textit{insignifiant socialement}, les \textit{laissers pour compte}. Ce n'est pas une fatalité, c'est une condition. Elle n'est pas un malheur, elle est une injustice. 

Scruter les signes des temps,part du pauvre à partir de Dieu et part de Dieu en partant du pauvre : on a une dialectique.

\paragraph{d) La théologie de la libération réprimée}
Elle inquiète les pouvoirs en place (Mgr Romero) et de la hiérarchie (à cause des liens avec communisme). 

\begin{quote}
    6 Aout 1984, instruction contre quelques aspects de la théologie de la libération :
    mise en garde contre la théologie de la libération.
    puis précision en 1986. Fermeture de \textit{Lumen Gentium} à Bruxelles.
\end{quote}

Le discours du Magistère reconnaît l'injustice mais est prudent sur l'aspect collectif de la rédemption. Le problème est moins le système que les personnes qui animent le système.

\paragraph{e) Sur l’inspiration marxiste de la TdL et la légitimation de la violence}

Gutierrez s'inscrit en faux contre ces critiques : S'il y a inspiration, il s'agit de l'inspiration de la foi chrétienne et non marxiste.
Pour penser les causes externes du sous-développement, le marxisme peut donner des clés. 
Mais pour les marxistes, la foi est aliénante alors que pour les TdlL, la foi est libérante.




\paragraph{f) Evolutions plus récentes}
Certes, elle a évolué parce que le contexte a évolué : effondrement du communisme, Rome s'inquiète moins du communisme que du relativisme, et les régimes militaires ont donné lieu à des régimes plus démocratiques.  Nées en résistance aux dictatures, elle doit se transformer. 

D'autres types de luttes : femme, noirs, indigène, écologie,... d'autres types d'aliénation. 
La théologie de la libération n'est pas morte mais est éclatée. Moins de grands théologiens qui la portent (à part le pape...). 

\paragraph{g) Acquis et pertinence actuelle de la théologie de la libération}

\begin{itemize}
    \item Option préférentielle pour les pauvres
    \item idée du péché structurel\mn{culture de mort : dur.}
\item partir du vécu (expérience de l'action catholique).
    
\end{itemize}



\paragraph{h) Les défis que la théologie doit aujourd’hui relever en priorité (selon Gutierrez).}

Les Eglises évangéliques ont évangélisé les pauvres : elles manquent de personnes pour penser les choses. Alors que les cathos sont les classes moyennes qui n'ont pas forcément envie de bouger. 2007.

Le monde a éclaté : comment continuer à faire vivre ses inspirations ?

Les théologiens qui viennent des pauvres essayent de relever ce défi.

\paragraph{Le pape François et la théologie de la libération} une théologie du Peuple. Synodalité.

% -----------------
\section{Introduction à la christologie de la libération}

\section{Une nouvelle image et une nouvelle foi en Christ}

\begin{Ex}
Faire la théologie avec les pauvres du Salvador
\end{Ex}

\paragraph{Ce n'est pas uniquement contextuel qui m'a fait élaborer une telle théologie}

\begin{quote}
    « En plus des textes, qui constituent généralement le fondement conventionnel des christologies, un
autre fondement fut très présent : la réalité dans laquelle il m’a été donné de vivre et de penser. Et
cette réalité a eu des conséquences spécifiques pour la christologie, bien au-delà de l’influence
évidente que la réalité qui nous entoure a toujours sur notre mode de penser » (Sobrino I 14).
\end{quote}


\paragraph{Une parenté avec Hauerwas}
Hauerwas était très critique des grandes théologies.
\begin{quote}
    « Je\sn{HAUERWAS, S., « Jesus : The
Story of the Kingdom » dans A Community of Character. Toward a Constructive Christian Social Ethic, Notre
Dame (Indiana) 1981, note 4 p. 37. Il s’agit d’un commentaire sur Sobrino, Christology at the Crossroads,
Maryknoll 1978.}. partage une profonde sympathie pour l’intention de Sobrino, en particulier lorsqu’il situe le
\textbf{discipulat} comme un motif christologique central.[Mais il n’est pas forcément d’accord avec lui sur
l’idée et la centralité du concept de « libération » dans la théologie]. Je suis aussi en consonance avec
son insistance sur le ‘Jésus historique’, bien que je me demande si sa thèse affirmant que la
compréhension de Jésus sur le royaume ait changé au milieu de son ministère puisse être établie
historiquement. Bien plus Sobrino semble ne pas se rendre compte que déterminer un ‘Jésus
historique’ séparé des évangiles est une abstraction historique. Cependant, il souligne à juste titre que
les formulations christologiques classiques, y compris les revendications sur l’incarnation, doivent
venir à la fin et non pas au début de notre réflexion christologique » 
\end{quote}

Celui qui suit Jésus Christ, nous dit quelque chose du Christ.

\paragraph{Sobrino est basque}, Jésuite, Salvador, US, Allemagne, ThD 1975 : "Croix Pannenberg". Prof à l'UCA. Lance une revue. Proche de Mgr Romero. 1989 : échappe à un assassinat où ses frères jésuites meurent.
Deux ouvrages condamnés par Rome. 


\subsection{Quels sont les aspects fondamentaux de cette nouvelle image ?}

\paragraph{Une nouvelle image du Christ} qui fait qu'on se configure à elle différemment. 
\begin{quote}
    Ce que j’appelle l’\textit{irruption} d’une image de Jésus ne fut pas la \textit{conclusion} d’un processus discursif
ou émotionnel mais l’explosion de quelque chose qui s’imposait de soi comme étant vrai et porteur
d’un \textit{potentiel} capable de générer une réalité non seulement au niveau de la personne, mais aussi du
groupe ; et je pense que ce dernier point est le plus novateur. Que cette irruption se soit produite nous
a fait prendre conscience d’un Jésus au \textit{caractère propr}e et spécifique, nous a permis de nous
imprégner d’esprit évangélique et de nous configurer, par la disponibilité, \textit{à une praxis} conforme à
cette image » (Sobrino I, 14).
\end{quote}

C'est une image réelle car il y eut bcp de réactions pour la supprimer.

\paragraph{Il faut prendre au sérieux la vie de Jésus} 


\subsubsection{3 traits sont essentiels} 
\paragraph{Jésus libérateur}
  Image performative de Jésus. Porter du fruit dans notre vie : nous devons être libérés des formes d'esclavage les pauvres du continent \textit{en vue d'être acteurs}. Image \textit{sotériologique} enraciné dans l'image de Jésus envoyé pour libérer les captifs(Lc 4, 18)

    \begin{quote}
        « Le traditionnel Christ souffrant a été vu, non plus seulement comme un symbole de la souffrance
avec lequel il était possible de s’identifier, mais encore et spécifiquement comme un symbole de
protestation contre la souffrance, et surtout, comme un symbole de libération (…). Le fait qu’existe
cette nouvelle image du Christ est le plus grand événement christologique en Amérique latine (…) »
(Sobrino I, 48-49).
    \end{quote}

 \textsc{Contre la souffrance}\sn{peut inquiéter les puissants. Cf frères Karamazov, le grand inquisiteur face à Jésus: \textit{tu leur as donné la liberté, nous avons été obligé de leur reprendre}}.

\paragraph{la foi en Christ, \textit{sequela Christi}} Il s'agit de suivre le Christ, vraiment. Si les gens ont donné leur vie, cela a plus de poids qu'une démonstration logique. 


\paragraph{La suite du Christ est par essence conflictuelle} Assez originelle. Témoigner du Christ libérateur nous expose à des puissances hostiles.

pas uniquement théorique : si la théologie ne nous fait pas vivre selon le christ, est ce que cette théologie est vraie ?


\paragraph{Déplore les théologies traditionnelles qui enferment et qui n'aident pas à suivre le Christ}
\begin{quote}
    « Même dans ses formes orthodoxes, la christologie peut se transformer en une machine faisant en
sorte que la foi ne permette plus aux croyants de reproduire dans leurs vies la réalité de Jésus ni de
bâtir dans l’histoire le royaume de Dieu annoncé par Jésus » (Sobrino I, 34). D’après Segundo, il
s’agit, au contraire, d’élaborer « un discours sur Jésus qui ouvre une voie pour qu’il soit considéré
comme témoin d’une vie plus humaine et plus libérée » (Sobrino I, 34).
\end{quote}

Critique cette image d'un Christ approprié par les puissants et laissant le Christ souffrant aux pauvres.
\textit{On a développé une christologie sans Jésus}. 

\begin{itemize}
    \item Un Christ abstrait ("le logos").  
    \item Est ce qu'on justifie la tragédie ? un Christ charitable, \textit{bon samaritain}. Mais \textit{la justice} ? et on oublie la prophétie de Jésus ("hypocrisie du Système"). 
    \item le Christ pouvoir. omnipotent. Le lieu de pouvoir est en haut car le Christ est en haut.
    \item un Christ réconciliateur. Affirmation limite et eschatologique à la fin des temps, mais dans l'histoire, des luttes.
    \item un christ absolument absolu : le Christ annonce le Royaume de Dieu. mais on peut oublier que Jésus, médiateur, n'est donc pas absolu par rapport au service de la venue du Royaume.
\end{itemize}

Il est attentif aux nuances : Jésus aime tout le monde mais demande la conversion.  Il s'adresse différemment à chacun. 

\paragraph{Comment Jésus se situe dans le tragique, dans les conflits ? } Est ce que cela nous donne les moyens de vivre dans les conflits ? 

% - -----------------------------------
\subsection{Un dépassement des christologies aliénantes}


\paragraph{les théologies qui ont domestiqué le Christ}



% - -----------------------------------
\subsection{L’image du Christ à Medellin et à Puebla}

\paragraph{1968 Medellin} Aborde la figure du Christ par rapport à sa figure salvifique, en utilisant l'expression \textit{libération} et non \textit{salut}. 


\begin{quote}
    « C’est Dieu lui-même qui, dans la plénitude des temps, envoie son Fils, pour qu’ayant pris chair, il
vienne \textsc{libérer} tous les hommes de tous les esclavages auxquels ils sont soumis par le péché,
l’ignorance, la faim, la misère et l’oppression, en un mot l’injustice et la haine qui ont leur origine
dans l’égoïsme humain » [Medellin, « Justice », n° 3].
\end{quote}
La libération n'est pas uniquement celle du péché.

\subparagraph{Partialité - concept de Medellin} Jésus a apporté le salut à tous les hommes (principe d'impartialité). Or, il s'est fait \textit{pauvre}, donc principe de partialité. L'universalité du salut place par les pauvres. 
\begin{quote}
    « Le Christ notre Sauveur a non seulement aimé les pauvres, mais, ‘étant riche, il s’est fait pauvre’, a
vécu dans la pauvreté, a centré sa mission sur l’annonce aux pauvres de leur libération et a fondé son
Église comme signe de cette pauvreté parmi les hommes [Medellin, « pauvreté de l’Église », n° 7].
\end{quote}

En allant un peu plus loin que G\&S, l'homme est celui qui anticipe dans ses conquêtes le signe annonciateur du Royaume : \mn{cf sur le refus de la souffrance à la mort, l'évolution du monde vers une humanisation ? } 
\begin{quote}
   « Le Christ activement présent dans notre histoire anticipe son geste eschatologique non seulement
dans l’impatiente aspiration de l’homme à sa rédemption totale, mais aussi dans ces conquêtes que,
comme autant de \textit{signes annonciateurs}, l’homme réalise progressivement à travers une \textit{activité
conduite dans l’amour} [Medellin, « Introduction », n° 5]. 
\end{quote}

On ne peut rejeter quelqu'un que s'il est présent. Mt 25 : identification de Jésus aux opprimés.  


\paragraph{1979 - Puebla} Option préférentielle pour les pauvres
\begin{quote}
    « Les pauvres sont les destinataires privilégiés de la mission de Jésus, et du simple fait d’être pauvres,
‘quelle que soit la situation morale ou personnelle dans laquelle ils peuvent se trouver’ (n° 1.142) ; il
est dit aussi que l’évangélisation des pauvres est signe et preuve par excellence de la mission de Jésus
(n° 1.142). De cette façon, on réaffirme la corrélation essentielle entre pauvres et mission de Jésus »
(Sobrino I, 64).
\end{quote}

Puebla introduit ce principe de partialité dans le Christ, les pauvres sont un quasi sacrement de Jésus : 

\begin{quote}
    « Les pauvres sont donc un \textsc{quasi-sacrement} en deux dimensions fondamentales de la mission de
Jésus : en premier lieu, ils appellent à la conversion, car leur propre réalité, comme celle de Jésus
crucifié, est la plus grande interpellation du chrétien et de l’être l’humain et en ce sens les pauvres
exercent une prophétie primaire en raison de ce qu’ils sont en tant que victimes.
En second lieu, ils offrent des réalités et des valeurs comme celles qui ont été offertes par Jésus, en ce
sens, ils sont porteurs d’un évangile, exercent une évangélisation primaire » (Sobrino I, 65).
\end{quote}


\begin{Synthesis}
    Contexte spécifique sud-américain. 
    
    Critique des images du Christ déformés qui justifient le pouvoir. Des théologies trop confortables. D'où parlent les théologiens ?

    
\end{Synthesis}


%----------------------------------
\section{Le lieu à partir duquel on élabore la christologie}

Quand on fait de la théologie, on part d'où ? D'un texte ? 
Il faut tenir deux choses fondamentales : 
\begin{itemize}
    \item ce que le passé nous a laissé (textes, Tradition de l'Eglise).
    \item la seconde, la réalité du Christ dans le présent. On ne peut pas se fonder sur des textes passés. Quelle réalité du Christ présent aujourd'hui ? 
\end{itemize}

La foi en Christ ne peut répéter la foi des apôtres mais se vivre.
Où rencontrons-nous le Christ vivant aujourd'hui ? 

\begin{quote}
    « Pour aborder son objet, Jésus-Christ, la christologie doit tenir compte de deux choses fondamentales.
La première, et la plus évidente, est ce que le passé nous a livré à son sujet, c-à-d des textes dans
lesquels est exprimée la révélation [il faudrait ajouter la tradition vivante de l’Église] ; la seconde,
dont on tient moins compte, est la réalité du Christ dans le présent, c-à-d sa présence actuelle dans
l’histoire à laquelle correspond la foi réelle en Christ » (Sobrino I, 67).
\end{quote}


\subsection{Le « lieu christologique » comme une médiation pour mieux interpréter le Christ}

Vivifier l'expérience que nous faisons du Christ.
\paragraph{le lieu} là où on va bien comprendre le Christ. l'Evangile nous parle car nous faisons une expérience de cet évangile. 

\paragraph{actualiser le message du Christ et l'enrichir}
\begin{Ex}
    Dire que Dieu est \textit{sauveur}, dans quel lieu aujourd'hui on peut vivre cette expérience salvifique
\end{Ex}

\begin{quote}
    « En étant cela et faisant cela, l’Église se transforme en sacrement par rapport au Christ et devient son
corps dans l’histoire. ‘La corporéité historique de l’Église implique que ‘prennent corps’ en elle la
réalité et l’action de Jésus-Christ afin qu’elle réalise une ‘incorporation’ de Jésus-Christ dans la réalité
de l’histoire’ » (Sobrino I, 77).
\end{quote}

\paragraph{Expérience de libération} On ne peut parler de la libération du Christ que si on fait cette expérience d'oppression et de libération. 

\begin{quote}
    « La christologie doit elle aussi se comprendre avant tout comme christopraxis, non pour annuler le
logos, mais bien pour que ce dernier éclaire la vérité du Christ à partir des impulsions données par le
Christ lui-même pour que la libération devienne réalité » (Sobrino I, 86).
\end{quote}

\paragraph{Dans les situations nouvelles, il faut réinterroger l'écriture} cf Nostra Aetate, aucune référence du Magistère mais référence des Ecritures. Ecriture pour vivifier aujourd'hui.


% -----------------------------------
\subsection{Lieu théologique et signe des temps}

\paragraph{des termes proches}
Ce n'est pas seulement une théologie contextuelle (théologie sud américaine, féminine).
\paragraph{présence plus forte}
Jésus n'est pas uniformément présent dans l'histoire et le temps. Logique de la partialité, un peu inhabituelle. 
\begin{quote}
    Emmaus, notre coeur n'était-il pas brulant
    Lc 24
\end{quote}


\paragraph{Vatican II et signes des temps}
 


\paragraph{première lecture des signes des temps : G\&S 4} ce qui caractérise une époque.

\begin{quote}
    1. Pour mener à bien cette tâche, l’Église a le devoir, à tout moment, de scruter les signes des temps et de les interpréter à la lumière de l’Évangile, de telle sorte qu’elle puisse répondre, d’une manière adaptée à chaque génération, aux questions éternelles des hommes sur le sens de la vie présente et future et sur leurs relations réciproques. Il importe donc de connaître et de comprendre ce monde dans lequel nous vivons, ses attentes, ses aspirations, son caractère souvent dramatique. Voici, tels qu’on peut les esquisser, quelques-uns des traits fondamentaux du monde actuel. G\& S 4
\end{quote}

\begin{Ex}
    théologie des religions, écologie
\end{Ex}

\paragraph{deuxième vision : signe des temps historico-théologale,}
 signe de la présence de Dieu. 
\begin{quote}
    1. Mû par la foi, se sachant conduit par l’Esprit du Seigneur qui remplit l’univers, le Peuple de Dieu s’efforce de discerner dans les événements, les exigences et les requêtes de notre temps, auxquels il participe avec les autres hommes, quels sont les signes véritables de la présence ou du dessein de Dieu. La foi, en effet, éclaire toutes choses d’une lumière nouvelle et nous fait connaître la volonté divine sur la vocation intégrale de l’homme, orientant ainsi l’esprit vers des solutions pleinement humaines. G\&S 11
\end{quote}

Histoire dans sa dimension sacramentelle (signe), à manifester Dieu.  Comment à travers ces évenements (pedocriminalité), Dieu nous fait signe.


\subsection{Le signe de la présence du Christ : son corps}

\paragraph{Eglise \textit{signifie} le Corps du Christ} Le Christ est rendu présent.

\begin{quote}
    « La christologie est fondamentalement ecclésiale, parce qu’elle se réalise à l’intérieur d’une
communauté, avec une foi réelle qui rend le Christ présent, et à l’intérieur d’une communauté-en devenir,
et c’est elle qui, fondamentalement, réinterprète sa foi, apprenant à l’exprimer et à la formuler
pour que cette foi soit de plus en plus féconde » (Sobrino I, 78).
\end{quote}

L'Eglise est en \textit{devenir}, elle n'est pas installée. 

\begin{Ex}
    Seminaire AlMowafaqa. L'Eglise en Algérie. Les Français sont arrivés avec les processions,... L'Eglise est transplantée mais elle ne fait plus signe.
\end{Ex}

\begin{quote}
    En Amérique latine, on reconnaît la présence actuelle du Christ dans les opprimés et l’ « on prend au
sérieux (…) le fait que la christologie soit aussi christologie du ‘corps’ du Christ. Pour le dire avec les
paroles théologiques de \href{https://fr.wikipedia.org/wiki/Ignacio_Ellacur%C3%ADa}{I. Ellacuria}\sn{ {Ignacio Ellacuría } (1930 – San Salvador, 16 novembre 1989) est un théologien jésuite hispano-salvadorien, philosophe et théologien qui fut professeur et recteur de l'Universidad Centroamericana "José Simeón Cañas" (UCA) et qui fut assassiné par une milice d'extrême-droite dans les dernières années de la guerre civile salvadorienne .}, \begin{quote}
    ‘ce peuple crucifié est le prolongement historique du serviteur
souffrant de Yahvé’
\end{quote} ou avec les paroles pastorales de Mgr Romero à des paysans terrorisés après un
massacre :  
    ‘Vous êtes l’image du divin transpercé’
 » (Sobrino I, 72).
\end{quote}


 
\subsection{L’Église des pauvres, comme principe épistémologique de la christologie}


\paragraph{Christologie dans une Eglise des pauvres}

\begin{quote}
    Lorsque l’Église et les pauvres sont placés dans une relation essentielle, alors surgit l’Église des
pauvres, et celle-ci devient le lieu ecclésial de la christologie latino-américaine » (Sobrino I, 78).
\end{quote}

\paragraph{mieux connaître le Christ} la suite réelle du Christ introduit à la vie réelle de Jésus. 



\begin{quote}
    « Au niveau du contenu, du fait que les pauvres sont les destinataires privilégiés de la mission de
Jésus, ce sont eux qui posent à la foi les questions fondamentales, et ils le font avec force pour
ébranler et mettre en mouvement toute la communauté dans le processus d’ ‘apprendre à apprendre’ ce
que peut être le Christ. Parce qu’ils sont les privilégiés de Dieu et que leur foi est différente de celle
des non-pauvres, les pauvres mettent en question, de l’intérieur de la communauté, la foi
christologique et lui donnent sa direction fondamentale » (Sobrino I, 79).
\end{quote}

\begin{quote}    
« Non seulement ‘l’image’ du Christ qu’ont les croyants, mais leur foi concrète, leur façon de
répondre à cette image et d’y correspondre dans la réalité de leur vie, aide la christologie à s’introduire
dans la réalité du Christ et à comprendre les textes qui parlent de lui » (Sobrino I, 73).
    
\end{quote}

\paragraph{Jésus Sud Américain, un Jésus partial} Dans ce contexte, c'est ce Jésus qui parle, celui des pauvres, destructeur des idoles,...
\begin{quote}
« Le contenu concret de la foi accomplie éclaire aussi des contenus du Christ. Ainsi la suite réelle
introduit le Jésus que l’on suit, le martyre réel introduit la réalité du martyr Jésus.. Pour cette raison,
en analysant la réalité du Christ, la christologie latino-américaine a mis l’accent sur un Jésus et non sur
un autre, un Jésus au contours concrets et différents de ceux d’autres christologies (partial au profit des
pauvres, dénonciateur et démasqueur d’idoles, miséricordieux et fidèle jusqu’au bout..) » (Sobrino I,
74).
\end{quote}


\paragraph{par isomorphisme, on arrive au Jésus réel} Méthode
\begin{quote}
    « D’un point de vue méthodologique, il fut très important qu’ (…) on ait perçu un isomorphisme
fondamental entre la réalité dans laquelle se sont inscrits la vie et le destin de Jésus selon les récits des
évangiles et la réalité dans laquelle beaucoup de chrétiens et de chrétiennes ont vécu et ont donné leur
vie. Cet isomorphisme fait que la réalité d’aujourd’hui devient condition de possibilité de textes
nouveaux sur Jésus de Nazareth. Et qu’elle rend possible de le voir, de l’entendre et de le toucher dans
des témoins, ce qui en définitive, est plus décisif » (Sobrino I, 17).
\end{quote}

\begin{Ex}
    Martyr d'Etienne a permis de mieux comprendre le martyr du Christ. 
\end{Ex}
\subsection{La christologie en vue de la christopraxis}

\paragraph{sequela Christi} Provient du Jésus historique. La théologie ne produit pas que des connaissances. Elle retourne à l'histoire pour la transformer.
Interdépendance de la théologie à la christopraxis. 

\paragraph{Intellectus amoris} Pas uniquement une \textit{intellectus fidei}, qui rende compte de notre foi en raison mais aussi une intelligence qui ouvre sur l'amour. 
La foi en Christ s'inscrit dans une pratique. 


\paragraph{Les pauvres : lumière qui permet ce qui fait voir} Nouvelles questions permettent d'interpeller l'Evangile. Dans la pauvreté, il n'y a pas de faux semblants. Les peuples crucifiés nous libèrent des idéologies comme justification de certaines positions.

 \section{Le « Jésus de l’histoire » point de départ de la christologie}

\begin{quote}
    « Le Christ se rend présent et cette Église est son corps dans l’histoire. Cependant, elle ne l’est pas
n’importe comment, mais dans la mesure où elle offre au Christ cette espérance, cette praxis
libératrice et cette souffrance qui peuvent le rendre présent comme ressuscité et comme crucifié »
(Sobrino I, 80).
\end{quote}

La théologie n'est pas uniquement une science mais \textit{une pratique}. 
Antonio Gramci, praticien du marxisme : il faut changer la culture, le reste suivra. \textit{Il faut gagner le combat culturel}.



\subsection{Le Jésus de l’histoire est nécessaire à la christologie}

\paragraph{On peut contempler le Christ dans le ciel mais on ne peut pas le suivre} Seul le Jésus de l'histoire bouge, avance, \textit{va à Capharnaüm}. \textit{Sequela Christi}, il faut que le Christ bouge. 

\textit{Déjà Pannenberg le disait}, on ne peut connaître le Christ sans passer par Jésus. 

\paragraph{Comprendre les affirmations transcendantes (Jésus est Christ, ...) par l'histoire}
\begin{quote}
    Affirmer que Jésus est le Christ, qu’il est l’expression de ce qui est véritablement divin et
véritablement humain, sont des affirmations limites qui, par nature, exigent un chemin de
connaissance. (…) Sans ce chemin préalable, ce que l’on confesse dans la foi n’aurait aucun sens
descriptible. C’est pourquoi, dans l’Ecriture, les affirmations limites transcendantes sont précédées
d’affirmations historiques » (Sobrino I, 92).
\end{quote}

Très important, et vu avec Pannenberg. 

\paragraph{Le NT, retour à l'histoire de Jésus} Dans les premiers Kerygmes, on ne parle pas de Jésus. Puis, de plus en plus, on va parler de Jésus. 
\begin{Ex}
    Paul parle peu de Jésus dans ses lettres : repas pascal, mariage,...
    Pourtant Sobrino dit qu'indirectement, S. Paul raconte le Christ en disant sa propre histoire. 
\end{Ex}

\paragraph{Lettre aux He} Jésus est lié à la souffrance, l'amour des frères,... C'est surtout dans les textes des Evangiles que nous avons l'histoire de Jésus qui permet de retourner à son histoire concrête. Les premiers chrétiens se sont rendus compte que pour théologiser Jésus, il a fallu l'historiciser et raconter son histoire dans les Evangiles. 

\paragraph{En Amérique Latine} même mouvement de retourner à la vie de Jésus. 


\paragraph{Besoin de vivifier la foi dans un contexte historique} purifier la théologie en rendant Jésus concret. Vérifier la foi en Christ dans l'histoire.


\subsection{Qu’est-ce qu’il faut entendre par le « Jésus de l’histoire » ?}


\subsection{La priorité de la pratique sur l’intériorité ; de l’agir sur l’être}


\subsection{La suite du Christ comme mystagogie}
C'est en marchant avec lui, qu'on le comprend. 

\begin{quote}
    « Nous pensons que l’on accède mieux à l’intériorité de Jésus (l’historicité de sa subjectivité) à partir
de l’extériorité de sa pratique (sa façon de faire l’histoire) qu’inversement » (Sobrino I, 119). « Les
christologies qui commencent méthodologiquement par la ‘personne’ de Jésus, ses attitudes
intérieures, sa relation au Père, retrouvent plus difficilement et de façon moins radicale les données
centrales des récits évangéliques : le royaume de Dieu, la partialité en faveur des pauvres, la pratique
prophétique et conflictuelle… » (Sobrino I, 120).
\end{quote}

\subsection{Le retour à Jésus dans le NT au Ier siècle et en Amérique latine au XXe s.}

\begin{quote}
    « Par le simple fait de reproduire la pratique de Jésus dans ce qu’elle a d’ultime et son historicité parce
que c’est celle de Jésus (…), on est en train d’accepter que Jésus soit une norme ultime, et donc de le
confesser comme quelque chose de réellement ultime ; on est en train de le confesser, même si cela est
implicite, comme le Christ, même s’il faut par la suite expliciter cette confession » (Sobrino I, 121).
\end{quote}


\paragraph{Evangélistes qui valorise le succès}

\paragraph{théologie de la sainteté qu'on retrouve ici} mais au lieu de les présenter d'une façon individuelle, on marche ensemble et on essaye de suivre le Christ ensemble. cf \href{https://www.jesuites.com/quatre-preferences-apostoliques-universelles-jesuites-dix-prochaines-annees/}{préférences apostoliques universelles}.
\mn{Règne de DIeu, enjeu catholique; quelle pratique aussi pour cette pratique collective}