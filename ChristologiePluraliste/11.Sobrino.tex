\Chapter{La christologie de la libération de Sobrino}{La fécondité sociale du Christ III}


\section{Bibliographie}

\begin{itemize}
    \item 
Théologies de la libération. Documents et débats, Paris 1985.
    \item BERTEN, I., « La théologie de la libération est-elle encore d’actualité ? », Lumière et Vie 273
(2007) 93-102.
    \item BOFF, C., Théorie et pratique. La méthode des théologies de la libération, Paris 1990
(l’original en Portugais est de 1977).
    \item BOFF, L., Jésus Christ libérateur, Paris 1974.
    \item CHEZA, M. – MARTINEZ SAAVEDRA, L. – SAUVAGE, P. (dir.), Dictionnaire historique de la
théologie de la libération. Les thèmes, les lieux, les acteurs, Namur – Paris 2017.
    \item GUTIERREZ, G., « La libération par la foi », Lumière et Vie 275 (2007) 5-15.
    \item MANZATTO, A. – XAVIER, D.-J., « Medellin, 50 ans après. Qu’en est-il de la théologie latinoaméricaine
? », Revue théologique de Louvain 50 (2019) 58-80.
    \item SCANNONE, J.-C., La théologie du peuple. Racines théologiques du pape François, Namur –
Paris 2017.
    \item SEGUNDO, J.-L., Jésus devant la conscience moderne. L’histoire perdue, Paris 1988.
    \item SOBRINO, J., Jésus Christ libérateur. Lecture historico-théologique de Jésus de Nazareth, tr.
par T. BENITO, Paris 2014. L’original en espagnol est de 1991.
    \item SOBRINO, J., La foi en Jésus Christ. Essai à partir des victimes, tr. par T. BENITO, Paris 2015.

\end{itemize}

% -----------------
\section{Introduction à la théologie de la libération}

\mn{11/4/23}
\begin{Synthesis}
Nous avons besoin d'être en dialogue mais aussi d'être une communauté vivante dans le monde d'aujourd'hui. Avec Hauerwas, nous avons vu comme faire \textit{histoire} et de tenir en partageant autour de nous.
\end{Synthesis}


\paragraph{Introduction}
Passage de la modernité à la post-modernité; \textit{une mondialisation qui produit des déchets}.

Fin des années 1960, deuxième conférence de la CELAM à Medellin en 1968 et la publication de G. Gutierrez, \textit{théologie de la libération, perspectives.}

\begin{quote}
    Comment dire aux pauvres que Dieu vous aime ? Gutierrez
\end{quote}

\paragraph{a) Cette théologie est au confluent de plusieurs événements et mouvements.} 1960, mouvement d'alphabetisation de culture populaire au Brésil. 
Coups d'Etat (1964 Brésil,...).

\begin{Synthesis}
    Un mouvement de développement brisé politiquement par les coups d'Etat.
\end{Synthesis}

La théorie de la \textit{dépendance} : ce ne sont pas des causes internes mais une dépendance du Nord qui ne veut pas notre développement.

Le deuxième évènement, c'est \textit{Vatican II}. Botzmann et Metz développent une théologie politique dans l'élan de VII. 

\textsc{Un Kairos} émerge.

\paragraph{b) La théologie de la libération : méthode et orientations}
Une théologie qui va accompagner les communautés de base, souvent engagées contre les dictatures militaires et le système capitaliste. Fournit des outils intellectuels et spirituels pour la \textit{lutte}. Il y a une dimension collective et engagée.

Ce n'est pas une théologie mais une \textit{manière de faire de la théologie}. Ce n'est pas une théologie déductive partant du dogme, ni historique, partant de l'histoire mais c'est une théologie \textit{inductive} qui développe par rapport aux opprimés. 

Il s'agit d'une compréhension non naive, une \textit{théologie là pour avoir une vision plus profonde}.

Des passages mis en valeur : Isaie, Elisée,
Amos, Magnificat, Mt 25.
Elle est plutôt classique en terme de théologie mais il y a une critique interne envers l'Eglise et son rapport au monde. 

\paragraph{c) Au sujet des pauvres}
Pas slt économique, le pauvre est l'\textit{insignifiant socialement}, les \textit{laissers pour compte}. Ce n'est pas une fatalité, c'est une condition. Elle n'est pas un malheur, elle est une injustice. 

Scruter les signes des temps,part du pauvre à partir de Dieu et part de Dieu en partant du pauvre : on a une dialectique.

\paragraph{d) La théologie de la libération réprimée}
Elle inquiète les pouvoirs en place (Mgr Romero) et de la hiérarchie (à cause des liens avec communisme). 

\begin{quote}
    6 Aout 1984, instruction contre quelques aspects de la théologie de la libération :
    mise en garde contre la théologie de la libération.
    puis précision en 1986. Fermeture de \textit{Lumen Gentium} à Bruxelles.
\end{quote}

Le discours du Magistère reconnaît l'injustice mais est prudent sur l'aspect collectif de la rédemption. Le problème est moins le système que les personnes qui animent le système.

\paragraph{e) Sur l’inspiration marxiste de la TdL et la légitimation de la violence}

Gutierrez s'inscrit en faux contre ces critiques : S'il y a inspiration, il s'agit de l'inspiration de la foi chrétienne et non marxiste.
Pour penser les causes externes du sous-développement, le marxisme peut donner des clés. 
Mais pour les marxistes, la foi est aliénante alors que pour les TdlL, la foi est libérante.




\paragraph{f) Evolutions plus récentes}
Certes, elle a évolué parce que le contexte a évolué : effondrement du communisme, Rome s'inquiète moins du communisme que du relativisme, et les régimes militaires ont donné lieu à des régimes plus démocratiques.  Nées en résistance aux dictatures, elle doit se transformer. 

D'autres types de luttes : femme, noirs, indigène, écologie,... d'autres types d'aliénation. 
La théologie de la libération n'est pas morte mais est éclatée. Moins de grands théologiens qui la portent (à part le pape...). 

\paragraph{g) Acquis et pertinence actuelle de la théologie de la libération}

\begin{itemize}
    \item Option préférentielle pour les pauvres
    \item idée du péché structurel\mn{culture de mort : dur.}
\item partir du vécu (expérience de l'action catholique).
    
\end{itemize}



\paragraph{h) Les défis que la théologie doit aujourd’hui relever en priorité (selon Gutierrez).}

Les Eglises évangéliques ont évangélisé les pauvres : elles manquent de personnes pour penser les choses. Alors que les cathos sont les classes moyennes qui n'ont pas forcément envie de bouger. 2007.

Le monde a éclaté : comment continuer à faire vivre ses inspirations ?

Les théologiens qui viennent des pauvres essayent de relever ce défi.

\paragraph{Le pape François et la théologie de la libération} une théologie du Peuple. Synodalité.

% -----------------
\section{Introduction à la christologie de la libération}

\section{Une nouvelle image et une nouvelle foi en Christ}

\begin{Ex}
Faire la théologie avec les pauvres du Salvador
\end{Ex}

\paragraph{Ce n'est pas uniquement contextuel qui m'a fait élaborer une telle théologie}

\begin{quote}
    « En plus des textes, qui constituent généralement le fondement conventionnel des christologies, un
autre fondement fut très présent : la réalité dans laquelle il m’a été donné de vivre et de penser. Et
cette réalité a eu des conséquences spécifiques pour la christologie, bien au-delà de l’influence
évidente que la réalité qui nous entoure a toujours sur notre mode de penser » (Sobrino I 14).
\end{quote}


\paragraph{Une parenté avec Hauerwas}
Hauerwas était très critique des grandes théologies.
\begin{quote}
    « Je\sn{HAUERWAS, S., « Jesus : The
Story of the Kingdom » dans A Community of Character. Toward a Constructive Christian Social Ethic, Notre
Dame (Indiana) 1981, note 4 p. 37. Il s’agit d’un commentaire sur Sobrino, Christology at the Crossroads,
Maryknoll 1978.}. partage une profonde sympathie pour l’intention de Sobrino, en particulier lorsqu’il situe le
\textbf{discipulat} comme un motif christologique central.[Mais il n’est pas forcément d’accord avec lui sur
l’idée et la centralité du concept de « libération » dans la théologie]. Je suis aussi en consonance avec
son insistance sur le ‘Jésus historique’, bien que je me demande si sa thèse affirmant que la
compréhension de Jésus sur le royaume ait changé au milieu de son ministère puisse être établie
historiquement. Bien plus Sobrino semble ne pas se rendre compte que déterminer un ‘Jésus
historique’ séparé des évangiles est une abstraction historique. Cependant, il souligne à juste titre que
les formulations christologiques classiques, y compris les revendications sur l’incarnation, doivent
venir à la fin et non pas au début de notre réflexion christologique » 
\end{quote}

Celui qui suit Jésus Christ, nous dit quelque chose du Christ.

\paragraph{Sobrino est basque}, Jésuite, Salvador, US, Allemagne, ThD 1975 : "Croix Pannenberg". Prof à l'UCA. Lance une revue. Proche de Mgr Romero. 1989 : échappe à un assassinat où ses frères jésuites meurent.
Deux ouvrages condamnés par Rome. 


\subsection{Quels sont les aspects fondamentaux de cette nouvelle image ?}

\paragraph{Une nouvelle image du Christ} qui fait qu'on se configure à elle différemment. 
\begin{quote}
    Ce que j’appelle l’\textit{irruption} d’une image de Jésus ne fut pas la \textit{conclusion} d’un processus discursif
ou émotionnel mais l’explosion de quelque chose qui s’imposait de soi comme étant vrai et porteur
d’un \textit{potentiel} capable de générer une réalité non seulement au niveau de la personne, mais aussi du
groupe ; et je pense que ce dernier point est le plus novateur. Que cette irruption se soit produite nous
a fait prendre conscience d’un Jésus au \textit{caractère propr}e et spécifique, nous a permis de nous
imprégner d’esprit évangélique et de nous configurer, par la disponibilité, \textit{à une praxis} conforme à
cette image » (Sobrino I, 14).
\end{quote}

C'est une image réelle car il y eut bcp de réactions pour la supprimer.

\paragraph{Il faut prendre au sérieux la vie de Jésus} 


\subsubsection{3 traits sont essentiels} 
\paragraph{Jésus libérateur}
  Image performative de Jésus. Porter du fruit dans notre vie : nous devons être libérés des formes d'esclavage les pauvres du continent \textit{en vue d'être acteurs}. Image \textit{sotériologique} enraciné dans l'image de Jésus envoyé pour libérer les captifs(Lc 4, 18)

    \begin{quote}
        « Le traditionnel Christ souffrant a été vu, non plus seulement comme un symbole de la souffrance
avec lequel il était possible de s’identifier, mais encore et spécifiquement comme un symbole de
protestation contre la souffrance, et surtout, comme un symbole de libération (…). Le fait qu’existe
cette nouvelle image du Christ est le plus grand événement christologique en Amérique latine (…) »
(Sobrino I, 48-49).
    \end{quote}

 \textsc{Contre la souffrance}\sn{peut inquiéter les puissants. Cf frères Karamazov, le grand inquisiteur face à Jésus: \textit{tu leur as donné la liberté, nous avons été obligé de leur reprendre}}.

\paragraph{la foi en Christ, \textit{sequela Christi}} Il s'agit de suivre le Christ, vraiment. Si les gens ont donné leur vie, cela a plus de poids qu'une démonstration logique. 


\paragraph{La suite du Christ est par essence conflictuelle} Assez originelle. Témoigner du Christ libérateur nous expose à des puissances hostiles.

pas uniquement théorique : si la théologie ne nous fait pas vivre selon le christ, est ce que cette théologie est vraie ?


\paragraph{Déplore les théologies traditionnelles qui enferment et qui n'aident pas à suivre le Christ}
\begin{quote}
    « Même dans ses formes orthodoxes, la christologie peut se transformer en une machine faisant en
sorte que la foi ne permette plus aux croyants de reproduire dans leurs vies la réalité de Jésus ni de
bâtir dans l’histoire le royaume de Dieu annoncé par Jésus » (Sobrino I, 34). D’après Segundo, il
s’agit, au contraire, d’élaborer « un discours sur Jésus qui ouvre une voie pour qu’il soit considéré
comme témoin d’une vie plus humaine et plus libérée » (Sobrino I, 34).
\end{quote}

Critique cette image d'un Christ approprié par les puissants et laissant le Christ souffrant aux pauvres.
\textit{On a développé une christologie sans Jésus}. 

\begin{itemize}
    \item Un Christ abstrait ("le logos").  
    \item Est ce qu'on justifie la tragédie ? un Christ charitable, \textit{bon samaritain}. Mais \textit{la justice} ? et on oublie la prophétie de Jésus ("hypocrisie du Système"). 
    \item le Christ pouvoir. omnipotent. Le lieu de pouvoir est en haut car le Christ est en haut.
    \item un Christ réconciliateur. Affirmation limite et eschatologique à la fin des temps, mais dans l'histoire, des luttes.
    \item un christ absolument absolu : le Christ annonce le Royaume de Dieu. mais on peut oublier que Jésus, médiateur, n'est donc pas absolu par rapport au service de la venue du Royaume.
\end{itemize}

Il est attentif aux nuances : Jésus aime tout le monde mais demande la conversion.  Il s'adresse différemment à chacun. 

\paragraph{Comment Jésus se situe dans le tragique, dans les conflits ? } Est ce que cela nous donne les moyens de vivre dans les conflits ? 

% - -----------------------------------
\subsection{Un dépassement des christologies aliénantes}


\paragraph{les théologies qui ont domestiqué le Christ}



% - -----------------------------------
\subsection{L’image du Christ à Medellin et à Puebla}

\paragraph{1968 Medellin} Aborde la figure du Christ par rapport à sa figure salvifique, en utilisant l'expression \textit{libération} et non \textit{salut}. 


\begin{quote}
    « C’est Dieu lui-même qui, dans la plénitude des temps, envoie son Fils, pour qu’ayant pris chair, il
vienne \textsc{libérer} tous les hommes de tous les esclavages auxquels ils sont soumis par le péché,
l’ignorance, la faim, la misère et l’oppression, en un mot l’injustice et la haine qui ont leur origine
dans l’égoïsme humain » [Medellin, « Justice », n° 3].
\end{quote}
La libération n'est pas uniquement celle du péché.

\subparagraph{Partialité - concept de Medellin} Jésus a apporté le salut à tous les hommes (principe d'impartialité). Or, il s'est fait \textit{pauvre}, donc principe de partialité. L'universalité du salut place par les pauvres. 
\begin{quote}
    « Le Christ notre Sauveur a non seulement aimé les pauvres, mais, ‘étant riche, il s’est fait pauvre’, a
vécu dans la pauvreté, a centré sa mission sur l’annonce aux pauvres de leur libération et a fondé son
Église comme signe de cette pauvreté parmi les hommes [Medellin, « pauvreté de l’Église », n° 7].
\end{quote}

En allant un peu plus loin que G\&S, l'homme est celui qui anticipe dans ses conquêtes le signe annonciateur du Royaume : \mn{cf sur le refus de la souffrance à la mort, l'évolution du monde vers une humanisation ? } 
\begin{quote}
   « Le Christ activement présent dans notre histoire anticipe son geste eschatologique non seulement
dans l’impatiente aspiration de l’homme à sa rédemption totale, mais aussi dans ces conquêtes que,
comme autant de \textit{signes annonciateurs}, l’homme réalise progressivement à travers une \textit{activité
conduite dans l’amour} [Medellin, « Introduction », n° 5]. 
\end{quote}

On ne peut rejeter quelqu'un que s'il est présent. Mt 25 : identification de Jésus aux opprimés.  


\paragraph{1979 - Puebla} Option préférentielle pour les pauvres
\begin{quote}
    « Les pauvres sont les destinataires privilégiés de la mission de Jésus, et du simple fait d’être pauvres,
‘quelle que soit la situation morale ou personnelle dans laquelle ils peuvent se trouver’ (n° 1.142) ; il
est dit aussi que l’évangélisation des pauvres est signe et preuve par excellence de la mission de Jésus
(n° 1.142). De cette façon, on réaffirme la corrélation essentielle entre pauvres et mission de Jésus »
(Sobrino I, 64).
\end{quote}

Puebla introduit ce principe de partialité dans le Christ, les pauvres sont un quasi sacrement de Jésus : 

\begin{quote}
    « Les pauvres sont donc un \textsc{quasi-sacrement} en deux dimensions fondamentales de la mission de
Jésus : en premier lieu, ils appellent à la conversion, car leur propre réalité, comme celle de Jésus
crucifié, est la plus grande interpellation du chrétien et de l’être l’humain et en ce sens les pauvres
exercent une prophétie primaire en raison de ce qu’ils sont en tant que victimes.
En second lieu, ils offrent des réalités et des valeurs comme celles qui ont été offertes par Jésus, en ce
sens, ils sont porteurs d’un évangile, exercent une évangélisation primaire » (Sobrino I, 65).
\end{quote}

%----------------------------------
\section{Le lieu à partir duquel on élabore la christologie}

\subsection{}
2.1 Le « lieu christologique » comme une médiation pour mieux interpréter le Christ

\subsection{}
2.2 Lieu théologique et signe des temps

\subsection{}
2.3 Le signe de la présence du Christ : son corps

\subsection{}
2.4 L’Église des pauvres, comme principe épistémologique de la christologie

\subsection{}
2.5 La christologie en vue de la christopraxis


 \section{Le « Jésus de l’histoire » point de départ de la christologie}

\subsection{}
3.1 Le Jésus de l’histoire est nécessaire à la christologie

\subsection{}
3.2 Qu’est-ce qu’il faut entendre par le « Jésus de l’histoire » ?

\subsection{}
3.3 La priorité de la pratique sur l’intériorité ; de l’agir sur l’être

\subsection{}
3.4 La suite du Christ comme mystagogie

\subsection{}
3.5 Le retour à Jésus dans le NT au Ier siècle et en Amérique latine au XXe s.
