\chapter{De Jésus au Christ - 2}
\mn{31/1/23}
\section{Eléments bibliographiques}
\begin{itemize}
    \item CHÉNO, R., Dieu au pluriel. Penser les religions, Paris 2017.
    \item COMMISSION THEOLOGIQUE INTERNATIONALE, Le christianisme et les religions, Rome
1997.
    \item DALFERTH, I. U., « Inkarnation : Der Mythos vom inkarnierten Gott » dans Der auferweckte
Gekreuzigte, Tübingen 1994, 1-37.
    \item DUPUIS, J., Vers une théologie chrétienne du pluralisme religieux, tr. par O. PARACHINI,
Paris 1997.
  \item HICK, J., The Myth of God Incarnate, Londres 1977.
  \item KNITTER, P. F., No Other Name ? A Critical Survey of Christian Attitudes Toward the World
Religions, Maryknoll 1985.
  \item KNITTER, P. F., « La théologie catholique des religions à la croisée des chemins », Concilium
203 (1986) 129-138CHÉNO, R., Dieu au pluriel. Penser les religions, Paris 2017.
  \item COMMISSION THEOLOGIQUE INTERNATIONALE, Le christianisme et les religions, Rome
1997.
  \item DALFERTH, I. U., « Inkarnation : Der Mythos vom inkarnierten Gott » dans Der auferweckte
Gekreuzigte, Tübingen 1994, 1-37.
  \item DUPUIS, J., Vers une théologie chrétienne du pluralisme religieux, tr. par O. PARACHINI,
Paris 1997.
  \item HICK, J., The Myth of God Incarnate, Londres 1977.
  \item KNITTER, P. F., No Other Name ? A Critical Survey of Christian Attitudes Toward the World
Religions, Maryknoll 1985.
  \item KNITTER, P. F., « La théologie catholique des religions à la croisée des chemins », Concilium
203 (1986) 129-138
\end{itemize}

%--------------------------------------- 
\section{Introduction}

Un certain nombre de chrétiens ont voulu renouvelé le christianisme à partir de l'histoire. 

Shleiermacher :  Comment finalement des personnes charismatiques peuvent avoir de l'influence ? Mais pas limité au Christ.
Harnack : A partir du moment où l'homme vit de la relation à Dieu, il est messie.
\begin{marginfigure}
\includegraphics[width=\textwidth]{ChristologiePluraliste/Images/TintinFindesTemps.jpg}
\caption{La figure Charismatique peut être un soucis ...}

\end{marginfigure}



%--------------------------------------- 
\section{L’émergence de la théologie du pluralisme religieux}

\begin{Def}[théologie du pluralisme religieux]
modèle du paradigme : à l'image des sciences, on évolue de paradigme en paradigme
\end{Def}

\begin{Ex}
On passe du paradigme ptomélique, à Newton puis relativité.
\end{Ex}

D'après Kuhn, \href{https://fr.wikipedia.org/wiki/La_Structure_des_r%C3%A9volutions_scientifiques}{La Structure des révolutions scientifiques} \mn{intéressant à lire}



%--------------------------------------- 
\subsection{La théorie des trois paradigmes}
  Quel paradigme : 
\begin{itemize}
    \item Vision Exclusiviste, ecclesiocentriste : Le salut pas possible hors de l'Eglise. K. Barth, courant évangéliste actuel (Manille, 1986, "rien ne permet de dire que le salut peut s'obtenir en dehors du Christ confessé explicitement"). \textit{Concile de Florence} au XVI. \textit{Hors de l'Eglise point de salut. }
    \item Vision Inclusiviste, christocentrisme. Au XX. Pas un seul chemin pour aller au Christ.
\item vision théocentrique ou pluraliste. Plusieurs voix pour aller à Dieu
    
\end{itemize}

%--------------------------------------- 
\subsection{Le promoteur du « pluralisme » : J. Hick}

 \paragraph{John Hicks - 2012} D'abord évangélique puis protestantisme libéral après l'Inde. Double conversion. 

%--------------------------------------- 
\subsection{Qui est Paul F. Knitter ?}
 
\paragraph{F. Knitter} Catholique, Chicago, 1966 Rome. Cincinnati. 

1985 : \textit{No Other Name ? } Actes : pas d'autre nom pour être sauvé 


%--------------------------------------- 
\section{La christologie de Knitter : L’unicité relationnelle de Jésus}


\paragraph{Question} comment Jésus est il unique ? 

\paragraph{Histoire} Comment ce Jésus crédible il y a deux mille ans, est il toujours crédible aujourd'hui ? Ce rapport à l'histoire est spécifique aux Occidentaux. Histoire avec un progrès.

\paragraph{Ces approches empêchent le vrai dialogue avec les autres religions} Il faut accepter être impacter à l'autre. Sinon, ce n'est pas un dialogue. Soutient le modèle théocentrique. Jésus : unique mais sans absorber les autre figures prophétiques. 


\paragraph{part des textes bibliques} en même temps historique et théologique


\paragraph{se met à la place des premiers disciples de Jésus} Aspect historique au service de la théologie

%--------------------------------------- 
\section{L’unicité relationnelle de Jésus et la christologie du NT}
%--------------------------------------- 
\subsection{Jésus était théocentrique}
 
 
\paragraph{Jésus est théocentrique} au XX, Jésus annonce le \textit{Royaume de Dieu} : il ne s'annonce pas lui-même. 
Jésus est centré sur le Père. Il est \textit{Regno-centrique} (proche du théocentrisme). 

%--------------------------------------- 
\subsection{Du royaume de Dieu au Fils de Dieu}

 \paragraph{transformation du message du Christ à sa mort}
Dans le NT, il y a un message Christocentrique. Jn 20;28 : "mon Seigneur et mon Dieu". Il y a dejà une subordination du Christ au Père. On pense parfois trop la distinction en pensant au concile de Nicée. 

\paragraph{Jésus aurait fait une expérience de Dieu au début de sa mission} mais cela ne dit pas une exclusivité mais une mission unique. Il s'est pensé le prophète eschatologique de la fin des temps ?  

\paragraph{Loisy} mettait le doigt sur un problème. Expérience de salut faite par ses chrétiens. Les chrétiens ont rencontré Jésus, à la base de cette expérience. 

%--------------------------------------- 
\subsection{La christologie est, depuis le début, dialogique, pluriforme et évolutive.}




 \paragraph{De la diversité des christologies à l’unité de la christologie}
 \paragraph{Revenir à la diversité des christologies}
4 visions de christologie par les chrétiens :

\subparagraph{Christologie Marana tha} contexte apocalytique, jésus vient juger à la fin du temps. Jésus est Seigneur (Mar).
\paragraph{Jésus divin} il est divin dans le sens où il est inspiré par Dieu : \textit{il fait des miracles}. Mais il n'est pas une personne divine au sens ontologique du terme.

\subparagraph{Sagesse ou logos} Jésus est la personnification de la sagesse dont le sens qu'il révèle Dieu.

\paragraph{Théologie pascale} elle n'est pas au centre pour Knitter. Les premiers chrétiens vont diffuser le message. 

%--------------------------------------- 
\section{Unicité et exclusivité du Christ dans le NT}


\paragraph{Pour Knitter, aucune des figures de l'AT n'exprime tout ce qu'est Jésus}
Il a fallu un langage (de l'AT, grec,...) pour dire Jésus. Et donc dialogue.

\paragraph{Cela entraîne une diversité de regards}.

\paragraph{Dimension critique} On va uniformiser ces christologies : on va passer de la vision eschatologique de fils de Dieu (à la fin) à une vision préexistante (le verbe de Dieu s'incarne). Et ceci à la deuxième génération des chrétiens. 



%--------------------------------------- 
\subsection{Le contexte historico-culturel} 


\paragraph{5 pistes d'interprétation des figures du Christ} Il y a dans chaque contexte culturel, on peut utiliser une image un peu différente.

Ce sont des \textit{mythes}, c'est à dire des images et des symboles (cela nous donne un accès) à Jésus. Le mystère du Christ est indéfinissable et donc il ne faut pas absolutiser une image au profit des autres : aucune image ne dit tout. 

\paragraph{Ce qui a été fécond au début, le dialogue, doit continuer aujourd'hui} pour avoir de nouvelles images qui disent Jésus aujourd'hui. Cela a été positif au départ. 

\begin{Synthesis}
Positivité du dialogue avec les religions et la culture au début du christiansime pour dire le Christ.
\end{Synthesis}

\paragraph{Historiciser le langage du Christ} Quand on fait une expérience très forte, on a tendance à dire qu'elle est définitive et exclusive. \mn{Voir discussion avec Charbel attalah : pourquoi les soufis spirituels sont exclusivistes ? }

\paragraph{Statut minoritaire des Juifs dans l'Empire} pour ne pas être absorbé, on utilise un langage de survie, c'est à dire on insiste sur le Christ \textit{définitif}, \textit{insurpassable}.


 
 %--------------------------------------- 
\subsection{« Unique et seul » - les traits du langage confessionnel}


 \paragraph{C'est dans un langage amoureux et enthousiaste} "tu es mon unique". Ce n'est pas un langage métaphysique.   Différent du langage de raison. C'est ainsi qu'il comprend les passages sur "sauveur", "unique".

 

%--------------------------------------- 
\section{La réinterprétation de l’incarnation dans la théologie actuelle}
 

%--------------------------------------- 
\section{La question de la résurrection}


 \paragraph{Sous-estime l'évènement pascal} Dieu intervient dans le monde

 \paragraph{Quand Rahner dit que Jésus accomplit l'homme} pourquoi seulement Jésus serait celui qui accomplit l'homme ? 

 \paragraph{C'est la foi qui est à l'origine de la résurrection} Bultmann. C'est l'expérience des disciples est première. Pour dire qu'il est médiateur, l'expérience des disciples va susciter les récits d'expérience d'apparition pour dire que \textit{Jésus est vivant}.

\begin{quote}
  Jésus est ressuscité dans la Foi des apôtres (Bultmann)  
\end{quote}

 \paragraph{Expérience de Jésus par les disciples toujours essentielle}








 
%--------------------------------------- 
\section{Critique de la proposition de la théologie pluraliste}

\begin{Synthesis}
    Est ce que Jésus témoigne de l'action lui même de Dieu ? Action de Dieu Première ?
    Ou au contraire, expérience religieuse des disciples qui est première.
\end{Synthesis}
 \subsection{Au point de vue philosophique : la question de la vérité}

 \subsection{Au point de vue de l’exégèse biblique et néotestamentaire}
 \subsection{Au point de vue de la théologie et du salut}
  \paragraph{Avec la théologie libérale, d'autres chemins du salut} L'expérience religieuse est absolutisée (l'action de Dieu est secondaire). Alors que pour les Chrétiens, la Foi chrétienne atteste de l'action de Dieu qui ressuscite Jésus d'entre les morts. C'est le \textbf{fondement de notre foi}. 
 \paragraph{Dans la théologie libérale, on confond la révélation de Dieu et la manifestation divine} Pour les Chrétiens, dernière parole de Dieu dans l'histoire. 
La résurrection est la fin de l'histoire : action définitive de Dieu; vision eschatologique. C'est la dernière parole. 