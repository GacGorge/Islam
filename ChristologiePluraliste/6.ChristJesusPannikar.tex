\chapter{Du Christ à Jésus II
L’approche de Panikkar
}

\mn{Christologies au défi de la culture pluraliste}

\section{Eléments bibliographiques :}

\begin{itemize}
    \item DUPUIS, J. Vers une théologie chrétienne du
pluralisme religieux, tr. par O. PARACHINI,
Paris 1997.
    \item PANIKKAR, R., Le Christ et l’hindouisme. Une présence cachée, Paris 1972
    \item PANIKKAR, R., Entre Dieu et le cosmos. Entretiens avec G. JARCZYK, Paris 20122.
    \item PANIKKAR, R., OEuvres VIII. Vision trinitaire et cosmothéandrique : Dieu-Homme-Cosmos, tr.
par M. BACCELLI, Paris 2013.
\end{itemize}

\section{Introduction}

\section{Au-delà d’une théorie de l’accomplissement : Le Christ inconnu
}

\subsection{Le Christ est présent dans les autres religions comme un principe caché}
 
\subsection{La réalité inconnue à la base de toutes les religions} 

\section{La vision de la réalité (ou du mystère) selon Panikkar}
\subsection{Le contexte vital}
 
\subsection{L’aporie des visions moniste et dualiste de la réalité}
 
\subsection{La voie indienne : « la vision a-dualiste » rejoint la voie chrétienne : « la vision
trinitaire »}
 
\section{La christologie de Panikkar : l’interprétation du Christ dans la vision
cosmothéandrique de la réalité} 

\subsection{Le Christ comme le symbole de la réalité humano-divino-cosmique}
  
\subsection{Quelle est la relation entre ce Christ et le Jésus historique ?}
 


a) Le présupposé : la distinction entre la religion et le mystère


b) Le Christ est plus grand que le Jésus historique


c) Comme Jésus, tout homme peut devenir Christ

\section{Evaluation critique de la christologie de Panikkar}

\section{Conclusion}


\section{Textes}

\begin{quote}
    « Si nous visons l’unité (…), à la longue la pluralité, réduite à l’irréel, se rebelle et fait valoir
ses droits. Si nous visons la pluralité (…), c’est la lutte de tous contre tous et la destruction
réciproque » (Panikkar, Vision trinitaire, 233).
\end{quote}

\begin{quote}
    
\end{quote}
\begin{quote}
    
\end{quote}
« La vision a-dualiste affirme : ni Un ni Multiple, parce que la réalité ne se soumet pas aux 
exigences de l’esprit et résout le dilemme en le niant parce qu’il n’y a pas l’Un sans la 
Multiplicité ni la Multiplicité sans l’Un, puisque la réalité est pure relation et que la polarité 
est ce qui la caractérise » (Panikkar, Vision trinitaire, 241).

\end{quote}
\begin{quote}
    
« Cette vision nous dit que la réalité n’est formée ni d’un bloc unique indistinct (…) ni de trois 
blocs ou d’un monde à trois niveaux – le monde de Dieu (ou de la Transcendance), le monde des 
hommes (ou de la Conscience) et le monde physique (ou de la Matière), comme s’il s’agissait d’un 
édifice à trois étages. La réalité est constituée par les trois dimensions en relation l’une avec 
l’autre – la perichorèsis trinitaire, de sorte que non seulement l’un n’existe pas sans l’autre, 
mais que toutes sont tressées, inter-dépendamment. Pris séparément, ou en soi, sans relation avec 
les autres dimensions de la réalité, Dieu, le monde et l’homme sont de simples abstractions de 
notre esprit » (Panikkar, Vision trinitaire, 16).

\end{quote}
\begin{quote}
    
« Tout est intégré, assumé, transfiguré. Rien n’est envoyé dans le futur : la présence tout entière 
est ici (…). Rien n’est laissé de côté ou considéré comme non rachetable, y compris le corps et la 
mémoire humaine. La transfiguration n’est pas la vision d’une réalité plus belle ni une évasion sur 
un plan plus élevé : elle est l’intuition totalement intégrée du tissu sans coutures de la réalité 
tout entière » (Panikkar, Vision trinitaire, 225).

\end{quote}
\begin{quote}
    
« Nous ne sommes jamais en dehors de la Trinité. Il n’y a rien en dehors de ce qu’on appelle Dieu 
ou, le divin (…) La réalité ultime est trinitaire : elle est divine, humaine et cosmique. C’est en 
cela même que consiste l’intuition cosmothéandrique. (…). Cosmothéandrique serait donc cette 
vision, cette expérience, de ce que nous sommes une partie de la Trinité, et qu’il y a trois 
dimensions du réel : une dimension d’infini et de liberté que nous appelons divine ; une dimension 
de conscience, que nous appelons humaine  et une dimension corporelle ou matérielle que nous 
appelons le cosmos » (Panikkar, Entre Dieu et le cosmos, 135).

\end{quote}
\begin{quote}
    
« Tout a la même origine ; tout est en relation ; l’univers tout entier est une famille, un macro- 
organisme : des liens de ‘sang’, pour ainsi dire, animent tout ce qui est. Nous sommes de la même 
race. Nous sommes les membres démembrés de son Corps. Notre devoir (…), c’est de re-member le Corps 
démembré, de le recomposer, c’est-à-dire de guérir et d’intégrer tous les membres disjoints de la 
réalité, éparpillés à travers espace et temps. L’énergie pour atteindre ce ‘salut’ peut venir de 
plusieurs directions, mais elle n’a qu’une seule source. Et c’est l’aventure de toute la réalité » 
(Panikkar, Vision trinitaire, 251).

\end{quote}
\begin{quote}
    
« Le Dieu créateur doit être interprété, non tel qu’il figure au début de la Genèse, mais tel que 
le présente saint Jean lorsqu’il dit du Verbe de Dieu que ‘toutes choses ont été faites par lui’ 
(Jn 1,3), il est clair […] que le Logos est incarné depuis le commencement, bien que dans
l’histoire il n’apparaisse qu’à la plénitude des temps. D’après la tradition elle-même, la réalité 
toute entière – le cosmos – est alors une christophanie, et pas simplement la théophanie d’un Dieu 
tel qu’on l’imagine dans une perspective plus ou moins gnostique » (Panikkar, Entre Dieu et le 
cosmos, 132).

\end{quote}
\begin{quote}
    
« Il y a, de façon certaine, d’importantes différences entre ces deux mythes, mais l’un et l’autre 
s’entendent pour dire qu’une Unité indifférenciée, un Principe mystérieux, sortit de la solitude, 
se dégagea de l’inactivité, créa, produisit, donna naissance à l’existence, au temps, à l’espace et 
à tout ce qui se meut en eux (…). Cette Origine crée, produit, engendre, et se divise précisément 
parce qu’elle ne veut pas être seule. Mais ce n’est possible que parce qu’elle devient conscience 
d’elle-même. Cette conscience rend le Principe conscient de soi, visible dans son reflet, 
c’est-à-dire réel. C’est un double mouvement : l’un au sein du Principe lui-même, et l’autre, pour 
ainsi dire, ‘vers’ l’extérieur. Dieu génère et crée ; Il se démembre et engendre le monde, l’Un 
devient la source cachée et produit la multiplicité. L’homme émerge de ce processus. L’homme a donc 
la même origine que le cosmos, la même source, c’est-à- dire le même pouvoir que le divin qui 
s’ébranle au commencement. Tous trois coexistent. ‘Avant’ la création, le Créateur n’était certes 
pas créateur ; avant les ‘nombreux’, l’Un n’était même pas un. Et pourtant ce dynamisme ne va que 
dans une seule direction : l’Un est dans l’origine, il est l’Origine, mais il ne l’est qu’en tant 
qu’origine. En soi il n’est rien. » (Panikkar, Vision trinitaire, 250).

\end{quote}
\begin{quote}
    
« Pour moi, en effet, le Christ n’est pas d’abord un être historique, achevé en lui-même et non 
soumis à évolution – ce qui fait de lui facilement un objet de superstition -, mais une réalité 
autrement vivante et personnelle. La considération d’un Christ à la fois divin et humain modifie en 
profondeur la vision que nous avons de Dieu, et du même coup aussi notre vision de l’homme. […] Je 
vois dans le Christ non seulement la révélation de ce qu’est l’homme, mais aussi de ce que Dieu est 
» (Panikkar, Entre Dieu et le cosmos, 132-133).

\end{quote}
\begin{quote}
    
« Sous leur aspect institutionnel, [les religions] en sont venues à construire des systèmes de 
doctrine en étroite dépendance par rapport à la culture ambiante ; de là aussi leurs différences, 
qui très souvent les rendent incompatibles. Il n’en va pas de la sorte avec la dimension mystique : 
parce qu’elle est en prise (…) sur le noyau ineffable de la réalité, il semble aller de soi que la 
vision qu’elle développe doit valoir pour tout le monde. Dans la partie je vois le tout. Et c’est 
juste – mais de ce totum je ne puis parler que in parte. […] Ce dont je ne prends pas assez 
conscience alors, c’est que je vois le totum in parte, le tout dans la partie, et souvent, hélas, 
je prends la partie pour le tout. » (Panikkar, Entre Dieu et le cosmos, 19-20).

\end{quote}
\begin{quote}
    
« Si l’on conçoit que le phénomène Christ est quelque chose de plus qu’un phénomène historique, il 
ne s’agit pas de revenir en arrière. Les religions, comme formes vivantes et expressions les plus 
profondes de la culture humaine, naissent, s’épanouissent, croissent, décroissent et se 
transforment » (Panikkar, Entre Dieu et le cosmos, 164).

\end{quote}
\begin{quote}
    
« Jésus a affirmé qu’avant Abraham il était. Or ce n’était pas le fils de Marie qui était avant 
Abraham. Dans l’eucharistie, il y a la présence réelle du Christ. Or celui qui communie ne consomme 
pas les protéines du fils de Marie. Le Christ, tel qu’en rend compte toute la tradition chrétienne, 
est alpha et oméga. Il est […] l’unique engendré et le premier né, celui qui dès le commencement a 
fait toutes choses. Ce Christ, une fois encore, n’est pas immédiatement identique à Jésus : il 
s’agit de les distinguer sans les séparer. Les chrétiens peuvent trouver le Christ dans et par 
Jésus – n’a-t-il pas dit : ‘Je suis la voie’ (Jn 14,6) ? Mais le Christ dépasse infiniment la 
figure de Jésus. Jésus est le Christ : ce sont les chrétiens qui le
confessent, c’est-à-dire l’itinéraire qu’ils ont à suivre. Christ, c’est le nom que les chrétiens 
donnent à ce mystère qu’ils ont découvert dans et à travers Jésus » (Panikkar, Entre Dieu et le 
cosmos, 36).

\end{quote}
\begin{quote}
    
« Les hommes en s’éveillant à la réalité ont découvert qu’il y avait bien plus que ce qui se laisse 
découvrir par les yeux, bien plus que ce que l’on pense au moyen de la raison, et qu’il y a encore 
quelque chose d’ineffable, et qui ne peut pourtant se manifester que si cette troisième dimension 
s’incarne, utilise des mots, des expressions que l’on trouve dans le monde sensible et dans le 
monde intellectuel. Et comme le monde sensible et le monde intellectuel sont vus et vécus de façon 
très différente, il suit de là […] qu’il y a multiplicité de religions. […] Les hommes ne sont pas 
tous faits de même manière ni ne voient les choses de façon uniforme. Chaque  personne  en  fait  
est  révélation,  dans  le  mouvement  même  de  son autocompréhension » (Panikkar, Entre Dieu et 
le cosmos, 18).

\end{quote}
\begin{quote}
    
« Je suis donc parfaitement convaincu que le Christ peut être considéré comme le symbole – 
c’est-à-dire la récapitulation ou l’abrégé – de toute la réalité. Mais aussi bien le Bouddha et 
d’autre encore. La diversité des religions, ce sont comme les couleurs différentes de cette réalité 
multidimensionnelle (…). Chaque religion a une prétention légitime de catholicité, au sens profond 
où elle permet à la personne concrète de parvenir à sa propre plénitude (salut,
perfection) (Panikkar, Entre Dieu et le cosmos, 20).





\end{quote}
 




























































Cours ISTR 2022-2023                 2

Christologies au défi de la culture pluraliste                          du Christ à Jésus II
Cours ISTR 2022-2023                 3
confessent, c’est-à-dire l’itinéraire qu’ils ont à suivre. Christ, c’est le nom que les chrétiens 
donnent à ce mystère qu’ils ont découvert dans et à travers Jésus » (Panikkar, Entre Dieu et le 
cosmos, 36).

« Les hommes en s’éveillant à la réalité ont découvert qu’il y avait bien plus que ce qui se laisse 
découvrir par les yeux, bien plus que ce que l’on pense au moyen de la raison, et qu’il y a encore 
quelque chose d’ineffable, et qui ne peut pourtant se manifester que si cette troisième dimension 
s’incarne, utilise des mots, des expressions que l’on trouve dans le monde sensible et dans le 
monde intellectuel. Et comme le monde sensible et le monde intellectuel sont vus et vécus de façon 
très différente, il suit de là […] qu’il y a multiplicité de religions. […] Les hommes ne sont pas 
tous faits de même manière ni ne voient les choses de façon uniforme. Chaque  personne  en  fait  
est  révélation,  dans  le  mouvement  même  de  son autocompréhension » (Panikkar, Entre Dieu et 
le cosmos, 18).

« Je suis donc parfaitement convaincu que le Christ peut être considéré comme le symbole – 
c’est-à-dire la récapitulation ou l’abrégé – de toute la réalité. Mais aussi bien le Bouddha et 
d’autre encore. La diversité des religions, ce sont comme les couleurs différentes de cette réalité 
multidimensionnelle (…). Chaque religion a une prétention légitime de catholicité, au sens profond 
où elle permet à la personne concrète de parvenir à sa propre plénitude (salut,
perfection) (Panikkar, Entre Dieu et le cosmos, 20).
