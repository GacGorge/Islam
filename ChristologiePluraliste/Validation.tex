%\chapter{l'Unique et ses témoins - Ch. Theobald}


Dans son livre \textit{Le Christianisme comme style, une manière de faire de la théologie en post-modernité}\cite{theobald_christianisme_2007}, Christoph Theobald prolonge sa théologie vers le dialogue avec les religions mono-theistes dans un chapitre que nous nous proposons d'étudier, intitulé : \textit{l'Unique et ses témoins, Jalons pour une théologie de la rencontre entre juifs, chrétiens et musulmans}.  




% --------------------------------------------------------------------------------------------------------------------------

\begin{comment}
\section{instruction}
    Rédaction d’un travail de 8 pages
Vous faites au préalable une recherche documentaire afin de choisir un article, un chapitre
d’un ouvrage ou un ouvrage dans lequel la christologie est interrogée par la culture
postmoderne et le pluralisme religieux.
Puis vous rédigez votre travail selon deux grandes parties :
\begin{itemize}
    \item  Présentez le document choisi et la manière dont l’auteur pense la confrontation de la
christologie à la culture actuelle.
    \item  Vous réagissez personnellement au texte en le discutant et le critiquant de manière
argumentée.
\end{itemize}

Le texte choisi doit être au préalable validé par le professeur
Merci de suivre les normes universitaires : « Normes de présentations de mémoires »
Votre travail écrit doit être déposé sur l’ENT (espace dédié) et la date limite pour la
remise de votre travail est le 1er mai 2023

POints d'attention : 
\begin{itemize}
    \item Christologie
    \item Récent (Grieu,...)
    \item question Ecologie ?
\end{itemize}
\end{comment} 


 

% --------------------------------------------------------------------------------------------------------------------------
\section{\textit{christianisme comme style}}
% --------------------------------------------------------------------------------------------------------------------------


% --------------------------------------------------------------------------------------------------------------------------
\subsection{Christoph Theobald}

\paragraph{Théologien jésuite allemand vivant en France.} Christoph Theobald est un théologien jésuite d'origine allemande et vivant en France. Véritable pont entre les deux cultures théologiques, il est en particulier connu pour sa traduction des oeuvres de Karl Rahner. Il est par ailleurs un spécialiste de Maurice Blondel et d'Alfred Loisy. Il est très marqué par la philosophie française, Merleau-Ponty, Lévinas et Ricoeur. 

\paragraph{Curieux de la théologie contemporaine} Il a été longtemps le directeur des RSR, la \textit{Revue des Sciences Religieuses}, présentant les dernières recherches de théologie fondamentale au public français. Il n'hésite pas à se confronter aux questions de son temps, comme celle de \textit{la violence entre religions} que nous allons étudier. 
  


% --------------------------------------------------------------------------------------------------------------------------
%\subsection{Bibliographie}
%\sn{Eventuellement, notez l’ensemble des textes et références auxquels vous avez eu recours pour préparer l’exposé, y compris vos sources pour la biographie et les sites internet visités. }




% --------------------------------------------------------------------------------------------------------------------------
\subsection{Contexte historique et textuel du chapitre étudié}


\paragraph{D'abord un colloque.} Dans sa \href{https://centresevres.com/content/uploads/2017/07/bibliographie-complete-de-christoph-theobald-sj-2021.pdf}{bibliographie}, on note que le texte étudié a eu trois moutures différentes publiées dont deux que nous étudierons : la version publiée en 1996 du colloque de rentrée au Centre Sèvres, \textit{L'unique et ses témoins, Judaïsme, christianisme et islam, histoire et théologie d'une rencontre }\cite{centre_sevres_paris_unique_1996} et un chapitre éponyme de \textit{Christianisme comme Style} \cite{theobald_christianisme_2007}. 


 \paragraph{Un contexte de violence entre religions.}

L'étude de la version publiée du colloque \cite{centre_sevres_paris_unique_1996} et celle du livre \cite{theobald_christianisme_2007} montre des différences significatives. Mais ces différences viennent-elles d'un changement de contexte ? La longue première note dans le livre sur la polémique de Ratisbonne pourrait le faire penser : En 2006, le pape Benoit XVI a donné dans cette ville une conférence intitulée \textit{Foi, Raison et Université}, en citant les échanges entre l’empereur  Manuel II Paléologue et un savant musulman sur le lien entre raison et religion. Benoit XVI s'oppose à la fois au rejet de la raison par certaines tendances de l'islam et  du christianisme et à la fois à ce qu'il appelle une "exclusion du divin" du champ de la raison en Occident \cite[p.782, note 1]{theobald_christianisme_2007}. Le chapitre du livre se veut une réponse par rapport à la première opposition du Pape.
Même si le contexte ne nous semble pas avoir été la raison des changements entre les deux textes étudiés, on peut noter dans le livre l'écho de la polémique de Ratisbonne quand l'A. mentionne que lorsque on nie les différences propres de chaque religion, on peut susciter \textit{ aujourd'hui des nouvelles violences et des soubresauts identitaires de la part de l'islam}.\cite[p. 787]{theobald_christianisme_2007}
\begin{comment}
    raison historique émerge au sein même du prophétisme biblique sous la forme d'une capacipté éthique d'apprentissage qui n'a pas le droit de dénioer a priori à l'un des interlocuteurs.

    
Cependant, la référence à la violence entre religions, contre-témoignage, ne semble pas faire directement référence à la vague de violence que connut le monde musulman à la suite de cette polémique, l'article du colloque en faisant aussi son point de départ.
\end{comment}

\paragraph{Un contexte philosophique questionnant les religions et en particulier le christianisme } La lecture de Ch. Theobald n'est pas toujours aisé car il est en dialogue constant avec la philosophie contemporaine dont il reprend souvent les termes. Ainsi, il nous semble que l'extrait suivant de \textit{Foi et Savoir} de J. Derrida paru en 1996 est intéressant pour comprendre le contexte philosophique avec lequel il discute (italiques ajoutés) : 

\begin{singlequote} 
[Dans] ce qui lie indissolublement l'idée de la moralité pure à la révélation chrétienne, Kant recourt à la logique d'un principe simple [\ldots]: pour se conduire de façon morale, il faut faire en somme comme si Dieu n'existait pas ou ne s'occupait plus de notre salut. Voilà qui est moral et qui est donc chrétien.[\ldots] N'est-ce pas une autre façon de dire que le christianisme ne peut répondre à sa vocation morale et la morale à sa vocation chrétienne \textit{qu'à endurer ici-bas, dans l'histoire phénoménale, la mort de Dieu,et, bien au-delà, des figures de la Passion ?}  [\ldots] Le judaïsme et l'islam seraient peut-être alors les deux derniers monothéismes à s'insurger encore contre tout ce qui, dans la christianisation de notre monde, signifie la mort de Dieu, la mort en Dieu, deux monothéismes non païens qui n' admettent pas plus la mort que la multiplicité en Dieu (la Passion, la Trinité, etc.), deux monothéismes encore assez étrangers au cœur de l' Europe gréco-chrétienne, pagano-chrétienne, assez étrangers à une Europe qui signifie la mort de Dieu, pour rappeler à tout prix que « monothéisme» signifie autant la foi en l'Un, et en l'Un vivant, que la croyance en un \textit{Dieu unique}.

Au regard de cette logique, de sa rigueur formelle et de ses possibles, Heidegger ne fraye-t-il pas un autre chemin ? Il insiste en effet dans \textit{Sein und Zeit} sur le caractère à la fois pré-moral (ou \emph{pré-éthique} [\ldots])  et pré-religieux de la « conscience » (Gewissen), de l' être-responsable-coupable-endetté (Schuldigsein) ou de l'attestation (Bezengung) originaires. \textit{On reviendrait ainsi en deçà de ce qui soude la morale à la religion, c'est-à-dire ici au christianisme. }Ce qui permet en principe de répéter la généalogie nietzschéenne de la morale, mais en la déchristianisant davantage là où ce serait nécessaire, en déracinant ce qui lui resterait de souche chrétienne. Stratégie d'autant plus retorse et nécessaire pour Heidegger que celui-ci n'en finit jamais de s'en prendre au christianisme ou de se déprendre de lui - avec d'autant plus de violence qu'il est trop tard, peut-être, pour dénier certains motifs archi-chrétiens de la répétition ontologique et de l'analytique existentiale.     \cite[p. 22]{derrida_foi_2001}
\end{singlequote}
Face à "cette mort de Dieu" qui serait la caractéristique du monothéisme chrétien, le judaïsme et l'islam promouvraient un monothéisme vivant. Pour Heidegger au contraire, ce serait cet homme coupable qui serait \textit{pré-éthique}, et non le christianisme qui viendrait après. Même si ce texte de J. Derrida n'est pas directement l'arrière fond du contexte de Ch. Theobald (l'A. cite ce livre mais dans un autre chapitre), il illustre le contexte, le vocabulaire et les questions que pose la philosophie  de la fin du XXè siècle : lien entre religion et éthique ou pré-éthique, assimilation du christianisme à la morale, question de la mort de Dieu portée par le christianisme, Dieu unique,\ldots  

 



\paragraph{Vivre ensemble dans une société plurielle} Le lien social n'est plus fondé sur le religieux mais sur une instance neutre, l'Etat, qui impose ses règles du jeu à toute rencontre. Penser théologiquement ce lien social, sans condamner le christianisme à une certaine marginalisation, est de l'ordre de l'énigme et requiert un certain "apophatisme" (penser ce qu'il n'est pas) \cite[p. 771]{theobald_christianisme_2007}, à l'image de l'article de foi sur l'Esprit qui est très court dans le Credo de Constantinople. La frontière entre la règle dogmatique et la créativité spirituelle et théologique ne peut qu'être une position fragile et inconfortable de l'Eglise \cite[p. 777]{theobald_christianisme_2007}.

\begin{comment}
    si on a le temps, lire Esprit 2004:17
\end{comment}
 
\paragraph{Le \textit{style}, indispensable expression des religions en post-modernité}  Pour éviter l'insignifiance, les chrétiens doivent affirmer non seulement leur foi mais aussi leur style de vie, qui se connait à sa méthode mais doit tenir le paradoxe d'une ouverture à l'imprévu.  \cite[p.10]{theobald_christianisme_2007} 


% --------------------------------------------------------------------------------------------------------------------------
\section{Présentation du texte}
% --------------------------------------------------------------------------------------------------------------------------

\paragraph{\textit{Christianisme comme Style}.} Théobald a publié le \textit{Christianisme comme style, une manière de faire de la théologie en postmodernité}\cite{theobald_christianisme_2007} en 2007. Le chapitre que nous étudions se trouve dans la partie IV ainsi introduite : 

 \begin{singlequote}
     Après avoir "ausculté" notre présent et désigné le \textit{kairos} qu'il représente (I) et après avoir réfléchi longuement à l'enracinement spirituel (II) et scripturaire (III) de la théologie chrétienne, le moment est venu d'aborder directement ce que celle-ci doit donner à penser aujourd'hui : le christianisme comme style qui ouvre à une intelligence de lui-même, libre et accessible à tous, petits et grands. 
     \cite[p 699]{theobald_christianisme_2007}
 \end{singlequote}
 
l'A. organise cette partie à partir du \textit{Credo}, en appliquant une \textit{herméneutique dogmatique} \cite[p. 700]{theobald_christianisme_2007}  qui fait le va et vient entre le dogme, versant normatif du mystère chrétien, les textes canoniques, l'histoire et la pratique actuelle de l'Eglise. Le plan porte la trace de cette question : 
\begin{itemize}
    \item A. Croire en Dieu
    \begin{singlequote}
        Celui qui [croit] se laisse atteindre, au coeur de sa propre existence, par ce que Dieu est en lui-même et qu'il rend accessible en Jésus de Nazareth, grâce à son Esprit de Sainteté \cite[p. 699]{theobald_christianisme_2007}
    \end{singlequote}
    \item  B. Dans l'Eglise 
    \begin{singlequote}
        dont la raison d'être est de percevoir et de susciter cette sainteté au sein même des sociétés humaines; ce qui suppose qu'elle soit au bon endroit \cite[p. 699]{theobald_christianisme_2007}
    \end{singlequote} 
    \item C \textit{située dans l'ouverture  messianique de la création}, qui reprend à frais nouveau une \textit{théologie de la Création}, 
\end{itemize}
 
La partie A commence par une réflexion  sur la foi en Dieu et de la Trinité en post-modernité, en lien avec l'\textit{éthos} chrétien : 
\begin{comment}
 Je voudrais donc penser la marque eschatologique de l'éthos chrétien comme \textit{lien constitutif} entre l'accès de la foi \textit{en l'intimité} même de "Dieu" visée par le vocabulaire de la "sainteté", et une réflexion engagée, par définition sans garantie, du sujet croyant  sur lui-même et sur l'Eglise \textit{dans la cité et dans l'histoire}. [\ldots]
\end{comment}
\begin{comment}
 
    Or, cette forme de vie - l'éthos chrétien - se caractérise en définitive par ce que la tradition biblique appelle "sainteté" : 
    \textit{dire Dieu est alors une manière de désigner la sainteté comme mystère du monde et de l'histoire} \cite[p. 705]{theobald_christianisme_2007}
     La \textit{règle dogmatique } implique à sa racine la \textit{force d'effacement} de l'Eglise au service d'une 
\end{comment}
\begin{comment}
   
    Comment caractériser « l’accès à la foi au Christ, compte tenu du point de départ éthique de la christologie, appelé par nos sociétés néo-libérales toujours tentées d’oublier les menaces qui continuent à peser sur le lien social ? On peut, dans la perspective d’une christologie pratique, le définir comme ‘mue d’identité’ qui s’exprime par le passage à un\textit{ style de vie messianique}, à une manière spécifique de se situer dans la société globale » (style**, 813).
 
\end{comment}

\begin{singlequote}
    On pourrait certes envisager une relation "binaire" entre l'homme et le tout Autre, relation d'alliance avec le "Dieu unique" dont Abraham reste le prototype, mais notre entrée - à égalité\;- dans une relation de familiarité avec Dieu, telle que Jésus de Nazareth l'a risquée avec son "Père", resterait inconcevable; or, c'est cet accès, surprenant et apparemment excessif pour nos possibilités humaines, que nous attribuons à l'"Esprit de Sainteté". \cite[p. 705]{theobald_christianisme_2007}
 \end{singlequote}
 
 Puis Theobald explore le lien entre foi trinitaire des chrétiens et lien social, pour enfin couvrir la question de la multiplicité des témoins et religions monothéistes, le chapitre qui nous intéresse : \textit{ l'Unique et ses témoins, Jalons pour une théologie de la rencontre entre juifs, chrétiens et musulmans.}

Après ce chapitre, il explore la partie proprement christologique du livre. 
% --------------------------------------------------------------------------------------------------------------------------
\paragraph{Problématique de la violence entre religions monothéistes.}

\begin{comment}
   Déterminer la problématique en vous inspirant des questions suivantes :
Après la phase de lecture pas à pas, vous construisez la question à laquelle l’auteur s’affronte :
-	pourquoi l’auteur se bat-il ?
-	quel problème essaie-t-il de régler, d’éclairer ? En général, il n’est pas difficile de trouver exposée la problématique en toute lettre dans le texte lui-même – parfois même de manière redondante ;
-	en fonction de quel contexte culturel, social, culturel, économique, politique l’auteur construit-il sa problématique ? Notez les événements déterminants auxquels il se réfère et les auteurs qu’il évoque – plus ou moins explicitement – comme ses alliés ou ses adversaires – et le cas échéant, renseignez-vous sur eux.  
\end{comment}
La problématique se déplace sensiblement entre l'article du colloque et le chapitre éponyme : le colloque tente d'éclairer la question de l'unicité de Dieu à travers plusieurs religions ou \textit{témoins} et pour cela, fait \textit{jouer} le système \textit{Dieu unique, témoin} (prophète,...) et \textit{tiers} (Egypte ou Nations pour Israel, juifs et grecs pour les Chrétiens,...). 
Le chapitre  du livre questionne quant à lui  \textit{l'énigme de la violence entre les trois témoins} et de la difficulté à communiquer entre religions \cite[p.780]{theobald_christianisme_2007}. Cette violence discrédite les religions monothéistes pour nos contemporains. Comme il n'est pas possible de répondre à la place de l'autre, la réponse viendra ici d'une réflexion proprement chrétienne sur la rencontre entre religions.

% --------------------------------------------------------------------------------------------------------------------------
\paragraph{Hypothèse que le mystère du Christianisme permet de définir une théologie de la rencontre.}

\begin{comment}
    
En fonction de la problématique de l’auteur, vous établissez la thèse (ou hypothèse) de l’article ou du texte étudié : quelle solution l’auteur apporte-t-il à sa question ? quelle perspective établit-il ? Là encore, il n’est pas difficile de trouver cette thèse exposée de manière explicite dans le texte lui-même. 

\end{comment}
 Méditant la figure de Melchisédech, son hypothèse est que le mystère de l'Incarnation et de la Trinité - différence fondamentale du christianisme par rapport au judaïsme et à l'islam - ,  est en même temps le lieu  où se définit une\textsc{ théologie de la rencontre} \cite[p. 793]{theobald_christianisme_2007}. 
 
 


% --------------------------------------------------------------------------------------------------------------------------
\subsection{Une argumentation autour de la rencontre comme apprentissage}

\begin{comment}
   Présentez la logique argumentative en fonction de laquelle l’auteur établit sa thèse (passe de l’hypothèse à la thèse vérifiée). 
\end{comment}

La logique argumentative est la suivante : 

\paragraph{Présentation des trois monothéismes}  L'auteur reprend l'hypothèse de classement du judaïsme, du christianisme et de l'islam présentée lors du colloque \cite[p.13]{centre_sevres_paris_unique_1996} et la systématise, appelant :
\begin{itemize}
    \item le judaïsme, \textsc{monothéisme éthique}, Israël se définissant par l'injonction éthique d\textit{'aimer l'étranger car en Egypte, vous fûtes des étrangers} (Dt 10,17-1).  
    \item le christianisme, \textsc{monotheisme méta-éthique} au sens où il insiste sur la communication de l'amour excessif de Dieu à tout être humain.
    \item l'islam, \textsc{monothéisme pre-éthique} car sa lutte pour l'unicité de Dieu surpasse toute préoccupation éthique. 
\end{itemize}
 A l'issue de cette première comparaison, une première difficulté de la communication entre religions apparaît car elles ne se placent pas aux mêmes plans. 


\paragraph{La comparaison entre religions doit se situer au niveau de leur \textit{style}.} Nous l'avons vu, la modernité impose ses règles du jeu dans la communication entre religions, et en particulier elle impose le \textit{comparatisme}, ce qui peut se traduire par une réelle violence. Chaque religion est donc invitée à relire son propre patrimoine et à entrer dans le \textit{jeu difficile d'une communication qui consiste désormais à conjuguer le regard interne à sa foi sur les deux autres traditions et la perspective externe des deux autres sur lui} \cite[p. 788]{theobald_christianisme_2007}.

\paragraph{Savoir apprendre de la rencontre.} Les sociétés modernes imposent que toute rencontre véritable soit un processus d'apprentissage. Mais un tel processus n'est pas extrinsèque aux religions monothéistes, à travers différentes tournures  comme la figure du prophète pour le judaïsme. Pour les chrétiens, Jésus est le grand "apprenant" par sa souffrance et son obéissance (He 5,8), et  aussi par ses rencontres rapportées par les synoptiques où Jésus \textit{apprend} des autres qui il est.  

\paragraph{Pratiquement Un processus d'apprentissage.} Comme pour tout style, cet apprentissage passe un processus; d'abord une purification de nos préjugés et le refus de la substitution de l'un ou l'autre témoin, de son exclusion ou de l'inclusion de l'autre dans sa propre mission. La \textit{Règle d'or} est alors l'étalon de la justesse de notre attitude dans la communication avec l'autre témoin. Puis, il s'agit de penser positivement nos liens, soit par la mystique, les courants spirituels traversant aisément les frontières entre les religions,  soit en pensant le jugement d'excellence que je porte sur ma propre religion sans qu'il ne produise de la violence. Pour cela, il faut  accepter que la raison de la multiplicité des \textit{témoins} fasse partie du dessein de Dieu et en rendre compte, chacun avec les ressources propres de sa religion. 

\begin{comment}
[Style de vie messianique]
Ce style de vie consiste à vivre l’hospitalité qui suscite la mue
d’identité des personnes rencontrées (par Jésus). « La foi chrétienne n’est pas une doctrine
(…) mais un ‘style de vie’ ou une manière de vivre de la sainteté même de Dieu : seule
l’expérience effective de l’Esprit de sainteté nous permet de confesser et de comprendre un
jour l’indépassable excellence du Fils unique du Père » (L’unique et ses témoins, 19 ).\cite[p. 794]{theobald_christianisme_2007}
    
\end{comment}


\paragraph{Application de ces principes et \textit{style de vie} chrétien}
l'A. propose alors une application pratique proprement chrétienne : il s'agit de penser la multiplicité des témoins, en ne s'arrêtant pas au nombre de trois mais en lien avec le mystère de l'Incarnation et de la Trinité, ce que Theobald appelle son \textit{hypothèse d'une théologie de la rencontre}. 
\begin{comment}
    
Jésus suscite l'opposition qui l'amènera à la croix. La notion de réciprocité se métamorphose en règle d'amour, de la \textit{mesure} à la \textit{démesure}. {Jn 10, 20 : il déraisonne. }

    « Sans le correctif du commandement d’amour (…) la Règle d’Or serait sans cesse tirée dans le sens d’une maxime utilitaire dont la formule serait do ut des, je donne pour que tu donnes. La règle : donne parce qu’il t’a été donné, corrige le afin que de la maxime utilitaire et sauve la Règle d’Or d’une interprétation perverse toujours possible » (Ricoeur, Amour et justice, 39). 



[Règle d’or]

     « Tout ce que vous désirez que les autres fassent pour vous, faites-le vous-mêmes
pour eux : voilà la Loi et les Prophètes » (Mt 7,12). 
Avec Jésus, on passe de la justice
(relation de réciprocité) à l’amour (démesure : l’amour des ennemis) et de l’amour à l’amour
définitif. Jésus accomplit donc la Loi et les Prophètes en accomplissant la règle d’or : il la
réalise définitivement en aimant ses ennemis.

Mt 7,12 : règle d'or
\end{comment}
\begin{comment}
    Comme énoncé dans les principes, après avoir pris au sérieux les éléments opposés aux titres de Jésus par l'islam et le judaïsme mais aussi l'espérance  qui apparaît dans les psaumes messianiques, l'A. avance que le processus d'apprentissage spécifique des chrétiens doit être selon le \textit{style de vie} chrétien. 
\end{comment}


 L'A. explicite alors ce qu'est le style chrétien, à partir de nombreuses références bibliques (Ps 110, 3; Mt 5, 20.44; Mt 7,12, Lc 3, 22) : la sainteté de Dieu à laquelle nous sommes appelés est l'amour démesuré qui n'attend aucune réciprocité. Elle dépasse donc la Règle d'or. Etre\textit{ engendré}, devenir fils de notre Père, c'est finalement reconnaître  cet appel démesuré à être \textit{comme} Dieu, toujours dans telle ou telle situation pratique. L'A. s'appuie sur la lettre aux Hébreux, où Jésus est à la fois \textit{associé} à l'unicité de Dieu (He 7, 2s) mais, par son sacerdoce selon l'ordre du roi Melchisédech, prince de la \textit{paix}, il ouvre à la multitude les chemins de la Sainteté de Dieu et leur permet d'être engendrés fils. 


\begin{comment}
    Engendrement.
Il s’agit du devenir « fils de votre Père ». Il a lieu quand on « réalise tout d’un coup
que l’appel démesuré à être connu comme Dieu, dans telle ou telle situation, est
toujours ‘à la mesure’ de chacun » (L’Unique et ses témoins, 200). Intercommuni-
cation : une autre manière de dire le dialogue avec cette nuance où l’on communique
à l’autre sa sainteté
\end{comment}
\paragraph{Style chrétien de la rencontre.} Ayant ainsi défini ce qu'est le style de vie chrétien, on peut alors revenir au processus d'apprentissage. La purification nécessaire sera pour les chrétiens, celle des peurs qui marquent notre \textit{mesure humaine}. Et de façon positive, le chrétien qui s'est affronté dans sa propre vie à la question de la sainteté et de la démesure divine, peut admettre que juifs et musulmans sont eux aussi, aux prises aux mêmes combats. Theobald définit finalement ce \textit{style chrétien de la rencontre}, comme le \textit{style} qui se caractérise par une singulière manière d'espérer la paix en affrontant la violence. Il conclut avec Philon d'Alexandrie (\textit{le mode de la victoire n'est pas le même pour tous mais que tous sont dignes d'estime}) en assignant aux chrétiens la mission de montrer l'unicité de chaque témoin. 








 
% --------------------------------------------------------------------------------------------------------------------------
\section{Discussions}
% --------------------------------------------------------------------------------------------------------------------------


\begin{comment}
    Reaction personnelle en discutant et critiquant de manière argumentée
\end{comment}


% --------------------------------------------------------------------------------------------------------------------------
\subsection{Méthode de Ch. Theobald.}

\paragraph{De nombreuses références scripturaires.} Ce qui frappe à la lecture, ce sont le nombre de citations bibliques. Dans notre texte, c'est quasiment une mini-exégèse du sacerdoce selon Melchisédech, à partir de la lettre aux Hébreux, mais aussi des psaumes messianiques. L'A. n'hésite pas à revisiter certains passages (comme He 5, 8) pour illustrer l'apprentissage de Jésus, alors que ce passage pose parfois difficulté par l'affirmation d'une souffrance rédemptrice.
De même, il s'intéresse aux \textit{récits}  de rencontre que propose les Evangiles, qui montre que Jésus se laisse transformer par les interpellations de ces interlocuteurs. Au delà des références des Evangiles synoptiques, on pourrait aussi mentionner certains passages de Jean comme le miracle de Cana, à l'initiative de sa mère, jugée inopportune par Jésus : \textit{"Qu’y a-t-il entre moi et toi, femme  ? mon heure n'est pas encore venue"} (Jn 2,2) mais qui le met en action. 

Ainsi, la nouveauté du christianisme, c'est une identité en relation; elle
\begin{singlequote}
     [\ldots] ne se réduit pas à l’identité christologique du Nazaréen mais se concentre dans le type de relation qu’il entretient avec ceux qui croisent sa route » \cite[p. 57]{theobald_christianisme_2007}.
\end{singlequote}

\paragraph{Une pensée en dialogue.} Nous l'avons vu en présentant le contexte, Theobald ne se limite pas à interroger la Bible mais entre en dialogue avec les philosophes contemporains : Habermas, René Girard, Levinas, Buber, J. Lambert sont ainsi cités dans notre texte. De la même manière, il dialogue avec des Pères de l'Eglise et penseurs de l'antiquité (Saint Basile de Césarée, Saint Augustin, Philon d'Alexandrie) et les Conciles (Nicée), fidèle à son annonce d'une herméneutique dogmatique à partir du \textit{credo}. Ces références, toujours marquées par un regard positif, n'alourdissent pas trop le texte (notes de bas de page) et montre le style théologique, \textit{esprit large et généreux} pour reprendre les \textit{Exercices}, qui s'applique à lui-même.



\paragraph{Influence des Exercices Spirituels ?} Les \textit{Exercices spirituels} de Saint Ignace ont pour but de \textit{chercher et trouver la volonté divine dans la disposition de sa vie} (ES 1). Cette   contemplation de l'action est la manière de procéder ignatienne \cite[p.9]{theobald_christianisme_2007}. La deuxième semaine des \textit{Exercices} propose à l'exercitant, la méditation de l'appel du Roi Temporel (ES 91) qui appelle les hommes à sauver le monde. Il est proposé au méditant de suivre le Christ de façon digne, écho de la Règle d'or :
\begin{singlequote}
    par conséquent si quelqu'un n'accueillait pas la requête d'un tel roi, combien il mériterait d'être blâmé par tout le monde
\end{singlequote}

Certains voudront s'attacher \textit{davantage} et vivre du don \textit{démesuré} de Dieu, en faisant le don de soi, et ainsi 
\begin{singlequote}
    vous imiter en endurant tous les outrages, tout blâme et toute pauvreté effective et spirituelle.
\end{singlequote}

Ce texte peut permettre de comprendre l'unicité de chacun, que celui qui donne les \textit{Exercices} doit aider à mettre en valeur. De façon paradoxale, plus nous répondons à l'appel au décentrement et à imiter la Sainteté de Dieu dans sa démesure, plus nous devenons unique aux yeux de notre prochain et à nos yeux \cite[p. 827]{theobald_christianisme_2007}. Et finalement, n'est ce pas ce que propose Theobald aux chrétiens dans leur rencontre avec le judaïsme et l'islam de :
\begin{singlequote}
    [\ldots] mettre en valeur l'unicité incomparable de chaque partenaire ? Tâche difficile qui peut  les conduire aujourd'hui encore dans l'expérience du don de soi. \cite[p. 797]{theobald_christianisme_2007}
\end{singlequote} 
 



% --------------------------------------------------------------------------------------------------------------------------
\subsection{Quelques éléments sur la christologie de Theobald}

\paragraph{Cohérence entre christologie et théologie des religions.}
Dans son livre \textit{Christianisme comme style}, la partie proprement christologique est assez tardive (p. 799 et ss) et surprenante puisque l'auteur nous propose non pas une relecture des Evangiles mais celle du récit de Joseph et ses frères. C'est que la christologie irrigue tout le livre : ainsi, nous avons déjà noté que notre texte est marqué par sa christologie, celle du \textit{style de l'hospitalité}, décentrement vécu de façon unique par Jésus.  

\paragraph{Démesure de l'amour de Dieu, message de Jésus.} Jésus dans sa vie a enseigné la démesure de l'amour de Dieu, à travers le sermon sur la montagne ou le bon Samaritain, \textit{qui met en jeu son existence au profit du blessé} .
En mourant sur la croix, Jésus réalise sa parole et la promesse de Dieu de nous donner ce qu'il est en Lui-même (Rahner dirait \textit{auto-communication de Dieu)} et révèle ainsi qu'il est le fils unique. La mort du Christ est donc le sceau de la parole du Christ, qui la rend irrévocable \cite[p.829]{theobald_christianisme_2007}.

\begin{comment}
    Theobald développe cette thèse à l’aide de références scripturaires habilement agencées avant de souligner l’aporie philosophique de l’hospitalité chez Derrida : « la loi inconditionnelle de l’hospitalité illimitée » d’un côté et « les lois de l’hospitalité avec ses droits et ses devoirs toujours conditionnés et conditionnels » de l’autre. Pour Theobald, le corollaire christologique d’une telle aporie est le suivant : « … l’unique sainteté hospitalière de Jésus qui porte tout le dogme chrétien est une manière de se situer au cœur de cette aporie » (p. 162).
\end{comment}



 
\begin{comment}
{L'apprentissage de Jésus dans ses rencontres} Pratiquement, en regardant comment Jésus \textit{apprend} lors de ses rencontres, nous pouvons préciser l'approche proposée par Theobald pour la rencontre avec les autres monothéismes. Theobald recense 6 étapes : 
    Christoph Theobald est un théologien catholique français. Il propose six étapes de la rencontre de Jésus dans son livre “La grâce de l’Esprit” : la rencontre, l’écoute, la conversion, la foi, la mission et la communion
    \begin{itemize}
    \item Apprentissage de Jésus (He 5, 8) dans ses rencontres, il vit la sainteté
    
\item Jésus ne s'impose pas mais permet à l'autre d'accéder à lui -même. Il va libérer l'autre de ses craintes et se dire. 
\item l'émergence de la Foi : quand on accède à son identité, elle commence à être sauvé. Foi en Dieu. cf la Femme adultère, \textit{Ta Foi t'a sauvé}. de l'hospitalité de Jésus arrive la Foi. Il prend des exemples dans l'Evangile : elle libère les personnes rencontrées et les ouvre à une vie nouvelle
\item refiguration, mue d'identité : conversion des personnes qui croisent Jésus, ces gens qui ont changé de vie en rencontrant Jésus. Chacun à adopter le style de vie messianique. Nous avons à devenir hospitalier et leur permettre d'advenir à leur identité.
\item les personnes qui accèdent à cette vie nouvelle   ne sont pas forcément tous disciples. Mais ces rencontres ne sont pas forcément un changement de vie. Refus. Comment alors maintenir le lien social quand il y a refus de la rencontre ? 
Théobald introduit la règle d'Or.
\end{itemize}
\end{comment}


\paragraph{Déplacement de l'amour vers l'apprentissage} La pensée de Theobald articule plusieurs concepts de fond récurrents dans les différents chapitres du livre : cohérence entre action et pensée (le style), la surabondance de l'amour de Dieu au delà de la règle d'or, hospitalité/sainteté, \ldots Mais comment les articuler quand on parle d'institutions (ce que sont les religions) ? Ainsi, P. Ricoeur \cite{ricoeur_soi-meme_1990} considère la justice (et non l'amour) comme la vertu des institutions. Ce qui explique que l'archétype de la rencontre individuelle de Jésus (avec la syro-phénicienne, le lépreux  ou Pierre) ne soit pas entièrement opérant quand il s'agit de la rencontre entre \textit{religions}. D'où le déplacement proposé par Theobald d'un style de l'hospitalité, individuel, à un style de l'apprentissage, plus pertinent dans les \textit{processus institutionnels}. C'est ainsi qu'on peut comprendre le pape François :
\begin{singlequote}
     "Donner la priorité au temps, c'est s'occuper d'initier des processus plutôt que de posséder des espaces" (Evangelii Gaudium, n°223)
\end{singlequote}
A contrario, la polémique de Ratisbonne montre le risque quand on oublie cette nécessaire différence entre institutions et individus. D'une certaine façon, c'est moins le fond qui a posé problème aux musulmans, que le fait de sembler "dialoguer" depuis Ratisbonne.


% --------------------------------------------------------------------------------------------------------------------------
\subsection{Quel positionnement par rapport aux approches classiques de la théologie des religions}
Comment situer la pensée de Theobald dans le champ de la théologie chrétienne visant à penser le pluralismes religieux ? 
R. Cheno \cite{cheno_dieu_2017} propose 4 positions chrétiennes face au pluralisme religieux :
\begin{itemize}
    \item la position exclusiviste, pour qui le salut chrétien est inséparable de l'appartenance à l'Eglise, position incarnée par K. Barth.
    \item la position inclusiviste, incarnée par K. Rahner et reprise par le Concile Vatican II  reconnaît un salut par le Christ hors de l'Eglise
    \item la position pluraliste, incarnée par P. Knitter, met au centre le Royaume de Dieu annoncé par Jésus-Christ. R. Cheno indique 3 ponts possibles, le pont "philosophico-historique", le pont "religieux-mystique" et le pont "éthico-pratique" dont nous étudierons la proximité avec la pensée de Ch. Theobald.
    \item la position post-libérale, avec G. Lindbeck \cite{lindbeck_nature_2002}, s'appuie sur le modèle culturo-linguistique pour souligner les différences et la singularité de chaque religion.
\end{itemize}


\paragraph{Par rapport à la position Inclusiviste de  K. Rahner} K. Rahner a introduit la notion de \textit{Chrétiens anonymes}, sauvés en dehors de l'Eglise visible. K. Rahner s'appuie essentiellement sur Mt 25. Jean Baptiste Metz estime que la notion [de chrétien anonyme] échoue à penser le contenu fondamental de la personne historique de Jésus et le récit chrétien spécifique qui engage une pratique chrétienne appuyée sur la vie de Jésus. \cite[p. 83]{cheno_dieu_2017} Or, Ch. Theobald est particulièrement sensible à l'importance du récit, que ce soit dans l'Ancien Testament ou les Evangiles : 
\begin{singlequote}
     Si l'unicité risque en effet de désespérer toujours nos tentatives de la saisir par le concept, son articulation sur le récit permet de l'approcher dans une perspective anthropologique. \cite{centre_sevres_paris_unique_1996}
\end{singlequote}


\begin{comment}
    il faut ajouter un quatrième qui ne fait pas nombre avec les précédents : le récit. Jean Lambert note l'importance capitale de la "narrativité sémitique dans les trois monothéismes, qui leur permet de gérer, chaque fois de manière spécifique, leur rapport au temps et à l'histoire. Paul Beauchamp montre dans \textit{L'un et l'autre Testament II }ie lien intime entre confession de l'unicité et récit: 
\begin{quote}
    "la nécessité, selon ses propres termes, que l'Un soit manifesté par le récit orientant le pluriel"10
\end{quote}. Si l'unicité risque en effet de désespérer toujours nos tentatives de la saisir par le concept, son articulation sur le récit permet de l'approcher dans une perspective anthropologique
\end{comment}


\paragraph{Par rapport au pont "philosophico-historique" } Dans le texte du colloque de 1995, Theobald fait un long développement sur la \textit{figuration}, notion empruntée à Stanislas Breton et qui fonctionne comme un "système" :
\begin{singlequote}
     Quand change un de ses éléments : le Dieu unique (1), le témoin (2) ou le tiers (3), c’est toute la figure qui se transforme. [\ldots] on fait souvent comme si on pouvait isoler le Dieu unique du reste de ses figures historiques : "peu importe les querelles entre les trois témoins, c'est au même Dieu unique que nous croyons". Ce qui précède montre le caractère primaire de ce type de raisonnement très répandu. Mais comment décrire alors, de manière plus précise, les ressemblances et les différences entre les trois figures historiques du Dieu unique ?\cite{centre_sevres_paris_unique_1996}
\end{singlequote}
Theobald est donc sensible au danger   d'une approche qui minimise les différences, critique souvent faite aux théologies pluralistes \cite[p. 114]{cheno_dieu_2017}. 
Pourtant, ce système à trois éléments peut faire penser au pont "philosophico-historique" de la position pluraliste : 
\begin{singlequote}
    Chaque religion est une réponse humaine parmi d'autres à cette unique Réalité divine (c'est la dimension philosophique) et elle a construit sa réponse propre, c'est à dire ses croyances et ses pratiques, dans des circonstances historiques et culturelles différentes (c'est la dimension historique). \cite[p. 91]{cheno_dieu_2017} 
\end{singlequote}
Theobald y répond en utilisant le concept de \textit{jugement réflexif} emprunté lui-aussi à Stanislas Breton :

\begin{singlequote}
    Mais une simple analyse des "contenus" de la "figure" historique de Dieu ne suffit pas vraiment pour désigner le centre de nos différences et de nos ressemblances. Il doit être cherché [\ldots] dans le "jugement réflexif" qui constitue le "témoin" dans son rapport à l'Unique et aux autres.  \cite{centre_sevres_paris_unique_1996}
\end{singlequote}
Ce jugement réflexif ou \textit{réfléchissant} est "comme retour du témoin sur lui-même et sur la qualité de son rapport à Dieu", le terme polysémique de \textit{réfléchissant} rendant compte que la confession en un Dieu unique exclut logiquement la multiplicité des dieux.  
\begin{comment}
    Que Dieu prenne une figure spécifique pour quelqu'un dans une relation de foi, cela ne dit encore rien de son "unicité". Dire que ce Dieu-là existe et qu'il n'existe qu'une seule fois, qu'il est unique, est donc un jugement second, porté par le témoin sur une "figure" de Dieu qui a déjà pris consistance pour lui. On peut l'appeler "jugement réfléchissant" parce qu'il est comme un retour du témoin sur lui-même et sur la qualité de son rapport à Dieu : il intervient, au cœur même de la figure, déjà parce qu'il est - logiquement - impossible de confesser l'Unique sans exclure en même temps, et au moins implicitement, la multiplicité des dieux qui sont donc toujours là comme son ombre éternellement surmontée.
\end{comment}

\paragraph{Par rapport au pont "religieux-mystique"} Le pont "religieux-mystique" défend l'idée que le divin dépasse ce que chaque religion peut expérimenter et définir dogmatiquement mais qu'il est le même au coeur de l'expérience mystique de chacun \cite[p. 92]{cheno_dieu_2017}.  Ch. Theobald mentionne ce pont  : 
\begin{singlequote}
    On a toujours enseigné, dans chacun des monothéismes. que l'Unique ne fait pas nombre avec ceux qui témoignent de lui. Ce type d'argument, si familier aux mystiques, consiste à critiquer nos représentations de Dieu, à développer toute une série de procédures à la fois intellectuelles et affectives pour approcher corporellement le mystère du Dieu tout autre. \cite[p. 791]{theobald_christianisme_2007}
\end{singlequote}
Il précise néanmoins que Dieu n'est pas ineffable dans le Christianisme puisque le Fils Unique nous l'a dévoilé (Jn 1, 18). 
De plus, Ch. Theobald mentionne le risque de la théologie pluraliste de niveler toute différence : 
\begin{singlequote}
    Peut-être doit-on ajouter que le spirituel, qui relativise les différences entre les trois monothéismes. risque de ne plus se laisser interroger par la discordance des témoignages qui discrédite toujours la cause elle-même.\cite[p. 791]{theobald_christianisme_2007}
\end{singlequote}

\begin{comment}
    
\paragraph{mystiques, facile de traverser les religions} On a toujours enseigné, dans chacun des monothéismes. que l'Unique ne fait pas nombre avec ceux qui témoignent de lui. Ce type d'argument, si familier aux mystiques, consiste à critiquer nos représentations de Dieu, à développer toute une série de procédures à la fois intellectuelles et affectives pour approcher corporellement le mystère du Dieu tout autre. Les spirituels de tradition différente se rencontrent sur ces voies à la fois ascétiques et mystiques; ils traversent aisément les frontières entre les religions parce que, pour eux, un espace infini de communication avec autrui s'est ouvert dans la différence indépassable entre Dieu et ce qu'on peut dire de lui.

\paragraph{mystique pas facile pour christianisme du fait de l'incarnation qui nous dévoile le père, qui n'est pas ineffable}
Relativement bien ajusté au monothéisme juif et musulman, ce type d'expérience mystique se heurte, en christianisme, au mystère de l'Incarnation: \begin{quote}
    « Personne n'a jamais vu Dieu; le Fils unique, qui est dans le sein du Père, nous l'a dévoilé » (Jn 1, 18).
\end{quote} 

\paragraph{risque du mystique d'éluder la question - car elle ne l'intéresse pas ? }Peut-être doit-on ajouter que le spirituel, qui relativise les différences entre les trois monothéismes. risque de ne plus se laisser interroger par la discordance des témoignages qui discrédite toujours la cause elle-même.
 
\end{comment}
 
\paragraph{Une pensée post-libérale} Comme le pluralisme post-libéral de Lindbeck, Ch. Theobald souligne  que tout dialogue doit reconnaître la consistance spécifique de chaque religion et non essayer de réduire son message propre au risque de perdre son identité (R. Cheno parle de désarticulation du dogme chrétien). Ch. Theobald n'abandonne rien du dogme chrétien mais par une relecture des récits sur Jésus, peut fonder un dialogue avec les autres religions monothéistes sur des bases proprement chrétiennes, celles de la\textit{ théologie de la rencontre}. 
 
% --------------------------------------------------------------------------------------------------------------------------
\subsection{Quel regard sur l'islam ?}

Si christianisme et judaïsme partage une véritable proximité, 

\begin{singlequote}
    ils ne peuvent pas pour autant évacuer l'énigmatique présence du troisième : Ismaël, fils d'Abraham (Ibrahim) et de Hâgar, ancêtre du peuple arabe, qui réclame sa « parenté» avec le premier, prototype de la foi en l'Unique. \cite[p. 786]{theobald_christianisme_2007}
\end{singlequote}

Quel regard chrétien Ch. Theobald propose-t-il sur l'islam ? 


\paragraph{Conséquence de l'absence de récit en Islam.} Ch. Theobald remarque le manque de récit dans le Coran : 
\begin{singlequote}
 Dans cette perspective d'un lien intrinsèque entre la qualité éthique de relation à l'Unique et le récit, la particularité stylistique du Coran qui n'obeit pas au modèle du grand récit mais se présente comme un ensemble multiforme de sourates, serait une confirmation sensible de son statut pré-éthique.
Clôturé sous les Umayyades (634-744)  qui cachaient systématiquement le travail de rédaction, le faisant remonter à la "mère du livre" en Dieu, le Coran indique la victoire définitive de l'Unique. Certes, il ne ferme pas définitivement l'histoire, mais il ne la rouvre pas sous le signe de la reconnaissance éthique de sa propre particularité historique. Ce qui ne signifie nullement - faut-il le répéter? -, que la préoccupation éthique soit absente de sa "figure monothéiste" \cite{centre_sevres_paris_unique_1996}.
\end{singlequote}

Or, pour Theobald, lecteur de P. Ricoeur, le texte et particulièrement le récit a un effet de \textit{reconfiguration} sur le lecteur, effet de découverte et de transformation. L'absence de récit dans le Coran limite donc cette capacité de reconfiguration, si présente dans le judaisme et le christianisme. 

\paragraph{Définir l'islam comme monothéisme pré-éthique.}  Ch. Theobald définit  l'islam comme monothéisme pré-éthique :
\begin{singlequote}
    La lutte de l'islam pour l'unicité de Dieu précède donc toute préoccupation éthique. Pour cette raison, on peut le qualifier de \textit{monothéisme pré-éthique}. Le jugement réflexif de sa foi s'exerce dans le champ pré-éthique en double sens d'un retour à l'immémorial et d'une confession de l'Unique pour laquelle l'éthique, sans être absente, ne peut jamais être de l'ordre du principe \cite{centre_sevres_paris_unique_1996}.
\end{singlequote}
Ch. précise dans son livre : 
\begin{singlequote}
    La lutte de l'islam pour l'unicité de Dieu précède donc toute préoccupation éthique. C'est en ce sens qu'on peut appeler son monothéisme pré-éthique. Il reste, de ce fait, soumis aux interrogations éthiques qui ne peuvent pas ne pas émerger de la rencontre effective entre les trois témoins. \cite[p. 786] {theobald_christianisme_2007}
\end{singlequote}

La force de cette appellation de \textit{pré-éthique} est de doter l'islam d'une unicité, d'\textit{nom propre}, qui tienne compte de l'insistance coranique sur l'unicité de Dieu. Elle souligne aussi que cette alliance (\emph{mithaq}) dite de "pré-éternité".

Quand on connaît l'importance du \textit{fiqh}, le droit islamique, on ne peut cependant qu'être étonné par cette définition, qui \textit{donne à penser}. Les islamologues notent traditionnellement les versets médinois du Coran comme plus juridiques : dans le contexte de la première communauté musulmane à Médine, il a fallu préciser les règles et devoirs de chacun. Certes, les références explicites à la loi sont très peu nombreuses dans le Coran (quelques questions d'héritage ou de taux à intérêt, d'ailleurs souvent problématiques par leur brieveté) et ce sont les très nombreux \textit{hadiths}, tradition orale de la vie de Mahomet, qui nourrissent le \textit{fiqh}. Peut-on néanmoins dire que l'islam est un monothéisme pré-éthique ? 


\begin{comment}
    Paul Ricoeur est un philosophe français qui a travaillé sur la notion de pré-éthique. Selon lui, la pré-éthique est une étape préalable à l’éthique qui permet de comprendre les conditions de possibilité de l’éthique. Elle se situe entre la métaphysique et l’éthique et vise à déterminer les conditions d’existence de l’éthique. La pré-éthique est donc une réflexion sur les fondements de l’éthique et sur les conditions qui permettent à l’éthique d’exister. Elle est également une réflexion sur les limites de l’éthique et sur les questions que l’éthique ne peut pas résoudre1.
    À notre avis, cette
altérité doit paradoxalement être étendue jusqu’au soi ; nous sommes obligé d’étendre le
terme « autrui » jusqu’à « moi », dans une sorte d’altérité réflexive, différente de celle,
pré-éthique, défendue par P. Ricoeur.
\end{comment}


 



  

% --------------------------------------------------------------------------------------------------------------------------
\section{Conclusion}
% --------------------------------------------------------------------------------------------------------------------------
A travers ce chapitre du \textit{christianisme comme style}, j'ai pu approfondir la pensée de Ch. Theobald et voir comment elle s'applique pratiquement à une question théologique précise. Il ne s'agit pas de rester à une théologie étherée mais d'accepter d'affronter des questions complexes, \textit{ici, la violence entre les trois monothéismes}, en utilisant le cadre théologique qui fonde sa pensée (\textit{hospitalité / Sainteté},...), et en utilisant d'autres sources pertinentes par rapport à sa question (par exemple ici, long processus d'apprentissage nécessaire à tout vrai dialogue) et s'appuyant sur des références scripturaires étayées. 

Partant d'une problématique contemporaine, la réponse proposée par Ch. Theobald est sans doute condamnée à être revisitée régulièrement mais elle est précieuse pour aider les chrétiens à vivre aujourd'hui  leur foi.




 




