\chapter{Validation}

\section{instruction}
 
Rédaction d’un travail de 8 pages
Vous faites au préalable une recherche documentaire afin de choisir un article, un chapitre
d’un ouvrage ou un ouvrage dans lequel la christologie est interrogée par la culture
postmoderne et le pluralisme religieux.
Puis vous rédigez votre travail selon deux grandes parties :
\begin{itemize}
    \item  Présentez le document choisi et la manière dont l’auteur pense la confrontation de la
christologie à la culture actuelle.
    \item  Vous réagissez personnellement au texte en le discutant et le critiquant de manière
argumentée.
\end{itemize}

Le texte choisi doit être au préalable validé par le professeur
Merci de suivre les normes universitaires : « Normes de présentations de mémoires »
Votre travail écrit doit être déposé sur l’ENT (espace dédié) et la date limite pour la
remise de votre travail est le 1er mai 2023

POints d'attention : 
\begin{itemize}
    \item Christologie
    \item Récent (Grieu,...)
    \item question Ecologie ?
\end{itemize}


\section{Thèmes possibles}

Habiter ses questions, entre l'individu Jésus et l'universalité. 
