\chapter{Validation}

\section{instruction}
 
Rédaction d’un travail de 8 pages
Vous faites au préalable une recherche documentaire afin de choisir un article, un chapitre
d’un ouvrage ou un ouvrage dans lequel la christologie est interrogée par la culture
postmoderne et le pluralisme religieux.
Puis vous rédigez votre travail selon deux grandes parties :
\begin{itemize}
    \item  Présentez le document choisi et la manière dont l’auteur pense la confrontation de la
christologie à la culture actuelle.
    \item  Vous réagissez personnellement au texte en le discutant et le critiquant de manière
argumentée.
\end{itemize}

Le texte choisi doit être au préalable validé par le professeur
Merci de suivre les normes universitaires : « Normes de présentations de mémoires »
Votre travail écrit doit être déposé sur l’ENT (espace dédié) et la date limite pour la
remise de votre travail est le 1er mai 2023

POints d'attention : 
\begin{itemize}
    \item Christologie
    \item Récent (Grieu,...)
    \item question Ecologie ?
\end{itemize}


\section{Thèmes possibles}

Habiter ses questions, entre l'individu Jésus et l'universalité. 






\section{Theobald}

 

\section{bibliographie}

\begin{itemize}
    \item RICOEUR, P., Amour et justice, Paris 20082.
   \item RICOEUR, P., Soi-même comme un autre, Paris 1990.
   \item THEOBALD, C., « Jésus n’est pas seul. Ouvertures » dans P. GIBERT – C. THEOBALD, Le cas
Jésus Christ. Exégètes, historiens et théologiens en confrontation, Paris 2002, 381-462.
   \item THEOBALD, C., Le christianisme comme style * et **. Une manière de faire de la théologie en
postmodernité, Paris 2007. \cite{theobald_christianisme_2007} 
   \item THEOBALD, C., Selon l’Esprit de sainteté. Genèse d’une théologie systématique, Paris 2015.
   \item THEOBALD, C., « L’unique et ses témoins : jalons pour une théologie de la rencontre entre
juifs, chrétiens et musulmans », Chemins de dialogue 7 (1996), p. 183-202.
\end{itemize}

\section{Introduction}


\paragraph{Approche libérale (et pluraliste) : partent de l'expérience religieuse} alors que notre foi chrétienne a une bipolarité : à la fois l'expérience de Jésus mais \textit{aussi l'expérience historique de la Résurrection de Jésus} : action de Dieu. D'une certaine façon, Dieu répond à Jésus à travers la Résurrection.

\paragraph{On ne peut donc pas partir de l'histoire} On ne part pas ici de Jésus historique mais on part des récits évangéliques, éclairés par la Résurrection, travaillés par elle.


\paragraph{Theobald : part des textes évangéliques} il ne s'agit pas de relativiser le Christ mais que le Christ suscite le dialogue. Il ne déconstruit pas les textes. Et il intègre l'expérience pascale parce que 

\paragraph{Théobald} Jésuite, traduction des oeuvres de Rahner en français. Oeuvre originale. Deux volumes : le Christianisme comme Style. Il essaye de penser le christianisme dans le contexte actuel.


%------------------------------------------------------------
\section{le contexte et le problème}

%------------------------------------------------------------

\subsection{Vivre ensemble dans une société plurielle}

\paragraph{Nécessité de fonder le lien social non sur le religieux mais sur une instance neutre} Pluralité des religions. "Neutralité" du lien social. Comment fait on société alors que nous sommes différents et de culture différente ? Une vraie actualité. 
\begin{quote}
    L’Église « doit désormais proposer la foi au Christ au sein de démocraties pluralistes, prenant donc la mesure de la laïcité de l’Etat moderne et de sa propre position de groupe social parmi d’autres. Précisons brièvement que le pluralisme des ‘convictions axiales’, religieuses ou non, et la cohabitation de leurs organisations dans une même société supposent la forme agnostique du  ‘lien social’ » (style**, 807).
\end{quote}
\begin{Def}[Lien social]
la forme « agnostique » : il s’agit de la neutralité de l’Etat (laïcité). L’Etat est
neutre pour permettre aux différentes traditions religieuses de vivre ensemble.
Lien social : la forme « énigmatique » : l’unité n’est pas disponible, mais elle se fera à la fin
(eschatologie).
\end{Def}
 
\begin{Def}[position agnostique]
  pas lié à une tradition religieuse  
\end{Def}

Il ne faut pas partir de la dogmatique mais éthique

%------------------------------------------------------------
\subsection{Le christianisme n’est plus le ciment de la société}

\paragraph{Nous ne sommes plus en chrétienté} JP II : appelle de ses voeux la \textit{civilisation de l'amour}.

\begin{quote}
    « Une telle situation appelle, selon nous, un déplacement éthique du point de départ de la christologie (...). Les chrétiens (…) doivent montrer qu’ils reconnaissent le caractère énigmatique du lien social, condition d’un véritable pluralisme, non par concession à des pressions externes mais par intime conviction. Or, cette reconnaissance intérieure exige d’eux une véritable reprise christologique » (style**, 808).
\end{quote}


La vision chrétienne cohabite avec d'autres traditions, laic,n Islam...
\begin{Ex}
Historiquement, racine chrétienne de l'Europe.  Mais ce n'est plus la matrice pour comprendre le monde dans lequel on vit.  
\end{Ex}

\paragraph{Christ Roi} 1925- Pie XI. "christologie glorieuse". Dogmatique qui pose aujourd'hui problème.
%------------------------------------------------------------
 \subsection{Le dilemme christologique}
 

\paragraph{soit on continue à affirmer l'unicité du Christ} mais alors comment le Christ peut rassembler le monde. Facteur de division de la société.

\paragraph{Soit on admet d'autres figures religieuses } et alors on relativise le Christ.

\paragraph{Défi qui n'est pas mince} Quel chemin ?

\paragraph{Théobald : l'unicité du Christ fonde un universalisme pluriel} On peut comprendre l'unicité du Christ autrement. 


%--------------------------------------------------
\section{Réflexions sur le Fils « Unique »}
%--------------------------------------------------

Penser son unité avec Dieu et son lien avec tous les hommes. Si on n'insiste que sur l'unité avec Dieu, quelle fraternité avec les hommes ?

%------------------------------------------------------------
\subsection{Le Fils unique et ses frères}

\paragraph{la préférence entraîne la violence}

\begin{quote}
   « L’histoire de Joseph et de ses frères raconte ce qui arrive quand un père préfère un de ses fils à tous les autres : ceux-ci prennent ‘l’Unique’ en haine, ne pouvant plus lui parler amicalement (Gn 37,3s.) » (\citep[p. 821]{theobald_christianisme_2007}.) 
\end{quote}
 
\paragraph{Comment être fils unique d'un père et avoir en même temps des frères ?} il faut penser les deux
\begin{quote}
    « Comment en effet aimer sans privilégier tel ou telle ? Mais préférer quelqu’un, c’est en même temps risquer que jalousie et violence se lèvent à ses et à nos côtés et que la fraternité soit mise à rude épreuve  […] Comment peut-on être ‘fils unique’ d’un Père et avoir en même temps des frères » (Syle**, 821).
\end{quote}

\begin{Prop}
En théologie, penser ensemble ET... ET...
\end{Prop}


\begin{quote}
    « Il nous faut (…) penser ensemble l’une en fonction de l’autre, l’\textit{unicité} de Jésus de Nazareth et la \textit{relation} qu’il entretient avec les siens et, par extension, avec tout être humain. Cela ne va pas de soi dans une tradition théologique qui a fréquemment distingué, voire séparé ces deux facettes d’une seule et même réalité » (Style**, 822).
\end{quote}


\paragraph{Le long parcours de Joseph préfigure Jésus} Élimination de Joseph (puits) et réconciliation. 
\begin{quote}
    « On ne peut être unique que ‘pour’ quelqu’un : Joseph est d’entrée de jeu unique pour Jacob et destiné à le devenir pour ses frères, eux devant entrer à leur tour dans l’expérience de la fraternité ; ce qui nécessite un long parcours qui prend l’allures d’un drame. Jésus est le Fils unique du Père et appelé à devenir l’Unique non seulement pour ses disciples qui reçoivent le nom de ‘frères’ ou d’ ‘amis’, mais encore pour une multitude » (Style**, 822).
\end{quote}


\begin{Synthesis}
Théobald ne part pas du dogme mais de l'éthique / Texte. Interroge les Ecritures et interroge le devenir historique de Jésus.
\end{Synthesis}


%------------------------------------------------------------
\subsection{L’ambivalence de l’unique dans le NT}

Ici, Théobald reprend la distinction de Stanislas Breton\sn{professeur ICP} : \textit{unicité de singularité} (chaque personne est unique) et \textit{unicité d'excellence}(attribué à Jésus dans l'Espace et le Temps) et en ce sens, touche tous les hommes.
\begin{Def}[Unicité de singularité]
 L’unicité de singularité renvoie au caractère unique de toute
personne, à sa singularité : « tout être humain est ‘unique’, au sens où toute rencontre d’autrui
doit surmonter le réflexe de comparaison et aboutir au respect de ce qu’il a d’unique »
\end{Def}
\begin{Def}[    Unicité d’excellence]
  « Le croyant ne doit-il pas accorder à Jésus, outre l’unicité de
singularité qu’il partage avec tout être humain, une ‘excellence’ dans l’espace et le temps ? »
\end{Def}

 




\paragraph{une unicité d'excellence en relation avec les autres} Jésus rentre en relation avec les autres. Théobald part de S. Jean. Tout d'abord le fils unique : 
\begin{Def}[Unicité du Christ]
    cette expression vise à souligner que Jésus est l’unique sauveur de tous
les hommes car il est le « Fils unique » de Dieu. Theobald, néanmoins, développe l’idée que
l’unicité du Christ passe par la reconnaissance de cette unicité par les hommes. L’unicité n’est
pas seulement une qualité métaphysique, mais elle implique une relation réciproque.
\end{Def}
 
\textit{monos} signifie en grec unique mais aussi \textit{seul}. 

\paragraph{Fécondité de l'unique} 
\begin{quote}
    Jn 12,24 : « Le grain de blé tombé en terre, s’il ne meurt pas il reste tout seul (\textit{monos}), vous dis-je, mais s’il meurt il donne beaucoup d’épis ». 
\end{quote}

\begin{quote}
    « La solution de l’ambivalence fondamentale, attachée à toute unicité – lieu où se loge l’ultime tentation de tout homme - , est donc le don de soi, mort du grain pour porter du fruit (…). L’unique n’est donc pas vraiment l’unique, au sens d’une unicité d’excellence, que s’il donne sa vie pour une multitude ; s’il donne sa vie pour que chacun puisse accéder à sa propre unicité » (Le cas, 454).
\end{quote}
Idée que l'unicité de Jésus est une unicité relationnelle : il donne sa vie pour que les autres aient aussi la vie. 

On voit ici comment Théobald relit et reinterprête les écritures avec un regard neuf. 

\begin{Prop}
    Quand on pratique la générosité du samaritain, on participe à l'unicité du Christ. 
\end{Prop}


%------------------------------------------------------------
\subsection{La sainteté et unicité}

Théobald appelle cette singularité, \textit{sainteté}.




\paragraph{Figure du Samaritain et de la démesure qui va jusqu'à aimer sur la croix}
\begin{quote}
    « Celui qui met en jeu son existence au profit du blessé, le Samaritain, devient unique, non seulement pour l’homme rencontré par hasard sur le chemin, mais aussi à ses propres yeux : la démesure de son geste qui défie toute obligation légale s’est avérée à sa mesure, mesure incomparable à celle du voisin » (style**, 827).
\end{quote}

Figure de Jésus. Il y a quelque chose d'excessif. 
Cela est marqué au moment du refus et du rejet qu'il revele qu'il est fils unique : \textit{aimez vos ennemis}. Et cela culmine sur la mort sur la croix. 


%------------------------------------------------------------
\subsection{L’unicité de Jésus et sa manière unique de communiquer la sainteté}


 Ce n'est pas la sainteté qui fait la différence car tout homme est appelé à la \textit{vision béatifique.} Ce qui est unique, c'est qu'il communique de façon unique et définitive de la Sainteté. 
 La sainteté est communiquée \textit{une fois pour toute}\sn{Hebreux}

\begin{quote}
    « Ce n’est donc pas la grâce qui fait la différence entre l’Unique et ses frères humains, ni la sainteté (...). La différence entre Lui et nous consiste seulement dans le fait que c’est Lui, le Fils unique, qui nous communique la sainteté et que c’est en Lui que l’unique promesse de Dieu de nous donner ce qu’il est en Lui-même est devenue dans  notre histoire réalité ‘irrévocable’ » (Style**, 829).
\end{quote}

Cf la distinction que faisaient les Pères de l'Eglise de l'Union hypostatique (en Jésus, les personnes divine et humaine) et l'invitation de Dieu dans les Saints.


\paragraph{Irrévocable} Interprétation de la mort du Christ, sceau de la parole du Christ. 

\paragraph{Comment les approches christologiques présentes doivent être complétées}
%------------------------------------------------------------
\section{Les insuffisances des approches christologiques récentes}
%------------------------------------------------------------

Une théologie, c'est aider les chrétiens à vivre dans un certain contexte. 

\begin{itemize}
    \item Isoler Jésus de ses frères. Seconde Quête
    \item les théologies inclusivistes (Danielou, Rahner), qui insistent sur l'aspect culturel.
\end{itemize}
%------------------------------------------------------------
\subsection{L’unicité de Jésus fondée sur sa relation à Dieu}

\paragraph{Seconde Quête} Käsemann, ... Les disciples de Bultmann qui cherchent le Jésus historique : montrer que Jésus était le seul de son espèce. Montrer que le Jésus de l'histoire et le Christ de la Foi, il y a une continuité. C'était une réponse à Reimarus en montrant qu'en étant historien, il y avait une christologie implicite dans le Jésus historique. 

Techniquement, on avait des critères d'historicité \mn{ex : Jésus discutant avec les pécheurs : on ne faisait pas cela avant ni après}. "Mon Père et votre Père" : distingue bien. "on vous a dit, moi je vous dis" : tous ces motifs qui vont isoler Jésus des autres.

\paragraph{Inconciemment} ces théologies ont mis l'unicité théologique de Jésus au détriment de sa relation avec les hommes.



%------------------------------------------------------------
\subsection{L’unicité de Jésus fondée sur sa dimension eschatologique}

\paragraph{Jésus, un avec les hommes} Rejoint le projet de Théobald de montrer le Christ en Relation. On retrouver cela en G\&S 22 (\textit{Le Christ, homme nouveau})
\begin{quote}
1. En réalité, le mystère de l’homme ne s’éclaire vraiment que dans le mystère du Verbe incarné. Adam, en effet, le premier homme, était la figure de celui qui devait venir, le Christ Seigneur. Nouvel Adam, le Christ, dans la révélation même du mystère du Père et de son amour, manifeste pleinement l’homme à lui-même et lui découvre la sublimité de sa vocation. Il n’est donc pas surprenant que les vérités ci-dessus trouvent en lui leur source et atteignent en lui leur point culminant.
2. « Image du Dieu invisible » (Col 1, 15) , il est l’Homme parfait qui a restauré dans la descendance d’Adam la ressemblance divine, altérée dès le premier péché. Parce qu’en lui la nature humaine a été assumée, non absorbée , par le fait même, cette nature a été élevée en nous aussi à une dignité sans égale. Car, par son incarnation, le Fils de Dieu s’est en quelque sorte uni lui-même à tout homme. Il a travaillé avec des mains d’homme, il a pensé avec une intelligence d’homme, il a agi avec une volonté d’homme, il a aimé avec un cœur d’homme. Né de la Vierge Marie, il est vraiment devenu l’un de nous, en tout semblable à nous, hormis le péché.
3. Agneau innocent, par son sang librement répandu, il nous a mérité la vie ; et, en lui, Dieu nous a réconciliés avec lui-même et entre nous, nous arrachant à l’esclavage du diable et du péché. En sorte que chacun de nous peut dire avec l’Apôtre : le Fils de Dieu « m’a aimé et il s’est livré lui-même pour moi » (Ga 2, 20). En souffrant pour nous, il ne nous a pas simplement donné l’exemple, afin que nous marchions sur ses pas, mais il a ouvert une route nouvelle : si nous la suivons, la vie et la mort deviennent saintes et acquièrent un sens nouveau.
4. Devenu conforme à l’image du Fils, premier-né d’une multitude de frères, le chrétien reçoit « les prémices de l’Esprit » (Rm 8, 23), qui le rendent capable d’accomplir la loi nouvelle de l’amour. Par cet Esprit, « gage de l’héritage » (Ep 1, 14), c’est tout l’homme qui est intérieurement renouvelé, dans l’attente de « la rédemption du corps » (Rm 8, 23) : « Si l’Esprit de celui qui a ressuscité Jésus d’entre les morts demeure en vous, celui qui a ressuscité Jésus Christ d’entre les morts donnera aussi la vie à vos corps mortels, par son Esprit qui habite en vous (Rm 8, 11) [36]. Certes, pour un chrétien, c’est une nécessité et un devoir de combattre le mal au prix de nombreuses tribulations et de subir la mort. Mais, associé au mystère pascal, devenant conforme au Christ dans la mort, fortifié par l’espérance, il va au-devant de la résurrection.
5. Et cela ne vaut pas seulement pour ceux qui croient au Christ, mais bien pour tous les hommes de bonne volonté, dans le cœur desquels, invisiblement, agit la grâce. En effet, puisque le Christ est mort pour tous [39] et que la vocation dernière de l’homme est réellement unique, à savoir divine, nous devons tenir que l’Esprit Saint offre à tous, d’une façon que Dieu connaît, la possibilité d’être associé au mystère pascal.
6. Telle est la qualité et la grandeur du mystère de l’homme, ce mystère que la Révélation chrétienne fait briller aux yeux des croyants. C’est donc par le Christ et dans le Christ que s’éclaire l’énigme de la douleur et de la mort qui, hors de son Évangile, nous écrase. Le Christ est ressuscité ; par sa mort, il a vaincu la mort, et il nous a abondamment donné la vie pour que, devenus fils dans le Fils, nous clamions dans l’Esprit : Abba, Père!
\end{quote}

\paragraph{Une vision une de l'anthropologie} Cela suppose que la vision du monde des Chinois est la même que la vision des Européens. Reste tributaire de la vision universelle des lumières. Il faut parler de plusieurs \textit{universalismes} ou visions du monde : donner un but, une vision du monde. Cela ne veut pas dire que le Concile Vatican II est dépassé : il applique le Concile non pas à la lettre mais comme méthode : \textit{signe des temps } à lire dans la culture actuelle.

Tout en reconnaissant l'intérêt de ces théologies, elles ne peuvent servir de \textit{point de départ}. Partir de l'éthique, pratique, \textit{plus modeste}.

\paragraph{Théologie pratique} qui n'exclue pas la vision théologique ou anthropologique. 


%------------------------------------------------------------
\section{Commencer par une christologie pratique ou éthique}
%------------------------------------------------------------

%------------------------------------------------------------
\subsection{Une nouvelle approche de la christologie}

\paragraph{Normalement, on part de la figure divine de Jésus} Lui propose une autre approche. Ne prennent pas en compte le \textit{chemin } des disciples pour arriver à la Foi, l'unicité relationnelle de Jésus, comme les frères de Joseph. \textit{je suis le Chemin}. Intégrer la genèse de la Foi dans la Christologie même. 
\paragraph{Evangile}
Il s'appuie sur les Evangiles, qui font partie de la Révélation.
Les récits ne nous disent pas qui est Jésus mais ont une pédagogie pour nous faire entrer dans la Foi. \mn{Théobald est Jésuite et les Evangiles ont un goût, histoire}

Il s'agit de faire la même expérience que les disciples :
    
\begin{quote}
    « Le ‘Jésus des historiens’ n’a pas l’actualité et l’autorité nécessaire pour réclamer la foi, c’est-à-dire pour que soient recrées les conditions de la décision jadis appelée par l’évangéliste sur la vérité de ce qu’il appelait la vie ou le royaume » (Le cas 442). Il y a donc une « impossibilité et plus encore une illégitimité de toute communication de la foi par l’histoire » (le Cas 445).
\end{quote}

En lisant Renan, on ne découvre pas la Foi. L'histoire n'est pas là pour accéder à la Foi. Cela peut nous aider à un moment donné. 
\begin{quote}
     Mais ceux-là ont été écrits pour que vous croyiez que Jésus est le Christ, le Fils de Dieu, et pour qu’en croyant, vous ayez la vie en son nom.
    Jn 20,31
\end{quote}


\paragraph{Rencontre de Jésus}
les textes bibliques, pas des textes, mais la parole de Dieu qui nous féconde.
\begin{quote}
    « [La nouveauté du christianisme] ne se réduit pas à l’identité christologique du Nazaréen mais se concentre dans le type de relation qu’il entretient avec ceux qui croisent sa route » (Style*, 57). 
\end{quote}

\paragraph{configuration du récit } Le texte a un effet sur le lecteur : 
\begin{Def}[Configuration]
« Dynamisme intégrateur qui tire une histoire une et complète d’un divers
d’incidents, autant dire transforme ce divers en une histoire une et complète » (Ricoeur,
Temps et Récit II, p. 18). Ici il s’agit de la mise en intrigue, laquelle appartient à la
configuration.
\end{Def}

\begin{Def}[Refiguration]
 
    « J’appelle refiguration l’effet de découverte et de transformation exercé par
le discours sur son auditeur ou son lecteur dans le processus de réception du texte » (Ricoeur,
Amour et justice, 50).
 
\end{Def}
\begin{quote}
    « Comment le soi se comprend-il en se contemplant dans le miroir que lui tend le livre ? » (Ricoeur, Amour et justice, 51). 
\end{quote}
En méditant les Evangiles, on rencontre le Christ. \mn{Arrière fond de Théobald} 



%------------------------------------------------------------
\subsection{Jésus a « impressionné » ses disciples}

\paragraph{Rencontre du Christ : impact, impressionne} La Théologie libérale a insisté sur cette partie. Jésus les a impressionné fortement car il vivait intensément cette relation \mn{Schleiermacher p. XX}

\paragraph{salut lié à la christologie} c'est parce que Jésus est en lien avec tous les hommes qu'il sauve. Et je suis poussé à agir comme lui. Théobald reprend Schleiermacher mais reprend l'Evangile.

%------------------------------------------------------------
\subsection{Le style de vie messianique de Jésus}


\begin{quote}
    Comment caractériser « l’accès à la foi au Christ, compte tenu du point de départ éthique de la christologie, appelé par nos sociétés néo-libérales toujours tentées d’oublier les menaces qui continuent à peser sur le lien social ? On peut, dans la perspective d’une christologie pratique, le définir comme ‘mue d’identité’ qui s’exprime par le passage à un\textit{ style de vie messianique}, à une manière spécifique de se situer dans la société globale » (style**, 813).
\end{quote}

\begin{Def}[Style de vie messianique]
Ce style de vie consiste à vivre l’hospitalité qui suscite la mue
d’identité des personnes rencontrées (par Jésus). « La foi chrétienne n’est pas une doctrine
(…) mais un ‘style de vie’ ou une manière de vivre de la sainteté même de Dieu : seule
l’expérience effective de l’Esprit de sainteté nous permet de confesser et de comprendre un
jour l’indépassable excellence du Fils unique du Père » (L’unique et ses témoins, 199).
\end{Def}

\begin{Def}[Mue d’identité]
Le fait que l’autre puisse se découvrir et accéder à son identité singulière.

\end{Def}

\begin{quote}
    « Les évangiles sont en fait des récits de conversion qui ne mettent pas seulement en scène l’itinéraire de Jésus mais aussi et surtout ce qu’il devient en et pour ceux et celles dont l’itinéraire croise le sien : l’accès au Christ se vit comme véritable \textit{mue d’identité} » (style**, 805). 
\end{quote}



\begin{quote}
    Elle engendre l’autre à son identité propre (conversion ?). L’identité d’une personne coïncide
avec l’émergence de la foi (« Ma fille, ta foi t’a sauvée »Mc 5,34), l’autre vit à partir de cette
foi (foi en Dieu), il devient lui-même en se décentrant. Cette conversion consiste à « suivre
Jésus », c’est-à-dire à adopter chacun à sa manière le style de vie messianique\sn{Le regard de Jésus n’est pas évident. « L’adopter effectivement est de l’ordre d’une véritable conversion ou
nécessite une inversion. L’oeuvre messianique consiste précisément dans la victoire sur cet aveuglement » (Style
73). L’incompréhension des disciples montrent bien la difficulté d’adopter l’hospitalité de Jésus (posture
d’apprentissage et dessaisissement de soi). On doit « pouvoir atteindre effectivement l’absence de mensonge ou
la concordance absolue entre pensées, paroles et actes, entre la ‘forme de vie’ de quiconque et son ‘fond’,
concordance, qui, par principe, est chaque fois unique et incomparable » (style 75).}. Jésus devient
l’unique à leurs yeux : c’est-à-dire celui qui leur a communiqué la sainteté. 
\end{quote}

\paragraph{6 étapes}
\begin{itemize}
    \item Apprentissage de Jésus (He 5, 8) dans ses rencontres, il vit la sainteté
    \begin{Def}[Sainteté]
 Capacité d’apprentissage ou dessaisissement de soi au profit d’une présence à
quiconque, ici et maintenant (voir Mc 8,35).
\end{Def}
\item Jésus ne s'impose pas mais permet à l'autre d'accéder à lui -même. Il va libérer l'autre de ses craintes et se dire. 
\item l'émergence de la Foi : quand on accède à son identité, elle commence à être sauvé. Foi en Dieu. cf la Femme adultère, \textit{Ta Foi t'a sauvé}. de l'hospitalité de Jésus arrive la Foi. Il prend des exemples dans l'Evangile : elle libère les personnes rencontrées et les ouvre à une vie nouvelle
\item refiguration, mue d'identité : conversion des personnes qui croisent Jésus, ces gens qui ont changé de vie en rencontrant Jésus. Chacun à adopter le style de vie messianique. Nous avons à devenir hospitalier et leur permettre d'advenir à leur identité.
\item les personnes qui accèdent à cette vie nouvelle \sn{le possédé à Gerasa reste et ne devient pas disciple de Jésus} ne sont pas forcément tous disciples. Mais ces rencontres ne sont pas forcément un changement de vie. Refus. Comment alors maintenir le lien social quand il y a refus de la rencontre ? 
Théobald introduit la règle d'Or.
\end{itemize}
 
\begin{Def}[Concept d’hospitalité ]
Quand l’hospitalité se produit, c’est l’accomplissement des temps
messianiques. Aussi, dans cette expérience de la rencontre avec Jésus, de l’hospitalité de
Jésus est engendrée la foi, le royaume de Dieu advient. « Les aveugles voient ». C’est
pourquoi on peut qualifier de « messianique » le style de vie hospitalier de Jésus car il libère
les gens rencontrés et les engendre à la foi pour une vie nouvelle.
\end{Def}

\begin{Def}[Engendrement]
Il s’agit du devenir « fils de votre Père ». Il a lieu quand on « réalise tout
d’un coup que l’appel démesuré à être connu comme Dieu, dans telle ou telle situation, est
toujours ‘à la mesure’ de chacun » (L’Unique et ses témoins, 200).
Intercommunication : une autre manière de dire le dialogue avec cette nuance où l’on
communique à l’autre sa sainteté.
\end{Def}
%------------------------------------------------------------
\subsection{Le style de vie de Jésus à l’épreuve du refus}

Jésus suscite l'opposition qui l'amènera à la croix. La notion de réciprocité se métamorphose en règle d'amour, de la \textit{mesure} à la \textit{démesure}. \mn{Jn 10, 20 : il déraisonne. }
\begin{quote}
    « Sans le correctif du commandement d’amour (…) la Règle d’Or serait sans cesse tirée dans le sens d’une maxime utilitaire dont la formule serait do ut des, je donne pour que tu donnes. La règle : donne parce qu’il t’a été donné, corrige le afin que de la maxime utilitaire et sauve la Règle d’Or d’une interprétation perverse toujours possible » (Ricoeur, Amour et justice, 39). 
\end{quote}


\begin{Def}[Règle d’or]
 \begin{quote}
     « Tout ce que vous désirez que les autres fassent pour vous, faites-le vous-mêmes
pour eux : voilà la Loi et les Prophètes » (Mt 7,12). 
 \end{quote} Avec Jésus, on passe de la justice
(relation de réciprocité) à l’amour (démesure : l’amour des ennemis) et de l’amour à l’amour
définitif. Jésus accomplit donc la Loi et les Prophètes en accomplissant la règle d’or : il la
réalise définitivement en aimant ses ennemis.
\end{Def}
Mt 7,12 : règle d'or
\begin{quote}
    
\end{quote}

La règle d'or existe dans d'autres cultures \mn{cf  \cite{kung_lethique_2009}}


Paul Ricoeur : sans 
\begin{quote}
    Do ut des : je donne pour que tu donnes. La règle \textit{donne parce qu'il t'a été donné} Sauve la règle d'or de l'utilitarisme (Ricoeur)
\end{quote}


%------------------------------------------------------------
\subsection{La dimension eschatologique du style de vie de Jésus}

\begin{Def}[Style de vie eschatologique]
« Sa sainteté hospitalière [va] jusque dans son ultime
dessaisissement de soi » (style*, 91). L’absence de mesure ou la démesure fait entrer
l’incomparable de Dieu dans l’histoire (voir style*, 95). « Vous serez parfaits comme votre
Père céleste est parfait » (Mt 5,48). En donnant sa vie, son style de vie est donc définitif,
eschatologique. Sa mort ne dit pas seulement l’une fois pour toutes, mais aussi qu’il
transcende le temps et l’espace en s’identifiant à toutes les victimes comme Mt 25 les
présente.
\end{Def}

 En quoi une dimension eschatologique : 


 \begin{quote}
     « Comment la manière d’être du Nazaréen – son type d’hospitalité absolument unique – a pu engendrer, non seulement la confession messianique des premiers chrétiens, mais encore leur perception du caractère définitif et ultime de ce qui est advenu dans leur rencontre avec lui » (Style 86).
 \end{quote}

C'est la mort de Jésus manifestant la perfection de sa sainteté, dessaisi de lui-même, en partant, pour laisser la place même à l'ennemi.
\begin{quote}
    domine jusqu'au coeur des ennemis
\end{quote}

 


\begin{quote}
    « Sa sainteté hospitalière [va] jusque dans son ultime dessaisissement de soi » (style*, 91). L’absence de mesure ou la démesure fait entrer l’incomparable de Dieu dans l’histoire (voir style*, 95). « Vous serez parfaits comme votre Père céleste est parfait » (Mt 5,48).
\end{quote}

\paragraph{il est lui-même le saint} il brise le refus, Temps et Espace. Il donne à celui qui ne peut pas rendre. Quand on donne à un pauvre, il ne peut pas rendre.


\section{Contexte}



%------------------------------------------------------------
\section{Conclusion}


Jésus n'empêche pas le dialogue mais le favorise. 
Style de vie hospitalier, se dessaisissant pour laisser l'autre advenir à lui-même. 


\begin{Synthesis}
    Une approche éthique, pragmatique partant de Jésus. Approche originale et adaptée à notre monde.
\end{Synthesis}


 


\section{L'unique et ses témoins }

\cite{theobald_christianisme_2007} Jalons pour une théologie de la rencontre entre juifs, chrétiens et musulmans

\paragraph{unicité de Dieu et le témoignage qui lui erst dû auprès des humains et à leur service : foi partagée}

Jésus : Mt 22,34-40
Jn 18,37
Ap 3,14 monothéisme trinitaire des chrétiens


\paragraph{différences entre les attestations} apparition successive dans l'histoire 

\paragraph{violence entre les trois témoins} \textit{liens inextricables} peut être l'origine de la violence (mimétique au sens de René Girard). 
\begin{quote}
    il nous faudra affronter l'énigme de la violence pour voir où se trouve la véritable difficulté à communiquer entre témpoins. 
\end{quote}
\paragraph{Possibilités et limites des rencontres } \textit{qu'as tu appris des deux autres et de la société sur toi-même ? } Aucun des témloins ne peut répondre à la place des deux autres.
Réflexion proprement chrétienne sur la \textit{rencontre} en méditant sutr la figure de Melchisedech, \textit{roi de Gloire} (Ps 23).

\subsection{Difficile communication}

\paragraph{pour éviter la violence, séparation dans le privé} Lumières sépare privé et le public.  \textit{Armistice. } 

\paragraph{la société moderne impose ses règles du jeu à toute rencontre} cf Habermas et Ratzinger à Munich 19 /01/2004.  Esprit 2004/07

\subsubsection{regards croisées dans la famille d'Abraham}

\paragraph{judaisme monotheisme éthique} ne reconnaît pas christianisme et islam comme ses héritiers (cf levinas). Identité propre :
\begin{quote}
    Aimez l'étranger, car au pays d'Egypte, vous fûtes des étrangers Dt 10,17-19
\end{quote}
imitant son Dieu par la justice faite à autrui.


pas besoin du Christianisme ou Islam mais confrontation a un impact. Indépendance mais aussi fragilité vis à vis d'eux. "mystère" du chant du serviteur.  

\paragraph{Christianisme, réalité méta-éthique du don de Soi  } \sn{besoin du Judaisme pour se comprendre "accomplissement de la Loi et des Prophètes" en JC.  Rm 11,18 réalité inouie et excecessive : l'unique Dieu est censé communiquer à la multitude la sainteté qui le constitue en lui-même, et tous peuvent désormais}
2. Le christianisme se situe entre l'aîné et le dernier. Il a d'abord besoin du judaïsme pour se comprendre lui-même parce qu'il ne peut ni raconter ni vivre « l'accomplissement de la Loi et des Prophètes » en Jésus le Christ sans se référer continuellement à la racine sur laquelle il a été greffé (Rm 11, 18) Selon la tradition chrétienne, la foi s'ouvre à une réalité inouie et excessive: l'unique Dieu est censé communiquer à la multitude la sainteté qui le constitue en lui-même, et tous peuvent désormais découvrir par la foi à quel point cette sainteté le habite déjà:
\begin{quote}
    « Vous serez parfaits comme votre Père céleste est parfait » (Mt 5, 48). 
\end{quote}
C'est cela « l'accomplissement de Loi et des Prophètes», vécu dans les gestes les plus quotidien Si on se réfère au vocabulaire de l'éthique, on peut donc appel le christianisme un monothéisme méta-éthique .
\begin{Def}[monothéisme méta-éthique]
    au sens il insiste sur la communication de l'agapé divine, de l'amour surabondant de Dieu à tout être humain
\end{Def}
 
Mais comment se réclamer de cette réalité méta-éthique du don de soi qui dépasse toutes nos mesures humaines ses reconnaître d'abord l'entière autonomie ou consistance de l'ordre éthique de la justice et du respect d'autrui? La fragilité du christianisme ne vient donc pas de sa dépendance par rapport à un autre qui le précède et qui existe à ses côtés: mais elle vient de la tentation d'indépendance qui l'a amené à se substituer à Israël, à se considérer comme le « véritable Israël »; et cela en dépit de l'avertissement paulinien dans l'épître aux Romains qui prévient l'Église contre l'orgueil: les chrétiens risquent d'oublier qu'ils tiennent, grâce à la foi sur une racine qui les porte (Rm 11, 20).

\sn{785}
Peut-être cet oubli n'a-t-il pas été sans influence sur la naissance de l'islam, comme le pensent certains théologiens chrétiens!. Par rapport aux musulmans, l'Église se trouve en tout cas dans une position analogue à celle que le judaïsme occupe par rapport à elle: elle n'a pas besoin de comprendre l'islam pour se comprendre elle-même. Elle a subi l'insensibilité du nouveau venu sur la scène religieuse qu'elle-même a montrée vis-à-vis de l'aîné qui la précède et qui existe à côté d'elle; affrontement d'autant plus violent qu'il oppose deux manières de concevoir « l'accomplissement ». Il faudra attendre le xie siècle2 pour que le christianisme commence à abandonner la répartition traditionnelle de l'humanité entre juifs, païens et chrétiens qui l'avait amené à identifier la foi musulmane à un vague paganisme monothéiste.

\paragraph{Islam, monotheisme pre-éthique}
3. L'islam enfin, le dernier-né des trois, a besoin du judaisme et du christianisme, des « gens du Livre » comme il dit, pour se comprendre lui-même. C'est sur leur trace qu'il affirme le caractère ultime de sa révélation; après les quatre grands prophètes « doués de constance », Noé, Abraham, Moise et Jésus, Muhammad est le dernier, le « sceau des prophètes » (sourate 33, 40), qui met fin aux intervalles entre les époques prophétiques et fixe la communauté des fidèles dans l'attente de l'Heure dernière. Cette dépendance avouée est la force de l'islam et en même temps le lieu de sa fragilité propre.
Force d'abord, parce que sa critique des juifs et des chrétiens se met plutôt en dépendance par rapport à une alliance (mithãq) dite de « pré-éternité », qui précède toute division historique entre judaïsme, christianisme et islam. Il n'y a donc pas de progrès historique dans la révélation, mais rappel ultime et définitif de ce qui a été oublié ou déformé: l'unicité absolue de Dieu, menacée par le polythéisme et tout ce qui lui ressemble, comme l'association du Christ à Dien Ce retour en deçà de l'histoire et de ses divisions, vers l'origine adamique du « pacte » de « pré-éternité », qui d'emblée fait de tout homme un «croyant», fonde l'universalité de l'islam. Celle-ci n'est plus fondée, comme dans le prophétisme juif, sur la qualité éthique de la « relation » de l'Unique avec son témoin et avec l'étranger; \textbf{elle s'appuie sur l'idée que tout homme porte à sa naissance, sceau imprimé par Dieu en son cœur, la proclamation de foi de la pré-éternité}: cette « religion naturelle », liée à la création comme première révélation d'en deçà des temps, est une prédisposition à recevoir l'islam.
Mais cette force qui permet de contourner sans cesse les divisions historiques en les relativisant par la référence obligée à une origine immémoriale constitue en même temps la \textit{fragilité} propre de l'islam: 
\begin{Def}[monothéisme pré-éthique]
\textit{sa lutte pour l'unicité de Dieu précède toute préoccupation éthique}. 
\end{Def}
C'est en ce sens qu'on peut appeler son monothéisme pré-éthique. Il reste, de ce fait, soumis aux interrogations éthiques qui ne peuvent pas ne pas émerger de la rencontre effective entre les trois témoins.
Cette rencontre s'avère donc comme extraordinairement difficile à cause de l'asymétrie entre les trois traditions: si les deux premiers, juifs et chrétiens, se retrouvent, au moins selon la perspective chrétienne, dans une proximité privilé
 
Cette rencontre s'avère donc comme extraordinairement difficile à cause de l'asymétrie entre les trois traditions si les deux premiers, juifs et chrétiens, se retrouvent, au moins selon la perspective chrétienne, dans une proximité privilégiée, ils ne peuvent pas pour autant évacuer l'énigmatique présence du troisième : Ismaël, fils d'Abraham (Ibrähim) et de Hâgar, ancêtre du peuple arabe, qui réclame sa « parenté» avec le premier, prototype de la foi en l'Unique.

\subsection{Rencontre et comparaison.}

\paragraph{pas de facilité à communiquer entre les 3 témoins par la modernité du fait du \textit{comparatisme} et une distance critique qui atteint chacun des trois religions}
La modernité occidentale a-t-elle facilité la communication entre les trois témoins? Ce n'est pas du tout sûr, El nous a appris cependant à « comparer » les figures du Dieu unique, en nous permettant ainsi de prendre une certaine distance par rapport à l'ensemble des trois traditions.
Nous savons qu'il est impossible de rencontrer l'as sans se comparer à lui; et puisque des rencontres entre juit chrétiens et musulmans ont existé depuis les débuts, les compe raisons entre différents interlocuteurs n'ont pas non pis manqué. Mais la communication généralisée qui caractérise as sociétés modernes transforme la comparaison en principe intellectuel. L'homme contemporain ne cesse de comparer ce qui lui est proposé, et cela jusque dans le domaine religieux. Le «comparatisme » systématique des sciences de la religion  constitue donc la face intellectuelle d'une société démocratique qui, pour éviter toute violence religieuse en son sein, n'accorde plus de privilèges à aucun des trois témoins mais se fonde désormais sur une conception a-religieuse ou a-gnostique du «lien social » qui la constitue. La distinction entre le public et le privé dans nos États laïcs s'appuie sur cette prise de distance critique par rapport à chacun des trois; ce qui explique pourquoi elle les atteint dans leur propre identité.
\paragraph{comparatisme pour éviter les violences ? pas sûr pour l'islam}
Il est sûr que la crainte de la violence religieuse a conduit tout au long du xixe et du xxe siècle vers un «comparatisme» critique de toute religion, critique en particulier du mono-théisme: dans leur commune obsession de «l'Unique», juifs, chrétiens et musulmans n'auraient cessé de lutter contre tout ce qui est pluriel, poursuivant, chacun à sa façon, le mirage d'une culture unifiée qui exclut ce qui est différent. Cette critique fait peu de cas des identités propres de chaque figure. 
Et comme elle avait produit déjà au siècle dernier des réactions très violentes de la part du christianisme et du judaïsme officiels, elle suscite aujourd'hui des nouvelles violences et des soubresauts identitaires de la part de l'islam. \sn{cf Ratisbonne}
\paragraph{identité propre entre les 3 témoins : on ne peut simplifier en disant qu'ils cherchent à éliminer le pluriel}Nous sommes là devant une alternative intellectuelle et a arque en particulier du monothéisme: dans leur commune obsession de « l'Unique», juifs, chrétiens et musulmans n'auraient cessé de lutter contre tout ce qui est pluriel, poursuivant, chacun à sa façon, le mirage d'une culture unifiée qui exclut ce qui est différent. Cette critique fait peu de cas des identités propres de chaque figure.
Et comme elle avait produit déjà au siècle dernier des réactions très violentes de la part du christianisme et du judaïsme officiels, elle suscite aujourd'hui des nouvelles violences et des soubresauts identitaires de la part de l'islam.
Nous sommes là devant une alternative intellectuelle et spirituelle tout à fait décisive pour nos sociétés modernes.
Comparer les trois « monothéismes » comme je viens de le faire (sans pouvoir d'ailleurs entrer dans les détails), cela conduit-il nécessairement à aplatir les différences et à produire de nouvelles violences ? Ou peut-on espérer que le «comparatisme » réussisse à mettre en valeur le mystère de nos identités? 

\paragraph{comparatisme pour mettre en valeur nos identités, diversité et pour se situer dans la société avec nos propres styles mais en fouillant dans nos ressources. }
\begin{Synthesis}
Mon pari est qu'il peut faire paraître avec une acuité toujours plus grande la diversité extraordinaire de nos manières de nous situer dans la vie commune et par rapport au « lien social ».     
\end{Synthesis}
À condition cependant que la communication entre ces différents \textbf{« styles »}, inévitable dans la société moderne, provoque non pas une fermeture définitive de certains mais suscite en chacun une véritable auto-interrogation2, un retour \sn{788} réflexif sur soi pour découvrir dans son propre patrimoine des ressources jusqu'alors inaperçues, permettant d'affronter le nouveau pluralisme radical.
Ainsi compris le « comparatisme » renvoie chaque tradition au niveau le plus profond de sa propre identité, à sa foi et à l'exercice (parfois autocritique) d'un retour sur soi. Chacun des trois est convié au jeu difficile d'une communication qui consiste désormais à conjuguer le regard interne à sa foi sur les deux autres traditions et la perspective externe des deux autres sur lui; exigence de communication déjà présente dans la célèbre Règle d'or: \begin{quote}
    «Tout ce que vous voulez que les autres fassent pour vous, faites-le pour eux ! » 
\end{quote}Certes, cette règle de réciprocité est au cœur de la préoccupation éthique du judaïsme; mais elle traverse aisément les frontières entre traditions parce qu'elle existe en toute culture. À ce titre, elle se trouve aujourd'hui au fondement de nos sociétés démocratiques\sn{Ricoeur règle d'or}.

\subsection{Savoir apprendre de l'autre}

\paragraph{différence avec la rencontre de la société : la société poblige à des règles de communication, la rencontre oblige à un processus d'apprentissage}
Quand nos sociétés démocratiques imposent aux trois témoins certaines règles de communication, la rencontre des deux autres invite chacun à entrer dans un long processus d'apprentissage. Peut-être l'histoire de la modernité leur fait elle-même d'abord comprendre que, loin de les éloigner de leur propre tradition, cette capacité d'apprendre constitue depuis toujours l'identité la plus profonde du témoin.

\subsubsection{L'interrogation prophétique.}

\paragraph{prophétisme, figure de l'apprentissage "à se mettre à la place d'autrui"}
En effet, cette disponibilité à se laisser enseigner par autrui n'a pas été simplement imposée de l'extérieur aux traditions monothéistes; elle est née en leur sein sous la figure du prophétisme. Dire que Dieu est unique suppose déjà une prise de conscience de haut niveau. Mais que la tradition juive découvre un jour que son Dieu ne fait acception de personne et qu'il désire avoir un témoin comme lui généreux envers tout homme, cela suppose un apprentissage éthique d'un tout autre ordre encore : la capacité du juste à « se mettre à la place d'autrui», qui lui vient du souvenir d'avoir déjà occupé cette position: « Aimez l'étranger, car au pays d'Égypte vous fûtes des étrangers » (Dt 10, 17-19).
\paragraph{Jésus, Grand apprenant (He) de la rencontre avec l'autre, de ses souffrance vers l'accomplissement}
Jésus a été, lui aussi, un grand « apprenant», selon les dires de l'épître aux Hébreux : \begin{quote}
    «Tout Fils qu'il était, il apprit par ses souffrances l'obéissance, et, conduit jusqu'à son propre accomplissement, il devint pour tous ceux qui lui obéissent cause de salut éternel » (He 5, 8).
\end{quote} 
Ce que l'épître aux Hébreux affirme avec vigueur, les évangiles synoptiques le racontent en montrant Jésus apprenant des autres qui il est : du lépreux (Mc 1, 40), de la femme hémorroisse (Mc 5, 30), de la Syro-Phénicienne (Mc 7, 29), de Pierre et de bien d'autres encore.
Cet apprentissage le conduit vers l'accomplissement, vers l'expérience méta-éthique du don de soi pour la multitude.

\paragraph{Musulman, interrogation du pacte avec Dieu }
Le musulman, enfin, réitère l'intransigeant jugement initial qui exclut tout pluriel de l'Unique. Lui aussi vit donc la «foi » dans une sorte d'interrogation constante, qui s'exerce, comme on l'a déjà noté, dans le champ pré-éthique du « pacte » avec Dieu, imprimé en tout homme avant sa naissance. C'est en ce sens qu'il faut entendre la réserve du Coran quand il s'adresse directement au prophète Muhammad: 
\begin{quote}
    « Dis: je ne suis qu'un Avertisseur. Il n'est de divinité que Dieu, l'Unique, l'Invincible » (sourate 38, 65).
\end{quote}
Aujourd'hui il faut donc mettre en valeur ce retour critique sur soi qui caractérise le prophétisme biblique et coranique, indépendamment de la tournure précise qu'il prend dans chaque cas: il constitue, au sein des trois traditions, ce lieu mystérieux où des rencontres imprévisibles avec d'autres pourront se nouer et provoquer un véritable apprentissage entre partenaires.
\sn{790}

\subsubsection{Les étapes de l'apprentissage.}

\paragraph{1. purifier nos préjugés. Rester dans la règle d'Or; apprentissage de la société moderne} 
1. Une première étape consisterait alors à nous purifier des préjugés qui ont entraîné la violence. Je les ai déjà nommés: il peut s'agir de schèmes de «substitution » ou d'«exclusion», quand l'un prétend se substituer brutalement à l'autre dans sa mission religieuse au sein de l'humanité; de façon plus subtile il peut s'agir aussi d'« inclure » l'autre dans sa propre mission, de l'enfermer par exemple dans un rôle de préparation. Une autre manière encore de sortir de la « Règle d'or» de nos rencontres, symétriquement opposée à la précédente, serait de dénier à l'un des interlocuteurs ou à tous les trois toute capacité d'apprentissage dans la société moderne\sn{important pôur l'écologie}. Nous avons vu que le véritable « comparatisme » permet de décrypter cette capacité spécifique d'un retour sur soi au cœur de la foi de chacun des trois.

\paragraph{2. repenser positivement nos liens}
2. La deuxième étape de l'apprentissage est plus difficile.
Il s'agit de repenser positivement nos liens. Or, nous touchons 1 aux limites structurelles de la communication entre les trois, aux limites aussi de notre capacité d'apprentissage.
À nouveau plusieurs possibilités se présentent.

\paragraph{mystiques, facile de traverser les religions} On a toujours enseigné, dans chacun des monothéismes. que l'Unique ne fait pas nombre avec ceux qui témoignent de lui. Ce type d'argument, si familier aux mystiques, consiste à critiquer nos représentations de Dieu, à développer toute une série de procédures à la fois intellectuelles et affectives pour approcher corporellement le mystère du Dieu tout autre. Les spirituels de tradition différente se rencontrent sur ces voies à la fois ascétiques et mystiques; ils traversent aisément les frontières entre les religions parce que, pour eux, un espace infini de communication avec autrui s'est ouvert dans la différence indépassable entre Dieu et ce qu'on peut dire de lui.

\paragraph{mystique pas facile pour christianisme du fait de l'incarnation qui nous dévoile le père, qui n'est pas ineffable}
Relativement bien ajusté au monothéisme juif et musulman, ce type d'expérience mystique se heurte, en christianisme, au mystère de l'Incarnation: \begin{quote}
    « Personne n'a jamais vu Dieu; le Fils unique, qui est dans le sein du Père, nous l'a dévoilé » (Jn 1, 18).
\end{quote} 

\paragraph{risque du mystique d'éluder la question - car elle ne l'intéresse pas ? }Peut-être doit-on ajouter que le spirituel, qui relativise les différences entre les trois monothéismes. risque de ne plus se laisser interroger par la discordance des témoignages qui discrédite toujours la cause elle-même.
\sn{791}

\paragraph{accepter que l'autre n'abandonne pas sa prétention à la prééminence mais accepter de penser cette prééminence sans produire la violence}Faut-il alors préserver la prééminence ou l'excellence de l'un des trois ? D'un simple point de vue anthropologique, je ne vois pas comment éviter le jugement d'excellence sur la tradition à laquelle j'appartiens. D'un point de vue théolo-gique, je ne vois pas non plus comment l'un des trois pourrait renoncer à ses prérogatives d'excellence sans renoncer à sa propre identité. La question se transforme alors en exigence de penser aujourd'hui « l'ultime sceau de la lignée prophétique », « l'accomplissement des Écritures » ou la « mission d'être lumière des nations » de telle manière que ces prérogatives irrépressibles ne produisent pas de violence.

\paragraph{3. \textit{pourquoi} trois témoins}
3. Je reviendrai sur cette tâche proprement théologique.
Cependant, ce traitement du « comment » de l'excellence ne doit pas nous détourner prématurément de la question du «pourquoi ». C'est la troisième étape sur le chemin de l'apprentissage: « Mais pourquoi finalement trois témoins ? » Certes, nos traditions ont quelques réponses à leur disposition pour «expliquer » la venue d'un deuxième et même d'un troisième témoin ou pour rendre raison de leur propre existence après l'arrivée d'un premier et d'un deuxième envoyé: on évoque la dissidence ou l'hérésie de l'héritier, l'aveuglement ou le péché de l'aîné, l'infidélité ou l'exagération des deux prédé-cesseurs.   Mais c'est une chose de raconter la venue progressive d'un premier, d'un deuxième et même d'un troisième, et c'est une autre de penser le « mystère » de leur cohabita-ton dans nos sociétés. N'est-ce qu'un accident de l'histoire, une contingence fortuite ? Ou faut-il y découvrir la main de Dieu, son «dessein » ? Et ce « dessein » est-il même concerné par l'avènement d'une société dont le retrait fondamental par rapport aux trois monothéismes suscite la question de leur «pourquoi» ? Le terme « mystère» n'est sûrement pas trop fort pour désigner la «chose», puisqu'il a déjà servi à Paul pour penser l'énigmatique présence d'Israël aux côtés des chrétiens (Rm 11, 25). Ce « mystère » n'est-il pas devenu plus insondable encore et plus impénétrable (Rm 11, 33-36) depuis que nous sommes «trois témoins » ? La présence des deux autres, que nous apprend-elle sur nous-mêmes et sur l'Unique \sn{792}
que nous ne puissions pas savoir par nous-mêmes ? Question adressée à chacun des trois et à laquelle aucun des trois ne peut répondre à la place des deux autres. 
\paragraph{quelle réponse en tant que Chrétien ?}Nous sommes donc reconduits vers la perspective interne qui ne peut être que chrétienne et théologique pour nous.

\subsection{Entrer dans le règne de l'incomparable}

\paragraph{ne pas compter, ce n'est pas le nombre trois qui est important, mais entrer dans la comparaison (pour y voir in fine un plan de Dieu ?)} Il n'est pas sûr qu'il y ait une réponse à la question «pourquoi trois ?» Peut-être faut-il même renoncer à compter.
On se souvient ici de l'embarras de saint Augustin quand il réfléchit dans son \textit{De Trinitate }sur la différence trinitaire en Dieu: \begin{quote}
    «Le Père, le Fils, le Saint-Esprit sont trois, nous cherchons donc: trois quoi? (tres quid) »
\end{quote}
L'extraordinaire difficulté vient de ce que le terme « personne » (trois personnes) est déjà un générique qui ne peut désigner l'absolu singularité de «chacun», comme le générique « monothéisme » n'arrive jamais à dire ce qu'est l'islam, le christianisme et le judaïsme.
Face à cette limite du nombre dans un domaine où on ne peut «co-numérer», saint Basile conseille de renoncer à compter.
Ce qu'il dit de l'Unique, il faut l'appliquer, me semble-t-il aux trois témoins: \begin{quote}
    « Et s'il faut tout de même compter, du moins que la vérité ne soit point falsifiée: ou bien qu'on honore en silence les choses ineffables, ou bien qu'on compte avec piété et respect. »
\end{quote}  Mais pour pouvoir renoncer à compter il faut bien d'abord compter jusqu'à trois, et pour entrer avec respect dans le règne de l'incomparable il faut bien commencer par comparer. C'est ce que nous avons fait jusqu'ici. Mais dans cette dernière partie je voudrais brièvement tracer un chemin de rencontre qui nous conduit de la comparaison à la découverte de l'incomparable singularité de chacun des trois témoins. \sn{793}
\begin{Synthesis}
Mon hypothèse est que la différence fondamentale du christianisme par rapport au judaïsme et à l'islam, le mystère de l'Incarnation et de la Trinité, est en même temps le lieu  
où se définit une\textsc{ théologie de la rencontre} à la hauteur des enjeux développés dans les deux premières parties. 
\end{Synthesis}

Commençons donc par prendre au sérieux cette différence.
\subsubsection{La contestation de l'unicité du Christ.}

\paragraph{partir du scandale de l'association de Jésus à Dieu pour l'Islam}
Considérer le «témoin » par excellence du christianisme, Jésus, comme un de la Trinité et l'associer à l'unique Dieu, voilà une manière bien scandaleuse, aux yeux des juifs et des musulmans, d'introduire le nombre en Dieu et de diviniser un parmi d'autres sur terre. L'islam a porté sa contestation au cœur même de cette foi: \begin{quote}
    «Ô gens du Livre! N'allez pas au-delà du bon sens dans votre religion. Ne proclamez que la vérité sur Dieu. Le Messie, Jésus fils de Marie, n'est qu'un envoyé de Dieu... Croyez donc en Dieu et en ses prophètes.
Ne dites jamais "trois". Arrêtez cette imposture... Dieu est un Dieu unique. Le Messie ne se sent pas indigne d'être serviteur de Dieu, comme le sont les anges les plus proches de Dieu» (sourate 4, 171s.)

« Il n'est pas concevable que Dieu se donne un fils» couriste 19 35). « Il n'a pas engendré; n'est Dieu» (sourate 4, l/Is.). « Ce n'est pas concevable que Dieu se donne un fils » (sourate 19,35). « Il n'a pas engendré; n'est égal à lui, personne » (sourate 112).
\end{quote}  
\paragraph{des psaumes messianiques comme réponse chrétienne}
Ces quelques versets du Coran nous renvoient, par contraste, à un autre ensemble, les textes messianiques des Écritures juives, et en particulier les psaumes 2 et 110:
\begin{quote}
    «Je proclame le décret du Seigneur. Il m'a dit: "Tu es mon fils; moi, aujourd'hui je t'ai engendré. Demande et je te donne en héritage les nations, pour domaine la terre entière" » (Ps 2, 7). 
    
    «Oracle du Seigneur à mon Seigneur: "Siège à ma droite !... Domine jusqu'au cœur de l'ennemi!" Le jour où paraît ta puissance, tu es prince, éblouissant de sainteté : "Comme la rosée qui naît de l'aurore, je t'ai engendré." Le Seigneur l'a juré dans un serment irrévocable: "Tu es prêtre à jamais selon l'ordre du roi Melchisédech" » (Ps 110, 1-4). 
\end{quote}
\paragraph{Et pour les juifs, laisser ouvert l'avenir du peuple messianique}
Si l'islam conteste l'idée même d'un « engendrement» en Dieu, et cela au nom de sa critique prééthique de toute association d'un pluriel à l'Unique, le judaïsme, lui, ne peut pas reconnaître l'accomplissement définitif de ces psaumes en l'itinéraire de Jésus parce qu'il doit laisser ouvert l'avenir du peuple messianique: au nom même de sa mission éthique il se méfie de tout ce qui occulte la situation d'exil de l'humanité qui durera jusqu'à la fin.

\paragraph{face à cette double contestation, quelle ressort meta éthique ?}
La rencontre de cette double contestation ne nous oblige pas à renoncer aux prérogatives du Christ; je l'ai déjà dit Mais alors en quoi consiste la « purification » de nos schème d'excellence (1re étape de toute rencontre)? Comment pouvons nous aborder positivement la communication avec les deux autres témoins (2e étape de rencontre)? Il est probable que l'enjeu de la rencontre n'est plus d'abord la réaffirmation d'une différence doctrinale mais la mise en œuvre discrète du caractère méta-éthique du christianisme dans l'histoire d la communication humaine : autrement dit, il s'agit pour le chrétiens de vivre de la sainteté même de Dieu, rien de plus e rien de moins. C'est seulement alors que peut leur paraître avec une acuité nouvelle, la capacité de l'Unique à engendre non seulement un «Fils unique » mais aussi avec lui «un multitude de fils ».
\paragraph{Foi Chrétienne comme style de vie, de vivre la sainteté}
On ne soulignera jamais assez le renversement de perspective qui vient d'être produit. La foi chrétienne n'est pas d'abord une doctrine (comme toute une tradition l'a prétendu) mais un « style de vie » ou une manière de vivre de la sainteté même de Dieu: seule l'expérience effective de l'Esprit de sainteté nous permet de confesser et de comprendre un jour l'indépassable excellence du Fils unique du Père. N'est-ce pas cela que la rencontre des autres témoins nous apprend?
Essayons donc de comprendre.
\subsubsection{Unique et la « multitude des fils» (He 2, 10).}
\paragraph{psaumes messianiques : communication de la sainteté de Dieu au roi et à son peuple}
L'enjeu des psaumes messianiques, cités à l'instant, est la communication de la sainteté de Dieu au roi et à son peuple:
\begin{quote}
    « Le jour où paraît ta puissance, tu es prince, éblouissant de sainteté» (Ps 110, 3). 
\end{quote}
S'adressant à la multitude, le Nouveau Testament l'a bien compris quand il appelle tous à être «parfaits comme votre Père céleste est parfait » (Mt 5, 48).
Ce vin nouveau de la perfection divine qui n'est rien d'autre que l'accomplissement inouï et excessif de la Loi (« la justice qui surpasse la justice » selon Mt 5, 20). le Sermon sur la montagne l'introduit dans les «outres» de la Règle d'or:
\begin{quote}
    «Tout ce que vous voulez que les hommes fassent pour vous, faites-le vous-mêmes pour eux : c'est la Loi et les Prophètes » (Mt 7, 12). 
\end{quote}
\paragraph{Sainteté de Dieu, démesurer au delà de la règle de réciprocité}
Cette règle de réciprocité qui se trouve au fondement de nos sociétés modernes peut en effet nous réserve quelques surprises, individuelles et collectives, quand on doit affronter l'antipathie d'autrui ou quand on entend subitement l'invitation à renoncer à toute réciprocité et à prendre sur soi la violence et la faute d'autrui: \begin{quote}
    « Aimez vos ennemis, et priez pour ceux qui vous persécutent, afin d'être vraiment les fils de votre Père aux cieux, car il fait lever son soleil sur les méchants et les bons, et tomber la pluie sur les justes et les injustes » (Mt 5, 44 s.).
\end{quote}
Qu'il s'agisse, dans la communication de la sainteté de Dieu à l'homme, d'un véritable \textit{engendrement} \sn{1. Cette communication de la sainteté même de Dieu fixe définitivement le sens du terme « engendrement », jusque dans la haute christologie de Nicée (« engendré non pas créé, de même nature que le Père »). La forme évangélique ou « narrative » de cet engendrement a été abordée dans le chapitre Il de la deuxième partie, p. 477 s. Je reviendrai aux chapitres v, vi et xin de cette partie à son enjeu christologique et eschatologique; voir plus loin, p. 826, p. 849 s. et p. 1036 s.}
(« devenir fils de votre Père»), on ne le découvre que progressivement: quand on réalise tout d'un coup que l'appel démesuré à être \textit{comme} Dieu, toujours dans telle ou telle situation, s'avère «à la mesure» de celui qui l'entend. 
\paragraph{unification traverse les forces de mort, de peur de l'autre et de nous mêmes. Nous nous "comparons", violence entre témoins}
Mais loin d'être d'emblée acquise, cette gracieuse unification affronte et traverse des violences et des forces de mort qui se déclarent précisément à la « limite » infiniment mobile entre la « démesure divine » et nos multiples « mesures humaines ». La peur face à l'inconnu
- celui de l'autre, de nous-mêmes et finalement de la mort - nous pousse à fixer nos frontières, à garder nos terrains et à entrer dans un jeu de comparaison, voire une lutte sans merci entre « semblables ». Jamais pourtant ce que l'un accueille de la «démesure», \textit{inscrite en tout être humain}, ne se laissera mesurer en fonction de ce qu'en conscience l'autre jugera «à sa mesure». Qui peut faire sortir l'homme de ses mortelles comparaisons, sinon celui qui atteint vraiment en lui la peur d'être soi-même, peur qui est sans doute la racine ultime des mystérieuses violences entre « témoins » ? Il faut entendre la voix d'un autre pour lâcher cette peur; voix qui certes doit émaner de quelqu'un devenu « proche » mais voix qui doit venir en même temps de plus loin: de Celui qui, ici et maintenant, se montre « à la mesure» d'un tel ou d'une telle, se faisant entendre à lui et en lui, avec la douceur et la discrétion
de l'Esprit: «Aujourd'hui, je t'ai engendré» (Lc 3, 22 et Ac 13, 33).

\paragraph{Expérience de l'engendrement : en Christologie, permet le pluriel (He). L'association  de Jésus à l'unicité de Dieu ne peut être séparé de berger de Paix et prêtre de Melchisédech}
Cette expérience inouïe de « l'engendrement » nous fait comprendre les deux versants intimement liés d'une théologie chrétienne de la rencontre. D'abord le versant christologique: la passion de certains auteurs du Nouveau Testament pour le « pluriel ». À ce propos l'épître aux Hébreux, citée tout au début, est tout à fait exemplaire parce qu'elle déplace la «filiation divine » vers la condition commune de tous: \begin{quote}
    «Il convenait, en effet, écrit-il, à celui pour qui et par qui tout existe et qui voulait conduire à la gloire une multitude de fils, de mener à l'accomplissement par des souffrances l'initiateur de leur salut. Car le sanctificateur et les sanctifiés ont tous une même origine; aussi ne rougit-il pas de les appeler frères» (He 2, 10 s.)
\end{quote}
Certes, l'unicité ou l'excellence de Jésus n'est pas niée dans ce texte étonnant: il reste pour l'épître aux Hébreux celui qui ouvre, au cœur de nos mesures humaines, l'accès à la sainteté démesurée de Dieu. Mais Jésus n'est «associé » à l'unicité de Dieu (He 7, 2 s.) que parce qu'il est en même temps berger de paix (roi de Salem) et prêtre selon l'ordre du roi Melchisédech, l'homme unique donc qui est entièrement façonné par le don de soi et appelé à s'effacer pour ouvrir dans l'humanité \textbf{les chemins multiformes de la sainteté de Dieu}.
\paragraph{Expérience de l'engendrement : spirituellement, pas de comparaison entre Fils et témoin }
Ensuite, le versant spirituel ou pneumatologique de la rencontre. La découverte que Dieu engendre une « multitude de fils » sur les chemins de sa sainteté amène à renoncer à toute comparaison entre fils et témoins. Dieu n'est-il pas à la mesure de tant et de tant de mesures humaines, devenues toutes, de ce fait, incomparables? Le Dieu unique des chrétiens est mystère du lien entre des incomparables \sn{voir p 77}. Mais il faut s'être affronté, dans sa propre  vie, à la question de la sainteté, de la démesure divine à ma mesure, pour pouvoir admettre que juifs et musulmans sont, eux aussi, aux prises avec un même combat. On sera alors conduit non seulement au respect dans la rencontre des deux autres témoins mais encore au risque d'y « laisser sa peau » : \begin{quote}
    « En Christ, je dis la vérité, je ne mens  pas, par l'Esprit saint ma conscience m'en rend témoignage », écrit Paul dans l'épître aux Romains. « Oui, je souhaiterais être anathème, être moi-même séparé du Christ pour mes frères..., eux qui sont les Israélites » (Rm 9, 1-5).
\end{quote}
Voilà ce que j'entends par « style chrétien de rencontre»; 
\begin{Def}[style chrétien de rencontre]
    «style» qui se caractérise par une singulière manière d'espérer la paix en affrontant la violence. 
\end{Def}
Il se « définit » au lieu même de la différence fondamentale du christianisme par rapport au judaïsme et à l'islam, dans le mystère de l'Incarnation et de la Trinité. Ce qui a été mon hypothèse.

\subsection{En guise de Conclusion : un jeu de compétition}

\paragraph{« compétition » autour de la sainteté de Dieu, le mode de victoire n'est pas le même pour tous}
La rencontre des trois monothéismes est en dernière instance une « compétition » autour de la sainteté de Dieu.
\begin{quote}
    «Les coureurs, dans le stade, courent tous mais un seul gagne le prix» (1 Co 9, 24),
\end{quote}
 écrit saint Paul qui affectionne la métaphore grecque des Jeux olympiques!. Le juif Philon d'Alexandrie l'a précédé quand il loue dans le \textit{De agricultura} 
 \begin{quote}
 «le seul concours olympique » qui « pourrait être appelé sacré à juste titre: non pas celui que célèbrent les gens d'Élide, mais celui qui vise à acquérir les vertus divines et vraiment olympiennes», en ajoutant, avec une rare finesse que « le mode de la victoire n'est pas le même pour tous mais que tous sont dignes d'estime2».    
 \end{quote}
  On connaît par ailleurs ce verset du Coran. \begin{quote}
      « Si Dieu avait voulu, il aurait fait de vous une seule communauté. Mais il a voulu vous éprouver par le don qu'il vous a fait. Cherchez à vous surpasser les uns les autres dans vos bonnes actions » (sourate 5, 48).
  \end{quote}

\paragraph{rôle des chrétiens : montrer l'unicité de chaque témoin}
Le rôle spécifique des chrétiens dans ce singulier « jeu de compétition », aux allures parfois dramatiques, ne serait-il pas de renoncer à compter et à comparer pour mettre en valeur l'unicité incomparable de chaque partenaire? Tâche difficile qui peut les conduire aujourd'hui encore dans l'expérience du don de soi.\sn{798}
\paragraph{mystère de la diversité des témoins : penser la pluralité des monothéismes}
Mais revenons, pour finir, à notre question: pourquoi des témoins (dernière étape de la rencontre)? Renoncer un jour à poser cette question et abandonner le « comptage », c'est certes accepter avec l'épître aux Romains que le « dessein » de Dieu est insondable et impénétrable. Mais il ne s'agit absolument pas d'un scepticisme qui renoncerait à penser vraiment pluralité des monothéismes. Penser le mystère des trois témoins c'est baliser, au sein d'une histoire faite de compétitions chemin de rencontre qui nous amène à vivre dans le règne trinitaire de l'incomparable.
\paragraph{Plan : on avait pensé la communauté ecclésiale, il faut penser le sans religion}
Ce qui précède nous a déjà conduit vers le deuxième lien de l'expérience trinitaire de la foi: la référence de la communauté ecclésiale à la sainteté messianique de Jésus de Nazareth.
Les deux dernières études de cette première étape entreront davantage dans la question christologique en faisant d'abord état de la transformation de la christologie catholique au xxe siècle et en revenant ensuite au rapport du Christ aux hommes religieux et à ceux qui sont « sans religion ».

 
\section{devoir}

\section{L'unique et ses témoins, le christianisme comme style}

\subsection{Christoph Theobald}

\paragraph{Théologien allemand vivant en France} Chrstoph Théobald est un jésuite d'origine allemande et vivant en France. Véritable pont entre les deux cultures théologiques, il est en particulier connu pour sa traduction des oeuvres de Karl Rahner. Il est aussi très marqué par la philosophie Française, Merleau-Ponty, Lévinas et Ricoeur.

\paragraph{Curieux de la théologie contemporaine} Il a été longtemps le directeur des RSR, la \textit{Revue des Sciences Religieuses}, présentant les dernières recherches de théologie fondamentale au public français. Il porte aussi les questions de son temps comme le montre l'un de ses derniers ouvrages, \cite{theobald_urgences_2017}. De même, le chapitre que nous étudions a d'abord fait partie d'un colloque \cite{centre_sevres_paris_unique_1996}

\paragraph{Un théologien Jésuite} Christoph Théobald est jésuite. Outre son intérêt pour les théologiens Karl Rahner et Urs van Balthazar, il est marqué par l'esprit \textit{Exercices Spirituels} et de suivre Jésus concret. Certains passages de \cite{theobald_christianisme_2007} nous ont rappelé les méditations de l'appel du roi temporel (ES 91)  qui propose \textit{l'imitation de Jésus en endurant tous les outrages, tout blâme et toute pauvreté (ES 98)}



\subsection{Bibliographie}
\sn{Eventuellement, notez l’ensemble des textes et références auxquels vous avez eu recours pour préparer l’exposé, y compris vos sources pour la biographie et les sites internet visités. }





\subsection{Nature du texte}


\paragraph{Un colloque} Dans sa \href{https://centresevres.com/content/uploads/2017/07/bibliographie-complete-de-christoph-theobald-sj-2021.pdf}{bibliographie}, on note que le texte a eu au moins trois moutures différentes : en 1995, lors du colloque de rentrée au Centre Sevres (numéro 60 de la bibliographie), publié en 1996 \cite{centre_sevres_paris_unique_1996} dans  une version différente du texte précédent.  

\paragraph{Puis un chapitre du \textit{Christianisme comme Style}} Théobald a publié le \textit{Christianisme comme style, une manière de faire de la théologie en postmodernité}\cite{theobald_christianisme_2007} en 2007. Le livre comment par une \textit{ouverture} sur le Christianisme comme \textit{Style}, reprenant le terme de Merleau-Ponty (verif). 

 \begin{quote}
     Après avoir "ausculté" notre présent et désigné le \textit{kairos} qu'il représente (I) et après avoir réfléchi longuement à l'enracinement spirituel (II) et scripturaire (III) de la théologie chrétienne, le moment est venu d'aborder directement ce que celle-ci doit donner à penser aujourd'hui : le christianisme comme style qui ouvre à une intelligence de lui-même, libre et accessible à tous, petits et grands. 
     \cite[p 699]{theobald_christianisme_2007}
 \end{quote}
\begin{quote}
    

... herméneutique dogmatique qui tente de penser le versant normatif du mystère chrétien en relation constante avec l'histoire et la pratique actuelle de l'Eglise.  700
\end{quote}
Suivant le le symbole de Nicée, il commence par une réflexion sur Croire en Dieu  :
\begin{quote}
    le christianisme est il un monotheisme ?" 703
\end{quote}
Mais n'y répond pas directement.


Le chapitre 3 de cette partie est le chapitre qui nous intéresse, l'Unique et ses témoins, Jalons pour une théologie de la rencontre entre juifs, chrétiens et musulmans.
Il est précédé par un premier chapitre sur Dieu en postmodernité, une manière de désigner la sainteté comme mystère du monde, et le chapitre 2 sur la foi trinitaire des chrétiens et l'énigme du lien social.
Il précède une reflexion sur la christologie. 




\subsection{Le contexte historique et textuel}

La publication dans le contexte de la polémique de la conférence donnée à Ratisboonne sur \textit{Foi, Raison et Université} du pape Benoit XVI du 12 septembre 2006, qui cite le dialogue publié par le professeur Khoury (de Münster) entre l’empereur by-
zantin lettré Manuel II Paléologue et un savant persan sur le lien entre raison et religion \sn{vérifier que les premières versions ne contiennent pas de référence à Benoit XVI}. Cette polémique entraîna une vague de violence dans le monde musulman. 
Situez la production du texte dans son contexte historique (date de la publication). A quelle occasion a-t-il été rédigé (suite à quel événement), dans quel contexte culturel et social, etc. ? A qui est-il adressé ?

Situez le texte dans son environnement littéraire, s’il est extrait d’un ouvrage ou d’un corpus. Présentez rapidement l’ouvrage, indiquez ce qui précède et ce qui suit le texte choisi, etc. Est-ce une traduction ? \mn{s'il y a 700 pages, dire les chapitres avant et après}



\section{Présentation}


	\subsection{5.1 La problématique }

Déterminer la problématique en vous inspirant des questions suivantes :
Après la phase de lecture pas à pas, vous construisez la question à laquelle l’auteur s’affronte :
-	pourquoi l’auteur se bat-il ?
-	quel problème essaie-t-il de régler, d’éclairer ? En général, il n’est pas difficile de trouver exposée la problématique en toute lettre dans le texte lui-même – parfois même de manière redondante ;
-	en fonction de quel contexte culturel, social, culturel, économique, politique l’auteur construit-il sa problématique ? Notez les événements déterminants auxquels il se réfère et les auteurs qu’il évoque – plus ou moins explicitement – comme ses alliés ou ses adversaires – et le cas échéant, renseignez-vous sur eux. 

\subsection{5.2	 La thèse ou plutôt hypothèse}

En fonction de la problématique de l’auteur, vous établissez la thèse (ou hypothèse) de l’article ou du texte étudié : quelle solution l’auteur apporte-t-il à sa question ? quelle perspective établit-il ? Là encore, il n’est pas difficile de trouver cette thèse exposée de manière explicite dans le texte lui-même. 

\subsection{5.3	 L’argumentation }


Présentez la logique argumentative en fonction de laquelle l’auteur établit sa thèse (passe de l’hypothèse à la thèse vérifiée).



\section{Reaction personnelle en discutant et critiquant de manière argumentée}

\paragraph{partir de Jésus}


\section{vision aux He} pertinent pour le Judaisme. pour l'islam ?
souffrance ? peut on penser la souffrance comme apprentissage ? peut être 
 

\section{Règle d'Or : dignité et perfection ? }

 \subsection{Medine et la Mecques} juridique vs anti juridique : on ne peut rester à l'ante juridique



 






