\chapter{Du Christ à Jésus IV
Les possibilités offertes par la christologie de Pannenberg}
 

 

\section{Bibiographie}
\begin{itemize}
    \item GANOCZY, A., Théologie en modernité. Une introduction à la pensée de Wolfhart
Pannenberg, Paris 2018.
GUE, X., La christologie de Wolfhart Pannenberg. De la modernité à la postmodernité,
Zürich 2016.
PANNENBERG, W., Esquisse d’une christologie, Paris 19992.
PANNENBERG, W., « Christologie et théologie », Les Quatre Fleuve 4 (1975), 85-99.
PANNENBERG, W., Théologie systématique **, Paris 2011.
\end{itemize}



\section{Introduction}


\subsection{a) Bilan des christologies du Logos dans le contexte du pluralisme religieux}

\paragraph{Le Christ est Jésus mais Jésus n'est pas le Christ} Jésus n'épuise pas le Christ. le Christ est plus grand que Jésus. Très clair chez Pannikar, mais même chez Dupuis, il y a un écart.

\paragraph{Vice caché} le Christ / logos est pensé en dehors de la figure historique de Jésus. 
\begin{Ex}[homme agréable à Dieu chez Kant]
On a aussi quelque chose d'extrinsèque à la vie de Jésus. 
\end{Ex}

\paragraph{Nestorius : unité entre l'humanité concrete et le fils de Dieu}
Le relachement de l'identité du Christ rappelle cette difficulté de Nestorius. 

\paragraph{il s'agit de penser une théologie du Logos pensé à partir de l'histoire de Jésus, de sa mort et de sa resurrection}



\subsection{Retour au problème : l’absolutisation d’une événement historique et singulier} 

\paragraph{Comment être fidèle à la fois de l'Eglise en partant d'une théologie du haut} 

\paragraph{Autre approche : Le Christ}

\paragraph{Règle de foi de l'Eglise doit nous servir de gramme} Lien définitif entre Dieu et l'homme : on ne peut pas le distendre. 

\paragraph{l'objectif : Desabsolutiser le Christ sans casser la divinité de Jésus
} 

%-----------------------------------
\section{De la question de la modernité à la question christologique postmoderne}

\paragraph{répondre directement au problème posé par le dialogue inter-religieux} Parfois, il est nécessaire de contourner le mur. Regarder dans le passé à des problèmes similaires. 

\paragraph{le problème post moderne ressemble à la question de la modernité} Montrer pourquoi.


\subsection{La question christologique de la modernité} Au XVIII, on se pensa la question entre la figure contingente, \textit{Jésus} et comment il peut être sauveur universel.

\paragraph{Lessing } il y a des lois universelle mais on ne peut pas prouver des lois universelles par un événement de l'histoire (un exemple pas une preuve) : 
\begin{quote}
    « Des vérités historiques accidentelles ne peuvent jamais devenir la preuve de vérités
rationnelles nécessaires ». (Lessing, Uber den Beweis des Geistes und der Kraft, 1777).
\end{quote}

L'histoire relativise : on préfère les lois universelles à des événements particuliers. 




\subsection{La question christologique de la postmodernité}

 
\paragraph{De nombreuses réflexions pour y répondre} Kant a proposé que Jésus était un exemple de loi morale. 
\paragraph{Crise moderniste} Au cours du XXè, l'irruption du problème historique dans la théologie catholique.

\paragraph{Un refoulement de la question historique} que fait Ponce Pilate dans le Credo ? Grain de sable. Montre l'enracinement historique de notre foi. \textit{Pourquoi finalement on peut affirmer le lien entre l'histoire et le discours (universel) sur Dieu ?} Il s'agit de sauver Jésus du relativisme. Tâche de la \textit{théologie fondamentale}.

\paragraph{comment annoncer Jésus dans un monde fragmenté ? } Benoit XVI
\begin{quote}
    Pour les occidentaux, le Christ est la vérité.
    Pour les asiatiques, on l'approche par la voie.
    Pour les africains, on est sensible à la vie. 
    \textit{Général Jésuite Adolfo Nicolás ?}
\end{quote}

Comment parle-t-on du Christ aujourd'hui ? 

\begin{Prop}

Annoncer un Christ qui soit reçu dans différentes cultures du monde. 
\end{Prop}
\subsection{L’homologie christologique entre la modernité et la postmodernité}

Si on fait de la théologie directe, on fait un combat de boxe. Partir par des biais et des détours pour répondre aux questions. 

\paragraph{un relativisme diachronique : l'histoire}
\paragraph{un relativisme synchronique : plusieurs visions}



\subsection{Qui est Wolfhart Pannenberg ?}
 1928 (Allemagne)-2014. Lutherien.
 En janvier 1944, expérience spirituelle. Après la guerre, il travaille sur la théologie médiévale (Dun Scott). 

 \paragraph{remettre l'histoire dans la théologie} On ne peut accéder à Dieu hors des événements de l'histoire. Une pensée post-Auschwitz. 

 \paragraph{Manifeste : Révélation comme histoire} Ces jeunes théologiens vont se distinguer des dialecticiens (K. Barth, Bultman), et vont affirmer l'importance du récit et de l'histoire. 

\paragraph{refonder la christologie à partir du Jésus de L'histoire} 1964 : publie sa christologie. Il va dialoguer avec les sciences, l'histoire,... tous les domaines du savoir. C'est roboratif (consistant). 

\paragraph{Oecuménisme} il a travaillé sur le document sur le salut en 1999. 


\section{La christologie de Pannenberg}

 \subsection{Histoire et résurrection de Jésus}

\paragraph{La revendication d’autorité de Jésus} Il annonce le Royaume. L'autorité de l'annonce du Royaume permet la justification du Royaume. Il s'appuie sur ce qu'il considère comme parole authentique : 
\begin{quote}
    Lc 9,26 : « Car si quelqu’un a honte de moi et de mes paroles, le Fils de l’homme aura honte
de lui quand il viendra dans sa gloire, et dans celle du Père et des saints anges ».
\end{quote}
\begin{quote}
    Le 12,8-9 : « Je vous le dis : quiconque se déclarera pour moi devant les hommes, le Fils de
l’homme aussi se déclarera pour lui devant les anges de Dieu ; mais celui qui m’aura renié par
devant les hommes sera renié par devant les anges de Dieu. »
\end{quote}
Pannenberg : Jésus se distingue du fils de l'homme. Il se présente comme un prophète eschatologique, annonçant le Royaume avec des signes avant coureur du royaume.  

\paragraph{L’attente d’une confirmation} Mais Jésus était une question ouverte. Il fallait donc attendre la venue de la venue du Royaume pour \textit{confirmer} sa revendication d'autorité. 
\begin{quote}
    « Toute l’action de Jésus restait donc suspendue à la vérification future de sa revendication
d’autorité, à une confirmation que Jésus ne pouvait pas donner, justement parce qu’il
s’agissait de la légitimation de sa propre personne, liée à la venue des événements annoncées
pour la fin » (Pannenberg, Esquisse, 71).
\end{quote}

\paragraph{La place centrale de la résurrection de Jésus} 
\begin{quote}
    « L’unité de Jésus avec Dieu n’est pas encore fondée par la prétention impliquée dans son
comportement prépascal, mais seulement par sa résurrection d’entre les morts » (Pannenberg,
Esquisse, 55).
\end{quote}
Historiquement, il se distinguait du fils de l'homme mais ensuite, à la lumière de la resurrection, les témoins ont trouvé la figure de Jésus du fils de l'homme. L'humanité de Jésus est compréhensible par le fait que la fin de l'histoire prend un sens de toute l'histoire.
\begin{Prop}
    L'identité narrative (Paul Ricoeur). Qui est quelqu'un ? on ne peut répondre qu'à la fin de sa vie et pour le dire, on raconte une histoire.
    Ici, la même chose. La mort et la résurrection de Jésus est la fin de l'histoire, mouvement de Dieu dans l'histoire. 
\end{Prop}
L'histoire où Dieu se révèle définitivement dans la résurrection de Jésus. 
\mn{
Vision totalisante de l'histoire (comme les marxistes)}

\begin{Synthesis}

\end{Synthesis}
\subsection{L’identité divine de Jésus est connue indirectement}

\paragraph{Par le biais de l'eschatologie, il peut faire le lien entre l'histoire d'en bas et la divinité} Dans la théologie classique, on avait oublié le Jésus historique (galiléen,...) et uniquement son essence (hypostase...). On connaissait Jésus sans connaître son histoire. 

\paragraph{Panneberg critique cette réduction} On peut dire que Jésus est le fils du père en regardant comment il a vécu. Critique de la neutralisation de l'histoire. 

\mn{interessant par rapport au dialogue avec l'islam. Explique pourquoi les chrétiens ont attaqué le mohammed concret. l'homme Jésus et sa vie est d'une certaine façon équivalente à la grammaire arabe ?}
\begin{quote}
    « L’unité de Jésus avec Dieu ne doit pas être conçue comme l’union de deux substances, mais
en ce sens que Jésus est Dieu en tant qu’il est cet homme » (Pannenberg, Esquisse, 359).


« L’unité de Jésus avec Dieu, l’unité du Jésus historique concret (toujours énigmatique à bien
des égards, mais si nettement caractérisé qu’il défie toute confusion) avec le Dieu de la Bible,
de l’AT, avec celui que Jésus appelait Père, cette unité ne se laisse découvrir que dans la
singularité historique de l’homme Jésus, de son message et de son destin (…) C’est en tant
qu’il est cet homme que Jésus est Dieu » (Pannenberg, Esquisse, 412).
\end{quote}

\begin{quote}
    « C’est seulement la communauté de personne avec le Père qui manifeste Jésus comme le Fils
de Dieu » (Pannenberg, Esquisse, 415).
\end{quote}
Il n'est pas directement le fils de Dieu, mais indirectement : On ne peut accéder à la réalité ontologique de Jésus que par son existence concrète. 

\paragraph{le verbe s'est fait histoire} on court-circuite souvent "il est le fils de Dieu". non, il est fils de Dieu car il a vécu tout cela jusqu'à la croix.
\textit{Cela donne du contenu au terme logos}

\subsection{La christologie du Logos}

Il se méfie d'une théologie du logos dans un premier temps car les pères apologétiques avaient "oublié" le Jésus historique par le logos.

\paragraph{La création fondée sur le Logos} Trinitairement, le fils s'auto distingue du Père. C'est parce que le fils se distingue du père que la Création est possible. 
\begin{quote}
    « Si, du point de vue noétique, la connaissance de la filiation éternelle de Jésus (…) repose
sur la singularité même de cette humanité dans sa relation avec le Père, on constate l’inverse
du point de vue ontologique : car, dans l’être, la filiation divine caractérise ce en quoi
l’existence de Jésus, unie au Père et cependant distincte de lui, trouve le fondement de son
unité et de son sens » (Pannenberg, Esquisse, 433).
\end{quote}
\begin{quote}
    « L’incarnation du Fils est la conséquence de son autodistinction trinitaire d’avec le Père. Sa
communion éternelle avec le Père est déjà médiatisée par cette libre autodistinction. De même
que l’autodistinction du Fils d’avec le Père constitue le principe de la possibilité de toute
réalité créée distincte de Dieu, de même est-elle à la source de son incarnation en Jésus de
Nazareth » (Pannenberg, Théologie systématique **, 440).
\end{quote}
Cette idée du logos vient de la méditation de la vie même de Jésus.


\paragraph{La logique d’autodistinction dans la vie et le destin de Jésus}  L'amour doit être libre donc il faut se distinguer. il ne s'est pas fait l'égal. il ne s'est pas annoncé lui-même. 

 \begin{quote}
     « Au centre du message se trouvent le Père et son Royaume qui vient et non une dignité que
Jésus aurait revendiquée pour sa propre personne, par laquelle il se ferait l’égal de Dieu (Jn
5,18). Jésus, en tant que simplement homme, s’est distingué du Père, comme Dieu unique, et
s’est soumis à l’exigence du Royaume qui vient » (Pannenberg, Théologie systématique **,
506).
 \end{quote}

\paragraph{Dieu se révèle dans la Croix} dans ce qui n'a pas de valeur. 


\section{Un Christ « dés-absolutisé » et un « singulier » Logos}

Pannenberg n'a pas réfléchi au dialogue avec la culture et les autres religions mais il a des semences.

\subsection{Les conséquences de la prise en compte de l’histoire en christologie} 

\paragraph{Christ Chemin et Christ Glorieux} En Christ, il y a toujours le chemin et le but. On avait vu avec Theobald, il faut faire le cheminement. Le risque est de court-circuiter ce chemin.  

\paragraph{Religions} On peut dialoguer si on tient compte de l'histoire de Jésus. 
\paragraph{En quoi est il unique médiateur ? }

\subsection{Il est médiateur en ne se faisant pas l’égal de Dieu}
Hymne aux Philippiens, 
\begin{quote}
    Ph 2,6
\end{quote}
A la différence d'Adam, il sort vainqueur de la tentation au désert. Mis à l'épreuve à la manière d'Adam. Il est emmené au désert par Dieu. Jésus apparait comme le nouvel Adam. \textit{Jésus ne se prend pas pour Dieu, il aurait chuté comme Adam}.

\paragraph{Les tentations reviennent à sa crucifixion} "descends de la Croix (Mc 15, 25-31)\sn{Mc 15,29-31 : « Hé ! Toi qui détruis le Sanctuaire et le rebâtis en trois jours, sauve-toi toimême
en descendant de la croix (…) Il en a sauvé d’autres, il ne peut pas se sauver luimême
! Le Messie, le roi d’Israël, qu’il descende maintenant de la croix, pour que nous
voyions et que nous croyions ».}. Les tentations sont au début et à la fin de son ministère. Il fallait qu'il se distingue de Dieu pour révéler Dieu. 
\begin{quote}
    « Dans la crucifixion de Jésus son autodistinction d’avec le Père, qui est également la
condition de son identification comme Fils de Dieu à la lumière de la résurrection, trouve son
acuité extrême » (Pannenberg, Postface dans Esquisse d’une christologie, 525).
\end{quote}


\subsection{Il est médiateur en annonçant que seul Dieu est sauveur
}

\paragraph{Jésus dénonce les puissants} et critique les riches, les autorités politiques au nom du Règne qui vient. Seul Dieu est le sauveur des Hommes.
\begin{quote}
    Magnificat : Lc 1,51-53 : « Le Puissant (…) a dispersé les hommes à la pensée orgueilleuse ; il a jeté les
puissants à bas de leurs trônes et il a élevé les humbles ; les affamés, ils a comblés de biens et
les riches, ils a renvoyés les mains vides ».
\end{quote}

\begin{quote}
    1 Co 1,17 : « Ce qui est folie de Dieu est plus sage que les hommes, et ce qui est faiblesse de
Dieu est plus fort que les hommes »
  
\end{quote}
\subsection{Il est médiateur en prenant soin de l’homme et en s’identifiant aux plus petits}

\paragraph{Jésus annonce le Règne de Dieu qui élève les pauvres}

\begin{quote}
    Mt 25
\end{quote}
 
\paragraph{Cela parle à tous de reconnaitre les plus pauvres} Si les péchés, les plus pauvres sont sauvés. Si les marges sont sauvées, alors tous sont sauvés.


\subsection{Il est médiateur par ses rencontres et la nouvelle fraternité qu’il initie entre les
hommes}

 On retrouve cela chez Théobald.

 \paragraph{Attitude de service des personnes} Il remet debout les personnes rencontrées. Il crée une nouvelle humanité, avec ses disciples. 

 \paragraph{médiation : il y a un décentrement fondamental} Il est médiateur en se mettant au dernier rang, en se distinguant de Dieu pour laisser les autres entrer.
 

\section{La médiation universelle du Christ décentrée et le dialogue}
