\chapter{De Jésus au Christ - 2}
\mn{31/1/23}
\section{Eléments bibliographiques}
\begin{itemize}
    \item CHÉNO, R., Dieu au pluriel. Penser les religions, Paris 2017.
    \item COMMISSION THEOLOGIQUE INTERNATIONALE, Le christianisme et les religions, Rome
1997.
    \item DALFERTH, I. U., « Inkarnation : Der Mythos vom inkarnierten Gott » dans Der auferweckte
Gekreuzigte, Tübingen 1994, 1-37.
    \item DUPUIS, J., Vers une théologie chrétienne du pluralisme religieux, tr. par O. PARACHINI,
Paris 1997.
  \item HICK, J., The Myth of God Incarnate, Londres 1977.
  \item KNITTER, P. F., No Other Name ? A Critical Survey of Christian Attitudes Toward the World
Religions, Maryknoll 1985.
  \item KNITTER, P. F., « La théologie catholique des religions à la croisée des chemins », Concilium
203 (1986) 129-138CHÉNO, R., Dieu au pluriel. Penser les religions, Paris 2017.
  \item COMMISSION THEOLOGIQUE INTERNATIONALE, Le christianisme et les religions, Rome
1997.
  \item DALFERTH, I. U., « Inkarnation : Der Mythos vom inkarnierten Gott » dans Der auferweckte
Gekreuzigte, Tübingen 1994, 1-37.
  \item DUPUIS, J., Vers une théologie chrétienne du pluralisme religieux, tr. par O. PARACHINI,
Paris 1997.
  \item HICK, J., The Myth of God Incarnate, Londres 1977.
  \item KNITTER, P. F., No Other Name ? A Critical Survey of Christian Attitudes Toward the World
Religions, Maryknoll 1985.
  \item KNITTER, P. F., « La théologie catholique des religions à la croisée des chemins », Concilium
203 (1986) 129-138
\end{itemize}

%--------------------------------------- 
\section{Introduction}

Un certain nombre de chrétiens ont voulu renouvelé le christianisme à partir de l'histoire. 

Shleiermacher :  Comment finalement des personnes charismatiques peuvent avoir de l'influence ? Mais pas limité au Christ.
Harnack : A partir du moment où l'homme vit de la relation à Dieu, il 



%--------------------------------------- 
\section{L’émergence de la théologie du pluralisme religieux}
%--------------------------------------- 
\subsection{La théorie des trois paradigmes}
 

%--------------------------------------- 
\subsection{Le promoteur du « pluralisme » : J. Hick}
 

%--------------------------------------- 
\subsection{Qui est Paul F. Knitter ?}
 


%--------------------------------------- 
\section{La christologie de Knitter : L’unicité relationnelle de Jésus}

%--------------------------------------- 
\section{L’unicité relationnelle de Jésus et la christologie du NT}

%--------------------------------------- 
\subsection{Jésus était théocentrique}
 
 
%--------------------------------------- 
\subsection{Du royaume de Dieu au Fils de Dieu}

 
%--------------------------------------- 
\subsection{La christologie est, depuis le début, dialogique, pluriforme et évolutive.}
 
a) De la diversité des christologies à l’unité de la christologie
b) Revenir à la diversité des christologies

%--------------------------------------- 
\section{Unicité et exclusivité du Christ dans le NT}

%--------------------------------------- 
\subsection{Le contexte historico-culturel} 
 
%--------------------------------------- 
\subsection{« Unique et seul » - les traits du langage confessionnel}
 

%--------------------------------------- 
\section{La réinterprétation de l’incarnation dans la théologie actuelle}
 

%--------------------------------------- 
\section{La question de la résurrection}
 
%--------------------------------------- 
\section{Critique de la proposition de la théologie pluraliste}
 
7.1 Au point de vue philosophique : la question de la vérité
7.2 Au point de vue de l’exégèse biblique et néotestamentaire
7.3 Au point de vue de la théologie et du salut