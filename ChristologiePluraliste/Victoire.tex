Frères,





(et soeurs),



Ce dimanche, nous écoutons l'Evangile du Bon Pasteur : 

\begin{singlequote}
En ce temps-là, Jésus déclara :
\begin{quote}
      « Amen, amen, je vous le dis :
celui qui entre dans l’enclos des brebis
sans passer par la porte,
mais qui escalade par un autre endroit,
celui-là est un voleur et un bandit.
    Celui qui entre par la porte,
c’est le pasteur, le berger des brebis.
    Le portier lui ouvre,
et les brebis écoutent sa voix.
Ses brebis à lui, il les appelle chacune par son nom,
et il les fait sortir.
    Quand il a poussé dehors toutes les siennes,
il marche à leur tête,
et les brebis le suivent,
car elles connaissent sa voix.
    Jamais elles ne suivront un étranger,
mais elles s’enfuiront loin de lui,
car elles ne connaissent pas la voix des étrangers. »
\end{quote}
  

    Jésus employa cette image pour s’adresser aux pharisiens,
mais eux ne comprirent pas de quoi il leur parlait.
C’est pourquoi Jésus reprit la parole :
\begin{quote}
    « Amen, amen, je vous le dis :
Moi, je suis la porte des brebis.
    Tous ceux qui sont venus avant moi
sont des voleurs et des bandits ;
mais les brebis ne les ont pas écoutés.
    Moi, je suis la porte.
Si quelqu’un entre en passant par moi,
il sera sauvé ;
il pourra entrer ; il pourra sortir et trouver un pâturage.
Le voleur ne vient que pour voler, égorger, faire périr.
Moi, je suis venu pour que les brebis aient la vie,
la vie en abondance. »
\end{quote}
 Jn 10,10
\end{singlequote}

Ce passage est aussi fameux car il a été repris par Nietzsche dans \textit{Also sprach Zarathustra} :

\begin{singlequote}
Mes yeux se sont ouverts : J’ai besoin de compagnons, de compagnons vivants, 
– non point de compagnons morts et de cadavres que je porte avec moi où je veux. Mais j’ai besoin de compagnons vivants qui me suivent, parce qu’ils veulent se suivre eux-mêmes partout où je vais. 

Mes yeux se sont ouverts : Ce n’est pas à la foule que doit parler Zarathoustra, mais à des compagnons ! 

Zarathoustra ne doit pas être le berger et le chien d’un troupeau ! 
C’est pour enlever beaucoup de brebis du troupeau que je suis venu. Le peuple et le troupeau s’irriteront contre moi : 

Zarathoustra veut être traité de brigand par les bergers. Je dis bergers, mais ils s’appellent les bons et les justes. Je dis bergers, mais ils s’appellent les fidèles de la vraie croyance. Voyez les bons et les justes ! Qui haïssent-ils le plus ? Celui qui brise leurs tables des valeurs, le destructeur, le criminel : 
– mais c’est celui-là le créateur. 

Voyez les fidèles de toutes les croyances ! Qui haïssent-ils le plus ? Celui qui brise leurs tables des valeurs, le destructeur, le criminel : 
– mais c’est celui-là le créateur. 
Des compagnons, voilà ce que cherche le créateur et non des cadavres, des troupeaux ou des croyants. Des créateurs comme lui, voilà ce que cherche le créateur, de ceux qui inscrivent des valeurs nouvelles sur des tables nouvelles. 
\end{singlequote}


\paragraph{Prendre Nietzsche au sérieux} Il convient tout d'abord de prendre Nietzsche au sérieux. Il y a ici une double critique qui s'adresse à tous les croyants mais en particulier aux chrétiens (et aux juifs) :
\begin{itemize}
    \item les chrétiens sont des brebis et non des compagnons, ou est leur liberté ?
    \item les chrétiens attaquent ceux qui brisent l'ordre établi.
\end{itemize}
Cette critique \textit{touche} car la religion a trop souvent été du côté des puissants et \textit{du parti de l'ordre}.
Qui voudrait d'un tel sauveur, qui préfère l'esclavage à la liberté ? 


\paragraph{Mais de quel Jésus parle Nietzsche} De la même façon que les zoroastres perses ne reconnaitraient pas la figure de Zoroastre présentée par Nietzche, est ce que le Jésus de Nietzsche ressemble à celui que nous connaissons ? Un philosophe contemporain a écrit : 

\begin{singlequote}
    Nietzsche concentre le message tout entier sur l'aujourd'hui, sur la présence vivante de Jésus. Toute attente, toute peine, tout renvoi pénible, toute future apocalypse est effacé de T« aujourd'hui, tu seras comme moi au Paradis » (Le 23,43) —
\end{singlequote}


\paragraph{Le Christianisme comme Style} Deux dangers guettent toute religion. 
\begin{itemize}
    \item Le premier, c'est de mettre la "main" sur Dieu ou notre salut : \textit{je suis la loi ou la "méthode" donc je suis sauvé}. A noter que cela concerne tout homme, le juif qui suit les 613 commandements de la torah, celui de \textit{comment se faire des amis} ou celui qui fait un \textit{ultra-trail} ou suis \textit{petit bambou}. C'est la critique de Nietzsche, de chercher notre salut en abandonnant notre liberté. C'est aussi l'expérience que l'on peut faire que la loi, le processus peut aider mais qu'en particulier sur les choses de l'esprit, elle ne marche pas.
    \item Mais le risque inverse, c'est celui pour être religion, d'être insignifiante, de ne pas proposer un \textit{chemin} vers Dieu. Ou plus subtile, de proposer un chemin de Salut qui passe par la connaissance, \textit{la gnose} : je suis \textit{initié} (parce que j'ai fait une licence de théologie,...), donc je suis \textit{sauvé}. Croire que je suis meilleur que les autres.
\end{itemize}

Face à ces deux risques, le christianisme est \textit{un style } :
\begin{itemize}
    \item c'est à dire avec une partie qui nous permet de reconnaître les chrétiens : non pas des habits, ni une nourriture spéciale, mais un accueil des plus pauvres, des malades, un rassemblement le dimanche,... 
    \item mais ce style se caractérise par son ouverture sur l'imprévu : se laisser surprendre par le Christ.
\end{itemize}

\paragraph{Suivre Jésus de Nazareth} Quelle familiarité avons-nous de Jésus ? La prière, la lecture 

\begin{singlequote}
    Et pour vous, qui suis-je ?
\end{singlequote}
Demande Jésus aux apôtres à Césarée de Philippe. 

Retour à Jésus, la mort du Christ, certes mais d'abord un homme qui a vécu, 


\begin{singlequote}
    La négligence dans la charge de cultiver et de garder une relation adéquate avec le voisin, envers lequel j’ai le devoir d’attention et de protection, détruit ma relation intérieure avec moi-même, avec les autres, avec Dieu et avec la terre LS 70
\end{singlequote}

Décentrement

Mort alors se comprend

1880 : Ste Thérèse Charles de Foucault, Jésus, amour.


\begin{singlequote}
    Pape François, Exhortation apostolique post-synodale « Amoris Laetitia » (2016), n° 267 (cité désormais comme AL). Voir aussi : « Il est nécessaire de développer des habitus. De même, les habitudes acquises depuis l’enfance ont une fonction positive, en aidant à ce que les grandes valeurs intériorisées se traduisent par des comportements extérieurs sains et stables. […] Le renforcement de la volonté et la répétition d’actions déterminées construisent la conduite morale, et sans la répétition consciente, libre et valorisée de certains bons comportements, l’éducation à cette conduite n’est pas achevée. Les motivations, ou bien l’attraction que nous sentons pour une valeur déterminée, ne deviennent pas une vertu sans ces actes adéquatement motivés » (AL, 266).
\end{singlequote}