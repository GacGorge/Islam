\chapter{L’approche postlibérale de Linbeck}
La fécondité sociale du Christ I

\mn{Christologies au défi de la culture pluraliste - La fécondité sociale du Christ I}

\section{Eléments bibliographiques}

\begin{itemize}
    \item 
BAUMAN, Z., La société assiégée, Paris 2014
    \item BOSS, M. – EMERY, G. et GISEL, P. (éd.), Postlibéralisme ? La théologie de George Lindbeck
et sa réception, Genève 2004.
    \item CHÉNO, R., Dieu au pluriel. Penser les religions, Paris 2017.
    \item FREI, H. W., The Identity of Jesus Christ. The Hermeneutical Bases of Dogmatic Theology,
Philadelphie 1975.
    \item LINDBECK, G.A., « Le cadre du désaccord catholique – protestant » dans Théologie
d’aujourd’hui et de demain, Paris 1967, 191-206.
    \item LINDBECK, G. A., La nature des doctrines. Religion et théologie à l’âge du postlibéralisme,
tr. par M. HEBERT, Paris 2002.
    \item MICHENER, Ronald T., « The Community ethics of Stanley Hauerwas » dans Postliberal
Theology. A Guide for the perplexed, Londres – New York 2013, 72-77.
\end{itemize}
% ------------------------------------


\section{Introduction : le christianisme et le contexte postmoderne}

\subsection{Le contexte de la postmodernité}
\subsection{Le christianisme dans ce nouveau contexte}
 
% ------------------------------------
\section{L’émergence de la théologie post-libérale}

\subsection{L’école de Yale}
\subsection{Une lecture « littérale » de l’Ecriture (H. Frei)}
\subsection{G. Lindbeck}
\subsection{Lindbeck et la promotion d’une Église « sectaire »}
 

% ------------------------------------
\section{La théologie de G. Lindbeck}
 \subsection{L’insistance sur la dimension sociale du christianisme}
 \subsection{Les Ecritures et la doctrine façonnent la communauté}
  \subsection{La religion comme un langage ou une culture}
 
a) Quelques constats sur la transmission de la foi
b) La religion est un « langage »

 \subsection{La théologie comme « grammaire »}
 