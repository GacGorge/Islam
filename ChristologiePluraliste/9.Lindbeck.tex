\chapter{L’approche postlibérale de Linbeck}
La fécondité sociale du Christ I

\mn{Christologies au défi de la culture pluraliste - La fécondité sociale du Christ I}

\section{Eléments bibliographiques}

\begin{itemize}
    \item 
BAUMAN, Z., La société assiégée, Paris 2014
    \item BOSS, M. – EMERY, G. et GISEL, P. (éd.), Postlibéralisme ? La théologie de George Lindbeck
et sa réception, Genève 2004.
    \item CHÉNO, R., Dieu au pluriel. Penser les religions, Paris 2017.
    \item FREI, H. W., The Identity of Jesus Christ. The Hermeneutical Bases of Dogmatic Theology,
Philadelphie 1975.
    \item LINDBECK, G.A., « Le cadre du désaccord catholique – protestant » dans Théologie
d’aujourd’hui et de demain, Paris 1967, 191-206.
    \item LINDBECK, G. A., La nature des doctrines. Religion et théologie à l’âge du postlibéralisme,
tr. par M. HEBERT, Paris 2002.
    \item MICHENER, Ronald T., « The Community ethics of Stanley Hauerwas » dans Postliberal
Theology. A Guide for the perplexed, Londres – New York 2013, 72-77.
\end{itemize}
% ------------------------------------


\section{Introduction : le christianisme et le contexte postmoderne}

\subsection{Le contexte de la postmodernité}

\paragraph{Dans les chapitres précédents, interroger la figure du Christ}
Christologie au centre de cette discussion jusqu’à présent. \mn{En thélogie, Mort et résurrection d’abord
Incarnation après
Intégrer l’historique ; la foi part de la résurrection
}


Tenir bon une foi où Jesus n’est pas un simple homme mais la christologie ne doit pas nous séparer des autres (« vis cachée d’une théologie qui nous couperait des autres »)

\paragraph{Vérité}
La vérité ne s’impose pas de façon évidente car elle se manifeste dans une version humble. On a été éduqué dans le fait que la vérité était évidente et finirait à s’imposer. Il y a un deuil à faire pour faire autrement. 

\begin{quote}
    « L’époque qui débuta par la construction du mur d’Hadrien ou de la Grande Muraille de Chine et se
termina par le mur de Berlin est terminée. Dans notre espace planétaire global, on ne peut plus tracer
de frontière derrière laquelle on pourrait se sentir vraiment en sécurité » (Zygmunt Bauman, La société
assiégée, Paris 2014, 24).
\end{quote}

\begin{quote}
    « L’un des effets peut-être les plus conséquents de cette nouvelle situation est la porosité et la fragilité
endémiques de toute limite, et la futilité innée, ou du moins la nature irrémédiablement provisoire et la
révocabilité incurable, de tout tracé de limite. Toutes les limites sont ténues, fragiles et poreuses »
(Zygmunt Bauman, La société assiégée, Paris 2014, 26).
\end{quote}

\begin{quote}
    Les défis que postule l’époque moderne entrant dans sa phase ‘liquide’ sont sans doute les plus
terrifiants de tous » (Zygmunt Bauman, La société assiégée, Paris 2014, 37).
\end{quote}

\paragraph{Expérience en France un peu différente : les deux France} Depuis la révolution, tension entre la France libérale et des villes et la France catholique, conservatrice et rurale.


\subsection{Le christianisme dans ce nouveau contexte}
\paragraph{Deux injonctions difficiles} à tenir ensemble, ambivalente : 
\begin{itemize}
    \item Nécessité du dialogue avec le monde. 
    \item et de l'autre, construction du corps, une identité fraternelle, une culture et une façon d'habiter le monde
\end{itemize}


\paragraph{nécessité du dialogue}

\paragraph{une théologie qui construit un corps et une fraternité} Fraternité sans exclure.

 \paragraph{Rahner vs Balthazar} Rahner : ne pas exclure les gens (christianisme anonyme) et Balthazar, identité forte ("jusqu'au Martyr").

 
% ------------------------------------
\section{L’émergence de la théologie post-libérale}

\subsection{L’école de Yale}

\paragraph{Ecole qui réagit contre le mouvement libéral} moderne.

\subsection{Une lecture « littérale » de l’Ecriture (H. Frei)}
\paragraph{The identity of Jesus Christ 1975} retour au récit biblique\mn{Il y avait déjà eu une réaction de Barth. Il ne s'agit pas d'interroger le texte, il s'agit de laisser l'écriture nous interroger.}. On note \textit{identity}. Il réagit contre une façon de faire de la théologie en regardant l'histoire des textes mais comment cela raconte une histoire (aujourd'hui, on parlerait de \textit{narrativité}). \textit{On revient à l'écriture}. 


\paragraph{lecture littérale doit être prise au sérieux} \mn{d'une certaine façon, Bultmann proposait une lecture "allégorique" avec le présupposé d'un sens caché.}. Le texte doit être accueilli sans tenir compte du contexte : c'est le texte qui nous transforme.


\subsection{G. Lindbeck}

\paragraph{Histoire} Né en 1923 à Luyang, en Chine. Jusqu'à 17 ans. Ces parents sont pasteurs luthériens, vivent en minorité. Cela a fortement influencé sa pensée. Il va étudier à Yale et à l'EHESS (Duns Scott). Il est observateur à VII en tant qu'observateur luthérien. Il sent qu'il y a un phénomène de marginalisation des religions. Il est critique d'une église \textit{constantinienne} et souhaite retrouver une Eglise diaspora. \textit{Retrouver son caractère distinctif}
\paragraph{la nature des doctrines - 1984}
\subsection{Lindbeck et la promotion d’une Église « sectaire »}

\paragraph{hypothèse sociologique : seule une Eglise "sectaire" peut survivre}. de vrais croyants soudés qui ne soit pas \textit{sectaires} mais qui ne vivent plus dans une Eglise liée à l'Etat ("Constantinienne")\mn{Récit mythique, la loi, les rites. sont pour les sociologues ce qui fait la religion}
. 

\paragraph{Comment le penser dans une orthodoxie théologique} Dans une Eglise de diaspora, minorité, il faut une foi plus profonde pour lutter contre les acides de la modernité.  La théologie est là pour \textit{consolider} la vie communautaire. Que cela suscite la cohésion du Groupe.

% ------------------------------------
\section{La théologie de G. Lindbeck}
 \subsection{L’insistance sur la dimension sociale du christianisme}

 \paragraph{Une théologie utilisée} On voit que des communautés utilisent le texte pour faire identité.
 
 \subsection{Les Ecritures et la doctrine façonnent la communauté}

 \paragraph{dimension sociale du Christianisme} Lubac (68) faisait le même diagnostic.

 \begin{quote}
     « Le dépassement de l’individualisme de la tradition occidentale attire notre attention sur un second
domaine de rapprochement (oecuménique), ecclésial cette fois. Non seulement la Bible, mais aussi
l’histoire et la sociologie, nous ont rendu beaucoup plus conscients de la nature sociale de l’homme
(…). Les êtres humains sont inévitablement conditionnés par les diverses communautés auxquelles ils
appartiennent, et par où ils prennent place dans l’histoire (…) » (Lindbeck, Le désaccord, 202, 1960).
 \end{quote}


 \paragraph{l'Eglise nous \textit{met au monde} socialement}
 \begin{quote}
     « C’est par le moyen de ce qui a été transmis dans et par la communauté chrétienne qu’une personne
devient consciemment et activement membre du corps du Christ. L’Église (…) comme communauté,
communion, est notre mère à tous – notre mère patrie – et c’est pourquoi sa continuité vivante, sa
catholicité et, en conséquence, son unité sont d’une extrême importance » (Lindbeck, Le désaccord,
202-203).
 \end{quote}

\paragraph{liturgie}
 L'homme est un animal social mais aussi \textit{liturgique} : rites, dévotions, Comment on est façonné par ses rites, on chante ensemble : tout cela nous façonne. 

\begin{quote}
    « Toute société qui est plus qu’une collection d’individus extérieurement reliés exprime et réalise son
unité, et transmet sa vie communautaire, non principalement par l’organisation et l’instruction (…),
mais avec le maximum de vigueur par des rites communs qui célèbrent les objets suprêmes de sa
dévotion. (…) C’est précisément dans le culte, dans la célébration liturgique de ces souvenirs et de ces
espérances, quand ils impliquent la participation active de toute l’assemblée, âme et corps, que même
au plan psychologique et sociologique, la communauté existe le plus intensément et le plus
efficacement » (Lindbeck, Le désaccord, 203).
\end{quote}

\paragraph{Image d'Israël} Diaspora. Petit peuple. Histoire narrative post-critique. Lindbeck calque les chrétiens sur ce modèle. Il critique la \textit{critique moderne} qui remettent en cause les traditions. La vie croyante ne peut pas être sans communauté.

\paragraph{Lindbeck peut aider à penser les communautés minoritaires} au Maroc, ... 

\begin{quote}
    « L’Ecriture crée son propre domaine de signification et (…) la tâche de l’interprétation est d’étendre
celui-ci à l’ensemble de la réalité (…). » (Lindbeck, La nature des doctrines, 153).

\end{quote}

  Il faut revenir à une certaine \textit{naïveté} dans la lecture des textes, la Création... Mais avec un risque de créer une \textit{cathosphère}, comme sur l'île Bouchard. Est-ce que cela vient de Lindbeck ? 

  \paragraph{Créer une contre-culture : un cosmos chrétien}
\begin{quote}
  « Plutôt que de convertir les Ecritures en catégories extrascripturaires, la théologie intratextuelle
redéfinit la réalité à l’intérieur d’un cadre scripturaire. C’est pour ainsi dire le texte qui absorbe le
monde et non l’inverse » (Lindbeck, La nature, 155).
\end{quote}
 Renversement herméneutique.  Métaphore des lunettes : la bible est la façon de regarder le monde.

La référence à la Bible uniquement peut néanmoins poser question et créer un monde "irréel" et coupé du monde.


 
  \subsection{La religion comme un langage ou une culture}

\paragraph{a) Quelques constats sur la transmission de la foi}

\subparagraph{comment on transmet la foi chrétienne dans une autre culture} si la religion est une culture. 
\begin{quote}
    « L’impossibilité d’une catéchèse efficace dans la situation actuelle est en partie le résultat de la
supposition implicite que savoir quelques bribes du langage d’une religion suffit pour la connaître
(alors même qu’il ne viendrait à l’idée de personne de faire cette supposition pour le latin » (Lindbeck,
La nature des doctrines, 179).
\end{quote}

 \paragraph{b) La religion est un « langage »}

\begin{quote}
    « On peut considérer une religion comme une sorte de cadre ou de médium culturel et/ou linguistique
qui façonne l’intégralité de la vie et de la pensée » (Lindbeck, La nature, 35).
\end{quote}
D'abord apprendre le langage pour faire une expérience religieuse authentique. Le tournant linguistique (Derrida) : il faut posséder le langage pour faire une expérience.




 \subsection{La théologie comme « grammaire »}

 \paragraph{Pour bien apprendre à parler, on a besoin de la théologie} Comme la grammaire, on apprend d'abord la langue et la grammaire vient après. Les doctrines Chrétiennes sont des règles. 

 \begin{quote}
     Une religion n’est pas d’abord un déploiement de croyances concernant le vrai et le bien (même si
elle les implique éventuellement) [approche cognitivo-propositionnelle], ou un symbolisme qui serait
l’expression d’attitudes, d’émotions ou de sentiments fondamentaux [approche expérientielle-expressive].
Une religion ressemble davantage à un idiome qui rend possible la description de réalités,
la formulation de croyances et l’expérience d’attitudes, d’émotions et de sentiments intérieures.
Comme une culture ou une langue, c’est plutôt un phénomène communautaire qui modèle la
subjectivité des individus avant d’en être la manifestation » (Lindbeck, La nature, 35-36).
 \end{quote}

\paragraph{aspect régulateur de la théologie}
 \paragraph{Cela ne concerne que les chrétiens} les Musulmans auront leur propre langue. 

 \begin{quote}
     Finalement, « une religion est surtout une parole extérieure, un verbum externum, qui modèle et donne
forme au soi et à son monde » (Lindbeck, La nature, 37).
 \end{quote}

 \begin{Prop}
 Lindbeck n'aborde par l'histoire (de la grammaire). Cela donne une vision un peu statique. 
 \end{Prop}