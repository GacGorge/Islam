
\subsection{Foi, raison et université le 12 septembre 2006}
\begin{quote}
    «C'est pour moi un moment émouvant de me retrouver une fois encore à l'université et de pouvoir y tenir une fois encore une conférence. Mes pensées se retournent de même vers les belles années au cours desquelles, après une belle période à l'Institut supérieur de Freising, j'ai commencé mon activité académique comme enseignant à l'université de Bonn. C'était encore le temps - 1959 - de l'ancienne université. Pour les différentes chaires il n'y avait ni assistants ni secrétaires, mais en revanche des rencontres directes avec les étudiants et avant tout des professeurs entre eux. Dans les salles des enseignants, on se rencontrait avant et après les cours. Les contacts avec les historiens, les philosophes, les philologues, et naturellement aussi entre les deux facultés de théologie, étaient très vivants.

Chaque semestre avait lieu ce qu'on appelait un Dies academicus, au cours duquel les professeurs de toutes les facultés se présentaient devant les étudiants de l'ensemble de l'université : ainsi devenait possible une réelle expérience de l'Universitas. À travers toutes les spécialisations, qui nous laissent parfois muets les uns envers les autres, nous faisions l'expérience de former cependant un tout, et qu'en tout nous travaillions avec la même raison dans toutes ses dimensions, avec le sentiment que nous avions à assumer une responsabilité commune dans l'usage correct de la raison - voilà ce que l'on pouvait vivre.

L'université était très fière de ses deux facultés de théologie. Il était clair qu'elles aussi, dans la mesure où elles s'interrogent sur la raison de la foi, accomplissent un travail qui appartient nécessairement au tout de l'Universitas scientiarum, même si tous ne pouvaient pas partager la foi dont les théologiens s'efforcent de montrer qu'elle s'ordonne à la raison commune. Ce lien interne avec le cosmos de la raison ne fut pas dérangé le jour où l'on entendit un de nos collègues déclarer que dans notre université existait une chose remarquable : deux facultés qui s'occupent de quelque chose qui n'existe même pas - de Dieu. Qu'à l'encontre d'un scepticisme aussi radical, il demeure nécessaire et raisonnable de s'interroger sur Dieu avec la raison, cela restait indiscutable dans l'ensemble de l'université.

Tout cela m'est revenu à l'esprit lorsque récemment j'ai lu une partie du dialogue publié par le professeur Khoury (de Münster) entre l'empereur byzantin lettré Manuel II Paléologue et un savant persan dans le camp d'hiver d'Ankara en 1391, sur le christianisme et l'islam, et sur leur vérité respective. L'empereur a sans doute mis par écrit le dialogue pendant le siège de Constantinople entre 1394 et 1402. On peut comprendre ainsi que ses propres exposés soient restitués de façon bien plus explicite que les réponses du lettré persan. Le dialogue s'étend à tout le domaine de ce qui est écrit dans la Bible et dans le Coran au sujet de la foi ; il s'intéresse en particulier à l'image de Dieu et de l'homme, mais aussi au rapport nécessaire entre les «trois Lois» : Ancien Testament - Nouveau Testament - Coran. Dans mon exposé, je ne voudrais traiter que d'un seul aspect - au demeurant marginal dans la rédaction du dialogue -, un aspect en lien avec le thème foi et raison qui m'a fasciné et me sert d'introduction à mes réflexions sur ce thème.

Dans le 7e dialogue édité par le professeur Khoury (dialexis, «controverse»), l'empereur en arrive parler du thème du djihad (guerre sainte). L'empereur savait certainement que dans la sourate 2, 256, il est écrit : «Pas de contrainte en matière de foi» - c'est l'une des sourates primitives datant de l'époque où Mohammed lui-même était privé de pouvoir et se trouvait menacé.

Mais l'empereur connaissait naturellement aussi les dispositions inscrites dans le Coran - d'une époque plus tardive - au sujet de la guerre sainte. Sans s'arrêter aux particularités, comme la différence de traitement entre «gens du Livre» et «incroyants», il s'adresse à son interlocuteur d'une manière étonnamment abrupte au sujet de la question centrale du rapport entre religion et contrainte. Il déclare : «Montre-moi donc ce que Mohammed a apporté de neuf, et alors tu ne trouveras sans doute rien que de mauvais et d'inhumain, par exemple le fait qu'il a prescrit que la foi qu'il prêchait, il fallait la répandre par le glaive.»

L'empereur intervient alors pour justifier pourquoi il est absurde de répandre la foi par la contrainte. Celle-ci est en contradiction avec la nature de Dieu et la nature de l'âme. «Dieu ne prend pas plaisir au sang, et ne pas agir raisonnablement (sunlogô) est contraire à la nature de Dieu. La foi est un fruit de l'âme, non du corps. Donc, si l'on veut amener quelqu'un à la foi, on doit user de la faculté de bien parler et de penser correctement, non de la contrainte et de la menace. Pour convaincre une âme raisonnable, on n'a besoin ni de son bras, ni d'un fouet pour frapper, ni d'aucun autre moyen avec lequel menacer quelqu'un de mort.»

La principale phrase dans cette argumentation contre la conversion par contrainte s'énonce donc ainsi : ne pas agir selon la raison contredit la nature de Dieu. Le professeur Théodore Khoury commente ainsi : pour l'empereur, «un Byzantin, nourri de la philosophie grecque, ce principe est évident. Pour la doctrine musulmane, Dieu est absolument transcendant, sa volonté n'est liée par aucune de nos catégories, fût-elle celle du raisonnable.» Khoury cite à l'appui une étude du célèbre islamologue français R. Arnaldez, affirmant qu'«Ibn Hazm ira jusqu'à soutenir que Dieu n'est pas tenu par sa propre parole, et que rien ne l'oblige à nous révéler la vérité : s'Il le voulait, l'homme devrait être idolâtre (1)».

Ici s'effectue une bifurcation dans la compréhension de Dieu et dans la réalisation de la religion, qui nous interpelle directement aujourd'hui. Est-ce seulement grec de penser qu'agir contre la raison est en contradiction avec la nature de Dieu, ou est-ce une vérité de toujours et en soi ? Je pense qu'en cet endroit devient visible l'accord profond entre ce qui est grec, au meilleur sens du terme, et la foi en Dieu fondée sur la Bible.

En référence au premier verset de la Genèse, Jean a ouvert le prologue de son Évangile avec la parole : «Au commencement était le Logos.» C'est exactement le terme qu'emploie l'empereur : Dieu agit avec logos. Logos désigne à la fois la raison et la Parole - une raison qui est créatrice et peut se donner en participation, mais précisément comme raison. Jean nous a ainsi fait don de la parole ultime du concept biblique de Dieu, parole dans laquelle aboutissent tous les chemins, souvent difficiles et tortueux, de la foi biblique, et où ils trouvent leur synthèse. Au commencement était le Logos, et le Logos est Dieu, nous dit l'évangéliste. La rencontre du message biblique et de la pensée grecque n'est pas un hasard. La vision de saint Paul à qui se fermèrent les chemins vers l'Asie et qui vit en songe au cours de la nuit un Macédonien et l'entendit l'appeler : «Viens à notre aide» (Actes 16, 6-10), cette vision peut être interprétée comme un condensé de la nécessaire rencontre interne entre foi biblique et questions grecques.

Cette rencontre était depuis longtemps en marche. Déjà le nom de Dieu très mystérieux émanant du buisson ardent, qui sépare ce Dieu de tous les dieux aux noms multiples et le nomme simplement l'Être, est une contestation du mythe, qui n'est pas sans analogie interne avec la tentative de Socrate de dépasser et de surmonter le mythe. Le processus commencé au buisson ardent parvient à une nouvelle maturité à l'intérieur de l'Ancien Testament durant l'Exil, où le Dieu d'Israël, alors privé de pays et de culte, se proclame comme le Dieu du ciel et de la terre et se présente avec une simple formule, dans la continuation de la parole du buisson ardent «Je le suis». Avec cette nouvelle confession de Dieu s'opère de proche en proche une clarification qui s'exprime efficacement dans le mépris des idoles, lesquelles ne sont que des ouvrages fabriqués par les hommes (cf. Ps 115).

C'est ainsi que la foi biblique à l'époque helléniste, s'étant opposée avec une extrême vigueur aux autorités hellénistes qui voulaient faire adopter par la contrainte les manières de vivre des Grecs et le culte de leurs divinités, alla de l'intérieur à la rencontre de la pensée grecque en ce qu'elle avait de meilleur pour un apaisement réciproque, telle qu'elle s'est en particulier réalisée plus tard dans la littérature sapientielle. Aujourd'hui, nous savons que la traduction de l'Ancien Testament de l'hébreu en grec réalisée à Alexandrie - la Septante - est plus qu'une simple traduction du texte hébreu (appréciée peut-être de façon pas très positive) ; à vrai dire, il s'agit d'un témoin textuel indépendant et d'un pas spécifique important de l'histoire de la Révélation, par lequel s'est réalisée cette rencontre d'une manière qui acquit une signification décisive pour la naissance et l'expansion du christianisme. En profondeur, il y va, dans la rencontre entre foi et raison, des lumières et de la religion authentiques. À partir de l'essence de la foi chrétienne et en même temps à partir de l'essence de l'hellénisme, qui s'était fondu avec la foi, Manuel II a pu effectivement déclarer : ne pas agir «avec le Logos» est en contradiction avec la nature de Dieu.

La probité exige qu'on doive considérer ici que, au cours du Moyen Âge tardif, se sont développées en théologie des tendances qui ont fait éclater cette synthèse entre le grec et le chrétien. Contre le soi-disant intellectualisme augustinien et thomiste commence, avec Duns Scot, une position du volontarisme qui conduisit finalement à dire que nous ne connaissons de Dieu que sa voluntas ordinata. Au-delà, il y a la liberté de Dieu, en vertu de laquelle il aurait également pu faire le contraire de tout ce qu'il a fait. Ici se dessinent des positions qui peuvent être rapprochées totalement de celles d'Ibn Hazm et qui peuvent tendre vers l'image d'un Dieu arbitraire, qui n'est pas tenu par la vérité et le bien. La transcendance et l'altérité de Dieu sont placées si haut que notre raison, notre sens du vrai et du bien ne sont plus de réels miroirs de Dieu, dont les possibilités mystérieuses, derrière ses décisions effectives, nous restent éternellement inaccessibles et cachées.

À l'encontre de cette position, la foi chrétienne a toujours affirmé fermement qu'entre Dieu et nous, entre son esprit créateur éternel et notre raison créée, il existe une réelle analogie, dans laquelle les dissimilitudes sont infiniment plus grandes que les similitudes, mais cela ne supprime pas l'analogie et son langage (cf. concile Latran IV). Dieu ne devient pas plus divin si nous l'éloignons dans un volontarisme pur et incompréhensible, mais le véritable Dieu est le Dieu qui s'est manifesté dans le Logos, et qui a agi et qui agit par amour envers nous. Certes, l'amour «surpasse» la connaissance et demande en conséquence de prendre en considération plus que la simple pensée (cf. Ep 3, 19), mais il reste néanmoins amour du Dieu- Logos ; c'est pourquoi le culte de Dieu chrétien est logiké latreia - culte de Dieu en accord avec la Parole éternelle et avec notre raison (cf. Rm 12, 1).

La rencontre intime qui s'est réalisée entre la foi biblique et les interrogations de la philosophie grecque n'est pas seulement un événement concernant l'histoire des religions, mais un événement décisif pour l'histoire mondiale qui nous concerne aussi aujourd'hui. Quand on considère cette rencontre, on ne s'étonne pas que le christianisme, bien qu'il soit né et ait connu un développement important en Orient, ait finalement trouvé son véritable impact grec en Europe. Nous pouvons aussi dire, à l'inverse : cette rencontre, à laquelle s'est ensuite ajouté l'héritage de Rome, a fait l'Europe et reste au fondement de ce qu'on peut appeler à juste titre l'Europe.

Cette thèse - que l'héritage grec critiquement purifié appartient à la foi chrétienne - fait face à l'exigence d'une déshellénisation qui domine de façon croissante le débat théologique depuis le début de l'époque moderne. Si l'on y regarde de plus près, on peut observer que ce programme de déshellénisation a connu trois vagues, sans doute liées, mais pourtant différentes les unes des autres dans leur fondement et dans leurs buts.

La déshellénisation apparaît d'abord en lien avec les fondements de la Réforme du XVIe siècle. Les réformés se sont situés face à la tradition scolastique de la théologie, qui avait totalement systématisé la foi sous la détermination de la philosophie, pour ainsi dire une détermination étrangère de la foi par une pensée qui n'émane pas d'elle. La foi n'apparaissait plus comme Parole vivante et historique, mais comme domiciliée dans un système philosophique. La scriptura sola recherche, à l'inverse, la forme originaire de la foi telle qu'elle est donnée originairement dans la Parole biblique. La métaphysique apparaît comme une assertion qui provient d'ailleurs et dont il faut libérer la foi, en sorte qu'elle soit de nouveau totalement elle-même. Avec une radicalité que ne pouvaient pas prévoir les réformés, Kant a fonctionné à partir de ce programme, quand il disait qu'il a dû écarter la pensée pour faire place à la foi. En cela, il a ancré la foi exclusivement dans la raison pratique et lui a dénié l'accès à la totalité de la réalité.

La théologie libérale des XIXe et XXe siècles apporta une deuxième vague dans le programme de déshellénisation, dont Adolf von Harnack est le plus éminent représentant. Au temps de mes études comme dans les premières années de mon activité académique, ce programme était aussi fortement à l'œuvre dans la théologie catholique. La distinction que faisait Pascal entre le Dieu des philosophes et le Dieu d'Abraham, d'Isaac et de Jacob, servait de point de départ. Dans ma leçon inaugurale à Bonn en 1959, j'ai essayé de m'en expliquer.

Je ne voudrais pas reprendre tout cela à nouveau ici. Mais je voudrais du moins essayer brièvement de faire ressortir la différence entre cette nouvelle et deuxième déshellénisation et la première. Comme pensée centrale apparaît, chez Harnack, le retour à Jésus simple homme et à son simple message, antérieurs à toutes les théologisations et aussi à l'hellénisation : ce simple message représente le vrai sommet du développement religieux de l'humanité. Jésus a congédié le culte pour la morale. Il est finalement présenté comme le père d'un message moral plein d'amitié pour les hommes. L'enjeu fondamental, c'est d'accorder de nouveau le christianisme avec la raison moderne, justement en le libérant des éléments apparemment philosophiques et théologiques, comme la foi en la divinité du Christ ou au Dieu trinitaire.

Dans la mesure où elle s'aligne ainsi sur une explication historico-critique du Nouveau Testament, la théologie a de nouveau droit de cité dans le cosmos de l'université : la théologie est, pour Harnack, essentiellement historique et ainsi rigoureusement scientifique. Ce qu'elle découvre sur le chemin de la critique de Jésus est pour ainsi dire l'expression de la raison pratique et par là elle a aussi sa place dans l'ensemble universitaire. À l'arrière-plan, on perçoit l'autolimitation moderne de la raison, telle qu'elle a trouvé son expression classique dans les Critiques de Kant, mais telle aussi qu'entre-temps elle a été radicalisée encore par la pensée scientifique.

Cette conception moderne de la raison repose sur la synthèse, confirmée par le succès technique, entre le platonisme (cartésianisme) et l'empirisme, pour le dire brièvement. D'un côté, on présuppose la structure mathématique de la matière, à savoir sa rationalité interne, qui rend possible de la comprendre et de l'utiliser comme force effective : ce présupposé fondamental est pour ainsi dire l'élément platonicien de la compréhension de la nature. De l'autre côté, il y va de la fonctionnalité de la nature pour nos intérêts, sur quoi seule la possibilité de la vérification ou de la falsification par l'expérience livre la certitude. Le poids entre les deux pôles peut être placé davantage sur l'un ou sur l'autre côté. Un penseur positiviste aussi rigoureux que J. Monod s'est décrit comme un platonicien convaincu, c'est-à-dire un cartésien.

Cela entraîne pour notre question deux orientations fondamentales. Seule la forme de certitude qui se donne dans le jeu concerté des mathématiques et de l'expérience autorise à parler de scientificité. Tout ce qui prétend être science doit se soumettre à ce critère. Aussi, les sciences qui se rapportent aux réalités humaines - telles que l'histoire, la psychologie, la sociologie, la philosophie - essaient de s'adapter à ce canon de la scientificité. Il est important encore, pour nos réflexions, que la méthode en tant que telle exclue la question de Dieu et la fasse apparaître comme non scientifique ou préscientifique. Mais par là, nous nous trouvons devant un rétrécissement du rayon de la science et de la raison qui doit être mis en question.

Nous allons y revenir. Il faut d'abord constater qu'essayer de faire de ce point de vue une théologie «scientifique» ne laisse subsister du christianisme qu'un fragment misérable. Mais nous devons dire plus : l'homme lui-même en cela est diminué. Car les questions humaines spécifiques : d'où venons-nous et où allons-nous, les questions de la religion et de la morale ne peuvent pas trouver une place dans la raison communément définie par la «science» et doivent être transférées dans la subjectivité. La subjectivité décide à partir de ses expériences ce qui lui paraît supportable d'un point de vue religieux, et la «conscience» subjective devient finalement l'unique instance éthique.

Mais, de cette manière, morale et religion perdent leur capacité de formation collective et relèvent de l'arbitraire. Cette situation est dangereuse pour l'humanité : nous le constatons en voyant les pathologies de la religion et de la raison, qui doivent nécessairement se manifester là où la raison est si réduite que les questions de la religion et de la morale ne relèvent plus de son domaine. Ce qui, dans les essais éthiques, provient des règles de l'évolution ou de la psychologie et de la sociologie est tout simplement insuffisant.

Avant d'en arriver aux conséquences ultimes auxquelles je tends en tout cela, je dois brièvement signaler la troisième déshellénisation, qui a lieu actuellement. Au regard de la rencontre avec la multiplicité des cultures, on dit volontiers aujourd'hui que la synthèse avec la culture de la Grèce a été une première inculturation, réalisée dans l'Église antique, qu'on ne devrait pas imposer aux autres cultures. Ce serait leur droit de contourner cette inculturation pour revenir au simple message du Nouveau Testament, afin de l'inculturer à nouveau dans leurs espaces. Cette thèse n'est pas simplement fausse, elle est exagérée et inexacte. Car le Nouveau Testament est écrit en grec et porte en lui-même la rencontre avec l'esprit grec qui avait mûri auparavant dans la formation de l'Ancien Testament. Bien sûr, il y a des couches dans le devenir de l'Église antique qui ne doivent pas entrer dans toutes les cultures. Mais les choix fondamentaux, qui concernent le lien de la foi avec la quête de la raison humaine, appartiennent à cette foi elle-même et sont adaptés à son développement.

J'en viens à ma conclusion. L'essai d'autocritique de la raison esquissé ici à gros traits n'implique pas du tout la conception selon laquelle il faudrait revenir en deçà de l'Aufklärung et congédier les vues de la modernité. La grandeur du développement moderne de l'esprit est reconnue sans restriction : nous sommes tous reconnaissants pour les grandes possibilités qu'elle a ouvertes à l'homme et pour les progrès de l'humanité qui nous sont offerts. L'éthique de la scientificité est en outre volonté d'obéissance envers la vérité et, par suite, expression d'une attitude fondamentale qui appartient aux choix fondamentaux du christianisme.

Il s'agit non d'un retrait, ni d'une critique négative, mais d'un élargissement de notre concept et de notre usage de la raison. Car avec toute la joie que nous éprouvons à la vue des nouvelles possibilités de l'homme, nous voyons aussi les dangers qui croissent avec ces possibilités et nous devons nous demander comment en devenir maîtres. Nous le pouvons seulement si raison et foi s'unissent d'une manière nouvelle ; si nous surmontons l'autolimitation de la raison à ce qui est falsifiable dans l'expérience, et si nous ouvrons de nouveau à la raison toute sa largeur. En ce sens, la théologie appartient à l'université non seulement comme discipline relevant de l'histoire et des sciences humaines, mais comme spécifiquement théologie, comme question sur la raison de la foi et à son large dialogue avec les sciences.

Ainsi seulement nous devenons capables d'un authentique dialogue entre cultures et religions, dont nous avons impérativement besoin. Dans le monde occidental domine largement l'opinion que seules la raison positiviste et les formes de la philosophie qui en dépendent sont universelles. Mais précisément, cette exclusion du divin hors de l'universalité de la raison est perçue, par les cultures profondément religieuses du monde, comme un mépris de leurs convictions les plus intimes. Une raison qui est sourde au divin et repousse les religions dans le domaine des sous-cultures est inapte au dialogue des cultures.

En outre, comme j'ai essayé de le montrer, la raison scientifique, avec son élément platonicien, porte en elle-même une question qui tend au-delà d'elle et des possibilités de sa méthode. Elle doit tout simplement accepter comme un donné la structure rationnelle de la matière, tout comme la correspondance entre notre esprit et les structures rationnelles qui règnent dans la nature, un donné sur lequel est fondée sa méthode. Mais la question «pourquoi il en est ainsi» demeure, et doit être transmise par les sciences de la nature à d'autres niveaux et à d'autres manières de penser - à la philosophie et à la théologie.

Pour la philosophie et d'une autre manière pour la théologie, l'écoute des grandes expériences et intuitions des traditions religieuses de l'humanité, en particulier de la foi chrétienne, est une source de connaissance, contre laquelle on ne se protégerait qu'en restreignant de façon inadmissible notre capacité d'écouter et de trouver des réponses. Il me vient ici à l'esprit un mot de Socrate à Phédon. Les discours précédents ayant évoqué beaucoup d'opinions philosophiques fausses, Socrate déclare : «On comprendrait aisément que quelqu'un, devant tant de faussetés, passât le restant de sa vie à haïr et à mépriser tous les discours sur l'être.» Mais de cette manière, il perdrait la vérité de l'être et s'attirerait un très grand dommage.

L'Occident est menacé depuis longtemps par le rejet des questions fondamentales de la raison et ne peut en cela que courir un grand danger. Le courage pour l'élargissement de la raison, non la dénégation de sa grandeur - tel est le programme qu'une théologie responsable de la foi biblique doit assumer dans le débat actuel. «Ne pas agir selon la raison (selon le Logos) s'oppose à la nature de Dieu», répliqua Manuel II, depuis sa vision chrétienne de l'image de Dieu, à son interlocuteur persan. C'est dans ce grand Logos, dans cette large raison que nous invitons nos partenaires au dialogue des cultures. La trouver toujours à nouveau, telle est la grande tâche de l'université. »

(Traduit de l'allemand par Marcel Neusch)

L'empereur byzantin et le sage persan

L'épisode du XIVe siècle raconté par Benoît XVI lors de sa conférence à Ratisbonne est rapporté dans un ouvrage publié en France en 1966 : Entretiens avec un musulman - 7e controverse, de Manuel II Paléologue, présentés et traduits par Théodore Khoury (Éditions du Cerf, coll. « Sources chrétiennes » n° 115). Ce document apologétique reprend les échanges - qui eurent effectivement lieu - entre l'empereur théologien byzantin Manuel II (1350-1425) et un savant persan (un mudarris) non identifié, dans un contexte de guerre civile fragilisant l'empire chrétien d'Orient face aux Turcs en train de l'envahir : Constantinople tombera en 1453. L'ouvrage examine les thèses faisant alors débat entre christianisme et islam, surtout leur valeur respective en matière morale et, par là, doctrinale.

NEUSCH Marcel
\end{quote}