\Chapter{Du Christ à Jésus III
L’approche de Jacques Dupuis}{distinction Logos ensarkos et Logos asarkos}


\section{Eléments bibliographiques :}
\begin{itemize}
    \item CONGRÉGATION POUR LA DOCTRINE DE LA FOI, Notification sur le livre du P. JACQUES
   \item DUPUIS S.J., « Vers une théologie chrétienne du pluralisme religieux » Paris, Cerf 1997,
Cité du Vatican 2001.
   \item CONGRÉGATION POUR LA DOCTRINE DE LA FOI, Déclaration “Dominus Iesus” sur l’unicité
et l’universalité salvifique de Jésus-Christ et de l’Église, Cité du Vatican 2000.
   \item DUPUIS, J., « Le débat christologique dans le contexte du pluralisme religieux », Nouvelle
revue théologique 113 (1991) 853-863.
   \item DUPUIS, J. Vers une théologie chrétienne du pluralisme religieux, tr. par O. PARACHINI, Paris
1997.
\end{itemize}


\section{Introduction}

\begin{Prop}[Proposition de Dupuis]
Sortir de l'alternative entre inclusivisme qui englobe l'autre et pluralisme qui sépare Jésus et Christ.
Pour cela, il va développer une théologie trinitaire. 
\end{Prop}



\paragraph{Jacques Dupuis} 1923-2004 Jésuite belge. Origène\sn{Théologie large d'Origène}. Inde 1959-84. Puis grégorienne.

\begin{quote}
    « Je pourrais vous rapporter le cas d’un théologien qui a vécu de longues années en Inde, il se signalait
par son attitude occidentale et conservatrice, de sorte que ses compagnons indiens ne l’appréciaient
guère. Or voici que, dès son retour à Rome, il est apparu comme un symbole de progressisme et
d’ouverture. Cela parce que, ayant vécu dans un autre contexte, il est conscient des limites de sa
tradition, et est capable dorénavant d’être un pont entre les deux cultures » (R. Panikkar, Entre Dieu et
le cosmos, 168).
\end{quote}

\begin{Prop}
    Les voyages forment la jeunesse. on ne peut plus penser comme avant en ayant rencontré l'autre.
\end{Prop}

\section{Le fondement trinitaire d’une théologie chrétienne du pluralisme religieux}


\paragraph{une réflexion théologique} 


\subsection{Le rapport entre Dieu et le Christ}

\paragraph{Christocentrisme et théocentrisme} Rappel que John Hicks et xx promeuvent le théocentrisme. Mais pour Dupuis, il ne faut pas opposer Christocentrisme et théocentrisme. Sans Jésus, il n'y a pas de père, mais sans le père, il n'y a pas de Christ.

\paragraph{tenir compte du rapport entre Jésus et Dieu mais dans une théologie trinitaire}

\begin{quote}
    « Dieu, et Dieu seul, est le mystère absolu et, comme tel, il est à la source, au coeur et au centre de
toute la réalité. S’il est vrai que l’homme Jésus est de manière unique le Fils de Dieu, il est également
vrai que Dieu est au-delà de Jésus. Lorsque l’on dit qu’il est au centre du mystère chrétien, cela ne doit
pas être compris en un sens absolu, mais dans l’ordre de l’économie des rapports librement entretenus
par Dieu avec l’humanité dans l’histoire. » (J. Dupuis, Vers une théologie chrétienne, 313).
\end{quote}

\subsection{Le rapport entre le Fils et l’Esprit}

\paragraph{lien intrinsèque entre don de l'esprit et don du Christ} Si on ne peut séparer deux \textit{économies}, deux \textit{moyens}, toutefois la distinction hypostatique sert de clé herméneutique
\begin{quote}
    Un exposé théologique en équilibré de la relation entre christologie et pneumatologie doit allier
divers éléments : d’une part, les rôles respectifs du Fils et de l’Esprit ne doivent pas être confondus
mais rester distincts, comme est distincte l’identité hypostatique de chacun d’eux. D’autre part il existe
entre eux un ‘rapport d’ordre’ qui, sans impliquer aucune subordination de l’un envers l’autre, traduit
dans l’économie divine l’ordre des relations éternelles d’origine hypostatique dans le mystère
intrinsèque de la Divinité » (J. Dupuis, Vers une théologie chrétienne, 315).
\end{quote}

Distinction de Calcédoine. Rapport d'ordre (taxitrinitaire : le père engendre le fils et le fils spire l'esprit (en théologie latine). 


le christ, source de la diversité, source de l'unité. Au modèle théocentrique, différentes voies qui convergent vers un dieu unique. 
\begin{itemize}
    \item un Dieu
    \item un Christ
    \item des chemins religieux
\end{itemize}

\begin{quote}
    « ‘Un Dieu, un Christ, des voies convergentes’ évoque à la fois le caractère fondateur de l’événement-
Christ comme garantie des multiples modes\sn{influence indienne, Avatar} d’automanifestation, d’auto-révélation et de don de soi
divins au genre humain, en une économie de salut aux nombreux aspects (…) à travers laquelle les
diverses voies tendent vers une convergence mutuelle dans le mystère divin absolu qui constitue leur
terme ultime, commun à toutes. » (J. Dupuis, Vers une théologie chrétienne, 319).
\end{quote}

\section{L’agir trinitaire dans l’histoire de l’humanité}
\paragraph{Histoire du salut} on pense naturellement à l'histoire du peuple juif, Daniel-Rops. Mais ici, il faut penser l'histoire du genre humain : Dieu veut le salut du monde entier. 

\begin{quote}
« Affirmer que le genre humain a été à la fois, et depuis le commencement, créé et appelé par Dieu à
partager la vie divine, fait partie de la tradition chrétienne ». (J. Dupuis, Vers une théologie chrétienne,
331).
« Il découle de l’universalité de cette condition humaine concrète qu’il n’y a qu’une seule histoire du
salut, de la révélation et de l’offre de foi, qui coexiste avec l’histoire du monde. » (J. Dupuis, Vers une
théologie chrétienne, 332).
\end{quote}

\paragraph{Rahner} histoire universelle transcendantale (un peu invisible) de l'histoire universelle catégoriale (Abraham,...). D'une certaine façon les autres traditions religieuses manifestent l'histoire universelle transcendantale.  L'Esprit Saint travaille déjà ailleurs.




\subsection{L’histoire du salut est coextensive à toute l’histoire de l’humanité}
 


\paragraph{déclaration d'Abu Dhabi} Les divisions ne sont pas l'oeuvre de Satan mais l'oeuvre de sagesse de Dieu.



\subsection{Les religions appartiennent de droit à l’histoire du salut}

\paragraph{Judaisme, pas slt propédeutique} On passe des préfigurations à l'accomplissement ? Israel est il uniquement une \textit{préparation évangélique} ? 
 \begin{quote}
     « La relation entre les rapports de Dieu avec les ‘gentils’ au long de l’histoire d’une part, et son
automanifestation dans l’histoire biblique de l’autre, est-elle celle d’une simple substitution de l’ombre
à la réalité ? Ou au contraire, s’agit-il dans le plan divin d’une interaction entre des éléments distincts
qui, s’ils ne représentent pas la réalité de la même manière, sont néanmoins inséparablement unis ? »
(J. Dupuis, Vers une théologie chrétienne, 322).
 \end{quote}
On ne peut apporter une réponse définitive et se mettre à la place de Dieu.

\paragraph{un sens permanent de cette pluralité des traditions religieuses dans le monde} critique des théologies de l'accomplissement (L'Eglise accomplit (purifie et dépasse) les autres religions). Déplacement d'accent ici. 

\begin{Synthesis}
les traditions manifestent le salut de Dieu transcendantal. 
\end{Synthesis}


\subsection{L’agir trinitaire et l’histoire du salut}

\paragraph{Garantie de l'unité de cet agir} Unité et multiplicité. Activité en même temps du \textit{logos} et de \textit{l'esprit} de Dieu. Medium de Dieu présent dans notre histoire. \sn{Irénée : Dieu nous façonne avec ses deux mains, logos et esprit}


\begin{quote}
    « L’active présence universelle du Logos divin avant l’événement-Christ est clairement affirmée par le
Prologue de l’évangile selon saint Jean ; il était ‘la vraie lumière qui, en venant dans le monde,
illumine tout homme’ (Jn 1,9). Cette affirmation johannique a été scrutée et développée par saint
Irénée pour qui non seulement les théophanies\sn{Chêne de Mambré, buisson ardent} de l’AT, mais toutes les manifestations divines dans
l’histoire du salut, à partir de la création, étaient des \textit{logophanies} La fonction révélatrice universelle du
Logos le rendait présent au genre humain tout au long de l’histoire agissante ne dût culminer qu’en
son avènement dans la chair en Jésus-Christ ». (J. Dupuis, Vers une théologie chrétienne, 337).
\end{quote}


\paragraph{Jean-Paul II et l'ES} Redemptoris Missio, 28 : 

\begin{quote}
    L'Esprit se manifeste d'une manière particulière dans l'Eglise et dans ses membres; \textit{cependant}\sn{souligner ce type d'adverbe car introduit souvent une nouveauté} sa
présence et son action sont universelles, sans limites d'espace ou de temps. Le Concile Vatican II
rappelle l'oeuvre de l'Esprit dans le \textbf{coeur de tout homme}, par les « semences du Verbe », dans les
\textbf{actions même religieuses}\sn{pour un hindouiste, quand il se tourne vers Dieu ?}, dans les efforts de l'activité humaine qui tendent vers la vérité, vers le bien,
vers Dieu. » (Jean-Paul II, Redemptoris Missio, 28).
\end{quote}
 On ne sait pas toujours l'importance de JP2 dans l'importance de l'Esprit Saint.


 \begin{quote}
     « Il n'est pas possible de se limiter aux deux mille ans écoulés depuis la naissance du Christ. Il faut
remonter en arrière, embrasser aussi toute l'action de l'Esprit Saint avant le Christ - depuis le
commencement - dans le monde entier et spécialement dans l'économie de l'Ancienne Alliance. Cette
action, en effet, en tout lieu et en tout temps, même en tout homme, s'est accomplie selon l'éternel
dessein de salut, dans lequel elle est étroitement unie au mystère de l'Incarnation et de la Rédemption;
ce mystère avait lui-même exercé son influence sur ceux qui croyaient au Christ à venir. » (Jean-Paul
II, Dominum et vivificantem, 53).
 \end{quote}
Les traditions religieuses peuvent être sous la mouvance de l'Esprit de Dieu.





\section{Les différents modes de l’autorévélation divine dans l’histoire}

\paragraph{Autorévélation Important chez Dupuis}


\subsection{Une Parole divine adressée à de nombreuses reprises}

\paragraph{Dieu parle} Reprend He et le met en lien avec St Jean. Approche rabbinique. 
\begin{quote}
    He : Dieu a parlé dans les prophètes au peuple d'Israël
\end{quote}
\begin{quote}
    Jn : le logos illumine tout homme. 
\end{quote}


 \begin{quote}
 
« [La] référence explicite à la façon dont Dieu, ‘à bien des reprises et de bien des manières (avait)
parlé autrefois’, et au Fils ‘par qui il a créé les mondes’, évoque de façon frappante l’affirmation du
Prologue de l’évangile selon Jean concernant le ‘Verbe’, par qui ‘tout fut’ (Jn 1,3) et qui ‘était la vraie
lumière qui, en venant dans le monde, illumine tout homme’ » (Jn 1,9). La similitude entre les deux
textes nous mène au-delà de la référence explicite, dans He, à la parole adressée par Dieu à Israël et
nous encourage à nous interroger au sujet d’une révélation divine qui ne serait pas limitée à l’histoire
biblique, mais s’étendrait à toute l’histoire du salut. » (Dupuis, Vers une théologie chrétienne, 357).
 \end{quote}







 
\subsection{Dieu se révèle trinitairement depuis le commencement}


\begin{quote}
    Si la révélation divine atteint sa plénitude qualitative en Jésus, c’est qu’aucune révélation du mystère
de Dieu ne peut égaler en profondeur ce qui est advenu quand le Fils divin incarné vécut en clé
humaine, dans une conscience humaine, sa propre identité de Fils de Dieu (…) Cette révélation n’est
pourtant pas absolue ; elle reste relative. La conscience humaine de Jésus, tout en étant celle du Fils,
reste une conscience humaine, et par conséquent limitée. (…) Aucune conscience humaine, pas même
la conscience humaine du Fils de Dieu, ne peut épuiser le mystère divin. D’autre part, c’est
précisément cette expérience humaine d’être le Fils, par rapport au Père, qui lui a permis de traduire en
paroles humaines le mystère de Dieu qu’il nous a révélé (…) Le mystère trinitaire ne pouvait être
révélé aux êtres humains que par le Fils incarné vivant en tant qu’homme son propre mystère de Fils
(…). » (J. Dupuis, Vers une théologie chrétienne, 379).
\end{quote}
 
\subsection{Le Logos incarné ne met pas fin aux autres modes de révélation divine}

 \paragraph{Jusqu'à maintenant, le logos dans l'AT, avant l'incarnation du verbe} mais on ne parlait pas de théophanie après l'incarnation. Or Dieu parle toujours par son verbe. DV : "Dieu ne cesse de parler à son épouse". Et donc pas pourquoi pas aux autres traditions. 

 \paragraph{le logos n'est pas \textit{épuisé} par l'incarnation du Christ} La plénitude de la révélation est donnée par le Christ mais elle n'est pas reçue. 

\paragraph{dimension qualitative atteinte en Jésus} Jésus du fait de sa conscience humaine limitée, n'a pas fait une révélation absolue ou quantitative : 


il laisse de l'espace aux autres.




\paragraph{continuité de la Révélation} En même temps.

 \begin{quote}
     « La plénitude qualitative – disons, l’intensité – de la révélation en Jésus-Christ n’empêche pas, même
après l’événement historique, une autorévélation divine continue par les prophètes et les sages d’autres
traditions religieuses (…). Cette autorévélation a eu lieu et continue d’avoir lieu dans l’histoire.
Toutefois, aucune révélation, ni avant, ni après le Christ, ne peut surpasser ni égaler celle qui a été
accordée en Jésus-Christ, le Fils divin incarné. » (J. Dupuis, Vers une théologie chrétienne, 379).
 \end{quote}

\paragraph{Critique de l'auto-révélation} autant le terme d'auto-révélation est légitime pour Jésus, Fils unique, autant quand c'est un prophète, Moise, il \textit{révèle} mais pas \textit{auto-révélation}.

\paragraph{Comique, Jésus, Prophète} pas à comprendre chronologiquement.
\paragraph{tout au long de l'histoire} Il va un peu plus loin que ses prédécesseurs.

% -------------------------
\section{L’action du Logos dans toute l’histoire et l’agir du Christ, le Logos incarné}
 
\subsection{L’émergence de l’idée d’une économie du Logos dans l’histoire de l’humanité}

\mn{comment distinguer l'action du logos et l'esprit}


\paragraph{a) Le témoignage du NT}
L'action du Logos et de l'Esprit (on ne sait pas trop comment) : 
\begin{quote}
    « L’action du Logos, l’oeuvre de l’Esprit et l’événement-Christ sont ainsi des aspects inséparables
d’une unique économie de salut. Le fait que selon la tradition paulinienne les êtres humains sont ‘créés
en Jésus-Christ’ (Ep 2,10), à qui appartient la primauté tant dans l’ordre de la création qu’en celui de
la re-création (Col 1,15-20), ne restreint pas, mais postule l’action anticipée du Verbe devant devenir
chair (Jn 1,9) et l’oeuvre universelle de l’Esprit » : ‘Sans l’ombre d’un doute, le Saint-Esprit était déjà
à l’oeuvre avant la glorification du Christ’, dit Vatican II (AG 4) » (J. Dupuis, Vers une théologie
chrétienne, 339).
\end{quote}

Pas très clair.

\paragraph{b) Le témoignage patristique} Saint Irénée et l'Evangile quadriforme du verbe. 

\begin{quote}
    Les mêmes traits se retrouvent aussi dans le Verbe de Dieu lui-même : aux patriarches qui existèrent
avant Moïse il \textit{parlait} selon sa divinité et sa gloire ; aux hommes qui vécurent sous la Loi \textit{il assignait}
une fonction sacerdotale et ministérielle ; ensuite pour nous, \textit{il se fit homme }; enfin, il envoya le don de
l’\textit{Esprit céleste} sur toute la terre, nous abritant ainsi sous ses propres ailes. En somme, telle se présente
l’activité du Fils de Dieu, telle aussi la forme des vivants, et telle la forme de ces vivants, tel aussi le
caractère de l’Evangile : quadruple forme des vivants, quadruple forme de l’Evangile, quadruple
forme de l’activité du Seigneur. Et c’est pourquoi quatre alliances furent données à l’humanité : la
première fut octroyée à Noé après le déluge ; la seconde le fut à Abraham sous le signe de la
circoncision ; la troisième fut le don de la Loi au temps de Moïse ; la quatrième enfin, qui renouvelle
l’homme et récapitule tout en elle, est celle qui, par l’Evangile, élève les hommes et leur fait prendre
leur envol vers le royaume céleste » (Irénée, Contre les hérésies, III, 11,8).
\end{quote}
Esprit céleste et logos sont quasiment la même chose.
Lien entre les 4 alliances et les 4 évangiles. Est ce que cela a été repris ?


\paragraph{c) Le fondement théologique de cet agir différencié : l’autocommunication divine} surtout dans les \textit{Alliance}

\begin{quote}
    « Les alliances sont entre elles comme autant de modes de l’engagement divin en faveur du genre
humain par son Logos. Ce sont des logophanies par lesquelles le Logos divin ‘répète’, pour ainsi dire,
son irruption dans l’histoire humaine par l’incarnation en Jésus-Christ. Comme telles, elles sont liées
entre elles, non pas comme l’ancien devenu obsolète à l’avènement du neuf qui se substitue à lui, mais
comme la semence qui contient déjà en promesse la plénitude de la plante qui sortira d’elle. » (J.
Dupuis, Vers une théologie chrétienne, 343).
\end{quote}
Dieu peut il s'auto- manifester plusieurs fois ? 
Lutte contre l'obsolescence de l'AT. Image de la semence, promesse. Alors que les Pères grecs avaient l'idée d'un fragment, semence, qu'il fallait être plusieurs. \mn{Attention aux images, tester leur robustesse. }

Distinction entre ombre et réalité et semence / plante. L'alliance avec Israël n'a jamais été révoqué (d'où l'Alliance). Mais ce qui est vrai pour Israel l'est-il pour les autres religions ? 


\paragraph{d) L’agir durable et différencié du Logos dans l’histoire}

\begin{quote}
    « Le Verbe ‘\textit{a habité parmi nous}’ (Jn 1,14) en Jésus-Christ ; mais la Sagesse avait auparavant pris
procession de tous les peuples et de toutes les nations, cherchant parmi eux un endroit où reposer la
tête (Si 24,6-7), et avait ‘établi [sa] demeure’ en Israël (Si 24,8-12). De même, Jésus-Christ est ‘le
chemin, la vérité et la vie’ (Jn 14,6) ; mais le Verbe qui est avant lui était, la vraie lumière qui, en
venant dans le monde, illumine tous les hommes’ (Jn 1,9). Encore, ‘en la période finale où nous
sommes’, Dieu ‘nous a parlé à nous en un Fils’ ; mais il avait parlé autrefois ‘à bien des reprises et de
bien des manières’ (He 1,1). ‘Il n’y avait pas encore d’Esprit’ parce que Jésus n’avait pas encore été
glorifié (Jn 7,39) ; mais il avait été présent dans ‘tous les êtres’ qui existent, bien avant (Sg 11,24 –
12,1). Jésus-Christ est ‘le témoin fidèle’ (Ap 1,5 ; 3,14) ; mais Dieu n’a pas ‘manqu[é] pourtant de
[…] témoigner sa bienfaisance’ en tout temps (Ac 14,17). » (J. Dupuis, Vers une théologie chrétienne,
446).
\end{quote}
Si on considère la sagesse comme étant le verbe, alors le verbe a à la fois habité parmi nous et touché le monde.

\paragraph{comment on met ensemble des affirmations théologiques ou bibliques} Comment on pense le "en même temps". 


\subsection{Le lien entre l’action du Verbe (et de l’Esprit) et l’incarnation du Verbe en Jésus-Christ}
\paragraph{a) Le problème de l’interprétation théologique d’énoncés bibliques contradictoires}

\begin{quote}
    « Jésus-Christ est le point culminant de l’intervention personnelle de Dieu en faveur du genre humain
(…). Mais l’Incarnation du Logos étant réellement présente dans l’intention divine, sa réalisation dans
le temps informe l’histoire entière des rapports de Dieu avec l’humanité » (J. Dupuis, Vers une
théologie chrétienne, 339).
\end{quote}

\paragraph{b) Une certaine réciprocité entre l’action du Logos et l’agir du Christ}
 


\begin{quote}
    « La centralité de l’événement-Christ n’obscurcit pas, mais appelle, et rehausse plutôt l’universelle
présence agissante du ‘Verbe de Dieu’ et de l’ ‘Esprit de Dieu’ dans toute l’histoire du salut, et
spécifiquement dans les traditions religieuses de l’humanité » (J. Dupuis, Vers une théologie
chrétienne, 336).
\end{quote}

Il faudrait donner des exemples. 

\paragraph{c) La révélation définitive en Jésus Christ aurait-elle besoin des autres révélations ?}

 \begin{quote}
     « A la question : la Parole de Dieu contenue dans les autres traditions religieuses a-t-elle valeur de
‘Parole de Dieu’ uniquement pour les membres de ces traditions ; ou peut-on penser que Dieu nous
parle à nous chrétiens également par l’intermédiaire des prophètes et des sages dont l’expérience
religieuse est la source des livres sacrés de ces traditions ?, une réponse simple peut être donnée : la
plénitude de la révélation contenue en Jésus-Christ ne dément pas pareille possibilité. » (J. Dupuis,
Vers une théologie chrétienne, 383-384).
 \end{quote}
La
plénitude de la révélation contenue en Jésus-Christ ne dément pas pareille possibilité. Suggère que les autres traditions permettent nous aider à mieux percevoir la révélation du Christ ? mais pas forcément évident.

\section{L’unité et la distinction entre le \textit{Logos ensarkos} et le \textit{Logos asarkos}}
\begin{Def}[logos asarkos]
Logos asarkos : logos a-carné

Logos ensarkos : logos incarné
\end{Def}
Analogie avec les sacrements. dieu s'est engagé dans les sacrements mais n'est pas engagé par les Sacrements (S. Thomas). Le baptême sacramentel n'epuise pas le baptême (GS 22,5) par des biais comme ignorent. 


\begin{quote}
    Le pouvoir sauveur de Dieu n’est pas exclusivement tenu par le signe universel que Dieu a conçu
pour son action salutaire. » (J. Dupuis, Vers une théologie chrétienne, 452-453).
 
\end{quote}


\subsection{L’explication « sacramentelle » de Jésus-Christ}
 

\subsection{L’interprétation « sacramentelle » du Logos}
 Dupuis applique l'idée des sacrements au Christ pour illustrer sa théorie.
\begin{quote}
    « Tandis que l’action humaine du Logos ensarkos est le sacrement universel de l’action salutaire de
Dieu, elle n’épuise pas l’action du Logos. Une action distincte du Logos asarkos continue d’exister.
Non pas certes en tant que constituant une économie du salut séparée, parallèle à celle qui est réalisée
dans la chair du Christ, mais comme l’expression du don gratuit surabondant et de l’absolue liberté de
Dieu. » (J. Dupuis, Vers une théologie chrétienne, 453).
\end{quote}


C'est intéressant car cela nous fait réfléchir.

\begin{Synthesis}
    
\end{Synthesis}

\section{Les critiques adressées à Dupuis}



\subsection{La mise en garde du Magistère romain}

\begin{quote}
    « La Congrégation pour la Doctrine de la Foi, après avoir accompli la procédure ordinaire de l’examen
dans toutes ses phases, a décidé de rédiger une Notification dans le but de sauvegarder la doctrine de la
foi catholique d’erreurs, d’ambiguïtés ou d’interprétations dangereuses. Cette Notification, approuvée
par le Saint Père durant l’audience du 24 novembre 2000, a été présentée au Père Jacques Dupuis et
acceptée par lui. En signant ce texte, l’Auteur s’est engagé à reconnaître les thèses énoncées et à s’en
tenir à l’avenir, dans ses activités théologiques et ses publications aux contenus doctrinaux indiqués
dans la Notification, dont le texte devra apparaître aussi dans les éventuelles réimpressions ou
rééditions du livre en question ainsi que dans ses traductions. » (CDF, Notification)
\end{quote}
\begin{quote}
« Il est […] contraire à la foi catholique non seulement d’affirmer une séparation entre le Verbe et
Jésus ou une séparation entre l’action salvifique du Verbe et celle de Jésus, mais aussi de soutenir la
thèse d’une action salvifique du Verbe comme tel, dans sa divinité, indépendamment de l’humanité du
Verbe incarné. » (CDF, Notification)
\end{quote}

\begin{quote}
    « Il est conforme à la doctrine catholique d’affirmer que les grains de vérité et de bonté qui se trouvent
dans les autres religions participent d’une certaine manière aux vérités contenues par/en Jésus-Christ.
Par contre, considérer que ces éléments de vérité et de bonté, ou certains d’entre eux, ne dérivent pas
ultimement de la médiation-source de Jésus-Christ, est une opinion erronée. » (CDF, Notification).
\end{quote}

\subsection{Les critiques sur l’argumentation de Dupuis}
\paragraph{l'analogie avec la sacramentaire peut poser question} Le Logos incarné sauve

\paragraph{Critique de Jn 1,9 }
X. Leon Dufour dit que c'est \textit{asarkos} mais de nombreux exegètes pensent que c'est le Christ ensarkos (Brown,...). Quand on parle du \textit{Logos} en Jn 1,9, on parle du \textit{logos (incarné)}. Et du coup, l'argumentation tombe. Dupuis reconnaît que cette interprétation est contestée.

\begin{quote}
    09 Le Verbe était la vraie Lumière, qui éclaire tout homme en venant dans le monde. Jn 1,9
\end{quote}




\paragraph{confusion entre le \textit{logos ensarkos} et l'Esprit} 


\begin{Synthesis}

    A essayé de tenir inclusivisme et pluralisme.
    
    Lien entre humanité et divinité du Christ. Spécificité du Christ. 
    
    Voir que l'Esprit travaille dans les autres religions mais de quelle façon. 

    Le dialogue avec les autres traditions aura une fécondité mais laquelle ?
\end{Synthesis}

