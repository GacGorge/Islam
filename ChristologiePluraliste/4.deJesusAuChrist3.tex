\Chapter{La christologie « pratique » de Theobald}{De Jésus au Christ III}
 

\section{bibliographie}

\begin{itemize}
    \item RICOEUR, P., Amour et justice, Paris 20082.
   \item RICOEUR, P., Soi-même comme un autre, Paris 1990.
   \item THEOBALD, C., « Jésus n’est pas seul. Ouvertures » dans P. GIBERT – C. THEOBALD, Le cas
Jésus Christ. Exégètes, historiens et théologiens en confrontation, Paris 2002, 381-462.
   \item THEOBALD, C., Le christianisme comme style * et **. Une manière de faire de la théologie en
postmodernité, Paris 2007. \cite{theobald_christianisme_2007} 
   \item THEOBALD, C., Selon l’Esprit de sainteté. Genèse d’une théologie systématique, Paris 2015.
   \item THEOBALD, C., « L’unique et ses témoins : jalons pour une théologie de la rencontre entre
juifs, chrétiens et musulmans », Chemins de dialogue 7 (1996), p. 183-202.
\end{itemize}

\section{Introduction}


\paragraph{Approche libérale (et pluraliste) : partent de l'expérience religieuse} alors que notre foi chrétienne a une bipolarité : à la fois l'expérience de Jésus mais \textit{aussi l'expérience historique de la Résurrection de Jésus} : action de Dieu. D'une certaine façon, Dieu répond à Jésus à travers la Résurrection.

\paragraph{On ne peut donc pas partir de l'histoire} On ne part pas ici de Jésus historique mais on part des récits évangéliques, éclairés par la Résurrection, travaillés par elle.


\paragraph{Theobald : part des textes évangéliques} il ne s'agit pas de relativiser le Christ mais que le Christ suscite le dialogue. Il ne déconstruit pas les textes. Et il intègre l'expérience pascale parce que 

\paragraph{Théobald} Jésuite, traduction des oeuvres de Rahner en français. Oeuvre originale. Deux volumes : le Christianisme comme Style. Il essaye de penser le christianisme dans le contexte actuel.


%------------------------------------------------------------
\section{le contexte et le problème}

%------------------------------------------------------------

\subsection{Vivre ensemble dans une société plurielle}

\paragraph{Nécessité de fonder le lien social non sur le religieux mais sur une instance neutre} Pluralité des religions. "Neutralité" du lien social. Comment fait on société alors que nous sommes différents et de culture différente ? Une vraie actualité. 
\begin{quote}
    L’Église « doit désormais proposer la foi au Christ au sein de démocraties pluralistes, prenant donc la mesure de la laïcité de l’Etat moderne et de sa propre position de groupe social parmi d’autres. Précisons brièvement que le pluralisme des ‘convictions axiales’, religieuses ou non, et la cohabitation de leurs organisations dans une même société supposent la forme agnostique du  ‘lien social’ » (style**, 807).
\end{quote}
\begin{Def}[Lien social]
la forme « agnostique » : il s’agit de la neutralité de l’Etat (laïcité). L’Etat est
neutre pour permettre aux différentes traditions religieuses de vivre ensemble.
Lien social : la forme « énigmatique » : l’unité n’est pas disponible, mais elle se fera à la fin
(eschatologie).
\end{Def}
 
\begin{Def}[position agnostique]
  pas lié à une tradition religieuse  
\end{Def}

Il ne faut pas partir de la dogmatique mais éthique

%------------------------------------------------------------
\subsection{Le christianisme n’est plus le ciment de la société}

\paragraph{Nous ne sommes plus en chrétienté} JP II : appelle de ses voeux la \textit{civilisation de l'amour}.

\begin{quote}
    « Une telle situation appelle, selon nous, un déplacement éthique du point de départ de la christologie (...). Les chrétiens (…) doivent montrer qu’ils reconnaissent le caractère énigmatique du lien social, condition d’un véritable pluralisme, non par concession à des pressions externes mais par intime conviction. Or, cette reconnaissance intérieure exige d’eux une véritable reprise christologique » (style**, 808).
\end{quote}


La vision chrétienne cohabite avec d'autres traditions, laic,n Islam...
\begin{Ex}
Historiquement, racine chrétienne de l'Europe.  Mais ce n'est plus la matrice pour comprendre le monde dans lequel on vit.  
\end{Ex}

\paragraph{Christ Roi} 1925- Pie XI. "christologie glorieuse". Dogmatique qui pose aujourd'hui problème.
%------------------------------------------------------------
 \subsection{Le dilemme christologique}
 

\paragraph{soit on continue à affirmer l'unicité du Christ} mais alors comment le Christ peut rassembler le monde. Facteur de division de la société.

\paragraph{Soit on admet d'autres figures religieuses } et alors on relativise le Christ.

\paragraph{Défi qui n'est pas mince} Quel chemin ?

\paragraph{Théobald : l'unicité du Christ fonde un universalisme pluriel} On peut comprendre l'unicité du Christ autrement. 


%--------------------------------------------------
\section{Réflexions sur le Fils « Unique »}
%--------------------------------------------------

Penser son unité avec Dieu et son lien avec tous les hommes. Si on n'insiste que sur l'unité avec Dieu, quelle fraternité avec les hommes ?

%------------------------------------------------------------
\subsection{Le Fils unique et ses frères}

\paragraph{la préférence entraîne la violence}

\begin{quote}
   « L’histoire de Joseph et de ses frères raconte ce qui arrive quand un père préfère un de ses fils à tous les autres : ceux-ci prennent ‘l’Unique’ en haine, ne pouvant plus lui parler amicalement (Gn 37,3s.) » (\citep[p. 821]{theobald_christianisme_2007}.) 
\end{quote}
 
\paragraph{Comment être fils unique d'un père et avoir en même temps des frères ?} il faut penser les deux
\begin{quote}
    « Comment en effet aimer sans privilégier tel ou telle ? Mais préférer quelqu’un, c’est en même temps risquer que jalousie et violence se lèvent à ses et à nos côtés et que la fraternité soit mise à rude épreuve  […] Comment peut-on être ‘fils unique’ d’un Père et avoir en même temps des frères » (Syle**, 821).
\end{quote}

\begin{Prop}
En théologie, penser ensemble ET... ET...
\end{Prop}


\begin{quote}
    « Il nous faut (…) penser ensemble l’une en fonction de l’autre, l’\textit{unicité} de Jésus de Nazareth et la \textit{relation} qu’il entretient avec les siens et, par extension, avec tout être humain. Cela ne va pas de soi dans une tradition théologique qui a fréquemment distingué, voire séparé ces deux facettes d’une seule et même réalité » (Style**, 822).
\end{quote}


\paragraph{Le long parcours de Joseph préfigure Jésus} Élimination de Joseph (puits) et réconciliation. 
\begin{quote}
    « On ne peut être unique que ‘pour’ quelqu’un : Joseph est d’entrée de jeu unique pour Jacob et destiné à le devenir pour ses frères, eux devant entrer à leur tour dans l’expérience de la fraternité ; ce qui nécessite un long parcours qui prend l’allures d’un drame. Jésus est le Fils unique du Père et appelé à devenir l’Unique non seulement pour ses disciples qui reçoivent le nom de ‘frères’ ou d’ ‘amis’, mais encore pour une multitude » (Style**, 822).
\end{quote}


\begin{Synthesis}
Théobald ne part pas du dogme mais de l'éthique / Texte. Interroge les Ecritures et interroge le devenir historique de Jésus.
\end{Synthesis}


%------------------------------------------------------------
\subsection{L’ambivalence de l’unique dans le NT}

Ici, Théobald reprend la distinction de Stanislas Breton\sn{professeur ICP} : \textit{unicité de singularité} (chaque personne est unique) et \textit{unicité d'excellence}(attribué à Jésus dans l'Espace et le Temps) et en ce sens, touche tous les hommes.
\begin{Def}[Unicité de singularité]
 L’unicité de singularité renvoie au caractère unique de toute
personne, à sa singularité : « tout être humain est ‘unique’, au sens où toute rencontre d’autrui
doit surmonter le réflexe de comparaison et aboutir au respect de ce qu’il a d’unique »
\end{Def}
\begin{Def}[    Unicité d’excellence]
  « Le croyant ne doit-il pas accorder à Jésus, outre l’unicité de
singularité qu’il partage avec tout être humain, une ‘excellence’ dans l’espace et le temps ? »
\end{Def}

 




\paragraph{une unicité d'excellence en relation avec les autres} Jésus rentre en relation avec les autres. Théobald part de S. Jean. Tout d'abord le fils unique : 
\begin{Def}[Unicité du Christ]
    cette expression vise à souligner que Jésus est l’unique sauveur de tous
les hommes car il est le « Fils unique » de Dieu. Theobald, néanmoins, développe l’idée que
l’unicité du Christ passe par la reconnaissance de cette unicité par les hommes. L’unicité n’est
pas seulement une qualité métaphysique, mais elle implique une relation réciproque.
\end{Def}
 
\textit{monos} signifie en grec unique mais aussi \textit{seul}. 

\paragraph{Fécondité de l'unique} 
\begin{quote}
    Jn 12,24 : « Le grain de blé tombé en terre, s’il ne meurt pas il reste tout seul (\textit{monos}), vous dis-je, mais s’il meurt il donne beaucoup d’épis ». 
\end{quote}

\begin{quote}
    « La solution de l’ambivalence fondamentale, attachée à toute unicité – lieu où se loge l’ultime tentation de tout homme - , est donc le don de soi, mort du grain pour porter du fruit (…). L’unique n’est donc pas vraiment l’unique, au sens d’une unicité d’excellence, que s’il donne sa vie pour une multitude ; s’il donne sa vie pour que chacun puisse accéder à sa propre unicité » (Le cas, 454).
\end{quote}
Idée que l'unicité de Jésus est une unicité relationnelle : il donne sa vie pour que les autres aient aussi la vie. 

On voit ici comment Théobald relit et reinterprête les écritures avec un regard neuf. 

\begin{Prop}
    Quand on pratique la générosité du samaritain, on participe à l'unicité du Christ. 
\end{Prop}


%------------------------------------------------------------
\subsection{La sainteté et unicité}

Théobald appelle cette singularité, \textit{sainteté}.




\paragraph{Figure du Samaritain et de la démesure qui va jusqu'à aimer sur la croix}
\begin{quote}
    « Celui qui met en jeu son existence au profit du blessé, le Samaritain, devient unique, non seulement pour l’homme rencontré par hasard sur le chemin, mais aussi à ses propres yeux : la démesure de son geste qui défie toute obligation légale s’est avérée à sa mesure, mesure incomparable à celle du voisin » (style**, 827).
\end{quote}

Figure de Jésus. Il y a quelque chose d'excessif. 
Cela est marqué au moment du refus et du rejet qu'il revele qu'il est fils unique : \textit{aimez vos ennemis}. Et cela culmine sur la mort sur la croix. 


%------------------------------------------------------------
\subsection{L’unicité de Jésus et sa manière unique de communiquer la sainteté}


 Ce n'est pas la sainteté qui fait la différence car tout homme est appelé à la \textit{vision béatifique.} Ce qui est unique, c'est qu'il communique de façon unique et définitive de la Sainteté. 
 La sainteté est communiquée \textit{une fois pour toute}\sn{Hebreux}

\begin{quote}
    « Ce n’est donc pas la grâce qui fait la différence entre l’Unique et ses frères humains, ni la sainteté (...). La différence entre Lui et nous consiste seulement dans le fait que c’est Lui, le Fils unique, qui nous communique la sainteté et que c’est en Lui que l’unique promesse de Dieu de nous donner ce qu’il est en Lui-même est devenue dans  notre histoire réalité ‘irrévocable’ » (Style**, 829).
\end{quote}

Cf la distinction que faisaient les Pères de l'Eglise de l'Union hypostatique (en Jésus, les personnes divine et humaine) et l'invitation de Dieu dans les Saints.


\paragraph{Irrévocable} Interprétation de la mort du Christ, sceau de la parole du Christ. 

\paragraph{Comment les approches christologiques présentes doivent être complétées}
%------------------------------------------------------------
\section{Les insuffisances des approches christologiques récentes}
%------------------------------------------------------------

Une théologie, c'est aider les chrétiens à vivre dans un certain contexte. 

\begin{itemize}
    \item Isoler Jésus de ses frères. Seconde Quête
    \item les théologies inclusivistes (Danielou, Rahner), qui insistent sur l'aspect culturel.
\end{itemize}
%------------------------------------------------------------
\subsection{L’unicité de Jésus fondée sur sa relation à Dieu}

\paragraph{Seconde Quête} Käsemann, ... Les disciples de Bultmann qui cherchent le Jésus historique : montrer que Jésus était le seul de son espèce. Montrer que le Jésus de l'histoire et le Christ de la Foi, il y a une continuité. C'était une réponse à Reimarus en montrant qu'en étant historien, il y avait une christologie implicite dans le Jésus historique. 

Techniquement, on avait des critères d'historicité \mn{ex : Jésus discutant avec les pécheurs : on ne faisait pas cela avant ni après}. "Mon Père et votre Père" : distingue bien. "on vous a dit, moi je vous dis" : tous ces motifs qui vont isoler Jésus des autres.

\paragraph{Inconciemment} ces théologies ont mis l'unicité théologique de Jésus au détriment de sa relation avec les hommes.



%------------------------------------------------------------
\subsection{L’unicité de Jésus fondée sur sa dimension eschatologique}

\paragraph{Jésus, un avec les hommes} Rejoint le projet de Théobald de montrer le Christ en Relation. On retrouver cela en G\&S 22 (\textit{Le Christ, homme nouveau})
\begin{quote}
1. En réalité, le mystère de l’homme ne s’éclaire vraiment que dans le mystère du Verbe incarné. Adam, en effet, le premier homme, était la figure de celui qui devait venir, le Christ Seigneur. Nouvel Adam, le Christ, dans la révélation même du mystère du Père et de son amour, manifeste pleinement l’homme à lui-même et lui découvre la sublimité de sa vocation. Il n’est donc pas surprenant que les vérités ci-dessus trouvent en lui leur source et atteignent en lui leur point culminant.
2. « Image du Dieu invisible » (Col 1, 15) , il est l’Homme parfait qui a restauré dans la descendance d’Adam la ressemblance divine, altérée dès le premier péché. Parce qu’en lui la nature humaine a été assumée, non absorbée , par le fait même, cette nature a été élevée en nous aussi à une dignité sans égale. Car, par son incarnation, le Fils de Dieu s’est en quelque sorte uni lui-même à tout homme. Il a travaillé avec des mains d’homme, il a pensé avec une intelligence d’homme, il a agi avec une volonté d’homme, il a aimé avec un cœur d’homme. Né de la Vierge Marie, il est vraiment devenu l’un de nous, en tout semblable à nous, hormis le péché.
3. Agneau innocent, par son sang librement répandu, il nous a mérité la vie ; et, en lui, Dieu nous a réconciliés avec lui-même et entre nous, nous arrachant à l’esclavage du diable et du péché. En sorte que chacun de nous peut dire avec l’Apôtre : le Fils de Dieu « m’a aimé et il s’est livré lui-même pour moi » (Ga 2, 20). En souffrant pour nous, il ne nous a pas simplement donné l’exemple, afin que nous marchions sur ses pas, mais il a ouvert une route nouvelle : si nous la suivons, la vie et la mort deviennent saintes et acquièrent un sens nouveau.
4. Devenu conforme à l’image du Fils, premier-né d’une multitude de frères, le chrétien reçoit « les prémices de l’Esprit » (Rm 8, 23), qui le rendent capable d’accomplir la loi nouvelle de l’amour. Par cet Esprit, « gage de l’héritage » (Ep 1, 14), c’est tout l’homme qui est intérieurement renouvelé, dans l’attente de « la rédemption du corps » (Rm 8, 23) : « Si l’Esprit de celui qui a ressuscité Jésus d’entre les morts demeure en vous, celui qui a ressuscité Jésus Christ d’entre les morts donnera aussi la vie à vos corps mortels, par son Esprit qui habite en vous (Rm 8, 11) [36]. Certes, pour un chrétien, c’est une nécessité et un devoir de combattre le mal au prix de nombreuses tribulations et de subir la mort. Mais, associé au mystère pascal, devenant conforme au Christ dans la mort, fortifié par l’espérance, il va au-devant de la résurrection.
5. Et cela ne vaut pas seulement pour ceux qui croient au Christ, mais bien pour tous les hommes de bonne volonté, dans le cœur desquels, invisiblement, agit la grâce. En effet, puisque le Christ est mort pour tous [39] et que la vocation dernière de l’homme est réellement unique, à savoir divine, nous devons tenir que l’Esprit Saint offre à tous, d’une façon que Dieu connaît, la possibilité d’être associé au mystère pascal.
6. Telle est la qualité et la grandeur du mystère de l’homme, ce mystère que la Révélation chrétienne fait briller aux yeux des croyants. C’est donc par le Christ et dans le Christ que s’éclaire l’énigme de la douleur et de la mort qui, hors de son Évangile, nous écrase. Le Christ est ressuscité ; par sa mort, il a vaincu la mort, et il nous a abondamment donné la vie pour que, devenus fils dans le Fils, nous clamions dans l’Esprit : Abba, Père!
\end{quote}

\paragraph{Une vision une de l'anthropologie} Cela suppose que la vision du monde des Chinois est la même que la vision des Européens. Reste tributaire de la vision universelle des lumières. Il faut parler de plusieurs \textit{universalismes} ou visions du monde : donner un but, une vision du monde. Cela ne veut pas dire que le Concile Vatican II est dépassé : il applique le Concile non pas à la lettre mais comme méthode : \textit{signe des temps } à lire dans la culture actuelle.

Tout en reconnaissant l'intérêt de ces théologies, elles ne peuvent servir de \textit{point de départ}. Partir de l'éthique, pratique, \textit{plus modeste}.

\paragraph{Théologie pratique} qui n'exclue pas la vision théologique ou anthropologique. 


%------------------------------------------------------------
\section{Commencer par une christologie pratique ou éthique}
%------------------------------------------------------------

%------------------------------------------------------------
\subsection{Une nouvelle approche de la christologie}

\paragraph{Normalement, on part de la figure divine de Jésus} Lui propose une autre approche. Ne prennent pas en compte le \textit{chemin } des disciples pour arriver à la Foi, l'unicité relationnelle de Jésus, comme les frères de Joseph. \textit{je suis le Chemin}. Intégrer la genèse de la Foi dans la Christologie même. 
\paragraph{Evangile}
Il s'appuie sur les Evangiles, qui font partie de la Révélation.
Les récits ne nous disent pas qui est Jésus mais ont une pédagogie pour nous faire entrer dans la Foi. \mn{Théobald est Jésuite et les Evangiles ont un goût, histoire}

Il s'agit de faire la même expérience que les disciples :
    
\begin{quote}
    « Le ‘Jésus des historiens’ n’a pas l’actualité et l’autorité nécessaire pour réclamer la foi, c’est-à-dire pour que soient recrées les conditions de la décision jadis appelée par l’évangéliste sur la vérité de ce qu’il appelait la vie ou le royaume » (Le cas 442). Il y a donc une « impossibilité et plus encore une illégitimité de toute communication de la foi par l’histoire » (le Cas 445).
\end{quote}

En lisant Renan, on ne découvre pas la Foi. L'histoire n'est pas là pour accéder à la Foi. Cela peut nous aider à un moment donné. 
\begin{quote}
     Mais ceux-là ont été écrits pour que vous croyiez que Jésus est le Christ, le Fils de Dieu, et pour qu’en croyant, vous ayez la vie en son nom.
    Jn 20,31
\end{quote}


\paragraph{Rencontre de Jésus}
les textes bibliques, pas des textes, mais la parole de Dieu qui nous féconde.
\begin{quote}
    « [La nouveauté du christianisme] ne se réduit pas à l’identité christologique du Nazaréen mais se concentre dans le type de relation qu’il entretient avec ceux qui croisent sa route » (Style*, 57). 
\end{quote}

\paragraph{configuration du récit } Le texte a un effet sur le lecteur : 
\begin{Def}[Configuration]
« Dynamisme intégrateur qui tire une histoire une et complète d’un divers
d’incidents, autant dire transforme ce divers en une histoire une et complète » (Ricoeur,
Temps et Récit II, p. 18). Ici il s’agit de la mise en intrigue, laquelle appartient à la
configuration.
\end{Def}

\begin{Def}[Refiguration]
 
    « J’appelle refiguration l’effet de découverte et de transformation exercé par
le discours sur son auditeur ou son lecteur dans le processus de réception du texte » (Ricoeur,
Amour et justice, 50).
 
\end{Def}
\begin{quote}
    « Comment le soi se comprend-il en se contemplant dans le miroir que lui tend le livre ? » (Ricoeur, Amour et justice, 51). 
\end{quote}
En méditant les Evangiles, on rencontre le Christ. \mn{Arrière fond de Théobald} 



%------------------------------------------------------------
\subsection{Jésus a « impressionné » ses disciples}

\paragraph{Rencontre du Christ : impact, impressionne} La Théologie libérale a insisté sur cette partie. Jésus les a impressionné fortement car il vivait intensément cette relation \mn{Schleiermacher p. XX}

\paragraph{salut lié à la christologie} c'est parce que Jésus est en lien avec tous les hommes qu'il sauve. Et je suis poussé à agir comme lui. Théobald reprend Schleiermacher mais reprend l'Evangile.

%------------------------------------------------------------
\subsection{Le style de vie messianique de Jésus}


\begin{quote}
    Comment caractériser « l’accès à la foi au Christ, compte tenu du point de départ éthique de la christologie, appelé par nos sociétés néo-libérales toujours tentées d’oublier les menaces qui continuent à peser sur le lien social ? On peut, dans la perspective d’une christologie pratique, le définir comme ‘mue d’identité’ qui s’exprime par le passage à un\textit{ style de vie messianique}, à une manière spécifique de se situer dans la société globale » (style**, 813).
\end{quote}

\begin{Def}[Style de vie messianique]
Ce style de vie consiste à vivre l’hospitalité qui suscite la mue
d’identité des personnes rencontrées (par Jésus). « La foi chrétienne n’est pas une doctrine
(…) mais un ‘style de vie’ ou une manière de vivre de la sainteté même de Dieu : seule
l’expérience effective de l’Esprit de sainteté nous permet de confesser et de comprendre un
jour l’indépassable excellence du Fils unique du Père » (L’unique et ses témoins, 199).
\end{Def}

\begin{Def}[Mue d’identité]
Le fait que l’autre puisse se découvrir et accéder à son identité singulière.

\end{Def}

\begin{quote}
    « Les évangiles sont en fait des récits de conversion qui ne mettent pas seulement en scène l’itinéraire de Jésus mais aussi et surtout ce qu’il devient en et pour ceux et celles dont l’itinéraire croise le sien : l’accès au Christ se vit comme véritable \textit{mue d’identité} » (style**, 805). 
\end{quote}



\begin{quote}
    Elle engendre l’autre à son identité propre (conversion ?). L’identité d’une personne coïncide
avec l’émergence de la foi (« Ma fille, ta foi t’a sauvée »Mc 5,34), l’autre vit à partir de cette
foi (foi en Dieu), il devient lui-même en se décentrant. Cette conversion consiste à « suivre
Jésus », c’est-à-dire à adopter chacun à sa manière le style de vie messianique\sn{Le regard de Jésus n’est pas évident. « L’adopter effectivement est de l’ordre d’une véritable conversion ou
nécessite une inversion. L’oeuvre messianique consiste précisément dans la victoire sur cet aveuglement » (Style
73). L’incompréhension des disciples montrent bien la difficulté d’adopter l’hospitalité de Jésus (posture
d’apprentissage et dessaisissement de soi). On doit « pouvoir atteindre effectivement l’absence de mensonge ou
la concordance absolue entre pensées, paroles et actes, entre la ‘forme de vie’ de quiconque et son ‘fond’,
concordance, qui, par principe, est chaque fois unique et incomparable » (style 75).}. Jésus devient
l’unique à leurs yeux : c’est-à-dire celui qui leur a communiqué la sainteté. 
\end{quote}

\paragraph{6 étapes}
\begin{itemize}
    \item Apprentissage de Jésus (He 5, 8) dans ses rencontres, il vit la sainteté
    \begin{Def}[Sainteté]
 Capacité d’apprentissage ou dessaisissement de soi au profit d’une présence à
quiconque, ici et maintenant (voir Mc 8,35).
\end{Def}
\item Jésus ne s'impose pas mais permet à l'autre d'accéder à lui -même. Il va libérer l'autre de ses craintes et se dire. 
\item l'émergence de la Foi : quand on accède à son identité, elle commence à être sauvé. Foi en Dieu. cf la Femme adultère, \textit{Ta Foi t'a sauvé}. de l'hospitalité de Jésus arrive la Foi. Il prend des exemples dans l'Evangile : elle libère les personnes rencontrées et les ouvre à une vie nouvelle
\item refiguration, mue d'identité : conversion des personnes qui croisent Jésus, ces gens qui ont changé de vie en rencontrant Jésus. Chacun à adopter le style de vie messianique. Nous avons à devenir hospitalier et leur permettre d'advenir à leur identité.
\item les personnes qui accèdent à cette vie nouvelle \sn{le possédé à Gerasa reste et ne devient pas disciple de Jésus} ne sont pas forcément tous disciples. Mais ces rencontres ne sont pas forcément un changement de vie. Refus. Comment alors maintenir le lien social quand il y a refus de la rencontre ? 
Théobald introduit la règle d'Or.
\end{itemize}
 
\begin{Def}[Concept d’hospitalité ]
Quand l’hospitalité se produit, c’est l’accomplissement des temps
messianiques. Aussi, dans cette expérience de la rencontre avec Jésus, de l’hospitalité de
Jésus est engendrée la foi, le royaume de Dieu advient. « Les aveugles voient ». C’est
pourquoi on peut qualifier de « messianique » le style de vie hospitalier de Jésus car il libère
les gens rencontrés et les engendre à la foi pour une vie nouvelle.
\end{Def}

\begin{Def}[Engendrement]
Il s’agit du devenir « fils de votre Père ». Il a lieu quand on « réalise tout
d’un coup que l’appel démesuré à être connu comme Dieu, dans telle ou telle situation, est
toujours ‘à la mesure’ de chacun » (L’Unique et ses témoins, 200).
Intercommunication : une autre manière de dire le dialogue avec cette nuance où l’on
communique à l’autre sa sainteté.
\end{Def}
%------------------------------------------------------------
\subsection{Le style de vie de Jésus à l’épreuve du refus}

Jésus suscite l'opposition qui l'amènera à la croix. La notion de réciprocité se métamorphose en règle d'amour, de la \textit{mesure} à la \textit{démesure}. \mn{Jn 10, 20 : il déraisonne. }
\begin{quote}
    « Sans le correctif du commandement d’amour (…) la Règle d’Or serait sans cesse tirée dans le sens d’une maxime utilitaire dont la formule serait do ut des, je donne pour que tu donnes. La règle : donne parce qu’il t’a été donné, corrige le afin que de la maxime utilitaire et sauve la Règle d’Or d’une interprétation perverse toujours possible » (Ricoeur, Amour et justice, 39). 
\end{quote}


\begin{Def}[Règle d’or]
 \begin{quote}
     « Tout ce que vous désirez que les autres fassent pour vous, faites-le vous-mêmes
pour eux : voilà la Loi et les Prophètes » (Mt 7,12). 
 \end{quote} Avec Jésus, on passe de la justice
(relation de réciprocité) à l’amour (démesure : l’amour des ennemis) et de l’amour à l’amour
définitif. Jésus accomplit donc la Loi et les Prophètes en accomplissant la règle d’or : il la
réalise définitivement en aimant ses ennemis.
\end{Def}
Mt 7,12 : règle d'or
\begin{quote}
    
\end{quote}

La règle d'or existe dans d'autres cultures \mn{cf  \cite{kung_lethique_2009}}


Paul Ricoeur : sans 
\begin{quote}
    Do ut des : je donne pour que tu donnes. La règle \textit{donne parce qu'il t'a été donné} Sauve la règle d'or de l'utilitarisme (Ricoeur)
\end{quote}


%------------------------------------------------------------
\subsection{La dimension eschatologique du style de vie de Jésus}

\begin{Def}[Style de vie eschatologique]
« Sa sainteté hospitalière [va] jusque dans son ultime
dessaisissement de soi » (style*, 91). L’absence de mesure ou la démesure fait entrer
l’incomparable de Dieu dans l’histoire (voir style*, 95). « Vous serez parfaits comme votre
Père céleste est parfait » (Mt 5,48). En donnant sa vie, son style de vie est donc définitif,
eschatologique. Sa mort ne dit pas seulement l’une fois pour toutes, mais aussi qu’il
transcende le temps et l’espace en s’identifiant à toutes les victimes comme Mt 25 les
présente.
\end{Def}

 En quoi une dimension eschatologique : 


 \begin{quote}
     « Comment la manière d’être du Nazaréen – son type d’hospitalité absolument unique – a pu engendrer, non seulement la confession messianique des premiers chrétiens, mais encore leur perception du caractère définitif et ultime de ce qui est advenu dans leur rencontre avec lui » (Style 86).
 \end{quote}

C'est la mort de Jésus manifestant la perfection de sa sainteté, dessaisi de lui-même, en partant, pour laisser la place même à l'ennemi.
\begin{quote}
    domine jusqu'au coeur des ennemis
\end{quote}

 


\begin{quote}
    « Sa sainteté hospitalière [va] jusque dans son ultime dessaisissement de soi » (style*, 91). L’absence de mesure ou la démesure fait entrer l’incomparable de Dieu dans l’histoire (voir style*, 95). « Vous serez parfaits comme votre Père céleste est parfait » (Mt 5,48).
\end{quote}

\paragraph{il est lui-même le saint} il brise le refus, Temps et Espace. Il donne à celui qui ne peut pas rendre. Quand on donne à un pauvre, il ne peut pas rendre.


%------------------------------------------------------------
\section{Conclusion}


Jésus n'empêche pas le dialogue mais le favorise. 
Style de vie hospitalier, se dessaisissant pour laisser l'autre advenir à lui-même. 


\begin{Synthesis}
    Une approche éthique, pragmatique partant de Jésus. Approche originale et adaptée à notre monde.
\end{Synthesis}


 



 





 




 







