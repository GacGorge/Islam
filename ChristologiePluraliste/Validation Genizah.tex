\chapter{L'unique et ses témoins }



\section{L'unique et ses témoins }

\cite{theobald_christianisme_2007} Jalons pour une théologie de la rencontre entre juifs, chrétiens et musulmans



 
\paragraph{unicité de Dieu et le témoignage qui lui erst dû auprès des humains et à leur service : foi partagée}

Jésus : Mt 22,34-40
Jn 18,37
Ap 3,14 monothéisme trinitaire des chrétiens




Aucun doute que le titre L'Unique et ses témoins dit ce que juifs, chrétiens et musulmans mettent au centre de leur "foi" : l'unicité de Dieu et le témoignage qui lui est dû auprès des humains et à leur service. Le judaïsme se réfère au Shema Israël : "Écoute Israël ! Le Seigneur notre Dieu est le Seigneur Un" (Dt 6, 4). On peut citer aussi l'un des passages du Deutero-Isaïe où Israël est appelé à "témoigner" dans le procès entre l'unique et les dieux des nations :

"'Vous êtes mes témoins", dit Yahvé (Es 43, 10. 12 ; 44, 8). "Mon serviteur, c'est vous que j'ai choisis afin que vous puissiez comprendre, avoir foi en moi et discerner que je suis bien tel : avant moi ne fut formé aucun dieu et après moi il n'en existera pas. C'est moi, c'est moi qui suis le Seigneur, en dehors de moi pas de sauveur" (Es 43, 10.11). Jésus s'inscrit, à sa manière, dans ce procès autour de l'unicité de Dieu comme le montre sa réponse au scribe sur le grand commandement et le second qui lui est semblable (Mt 22, 34-40). Mais il faut tout de suite ajouter que le récit johannique du procès devant Pilate met dans la bouche de Jésus une déclaration sur son identité de témoin: 'Je suis né et venu dans le monde pour rendre témoignage à la vérité" (18, 37), et que l'Apocalypse lui donne même le titre de "témoin fidèle et véritable, Principe de la création de Dieu" (Ap 3, 14) ; textes qui marquent, avec bien d'autres, le point de départ du monothéisme trinitaire des chrétiens. Quant à l'islam, il est fondé sur un double "témoignage" (shahâda) à rendre publiquement, devant témoins et si possible au tribunal, en cas de conversion: \begin{quote}
    "Je témoigne que point de divinité - si ce n'est Dieu - et je témoigne que Muhammad est envoyé de Dieu".
\end{quote}  Citons, pour illustration, la très célèbre sourate 112, initialement dirigée contre le polythéisme des Mecquois et interprétée ensuite comme polémique anti-chrétienne: \begin{quote}
    "Au nom de Dieu le Clément, le bouche de Jesus une declaration sur son identité de temoin: je suis ne et nu dans le monde pour rendre témoignage à la vérité" (18, 37)
\end{quote}, et que l'Apocalypse lui donne même le titre de "témoin fidèle et véritable, Principe de la création de Dieu" (Ap 3, 14) ; textes qui marquent, avec bien d'autres, le point départ du monothéisme trinitaire des chrétiens. Quant à l'islam, il est fondé sur un double "témoignage" (shahâda) à rendre publiquement, devant moins et si possible au tribunal, en cas de conversion: \begin{quote}
    "Je témoigne que int de divinité - si ce n'est Dieu - et je témoigne que Muhammad est envoyé de Dieu"
\end{quote}. Citons, pour illustration, la très célèbre sourate 112, initialenent dirigée contre le polythéisme des Mecquois et interprétée ensuite mme polémique anti-chrétienne:
\begin{quote}
    "Au nom de Dieu le Clément, le miséricordieux. Dis : Il est dieu Un (ahad), dieu l'Impénétrable (samad). Il n'a s engendré; n'est égal à lui - personne"
\end{quote}

\paragraph{différences entre les attestations - apparition successive dans l'histoire } 
Les ressemblances entre ces trois attestations de l'Unique sont manifestes, mais aussi leurs différences. Les trois témoins sont en effet reliés entre eux par des liens subtils de "parenté", et si leur apparition progressive dans l'histoire semble  d'abord donner aux plus jeunes le privilège du "une fois pour toutes" celui ou sur ceux qui le précèdent, l'aîné ou les aînés réagissent, par contrecoup, en se pensant plus fondateurs. Peut-être ces liens inextricables entre semblables, éprouvés par chacun d'eux comme une menace plus forte que leurs différence, expliquent ils des violences si paradoxales entre "Religions" qui se réclament de la valeur fondamentale de la paix ?


\paragraph{violence entre les trois témoins \textit{liens inextricables} peut être l'origine de la violence (mimétique au sens de René Girard). } 
\begin{quote}
    il nous faudra affronter l'énigme de la violence pour voir où se trouve la véritable difficulté à communiquer entre témoins. 
\end{quote}

Nous voila au coeur du sujet. l’énigme de la violence entre les trois témoins. Nous sentons à quel point elle discrédite leur “cause" devant des contemporains qui préfèrent souvent l'agnosticisme ou le bouddhisme à l'attestation de l'unique.  La longue serie d'inimitiés entre musulmans, chrétiens et juifs,  a Jerusalem, ville de la paix (Salem) ou ailleurs, nous init chretientre modestes et audacieux en même temps quand nous parlerons de "rencontre", et même de "théologie de la rencontre Mais, avant d'affronter l'énigme de la violence, nous aurons eu, dans un premier temps, à présenter les "trois témoins", à les comparer, à nous demander si on peut les regrouper sous le même concept de "mono-théisme Quel lien y a-t-il donc entre ces trois "figures" de Dieu? Où se trouve la véri. table difficulté à communiquer entre témoins? Peut-être notre commune histoire déjà longue est-elle en mesure de susciter notre capacité d'apprentis-sage. Je voudrais aborder ce grand pari de notre époque, en réfléchissant sur les possibilités et les limites de nos rencontres. 


\paragraph{Possibilités et limites des rencontres } \textit{qu'as tu appris des deux autres et de la société sur toi-même ? } Aucun des témoins ne peut répondre à la place des deux autres.
Réflexion proprement chrétienne sur la \textit{rencontre} en méditant sur la figure de Melchisedech, \textit{roi de Gloire} (Ps 23).


À la question : qu'as-tu appris des autres et de la société sur toi-même? aucun des témoins ne peut répondre à la place des deux autres. C'est pourquoi je proposerai, pour finir, une réflexion chrétienne sur la "rencontre", en méditant sur la figure de Melchisédek, roi de Salem, roi de justice et roi de paix, d'après l'étymologie donnée dans la lettre aux Hébreux "Qui est ce roi de gloire?", se demande le psaume 23. Notre réponse théologique à cette question, qui touche en même temps à la différence fondamentale entre les trois "monothéismes", devra témoigner de ce que la rencontre des deux autres nous aura appris sur notre propre manière d'espérer la paix et sur notre capacité à gérer des situations de violence.
\subsection{Difficile communication}

\paragraph{pour éviter la violence, séparation dans le privé} Lumières sépare privé et le public.  \textit{Armistice. } 

\paragraph{la société moderne impose ses règles du jeu à toute rencontre} cf Habermas et Ratzinger à Munich 19 /01/2004.  Esprit 2004/07

\subsubsection{regards croisées dans la famille d'Abraham}

\paragraph{judaisme monotheisme éthique} ne reconnaît pas christianisme et islam comme ses héritiers (cf levinas). Identité propre :
\begin{quote}
    Aimez l'étranger, car au pays d'Egypte, vous fûtes des étrangers Dt 10,17-19
\end{quote}
imitant son Dieu par la justice faite à autrui.


pas besoin du Christianisme ou Islam mais confrontation a un impact. Indépendance mais aussi fragilité vis à vis d'eux. "mystère" du chant du serviteur.  

\paragraph{Christianisme, réalité méta-éthique du don de Soi  } \sn{besoin du Judaisme pour se comprendre "accomplissement de la Loi et des Prophètes" en JC.  Rm 11,18 réalité inouie et excecessive : l'unique Dieu est censé communiquer à la multitude la sainteté qui le constitue en lui-même, et tous peuvent désormais}
2. Le christianisme se situe entre l'aîné et le dernier. Il a d'abord besoin du judaïsme pour se comprendre lui-même parce qu'il ne peut ni raconter ni vivre « l'accomplissement de la Loi et des Prophètes » en Jésus le Christ sans se référer continuellement à la racine sur laquelle il a été greffé (Rm 11, 18) Selon la tradition chrétienne, la foi s'ouvre à une réalité inouie et excessive: l'unique Dieu est censé communiquer à la multitude la sainteté qui le constitue en lui-même, et tous peuvent désormais découvrir par la foi à quel point cette sainteté le habite déjà \sn{voir définition p. \pageref{Mt5Def1} et \pageref{Mt5Def2}. Surtout même passage plus développée \pageref{Mt5Def3}}:
\begin{quote}
    « Vous serez parfaits comme votre Père céleste est parfait » (Mt 5, 48).  
\end{quote}
C'est cela « l'accomplissement de Loi et des Prophètes», vécu dans les gestes les plus quotidien Si on se réfère au vocabulaire de l'éthique, on peut donc appel le christianisme un monothéisme méta-éthique .
\begin{Def}[monothéisme méta-éthique]
    au sens il insiste sur la communication de l'agapé divine, de l'amour surabondant de Dieu à tout être humain
\end{Def}
 
Mais comment se réclamer de cette réalité méta-éthique du don de soi qui dépasse toutes nos mesures humaines ses reconnaître d'abord l'entière autonomie ou consistance de l'ordre éthique de la justice et du respect d'autrui? La fragilité du christianisme ne vient donc pas de sa dépendance par rapport à un autre qui le précède et qui existe à ses côtés: mais elle vient de la tentation d'indépendance qui l'a amené à se substituer à Israël, à se considérer comme le « véritable Israël »; et cela en dépit de l'avertissement paulinien dans l'épître aux Romains qui prévient l'Église contre l'orgueil: les chrétiens risquent d'oublier qu'ils tiennent, grâce à la foi sur une racine qui les porte (Rm 11, 20).

\sn{785}
Peut-être cet oubli n'a-t-il pas été sans influence sur la naissance de l'islam, comme le pensent certains théologiens chrétiens!. Par rapport aux musulmans, l'Église se trouve en tout cas dans une position analogue à celle que le judaïsme occupe par rapport à elle: elle n'a pas besoin de comprendre l'islam pour se comprendre elle-même. Elle a subi l'insensibilité du nouveau venu sur la scène religieuse qu'elle-même a montrée vis-à-vis de l'aîné qui la précède et qui existe à côté d'elle; affrontement d'autant plus violent qu'il oppose deux manières de concevoir « l'accomplissement ». Il faudra attendre le xie siècle2 pour que le christianisme commence à abandonner la répartition traditionnelle de l'humanité entre juifs, païens et chrétiens qui l'avait amené à identifier la foi musulmane à un vague paganisme monothéiste.

\paragraph{Islam, monotheisme pre-éthique}
3. L'islam enfin, le dernier-né des trois, a besoin du judaisme et du christianisme, des « gens du Livre » comme il dit, pour se comprendre lui-même. C'est sur leur trace qu'il affirme le caractère ultime de sa révélation; après les quatre grands prophètes « doués de constance », Noé, Abraham, Moise et Jésus, Muhammad est le dernier, le « sceau des prophètes » (sourate 33, 40), qui met fin aux intervalles entre les époques prophétiques et fixe la communauté des fidèles dans l'attente de l'Heure dernière. Cette dépendance avouée est la force de l'islam et en même temps le lieu de sa fragilité propre.
Force d'abord, parce que sa critique des juifs et des chrétiens se met plutôt en dépendance par rapport à une alliance (mithãq) dite de « pré-éternité », qui précède toute division historique entre judaïsme, christianisme et islam. Il n'y a donc pas de progrès historique dans la révélation, mais rappel ultime et définitif de ce qui a été oublié ou déformé: l'unicité absolue de Dieu, menacée par le polythéisme et tout ce qui lui ressemble, comme l'association du Christ à Dien Ce retour en deçà de l'histoire et de ses divisions, vers l'origine adamique du « pacte » de « pré-éternité », qui d'emblée fait de tout homme un «croyant», fonde l'universalité de l'islam. Celle-ci n'est plus fondée, comme dans le prophétisme juif, sur la qualité éthique de la « relation » de l'Unique avec son témoin et avec l'étranger; \textbf{elle s'appuie sur l'idée que tout homme porte à sa naissance, sceau imprimé par Dieu en son cœur, la proclamation de foi de la pré-éternité}: cette « religion naturelle », liée à la création comme première révélation d'en deçà des temps, est une prédisposition à recevoir l'islam.
Mais cette force qui permet de contourner sans cesse les divisions historiques en les relativisant par la référence obligée à une origine immémoriale constitue en même temps la \textit{fragilité} propre de l'islam: 
\begin{Def}[monothéisme pré-éthique]
\textit{sa lutte pour l'unicité de Dieu précède toute préoccupation éthique}. 
\end{Def}
C'est en ce sens qu'on peut appeler son monothéisme pré-éthique. Il reste, de ce fait, soumis aux interrogations éthiques qui ne peuvent pas ne pas émerger de la rencontre effective entre les trois témoins.
Cette rencontre s'avère donc comme extraordinairement difficile à cause de l'asymétrie entre les trois traditions: si les deux premiers, juifs et chrétiens, se retrouvent, au moins selon la perspective chrétienne, dans une proximité privilégiée, ils ne peuvent pas pour autant évacuer l'énigmatique présence du troisième : Ismaël, fils d'Abraham (Ibrähim) et de Hâgar, ancêtre du peuple arabe, qui réclame sa « parenté» avec le premier, prototype de la foi en l'Unique.

\subsection{Rencontre et comparaison.}

\paragraph{pas de facilité à communiquer entre les 3 témoins par la modernité du fait du \textit{comparatisme} et une distance critique qui atteint chacun des trois religions}
La modernité occidentale a-t-elle facilité la communication entre les trois témoins? Ce n'est pas du tout sûr, Elle nous a appris cependant à « comparer » les figures du Dieu unique, en nous permettant ainsi de prendre une certaine distance par rapport à l'ensemble des trois traditions.
Nous savons qu'il est impossible de rencontrer l'autre sans se comparer à lui; et puisque des rencontres entre juifs chrétiens et musulmans ont existé depuis les débuts, les compe raisons entre différents interlocuteurs n'ont pas non pis manqué. Mais la communication généralisée qui caractérise as sociétés modernes transforme la comparaison en principe intellectuel. L'homme contemporain ne cesse de comparer ce qui lui est proposé, et cela jusque dans le domaine religieux. Le «comparatisme » systématique des sciences de la religion  constitue donc la face intellectuelle d'une société démocratique qui, pour éviter toute violence religieuse en son sein, n'accorde plus de privilèges à aucun des trois témoins mais se fonde désormais sur une conception a-religieuse ou a-gnostique du «lien social » qui la constitue. La distinction entre le public et le privé dans nos États laïcs s'appuie sur cette prise de distance critique par rapport à chacun des trois; ce qui explique pourquoi elle les atteint dans leur propre identité.
\paragraph{comparatisme pour éviter les violences ? pas sûr pour l'islam}
Il est sûr que la crainte de la violence religieuse a conduit tout au long du xixe et du xxe siècle vers un «comparatisme» critique de toute religion, critique en particulier du mono-théisme: dans leur commune obsession de «l'Unique», juifs, chrétiens et musulmans n'auraient cessé de lutter contre tout ce qui est pluriel, poursuivant, chacun à sa façon, le mirage d'une culture unifiée qui exclut ce qui est différent. Cette critique fait peu de cas des identités propres de chaque figure. 
Et comme elle avait produit déjà au siècle dernier des réactions très violentes de la part du christianisme et du judaïsme officiels, elle suscite aujourd'hui des nouvelles violences et des soubresauts identitaires de la part de l'islam. \sn{cf Ratisbonne}
\paragraph{identité propre entre les 3 témoins : on ne peut simplifier en disant qu'ils cherchent à éliminer le pluriel}Nous sommes là devant une alternative intellectuelle et a arque en particulier du monothéisme: dans leur commune obsession de « l'Unique», juifs, chrétiens et musulmans n'auraient cessé de lutter contre tout ce qui est pluriel, poursuivant, chacun à sa façon, le mirage d'une culture unifiée qui exclut ce qui est différent. Cette critique fait peu de cas des identités propres de chaque figure.
Et comme elle avait produit déjà au siècle dernier des réactions très violentes de la part du christianisme et du judaïsme officiels, elle suscite aujourd'hui des nouvelles violences et des soubresauts identitaires de la part de l'islam.
Nous sommes là devant une alternative intellectuelle et spirituelle tout à fait décisive pour nos sociétés modernes.
Comparer les trois « monothéismes » comme je viens de le faire (sans pouvoir d'ailleurs entrer dans les détails), cela conduit-il nécessairement à aplatir les différences et à produire de nouvelles violences ? Ou peut-on espérer que le «comparatisme » réussisse à mettre en valeur le mystère de nos identités? 

\paragraph{comparatisme pour mettre en valeur nos identités, diversité et pour se situer dans la société avec nos propres styles mais en fouillant dans nos ressources. }
\begin{Synthesis}
Mon pari est qu'il peut faire paraître avec une acuité toujours plus grande la diversité extraordinaire de nos manières de nous situer dans la vie commune et par rapport au « lien social ».     
\end{Synthesis}
À condition cependant que la communication entre ces différents \textbf{« styles »}, inévitable dans la société moderne, provoque non pas une fermeture définitive de certains mais suscite en chacun une véritable auto-interrogation2, un retour \sn{788} réflexif sur soi pour découvrir dans son propre patrimoine des ressources jusqu'alors inaperçues, permettant d'affronter le nouveau pluralisme radical.
Ainsi compris le « comparatisme » renvoie chaque tradition au niveau le plus profond de sa propre identité, à sa foi et à l'exercice (parfois autocritique) d'un retour sur soi. Chacun des trois est convié au jeu difficile d'une communication qui consiste désormais à conjuguer le regard interne à sa foi sur les deux autres traditions et la perspective externe des deux autres sur lui; exigence de communication déjà présente dans la célèbre Règle d'or: \begin{quote}
    «Tout ce que vous voulez que les autres fassent pour vous, faites-le pour eux ! » 
\end{quote}Certes, cette règle de réciprocité est au cœur de la préoccupation éthique du judaïsme; mais elle traverse aisément les frontières entre traditions parce qu'elle existe en toute culture. À ce titre, elle se trouve aujourd'hui au fondement de nos sociétés démocratiques\sn{Ricoeur règle d'or}.

\subsection{Savoir apprendre de l'autre}

\paragraph{différence avec la rencontre de la société : la société poblige à des règles de communication, la rencontre oblige à un processus d'apprentissage}
Quand nos sociétés démocratiques imposent aux trois témoins certaines règles de communication, la rencontre des deux autres invite chacun à entrer dans un long processus d'apprentissage. Peut-être l'histoire de la modernité leur fait elle-même d'abord comprendre que, loin de les éloigner de leur propre tradition, cette capacité d'apprendre constitue depuis toujours l'identité la plus profonde du témoin.

\subsubsection{L'interrogation prophétique.}

\paragraph{prophétisme, figure de l'apprentissage "à se mettre à la place d'autrui"} \sn{voir p. \pageref{Dt10Prophete}}
En effet, cette disponibilité à se laisser enseigner par autrui n'a pas été simplement imposée de l'extérieur aux traditions monothéistes; elle est née en leur sein sous la figure du prophétisme. Dire que Dieu est unique suppose déjà une prise de conscience de haut niveau. Mais que la tradition juive découvre un jour que son Dieu ne fait acception de personne et qu'il désire avoir un témoin comme lui généreux envers tout homme, cela suppose un apprentissage éthique d'un tout autre ordre encore : la capacité du juste à « se mettre à la place d'autrui», qui lui vient du souvenir d'avoir déjà occupé cette position: « Aimez l'étranger, car au pays d'Égypte vous fûtes des étrangers » (Dt 10, 17-19).
\paragraph{Jésus, Grand apprenant (He) de la rencontre avec l'autre, de ses souffrance vers l'accomplissement}
Jésus a été, lui aussi, un grand « apprenant», selon les dires de l'épître aux Hébreux : \begin{Ecriture}[He 5, 8]
\begin{quote}
       «Tout Fils qu'il était, il apprit par ses souffrances l'obéissance, et, conduit jusqu'à son propre accomplissement, il devint pour tous ceux qui lui obéissent cause de salut éternel » 
\end{quote}
 
\end{Ecriture} 
Ce que l'épître aux Hébreux affirme avec vigueur, les évangiles synoptiques le racontent en montrant Jésus apprenant des autres qui il est : du lépreux (Mc 1, 40), de la femme hémorroisse (Mc 5, 30), de la Syro-Phénicienne (Mc 7, 29), de Pierre et de bien d'autres encore.
Cet apprentissage le conduit vers l'accomplissement, vers l'expérience méta-éthique du don de soi pour la multitude.

\paragraph{Musulman, interrogation du pacte avec Dieu }
Le musulman, enfin, réitère l'intransigeant jugement initial qui exclut tout pluriel de l'Unique. Lui aussi vit donc la «foi » dans une sorte d'interrogation constante, qui s'exerce, comme on l'a déjà noté, dans le champ pré-éthique du « pacte » avec Dieu, imprimé en tout homme avant sa naissance. C'est en ce sens qu'il faut entendre la réserve du Coran quand il s'adresse directement au prophète Muhammad: 
\begin{Ecriture}[sourate 38, 65]
    « Dis: je ne suis qu'un Avertisseur. Il n'est de divinité que Dieu, l'Unique, l'Invincible » 
\end{Ecriture}
Aujourd'hui il faut donc mettre en valeur ce retour critique sur soi qui caractérise le prophétisme biblique et coranique, indépendamment de la tournure précise qu'il prend dans chaque cas: il constitue, au sein des trois traditions, ce lieu mystérieux où des rencontres imprévisibles avec d'autres pourront se nouer et provoquer un véritable apprentissage entre partenaires.
\sn{790}

\subsubsection{Les étapes de l'apprentissage.}

\paragraph{1. purifier nos préjugés. Rester dans la règle d'Or; apprentissage de la société moderne} 
1. Une première étape consisterait alors à nous purifier des préjugés qui ont entraîné la violence. Je les ai déjà nommés: il peut s'agir de schèmes de «substitution » ou d'«exclusion», quand l'un prétend se substituer brutalement à l'autre dans sa mission religieuse au sein de l'humanité; de façon plus subtile il peut s'agir aussi d'« inclure » l'autre dans sa propre mission, de l'enfermer par exemple dans un rôle de préparation. Une autre manière encore de sortir de la « Règle d'or» de nos rencontres, symétriquement opposée à la précédente, serait de dénier à l'un des interlocuteurs ou à tous les trois toute capacité d'apprentissage dans la société moderne\sn{important pôur l'écologie}. Nous avons vu que le véritable « comparatisme » permet de décrypter cette capacité spécifique d'un retour sur soi au cœur de la foi de chacun des trois.

\paragraph{2. repenser positivement nos liens}
2. La deuxième étape de l'apprentissage est plus difficile.
Il s'agit de repenser positivement nos liens. Or, nous touchons 1 aux limites structurelles de la communication entre les trois, aux limites aussi de notre capacité d'apprentissage.
À nouveau plusieurs possibilités se présentent.

\paragraph{mystiques, facile de traverser les religions} On a toujours enseigné, dans chacun des monothéismes. que l'Unique ne fait pas nombre avec ceux qui témoignent de lui. Ce type d'argument, si familier aux mystiques, consiste à critiquer nos représentations de Dieu, à développer toute une série de procédures à la fois intellectuelles et affectives pour approcher corporellement le mystère du Dieu tout autre. Les spirituels de tradition différente se rencontrent sur ces voies à la fois ascétiques et mystiques; ils traversent aisément les frontières entre les religions parce que, pour eux, un espace infini de communication avec autrui s'est ouvert dans la différence indépassable entre Dieu et ce qu'on peut dire de lui.

\paragraph{mystique pas facile pour christianisme du fait de l'incarnation qui nous dévoile le père, qui n'est pas ineffable}
Relativement bien ajusté au monothéisme juif et musulman, ce type d'expérience mystique se heurte, en christianisme, au mystère de l'Incarnation: \begin{quote}
    « Personne n'a jamais vu Dieu; le Fils unique, qui est dans le sein du Père, nous l'a dévoilé » (Jn 1, 18).
\end{quote} 

\paragraph{risque du mystique d'éluder la question - car elle ne l'intéresse pas ? }Peut-être doit-on ajouter que le spirituel, qui relativise les différences entre les trois monothéismes. risque de ne plus se laisser interroger par la discordance des témoignages qui discrédite toujours la cause elle-même.
\sn{791}

\paragraph{accepter que l'autre n'abandonne pas sa prétention à la prééminence mais accepter de penser cette prééminence sans produire la violence}Faut-il alors préserver la prééminence ou l'excellence de l'un des trois ? D'un simple point de vue anthropologique, je ne vois pas comment éviter le jugement d'excellence sur la tradition à laquelle j'appartiens. D'un point de vue théolo-gique, je ne vois pas non plus comment l'un des trois pourrait renoncer à ses prérogatives d'excellence sans renoncer à sa propre identité. La question se transforme alors en exigence de penser aujourd'hui « l'ultime sceau de la lignée prophétique », « l'accomplissement des Écritures » ou la « mission d'être lumière des nations » de telle manière que ces prérogatives irrépressibles ne produisent pas de violence.

\paragraph{3. \textit{pourquoi} trois témoins}
3. Je reviendrai sur cette tâche proprement théologique.
Cependant, ce traitement du « comment » de l'excellence ne doit pas nous détourner prématurément de la question du «pourquoi ». C'est la troisième étape sur le chemin de l'apprentissage: « Mais pourquoi finalement trois témoins ? » Certes, nos traditions ont quelques réponses à leur disposition pour «expliquer » la venue d'un deuxième et même d'un troisième témoin ou pour rendre raison de leur propre existence après l'arrivée d'un premier et d'un deuxième envoyé: on évoque la dissidence ou l'hérésie de l'héritier, l'aveuglement ou le péché de l'aîné, l'infidélité ou l'exagération des deux prédé-cesseurs.   Mais c'est une chose de raconter la venue progressive d'un premier, d'un deuxième et même d'un troisième, et c'est une autre de penser le « mystère » de leur cohabita-ton dans nos sociétés. N'est-ce qu'un accident de l'histoire, une contingence fortuite ? Ou faut-il y découvrir la main de Dieu, son «dessein » ? Et ce « dessein » est-il même concerné par l'avènement d'une société dont le retrait fondamental par rapport aux trois monothéismes suscite la question de leur «pourquoi» ? Le terme « mystère» n'est sûrement pas trop fort pour désigner la «chose», puisqu'il a déjà servi à Paul pour penser l'énigmatique présence d'Israël aux côtés des chrétiens (Rm 11, 25). Ce « mystère » n'est-il pas devenu plus insondable encore et plus impénétrable (Rm 11, 33-36) depuis que nous sommes «trois témoins » ? La présence des deux autres, que nous apprend-elle sur nous-mêmes et sur l'Unique \sn{792}
que nous ne puissions pas savoir par nous-mêmes ? Question adressée à chacun des trois et à laquelle aucun des trois ne peut répondre à la place des deux autres. 
\paragraph{quelle réponse en tant que Chrétien ?}Nous sommes donc reconduits vers la perspective interne qui ne peut être que chrétienne et théologique pour nous.

\subsection{Entrer dans le règne de l'incomparable}

\paragraph{ne pas compter, ce n'est pas le nombre trois qui est important, mais entrer dans la comparaison (pour y voir in fine un plan de Dieu ?)} Il n'est pas sûr qu'il y ait une réponse à la question «pourquoi trois ?» Peut-être faut-il même renoncer à compter.
On se souvient ici de l'embarras de saint Augustin quand il réfléchit dans son \textit{De Trinitate }sur la différence trinitaire en Dieu: \begin{quote}
    «Le Père, le Fils, le Saint-Esprit sont trois, nous cherchons donc: trois quoi? (tres quid) »
\end{quote}
L'extraordinaire difficulté vient de ce que le terme « personne » (trois personnes) est déjà un générique qui ne peut désigner l'absolu singularité de «chacun», comme le générique « monothéisme » n'arrive jamais à dire ce qu'est l'islam, le christianisme et le judaïsme.
Face à cette limite du nombre dans un domaine où on ne peut «co-numérer», saint Basile conseille de renoncer à compter.
Ce qu'il dit de l'Unique, il faut l'appliquer, me semble-t-il aux trois témoins: \begin{quote}
    « Et s'il faut tout de même compter, du moins que la vérité ne soit point falsifiée: ou bien qu'on honore en silence les choses ineffables, ou bien qu'on compte avec piété et respect. »
\end{quote}  Mais pour pouvoir renoncer à compter il faut bien d'abord compter jusqu'à trois, et pour entrer avec respect dans le règne de l'incomparable il faut bien commencer par comparer. C'est ce que nous avons fait jusqu'ici. Mais dans cette dernière partie je voudrais brièvement tracer un chemin de rencontre qui nous conduit de la comparaison à la découverte de l'incomparable singularité de chacun des trois témoins. \sn{793}
\begin{Synthesis}
Mon hypothèse est que la différence fondamentale du christianisme par rapport au judaïsme et à l'islam, le mystère de l'Incarnation et de la Trinité, est en même temps le lieu  
où se définit une\textsc{ théologie de la rencontre} à la hauteur des enjeux développés dans les deux premières parties. 
\end{Synthesis}

Commençons donc par prendre au sérieux cette différence.
\subsubsection{La contestation de l'unicité du Christ.}

\paragraph{partir du scandale de l'association de Jésus à Dieu pour l'Islam}
Considérer le «témoin » par excellence du christianisme, Jésus, comme un de la Trinité et l'associer à l'unique Dieu, voilà une manière bien scandaleuse, aux yeux des juifs et des musulmans, d'introduire le nombre en Dieu et de diviniser un parmi d'autres sur terre. L'islam a porté sa contestation au cœur même de cette foi: \begin{quote}
    «Ô gens du Livre! N'allez pas au-delà du bon sens dans votre religion. Ne proclamez que la vérité sur Dieu. Le Messie, Jésus fils de Marie, n'est qu'un envoyé de Dieu... Croyez donc en Dieu et en ses prophètes.
Ne dites jamais "trois". Arrêtez cette imposture... Dieu est un Dieu unique. Le Messie ne se sent pas indigne d'être serviteur de Dieu, comme le sont les anges les plus proches de Dieu» (sourate 4, 171s.)

« Il n'est pas concevable que Dieu se donne un fils» couriste 19 35). « Il n'a pas engendré; n'est Dieu» (sourate 4, l/Is.). « Ce n'est pas concevable que Dieu se donne un fils » (sourate 19,35). « Il n'a pas engendré; n'est égal à lui, personne » (sourate 112).
\end{quote}  
\paragraph{des psaumes messianiques comme réponse chrétienne}
Ces quelques versets du Coran nous renvoient, par contraste, à un autre ensemble, les textes messianiques des Écritures juives, et en particulier les psaumes 2 et 110:
\begin{quote}
    «Je proclame le décret du Seigneur. Il m'a dit: "Tu es mon fils; moi, aujourd'hui je t'ai engendré. Demande et je te donne en héritage les nations, pour domaine la terre entière" » (Ps 2, 7). 
    
    «Oracle du Seigneur à mon Seigneur: "Siège à ma droite !... Domine jusqu'au cœur de l'ennemi!" Le jour où paraît ta puissance, tu es prince, éblouissant de sainteté : "Comme la rosée qui naît de l'aurore, je t'ai engendré." Le Seigneur l'a juré dans un serment irrévocable: "Tu es prêtre à jamais selon l'ordre du roi Melchisédech" » (Ps 110, 1-4). 
\end{quote}
\paragraph{Et pour les juifs, laisser ouvert l'avenir du peuple messianique}
Si l'islam conteste l'idée même d'un « engendrement» en Dieu, et cela au nom de sa critique prééthique de toute association d'un pluriel à l'Unique, le judaïsme, lui, ne peut pas reconnaître l'accomplissement définitif de ces psaumes en l'itinéraire de Jésus parce qu'il doit laisser ouvert l'avenir du peuple messianique: au nom même de sa mission éthique il se méfie de tout ce qui occulte la situation d'exil de l'humanité qui durera jusqu'à la fin.

\paragraph{face à cette double contestation, quelle ressort meta éthique ?}
La rencontre de cette double contestation ne nous oblige pas à renoncer aux prérogatives du Christ; je l'ai déjà dit Mais alors en quoi consiste la « purification » de nos schème d'excellence (1re étape de toute rencontre)? Comment pouvons nous aborder positivement la communication avec les deux autres témoins (2e étape de rencontre)? Il est probable que l'enjeu de la rencontre n'est plus d'abord la réaffirmation d'une différence doctrinale mais la mise en œuvre discrète du caractère méta-éthique du christianisme dans l'histoire d la communication humaine : autrement dit, il s'agit pour le chrétiens de vivre de la sainteté même de Dieu, rien de plus e rien de moins. C'est seulement alors que peut leur paraître avec une acuité nouvelle, la capacité de l'Unique à engendre non seulement un «Fils unique » mais aussi avec lui «un multitude de fils ».
\paragraph{Foi Chrétienne comme style de vie, de vivre la sainteté}
On ne soulignera jamais assez le renversement de perspective qui vient d'être produit. La foi chrétienne n'est pas d'abord une doctrine (comme toute une tradition l'a prétendu) mais un « style de vie » ou une manière de vivre de la sainteté même de Dieu: seule l'expérience effective de l'Esprit de sainteté nous permet de confesser et de comprendre un jour l'indépassable excellence du Fils unique du Père. N'est-ce pas cela que la rencontre des autres témoins nous apprend?
Essayons donc de comprendre.
\subsubsection{Unique et la « multitude des fils» (He 2, 10).}
\paragraph{psaumes messianiques : communication de la sainteté de Dieu au roi et à son peuple}
L'enjeu des psaumes messianiques, cités à l'instant, est la communication de la sainteté de Dieu au roi et à son peuple:
\begin{quote}
    « Le jour où paraît ta puissance, tu es prince, éblouissant de sainteté» (Ps 110, 3). 
\end{quote}
S'adressant à la multitude, le Nouveau Testament l'a bien compris quand il appelle tous à être «parfaits comme votre Père céleste est parfait » (Mt 5, 48).
Ce vin nouveau de la perfection divine qui n'est rien d'autre que l'accomplissement inouï et excessif de la Loi (« la justice qui surpasse la justice » selon Mt 5, 20). le Sermon sur la montagne l'introduit dans les «outres» de la Règle d'or:
\begin{quote}
    «Tout ce que vous voulez que les hommes fassent pour vous, faites-le vous-mêmes pour eux : c'est la Loi et les Prophètes » (Mt 7, 12). 
\end{quote}
\paragraph{Sainteté de Dieu, démesurer au delà de la règle de réciprocité}
Cette règle de réciprocité qui se trouve au fondement de nos sociétés modernes peut en effet nous réserve quelques surprises, individuelles et collectives, quand on doit affronter l'antipathie d'autrui ou quand on entend subitement l'invitation à renoncer à toute réciprocité et à prendre sur soi la violence et la faute d'autrui: \begin{quote}
    « Aimez vos ennemis, et priez pour ceux qui vous persécutent, afin d'être vraiment les fils de votre Père aux cieux, car il fait lever son soleil sur les méchants et les bons, et tomber la pluie sur les justes et les injustes » (Mt 5, 44 s.).
\end{quote}
Qu'il s'agisse, dans la communication de la sainteté de Dieu à l'homme, d'un véritable \textit{engendrement} \sn{1. Cette communication de la sainteté même de Dieu fixe définitivement le sens du terme « engendrement », jusque dans la haute christologie de Nicée (« engendré non pas créé, de même nature que le Père »). La forme évangélique ou « narrative » de cet engendrement a été abordée dans le chapitre Il de la deuxième partie, p. 477 s. Je reviendrai aux chapitres v, vi et xin de cette partie à son enjeu christologique et eschatologique; voir plus loin, p. 826, p. 849 s. et p. 1036 s.}
(« devenir fils de votre Père»), on ne le découvre que progressivement: quand on réalise tout d'un coup que l'appel démesuré à être \textit{comme} Dieu, toujours dans telle ou telle situation, s'avère «à la mesure» de celui qui l'entend. 
\paragraph{unification traverse les forces de mort, de peur de l'autre et de nous mêmes. Nous nous "comparons", violence entre témoins}
Mais loin d'être d'emblée acquise, cette gracieuse unification affronte et traverse des violences et des forces de mort qui se déclarent précisément à la « limite » infiniment mobile entre la « démesure divine » et nos multiples « mesures humaines ». La peur face à l'inconnu
- celui de l'autre, de nous-mêmes et finalement de la mort - nous pousse à fixer nos frontières, à garder nos terrains et à entrer dans un jeu de comparaison, voire une lutte sans merci entre « semblables ». Jamais pourtant ce que l'un accueille de la «démesure», \textit{inscrite en tout être humain}, ne se laissera mesurer en fonction de ce qu'en conscience l'autre jugera «à sa mesure». Qui peut faire sortir l'homme de ses mortelles comparaisons, sinon celui qui atteint vraiment en lui la peur d'être soi-même, peur qui est sans doute la racine ultime des mystérieuses violences entre « témoins » ? Il faut entendre la voix d'un autre pour lâcher cette peur; voix qui certes doit émaner de quelqu'un devenu « proche » mais voix qui doit venir en même temps de plus loin: de Celui qui, ici et maintenant, se montre « à la mesure» d'un tel ou d'une telle, se faisant entendre à lui et en lui, avec la douceur et la discrétion
de l'Esprit: «Aujourd'hui, je t'ai engendré» (Lc 3, 22 et Ac 13, 33).

\paragraph{Expérience de l'engendrement : en Christologie, permet le pluriel (He). L'association  de Jésus à l'unicité de Dieu ne peut être séparé de berger de Paix et prêtre de Melchisédech}
Cette expérience inouïe de « l'engendrement » nous fait comprendre les deux versants intimement liés d'une théologie chrétienne de la rencontre. D'abord le versant christologique: la passion de certains auteurs du Nouveau Testament pour le « pluriel ». À ce propos l'épître aux Hébreux, citée tout au début, est tout à fait exemplaire parce qu'elle déplace la «filiation divine » vers la condition commune de tous: \begin{quote}
    «Il convenait, en effet, écrit-il, à celui pour qui et par qui tout existe et qui voulait conduire à la gloire une multitude de fils, de mener à l'accomplissement par des souffrances l'initiateur de leur salut. Car le sanctificateur et les sanctifiés ont tous une même origine; aussi ne rougit-il pas de les appeler frères» (He 2, 10 s.)
\end{quote}
Certes, l'unicité ou l'excellence de Jésus n'est pas niée dans ce texte étonnant: il reste pour l'épître aux Hébreux celui qui ouvre, au cœur de nos mesures humaines, l'accès à la sainteté démesurée de Dieu. Mais Jésus n'est «associé » à l'unicité de Dieu (He 7, 2 s.) que parce qu'il est en même temps berger de paix (roi de Salem) et prêtre selon l'ordre du roi Melchisédech, l'homme unique donc qui est entièrement façonné par le don de soi et appelé à s'effacer pour ouvrir dans l'humanité \textbf{les chemins multiformes de la sainteté de Dieu}.
\paragraph{Expérience de l'engendrement : spirituellement, pas de comparaison entre Fils et témoin }
Ensuite, le versant spirituel ou pneumatologique de la rencontre. La découverte que Dieu engendre une « multitude de fils » sur les chemins de sa sainteté amène à renoncer à toute comparaison entre fils et témoins. Dieu n'est-il pas à la mesure de tant et de tant de mesures humaines, devenues toutes, de ce fait, incomparables? Le Dieu unique des chrétiens est mystère du lien entre des incomparables \sn{voir p 77}. Mais il faut s'être affronté, dans sa propre  vie, à la question de la sainteté, de la démesure divine à ma mesure, pour pouvoir admettre que juifs et musulmans sont, eux aussi, aux prises avec un même combat. On sera alors conduit non seulement au respect dans la rencontre des deux autres témoins mais encore au risque d'y « laisser sa peau » : \begin{quote}
    « En Christ, je dis la vérité, je ne mens  pas, par l'Esprit saint ma conscience m'en rend témoignage », écrit Paul dans l'épître aux Romains. « Oui, je souhaiterais être anathème, être moi-même séparé du Christ pour mes frères..., eux qui sont les Israélites » (Rm 9, 1-5).
\end{quote}
Voilà ce que j'entends par « style chrétien de rencontre»; 
\begin{Def}[style chrétien de rencontre]
    «style» qui se caractérise par une singulière manière d'espérer la paix en affrontant la violence. 
\end{Def}
Il se « définit » au lieu même de la différence fondamentale du christianisme par rapport au judaïsme et à l'islam, dans le mystère de l'Incarnation et de la Trinité. Ce qui a été mon hypothèse.

\subsection{En guise de Conclusion : un jeu de compétition}

\paragraph{« compétition » autour de la sainteté de Dieu, le mode de victoire n'est pas le même pour tous}
La rencontre des trois monothéismes est en dernière instance une « compétition » autour de la sainteté de Dieu.
\begin{quote}
    «Les coureurs, dans le stade, courent tous mais un seul gagne le prix» (1 Co 9, 24),
\end{quote}
 écrit saint Paul qui affectionne la métaphore grecque des Jeux olympiques!. Le juif Philon d'Alexandrie l'a précédé quand il loue dans le \textit{De agricultura} 
 \begin{quote}
 «le seul concours olympique » qui « pourrait être appelé sacré à juste titre: non pas celui que célèbrent les gens d'Élide, mais celui qui vise à acquérir les vertus divines et vraiment olympiennes», en ajoutant, avec une rare finesse que « le mode de la victoire n'est pas le même pour tous mais que tous sont dignes d'estime2».    
 \end{quote}
  On connaît par ailleurs ce verset du Coran. \begin{quote}
      « Si Dieu avait voulu, il aurait fait de vous une seule communauté. Mais il a voulu vous éprouver par le don qu'il vous a fait. Cherchez à vous surpasser les uns les autres dans vos bonnes actions » (sourate 5, 48).
  \end{quote}

\paragraph{rôle des chrétiens : montrer l'unicité de chaque témoin}
Le rôle spécifique des chrétiens dans ce singulier « jeu de compétition », aux allures parfois dramatiques, ne serait-il pas de renoncer à compter et à comparer pour mettre en valeur l'unicité incomparable de chaque partenaire? Tâche difficile qui peut les conduire aujourd'hui encore dans l'expérience du don de soi.\sn{798}
\paragraph{mystère de la diversité des témoins : penser la pluralité des monothéismes}
Mais revenons, pour finir, à notre question: pourquoi des témoins (dernière étape de la rencontre)? Renoncer un jour à poser cette question et abandonner le « comptage », c'est certes accepter avec l'épître aux Romains que le « dessein » de Dieu est insondable et impénétrable. Mais il ne s'agit absolument pas d'un scepticisme qui renoncerait à penser vraiment pluralité des monothéismes. Penser le mystère des trois témoins c'est baliser, au sein d'une histoire faite de compétitions chemin de rencontre qui nous amène à vivre dans le règne trinitaire de l'incomparable.
\paragraph{Plan : on avait pensé la communauté ecclésiale, il faut penser le sans religion}
Ce qui précède nous a déjà conduit vers le deuxième lien de l'expérience trinitaire de la foi: la référence de la communauté ecclésiale à la sainteté messianique de Jésus de Nazareth.
Les deux dernières études de cette première étape entreront davantage dans la question christologique en faisant d'abord état de la transformation de la christologie catholique au xxe siècle et en revenant ensuite au rapport du Christ aux hommes religieux et à ceux qui sont « sans religion ».


\subsection{Article original - pas dans le livre}

\subsubsection{I. Judaïsme, christianisme et islam: peut on les regrouper sous le même concept de "monothéisme"?}

\paragraph{les comparer synchroniquement}
Pour répondre à cette première question, je voudrais montrer comment le nom de l'Unique s'inscrit, chaque fois, dans une "figure spécifique", c'est a dire dans un ensemble à la fois linguistique et social. Je ne me soucie donc point encore de la connexion interne entre les trois figures "mono-théistes, mais je les considère d'un point de vue strictement synchronique, c'est-à-dire que je les juxtapose dans un système de comparaison assez élémentaire, en me demandant comment analyser leurs ressemblances et leurs différences.

\subsubsection{Donner "figure" à l'Unique}

Pour cette première comparaison, j'emprunte à Stanislas Breton2 une distinction assez commode entre "figuration" et "jugement réfléchissant", dont le sens se dévoilera progressivement.

\paragraph{d'abord les témoins donnent figure à l'Absent}
Restons d'abord au niveau de la "figuration' : Dieu existe - concrètement
 dans la relation de foi de ceux et celles qui se réclament de Lui. Ce sont ses témoins qui lui donnent "figure dans notre histoire: ils ne constatent pas son existence mais rendent présent l'Absent, et cela à leurs propres risques et périls; ils sont radicalement, ce qui veut dire par leur vie et leur mort, impliqués dans ce qu'il font. C'est pourquoi Stanislas Breton\sn{théologien et philosophe} met au cœur du processus qui donne "figure" à Dieu, non seulement son "nom" mais encore ce qu'il appelle des "descriptions relatives"3 comme celles-ci: "Yahvé, le sauveur d'Israël"; "Dieu, le père de Jésus-Christ"; et pourquoi ne pas ajouter à son énumération "le Dieu dont Muhammad est l'envoyé ? 
 
 \paragraph{\textit{l'être vers} du témoin et \textit{l'être dans} support territorial}Peu importe donc, pour le moment, la forme exacte de cette relation d'alliance, différente dans chaque cas; elle se décompose en tout cas en deux versants ou directions\sn{je suis encore très librement l'analyse logique de Breton } l'être vers" Dieu du témoin, mais aussi "l'être dans" qui enracine la relation dans un support territorial.

 
\begin{quote}
    "Les deux aspects sont indissociables, écrit Breton. C'est le même sujet qui tantôt est comme aspiré, dans l'oubli de soi, par l'élan (l'arsis) qui l'emporte; et qui tantôt, par un mouvement opposé, fixe cet élan et l'individualise dans le support d'une terre et la durée d'un groupe ethnique"4.
\end{quote}

\paragraph{de \textit{figure spécifique individuelle} à l\textit{unicité}}
Que Dieu prenne une figure spécifique pour quelqu'un dans une relation de foi, cela ne dit encore rien de son "unicité". Dire que ce Dieu-là existe et qu'il n'existe qu'une seule fois, qu'il est unique, est donc un jugement second, porté par le témoin sur une "figure" de Dieu qui a déjà pris consistance pour lui. On peut l'appeler "jugement réfléchissant" parce qu'il est comme un retour du témoin sur lui-même et sur la qualité de son rapport à Dieu : il intervient, au cœur même de la figure, déjà parce qu'il est - logiquement - impossible de confesser l'Unique sans exclure en même temps, et au moins implicitement, la multiplicité des dieux qui sont donc toujours là comme son ombre éternellement surmontée.

\paragraph{exclusion des autres Dieux}
Les formules négatives des témoignages citées plus haut en sont la trace dans le langage: "Avant moi ne fut formé aucun dieu et après moi il n'en existera pas" (Es 43, 10.11) ; et la forme du "procès" de ces textes en est la marque socio-politique. Nous touchons-là, me semble-t-il, l'essentiel du "mono-théisme". Spinoza le confirme à sa manière quand, anticipant certaines critiques modernes, il écrit dans sa lettre 50:
\begin{quote}
Vous me demandez quelle différence il y a entre Hobbes et moi quant à la politique : cette différence consiste en ce que je maintiens toujours le droit naturel et que je n’accorde dans une cité quelconque de droit au souverain sur les sujets que dans la mesure où, par la puissance, il l’emporte sur eux ; c’est la continuation de l’état de nature.
Pour ce qui concerne la démonstration par laquelle dans l’Appendice des Principes de Descartes démontrés géométriquement, j’établis que Dieu ne peut être appelé que très improprement seul et unique, je réponds qu’une chose ne peut être dite seule et unique à l’égard de l’essence mais seulement à l’égard de l’existence. Nous ne concevons en effet les choses comme existant en un certain nombre d’exemplaires qu’après les avoir ramenées à un genre commun. Qui tient en main par exemple un sou et un écu, ne pense pas au nombre deux s’il ne range le sou et l’écu sous une même dénomination, celle de pièce de monnaie. Alors seulement il pourra dire qu’il a deux pièces de monnaie, l’écu et le sou étant tous deux dénotés par ce terme [1].
De là suit manifestement qu’une chose ne peut être dite seule et unique avant qu’on en ait conçu quelque autre ayant même définition (comme on dit) que la première. Mais l’existence de Dieu étant son essence même, \textbf{il est certain que dire que Dieu est seul et unique montre ou qu’on n’a pas de lui une idée vraie, ou que l’on parle de lui improprement.}
Pour ce qui est de cette idée que la figure est une négation mais non quelque chose de positif, il est manifeste que la pure matière considérée comme indéfinie ne peut avoir de figure et qu’il n’y a de figure que dans des corps finis et limités. Qui donc dit qu’il perçoit une figure, montre par là seulement qu’il conçoit une chose limitée, et en quelle manière elle l’est. Cette détermination donc n’appartient pas à la chose en tant qu’elle est, mais au contraire elle indique à partir d’où la chose n’est pas. La figure donc n’est autre chose qu’une limitation et, toute limitation étant une négation, la figure ne peut être, comme je l’ai dit, autre chose qu’une négation.
 
\end{quote}
Pourquoi ? Parce que, si quelqu'un dit que Dieu est un ou unique, il l'insère d'abord, de manière irrespectueuse de \textbf{qunagination} morcelante de la série des nombres ; en écartant ensuite pluriel de Dieu, il fait finalement dependre l'Unique de cet acte de négation.
\paragraph{force du prophétisme d'ouverture à l'universel par le témoin}
Quoi qu'il en soit de cette critique radicale sur laquelle nous reviendrons, il faut surtout noter que le "jugement réflexif" a connu de grandes transformations. C'est dans le prophétisme qu'il a été doté d'une véritable force historique d'ouverture vers l'universel : la prophétie étend la relation d'alliance toutes les "nations"; mais elle le fait par la médiation du "témoin" (qui garde donc le privilège donc le privilège d'unicité reçu de son Dieu). 
\paragraph{par le prophète, dimension éthique; itré du neant, le témoin répond en tirant du néant les autres. Rôle de la partialité}
Cette ouverture est tout à fait paradoxale dans la mesure où elle change le visage même de "l'Unique” par l'intervention d'une nouvelle composante: la qualification éthique de la "relation" entre Dieu, son témoin et les autres. \label{Dt10Prophete} \begin{quote}
    "Le Seigneur *votre* Dieu ne fait acception de personne et ne se laisse pas corrompre par des présents. C'est lui qui fait droit à l'orphelin et à la veuve, et il aime l'étranger, auquel il donne pain et vêtement. Aimez l'étranger, car au pays d'Égypte *vous fûtes des étrangers*" (Dt 10, 17-19).
\end{quote} 
Se souvenant de son propre néant d'où l'élection le tire, le témoin ne peut répondre de l'alliance unique qu'en imitant son Dieu par la justice faite à autrui. Le "jugement réflexif" qui exclut le pluriel de la figure de Dieu est toujours aussi rigoureux mais il prend corps dans l’obligation éthique de faire surgir du "néant" dans lequel se trouve l'étranger ou l'opprimé le règne de Dieu parmi les nations.

\begin{Def}[jugement réflexif]
    jugement ("seul Dieu") le retour critique sur soi du prophétisme, est le lieu théologique ou des rencontres imprévisibles avec d'autres témoins pourront se nouer, conduise jusqu'à une véritable capacité d'apprendre auprès des autres \textbf{partenais} quelque chose sur sa propre identité.
\end{Def}
Il serait certainement dangereux d’inscrire le “jugement réflexif” du prophétisme dans une sorte d’évolution linéaire qui irait d’une théologie nationaliste ou exclusiviste de l’élection par le “compromis réflexif”, le retour critique sur soi du prophétisme, est le lieu théologique ou des rencontres imprévisibles avec d'autres témoins pourront se nouer, conduise jusqu'à une véritable capacité d'apprendre auprès des autres partenais quelque chose sur sa propre identité.

\subsubsection{Comment analyser les ressemblances et les différences entre les trois ?}
 
 
\paragraph{figuration, un système avec Dieu unique, le témoin, le tiers et chacun joue un rôle}
Retenons surtout de cette approche très élémentaire du "monothéisme” que le processus de "figuration" fonctionne comme un "système". Quand change un de ses éléments : le Dieu unique (1), le témoin (2) ou le tiers (3), c’est toute la figure qui se transforme. 
\paragraph{une brèche car on ne peut isoler le Dieu unique}
Il faut le souligner parce que cette remarque ouvre une première brèche dans la façon courante de parler de trois "monothéismes" abrahamiques: on fait souvent comme si on pouvait isoler le Dieu unique du reste de ses figures historiques : "peu importe les querelles entre les trois témoins, c'est au même Dieu unique que nous croyons". Ce qui précéde montre le caractère primaire de ce type de raisonnement très répandu. Mais comment décrire alors, de manière plus précise, les ressemblances et les différences entre les trois figures historiques du Dieu unique ?

\paragraph{partir du tiers ?}
On pourrait - première possibilité - partir d'un des pôles, par exemple du "rôle" joué par "le tiers", et voir comment la place qui lui est réservée se répercute sur l'ensemble de la "figure". Israël parle de l'Égypte, des nations, de l'étranger, etc., expressions qui véhiculent des valeurs diversifiées, pouvant aller de l'accusation d'idolâtrie jusqu'à l'intégration de la sagesse étrangère des nations. Le christianisme primitif, de son côté, utilise la distinction entre juifs et grecs, juifs et païens ou circoncis et incirconcis. Pour l'épître aux Ephésiens, par exemple, le Christ est celui qui, par son sang, a détruit le mur de la séparation, créant ainsi l'Église: \begin{quote}
    "Vous n'êtes plus des étrangers, ni des émigrés : vous êtes concitoyens des saints, vous êtes de la famille de Dieu" (Ep 2, 9).
\end{quote}

\paragraph{Rm à la différence d'Ep pense l'élection d'Israël}
L'épître aux Romains, par contre, reste en retrait par rapport à une telle vision d'intégration totale: elle met en valeur le "mystère" de l'autre, qui est le "mystère" de l'élection d'Israël: 
\begin{quote}
    "car les dons et l'appel de Dieu sont irrévo-cables" (Rm 11, 29).
\end{quote} 

L'islam, enfin, a développé ses propres catégories pour situer les "autres" : retournant à Ismaël, fils d'Abraham (Ibrâhim) et de Hâgar, ancêtre du peuple arabe, il renouvelle le procès intransigeant contre le polythéisme païen et donne aux juifs et aux chrétiens le statut particulier des "gens du livre": expression qui suppose l'idée (d'origine apocalyptique) d'un livre sacré ("mère du livre", "livre caché" ou "table gardée"), caché depuis toujours dans le mystère incommunicable de Dieu et révélé dans le Coran à jamais incorruptible et incréé. Le livre de Moise et l'Évangile de Jésus qui le confirme ne sont qu'une partie de cette Écriture fondamentale. Si l'islam reproche aux juifs leur "infidélité" à ce qui était à l'origine, et aux chrétiens leur "exagération", il leur accorde cependant un certain respect et en tout cas quelques  privilèges.

\paragraph{\textit{l'Être dans} un support, livre ou terre}
Ce n'est pas le moment d'affiner ce type de comparaison qui doit aussi porter sur les deux versants de la relation du témoin à l'Unique : son "être vers", ou sa liberté, et celle de Dieu, mais aussi "l'être dans" de cette relation dans un support, qu'il s'agisse du livre avec la question herméneutique, ou de la terre avec les problèmes de cohabitation ou d'expulsion. Mais une simple analyse des "contenus" de la "figure" historique de Dieu ne suffit pas vraiment pour désigner le centre de nos différences et de nos ressemblances. 
\paragraph{jugement réflexif du témoin dans son rapport à Dieu et tiers}
Il doit être cherché - deuxième type d'approche - dans le "jugement réflexif" qui constitue le "témoin" dans son rapport à l'Unique et aux autres. Je proposerais donc volontiers, en m'excusant des raccourcis que je dois prendre, une hypothèse de classement.

\subsection{Une hypothèse de classement}
 
*pas dans le livre mais plus assertif*

\paragraph{prophétisme juif et éthique car lie confession et combat pour autrui (Dt) }
Le prophétisme juif se situe dans le champ de l'éthique, dans la mesure il lie intimement la confession de l'Unique et le combat pour autrui. C e lien intime, nous l'avons déjà dit, est comme une “puissance de réalisation” (ou de libération) qui fait advenir Dieu comme visage unique pour ceux et celles qui \textit{sont sans visage humain}.

\paragraph{Chrétien : don inoui et excessif de Dieu unique communiquant la surabondance qui le constitue}
Selon la tradition chrétienne, la foi s'ouvre à une réalité inouie et excessive: l’unique Dieu est censé communiquer gratuitement à la multitude la sainteté qui le constitue en lui-même, et tous peuvent désormais déc par la foi à quel point cette sainteté les habite déjà :  \label{Mt5Def3}
\begin{quote}
    "Vous serez parfait comme votre Père céleste est parfait" (Mt 5, 48).
\end{quote} 
C'est cela "l'accomplissement de loi et des prophètes", vécu dans les gestes les plus quotidiens. Si on se réle vocabulaire de l'éthique, on peut donc appeler le christianisme un m théisme méta-éthique, terme qui désigne un "excès", une "démesure" au "accomplissement" à l'intérieur même du champ de l'éthique: le don des visé par la symbolique du "sang versé" (Ephésiens) ou communication l'agape divine, de l'amour surabondant de Dieu à tout être humain.
\paragraph{islam}
Quant à l'islam, il a perçu quelque chose de ce jugement d'excès quand il accuse les chrétiens d'exagération": 
\begin{Ecriture}[sourate 5, 77]
    " gens du livre, n'exagérez pas te religion ! La vérité, rien d'autre !" 
\end{Ecriture}
ou encore : \begin{Ecriture}[sourate 4, 171]
  "À gens du livre N'allez pas au-delà du bon sens dans votre religion. Ne proclamez que la vérité sur Dieu. Le Messie, Jésus fils de Marie, n'est qu'un envoyé de Dieu.
\end{Ecriture}

Les envoyés de Dieu ont charge de rappeler aux hommes le pacte a "'alliance (mithaq) dite de "pré-éternité" qui précède toute division historique entre judaisme, christianisme et islam, et qui date du moment où Die nique une descendance des reins des fils d'Adam et qu'Il les fit témoigner à l’encontre d'eux-mêmes, disant : \begin{Ecriture}[sourate 7, 172]
    Ne suis-je point votre Seigneur? Ils répondirent: Oui, nous en témoignons"
\end{Ecriture} . Il n'y a donc pas de progrès dans la révélation, mais rappel ultime et définitif de ce qui a été oublié ou déformé: l'unicité absolue de Dieu, menacée par le polythéisme et tout ce qui lui ressemble, comme l'association du Christ à Dieu. Ce retour, en deçà de l'histoire et de ses divisions, vers l'origine adamique du "pacte" qui fait de l'homme d'emblée un "croyant", fonde l'universalité de l'islam. \paragraph{En Islam, Dieu imprime son sceau au coeur de chaque homme de toute éternité}Celle-ci n’est plus basée, comme dans le prophétisme juif, sur la qualité éthique de la "relation" de l'Unique avec son témoin et avec l'étranger; elle s'appuie sur l'idée que tout homme porte à sa naissance, sceau imprimé par Dieu en son coeur, la proclamation de foi de la pré-éternité : cette "religion naturelle", liée à la création comme première révélation d'en deçà des temps, est une prédisposition à recevoir l'islam.
\paragraph{Unicité d'abord (rapport à Dieu avant rapport aux autres)}
La lutte de l'islam pour l'unicité de Dieu précède donc toute préoccupation éthique. Pour cette raison, on peut le qualifier de monothéisme pre éthique. Le jugement réflexif de sa foi s'exerce dans le champ pré-éthique en double sens d'un retour à l'immémorial et d'une confession de l'Unique pour laquelle l'éthique, sans être absente, ne peut jamais être de l'ordre du principe Le registre du temps, qui vient d'apparaître dans ce classement (l'accomplissement ou l'alliance de pré-éternité), nous oblige à apporter une dernière précision. 

\paragraph{Rajouter le récit, pluriel, à Dieu, tiers et prophète pour éviter de tourner au concept } Aux trois éléments fondamentaux de la figure du "mono-théisme", l'Unique, son témoin et autrui, il faut ajouter un quatrième qui ne fait pas nombre avec les précédents : le récit. Jean Lambert note l'importance capitale de la "narrativité sémitique dans les trois monothéismes, qui leur permet de gérer, chaque fois de manière spécifique, leur rapport au temps et à l'histoire. Paul Beauchamp montre dans \textit{L'un et l'autre Testament II }ie lien intime entre confession de l'unicité et récit: 
\begin{quote}
    "la nécessité, selon ses propres termes, que l'Un soit manifesté par le récit orientant le pluriel"10
\end{quote}. Si l'unicité risque en effet de désespérer toujours nos tentatives de la saisir par le concept, son articulation sur le récit permet de l'approcher dans une perspective anthropologique, basée sur la différence sexuelle et la loi de l'inceste \sn{lecture datée ?}:
\begin{quote}
    "comment le corps et le récit sont-ils impliqués l'un dans l'autre par la rencontre de l'homme et de la femme?"11
\end{quote}
, demande Paul Beauchamp; et il montre comment l'expérience de l'unique émerge sur la trajectoire du récit, précisément dans cette rencontre. Pour ma part, j'insisterais sur le \textbf{lien intrinsèque entre la qualification éthique de la relation au Dieu unique et le récit }: est-il possible de faire l'expérience prophétique de l'unicité de Dieu dans un geste qui fait bénéficier autrui de sa justice, sans raconter - et faire mémoire - qu'on a été soi-même bénéficiaire de son élection: 
\begin{Ecriture}[Dt 10, 17-19]
    "Aimez l'étranger, car au pays d'Égypte vous fûtes des étrangers" 
\end{Ecriture}

\paragraph{Eviter une vision évolutioniste qui sous estime le poids anthropologique de la narrativité}
Ce n'est pas le lieu de préciser davantage l'articulation entre l'éthique et le récit 12 qui se situe en effet à l'arrière-plan de la distinction entre les catégories d'éthique, de méta-éthique et de pré-éthique. Je voudrais au moins suggérer que l'analyse logique de Breton, dont je me suis inspiré, ne résiste pas toujours à la tentation d'une vision évolutionniste de l'histoire religieuse de humanité, peut-être parce qu'elle sous-estime quelque peu le poids anthropologique de la narrativité biblique. 

\paragraph{Quand l'amour, "les entrailles" est pour le prochain et non l'ami ou le conjoint}Tracer une ligne qui, partant d'une théologie \textbf{exclusiviste} de l'élection, et passant par le \textbf{"compromis prophétique"} qui aurait tardé encore l'excellence de l'unique, aboutirait à une éthique basée sur le seul principe d'une singularité en relation avec d'autres singularités religieuses 13, n'est-ce pas renoncer trop vite à l'idée que l'excellence de l'unique, composante fondamentale de l'amour humain (philia), reste présente jusque dans expérience de l'agapè, quand c'est le "prochain", et non plus seulement le conjoint ou l'ami, qui fait appel à mes "entrailles". L'histoire d'amour du grand récit des Écritures juives et des Écritures chrétiennes (lues dans la perspective du Cantique des cantiques) ne se laisse pas totalement intégrer dans l’expérience d’un "\textit{amor intellectualis}", telle que Breton semble le proposer à la suite de Spinoza14. Elle balise plutôt le "travail" continuel de "jugement réfléchissant" à l'intérieur de nos rencontres les plus imprévisibles. 

\paragraph{le manque de récit du Coran participe à sa vision pre-éthique} Dans cette perspective d'un lien intrinsèque entre la qualité éthique de relation à l'Unique et le récit, la particularité stylistique du Coran qui n'obeit pas au modèle du grand récit mais se présente comme un ensemble mule forme de sourates, serait une confirmation sensible de son statut pré-éthique.
Clôturé sous les Umayyades (634-744)15 qui cachaient systématiquement le travail de rédaction, le faisant remonter à la "mère du livre" en Dieu, le Coran indique la victoire définitive de l'Unique. Certes, il ne ferme pas définitivement l'histoire, mais il ne la rouvre pas sous le signe de la reconnaissance éthique de sa propre particularité historique. Ce qui ne signifie nullement - faut-il le répéter? -, que la préoccupation éthique soit absente de sa "figure monothéiste".

Peut-on regrouper l'Unique et ses témoins sous le même concept "monothéisme" ? Si la réponse est affirmative, il faut la nuancer beaucoup nous venons de le voir. Elle fait, en tout cas, apparaître un jeu de ressemblances et de différences qui rend le problème du lien historique et théologique entre les trois "figures" de plus en plus difficile.
\section{genezah}

comment être fils unique d'un père et avoir en même temps des frères ? il s'agit d'articuler sainteté  et unicité dans la fécondité de l'unicité de Jésus, qui donne sa vie pour que les autres aient aussi la vie. Il y a de l'ordre de l'excès,  et non la conséquence d'une obligation légale ou éthique; c'est la démesure celle du bon Samaritain
