\chapter{De Jésus au Christ}

\section{Eléments bibliographiques}
\begin{itemize}
    \item HARNACK, A., L’essence du christianisme, tr. par J.-M. TÉTAZ, Genève 2015.
    \item HAZARD, P., La crise de conscience européenne 1680-1715, Paris 1961.
    \item LESSING, G. – REIMARUS, Fragments d’un anonyme de Wolfenbüttel, tr. par M. GÉRAUD,
Paris 2018.
    \item REYMOND, B., A la découverte de Schleiermacher, Paris 2008.
    \item SCHLEIERMACHER, F., Discours sur la religion à ceux de ses contempteurs qui sont des
esprits cultivés, tr. par I. J. ROUGE, Paris 1944
    \item ID., La cohérence de la foi chrétienne, tr. par B. REYMOND, Genève 2018.
    \item SCHWEITZER, A., « Histoire des recherches sur la vie de Jésus. Considération finale », ETR
69 (1994) 153-164 (tr. par J-P. SORG).
\end{itemize}

%--------------------------------------- 
\section{Introduction}

\paragraph{Une culture de dialogue, c'est accepter de se remettre en cause} Sinon, on est dans une \textit{apologie}, une \textit{dispute}. 

\paragraph{Comment comprendre le dogme ?} \textit{Si Jésus est vrai Dieu}, où on accepte ou on accepte pas.  Aussi, il ne faut pas partir du dogme mais les conditions actuelles nous obligent à repartir de Jésus et l'histoire de Jésus comme un lieu de débat. \textit{Qui dites-vous que je suis}. Le recevoir dans le dialogue.

\paragraph{le dogme au service de la parole de Dieu}

\paragraph{Approche du cours, ne pas partir du dogme mais du dialogue sur la personne de Jésus} Dans le XIX, un dialogue entre la culture chrétienne et la culture des lumières : 
\begin{itemize}
    \item Pour sortir du dilemne entre \textit{dogmatisme} et du \textit{rationalisme} (Jésus comme maître de moral), il fallait revenir à l'histoire de Jésus et montrer que Jésus est la source du Christianisme.

    \item Or ce retour à l'histoire concrête (à l'homme Jésus), conduit à une certaine relativisation : \textit{Jésus n'est pas l'unique}
\end{itemize}





%--------------------------------------- 
\section{La remise en cause historique de la foi en Christ}

\paragraph{Approche théologique} C'est toujours intéressant de ne pas affronter la question directement en théologie. Ici, l'astuce du chemin est de montrer comment le \textit{débat} a eu lieu.


\subsection{Le mouvement unitarien à l’origine de la crise du dogme christologique}

\begin{Def}[Unitarien]
mouvement anti-trinitaire, qui insiste sur l'unité de Dieu, qui part d'Italie au XVI. Ils vont en Pologne, Grande Bretagne (Newton) et USA (Jefferson, Dickens).
\end{Def}
On l'appelle aussi le \textit{socinianisme}, de Sozzini (1539-1604). 
Ils vont au delà de la Réforme : 

\begin{quote}
    « Quand au Christ, s’il est fils de Dieu il n’a pas la nature divine, “car cela n’est pas seulement contraire à la droite raison, mais c’est aussi contraire à la Sainte Écriture » (Catéchisme de Rakow, question 72).
\end{quote}

Jésus intercède pour l'homme et il n'est donc pas Dieu. A la différence d'Arius qui considérait Jésus comme première des créatures de Dieu et non pas un homme, ici, Jésus est homme. On fait appel à la raison et l'écriture.

\paragraph{Il fallut répondre à ces unitarien} en repartant des Ecritures.




\subsection{La critique des traditions et des dogmes : la Renaissance et les Lumières} 


\paragraph{La Renaissance revient aux sources} avec l'humanisme (Erasme) et Luther (qui revient aux Ecritures). L'idée est que \textit{l'eau est plus pure à la source}. On remet en cause les traditions entassées.

\paragraph{Les Lumières vont valoriser l'autonomie de la Raison} au lieu de s'appuyer sur l'autorité des autres. Doute méthodique de Descartes.  De plus en plus, on va invoquer les Ecritures contre les dogmes.


\paragraph{L'Ecriture va elle même être fragilisée} Richard Simon, Oratorien (1711) publie \textit{l'histoire critique du vieux testament} (1630). Il s'aperçoit qu'il y a des problèmes, en particulier que Moise ne pouvait avoir écrit le Pentateuque. 





\subsection{La crise du principe scripturaire}  
\paragraph{Question herméneutique au dépend de l'autorité} Chez les catholiques, le magistère.  Mais les protestants, problème du critère (Luther dit : \textit{la Foi sauve}). 

\subsection{La rupture entre la vérité universelle et la révélation historique}  
\paragraph{XVIII : siècle de la Loi} ce qui est stable, découverte avec Newton des lois qui gouvernent le monde. on a du mal à penser l'intervention de Dieu dans le monde. 

\paragraph{Signification universelle de Jésus est posée} Lessing oppose une vérité universelle à un événement historique : 

\begin{quote}
   « Des vérités historiques accidentelles ne peuvent jamais devenir la preuve de vérités rationnelles nécessaires » (Lessing, Über den Beweis des Geistes und der Kraft, 1777). 
\end{quote}

Les vérités sur Dieu sont inaccessibles à un événement historique, fut il Jésus. E. Kant dira que \textit{Jésus c'est celui qui a incarné les valeurs mais quand on a la loi morale, on n'a plus besoin de lui}.

\subsection{La thèse de Reimarus et l’aboutissement christologique de la crise} 

\paragraph{Reimarus pousse à l'extrême} Il est publié à titre postume, avec une intention radicale entre Jésus et les apôtres. 
\mn{LESSING, G. – REIMARUS, Fragments d’un anonyme de Wolfenbüttel, tr. par M. GÉRAUD, Paris 2018.}

A partir de là, il va expliquer ce qui s'est passé (ce sont les apôtres qui fondent le christianisme). Rupture maximale entre Jésus et le Christ de la Foi.

\begin{quote}
    « J’ai montré jusque-là que le nouveau système modifié des apôtres, celui d’un libérateur spirituel souffrant qui doit ressusciter de la mort et après son Ascension au ciel reviendra bientôt du ciel avec une grande force et dans une grande gloire, est inventée et fausse dans son premier fondement essentiel, à savoir la résurrection d’entre les morts : 1) parce que le témoignage prétendument extérieur de la garde romaine, chez Mathieu, est en soi souverainement absurde, et n’est jamais mentionné par aucun des autres évangélistes et apôtres, mais que ce système est contredit par nombre de circonstances, si bien qu’il reste plutôt tout à fait possible et fort vraisemblable (…) que les disciples de Jésus étaient venus la nuit, avaient volé le cadavre et dit ensuite qu’il avait ressuscité ; 2) parce que les disciples de Jésus eux-mêmes, comme témoins de la résurrection, non seulement varient beaucoup sur les points principaux de leur déclaration, mais encore se contredisent aussi entre eux de plusieurs manières, clairement et grossièrement (…) (Reimarus, 105).
\end{quote}


\begin{quote}
    (…) Il ne pouvait manquer que quelques Juifs qui compilaient les différentes descriptions, en viennent à penser que leur Messie viendrait deux fois, et ce sous une forme radicalement différente. On conçoit donc avec évidence que les apôtres se servirent désormais de ce système, bien que peu l’aient adopté, car leur premier système, goûté par la plupart, était réfuté par son issue ; et qu’ils se sont donc aussi promis de Jésus, comme étant le Messie, après sa mort, une autre venue glorieuse. Il faut en outre savoir que les Juifs pensent que la résurrection des morts suivrait l’autre venue du Messie, qui à ce moment jugerait les morts et les vivants : et alors commencerait le royaume des cieux ou l’autre monde (…). Les apôtres devaient donc eux aussi, grâce à leur nouveau système, promettre une autre venue du Christ dans les nuées du ciel, où ce qu’ils avaient maintenant espéré en vain serait exaucé, et où ses adeptes croyants, une fois le tribunal passé, devaient hériter du royaume » (Reimarus, 107).
\end{quote}
%--------------------------------------- 
\section{Retrouver la signification universelle de Jésus}

\paragraph{Comment refaire le lien ? }


\subsection{Schleiermacher : le Christ romantique}  

\paragraph{Courant libéral } au début du XIX. : courant libre par rapport au dogme et le discuter par rapport à la modernité. \textit{Annoncer la Foi dans la culture de l'époque}, c'est prendre des risques, entre répéter et personne ne comprend ou bien s'adapter trop à la culture.

\paragraph{F. Schleiermacher} plus grand théologien du XIX. \textit{Le Christ romantique}. Il est passé par les \textit{frères moraves} (équivalent des charismatiques). En 1799,  il publie le \textit{Discours sur la religion à ceux de ses contempteurs qui sont des esprits cultivés}.
\mn{SCHLEIERMACHER, F., Discours sur la religion à ceux de ses contempteurs qui sont des esprits cultivés, tr. par I. J. ROUGE, Paris 1944}

Il ouvre une nouvelle voie en montrant que l'\textit{homme est essentiellement religieux}. Figure concrète de Jésus et le Christ pour tous les hommes.

\paragraph{L’idée de « sentiment » (Das Gefühl)} Il refuse la métaphysique classique et de l'autre côté la morale (de Kant). Ce qui est premier, c'est l'experience d'une \textit{dépendance radicale avec le divin}. A partir de cette expérience fondamentale, il s'agit de donner des mots à cette expérience spirituelles. Le langage religieux provient de cette expérience mais ne peut l'enfermer.

\begin{Prop}
Est ce que l'expérience mystique est première ou bien est ce que le langage religieux permet de comprendre l'expérience mystique ? 
\end{Prop}

\paragraph{La conscience divine en Jésus} Il y a une conscience divine en Jésus, transparence en Jésus, immédiateté (proximité de Dieu, du Royaume de Dieu).
En Jésus, son humanité possède la capacité de communiquer son expérience aux autres.



\paragraph{Le Christ médiateur et rédempteur} Jésus va révéler à tous les hommes cette dépendance absolue envers Dieu. Un impact collectif sur les hommes, \textit{contagion de son expérience}. 
\begin{Ex}
Quelqu'un qui est lumineux, charismatique, irradie autour de lui, ces disciples deviennent aussi lumineux.
\end{Ex}

\paragraph{le Christ devient le sommet de l'expérience religieuse} Il a eu la foi parfaite et il transmet son expérience. Mais on n'a pas de foi en Jésus, pour Schleiermacher.

Pour la question de la rédemption, la croix n'est pas essentiel pour Schleiermacher. Le Rédempteur c'est celui qui nous met en communication avec Dieu. 





\begin{quote}
    « Il doit, après chacune de ses excursions dans l’Infini, \textit{extérioriser l’impression qu’il en rapporte, de façon à faire d’elle ainsi un objet communicable par l’image ou la parole} (…) et il doit donc aussi, involontairement et pour ainsi dire dans l’état d’enthousiasme\sn{habité par Dieu} (…) figurer pour autrui ce qui lui est arrivé, sous une forme sensible, en poète ou en voyant, en orateur ou en artiste. Un tel homme est un véritable prêtre du Très-Haut, qu’il rend plus accessible à ceux qui ne sont pas habitués à saisir que le fini et sa valeur minime ; il leur présente les choses célestes et éternelles comme des objets de jouissance et de communion, comme la seule source inépuisable de ce vers quoi tend toute leur aspiration supérieure. Il vise ainsi à éveiller le germe somnolent de la meilleure humanité, à allumer l’amour du Très-Haut, à transformer la vie ordinaire en une vie plus haute, à réconcilier les fils de la terre avec le ciel\sn{Ici la rédemption} (…) C’est ici la prêtrise supérieure, celle qui fait connaître l’âme intime de tous les mystères spirituels, et dont la voix descend des hauteurs du royaume de Dieu » (Schleiermacher, Discours, 125-126).
\end{quote}





\paragraph{Les ambiguïtés de la pensée de Schleiermacher} 


\paragraph{Reprend le dogme de Chalcédoine} ce qui fait dire que certains théologiens disent qu'il est tout à fait orthodoxe.

\begin{quote}
    « Quand je considère, dans les descriptions tronquées de sa vie, la sainte figure de celui qui est le sublime auteur de ce qu’il y a jusqu’ici de plus magnifique dans la religion, ce que j’admire ce n’est pas la pureté de sa doctrine morale ; celle-ci n’a fait après tout qu’exprimer ce que tous les hommes parvenus à la conscience de leur nature spirituelle ont de commun avec lui, et à quoi ne peuvent ajouter une plus grande valeur ni le fait de l’exprimer, ni celui d’être le premier à l’exprimer (…). Tout cela n’est que choses humaines. Mais ce qui est véritablement divin, {c’est la magnifique clarté qu’a revêtue dans son âme la grande idée qu’il était venu représenter, l’idée que tout ce qui est fini a besoin, pour sa liaison avec la Divinité, de médiations supérieures.} (…) Considérons seulement la vivante intuition de l’Univers\sn{ie Dieu}, qui remplissait toute son âme, sous la forme parfaite sous laquelle nous la trouvons en lui. Si tout fini, pour ne pas s’éloigner toujours plus de l’Univers et ne pas aller s’éparpillant dans le vide et le néant, pour maintenir sa liaison avec le Tout et s’élever à la conscience de cette liaison, a besoin de la médiation d’un élément supérieur, s’il en est ainsi, cet élément médiateur, qui ne doit pas avoir lui-même à son tour besoin d’une médiation, ne peut absolument pas n’être que fini ; \textbf{il doit appartenir aux deux règnes : il doit participer de la nature divine de même et dans le même sens qu’il participe de la nature finie.} Or, que voyait-il autour de lui qui ne fût fini et soumis au besoin de la médiation ? Et où se trouvait un principe de médiation en dehors de Lui ? Personne ne connaît le Père que le Fils, et celui auquel Il veut le révéler. Cette conscience du caractère unique de sa religiosité, du caractère primordial de sa conception, et de sa force dont celle-ci était douée pour se communiquer et susciter de la religion, a été chez lui à la fois la conscience et de sa fonction médiatrice et de sa divinité » (Schleiermacher, Discours, 316-317).
\end{quote}

\paragraph{Jésus n'est pas un médiateur exclusif pour Schleiermacher} Il ouvre l'idée que la figure du médiateur dépasse la figure historique de Jésus. Médiateur \textit{par excellence} mais non unique.

\begin{quote}
    « Avec cette foi en lui-même, qui peut s’étonner qu’il ait été certain non seulement d’être le médiateur pour beaucoup d’êtres, mais encore de laisser derrière lui une grande école, qui déduirait de sa religion à Lui sa religion à elle, toute semblable (…). Mais \textbf{il n’a jamais affirmé être l’unique objet} de l’application de son idée, être le seul médiateur, et il n’a jamais confondu son école avec sa religion. (…) Aujourd’hui encore il devrait en être ainsi : celui qui pose cette même intuition comme base de sa religion est un chrétien, sans considération d’école, qu’il déduise historiquement sa religion de lui-même ou de n’importe qui d’autre. Le Christ n’a jamais donné les intuitions et les sentiments qu’il pouvait communiquer lui-même comme contenant tout ce que devait embrasser la religion qui devait sortir de son intuition fondamentale ; il a toujours engagé à tenir compte de la vérité qui viendrait après lui » (Discours, 318-319).
\end{quote}

\subsection{La christologie libérale de A. von Harnack (1851-1930)}  
\paragraph{L’essence du christianisme} 
Publication de son cours à Humbold à tous les étudiants de Humbold (Berlin).  
\mn{HARNACK, A. (von), L’Essence du christianisme, tr. par J.-M. TÉTAZ, Genève 2015.
}
On ne peut réduire l'essence du Christianisme aux Evangiles mais aussi à toute l'influence qu'il a eu par la suite. Il faut prendre toute l'histoire. \textit{Pour connaître Jésus, il faut regarder les chrétiens aujourd'hui}.
Plus une personnalité est impressionnante, plus il \textit{impressionne} d'autres et donc c'est en voyant comment les autres ont été touchés, qu'on peut accéder à l'essence de cette personnalité.



\begin{quote}
    « Toute grande personnalité influente ne révèle une partie de son essence qu’en ceux sur lesquels elle agit. On peut même dire que plus une personnalité est impressionnante et plus elle intervient dans la vie intérieure d’autres personnes, moins il possible de reconnaître la
totalité de son essence à ses seules paroles et ses seuls faits et gestes (…) Des forces ont été libérées non une seule fois, mais toujours à nouveau, alors nous devons aussi prendre en compte toutes les productions ultérieures de son esprit. Car ce n’est pas d’une d’ ‘doctrine’ qu’il s’agit, qui aurait été transmise et répétée de façon monotone, ou modifiée arbitrairement, mais bien d’une vie qui, toujours ranimée, brûle de son propre feu » (Harnack, L’essence, 93- 94).
\end{quote}


{Discerner le Christ dans la vie : image du noyau du fruit} Il faut dépouiller le fruit. 

\begin{quote}
    « Dans la prédication de Jésus à propos du Royaume de Dieu, (l’historien) doit séparer ce qui est propre de ce qui est transmis, le noyau de l’écorce (…). Le Royaume de Dieu vient lorsqu’il vient vers les individus, lorsqu’il pénètre dans leur âme et en prend possession. Le Royaume de Dieu est certes domination de Dieu – mais c’est la domination du Dieu saint dans les cœurs des individus, c’est Dieu lui-même avec sa force. Tout drame au sens extérieur du monde, au sens de l’histoire du monde, a disparu ; tout ce qui concernait une expérience future extérieure a été englouti » (Harnack, L’essence, 122).
\end{quote}
\paragraph{La prédication de Jésus} On peut résumer en (p.119):
\begin{itemize}
    \item Le Royaume de Dieu et sa venue
    \item Dieu le Père et la valeur infinie de l’âme humaine
    \item La justice supérieure et le commandement d’amour
\end{itemize}


\paragraph{Le Royaume de Dieu et sa venue} Non pas un royaume exterieur mais un royaume intérieur et individualiste. Il ne s'agit pas de trouver le Règne (pas de dimension sociale). Ce Royaume vient vers les personnes humiliées.




\paragraph{Dieu le Père et la valeur infinie de l’âme humaine}  filiation divine ("inscrit au ciel", "cheveux par terre"). Le Père est la providence : 

\begin{quote}
    « Qui a le droit de dire ‘mon Père’ à l’être qui régit le Ciel et la Terre est lui-même élevé au- dessus du Ciel et de la Terre et possède une valeur qui est plus haute que tout le mystère du monde » (Harnack, L’essence, 129).
\end{quote}

Il ne fait pas la différence entre Jésus et Père et les disciples. L'âme a donc une valeur infinie quand elle se tourne vers Dieu car elle devient "fils bien-aimé". La filiation entre Dieu et Jésus, \textit{unique} et la notre, \textit{adoptive} n'est pas différenciée ici.




\paragraph{La justice supérieure et le commandement d’amour}  

\begin{quote}
    « L’amour du prochain est sur terre la seule façon de mettre en œuvre l’amour de Dieu qui vit dans l’humilité » (Harnack, L’essence, 134).
\end{quote}
Il essaye de dire l'essentiel, le reste est ajouté.

\paragraph{La christologie}   Que devient la christologie ? Quelle auto-compréhension de Jésus ? Pour Harnack, il s'agit uniquement de "garder ses commandements". \textit{Décentrement} de Jésus qui renvoie les disciples au Père. Il va conclure que : 
\begin{quote}
    « Sa conscience d’être le Fils de Dieu n’est donc rien d’autre que la conséquence pratique de la connaissance de Dieu comme le Père et comme son Père (…). Jésus est persuadé de connaître Dieu comme personne ne l’a connu avant lui ; et il sait qu’il a la vocation de communiquer à tous les autres en parole et en acte cette connaissance de Dieu – et donc la filialité divine. » (Harnack, L’essence, 169).
\end{quote}

On retourne ici à Schleiermacher : nous participons comme Jésus, au même niveau que lui.  

Il est en deçà d'une dogmatique classique : 
\begin{quote}
    « Combien s’éloigne-t-on alors de ses idées et des prescriptions lorsqu’on fait précéder l’Evangile d’une confession ‘christologique’ et qu’on enseigne qu’il faut d’abord avoir des idées correctes sur le Christ avant de pouvoir venir à l’Evangile ! C’est inverser l’ordre des choses. On ne peut avoir des idées et des doctrines ‘correctes’ sur le Christ que dans la mesure où l’on a commencé à vivre conformément à son Evangile » (Harnack, L’essence, 181).
\end{quote}

Critique assez forte de la dogmatique.


\subsection{La figure « libérale » de Jésus dissoute par la question eschatologique}  

\paragraph{Découverte récente et renforcement de l'eschatologie} Jésus ne croyait pas à un royaume intérieur :

\begin{quote}
    « Le Jésus de Nazareth, qui vint comme un Messie, qui annonça la moralité du royaume de Dieu, qui fonda le royaume des cieux sur la terre et qui mourut afin de consacrer son œuvre, n’a jamais existé. C’est une figure qui a été esquissée par le rationalisme, rendue vivant par le libéralisme et habillée historiquement par la théologie moderne » (A. \textsc{Schweitzer}, Geschichte der Leben-Jesu-Forschung, 620.)
\end{quote}
%--------------------------------------- 
\section{Conclusion}






 
