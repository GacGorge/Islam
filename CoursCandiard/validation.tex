\chapter{Validation}


\begin{quote}
Le devoir, d'environ 10 000 caractères (espaces compris), est à rendre
avant le 16 mai.
\end{quote}
\paragraph{Sujet} : Choisissez deux preuves de l’existence de Dieu développées par des auteurs
musulmans classiques ; comparez-les et montrez leurs implications théologiques.

\hypertarget{ressources}{%
\section{Ressources}\label{ressources}}

\begin{quote}
- La preuve développée par Abū al-Ḥasan al-Ašʿarī a été donnée dans la
séquence 4 ;
\end{quote}

\begin{itemize}
\item
  \begin{quote}
  La preuve proposée par le théologien ašʿarite al-Ǧuwaynī (m. en 1085)
  est donnée à la suite de ce document ;
  \end{quote}
\item
  \begin{quote}
  La preuve du philosophe Avicenne fait partie des preuves les plus
  célèbres et les plus commentées : vous devriez la trouver sans peine
  dans la littérature secondaire.
  \end{quote}
\end{itemize}

\begin{quote}
Il vous est naturellement possible d'en utiliser d'autres !
\end{quote}

\hypertarget{bibliographie-indicative}{%
\section{Bibliographie indicative}\label{bibliographie-indicative}}

\begin{quote}
H. Davidson, \emph{Proofs for Eternity, Creation and the Existence of
God in Medieval Islamic and Jewish Philosophy}, Oxford University Press,
1987.

W. Hallaq, ``Ibn Taymiyya on the Existence of God'', in \emph{Acta
Orientalia} 52 (1991), pp. 49-69

A. Shihadeh, ``The Existence of God'', in T. Winter, \emph{The Cambridge
Companion to Classical Islamic Theology}, pp. 197-217.
\end{quote}

\hypertarget{annexe}{%
\section{Annexe}\label{annexe}}

\begin{quote}
\textbf{PREUVE DE L'EXISTENCE DE DIEU} (Al-Ğuwaynī, \emph{Kitāb
al-Iršād}, 4 ; trad. Luciani)

Maintenant qu'il est prouvé que le monde est contingent et qu'il a eu un
commencement, il s'ensuit que le contingent peut exister ou ne pas
exister, et que quel soit le moment où il se produit, il aurait pu se
produire à un moment antérieur ; que l'existence du contingent aurait pu
être retardée de plusieurs heures au-delà de ce moment.

Si donc l'existence possible se produit, au lieu d'une prolongation
également possible de la non-existence, l'esprit saisit, comme une chose
évidente, que (pour se produire) l'existence a eu besoin d'une principe
déterminant (\emph{muḫaṣṣiṣ}) qui détermine sa réalisation. C'est là une
chose qui apparaît nécessairement, sans qu'il y ait besoin de faire des
distinctions ou d'employer le raisonnement.

Une fois admis le principe général que le contingent exige un principe
déterminant, il faut nécessairement que ce principe soit : ou bien une
cause nécessitant la réalisation de la contingence, comme la cause
nécessite son effet ; ou bien une force physique, comme le pensent les
naturalistes ; ou bien enfin un agent libre. Or il est faux que ce
principe déterminant agisse à la manière d'une cause. La cause en effet
nécessite son effet d'une façon simultanée. Si on supposait que le
principe déterminant fût une cause, celle-ci serait forcément ou
éternelle, ou contingent. Dans le premier cas, elle aurait dû
nécessairement provoquer l'existence du monde de toute éternité, ce qui
conduirait à admettre l'éternité du monde. Or nous avons fourni les
preuves de sa contingence. Si le principe était contingent, il aurait
lui-même besoin d'un principe déterminant, et ainsi de suite à l'infini.

Quant à ceux qui prétendent que le principe déterminant est une force
physique, leur théorie
est inadmissible. {[}\ldots{]}

S'il est faux, par conséquent, que le principe déterminant du contingent
soit une cause nécessitante, ou une force physique qui lui donne par
elle-même l'existence, mais involontairement, il s'ensuit d'une manière
certaine que le principe déterminant des choses

contingentes est un agent qui agit sur elles librement, qui leur assigne
spécialement, en les produisant, certains attributs et certains moments.
\end{quote}
