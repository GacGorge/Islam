\chapter{Matériaux pour validation}

 

\begin{quote}
Il vous est naturellement possible d'en utiliser d'autres !
\end{quote}

\hypertarget{bibliographie-indicative}{%
\section{Bibliographie indicative}\label{bibliographie-indicative}}

\begin{quote}
H. Davidson, \emph{Proofs for Eternity, Creation and the Existence of
God in Medieval Islamic and Jewish Philosophy}, Oxford University Press,
1987.

W. Hallaq, ``Ibn Taymiyya on the Existence of God'', in \emph{Acta
Orientalia} 52 (1991), pp. 49-69

A. Shihadeh, ``The Existence of God'', in T. Winter, \emph{The Cambridge
Companion to Classical Islamic Theology}, pp. 197-217.
\end{quote}
 

% -------------------------------------------------------------
\section{Al Guwayni} 

\subsection{la preuve}
\begin{quote}
\textbf{PREUVE DE L'EXISTENCE DE DIEU} (Al-Ğuwaynī, \emph{Kitāb
al-Iršād}, 4 ; trad. Luciani)

Maintenant qu'il est prouvé que le monde est contingent et qu'il a eu un
commencement, il s'ensuit que le contingent peut exister ou ne pas
exister, et que quel soit le moment où il se produit, il aurait pu se
produire à un moment antérieur ; que l'existence du contingent aurait pu
être retardée de plusieurs heures au-delà de ce moment.

Si donc l'existence possible se produit, au lieu d'une prolongation
également possible de la non-existence, l'esprit saisit, comme une chose
évidente, que (pour se produire) l'existence a eu besoin d'une principe
déterminant (\emph{muḫaṣṣiṣ}) qui détermine sa réalisation. C'est là une
chose qui apparaît nécessairement, sans qu'il y ait besoin de faire des
distinctions ou d'employer le raisonnement.

Une fois admis le principe général que le contingent exige un principe
déterminant, il faut nécessairement que ce principe soit : ou bien une
cause nécessitant la réalisation de la contingence, comme la cause
nécessite son effet ; ou bien une force physique, comme le pensent les
naturalistes ; ou bien enfin un agent libre. Or il est faux que ce
principe déterminant agisse à la manière d'une cause. La cause en effet
nécessite son effet d'une façon simultanée. Si on supposait que le
principe déterminant fût une cause, celle-ci serait forcément ou
éternelle, ou contingent. Dans le premier cas, elle aurait dû
nécessairement provoquer l'existence du monde de toute éternité, ce qui
conduirait à admettre l'éternité du monde. Or nous avons fourni les
preuves de sa contingence. Si le principe était contingent, il aurait
lui-même besoin d'un principe déterminant, et ainsi de suite à l'infini.

Quant à ceux qui prétendent que le principe déterminant est une force
physique, leur théorie
est inadmissible. {[}\ldots{]}

S'il est faux, par conséquent, que le principe déterminant du contingent
soit une cause nécessitante, ou une force physique qui lui donne par
elle-même l'existence, mais involontairement, il s'ensuit d'une manière
certaine que le principe déterminant des choses
contingentes est un agent qui agit sur elles librement, qui leur assigne
spécialement, en les produisant, certains attributs et certains moments.
\end{quote}
 
 
 \begin{Synthesis}
 -Dieu comme principe déterminant. Il y a besoin d'un principe, démo par récurrence. Ressemble à la première démo d'Avicenne
 Comment Al gazali juge il cette démo ? 
 \end{Synthesis}



\subsection{ce qui wiki en dit}
Par exemple, il tente de démontrer l'existence de Dieu par la raison. Ce fait, à lui seul, montre qu'il a utilisé les outils de la philosophie. En outre, son raisonnement indique qu'il connaissait les concepts néo-platoniciens. Il commence par le constat de la contingence du monde : les choses auraient pu aussi bien ne pas être, ou être autrement. 
\begin{Synthesis}
 est ce la même chose d'Avicenne
\end{Synthesis}

Il faut donc supposer une intervention arbitraire, celle d'un créateur qui a fait un choix, celui de faire exister le monde, à un moment et sous une forme plutôt qu'une autre. L'idée d'un Dieu artisan rappelle le démiurge platonicien ; le concept de contingence a été introduit par Avicenne dans sa propre preuve de l'existence de Dieu. 


Pour Frank Griffel, l'influence directe d'Avicenne ne fait pas de doute : 
\begin{quote}
    « Al-Juwaynî fut le premier théologien musulman à étudier sérieusement les livres d'Avicenne »45(p. 47).
\end{quote} 
Mais selon Jan Thiele, al-Juwayni a pu aussi être inspiré par le mu'tazilite al-Basrī12. Paul Heck tranche ce débat en soulignant que l'atmosphère intellectuelle dans laquelle baignait al-Juwaynî était déjà imprégnée de l'influence d'Avicenne3. 
\begin{Synthesis}
Et Al-Juwaynī, s'il n'a pas inventé cette preuve, est du moins le premier des ash'arites à y avoir recours35.
\end{Synthesis}


Averroès en reconnaît l'originalité dans Al-Kashf ʿan manāhij al-adilla fī ʿaqāʾid al-milla. Il la réfute cependant, en mettant en doute la prémisse sur laquelle toute l'argumentation repose : l'ide que le monde est contingent. Dieu, en effet, dans sa sagesse, a pu choisir de créer le seul monde qui comportait le plus de perfection, et qui par conséquent s'imposait35.
\begin{Synthesis}
A la différence d'Averroes, ce n'est pas le meilleur des mondes. Explique le besoin de recourir au Coran pour éclairer certains comportements, en particulier des élites
\end{Synthesis}



\begin{Synthesis}
Al As'ari
- L'homme est changeant
- référence explicite au Coran sur ce que c'est l'homme
- après cette démonstration, ce qu'on sait de l'homme : c'est le Coran qui nous le donne mais aussi la raison.
\end{Synthesis}


\section{Démonstration d'avicenne}
voir aussi \href{https://books.google.fr/books?id=6jcTAQAAMAAJ&pg=RA1-PA6&lpg=RA1-PA6&dq=Inquirit+enim+universale+et+particulare,+potentiam+et+effectum,+possibile+et+necesse,+et+cetera&source=bl&ots=caRb6_MMwQ&sig=ACfU3U0jQ0xx5Be712aqaVYgR5qLnQa5Ng&hl=fr&sa=X&ved=2ahUKEwiGq8qvyNT3AhWvy4UKHXIHB5cQ6AF6BAgNEAM#v=onepage&q=vehementia&f=false}{Texte latin}

\begin{quote}
    la métaphysique d'avicenne fait partie du \textit{livre la la guérison}, Kitâb al-Shifâ'). La philosophie est une thérapeutique : la médecine ne vise que la guérison des corps. N'a-t-on pas besoin aussi d'une science capable de guérir l'esprit des hommes, l'esprit de ceux qui vivent dans le doute, l'incertitude et l'erreur ? 
\end{quote}
A la différence des intentions premières de la métaphysique, la logique utilise les intentions secondes. 

Suppose la physique : génération et corruption, altérité, lieu , temps sont justifiés en physique. 

\begin{quote}
    D'ailleurs, comment arrive t on à saisir les causes ? On ne peut y arriver qu'en partant des êtres causés données dans l'expérience. on aboutit ainsi à la même impasse que pour l'existence de Dieu : comment les causes au sens absolu pourraient elles être l'objet de la métaphysique puisque leur existence n'est pas donnée comme une évidence initiale, mais doit être démontrée.
    \ldots ce qu'on constat par la perception sensible, c'est la concomitance de certains phénomènes. On ne peut en conclure directement que l'un des deux est la cause de l'autre; surtout dans le cas où des phénomènes sont fréquemment associés, on sera porté à les croire unis par un lien de causalité. Pourtant la fréquence de cette union ne nous fournit concernant ce lien causal qu'une certaine probabilité, elle ne conduit pas à une connaissance certaine. 
\end{quote}

\begin{quote}
   qu'on ne s'y méprenne pas toutefois, cette doctrine d'Avicenne ne signifie pas que de façon absolue la puissance serait antérieure à l'acte. p 51
   
   L'existant possible, quant à son essence, restera toujours ce qu'il est, à savoir un existant possible : le fait d'être un existant possible n'est pas un caractère accidentel et passager (meta 1, 7 quicquid enim est possibile esse, respectu sui, semper est possibile esse.). Il est donc exclu que l'existant possible devienne un jour un être de soi nécessaire (p 52)
   il faut une cause extérieure pour devenir nécessaire de façon permanente. Durant une période limité, s(sans cause extérieure) à la condition qu'il y ait en lui un principe matériel. 53
\end{quote}

\paragraph{comment peut on concevoir le rapport entre le possible et sa cause ?}
\begin{quote}
    Avicenne répond que le possible n'existe que si par rapport à sa cause il est nécessaire (Meta I,6 possibile esse per se habet causam). Le possible considéré en lui-même n'est déterminé à l'existence ni à la non-existence. il faut cependant qu'il y ait un facteur qui le détermine dans un sens ou dans l'autre : ne facteur ne peut être que quelque chose de distinct de la nature du possible, une cause extérieure. (54)
    C'est donc cette cause qui déterminera le possible à l'une des deux alternatives, car si elle n'était pas capable de le faire, il faudrait faire appel à une troisième cause et ainsi à l'infini. 
    La thèse d'Avicenne signifie donc que le possible existera si sa cause le fait exister et qu'il n'existera pas si sa cause le fait pas exister.

\end{quote}
\paragraph{    Cette position ne conduit elle pas à un déterminisme universelle ?}
\begin{quote}
Si l'exisence ou la non existence dépendent entièrement de la cause extérieure, ne faut il pas dire que tout est fixé et qu'il n'y a plus aucune marge laissée à la contingence ? nous ne le croyons pas : lorsque notre auteur parle de différentes espèces de puissance il fait une distinction entre les oeuvres de la nature, ce qui est acquis par habitude, les produits de l'art et les résultats du hasard; cependant l'acquisition d'une habitude et l'apprentissage d'une capacité technique se font de la même manière. Si le possible est amené à l'existence sous l'action d'une cause, il reste à préciser de quelle nature est cette action : activité déterminée ou initiative autonome et libre.  ... ceci n'exclut pas cependant que l'intervention de cette cause puisse être une initiative autonome.55
    
    Le possible ne trouve pas sa raison d'être en lui-même, alors que le nécessaire la possède dans sa constitution même. En outre, ce qui est de soi nécessaire ne peut être égale ou équivalente à une autre existence (55 ( 1,6)
    Simplicité de l'existant nécessaire (il s'oppose au possible qui inclut toujours une certaine composition que ce soit celle de nbature et d'existence ou celle de matière et de forme. ). 
    Avicenne en conclut que l'existant nécessaire n'est pas relatif, pas changeant (il ne peut passer de puissance à acte puisque sous aucun aspect il n'est en puissance), il n'est pas multiple  et il ne partage l'existence qui lui est propre avec aucun autre être. 56
    
    Tous ces arguments d'avicenne se réduisent en somme à une considération fondamentale : \textsc{l'être nécessaire dont il est question est un être qui est de soi nécessaire}. par son essence, il est ce qu'il est à savoir existant nécessaire. 62
    C'est pourquoi Aristote admet un acte pur qui est à la base du devenir qui se produit dans le monde. Avicenne a voulu approfondir cette théorie d'Aristote.
    

\end{quote}

\paragraph{la question de l'être}

\begin{quote}
    la notion d'être générique ou spécifique, c'est à dire une notion capable d'être déterminée par l'addition de certains caractères ou qualificatifs.
    être ; sens univoque ? caractère analogique de l'être (Thomas d'Aquinà). Le problème est important car la validité et le sens du langage métaphysique sur Dieu en dépendent (65)
    Pour avicenne, sens générique et univoque.
\end{quote}

% ----------------------------------------------------------------------------

\subsection{Une généalogie de la métaphysique moderne à l'époque de Duns Scot (XIIIe - XIVe siècle)}
\url{https://www.cairn.info/etre-et-representation--9782130504566-page-327.htm}

Être et représentation
Une généalogie de la métaphysique moderne à l'époque de Duns Scot (XIIIe - XIVe siècle)
Par Olivier Boulnois
Année : 1999
Pages : 544
Collection : Épiméthée
Éditeur : Presses Universitaires de France



40Pour Avicenne, l’idéal de la connaissance de Dieu est celui de la plus haute science, qui procéderait de manière géométrique, par une déduction à partir de propositions primitives évidentes par elles-mêmes, suivant le modèle des Éléments d’Euclide. Avicenne forme le projet d’une métaphysique en soi, qui atteindrait l’évidence de Dieu à la mesure de son intelligibilité, en apercevant la nécessité de son existence dans la contemplation de son essence. Comme la démonstration géométrique, elle atteindrait ses conclusions à partir de la définition de l’essence. Mais la réalité de la preuve ne peut se maintenir au niveau de cet air raréfié. Elle doit être obtenue à partir de l’expérience. Même 
\begin{quote}
    « s’il est presque évident (manifestum) par soi pour l’intelligence que tout ce qui commence a un principe, ce n’est pas pour cela que [cette proposition] doit être évidente par soi à la manière dont beaucoup de réalités géométriques prouvent les autres dans le livre d’Euclide » \sn{Avicenne, Philosophia prima I, 1 (I, 8, 33-36) : « Si paene fuerit manifestum per se apud intelligentiam quod quidquid coepit habet principium aliquod, ideo debet esse manifestum per se, sicut multa ex rebus geometricis per quae probantur cetera in libre Euclidis… »}.
\end{quote} 


\begin{Synthesis}
Une démonstration géométrique. 
\end{Synthesis}
41 Malheureusement, cette métaphysique more geometrico, qui irait a priori, des principes aux conséquences, ne nous est pas accessible.
\begin{quote}
     « Mais nous, en raison de la faiblesse de notre âme, nous ne pouvons pas commencer par la voie démonstrative elle-même, qui procède des principes aux conséquents, et de la cause au causé, sauf dans certains ordres d’universalité au sein de ce qui est, sans descendre dans le particulier (sine praecisione). » \sn{Philosophia prima I, 3 (I, 23, 37-41).} A défaut, nous pouvons passer de l’expérience à ses conditions de possibilité rationnelles, en discernant sous le sensible des ordres d’universalité \sn{In Met. I, q. 1, [6] 22 (III, 22) : « Deum esse desperatum cognosci non est, nec quaesitum in alia scientia, nec in ista secundum se, quamvis quoad nos fait notum ex effectibus, sicut procedit ratio. Potest enim aliquid secundum se notius, fieri nobis notum ex aliis nobis notioribus. »}.
\end{quote}


42En effet, nous avons des notions élémentaires, inscrites d’emblée dans l’intellect : chose, étant, nécessaire (necesse) \sn{Philosophia prima I, 5 (I, 31-32, 2-4).} . Avicenne préconise d’utiliser des couples de propriétés relatives et disjonctives : 
\begin{quote}
    « Cette science étudie les notions (intentiones) qui ne proviennent pas des accidents propres de ces causes en tant que causes. Elle étudie en effet l’universel et le particulier, la puissance et l’acte (effectum), le possible et le nécessaire, etc. » \sn{Philosophia prima I, 1 (I, 6, 12-17) : « Inquirit enim universale et particulare, potentiam et effectum, possibile et necesse, et cetera. »}
\end{quote}
 A chaque fois, le terme aperçu dans la créature permet d’inférer l’existence de son corrélat, qui est Dieu. Selon ce schéma, on constituera a priori des différences d’ordre au sein de l’étant. Dans chaque cas, il y aura un antérieur et un postérieur, Dieu et la créature prise sous un certain « ordre d’universalité », selon le vocabulaire même d’Avicenne. Cette dualité peut se dire en une série de couples transcendantaux, qui permettent tous d’inférer l’existence de Dieu. 
 \begin{Synthesis}
 Discussion sur les attributs de Dieu ? On aperçoit un terme dans la créature et on regarde le corrélat divin.
 Necessaire : le plus important; "la véhemence / l'intensité à exister"
 \end{Synthesis}
 
 – Or le plus haut objet de pensée est le nécessaire (necesse), car l’objet de la science est par excellence ce qui ne peut pas être autrement qu’il n’est ; et le nécessaire garde des affinités avec l’être transcendantal, car il signifie la « vehementiam essendi », la détermination à exister, l’affirmation de l’être. Il est plus connaissable que le possible et l’impossible, car l’être est plus connu que le non-être, l’être étant connu par soi-même et le non-être par lui \sn{Philosophia prima I, 5 (I, 41, 79-82) ; cf. Aristote, Seconds analytiques I, 2, 71 b 15.}. – Le necesse, comme nécessairement être, est convertible avec l’actualité d’être. Mais à cette continuité il faut ajouter une disjonction : au sein de l’étant, certains sont par eux-mêmes possibles, non nécessaires, tandis que les autres existent par eux-mêmes nécessairement (necesse esse per se). Le nécessairement-être par soi est sans cause, il est nécessaire par lui-même, incomparable, sans égal \sn{Philosophia prima I, 6 (I, 43, 9-18).}. 
 
 – De plus, le concept de nécessairement-être permet de démontrer l’unicité de son objet \sn{Cela peut être prouvé par l’absurde : s’il y a plusieurs nécessairement êtres distincts, ils sont distincts par un ajout fait à leur essence. Cet élément est-il nécessaire à leur nécessité propre ? Si oui, ils n’en diffèrent donc pas. Sinon, il faut distinguer deux aspects : la nécessité additive qui caractérise l’un d’eux est ce qui l’affecte et le rend nécessaire, alors qu’ils ont en eux-mêmes un fond commun non-nécessaire. Ce qui veut dire que l’autre n’est pas nécessaire, hypothèse contraire aux prémisses.} : celui qui est nécessairement, comme détermination à exister par lui-même, est unique, et réciproquement, tout ce qui est nécessairement par soi est lui-même \sn{Philosophia prima I, 7 (I, 54, 33-35) : « Si autem fuerit per se ipsum quod ipsum est necesse esse, erit idem ipsum tunc quicquid est necesse esse. »}. Le nécessairement-être est un étant dont l’essence est la nécessité ; il est donc numériquement unique ; son mode d’être n’est communicable à aucun autre \sn{Philosophia prima I, 7 (I, 53, 20). Comme le dira justement Schelling, « Dieu n’est pas simplement l’étant nécessaire, mais il est nécessairement l’étant nécessaire » ; Philosophie der Offenbarung, I, 8, SW, t. XIII, p. 159 ; tr. fr. Philosophie de la Révélation (I, 185).}. La voie part donc de notre premier concept, celui de l’étant, et aboutit nécessairement à l’unique nécessairement-être.



43Peut-on inventer une démonstration plus haute que celle d’Avicenne, conforme à son projet de science métaphysique en soi ? La science métaphysique, qui démontre l’existence de Dieu, peut-elle procéder a priori, comme la géométrie ? Bonaventure a entrevu la possibilité d’une démonstration a priori de Dieu, mais sans lui accorder un privilège particulier dans ses nombreux arguments en faveur de l’existence de Dieu \sn{QD De mysterio Trinitatis q. 1, a. 1 (V, 45 a) : trois voies permettent de démontrer l’existence de Dieu ; 1 / l’innéité de Dieu dans la pensée : il y a dans toutes les âmes rationnelles une certitude de l’existence de Dieu ; 2 / la relation du postérieur à l’antérieur ; 3 / l’argument anselmien : l’existence de Dieu est une vérité telle qu’on ne peut pas penser Dieu sans penser qu’il existe.}. Selon la seconde voie vers Dieu, fondée sur la relation de l’antérieur et du postérieur, toute créature implique l’existence de son principe et proclame son existence comme une vérité indubitable. 
\begin{quote}
    « S’il y a un étant postérieur, il y aussi un étant antérieur, car le postérieur n’existe que par l’antérieur (a priori) : si donc l’universalité des postérieurs existe, il est nécessaire qu’un premier existe. Si donc il est nécessaire de poser qu’il y a quelque chose d’antérieur et de postérieur dans les créatures, il est nécessaire que l’universalité des créatures implique et proclame le premier principe. » [75]
\end{quote}
 Ce texte joue sur deux sens de l’apriorité : être au préalable, venir d’un antérieur. Retourner vers l’intelligible, c’est pour Bonaventure remonter à la fois dans l’ordre de l’être et dans celui de l’être, se hausser vers l’universel et retrouver le divin, puisque l’intelligible pur est à la fois la condition de ma pensée (illumination) et celle de l’être (émanation). L’ambiguïté de la voie a priorivient du fait que Bonaventure admet aussi l’innéité du concept de Dieu dans l’âme : si l’a priori est atteint par inférence à partir du postérieur, il est aussi d’avance (a priori) connu comme son corrélat. En donnant une interprétation ontologique de l’a priori, Bonaventure identifie l’être premier et l’être universel [76] : l’être par excellence (Dieu) et l’être comme premier objet de la pensée coïncident tangentiellement, dans ce qui constitue littéralement le principe de l’analogie.

44Mais Bonaventure ne s’attarde pas sur les questions de méthode impliquées par une telle démarche, ni sur la relation entre cet argument et les suivants : il les énumère et les juxtapose sans s’interroger sur le fait que le premier donne une structure générale alors que les autres en sont des réalisations particulières. L’a priori est donc entrevu en un moment décisif de la seconde voie, mais il n’est pas pensé comme son principe constitutif [77]. Or cette voie repose sur des couples de concepts a priori tirés d’un aspect du réel, et permettant d’inférer l’existence de leur terme premier : celui-ci est donc connu, non comme cause, à la manière d’Aristote ou de Thomas d’Aquin, mais comme principe, en tant que premier dans un couple de différences construit a priori, par une relation d’antéropostériorité. La « voie de la raison » remonte de l’être imparfait à l’être parfait, mais elle peut prendre pour point de départ de multiples concepts : les créatures sont par essence déficientes, mais par essence elles impliquent le concept de la perfection correspondante [78]. L’intellect pense partir de données purement sensibles, l’être muable, relatif, composé, contingent, mais ces imperfections apparaissent comme imparfaites parce qu’il possédait déjà, antérieurement, a priori, le concept des perfections qui les mesurent. Lorsqu’il entreprend de démontrer l’existence de Dieu, l’intellect prend conscience du fait qu’il la connaissait déjà.

 

[75]
De mysterio Trinitatis q. 1, a. 1, arg. 11 (V, 46b) : « Si est ens posterius, est et ens prius, quia posterius non est nisi a priori : si ergo est universitas posteriorum, necesse est, esse ens primum. Si ergo necesse est ponere aliquid esse prius et posterius in creaturis ; necesse est universitatem creaturarum inferre et clamare primum principium. »
[76]
L’Itinéraire de l’esprit vers Dieu V, 3, identifie l’être pur (esse purum) et l’être divin (esse divinum) : « Esse nominat ipsum purum actum entis : esse igitur est quod primo cadit in intellectu […]. Sed hoc non est esse particulare, […] nec esse analogum […] restat igitur, quod illud esse est esse divinum » (Duméry, p. 84). Collationes in Hexaemeron X, 10< | > : « Primum speculabile est Deum esse. Primum nomen Dei est esse, quod est manifestissimum et perfectissimum, ideo primum ; unde nihil manifestius » (tr. fr. p. 269). Bonaventure joue sur deux sens de « esse » (être/exister) : rien n’est plus manifeste que l’existence de Dieu, parce que l’être est le nom le plus propre de son essence, et sur deux sens de « premier » : le plus manifeste (manifestissimum) et le plus excellent (perfectissimum).



Avicenne
- démonstration géometrique : pas besoin du coran pour démontrer Dieu. 
- Dieu est accessible à l'homme sans le Coran
- mais pas très accessible (la démonstration de Al Guwayni est trop simple ?) voir si il lui répond vraiment

- parler de l'intellect 



\hypertarget{annexe}{%
\section{Annexe}\label{annexe}}

\section{Les différentes preuves de Dieu EU}
Au long de l'histoire de la philosophie, les preuves de l'existence de Dieu varient selon le type d'argument choisi pour les fonder. Le philosophe peut partir de l'expérience qu'il fait de la contingence du monde, et en inférer, se plaçant à différents points de vue, l'existence nécessaire d'un Dieu soutenant dans l'être et expliquant à la pensée la contingence de l'expérimenté. C'est ainsi que tout mouvement (entendons tout devenir) impliquerait, d'objet mû en moteur et de moteur en objet mû, la nécessité d'un premier moteur immobile, suprême cause. Il en va pareillement pour l'existence même de ce qui se meut, qui impliquerait l'agir absolu d'une pure existence subsistante, seule capable de faire passer tout être du simple et indigent pouvoir d'exister à l'existence de fait. Ainsi raisonne Aristote (Physique, VIII ; Métaphysique, Λ), repris, à travers Avicenne et Maimonide, par Thomas d'Aquin (ce sont les trois premières voies du Contra Gentiles et de la Somme théologique). Tel est l'esprit des preuves dites cosmologiques ou a contingentia mundi.

\begin{Synthesis}
Preuve cosmologique : de Aristote à Avicenne : il faut une cause.
\end{Synthesis}


Le philosophe peut aussi considérer l'ordre du monde, la finalité qu'il y discerne, la beauté qu'il y contemple, etc., et, se refusant à y voir l'effet du hasard, affirmer l'action suprêmement intelligente d'un Dieu organisateur ultime du cosmos. Cet argument, dit téléologique ou physico-théologique, s'enracine chez Platon et Aristote. Thomas d'Aquin l'utilise (cinquième voie de la Somme théologique) ; Leibniz ne le néglige point (Nouveaux Essais) ; Voltaire y acquiesce. Le philosophe peut encore être sensible au fait que les perfections qu'il constate dans le monde s'y manifestent selon des degrés et, de là, inférer la nécessaire existence d'un absolu divin de perfection. On trouverait des amorces de cet argument chez Platon et Aristote, chez les néo-platoniciens et ceux qui, comme Augustin (De libero arbitrio), ont subi leur influence. Anselme de Cantorbéry (Monologion), Thomas d'Aquin (quatrième voie de la Somme théologique), Descartes, Bossuet l'ont utilisé ou s'en sont inspirés. Ces différentes preuves n'en font qu'une, dans la mesure où elles ont en commun d'aller de l'expérience prise comme conséquence à son principe ; elles procèdent a posteriori. 
\begin{Synthesis}
Partir du monde et refuser d'y voir le hasard : Al Ashari (la boue ne se transforme pas en brique)
\end{Synthesis}
Mais certains penseurs inversent le processus et, considérant la seule idée de Dieu et ses notes constitutives, en infèrent l'existence nécessaire de ce Dieu sans qui, selon eux, il ne saurait y avoir d'idée de Dieu. Anselme de Cantorbéry (Proslogion), le premier, utilisa cet argument, qui fut retrouvé plus tard, sans doute au niveau de la critique qu'en fit le thomisme, par Descartes, lequel le présenta sous des formes originales (Discours de la méthode, 4 ; Méditation cinquième). 
\begin{Synthesis}
Idée de Dieu ne peut être sans Dieu : preuve ontologique pour Kant et en fait caché dans toutes les autres.
\end{Synthesis}
Leibniz l'utilisa selon ses propres perspectives (Nouveaux Essais, Monadologie). Procédant de l'idée à l'existence, cet argument a priori a été dit par Kant « ontologique ». À côté de ces preuves classiques, de nature logique, il faut mentionner la preuve dite morale, où la postulation d'un Dieu apparaît comme seule capable d'accomplir les requêtes de la conscience morale : c'est la position de Kant (Critique de la raison pratique), ainsi que le chemin philosophique vers Dieu que trace M. Nédoncelle à partir de l'existence, inexplicable autrement, selon lui, de l'ordre des personnes humaines.

Les diverses preuves de l'existence de Dieu ont toujours rencontré faveur et défaveur, soit en particulier, soit en bloc. 
\begin{quote}
    « La critique la plus cohérente et la plus ferme qui ait été jamais opposée aux preuves traditionnelles [...] est celle que Kant développe dans la Critique de la raison pure, et dont Hegel lui-même reconnaît qu'elle est la seule à avoir écarté de façon « scientifique » ces preuves » (D. Dubarle)
\end{quote}
. Kant, en effet, fait apparaître sous les preuves cosmologique et téléologique, apparemment autonomes, un recours subreptice et obligé à « cette malheureuse preuve ontologique » (Critique de la raison pure), elle-même dépourvue de toute valeur, dès lors que l'existence réelle de Dieu, qu'on est censé découvrir impliquée dans le concept de Dieu, n'est point, en fait, une perfection analytiquement déductible, mais bien une détermination extérieure au concept analysé, et d'un autre ordre que lui. Du concept à l'existence, la conséquence ne vaut pas. Thomas d'Aquin, après Gaunilon, avait d'ailleurs opposé à l'argument d'Anselme une objection analogue. Lui, du moins, accordait une valeur probante aux preuves a posteriori, où l'affirmation de Dieu venait tout naturellement à sa place au sein d'une représentation prégaliléenne de l'Univers, aujourd'hui sans rapport avec les modernes cosmologies.
\begin{Synthesis}
 Anselme et Avicenne : preuves a posteriori : représentation pregaliléenne de l'univers.
\end{Synthesis}

D'un point de vue plus général, le propos même de prouver l'existence de Dieu se voit opposer des objections de principe. Ce peut être en raison de la transcendance de l'objet, qui le fait échapper par définition à toute insertion dans un plan purement intellectuel. Le croyant Pascal (dont le pari n'entendait rien prouver) n'accordait aux preuves traditionnelles aucune valeur probante : « Une heure après, ils craignent de s'être trompés » (éd. Brunschvicg, 543) ; Kierkegaard, pas davantage. L'incroyant, enfin, peut-il être convaincu par une démarche rationnelle qu'il voit, chez le croyant, se déroulant tout entière portée par une foi préalable qu'il ne partage point ? Au croyant la foi donne non une image du monde, mais le monde tel qu'il est, avec Dieu en son centre ; cela, l'incroyant le sait ; il ne peut dès lors regarder la vision du monde engendrée par la foi autrement que sous l'angle d'une axiomatique.
\begin{Synthesis}
Objection de principe aux preuves de l'existence de Dieu qui ne respectent pas la transcendance de Dieu. Cf Levinas et homme Dieu
\end{Synthesis}

POUR CITER L’ARTICLE
Lucien JERPHAGNON, « DIEU PREUVES DE L'EXISTENCE DE », Encyclopædia Universalis [en ligne], consulté le 7 mai 2022. URL : http://www.universalis-edu.com.icp.idm.oclc.org/encyclopedie/preuves-de-l-existence-de-dieu/


\section{Preuve de l'existence de Dieu Avicenne Wikipedia}

Proof of the Truthful
From Wikipedia, the free encyclopedia



The Proof of the Truthful[1] (Arabic: \TArabe{ برهان الصديقين,} romanized: burhān al-ṣiddīqīn,[2] also translated Demonstration of the Truthful[2] or Proof of the Veracious,[3] among others) is a formal argument for proving the existence of God introduced by the Islamic philosopher Avicenna (also known as Ibn Sina, 980–1037). Avicenna argued that there must be a "necessary existent" (Arabic:\TArabe{ واجب الوجود,} romanized: wājib al-wujūd), an entity that cannot not exist.[4] The argument says that the entire set of contingent things must have a cause that is not contingent because otherwise it would be included in the set. Furthermore, through a series of arguments, he derived that the necessary existent must have attributes that he identified with God in Islam, including unity, simplicity, immateriality, intellect, power, generosity, and goodness.[5]

\begin{Synthesis}
La démonstration d'un Dieu par géométrie, puis les principales propriétés de Dieu.
A noter Levinas, l'homme Dieu, sur l'unité sans entrer dans l'ordre.
\end{Synthesis}
 
Critics of the argument include Averroes, who objected to its methodology, Al-Ghazali, who disagreed with its characterization of God, and modern critics who state that its piecemeal derivation of God's attributes allows people to accept parts of the argument but still reject God's existence. There is no consensus among modern scholars on the classification of the argument; some say that it is ontological while others say it is cosmological.[6]



Origin
The argument is outlined in Avicenna's various works. The most concise and influential form is found in the fourth "class" of his Remarks and Admonitions (Al-isharat wa al-tanbihat).[7] It is also present in Book II, Chapter 12 of the Book of Salvation (Kitab al-najat) and throughout the Metaphysics section of the Book of Healing (al-Shifa).[8] The passages in Remarks and Admonitions draw a distinction between two types of proof for the existence of God: the first is derived from reflection on nothing but existence itself; the second requires reflection on things such as God's creations or God's acts.[1][9] Avicenna says that the first type is the proof for "the truthful", which is more solid and nobler than the second one, which is proof for a certain "group of people".[10][11] According to the professor of Islamic philosophy Shams C. Inati, by "the truthful" Avicenna means the philosophers, and the "group of people" means the theologians and others who seek to demonstrate God's existence through his creations.[10] The proof then became known in the Arabic tradition as the "Proof of the Truthful" (burhan al-siddiqin).[2]

Argument
The necessary existent
Avicenna distinguishes between a thing that needs an external cause in order to exist – a contingent thing – and a thing that is guaranteed to exist by its essence or intrinsic nature – a necessary existent.[12] The argument tries to prove that there is indeed a necessary existent.[12] It does this by first considering whether the opposite could be true: that everything that exists is contingent. Each contingent thing will need something other than itself to bring it into existence, which will in turn need another cause to bring it into existence, and so on.[12] Because this seemed to lead to an infinite regress, cosmological arguments before Avicenna concluded that some necessary cause (such as God) is needed to end the infinite chain.[13] However, Avicenna's argument does not preclude the possibility of an infinite regress.[12][13]

Instead, the argument considers the entire collection (jumla) of contingent things, the sum total of every contingent thing that exists, has existed, or will exist.[12][13] Avicenna argues that this aggregate, too, must obey the rule that applies to a single contingent thing; in other words, it must have something outside itself that causes it to exist.[12] This cause has to be either contingent or necessary. It cannot be contingent, though, because if it were, it would already be included within the aggregate. Thus the only remaining possibility is that an external cause is necessary, and that cause must be a necessary existent.[12]

Avicenna anticipates that one could reject the argument by saying that the collection of contingent things may not be contingent. A whole does not automatically share the features of its parts; for example, in mathematics a set of numbers is not a number.[14] Therefore, the objection goes, the step in the argument that assumes that the collection of contingent things is also contingent, is wrong.[14] However, Avicenna dismisses this counter-argument as a capitulation, and not an objection at all. If the entire collection of contingent things is not contingent, then it must be necessary. This also leads to the conclusion that there is a necessary existent, the very thing Avicenna is trying to prove. Avicenna remarks, "in a certain way, this is the very thing that is sought".[14]

From the necessary existent to God
Bismillahir Rahmanir Rahim
Part of a series on
God in Islam
"Allah" in Arabic calligraphy
Allah Jalla Jalālah
in Arabic calligraphy
List
Allah
Names
Phrases and expressions
Theology
Oneness
Islamic creed
Transcendence
Denial of Divine attributes
Anthropomorphism
Allah-green.svg Islam portal • Category
vte
The limitation of the argument so far is that it only shows the existence of a necessary existent, and that is different from showing the existence of God as worshipped in Islam.[5] An atheist might agree that a necessary existent exists, but it could be the universe itself, or there could be many necessary existents, none of which is God.[5] Avicenna is aware of this limitation, and his works contain numerous arguments to show the necessary existent must have the attributes associated with God identified in Islam.[14]

For example, Avicenna gives a philosophical justification for the Islamic doctrine of tawhid (oneness of God) by showing the uniqueness and simplicity of the necessary existent.[15] He argues that the necessary existent must be unique, using a proof by contradiction, or reductio, showing that a contradiction would follow if one supposes that there were more than one necessary existent. If one postulates two necessary existents, A and B, a simplified version of the argument considers two possibilities: if A is distinct from B as a result of something implied from necessity of existence, then B would share it, too (being a necessary existent itself), and the two are not distinct after all. If, on the other hand, the distinction resulted from something not implied by necessity of existence, then this individuating factor will be a cause for A, and this means that A has a cause and is not a necessary existent after all. Either way, the opposite proposition resulted in contradiction, which to Avicenna proves the correctness of the argument.[16] Avicenna argued that the necessary existent must be simple (not a composite) by a similar reductio strategy. If it were a composite, its internal parts would need a feature that distinguishes each from the others. The distinguishing feature cannot be solely derived from the parts' necessity of existence, because then they would both have the same feature and not be distinct: a contradiction. But it also cannot be accidental, or requiring an outside cause, because this would contradict its necessity of existence.[17]

Avicenna derives other attributes of the necessary existent in multiple texts in order to justify its identification with God.[5] He shows that the necessary existent must also be immaterial,[5] intellective,[18] powerful,[5] generous,[5] of pure good (khayr mahd),[19] willful (irada),[20] "wealthy" or "sufficient" (ghani),[21] and self-subsistent (qayyum),[22] among other qualities. These attributes often correspond to the epithets of God found in the Quran.[21][22] In discussing some of the attributes' derivations, Adamson commented that "a complete consideration of Avicenna's derivation of all the attributes ... would need a book-length study".[23] In general, Avicenna derives the attributes based on two aspects of the necessary existent: (1) its necessity, which can be shown to imply its sheer existence and a range of negations (e.g. not being caused, not being multiple), and (2) its status as a cause of other existents, which can be shown to imply a range of positive relations (e.g. knowing and powerful).[24]

Reaction
Reception
Present-day historian of philosophy Peter Adamson called this argument one of the most influential medieval arguments for God's existence, and Avicenna's biggest contribution to the history of philosophy.[4] Generations of Muslim philosophers and theologians took up the proof and its conception of God as a necessary existent with approval and sometimes with modifications.[4] The phrase wajib al-wujud (necessary existent) became widely used to refer to God, even in the works of Avicenna's staunch critics, a sign of the proof's influence.[2] Outside the Muslim tradition, it is also "enthusiastically"[2] received, repeated, and modified by later philosophers such as Thomas Aquinas (1225–1274) and Duns Scotus (1266–1308) of the Western Christian tradition, as well by Jewish philosophers such as Maimonides (d. 1204).[2][4]

Adamson said that one reason for its popularity is that it matches "an underlying rationale for many people's belief in God",[2] which he contrasted with Anselm's ontological argument, formulated a few years later, which read more like a "clever trick" than a philosophical justification of one's faith.[2] Professor of medieval philosophy Jon McGinnis said that the argument requires only a few premises, namely, the distinction between the necessary and the contingent, that "something exists", and that a set subsists through their members (an assumption McGinnis said to be "almost true by definition").[25]

Criticism
The Islamic Andalusi philosopher Averroes or Ibn Rushd (1126–1198) criticized the argument on its methodology. Averroes, an avid Aristotelian, argued that God's existence has to be shown on the basis of the natural world, as Aristotle had done. According to Averroes, a proof should be based on physics, and not on metaphysical reflections as in the Proof of the Truthful.[26] Other Muslim philosophers such as Al-Ghazali (1058–1111) attacked the argument over its implications that seemed incompatible with God as known through the Islamic revelation. For example, according to Avicenna, God can have no features or relations that are contingent, so his causing of the universe must be necessary.[26] Al-Ghazali disputed this as incompatible with the concept of God's untrammelled free will as taught in Al-Ghazali's Asharite theology.[27] He further argued that God's free choice can be shown by the arbitrary nature of the exact size of the universe or the time of its creation.[27]

Peter Adamson offered several more possible lines of criticism. He pointed out that Avicenna adopts a piecemeal approach to prove the necessary existent, and then derives God's traditional attribute from it one at a time. This makes each of the arguments subject to separate assessments. Some might accept the proof for the necessary existent while rejecting the other arguments; such a critic could still reject the existence of God.[15] Another type of criticism might attack the proof of the necessary existent itself. Such a critic might reject Avicenna's conception of contingency, a starting point in the original proof, by saying that the universe could just happen to exist without being necessary or contingent on an external cause.[26]

Classification
German philosopher Immanuel Kant (1724–1804) divided arguments for the existence of God into three groups: ontological, cosmological, or teleological.[28] Scholars disagree on whether Avicenna's Proof of the Truthful is ontological, that is, derived through sheer conceptual analysis, or cosmological, that is, derived by invoking empirical premises (e.g. "a contingent thing exists").[5][25][28] Scholars Herbert A. Davidson, Lenn E. Goodman, Michael E. Marmura, M. Saeed Sheikh, and Soheil Afnan argued that it was cosmological.[29] Davidson said that Avicenna did not regard "the analysis of the concept necessary existent by virtue of itself as sufficient to establish the actual existence of anything in the external world" and that he had offered a new form of cosmological argument.[29] Others, including Parviz Morewedge, Gary Legenhausen, Abdel Rahman Badawi, Miguel Cruz Hernández, and M. M. Sharif, argued that Avicenna's argument was ontological.[28] Morewedge referred to the argument as "Ibn Sina's ontological argument for the existence of God", and said that it was purely based on his analytic specification of this concept [the Necessary Existent]."[28] Steve A. Johnson and Toby Mayer said the argument was a hybrid of the two.[25][28]

References

\section{avicenne wikipedia}

Avicenne reprend la théorie aristotélicienne des quatre causes. Mais il est le premier à concevoir la causalité efficiente de Dieu (c'est-à-dire une causalité créatrice), par opposition à la causalité motrice aristotélicienne (qui était seulement un principe de mouvement, mais non une cause d'existence ex nihilo)55,56.

Ibn-Sina distingue ainsi la philosophie naturelle, ou la physique, et la théologie, ou la métaphysique. Le métaphysicien tient un discours sur la cause très différent du naturaliste :

« Par “agent”, le métaphysicien ne veut pas seulement dire le principe du mouvement, comme le naturaliste veut le dire, mais le principe et l'origine de l'existence, comme dans le cas de Dieu à l'égard du monde57. »

Spécialiste de sa pensée, Kara Richardson donne une définition importante et contextualisée : « In his Metaphysics, Ibn Sīnā defines each of the four causes in relation to the subject studied in metaphysics : the existent qua existent. He defines the efficient cause or agent as that which gives or bestows the existence of something distinct from it58. »

C'est en ce sens qu'Avicenne écrit : « La cause est pour l’existence seulement » (Kitāb al-Shifā, ou Livre de la guérison, Livre VI, chap. 1).

Les théologiens chrétiens, tels que Albert le Grand et Thomas d'Aquin, le citent dans leurs œuvres et lui sont redevables de cette invention majeure59.

L'essence, pour Avicenne, est non-contingente, ne dépendant que d'elle-même. Possible est chaque essence dans son potentiel à être. Pour qu'une essence soit actualisée dans une instance (une existence), il faut un accident nécessaire. Cette relation de cause à effet, toujours parce que l'essence n'est pas contingente, est inhérente à l'essence elle-même. Ainsi il doit exister une essence nécessaire en elle-même pour que l'existence puisse être possible : l'Être nécessaire, ou encore Dieu60.

L'Être nécessaire est Un - c'est à la fois la conception qu'en a Plotin (το ου), mais aussi le dogme musulman (tawhīd, unicité et unité). La difficulté est alors d'expliquer l'origine de la pluralité des êtres. Comment, de l'unité, peut naître la multiplicité54,61 ?

L'Être nécessaire crée la Première Intelligence par émanation (ou « procession »), notion typiquement plotinienne54. Cette définition altère profondément la conception de la création : il ne s'agit plus d'une divinité créant par caprice, mais d'une pensée divine qui se pense elle-même ; le passage de ce premier être à l'existant est une nécessité et non plus une volonté. Le monde émane alors de Dieu par surabondance de Son Intelligence, suivant ce que les néoplatoniciens ont nommé émanation : une causalité immatérielle. La venue au monde par procession ou émanation heurte la théologie asharite qui souligne le volontarisme divin : la Création est l'effet de la volonté libre et arbitraire d'Allah54,62. C'est pourquoi Fakhr al-Din Al-Razi, qui introduit certaines thèses d'Avicenne dans la théologie asharite, ne suit pas Ibn Sina sur ce point : l'idée d'une Création nécessaire, et non volontaire, ne convient pas à sa représentation de la Toute-puissance divine63. Avicenne s'inspire des travaux d'al-Farabi, mais à cette différence que c'est l'Être nécessaire qui est à l'origine de tout (voir plus bas les Dix intelligences)60. Cette perspective serait donc plus compatible avec le Coran.

« Chaque Intelligence, à l'exception de la dernière de la série, engendre en premier lieu l'Intelligence qui lui est immédiatement inférieure à travers l'acte par lequel elle connaît le Premier Être, puis l'âme de sa sphère à travers l'acte par lequel elle se connaît comme nécessaire en vertu du Premier Être, et en troisième lieu le corps de cette sphère à travers l'acte par lequel elle se connaît comme possible en elle-même64. »

La création de la pluralité va procéder de cette Première Intelligence.

La Première Intelligence, en contemplant le principe qui la fait exister nécessairement (c'est-à-dire Dieu), donne lieu à la Deuxième Intelligence.
La Première Intelligence, en se contemplant comme émanation de ce principe, donne lieu à la Première Âme, qui anime la sphère des sphères (celle qui contient toutes les autres).
La Première Intelligence, en contemplant sa nature d'essence rendue possible par elle-même, c'est-à-dire la possibilité de son existence, crée la matière qui emplit la sphère des sphères, c'est la sphère des fixes.
L'existence de Dieu: l'argument par la contingence
Dans "Le livre de la délivrance" (un condensé du "Livre de la guérison"), Avicenne développe un argument original pour l'existence de Dieu. L'arrière-plan de l'argument est le point de vue d'Avicenne selon lequel l'existence, la nécessité et la possibilité sont mieux connues de nous que tout ce que nous pourrions dire pour les élucider. En particulier, l'affirmation de l'existence d'une chose ou d'une autre est plus manifestement correcte que ne le serait tout argument que nous pourrions donner pour justifier cette affirmation. Et les notions de nécessité et de possibilité sont plus fondamentales que toute autre notion à laquelle nous pourrions faire appel pour tenter de les définir. Néanmoins, Avicenne pense que nous pouvons dire quelque chose pour décrire les notions de nécessité et de possibilité, même si nous ne pouvons pas les définir strictement65.

Le philosophe thomiste Edward Feser résume l'argument d’Avicenne comme suit65:

Quelque chose existe.
Tout ce qui existe est soit possible, soit nécessaire.
Si cette chose qui existe est nécessaire, alors il y a un existant nécessaire.
Tout ce qui est possible a une cause.
Donc, si cette chose qui existe est possible, alors elle a une cause.
La totalité des choses possibles est soit nécessaire en soi, soit possible en soi.
La totalité ne peut être nécessaire en elle-même puisqu'elle n'existe que par l'existence de ses membres.
Ainsi, la totalité des choses possibles est possible en elle-même.
Donc la totalité des choses possibles a une cause.
Cette cause est soit interne à la totalité, soit externe à celle-ci.
Si elle est interne à la totalité, alors elle est soit nécessaire, soit possible.
Mais elle ne peut dans ce cas être nécessaire, car la totalité est constituée de choses possibles.
Et elle ne peut pas non plus dans ce cas être possible, puisqu'en tant que cause de toutes les choses possibles, elle serait dans ce cas sa propre cause, ce qui la rendrait nécessaire et non possible après tout, ce qui est une contradiction.
Ainsi, la cause de la totalité des choses possibles n'est pas interne à cette totalité, mais externe à elle.
Mais si elle est en dehors de la totalité des choses possibles, alors elle est nécessaire.
Il y a donc un existant nécessaire.
Comme le note Jon McGinnis66, parmi les caractéristiques distinctives de cet argument, il y a le fait que non seulement il n'exige pas une prémisse à l'effet qu'un infini réel est impossible comme le font souvent les arguments cosmologiques, mais qu'il ne repose pas non plus sur une prémisse à l'effet que le monde des choses possibles est ordonné (comme le fait un argument téléologique), ou qu'il est en mouvement (comme le fait un argument aristotélicien du mouvement), ou qu'il est multiple par opposition à unifié (comme le pourrait un argument néoplatonicien). Son but est de montrer que si quelque chose existe, il doit alors y avoir un être nécessaire65.

L’Être Nécessaire, selon Avicenne, doit être unique. Car supposons qu'il y ait deux ou plusieurs Êtres Nécessaires. Il faudrait alors que chacune ait un aspect qui la différencie de l'autre - quelque chose que cet Être Nécessaire a et que l'autre n'a pas. Dans ce cas, ils devraient avoir des parties. Mais une chose qui a des parties n'est pas nécessaire en elle-même, puisqu'elle existe par ses parties et ne serait donc nécessaire que par elles. Puisque l'Être Nécessaire est nécessaire en lui- même, il n'a pas de parties, et n'a donc rien par lequel un Être Nécessaire pourrait même en principe différer d'un autre. Il ne peut donc y en avoir plus d'un65.

De toute évidence, il s'ensuit également que l'Être Nécessaire, étant sans parties, est simple ou non-composé. L'Être Nécessaire doit aussi être immatériel, et donc incorporel. Car la matière n'existe que dans la mesure où elle a une forme, et ce qui est composé de forme et de matière n'est pas simple mais composite65.

Aussi, La bonté, pour l'aristotélicien, doit être définie en termes de la fin vers laquelle une chose pointe comme une cause finale. Or, une partie de la métaphysique plus générale d'Avicenne est la thèse selon laquelle toute chose existante "désire" ou vise à s'approcher de l'existence nécessaire autant qu'elle le peut. Mais alors ce qu'elle désire ou vise est de se rapprocher de l'Être Nécessaire, qui en tant qu'objet de ce désir ou de ce but est le bien le plus élevé. L'Être nécessaire doit également être parfait dans la mesure où pour Avicenne, la perfection est une question de ce qui complète une chose par rapport à son existence. Un gland est d'autant plus parfait qu'il est proche d'être un chêne, la Vénus de Milo serait plus parfaite si elle avait ses bras, et ainsi de suite. Mais l'Être Nécessaire, étant absolument nécessaire en lui-même, ne manque de rien en ce qui concerne son existence65.

Influence d'Avicenne sur le kalām
L'influence de la philosophie d'Avicenne dans la théologie rationnelle asharite a été croissante. C'est surtout avec Al-Juwayni que les concepts avicenniens commencent à pénétrer dans le kalām67. La preuve de l'existence de Dieu par la contingence du monde témoigne de cette influence68. Si Al-Ghazālī condamne certaines des thèses d'Avicenne, cela ne l'empêche pas de lui en emprunter d'autres - sans le nommer69. Faḫr ad-Dīn ar-Rāzī n'aura pour sa part aucune réticence à se référer explicitement à Ibn Sinā70. Aux yeux de Louis Gardet, c'est cette place des concepts et méthodes des falāsifa, en particulier Avicenne, qui distingue, parmi les deux grandes périodes de la théologie acharite, celles des modernes71. Preuve de la progression des idées d'Avicenne, le théologien mutazilite al-Malāhimī, au xiie siècle, voit cette influence grandir avec inquiétude, car il reproche au philosophe de dénaturer l'islam72.

Philosophie de l'être
Selon Marie-Dominique Philippe, Avicenne était un croyant au Dieu-Créateur dans l’Islam. La foi d’Avicenne ne l’empêche pas d'utiliser la métaphysique d’Aristote. Mais au contraire, il s’en sert. Il ajoute qu’Avicenne ne fait pas la distinction entre la théologie et métaphysique. Chez Avicenne, il y a un passage de la métaphysique à la théologie comme une sorte d’enveloppement73.

Angélologie
Hiérarchie des dix sphères

Hiérarchie de dix sphères célestes d'après le système de Ptolémée, illustration de Cosmographia, Anvers, 1524, de Petrus Apianus.
Avicenne s'inspire plus particulièrement de l'angélologie d'al-Farabi. L'univers est constitué d'une hiérarchie de mondes sphériques, animés par des Âmes célestes (anges et archanges) procédant du principe divin, et motrices des cieux.

La triple contemplation de la Première Intelligence instaure les premiers degrés de l'être. Elle se répète, donnant naissance à la double hiérarchie :

hiérarchie supérieure, qu'Avicenne désigne comme les Chérubins (Kerubim) ;
hiérarchie inférieure, qu'Avicenne désigne comme les Anges de la magnificence ;
Ces âmes animent les cieux, mais elles sont dépourvues de la perception du sensible ; elles se situent entre pur intelligible et sensible, et elles se caractérisent par leur imagination, qui leur permet de désirer l'intelligence dont elles procèdent. Le mouvement éternel qu'elles impriment aux cieux résulte de leur recherche toujours inassouvie de cette intelligence qu'elles désirent atteindre.

Elles sont à l'origine des visions des prophètes, par exemple. « Il y a donc, dit Avicenne, pour chaque sphère céleste une âme motrice qui intellige [saisit par son intelligence] le bien et qui, à cause de son corps, est douée d'imagination, c'est-à-dire des représentations des particuliers et une volonté des particuliers »74. Le point de départ, ici, était la cosmologie d'Aristote : Dieu est une substance immobile, un premier moteur unique, immobile, qui meut en tant qu'objet de désir et d'intellection du premier ciel, qui est la substance de la circonférence la plus extérieure de l’Univers, à savoir la sphère des étoiles fixes75.

Cette hiérarchie correspond aux Dix Sphères englobantes (Sphère des Sphères, Sphère des Fixes, sept Sphères planétaires, Sphère sublunaire).

Dixième intelligence et intellect
La Dixième Intelligence76, issue de l'Intelligence du 9° ciel (la Lune), mais sans fonction astronomique, revêt une importance singulière: aussi appelée Intellect agent ou l'Ange, et associée à Gabriel dans le Coran, elle se situe si loin du Principe que son émanation éclate en une multitude de fragments. En effet, de la contemplation de l'Ange par lui-même, en tant qu'émanation de la neuvième Intelligence, n'émane pas une âme céleste, mais les âmes humaines. Alors que les Anges de la Magnificence sont dépourvus de sens, les âmes humaines ont une imagination sensuelle, sensible, qui leur confère le pouvoir de mouvoir les corps matériels60.

Pour Avicenne, l'intellect humain n'est pas forgé pour l'abstraction des formes et des idées. L'homme est pourtant intelligent en puissance, mais seule l'illumination par l'Ange leur confère le pouvoir de passer de la connaissance en puissance à la connaissance en acte. Toutefois, la force avec laquelle l'Ange illumine l'intellect humain varie :

les prophètes, inondés de l'influx au point qu'il irradie non plus seulement l'intellect rationnel mais aussi l'imagination, réémettent à destination des autres hommes cette surabondance ;
d'autres reçoivent tant d'influx, quoique moins que les prophètes, qu'ils écrivent, enseignent, légifèrent, participant aussi à la redistribution vers les autres ;
d'autres encore en reçoivent assez pour leur perfection personnelle ;
et d'autres, enfin, si peu qu'ils ne passent jamais à l'acte.
Selon cette conception, l'humanité partage un et un seul intellect agent, c'est-à-dire une conscience collective. Le stade ultime de la vie humaine, donc, est l'union avec l'émanation angélique. Ainsi, cette âme immortelle confère, à tous ceux qui ont fait de la perception de l'influx angélique une habitude, la capacité de surexistence, c'est-à-dire l'immortalité.

Pour les néo-platoniciens, dont Avicenne fait partie, l'immortalité de l'âme est une conséquence de sa nature, et pas une finalité77.

Pour sa part, à la différence d'Avicenne, Averroès va dégager l'aristotélisme des ajouts platoniciens qui s'étaient greffés sur lui : point d'émanatisme chez lui.


\section{Avicenne}
\cite{PolDroit:voyage}

V. Dans les têtes des philo­sophes arabo-musulmans
Surlignement (bleu) - La lecture minutieuse des textes grecs conduit plusieurs générations de philo­sophes arabes à les prolonger, en les interprétant en relation avec la révélation coranique. 
On lui doit en effet , dans ce domaine , de grandes innovations . D’abord , parmi les divisions de l’être , la distinction cruciale entre « être possible » et « être nécessaire » . En réexaminant les catégories d’Aristote , Avicenne montre comment elles se trouvent en quelque sorte traversées , ou surpassées , par cette division entre les êtres qui ne portent pas
 
en eux leur cause et l’être nécessaire par lui - même , de par sa propre essence . La réflexion du philosophe chemine du couple conceptuel « possible - nécessaire » au couple « essence - existence » , qui lui doit également son émergence dans la tradition philosophique , entamant ainsi une très longue histoire , dont l’époque contemporaine porte toujours les marques . On voit aussi le génie d’Avicenne approcher une théorie de la conscience et du Je antérieure au « cogito » de Descartes , à travers son hypothèse de « l’homme volant » . Il imagine un homme créé d’un coup , dans le vide , sans perception provenant d’un monde extérieur ni de son propre corps . Cet homme dépourvu de toute sensation et comme privé de corps n’en aurait pas moins , soutient Avicenne , la conscience d’exister , d’être lui - même , et aucun autre . Mort relativement jeune , à cinquante - sept ans , au cours d’une expédition militaire qu’il accompagnait , le philosophe repose à Hamadan , dans l’actuel Iran , à mi - chemin entre Téhéran et Bagdad , où un mausolée monumental a été édifié en 1952 . Car la gloire d’Avicenne ne s’est pas ternie , et il continue d’être célébré comme esprit universel , même si , dans l’histoire des philosophes arabo - musulmans , il a rencontré des adversaires farouches .


\section{La preuve développée par Abū al-Ḥasan al-Ašʿarī}

  
  \subsection{L'existence de Dieu}
  
 

C'est par une preuve de l'existence de Dieu qu'al-Ašʿarī ouvre un de ses
ouvrages majeurs d'exposition systématique de sa théologie, le
\emph{Kitāb al-Lumaʿ}. La démarche nous
semble naturelle, mais elle n'est alors pas si commune. Al-Ašʿarī n'est
pas le premier à proposer une preuve de l'existence de Dieu (il semble
que ce premier soit, en islam, le théologien zaydite Qāsim ibn Ibrāhīm),
la démarche n'est pas fréquente : beaucoup de théologiens considèrent
que l'existence de Dieu est le fondement même du savoir, une donnée
évidente par elle-même.




\begin{quote}
   Question : Quelle est la preuve que la création a un auteur qui l'a
créée, et un organisateur qui l'a organisée ?\sn{PREUVE DE L'EXISTENCE DE DIEU (al-Ašʿarī, \emph{Kitāb al-Lumaʿ},
§ 3).}

Réponse : La preuve est la suivante. L'être humain, même quand il est au
sommet de sa perfection, a d'abord été successivement du sperme, un
caillot, un petit amas, et enfin de la chair et du sang. Nous savons que
ce n'est pas lui-même qui s'est fait passer d'un état à un autre. En
effet, nous voyons que, alors même qu'il est haut plus au degré de sa
force et de son intelligence, il est incapable de se fabriquer des yeux
pour voir ou des oreilles pour entendre, ni de se créer un membre
quelconque. C'est bien la preuve qu'il était encore plus incapable de le
faire avant même d'avoir acquis sa force et son intelligence : ce qu'il
est incapable de faire dans son état de perfection, à plus forte raison
en sera-t-il incapable dans un état de faiblesse.

De plus, nous voyons que l'homme est d'abord un enfant, puis un jeune,
puis un adulte, enfin un vieillard, et nous savons qu'il ne se fait pas
passer lui-même de l'état de jeunesse à celui de maturité ou de grand
âge, puisque s'il s'efforçait de quitter la maturité ou le grand âge
pour revenir à la jeunesse, il ne le pourrait pas. Ce qui prouve bien
qu'il ne se fait pas passer de lui-même d'un état à un autre, et qu'il y
a quelqu'un qui le fait passer d'un état à un autre et qui l'a organisé
comme il est, car il est impossible qu'il passe d'un état à un autre
sans l'aide de quelqu'un qui le change et l'organise.

On peut prendre un exemple pour l'expliquer : le coton ne peut pas
devenir du fil, et le fil du tissu, sans l'aide d'un fileur et d'un
tisserand. L'homme qui achète du coton en espérant le voir se changer en
fil puis en tissu sans l'aide d'un fileur et d'un tisserand est un fou,
tout comme celui qui, dans un désert, s'attend à voir la boue se changer
en briques qui s'empileraient d'elles-mêmes, sans l'aide d'un briquetier
et d'un maçon, est un imbécile. 
\end{quote}

\section{Avicenne - Intellect agent}

Alfarabi, Avicenna, and Averroes, on Intellect: Their Cosmologies...
par Davidson, Herbert A
1992
A study of problems revolving around the subject of intellect in the philosophies of Alfarabi, Avicenna, and Averroes, this book pays particular attention to the way in which these...

Intellect agent : explique le monde sublunaire, depuis la première cause. 

shifa' : le problème d'expliquer comment un univers pluriel peut dériver d'une unique cause a été posé par Plotin. 
\begin{quote}
    from the one, insofar as it is one, only one can come into existence (yuhad)
\end{quote}
Les émanations successives permettent d'expliquer comment, de ce principe. La plurarité entre parce que les êtres incorporels suivant la première cuase ont plusieurs pensées.

\begin{Def}[Intellect actif]
\begin{itemize}
    \item emanating cause of the matter of the sublunar world
    \item the emanating cause of natural forms appearing in matter, inlcuding the souls of plants, animals, and man
    \item the cause of the actualization of the human intellect.
\end{itemize}
\end{Def}

\section{Avicenna’s Proof of the Existence of God: Problem 7}

In: Doubts on Avicenna
Author: Ayman Shihadeh
Type: Chapter
Pages: 143–155
 
 \begin{quote}
     al Mas'udi's central objection -- that an infinite series cannot be treated as a self contained whole, at least not in Avicenna's matter of facter manner - seems quite compelling, there is much less mileage in his analogical ad hominem argument, as it fails to fulfuil the principal argument of this type of argument, namely, that it should start from the adversary's own view.
 \end{quote}
 
 
 \section{The Cambridge Companion to Classical Islamic Theology (Cambridge Companions to Religion)}
 
 

Tim Winter
Part II Themes
 \begin{quote} > Page 198 · Emplacement 4943
A convenient starting - point will be a categorisation of proofs provided by Fakhr al - Dīn al - Rāzī ( d . 1210 ) , an outstanding philosopher and mutakallim , who surveyed and assessed the previous philosophical and theological dialectic more systematically and insightfully than did his predecessors . He distinguishes between four categories : ( 1 ) arguments from the creation of the attributes of things ( a subspecies of the argument from design ) ; ( 2 ) arguments from the creation of things ; ( 3 ) arguments from the contingency of the attributes of things ( a subspecies of the argument from particularisation ) ; and ( 4 ) arguments from the contingency of things . 4 The first type will be discussed below under “ Common teleological arguments ” ; the second and third under “ Kalām cosmological arguments ” ; and the fourth under “ Avicenna’s argument from contingency ” .
COMMON TELEOLOGICAL ARGUMENTS An argument from design , or a so - called teleological argument , is one which argues from manifestations
\end{quote} 
\subsection{KALĀM COSMOLOGICAL ARGUMENTS}
\begin{quote} > Page 204 · Emplacement 5095
 The early mutakallimūn developed characteristic doctrines and methods of argument ( some of which we will encounter below ) , which formed the speculative frameworks in which they expounded their proofs for the existence of God . Generally , arguments from design were either omitted or accorded secondary importance in kalām works , since they proved only the existence of a “ designer ” , but not the generation of matter and hence creation ex nihilo , and because they were often seen to lack methodological rigour . Instead , the kalām argument par excellence became the argument from creation ex nihilo , or temporal generation (  udūth ) , 31 and the closely related argument from particularisation – both cosmological arguments , since they prove the existence of God starting from the existence of other beings .
\end{quote} 
\paragraph{Arguments from creation ex nihilo}
 \begin{quote} > Page 205 · Emplacement 5105
The basic argument from creation goes as follows . The world is temporally originated (  ādith ) . All that is temporally originated requires a separate originator . Therefore , the world requires a separate originator . This originator must be pre - eternal . Otherwise , if it too is generated , then , by the same reasoning , it will require another originator ; and ultimately the existence of a pre - eternal originator has to be admitted . Both premises in the argument were surrounded by complex discussions , both among theologians , and between them and the philosophers . In what follows , some of the discussions that appeared among the mutakallimūn surrounding the two premises in this proof are examined . That the world is temporally originated Several arguments were advanced in support of this doctrine ( the minor premise in the above proof ) mostly on the basis of the early kalām physical theory that , apart from God , all beings are bodies consisting of both atoms and accidents present
in them . 32 The most commonly used is the so - called argument from accidents ( a‘rā  ) , apparently developed by the Mu‘tazilite Abū Hāshim al - Jubbā’ī , which establishes the generation of atoms on the basis of four principles , as follows : ( a ) Accidents exist in bodies . ( b ) Accidents are generated . ( c ) Bodies cannot be devoid of , or precede , accidents . ( d ) What cannot be devoid of , or precede , what is generated is likewise generated . 33 Earlier mutakallimūn seem to hold that the generation of the world follows from these contentions directly . Yet , as Averroes points out , this line of reasoning involves an equivocation : what is found to be generated in the fourth principle is the single body that necessarily has a particular accident known to be generated , rather than bodies as such , and consequently the world as a whole , as in the conclusion . 34 Indeed , he points out , it will still be conceivable for the world to be pre - eternal , involving infinitely regressing series of temporally originated things (  awādith lā awwala lahā ) .
Later mutakallimūn , as Averroes notes , became more aware of this gap in the proof , and attempted , apparently starting from Juwaynī , to address it by arguing that a pre - eternal series of accidents is inconceivable . 35 Several arguments are found in later works of kalām that support this contention ; the following two are recorded in a later Mu‘tazilite source . For instance , it is argued , rather opaquely , that the whole must be characterised by the same attributes that necessarily characterise each of its individual parts ; for instance , if something consists entirely of black parts , it too must be black . Therefore , since each part of the world is generated and has a beginning , the whole world too must be generated and have a beginning . The infinite regress of accidents is also refuted using proofs from the impossibility of an infinite number , some of which were apparently adopted from John Philoponus ( d . c . 570 ) . 36 For instance , it is argued : When today’s events are combined with past events , these will increase ; without today’s events , they will
diminish . Increase and diminution in what is infinite are inconceivable . This indicates that [ the series of past events ] is finite with respect to its beginning . This is the proof also for the finiteness of the magnitude of the earth and other bodies ; for it is possible to conceive of increase and diminution in them . 37 Many later Ash‘arites adopted Juwaynī’s modified version of the argument for creation ex nihilo , which most theologians treated as an article of faith . Yet this doctrine soon became the centre of conflict between the theologians and most philosophers , who defended the pre - eternity of the world , as the interaction between the two traditions increased . Doubts were raised around the arguments for creation , to the extent that in one of his latest works Rāzī examines all the relevant arguments and counterarguments and admits that no rational or revealed evidence proves either the creation or pre - eternity of the world . 38 Under his influence , it seems , Ibn Taymiyya asserts that no rational or revealed evidence proves the inconceivability of the infinite regress of accidents , apparently suspending judgement on the subject .
without hesitation or reflection ” .

\subsection{Note}
Note - 10 The existence of God > Page 207 · Emplacement 5153
Kahnemann sysstemmem 1
\end{quote} \begin{quote} > Page 207 · Emplacement 5163
this archetypical kalām analogy ( an instance of inferring the “ unobservable ” from the “ observable ” ,
\end{quote} \begin{quote} > Page 207 · Emplacement 5171
My act requires me ( its originator ) , we are told , because it occurs according to my motives ; this connection affirms the judgement in the original case . But in what respect exactly does my act depend on me ? Does it depend on me because it is temporally originated , or for some other ground ?
Therefore , my act depends on me in this respect only , and the ground will thus be affirmed in the original case . It may seem strange to argue for the existence of God from human acts , rather than from the need of natural events generally for causes . Yet this oblique way is forced on those Mu‘tazilites who employ this argument by
physics : many of them reject natural causality , and affirm that God creates all generated things , except accidents produced by the power of living creatures . Hence , when I move my pen , my power will generate the accident of motion in it ; however , when running water moves a pebble , the accident of motion in the pebble will be generated by God’s power , not by the water . Our acts , therefore , provide the only case where we can observe both the originated thing and its originator and conclude that the former is generated by the latter . The existence of the creator will then be the only explanation for the generation of the existence of other accidents and all atoms , as ‘ Abd al - Jabbār writes : “ Everything that is [ beyond the capacity of created beings ] is evidence for Him . ” 43 Mu‘tazilites criticised Ash‘arites on account of their contention that human acts are generated by divine , rather than human , power : since they cannot affirm that power generates things in the “ observable ” realm , they cannot affirm the
They will be unable to accept the causal premise in the argument , and will thus fail both to explain the world as a divine act and to prove the existence of God . Juwaynī retorts that Ash‘arites use the closely related particularisation argument , which does not resort to the above analogy . 44 Ash‘arites indeed rarely use this basic argument from creation , involving the major premise , “ What is originated requires an originator ” , except in an informal and non - technical manner . Rāzī attacks each step in the above analogical argument , arguing at length that “ coming into being ” cannot be the ground for a thing’s requiring a cause . 45
Arguments from particularisation This is the main form of argument used by early Ash‘arites , and is often used by Mu‘tazilites and later Ash‘arites . It turns on the notion of particularisation ( takh  ī  ) , which has its background in a trend distinctly characteristic
classical kalām , stemming from the sense that randomness of any kind , in either quantity or quality , is inconceivable .
\end{quote} 
\begin{Synthesis}
Note - 10 The existence of God > Page 209 · Emplacement 5201
Aleatoire impossible ?
\end{Synthesis}
\begin{quote} > Page 209 · Emplacement 5201
Every seemingly random fact about the world or things therein thus calls for explanation . Different instances of this type of proof cite different facts . The earliest arguments were relatively simple and departed from the atomist framework of classical kalām , as in the following two arguments advanced by the Ash‘arite theologian al - Bāqillānī ( d . 1013 ) .
Note - 10 The existence of God > Page 209 · Emplacement 5203
Vision atomiste
\end{quote} \begin{quote} > Page 209 · Emplacement 5204
He argues that we observe identical things coming into being at different times . If the occurrence of one thing at a particular moment is due to an intrinsic quality thereof , all similar things should occur at the same time . It thus appears that nothing intrinsic to the thing itself could make it more likely to occur at a particular moment rather than at another moment , or more likely to occur at a given moment than another , similar thing . Therefore , there must be an external voluntary effecter , who causes particular things to occur at particular moments .
\end{quote} 

\begin{quote} > Page 209 · Emplacement 5219
There is no natural necessity determining the way things actually are . All things , rather , consist of identical atoms and of different accidents present in them , which come in and out of existence at every moment .
\end{quote} \begin{quote}> Page 210 · Emplacement 5222
As mentioned , the general particularisation argument can take different types of facts as its point of departure . The foregoing examples focus on the when and how with respect to the generation of things . In later , more sophisticated , arguments advanced by Juwaynī , the same lines of reasoning are applied to the world as a whole , which allows
\end{quote} \begin{quote}> Page 210 · Emplacement 5225
him to transcend the occasionalistic bias of earlier particularisation arguments . He argues , first , that since the world is generated , it must have come into being at a particular point in time . This implies that a separate particularisation agent must exist to select this particular moment for creating the world out of other possible moments . Such selection can only be made by a voluntary agent . An unchanging , non - voluntary pre - eternal cause will necessitate its effect and will thus produce a pre - eternal world ; yet the world , Juwaynī argues , has been shown to be temporally originated . 47 This argument faces the problem that it implies that time existed before creation , a doctrine that was subject to much debate . 48
\end{quote}
Note - 10 The existence of God > Page 210 · Emplacement 5226
Ce que jee comprend de la parti cularisatiokn temps deux choses qrriivrnn tq des moments differents. Donx ce nest pas dqns leurr essence majis cekkkee de dieu
Note - 10 The existence of God > Page 210 · Emplacement 5231
A partir de la particukariisatiion dieu eset xelui qui chosit ce momeenet parrixxuxukier




 \begin{quote}> Page 210 · Emplacement 5232
Elsewhere , Juwaynī also argues that if we observe the world , we find that it consists of things that have great variety in their attributes , composition and circumstances . None of these , however , is necessary , as the mind can imagine all things being otherwise . It becomes evident , he continues , that since the world is possible , “ it will require a determinant [ muqta  ī ] , which determines it in the way it
actually is ” . What could exist in different possible ways cannot exist randomly ( ittifāqan ) , without a determinant , in one particular way . 49 Again , the determinant has to be a voluntary agent ; for a non - voluntary factor will necessitate a uniform , undifferentiated effect , whereas this world consists of highly complex parts , which do not behave in simple , uniform ways . 50 Ghazālī writes with reference to the notion of particularisation : “ The world came into existence whence it did , having the description with which it came to exist , and in the place in which it came to exist , through will , will being an attribute whose function is to differentiate a thing from its similar . ” 51 Such particularisation arguments , which refer to characteristics of the world or things therein differ crucially from arguments from design . The latter focus on aspects of perfection , masterly production , or providence in the world . Particularisation arguments , by contrast , depart from the mere fact that existents in this world , regardless of their perfection , imperfection , goodness or badness , are possible , since they exist in one particular way
rather than another , and thus require an external factor to select this possibility over all other possibilities . Such arguments aim only at proving that the world has a voluntary producer , whereas arguments from design seek to prove that the world must have a wise , powerful and good producer . Finally , Juwaynī goes further to develop a third argument by applying the particularisation principle to the fact that the world exists . In this crucial modification to the particularisation argument , he frees it completely from the constraints of atomist physics . He first demonstrates that the world is temporally originated , then writes : What is temporally originated is a possible existent ( jā’iz al - wujūd ) ; for it is possible to conceive its existence rather than its non - existence , and it is possible to conceive its non - existence rather than its existence . Thus , since it is characterised by possible existence rather than possible non - existence , it will require a particularising factor ( mukha   i  ) , viz . the Creator , be He exalted . 52 The argument departs from the fact that the world exists , regardless of what it consists of and the way
in which it exists . Since it is equally possible that the world did not exist , the fact that it does exist points to an external factor which effected one of the two possibilities . In this argument , Juwaynī marries the argument from creation ex nihilo to the particularisation argument , which allows him , as an Ash‘arite , to argue that the world requires an originator because it is temporally originated , without resorting to the Mu‘tazilite analogy from human action . More crucially , Juwaynī’s modified argument brings the particularisation argument close to Avicenna’s argument from contingency , paving the way for a synthesis of the two arguments in later kalām .
\end{quote} 

\begin{Synthesis}
Note - 10 The existence of God > Page 210 · Emplacement 5233
Dieu comme celui quu fait advenirr ce qui doitlp
\end{Synthesis}
\subsection{AVICENNA’S ARGUMENT FROM CONTINGENCY }
\begin{quote}> Page 211 · Emplacement 5262
AVICENNA’S ARGUMENT FROM CONTINGENCY The central proof for the existence of God that Avicenna puts forth is the proof from contingency ( imkān ) . In line with the Neoplatonic tradition , he attempts to prove an ultimate efficient cause for bringing the world into being , rather than a cause for motion in the world , as Aristotle does . Unlike
most other proofs , this proof depicts God as a non - voluntary First Cause , which produces the world from pre - eternity by Its essence . Thus , despite its great influence on later Muslim thought , the proof had to be adjusted to conform to more orthodox conceptions of God . Avicenna claims to advance a purely metaphysical proof ( as opposed to a physical proof ) , one that rests purely on an analysis of the notion of existence qua existence , without consideration of any attributes of the physical world . 53 He writes : Reflect on how our proof for the existence and oneness of the First and His being free from attributes did not require reflection on anything except existence itself and how it did not require any consideration of His creation and acting even though the latter [ provide ] evidential proof for Him . This mode , however , is more reliable and noble , that is , where when we consider the state of existence , we find that existence inasmuch as it is existence bears witness to Him , while He thereafter bears witness to all that comes after Him in existence . 54
If true , this characterisation would set the proof apart from all contemporaneous , cosmological and teleological proofs . In contemporary terminology , it would qualify it to be an ontological proof , that is to say , a proof which argues for the existence of God entirely from a priori premises and makes no use of any premises that derive from our observation of the world . Recent studies of Avicenna’s proof , however , differ on whether the argument is cosmological or indeed ontological . 55 As we will see , doubt with regard to the purported fundamental novelty of Avicenna’s proof was expressed centuries ago . The proof rests on conceptions that , Avicenna contends , are primary in the mind , intuited without need of sensory perception and mental cogitation , namely “ the existent ” and “ the necessary ” . The conception “ the possible ” , being what is neither necessary nor impossible , is either equally primary , or derived directly from the conception “ the necessary ” . An existent , by virtue of itself , is either possibly existent , or necessarily existent . If we posit an existent that is necessary in itself , then , Avicenna argues ,
it will have to be uncaused , absolutely simple , one and unique . If we posit an existent that is possible in itself , it will have to depend for its existence on another existent . The latter will be its cause , not in the sense of being an antecedent accidental cause for its temporal generation , but as a coexistent essential cause for its continuous existence . If this cause is itself a possible existent , it will have to exist by virtue of another . The series of actual existents , Avicenna argues , cannot continue ad infinitum , but must terminate in an uncaused existent that is necessary in itself . But why does a possible existent require a cause to exist ? Avicenna proves this using the argument from particularisation , apparently borrowed from kalām . A possible existent can exist or not exist . It will exist only once “ the scale is tipped ” by an external cause such that its existence becomes preponderant over its non - existence . When this occurs , its existence will be “ necessitated ” by its cause . Now , the proof for the existence of God runs as follows . There is no doubt that there is existence . Every existent , by virtue of itself , is either possible
or necessary . If necessary , then this is the existent being sought , namely God . If possible , then it will ultimately require the necessary existent in order to exist . In either case , God must exist . 56 Apparently based entirely on an analysis of a priori conceptions and premises , the proof will appear ontological . However , other considerations suggest that the proof is fundamentally cosmological . For instance , the deliberately abstract and unexplained premise , “ There is no doubt that there is existence ” , appears to derive from our knowledge that “ there is no doubt that something exists ” , or it may even mean the same as the latter statement . 57 When the proof then goes on to appeal to the dichotomy of possible existence and necessary existence , it branches into two hypothetical directions : that this indubitable existence is either possible or necessary . But this then begs the following question : if our indubitable knowledge that there actually is existence is examined , will this existence turn out to be possible or necessary ? In other words , will this knowledge derive from our awareness ( no matter how primitive and abstract ) of possible
existents or necessary ones ? Of course , we cannot be aware of necessary existents ; therefore , our indubitable knowledge of existence must relate to our awareness of possible existence . Inevitably , it seems , the proof reasons on the basis of possible existence using the causal premise , which explains the existence of possible existents by reference to a necessary existent . It hence appears to hinge on the existence of things other than God to prove His existence . Indeed , eight centuries ago , Rāzī wrote that all proofs for the existence of God depart from facts about the world , except that Avicenna had claimed to have advanced a fundamentally new proof purportedly based on a consideration of existence qua existence , without consideration of things other than God . He quotes Avicenna’s above statement to this effect . This claim , however , invites two objections from Rāzī . First , this proof depends on a causal premise : the proof in fact “ infers the existence of the necessary [ existent ] from the [ actual ] existence of the contingent ” . Second , even if it proves a necessary existent , one will still need

to demonstrate that it is other than the physical things perceptible in this world ( this recalls the series of proofs , already referred to , which Avicenna advances for the simplicity , oneness and uniqueness of the necessary existent ) . 58 In other words , the argument presupposes these different considerations about the world : one should prove that the world is not necessarily existent , but contingent , and that a contingent requires a necessary existent to exist , before concluding that God , therefore , exists . A good proof indeed , Rāzī would add , but not an ontological one . Nevertheless , even if such criticisms are accepted , Avicenna should nonetheless be credited with the first attempt ever to advance such a proof . 59
\end{quote}