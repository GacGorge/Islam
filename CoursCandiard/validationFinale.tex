\chapter{Les preuves de l'existence de Dieu de al-Ǧuwaynī et Avicenne}
%{\Large\textbf{Les preuves de l'existence de Dieu de al-Ǧuwaynī et Avicenne }}
\mn{Guillaume Gorge - ISTR 2022 - Introduction à la théologie musulmane à la période classique - 16 mai 2022}

 

% -------------------------------------------------------------------------------------------------------
\section{Introduction}

La métaphysique en Islam (\emph{Kalām}) est aujourd'hui assez délaissée. Pourtant, cette métaphysique a connu une développement important, en particulier au 9è et 10è siècle, autour de deux questions : 
\begin{itemize}
    \item la transcendance de Dieu : si Dieu est transcendant, comment est-il possible d'en dire quelque chose ?
    \item la possibilité du libre-arbitre de l'homme face à la toute puissance de Dieu
\end{itemize}
Une troisième question anime le \emph{Kalām} : peut-on démontrer l'existence de Dieu ?  Nous voudrions étudier cette dernière question à travers deux démonstrations : 
\begin{itemize}
    \item la démonstration de \emph{al-Ǧuwaynī} , maître de d'Al-Ghazâlî et intermédiare entre les "anciens" et les "modernes" ash'arites.
    \item la démonstration d'Avicenne 
\end{itemize}
En montrant les présupposés de ces démonstrations, nous essayerons de montrer qu'elles éclairent finalement les deux premières questions, à la fois la transcendance de Dieu et la possibilité pour l'homme de libre-arbitre.


 

 
% -------------------------------------------------------------------------------------------------------
\section{Les démonstrations de al-Ǧuwaynī}



\paragraph{al-Ǧuwaynī.} Grand théologien ash'arite, al-Ǧuwaynī (1028-1085) dit \textit{l'imam des deux sanctuaires}  suite son séjour en Arabie, propose une approche plus rationnelle que les premiers ash'arites. Ainsi, alors que la démonstration de l'existence de Dieu de Al-Ashari proposait une démonstration basée sur la connaissance scientifique de l'époque et le Coran, la démonstration de al-Ǧuwaynī est plus conceptuelle.


\paragraph{la raison comme commandement divin Ash'arite.} Pour les Ash'arites, la réflexion constitue un devoir religieux (shar'i), à la différence des Mu'tazilites qui présentaient la réflexion comme un devoir mais qui s'applique à tout homme et pas seulement découlant d'une prescription religieuse (\cite{Cambridge:ClassicalIslamicTheology}). La place de la raison est l'un des points de désaccord avec les Hanbalites, courant qui s'interdit de faire de la raison une source du droit religieux. 

Ainsi, les théologiens traditionalistes comme Ibn Taymiyya  (pour qui la connaissance de Dieu est intuitive et immédiate par la vertu de la nature de l'homme \emph{fitra})   jugent vaines les preuves de l'existence de Dieu mais, de façon plus étonnante nous trouvons aussi des théologiens classiques comme Al Ghazali, pourtant l'élève de al-Ǧuwaynī, qui affirme que l'homme connaît Dieu à travers la \textit{fitra} sans besoin de raisonnement discursif. La preuve de Dieu peut néanmoins être nécessaire pour l'homme assailli par le doute \cite[p.197]{Cambridge:ClassicalIslamicTheology}


\paragraph{Classification des preuves de l'Existence de Dieu} 
Suivant A. Shihadeh \cite[198]{Cambridge:ClassicalIslamicTheology}, nous reprenons la catégorisation des preuves fournies par Fakhr al-Dīn al-Rāzī (m. 1210) , philosophe et \textit{mutakallim} (savant du Kalām, ou théologien), qui distingue les catégories suivantes : 
\begin{itemize}
    \item les preuves par la création des attributs des choses  ;   
    \item les preuves par la création des choses ;    
    \item les preuves par de la contingence des attributs des choses   ;  
    \item les preuves tirées de la contingence des choses (argument d'Avicenne).   
\end{itemize}
Les preuves de la première catégorie, historiquement la plus anciennes,  sont basées sur l'observation du monde et du refus d'y voir le fruit du hasard ou d'une entière contingence. Ils y décèlent au contraire son action et son \textit{dessein}. Ces preuves s'inspirent de nombreux versets du Coran (par exemple Co 2, 164). On trouve par exemple la démonstration d'Al-Ashari. Généralement ces preuves, mêmes si elles sont populaires parmi les musulmans, du fait de leur souffle coranique, sont délaissées par les théologiens du \textit{Kalām} car elles ne prouvent que l'existence d'un \textit{concepteur du dessein} (en anglais \textit{designer}) et non d'une création ex-nihilo (\cite[p.204]{Cambridge:ClassicalIslamicTheology}). Les théologiens musulmans comme Al-Ğuwaynī privilégient donc un autre type de démonstration, de type cosmologique, qui regroupent les preuves des catégories 2 et 3 de al-Rāzī.

\paragraph{\emph{Kitāb al-irshād} utilisant les outils de la logique} C'est à cette démonstration cosmologique que s'attelle  al-Ǧuwaynī dans son livre \textit{Kitāb al-irshād}, dont le plan reprend la structure des livres des   mu'tazilites. Il y établi tout d'abord que le raisonnement fait partie des obligations de tout homme. Puis il présente la démonstration de l'existence de Dieu, en utilisant la logique comme outil dans sa démonstration. Il est en cela à l'intersection entre les ash'arites \textit{anciens} qui répugnaient à utiliser les outils philosophiques comme la logique, et les mutakalimins \textit{modernes} qui se basent sur la logique d'Aristote.

\subsection{la démonstration proprement dite}

\begin{quote}
    Maintenant\sn{ (Al-Ğuwaynī, \emph{Kitāb
al-Iršād}, 4 ; trad. Luciani)} qu'il est prouvé que le monde est contingent et qu'il a eu un
commencement, il s'ensuit que le contingent peut exister ou ne pas
exister, et que quel soit le moment où il se produit, il aurait pu se
produire à un moment antérieur ; que l'existence du contingent aurait pu
être retardée de plusieurs heures au-delà de ce moment.

Si donc l'existence possible se produit, au lieu d'une prolongation
également possible de la non-existence, l'esprit saisit, comme une chose
évidente, que (pour se produire) l'existence a eu besoin d'une principe
déterminant (\emph{muḫaṣṣiṣ}) qui détermine sa réalisation. C'est là une
chose qui apparaît nécessairement, sans qu'il y ait besoin de faire des
distinctions ou d'employer le raisonnement.
\end{quote}

\paragraph{la démonstration Ex nihilo} La démonstration part de l'expérience du \textit{temps} qui préexiste au monde : le monde n'a pas été créé de toute éternité mais dans le temps (cf \textit{l'existence du contingent aurait pu être retardée de plusieurs heures au delà de ce moment}). 

\paragraph{Une impossibilité à penser le hasard} La première partie de la démonstration, avancée initialement par le théologien Ash'arite al-Bāqillānī (d. 1013) a comme présupposé une conclusion du \emph{Kalām} classique que le hasard est inconcevable et que tout fait doit être expliqué même celui qui parait le fruit du hasard. Pour al- Bāqillānī,  on observe des choses identiques se produisant à des temps différents. Si la survenance d'une chose est due à une de ses qualités intrinsèques, toutes les choses similaires doivent survenir au même moment. Il faut donc qu'il y ait une cause externe libre qui cause ces choses particulières de survenir à un moment spécifique. \cite[p.209]{Cambridge:ClassicalIslamicTheology} Cette vision, étonnante à nos yeux modernes, vient de la vision atomiste du monde : toute chose consiste en des atomes identiques et en des accidents différents présents en eux, atomes qui viennent à l'existence à chaque moment. Comme les atomes ont des possibilités infinies, il faut bien un facteur externe, \textit{principe déterminant}, car il détermine ou spécifie (\emph{takhīs}), les propriétés et les accidents des choses.

Al-Ğuwaynī reprend cette démonstration en l'étendant au monde puis continue sa démonstration : 

\begin{quote}
Une fois admis le principe général que le contingent exige un principe
déterminant, il faut nécessairement que ce principe soit : ou bien une
cause nécessitant la réalisation de la contingence, comme la cause
nécessite son effet ; ou bien une force physique, comme le pensent les
naturalistes ; ou bien enfin un agent libre. Or il est faux que ce
principe déterminant agisse à la manière d'une cause. La cause en effet
nécessite son effet d'une façon simultanée. Si on supposait que le
principe déterminant fût une cause, celle-ci serait forcément ou
éternelle, ou contingent. Dans le premier cas, elle aurait dû
nécessairement provoquer l'existence du monde de toute éternité, ce qui
conduirait à admettre l'éternité du monde. Or nous avons fourni les
preuves de sa contingence. Si le principe était contingent, il aurait
lui-même besoin d'un principe déterminant, et ainsi de suite à l'infini.
    
\end{quote}


 \paragraph{Une solution possible : la série infinie de principes contingents} Al-Ğuwaynī ne saute pas directement à la conclusion de l'existence d'un principe déterminant. Car, il est possible d'imaginer un monde existant de toute éternité en postulant une série infinie de \textit{principes contingents}, une opposition que mentionnera Averroes. Al-Ğuwaynī, conscient de la faiblesse de la démonstration classique du monde contingent, mentionne ici cette possibilité pour l'écarter.
Un monde qui n'existe pas de toute éternité entraîne donc l'existence d'un \textit{agent qui agit librement} sur les choses contingentes, en leur assignant certains attributs et certains moments :

\begin{quote}
Quant à ceux qui prétendent que le principe déterminant est une force
physique, leur théorie
est inadmissible. {[}\ldots{]}

S'il est faux, par conséquent, que le principe déterminant du contingent
soit une cause nécessitante, ou une force physique qui lui donne par
elle-même l'existence, mais involontairement, il s'ensuit d'une manière
certaine que le principe déterminant des choses
contingentes est un agent qui agit sur elles librement, qui leur assigne
spécialement, en les produisant, certains attributs et certains moments.
\end{quote}
 
 
\subsection{Conséquence de cette démonstration}

\paragraph{Conséquence sur la question du libre Arbitre}

La position d'Al-Ash'ari est fataliste - toute action, y compris les actes humains, sont déterminés par la volonté divine. En particulier, Al-Ashari déduit de la toute-puissance divine, l'attribution du mal et de l'injustice à la volonté de Dieu, mais se pose alors le problème de la responsabilité humaine et de la justice de Dieu à la fin des temps : si l'homme n'est pas responsable, il est injuste de le condamner. Or Dieu est juste. Face aux mu'tazilites qui reconnaissent le libre arbitre humain, Al-Ğuwaynī ne peut se satisfaire de la position d'Al-Ashari et cherche un compromis qui préserve l'idée de puissance de Dieu, tout en faisant davantage de place à Sa justice. Sans rentrer dans l'approche qu'il retient, le fait que le principe déterminant soit appliqué au monde et non directement à toute chose, permet de laisser un espace pour la volonté de l'homme et donc pour sa responsabilité et in fine la justice de son jugement.

\paragraph{Conséquence sur la question des attributs divins}
La position de Al-Ash'ari sur les attributs divins, se veut intermédiaire entre l'approche des mu'tazilites qui purifient (\emph{tanzih}) les attributs indignes de Dieu et les Hanbalites qui se tiennent à la lettre du Coran, mais acceptent l'équivocité des termes, quand ils sont appliqués à Dieu et à l'homme. Al-Ash'ari prend au sérieux les termes et fait le choix de leur unicité (si Dieu est \textit{voyant}, c'est qu'il a la vue), mais refuse l'approche des mu'tazilites qui considèrent que ces attributs font partie de l'essence de Dieu. Mais l'approche d'Al-Ash'ari laisse un peu sur sa faim.
Al-Juwaynī reprend la distinction grammatical  al-Jubbā' qui distingue l'essence de l'attribut du \textit{hal} ou \textit{accusatif d'état}. Cela  permet de distinguer les attributs de Dieu liés à son essence (l'existence, l'unicité, le savoir, l'éternité) et ceux qui sont possibles (la vue, l'ouïe, la parole,..). Al-Ğuwaynī reprend alors la définition de la propriété, ce qui est commun à des choses pourtant distinctes. La propriété est distincte et indépendante de l'entité qu'elle caractérise. Elle ne peut être dite ni existante, ni non-existante. 

% -------------------------------------------------------------------------------------------------------
\section{Les démonstrations d'Avicenne}

\paragraph{Avicenne.} Ibn Sina (980-1037) est probablement la figure la plus marquante de le \textit{falsafa}, ce courant théologique musulman nourri de la philosophie grecque, en particulier d'Aristote et, pour la partie métaphysique, de textes du philosophe néoplatonicien Plotin, faussement attribués à Aristote. 
Sa démonstration de l'existence de Dieu aura une postérité considérable, non seulement en Islam mais aussi sur la théologie Occidentale.

\paragraph{le courant de la \textit{Falsafia}} A la différence de l'Occident où la philosophie est une faculté d'université, certes ancillaire à la théologie, mais autonome, la \textit{falsafia} est une école musulmane concurrente des autres, ce qui fera rejeter leurs outils quand leurs conclusions ne seront plus jugées recevables par la plupart des théologiens musulmans.  

\paragraph{Le livre des guérisons}
    Avicenne développe sa métaphysique dans son  \textit{livre la la guérison}, \emph{Kitâb al-Shifâ'}. Ce livre permet d'apprécier l'érudition d'Avicenne que ce soit en médecine ou en théologie. La partie sur la métaphysique est connue en occident sous son nom latin \textit{Liber de philosophia prima sive scientia divina}.  Avicenne intègre donc la métaphysique dans un ensemble plus large, sur la guérison. Il manifeste ainsi sa vision de la philosophie comme une thérapeutique : la guérison ne concerne pas uniquement le corps mais aussi de l'esprit.


\subsection{Démonstration d'Avicenne}

\paragraph{Une démonstration \textit{géométrique}} Avicenne cherche à faire une démonstration de l'existence de Dieu dans la logique de la \textit{géométrie} axiomatique d'Euclide, et en se basant sur la définition de la notion de l'existence en tant existence sans recours à l'expérience.
. 
\begin{quote}
    [même] « s’il est presque évident (manifestum) par soi pour l’intelligence que tout ce qui commence a un principe, ce n’est pas pour cela que [cette proposition] doit être évidente par soi à la manière dont beaucoup de réalités géométriques prouvent les autres dans le livre d’Euclide » \sn{Avicenne, Philosophia prima I, 1 (I, 8, 33-36)}.
\end{quote} 
 

Mais il doit reconnaître la difficulté de l'approche : 
 
\begin{quote}
     « Mais nous, en raison de la faiblesse de notre âme, nous ne pouvons pas commencer par la voie démonstrative elle-même, qui procède des principes aux conséquents, et de la cause au causé, sauf dans certains ordres d’universalité au sein de ce qui est, sans descendre dans le particulier (sine praecisione). » \sn{Philosophia prima I, 3 (I, 23, 37-41).} 
\end{quote}

\paragraph{La méthode géométrique} L'idée d'Avicenne est de partir de notions disjonctives qui permettent d'appliquer le principe logique "ou bien ou bien". Ces principes disjonctifs seront par exemple : \textit{possible} et \textit{nécessaire}, \textit{interne} et \textit{externe}. \sn{Philosophia prima I, 1 (I, 6, 12-17) : « Inquirit enim universale et particulare, potentiam et effectum, possibile et necesse, et cetera. »} 

La catégorie la plus importante étant  le nécessaire (necesse), car il garde des affinités avec l’être transcendantal, pensé comme le « vehementiam essendi », la détermination à exister, l’affirmation de l’être (\cite{Boulnois:EtreRepresentationAvicenne}).

En utilisant les catégories de la géométrie euclidienne, Avicenne part de l'axiome de \textit{l'étant} : \textit{quelque chose existe}. 
Au sein de l’étant, Avicenne ajoute une disjonction : certains sont par eux-mêmes possibles, non nécessaires, tandis que les autres existent par eux-mêmes nécessairement (\textit{necesse esse per se}). Si cette chose qui existe est nécessaire, alors il y a un existant nécessaire, CQFD. 
\paragraph{Tout ce qui est possible a une cause} : le possible n'existe que si par rapport à sa cause il est nécessaire \sn{Philosophia prima I,6}. Le possible considéré en lui-même n'est déterminé ni à l'existence ni à la non-existence. Il faut cependant qu'il y ait un facteur qui le détermine dans un sens ou dans l'autre : un facteur ne peut être que quelque chose de distinct de la nature du possible, une cause extérieure.
    C'est donc cette cause qui déterminera le possible à l'une des deux alternatives, car si elle n'était pas capable de le faire, il faudrait faire appel à une troisième cause et ainsi à l'infini. 
    La thèse d'Avicenne signifie donc que le possible existera si sa cause le fait exister et qu'il n'existera pas si sa cause le fait pas exister.
    
Donc si cette chose qui existe est possible, alors elle a une cause.  A noter que l'existant possible, quant à son essence, restera toujours ce qu'il est, à savoir un existant possible : le fait d'être un existant possible n'est pas un caractère accidentel et passager \sn{Philosophia prima 1, 7}. l'existant possible ne peut donc devenir un être de soi nécessaire \cite[p. 52]{Avicenne:latinus}.
 
\paragraph{Passage à la totalité des choses possibles} Puis Avicenne passe à la totalité des choses possibles; cette totalité est soit nécessaire en soi, soit possible en soi. Or, en logique, la totalité ne peut être nécessaire en elle-même  puisqu'elle n'existe que par l'existence de ses membres, sans existence propre. Ainsi, la totalité des choses possibles a une cause.
Cette cause est soit interne à la totalité, soit externe à celle-ci.
Si elle est interne à la totalité, alors elle est soit nécessaire, soit possible.
Mais elle ne peut dans ce cas être nécessaire, car la totalité est constituée de choses possibles.
Et elle ne peut pas non plus dans ce cas être possible, puisqu'en tant que cause de toutes les choses possibles, elle serait dans ce cas sa propre cause, ce qui la rendrait nécessaire et non possible après tout, ce qui est une contradiction.
Ainsi, la cause de la totalité des choses possibles n'est pas interne à cette totalité, mais externe à elle.
Mais si elle est en dehors de la totalité des choses possibles, alors elle est nécessaire.
Il y a donc un existant nécessaire ou un \textit{nécessairement-être par soi}, qui est sans cause,  nécessaire par lui-même, incomparable, sans égal \sn{Philosophia prima I, 6 (I, 43, 9-18).}. 
 
 
\paragraph{Unicité du nécessairement-être.} De plus, le concept de nécessairement-être permet de démontrer l’unicité de son objet par l'absurde \sn{Philosophia prima I, 7 (I, 54, 33-35)}. 

% ----------------------------------------------
\subsection{Conséquence de cette démonstration}


\paragraph{Conséquence sur le libre Arbitre}

   Cette position ne conduit-elle pas à un déterminisme universelle ? En effet \sn{\cite[p.55]{Avicenne:latinus}},
si l'existence ou la non existence dépendent entièrement de la cause extérieure, ne faut il pas dire que tout est fixé et qu'il n'y a plus aucune marge laissée à la contingence ? En fait, Avicenne propose plusieurs limites.

Tout d'abord, d'un point de vue cosmologique, Avicenne reprend les 10 sphères célestes de Ptolémée, qui permettent de passer de la toute puissance du \textit{nécessairement-être} à l'homme : dans le monde sublunaire dans lequel nous vivons, la Dixième Intelligence  issue de l'Intelligence du 9° ciel (la Lune), mais sans fonction astronomique, revêt une importance singulière. Elle est aussi appelée \textit{intellect agent} et associée à Gabriel dans le Coran. Mais elle se situe si loin du Principe que son émanation éclate en une multitude de fragments. A chaque sphère, la part du principe diminue, ce qui permet le hasard et la liberté de l'homme.

\paragraph{Attributs de Dieu}
La critique de la démonstration par Al-Ghazali insiste sur l’incompatibilité avec Dieu tel qu'on le connait dans la religion musulmane. Comme Dieu est le \textit{nécessairement-être}, il ne peut avoir de propriétés ou relations contingentes : la causalité de l'univers doit être nécessaire. Al Ghazali réfute ceci comme étant incompatible avec le concept de volonté sans limite de Dieu tel que l'enseigne la théologie ash'arite, volonté sans limite qui aurait permis à Dieu ne pas créer l’univers.

% -------------------------------------------------------------------------------------------------------
\section{Conclusion}

Nous avons voulu montrer que ces démonstrations sont des clés de voûte de la métaphysique de ces théologiens, qui synthétisent les \textit{axiomes} de leurs pensées : « A quoi on tient en vrai ». Or, cette métaphysique, ce \emph{Kalām} est à son tour la source qui permet d’articuler les différentes sources du droit et in fine l’action du musulman.
La sophistication de ces démonstrations est là pour inclure certains principes au cœur de la Foi et non pour convaincre \textit{principalement} de l’existence de Dieu. 

Pour sortir d'un Islam pensé autour du droit, beaucoup de musulmans appellent de leur voeux un renouveau du \emph{Kalām}. Les \textit{preuves de l'existence de Dieu}, débarrassées de leurs visées apologétiques, peuvent être des \textit{synthèses utiles} des axiomes et fondements du \emph{Kalām}, permettant confrontations et discussions théologiques pour ensuite irriguer les autres questions métaphysiques, et in fine vivifier la foi musulmane et lui permettre de répondre aux nouvelles questions qu'elle doit affronter. 
	
