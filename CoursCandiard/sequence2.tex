 
 
\chapter{Le muʿtazilisme} 


\mn{Candiard : seance 2}
Nous avons terminé la séance précédente en évoquant la figure aux
facettes multiples d'al- Ḥasan al-Baṣrī\sn{p. \pageref{Theol:HasanAlBasri}}, un savant musulman de la ville
irakienne de Baṣra actif entre la fin du VII\textsuperscript{e} siècle
et le début du VIII\textsuperscript{e}. C'est précisément dans son
milieu le plus proche que va apparaître, avec \emph{Wāṣil ibn ʿAṭāʾ} (m. en
748), un de ses disciples, la première véritable école de théologie
islamique. 
\begin{Def}[Origine des muʿtazilites]
On raconte que le Wāṣil, le disciple, s'était éloigné du
maître, et ce dernier l'aurait constaté avec tristesse : « Wāṣil s'est
éloigné de nous », donnant à la nouvelle école le nom qui lui restera :
les « séparés », les « éloignés », en arabe les muʿtazilites. 
\end{Def}
Si
l'anecdote est très incertaine, et l'origine du nom toujours sujette à
débat\sn{ Si l'on en reste quoi qu'il en soit au stade des hypothèses sur la
naissance du mouvement, les chercheurs contemporains (notamment Josef
van Ess) tendent plutôt à comprendre cet « éloignement » comme une forme
de neutralité politique dans les luttes de pouvoir du temps. Toutefois,
si ce positionnement politique attentiste a pu définir les premiers
muʿtazilites, le mouvement n'a été que marginalement un mouvement
politique.}, il semble bien que l'école naissante ait en tout cas rapidement
fait voir la nouveauté de sa proposition.

L'école muʿtazilite est certainement la première véritable école de
théologie islamique, avec un ensemble cohérent de doctrines couvrant
différents domaines, des maîtres reconnus et une forme d'affiliation,
mais aussi différents courants nourrissant un riche débat interne et
externe. À la cour des califes abbassides, elle joue un rôle de tout
premier plan dans l'animation du débat intellectuel et religieux. Elle
est restée importante, depuis son apparition à la fin du
VIII\textsuperscript{e} siècle jusqu'au XII\textsuperscript{e} siècle
dans le monde sunnite, où elle a progressivement disparu par la suite,
tout en conservant des survivances dans le monde chiite. Au XX\textsuperscript{e} siècle et jusqu'à présent, des
mouvements néo-muʿtazilites se réclameront de
son héritage, mais sans continuité historique avec leur ancêtre
revendiqué.

\paragraph{Problème de sources }Cette disparition du mouvement pendant plusieurs siècles et sa
réputation hétérodoxe expliquent que nous rencontrions une fois encore
un problème de sources à son égard. Contrairement au mouvement qadarite,
nous disposons de textes théologiques muʿtazilites de première main,
mais en nombre insuffisant par rapport à l'influence que ce mouvement a
pu avoir, et plusieurs périodes de l'école nous restent mal connues
(nous ne disposons ainsi pas d'un seul traité théologique muʿtazilite
complet datant du premier siècle du mouvement : pour cette période
importante de formation de l'école, il faut nous contenter d'extraits,
souvent transmis par des doxographes ou des adversaires). Notre
connaissance du mouvement repose sur une patiente reconstruction des
sources, pour laquelle il faut mentionner ici l'importance du travail de
l'islamologue allemand Josef van Ess, dont nous avons déjà parlé. Ses
volumes, intitulés \emph{Theologie und Gesellschaft im 2. und 3.
Jahrhundert Hidschra}, traduits en anglais (mais hélas pas en français
!\sn{2 Le lecteur purement francophone, ou pressé, se consolera en lisant de
suggestives conférences, regroupées dans J. van Ess, 2002.}), sont un outil indispensable pour quiconque entend travailler sur le
mouvement. Toutefois, la recherche progresse, et la découverte de
manuscrits de textes muʿtazilites importants, restés inconnus à prendre
la poussière pendant des siècles, est loin d'être terminée.

\hypertarget{le-rationalisme-muux2bftazilite}{%
\section{Le rationalisme
muʿtazilite}\label{le-rationalisme-muux2bftazilite}}

Le muʿtazilisme est d'abord connu comme un \emph{rationalisme}\mn{\begin{Def}[Rationalisme]  le recours à la raison logique et dialectique comme voie préférentielle d’accès à la vérité. Pour le rationalisme islamique, cette méthode n’exclut pas l’existence d’une Révélation de Dieu, par le Coran, mais il privilégie les méthodes rationnelles, plutôt que le recours à la tradition, pour interpréter et expliquer le sens du texte saint. 
\end{Def}}. Cette
caractérisation n'est pas fausse, mais elle peut aussi induire en erreur
; une partie de la popularité du muʿtazilisme en Occident repose sur des
malentendus liés aux ambiguïtés de cette notion très vaste. Le
rationalisme des muʿtazilites n'est pas celui des philosophes des
Lumières européennes du XVIII\textsuperscript{e} siècle, encore moins
celui des libres-penseurs en lutte contre les religions. Si l'islam
médiéval a connu des formes de scepticisme religieux, de critique de la
religion voire certaines expressions d'athéisme, le muʿtazilisme n'a
rien à voir avec cela. De même, le mouvement est très différent du
mouvement philosophique islamique hérité de la pensée grecque, qu'on
appelle d'un mot grec arabisé la «\emph{falsafa}», bien qu'il ait
recours à des outils (en particulier des outils logiques) fournis par la
philosophie.

Le rationalisme des muʿtazilites est un \emph{rationalisme théologique},
profondément religieux. Il ne s'agit pas non plus d'une théologie
naturelle, purement conceptuelle, qui ne prendrait pas en compte les
textes sacrés de l'islam. Le mouvement, au contraire, se donne pour
objectif de rendre compte de la vérité de la révélation coranique, qui
n'est pas questionnée ni remise en cause, mais dont on cherche au
contraire à comprendre le sens authentique. Pour comprendre adéquatement
le sens du Livre saint, qui appelle nécessairement des explications, les
muʿtazilites se distinguent des traditionnistes (les «proto-sunnites»
précédemment mentionnés) qui se fondent sur les paroles du Prophète, en
préférant employer à cette fin la raison.

Le raisonnement apparaît donc comme un instrument au service de la
révélation, et certainement pas un moyen de la contourner ou de la
limiter. La raison va aider le croyant à déterminer le sens véritable de
versets coraniques difficiles, qu'il ne faut pas comprendre en leur sens
obvie. Quand le Coran dit, par exemple, que Dieu est assis sur un trône,
il n'entend pas nous faire admettre que Dieu possède un corps capable de
s'asseoir, ni qu'il se trouve en un lieu précis : le comprendre ainsi,
comme le fait la croyance populaire, n'est pas digne de Dieu, dont nous
savons qu'il n'a ni lieu ni corps. Le rationalisme des muʿtazilites
cherche donc à veiller à la cohérence de la doctrine musulmane (qui ne
peut à la fois tenir que Dieu n'a pas de corps, et en même qu'il est
\emph{littéralement} assis sur un trône) et à la purification des
représentations humaines de Dieu.

\begin{Synthesis}
Le muʿtazilisme est un rationalisme théologique, qui cherche à mettre en cohérence la doctrine musulmane, en s'appuyant sur la Raison plutôt que sur les paroles du Prophète. Il n'est pas une spéculation philosophique ("falsafa") car il est profondément religieux.
\end{Synthesis}
\hypertarget{les-cinq-principes-du-muux2bftazilisme}{%
\section{Les « cinq principes » du
muʿtazilisme}\label{les-cinq-principes-du-muux2bftazilisme}}

Ce souci de cohérence rationnelle va pousser les auteurs muʿtazilites à
dégager des principes doctrinaux dans lesquels ils se reconnaissent.
Deux d'entre eux les caractérisent au premier chef : les muʿtazilites se
désignent volontiers eux-mêmes, en arabe, comme les
\textsc{« partisans de l'unicité et de la justice {[}de Dieu{]} » }(\emph{ahl
al-tawḥīd wa-l-ʿadl}). On reviendra plus longuement, dans le prochain
paragraphe, sur les conséquences qu'ils tirent de leur attachement
intransigeant à l'unicité divine. L'affirmation de la justice de Dieu,
elle, se traduit par le refus de la prédestination et par la doctrine du
libre-arbitre (si Dieu est juste, alors il ne peut punir et récompenser
que des hommes responsables de leurs actes), qu'ils héritent du
mouvement des qadarites, dont nous avons parlé à la séquence précédente,
mais qu'ils reformulent de façon incomparablement plus précise et
complexe, dans le cadre d'une théorie atomiste du monde.

Les muʿtazilites prennent bientôt position dans un troisième débat
théologique, qui a divisé la communauté musulmane dès ses premières
années : le statut des musulmans coupables de péchés majeurs
(\emph{kabāʾir}). Faut-il considérer que, s'étant par leur péché
révoltés contre la Loi divine, ils ont donc cessé d'être musulmans et
doivent être traités comme des apostats ? Quand on sait que l'apostasie
est généralement punie de mort, c'est une solution évidemment radicale.
Faut-il au contraire laisser le jugement à Dieu, qui seul connaît le
fond des cœurs, et rappeler que, bien que pécheur, le coupable reste un
croyant ? Dans ce débat, les muʿtazilites proposent une solution de
compromis, celle de « l'état intermédiaire » (\emph{al- manzila bayn
al-manzilatayn}) : Wāṣil et ses successeurs considèrent que le coupable
est un
« pécheur grave » (\emph{fāsiq}), certainement destiné aux peines de
l'Enfer, mais refusent que cela
modifie son statut légal de musulman.

Progressivement, deux autres principes viennent s'ajouter à ces trois
premiers, pour
former les fameux « cinq principes », peut-être formalisés pour la
première fois par le
muʿtazilite Abū al-Huḏayl (m. en 840), qui constitueront la base commune
des penseurs

muʿtazilites. Ces deux principes sont :

\begin{itemize}
\item

  « la promesse et la menace » (\emph{al-waʿd wa-l-waʿīd}), sur la
  réalité des peines de l'Enfer et des
délices du Paradis annoncées dans le Coran ;


\item
 
  « la commanderie du bien et le pourchas du mal » (\emph{al-amr
  bi-l-maʿrūf wa-l-nahī ʿan al- munkar}), qui justifie le souci d'ordre
  public et l'intervention dans la moralité d'autrui. C'est ce dernier
  principe, sans doute, qui justifiera que la doctrine muʿtazilite
  puisse être imposée au besoin par la force (cf. le paragraphe 5 de
  cette séquence) ; il confirme que les muʿtazilites, rationalistes, ne
  sont pas pour autant des libéraux !

\end{itemize}

Ces principes communs n'empêcheront pas le mouvement muʿtazilite
d'abriter en son sein des opinions divergentes. Au fil du temps, après
les premières générations, des écoles distinctes vont même apparaître et
s'opposer, en particulier entre les muʿtazilites de Baṣra, au sud de
l'Irak, et ceux de la capitale de l'empire, Bagdad.

\begin{Def}[Cinq Principes muʿtazilites]
Formalisé par Abū al-Huḏayl :
\begin{itemize}
    \item \emph{tawḥīd}, intransigeance sur l'unité de Dieu
    \item \emph{ʿadl}, la {Justice de Dieu} et la doctrine du libre-arbitre
    \item le musulman coupable de péchés majeures (\emph{kabāʾir}) est un pécheur grave (\emph{fāsiq}) mais reste Musulman.
    \item dimension eschatologique : réalité des peines de l'Enfer et des délices du Paradis
    \item commanderie du bien et pourchas du mal, et intervention sur la moralité d'autrui. (<> libéralisme)
\end{itemize}

\end{Def}

\hypertarget{la-question-de-dieu}{%
\section{La question de Dieu}\label{la-question-de-dieu}}

Mais la question essentielle, pour les muʿtazilites, est la question
théologique par excellence : 
\begin{quote}
    comment le langage humain peut-il parler
adéquatement de Dieu ?
\end{quote}
 L'absolue transcendance de Dieu, qui est
radicalement différent de sa créature comme le rappelle le Coran (« Rien
n'est semblable à lui », dit le verset 11 de la sourate 42), n'interdit
pas au langage de parler de lui, estiment-ils, puisque Dieu lui-même
s'est révélé dans un livre, le Coran, écrit dans la langue des humains ;
mais la lecture de ce livre suppose que le lecteur respecte cette
transcendance, et purifie donc sa propre lecture de tout ce qu'elle peut
prêter de faux à Dieu, en le faisant ressembler au monde créé. Cette
opération de purification s'appelle en arabe le \emph{tanzīh}\sn{\begin{Def}[Tanzīh] Ce mot arabe, qui renvoie à l’idée d’écarter, d’éloigner, désigne en théologie le processus de purification par lequel on refuse d’attribuer à Dieu tout ce qui est indigne de lui, tout ce qui ne respecte pas son absolue transcendance (en particulier, donc, des caractères anthropomorphiques) ; par extension, le mot en vient à désigner cette transcendance elle-même. 
\end{Def}
\begin{Def}[Tašbīh]
 Ce mot arabe (qui signifie « trouver ressemblant, déclarer proche, assimiler ») désigne l’attitude par laquelle le croyant peut décrire Dieu comme il décrirait une créature (par l’anthropomorphisme, par exemple), au risque de ne pas respecter sa transcendance. C’est précisément ce que les muʿtazilites, soucieux de respecter par-dessus tout la transcendance divine, cherchent à éviter.  
\end{Def}}, c'est-à-dire « l'éloignement » de tout ce qui ne convient pas, de ce qui
serait indigne de Dieu. Elle s'oppose symétriquement au \emph{tašbīh},
« l'assimilation » de Dieu aux créatures, à qui on voudrait le faire
ressembler. Les muʿtazilites ne manqueront pas d'en accuser leurs
adversaires, rétifs à l'opération de purification qu'ils proposent ;
mais à l'inverse, leurs adversaires accuseront volontiers les
muʿtazilites, tout occupés à éviter le \emph{tašbīh}, de verser dans
l'excès inverse, qu'on appelle le \emph{taʿṭīl}*, ou « dépouillement »,
c'est-à-dire le refus de laisser à Dieu des attributs qu'il s'est
pourtant lui-même donnés.
\begin{Def}[Taʿṭīl ]
Ce mot arabe, qui signifie « dépouillement,
désigne l'erreur symétrique à celle du \emph{tašbīh}. Il ne s'agit plus
cette fois de prêter à Dieu les qualités des créatures, mais de refuser
d'accepter pour Dieu le moindre attribut, puisque tous nos mots sont par
définition faits pour décrire des créatures. Le risque est alors de ne
plus rien pouvoir dire sur Dieu. Les muʿtazilites, très soucieux
d'éviter le \emph{tašbīh}, se verront souvent accusés par leurs
adversaires de verser dans le \emph{taʿṭīl}.
\end{Def}


\emph{Tašbīh} et \emph{taʿṭīl}, ces deux
opposés dont l'un prend trop littéralement les expressions du Coran
quand l'autre ne leur fait pas assez crédit, sont l'un et l'autre des
péchés. Ils montrent la ligne de crête que constitue, pour les
musulmans, la lecture du Coran : on a vite fait d'être accusé de tomber
d'un côté ou de l'autre ; et à trop craindre l'un, on peut vite
s'engouffrer dans l'autre.

Or le Dieu révélé par le Coran offre à première vue bien des prises au
\emph{tašbīh}, à la ressemblance avec les créatures. Tout d'abord parce
que de nombreux versets en donnent une description anthropomorphique* :
Dieu a des mains, il est assis sur un trône, ou il se met en colère. À
cet égard, la purification opérée par les muʿtazilites est aisée : à
l'instar des lectures chrétiennes du Dieu de la Bible, il s'agit de voir
dans ces expressions des métaphores. La main de Dieu ne désigne pas une
main corporelle, mais son action ou sa grâce ; son assise sur un trône
signifie sa puissance\ldots{} Ces attributs anthropomorphiques sont
également nombreux dans les \emph{ḥadīṯ}-s, ces propos rapportés du
Prophète qui occupent tant les traditionnistes : les muʿtazilites
n'hésitent pas à les nier, et font, pour plusieurs d'entre eux (comme
al-Naẓẓām, m. autour de 840), très peu de cas de cette littérature
qu'ils jugent au mieux secondaire vis-à-vis du Coran, au pire inutile et
nuisible.

On peut juger un peu naïve cette attention portée aux caractères
corporels de Dieu, mais la question a des conséquences importantes. En
effet, fera-t-on remarquer bien vite aux muʿtazilites, si Dieu n'est en
rien corporel, alors que faire des promesses du Coran et du \emph{ḥadīṯ}
selon lesquelles, au paradis, nous verrons Dieu ? Si Dieu n'est pas
visible, s'il n'est pas accessible aux sens, cette promesse est-elle
donc vide de signification ? Faut-il y voir une simple image, au risque
de jeter un doute sur la réalité de la récompense promise au croyant ?
Le Coran n'est pas avare de descriptions très détaillées des joies du
Paradis comme des
tourments de l'Enfer ; faut-il n'y voir là encore que des métaphores ?
\textsc{La « purification »
muʿtazilite provoque des inquiétudes.
}
Le Coran attribue également à Dieu bien d'autres qualités qui ne sont
pas corporelles et dont il semble difficile de faire une interprétation
simplement métaphorique. Trois attributs divins en particulier
retiennent l'attention : la Science, la Puissance et la Vie. Il est
difficile au lecteur du Coran d'ignorer que, dans ce livre, Dieu est
constamment présenté comme savant, puissant et vivant. Or pour les
muʿtazilites, cela entraîne plusieurs difficultés. Faut-il comprendre
cette connaissance divine à la manière d'une connaissance humaine ? quel
est l'objet de sa science ? s'il connaît les objets changeants et
temporels du monde, comment reste-t-il, lui, immuable et éternel ? Mais
une difficulté surpasse bientôt toutes les autres. Les muʿtazilites font
de l'unicité divine (\emph{tawḥīd}), qui est une affirmation très nette
maintes fois répétée par le Coran, l'élément essentiel de leur
théologie, et la pierre de touche de leur lecture. Si l'on affirme que
Dieu a une Science, une Puissance ou une Vie, ne risque-t-on pas de
compromettre cette unicité absolue, en affirmant qu'existent, à côté de
Dieu, d'autres entités éternelles associées à Dieu ? Sauf à supposer que
la Vie de Dieu n'est pas elle-même éternelle, ce qui n'a évidemment
aucun sens. Mais si Dieu a une Vie éternelle, une Puissance éternelle,
une Science éternelle, le ciel commence à se peupler d'entités
semi-divines, qui ne sont pas Dieu mais qui ne sont pas pour autant des
créatures. Quel \textbf{statut ontologique}* faut- il leur réserver ?

Dès la naissance du mouvement, les muʿtazilites vont refuser,
précisément, d'en faire des entités à part. Ils acceptent de dire que
Dieu est savant, puissant et vivant, non qu'il serait le possesseur
d'une Science, d'une Puissance et d'une Vie qui existeraient par
elles-mêmes. Dieu est savant, mais il ne faut pas chercher, au ciel ou
sur la terre, une réalité distincte qui serait la « Science divine ». Il
s'agit d'éviter de donner à Dieu des associés. Mais ce refus est loin de
régler toute la question, et il se limiterait à un jeu de langage si
l'on n'explicitait pas ce que signifie véritablement une expression
comme \emph{Dieu est savant} ou \emph{Dieu est vivant}.

\paragraph{Une première reponse, théologie Négative } Un muʿtazilite relativement marginal, Ḍirār ibn ʿAmr (m. en 815),
proposa une première réponse par une approche négative. Pour lui, quand
le Coran dit que \emph{Dieu est savant}, il ne dit pas quelque chose de
positif, car il nous est impossible de savoir ce que connaître signifie
pour Dieu, et nous en venons à plaquer sur Dieu nos conceptions humaines
et limitées du savoir. Pour autant, le Coran nous dit quelque chose
d'important : quand il affirme que \emph{Dieu est savant}, il nie qu'il
soit ignorant ; quand il dit que \emph{Dieu est puissant}, il écarte de
lui toute idée de faiblesse ; quand il répète que \emph{Dieu est
vivant}, il nie en lui toute espèce de mort. Les affirmations du Coran
relatives à Dieu sont donc pour lui des négations de négations, le refus
du contraire de l'attribut appliqué à Dieu.

Cette théologie négative, dont les racines sont à chercher dans les
traditions grecque et chrétienne, sera adoptée par les théologiens
muʿtazilites, mais bientôt complétée par une approche plus affirmative,
proposée par Abū al-Huḏayl. Comment Dieu peut-il être savant sans
qu'existe une Science divine ? C'est que, répond Abū al-Huḏayl,
\begin{quote}
    « il est
savant d'une science qui est Lui-même, puissant d'une puissance qui est
Lui-même, vivant d'une vie qui est lui-même »\sn{
 Cité dans les \emph{Maqālāt} d'al-Ašʿarī (165)
}.
\end{quote} 
L'attribut divin est
identifié à l'Essence même de Dieu. En d'autres termes, dire que Dieu
est savant, c'est affirmer que Dieu est.

Cette solution, que Abū al-Huḏayl combine avec l'approche négative, sera
vite adoptée par l'ensemble du mouvement, car elle permet de rendre
compte des formulations de la révélation coranique sans pour autant
risquer de supposer des entités éternelles associées à Dieu. Entre le
Créateur et ses créatures, il n'y a pas de troisième terme, d'êtres
intermédiaires. Mais cette solution ne va pas sans difficultés. Si les
attributs divins comme la Science ou la Puissance renvoient tous à la
seule Essence divine, alors leur distinction perd toute signification :
\emph{Dieu est savant} signifie la même chose que \emph{Dieu est
puissant}. On ne comprend guère, dès lors, pourquoi la Révélation a pris
la peine d'employer des termes distincts. Mais surtout, identifier tout
ce que le Coran dit de Dieu à sa seule Essence par nature
inconnaissable, c'est faire de Dieu un être lointain et très abstrait,
auquel il est difficile de
demander aux croyants de donner leur foi avec enthousiasme.

\begin{Synthesis}
La question des attributs anthropomorphiques se traduit dans la théologie \mzt par un effort de purification (Tanzīh) pour enlever ce qui est indigne de Dieu et l'assimilerait à une créature (tašbīh). Mais comment interpréter la vision de Dieu au Paradis et les affirmations du Coran selon que \textit{Dieu est savant, puissant et vivant} ?   une théologie négative \textit{Dieu n'est pas ignorant} puis une théologie ontologique \textit{Dieu est la science}\mn{cf définition chrétienne de \textit{Dieu amour}, \textit{Dieu tout puissant}. Avons nous le même problème ? cf Varillon ?} mais avec le risque de perdre tout sens au mot science, puissance et vie
\end{Synthesis}
\hypertarget{la-question-du-coran-cruxe9uxe9}{%
\section{La question du Coran
créé}\label{la-question-du-coran-cruxe9uxe9}}

Dans la mémoire collective, en particulier chez les musulmans, les
muʿtazilites restent d'abord associés à leur doctrine du « Coran créé »,
sans qu'on comprenne toujours de quoi il est question dans ce débat. Il
n'est pas rare de lire dans la presse que cette doctrine, rejetée par
l'orthodoxie islamique, aurait permis de développer des lectures
historico-critiques du Coran\ldots{} Ces affirmations sont parfaitement
fantaisistes, car dans cette querelle du « Coran créé », il n'est en
fait pas tellement question du Coran lui-même ou de son interprétation :
il s'agit d'une ramification du débat sur les attributs divins, et de
l'effort des muʿtazilites pour maintenir l'absolue unicité de Dieu.

\paragraph{Tous les musulmans s'accordent à voir dans le Coran la Parole de Dieu.}
Mais, comme pour les autres attributs divins, quel statut ontologique
faut-il donner à cette Parole ? De deux choses l'une, estiment les
muʿtazilites : ou bien elle est Dieu, ou bien elle est une créature, car
il n'existe pas de statut intermédiaire. Si la Parole de Dieu est
éternelle et incréée, n'est-ce pas une manière de donner raison aux
chrétiens qui, dans les controverses sur la Trinité, affirment que Dieu
a un Verbe, une Parole, qui lui est co-éternel ? Peut-on affirmer que le
Coran est Dieu ? Cela paraît impossible aux muʿtazilites, qui adoptent
donc l'opinion contraire : dans la grande division de toute chose entre
Créateur et créatures, le Coran est une créature. Ils savent qu'ils
s'engagent par là même devant des discussions sans fin (si la Parole de
Dieu n'est pas éternelle, cela signifie-t-il qu'il fut un temps où Dieu
ne parlait pas ?), mais c'est la cohérence de leur affirmation de
l'unicité absolue qui est en jeu. La Parole est un attribut comme un
autre, à ceci près toutefois que contrairement à la Science ou la Vie,
cet attribut prend une forme concrète et subsiste dans un Livre. Il
était facile d'affirmer que Dieu était savant, mais que sa Science
n'existait pas ; mais peut-on dire que Dieu parle, mais que sa Parole
n'existe pas, alors qu'on a le Coran sous les yeux ?

Quand le muʿtazilisme deviendra un enjeu politique (voir paragraphe 5 de
la séquence), c'est sur ce point -- sans doute plus concret pour les
croyants que l'abstraite identité de l'Essence et des attributs -- que
se cristalliseront les ralliements et les oppositions à la doctrine
muʿtazilite, ce qui explique sans doute sa place dans les mémoires.


\begin{Synthesis}
Pour préserver l'unicité de Dieu, la Parole de Dieu et le Coran doivent être une créature. Car sinon, on risque donner crédit au christianisme qui affirme que la Parole co-existe avec Dieu de toute éternité. \mn{D'une certaine manière, est ce que la théologie \mzt n'a pas donné trop de crédit aux questions chrétiennes ?}
\end{Synthesis}

\hypertarget{luxe9cole-muux2bftazilite-et-le-pouvoir-califal}{%
\section{L'école muʿtazilite et le pouvoir
califal}\label{luxe9cole-muux2bftazilite-et-le-pouvoir-califal}}

Le mouvement muʿtazilite apparaît à la fin de la dynastie omeyyade, sans
lien particulier avec le pouvoir : c'est un mouvement d'intellectuels,
probablement assez peu nombreux. Mais alors que les qadarites avaient
provoqué la suspicion et la répression du pouvoir omeyyade, les
muʿtazilites semblent au contraire vus d'un bon œil par la dynastie
abbasside qui lui succède en 749. L'ambition des califes de faire de
leur nouvelle capitale, Bagdad, le centre du monde intellectuel et
culturel s'accommode bien de cette tentative de donner à l'islam une
théologie spéculative. Sous le règne du célèbre calife Hārūn al-Rašīd
(m. en 809) \sn{5\textsuperscript{ème} calife abbaside. A noter qu'il imposa le papier dans tout l'empire, papier découvert par le contact avec la Chine}, les muʿtazilites participent aux nombreux débats
intellectuels qui fleurissent dans les grandes maisons de la capitale --
notamment chez les puissants vizirs, les Barmécides. Ils y occupent
souvent un rôle de tout premier plan.

Mais la visée des souverains abbassides est plus vaste : contrairement à
leurs prédécesseurs omeyyades, qui régnaient sur un empire de conquêtes,
ils entendent faire de l'islam le ciment idéologique de l'unité de
l'empire. Les grandes institutions classiques de l'islam se mettent
alors en place. Dans ce contexte, le rationalisme théologique des
muʿtazilites apparaît au calife al-Maʾmūn (m. en 833), fils de Hārūn,
comme une doctrine susceptible d'unifier l'empire sous son autorité.
Sans doute y voyait-il également une manière d'affirmer l'autorité du
calife en matière proprement religieuse, contre la concurrence de la
classe émergente des ulémas.

Professant la foi muʿtazilite en matière d'attributs divins (et donc sur
la question du Coran créé), le calife décide de l'imposer à ses sujets,
par la force au besoin. Quelques mois
avant sa mort, une forme d'inquisition muʿtazilite entreprend de
s'assurer de la foi des fonctionnaires, en particulier des juges et des
autorités religieuses, sur la question du Coran.
\begin{Def}[miḥna]
On désigne en général
cette persécution par le mot arabe de \emph{miḥna}, qui désigne «
l'épreuve » imposée aux croyants refusant la théologie muʿtazilite et la
profession de foi exigée sur la création du Coran. \mn{\href{Mihna sous Wikipedia}{https://fr.wikipedia.org/wiki/Mihna}}
\end{Def}  Elle a laissé de très
mauvais souvenirs dans la mémoire islamique, et écorne sérieusement
l'image de rationalistes libéraux que peuvent avoir les muʿtazilites
dans une certaine historiographie occidentale.

Cette persécution religieuse, poursuivie par les successeurs
d'al-Maʾmūn, durera une quinzaine d'années, avant que les califes ne
doivent y renoncer, reconnaissant l'échec de la tentative qui s'est
heurtée à une opposition déterminée, sur laquelle nous reviendrons dans
la prochaine séance.

\begin{Synthesis}
Face au pouvoir des Oulemas, les califes abassides (809-850) adoptent la théologie \mzt et vont imposer une véritable inquisition et une persécution (Mihna) autour de l'affirmation du \textit{Coran créé}
\end{Synthesis}

\section{Lecture}
\paragraph{Le « Credo » \mzt} : Al-Ašʿarī, Maqālāt al-islāmiyyīn, éd.
Helmut Ritter, 1980,p. 155-156.\sn{Trad. Badawi, \emph{Histoire de la philosophie en islam}, p. 24-25,
légèrement corrigée.}
 
\emph{Vous allez à présent lire votre premier texte (très probablement
!) de théologie musulmane. Il s'agit d'un texte très célèbre, résumant
la doctrine muʿtazilite sur la question de Dieu ; pour cette raison, on
le désigne en général comme « le} Credo \emph{des muʿtazilites ». Cette
appellation ne doit toutefois pas nous tromper : ce texte n'est pas
écrit par un auteur muʿtazilite, mais par un adversaire du mouvement, le
grand théologien Abū al-Ḥasan al-Ašʿarī (m. en 936). Nous reparlerons de
cet auteur, qui a été longtemps muʿtazilite lui-même et connaît leurs
doctrines de l'intérieur. Dans son ouvrage} Maqālāt al-
islāmiyyīn\emph{, il s'efforce de rendre compte des positions de
différents courants de théologie islamique ; cette page, qui résume les
positions communes à tous les muʿtazilites, semble fidèle à leur
doctrine.}

\emph{Ce texte peut paraître d'abord difficile, mais si vous avez lu
attentivement les pages qui précèdent, vous avez en main les principales
clefs pour en comprendre le sens, et il ne devrait pas vous résister
très longtemps.}


 \begin{quote}
     Les muʿtazilites sont unanimes à professer que Dieu est un et « n'a pas
de pareil ; il est entendant et voyant » (Coran 42, 11).

Il n'est ni corps, ni personne, ni forme, ni chair, ni sang, ni
substance, ni accident ; il n'a ni couleur, ni goût, ni odeur, ni
toucher, ni chaleur, ni froid, ni humidité, ni sécheresse, ni longueur,
ni largeur, ni assemblage, ni séparation. Il ne se meut ni ne repose. Il
ne se divise pas ; il n'a pas de parties, ni d'organes, ni direction, ni
gauche, ni droite, ni devant, ni derrière, ni haut, ni bas. Aucun lieu
ne peut L'envelopper. Il n'est pas soumis au temps. On ne peut pas Le
toucher. Il n'est ni solitaire, ni en un lieu. On ne peut pas Lui
attribuer une qualité humaine qui implique la contingence. On ne peut
pas Le qualifier de fini, ni dire qu'Il a une étendue, ou une direction.
Il n'est pas limité. Il n'est pas engendré, et Il n'a pas engendré.
Aucun espace ne Le limite ; aucun rideau ne Le cache. Les sens ne
peuvent pas Le percevoir. On ne peut pas Le mesurer aux hommes ; Il ne
leur ressemble d'aucune manière que ce soit. Aucune maladie ne Le
frappe, aucune difformité ne L'atteint. Tout ce qu'on peut concevoir par
la raison ou l'imagination ne Lui ressemble en rien. De toute éternité,
il existe avant toute création. Il est éternellement savant, puissant et
vivant.


Il est une chose, mais pas comme les choses ; savant, puissant et
vivant, mais pas comme
les savants, les puissants et les vivants. Seul Éternel, rien d'autre
que Lui n'étant éternel, aucun autre
Dieu que Lui n'existant ; il n'a aucun associé dans son royaume, aucun
ministre de son règne, aucune aide qui L'aiderait à créer ce qu'Il a
créé. Il n'a pas créé la créature selon un modèle préexistant. Aucune
créature n'est pour lui plus facile à créer qu'une autre, ni plus
difficile. Il ne Lui convient pas de poursuivre des gains ; aucun mal ne
peut L'atteindre. Il n'éprouve ni joie, ni plaisir, ni peine, ni mal. Il
n'a pas de limite, ni de fin. Il ne peut pas périr, ni être atteint de
défaut ou d'impuissance. Il ne cohabite pas avec une femme, et il n'a ni
compagne ni enfants.
 \end{quote}


Pour faciliter la compréhension, j'ai divisé le texte en quatre
paragraphes assez inégaux, mais qui correspondent au mouvement du texte.
Le premier rappelle, avec le verset coranique que nous avons déjà cité
plusieurs fois, la clef d'interprétation essentielle des muʿtazilites :
l'affirmation de sa transcendance absolue à l'égard de toute créature.

 

 

Le second paragraphe correspond ensuite au mouvement de purification, de
\emph{tanzīh}, qui découle de cette affirmation : on nie de Dieu tout ce
qui est indigne de sa transcendance -- toute dimension corporelle, locale, temporelle ou sensible en
particulier, mais aussi tout ce que peut produire l'imagination ou la
raison. De plus, dès la première ligne, le texte exclut à propos de Dieu
certaines catégories d'origine philosophique, comme la substance et
l'accident : on entend ainsi écarter de Dieu toute idée de composition.
Son unicité s'accompagne donc d'une parfaite simplicité.

Un troisième paragraphe, très bref, vient compléter cette voie négative
par une voie affirmative. On ne sera pas surpris de la trouver bien
moins détaillée que la voie négative : le Dieu des muʿtazilites reste
largement inconnaissable. De lui toutefois, on peut affirmer l'éternité,
ainsi que les trois attributs essentiels que nous avons longuement
mentionnés : il est savant, puissant et vivant.

Cette affirmation est aussitôt tempérée par le quatrième paragraphe,
dont les premières formules pourraient nous paraître mystérieuses si
nous ne connaissions pas la théorie de l'identité de l'Essence et des
attributs. Dieu est savant, puissant et vivant, mais
« non pas comme les savants, les puissants et les vivants {[}parmi les
créatures{]} » : chez ces derniers, leur science, leur puissance ou leur
vie sont évidemment distincts de leur essence, à la différence de Dieu,
dont l'isolement ontologique vis-à-vis de tout être est rappelé.

%----------------------------------------------------
\section{Lexique}
\begin{quote}
\textbf{Anthropomorphisme :} Cela consiste à attribuer à Dieu des
attributs imités de ceux de l'homme, comme des éléments corporels (une
main, des yeux\ldots) ou des émotions (la colère, la pitié). Le Coran,
tout comme la Bible, use fréquemment d'un langage anthropomorphique pour
parler de Dieu. Les lectures chrétiennes de la Bible ont, depuis
l'Antiquité, conclu qu'il s'agissait de métaphores, d'images, qu'il
convient de ne pas prendre au sens littéral. En monde musulman, le sens
à donner à ces expressions a fait l'objet de débats plus âprement
disputés.

\textbf{Ontologie :} Il s'agit d'une partie de la philosophie, qui
étudie spécialement ce que signifie le mot « être ». S'interroger sur le
statut ontologique de quelque chose ou de quelqu'un, c'est se demander
sous quelle modalité il existe. Ainsi, la couleur rouge n'existe que
dans des objets, contrairement à une voiture, qui existe par elle-même.
Ou encore, un personnage de fiction, comme Jean Valjean, n'existe pas
comme ma sœur existe (elle qui n'est pas fictive, mais bien de chair et
d'os). Un concept existe-t-il ? En quel sens ? C'est autour de ces
questions que s'élaborent, dans les systèmes philosophiques ou
théologiques, des ontologies parfois très différentes.

\textbf{Rationalisme :} Il peut être défini comme le recours à la raison
logique et dialectique comme voie préférentielle d'accès à la vérité.
Pour le rationalisme islamique, cette méthode n'exclut pas l'existence
d'une Révélation de Dieu, par le Coran, mais il privilégie les méthodes
rationnelles, plutôt que le recours à la tradition, pour interpréter et
expliquer le sens du texte saint.

\end{quote}



%--------------------------------------------
\section{Bibliographie de la séquence 2}


\paragraph{Instruments de travail}
J. van Ess, Theologie und Gesellschaft im 2. und 3. Jahrhundert Hidschra : eine Geschichte des religiösen Denkens im frühen Islam, Berlin, De Gruyter, 1991 (6 vol.).


Trois articles du Oxford Handbook of Islamic Theology (sous la dir. de S. Schmidtke, 2016) sont
consacrés au mouvement muʿtazilite.


Études


A. Amir-Moezzi et S. Schmidtke, « Rationalisme et théologie dans le monde musulman médiéval. Bref état des lieux », in Revue de l’histoire des religions (2009/4), pp. 613-638.
J. van Ess, Les prémices de la théologie musulmane, Paris, Albin Michel, 2002.
R. Frank, « The divine attributes according to the teaching of Abū l-Hudhayl al-ʿAllāf », in Le Muséon 82 (1969), pp. 3-41.

\section{Questions de compréhension de la séquence}
     Après une lecture de la séquence, vous devriez être capable de répondre en une dizaine de lignes à chacune des questions suivantes. Ces questions ne servent qu’à votre auto-évaluation : il n’est pas obligatoire d’y répondre, et elles ne font pas l’objet d’une correction.  1. En quel sens peut-on dire que le mouvement muʿtazilite est rationaliste ?  2. Pourquoi l’existence d’attributs divins — anthropomorphiques, comme les mains, ou essentiels, comme la vie — pose-t-il problème aux muʿtazilites ?  3. Quelles stratégies les muʿtazilites mettent-ils en place pour donner un sens aux « attributs divins » présents dans le Coran et les ḥadīṯ-s ? 