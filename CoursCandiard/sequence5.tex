 \chapter{L’ašʿarisme classique}

 
\mn{Séance 5} 


\paragraph{Déclin des mu'tazilites et falsafa} Avec l'entrée en scène de l'école ašʿarite au début du Xe siècle, les
principaux acteurs du débat théologique islamique classique nous sont
désormais tous connus. Les siècles qui suivent vont voir le succès
grandissant de ce dernier arrivé, capable d'assimiler méthodes et
doctrines de certains de ses adversaires dans une synthèse de plus en
plus élaborée : c'est le cas des muʿtazilites, qui perdent leur
vitalité, mais aussi des philosophes d'expression arabe, en particulier
Avicenne, dont l'influence est grande sur l'ensemble de la pensée
islamique.


\hypertarget{le-lent-duxe9clin-du-muux2bftazilisme}{%
\section{Le lent déclin du
muʿtazilisme}\label{le-lent-duxe9clin-du-muux2bftazilisme}}


\paragraph{Ecole Mu'tazilite brillante au $X^è$} La résistance ḥanbalite et l'échec de la \emph{miḥna} --- cette ambition
des califes de faire du muʿtazilisme la doctrine orthodoxe de l'empire
---, pas plus que l'apparition de l'école ašʿarite, n'ont signifié la
disparition de l'école muʿtazilite. C'est au Xe siècle que l'école
brille au contraire de son éclat le plus grand ! Le débat s'est
poursuivi, au sein de l'école muʿtazilite, sur la question des attributs
divins, avec des auteurs comme al-Naẓẓām ou Abū ʿAlī al-Ǧubbāʾī, le
maître du jeune al-Ašʿarī, qui ont encore durci la position muʿtazilite,
allant jusqu'à dire qu'affirmer « Dieu est éternel » revient à dire « le
noir est noir » : c'est une pure tautologie, car les attributs divins
n'ont aucune autre réalité que l'essence divine.

\paragraph{Un détour par la grammaire pour éviter la tautologie des attributs divins}Cette position ne satisfait pourtant pas tout à fait le fils d'Abū ʿAlī,
Abū Ḥāšim al-Ǧubbāʾī (m. en 933) : la diversité des attributs divins
devient, avec l'approche muʿtazilite traditionnelle, proprement
inexplicable ; si pour Dieu, être savant et être puissant ne signifient
absolument pas autre chose qu'être Dieu, on ne comprend pas pourquoi la
révélation ne s'est pas contentée de dire « Dieu est Dieu ». C'est à la
grammaire arabe, qui connaît depuis longtemps des « accusatifs d'état »
(le \emph{ḥāl}), qu'il emprunte la solution en élaborant la doctrine des
« états » (\emph{aḥwāl}, le pluriel de \emph{ḥāl}), en se fondant sur
une nouvelle catégorie ontologique : dire que Dieu est savant ne
signifie pas nécessairement, comme le craignaient ses devanciers, dire
qu'il \emph{possède une science}, mais qu'il est \emph{en état de
science}. Le mode ne suppose pas l'existence d'une entité ontologique
propre qui soutient l'attribut. Il est dès lors permis d'accepter que
les attributs divins aient chacun un sens particulier, sans impliquer
l'existence d'êtres supplémentaires.

Cette solution ingénieuse peut satisfaire assez largement, et elle sera
promue par une sous-école muʿtazilite portant le nom de son maître, la
Bahšamiyya. Mais elle aura un impact bien plus vaste : rejetée par
al-Ašʿarī, contemporain et adversaire d'Abū Hāšim, elle sera adoptée
avec des nuances par des théologiens ašʿarites postérieurs :
al-Bāqillānī (m. en 1013), puis al-Ǧuwaynī (m. en 1085), ce qui lui
assurera une postérité plus longue que l'existence du muʿtazilisme
lui-même.

\paragraph{Un lent déclin post 1018} Le déclin de l'école muʿtazilite est pourtant un lent processus, qu'il
ne faut pas anticiper par téléologie \sn{en considérant la finalité (disparition de l'école mu'tazilite.}. On a parfois surévalué la portée
de la proclamation du calife al-Qādir, en 1018, d'une profession de foi
(la \emph{risāla al-qādiriyya}, du nom du souverain) condamnant le
muʿtazilisme et le chiisme, et faisant du ḥanbalisme la doctrine
officielle. Le calife abbasside s'efforçait alors d'échapper à la
tutelle des émirs bouyides, chiites, qui depuis des décennies
gouvernaient à Bagdad en marginalisant le pouvoir du calife, et
cherchait à le faire par le biais d'une « restauration sunnite » ;
l'influence retrouvée du calife al-Qādir reste cependant limitée, à la
fois géographiquement (le pouvoir des califes se limite essentiellement
à la ville de Bagdad) et dans sa portée. De plus, la condamnation de
l'école par le calife n'empêchera pas l'enseignement d'un autre grand
maître muʿtazilite, à Baṣra, Abū al-Ḥusayn al-Baṣrī (m. en 1044). La
condamnation n'empêchera pas non plus les muʿtazilites de revenir un
moment au pouvoir avec le sultan turc Tuġril bey, fondateur de l'empire
seldjoukide, qui va jusqu'à
faire revivre la \emph{miḥna} (de 1054 à 1063), en persécutant
prioritairement les ašʿarites ; mais cet
épisode est le dernier exemple d'un appui politique en faveur des
muʿtazilites.

Par la suite, c'est l'ašʿarisme qui profitera du soutien du pouvoir,
marginalisant progressivement les muʿtazilites, du moins dans le monde
sunnite, où il s'éteint peu à peu dans le courant du XIIe siècle ; mais
il reste très influent, en revanche, dans la théologie des différentes
confessions chiites, qu'il imprègne et nourrit profondément.


\hypertarget{aux-marges-de-la-thuxe9ologie-les-falux101sifa}{%
\section{\texorpdfstring{Aux marges de la théologie : les
\emph{falāsifa}}{Aux marges de la théologie : les falāsifa}}\label{aux-marges-de-la-thuxe9ologie-les-falux101sifa}}


Il faut dire ici un mot d'une autre école, sans doute plus célèbre que
celles que nous avons évoquées jusqu'à présent, bien qu'elle ne soit pas
à proprement parler une école de théologie : la \emph{falsafa} (une
arabisation du grec « philosophie »), c'est-à-dire l'école des
philosophes islamiques.
\mn{
    Certes, il est exact que ce sont bien, en un sens, des philosophes appartenant pleinement à la sphère théorique de la philosophie gréco-européenne. Ils en sont partie intégrante, en raison de leurs axiomes rationnels, de leurs références communes au corpus grec, de leur partage des mêmes règles logiques, des mêmes options métaphysiques. De ce point de vue, il n’est pas excessif de dire que ce sont des Européens de langue arabe. Toutefois, ils écrivent dans un contexte singulier, celui de la révélation islamique, et leurs œuvres produisent, dans ce contexte, des effets particuliers. D’autre part, ils ne se contentent pas de lire et commenter. Leur part d’invention existe, on l’a vu, du fait même qu’ils se situent à l’entrecroisement de l’héritage grec et de la pensée islamique. Ces héritiers des Grecs ont une « part d’ailleurs » qui oriente leurs lectures et leur compréhension, et donc leurs concepts.
 \cite{PolDroit:voyage} (p. 287). 
}
La philosophie et la théologie sont pour nous, non des écoles, mais des
disciplines académiques distinctes par leur objet et leurs méthodes. Il
n'en va pas de même pour le monde islamique classique, qui n'a pas fait
de la philosophie, comme l'Occident médiéval, une de ses facultés
universitaires. La philosophie y reste longtemps l'apanage d'un groupe,
les \emph{falāsifa} (philosophes, dont le singulier est
\emph{faylasūf}), qui se reconnaissent dans les thèses de leur école et
se méfient en général de la théologie spéculative, le \emph{kalām}. Nous
avons déjà insisté sur le fait que les muʿtazilites, quelle que fût leur
confiance dans les vertus de la raison, n'étaient nullement des
philosophes, mais des hommes pieux cherchant à interpréter le Coran de
la manière la plus respectueuse de la transcendance divine. 

\paragraph{D'abord des philosophes se nourrissant de la pensée grecque}Les philosophes islamiques sont certainement, pour la plus grande partie
d'entre eux, des croyants musulmans sincères, mais ce n'est pas d'abord
dans la révélation coranique qu'ils recherchent la vérité. Leur source
première est, de leur propre aveu, la pensée grecque, telle que les
traductions ordonnées par les premiers califes abbassides à la fin du
VIIIe siècle et au début du IXe siècle l'ont donné à connaître.
L'expression « pensée grecque » nous paraîtra
sans doute trop large, tant Platon, Aristote, les stoïciens ou Plotin
sont pour nous des auteurs distincts, aux perspectives et aux idées bien
différentes. Les \emph{falāsifa} ne l'entendent pas ainsi, et
considèrent au contraire que les Grecs étaient porteurs d'une doctrine
sur laquelle ils s'accordaient pour l'essentiel. \mn{Les points d’ancrage de l’islam des Lumières pourraient s’énoncer ainsi : les Grecs ont formulé des connaissances vraies, elles ne sauraient contredire la révélation coranique, au contraire elles permettent de l’approfondir, de la comprendre par d’autres voies et de progresser non seulement dans l’analyse théorique mais dans l’accomplissement spirituel.
\cite{PolDroit:voyage} (p. 289).  }. 
Surmonter les
contradictions, qui pour eux ne sont qu'apparentes, entre Platon et
Aristote est un exercice habituel de la philosophie islamique. Elle le
peut d'autant plus aisément qu'elle s'appuie pour cela sur une première
synthèse originale, proposée à la fin de l'Antiquité par des auteurs
comme Plotin ou Porphyre : ce qu'on désigne en général sous le nom de
néoplatonisme. La \emph{falsafa} est à plus d'un titre une continuation
originale du néoplatonisme antique.

Parmi les éléments qui obligent ces penseurs à renouveler la pensée de
leurs devanciers grecs, la religion musulmane est un défi, qui pousse
les philosophes à montrer que leur doctrine, bien que née dans un
contexte païen, n'est pas incompatible avec la foi musulmane, et même
qu'elle s'accorde parfaitement bien avec elle, voire qu'elle en
explicite la vérité la plus profonde 

\paragraph{Al-Kindi } \label{Theo:AlKindi1}Le premier des grands philosophes
de langue arabe, al-Kindī \mn{voir p. \pageref{Theo:AlKindi}}(m. en 873), entreprend, dans sa
\emph{Philosophie première}, de montrer l'unité du Dieu de la
\emph{Métaphysique} d'Aristote, du Dieu émanatiste des néoplatoniciens
et du Dieu révélé du Coran. Cela l'amène à abandonner plusieurs thèses
de ses devanciers qui lui paraissent peu compatibles avec le Dieu de la
foi islamique, en particulier l'éternité du monde.

\paragraph{Al-Farabi} \label{Theo:AlFarabi1}Son successeur, le Persan al-Fārābī\mn{voir p. \pageref{Theo:AlFarabi}} (m. en 950), appelé le « Second
maître » (le premier étant Aristote), se montrera un philosophe plus
conséquent, en mettant au point une synthèse néoplatonicienne plus
complète, qui exercera la plus grande influence sur un troisième
philosophe, Avicenne.
\paragraph{Avicenne}, \label{Theo:Avicenne1} ou Ibn Sīnā (m. en 1037), dont l'importance sur
l'histoire de la philosophie est considérable. Dans son œuvre majeure,
le \emph{Livre de la guérison} (\emph{Kitāb al-Šifāʾ}), Avicenne
détaille les différentes parties de la philosophie, dont la
métaphysique, qu'il appelle
« \textit{{[}science des{]} choses divines} » (\emph{ilāhiyyāt}). Il est vrai
que, pour ce qui est de la théologie proprement dite, la
\emph{Métaphysique} d'Aristote, qui sert de cadre de référence à
Avicenne, n'est
pas très bavarde ; Dieu n'en est pas absent, mais des questions comme la
création ou les attributs divins ne sont pas abordées. Avicenne s'appuie
alors sur un autre ouvrage qui circule en arabe sous le nom de «
\emph{Théologie} d'Aristote », bien que cette attribution au grand
philosophe grec soit en réalité inexacte : il s'agit essentiellement de
textes du philosophe néoplatonicien Plotin, qui écrivait sept siècles
après Aristote, et dont la pensée est très différente.

\paragraph{Preuve de l'existence de Dieu comme Etre nécessaire et cause de l'existence de tout être}
Au moyen d'une preuve de l'existence de Dieu extrêmement célèbre, qui
aura un retentissement considérable jusque dans l'islam contemporain,
bien plus que la preuve d'al- Ašʿarī présentée dans la séquence
précédente, Avicenne établit que Dieu est l'Être nécessaire par soi, qui
n'a besoin d'aucune cause, mais qui cause l'existence de tous les autres
êtres (qui ne sont donc pas des êtres nécessaires, mais des êtres
possibles, dont l'existence est rendue effective grâce à la cause
première de toute existence qu'est l'Être nécessaire).

De ce Dieu assez abstrait, que nous connaissons d'abord par une enquête
rationnelle, Avicenne tient à écarter toute forme de contingence\sn{
\begin{Def}[Contingence] En philosophie, la contingence indique la possibilité qu’une chose arrive ou n’arrive pas. Elle s’oppose à la nécessité, par laquelle les choses arrivent sans aucun doute. On débat notamment de la contingence ou de la nécessité du monde : est-il le résultat d’une stricte causalité nécessaire, ou dépend-il de la volonté de Dieu qui aurait pu vouloir qu’il n’existât pas, ou qu’il fût différent ? 
\end{Def}
}, ce
qui l'amène à remettre en question la doctrine classique de la création
: cette dernière suppose, de la part de Dieu, un acte volontaire, qui
aurait pu aussi bien ne pas se produire, et ce genre de fantaisie
arbitraire lui paraît indigne de Dieu, pour qui tout est nécessaire. On
objectera que le Coran, qu'Avicenne reconnaît comme Révélation divine,
parle bien de création, mais pour Avicenne, il le fait pour simplifier,
à destination de la masse ignorante, une réalité plus complexe, qui
procède de l'être même de Dieu, d'une nécessité propre à cet être, et
non d'un acte volontaire. On parle alors d'émanation* (\emph{fayḍ})
plutôt que de création ; et comme il est nécessaire en Dieu, il a
toujours eu lieu : le monde, né de l'émanation divine, est
nécessairement éternel, et ne peut avoir de commencement, contrairement
à la foi musulmane commune.
\begin{Def}[Emanation (fayḍ)]  Doctrine d’origine néoplatonicienne, qui décrit l’origine du monde non plus sous le mode coranique de la création par la parole (Dieu dit, et les choses sont), mais en un processus graduel et spontané où, sur le modèle de l’émanation physique, le monde vient à exister d’une surabondance de Dieu. De cette émanation naît une hiérarchie des êtres, dont la dignité dépend de leur place dans la chaîne des émanations et donc de leur proximité avec Dieu : l’Intellect premier, premier émané, est au sommet de cette hiérarchie, dont le monde matériel sublunaire occupe la dernière place. 
\end{Def}
Ce processus d'émanation est-il donc involontaire en Dieu, puisqu'il ne
repose pas sur un
choix volontaire et libre de Dieu, mais sur une nécessité interne divine ? Poser la question,
pour Avicenne, c'est montrer qu'on ne respecte pas l'unicité divine (le
fameux \emph{tawḥīd} islamique, auquel le philosophe est lui aussi
attaché), qui suppose de refuser les distinctions abstraites entre des
attributs ou des facultés en Dieu : en Dieu, on ne peut distinguer d'une
part l'intelligence et d'autre part la volonté. L'émanation n'est pas
autre chose que l'acte par lequel Dieu se comprend lui-même. De cet acte
va découler un ensemble hiérarchisé de réalités, commençant par
l'Intellect premier (qui n'est autre que la compréhension que Dieu a de
lui-même) suivi d'une série d'Intellects séparés éternels identifiés aux
astres (ou, dans le discours religieux, aux anges), et se poursuivant
dans notre monde sublunaire marqué par la matière, et donc la génération
et la corruption.

Ce n'est pas le lieu de décrire dans le détail cette hiérarchie du réel,
assez complexe, qui ordonne à la fois la physique et la métaphysique du
philosophe. Elle nous paraît bizarre, presque religieuse et bien peu
philosophique, mais on ne peut l'évacuer de la pensée d'Avicenne \sn{voir p. \pageref{Theo:Avicenne}} !
Retenons surtout ici que le rôle de l'Intellect premier est essentiel,
parce qu'il permet le passage de l'Un divin au multiple du créé, et
c'est par lui que Dieu est connaissable. En conséquence, la question
théologique, celle des capacités de la raison et du langage humains à
parler adéquatement de Dieu, devient radicalement différente. La
théologie musulmane se débattait depuis l'origine avec les impasses de
l'absolue transcendance de Dieu : puisque « rien n'est semblable à Lui
», nous ne savons pas comment en parler. Or pour Avicenne, la
compréhension que Dieu a de lui-même, en donnant naissance à ces
Intelligences qui reflètent Dieu comme des miroirs, nous fournit les
médiations nécessaires pour que notre intelligence limitée puisse parler
adéquatement de Dieu.

Il est nécessaire d'évoquer, même très brièvement comme ici, la
philosophie d'Avicenne quand on fait l'histoire de la théologie
islamique, parce l'influence qu'elle aura sur la théologie va être
incalculable. Il convient de noter du reste que l'influence fonctionne
dans les deux sens : même si la \emph{falsafa} trouve ses sources
doctrinales chez les auteurs grecs, il est incontestable qu'elle a
également trouvé des modèles d'argumentation et d'exposition dans
les traités de \emph{kalām}, notamment muʿtazilite. Mais l'empreinte
d'Avicenne sur la théologie qui
le suit ne saurait être surévaluée, et cela de deux façons :


\begin{itemize}
\item
  
  Bien des musulmans remarqueront qu'il y a loin de l'Être nécessaire
  par soi au Dieu du Coran, trouvant le Dieu d'Avicenne encore plus
  conceptuel et abstrait que le Dieu des muʿtazilites, et très éloigné
  du Dieu vivant auquel va leur dévotion. La critique des thèses
  avicenniennes va occuper beaucoup de place dans les traités de
  théologie, ḥanbalites ou ašʿarites.
  
\item
  
  après Avicenne, les outils (notamment logiques), les concepts et les
  catégories de la philosophie occupent une place de plus en plus grande
  chez les auteurs du \emph{kalām}.
  
\end{itemize}


Un théologien ašʿarite de tout premier ordre va résumer à lui seul cette
relation

ambivalente, faite de polémique et de dépendance : Abū Ḥāmid al-Ġazālī
(m. en 1111).


\hypertarget{le-duxe9veloppement-de-luxe9cole-aux161ux2bfarite-la-pensuxe9e-de-ux121azux101lux12b}{%
\section{Le développement de l'école ašʿarite : la pensée de
Ġazālī}\label{le-duxe9veloppement-de-luxe9cole-aux161ux2bfarite-la-pensuxe9e-de-ux121azux101lux12b}}


L'historiographie arabe, depuis le XVe siècle, distingue deux grandes
périodes dans l'histoire de l'école ašʿarite, que sépare précisément le
moment avicennien :


\begin{itemize}
\item
  
  les « anciens », comme al-Ašʿarī lui-même, ou encore al-Bāqillānī (m.
  en 1013), encore préservés de la contagion avicennienne ;
    
\item
  
  les « modernes », comme al-Ġazālī ou Faḫr al-Dīn al-Rāzī (m. en 1210),
  qui malgré les préventions du premier à l'égard des philosophes ont
  introduit leurs méthodes et leurs outils dans la réflexion
  théologique. Il est certain que ces « modernes » sont nettement plus
  rationalistes que ne l'étaient les premiers ašʿarites. De leur temps,
  faute de combattants, la polémique contre les muʿtazilites a pris fin,
  et ils se font parfois les champions, contre les théologiens
  traditionnalistes, de l'usage de la raison en théologie.
  
\end{itemize}


\paragraph{al-Ǧuwaynī fait une distinction entre Parole éternelle et Coran prononcé, qui est créé}Al-Ġazālī est l'élève d'un autre grand théologien ašʿarite, al-Ǧuwaynī
(m. en 1085), dit
\textit{« l'imam des deux sanctuaires »} après un séjour en Arabie, chez qui se
dessinait déjà, sur le
sujet des attributs divins, une doctrine plus précise\sn{On cite classiquement sa distinction entre les « attributs de
l'essence » (\emph{ṣifāt nafsiyya}) et les « attributs de qualité »
(\emph{ṣifāt maʿnawiyya}), mais elle joue un rôle relativement mineur
dans la réalité de son approche des attributs divins.} et plus
rationaliste que celle que proposait l'ašʿarisme primitif. Affirmant en
Dieu l'absence d'étendue, il en vient à légitimer la lecture
métaphorique des versets anthropomorphiques du Coran et des
\emph{ḥadīṯ}-s qui supposent une forme d'étendue, à la manière des
muʿtazilites ; et, tenant de la thèse du Coran incréé, il procède
toutefois à une distinction (qu'al-Ašʿarī refusait) entre le Coran
éternel, céleste, qui est la Parole incréée de Dieu (\emph{kalām nafsī})
et le Coran prononcé par celui qui le récite, qui quant à lui est créé
et non éternel (\emph{kalām lafẓī}).

Son élève al-Ġazālī, dont la vie nous est bien connue par une petite
autobiographie intellectuelle fort attachante (\emph{Al-Munqiḏ min
al-ḍalāl}, traduite en français sous le titre \emph{Erreur et
délivrance}), va bien plus loin. C'est un bon connaisseur de la
philosophie, et en particulier de la pensée d'Avicenne : il y a consacré
un ouvrage, les \emph{Intentions des philosophes}, qui essaie de rendre
compte de façon neutre des thèses philosophiques. L'ironie du sort
voudra que seul ce livre soit traduit en latin : Ġazālī, dit « Algazel
», sera considéré en Occident, pendant des siècles, comme un philosophe,
alors que de son aveu même, ce livre lui avait servi de préparation à sa
grande réfutation des thèses philosophiques, l'\emph{Incohérence des
philosophes} (\emph{Tahāfut al-falāsifa}). Dans ce dernier livre, il
s'en prend à vingt opinions trouvées essentiellement chez Avicenne, et
qu'il juge contraires à la foi musulmane. Seize d'entre elles sont de
simples erreurs, mais les quatre autres (notamment la doctrine de
l'éternité du monde) sont plus graves : les professer, ce n'est pas
seulement commettre une erreur, mais rejeter ouvertement la foi
musulmane, quitter l'islam. Les philosophes ne sont donc pas à ses yeux
un simple groupe hétérodoxe, mais des apostats -- ce qui est une
accusation très grave. 
\begin{Def}[apostasie]
L’apostasie est l’abandon d’une appartenance religieuse. Dans l’islam médiéval, les différentes écoles de droit s’accordent à considérer, sur la base du Coran et des ḥadīṯ-s qu’une apostasie publique et certaine mérite la peine de mort, tout en fixant à l’application de cette dernière des conditions qui la rendent effectivement rare ; les mêmes juristes prévoient donc d’autres peines, plus généralement appliquées, comme la perte du droit de propriété et la dissolution du mariage. Face à des peines si lourdes, les écoles de droit et de théologie répugnent en général à accuser d’apostasie à la légère, et ceux qui traitent d’infidèles (kāfir, pl. kuffār) et donc d’apostats leurs coreligionnaires sous prétexte qu’ils sont en désaccord avec eux ne sont guère appréciés : ceux qui pratiquent cette forme d’excommunication (takfīr) sont considérés comme des fanatiques 
\end{Def}

Il convient de noter que l'attaque de Ġazālī
n'est pas un rejet des méthodes philosophiques, au contraire. Comme le
titre de son ouvrage l'indique, il reproche essentiellement aux
philosophes de ne pas être fidèles à leurs propres principes : alors
qu'ils
ont, dans leur logique, établi les règles de la démonstration véritable,
ils affirment des thèses qu'ils sont en fait incapables de démontrer.
Forts du prestige de la logique, ils font croire à leurs disciples que
tout leur enseignement repose sur des preuves irréfutables, alors que
les points sur lesquels, au dire de Ġazālī, ils s'éloignent de la foi
islamique ne sont que des opinions personnelles que n'appuie aucune
démonstration rationnelle. C'est donc en quelque sorte au nom des
principes de la philosophie que Ġazālī condamne les philosophes, au
premier rang desquels Avicenne.

\paragraph{Une diffusion des outils philosophiques dans le kalam}La condamnation d'al-Ġazālī n'entraîne pas, comme on l'a parfois
prétendu, la disparition de la \emph{falsafa} en terre d'islam. Dans les
périodes qui nous intéressent, le mouvement aura encore des
représentants de toute première importance, comme Averroès (m. en 1198),
qui répondra du reste à Ġazālī dans son \emph{Incohérence de
l'incohérence}, ou Suhrawardī (m. en 1191). Et elle est plutôt le signe
de la diffusion des méthodes philosophiques chez les théologiens du
\emph{kalām} qu'un coup d'arrêt porté à cette diffusion.

La richesse conceptuelle de la pensée d'al-Ġazālī va lui permettre de
réfléchir de façon nuancée sur le travail théologique lui-même. Nous
allons nous arrêter un moment sur un court ouvrage de Ġazālī où le
théologien élabore un cadre conceptuel plus complexe que ceux que nous
avons jusque-là rencontrés.

\paragraph{Une réduction du takfir à déclarer le Coran et les hadits comme mensongers}Dans le \emph{Critère de distinction entre l'islam et l'incroyance},
Ġazālī affronte la question du \emph{takfīr}, qu'on traduit généralement
par excommunication : c'est l'opération par laquelle on déclare un
musulman « \emph{kāfir} », c'est-à-dire incroyant et donc apostat. Dans
les débats théologiques, cette accusation peut vite fuser, pour des
raisons doctrinales : le tenant de telle ou telle thèse que je juge
incompatible avec ma compréhension de l'islam (par exemple, avec ma
compréhension de la véritable unicité divine) va vite m'apparaître comme
un \emph{kāfir}. L'accusation prête pourtant à conséquence : si un
musulman est convaincu d'apostasie, il mérite théoriquement la mort, et
à tout le moins la perte de toute propriété et de tout droit civil.
Ġazālī estime donc que cette accusation ne doit pas être faite à la
légère, en se fondant
notamment sur un \emph{ḥadīt} dans lequel le Prophète déclare que, quand
un musulman traite un autre de \emph{kāfir}, alors l'un des deux hommes
l'est réellement --- sous-entendu : celui qui accuse à tort d'un crime
aussi grave est lui-même un \emph{kāfir}\ldots{} Ġazālī souhaite donc
établir le critère juste qui permettra de ne pas accuser à tort,
simplement à cause d'un désaccord entre théologiens. Et son premier
critère est tout simple : est un musulman authentique quiconque professe
que Dieu, dans le Coran, et son Prophète, dans les \emph{ḥadīṯ}-s,
disent la vérité. Le \emph{kāfir} est seulement celui qui accuse Dieu ou
son Prophète de mensonge.

La question, nous le savons, n'est pas aussi simple, car les théologiens
ne s'accordent pas sur le sens à donner aux textes révélés :
muʿtazilites, ḥanbalites, ašʿarites, tous prétendent en expliciter le
sens véritable, contrairement à leurs adversaires, et nous ne sommes
guère plus avancés qu'au commencement :
\begin{quote}
    LES ACCUSATIONS D’INCROYANCE\sn{Al-Ġazālī, Critère de distinction, p. 42}
    Le ḥanbalite accuse l’ašʿarite d’incroyance, estimant qu’il a accusé le Messager de mensonge à propos de la thèse selon laquelle Dieu résiderait « dans les hauteurs », et de celle relative à son « assise sur le Trône ». L’ašʿarite l’accuse de même, prétendant que le ḥanbalite est anthropomorphiste et qu’il a accusé de mensonge le Messager, puisque rien n’est semblable à Dieu. D’autre part, l’ašʿarite accuse le muʿtazilite d’incroyance, estimant que celui-ci a contredit le Messager au sujet de la possibilité de voir Dieu et de la nécessité de lui reconnaître la science, la puissance et les autres attributs séparément de son essence. Quant au muʿtazilite, il accuse l’ašʿarite d’incroyance, considérant qu’affirmer l’existence d’entités séparées revient à accuser d’incroyance les Anciens et de mensonge le Messager au sujet de l’unicité divine.
\end{quote}
Ce que ces écoles ignorent, explique Ġazālī, c'est que dire qu'un énoncé
est vrai ou faux n'est pas aussi simple que ça en a l'air, parce que le
réel est complexe. \begin{quote}
    « L'existence (\emph{al-wuǧūd}) comporte cinq degrés
(\ldots) : l'existence est soit essentielle, sensible, imaginaire,
rationnelle ou métaphorique. » \sn{
P. 42
}
\end{quote}
Le premier degré est l'existence
concrète. Le second, ce que nos sens perçoivent (ainsi, dans le rêve ou
le délire, je vois effectivement quelque chose) ; le troisième,
ce que je parviens à imaginer ; le quatrième renvoie à la raison d'être
des choses (« je prends la plume » signifie « j'écris », et non pas que
j'attrape dans ma main le plumage d'un volatile) ; le cinquième est une
image, une comparaison. Puisqu'il existe cinq degrés de l'être, il
existe donc cinq manières de dire qu'un énoncé est vrai. Si je déclare
avoir vu Jean-Paul II la nuit dernière, bien peu vont comprendre que je
prétends l'avoir croisé dans ma vie ordinaire, et m'accuser de mensonge
; les plus malveillants penseront que je perds la tête et que j'ai des
visions, et les plus astucieux que j'ai rêvé de lui : dans un cas comme
dans l'autre, je ne mens pas, mais j'énonce une vérité du deuxième
niveau. Or, dit Ġazālī, être incroyant, être un \emph{kāfir}, c'est
affirmer qu'une parole du Coran ou des \emph{ḥadīṯ}-s n'est vraie à
aucun de ces cinq degrés.

Ġazālī présente quelques exemples de chacun de ces degrés appliqués aux
textes sacrés.

Je n'en conserve ici que deux, le premier et le quatrième niveaux :

\begin{Ex}
[Les degrés de l'existence] (Al-Ġazālī, Critère de distinction\sn{ pp. 50-52. })

\begin{quote}
    Quant à l’existence essentielle, elle ne nécessite pas d’exemple : c’est ce qui est compris sous sa forme littérale, et ne s’interprète donc pas. C’est l’existence véritable et absolue, tel par exemple ce qu’a annoncé le Messager au sujet du Trône, du Siège et des sept cieux ; ils sont à prendre au sens littéral, sans interprétation, puisque ce sont des corps existants en eux-mêmes, qu’ils soient ou non perçus par les sens ou l’imagination. […] De l’existence rationnelle, nombreux sont les exemples. […] Le second exemple est le ḥadīṯ du Prophète : « Dieu a, pendant quarante matins, pétri de sa main l’argile dont il a tiré Adam », qui affirme que Dieu a une main. Celui qui établit la démonstration qu’il est impossible que Dieu ait une main, en tant qu’organe sensible ou imaginable, concevra donc l’existence en Dieu d’une main spirituelle et rationnelle. J’affirme donc le sens de la main, sa réalité et son esprit, à l’exclusion de sa forme concrète. L’esprit de la main et son sens sont ce par quoi on exerce une force violente, on fait, on donne ou on enlève quelque chose. Or Dieu donne et prive par l’intermédiaire de ses anges.
\end{quote}  
\end{Ex}

Chacun des degrés est donc utile quand on se trouve face aux textes
sacrés. Ġazālī sait
bien que tous les théologiens ne sont pas d'accord là-dessus : les
ḥanbalites, en particulier,
tiennent que les textes révélés décrivent tous une vérité du premier
type, ce qu'il appelle le degré « essentiel » de l'existence,
l'existence concrète : ils doivent donc être compris toujours au sens
littéral (\emph{ẓāhir}), tandis que le recours à l'interprétation
(\emph{taʾwīl}), c'est-à-dire aux quatre autres degrés d'existence, est
à rejeter absolument. Cette position semble robuste : de quel droit
refuse-t-on à Dieu de parler de manière essentielle, au sens propre ?
Néanmoins, il ne manque pas de souligner que l'ancêtre éponyme des
ḥanbalites, le grand Aḥmad Ibn Ḥanbal lui-même, ne peut être totalement
fidèle à cette profession de foi littéraliste, et qu'il a lui aussi
recours au \emph{taʾwīl}, à l'interprétation non-littérale, dans sa
version la moins littérale du reste (les quatrième et cinquième degrés),
même s'il limite cet usage à trois \emph{ḥadīṯ}-s seulement. A titre
d'exemple, le premier d'entre eux est une parole du Prophète déclarant
que « la pierre noire {[}de la Kaʿba, à la Mecque{]} est la main droite
de Dieu sur terre ». Pour Ibn Ḥanbal, c'est une métaphore : de même
qu'on embrasse la main droite en signe d'intimité, ainsi on embrasse la
pierre noire pour se rapprocher de Dieu. \begin{quote}
    « Ce mode d'existence est celui
que nous avons appelé métaphorique ; c'est la figure la plus éloignée
{[}du sens littéral{]} en interprétation (\emph{taʿwīl}). Observe donc
comme l'homme le plus éloigné du \emph{taʾwīl} y a été contraint ! »\sn{P. 58.}
\end{quote}
\begin{Prop}[Al Gazali montre que même Ibn Hanbal a utilisé le niveau métaphorique]
Certes, Ibn Ḥanbal limite le \emph{taʾwīl} à trois textes, ce qui est un
usage bien moindre que celui des autres écoles, mais cela suffit à
interdire aux ḥanbalites de refuser jusqu'au principe de \emph{taʾwīl} :
tous les musulmans y ont recours, et c'est donc un principe en soi
légitime.
\end{Prop}


Mais légitime en soi ne signifie pas évidemment pas légitime dans tous
les cas. Ġazālī a réduit fortement le domaine de l'accusation
d'incroyance, mais il ne pense pas pour autant que tous le reste est
vrai ! Le sens littéral reste évidemment préférable, et l'usage du
\emph{taʾwīl} est soumis à une règle simple : on ne peut y recourir que
lorsque le sens littéral est impossible.

Le sens littéral, premier, c'est l'existence essentielle : si c'est ce
sens qui est retenu, alors il englobe tous les autres. Mais si c'est
impossible, on a recours au sens sensible : il comprendra alors tous les
modes qui le suivent. Et si on ne le peut pas, on recourra aux modes
imaginaire ou rationnel ou encore, si aucun d'eux
ne convient, à la métaphore. Nulle permission toutefois de s'écarter
d'un mode d'existence vers celui qui
lui est inférieur, sans en démontrer la nécessité.\sn{P. 62
}

\paragraph{Sens littéral préférable sauf quand ce sens est \textit{impossible}} Ġazālī ne met pas les différents degrés sur le même plan : il y a bien
une hiérarchie entre eux, et le sens littéral est préférable à la
métaphore. Les sens interprétatifs ne sont utilisables que lorsque le
sens littéral est impossible, et au sein même de ces sens, on doit
toujours préférer le sens supérieur quand il est possible. Cela semble
assez consensuel, et la question est : comment déterminer qu'un sens est
impossible ? Car c'est là, bien sûr, que les oppositions entre les
écoles vont se cristalliser. Pour Ġazālī, la chose est très claire :
quand l'impossibilité est démontrée. La démonstration (\emph{burhān}) à
laquelle il pense est celle de la logique d'Aristote, celle que le
Stagirite appelle \emph{apodeixis} et qui, pour les philosophes, conduit
seule à la certitude absolue. 
\begin{Def}[apodeixis]
L'\emph{apodeixis}, c'est la manière de
procéder du mathématicien grec ancien Euclide, l'inventeur de la
géométrie : un point de départ indiscutable, qu'on nomme axiome (le tout
est plus grand que la partie, les parallèles ne se rencontrent pas),
dont on déduit par une méthode rigoureuse, qu'on appelle le syllogisme,
d'autres vérités tout aussi incontestables.
\end{Def}


\paragraph{La raison est implicitement le critère du niveau herméneutique}Pour Ġazālī, la clef de voûte de l'herméneutique présentée ici tient
donc à cette possibilité logique d'atteindre, grâce à la démonstration
logique, une forme de connaissance certaine. Cela peut sembler du bon
sens, mais cela revient au final à faire de la logique le critère ultime
permettant de juger de la révélation : certains ḥanbalites ne manqueront
pas de relever cette marque de rationalisme qui met la Parole de Dieu
au-dessous de la raison humaine, puisque c'est cette dernière qui permet
d'en estimer la vérité. Qu'al-Ġazālī continue à tenir, comme les
ḥanbalites ou les premiers ašʿarites, le sens littéral du Trône de Dieu,
comme on l'a vu, n'empêche pas une évolution méthodologique extrêmement
de grande importance.

Mais si ce cadre proposé est bien loin de la pensée d'al-Ašʿarī, ne
serait-ce que par sa sophistication ontologique et herméneutique, il
conserve un élément commun avec le
positionnement du maître : c'est un cadre conciliant, qui permet de
théoriser la possibilité d'opinions divergentes, pas toutes également
vraies, mais néanmoins soutenables en islam, et ne méritant pas
l'excommunication ; la pensée d'al-Ġazālī permet à la diversité de fait
de l'empire islamique, et de la pensée islamique, d'être conçue par un
esprit en quête de vérité. Mais il n'est de tolérance véritable qui ne
se fonde sur une définition préalable de ce qui est intolérable. Les
frontières du \emph{kufr}, de l'incroyance, sont repoussées, mais elles
ne sont pas abolies : ceux qui disent que Dieu ou son Prophète mentent
ont, eux, quitté l'islam. Mais existent-ils ? Hélas oui : et sur ce
point, al-Ġazālī retrouve ses vieux adversaires, les philosophes
islamiques. Ces derniers, en effet, croient que la résurrection des
corps est impossible, alors qu'elle est affirmée par la révélation
musulmane. Que leur reproche Ġazālī ? D'une part, ils se fondent, pour
refuser un dogme de la foi, sur une démonstration très imparfaite, de
sorte que leur position est une opinion plus qu'une connaissance
certaine. Et d'autre part, les philosophes ne nient pas que le Prophète
ait annoncé la résurrection des corps ; ils ne disent pas qu'il faut
interpréter autrement les textes qui l'annoncent ; mais il estiment que
si le Prophète a annoncé ces mensonges, c'était dans une bonne intention
: s'adressant à des hommes grossiers, ne pouvant imaginer que des
récompenses ou des châtiments charnels, le Prophète a cru nécessaire de
leur annoncer une résurrection corporelle pour les pousser à bien agir ;
mais le véritable sage, lui, sait que le corps ne ressuscitera pas. Pour
Ġazālī, les philosophes accusent dont le Prophète de mensonge, bien
qu'ils lui donnent des excuses : les voilà donc sortis de l'islam.

On le constate, la théologie islamique a su sortir de l'aporie initiale
sur la transcendance divine qu'elle semblait condamnée à répéter sans
fin, pour atteindre un niveau d'élaboration métaphysique et
herméneutique élevé. Elle n'y a pas perdu son goût prononcé pour la
controverse, dont on a vu dans cette séquence quelques éléments sur le
versant le plus rationaliste de la théologie. Mais la théologie
traditionnaliste, on le verra, n'est pas en reste !

