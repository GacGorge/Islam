\chapter{Introduction}


\mn{Adrien Candiard S2 : 12 heures, 6 semaines, 2 ECTS K. Théologie et connaissance des grandes religions}


L’existence d’une théologie musulmane fait l’objet de mises en doute, parfois chez les penseurs musulmans eux-mêmes, qui soulignent la priorité du droit sur les considérations proprement théologiques. La question théologique, celle du discours humain sur Dieu, a pourtant occupé les savants musulmans de l’époque classique. Le cours visera à faire découvrir ces controverses dont les conséquences sont encore considérables dans tous les domaines des sciences islamiques.


\paragraph{Compétences à acquérir à l’issue de l’enseignement}
\bi
\item Connaître les principales problématiques de la théologie musulmane classique
\item Lire un texte théologique ancien (vocabulaire, notions, style).
\item Thématiser une problématique théologique
\item Acquérir une culture générale théologique
\item Evaluer, à la lumière des connaissances acquises en théologie chrétienne, une démonstration de l’existence de Dieu
\item Faire le lien entre la théologie classique et les problématiques contemporaines
\ei
\paragraph{Sommaire et thèmes}

\bi 
\item Séance 1 : Introduction générale : y a-t-il une théologie musulmane ? 
Les origines du kalām
\item Séance 2 : Le muʿtazilisme
Étude du « credo » muʿtazilite
\item Séance 3 : La réaction ḥanbalite
		Étude d’une profession de foi attribuée à Ibn Ḥanbal
\item Séance 4 : La conciliation ašʿarite
\item Séance 5 : L’ašʿarisme classique 
Étude de la preuve de l’existence de Dieu de Ğuwaynī
\item Séance 6 : La théologie traditionaliste
\ei

\paragraph{Pédagogie et méthodologie (incluant les compléments : TD, numérique, etc.)}




Ouvrages à lire au cours de l’enseignement (5 au maximum)
La bibliographie francophone sur le sujet est malheureusement lacunaire, et datée. On consultera avec profit les deux ouvrages suivants, qui ne sont pas à lire en entier ; des articles ou des chapitres seront indiqués et joints au cours lui-même.
\cite{Gardet:IntroductionTheoMusulmane}


D. GIMARET, Les noms divins en islam, Paris, Cerf, 1988.

\paragraph{Mode d’évaluation}
L’évaluation consistera en un travail de recherche et d’analyse d’environ 10 000 signes.


\chapter{Séance 1 : Introduction générale : y a-t-il une théologie musulmane ? }
\mn{Séquence 1}
Selon Gardet - Anawati\sn{\cite{Gardet:IntroductionTheoMusulmane}, p. 26}
\begin{quote}
    \begin{enumerate}
        \item une période de préformation à Médine
        \item la période de fermentation (ou la rencontre avec la théologie chrétienne - Damas)
        \item la période héroique ; le conflit mu'tazilisme - traditionalisme (ou la rencontre avec la philosophie grecque)
        \item le triomple de l'ash'arisme
        \item l'éclectisme ghazâlien et la voie dite \textit{des modernes}
        \item la période de conservatisme figé
        \item la période de modernisme (ou \textit{réformisme}
    \end{enumerate}
\end{quote}


%------------------------------------------------------------------------------


L'existence même d'une théologie islamique fait aujourd'hui l'objet de
nombreux doutes, aussi bien chez des auteurs d'articles de vulgarisation
sur l'islam que chez beaucoup de musulmans.

Il est pourtant certain qu'il existe, dans les sciences islamiques
classiques, une science appelée en arabe \emph{ʿilm al-kalām} :
\begin{Def}[ʿilm al-kalām]
Littéralement « la science du discours {[}théologique{]} » (plus
vraisemblablement que « science de la Parole {[}divine{]} »). 

On la
désigne également par son contenu essentiel, comme \emph{ʿilm al-tawhīd}
(« science de l'unicité {[}divine{]} »), ou encore, avec des nuances de
sens variant selon les lieux et les temps, \emph{uṣūl al-dīn} (« les
fondements de la religion »). 
\end{Def}

Cette science, avec ses courants, ses
grands noms et ses hérétiques, ses ouvrages de référence, ses
évolutions, ses méthodes propres, a organisé le débat intellectuel au
sein de l'islam essentiellement autour de deux questions essentielles :
\textsc{notre connaissance de Dieu, de sa nature et de ses attributs ; la
liberté de l'homme face à la toute-puissance de Dieu}. Notre cours
s'intéressera au premier chef à la première de ces questions, la plus
directement théologique au sens propre. Puisque cette discipline (le
\emph{ʿilm al-kalām}, ou plus simplement le \emph{kalām}) a existé dans
l'islam médiéval, comment est-il possible que certains affirment que la
théologie islamique n'existe pas ?

Il convient d'abord de remarquer que la science du \emph{kalām}
appartient bien plus à l'histoire de la pensée musulmane qu'à son
actualité, tout au moins dans le monde sunnite. Les penseurs de l'islam
contemporain ne se passionnent guère pour ces débats anciens et
abstraits, et préfèrent en général faire porter leurs efforts sur des
questions plus directement
en prise avec les enjeux d'actualité, comme des débats juridiques
(droits des femmes, droits de l'Homme, statut de la violence...) ou
herméneutiques (comment faut-il lire le Coran et les autres textes de la
tradition islamique ?). On considère couramment, dans le monde arabe,
que le \emph{kalām} est une discipline achevée, qui a trouvé une
solution à toutes les questions qu'elle soulevait et n'est, depuis le
XIVe siècle et la mort d'un de ses derniers représentants, ʿAḍud al-Dīn
al-Īǧī (m. en 1355), qu'un objet de pure curiosité historique. Elle
aurait réglé des problèmes concernant le dogme, mais ne serait pas une
discipline à part entière, d'autant plus inépuisable que son objet est
infini. C'est dans cette optique, celle d'une science achevée, figée,
pour ainsi dire morte, qu'elle est généralement enseignée aujourd'hui.

Il convient aussi de noter que la définition devenue progressivement
classique de l'objet de cette science lui confère une ambition
relativement limitée. C'est le cas de la définition qu'en donne le grand
savant maghrébin du XVe siècle, Ibn Ḫaldūn \label{Theol:IbnKhaldun2}:

\begin{quote}
La science du \emph{kalām} est une science qui fournit les moyens de
prouver les dogmes de la foi par des arguments rationnels, et de réfuter
les innovateurs qui, en ce qui concerne les croyances, s'écartent de la
doctrine suivie par les anciens et par les gens de la tradition.
L'essence même de ces dogmes est la profession de l'unicité de
Dieu.\sn{Ibn Ḫaldūn, \emph{Al-Muqaddima}, voir p. \pageref{theol:IbnKhaldun1}}
\end{quote}

L'objectif ici fixé au \emph{kalām} est essentiellement apologétique et
défensif : il s'agit d'appuyer par des arguments rationnels des dogmes
déjà connus par révélation, et réfuter les hérétiques qui mettent ces
dogmes en doute. Cette science n'est ici qu'instrumentale, au service
d'un catéchisme déjà établi. Elle est très en-deçà des ambitions que le
christianisme médiéval en Occident a attribuées à ses facultés de
théologie, pour qui la théologie conduit à une connaissance effective de
Dieu. Il est d'ailleurs certain que le \emph{kalām} n'a jamais occupé la
place centrale, au sein des sciences islamiques, où trônait la théologie
dans l'Université médiévale d'Occident. Pour autant, il ne faut pas
oublier que la définition donnée par Ibn Ḫaldūn est le fait d'un penseur
qui n'estime guère cette discipline et qui entend ici la dénigrer. Il
n'est pas
lui-même théologien. Les théologiens de l'islam n'auraient probablement
pas donné de leur
science une définition si restrictive.

\begin{Synthesis}
Le kalām s'intéresse à notre connaissance de Dieu, de sa nature et de ses attributs, ainsi qu'à la liberté de l'homme. Question qui n'a jamais été aussi centrale qu'en Occident et qui semble fermée aujourd'hui, défensive, \textit{apologétique} les penseurs de l'islam préférant travailler les questions éthiques.
\end{Synthesis}

\paragraph{Possibilité d'une théologie quand le Coran est la Parole même de Dieu, incréée et sans médiation}
Mais s'élève encore, à l'égard de la théologie islamique, un doute plus
substantiel, plus fondamental. Une théologie au sens propre, une science
qui prendrait Dieu pour objet, est- elle seulement possible en islam ?
La présence d'un texte révélé, le Coran, qui est la Parole même de Dieu,
éternelle et incréée, sans passer par les médiations humaines de
l'inspiration que le christianisme attribue à ses propres Écritures, ne
rend-elle pas vaines toutes les tentatives d'appréhender Dieu par des
raisonnements discursifs ? Une théo-logie, un
« discours rationnel sur Dieu », suppose que la raison peut en dire
quelque chose. Or, ne cesse d'affirmer le Coran, « rien n'est semblable
à Lui » (Coran 42, 11). L'absolue transcendance de Dieu ne le place-t-il
pas très loin de nos approches rationnelles, inaccessible à toutes nos
tentatives ?

Dans sa fameuse conférence donnée à Ratisbonne en septembre 2006, qui ne
portait du reste pas du tout sur l'islam, le pape Benoît XVI s'était, en
introduction, aventuré sur ce terrain\sn{Pour une mise en perspective de cette conférence, voir Ch. Jambet,
2006.}. Citant une controverse
islamo-chrétienne médiévale, il soulignait les conséquences très
concrètes de nos conceptions de Dieu. Si Dieu est accessible à notre
raison, alors nous pouvons en parler, échanger à son sujet des arguments
rationnels, nous employer à nous convaincre réciproquement, sans
recourir à la violence. Si, en revanche, on estime que la nature de Dieu
est par nature inconnaissable, et qu'on ne connaît de lui que sa volonté
(exprimée par révélation), alors la violence est presque inéluctable :
pour étendre une religion, on ne peut faire appel à des arguments
rationnels, et il ne reste que le sabre. À cette analyse fine des
rapports entre la question théologique et la violence, le pape ajoutait,
en citant l'éditeur de la controverse, que pour la doctrine musulmane,
la transcendance de Dieu le rend inaccessible à notre raison. C'est sur
ce dernier point que l'affirmation est discutable. S'il est certain que
l'affirmation de la transcendance de Dieu est un élément central de la
foi
musulmane, \textsc{l'idée que cette transcendance interdit l'entrée de la raison
humaine dans le domaine théologique proprement dit est loin d'avoir fait
consensus chez les auteurs musulmans classiques}. C'est même autour
d'elle que s'organiseront la plupart des débats que nous allons étudier
ici, et l'on constatera que la position selon laquelle la transcendance
divine empêche la raison humaine de s'élever jusqu'à Dieu n'est qu'une
option parmi d'autres chez les musulmans, et une option âprement
discutée au fil des siècles par les théologiens de l'islam, les auteurs
du \emph{kalām}. \textbf{L'effort d'articulation entre la transcendance divine,
si nettement affirmée par la révélation coranique, et les capacités de
la raison humaine constitue le cœur du débat théologique de l'islam
médiéval.}

Ces débats anciens portant sur les capacités du langage humain à dire
Dieu adéquatement, désignés sous le titre classique de « question des
attributs divins » (ou des
«noms divins»), sont souvent abstraits et techniques, mais ils ne sont
nullement inaccessibles pour un lecteur contemporain préparé à les lire
; ils démontrent une inventivité et une liberté d'esprit qui ont de quoi
nous surprendre, si nous les abordons en nous figurant le moyen-âge
comme une époque barbare ou obtuse. Les étudier, comme nous le ferons
ici, c'est découvrir une diversité d'approches bien souvent ignorée
aujourd'hui des musulmans eux-mêmes.

Ce cours sera consacré à la seule période classique, du VIIIe au XIVe
siècle de l'ère chrétienne, c'est-à-dire la période pendant laquelle la
« science du \emph{kalām} » est active. Cela ne signifie pas que la
théologie ait par la suite tout à fait disparu du monde musulman, mais
elle a pris d'autres formes, bien moins systématiques. Le cours sera de
plus essentiellement centré sur le monde sunnite, bien que la frontière
confessionnelle aujourd'hui si nette entre sunnisme et chiisme ait moins
de pertinence pour la période étudiée : des auteurs essentiels, comme le
philosophe Avicenne ou le théologien et scientifique Naṣīr al-Dīn
al-Ṭūsī, étaient chiites, mais leur pensée a exercé une influence qui ne
s'est pas cantonnée au seul monde chiite, loin de là.

L'étude du \emph{kalām} se heurte encore à une autre difficulté pour un
étudiant francophone. Les sources, presque toutes écrites en arabe (ou,
de manière plus exceptionnelle, en persan), ne sont presque jamais
traduites en français, et assez rarement en anglais. Quant aux études
sur le sujet, si l'\emph{Introduction à la théologie musulmane}\sn{\cite{Gardet:IntroductionTheoMusulmane}} de Louis
Gardet et Georges Anawati avait été en 1948 une œuvre pionnière, on ne
dispose plus guère de manuel à jour sur la question en langue française,
et la plupart des études les plus récentes sont publiées en anglais. On
indiquera toutefois pour chaque séquence quelques indications
bibliographiques, privilégiant les ouvrages les plus susceptibles de
vous être accessibles.

L'islam ne naît pas de la spéculation de théologiens subtils. Si le
contexte précis de l'apparition de l'islam nous est très mal connu,
faute de sources historiques fiables\sn{Un point récent sur les débats historiographiques relatifs à l'islam
primitif est proposé dans un petit ouvrage à la fois très bien informé
et très pédagogique, qui permet de s'épargner bien des errances dans un
domaine où l'imagination et l'idéologie fonctionnent à plein : Françoise
Micheau, 2012.}, on sait qu'il naît dans un
contexte à la fois influencé par les débats théologiques de l'Antiquité
tardive et pauvre en culture écrite. Le Coran est certainement marqué
par ces débats, mais ni Muḥammad ni ses compagnons, pas plus que les
conquérants arabes partis au VII\textsuperscript{e} siècle à la conquête
du monde, ne nous ont laissé trace d'une réflexion théologique
discursive. Il faudra plusieurs décennies pour que se formalise une
réflexion de cet ordre et qu'apparaissent les premiers signes de ce que
deviendra la science du \emph{Kalām} ; et de ces débuts, on ne peut
parler qu'avec prudence : si nous connaissons des noms et des courants,
rapportés par des historiens plus tardifs, les textes théologiques
eux-mêmes des époques les plus anciennes ont le plus souvent disparu.
Pour plusieurs de ces courants, on doit se fier aux rapports de leurs
adversaires, ou à des textes mal attribués et probablement écrits bien
plus tard.
\begin{Synthesis}
Le kalām a essayé d'articuler l'affirmation forte de la transcendance divine et de la raison, à travers la question des attributs divins, entre le VIII et XIV.
\end{Synthesis}

\hypertarget{une-origine-interne-ou-externe}{%
\section{Une origine interne ou externe
?}\label{une-origine-interne-ou-externe}}

L'apparition des premières discussions théologiques en islam correspond
à un changement de contexte par rapport à celui de l'apparition de
l'islam. La religion nouvelle est sortie d'Arabie et s'est confrontée,
dans de grands centres urbains très éduqués (Damas, Alexandrie,
Antioche, Ctésiphon...), à des intellectuels chrétiens, juifs ou
mazdéens rompus à la controverse métaphysique et religieuse. Qu'on songe
aux interminables débats qui ont occupé les chrétiens depuis le concile
de Nicée (325), relatifs à la Trinité ou aux deux natures du Christ :
ces querelles ont développé dans l'Église une dialectique très élaborée,
assez
éloignée de la façon très simple qu'a le Coran d'affirmer
l'unicité*\sn{Les mots suivis d'un astérisque font l'objet d'une explicitation dans
le lexique de la séquence (document joint).} de Dieu. Il est donc tentant d'attribuer la
naissance de la théologie islamique à cette rencontre : défiés par des
concurrents redoutablement équipés, les musulmans auraient dû développer
rapidement leur propre système d'argumentation et de défense, qui
donnera naissance à la « science du Kalām ».

Cette hypothèse est séduisante et largement recevable, mais
l'universitaire allemand Josef van Ess, l'une des autorités les plus
considérables dans ce domaine, a relevé ses fragilités\sn{J. van Ess, 1974. Voir aussi J. van Ess, 1977.} : l'examen du
peu de textes anciens dont nous disposons, estime-t-il, ne laisse rien
entrevoir d'une polémique interreligieuse. À cette origine externe et
polémique, il privilégie la piste d'une origine interne : ces débats
naissent naturellement entre les musulmans eux- mêmes, en fonction de
difficultés doctrinales propres à l'islam.

L'historien britannique Michael Cook nuancera bientôt à son tour les
conclusions de Josef van Ess\sn{M. Cook, 1980.}. Il souligne que les textes qu'il évoque
sont très probablement inauthentiques, et qu'il est difficile, en
l'absence de textes assurés, d'opiner dans un sens ou dans l'autre. Cook
remarque en revanche que les manières d'argumenter et d'écrire, très
caractéristiques, qui organiseront la discussion théologique en islam
sont déjà présentes dans des ouvrages de polémiques christologiques
antérieurs à l'islam, en particulier dans le monde chrétien de langue
syriaque. Pour expliquer cette permanence des outils rhétoriques, Cook
propose deux explications qui ne sont d'ailleurs pas exclusives l'une de
l'autre : des musulmans ont pu, en débattant avec les chrétiens, adopter
leur manière d'argumenter ; mais il se peut aussi que des théologiens
chrétiens se soient convertis à l'islam, et aient apporté avec eux leur
manière de débattre et d'écrire.


Sans doute ne faut-il pas chercher à trancher trop nettement entre ces
deux
approches, qui permettent ensemble de rendre compte de la complexité
d'un contexte : la
théologie islamique ne naît pas de rien, et elle est marquée par des
méthodes et des questionnements qui la précèdent ; mais cela ne signifie
nullement qu'elle n'apparaisse que par souci d'imitation. Comme la
théologie chrétienne avait su se saisir d'un outillage philosophique
grec qui la précédait pour faire naître des pensées originales et
profondément chrétiennes, l'islam s'approprie les outils dialectiques et
théologiques qu'il trouve à sa disposition au service d'un
questionnement porté par la révélation coranique.

 
  \subsection{La question du \emph{qadar}}
 

De ce point de vue, il est notable que le premier débat théologique qui
nous soit connu porte sur un embarras que les musulmans trouvent dans la
lecture du Coran. Ce dernier affirme avec force l'absolue
toute-puissance divine, s'opposant ouvertement au fatalisme du paganisme
arabe antéislamique qui plaçait le destin au-dessus des hommes et des
dieux. Mais cette affirmation se heurte bientôt à l'expérience du mal :
en constatant que le mal existe, peut-on tenir à la fois la
toute-puissance et la bonté de Dieu ? N'est-on pas obligé de relativiser
l'un ou l'autre de ces attributs divins ?

Cette question, présente dans tous les monothéismes, prend dans le cas
de l'islam une couleur particulière. Le Coran n'affirme-t-il pas, par
exemple, que
\begin{quote}
    « Dieu guide qui Il veut et égare qui Il veut » (Coran 14,
4) ?
\end{quote} 
Cela signifie-t-il que Dieu est l'auteur du péché de l'homme, que
pourtant il punit des peines de l'Enfer ? À cette question, un
traditionniste de Baṣra, en Irak, Maʿbad ibn ʿAbdallah al-Juhanī (m. en
699) \sn{Ma'bad ibn Abdullah al-Juhani (en arabe : \TArabe{ معبد الجهني}) , mort en 691, appartenait à la tribu des Juhainah qui vivait autour de la ville de Médine. Il a été crucifié par les ordres du calife omeyyade Abd al-Malik à Damas. Il était le premier homme, après Sinbuya, à parler du Qadar (libre arbitre humain).}, va donner une réponse jugée rapidement hétérodoxe : \textsc{Dieu est bien
le créateur de tout ce qui existe, à l'exception toutefois du péché de
l'homme, qui est de sa seule responsabilité.} Pour autant qu'on le sache,
en l'absence de textes de sa plume qui soit parvenu jusqu'à nous, sa
pensée sur le sujet ne semble pas nettement plus élaborée ; elle
n'atteint pas la sophistication que lui donnera bientôt une autre école,
le muʿtazilisme, qui reprendra et développera ces thèses. Il ne semble
pas même qu'il ait construit un système plus général, en s'interrogeant
sur la réciproque : les actes
louables de l'homme lui sont-ils également imputables, ou sont-ils, eux,
à attribuer directement à Dieu ? Il semble que son effort soit plus
théologique (peut-on exonérer Dieu du mal ?) qu'anthropologique (l'homme
est-il responsable de ses actes ?).

Son exécution en 699, probablement due à ses positions hétérodoxes (bien
que l'ensemble des auteurs anciens ne s'accorde pas sur cela),
n'empêchera pas un groupe de penseurs (comme le haut fonctionnaire
Ġaylān al-Dimašqī, qui sera lui aussi exécuté) de tenir et de développer
ces idées, en élaborant une doctrine du libre arbitre de l'homme. Pour
eux, l'homme est responsable de ses actes, qui ne sont donc pas
prédéterminés et décidés par Dieu : l'homme est libre d'agir, de choisir
le bien ou le mal, et c'est ce choix que Dieu jugera au dernier jour.L'homme possède donc un \emph{qadar}.
\begin{Def}[qadar]
Un qadar, c'est-à-dire la faculté ou le
pouvoir de se déterminer soi-même.
\end{Def}
 On appellera donc ses partisans les «
qadarites » --- appellation particulièrement malheureuse, car par la
suite, le mot \emph{qadar} (qui en arabe veut seulement dire « pouvoir») va changer radicalement de signification : il en viendra à désigner
non plus le pouvoir de l'homme de choisir le bien ou le mal, mais au
contraire le pouvoir que Dieu a de déterminer les actions des hommes, la
prédestination, c'est-à-dire précisément ce que les qadarites refusent !

Qu'on ait pu tenir cette position alors qu'elle semble aller contre la
citation coranique citée plus haut (« Dieu guide qui Il veut et égare
qui Il veut ») doit nous inciter à une forme de prudence face à nos
réflexes. Il ne nous appartient évidemment pas de juger de l'orthodoxie
de telle ou telle doctrine : remarquons simplement qu'elles ont existé
au sein de l'islam, à certaines époques, dans certains contextes. De
plus, cela nous amène à remarquer que la lettre du Coran n'est pas si
évidente qu'elle en a l'air quand on le lit en traduction. Des
continuateurs des Qadarites ne manqueront pas de faire remarquer, en
effet, que le verset coranique est susceptible d'une autre lecture,
grammaticalement tout aussi plausible :
\begin{quote}
    « Dieu guide qui le veut {[}=
qui veut être guidé{]} et égare qui le veut {[}= qui veut être égaré{]}
».
\end{quote} 
Cela change évidemment les choses ! On constate ainsi que notre lecture même
du texte coranique
peut dépendre de nos présupposés théologiques\ldots{}

Aucun texte qadarite ne nous est parvenu, si l'on excepte quelques
traités à l'authenticité très controversée\sn{Le plus important d'entre eux est le traité attribué à Ḥasan al-Baṣrī,
\emph{Risālat al-qadar} \label{Theol:HasanAlBasri}(« traité sur la prédestination »).}, et il est difficile de
faire la part, dans l'opposition farouche qu'ils ont rencontrée, entre
le refus strictement doctrinal et l'antagonisme politique. Les qadarites
s'impliqueront en effet dans le jeu politique, en particulier au cours
d'une guerre civile qui a divisé l'empire des Omeyyades \sn{(désignée sous
le terme de « troisième \emph{fitna} », qui s'ouvre en 744 et ne
s'achève que par le triomphe d'une nouvelle dynastie, les Abbassides, en
749)} ; l'usurpateur à l'origine de la guerre, le calife Yazīd III, avait
en effet clairement pris parti pour la doctrine qadarite, jusque-là
sévèrement réprimée, et il obtint le soutien de tous ses sympathisants,
mais son échec marquera la fin du mouvement, dont les idées seront
toutefois reprises, de manière plus élaborée, par l'école muʿtazilite.
On a parfois souligné l'impact proprement politique de la doctrine du
libre arbitre, qui aurait menacé l'obéissance aveugle, de droit divin,
exigée par les Omeyyades, mais cette hypothèse est incertaine.

Beaucoup d'auteurs musulmans reprochent aux qadarites d'avoir purement
et simplement adopté la doctrine chrétienne du libre arbitre dans un
contexte musulman, et vont jusqu'à attribuer la doctrine de Maʿbad, le
premier qadarite, à l'enseignement d'un maître chrétien. Cette
affirmation n'est pas aberrante, car confronté à une difficulté
analogue, le christianisme a effectivement développé une nette doctrine
de la liberté d'action de l'être humain ; il ne manquera pas de
reprocher à l'islam sa méconnaissance de cette liberté et sa croyance
fataliste en la prédestination. Pour autant, cette solution n'est pas
nécessairement empruntée au christianisme : elle fait partie des options
possibles devant la difficulté théologique posée par le mal. Et il est
clair que le soupçon d'une origine chrétienne a, sous la plume des
auteurs musulmans qui le rapportent, un objectif polémique : il s'agit
de
montrer que cette doctrine est radicalement étrangère à l'islam
originel, et qu'elle est par conséquent à rejeter comme hétérodoxe.
Cette intention polémique manifeste incite naturellement à la prendre
avec quelques réserves.

Encore mal connu, le mouvement qadarite témoigne néanmoins, dès le
premier siècle de l'islam, de l'émergence d'une réflexion théologique
cherchant à s'organiser, et de réactions énergiques, voire brutales, à
son égard. Mais remarquons que la première naissance de la théologie
islamique s'interroge davantage sur la liberté de l'homme que sur la
nature de Dieu. Cette question n'interviendra que dans un second temps.

\begin{Synthesis}
Le mouvement qadarite est le premier à produire une réflexion théologique, en proposant le \emph{qadar}, la puissance de l'homme de décider en libre arbitre. Néanmoins, le contexte politique (pour le calife Yasid III usurpateur et finalement battu par les abbasides), et une polémique accusant ce mouvement d'être crypto-chrétien l'a fait disparaitre.
\end{Synthesis}
 
 %--------------------------------------------------------------------
\section{Ǧahm Ibn Ṣafwān et la pensée rationnelle en
théologie} 
\label{Theol:JahmIbnSafwan}

Un autre nom doit être cité dans cette préhistoire de la théologie
musulmane : celui de Ǧahm Ibn Ṣafwān\sn{Aussi translittéré Jahm ibn Ṣafwān \TArabe{ (جَهْم بن صَفْوان)} } (m. en 746), actif dans les mêmes
décennies que le mouvement qadarite mais dans une tout autre région de
l'empire, son extrémité orientale, le Ḫurāsān (région correspondant à
l'est de l'Iran et l'Afghanistan). Personnage mal connu, dont aucun
écrit ne nous est parvenu et sur lequel tout notre savoir vient de ses
très nombreux adversaires, Ǧahm est, de l'avis général, le premier
auteur à avoir présenté la pensée rationnelle (\emph{ʿaql}) comme un
moyen de parvenir à des vérités théologiques, pouvant conduire à
corriger des lectures jugées trop littérales du texte coranique.

\begin{Def}[Neoplatonisme]
doctrine philosophique de l’Antiquité tardive, élaborée notamment
par le philosophe égyptien de langue grecque Plotin (IIIe
siècle ap. J.-C.), qui réalise une
synthèse des pensées de Platon et d’Aristote autour d’une réflexion métaphysique et
mystique prônant le retour à l’Un divin. C’est à partir de cette doctrine que s’élaborera
d’abord la falsafa, ou « philosophie islamique ».
\end{Def}
Plusieurs auteurs, anciens et contemporains, soulignent l'influence de
la philosophie grecque sur la pensée de Ǧahm. Cette influence, en
particulier celle du néoplatonisme* de l'Antiquité tardive, est
incontestable, mais elle ne doit pas occulter la pensée profondément
originale développée par le théologien persan, qui s'affranchit très
volontiers des cadres philosophiques reçus (comme la distinction
aristotélicienne essentielle de la substance et de l'accident). On a
aussi noté qu'il reçoit l'héritage des discussions patristiques
chrétiennes, en particulier sur les questions trinitaires. Sur une
question qui prendra, comme nous le verrons, une grande ampleur par la
suite, Ǧahm s'appuie explicitement sur des débats qu'il mène
contre des chrétiens : il affirme que \textsc{le Coran, Parole de Dieu, est créé
et non pas éternel, par souci de cohérence }; en effet, pour la foi
musulmane, le Christ, que le Coran qualifie de
« Parole et Esprit de Dieu » (Coran 4, 171), est une créature, donc le
Coran lui-même doit l'être aussi. Affirmer l'éternité du Coran, ce
serait risquer de donner raison aux chrétiens qui affirment que le
Christ est davantage qu'une créature.

Par les œuvres de ses adversaires, nous connaissons un grand nombre de
thèses de Ǧahm Ibn Ṣafwān, mais hors de leur contexte et sans
continuité. Retrouver la cohérence de son œuvre est un travail minutieux
qui provoque beaucoup de débats entre les chercheurs, et il n'est pas
question d'en examiner ici le détail. On peut en revanche retenir qu'il
identifie le premier le point névralgique qui ne cessera d'occuper la
théologie musulmane dans les siècles qui suivront. Le Coran,
remarque-t-il, attribue à Dieu des caractéristiques qui ne sont pas
compatibles avec sa parfaite simplicité, qui exclut en lui toute
composition et toute multiplicité. C'est le cas des attributs
anthropomorphiques, qui supposent en Dieu des éléments corporels (une
main, l'assise sur un trône\ldots) ; mais c'est aussi le cas d'attributs
essentiels, comme la vie, la science ou la puissance. Ǧahm résout la
difficulté par une affirmation radicale : Dieu n'étant pas une chose,
rien ne peut lui être logiquement attribué, et il ne peut être décrit
par des attributs. Le Coran ne peut chercher à faire ce qui est
logiquement impossible, et les passages en question ne signifient donc
pas ce qu'ils semblent signifier. On le constate, les éléments qui
guident son interprétation du texte sacré sont logiques et ontologiques.
Au nom de sa conception du monde, accessible à la raison humaine, il
entend corriger le sens obvie du Coran. À notre connaissance, il est le
premier auteur à poser sur la révélation ce regard critique, qui en fait
à tous effets le premier théologien de l'islam.

Pour autant, à l'exception de quelques disciples, Ǧahm n'aura guère de
postérité. Les principaux courants de théologie islamique qui viendront
par la suite échangeront le qualificatif de « ǧahmite » comme une
véritable insulte : les tenants d'une lecture plus
littérale du texte sacré, mais aussi les théologiens rationalistes, tous
refuseront de s'associer
à cet ancêtre mal reconnu de la théologie islamique.

\begin{Synthesis}
Le premier théologien de l'Islam, Ǧahm essaye de penser rigoureusement les noms de Dieu, qu'il refuse de lui attribuer par cohérence de la raison. De la même façon, il juge le Coran créé par cohérence avec Jésus, parole et Esprit de Dieu pour le Coran et considéré par les musulmans comme créature.
\end{Synthesis}

 %----------------------------------------------
\section{Une autre voie : les débuts du sunnisme} 

\begin{Def}[Sunna]
Ce terme arabe, qui signifie « voie » ou « tradition », désigne en contexte
islamique la « sunna du Prophète », c’est-à-dire les textes qui permettent au croyant de
connaître la pensée et les actions de Muḥammad. On désigne par là deux grands corpus de
textes : la Sīra, ou biographie du Prophète ; et les ḥadīṯ-s, anecdotes ou propos rapportés de
Muḥammad. Tous ces textes sont, pour la tradition musulmane, authentifiés par une chaîne
de garants, jugés fiables, qui ont transmis leur contenu de bouche à oreille, jusqu’à la fixation
par écrit au IXe
siècle.
\end{Def}

\begin{Def}[sunnisme]
Le terme, qui désigne la doctrine des partisans de la Sunna , a pris deux sens qu’il convient de distinguer :
\bi
\item Dans le vocabulaire théologique, il désigne ceux qui préfèrent fonder leurs
opinions religieuses sur la Sunna plutôt que sur des procédures rationnelles, c’està-dire les courants traditionnalistes de la théologie musulmane.
\item Dans le monde contemporain, le terme renvoie à la division de l’islam en deux
confessions principales, les sunnites (majoritaires, autour de 80 % des musulmans)
et les chiites. Cette division d’abord politique, datant des origines de l’islam, a pris
une signification religieuse, et concerne les rites, les croyances et l’organisation
religieuse.
\ei
\end{Def}


Ce tour d'horizon de la pensée théologique naissante, pour n'être pas
trop incomplet, se doit de mentionner, à côté de ces premiers efforts de
théologie discursive et rationnel, ce qui y constitue peut-être une
réaction. Alors que l'expansion de l'empire arabe mettait l'islam au
contact de nombreuses pensées et pratiques religieuses, l'Arabie
commençait à se percevoir comme le gardien de la mémoire du Prophète et
de ses Compagnons. Autour de Médine, ancienne capitale délaissée par les
califes omeyyades au profit de la lointaine Damas, s'organise à la fois
une réaction politique et une contre-offensive intellectuelle. ʿAbdallah
Ibn Zubayr, le fils d'un célèbre compagnon de Muḥammad, prend la tête
d'une rébellion contre les Omeyyades qui l'amènera à régner pendant près
de dix ans sur une partie considérable de l'empire, jusqu'à sa défaite
et sa mort en 692.

\begin{Def}[Traditionnistes]
 savants (ou « ulémas ») qui veillent à la
transmission des textes de la Sunna. Ce n’est donc pas un synonyme de « traditionnalistes », 
bien que naturellement, beaucoup de ces savants (mais pas tous !) sont aussi des partisans de
l’autorité de la tradition. 
\end{Def}

L'échec de cette tentative d'une reprise de pouvoir par les clans
demeurés en Arabie n'empêchera pas en revanche les lettrés d'Arabie de
mettre en place les moyens de transmission de la mémoire des actions et
des paroles du Prophète, qu'on appelle la \emph{sunna}* (ou tradition)
prophétique. La ville de Médine, capitale de Muḥammad, se sent investie
d'une mission particulière dans ce domaine. Hors d'Arabie, en
particulier en Iraq, la demande est d'ailleurs grande de garder vivante
la mémoire des temps prophétiques. Ces récits, pieusement recueillis et
transmis par des savants spécialisés, les traditionnistes* ou
transmetteurs de ḥadīṯ (\emph{muḥaddiṯūn}), vont peu à peu acquérir une
valeur juridique et théologique considérable. Ces « savants » (en arabe,
\emph{ʿulamāʾ}, qui a donné le français « les ulémas ») s'organisent
bientôt en écoles, et vont exercer une très grande autorité au sein de
l'islam. Dans cette période de formation, on parle à leur sujet de «proto-sunnisme» : le
sunnisme* théologique, qui se caractérise par le \textsc{primat donné aux
traditions prophétiques}, se met en place.

À cette insistance sur la \emph{sunna} correspond le plus souvent à une
manière bien différente de concevoir l'accès que l'homme peut avoir à
Dieu. Si la révélation coranique ne suffit pas à l'homme, alors il faut
encore que le complément soit lui-même révélé ou d'origine divine : la
mission du Prophète ne consiste pas seulement à donner le Coran, mais
encore à l'expliquer de la meilleure des manières. Les traditions
prophétiques ont donc pour vocation de répondre à la plupart des
questions religieuses, qu'elles soient juridiques ou métaphysiques. Ces
savants gardiens de la mémoire se méfient des innovations que prétendent
apporter les tenants d'opinions nouvelles. La théologie discursive et
rationnelle qui, on l'a vu, commence à naître apparaît dans ce contexte
comme un abandon des richesses de la tradition prophétique.

C'est une autre manière d'approcher Dieu qui se met ainsi en place. Par
le développement de méthodes extrêmement structurées, elle démontre
qu'elle n'a pas d'hostilité de principe au fonctionnement rationnel,
loin de là ; mais la raison humaine, surtout formalisée par la logique
grecque, ne lui paraît une voie d'accès à Dieu assurée comme l'est la
mémoire prophétique.

Gardons-nous cependant d'opposer trop nettement ces approches et les
groupes sociaux qui les portent, car si la distinction est commode, la
réalité est moins tranchée. Maʿbad al-Juhanī, le premier qadarite qui
nous soit connu, est lui-même traditionniste ; et une personnalité
importante du premier siècle de l'islam, al-Ḥasan al-Baṣrī (m. en 728),
semble s'être trouvé à la croisée de ces différents univers qui ne sont
pas encore antagonistes ni mutuellement exclusifs : ascète, penseur,
transmetteur de traditions, il sera revendiqué comme figure fondatrice
aussi bien par les soufis que les théologiens ou les traditionnalistes.
S'il est difficile de les départager, car une fois de plus on ne possède
pas de lui de texte
authentique, il semble surtout que cette personnalité ait pu traverser
des frontières qui deviendront par la suite plus fermes.

Je souhaite conclure cette première séance par une remarque de méthode.
Il est tentant de rechercher, dans cette diversité qui s'amorce, la voie
la plus authentiquement islamique ; on l'identifiera alors généralement
dans les milieux des traditionnistes, plus que dans les approches plus
directement théologiques, qui peuvent nous sembler trop mêlées de
philosophie grecque ou d'influences chrétiennes plus ou moins directes.
Il convient de se garder de ces préjugés : ces rencontres font partie de
l'histoire intellectuelle de l'islam, et rêver à un islam originel pur
est aussi illusoire que de regretter un christianisme vierge de toute
rencontre avec la pensée grecque. La réaction « sunnite » est elle-même
conditionnée par cette rencontre de l'islam avec d'autres pensées qui
vont l'aider à structurer sa réflexion, à travers des débats que
l'empire abbasside, mis en place en 749-750, va encourager à se
développer.
\begin{Synthesis}
Le mouvement sunnite se construit dans un contexte politique où Médine est délaissée. La tradition est vue comme un passage plus sure que la raison. Ne pas opposer trop fortement les oppositions à cette époque (al-Ḥasan al-Baṣrī revendiqué par les soufis, les théologiens et les traditionnalistes). Ne pas considérer les traditionnistes comme l'islam pur : toujours un contexte.
\end{Synthesis}


\section{Conférence IDEO -  à quoi sert la théologie islamique ?}
\mn{le 28/4/22   Adrien candiard}
\begin{quote}

Pas de place sur le discours réflexif sur Dieu et divin
 
Kalam discours divin pensée humaine peut s’approcher de dieu
Deux dossiers
	⁃	transcendance de dieu : si dieu est transcendant comment est il possible d’en dire quelque chose. Mais le coran manifestation de dieu
	⁃	Transcendance vs intelligibilité 
	⁃	9-10 siècle : période patristique de l'eglise
	⁃	Liberté humaine face à la toute puissance de dieu
	⁃	Face au jugement on ne peut pas dire que tout dépend de dieu et qu’il y a de la liberté 
	⁃	Toute puissance de Dieu et responsabilité de l’homme

Il y a des musulmans qui disent que pas de théologie et uniquement du droit : salafiste
Plus de pratique en islam du kalam
Kalam science morte au xiv ? On peut faire de l’histoire du Kalam ?
Définition restrictive du xv : science apologétique d’arguments rassembles pour défendre les dogmes
Mais ibn khaldum : inventaire de cette définition mais de façon polémique 
Victoire d’un courant anti théologique

En christianisme Justin vs Tatien. Tatien anti théologique mais pas de descendance
Mo atazilisme : courant rationnel
Cela va plaire au calife
Des opposants : le prophète ne faisait pas de théologie. Travailler sur le hadith 
Suivre les traditions du prophète
Ibn hanbal saint patron des traditionalistes

Une école médiane acharisme 
Légitimer la théologie

On ne peut pas faire disparaître une question. Parfois une question n’intéresse plus

Ibn rudama : ce dont il est possible de discuter. Extrémiste hanbalisme. Faire des recherches sur le droit

La fin du kalam repoussée : on a d’abord pensé au XIX que plus rien après averroes 
Mais xiv plein de choses

XIX relecture et retour aux sources image assez négative de la théologie médiévale
Contact avec l’Occident : repli sur le droit religieux car la métaphysique est une forte concurrence un peu comme théologie avec Nietzsche

Théologie islamique : on arrête de faire de la théologie au nom du salafisme
Aujourd’hui cette anti théologie n’est pas consciente et du coup on ne sait pas qu’on est partisan de cette théologie
Étudier les autres possibilités est intéressant
Libération pour bcp musulmans

Nous catholiques au XIX par rapport à la modernité: renouveau thomiste et patristique. Liberté de pensée par rapport aux questions de l’époque
« A quoi on tient en vrai »
Le raccourci de certains théologiens musulmans : faire le court circuit pour repenser les choses : obsession juridique. Comment on sort du droit ? En faisant de la théologie. Et pour cela retour à la tradition
Voile
Le coran la dit donc il faut se voiler. Voile : hijab n’est pas dans le coran. Rideau
Zina : qu’il faut cacher
Jilbab pour les filles et femmes du prophète 
Ibn el jahouzi : voile non mentionné. Dimension non religieuse

Des écrits au XX. Deux raisons : les femmes vont occuper plus de rôle dans l’espace publique.
Et renouveau de l’anti théologie. Le voile est une question de théologie. Question de la foi
	⁃	Saint Paul : la foi et les œuvres
	⁃	On comprend croyant non pratiquant
	⁃	Islam : cette distinction bcp moins centrale. Ce qui fait un musulman ? 
	⁃	Si on est pêcheur, est on sauvé ? Les khalirites : islam loi donc pas faire la loi c’est ne pas être musulman
	⁃	Acharite : on ne sait pas ce qu’il y a dans le cœur de l’homme. Hanbalite : de Dieu on ne connaît que la volonté.
	⁃	Comment faire pour avoir la foi quand on ne fait pas d’actions. Porter la barbe : inscrite dans sa propre chair ce que Dieu estime non. Quand je dors j’ai la barbe donc j’ai la foi
	⁃	Voile : manière d’avoir la foi permanente. Faire de la théologie musulmane
	⁃	Équivalent du scapulaire
	⁃	Dialogue inter religieuavicen


Avicenna : outillage mental qui a changé la façon dont tout le monde a vu le réel. On a un outil, on doit l’utiliser. Peut être a dévalué la théologie ottomane aux yeux des religieux car elle était très fortement philosophise
Grande différence notre occident et islam. Occident ; philo et théologie : deux domaines. Métaphysique, un peu au milieu. Cela a permis à la philosophie de s’acclimater en occident

Falsafia : pas une faculté d’université, c’est une école concurrente des autres. Du coup sent le souffre. Pas d’articulation.

\end{quote}