\chapter{Introduction}


\mn{Adrien Candiard S2 : 12 heures, 6 semaines, 2 ECTS K. Théologie et connaissance des grandes religions}


L’existence d’une théologie musulmane fait l’objet de mises en doute, parfois chez les penseurs musulmans eux-mêmes, qui soulignent la priorité du droit sur les considérations proprement théologiques. La question théologique, celle du discours humain sur Dieu, a pourtant occupé les savants musulmans de l’époque classique. Le cours visera à faire découvrir ces controverses dont les conséquences sont encore considérables dans tous les domaines des sciences islamiques.


\paragraph{Compétences à acquérir à l’issue de l’enseignement}
\bi
\item Connaître les principales problématiques de la théologie musulmane classique
\item Lire un texte théologique ancien (vocabulaire, notions, style).
\item Thématiser une problématique théologique
\item Acquérir une culture générale théologique
\item Evaluer, à la lumière des connaissances acquises en théologie chrétienne, une démonstration de l’existence de Dieu
\item Faire le lien entre la théologie classique et les problématiques contemporaines
\ei
\paragraph{Sommaire et thèmes}

\bi 
\item Séance 1 : Introduction générale : y a-t-il une théologie musulmane ? 
Les origines du kalām
\item Séance 2 : Le muʿtazilisme
Étude du « credo » muʿtazilite
\item Séance 3 : La réaction ḥanbalite
		Étude d’une profession de foi attribuée à Ibn Ḥanbal
\item Séance 4 : La conciliation ašʿarite
\item Séance 5 : L’ašʿarisme classique 
Étude de la preuve de l’existence de Dieu de Ğuwaynī
\item Séance 6 : La théologie traditionaliste
\ei

\paragraph{Pédagogie et méthodologie (incluant les compléments : TD, numérique, etc.)}




Ouvrages à lire au cours de l’enseignement (5 au maximum)
La bibliographie francophone sur le sujet est malheureusement lacunaire, et datée. On consultera avec profit les deux ouvrages suivants, qui ne sont pas à lire en entier ; des articles ou des chapitres seront indiqués et joints au cours lui-même.
\cite{Gardet:IntroductionTheoMusulmane}


D. GIMARET, Les noms divins en islam, Paris, Cerf, 1988.

\paragraph{Mode d’évaluation}
L’évaluation consistera en un travail de recherche et d’analyse d’environ 10 000 signes.


\chapter{le Kalâm}

Selon Gardet - Anawati\sn{\cite{Gardet:IntroductionTheoMusulmane}, p. 26}
\begin{quote}
    \begin{enumerate}
        \item une période de préformation à Médine
        \item la période de fermentation (ou la rencontre avec la théologie chrétienne - Damas)
        \item la période héroique ; le conflit mu'tazilisme - traditionalisme (ou la rencontre avec la philosophie grecque)
        \item le triomple de l'ash'arisme
        \item l'éclectisme ghazâlien et la voie dite \textit{des modernes}
        \item la période de conservatisme figé
        \item la période de modernisme (ou \textit{réformisme}
    \end{enumerate}
\end{quote}