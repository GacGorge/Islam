\section{Burkini : celles qui sont forcées et celles qui ne peuvent pas le porter}
\mn{La Croix Louise Gallorini, traductrice et docteure en littérature arabe classique}


La pièce la moins utilisée de ma garde-robe est celle qui fait le plus parler d’elle : le burkini. Dans un pays historiquement passionné par la mode, il n’est peut-être guère remarquable que ce nouveau type d’habillement venu de l’autre bout du monde passionne les foules. Je suis musulmane, et j’avais acheté un burkini non pas pour des raisons religieuses, mais pour des raisons de santé : ma peau ne supporte pas le soleil très longtemps, et cet habit m’a d’abord paru une bonne alternative (et écolo) aux tartinades régulières de crèmes chères (et nocives pour les océans). Je ne suis pas surprise qu’il ait été inventé en Australie, où les vêtements de surfeur m’avaient plu pour cette exacte raison : ils recouvrent le corps et règlent le problème des coups de soleil – surtout pour les Australiens, conscients du danger du trou dans la couche d’ozone au-dessus de leurs têtes…

Plus largement, le voile relève pour moi de la même problématique en France. Autant de voiles, autant d’histoires qui se placent sur un large spectre situé entre deux limites : celles qui sont forcées de le porter (en France, par pressions familiales ou sociales) et celles qui aimeraient le porter mais ne peuvent pas (pressions similaires mais inverses).

Le premier exemple est connu, et c’est la raison pour laquelle beaucoup partent en guerre contre le burkini… Le deuxième l’est beaucoup moins. C’est ce qui rend pour moi le voile ou le burkini comme un exemple trop peu fiable d’un symptôme d’oppression de la femme, surtout en France. D’une part, qui fréquente les milieux musulmans réalise très vite qu’un « vrai » islamiste ne laisserait de toute façon pas sortir une femme à la plage ou à la piscine, même recouverte d’un burkini.

D’autre part, l’oppression de la femme peut être parfaitement invisible : une amie française d’origine maghrébine qui, voulant mettre le voile par conviction religieuse personnelle, s’est vue essuyer un refus catégorique de la part de ses parents, qui voulaient lui éviter tout problème. Elle n’a pu qu’obtempérer. De l’extérieur, elle paraît donc parfaitement intégrée à la société française, modèle de la femme libérée… Pourtant son cas relève tout autant de ce « droit de regard » sur le corps des femmes que l’on dénonce à juste titre. Au nom de sa libération, on lui dicte encore comment s’habiller.

D’autres comme moi ne rencontrent heureusement aucune interdiction familiale, mais la société me limite plus : ne considérant pas le voile comme une obligation religieuse, je préfère m’épargner les soucis qui viennent avec le fait de le porter. Trouver un logement et un travail facilement me paraît plus important que de pouvoir utiliser toute la gamme de ma collection de voiles, rangés eux aussi dans mon placard. J’accepte donc cette contrainte, considérant que tout membre d’une société est contraint d’une manière ou d’une autre, que c’est ainsi qu’on peut faire société.

Ce débat vient aussi à l’intersection du débat sur la laïcité, souvent mal comprise : si je parle ici du cas des signes religieux islamiques, grands favoris des débats télé mais qu’un citoyen musulman doit pouvoir théoriquement porter dans l’espace public, cette pression n’est pas unique aux musulmans. Une femme de culture chrétienne de mon entourage n’ose plus porter de croix trop visibles, non pas à cause de voisins musulmans ou athées, mais tout simplement parce qu’elle pense que la loi interdit le port de signes religieux dans l’espace public… !

Quant à savoir si le voile ou le burkini est nécessaire à la pratique religieuse, cette question relève du théologique, et la réponse est variable d’une époque à une autre, d’un pays à l’autre, d’une culture à l’autre, et tient plus de la considération personnelle et de sa propre relation au divin et aux textes (surtout en islam sunnite, sans système clérical). Je considère personnellement le voilement comme une des expressions possibles du devoir de modestie coranique, injonction qui peut se traduire de diverses manières selon le temps et le lieu, et qui s’applique aux deux sexes (fait parfois ignoré à la fois par les pro-voile et les anti-voile).

Si cette interprétation me rend la vie plus facile – un jean et un tee-shirt suffisent à passer pour modeste en France –, je peux en revanche compatir avec le cas des musulmanes françaises qui considèrent que le voile est une partie non négociable de leur pratique, et pour qui l’espace public en devient beaucoup plus difficile d’accès, ce que je déplore. Il faudrait pouvoir laisser ce débat de (dé)voilement, en France laïque, là où il est pertinent : débat théologique intra-islamique.