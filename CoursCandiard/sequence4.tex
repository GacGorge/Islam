 

\hypertarget{suxe9ance-4}{%
\chapter{Abū al-Ḥasan al-Ašʿarī}\label{suxe9ance-4}}

Les termes du débat théologique de l'islam classique semblent
définitivement posés par l'opposition que nous avons vu prendre forme
entre deux courants irréconciliables, le muʿtazilisme et le ḥanbalisme.
Entre le rationalisme des premiers et le traditionalisme des seconds,
aucun terrain non seulement d'entente, mais même de discussion, ne
semble disponible ; la violence apparaît bientôt comme le seul mode de
confrontation possible entre ces deux conceptions irréductibles de la
vérité.

C'est pourtant une troisième école, née au début du X\textsuperscript{e}
siècle, qui parviendra progressivement à s'imposer comme la véritable
expression de l'orthodoxie islamique, dans le monde sunnite du moins,
et à le demeurer à peu près sans conteste jusqu'au XX\textsuperscript{e}
siècle. 
\begin{Def}[Orthodoxie]
: mot formé sur deux termes grecs signifiant « opinion droite », et désigne la doctrine considérée comme conforme à la vérité par l’autorité compétente. En ce sens, le contraire de l’orthodoxie est l’hétérodoxie. Le terme est employé aujourd’hui largement (on parle ainsi d’économiste orthodoxe quand son enseignement suit la doctrine jugée dominante), mais la notion est d’abord religieuse. En christianisme, le mot a pris un sens particulier et désigne les Églises en communion avec le patriarche de Constantinople, qui se considèrent elles-mêmes comme porteuses de la véritable doctrine ; mais naturellement, toutes les autres Églises se considèrent elles-mêmes comme orthodoxes au sens propre. Dans l’islam sunnite, la difficulté de définir une orthodoxie tient à l’absence initiale d’autorité religieuse indiscutable, de magistère. Ce magistère émergera progressivement au fil des siècles, formé par le consensus des savants (les ulémas) qui se reconnaissent mutuellement comme légitimes. Dans ce cadre, l’orthodoxie est la norme doctrinale validée par le plus grand nombre de ces ulémas et leurs institutions les plus prestigieuses.
\end{Def}


Appelée d'après le nom de son fondateur, Abū al-Ḥasan al-Ašʿarī
(m. en 936), cette école créative, capable de trouver une voie au milieu
de cette opposition radicale, offre à la théologie islamique une forme
nouvelle\sn{Comme toujours, il ne faut
pas surestimer cette nouveauté : des penseurs, comme Ibn Kullāb (m. en
855), ont déjà combiné l'affirmation de la réalité des attributs divins
et une pratique de la théologie spéculative. Mais aucun n'aura la
postérité spectaculaire de l'ancien muʿtazilite de Baṣra.}, capable d'agréger des approches assez différentes.
L'ašʿarisme, dont on n'envisagera aujourd'hui que les commencements,
connaîtra des évolutions qui en affecteront souvent le contenu, comme
nous aurons l'occasion de le voir par la suite, mais il conservera
toujours cette dimension conciliatrice et cette capacité de faire
cohabiter sereinement des positions apparemment incompatibles ; c'est à
ce titre une école qui convient au gouvernement du vaste monde
islamique, lieu d'une grande diversité ethnique, religieuse, théologique
et spirituelle.



\hypertarget{un-ancien-muux2bftazilite}{%
\section{Un ancien muʿtazilite}\label{un-ancien-muux2bftazilite}}

L'école muʿtazilite représentait, à la fin du IX\textsuperscript{e}
siècle, la seule véritable voie pour une théologie spéculative. Sous
l'impulsion de quelques maîtres, comme Abū ʿAlī al-Ǧubbāʾī (m. en 915) à
Baṣra, dans le sud de l'Irak, elle se développe sous la forme d'un
système de théologie scolastique. C'est parmi les disciples de ce
dernier maître qu'est formé le jeune Abū al-Ḥasan al-Ašʿarī, descendant
d'un important compagnon du Prophète, qui fait partie de l'école
muʿtazilite jusqu'à ses quarante ans. Alors qu'il passe pour un des
disciples les plus en vue d'al-Ǧubbāʾī, auquel il pourrait succéder à la
tête des muʿtazilites de Baṣra, il quitte bruyamment l'école, à propos
d'un désaccord théologique majeur. Lequel ? Les sources ne s'accordent
pas sur ce point, et l'on possède deux versions de cette rupture, qui
insistent chacune sur un des deux points essentiels de la théologie
muʿtazilite : la doctrine du libre- arbitre et la négation des attributs
divins. C'est le second récit que je vous propose de lire à présent : il
raconte comment al-Ašʿarī, déjà assailli par le doute à l'égard de son
école d'origine, décide finalement de la quitter suite à une série de
rêves où lui apparaît le Prophète lui-même. Bien qu'écrit à la première
personne, le récit n'est pas d'al-Ašʿarī lui-même, mais d'un de ses
disciples qui prend la parole en son nom. Si nous avons conservé (enfin
!) un certain nombre de textes théologiques du maître lui-même\sn{Sur la centaine de titres d'œuvres que nous connaissons d'al-Ašʿarī,
moins d'une dizaine de textes sont parvenus jusqu'à nous. Parmi eux, on
compte les \emph{Opinions des musulmans} (Maqālāt al-islāmiyyīn), un
traité d'hérésiographie qui est notamment un précieux témoignage des
positions muʿtazilites ; l'\emph{Elucidation des fondements de la
religion} (Kitāb al-ibāna ʿan uṣul al-diyāna), un traité où al-Ašʿarī se
montre proche des positions ḥanbalites ; les \emph{Éclairs de polémique
contre les hérétiques et les innovateurs} (Kitāb al-lumaʿ fī al-radd
ʿalā ahl al-zayġ wa-l-bidʿa), son œuvre la plus originale. Aucun de ces
ouvrages n'est, à ce jour, traduit en français.}, les
récits relatifs à sa vie sont l'œuvre de son école, et ont un caractère
hagiographique marqué. Quoi qu'il en soit de la réalité historique
précise des rêves ici racontés, le texte n'en est pas moins exact sur la
doctrine d'al-Ašʿarī et son évolution.




\subsection{trois rêves d'Al-As'Ari} \sn{Ibn ʿAsākir, \emph{Tabiyīn kaḏib al-muftarī}, p. 42}

\begin{quote}
    

Alors que je me demandais s'il fallait revenir du muʿtazilisme et que
j'examinais leurs preuves et la démonstration de leur erreur, je vis le
Prophète de Dieu dans mon sommeil dans les premiers jours du Ramaḍān. Il
me dit : « Abū al-Ḥasan, as-tu copié le \emph{ḥadīṯ} ? » Je répondis :

« Oui, Prophète de Dieu ! » Il reprit : « Et n'as-tu pas copié que Dieu,
le Très Haut, sera vu au dernier jour\sn{Le texte fait référence à un débat entre les muʿtazilites et les
ḥanbalites. Ces derniers considèrent qu'il faut comprendre littéralement
les promesses prophétiques, selon lesquelles l'homme, au Paradis, verra
son Créateur ; les premiers, au contraire, considèrent que c'est
impossible, et qu'il convient d'interpréter ces versets en un sens
métaphorique.} ? » J'ai
répondu : « Si, Prophète de Dieu. » Il me dit encore : « Alors qu'est-ce
qui te retient de tenir cette affirmation pour vraie ? » Je dis : « Ce
sont des arguments intellectuels qui m'en ont empêché : j'ai donc
interprété ce qui est dit. » Il me dit : « Et n'y a-t-il pas aussi des
arguments intellectuels qui te disent de croire que Dieu peut être vu au
dernier jour ? » Je dis : « Si, Prophète de Dieu, mais ce sont des
arguments douteux. » Il me dit : « Médite-les et tu verras avec clarté
que ce ne sont pas des arguments douteux, mais des preuves. » Puis il
disparut à mes yeux.

À mon réveil, j'étais terrifié. Je me mis à méditer sur ce que m'avait
dit le Prophète, et je trouvais la situation comme il l'avait décrite :
dans mon cœur, les preuves de l'affirmation {[}des attributs{]} se
renforçaient, tandis que celles de la négation s'affaiblissaient. Je me
taisais et ne montrais rien aux gens, mais j'étais perplexe. Puis alors
que nous entrions dans la deuxième dizaine du mois de Ramaḍān, je vis le
Prophète qui me dit : « Abū al-Ḥasan, as-tu fait ce que je t'ai dit ? »
Je répondis : « Prophète de Dieu, la situation est comme tu le disais !
» Il me dit :

« Réfléchis aux autres questions, et sois attentif. » Je m'éveillai, me
levai et pris tous mes livres de \emph{kalām}, je les mis de côté et me
consacrai aux livres de \emph{ḥadīṯ}, d'explication du Coran et de droit
religieux. Toutefois, je réfléchis aux autres questions, comme il me
l'avait ordonné.

Quand nous entrâmes dans la troisième dizaine du mois, je le vis de
nouveau au cours de la Nuit du Destin\sn{ \emph{Laylat al-Qadr}, « la nuit du Destin », une des dernières nuits
du Ramaḍān, considérée comme la plus sacrée.} et il me dit,
l'air contrarié : « Qu'as-tu fait de ce que je t'ai dit ? » Je répondis
:
\begin{quote}
    « Prophète de Dieu, j'ai continué à penser à ce que tu m'as dit. J'ai
réfléchi sur ces questions. Mais j'ai rejeté et mis de côté tout le
\emph{kalām} pour me consacrer au droit religieux. »
\end{quote}
 Il me répondit avec
colère : « Et qui t'a ordonné cela ? Écris des livres, et réfléchis de
la manière que je t'ai ordonnée : c'est ma religion et ma vérité que
j'apporte. » Je m'éveillai et dès lors j'entrepris d'écrire des livres
pour défendre et exposer la doctrine vraie.
\end{quote}



Ce récit de la « conversion » d'al-Ašʿarī en souligne deux points
essentiels :


\begin{itemize}
\item
  
  d'une part, sur la question de la vision de Dieu comme sur « les
  autres questions » (très certainement relatives aux attributs divins),
  al-Ašʿarī renonce complètement aux thèses muʿtazilites : il passe avec
  armes et bagages aux thèses de l'adversaire, en adoptant les positions
  des « partisans de la Tradition », les sunnites\sn{C'est à partir d'al-Ašʿarī que le terme « sunnisme » (\emph{ahl
al-sunna}, les « gens de la Tradition) cesse de désigner les seuls
ḥanbalites. Les ašʿarites sont eux aussi des sunnites. La diversité
grandit au sein du sunnisme : il compte désormais non seulement des
opposants, mais aussi des partisans de la théologie discursive. C'est
une première étape vers le sens moderne du mot.}.
  
\item
  
  pour autant, d'autre part, le Prophète apparu en rêve ne semble pas
  vouloir se satisfaire de cette pure et simple capitulation : il lui
  demande expressément de continuer à pratiquer cette théologie
  spéculative que les ḥanbalites n'apprécient pas, mais de la mettre au
  service de leur doctrine traditionnaliste. En d'autres termes, il lui
  enjoint de retourner contre les muʿtazilites leurs propres armes
  dialectiques. Car si les muʿtazilites sont dans l'erreur, ce n'est pas
  parce qu'ils emploient un instrument inadéquat (la raison), mais parce
  qu'ils l'emploient mal : la raison bien comprise vient en effet
  soutenir les thèses traditionnalistes. Le recours au récit de rêve ne
  doit pas nous apparaître comme une renonciation au rationalisme :
  l'apparition en rêve est un lieu commun extrêmement fréquent de la
  littérature arabe, y compris la plus rationaliste (ainsi, les
  philosophes racontaient que le calife al-Maʾmūn avait entrepris de
  faire traduire les philosophes grecs après avoir reçu, en rêve, une
  visite d'Aristote !).
  
\end{itemize}

Al-Ašʿarī a rejoint le camp des partisans de la Tradition, les «
sunnites », sans pour autant devenir ḥanbalite : en quittant les
muʿtazilites, il ne va pas abandonner le champ de la théologie
spéculative. Mais il est contraint de combattre sur deux fronts à la
fois : contre les muʿtazilites, qui le considèrent comme un traître, il
affirme la réalité des attributs divins ; contre les ḥanbalites, qui
répugnent à accueillir comme un des leurs ce théologien, il plaide
la cause de la raison spéculative et de la science théologique\sn{Il le fait notamment dans son traité \emph{Pourquoi il est excellent
de s'engager dans la théologie} (Risāla fī istiḥsān al-ḫawḍ fī ʿilm
al-kalām), où il répond énergiquement aux objections des ḥanbalites qui
considèrent que, si la théologie spéculative était utile, le Prophète et
ses compagnons l'aurait pratiquée. Al-Ašʿarī rétorque que, certes, ils
n'en n'ont pas rédigé de traités, mais qu'ils ont en réalité consacré
beaucoup de temps à ces questions de toute première importance pour tout
croyant.}. Une
position difficile, qui lui vaut de très nombreux adversaires.

Il convient de noter qu'un autre récit de la conversion d'al-Ašʿarī
porte sur un autre point de doctrine en débat, la question du
libre-arbitre et de la prédestination. Sur ce point, al-Ašʿarī se
montrera créatif, proposant une position intermédiaire entre ces deux
extrêmes (la doctrine de l'« acquisition » (\emph{kasb}), par l'homme,
des actes créés pour lui par Dieu), qui s'efforce de respecter à la fois
la toute-puissance divine et la responsabilité de l'homme dans ses
actions.

\hypertarget{la-thuxe9ologie-dal-aux161ux2bfarux12b}{%
\section{La théologie
d'al-Ašʿarī}\label{la-thuxe9ologie-dal-aux161ux2bfarux12b}}

Sur la question théologique centrale, sur les capacités du langage
humain à parler adéquatement de Dieu, al-Ašʿarī adopte donc les thèses
des ḥanbalites, mais il entreprend de les défendre à l'aide d'une
argumentation rationnelle. Ce que nous avons conservé de cette dernière
garde un fort caractère apologétique, où il est moins question de rendre
compte du sens profond de la révélation que détruire les attaques des
muʿtazilites et montrer que les formulations révélées ne sont pas
nécessairement aberrantes quand on les prend en leur sens littéral, ni
qu'elles conduisent nécessairement à une approche anthropomorphique. On
va le constater, al-Ašʿarī n'est pas un métaphysicien de premier ordre.
Il semble avoir presque tout ignoré du mouvement philosophique qui s'est
développé dans le monde musulman, et qui n'a eu d'influence notable ni
sur ses thèses, ni sur ses méthodes.

 
  
  \subsection{L'existence de Dieu}
  
 

C'est par une preuve de l'existence de Dieu qu'al-Ašʿarī ouvre un de ses
ouvrages majeurs d'exposition systématique de sa théologie, le
\emph{Kitāb al-Lumaʿ}. La démarche nous
semble naturelle, mais elle n'est alors pas si commune. Al-Ašʿarī n'est
pas le premier à proposer une preuve de l'existence de Dieu (il semble
que ce premier soit, en islam, le théologien zaydite Qāsim ibn Ibrāhīm),
la démarche n'est pas fréquente : beaucoup de théologiens considèrent
que l'existence de Dieu est le fondement même du savoir, une donnée
évidente par elle-même.




\begin{quote}
   Question : Quelle est la preuve que la création a un auteur qui l'a
créée, et un organisateur qui l'a organisée ?\sn{PREUVE DE L'EXISTENCE DE DIEU (al-Ašʿarī, \emph{Kitāb al-Lumaʿ},
§ 3).}

Réponse : La preuve est la suivante. L'être humain, même quand il est au
sommet de sa perfection, a d'abord été successivement du sperme, un
caillot, un petit amas, et enfin de la chair et du sang. Nous savons que
ce n'est pas lui-même qui s'est fait passer d'un état à un autre. En
effet, nous voyons que, alors même qu'il est haut plus au degré de sa
force et de son intelligence, il est incapable de se fabriquer des yeux
pour voir ou des oreilles pour entendre, ni de se créer un membre
quelconque. C'est bien la preuve qu'il était encore plus incapable de le
faire avant même d'avoir acquis sa force et son intelligence : ce qu'il
est incapable de faire dans son état de perfection, à plus forte raison
en sera-t-il incapable dans un état de faiblesse.

De plus, nous voyons que l'homme est d'abord un enfant, puis un jeune,
puis un adulte, enfin un vieillard, et nous savons qu'il ne se fait pas
passer lui-même de l'état de jeunesse à celui de maturité ou de grand
âge, puisque s'il s'efforçait de quitter la maturité ou le grand âge
pour revenir à la jeunesse, il ne le pourrait pas. Ce qui prouve bien
qu'il ne se fait pas passer de lui-même d'un état à un autre, et qu'il y
a quelqu'un qui le fait passer d'un état à un autre et qui l'a organisé
comme il est, car il est impossible qu'il passe d'un état à un autre
sans l'aide de quelqu'un qui le change et l'organise.

On peut prendre un exemple pour l'expliquer : le coton ne peut pas
devenir du fil, et le fil du tissu, sans l'aide d'un fileur et d'un
tisserand. L'homme qui achète du coton en espérant le voir se changer en
fil puis en tissu sans l'aide d'un fileur et d'un tisserand est un fou,
tout comme celui qui, dans un désert, s'attend à voir la boue se changer
en briques qui s'empileraient d'elles-mêmes, sans l'aide d'un briquetier
et d'un maçon, est un imbécile. 
\end{quote}


L'argumentation d'al-Ašʿarī n'est pas abstraite ou logique : elle part
de l'expérience la plus commune. Si nous étions toujours identiques,
nous pourrions nous croire éternels, sans Créateur, mais nous changeons,
du sperme initial jusqu'au vieillard, en passant par l'enfant. Or les
choses ne changent pas toutes seules : il faut un agent pour penser et
réaliser ce changement, et l'homme ne saurait être cet agent pour
lui-même. L'argument est simple, concret, appuyé qui plus est sur deux
comparaisons très claires.

Pour nous, la principale faiblesse de la preuve proposée par al-Ašʿarī
tient évidemment à son ancrage dans une conception scientifique qui nous
semble dépassée : de l'embryon à l'adulte, n'assiste-t-on pas plus à une
évolution progressive, inscrite dans les caractères mêmes du vivant,
plutôt qu'à des transformations d'un état à un autre ? Mais l'argument
est aussi ancré dans des conceptions tirées du Coran, auquel les
allusions sont en fait très nombreuses, en particulier dans le
vocabulaire employé\sn{9 On relève en particulier la présence, comme sous-texte évident de
l'argument, les versets 12-14 de la sourate 23 : « Nous avons créé
l'homme d'argile fine, puis nous en avons fait une goutte de sperme
contenue dans un réceptacle solide ; puis, de cette goutte, nous avons
fait un caillot de sang, puis, de cette masse nous avons créé des os ;
nous avons revêtu les os de chair, produisant ainsi une autre création.
»}. Cet ancrage coranique est un élément important de
la théologie d'al-Ašʿarī, qui n'exclut nullement le recours à
l'argumentation rationnelle.

 
  \subsection{La question anthropomorphique}
 

L'existence de Dieu étant acquise, il s'agit de comprendre comment nous
pouvons en parler. Al-Ašʿarī se heurte bien vite au problème
traditionnel des attributs divins, comme on le voit dans le texte
suivant, à propos des «mains» de Dieu dont il est souvent question
dans le Coran. On se souvient que les ḥanbalites affirmaient que,
puisque Dieu le disait lui-même dans le Coran, alors il avait
nécessairement des mains, encore qu'on en ignorât le
« comment ». Les muʿtazilites leur répondait qu'accepter le sens
littéral du texte revenait à reconnaître que Dieu avait des membres, et
donc un corps, et donc à tomber dans l'anthropomorphisme le plus indigne
du Dieu transcendant : il fallait donc en avoir une
lecture nécessairement métaphorique. L'argumentation d'al-Ašʿarī ne
semble guère résoudre
le dilemme, et peut sembler décevante :
\begin{quote}
    DIEU A-T-IL DES MAINS ? (AL-ASʿARI, Ibāna, p. 40)
Réponds-leur : « Pourquoi refusez-vous que Dieu, quand il parle de ses “deux mains’’, veuille signifier autre chose que ses ‘‘deux bénédictions’’ ? » S’ils te disent : « C’est parce que, si la main n’est pas une bénédiction, alors elle ne peut être qu’un membre. », dis-leur : « Et sur quoi se fonde votre affirmation selon laquelle, si ce n’est pas une bénédiction, alors elle ne peut être qu’un membre ? »

\end{quote}
Le théologien, ici, n'invente pas une théorie originale et neuve. Il se
contente de desserrer l'étau dans lequel le raisonnement des
muʿtazilites piégeait ceux qui tenaient au sens littéral du texte
coranique. Ou bien vous admettez le sens métaphorique, disaient-ils, ou
bien vous croyez que Dieu a des membres et un corps : il n'y a pas
d'autre choix possible. L'effort d'al-Ašʿarī consiste au contraire à
nier cette nécessaire alternative, qui ne lui paraît fondée sur aucun
argument. La prétention des muʿtazilites à détenir la vérité rationnelle
est usurpée, et il convient de le montrer en refusant le cadre dans
lequel ils enferment leurs adversaires. Al-Ašʿarī ne croit pas que Dieu
ait un corps ; et prendre au sérieux le texte coranique, sans en faire
des lectures métaphoriques, n'implique pas nécessairement de croire
qu'il en a un. Al-Ašʿarī ne va pas plus loin : il n'explique pas ce que
peut être une main qui n'est ni une allégorie, ni un membre corporel. Il
lui suffit de souligner qu'il n'y a pas d'obligation à accepter cette
alternative-là.

Voici comment il argumente sur la question de la possibilité pour
l'homme de voir Dieu au paradis. Un \emph{ḥadīṯ} du Prophète l'affirme
positivement, mais pour les muʿtazilites, la vision de Dieu par l'homme
au paradis est impossible à admettre, car une fois encore, affirmer que
Dieu est accessible à nos sens, c'est supposer à Dieu une forme de
corps.

\begin{quote}
    SUR LA VISION DE DIEU (Al-Ašʿarī, Ibāna, p. 15 – trad. Michel Allard)
Il n’y a aucun existant tel qu’il ne soit possible à Dieu de nous le faire voir ; ce qui est impossible, c’est seulement de voir le néant. Mais comme Dieu est réellement existant, il n’est pas impossible qu’il se fasse voir lui-même à nous.
Autre argument : Dieu voit les choses ; mais s’il voit des choses, celui qui ne voit pas les choses ne se voit pas lui-même ; mais s’il se voit lui-même, il lui est possible de se faire voir lui-même à nous. La raison en est que celui qui ne se connaît pas lui-même, ne connaît pas les choses ; et comme Dieu connaît les choses, il se connaît lui-même. Pour la même raison, celui qui ne se voit pas lui-même, ne voit pas les choses, et comme Dieu voit les choses, il se voit lui-même, et comme il se voit lui-même, il lui est possible de se faire voir lui-même à nous. De même, parce qu’il se connaît lui-même, il lui est possible de se faire connaître à nous.

\end{quote}

Le premier argument est très simple : rien n'est logiquement invisible,
sinon ce qui n'existe pas ; puisque Dieu existe, il n'est pas
nécessairement invisible. Là encore, il s'agit de contrer l'approche
muʿtazilite : il n'est pas nécessaire, comme ils l'affirment, de croire
que la vision de Dieu est rationnellement impossible. Le deuxième
argument, de présentation plus sophistiquée, n'est en fait pas moins
simple : Dieu se voit lui-même, donc il est possible de le voir. Quant à
l'impossibilité, pour des yeux de chair, de réaliser une telle vision,
al-Ašʿarī la confronte habilement à toute-puissance de Dieu, qui peut
tout ce qui n'est pas logiquement impossible. Dans les deux cas,
l'argument repose sur un postulat : les verbes « voir » et « être vu »,
s'agissant de Dieu, ont le même sens que lorsque on l'applique aux
créatures, c'est-à- dire le sens que nous connaissons d'ordinaire. C'est
un choix que ses adversaires ne manqueront pas d'accuser de
\emph{tašbīh}, d'assimilation de Dieu aux créatures, qui ne respecte pas
l'absolue transcendance divine.


  \subsection{Les attributs et l'essence divine}
  


Ce dernier exemple souligne un élément essentiel de la méthode
théologique d'al-Ašʿarī : confronté à la question du sens des mots
humains quand on les applique à Dieu, il fait le choix de l'univocité*.
\begin{Def}[Univocité-équivocité] Est dit univoque un terme qui, appliqué à plusieurs êtres différents, conserve toujours le même sens ; à l’inverse, est dit équivoque un terme qui comporte plusieurs significations. En théologie, il est classique de discuter de l’univocité ou de l’équivocité des mots employés à la fois pour Dieu et pour les créatures. Par définition, nous ne connaissons des mots que nous employons que le sens relatif aux créatures : nous savons ce que signifie l’adjectif « bon » appliqué à un aliment ou à une personne (et déjà, nous trouvons deux sens tout à fait différents) ; appliqué à Dieu, l’adjectif prend-il un sens radicalement différent (équivocité), qui risquerait alors de nous être inaccessible, ou a-t-il le même sens que pour un bon vin ou un bon grand-père (univocité), au risque de méconnaître la transcendance divine ? 
\end{Def}
Cette question traverse toutes les théologies. Comment comprendre, par
exemple, le sens de l'adjectif « bon », quand on dit que « Dieu est bon
» ? Dire que la bonté de Dieu a le même sens que la bonté des créatures,
c'est affirmer l'\textbf{univocité} des termes appliqués à Dieu ou aux
créatures. Dire que la bonté de Dieu n'a pas le même sens que celle des
créatures, qu'on emploie deux fois le même mot pour exprimer des
réalités différentes (comme on parle de
« mineurs » à la fois pour les moins de dix-huit ans et pour les
ouvriers des mines), c'est tenir pour l'\textbf{équivocité}*. La
première de ces positions, l'univocité, a le défaut de ne guère
respecter la transcendance divine (nous avons rappelé plusieurs fois ce
verset coranique qui l'affirme si nettement : « Rien n'est semblable à
Lui », Coran 42, 11). Mais si la seconde, l'équivocité, ne rabaisse pas
la bonté de Dieu (ou sa sagesse, ou sa puissance...) au simple niveau de
la créature, si elle n'en fait pas une chose banale, elle en fait en
revanche quelque chose d'incompréhensible, car en fait de bonté, nous ne
connaissons que celle de la créature. Dire que la bonté de Dieu est
toute différente, qu'on emploie le même mot pour parler de deux réalités
sans lien entre elles, c'est admettre que nous ne savons pas ce que veut
dire « bonté » dans ce cas, car la seule bonté que nous connaissons,
c'est celle des créatures\sn{Entre ces deux possibilités, il en existe une troisième :
l'\textbf{analogie}, pour laquelle la bonté de Dieu et celle d'une
créature sont certes différentes, mais ont quelque chose en commun.
Ainsi, la bonté des créatures nous fait percevoir quelque chose de la
bonté de Dieu, même si cette dernière est incomparablement plus grande
et plus parfaite. Mais en dépit des efforts des chercheurs, pour qui
cette position médiane aurait semblé convenir à la manière équilibrée
d'al-Ašʿarī d'aborder les questions, on n'en a pas trouvé trace dans les
œuvres qui nous sont parvenues. Elle sera cependant utilisée par la
suite par des théologiens musulmans de différentes écoles.}.

Alors les muʿtazilites niaient la réalité des attributs divins, en les
assimilant à l'Essence divine, et que les ḥanbalites optaient pour
l'équivocité, al-Ašʿarī s'en tient à l'univocité, sans laquelle le
langage humain lui paraît condamné à l'insignifiance. Dès lors, les
règles du langage humain deviennent des règles pour la théologie.
Affirmer, comme le font les muʿtazilites, que Dieu est savant, mais
qu'il n'a pas de science, ou qu'il est puissant mais n'a pas de
puissance, c'est pour al-Ašʿarī dire quelque chose d'incompréhensible.
Quand on dit
d'un homme qu'il est savant, constate-t-il, cela signifie nécessairement
qu'il a une certaine science. Il en va de même pour Dieu, sans quoi les
mots n'auraient plus aucun sens.

Cela permet à al-Ašʿarī de déduire même des attributs que les ḥanbalites
n'attribuent pas à Dieu, parce qu'ils ne sont pas explicitement dans le
Coran. Ce dernier qualifie plusieurs fois Dieu de « voyant »
(\emph{baṣīr}), mais il ne parle jamais de la « vue » (\emph{baṣar}) de
Dieu ; les ḥanbalites refusent donc de parler de cette dernière,
puisqu'elle n'est pas révélée, mais pour al-Ašʿarī, le fait d'être
voyant implique nécessairement la vue.

Quel statut ontologique accorder à cette science, à cette puissance, à
cette vie, qui existent éternellement à côté de Dieu ? On se souvient
que les muʿtazilites, pour maintenir la stricte unicité divine, les
identifiaient purement et simplement à l'essence de Dieu. Al-Ašʿarī
refuse cette solution, qui prive chacun des attributs divins de toute
substance propre, de toute particularité (puisqu'alors, pour Dieu, la
science n'est pas autre chose que l'être). Si ces attributs éternels ne
sont pas Dieu, répondent les muʿtazilites, c'est donc qu'il existe à
côté de Dieu d'autres entités éternelles ? Non pas, répond al-Ašʿarī par
une formule relativement énigmatique, mais commode : les attributs ne
sont « ni Dieu lui-même, ni autre chose que Dieu » (\emph{lā ʿaynuhu,
wa-lā ġayruhu}). Le théologien refuse ainsi de se soumettre à une
alternative qui, cette fois encore, lui semble inacceptable ; mais il
n'explique guère comment il y échappe sur le fond.


  \subsection{Le Coran incréé}


Sur cette question si longtemps et si âprement débattue au cours de la
\emph{miḥna}, al-Ašʿarī se range du côté des ḥanbalites, dont il accepte
dans certains textes (notamment dans l'\emph{Ibāna}, le plus ḥanbalite
de ses traités) les formulations les plus radicales : le Coran, Parole
de Dieu, est éternel dans l'ensemble de ses manifestations.

L'effort d'al-Ašʿarī va consister à appuyer cette profession de foi sur
des arguments rationnels qu'il s'efforce alors d'opposer à ses
adversaires muʿtazilites, comme dans l'exemple que je vous propose ici.
\begin{quote}
    COMMENT SAVOIR QUE LE CORAN N’A PAS ETE CREE ? (AL-ASʿARI, Ibāna, p. 23)
Autre argument. Dieu a dit : « Dis : Il est le Dieu unique, Dieu le Rocher, il n’a pas engendré ni été engendré, et nul n’a pu l’égaler. » (Coran, sourate 112) Comment donc le Coran serait-il créé, alors que le nom de Dieu est dans le Coran ? Cela impliquerait nécessairement que les noms de Dieu soient créés ; or si ses noms sont créés, alors son unicité est créée, et de même pour sa science ou sa puissance.
\end{quote}


L'argument proposé ici est étrange, et ne vous paraîtra pas
nécessairement très convaincant. Il repose sur une idée simple : une
réalité n'existe pas sans qu'existe un mot pour la dire. Or le Coran
contient des vérités éternelles, comme ces attributs divins (l'unicité,
la science, la puissance), qui sont, comme on l'a vu, éternels. Si donc
ces mots du Coran qui disent Dieu avaient été créés, la réalité à
laquelle ils renvoient devrait l'être aussi, puisque les attributs ne
peuvent exister sans un mot qui les dise. Et si le mot existe
éternellement, comment refuser que le Coran, qui les contient et les
exprime, soit lui aussi éternel ? Là encore, l'argument repose sur
l'univocité, et suppose que, quand nous disons que Dieu parle, cela a le
même sens que lorsqu'un homme parle.

Al-Ašʿarī ne s'est pas imposé, de son vivant, comme le conciliateur
capable de réaliser la synthèse des adversaires farouches qui
s'opposaient sur le terrain de la théologie musulmane ; il a été, au
contraire, rejeté par les deux camps. Mais son école, en proposant une
véritable théologie discursive pour défendre les thèses sunnites, s'est
progressivement imposée, comme nous le verrons, comme la véritable
orthodoxie islamique, tenante d'un
« juste milieu » qui aimera se présenter comme la voie du bon sens entre
deux positions extrêmes et à ce titre déraisonnables.

De ce fait, l'influence d'al-Ašʿarī aura été considérable dans
l'histoire de la théologie musulmane. Cela ne signifie pas toutefois
qu'il se soit agi d'un théologien de grande envergure. Ses raisonnements
sont souvent plus dialectiques, sinon polémiques, que
véritablement explicatifs. Les contradictions qui existent, d'un texte à
un autre, ont provoqué beaucoup de discussions chez les chercheurs, qui
ont pu de ce fait présenter des visages très contrastés du théologien.
Ainsi, au début du XXe siècle, Ignaz Goldziher pouvait estimer que
« sa doctrine distille quelques gouttes de rationalisme dans l'huile de
l'orthodoxie », tandis que quelques décennies plus tard, Josef Schacht y
voyait à l'inverse un « muʿtazilite orthodoxe » ! Sans doute faut-il
voir dans ces contradictions des évolutions, qui nous sont mal connues,
dans la pensée d'un théologien peu armé pour les développements bien
construits, mais habité d'une intuition d'une extrême fécondité.


\section{Synthèse}

  1. En quoi la doctrine d’al-Ašʿarī diffère-t-elle du rationalisme des muʿtazilites et du traditionalisme des ḥanbalites ?  
  \begin{Synthesis}
  
  al-Ašʿarī rejette le prima de la raison sur la Tradition. En ce sens, il est clairement sunnite. Cependant, il considère qu'on ne peut rester dans une position fidéiste, sans essayer de rendre raison de notre Foi. 
  \end{Synthesis}
  
  2. La preuve de l’existence de Dieu proposée par al-Ašʿarī paraît-elle convaincante ?  \begin{Synthesis}
 La preuve proposée par al-Ašʿarī s'appuie sur l'expérience concrète que l'homme change et qu'il y a besoin d'un agent extérieur pour faire ce changement. Il s'appuie aussi sur de nombreuses références coraniques. Mais pour nous, cette preuve de Dieu est ancrée dans une conception scientifique qui nous semble
dépassée : de l’embryon à l’adulte, n’assiste-t-on pas plus à une évolution pro-
gressive, inscrite dans les caractères mêmes du vivant, plutôt qu’à des transformations d’un état à un autre ? 
 
  \end{Synthesis}
  
  3. Quel sens et quelles conséquences a le choix, par al-Ašʿarī, de l’univocité pour les attributs divins ? 
  
  \begin{Synthesis}
  al-Ašʿarī choisit une interprétation univoque des noms de Dieu : si l'on parle de sagesse de Dieu, elle est à comprendre comme celle de l'homme, sinon le sens équivoque ne nous permet de rien dire de Dieu (approche Hanbalite). Cela lui permet d'en découler un certain nombre d'attributs divins, comme la vision (puisque Dieu est appelé voyant).
  A noter que al-Ašʿarī ne semble pas utiliser l'analogie que ses successeurs utiliseront, position médianne entre l'équivocité des Hanbalites ("Bonté divine" n'a rien à voir avec "bonté humaine") et Mu'tazilite (certains attributs ne peuvent associées à Dieu car anthropomorphisme).
  \end{Synthesis}
  
  
  \section{Bibliographie}
  \subsection{Instruments de travail}
    W. Montgomery Watt est l’auteur, dans l’Encyclopédie de l’islam (2e édition), de deux articles relativement sommaires, consacrés au fondateur (« al-Ashʿarī (Abū al-Ḥasan) », vol. 1-2, pp. 714-716) et à l’école elle-même (« Ashʿariyya », même vol., pp. 717-718).  L’article consacré aux débuts de l’ašʿarisme dans le Oxford Handbook of Islamic Theology (sous la dir. de S. Schmidtke, 2016), « Between Cordoba and Nīsābūr », par J. Thiele (pp. 225-241) laisse un espace limité à la naissance du mouvement. 
    \subsection{Études }
    M. Allard, Le problème des attributs divins dans la doctrine d'al-Ašʿarī et de ses premiers grands disciples, Beyrouth, Imprimerie catholique, 1965. 
    
    R. McCarthy, The Theology of al-Ashʿarī : The Arabic Texts of al-Ashʿarī's "Kitāb al-Lumaʿ" and "Risālat Istiḥsān al-Khawḍ fī ʿIlm al-Kalām" with Briefly Annotated Translations and Appendices Containing Material Pertinent to the Study of al-Ashʿarī, Beyrouth, Imprimerie catholique, 1953. 
    
    D. Gimaret, La doctrine d’al-Ashʿarī, Paris, Le Cerf, 1990. 
    
    G. Makdisi, « Ashʿarī and the Ash'arites in Islamic Religious History », in Studia islamica 17 (1962), pp. 37-80. 