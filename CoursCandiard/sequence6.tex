\chapter{La théologie traditionaliste}

\hypertarget{suxe9ance-6}{%
\mn{Séance 6}\label{suxe9ance-6}}

Le succès de l'ašʿarisme a longtemps occulté, notamment dans
l'historiographie occidentale, l'existence et le développement d'une
véritable théologie traditionaliste, à tort considérée comme marginale
ou négligeable --- voire inexistante ou intellectuellement sans relief.
Nous avons déjà eu l'occasion de souligner que, dès son origine, le
courant ḥanbalite n'est nullement une école obscurantiste bornée : elle
soulève des objections réelles à la tentative de purification
muʿtazilite, et exprime une vision du monde cohérente, bien qu'assez
étrangère pour nous. Au fil des siècles, ce mouvement se maintient à
côté de l'ašʿarisme triomphant, avec lequel il entretient un dialogue
souvent passionné, mais assez fécond, qui verra naître plusieurs
penseurs de premier ordre.

\hypertarget{le-cas-dibn-ux1e25azm}{%
\section{Le cas d'Ibn Ḥazm}\label{le-cas-dibn-ux1e25azm}}

\paragraph{Ibn Ḥazm de Cordoue}Nous avons jusqu'à présent identifié le littéralisme et le
traditionalisme ḥanbalite. Cela n'est pas tout à fait juste, et il
convient de mentionner ici un théologien andalou d'une grande
originalité, et pour tout dire assez inclassable, Ibn Ḥazm de Cordoue \label{Theol:IbnHazmCordoue}
(m. en 1064), qui fut aussi un juriste apprécié et un poète plus célébré
encore\sn{Il est l'auteur d'un livre poétique et personnel sur l'amour,
extrêmement original, \emph{Le collier de la colombe}, plusieurs fois
traduit en français.}.

\paragraph{Une école juridique littéraliste sans analogie}Né dans une Espagne musulmane dont la longue stabilité, sous la
domination omeyyade (756-1031), s'effondrait brutalement, Ibn Ḥazm
reçoit une éducation soignée à la cour des derniers souverains omeyyades
de Cordoue. Écrivain prolixe, polémiste ardent et parfois
injuste, Ibn Ḥazm s'affilie, en droit musulman, à une école juridique
dont il est dans l'histoire presque l'unique représentant qui nous soit
connu : l'école ẓahirite (de l'arabe \emph{ẓāhir}, qui désigne le sens
littéral). Cette cinquième école\sn{Les quatre écoles classiques du droit musulman sont le hanafisme, le
shafiʿisme, le malékisme et le ḥanbalisme (on se souvient que ce dernier
est à la fois une école juridique et une école théologique). Au cours de
l'histoire, quelques autres écoles de droit islamique ont existé ; le
ẓahirisme est la principale.}, qui se réfère au juriste du IXe
siècle Dāwūd al-Ẓāhirī, se distingue par un strict littéralisme dans sa
lecture des sources du droit, le Coran et les \emph{ḥadīṯ}-s, et le
refus du raisonnement par analogie. Ce dernier suppose en effet d'entrer
dans les intentions même de Dieu, de comprendre la logique de ses
commandements. Ainsi, la plupart des écoles considèrent que, si Dieu
interdit la consommation du vin, il interdit du même coup celle de toute
substance qui aura des effets comparables. Or Ibn Ḥazm estime au
contraire que les\textsc{ commandements divins sont arbitraires}, et qu'il faut
les prendre comme tels : chercher à comprendre les raisons de Dieu
serait aussi vain que prétentieux. Il faut se contenter d'obéir
strictement : si Dieu avait voulu interdire davantage, il l'aurait dit,
tout simplement.

Ce littéralisme n'est nullement ennemi de la raison. Au contraire, Ibn
Ḥazm le conjugue à un usage très informé de la logique aristotélicienne,
ce qui n'est contradictoire qu'en apparence. L'enjeu de notre salut est
tel, explique-t-il, que Dieu ne peut avoir permis la moindre ambiguïté :
sa Loi ne peut donc avoir qu'un seul sens, qui est le sens littéral.
Chercher des interprétations et des métaphores, c'est instiller le doute
et l'incertitude dans un domaine qui ne peut pas les supporter. Il faut
donc lire les textes révélés au sens premier, selon les canons de la
logique la plus stricte.

\paragraph{Polémique anti-juive et Chrétienne avec une lecture littéraliste}C'est sur ce fondement singulier de littéralisme et de logique qu'Ibn
Ḥazm rédige son œuvre la plus célèbre, le volumineux \emph{Kitāb
al-fiṣal fī al-milal wa-l-aḥwāʾ wa-l-niḥal}, un monument de polémique
religieuse. Il y fonde sur une argumentation explicitement rationnelle
une réfutation vigoureuse des doctrines opposées à la sienne : les
courants théologiques de l'islam avec lesquels il est en désaccord (le
muʿtazilisme et l'ašʿarisme en
premier lieu, mais aussi un grand nombre d'autres mouvements), et les
autres religions, à commencer par le judaïsme et le christianisme. Il
s'en prend notamment, de façon originale, à l'irrationalité des
Écritures juives et chrétiennes, qu'il a lues avec grande attention mais
selon son système herméneutique strictement littéraliste, et il entend
par cette polémique souvent féroce ramener ces égarés dans la voie de la
vérité, au moyen d'une argumentation logique.

C'est la même méthode, élaborée d'abord dans un cadre juridique, qu'Ibn
Ḥazm applique dans un cadre théologique. Sur la question des attributs
divins, des capacités de notre langage à parler de Dieu, Ibn Ḥazm
s'oppose au choix d'al-Ašʿarī sur l'univocité (les termes ont le même
sens, qu'on les applique à Dieu ou à la créature) et prend très
clairement le parti opposé, celui de l'équivocité : quand nous parlons
de la science de Dieu, de sa puissance, de sa vie, ces mots-là n'ont
absolument pas le même sens que lorsque nous évoquons la science, la
puissance ou la vie d'un être créé.

\paragraph{Sur les attribut divins, le choix de l'équivocité fait qu'ils n'ajoutent rien sur l'essence de Dieu mais sont des noms propres}Ce choix de l'équivocité signifie-t-il donc que, lorsque le Coran
qualifie Dieu de « savant », il ne dit rien d'intelligible, comme le
soupçonnent les adversaires de l'équivocité par un raisonnement par
l'absurde ? Ibn Ḥazm semble accorder à l'attribut une signification
minimale, par une formule qu'il emprunte formellement aux muʿtazilites :
« La science de Dieu n'est rien d'autre que Dieu lui-même. » Il refuse
ainsi de donner aux attributs divins une quelconque forme d'autonomie ou
d'existence propre. Mais on ne doit pas déduire de cette formule
lapidaire qu'il identifie les attributs de Dieu à son essence, à la
manière des muʿtazilites. Il avait au contraire refusé, faute de
fondement scripturaire, cette forme d'identification. Sa position est
plus prudente : les attributs ne sont que des désignations sans contenu
concret. Quand le texte révélé déclare que Dieu est savant, il dit que
Dieu veut être désigné ainsi. L'attribut ne nous dit rien de la nature
de Dieu ; il s'agit simplement d'un nom que Dieu nous demande d'utiliser
à son égard. Les attributs divins sont donc, d'une certaine façon, des
noms propres. Quand Dieu les révèle, il commande simplement qu'on fasse
usage
de ces noms propres, sans pour autant les comprendre. Ibn Ḥazm place la
parole qui prend Dieu pour objet du côté de la volonté (obéissance au
commandement divin) plutôt que de celui de l'intelligence.

Dans ces conclusions, Ibn Ḥazm semble relativement proche de l'école
ḥanbalite, donc nous avons souligné l'attachement à l'équivocité. Mais
son point de départ et son cadre intellectuel sont radicalement
différents : c'est au nom d'un rationalisme strict (Ibn Ḥazm prétend
qu'il n'affirme rien qui ne soit fondé sur une démonstration rationnelle
au sens de la logique d'Aristote) qu'il parvient à partager les
conclusions de l'école la plus hostile au rationalisme.

Auteur polémique très isolé de son vivant, Ibn Ḥazm n'aura guère de
postérité, en théologie comme en droit. Mais il représente un effort
particulièrement original de faire cohabiter, au sein de l'islam,
l'absolue transcendance divine et les capacités de la raison humaine.

\hypertarget{la-luxe9gitimituxe9-de-la-thuxe9ologie}{%
\section{La légitimité de la
théologie}\label{la-luxe9gitimituxe9-de-la-thuxe9ologie}}

Quant à l'école théologique se réclamant d'Ibn Ḥanbal, née comme on l'a
vu précédemment en réaction à la prétention des muʿtazilites à dire
l'orthodoxie musulmane, elle connaît des évolutions inattendues dans les
siècles qui suivent ces temps héroïques. Si beaucoup de ses
représentants restent attachés, avec le strict traditionalisme du
fondateur, au refus de toute théologie spéculative, on voit aussi peu à
peu émerger une théologie dialectique ḥanbalite, très influencée dans
ses méthodes par le \emph{kalām} ašʿarite et muʿtazilite. Ce phénomène
s'explique certainement, au moins en partie, par les nécessités de la
polémique. Des auteurs comme Abū Yaʿlā (m. en 1066) et Ibn ʿAqīl (m. en
1119), aux prises avec leurs adversaires, essentiellement chiites et
ašʿarites, développent contre eux une théologie de plus en plus
cohérente et sophistiquée. Si l'on garde à l'esprit que cette période
voit s'éteindre progressivement le muʿtazilisme, tandis que l'ašʿarisme
évolue dans un sens
de plus en plus rationaliste, on ne sera pas surpris de voir ces auteurs
tenir parfois des positions correspondant à celles d'al-Ašʿarī lui-même,
progressivement abandonnées par les théologiens ašʿarites.

Toutefois, ce développement d'une théologie spéculative ḥanbalite est
loin de faire l'unanimité au sein de l'école. En 1072, Ibn ʿAqīl est
contraint par les maîtres ḥanbalites les plus rigoristes à une
spectaculaire rétractation publique : il abjure tout ce qui, dans ses
ouvrages, dénote une influence du muʿtazilisme et déclare vouloir
désormais s'en tenir au strict traditionalisme antirationaliste primitif
de l'école. Après cet éclat, la méthodologie spéculative ne disparaîtra
jamais tout à fait de l'école ḥanbalite, mais elle se fera bien plus
discrète ; l'influence du \emph{kalām} n'y sera plus ouvertement
revendiquée.

\paragraph{Une tendance Hanbalite à charge des théologiens}Des auteurs ḥanbalites, en revanche, n'hésiteront pas à instruire à
charge le procès de la théologie, en lui reprochant avant tout son
manque d'ancrage scripturaire. C'est en particulier le cas du maître
ḥanbalite de Damas Ibn Qudāma (m. en 1223), dans son \emph{Taḥrīm al-
nazar fī kutub ahl al-kalām}\sn{3 ʿAbd Allāh ibn Aḥmad b. Qudāma al-Maqdisī, \emph{Censure of
speculative theology}, 1962.}. Le juriste raconte avoir été instruit de
la rétractation d'Ibn ʿAqīl, survenue quelques décennies plus tôt, et en
tire une attaque en règle contre la spéculation (\emph{al-naẓar}) en
matière théologique. Si les pieux ancêtres (\emph{salaf}) de la
communauté musulmane s'en sont abstenus, c'est que parce que Dieu nous
est radicalement inaccessible, car il diffère radicalement de « tout ce
qui se trouve dans le cœur ou l'imagination »\sn{Ibid., p. 42 (arabe).}.
\begin{quote}
Ibn Qudāma, Ce dont il est possible de discuter Si vous voulez des discussions (al-kalām) et l’accroissement de votre science, faites des recherches en droit (fiqh), ses questions et ses règles, sur l’héritage d’un intestat et ses normes, sur l’héritage acquis, sur les divisions du domaine d’un défunt, sur les questions de reconnaissance, sur le droit d’héritage d’un affranchi et ses multiples aspects, puis les testaments et leurs règles. Et puis, après cela, les questions d’algèbre, de calcul et de mesure des terrains.  Avec ces domaines, vous avez toute latitude pour éviter de vous consacrer (al-ḫawḍ) à ce qui vous est interdit : des questions que vos ancêtres (salaf) n’ont pas discutées, que les imams ont désapprouvées, qui ne vous conduira pas au bien mais ne pourra que vous mener à l’hérésie (bidʿa) où vous serez sous la conduite du diable.  
    \sn{\emph{Ibid.}, p. 65 (arabe).
}
\end{quote}
Dans cette conclusion, qui se poursuit en promettant les pires
châtiments éternels à ceux qui pratiqueraient la théologie spéculative,
Ibn Qudāma délimite le champ des spéculations autorisées : il s'agit en
premier lieu des réflexions juridiques, et secondairement mathématiques.
L'existence de ces dernières nous indique que nous ne sommes pas devant
un refus général de la rationalité, mais bien de son usage dans le
domaine de la théologie, réservé à la réception simple de la révélation.

\hypertarget{la-thuxe9ologie-dibn-taymiyya}{%
\section{La théologie d'Ibn
Taymiyya}\label{la-thuxe9ologie-dibn-taymiyya}}

\paragraph{Un théologien très populaire chez les salafistes}Le nom d'Ibn Taymiyya, théologien mort en 1328, déclenche les passions,
et pas seulement parmi les chercheurs. Au printemps 2020, des millions
d'Egyptiens se sont enthousiasmés pour une série diffusée pendant le
mois de ramadan, \emph{Le Choix}, à la gloire de l'armée nationale, dont
un épisode donne à voir une véritable dispute d'exégèse théologique :
les jihadistes que combat l'armée au Sinaï ont-ils raison de se réclamer
de la pensée d'Ibn Taymiyya, ou en font-ils une lecture faussée,
contrairement aux imams de l'armée qui en traduisent plus fidèlement la
pensée ? Inutile de dire que c'est le seul des auteurs évoqués dans ce
cours qui ait eu droit aux honneurs d'une série grand public\ldots{}

\paragraph{Retrouver le contexte d'Ibn Taymiyya}Il faut dire qu'il est de longue date la référence du monde salafiste,
un mouvement de réforme de l'islam apparu au XIXe siècle et aujourd'hui
en plein succès, et que les assassins --- jihadistes --- du président
égyptien Anouar al-Sadate, tué en 1981, ont justifié la légalité
religieuse de leur acte en s'appuyant sur une de ses fatwas. Un
dangereux extrémiste, donc ! Mais leurs adversaires ne veulent pas le
lui abandonner : ces textes, écrits à la fin du XIIIe siècle, contre le
sauvage envahisseur mongol, ont été indument sortis de leur contexte.

Curieusement, on retrouve cette division jusque chez les chercheurs :
les uns dénoncent un polémiste borné, ennemi de la raison et peut-être
de toute pensée ; les autres remarquent son immense culture, y compris
philosophique, et y voient un penseur immense injustement décrié. Une
chose est sûre : il mérite qu'on s'y arrête un instant, bien que
l'interprétation de cet auteur soit difficile, à cause de l'abondance de
son œuvre, mais surtout de son refus d'organiser sa pensée en un système
ordonné.

\paragraph{contre Chrétiens, mongoles et soufis}Ibn Taymiyya est né dans une famille de juristes ḥanbalites, et c'est
dans ce milieu qu'il prend très tôt sa place comme un savant et un
enseignant reconnu, à Damas. Il se fait connaître particulièrement par
son intransigeance à l'égard des chrétiens, et surtout par la virulence
de sa résistance aux invasions mongoles qui menacent alors plus d'une
fois la Syrie mamelouke. Mais il est bientôt pris dans des discussions
plus théologiques, qui l'opposent en particulier à de puissantes
confréries soufies et qui lui vaudront de nombreux et longs séjours en
prison, au Caire, à Alexandrie et à Damas, où il meurt en 1328, laissant
une œuvre foisonnante et polémique.

Bien que ḥanbalite, Ibn Taymiyya ne s'inscrit pas dans la lignée d'Ibn
Qudāma que nous venons d'évoquer : la théologie est, d'après lui, une
activité parfaitement acceptable, et même recommandée. En cela, Ibn
Taymiyya ne reste pas d'une exacte fidélité à la pensée d'Ibn Ḥanbal
lui-même, dont nous avons vu la méfiance à l'égard de l'activité
théologique, associée pour lui il est vrai au seul muʿtazilisme.
L'explicitation rationnelle du donné révélé ne fait nullement peur à Ibn
Taymiyya, dont la culture théologique et même philosophique est à vrai
dire prodigieuse. Mais, estime-t-il, encore faut-il savoir faire de la
théologie correctement, et c'est là que commencent les difficultés qui
vont le mettre en opposition féroce avec la quasi- totalité des auteurs
qui l'ont précédé dans ce domaine --- avec quelques bêtes noires : les
philosophes islamiques, en particulier Avicenne ; le philosophe et
mystique andalou Ibn ʿArabī (m. en 1240) ; le théologien ašʿarite Faḫr
al-Dīn al-Rāzī (m. en 1210).

\paragraph{préférer le réel aux idées}Que leur reproche-t-il principalement ? De prendre leurs idées pour des
réalités. L'exercice de la pensée suppose un nécessaire outillage
mental, dont il ne met nullement en cause la nécessité : des catégories,
comme le genre et l'espèce, la substance et l'accident, la forme et la
matière ; les nombres abstraits des mathématiciens ; ou les universaux,
comme par exemple « l'humanité » que partagent tous les hommes
particuliers, et qui fait qu'on dit d'Alfred ou de Gustave qu'ils sont
des hommes. L'erreur principale de tous ceux qu'il qualifie de «
rationalistes », c'est de croire que tous ces objets mentaux existent
réellement, qu'ils sont des choses, qu'ils ont une influence sur le
réel. Certains vont jusqu'à leur accorder davantage d'existence à ces
objets intelligibles qu'aux objets sensibles, que nous rencontrons
chaque jour : c'est la théorie platonicienne des idées, mais c'est au
fond ce qu'il croit voir derrière la théorie philosophie des dix
Intellects séparés, dont Avicenne veut faire des êtres véritables. Pour
Ibn Taymiyya, cette confusion est extrêmement dommageable, parce qu'elle
lance des intelligences brillantes sur un terrain purement imaginaire,
où elles se déploient dans les seules limites de la cohérence, mais non
pas du réel. Or la cohérence mentale n'a que peu à voir avec le réel !
Prouver, comme le fait Avicenne, la parfaite cohérence du concept d'Être
nécessaire, cela n'a guère à voir avec une preuve effective de
l'existence du Dieu vivant. On n'est pas toujours très loin de Pascal,
qui opposait le « Dieu d'Abraham, d'Isaac et de Jacob » au « Dieu des
philosophes ».

Poursuivre de telles chimères au détriment du réel, c'est risquer de
tomber bientôt dans le non-sens. Ainsi, par souci de cohérence sur
l'unicité divine et sur sa transcendance, ne faudrait-il pas refuser à
Dieu jusqu'à l'existence ? Sans cela, on attribue à Dieu un universel
qu'il partage avec tous les autres existants. Mais alors, remarque Ibn
Taymiyya avec malice, si on veut éviter d'assimiler Dieu à d'autres
existants, comme l'homme, on en est réduit à l'assimiler à ce qui
n'existe pas, comme les licornes ou les triangles à quatre côtés ? Cette
difficulté amène des auteurs, comme le théologien ismaélien
al-Siǧistānī, à déclarer que Dieu n'est ni existant, ni non-existant ---
une affirmation contraire à une règle de base de la logiqu
d'Aristote, celle du tiers exclu, souligne Ibn Taymiyya non sans
gourmandise. Comment sortir de l'aporie ? En remarquant que ces
questions n'ont aucun sens : elles se fondent sur un présupposé
aberrant, selon lequel l'existence existe. Or l'existence n'existe pas.
Ce qui existe, ce sont des existants : l'arbre, ma belle-mère, la
voiture neuve de mon voisin.

Les conséquences de cette conviction ontologique, qu'il ne formalise
jamais en un système ontologique (car le système, ce serait encore une
concession à ces objets mentaux qui pourtant n'existent pas !), vont
plus loin dans le domaine de la théologie. On se souvient qu'elle est
née par souci de respecter l'unicité divine, et de la purifier des
attributs divins anthropomorphiques mais aussi de l'attribution à Dieu
d'entités intelligibles coéternelles comme la science, la puissance et
la vie. Elle se poursuit chez les \emph{falāsifa} et les ašʿarites
« modernes » dans le refus de diviser l'unicité de l'essence divine par
toute forme de composition, qu'il s'agisse de la distinction de facultés
(l'intelligence et la volonté) ou de la présence en Dieu d'acte et de
puissance, ou de forme et de matière, ou d'espèce et de genre. Ces
débats interminables et indécidables reposent sur une malheureuse
confusion : \textsc{il ne sert à rien de protéger Dieu d'association avec des
entités purement mentales}. Quand le Coran dit que Dieu est un, il parle
simplement d'unicité personnelle : il dit quelque chose de très simple,
que tous comprennent --- à savoir qu'il n'y a pas plusieurs dieux, mais
un seul. Extrapoler sur cette affirmation, comme l'ont fait d'abord les
muʿtazilites, puis tous les théologiens, pour en déduire qu'il faut nier
en Dieu la science ou la puissance, c'est confondre des concepts ---
dont l'existence est purement mentale --- avec le réel. Il faut être
aveuglé par un intellectualisme maladif pour se poser des questions sur
le statut ontologique de la science de Dieu ou sa puissance : ce ne sont
que des concepts, à l'existence évidemment mentale. En dernière analyse,
pour Ibn Taymiyya, le débat sur les attributs divins, que nous avons vu
se déployer au fil des siècles, ne porte sur rien de réel : c'est une
longue discussion privée de toute signification.

Le goût montré par Ibn Taymiyya pour l'existence concrète l'amène à
considérer que n'existent réellement que les objets accessibles au sens.
Dieu, l'existant par excellence, est-il donc accessible aux sens ? et
les anges ? et toutes les autres réalités invisibles dont parle la
religion ? Ibn Taymiyya ne nie pas que Dieu ou les anges nous soient
invisibles, mais c'est une situation provisoire, qui tient moins à leur
nature qu'à notre faiblesse : ils peuvent déjà se rendre visibles en
rêve, et surtout, selon la promesse du Coran et des \emph{ḥadīṯ}-s,
l'homme pourra les voir au paradis. Si Ibn Taymiyya reste prudent,
fidèle à la tradition ḥanbalite, sur l'attribution à Dieu d'un corps,
dans la mesure où la Révélation islamique n'emploie jamais ce terme, il
n'en reste pas moins convaincu que Dieu n'est pas une réalité purement
intelligible, comme le présentent les philosophes, mais qu'il est pourvu
de localisation et d'étendue, et qu'il est accessible aux sens capables
de le percevoir. Mais admettre que Dieu soit quelque part, qu'il ait
véritablement des mains et des yeux, n'est-ce pas tomber dans
l'anthropomorphisme, dont on accuse si fréquemment la théologie
traditionnaliste ? Au contraire, répond Ibn Taymiyya : ce sont les
partisans d'une lecture métaphorique de ces attributs corporels, comme
les muʿtazilites et les philosophes, qui sont les vrais
anthropomorphistes ! Ils sont en effet incapables d'imaginer ces
attributs autrement que dans la stricte imitation de ce qu'ils sont chez
l'être humain. Toutefois, si l'on veut pousser plus loin et interroger
Ibn Taymiyya sur le sens que prennent ces attributs appliqués à Dieu, il
se réfugie dans la traditionnelle réponse du \emph{bilā kayfa}, du «
sans comment », à laquelle il s'efforce de donner une cohérence
théorique en distinguant le sens de l'attribut, qui est connu, de sa
modalité, qui ne l'est pas.

Puisque la question du statut ontologique des attributs de Dieu est,
pour Ibn Taymiyya, une question vide de sens, il ne voit plus d'obstacle
à attribuer à Dieu ses prédicats traditionnels, de science ou de
puissance par exemple. Il serait au contraire insultant de refuser à
Dieu des perfections que nous nous autorisons à donner à ses créatures.
Dira-t-on d'un professeur qu'il est savant, tandis que nous refuserons
de le dire de Dieu, qui en sait
infiniment plus que le professeur ? Ce serait bien mal rendre compte de
la grandeur de Dieu ! Ibn Taymiyya se tient donc résolument du côté des
théologiens qui affirment la réalité des attributs divins.

\paragraph{Une approche originale sur les noms de Dieu}Jusque-là, il ne se distingue guère des autres penseurs
traditionnalistes. Mais alors que ces derniers, pour la plupart,
considéraient qu'il faut affirmer des attributs dont on ignore la
véritable signification, Ibn Taymiyya met sur pied une approche plus
subtile, qui passe par une distinction entre la \emph{signification} de
l'attribut et sa \emph{modalité}. La modalité --- comment Dieu est-il
assis sur un trône ? ou encore : comment Dieu est-il sage ? --- ne nous
est pas accessible, affirme Ibn Taymiyya avec fidélité à la doctrine du
« sans comment », \emph{bilā kayfa}, dont il précise d'ailleurs la
raison. Si la modalité, le « comment », ne peut nous être connu, c'est
précisément parce que les attributs n'ont pas de réalité ontologique
propre, et qu'ils n'existent que dans les essences. Comme l'essence de
Dieu nous est par nature inaccessible, nous ne pouvons comprendre
comment Dieu est vivant ou sage. Mais cela ne veut pas dire --- et ici,
Ibn Taymiyya se distingue de l'école ḥanbalite traditionnelle --- que
nous ne connaissons pas la signification de ces attributs. Au contraire,
nous savons ce que signifient la sagesse ou la vie. Et nous savons ce
qui sépare la sagesse des créatures de celle du Créateur : la seconde
est la perfection de la première, son expression sans aucun de ces
défauts qui ne manquent pas de se trouver dans toutes les qualités des
créatures. Nous n'avons pas de connaissance précise, par nature, des
perfections divines, mais nous pouvons nous en faire une idée : 
\begin{Synthesis}
il
s'agit des qualités des créatures poussées à un point de perfection que
nous n'imaginons pas, mais qu'un raisonnement \emph{a fortiori} permet
tout de même de pressentir.


De cette façon, Ibn Tamiyya échappe à l'alternative déjà rencontrée
entre équivocité et univocité : les attributs de Dieu n'ont ni un sens
radicalement différent de celui qu'on applique aux créatures, ni
précisément le même sens ; ils ont un sens analogue, où le sens des mots
appliqués aux créatures nous met sur la voie du sens qu'ils prennent
appliqués à Dieu.
\end{Synthesis}
Cette solution existera également, et parfois presque
dans les mêmes termes, chez des
théologiens chrétiens médiévaux, notamment Thomas d'Aquin. C'est dire si
l'on aurait tort de caricaturer la théologie traditionnaliste en
l'enfermant dans une étroite défense du fidéisme, rejetant par principe
l'effort théologique.
\paragraph{Lecture}



\emph{Suit un extrait d'une profession de foi d'Ibn Taymiyya, adressée
aux habitants de Palmyre (Tadmur).}



\begin{quote}
    
 
Notre position\sn{IBN TAYMIYYA, \emph{Al-Risāla al-tadmuriyya}}  sur les attributs divins est la même que sur l'essence
(\emph{al-ḏāt}) {[}de Dieu{]}. « Rien n'est semblable à Dieu » (Q 42,
11), ni quant à l'essence, ni quant aux attributs, ni quant aux actes.
Et donc s'il a véritablement une essence, elle n'est pas semblable aux
autres essences, et cette essence est qualifiée par des attributs
véritables qui ne sont pas semblables aux attributs des autres essences.

À la question : « Comment Dieu peut-il être assis sur un trône ? », il
faut répondre, comme l'ont fait Rabīʿa\sn{Rabīʿa ibn Abī ʿAbdul Raḥmān Farrūḫ al-Madanī, muftī
à Médine, mort au milieu du VIIIe siècle.}, Mālik et
d'autres encore : Le fait qu'il soit assis est connu, mais le « comment » est inconnu, parce que c'est une question qui porte sur ce que l'homme
ne peut pas connaître, et à laquelle il est donc impossible de répondre.

Ainsi, à la question : « Comment le Seigneur descend-il jusqu'au ciel
inférieur ? », nous rétorquons : « Et Dieu lui-même, comment est-il ? »
Et si notre interlocuteur répond : « Je ne connais pas le comment de
Dieu », on lui dira : « Et nous, nous ne connaissons pas le comment de
sa descente du ciel. »

En effet, connaître le « comment » des attributs, cela suppose de
connaître d'abord le
« comment » de Celui qui est qualifié par les attributs : c'en est une
partie et une conséquence. Alors comment viens-tu me demander de savoir
comment il entend, il voit, il parle, il descend du ciel ou il s'assoit
sur son trône, alors que tu ne sais rien du « comment » de son essence !

Si tu considères que Dieu a une véritable essence, qui existe par
elle-même, requérant des attributs de perfection, à laquelle rien n'est
semblable, alors sa manière d'entendre, de voir, de parler, de descendre
et de s'asseoir existe aussi par elle-même, et Dieu doit être qualifié
par des attributs de perfection auxquels ne peuvent être assimilés ni la
manière d'entendre des créatures, ni celle de voir, de parler, de
descendre ou de s'asseoir.
\end{quote}
 
\emph{Le début du texte n'a rien pour surprendre, surtout de la part
d'un docteur ḥanbalite : Ibn Taymiyya réaffirme le caractère unique et
transcendant de Dieu, jusque dans son essence, et résume la position
classique du « sans comment ». Mais il se démarque ensuite des autres
traditionnalistes : compte tenu de la réalité de l'essence divine, ce «
sans comment », qu'il défend par une argumentation rationnelle,
n'apparaît pas comme une position fidéiste, indiscutable, mais comme la
position la plus rationnelle !}


 \section{Conclusion}

La théologie islamique existe-t-elle ?, nous demandions-nous à
l'introduction de ce parcours. S'il paraît difficile, au terme, de
conclure par la négative, peut-être avez-vous perçu ce qui a pu amener à
se poser une telle question. Les théologiens de l'islam classique ont eu
à faire face au paradoxe de la transcendance divine : si nous pouvons en
parler, alors ce n'est pas vraiment Dieu ; et si nous ne pouvons pas en
parler, il n'est au fond rien pour nous. Là où les théologiens chrétiens
avaient à leur disposition la ressource du Verbe, permettant à la
transcendance divine et au monde créé de communiquer, les théologiens
musulmans ont dû au contraire s'efforcer de parler de Dieu en gardant
constamment sous les yeux ce verset du Coran que nous avons plus d'une
fois rencontré : « Rien n'est semblable à Lui. »

Nous avons vu plusieurs stratégies se développer face à ce paradoxe de
la transcendance. Celle des philosophes de l'islam a été d'atténuer en
quelque sorte cette transcendance, en proposant des intermédiaires ---
les dix Intellects séparés nés de l'émanation divine --- entre le Dieu
inconnaissable, parfaitement un, et le monde de la multiplicité où nous
vivons. Ces Intellects, assurant l'intelligibilité de Dieu, permettent
de produire dans le monde créé des théophanies, des manifestations de
Dieu, qui nourriront de vastes courants de la spiritualité islamique.
Notamment à travers l'œuvre d'un philosophe et mystique de grande
importance, l'andalou Ibn ʿArabī (mort en ), cette approche nourrira
aussi bien les grandes écoles de théologie chiite qui verront le jour en
Iran au XVII\textsuperscript{e} siècle que de nombreuses confréries
soufies.

Au contraire de cette approche, deux écoles très opposées entre elles
ont fait de l'absolue transcendance de Dieu le cœur de leur théologie,
mais selon des méthodes et avec des résultats très différents. Au nom de
cette intransigeante transcendance, les muʿtazilites engagent un
exigeant processus de purification des conceptions religieuses, y
compris celles que portent les mots de la Révélation, qui provoque bien
des résistances : beaucoup ont ainsi
le sentiment de réduire Dieu à une abstraction inatteignable, bien
éloignée du Dieu actif et volontaire du Coran. C'est le même souci de
respecter la transcendance de Dieu qui guide leurs adversaires, les
ḥanbalites, et les incite à proposer une fidélité au texte divin,
au-delà des compréhensions toujours limitées que nous pouvons en avoir.
La confrontation de ces deux mouvements, parfois violente, les amènera à
approfondir et à nuancer ces approches radicales de la transcendance
divine, ouvrant des dépassements nouveaux du paradoxe que nous avons
mentionné.

L'école ašʿarite, devenue peu à peu l'orthodoxie de l'islam sunnite,
n'est pas nécessairement plus créative que ses rivales, mais elle sait
faire preuve d'une souplesse incomparable qui lui permet d'incorporer,
au fil de ses longues évolutions, les raisons des uns et des autres. Née
comme l'école d'une impossible conciliation entre les muʿtazilites et
les ḥanbalites, elle s'enrichit progressivement d'influences
philosophiques : la cohérence de l'école y perd, mais elle s'enrichit
d'une pluralité de points de vue qui précise son approche de la
complexité du réel.

Ce cours n'est qu'une introduction, trop brève même pour citer tous les
auteurs dont il aurait fallu rendre compte et pour entrer dans le détail
de pensées souvent très subtiles. Mais s'il permet de saisir l'enjeu de
cette question si centrale, celle du langage sur Dieu, et la finesse et
la liberté avec laquelle des penseurs très différents ont cherché à y
apporter une réponse de croyants, il n'aura pas manqué son but. Merci
pour votre attention !

\begin{Synthesis}
Face à la nécessité de conserver l'absolue transcendance de Dieu, les mu'tazilites vont purifier les concepts mais vont engendrer une résistance face à une abstraction inatteignable, éloignée du Coran. Les hanbalites vont proposer une approche fidèle au texte mais où les attributs ne se réfèrent à aucune réalité connue de l'homme.
Les as'arites, avec leur approche univoque vont proposer un "in medio stat virtus".
\end{Synthesis}