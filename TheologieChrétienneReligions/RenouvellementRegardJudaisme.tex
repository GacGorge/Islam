
\chapter{Le renouvellement du regard de l'Église sur le judaïsme au XXe siècle}


\hypertarget{eluxe9ments-bibliographiques}{%
\section{Eléments bibliographiques}\label{eluxe9ments-bibliographiques}}


ANDREVON, T. et KRISEL (dir.), W., \emph{Réflexion juives sur le
christianisme}, Genève, Labor et fides, 2021.

ARTIGES, D., « Les tentations antivijuives de la pensée chrétienne (I).
La contribution de Friedrich Wilhelm Marquardt », \emph{NRT} 142 (2020)
425-436.

ARTIGES, D., « Les tentations antivijuives (II). L'exclusivisme
institutionnel », \emph{NRT} 142 (2020) 606-622.

BOYARIN, D., \emph{La partition du judaïsme et du christianisme}, Paris,
Cerf, 2012.

COMMISSION BIBLIQUE, \emph{Le Peuple juif et ses saintes Ecritures dans
la Bible chrétienne}, Rome 2001.

COMMISSION POUR LES RELATIONS RELIGIEUSES AVEC LE JUDAÏSME,
\emph{Orientations et}
\emph{suggestions pour l'application de la déclaration conciliaire}
Nostra Aetate \emph{n° 4}, Rome 1\textsuperscript{er} décembre 1974.

COMMISSION POUR LES RELATIONS RELIGIEUSES AVEC LE JUDAÏSME, \emph{Notes
pour une}
\emph{correcte présentation des Juifs et du judaïsme dans la prédication
et la catéchèse de l'Église catholique}, Rome 24 juin 1985.

COMMISSION POUR LES RELATIONS RELIGIEUSES AVEC LE JUDAÏSME, \emph{Nous
nous}
\emph{souvenons : une réflexion sur la Shoah}, Rome 16 mars 1998.

COMMISSION POUR LES RELATIONS RELIGIEUSES AVEC LE JUDAÏSME,
\emph{Quarantième}

\emph{anniversaire de} Nostra Aetate, Conférence du cardinal Jean-Marie
Lustiger Rome, le 27 octobre 2005.

COMMISSION POUR LES RELATIONS RELIGIEUSES AVEC LE JUDAÏSME, \emph{« Les
dons et l'appel}
\emph{de Dieu sont irrévocables » (Rm 11,29). Une réflexion théologique
sur les rapports entre catholiques et juifs à l'occasion du 50e
anniversaire de} Nostra ætate (n. 4), Rome 2015.

CONCILE VATICAN II, \emph{Déclaration sur les religions
non-chrétiennes}. Nostra Aetate, 1965. DUJARDIN, J., \emph{L'Église
catholique et le peuple juif. Un autre regard}, Paris 2003.

LUBAC (de), H., « Un nouveau `front' religieux. Israël et la foi
chrétienne » dans \emph{Résistance chrétienne au nazisme. Œuvres
complètes XXXIV}, Paris, Cerf, 2006, p. 151-193.

« Christianisme et judaïsme depuis \emph{Nostra Aetate} »,
\emph{Recherches de science religieuse} 103 (2015).

« La christologie après Auschwitz un programme », \emph{Recherches de
science religieuse} 105 (2017), 5-90.


\hypertarget{introduction}{%
\section{Introduction}\label{introduction}}




  
  \subsection{La remise en cause de l'attitude chrétienne vis-à-vis du
  judaïsme}
  

  
  
  
    
    \emph{Le regard classique des chrétiens sur le Juifs}
    
  
    
    \paragraph{Aux origines du dialogue : le choc de la Shoah}
    
  

  
  \subsection{Le concile Vatican II et \emph{Nostra Aetate} 4}
  

  
  \def\labelenumii{\arabic{enumii}.}
  
    
    \paragraph{La genèse de cette déclaration}
    
  
    
    \paragraph{Les obstacles à la déclaration sur les juifs}
    
  
    
    \paragraph{Commentaire de} Nostra Aetate \emph{4}
    
  
 ~
  \hypertarget{la-ruxe9ception-de-na-4-et-ses-consuxe9quences}{%
  \section{La réception de NA 4 et ses
  conséquences}\label{la-ruxe9ception-de-na-4-et-ses-consuxe9quences}}

  
  \def\labelenumii{\arabic{enumii}.}
  
    
    \paragraph{Fécondité pratique}
    
  
    
    \paragraph{Les interprétations et les développements de NA}
    
  
 ~
  \hypertarget{luxe9tat-de-la-question-des-relations-entre-luxe9glise-catholique-et-le-judauxefsme-les-dons-et-lappel-de-dieu-sont-irruxe9vocables}{%
  \section{L'état de la question des relations entre l'Église catholique
  et le judaïsme : « Les dons et l'appel de Dieu sont irrévocables
  »}\label{luxe9tat-de-la-question-des-relations-entre-luxe9glise-catholique-et-le-judauxefsme-les-dons-et-lappel-de-dieu-sont-irruxe9vocables}}

  
 
  
    
    \paragraph{Statut du document}
    
  
    
    \paragraph{Une relecture de l'histoire de la relation entre juifs et
    chrétiens}
    

    
    \def\labelenumiii{\alph{enumiii}.}
    
      
      Le judaïsme n'est pas une autre religion
      
    
      
      La « judaïté » de Jésus et sa « nouveauté »
      
    
      
      Le christianisme est né au sein du judaïsme
      
    
      
      Vers la séparation
      
    
      
      La théorie de la substitution ou du remplacement
      
    
  
    
    \paragraph{Les questions théologiques}
    

    
    \def\labelenumiii{\alph{enumiii}.}
    
      
      Quelle est valeur théologique de la révélation dont témoigne le
      judaïsme ?
      
    
      
      Quel est le rapport entre Ancien et Nouveau Testament ?
      
    
      
      Quel est le rapport entre ancienne et nouvelle Alliance ?
      
    
      
      Comment les juifs sont-ils sauvés alors qu'ils ne confessent pas
      le Christ ?
      
    
  

