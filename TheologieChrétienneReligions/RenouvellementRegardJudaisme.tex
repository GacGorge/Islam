
\chapter{Le renouvellement du regard de l'Église sur le judaïsme au XXe siècle}
\mn{1 ISTR  2021-2022  Théologie  chrétienne  des  religions  C-1 }

\hypertarget{eluxe9ments-bibliographiques}{%
\section{Eléments bibliographiques}\label{eluxe9ments-bibliographiques}}


ANDREVON, T. et KRISEL (dir.), W., \emph{Réflexion juives sur le
christianisme}, Genève, Labor et fides, 2021.

ARTIGES, D., « Les tentations antivijuives de la pensée chrétienne (I).
La contribution de Friedrich Wilhelm Marquardt », \emph{NRT} 142 (2020)
425-436.

ARTIGES, D., « Les tentations antivijuives (II). L'exclusivisme
institutionnel », \emph{NRT} 142 (2020) 606-622.

BOYARIN, D., \emph{La partition du judaïsme et du christianisme}, Paris,
Cerf, 2012.

COMMISSION BIBLIQUE, \emph{Le Peuple juif et ses saintes Ecritures dans
la Bible chrétienne}, Rome 2001.

COMMISSION POUR LES RELATIONS RELIGIEUSES AVEC LE JUDAÏSME,
\emph{Orientations et}
\emph{suggestions pour l'application de la déclaration conciliaire}
Nostra Aetate \emph{n° 4}, Rome 1\textsuperscript{er} décembre 1974.

COMMISSION POUR LES RELATIONS RELIGIEUSES AVEC LE JUDAÏSME, \emph{Notes
pour une}
\emph{correcte présentation des Juifs et du judaïsme dans la prédication
et la catéchèse de l'Église catholique}, Rome 24 juin 1985.

COMMISSION POUR LES RELATIONS RELIGIEUSES AVEC LE JUDAÏSME, \emph{Nous
nous souvenons : une réflexion sur la Shoah}, Rome 16 mars 1998.

COMMISSION POUR LES RELATIONS RELIGIEUSES AVEC LE JUDAÏSME,
\emph{Quarantième}

\emph{anniversaire de} Nostra Aetate, Conférence du cardinal Jean-Marie
Lustiger Rome, le 27 octobre 2005.

COMMISSION POUR LES RELATIONS RELIGIEUSES AVEC LE JUDAÏSME, \emph{« Les
dons et l'appel}
\emph{de Dieu sont irrévocables » (Rm 11,29). Une réflexion théologique
sur les rapports entre catholiques et juifs à l'occasion du 50e
anniversaire de} Nostra ætate (n. 4), Rome 2015.

CONCILE VATICAN II, \emph{Déclaration sur les religions
non-chrétiennes}. Nostra Aetate, 1965.

DUJARDIN, J., \emph{L'Église
catholique et le peuple juif. Un autre regard}, Paris 2003.

LUBAC (de), H., « Un nouveau `front' religieux. Israël et la foi
chrétienne » dans \emph{Résistance chrétienne au nazisme. Œuvres
complètes XXXIV}, Paris, Cerf, 2006, p. 151-193.

« Christianisme et judaïsme depuis \emph{Nostra Aetate} »,
\emph{Recherches de science religieuse} 103 (2015).

« La christologie après Auschwitz un programme », \emph{Recherches de
science religieuse} 105 (2017), 5-90.

\hypertarget{introduction}{%
\section{Introduction}\label{introduction}}

On est dans une remise en cause de la théologie de l'accomplissement. Nouvelle tournure dans le dernier tiers du XX. Et en particulier du regard sur le judaisme. Pourquoi au moment où elle est largement reçue, cette théologie devient problématique.

\paragraph{Des évènements exterieurs} qui obligent à bouger. 
\begin{itemize}
    \item Les grandes traditions religieuses \textit{perdurent}, à commencer par le judaisme. 
    \item l'écroulement de la domination européenne avec 2 guerres et la Shoah, la décolonisation. 
\end{itemize}

Le constat d'une alterité durable, et donc d'une version universaliste du Christianisme, oblige à une nouvelle attitude, dialogale ?

C. Geffré
\begin{quote}
    On doit être prêt à reconnaître la non catholicité de l'Eglise\sn{elle est née face à l'altérité}.
\end{quote}

\paragraph{L'alterité d'Israël} ne nous permet-elle pas de penser le rapport aux autres religions de façon analogique ? 
\paragraph{Pluralisme religieux de principe} et non de fait car les religions contiennent quelque chose d'irréductible. Il ne s'agit pas d'absorber l'autre mais les reconnaître dans leur différence. 

\paragraph{Plan de Dieu} Le Pape François\sn{\href{https://www.cath.ch/newsf/pape-francois-dieu-a-permis-quil-y-ait-de-nombreuses-religions/}{Dieu a permis de nombreuses religions}} Gustav Thiels(1966).Ils font peut être partie du plan de Dieu.

  
  \subsection{La remise en cause de l'attitude chrétienne vis-à-vis du
  judaïsme}
  

  
  
  
    
    \paragraph{Le regard classique des chrétiens sur le Juifs}
    On a pensé les juifs pendant des siècles selon trois idées : 
    \begin{itemize}
        \item Infidélité (\textit{perfides Juifs})
        \item Culpabilité
        \item la substitution (le peuple d'israel est substituée par l'Eglise).
    \end{itemize}
  Marie-Laure Durand s'est pensé comme le nouveau Peuple de Dieu.
  \begin{quote}
      Se définir en \textit{substituant}, c'est se définir par rapport à l'autre, c'est s'affirmer en le niant la relation incontournable qui lient l'Eglise au Peuple juif. 
  \end{quote}
    Les Juifs sont toujours présents dans notre ADN. On va utiliser la Loi, ... juive.
    
    \paragraph{Chercher à les convertir} Les rapports avec les juifs n'étaient pas bienveillants.  Asymétrie dans la relation. 
    Dujardin : pas un dialogue, là où un des deux interlocuteurs usent sa position dominante pour imposer sa vue.
    
    \paragraph{Aux origines du dialogue : le choc de la Shoah}
    Va réveiller les consciences. Après Vatican II, 
    
    \subparagraph{Des précurseurs} Bcp d'hommes et femmes ont aidé les juifs, par charité évangélique. Mais on va les interroger avec le temps, non seulement les aider car ils sont dans le besoin mais le caractère profond du génocide va questionner de façon indirecte sur le rôle de l'Eglise.
    
    \subparagraph{Jules Isaac : 1948, Jésus et Israël} Invite les chrétiens à s'interroger sur leur responsabilité : \textit{l'enseignement du mépris}. 
    
    \subparagraph{Déclaration de Seelisberg} autour de J. Isaac \sn{\href{https://fr.wikipedia.org/wiki/Conf\%C3\%A9rence_de_Seelisberg}{Conférence de Seelisberg}}
    
    \subparagraph{Oremus per Perfidus iudeis} Jean XXIII (1957) supprime le terme de Perfidus de la prière du vendredi saint. 
    
    \subparagraph{Résistance chrétienne au Nazisme}
    \textit{de Lubac}\sn{1942, oeuvres complètes, Résistance chrétienne au nazisme} s'interroge sur l'antisémitisme nazi qui détruit aussi le monothéisme chrétien.  Comment haïr la promesse sans haïr l'accomplissement ?  
  
  Grâce à ces précurseurs, un des textes les plus originaux du Concile va naître.
  

  
  \subsection{Le concile Vatican II et \emph{Nostra Aetate} 4}
  

  
  \def\labelenumii{\arabic{enumii}.}
  
    
    \paragraph{La genèse de cette déclaration}
    
  
    
    \paragraph{Les obstacles à la déclaration sur les juifs}
    
  
    
    \paragraph{Commentaire de} Nostra Aetate \emph{4}
    
  
 ~
  \hypertarget{la-ruxe9ception-de-na-4-et-ses-consuxe9quences}{%
  \section{La réception de NA 4 et ses
  conséquences}\label{la-ruxe9ception-de-na-4-et-ses-consuxe9quences}}

  
  \def\labelenumii{\arabic{enumii}.}
  
    
    \paragraph{Fécondité pratique}
    
  
    
    \paragraph{Les interprétations et les développements de NA}
    
  
 ~
  \hypertarget{luxe9tat-de-la-question-des-relations-entre-luxe9glise-catholique-et-le-judauxefsme-les-dons-et-lappel-de-dieu-sont-irruxe9vocables}{%
  \section{L'état de la question des relations entre l'Église catholique
  et le judaïsme : « Les dons et l'appel de Dieu sont irrévocables
  »}\label{luxe9tat-de-la-question-des-relations-entre-luxe9glise-catholique-et-le-judauxefsme-les-dons-et-lappel-de-dieu-sont-irruxe9vocables}}

  
 
  
    
    \paragraph{Statut du document}
    
  
    
    \paragraph{Une relecture de l'histoire de la relation entre juifs et
    chrétiens}
    

    
    \def\labelenumiii{\alph{enumiii}.}
    
      
      Le judaïsme n'est pas une autre religion
      
    
      
      La « judaïté » de Jésus et sa « nouveauté »
      
    
      
      Le christianisme est né au sein du judaïsme
      
    
      
      Vers la séparation
      
    
      
      La théorie de la substitution ou du remplacement
      
    
  
    
    \paragraph{Les questions théologiques}
    

    
    \def\labelenumiii{\alph{enumiii}.}
    
      
      Quelle est valeur théologique de la révélation dont témoigne le
      judaïsme ?
      
    
      
      Quel est le rapport entre Ancien et Nouveau Testament ?
      
    
      
      Quel est le rapport entre ancienne et nouvelle Alliance ?
      
    
      
      Comment les juifs sont-ils sauvés alors qu'ils ne confessent pas
      le Christ ?
 
 
 \chapter{Nostra Aetate  et  l’approche  dialogale} 
\section{Bibliographie}

AVELINE,  J.-M.,  «  Evolution  des  problématiques  en  théologie  des  religions  »,  RSR  94  (2006) 499-522. 

COMEAU, G.,  Le  dialogue  interreligieux, Namur 2008. COMMISSION  THEOLOGIQUE  INTERNATIONALE,  Le  christianisme  et  les  religions,  Rome 1997. 

CONSEIL  PONTIFICAL  POUR  LE  DIALOGUE  INTERRELIGIEUX  ET  CONGRÉGATION  POUR L’ÉVANGELISATION  DES  PEUPLES,  Dialogue  et  annonce.  Réflexions  et  orientations concernant  le  dialogue  interreligieux  et  l’annonce  de  l’Evangile, Rome  1991. 

COUREAU, T-M.,  Le  salut  comme  dia-logue. De  saint  Paul  VI à François, Paris  2018. 

DUPUIS, J.,  Vers  une  théologie  chrétienne  du pluralisme  religieux, Paris  1997. 

GEFFRE, C.,  De  Babel  à Pentecôte. Essais  de  théologie  interreligieuse, Paris  2006. 

GEFFRE,  C.,  «  Où  en  est  la  théologie  des  religions  vingt  ans  après  Assise  ?  »  dans  BOUSQUET, F.  et  LA  HOUGUE  (de),  H.  (éd),  Le  dialogue  interreligieux.  Le  christianisme  face  aux autres  traditions, Paris  2009, 173-200. 

GEFFRE, C.,  Le  christianisme  comme  religion de  l’Evangile, Paris  2012. JEAN-PAUL  II,  Redemptoris  Missio,  Rome  1991. PAUL  VI,  Ecclesiam  suam, Rome  1964. 

SCHEUER,  J.,  «  A  50  ans  de  Nostra  Aetate.  Dialogue  interreligieux  et  théologie  des religions  »,  Revue  théologie  de  Louvain  46 (2015)153-177. 
%--------------------------------------------------------------------    
\section{Nostra Aetate  :  un  regard  positif sur  les  autres  religions}
    
 \subsection{Commentaire  du Nostra Aetate }  
 
 a)  Le  plan  de  la  déclaration b)  Un parallèle  avec  Rm  1 c)  De  Lumen Gentium  (1964) à  Nostra aetate  (1965) d)  Le  «  commun  »  entre  les  religions e)  Invitation au dialogue
 
 
 \subsection{Pourquoi  la  relation  avec  le  judaïsme  fonde  plus  largement  la  théologie  des religions  ?}  
  \subsection{Pluralisme  religieux  et  regard positif  de  l’Église}  
  
    
 %--------------------------------------------------------------------    
  
\section{L’Église  s’ouvre  au  dialogue  :  Ecclesiam  suam}
 
 
\subsection{Une  culture  autoritaire}
 
\subsection{ Des  figures  prophétiques}

 
\subsection{Le  modèle  du dialogue  œcuménique}

\subsection{Vatican II et  Ecclesiam  suam}
  a)  Vatican II b)  Ecclesiam  suam  (1964) c)  Réception critique  d’Ecclesiam  suam
  
      
 
 %--------------------------------------------------------------------    
  
\section{Réflexions  sur  la  dimension  théologique  du  dialogue}
  %--------------------------------------------------------------------    
  
\section{La remise  en  cause  de  l’Occident et  de  la modernité }
 
    
  
\section{Texte}

Texte  1

\begin{quote}
    «  Quant  à  ceux  qui  n’ont  pas  encore  reçu  l’Evangile,  sous  des  formes  diverses,  eux  aussi  sont ordonnés  au  peuple  de  Dieu.  Et,  en  premier  lieu,  ce  peuple  qui  reçut  les  alliances  et  les promesses,  et  dont  le  Christ  est  issu  selon  la  chair  (cf.  Rm  9,4-5),  peuple  très  aimé  du  point  de vue  de  l’élection,  à  cause  des  pères,  car  Dieu  ne  regrette  rien  de  ses  dons  ni  de  son  appel  (cf. Rm 11,28-29)  »  (LG  16). 
\end{quote}


Texte  2 
\begin{quote}
    «  Pour  que  les  juifs  et  les  autres  infidèles  se  convertissent  à  la  foi  orthodoxe  et  que  ceux  qui s’y  seront  convertis  persistent  en  elle  avec  constance,  décidant  d’y  pourvoir  par  ces  salutaires dispositions,  statue  en  premier  lieu  que  tous  les  diocésains  délèguent  plusieurs  fois  par  an  des personnes  bien  instruites  dans  les  divines  écritures  aux  endroits  où  vivent  des  juifs  ou  d’autres infidèles,  pour  y  prêcher  et  y  expliquer  la  vérité  de  la  foi  catholique  de  telle  manière  que  ces infidèles  qui  les  entendent  puissent  reconnaître  leurs  erreurs.  Qu’ils  obligent  les  infidèles  des deux  sexes  ayant  l’âge  de  raison  à  assister  à  cette  prédication  sous  peine  d’interdiction  du commerce  pour  eux  parmi  les  fidèles  ainsi  que  d’autres  punitions  appropriées  (…)  Pour  que soit  évitée  une  excessive  fréquentation  avec  [les  juifs],  qu’ils  soient  contraints  d’habiter  dans certains  quartiers  des  villes  et  des  bourgs,  séparés  de  la  cohabitation  avec  les  chrétiens,  et aussi  éloignés  des  églises.  Et  que  le  dimanche  et  les  autres  jours  de  fêtes  solennelles  ils n’aient  pas  la  hardiesse  de  tenir  boutique  ouverte  ou  de  travailler  en  public  »  (Décret  sur  les juifs, Concile  de  Bâle, session XIX). 
\end{quote}

Texte  3 
\begin{quote}
    L’Église  du  Christ,  en  effet,  reconnaît  que  les  prémices  de  sa  foi  et  de  son  élection  se trouvent,  selon  le  mystère  divin  du  salut,  chez  les  patriarches,  Moïse  et  les  prophètes.  Elle confesse  que  tous  les  fidèles  du  Christ,  fils  d’Abraham  selon  la  foi  (Ga  3,  7),  sont  inclus  dans la  vocation  de  ce  patriarche,  et  que  le  salut  de  l’Église  est  mystérieusement  préfiguré  dans  la sortie  du  peuple  élu  hors  de  la  terre  de  servitude.  C’est  pourquoi  l’Église  ne  peut  oublier qu’elle  a  reçu  la  révélation  de  l’Ancien  Testament  par  ce  peuple  avec  lequel  Dieu,  dans  sa miséricorde  indicible,  a  daigné  conclure  l’antique  Alliance,  et  qu’elle  se  nourrit  de  la  racine de  l’olivier  franc  sur  lequel  ont  été  greffés  les  rameaux  de  l’olivier  sauvage  que  sont  les Gentils  (Rm  11,  17-24).  L’Église  croit,  en  effet,  que  le  Christ,  notre  paix,  a  réconcilié  les  Juifs et  les  Gentils  par sa  croix et  en lui-même, des  deux, a  fait  un seul  (Ep  2, 14-16). (NA  4) L’Église  a  toujours  devant  les  yeux  les  paroles  de  l’apôtre  Paul  sur  ceux  de  sa  race  «  à  qui appartiennent  l’adoption  filiale,  la  gloire,  les  alliances,  la  législation,  le  culte,  les  promesses  et les  patriarches,  et  de  qui  est  né,  selon  la  chair,  le  Christ  »  (Rm  9,  4-5),  le  Fils  de  la  Vierge Marie.  Elle  rappelle  aussi  que  les  Apôtres,  fondements  et  colonnes  de  l’Église,  sont  nés  du peuple  juif,  ainsi  qu’un  grand  nombre  des  premiers  disciples  qui  annoncèrent  au  monde l’Évangile  du Christ  (NA  4). 

\end{quote}

Texte  4 
\begin{quote}
    «  Ce  qui  a  été  commis  durant  sa  Passion  ne  peut  être  imputé  ni  indistinctement  à  tous  les  Juifs vivant  alors,  ni  aux  Juifs  de  notre  temps.  S’il  est  vrai  que  l’Église  est  le  nouveau  peuple  de Dieu,  les  Juifs  ne  doivent  pas,  pour  autant,  être  présentés  comme  réprouvés  par  Dieu  ni maudits, comme  si  cela  découlait  de  la  Sainte  Ecriture  »  (NA  4). 

\end{quote}


Texte  6 \begin{quote}
    «  C’est  pourquoi  le  dialogue  juif-chrétien  ne  peut  être  qualifié  qu’avec  beaucoup  de  réserves de  «  dialogue  interreligieux  »  au  sens  propre  ;  il  faudrait  parler  plutôt  d’un  dialogue  «  intrareligieux  »  ou  «  intra-familial  »  sui  generis.  Dans  son  discours  du  13  avril  1986  à  la synagogue  de  Rome,  Saint  Jean-Paul  II  a  décrit  cette  situation  en  ces  termes  :  Vous  êtes  nos frères  préférés, et  d’une  certaine  manière, on pourrait  dire  nos  frères  aînés  »  (Les  dons, 20). 
\end{quote} 

Texte  7 
\begin{quote}
    «  Parmi  les  Pères  de  l’Église,  la  théorie  dite  du  remplacement  ou  supersessionisme  gagna progressivement  du  terrain  jusqu’à  représenter  au  Moyen-Âge  le  fondement  théologique courant  du  rapport  entre  christianisme  et  judaïsme  :  les  promesses  et  les  engagements  de  Dieu ne  s’appliquaient  plus  à  Israël  qui  n’avait  pas  reconnu  en  Jésus  le  Messie  et  le  Fils  de  Dieu, mais  avaient  été  reportés  sur  l’Église  de  Jésus  Christ,  devenue  désormais  le  véritable  « Nouvel  Israël  », le  nouveau peuple  élu de  Dieu.  »  (Les  dons,  17).   
\end{quote}



Texte  8 \begin{quote}
    «  Dieu  s’étant  révélé  à  travers  sa  Parole,  il  peut  être  compris  par  l’humanité  dans  les  situations historiques  concrètes.  Cette  parole  invite  tous  les  hommes  à  répondre.  Si  leur  réponse  est  en accord  avec  la  parole  de  Dieu,  ils  ont  une  relation  juste  avec  lui.  Pour  les  juifs,  cette  parole peut  être  apprise  grâce  à  la  Torah  et  aux  traditions  qui  en  découlent.  La  Torah  donne  des instructions  pour  une  vie  réussie  dans  une  relation  juste  avec  Dieu.  Celui  qui  observe  la  Torah a  la  plénitude  de  vie  (cf.  Pirqe  Avot  II,  7).  Et  surtout,  en  observant  la  Torah,  les  juifs  prennent part  à  la  communion  avec  Dieu.  À  ce  propos,  le  Pape  François  a  dit  :  «  Les  confessions chrétiennes  trouvent  leur  unité  dans  le  Christ  ;  le  judaïsme  trouve  son  unité  dans  la  Torah.  Les chrétiens  croient  que  Jésus  Christ  est  la  Parole  de  Dieu  qui  s’est  faite  chair  dans  le  monde  ; pour  les  juifs,  la  Parole  de  Dieu  est  surtout  présente  dans  la  Torah.  Ces  deux  traditions  de  foi ont  pour  fondement  le  Dieu  unique,  le  Dieu  de  l’Alliance,  qui  se  révèle  aux  hommes  à  travers sa  Parole.  Dans  la  recherche  d’une  juste  attitude  envers  Dieu,  les  chrétiens  s’adressent  au Christ  comme  source  de  vie  nouvelle,  les  juifs  à  l’enseignement  de  la  Torah  »  (Discours  aux participants  au  Congrès  international  du  Conseil  international  des  chrétiens  et  des  juifs,  30 juin 2015).  »  (Les  dons, 24). 
\end{quote} 

Texte  9 \begin{quote}
    «  Le  judaïsme  et  la  foi  chrétienne,  telle  qu’elle  est  exposée  dans  le  Nouveau  Testament,  sont deux  modalités  par  lesquelles  le  peuple  de  Dieu  fait  siennes  les  Écritures  sacrées  d’Israël.  Le Livre  que  les  chrétiens  appellent  Ancien  Testament  se  prête  donc  à  l’une  et  à  l’autre  de  ces modalités.  Ainsi,  toute  réponse  à  la  parole  salvifique  de  Dieu,  qui  serait  en  accord  avec  l’une ou  l’autre  de  ces  traditions,  peut  ouvrir  un  accès  à  Dieu,  même  s’il  dépend  de  son  conseil  de salut  de  déterminer  de  quelle  manière  il  entend  sauver  les  hommes  en  chaque  circonstance. Que  sa  volonté  de  salut  est  universelle  est  confirmé  par  les  Écritures  (cf.  par  ex.  Gn  12,  1-3  ; 2 ISTR  2021-2022  Théologie  chrétienne  des  religions  C-1 Is  2,  2-5  ;  1  Tim  2,  4).  Il  faut  donc  en  conclure  qu’il  n’y  a  pas  deux  voies  vers  le  salut,  selon l’expression  :  «  Les  juifs  suivent  la  Torah,  les  chrétiens  suivent  le  Christ  ».  La  foi  chrétienne proclame  que  l’oeuvre  de  salut  du  Christ  est  universelle  et  s’étend  à  tous  les  hommes.  La parole  de  Dieu  est  une  réalité  une  et  indivisible,  qui  prend  une  forme  concrète  dans  chaque contexte  historique  particulier. (Les  dons, 25). 
\end{quote} 

Texte  10 \begin{quote}
    «    On  peut  dire  que  Jésus  Christ  porte  en  lui  la  racine  vivante  de  cet  «  olivier  franc  »  mais aussi,  en  un  sens  encore  plus  profond,  que  toute  la  promesse  a  sa  racine  en  lui  (cf.  Jn  8,  58). Cette  image  représente  pour  Paul  la  clé  d’interprétation  décisive  du  rapport  entre  judaïsme  et christianisme  à  la  lumière  de  la  foi.  À  l’aide  de  cette  image,  Paul  entend  exprimer  la  dualité de  l’unité  et  de  la  divergence  entre  Israël  et  l’Église.  Car  cette  image  montre  d’une  part  que les  rameaux  sauvages  de  l’olivier  ne  sont  pas  nés  de  la  plante  sur  laquelle  ils  ont  été  greffés,  et que  leur  situation  nouvelle  représente  une  nouvelle  réalité  et  une  nouvelle  dimension  de l’œuvre  salvifique  de  Dieu,  de  telle  sorte  que  l’Église  chrétienne  ne  peut  pas  être  considérée simplement  comme  une  branche  ou  un  fruit  d’Israël  (cf.  Mt  8,  10-13).  Et  d’autre  part,  elle montre  aussi  que  l’Église  tire  sa  substance  et  sa  force  de  la  racine  d’Israël  et  que  les  rameaux greffés  se  flétriraient  et  risqueraient  de  se  dessécher  s’ils  étaient  séparés  de  la  racine  d’Israël (cf.  Ecclesia in Medio Oriente, n. 21).  »  (Les  dons, 34).   
\end{quote} 

Texte  11 \begin{quote}
    «  Un  autre  point  central  doit  continuer  à  être  pour  les  catholiques  la  question  théologique hautement  complexe  de  savoir  comment  concilier  de  façon  cohérente  leur  croyance  dans  la mission  salvifique  universelle  de  Jésus  Christ  avec  l’article  de  foi  selon  lequel  Dieu  n’a jamais  révoqué  son  alliance  avec  Israël.  Pour  l’Église,  le  Christ  est  le  Rédempteur  de  tous.  En conséquence,  il  ne  peut  y  avoir  deux  voies  menant  au  salut  puisque  le  Christ  est  venu  sauver les  gentils  mais  également  les  juifs.  Nous  sommes  confrontés  ici  au  mystère  de  l’oeuvre  de Dieu  :  il  ne  s’agit  pas  de  déployer  des  efforts  missionnaires  pour  convertir  les  juifs,  mais plutôt  d’attendre  l’heure  voulue  par  le  Seigneur  où  nous  serons  tous  unis  et  où  «  tous  les peuples  [l’]invoqueront  […]  d’une  seule  voix  et  le  serviront  sous  un  même  joug  »  (Nostra Ætate, n. 4).  »  (Les  dons, 37). 
\end{quote}   

Texte  12 \begin{quote}
     «  L’Église  a  été  amenée  à  considérer  l’évangélisation  des  juifs,  qui  croient  dans  le  Dieu unique,  d’une  manière  différente  de  celle  auprès  des  peuples  ayant  une  autre  religion  et  une autre  vision  du  monde.  En  pratique,  cela  signifie  que  l’Église  catholique  ne  conduit  et  ne promeut  aucune  action  missionnaire  institutionnelle  spécifique  en  direction  des  juifs.  Mais alors  que  l’Église  rejette  par  principe  toute  mission  institutionnelle  auprès  des  juifs,  les chrétiens  sont  néanmoins  appelés  à  rendre  témoignage  de  leur  foi  en  Jésus  Christ  devant  les juifs,  avec  humilité  et  délicatesse,  en  reconnaissant  que  les  juifs  sont  dépositaires  de  la  Parole de  Dieu  et  en  gardant  toujours  présente  à  l’esprit  l’immense  tragédie  de  la  Shoah.  »  (Les  dons, 40). 
\end{quote} 


\section{Textes NE}

Texte  1 «  Ce  que  l’on  peut  connaître  de  Dieu  est  pour  eux  manifeste  :  Dieu  le  leur  a  manifesté.  En effet,  depuis  la  création  du  monde,  ses  perfections  invisibles,  éternelle  puissance  et  divinité, sont  visibles  dans  ses  œuvres  pour  l’intelligence  ;  ils  sont  donc  inexcusables,  puisque, connaissant  Dieu,  ils  ne  lui  ont  rendu  ni  la  gloire  ni  l’action  de  grâce  qui  reviennent  à  Dieu  » (Rm  1, 19-21). 


Texte  2 «  Depuis  les  temps  les  plus  reculés  jusqu’à  aujourd’hui,  on  trouve  dans  les  différents  peuples une  certaine  sensibilité  à  cette  force  cachée  qui  est  présente  au  cours  des  choses  et  aux événements  de  la  vie  humaine,  parfois  même  une  reconnaissance  de  la  Divinité  suprême,  ou encore  du  Père.  Cette  sensibilité  et  cette  connaissance  pénètrent  leur  vie  d’un  profond  sens religieux.  Quant  aux  religions  liées  au  progrès  de  la  culture,  elles  s’efforcent  de  répondre  aux mêmes  questions  par des  notions  plus  affinées  et  par un langage  plus  élaboré  »  (NA  2). 


Texte  3 «  L’Église  catholique  ne  rejette  rien  de  ce  qui  est  vrai  et  saint  ces  religions.  Elle  considère avec  un  respect  sincère  ces  manières  d’agir  et  de  vivre,  ces  règles  et  ces  doctrines  qui, quoiqu’elles  diffèrent  en  beaucoup  de  points  de  ce  qu’elle-même  tient  et  propose,  cependant apportent  souvent  un rayon de  la  vérité  qui  illumine  tous  les  hommes  »  (NA  2). 


Texte  4 «  Dans  l’exhortation  apostolique  Evangelii  nuntiandi,  le  terme  ‘évangélisation’  est  utilisé  de différentes  manières.  Il  signifie  «porter  la  Bonne  Nouvelle  à  toute  l’humanité  et,  par  son impact,  transformer  du  dedans,  rendre  neuve  l’humanité  elle-même»  (Evangelii  nuntiandi, 18).  Ainsi,  par  l’évangélisation,  l’Eglise  ‘cherche  à  convertir,  par  la  seule  énergie  divine  du Message  qu’elle  annonce,  les  consciences  personnelles  et  collectives,  les  activités  dans lesquelles  les  hommes  sont  engagés,  leurs  manières  de  vivre,  et  les  milieux  concrets  dans lesquels  ils  vivent’  (ibid.).  […]  Il  y  a  donc  un  sens  large  du  concept  d’évangélisation. Cependant,  dans  le  même  document,  évangélisation  est  aussi  pris  dans  un  sens  plus  spécifique comme  ‘l’annonce  claire  et  sans  ambiguïté  du  Seigneur  Jésus’  (Evangelii  nuntiandi,  22). L’exhortation  dit  que  ‘cette  annonce,  kerygma,  prédication  et  catéchèse,  occupe  une  place tellement  importante  dans  l’évangélisation  qu’elle  en  est  souvent  synonyme;  cependant,  ce n’est  qu’un aspect  de  l’évangélisation’  (ibid.).  »  (Dialogue  et  annonce  n.8) 

Texte  5 «  À  l’origine,  Saint  Jean  XXIII  avait  proposé  que  le  Concile  promulgue  un  Tractatus  de Iudaeis,  mais  à  la  fin  il  fut  décidé  que  Nostra  Ætate  prendrait  en  considération  toutes  les grandes  religions  mondiales.  Cependant  le  quatrième  article  de  cette  Déclaration  conciliaire, qui  préfigure  un  nouveau  rapport  théologique  avec  le  judaïsme,  représente  en  quelque  sorte  le coeur  de  ce  document  où  une  place  est  faite  également  aux  rapports  de  l’Église  catholique avec  les  autres  religions.  En  ce  sens,  les  rapports  de  l’Église  catholique  avec  le  judaïsme peuvent  être  considérés  comme  le  catalyseur  qui  a  poussé  le  Concile  à  déterminer  ses  rapports avec  les  autres  grandes  religions  mondiales.  »  (Les  dons  sont  irrévocables, 19).   


Texte  6 «  Aujourd’hui,  dans  notre  monde  caractérisé  par  la  rapidité  des  communications,  la  mobilité des  peuples,  l’interdépendance,  il  existe  une  nouvelle  prise  de  conscience  du  pluralisme religieux.  Les  religions  ne  se  contentent  pas  tout  simplement  d’exister  ou  même  de  survivre. En  certains  cas,  elles  manifestent  un  réel  renouveau.  Elles  continuent  à  inspirer  et  à  influencer la  vie  de  millions  de  leurs  membres.  Dans  le  contexte  actuel  du  pluralisme  religieux,  on  ne peut  donc  pas  oublier  le  rôle  important  que  jouent  les  traditions  religieuses.  »  (Dialogue  et annonce  n.4) 


Texte  7 «  Pour  un  dialogue  chrétien  vers  l’extérieur,  c’est-à-dire  qui  sort  de  la  sphère  de  la  foi  pour aller  vers  le  monde  profane,  on  manquait,  sur  le  plan  du  magistère,  d’expérience  et  de modèles.  Le  type  même  du  langage  magistériel  est,  depuis  le  début,  d’un  côté  le  symbole, c’est-à-dire  la  confession  de  foi  reçue  par  tradition  et  qui  lie  celui  qui  l’accepte,  et  de  l’autre côté  l’anathème  qui  exclut.  Dans  les  deux  cas,  il  s’agit  d’une  forme  d’expression  qui  n’a  de sens  qu’à  l’intérieur  de  la  sphère  de  la  foi,  parce  qu’elle  repose  sur  la  prise  en  compte  de l’autorité  de  la  foi.  »  (J. RATZINGER,  Mon Concile, p. 225-226). 



Texte  8   «  Cette  distinction  d'avec  le  monde  n'est  pas  séparation.  Bien  plus,  elle  n'est  pas  indifférence, ni  crainte,  ni  mépris.  Quand  l'Eglise  se  distingue  de  l'humanité,  elle  ne  s'oppose  pas  à  elle  ;  au contraire  elle  s'y  unit.  Il  en  est  de  l'Eglise  comme  d'un  médecin  :  connaissant  les  pièges  d'une maladie  contagieuse,  le  médecin  cherche  à  se  garder  lui-même  et  les  autres  de  l'infection  ; mais  en  même  temps  il  s'emploie  à  guérir  ceux  qui  en  sont  atteints  ;  de  même  l'Eglise  ne  se réserve  pas  comme  un  privilège  exclusif  la  miséricorde  à  elle  concédée  par  la  bonté  divine  ; elle  ne  tire  pas  de  son  propre  bonheur  une  raison  de  se  désintéresser  de  qui  ne  l'a  pas  atteint, mais  elle  trouve  dans  son  propre  salut  un  motif  d'intérêt  et  d'amour  envers  tous  ceux  qui  lui sont  proches  et  pour  tous  ceux  que,  dans  son  effort  de  communion  universelle,  il  lui  est possible  d'approcher.  »  (Ecclesiam  Suam  § 65). 

Texte  9 «  La  Révélation  qui  est  la  relation  surnaturelle  que  Dieu  lui-même  a  pris  l’initiative d’instaurer  avec  l’humanité,  peut  être  représentée  comme  un  dialogue  dans  lequel  le  Verbe  de Dieu  s’exprime  par  l’Incarnation,  et  ensuite  par  l’Evangile.  (…)  L’histoire  du  salut  raconte précisément  ce  dialogue  long  et  divers  qui  part  de  Dieu  et  noue  avec  l’homme  une conversation  variée  et  étonnante.  C’est  dans  cette  conversation  du  Christ  avec  les  hommes  (cf. Ba  3,38)  que  Dieu  laisse  comprendre  quelque  chose  de  lui-même,  le  mystère  de  sa  vie  (…)  » (Ecclesiam  Suam  § 72). 


Texte  10 «  Cette  forme  de  rapport  indique  une  volonté  de  courtoisie,  d'estime,  de  sympathie,  de  bonté de  la  part  de  celui  qui  l'entreprend  ;  elle  exclut  la  condamnation  a  priori,  la  polémique offensante  et  tournée  en  habitude,  l'inutilité  de  vaines  conversations.  Si  elle  ne  vise  pas  à obtenir  immédiatement  la  conversion  de  l'interlocuteur  parce  qu'elle  respecte  sa  dignité  et  sa liberté,  elle  vise  cependant  à  procurer  son  avantage  et  voudrait  le  disposer  à  une  communion plus  pleine  de  sentiments  et  de  convictions.  »  (Ecclesiam  Suam  § 81). 


Texte  11 2   ISTR  2021-2022  Théologie  chrétienne  des  religions  C  2 «  Puis,  autour  de  nous  nous  voyons  se  dessiner  un  autre  cercle  immense,  lui  aussi,  mais  moins éloigné  de  nous  :  c'est  avant  tout  celui  des  hommes  qui  adorent  le  Dieu  unique  et  souverain, celui  que  nous  adorons  nous  aussi  ;  Nous  faisons  allusion  aux  fils,  dignes  de  Notre  affectueux respect,  du  peuple  hébreu,  fidèles  à  la  religion  que  Nous  nommons  de  l'Ancien  Testament  ; puis  aux  adorateurs  de  Dieu  selon  la  conception  de  la  religion  monothéiste  -  musulmane  en particulier  -  qui  méritent  admiration  pour  ce  qu'il  y  a  de  vrai  et  de  bon  dans  leur  culte  de  Dieu ; et  puis  encore  aux  fidèles  des  grandes  religions  afro-asiatiques.  Nous  ne  pouvons évidemment  partager  ces  différentes  expressions  religieuses,  ni  ne  pouvons  demeurer indifférent,  comme  si  elles  s'équivalaient  toutes,  chacune  à  sa  manière,  et  comme  si  elles dispensaient  leurs  fidèles  de  chercher  si  Dieu  lui-même  n'a  pas  révélé  la  forme  exempte d'erreur,  parfaite  et  définitive,  sous  laquelle  il  veut  être  connu,  aimé  et  servi  ;  au  contraire,  par devoir  de  loyauté,  nous  devons  manifester  notre  conviction  que  la  vraie  religion  est  unique  et que  c'est  la  religion  chrétienne,  et  nourrir  l'espoir  de  la  voir  reconnue  comme  telle  par  tous ceux qui  cherchent  et  adorent  Dieu.  »  (Ecclesiam  Suam  § 111). 


Texte  12 «  Nous  ne  voulons  pas  refuser  de  reconnaître  avec  respect  les  valeurs  spirituelles  et  morales des  différentes  confessions  religieuses  non  chrétiennes  ;  nous  voulons  avec  elles  promouvoir et  défendre  les  idéaux  que  nous  pouvons  avoir  en  commun  dans  le  domaine  de  la  liberté religieuse,  de  la  fraternité  humaine,  de  la  sainte  culture,  de  la  bienfaisance  sociale  et  de  l'ordre civil.  Au  sujet  de  ces  idéaux  communs,  un  dialogue  de  notre  part  est  possible  et  nous  ne manquerons  pas  de  l'offrir  là  où,  dans  un  respect  réciproque  et  loyal,  il  sera  accepté  avec bienveillance.  »  (Ecclesiam  Suam  § 112). 


Texte  13 «  Dans  un  contexte  de  pluralisme  religieux,  le  terme  de  dialogue  signifie  «l’ensemble  des rapports  interreligieux,  positifs  et  constructifs,  avec  des  personnes  et  des  communautés  de diverses  croyances,  afin  d’apprendre  à  se  connaître  et  à  s’enrichir  les  uns  les  autres» (Dialogue  et  mission,  3),  tout  en  obéissant  à  la  vérité  et  en  respectant  la  liberté  de  chacun.  Il implique  à  la  fois  le  témoignage  et  l’approfondissement  des  convictions  religieuses respectives.  »  (Dialogue  et  annonce, n. 9). 


Texte  14 Par  le  dialogue,  les  chrétiens  et  les  autres  sont  invités  à  approfondir  les  dimensions  religieuses de  leur  engagement  et  à  répondre,  avec  une  sincérité  croissante,  à  l’appel  personnel  de  Dieu  et au  don  gratuit  qu’il  fait  de  lui-même,  don  qui  passe  toujours,  comme  notre  foi  nous  le  dit,  par la  médiation  de  Jésus  Christ  et  l’œuvre  de  son  Esprit.  »  (Dialogue  et  annonce  n.  40).  «  Etant donné  cet  objectif,  à  savoir  une  conversion  plus  profonde  de  tous  à  Dieu,  le  dialogue interreligieux possède  sa  propre  valeur.  »  (Dialogue  et  annonce,  n. 41). 