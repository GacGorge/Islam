\chapter{Validation}

Deux possibilités : 


1. Une brève dissertation (environ 6 pages) A partir d’un ou deux textes sur la question, essayez de répondre à la question suivante : « Quel l’enjeu du thème du Royaume de Dieu dans la théologie des religions ? ». Voici quelques références (mais il est possible d’en chercher d’autres) : 

VILLEMIN, L. et CHEVALLIER, G. , « La distinction « incorporé à / ordonné à » dans Lumen Gentium : quelles conséquences pour la compréhension du rapport Eglise / Royaume ? », in Christoph THEOBALD (dir.), Pourquoi l’Eglise ? La dimension ecclésiale de la foi dans l’horizon du salut, Paris : Bayard, 2014, pp. 165-196. 


DUPUIS, J., « Le règne de Dieu, les religions et l’Église » dans Vers une théologie chrétienne du pluralisme religieux, Paris, Cerf, 1997, p. 501-541. 

PANNENBERG, W., « Le Royaume de Dieu et l’Église » dans Théologie et royaume de Dieu, Genève, Labor et fides, 2021, p. 73-144. 

JEAN-PAUL II, Lettre encyclique Redemptoris missio, 1990. 

CDF, Dominus Iesus, Rome, 2000. CTI, Le christianisme et les religions, Rome, 1997 



Votre dissertation consistera donc à répondre à la question de manière argumentée et comportera une introduction, un développement (deux à quatre parties) et une conclusion.  


 % ------------------------------------------------------------------------------------------------------------------------------------------------------------------------------------    
\section{Genizah}    
% ------------------------------------------------------------------------------------------------------------------------------------------------------------------------------------
 \begin{quote}
     Contre une conception individualiste de la justice et de l’amour Le royaume de Dieu signifie tout à fait autre chose que le formalisme d’une idée de domination divine. Il est la réalité la plus concrète du droit et de l’amour parmi les êtres humains. Il n’y va pas seulement de la relation privée des individus, par opposition aux institutions publiques dans la vie sociale. Une conception individualiste de la justice et de l’amour devient particulièrement dangereuse quand elle s’unit à un dualisme fondamental entre la religion et la société. Le comportement subjectif et les institutions sociales ne peuvent pas être séparés. Le comportement subjectif se rapporte toujours aux institutions sociales et survient dans la plupart des cas en endossant des rôles sociaux bien précis qui, de leur côté, sont soumis à la structure des aménagements sociaux. Parmi les manquements de nombreuses analyses académiques au sujet des questions éthiques, on trouve souvent la concentration abstraite sur le comportement de l’individu et ses motivations privées, comme si ces éléments n’entretenaient pas un rapport étroit avec les formes données de la vie sociale7. Le royaume de Dieu concerne aussi de manière fondamentale les institutions et les formes de vie sociale. La justice et l’amour n’ont pas seulement affaire aux individus, mais aussi aux structures du vivre-ensemble des êtres humains. Dans cette mesure, le royaume a sans conteste un caractère politique. (pp. 89-90). 
 \end{quote}
 
 \begin{quote}
     Dans la vie individuelle comme dans la vie sociale, l’être humain a besoin d’institutions honnêtes (honest institutions) . Une institution honnête est une institution qui démasque les limites des formes présentes de la vie sociale et politique et renvoie les êtres humains à la réalité ultime englobant leur destination dernière.\cite[pp. 98]{Pannenberg:RoyaumeDieu} 
 \end{quote}
 
 \begin{quote}
     L’Église ne nourrit ici aucune illusion tendant à nier le caractère provisoire des réalisations présentes. Elle offre au contraire la possibilité de vivre déjà maintenant de la certitude du futur de Dieu, sans se laisser tromper par l’expérience du provisoire. En ce sens, l’Église comme communauté religieuse particulière est nécessaire à côté des autres groupements et institutions de la société. L’Église est nécessaire tant que la vie sociale et politique n’incarne pas cet accomplissement parfait de la destination humaine seule capable de réaliser le royaume de Dieu dans l’histoire humaine. On voit clairement que, dans cette perspective, l’Église n’est certes pas éternelle, mais nécessaire pour le temps en deçà de la plénitude du royaume de Dieu. L’Église doit assumer dans la société une fonction vitale, critique et constructive. Sa fonction critique ne se tient pas par hasard au premier rang, d’un point de vue historique. L’Église doit sans cesse attirer l’attention sur les limites de toute société donnée.\cite[pp. 99-100]{Pannenberg:RoyaumeDieu} 
\end{quote}

\paragraph{Habiter le monde}
\begin{quote}
    Le plus grand danger est le retrait de l’Église hors des problèmes de la société. À première vue, un tel retrait sur soi-même peut apparaître comme une forme radicale de diastase et donc d’opposition au monde et de critique de la société existante.\cite[p. 100]{Pannenberg:RoyaumeDieu}
\end{quote}
 
 
 \paragraph{Respecter l'héritage culturel et spirituel}
 \begin{quote}
      Le futur du règne de Dieu, en se tournant vers le monde d’une manière rédemptrice, vient à sa rencontre sur le mode positif de l’approbation. Malgré toute la déchéance de l’être humain, il n’est pas possible, vu le message néotestamentaire de l’amour de Dieu pour le monde, de voir cette relation fondamentale de Dieu autrement que de manière positive. Cela veut dire : pour l’amour du royaume de Dieu, l’Église doit résister à la tentation de mépriser l’héritage social et culturel. D’habitude,\cite[p. 105]{Pannenberg:RoyaumeDieu}. 
 \end{quote}


\begin{quote}
    Dieu. Parce que l’Église transmet à l’individu la participation au salut futur, la vie ecclésiale devrait déjà maintenant faire connaître le signe de l’identité et de l’intégrité de l’être humain anticipées en elle. Une telle plénitude [Ganzsein] de l’être humain – l’équivalent du salut – n’a pas seulement à faire avec le confort de l’individu, mais inclut également les institutions sociales et leur perception appropriée.\cite[pp. 106-107]{Pannenberg:RoyaumeDieu} 
\end{quote}

\begin{quote}
    La force de l’amour n’est ni la propriété ni le privilège de l’Église.\cite[p. 112]{Pannenberg:RoyaumeDieu}
\end{quote}

\begin{quote}
    Partout où la liberté de l’Esprit s’impose – dans les Églises ou en dehors de leurs murs –, la vie conquiert une nouvelle intégrité et brille comme un tout, d’une manière toute neuve. Et malgré toutes les défaillances chrétiennes, l’Église conserve toujours la promesse, pour toute l’humanité, selon laquelle l’Esprit de la vie va se manifester non seulement ici et là, mais qu’il va en définitive se donner à l’humanité de manière durable. \cite[p. 113]{Pannenberg:RoyaumeDieu}
\end{quote}

\paragraph{La guérison par l’Esprit : un cas marginal ?}

\begin{quote}
    Quand on parle de la puissance de guérison de l’Esprit, on ne doit pas se limiter au cercle de vie privé. Les forces d’intégration issues de la foi authentique se laissent aussi observer dans la vie publique et sociale.\cite[p. 116]{Pannenberg:RoyaumeDieu}
\end{quote}



\begin{quote}
    Cette conscience peut faire naître une nouvelle unité des chrétiens, sans uniformité. Cette unité correspond à l’unité du Dieu trinitaire. Elle pourrait devenir dans le monde présent le symbole de la solution proposée au problème le plus angoissant de toutes les sociétés modernes : comment atteindre et garantir l’unité, sans sacrifier la pluralité de la vie ?\cite[p. 144]{Pannenberg:RoyaumeDieu}
\end{quote}