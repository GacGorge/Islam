
\chapter{Le renouvellement du regard de l'Église sur le judaïsme au XXe siècle}
\mn{1 ISTR  2021-2022  Théologie  chrétienne  des  religions  C-1 }

\hypertarget{eluxe9ments-bibliographiques}{%
\section{Eléments bibliographiques}\label{eluxe9ments-bibliographiques}}


ANDREVON, T. et KRISEL (dir.), W., \emph{Réflexion juives sur le
christianisme}, Genève, Labor et fides, 2021.

ARTIGES, D., « Les tentations antivijuives de la pensée chrétienne (I).
La contribution de Friedrich Wilhelm Marquardt », \emph{NRT} 142 (2020)
425-436.

ARTIGES, D., « Les tentations antivijuives (II). L'exclusivisme
institutionnel », \emph{NRT} 142 (2020) 606-622.

BOYARIN, D., \emph{La partition du judaïsme et du christianisme}, Paris,
Cerf, 2012.

COMMISSION BIBLIQUE, \emph{Le Peuple juif et ses saintes Ecritures dans
la Bible chrétienne}, Rome 2001.

COMMISSION POUR LES RELATIONS RELIGIEUSES AVEC LE JUDAÏSME,
\emph{Orientations et}
\emph{suggestions pour l'application de la déclaration conciliaire}
Nostra Aetate \emph{n° 4}, Rome 1\textsuperscript{er} décembre 1974.

COMMISSION POUR LES RELATIONS RELIGIEUSES AVEC LE JUDAÏSME, \emph{Notes
pour une}
\emph{correcte présentation des Juifs et du judaïsme dans la prédication
et la catéchèse de l'Église catholique}, Rome 24 juin 1985.

COMMISSION POUR LES RELATIONS RELIGIEUSES AVEC LE JUDAÏSME, \emph{Nous
nous souvenons : une réflexion sur la Shoah}, Rome 16 mars 1998.

COMMISSION POUR LES RELATIONS RELIGIEUSES AVEC LE JUDAÏSME,
\emph{Quarantième}

\emph{anniversaire de} Nostra Aetate, Conférence du cardinal Jean-Marie
Lustiger Rome, le 27 octobre 2005.

COMMISSION POUR LES RELATIONS RELIGIEUSES AVEC LE JUDAÏSME, \emph{« Les
dons et l'appel}
\emph{de Dieu sont irrévocables » (Rm 11,29). Une réflexion théologique
sur les rapports entre catholiques et juifs à l'occasion du 50e
anniversaire de} Nostra ætate (n. 4), Rome 2015.

CONCILE VATICAN II, \emph{Déclaration sur les religions
non-chrétiennes}. Nostra Aetate, 1965.

DUJARDIN, J., \emph{L'Église
catholique et le peuple juif. Un autre regard}, Paris 2003.

LUBAC (de), H., « Un nouveau `front' religieux. Israël et la foi
chrétienne » dans \emph{Résistance chrétienne au nazisme. Œuvres
complètes XXXIV}, Paris, Cerf, 2006, p. 151-193.

« Christianisme et judaïsme depuis \emph{Nostra Aetate} »,
\emph{Recherches de science religieuse} 103 (2015).

« La christologie après Auschwitz un programme », \emph{Recherches de
science religieuse} 105 (2017), 5-90.

\hypertarget{introduction}{%
\section{Introduction}\label{introduction}}

On est dans une remise en cause de la théologie de l'accomplissement. Nouvelle tournure dans le dernier tiers du XX. Et en particulier du regard sur le judaisme. Pourquoi au moment où elle est largement reçue, cette théologie devient problématique.

\paragraph{Des évènements exterieurs} qui obligent à bouger. 
\begin{itemize}
    \item Les grandes traditions religieuses \textit{perdurent}, à commencer par le judaisme. 
    \item l'écroulement de la domination européenne avec 2 guerres et la Shoah, la décolonisation. 
\end{itemize}

Le constat d'une alterité durable, et donc d'une version universaliste du Christianisme, oblige à une nouvelle attitude, dialogale ?

C. Geffré
\begin{quote}
    On doit être prêt à reconnaître la non catholicité de l'Eglise\sn{elle est née face à l'altérité}.
\end{quote}

\paragraph{L'alterité d'Israël} ne nous permet-elle pas de penser le rapport aux autres religions de façon analogique ? 
\paragraph{Pluralisme religieux de principe} et non de fait car les religions contiennent quelque chose d'irréductible. Il ne s'agit pas d'absorber l'autre mais les reconnaître dans leur différence. 

\paragraph{Plan de Dieu} Le Pape François\sn{\href{https://www.cath.ch/newsf/pape-francois-dieu-a-permis-quil-y-ait-de-nombreuses-religions/}{Dieu a permis de nombreuses religions}} Gustav Thiels(1966).Ils font peut être partie du plan de Dieu.

  
  \subsection{La remise en cause de l'attitude chrétienne vis-à-vis du
  judaïsme}
  

  
  
  
    
    \paragraph{Le regard classique des chrétiens sur le Juifs}
    On a pensé les juifs pendant des siècles selon trois idées : 
    \begin{itemize}
        \item Infidélité (\textit{perfides Juifs})
        \item Culpabilité
        \item la substitution (le peuple d'israel est substituée par l'Eglise).
    \end{itemize}
  Marie-Laure Durand s'est pensé comme le nouveau Peuple de Dieu.
  \begin{quote}
      Se définir en \textit{substituant}, c'est se définir par rapport à l'autre, c'est s'affirmer en le niant la relation incontournable qui lient l'Eglise au Peuple juif. 
  \end{quote}
    Les Juifs sont toujours présents dans notre ADN. On va utiliser la Loi, ... juive.
    
    \paragraph{Chercher à les convertir} Les rapports avec les juifs n'étaient pas bienveillants.  Asymétrie dans la relation. 
    Dujardin : pas un dialogue, là où un des deux interlocuteurs usent sa position dominante pour imposer sa vue.
    
    \paragraph{Aux origines du dialogue : le choc de la Shoah}
    Va réveiller les consciences. Après Vatican II, 
    
    \subparagraph{Des précurseurs} Bcp d'hommes et femmes ont aidé les juifs, par charité évangélique. Mais on va les interroger avec le temps, non seulement les aider car ils sont dans le besoin mais le caractère profond du génocide va questionner de façon indirecte sur le rôle de l'Eglise.
    
    \subparagraph{Jules Isaac : 1948, Jésus et Israël} Invite les chrétiens à s'interroger sur leur responsabilité : \textit{l'enseignement du mépris}. 
    
    \subparagraph{Déclaration de Seelisberg} autour de J. Isaac \sn{\href{https://fr.wikipedia.org/wiki/Conf\%C3\%A9rence_de_Seelisberg}{Conférence de Seelisberg}}
    
    \subparagraph{Oremus per Perfidus iudeis} Jean XXIII (1957) supprime le terme de Perfidus de la prière du vendredi saint. 
    
    \subparagraph{Résistance chrétienne au Nazisme}
    \textit{de Lubac}\sn{1942, oeuvres complètes, Résistance chrétienne au nazisme} s'interroge sur l'antisémitisme nazi qui détruit aussi le monothéisme chrétien.  Comment haïr la promesse sans haïr l'accomplissement ?  
  
  Grâce à ces précurseurs, un des textes les plus originaux du Concile va naître.
  

  
  \subsection{Le concile Vatican II et \emph{Nostra Aetate} 4}
  
\mn{\href{https://fr.wikipedia.org/wiki/Nostra\_\%C3\%86tate}{Wikipedia sur NA}}
  
   Nostra Aetate est un document promulgué par Vatican II.
    
    \paragraph{La genèse de cette déclaration} A l'origine, des commissions préparatoires mais aucune commission n'envisageait d'écrire quelque chose sur les juifs (début 60).
    Or, Jules Isaac va rencontrer le 13 juillet 1960 le Pape Jean XXIII et à la suite de cette rencontre, une commission spécifique sur cette question. Il confie cette commission au \href{https://fr.wikipedia.org/wiki/Augustin\_Bea}{Cardinal Béa} \sn{Bibliste, secrétariat pour l'unité des chrétiens}.
    Le document préparé va être mis de côté suite aux potentielles répercutions pour les populations chrétiennes au Moyen-Orient.
    
    En 1963, on pense qu'on pourrait mettre un chapitre sur les Juifs dans le document sur l'Oecuménisme. Mais ce document va rencontrer l'opposition des patriarches des Eglises Orientales. D'autant plus que la minorité conservatrice va s'opposer à ce type de textes.
    On introduit dans Lumen Gentium un paragraphe positif sur les juifs, mais le Pape souhaite un document séparé. De plus en plus, une opposition théologique.
    
    \begin{quote}
    «  Quant  à  ceux  qui  n’ont  pas  encore  reçu  l’Evangile,  sous  des  formes  diverses,  eux  aussi  sont ordonnés  au  peuple  de  Dieu.  Et,  en  premier  lieu,  ce  peuple  qui  reçut  les  alliances  et  les promesses,  et  dont  le  Christ  est  issu  selon  la  chair  (cf.  Rm  9,4-5),  peuple  très  aimé  du  point  de vue  de  l’élection,  à  cause  des  pères,  car  Dieu  ne  regrette  rien  de  ses  dons  ni  de  son  appel  (cf. Rm 11,28-29)  »  (LG  16). 
    \end{quote}
    
    En 1963, ne pas oublier le contexte de \textit{le Vicaire}, pièce sur le rôle contreversé de Pie XII pendant la seconde guerre mondiale.
    
    En 1965, 7ème version du texte de NA, promulgué le 28 octobre 1965. Le texte va passer parce qu'on retirer du texte la condamnation de l'expression \textit{peuple déicide}, initialement dans le texte. 
    
  
    
    \paragraph{Les obstacles à la déclaration sur les juifs} Pourquoi y-a-t-il eu des réactions théologiques ?
    \begin{itemize}
        \item Aucune tradition ecclésiale ne peut s'appuyer sur cette position. Un rapport à la Tradition assez nouveau. Aucune référence à un concile, un pape, un docteur de l'Eglise. Il y a des réfèrences à \emph{Rm}. Mais c'est comme si depuis le Concile de Jérusalem (Ac 15), l'Eglise n'avait jamais réfléchi à son rapport avec le judaisme.
        
        \item Les condamnations du passé. Par exemple, Latran IV (1215), on est très critique vis à vis des Juifs (\textit{l'usure des juifs}. On était banquier, on n'était pas sauvé). Les juifs doivent se distinguer des chrétiens par les habits. Inaptitude des juifs aux emplois publiques (pas de responsabilité sur les chrétiens), interdictions aux Juifs devenus chrétiens de redevenir Juif.
        De même le concile de Bâle (1434), un texte sur les juifs et les néophytes.
  \begin{quote}
    «  Pour  que  les  juifs  et  les  autres  infidèles \sn{(Décret  sur  les juifs, Concile  de  Bâle, session XIX)}  se  convertissent  à  la  foi  orthodoxe  et  que  ceux  qui s’y  seront  convertis  persistent  en  elle  avec  constance,  décidant  d’y  pourvoir  par  ces  salutaires dispositions,  statue  en  premier  lieu  que  tous  les  diocésains  délèguent  plusieurs  fois  par  an  des personnes  bien  instruites  dans  les  divines  écritures  aux  endroits  où  vivent  des  juifs  ou  d’autres infidèles,  pour  y  prêcher  et  y  expliquer  la  vérité  de  la  foi  catholique  de  telle  manière  que  ces infidèles  qui  les  entendent  puissent  reconnaître  leurs  erreurs.  Qu’ils  obligent  les  infidèles  des deux  sexes  ayant  l’âge  de  raison  à  assister  à  cette  prédication  sous  peine  d’interdiction  du commerce  pour  eux  parmi  les  fidèles  ainsi  que  d’autres  punitions  appropriées  (…)  Pour  que soit  évitée  une  excessive  fréquentation  avec  [les  juifs],  qu’ils  soient  contraints  d’habiter  dans certains  quartiers  des  villes  et  des  bourgs,  séparés  de  la  cohabitation  avec  les  chrétiens,  et aussi  éloignés  des  églises.  Et  que  le  dimanche  et  les  autres  jours  de  fêtes  solennelles  ils n’aient  pas  la  hardiesse  de  tenir  boutique  ouverte  ou  de  travailler  en  public  »  
\end{quote}
    Certes ce n'est pas un texte dogmatique mais de discipline, fortement vexatoire.
        
    \item où placer cette déclaration sur les Juifs ? Dans Lumen Gentium, car a à voir avec l'origine de l'Eglise ? Ou dans le décret sur l'oecuménisme, la séparation du judaisme étant la première "cession" ? ou dans un texte sur les religions ? 
    D'ailleurs, post concile, le lien avec le judaisme a été mis dans le lien avec l'Oecuménisme.
     
    \end{itemize}
  
    
    \paragraph{Commentaire de Nostra Aetate \emph{4}} On part d'une méditation sur l'Eglise pour penser la relation au judaisme. D'une certaine façon, cela relit avec l'Eglise.  
    
  \begin{quote}
    L’Église  du  Christ,  en  effet,  reconnaît  que  les  prémices  de  sa  foi  et  de  son  élection  se trouvent,  selon  le  mystère  divin  du  salut,  chez  les  patriarches,  Moïse  et  les  prophètes.  Elle confesse  que  tous  les  fidèles  du  Christ,  fils  d’Abraham  selon  la  foi  (Ga  3,  7),  sont  inclus  dans la  vocation  de  ce  patriarche,  et  que  le  salut  de  l’Église  est  mystérieusement  préfiguré  dans  la sortie  du  peuple  élu  hors  de  la  terre  de  servitude.  C’est  pourquoi  l’Église  ne  peut  oublier qu’elle  a  reçu  la  révélation  de  l’Ancien  Testament  par  ce  peuple  avec  lequel  Dieu,  dans  sa miséricorde  indicible,  a  daigné  conclure  l’antique  Alliance,  et  qu’elle  se  nourrit  de  la  racine de  l’olivier  franc  sur  lequel  ont  été  greffés  les  rameaux  de  l’olivier  sauvage  que  sont  les Gentils  (Rm  11,  17-24).  L’Église  croit,  en  effet,  que  le  Christ,  notre  paix,  a  réconcilié  les  Juifs et  les  Gentils  par sa  croix et  en lui-même, des  deux, a  fait  un seul  (Ep  2, 14-16). (NA  4) 
    
    L’Église  a  toujours  devant  les  yeux  les  paroles  de  l’apôtre  Paul  sur  ceux  de  sa  race  «  à  qui appartiennent  l’adoption  filiale,  la  gloire,  les  alliances,  la  législation,  le  culte,  les  promesses  et les  patriarches,  et  de  qui  est  né,  selon  la  chair,  le  Christ  »  (Rm  9,  4-5),  le  Fils  de  la  Vierge Marie.  Elle  rappelle  aussi  que  les  Apôtres,  fondements  et  colonnes  de  l’Église,  sont  nés  du peuple  juif,  ainsi  qu’un  grand  nombre  des  premiers  disciples  qui  annoncèrent  au  monde l’Évangile  du Christ  (NA  4). 

\end{quote}


Lien spirituel qui subsiste et pas seulement un lien historique, lien toujours vivant. On l'illustre par des versets de Saint Paul (olivier franc,... ). Des références à l'écriture (Ga, Rm) mais un peu ambivalent (on est toujours dans la théologie de l'Accomplissement). Mais avec l'olivier franc, on a toujours une actualité de cette nourriture de l'Eglise par la Synagogue. De même \textit{à qui appartient} est au présent. On a une alternance entre une approche historique (quand le texte est au passé) et une approche spirituelle qui demeure (au présent).

\subparagraph{Malédiction des Juifs} On dit qu'il y a le nouveau peuple de Dieu mais le peuple n'est pas rejeté et il est toujours \textit{peuple de Dieu} pour Dieu. briser un antisémitisme qui s'appuyerait sur les Ecritures. L'Eglise ne tranche pas sur le statut du Peuple Juif aujourd'hui

\begin{quote}
    «  Ce  qui  a  été  commis  durant  sa  Passion  ne  peut  être  imputé  ni  indistinctement  à  tous  les  Juifs vivant  alors,  ni  aux  Juifs  de  notre  temps.  S’il  est  vrai  que  l’Église  est  le  nouveau  peuple  de Dieu,  les  Juifs  ne  doivent  pas,  pour  autant,  être  présentés  comme  réprouvés  par  Dieu  ni maudits, comme  si  cela  découlait  de  la  Sainte  Ecriture  »  (NA  4). 

\end{quote}
\subparagraph{pas de mention à la Shoah dans NA} 

 ~
 
  \section{La réception de NA 4 et ses
  conséquences}

    
    \paragraph{Fécondité pratique}
Il y a eu une fécondition forte de ce texte (vendredi saint, reintrodction de l'AT, révision des manuels de catéchisme..). En 1969, en France, comité episcopal des relations avec les Juifs. Revue \textit{sens} de l'amitié judeo-chrétienne.     
  
    
    \paragraph{Les interprétations et les développements de NA}
    A partir de NA, différents documents (1974, 85,...). En 1986, discours de Jean-Paul II à la synagogue de Rome : "frères ainés". "Pas extrinsèque". 1998 : \textit{nous nous souvenons} face à la Shoah. 
    Interprétation juive des Ecritures.
    2015 : don et l'appel.
  
 ~
  \hypertarget{luxe9tat-de-la-question-des-relations-entre-luxe9glise-catholique-et-le-judauxefsme-les-dons-et-lappel-de-dieu-sont-irruxe9vocables}{%
  \section{L'état de la question des relations entre l'Église catholique
  et le judaïsme : « Les dons et l'appel de Dieu sont irrévocables
  »}\label{luxe9tat-de-la-question-des-relations-entre-luxe9glise-catholique-et-le-judauxefsme-les-dons-et-lappel-de-dieu-sont-irruxe9vocables}}

  
 
  
    
    \paragraph{Statut du document}
    2015 pour les 50 ans de Nostra Aetate.  Document le plus récent, publié par la commission religieuse avec le Judaisme.
    La première partie est une relecture des relations depuis 50 ans; la deuxième partie est théologique. il ne s'agit pas d'un document doctrinal ou magistériel, mais un document pour susciter la réflexion.
  
    
    \paragraph{Une relecture de l'histoire de la relation entre juifs et
    chrétiens}
    On ne peut pas considérer le judaïsme comme une religion à côté de la notre. Plutôt un dialogue intra-religieux ou intra-familial. Le mot religion est un terme romain et non chrétien ou juif. 
    \begin{quote}
    «  C’est  pourquoi  le  dialogue  juif-chrétien  ne  peut  être  qualifié  qu’avec  beaucoup  de  réserves de  «  dialogue  interreligieux  »  au  sens  propre  ;  il  faudrait  parler  plutôt  d’un  dialogue  «  intrareligieux  »  ou  «  intra-familial  »  sui  generis.  Dans  son  discours  du  13  avril  1986  à  la synagogue  de  Rome,  Saint  Jean-Paul  II  a  décrit  cette  situation  en  ces  termes  :  Vous  êtes  nos frères  préférés, et  d’une  certaine  manière, on pourrait  dire  nos  frères  aînés  »  (Les  dons, 20). 
\end{quote} 
    
    Ce nouveau regard nous oblige à revoir la façon dont nous écrivons l'histoire : \textit{une histoire de rupture où l'autre est coupable}. On réaffirme la judaïté de Jésus
 
    
      
      Le judaïsme n'est pas une autre religion
      
    
      
      La « judaïté » de Jésus et sa « nouveauté »
      
    
      
      Le christianisme est né au sein du judaïsme
      
    
      
    \subparagraph{Vers la séparation}  Jusqu'au concile de Nicée, la séparation n'est pas aussi marquée. Pas de contradiction à l'époque entre la pratique juive et la Foi chrétienne \sn{BOYARIN, D., La partition du judaïsme et du christianisme, Paris, Cerf, 2012.}
      \begin{quote}
    «  Parmi  les  Pères  de  l’Église,  la  théorie  dite  du  remplacement  ou  supersessionisme  gagna progressivement  du  terrain  jusqu’à  représenter  au  Moyen-Âge  le  fondement  théologique courant  du  rapport  entre  christianisme  et  judaïsme  :  les  promesses  et  les  engagements  de  Dieu ne  s’appliquaient  plus  à  Israël  qui  n’avait  pas  reconnu  en  Jésus  le  Messie  et  le  Fils  de  Dieu, mais  avaient  été  reportés  sur  l’Église  de  Jésus  Christ,  devenue  désormais  le  véritable  « Nouvel  Israël  », le  nouveau peuple  élu de  Dieu.  » \sn{(Les  dons,  17). }   
\end{quote}
\begin{Synthesis}
    A partir du moment on rejette la théorie de la substitution, on rejette un système simple. mais par quoi le remplacer ?
    Nouvelle théorie de l'accomplissement : L'Eglise accomplit toutes les autres traditions religieuses. Demeure l'altérité, qui empêche une vision totalisante
\end{Synthesis}
    
      
       \subparagraph{La théorie de la substitution ou du remplacement} A partir de Nicée, il y a une théologie du remplacement (\textit{parabole des vignerons homicides}) : Nouveau Peuple de Dieu
      
    
  
    
    \paragraph{Les questions théologiques}
    

   
    
      
   
      
     \subparagraph{Quelle est valeur théologique de la révélation dont témoigne le
      judaïsme ?}
      il s'agit de penser l'AT non pas comme dépassé par le NT. Pas uniquement une histoire, témoin de la Révélation mais une \textit{source} toujours actuelle.
      Par rapport à la Parole de Dieu, plusieurs façons de le recevoir : 
      
          \begin{quote}
    «  Dieu  s’étant  révélé  à  travers  sa  Parole,  il  peut  être  compris  par  l’humanité  dans  les  situations historiques  concrètes.  Cette  parole  invite  tous  les  hommes  à  répondre.  Si  leur  réponse  est  en accord  avec  la  parole  de  Dieu,  ils  ont  une  relation  juste  avec  lui.  Pour  les  juifs,  cette  parole peut  être  apprise  grâce  à  la  Torah  et  aux  traditions  qui  en  découlent.  La  Torah  donne  des instructions  pour  une  vie  réussie  dans  une  relation  juste  avec  Dieu.  Celui  qui  observe  la  Torah a  la  plénitude  de  vie  (cf.  Pirqe  Avot  II,  7).  Et  surtout,  en  observant  la  Torah,  les  juifs  prennent part  à  la  communion  avec  Dieu.  À  ce  propos,  le  Pape  François\sn{(Discours  aux participants  au  Congrès  international  du  Conseil  international  des  chrétiens  et  des  juifs,  30 juin 2015).}  a  dit  :  \begin{quote}
        «  Les  confessions chrétiennes  trouvent  leur  unité  dans  le  Christ  ;  le  judaïsme  trouve  son  unité  dans  la  Torah.  Les chrétiens  croient  que  Jésus  Christ  est  la  Parole  de  Dieu  qui  s’est  faite  chair  dans  le  monde  ; pour  les  juifs,  la  Parole  de  Dieu  est  surtout  présente  dans  la  \textit{Torah}.  Ces  deux  traditions  de  foi ont  pour  fondement  le  Dieu  unique,  le  Dieu  de  l’Alliance,  qui  se  révèle  aux  hommes  à  travers sa  Parole.  Dans  la  recherche  d’une  juste  attitude  envers  Dieu,  les  chrétiens  s’adressent  au Christ  comme  source  de  vie  nouvelle,  les  juifs  à  l’enseignement  de  la  \textit{Torah}  »    »
    \end{quote}  (Les  dons, 24). 
\end{quote}
    La Loi \textit{Torah} \sn{ volonté de Dieu sur l'homme. En ce sens, Jésus a accomplit la Torah} est vive et il faut respecter la façon dont les juifs accueillent cette Parole et la modalité juive, la voie juive d'accès à la Parole n'est pas un autre accès car unité du salut en Christ : 
Parole, \emph{dabar} une parole performatrice .


      \subparagraph{Quel est le rapport entre Ancien et Nouveau Testament ?}
      La logique du Christianisme a pensé le Nouveau Testament comme l'\textit{accomplissement} de l'AT. 
      Les deux interprétations de l'AT demeurent valides : 
        \begin{quote}
    «  Le  judaïsme  et  la  foi  chrétienne,  telle  qu’elle  est  exposée  dans  le  Nouveau  Testament,  \textit{sont deux  modalités  par  lesquelles  le  peuple  de  Dieu  fait  siennes  les  Écritures  sacrées  d’Israël.}  Le Livre  que  les  chrétiens  appellent  Ancien  Testament  se  prête  donc  à  l’une  et  à  l’autre  de  ces modalités.  Ainsi,  toute  réponse  à  la  parole  salvifique  de  Dieu,  qui  serait  en  accord  avec  l’une ou  l’autre  de  ces  traditions,  peut  ouvrir  un  accès  à  Dieu,  même  s’il  dépend  de  son  conseil  de salut  de  déterminer  de  quelle  manière  il  entend  sauver  les  hommes  en  chaque  circonstance. Que  sa  volonté  de  salut  est  universelle  est  confirmé  par  les  Écritures  (cf.  par  ex.  Gn  12,  1-3  ; 2 Is  2,  2-5  ;  1  Tim  2,  4).  Il  faut  donc  en  conclure  qu’il  n’y  a  pas  deux  voies  vers  le  salut,  selon l’expression  :  \begin{quote}
        «Les  juifs  suivent  la  Torah,  les  chrétiens  suivent  le  Christ»
    \end{quote}.  La  foi  chrétienne proclame  que  l’oeuvre  de  salut  du  Christ  est  universelle  et  s’étend  à  tous  les  hommes.  La parole  de  Dieu  est  une  réalité  une  et  indivisible,  qui  prend  une  forme  concrète  dans  chaque contexte  historique  particulier. (Les  dons, 25). 
\end{quote} 
    Le document défend une certaine altérité dans l'interprétation. on ne peut pas fusionner la lecture juive et la lecture chrétienne. On essaye de penser l'alternative durable et non pas de penser que l'un est dans la vérité et l'autre dans l'erreur. Nous ne connaissons pas les plans de Dieu mais il est possible que cette altérité soit voulue par Dieu.
    
    
    
      
     \subparagraph{Quel est le rapport entre ancienne et nouvelle Alliance ?}
      Le document va essayer de penser continuité et \textit{altérité}. L'alliance avec Abraham est constitutive. Par les Christianisme, l'alliance Abrahamique a été étendue au monde. 
      
    \begin{quote}
         On peut dire que Jésus Christ porte en lui la racine vivante de
cet « olivier franc » mais aussi, en un sens encore plus profond, que
toute la promesse a sa racine en lui (cf. Jn 8, 58). Cette image re-
présente pour Paul la clé d’interprétation décisive du rapport entre
judaïsme et christianisme à la lumière de la foi. À l’aide de cette
image, Paul entend exprimer la dualité de l’unité et de la divergence
entre Israël et l’Église. Car cette image montre d’une part que les
rameaux sauvages de l’olivier ne sont pas nés de la plante sur la-
quelle ils ont été greffés, et que leur situation nouvelle représente
une nouvelle réalité et une nouvelle dimension de l’œuvre salvifique
de Dieu, de telle sorte que l’Église chrétienne ne peut pas être consi-
dérée simplement comme une branche ou un fruit d’Israël (cf. Mt
8, 10-13). Et d’autre part, elle montre aussi que l’Église tire sa substance et sa force de la racine d’Israël et que les rameaux greffés se
flétriraient et risqueraient de se dessécher s’ils étaient séparés de la
racine d’Israël (cf. Ecclesia in Medio Oriente, n. 21). » (Les dons,
34).
    \end{quote}
    
    Une image de \textit{l'olivier franc} qui donne à penser, comme dirait Ricoeur. 
      L'Eglise a besoin d'Israel pour ne pas être dans l'idéologie et être ancré dans l'histoire. Mais Israël a besoin de l'Eglise pour l'universalisme.
      Il faut les deux.
      \begin{Synthesis}
      Jésus : particularisme. Et Christ universel. Mais si on ne rattache pas le Christ au Jésus humain, on en fait une idéologie. Et uniquement à regarder Jésus et son particularisme, comment nous rejoint-il ?
      \end{Synthesis}
      D'une certaine façon, Israël nous permet de ne pas oublier l'ancrage dans l'histoire "passer par le particularisme d'Abraham".
      
      
      \subparagraph{Comment les juifs sont-ils sauvés alors qu'ils ne confessent pas
      le Christ ?}
 Le document botte un peu en touche "mystère insondable": 
 
 
 \begin{quote}
    «  Un  autre  point  central  doit  continuer  à  être  pour  les  catholiques  la  question  théologique hautement  complexe  de  savoir  comment  concilier  de  façon  cohérente  leur  croyance  dans  la mission  salvifique  universelle  de  Jésus  Christ  avec  l’article  de  foi  selon  lequel  Dieu  n’a jamais  révoqué  son  alliance  avec  Israël.  Pour  l’Église,  le  Christ  est  le  Rédempteur  de  tous.  En conséquence,  il  ne  peut  y  avoir  deux  voies  menant  au  salut  puisque  le  Christ  est  venu  sauver les  gentils  mais  également  les  juifs.  Nous  sommes  confrontés  ici  au  mystère  de  l’oeuvre  de Dieu  :  il  ne  s’agit  pas  de  déployer  des  efforts  missionnaires  pour  convertir  les  juifs,  mais plutôt  d’attendre  l’heure  voulue  par  le  Seigneur  où  nous  serons  tous  unis  et  où  «  tous  les peuples  [l’]invoqueront  […]  d’une  seule  voix  et  le  serviront  sous  un  même  joug  »  (Nostra Ætate, n. 4).  »  (Les  dons, 37). 
\end{quote}   
Il ne s'agit pas de faire des efforts missionnaires pour convertir les juifs : 
\begin{quote}
     «  L’Église  a  été  amenée  à  considérer  l’évangélisation  des  juifs,  qui  croient  dans  le  Dieu unique,  d’une  manière  différente  de  celle  auprès  des  peuples  ayant  une  autre  religion  et  une autre  vision  du  monde.  En  pratique,  cela  signifie  que  l’Église  catholique  ne  conduit  et  ne promeut  aucune  action  missionnaire  institutionnelle  spécifique  en  direction  des  juifs.  Mais alors  que  l’Église  rejette  par  principe  toute  mission  institutionnelle  auprès  des  juifs,  les chrétiens  sont  néanmoins  appelés  à  rendre  témoignage  de  leur  foi  en  Jésus  Christ  devant  les juifs,  avec  humilité  et  délicatesse,  en  reconnaissant  que  les  juifs  sont  dépositaires  de  la  Parole de  Dieu  et  en  gardant  toujours  présente  à  l’esprit  l’immense  tragédie  de  la  Shoah.  »  (Les  dons, 40). 
\end{quote} 
 La question de la permanence du judaisme est un \textit{caillou dans la chaussure}. nous empêche une pensée trop systématique où tout entrerait bien. Cela remet en cause les choses. 
 \begin{Synthesis}
 La permance du judaïsme nous met dans une situation \textit{d'altérité}, \textit{eschatologique} (car il faudra attendre la fin des temps).
 \end{Synthesis}
 
 