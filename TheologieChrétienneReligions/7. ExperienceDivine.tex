\chapter{La religion à partir de l’expérience divine}

\mn{21/3/22 THéologie des Religions C3}

\section{Bibliographie}
COMMISSION THÉOLOGIQUE INTERNATIONALE, Le christianisme et les religions, Rome 1997. 

DULLES, A., Théologies de la Révélation, Artège, Perpignan, 2012. 

DURAND, M-L., « Le rapport Église/peuple juif comme paradigme du rapport aux autres religions et quasi-religions séculières ? » dans E. PISANI (dir.), Maximum illud. Aux sources d’une nouvelle ère missionnaire, Paris, Cerf, 2020, p. 141-152. 

GEFFRE, C, Le christianisme comme religion de l’Evangile, Cerf, Paris, 2012. 

GROUPE DE RECHERCHE ISLAMO-CHRETIEN (GRIC), « L’Ecriture des uns vue par la foi des autres », Ces Ecritures qui nous questionnent, la Bible et le Coran, Le Centurion, Paris, 1987 

OTTO, R., Le sacré. L’élément non rationnel dans l’idée du divin et sa relation avec le rationnel, tr. par A. JUNDT, Payot, Paris, 2001. 

RAHNER, K., Traité fondamental de la foi. Introduction au concept de christianisme, Centurion, Paris, 1983. 

SABATIER, A., Esquisse d’une philosophie de la religion d’après la psychologie et l’histoire, Paris, Librairie Fischbarer, 1897. 

SCHLEIERMACHER, D. F., Discours sur la religion à ceux de ses contempteurs qui sont des esprits cultivés, tr. par I. ROUGE, Paris 1944. 

SIEGWALT, G., « Mohammed, prophète pour le christianisme ? », Le défi interreligieux. L’Église chrétienne, les religions et la société laïque. Ecrits théologiques I, Paris, Cerf, 2014 



\section{Introduction générale }


(pour les 3 prochains chapitres) 

L'approche dialogale est un tournant, pour la théologie chrétienne. Mais il n'est pas encore reçu.


\paragraph{Sur le plan de la méthode.  } 
 Dans la mesure où le dialogue nécessite une certaine égalité entre partenaires.  Il convient de partir d'un langage ou d'un lieu commun. Il s'agit de définir ce qu'est \textit{la religion} pour définir ce lieu commun. Si chacun parle sa langue, ce sont deux monologues.
 
 Dans Fides et Ratio 104, Jean-Paul II :
 \begin{quote}
     104. La pensée philosophique est souvent l'unique terrain d'entente et de dialogue avec ceux qui ne partagent pas notre foi. Le mouvement philosophique contemporain requiert l'engagement résolu et compétent de philosophes croyants capables de reconnaître les aspirations, les ouvertures et les problématiques de ce moment de l'histoire. 
 \end{quote}

Bien sûr ce commun ne saurait être un commun que l'Eglise aurait en plenitude et les autres en fragment. La détermination de ce commun suppose un décentrement.


\paragraph{Sur le plan du contenu.  } 


conditions pour un dialogue, pas une égalité de principe :  

\begin{quote}
    « Pour rendre ce dialogue possible, les représentants de ces théologies pensent qu’il faut éliminer, chez les chrétiens, toute prétention de supériorité et d’absoluité : il faut considérer que toutes les religions ont une égale valeur. Ils pensent que considérer Jésus comme le Sauveur et l’unique médiateur de tous les hommes constitue une prétention de supériorité. » (CTI 1997, n. 93).   
\end{quote}
Il ne s'agit pas de s'autocensurer pour se décentrer . Il faut être soi-même. 




Evaluer théologiquement des autres traditions, les comparer. mais le centre de cette rencontre, c'est renouveler notre propre théologie chrétienne. 

L'autre peut m'aider à renouveler mon regard sur moi-même. La théologie des religions redevient alors une théologie chrétienne. Le dialogue nous permet de rester dans la fidélité.

\begin{quote}
    signification théologique... sens théologique de la lutte des autres religions contre le christianisme \ldots parce que le christianisme aurait oublié quelque chose. 
\textit{    Thils}
\end{quote}

\begin{quote}
    « Si les chrétiens entretiennent une telle ouverture et s’ils acceptent d’être mis eux-mêmes à l’épreuve, ils deviendront capables de cueillir les fruits du dialogue. Ils découvriront alors avec admiration tout ce que Dieu, par Jésus Christ et en son Esprit, a réalisé et continue à réaliser dans le monde et dans l’humanité tout entière. Loin d’affaiblir leur foi chrétienne, le vrai dialogue l’approfondira. Ils deviendront toujours plus conscients de leur identité chrétienne et percevront plus clairement ce qui est propre au message chrétien. Leur foi gagnera de nouvelles dimensions, tandis qu’ils découvriront la présence agissante du mystère de Jésus Christ au-delà des frontières visibles de l’Eglise et du bercail chrétien. » \textit{(Dialogue et annonce, n. 50). }
\end{quote}
Pas toujours facile. 
\paragraph{quelques visions}
» F.BOUSQUET, \sn{« Le concile Vatican II : l’Église entre en ‘conversation’ avec les religions » dans PISANI (dir), Religions et dialogues, p. 32. } : 
\begin{quote}
    « L’attitude prônée par \textit{Nostra aetate} de la conversation , du dialogue positif, montre qu’on en a fini avec la compréhension de soi comme clos, parfait, achevé, n’ayant par conséquent rien à attendre de ‘l’autre’ […] Rester altéré de l’ ‘autre’ (non pas : aliéné par l’autre), est un signe positif d’une compréhension du mystère […] comme ce qui nous déborde, ce qui, bien loin d’arrêter de pensée invite à penser davantage et à vivre plus profond […] 
\end{quote}
L'importance de rencontrer l'autre : 
\begin{quote}
    « Si quelque chose se révèle transposable (du rapport Église-judaïsme vers le rapport Église autres religions), c’est ce que l’Église est devenue dans ce processus. Finalement, non pas tant ce qu’elle dit de l’autre mais ce que la rencontre avec l’existence de l’autre a changé d’elle-même (\ldots) Ce qui est transposable ailleurs et avec d’autres, c’est toujours ce que l’on comprend de soi, de ses propres blocages, de ses propres obscurcissements et c’est cela qui permet de voir de façon neuve la réalité et d’y répondre » (M.-L. Durand, 148). \textbf{}
\end{quote}

\begin{rem}
Il faut ajouter le dialogue de vie (dans l'action) et dans le dialogue spirituel.  Il y a déjà quelque chose de commun avec les autres. 
\end{rem}

la théologie chrétienne est renouvelée par le dialogue avec les autres religions. 
\begin{Synthesis}
La théologie chrétienne des religions est plus la manière dont on renouvelle notre théologie suite au dialogue avec l'autre
\end{Synthesis}


\paragraph{La détermination d’un « commun » } 
Dans ce premier chapitre, nous partirons du commun suivant : "expérience spirituelle du Divin". Une telle approche reconnaît une réalité à partir du moment où elle est fondée spirituellement. 
Elle permet de faire la distinction entre l'expérimentation de la vérité et la vérité elle-même.
On symbolise une expérience qui est ineffable. 

La théologie des religions basée sur la manifestation divine elle-même.

% -------------------------------------------
\section{A l’origine des religions, l’expérience spirituelle}

Qu'avons nous de commun avec les autres traditions religieuses, à partir du moment où elles ne sont pas dans une démarche agnostique, l'étude des \textsc{phénomènes religieux} permet de définir un \textit{commun} objectif avec les autres religions. 

Au XIX et XX, plusieurs auteurs ont travaillé le sujet. Nous nous arrêterons sur : 
\begin{itemize}
    \item Fridriech Schleiermacher
    \item Otto
\end{itemize}

\subsection{L’approche « romantique » de Schleiermacher} 
\paragraph{Fridriech Schleiermacher}, 17.., théologien réformé, Berlin. Inspirateur de la théologie libérale. En 1799, publie \textit{le discours sur la Religion à ceux de ces contempteurs qui sont des esprits éclairés}. Nouvelle voie pour la théologie en s'appuyant sur le fait que l'homme est essentiellement religieux.

\paragraph{Das Gefühl} le sentiment, élément important pour lui. Il ne part ni de la métaphysique ni de la morale, il part de l'expérience de l'impression (empreinte) de l'infini, qui se manifeste dans tous les genres de manifestations religieux \sn{SCHLEIERMACHER, Discours sur la religion à ceux de ses contempteurs qui sont des esprits cultivés, Paris 1944, 150-151. } :

\begin{quote}
    « La religion n’apparaît nulle part à l’état pur ; tout ce dont vous parlez là, ce ne sont que des adhérences étrangères, dont notre tâche est précisément de la dégager » (Discours, 149). 
    
    « La religion, pour entrer dans son bien propre, renonce à toute prétention sur tout ce qui appartient à la métaphysique et à la morale, et restitue tout ce qu’on lui a incorporé de force. Elle ne cherche pas à déterminer et expliquer l’univers d’après sa nature à lui comme fait la métaphysique ; elle ne cherche pas à le perfectionner et l’achever par le développement de la liberté et du divin libre-arbitre de l’homme ainsi que fait la morale. En son essence, elle n’est ni pensée ni action, mais contemplation intuitive et sentiment. Elle veut contempler intuitivement l’Univers [l’Infini] ; elle veut l’épier pieusement dans les manifestations et les actes qui lui sont propres ; elle veut se laisser, dans une passivité d’enfant, \textit{saisir et envahir} par ses influences directes. Ainsi donc elle est l’opposé de la métaphysique et de la morale dans tout ce qui constitue son essence et dans tout ce qui caractérise ses effets. Les deux autres ne voient dans tout l’univers que l’homme centre de toutes les connexions, comme condition de toute existence et cause de tout devenir. Elle veut, elle, dans l’homme, non moins que dans tout autre être particulier et fini, voir l’Infini, le décalque, la représentations de l’Infini » 
\end{quote}

Il cherche \textit{l'essence de la Religion}, ce qui est son fondement \sn{Chateaubriant}. Contemplation intuitive des sentiments.
La piété n'est ni un savoir ni un agir mais la détermination de la conscience de soi immédiate.

\begin{Def}[conscience immédiate]
La place de l’ancienne psychologie rationnelle est occupée par la conscience immédiate de soi comme sentiment, qui comprend aussi l’ontologie, « parce que cette construction de l’être fini en général ne nous est donnée que dans la condition fondamentale de notre être " Dialectique, p. 204.. 
\end{Def}

L'inventeur de l'herméneutique, c'est Schleiermacher, à l'origine de la source jaillissante. 

\begin{Def}[religion]
C'est le passage à l'expression collective d'une expérience individuelle
\end{Def}

\paragraph{médiateur } On a besoin de médiateurs pour prendre conscience de cette expérience.

\begin{quote}
    « Il doit, après chacune de ses excursions dans l’Infini, extérioriser l’impression qu’il en rapporte, de façon à faire d’elle ainsi un objet communicable par l’image ou la parole (…) et il doit donc aussi, involontairement et pour ainsi dire dans l’état d’enthousiasme (…) figurer pour autrui ce qui lui est arrivé, sous une forme sensible, en poète ou en voyant, en orateur ou en artiste. Un tel homme est un véritable prêtre du Très-Haut\sn{étonnant comme expression}, qu’il rend plus accessible à ceux qui ne sont pas habitués à saisir que le fini et sa valeur minime ; il leur présente les choses célestes et éternelles comme des objets de jouissance et de communion, comme la seule source inépuisable de ce vers quoi tend toute leur aspiration supérieure. Il vise ainsi à éveiller le germe somnolent de la meilleure humanité, à allumer l’amour du Très-Haut, à transformer la vie ordinaire en une vie plus haute, à réconcilier les fils de la terre avec le ciel (…) C’est ici la prêtrise supérieure, celle qui fait connaître l’âme intime de tous les mystères spirituels, et dont la voix descend des hauteurs du royaume de Dieu » (Schleiermacher, Discours, 125-126).  
\end{quote}
L'homme qui a fait l'expérience de Dieu doit le communiquer, la traduire, celle de l'ineffable dans du fini.

Jésus est l'archétype du grand prêtre, qui transmet aux hommes l'expérience de Dieu.


L'Eglise provienne des personnes charismatiques qui éveillent des veilleurs.

\paragraph{Les autres religions sont à leur source des religions qui contiennent son essence}

l'herméneutique de l'âge romantique, c'est la source qui est véritable.  A influencé l'historico critique qui cherche la source.

\begin{quote}
    « Toute religion que vous désignez par un nom et un caractère distincts, et qui peut-être a depuis longtemps dégénéré en un code d’usages vides de sens, en un système de concepts et de théories abstraites ; si vous remontez à sa source (…) vous trouverez que toutes les scories mortes ont été jadis des jets ardents du feu intérieur, que toutes contiennent de la religion, et plus ou moins de ce qui constitue sa véritable essence (…) ; que chacune a été une des formes que la religion éternelles et infinie a nécessairement dû revêtir, au milieu d’êtres finis et limités » (Discours, 281). 
\end{quote}


\paragraph{Une influence forte sur la modernité et la condition de dialogue des religions}

%------------------------------
\subsection{L’expérience du sacré selon R. Otto}

\paragraph{Rudolf Otto} 150 ans plus tard. 1860-1937. Théologiens. Son livre : \textit{le Sacré}, influencé par Scheiermacher et Fries. Le divin est le tout autre pour l'homme. le Sacré ne peut être atteint par la raison. Il met en exergue la catégorie du sacré. 

\begin{quote}
    « Cette catégorie est complexe [numineux] ; elle comprend un élément, d’une qualité absolument spéciale, qui se soustrait à tout ce que nous avons appelé rationnel, est complètement inaccessible à la compréhension conceptuelle et, en tant que tel, constitue un arrêton, quelque chose d’ineffable » (Otto, 25). 
\end{quote}

\paragraph{L’invention du « numineux »}


\begin{Def}[numineux]
ce qui est irréductible. 
\end{Def}

Le \textit{numen } pour Kant, c'est ce qui s'oppose au \textit{phenomene}, seule voix d'accès au \textit{numen}. 

Otto critique la théologie chrétienne, qui cherche une hyper rationalité et qui perd de vue le \textit{numen}

\paragraph{Le numineux est « mysterium » }

Pas un sentiment de dépendance comme Scheiermacher mais \textit{mystère} quand il se fait proche de l'homme.
Il va caractériser le mysterium quand il s'approche de l'homme : \emph{Etonnement}



\subparagraph{Le mysterium tremendum } étonnement qui nous saisit, bouche bée, interdit. Références bibliques : 
\begin{quote}
Les disciples\sn{    Mc 10,32} étaient en route pour monter à Jérusalem ; Jésus marchait devant eux ; ils étaient saisis de frayeur, et ceux qui suivaient étaient aussi dans la crainte. Prenant de nouveau les Douze auprès de lui, il se mit à leur dire ce qui allait lui arriver.

\end{quote}

Pour Otto, il ne s'agit pas d'une peur normale, mais une peur religieuse. 
\begin{quote}
    Ce saisissement « provoque dans la conscience, comme réaction, le ‘sentiment de l’état de créature’ (…), le sentiment de notre néant, de notre effacement devant l’objet dont nous avons pressenti, dans la ‘terreur’, le caractère terrifiant et la grandeur » (Otto, 43).  
\end{quote}
\subparagraph{mysterium fascinans} le mystère qui captive, fascine. Face à l'élément révulsif du tremendum, un complémentaire : 


\begin{quote}
    « Le mystère n’est pas seulement pour  (la créature) l’étonnant, il est le merveilleux. A côté de l’élément troublant apparaît quelque chose qui séduit, entraîne, ravit étrangement, qui croît en intensité jusqu’à produire le délire et l’ivresse ; c’est l’élément dionysiaque de l’action du numen. Nous l’appellerons le ‘fascinant’ » (Otto 70). 
\end{quote}

\subparagraph{mysterium augustum} respect, distance. Cite Job, 

\begin{quote}
    « (Ce passage de Job  38) exprime (…) ce qu’il y a d’absolument stupéfiant, presque démoniaque, de complètement incompréhensible, d’énigmatique dans l’éternelle puissance créatrice, ce qu’il y a en elle d’incalculable, de ‘tout autre’, mais qui émeut l’âme dans ses profondeurs, la fascine et la remplit en même temps du plus profond respect. C’est là le sens de tout le passage ; il s’agit du ‘mystère’ tout ensemble fascinant et auguste » (Otto, 143). 
\end{quote}



\paragraph{Le processus de rationalisation}

\subparagraph{domestiquer le numineux} On dit qu'on a peur de Dieu parce qu'on a péché, on rationalise par l'éthique ce qui est d'un autre ordre. 
\begin{quote}
    Cette colère « que l’on appelle trop souvent ‘naturelle’ et qui en réalité n’est rien moins que naturelle puisqu’elle est numineuse, se rationalise en se saturant d’éléments éthiques, d’ordre rationnel, ceux de la justice divine, justice distributive qui punit les transgressions morales. (…) Dans la ‘colère de Dieu’ vibre et brille toujours l’élément non rationnel qui lui donne un caractère effrayant que ‘l’homme naturel’ ne peut sentir » (Otto, 45).  
\end{quote}

De la même façon, le numineux fascinent en grâce, Dieu d'amour. Mais cette moralisation n'est pas la suppression du numineux.

On n'a pas pu camoufler complétement le numineux. Par exemple, dans le \textit{royaume de Dieu} ; 
\begin{quote}
    « Dans l’Evangile de Jésus s’est achevé le processus tendant à rationaliser, à moraliser et à humaniser l’idée de Dieu (…). Il a saturé le numineux, toujours plus abondamment et plus complètement, de valeurs rationnelles profondes, exprimées en de clairs prédicats (…) Mais ce serait une erreur de croire que cette rationalisation est une élimination du numineux » (Otto, 147). 
\end{quote}

\paragraph{Les tentatives de supprimer le numineux}

Dieu est Esprit. Hegel voyait dans le christianisme la religion par excellence car Esprit = raison. 
Otto insiste sur le Dieu de Pascal, qui ne se réduit pas par la raison : 
\begin{quote}
    « Si les sentiments du non-rationnel du numineux se manifestent dans toute religion, ils apparaissent avec une vigueur particulière dans la religion sémitique et surtout dans celle de la Bible. Ici le mystérieux pénètre et anime puissamment les idées des démons et des anges qui représentent le ‘tout autre’ dont notre monde est entouré, dominé et imprégné ; il fait sentir sa force dans l’attente de la fin du monde et dans l’idéal du royaume de Dieu qui s’oppose à l’ordre naturel, soit comme futur, soit comme éternel, mais toujours comme quelque chose d’absolument merveilleux et de ‘tout autre’ » (Otto, 133). 
\end{quote}


\begin{Ex}
La colère de Dieu, exemple de Numineux, a été ensuite rationalisé en disant que c'est parce que nous avons péché.
\end{Ex}
\paragraph{L’universalité de l’expérience du numineux }
 \sn{le 28/3/22 rappel de la méthode, se décentrer pour entrer en dialogue avec l'autre. }
 Expérience universelle mais expression différente.
 \begin{quote}
     « Si les sentiments du non-rationnel du numineux se manifestent dans toute religion, ils apparaissent avec une vigueur particulière dans la religion sémitique et surtout dans celle de la Bible. Ici le mystérieux pénètre et anime puissamment les idées des démons et des anges qui représentent le ‘tout autre’ dont notre monde est entouré, dominé et imprégné ; il fait sentir sa force dans l’attente de la fin du monde et dans l’idéal du royaume de Dieu qui s’oppose à l’ordre naturel, soit comme futur, soit comme éternel, mais toujours comme quelque chose d’absolument merveilleux et de ‘tout autre’ » (Otto, 133). 
 \end{quote}
Autrement dit, Otto reconnaît dans la révélation biblique, une dimension du numineux (anges, eschatologie,..) mais cette expérience dans les autres religions comme l'Islam : 
\begin{quote}
    « Aucune religion n’a autant d’affinité pour cette idée (de la prédestination) que l’islam.  Ce qui donne à l’islam son caractère propre, c’est le fait qu’en lui le côté rationnel et aussi le côté moral de l’idée de Dieu n’ont jamais pu acquérir le relief fermement accusé qu’ils ont pris par exemple dans le judaïsme ou dans le christianisme » (Otto, 159). « En Allah, le numineux l’emporte absolument. On critique l’islam en disant qu’en lui le commandement moral porte le caractère du ‘contingent’ et ne vaut qu’en vertu de la volonté contingente de la divinité. La critique est juste, mais le fait donc il s’agit n’a aucun rapport avec la ‘contingence’. Il s’explique au contraire par le fait qu’en Allah l’élément non rationnel du numineux l’emporte encore trop sur l’élément rationnel et n’est pas encore suffisamment schématisé et tempéré par ce dernier, dans le cas présent (la prédestination) par l’élément moral comme dans le christianisme » (Otto, 159-160). 
\end{quote}
Le judaïsme et le christianisme ont essayé de mettre sur un plan moral voire métaphysique (pour le christianisme). 
On va domestiquer le caractère numineux de certaines religions (surtout les religions qui valorisent le non rationnel). 

Un psychologue, Abraham Maslow, affirmait : 
\begin{quote}
    les religions... sont les mêmes ... car une expérience commune du mystère, essence de la religion
\end{quote}


\section{La révélation de Dieu en tout homme}

On passe de l'expérience mystique à la révélation.
Le christianisme n'est plus une religion à part mais fait partie de l'histoire des religions.
Certes, leur différences montrent qu'il y a différentes réponses au mystère. 

On peut penser qu'à partir d'une même expérience, plusieurs réponses sont possibles.

\paragraph{Confrontation avec les sciences} C'est dans le dialogue avec les sciences que c'est développé la phénoménologie des religions, On n'est pas encore dans le dialogue des religions mais dans la science des religions.

Un certain nombre de théologiens ont essayé de rentre témoignage du christianisme.

\subsection{La révélation divine en l’homme : l’impulsion de Schleiermacher} 

Il ouvre  par le \textit{sentiment religieux}, l'impact du divin sur l'humain. 
\begin{quote}
    « Avec cette foi en lui-même, qui peut s’étonner que [Jésus] ait été certain non seulement d’être le médiateur pour beaucoup d’êtres, mais encore de laisser derrière lui une grande école, qui déduirait de sa religion à Lui sa religion à elle, toute semblable (…). Mais il n’a jamais affirmé être l’unique objet de l’application de son idée, être le seul médiateur […] Le Christ n’a jamais donné les intuitions et les sentiments qu’il pouvait communiquer lui-même comme contenant tout ce que devait embrasser la religion qui devait sortir de son intuition fondamentale ; il a toujours engagé à tenir compte de la vérité qui viendrait après lui » (Schleiermacher, Discours, 318-319). 
\end{quote}
En évoquant Jésus, Schleiermacher ouvre à la possibilité d'autres médiateurs. La religion véritable dépasse le christianisme. 


\subsection{Religion et révélation : Auguste Sabatier} 


On ne peut pas sortir la religion de la révelation.

\begin{quote}
    « Si la piété ‘c’est Dieu sensible au cœur’, il est évident qu’il y a, dans toute piété, quelque manifestation positive de Dieu. Les idées de religion et de révélation restent donc corrélatives et religieusement inséparables. La religion n’est rien d’autre que la révélation subjective de Dieu dans l’homme, et la révélation c’est la religion objective en Dieu. C’est le rapport de la forme et de l’objet, de l’effet et de la cause organiquement unis ; c’est un seul et même phénomène psychologique, lequel ne peut subsister ni se produire que par leur rencontre » (Sabatier, 34). 
\end{quote}

L'action de Dieu se manifeste dans la nuit même mais on ne peut acceder à la révélation qu'en acceptant le subjectif de l'homme qui fait une expérience.

\begin{itemize}
    \item Pas de religion sans révélation
    \item universalité des religions
    \item plus on accueille cette révélation, plus elle va progresser dans sa pureté
\end{itemize}

\paragraph{Universel}
\begin{quote}
    « Je conçois donc que la révélation soit aussi universelle que la religion elle-même (…) Aucune forme de piété n’est vide ; aucune religion n’est absolument fausse ; aucune prière n’est vaine. Encore une fois, la révélation est dans la prière et progresse avec la prière. D’une révélation obtenue dans une première prière, naît une prière plus pure, et de celle-ci une révélation plus haute. Ainsi la lumière grandit avec la vie, la vérité avec la piété » (Sabatier, 34). 
\end{quote}

\paragraph{Et expérience de révélation Intérieure}


Elle part du présupposé que \textit{Dieu est Esprit} : 
cf Jn 4
\begin{quote}
    Mais l'heure vient, et elle est déjà venue, où les vrais adorateurs adoreront le Père en esprit et en vérité; car ce sont là les adorateurs que le Père demande. 24Dieu est Esprit, et il faut que ceux qui l'adorent l'adorent en esprit et en vérité.
\end{quote}

La révélation est intérieure car Dieu n'ayant pas d'expérience phénoménale, il ne peut se

Et Jésus ?
\paragraph{Jésus nous révèle Dieu mais n'est pas Dieu qui se révèle}
L'Esprit se saisit des prophètes, mais de façon intermittente.
En Jésus, de façon continue.

\begin{quote}
    « Cette évolution paraît achevée dans l’âme du Christ. Ici l’inspiration cesse d’être miraculeuse, sans cesser d’être surnaturelle. Elle ne se produit plus par accès ni par intermittence […] Etant continue, l’inspiration devient normale. L’ancien conflit de l’Esprit divin et de l’esprit humain s’évanouit. L’action immanente et constante de l’un se manifeste dans l’activité régulière et féconde de l’autre. Dieu vit et travaille dans l’homme, l’homme vit et travaille en Dieu. La religion et la nature, la voix divine et celle de la conscience, le sujet et l’objet de la révélation se pénètrent et ne font plus qu’un. La révélation suprême de Dieu éclate dans la plus haute des  consciences et dans la plus belle des vies humaines » (Sabatier, 40). 
\end{quote}
Il ne prend pas le vocabulaire du dogme de Chalcédoine. Mais On risque d'avoir une forme \textit{adoptianisme}, idée que Dieu avait adopté Jésus en envoyant son esprit.
Même si pour les Pères de l'Eglise, Esprit = Sagesse = Logos. Si Esprit = Logos, on retrouve presque l'union apostatique. 


\begin{quote}
    Texte 21 « [Dieu] n’a pas laissé tomber (l’Evangile) du ciel ; il ne l’a pas envoyé par l’intermédiaire d’un ange ; il fait naître Jésus des flancs mêmes de la race humaine, et Jésus nous a donné l’Evangile éclos au fond de son cœur. Ainsi Dieu se révèle dans les grandes consciences que son Esprit fait surgir l’une après l’autre, qu’il emplit et qu’il illumine » (Sabatier, 54). 
\end{quote}

Sabatier est conscient que l'expérience spirituelle peut être différente de l'un à l'autre : 
\begin{quote}
    « La révélation de Dieu, faite sur un point et dans une conscience, se prolonge et rayonne infailliblement. L’ébranlement donné à une âme retentit dans toutes les âmes sœurs qui se mettent à vibrer et à rendre le même son. Une conscience illuminée devient illuminatrice à son tour […] Ainsi la révélation intérieure devient chaîne, tradition continue et, s’incarnant dans chaque génération humaine, reste non seulement le plus riche des héritages, mais la puissance historique la plus féconde » (Sabatier, 56). 
\end{quote}

Approche psychologique mais qui remet en cause ce qui est un peu trop extérieur.
\paragraph{Une condamnation}
Dans le monde catholique, on va leur appliquer le terme de \textit{moderniste} qui va être condamné au début du XXe siècle, car elle remet en cause beaucoup de choses.
Mais Jean-Paul II \sn{ 9 septembre 98} : 
\begin{quote}
Dans toutes les expériences religieuses authentiques, la manifestation la plus caractéristique est la prière. En raison de l'ouverture constitutive de l'esprit humain à l'action par laquelle Dieu l'invite à se transcender, nous pouvons considérer que «toute prière authentique est suscitée par l'Esprit Saint, qui est mystérieusement présent dans le cœur de chaque homme» (Allocution aux membres de la Curie romaine, 22 décembre 1986, n. 11, in Insegnamenti IX/2 [1986], p. 2028).
   
\end{quote}

\subsection{L’expérience religieuse et du pluralisme religieux selon J. Hick}


\paragraph{John Hick} promoteur de la théologie plurielle. E \sn{\href{https://fr.wikipedia.org/wiki/John_Hick}{John Hick en wikipedia}} contre les théologies trop dogmatiques du Christianisme.


la manière dont on interprète une révélation fait partie de l'expérience religieuse. La révélation est à la fois l'action de Dieu et notre acceptation/expérience de cette expérience de Dieu.





\subsection{Remarques conclusives }

\begin{Synthesis}
Cette expérience de la Révélation, ce que les chrétiens appelent la grâce, expérience immédiate de Dieu qui se rend présent à l'homme en minimisant les médiations créées.
Mais en même temps, on a besoin de prophètes qui vont révéler en nous ce que nous avons expérimenté mais pas perçu.
\end{Synthesis} 

Tout le monde peut faire l'expérience mystique, encore faut il s'ouvrir à cette grâce, et l'autre (prophète) nous aide à cela.

Ac 17,28
\begin{quote}
    Il a voulu qu'ils cherchassent le Seigneur, et qu'ils s'efforçassent de le trouver en tâtonnant, bien qu'il ne soit pas loin de chacun de nous, 28car en lui nous avons la vie, le mouvement, et l'être. C'est ce qu'ont dit aussi quelques-uns de vos poètes: De lui nous sommes la race... 
\end{quote}


Rm 2, 14

\begin{quote}
    14Quand les païens, qui n'ont point la loi, font naturellement ce que prescrit la loi, ils sont, eux qui n'ont point la loi, une loi pour eux-mêmes;
\end{quote}

Cette approche a permis un regard plus positif sur les autres religions, qui permettent l'auto communication de Dieu dans les autres religions. 

\paragraph{Critiques} On relative l'expression de l'expérience; Or, l'expérience mystique passe par l'extérieur, le langage,... la manière dont on interprète l'expérience fait partie de l'expérience.

\begin{Ex}
Jésus a fait l'expérience de Dieu. Jésus exprime sa communication avec Dieu dans toute sa vie.
On peut donc pas séparer l'interprétation de l'expérience de l'expérience.
Et du coup, on est plus proche des dogmes. La symbolisation fait partie de l'expérience spirituelle. 
\end{Ex}

%-------------------------------
\section{Le renouvellement de la théologie face à la question de l’universalité de l’expérience de Dieu}

\begin{quote}
    « L’expérience de Dieu, dans la mesure où il s’agit de l’œuvre transformatrice de Dieu en nous, peut être appelée grâce, et la grâce, dans la mesure où elle produit une nouvelle perception de Dieu, est Révélation » (Dulles, 216). 
\end{quote}
\begin{quote}
    « En partie à cause de la distinction entre Révélation et doctrine, les théoriciens de cette approche se sont permis, à l’égard des religions non-bibliques, une attitude plus positive qu’à l’accoutumée en milieu chrétien. Il devenait possible de soutenir que non seulement le christianisme, mais toutes les religions, selon des degrés variés de pureté et d’intensité, témoignaient de l’auto-manifestation du Dieu vivant. Cette attitude a grandement contribué au dialogue interreligieux » (Dulles, 229). 
\end{quote}
\begin{quote}
    « Le concept théologique de révélation ne peut pas être confondu avec celui de la phénoménologie religieuse (« religions de révélation », c’est-à-dire celles qui se considèrent comme fondées sur une révélation divine). Ce n’est que dans le Christ et dans son Esprit que Dieu s’est donné totalement aux hommes » (CTI 1997, n. 88) 
\end{quote} 


\subsection{La réinterprétation de la dichotomie nature-surnaturel}
 

\paragraph{ La théologie du concile Vatican I. }

Le concile V I canonise deux types de voie : 
\begin{quote}
    « Dieu, principe et fin de toutes choses, peut être connu avec certitude par la lumière naturelle de la raison humaine à partir des choses créées (…). Toutefois, il a plu à sa sagesse et à sa bonté de se révéler lui-même au genre humaine ainsi les décrets éternels de sa volonté par une autre voie, surnaturelle celle-là » (…) Cette Révélation surnaturelle est contenue (…) ‘dans les livres écrits et dans les traditions non écrites qui, reçues par les apôtres de la bouche du Christ lui-même, ou transmises comme de main en main par les apôtres sous la dictée de l’Esprit Saint, sont parvenues jusqu’à nous’ » (Vatican I, Dei Filius, 3004-3006).
\end{quote}




\paragraph{La grille de lecture « classique » pour la théologie des religions }
Problème de cette vision, c’est de séparer les deux : la Raison semble incapable de nous faire connaitre autre chose qu’un dieu idolatrique (cf Rm 1)
Avec Danielou, dans la théologie de l’histoire, les religions naturelles sont propedeutiques à la religion surnaturelle.
 


\subsection{Vatican II : La révélation en Esprit et en vérité est salvifique}
 
 

\paragraph{ Dimension trinitaire de la révélation }



va rééquilibrer l'idée de révélation
\begin{quote}
    Le Saint-Esprit « qui touche le cœur et le tourne vers Dieu, ouvre les yeux de l’esprit et donne ‘à tous la douceur de consentir et de croire à la vérité’. Afin de rendre toujours plus profonde l’intelligence de la révélation, l’Esprit-Saint ne cesse, par ses dons, de rendre la foi plus parfaite » (Dei Verbum 5). 
\end{quote}
\paragraph{Révélation et salut }
Dieu se révèle non pour se faire connaître mais pour qu'on participe à la vie divine. Pas de salut sans révélation, pas de révélation sans salut. 



\paragraph{Universalité du salut } le salut est pour tous les hommes. 
\begin{quote}
      « Puisque le Christ est mort pour tous et que la vocation dernière de l’homme est réellement unique, à savoir divine, nous devons tenir\sn{dogmatique} que l’Esprit-Saint offre à tous, d’une façon que Dieu connaît, la possibilité d’être associé au mystère pascal » (Gaudium et spes 22,5). 
\end{quote}

On ne peut pas penser l'universalité de la révélation dans des limites trop strictes. 
\paragraph{L’unité de l’histoire du salut et de la révélation} 




\subsection{Révélation transcendantale et révélation catégorielle}

intégrer dans le dialogue les aspects positifs de la polémique modernisme. C'est Rahner qui a le mieux articulé les requetes du monde contemporain.

\begin{quote}
    « Si Dieu crée l’autre et le crée par suite comme fini, si Dieu crée l’esprit qui, par sa transcendance (…) connaît l’autre comme fini, et pour cette raison sépare en même temps ce fondement comme qualitativement tout autre, comme mystère sacré ineffable, par là est déjà donnée une certaine annonce de Dieu comme mystère infini » (Rahner, Traité, 197).   Ici Dieu n’est appréhendé comme mystère que par analogie, dans la mesure où il n’est su en lui-même que par dépassement négatif du fini et par référence médiatisante, et non par relation immédiate directe à lui […]« par-delà cette ‘révélation naturelle’, qui présente à proprement parler Dieu comme question (et non comme réponse), existe la Révélation proprement dite de Dieu (…) La Révélation proprement dite ouvre (…) – étant présupposés le monde et l’esprit transcendantal – ce qui en elle et pour l’homme est encore inconnu : la réalité intérieure de Dieu et son attitude personnelle et libre à l’égard de la créature spirituelle » (Rahner, Traité, 197-198).   
\end{quote}

\paragraph{Trois niveaux }
\begin{itemize}
    \item L'homme est ouvert à Dieu \emph{Capax Dei}\sn{Saint Irénée}
    \item La \textsc{Révélation objective}, historique de Jésus Christ ("catégoriale" ou "spéciale")
    \item \textsc{Révélation transcendantale }: reconnaissance de cette Révélation objective via une intériorité subjective, intérieure, spirituelle. Sans cette grâce, on voit Jésus et on ne comprend rien. 
\end{itemize}
\begin{quote}
    « Cette Révélation présente deux aspects (un aspect transcendantal et un aspect historique) qui sont distincts et en cohérence, qui sont l’un et l’autre nécessaire pour qu’il y ait simplement Révélation » (Rahner, Traité, 198). 
\end{quote}
\begin{Synthesis}
Il n'y a pas de Révélation sans réception
\end{Synthesis}


\paragraph{Que signifie l’aspect transcendantal de la Révélation ? }

Pour Rahner, Dieu se donne pas de façon extérieure mais quand il s'approche de l'homme, il le transforme intérieurement.

Idem liberté : la liberté se vérifie quand nous prenons des choix. Passer de la puissance à l'acte. Nous agissons imparfaitement librement mais en Jésus, la grâce de Dieu a été accueillie totalement et c'est pour cela que sa vie est révélation insurpassable.
 
Jésus Christ : Parole de Dieu reçue à 100\% et historicisé;



\paragraph{Que signifie l’aspect historique, catégorial de la Révélation ? }
\begin{quote}
    « En Jésus sont parvenues à leur point culminant (…) la communication gracieuse de Dieu à l’homme et son auto-explication catégoriale dans la dimension de ce qui est corporellement saisissable et relevant du social » (Rahner, Traité, 202). 
\end{quote}

En Jésus-Christ, l'homme qui a totalement fait la volonté de son père. A partir de là, il a historicisé cette volonté dans son action et son histoire. Il est l'accomplissement de la Révélation dans ces deux dimensions catégoriale et transcendentale.



\begin{Synthesis}
Rahner essaye de réconcilier la vision moderniste qui insiste sur la Révélation transcendantale mais en laissant de côté la Révélation catégorielle et le dogme qui insiste sur la Révélation catégorielle. 
\end{Synthesis}
Le christianisme est une religion universelle


%------------------------
\section{L’articulation entre la révélation historique et la révélation universelle et spirituelle dans le contexte pluraliste}

\subsection{Les religions proviennent de la révélation transcendantale mais ne témoignent pas de la révélation spéciale }
 
 
 \paragraph{Schéma à trois niveaux}
 \begin{quote}
     « En toute religion se rencontre la tentative (…) de médiatiser historiquement, de réfléchir et d’exposer dans des propositions la Révélation originaire, non réfléchie et non objective. Dans toutes les religions, l’on trouve des moments isolés de cette sorte de médiation et d’autoréflexion de la relation surnaturellement transcendantale de l’homme à Dieu par l’autocommunication de Dieu » (Rahner, Traité, 201). 
 \end{quote}

Autrement dit, cette révélation intérieure va se manifester dans les autres traditions religieuses : signe que le travail de la grâce porte du fruit, qui sera différent selon les cultures et les circonstances.

La révélation transcendentale étant dans toute religion, elles sont surnaturelles. Mais comme elles sont médiatisées par l'homme et son péché, seul JC a objectivé dans toute sa pureté la Révélation divine.


\subsection{La révélation dans le document de la CTI sur les religions non-chrétiennes (1997) }

\paragraph{sens strict}
\begin{quote}
    «  La spécificité et le caractère non réitérable de la Révélation divine en Jésus-Christ sont fondés sur le fait que c’est seulement en sa personne que se réalise l’autocommunication de Dieu Trinité. De là vient qu’on ne peut parler de Révélation de Dieu au sens strict que lorsque Dieu se donne lui-même. » (CTI 1997, § 88). 
\end{quote}

\paragraph{Sens étendu}
On suggère l'idée de révélation transcendantale : 
\begin{quote}
    « Dieu s’est donné à connaître et continue de se faire connaître aux hommes de nombreuses manières : à travers les œuvres de la création[169], à travers les jugements de la conscience[170], etc. Dieu peut éclairer les hommes sur des chemins divers. La fidélité de Dieu peut donner lieu à une certaine connaissance par connaturalité. Les traditions religieuses ont été marquées par « bien des personnes sincères, inspirées par l’Esprit de Dieu[171] ». […] On ne peut exclure en elles des éléments d’une véritable connaissance de Dieu, même avec des imperfections[172]. La dimension de connaissance ne peut être totalement absente là où nous reconnaissons des éléments de grâce et de salut. » (CTI 1987, § 90)
\end{quote}

On a bien une dimension surnaturelle de salut. 

\subsection{En quoi les autres religions ont-elles une raison d’être du point de vue chrétien ?}  

a) La raison eschatologique des autres traditions religieuses


b) La raison prophétique 


c) La raison trinitaire  


\section{textes}

Texte 13 « Si la révélation divine atteint sa plénitude qualitative en Jésus, c’est qu’aucune révélation du mystère de Dieu ne peut égaler en profondeur ce qui est advenu quand le Fils divin incarné vécut en clé humaine, dans une conscience humaine, sa propre identité de Fils de Dieu » (Dupuis 378-379).



Texte 14  « La plénitude qualitative (…) de la révélation en Jésus-Christ n’empêche pas, même après l’événement historique, une autorévélation divine continue par les prophètes et les sages d’autres traditions religieuses, par exemple le prophète Mohammed. Cette autorévélation a eu lieu et continue d’avoir lieu dans l’histoire. Toutefois, aucune révélation, ni avant, ni après le Christ, ne peut surpasser ni égaler celle qui a été accordée en Jésus-Christ, le Fils divin incarné » (Dupuis, 379).



Texte 15  « Il y a place pour une complémentarité de la Parole de Dieu, non seulement entre les deux Testaments de la Bible chrétienne, mais également entre les Ecritures bibliques et  non bibliques. Ces dernières peuvent contenir des aspects du mystère divin que la Bible, Nouveau Testament compris, ne met pas en valeur de façon égale. » (Dupuis 383).


Texte 16  « Il bénit la pluralité des races, des langues, des cultures et des nations. Et si les religions sont toujours enracinées dans des cultures, il est permis de penser qu’il bénit aussi la pluralité des formes religieuses. C’est comme si la pluralité des cultures, des langues, des nations et des religions était nécessaire pour traduire les richesses multiformes du mystère de Dieu. L’événement de la Pentecôte était justement la manifestation du même Dieu unique à partir d’une humanité plurielle. Mais dans l’attente du retour du Christ, Dieu continue à se raconter ‘à plusieurs reprises et de bien des manières’, à partir du devenir de tous les peuples de la terre » (Geffré, 134). 



Texte 17 « Le dialogue est le triomphe sur le monologue, sur l’autosuffisance de l’individu tendant s’absolutiser et qui par le dialogue s’inscrit dans la communauté humaine, renonçant à sa superbe arrogante d’être lui-même Dieu […] Le dialogue est un fait prophétique parce qu’il  fait place à autre chose que moi, à l’autre (minuscule) et ainsi également à l’Autre (majuscule) » (Siegwalt, 329). « L’affirmation monothéiste est, dans l’islam dans sa vérité (…) une affirmation prophétique par le fait qu’elle rejette toute idolâtrie, aussi toute théo-idolâtrie, laquelle est la réduction de Dieu à une idole, à la disposition de l’être humain » (Gérard SIEGWALT, « Mohammed, prophète pour le christianisme ? », Le défi interreligieux. L’Église chrétienne, les religions et la société laïque. Ecrits théologiques I, Paris, Cerf, 2014, p. 333 (Section III). 