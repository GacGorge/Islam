\chapter{Une  théologie  universelle  et inclusive   }



\subsection{Eléments bibliographiques}

AVELINE, J.-M., \emph{L'enjeu christologique en théologie des religions.
Le débat Tillich -- Troeltsch}, Paris, Cerf, 2003.

CLÉMENT D'ALEXANDRIE, \emph{Stromates I à V}, Sources Chrétiennes,
Paris. DANIÉLOU, J., \emph{Essai sur le mystère de l'histoire}, Paris
1953.

DANIÉLOU, J., \emph{Le mystère du salut des nations}, Paris 1948.

DUPUIS, J. \emph{Vers une théologie chrétienne du pluralisme religieux},
tr. par O. PARACHINI, Paris 1997.

FÉDOU, M., \emph{La voie du Christ, Genèses de la christologie dans le
contexte religieux de} l'Antiquité du IIe siècle au début du IVe siècle,
Paris 2006.

IRENÉE DE LYON, \emph{Contre les Hérésies. Dénonciation et réfutation de
la prétendue gnose au nom menteur}, tr. par A. ROUSSEAU, Paris
1991\textsuperscript{3}.

JUSTIN, \emph{Œuvres complètes}, Paris 1994.

LUBAC, H. (de), \emph{Le fondement théologique des missions}, Paris

ORIGENE, \emph{Traité des principes I}, Sources Chrétiennes 252, Paris
1978. 

POUDERON, B., \emph{Les apologistes grecs du II\textsuperscript{e}
siècle}, Paris 2005.

RÖMER, T., \emph{L'invention de Dieu}, Paris 2014.

THEOBALD, C., \emph{Selon l'Esprit de sainteté. Genèse d'une théologie
systématique}, Paris 2015. 

TROELTSCH, E., \emph{Histoire des religions
et destin de la théologie, Œuvres III}, tr. de J.-M.

TETAZ, Paris -- Genève, Cerf -- Labor et fides, 1996.

 
%------------------------------------------
\hypertarget{introduction}{%
\section{Introduction}\label{introduction}}

Dire \emph{Dieu est Unique}, il nous faut tenir aussi que \emph{Rien n'échappe à Dieu}
  

  
\paragraph{Discours de Paul à Athènes}

\begin{quote}
Ac 17,24-28 : « Le Dieu qui a créé l'univers et tout ce qui s'y trouve
(\ldots) donne à tous la vie et le souffle, et tout le reste (\ldots) A
partir d'une seul homme il a créé tous les peuples pour habiter toute la
surface de la terre, il a défini des temps fixes et tracé des limites de
l'habitat des hommes ; c'était pour qu'ils cherchent Dieu ; peut-être
pourraient-ils le découvrir en tâtonnant, lui qui, en réalité, n'est pas
loin de chacun de nous car c'est en lui que nous avons la vie, le
mouvement et l'être, comme l'ont dit certains de vos poètes : `car nous
sommes de sa race' ».
\end{quote}
   
   L'exclusivité de Dieu est aussi son inaccessibilité, du fait de son universalité.
   \subparagraph{première partie} Il fait référence à Epiménide, vision philosophique ajoutée à la vision théologique.
   \subparagraph{Car nous tirons de lui notre origine} Aratos.
    
    
On essaye de chercher en l'autre comme Dieu se manifeste.
%------------------------------------------
\section{La vision universaliste dans l'AT et le NT}




    \subsection{La théorie des trois étapes de la manifestation de Dieu}
 
    
Une autre façon de réagir à l'exil se retrouve dans la réaction sacerdotale. 

\paragraph{Rappel de la théorie documentaire} Willhausen a découpé en 4 les textes de l'AT : 
\begin{itemize}
    \item Elohiste
    \item Yahwiste
    \item Sacerdotale
    \item deuteronome
\end{itemize}
Cette approche a été remise en cause (en particulier par Romer).


Pendant et après l'exil, on a rassemblé les textes pour répondre à la crise. Les composantes traditionnelles ont proposées une réponse \textit{sacerdotale}, aller aux sources. Ecrit par le milieu des prêtres, au moment de l'exil. Relire à partir de l'exil de comment les textes sont écrits : 

\begin{quote}
« Pour le milieu sacerdotal, seul compte \textit{le temps des origines} (origine
du monde, temps des Patriarches et de Moïse). (\ldots) Pour lui, tout
est donné, établi dès les origines : l'interdit de consommer le sang
(\ldots), la circoncision (\ldots), la Pâque (\ldots) ainsi que les lois
rituelles et sacrificielles, et tout est révélé au peuple dans le désert
par l'intermédiaire de Moïse. » (Römer\sn{Historien, il fait des hypothèses. on peut ne pas être d'accord}, 296).
\end{quote}

Pour le milieu sacerdotal, tout a été révélé dès l'origine. 


\begin{quote}
« A l'opposé du discours deutéronomiste, qui insiste sur une ségrégation
stricte entre le peuple de Yhwh et les autres peuples, {le
milieu des prêtres présente un discours monothéiste} \textit{inclusif
qui cherche à définir la place et le rôle d'Israël et de Yhwh au milieu
de tous les peuples et de leurs dieux respectifs}. Dans ce
but, il développe, à l'aide des noms divins, `trois cercles' ou trois
étapes de la manifestation de Yhwh » (Römer, 297).
\end{quote}
 
 On n'affirme pas un Dieu contre les autres, mais un Dieu qui rassemble. Dans ce but, il développe trois noms divins (Ex 6,2-3):
 
 \begin{quote}
     « Elohim parla à Moïse et lui dit : je suis Adonaï ! Je suis apparu à Abraham, à, Isaac et à Jacob comme El Shaddaï, mais mon nom d’Adonaï ne leur a pas été connu ». (Ex. 6,2-3) 
 \end{quote}

Au lieu de penser que comme Willhausen que c'est trois couches d'écriture, Römer pense que c'est le seul rédacteur sacerdotal.

      \paragraph{Elohim} Pluriel de \textit{El}, nom propre d'un Dieu en phénicie et Cana. nom commun aussi. utilisé dans tout le monde sémitique. Yhwh se révèle comme Elohim. cf Gen 1 : \emph{Bereshit bara Elohim}. Tous les deiux peuvent être une manifestations du Dieu unique. Tous les peuples rendant un culte à un Dieu créateur révèrent sans le savoir un culte à Yhwh.
      Elohim est un pluriel, pas une survivance polytheiste, mais une manifestation que Dieu se manifeste sous differentes formes (?).
      
     
      \paragraph{El Shadday} El : le Dieu Shadday. Le Dieu des montagnes. La circoncision.
     
      
     
      \paragraph{Yhwh} le nom est révélé à Moise au buisson ardent (Ex 3, 1-15). une part mystérieuse. Envoi de Moise en mission, il n'abandonne pas son Peuple. un nom qui est mystérieux, pas un objet mais celui qu'on ne peut pas saisir. La Paque.
      
      \paragraph{Découplage du politique et du religieux} les trois termes sont donnés à Israël avant son organisation en pays, et donc il n'y a pas besoin d'un Etat pour vénérer Yhwh. Il y a un étalement de l'économie, et le véritable culte, celui d'Israel, avec la Pâque, n'exclue pas les autres formes de vénérations, que ce soit la circoncision.
      
      \begin{Synthesis}
      La pensée sacerdotale pense une théologie inclusive dont Israël serait le véritable culte
      \end{Synthesis}
      On peut discuter, il y a des tensions dans le texte biblique. Eviter d'unifier car il y a un surplus de sens qui permet de bouger.
      
      
  
   
    
    \subsection{Les quatre alliances scellées par Dieu}
    
    Jadis, les pères de l'Eglise avaient soulignés 4 alliances dans l'AT, avec une progression dans les Alliances : 
    
\begin{quote}
« Quatre alliances furent données à l'humanité : la première avant le
déluge, au temps d'Adam ; la seconde, après le déluge, au temps de Noé :
la troisième, qui est le don de la Loi, au temps de Moïse ; la quatrième
enfin, qui renouvelle l'homme et récapitule tout en elle, celle qui, par
l'Evangile, élève les hommes et leur fait prendre leur envol vers le
royaume céleste » (Irénée, \emph{Contre les Hérésies}, III, 11, 8).
\end{quote}
  

\paragraph{ Adam, Gen 1-5} Le terme Alliance n'est pas mentionné. Néanmoins, dans le Siracide : 
      \begin{quote}
          Si 17,1-2.12 : « Le Seigneur a tiré l’homme de la terre (...) Il a assigné aux hommes un nombre précis de jours et un temps déterminé (...) Il a conclu avec
eux une alliance éternelle et leur a fait connaître ses jugements ».
      \end{quote}
      De même en Jérémie : 
      \begin{quote}
          Jr 33,20-26 : « Ainsi parle Yahvé. Si vous pouvez rompre mon alliance avec le
jour et mon alliance avec la nuit, de sorte que le jour et la nuit n’arrivent plus
au temps fixé, mon alliance sera aussi rompue avec David mon serviteur, de
sorte qu’il n’aura plus de fils régnants sur son trône, ainsi qu’avec les lévites,
les prêtres qui assurent mon service ».
      \end{quote}
     
      
\paragraph{Noé, Gn 5-9} L'arc en ciel. Une alliance universelle. \pageref{Alterite}. Danielou avait souligné les Saints d'Israel \sn{DANIÉLOU, J., Le mystère du salut des nations, Paris 1948}. Les chrétiens sont pour les juifs dans l'Alliance Noashique. 
      \begin{quote}
          « Dieu parla ainsi à Noé et à ses fils : `Voici que j'établis mon
alliance avec vous et avec vos descendants après vous (\ldots) Et Dieu
dit : `Voici le signe de l'alliance que j'institue entre moi et vous et
tous les êtres vivants qui sont avec vous, pour les générations à venir
: je mets mon  arc dans la nuée et il deviendra un signe d'alliance entre moi et la
terre (\ldots) Quand l'arc sera dans la nuée, je le verrai et me
souviendrai de l'alliance éternelle qu'il y a entre Dieu et tous les
êtres vivants, en somme toute chair qui est sur la terre. (Gn
9,8-9.12-13.16).
      \end{quote}
     
      
\paragraph{Abraham, Gn 12 et Moïse}
      
  
\begin{Synthesis}
Dieu est unique donc il est universel. On part du plus large jusqu'à un peuple d'Israel. Mais une Alliance inclusive. 
\end{Synthesis}   
    
    \subsection{Une économie universelle : Parole, Sagesse et Esprit}
    
Dieu se tourne vers nous avec des manifestations, avec les trois attributs, paroles, Sagesse et Esprit. Peu à peu, on les a transformé en personne (\emph{hypostasiés}). Y-a-t-il une démarche universelle ?
  

     
      
      \paragraph{La Parole de Dieu.} \emph{Dabar}, les paroles de Dieu.
      
      \begin{quote}
          «  Ainsi  se  comporte  ma  Parole  du  moment  qu’elle  sort  de  ma  bouche  :  elle  ne  retourne  pas vers  moi  sans  résultat,  sans  avoir  exécuté  ce  qui  me  plaît  et  fait  aboutir  ce  pour  quoi  je  l’avais envoyée  »  (Is  55,11).   
      \end{quote}
      La parole est \textit{performative}, elle fait quelque chose. Ce n'est pas seulement la parole adressée à Israél mais à toute la création. 
     
      
      \paragraph{La Sagesse de Dieu.} Surtout dans les écrits sapientiaux, plus tardifs, au contact du monde hellenistique.
      \begin{quote}
      
  
22 L'Éternel m'a créée la première de ses oeuvres, Avant ses oeuvres les plus anciennes.

23 J'ai été établie depuis l'éternité, Dès le commencement, avant l'origine de la terre.

24 Je fus enfantée quand il n'y avait point d'abîmes, Point de sources chargées d'eaux;

          Pr 8
      \end{quote}
      La sagesse de Yhwh est personnifié et n'est pas limité à Israël.
     
      
      \paragraph{L'Esprit de Dieu.} \emph{Ruah}, souffle, vent, esprit.  
      
      \begin{quote}
      La terre était informe et vide: il y avait des ténèbres à la surface de l'abîme, et l'esprit de Dieu se mouvait au-dessus des eaux.
          Gn 1, 2 
      \end{quote}
      
      \begin{quote}
      30 Tu envoies ton souffle: ils sont créés, Et tu renouvelles la face de la terre.
          Ps 104, 30
      \end{quote}
      Energie vivifiante (Joel x)
      
      \begin{quote}
      7 L'Éternel Dieu forma l'homme de la poussière de la terre, il souffla dans ses narines un souffle de vie et l'homme devint un être vivant.
          Gn 2, 7
      \end{quote}
  Interprétation pas simple mais idée que Dieu donne la vie, donc côté universel.
  
  \begin{Synthesis}
  Dieu instaure un rapport avec le monde par sa parole, sa sagesse et son esprit.
  \end{Synthesis}
   
    
    \subsection{Un universalisme manifesté par Israël}
    
    Le Peuple Elu s'ouvre aussi au monde : 
    
    Jacques Dupuis :
    \begin{itemize}
        \item \textsc{Universalité d'attraction}, centralisée Is 2,1ss : toutes les nations afflueront vers Jérusalem. C'est plutôt cette dimension qui prime.
        \item Universalisme : Malachie 
        \begin{quote}
            Ml  1,11  :  «  Car  du  Levant  au  Couchant,  grand  est  mon  nom  parmi  les  nations.  En  tout  lieu  un sacrifice  d’encens  est  présenté  à  mon  nom,  ainsi  qu’une  offrande  pure,  car  grand  est  mon  nom parmi  les  nations, dit  le  Seigneur, le  tout-puissant  ?  ». 
        \end{quote}
    \end{itemize}
   
   Tension entre particularisme (singularité) du Peuple Elu et l'universalisme. Ce n'est pas parce qu'on est mis à part qu'il n'y a pas une dimension universaliste. Comment être notre propre nous-même tout en ouvert aux autres ? 
    
    \subsection{Comment les premières communautés chrétiennes ont-elle vécu la
    tension particularisme-universalité ?}
    
    Regardons maintenant le NT : 
    \paragraph{Jésus n'a pas été très universel} Il a été en Galilée et Judée. Visée de rassembler les 12 tribus d'Israël (cf decapole). Pas de notions d'aller \textit{évangéliser}.
    \begin{quote}
         Il répondit: Je n'ai été envoyé qu'aux brebis perdues de la maison d'Israël.   Mt 15,24 
    \end{quote}
  Le ministère de Jésus touche d'abord la conversion du peuple d'Israël. 
  
  \paragraph{Les premières Eglises vont se tourner vers les autres}
  \begin{quote}
      44 Comme Pierre prononçait encore ces mots, le Saint Esprit descendit sur tous ceux qui écoutaient la parole.
      Ac 10 Corneille
  \end{quote}
    L'Esprit agit en dehors de la communauté d'Israël.
    
\paragraph{Paul} se tourne résolument vers la Mission. 
\begin{quote}
    Ac 17
\end{quote}



     
      
\paragraph{Quelle attitude l'Église apostolique va-t-elle adopter vis-à-vis
      des autres
« religions » ?}


\paragraph{L'universalisme se lit en filigrane dans certains passages
  évangéliques}
  Il faut souligner que c'est le mystère pascal qui change la donne. Avec la résurrection, la bonne nouvelle devient universelle.
\paragraph{avant la résurrection, Accueillir les étrangers} Jésus n'était pas fermé aux étrangers. 
\paragraph{Après la résurrection, aller vers les Nations} 
\begin{quote}
    Mt 28 Allez et de toutes les nations, faites des disciples
\end{quote}
 
  \paragraph{Les samaritains}
  
 Dupuis souligne le rôle des samaritains, "Esprit et Vérité". Lc 17, le samaritain lépreux. 
  

 %-----------------------------------------
  \section{La « christologisation » de l'universalisme
  biblique} 



   
    
    \subsection{Le problème}
    
   Loisy  : "Jésus a annoncé la Bonne Nouvelle et c'est l'Eglise qui est arrivé".
   
   Comment penser le Christ universel dans les autres Religions. Comment reconnaître la présence du Christ alleurs, alors qu'il s'est incarné en Judée.
   
   \begin{quote}
   22 Paul, debout au milieu de l'Aréopage, dit: Hommes Athéniens, je vous trouve à tous égards extrêmement religieux.

23 Car, en parcourant votre ville et en considérant les objets de votre dévotion, j'ai même découvert un autel avec cette inscription: A un dieu inconnu! Ce que vous révérez sans le connaître, c'est ce que je vous annonce.
24 Le Dieu qui a fait le monde et tout ce qui s'y trouve, étant le Seigneur du ciel et de la terre, n'habite point dans des temples faits de main d'homme;
25 il n'est point servi par des mains humaines, comme s'il avait besoin de quoi que ce soit, lui qui donne à tous la vie, la respiration, et toutes choses.
26 Il a fait que tous les hommes, sortis d'un seul sang, habitassent sur toute la surface de la terre, ayant déterminé la durée des temps et les bornes de leur demeure;
27 il a voulu qu'ils cherchassent le Seigneur, et qu'ils s'efforçassent de le trouver en tâtonnant, bien qu'il ne soit pas loin de chacun de nous,
28 car en lui nous avons la vie, le mouvement, et l'être. C'est ce qu'ont dit aussi quelques-uns de vos poètes: De lui nous sommes la race...
29 Ainsi donc, étant la race de Dieu, nous ne devons pas croire que la divinité soit semblable à de l'or, à de l'argent, ou à de la pierre, sculptés par l'art et l'industrie de l'homme.
30 Dieu, sans tenir compte des temps d'ignorance, annonce maintenant à tous les hommes, en tous lieux, qu'ils aient à se repentir,
31 parce qu'il a fixé un jour où il jugera le monde selon la justice, par l'homme qu'il a désigné, ce dont il a donné à tous une preuve certaine en le ressuscitant des morts...
32 Lorsqu'ils entendirent parler de résurrection des morts, les uns se moquèrent, et les autres dirent: Nous t'entendrons là-dessus une autre fois.
       Ac 17, 21-32
   \end{quote}

    
    \subsection{Le Logos : un pont entre la tradition biblique et
    l'hellénisme}

    

  

     
      
      \paragraph{L'idée de Logos dans le NT}
      
      Premier pont fait d'un monde à l'autre. Apparaît peu dans le NT (5 fois dans le prologue de Jn, une fois dans une lettre de Jn, une fois dans l'Ap.).
    Prologue : lié à la littérature sapientelle de l'AT.
      
     
      
      \paragraph{La convergence du christianisme et de la philosophie grecque}
     
         Cette dimension médiatrice pour la création est importante bien sûr pour le caractère universel du logos qui va être mobilisé par les pères apologistes dans leur dialogue avec le monde culturel grec les apologistes ce sont des des pères de l'église mais du 2e et 3e siècle. c'est souvent et les gens des intellectuels formés un philosophie grecque qui vont essayer de acclimater de transmettre la foi dans un langage philosophique les premiers témoins de veut dire témoin au niveau de des intellectuels de ce passage de cette inculturation alors avec plus ou moins de bonheur  
  
   \begin{Def}[Logos chez les Stoiciens]
    Pour les stoiciens, le logos est l'ordre du monde, ce qui organise le monde.
   \end{Def}
   De plus, on va utiliser l'idée néoplatonicien d'un être intermédiaire entre le Dieu un et l'universalité du monde.  
    
    \subsection{Universalité du Logos et le thème des « semences du Verbe »}
    

  

     
      
       \paragraph{Le thème du Logos disséminé : Justin et les autres}
       Justin, école de Philosophie à Rome. 
      
     \begin{quote}
         «  Ceux  qui  ont  vécu  selon  le  Logos  sont  chrétiens,  même  s’ils  ont  été  tenus  pour  athées, comme  par  exemple,  chez  les  Grecs,  Socrate,  Héraclite,  et  d’autres  pareils,  et,  chez  les Barbares,  Abraham,  Ananias,  Azarias,  Misaël,  Elie  et  quantité  d’autres,  dont  nous  renonçons pour  l’instant  à  énumérer  les  œuvres  et  les  noms  (…).  Dès  lors  aussi,  ceux  qui,  parmi  les hommes  des  temps  passés,  ont  vécu  loin  du  Logos,  furent  mauvais,  ennemis  du  Christ, meurtriers  de  ceux  qui  vivaient  selon  le  Logos,  tandis  que  ceux  qui  ont  vécu  et  qui  vivent encore  selon  le  Logos  sont  chrétiens,  sans  crainte  et  sans  inquiétude  »  (Justin,  Apologie,  46,34).   
     \end{quote}
      Intéressant de voir l'universalité du Christ et toute personne peut être touché. 
    
      
      L'idée de semence, la parole est semée partout mais selon les terrains cela lève plus ou moins bien.
      \begin{quote}
      Tout ce qui a été dit de bon dans chaque école, c'est ce que j'appelle philosophie. Clément, Stromates
          
      \end{quote}
      Il est l'intelligence de Dieu. Deuxièmement il est universel parce que c'est lui qui par lui que tout est créé l'ordre du monde. Troisièmement il va être progressivement manifesté dans l'histoire donc un aspect historique c'est à dire que on va découvrir qui est le logos qui va se rendre visible. 
 

L'homme lui-même a un logos, une raison, et peut donc suivre le Christ. 
  ~
  
       \paragraph{Une économie du Logos}
       Plus on se tourne vers le logos, plus on se tourne vers Dieu. Analogie entre la philosophie pour les paiens, préparatoire, propédeutique, comme la Loi pour les juifs. 
        \begin{quote}
      «  Avant  l’avenue  du  Seigneur  la  philosophie  était  indispensable  aux  grecs  pour  les  conduire  à la  justice  ;  maintenant  elle  devient  utile  pour  les  conduire  à  la  vénération  de  Dieu.  Elle  sert  de formation  préparatoire  aux  esprits  qui  veulent  gagner  leur  foi  par  la  démonstration  (…).  Dieu est  la  cause  de  toutes  les  bonnes  choses,  des  unes  immédiatement  et  pour  elles-mêmes, comme  de  l’AT  et  du  NT,  des  autres  par  corollaire,  comme  de  la  philosophie.  Peut-être  même la  philosophie  a-t-elle  été  donnée  elle  aussi  comme  un  bien  direct  aux  Grecs,  avant  que  le Seigneur  eût  élargi  son  appel  jusqu’à  eux  :  car  elle  faisait  leur  éducation,  tout  comme  la  Loi celle  des  juifs,  pour  aller  au  Christ.  La  philosophie  est  un  travail  préparatoire  ;  elle  ouvre  la route  à  celui  que  le  Christ  rend  ensuite  parfait  (…).  Il  n’y  a,  certes,  qu’une  route  de  la  vérité, mais  elle  est  comme  un  fleuve  intarissable,  vers  lequel  débouchent  les  autres  cours  d’eau venus  d’une  peu partout  »  (Clément  d’Alexandrie,  Stromates,  I, 5,1-3). 
  \end{quote}
  
  
  La progressivité de la Révélation : les théophanies de l'AT sont des logophanies, des manifestations du Logos, par exemple c'est le logos qui se promenait dans le Jardin et parlait avec Adam, Mambré, phileuxelia, c'est le \textit{logos}, Moise : verbe qui parle à Moise \textit{comme à un ami}.
  \begin{quote}
      «  Si  (…)  la  venue  du  Roi  est  annoncée  par  avance  par  les  serviteurs  que  l’on  envoie,  c’est pour  la  préparation  de  ceux  qui  auront  à  accueillir  leur  Seigneur.  Mais  lorsque  le  Roi  est arrivé,  que  ses  sujets  ont  été  remplis  de  la  joie  annoncée,  qu’ils  ont  reçu  de  lui  la  liberté,  qu’ils ont  bénéficié  de  sa  vue,  entendu  ses  paroles  et  joui  de  ses  dons,  alors  du  moins  pour  les  gens sensés,  ne  se  pose  plus  la  question  de  savoir  ce  que  le  Roi  a  apporté  de  nouveau  par  rapport  à ceux qui  annoncèrent  sa  venue  »  (Irénée,  Contre  les  hérésies,  VI, 34, 1). 
  \end{quote}
  
  Les logophanies sont des anticipations des christophanies, le verbe fait chair.
  
  \begin{quote}
      dimension anticipée. 
  \end{quote}
 %-----------------------------------------
  \section{L'universalisme chrétien à l'épreuve du salut} 

On va remonter vers nous dans l'histoire. 

    \subsection{Universalité de l'annonce de l'Evangile et non-universalité du
    salut}
    Dans les premiers siècles du Christianisme, on essayait de penser la dimension du Christ dans une logique d'Evangélisation. Mais à partir du moment où le Christianisme est religion d'Etat, on pense que \textit{le salut n'est pas universel car on l'a annoncé à tous}. Le salut effectif n'est pas universel, à partir du moment où on n'accepte pas le logos annoncé par l'Eglise.
    
    \begin{itemize}
        \item Avant JC : le Christ est présent par \textit{anticipation} par la figure des prophètes : on pouvait être saint car on croit à celui qui vient.
        \item Après JC : Le Christ est rendu présent par l'Eglise et on a donc besoin d'entrer dans l'Eglise pour être sauvé.
    \end{itemize}
    
    \begin{quote}
        livre saint des Hébreux...
        
        Augustin (B1 ?)
    \end{quote}
   Avant le Christ, tout le monde pouvait se tourner vers le Christ.
    
    \begin{quote}
    Foi implicite avant l'arrivée du Christ.
        St Thomas d'Aquin, de veritate (B2 ?)
    \end{quote}
    Quelqu'un qui obeit à sa conscience, Dieu va se manifester directement à lui soit va lui envoyer un missionnaire.  Ceux qui ne croient pas en Christ, sont coupables car le Christ a été annoncé à tout le monde connu. Certains l'ont refusé intentionnellement. 
    
    \textit{Hors de l'Eglise, point de Salut} s'applique :
    \begin{quote}
        B3
        Concile de florence - Cantate Domino\sn{La bulle Cantate Domino (4 février 1442) et les enjeux éthiopiens du concile de Florence L’étude de la bulle d’union entre les Églises copte et romaine (Cantate Domino) montre que les Coptes recherchaient, entre autre, une prise de position du pape sur des litiges qui les opposaient, à cette époque, aux Éthiopiens, en particulier la célébration du double sabbat. Lettres et discours tenus par les envoyés orientaux confirment la concurrence entre les deux délégations. Au-delà des enjeux théologiques et rituels, les Coptes cherchaient à conforter la suprématie du patriarche d’Alexandrie sur l’Éthiopie, les Éthiopiens voulaient affirmer leur autonomie par rapport à ce même patriarche. Que la parole pontificale ait pu être sollicitée dans cette affaire prouve le prestige acquis, au xve siècle, par la papauté auprès des chrétiens d’Orient. Le peu de cas que fit le roi d’Éthiopie, Zär’ä Ya’eqob, des prescriptions d’Eugène IV, en démontre les limites.}
    \end{quote}
    
    \begin{quote}
        Ratzinger
    \end{quote}
    
    Ainsi les autres traditions ne sont pas des semences du verbe mais des groupements d'individus qui refusent le Verbe. Cette question du salut
    
    \subsection{Changement de paradigme : vers la possibilité du salut
    universel}
    

  

     
      
       \paragraph{La non culpabilité des indiens et la question d'un salut pour les
      « infidèles »}
      La découverte du Nouveau Monde : les indiens d'Amérique ne sont pas coupables. Influence considérable sur la perception du salut. Comment ces personnes peuvent elles être justifiées alors qu'elles n'ont pas entendues parler du Salut : \textit{ils n'ont pas pu refuser le Christ}
     
      
       \paragraph{L'humanisme et la remise en cause d'une vision de l'enfer}
      Sur un temps long, avec les Lumières, on s'éloigne de la \textit{Divine Comédie }de Dante. On n'accepte plus d'envoyer les gens en Enfer,  Au XX, il y a tellement de gens qui souffrent dans leur vie terrestre, que cela devient aberrant de les envoyer en Enfer; Dieu d'amour. 
  
   
    
    \subsection{La conséquence sur le regard chrétien vis-à-vis des autres
    traditions}
    
   
    On passe d'un salut individualiste ("il a fait le bien autour de lui") à une vision plus positive des grandes traditions religieuses comme positive, car cela fait vivre les gens. Eventuelle vision salvifique des autres traditions.
    
    
    
    \subsection{L'universalité du christianisme et les autres religions} 
    
    Au XIX-XX, deux facteurs :
    \begin{itemize}
        \item contexte du Christianisme au sommet avec la civilisation Européenne. \textsc{Supériorité}
        \item théologie plus historique, va voir de façon plus positive les autres religions. On redécouvre cette progressivité du salut, pas \textit{oui-non}. Influence de Hegel et de l'Idéalisme Allemand. \textsc{Dans l'histoire émerge la vraie religion}
    \end{itemize}
    
    Dans ce contexte se développe l'absoluité du christianisme.


 %-----------------------------------------
  \section{L' « absoluité » du christianisme dans la théologie
  protestante} 
 
 Pas forcément très connu en France, une vision idéaliste de la religion.  
 \begin{Def}[absoluité]
  Quelle est la religion idéale ? Qu'elle est l'essence de la religion ? et on compare les religions concrète à cette religion idéale. Hegel
 \end{Def}
  Le christianisme est placé dans une dimension historique.
  L'histoire comme une totalité qui va vers un aboutissement et une réalisation. Quelque chose émerge. 
  \begin{quote}
      Ernst Troeltsch. Va décrire cette approche
  \end{quote}
  Il y a cette religion idéale et elle est plus ou moins présente dans toutes les religions du monde. Une autre idée d'exprimer les semences du verbe. 
  
  
  \begin{quote}
      Troeltsch. 
  \end{quote}
  Ce que les autres ont de bien, le christianisme l'a déjà.
  
  
  \section{Les théologies catholiques de l'accomplissement au XXe siècle}
  
Le courant inclusiviste considère que les autres religions peuvent apporter une contribution à l'Eglise universelle (bienveillance). Retour aux semences du Verbe.


   
    
    \subsection{La théologie de Jean Daniélou}
    
Danielou, Grand théologien jésuite, mort en 1974, a fondé l'ISTR. Va s'interesser aux autres religions. 
  

     
      
      \paragraph{La théologie de l'histoire} Le Christianisme appartient à l'histoire comme les autres traditions religieuses. On peut aussi avoir une approche théologique de l'histoire. l'unité de l'histoire se fait théologiquement : 
      \begin{quote}
      
      Le christianisme n’est pas un moment mais la totalité de l’histoire. Avec lui, la fin est déjà là. Le Christianisme dit la fin, vers où l’histoire converge. … Alpha et Omega



\textit{Essai de théologie de l'histoire, 1953}
      \end{quote}
      
     
      
      \paragraph{« Christologisation » de l'histoire}
      Si l’histoire entre dans le plan de Dieu, elle doit être ordonnée christologiquement. Pas uniquement théocentrique de l’histoire mais christocentrique. 
     
      
      \paragraph{Les différentes phases dans de l'histoire du salut et la place des
      autres religions}
      
     
      
      \paragraph{Le christianisme accomplit les autres traditions}
      
  
   
    
    \subsection{La théologie de Henri de Lubac}
    
   
    
    \subsection{Le concile Vatican II}
    

 
  \section{Les limites des théologies de
  l'accomplissement}



   
    
    \subsection{Les critiques des théologiens pluralistes.}
    
   
    
    \subsection{Une critique de la théorie de l'accomplissement dans Vatican
    II}
    

 




