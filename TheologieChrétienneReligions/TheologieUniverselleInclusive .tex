\chapter{Une  théologie  universelle  et inclusive   }



\subsection{Eléments bibliographiques}

AVELINE, J.-M., \emph{L'enjeu christologique en théologie des religions.
Le débat Tillich -- Troeltsch}, Paris, Cerf, 2003.

CLÉMENT D'ALEXANDRIE, \emph{Stromates I à V}, Sources Chrétiennes,
Paris. DANIÉLOU, J., \emph{Essai sur le mystère de l'histoire}, Paris
1953.

DANIÉLOU, J., \emph{Le mystère du salut des nations}, Paris 1948.

DUPUIS, J. \emph{Vers une théologie chrétienne du pluralisme religieux},
tr. par O. PARACHINI, Paris 1997.

FÉDOU, M., \emph{La voie du Christ, Genèses de la christologie dans le
contexte religieux de} l'Antiquité du IIe siècle au début du IVe siècle,
Paris 2006.

IRENÉE DE LYON, \emph{Contre les Hérésies. Dénonciation et réfutation de
la prétendue gnose au nom menteur}, tr. par A. ROUSSEAU, Paris
1991\textsuperscript{3}.

JUSTIN, \emph{Œuvres complètes}, Paris 1994.

LUBAC, H. (de), \emph{Le fondement théologique des missions}, Paris

ORIGENE, \emph{Traité des principes I}, Sources Chrétiennes 252, Paris
1978. 

POUDERON, B., \emph{Les apologistes grecs du II\textsuperscript{e}
siècle}, Paris 2005.

RÖMER, T., \emph{L'invention de Dieu}, Paris 2014.

THEOBALD, C., \emph{Selon l'Esprit de sainteté. Genèse d'une théologie
systématique}, Paris 2015. 

TROELTSCH, E., \emph{Histoire des religions
et destin de la théologie, Œuvres III}, tr. de J.-M.

TETAZ, Paris -- Genève, Cerf -- Labor et fides, 1996.

 
%------------------------------------------
\hypertarget{introduction}{%
\section{Introduction}\label{introduction}}

Dire \emph{Dieu est Unique}, il nous faut tenir aussi que \emph{Rien n'échappe à Dieu}
  

  
\paragraph{Discours de Paul à Athènes}

\begin{quote}
Ac 17,24-28 : « Le Dieu qui a créé l'univers et tout ce qui s'y trouve
(\ldots) donne à tous la vie et le souffle, et tout le reste (\ldots) A
partir d'une seul homme il a créé tous les peuples pour habiter toute la
surface de la terre, il a défini des temps fixes et tracé des limites de
l'habitat des hommes ; c'était pour qu'ils cherchent Dieu ; peut-être
pourraient-ils le découvrir en tâtonnant, lui qui, en réalité, n'est pas
loin de chacun de nous car c'est en lui que nous avons la vie, le
mouvement et l'être, comme l'ont dit certains de vos poètes : `car nous
sommes de sa race' ».
\end{quote}
   
   L'exclusivité de Dieu est aussi son inaccessibilité, du fait de son universalité.
   \subparagraph{première partie} Il fait référence à Epiménide, vision philosophique ajoutée à la vision théologique.
   \subparagraph{Car nous tirons de lui notre origine} Aratos.
    
    
On essaye de chercher en l'autre comme Dieu se manifeste.
%------------------------------------------
\section{La vision universaliste dans l'AT et le NT}




    \subsection{La théorie des trois étapes de la manifestation de Dieu}
 
    
Une autre façon de réagir à l'exil se retrouve dans la réaction sacerdotale. 

\paragraph{Rappel de la théorie documentaire} Willhausen a découpé en 4 les textes de l'AT : 
\begin{itemize}
    \item Elohiste
    \item Yahwiste
    \item Sacerdotale
    \item deuteronome
\end{itemize}
Cette approche a été remise en cause (en particulier par Romer).


Pendant et après l'exil, on a rassemblé les textes pour répondre à la crise. Les composantes traditionnelles ont proposées une réponse \textit{sacerdotale}, aller aux sources. Ecrit par le milieu des prêtres, au moment de l'exil. Relire à partir de l'exil de comment les textes sont écrits : 

\begin{quote}
« Pour le milieu sacerdotal, seul compte \textit{le temps des origines} (origine
du monde, temps des Patriarches et de Moïse). (\ldots) Pour lui, tout
est donné, établi dès les origines : l'interdit de consommer le sang
(\ldots), la circoncision (\ldots), la Pâque (\ldots) ainsi que les lois
rituelles et sacrificielles, et tout est révélé au peuple dans le désert
par l'intermédiaire de Moïse. » (Römer\sn{Historien, il fait des hypothèses. on peut ne pas être d'accord}, 296).
\end{quote}

Pour le milieu sacerdotal, tout a été révélé dès l'origine. 


\begin{quote}
« A l'opposé du discours deutéronomiste, qui insiste sur une ségrégation
stricte entre le peuple de Yhwh et les autres peuples, {le
milieu des prêtres présente un discours monothéiste} \textit{inclusif
qui cherche à définir la place et le rôle d'Israël et de Yhwh au milieu
de tous les peuples et de leurs dieux respectifs}. Dans ce
but, il développe, à l'aide des noms divins, `trois cercles' ou trois
étapes de la manifestation de Yhwh » (Römer, 297).
\end{quote}
 
 On n'affirme pas un Dieu contre les autres, mais un Dieu qui rassemble. Dans ce but, il développe trois noms divins (Ex 6,2-3):
 
 \begin{quote}
     « Elohim parla à Moïse et lui dit : je suis Adonaï ! Je suis apparu à Abraham, à, Isaac et à Jacob comme El Shaddaï, mais mon nom d’Adonaï ne leur a pas été connu ». (Ex. 6,2-3) 
 \end{quote}

Au lieu de penser que comme Willhausen que c'est trois couches d'écriture, Römer pense que c'est le seul rédacteur sacerdotal.

      \paragraph{Elohim} Pluriel de \textit{El}, nom propre d'un Dieu en phénicie et Cana. nom commun aussi. utilisé dans tout le monde sémitique. Yhwh se révèle comme Elohim. cf Gen 1 : \emph{Bereshit bara Elohim}. Tous les deiux peuvent être une manifestations du Dieu unique. Tous les peuples rendant un culte à un Dieu créateur révèrent sans le savoir un culte à Yhwh.
      Elohim est un pluriel, pas une survivance polytheiste, mais une manifestation que Dieu se manifeste sous differentes formes (?).
      
     
      \paragraph{El Shadday} El : le Dieu Shadday. Le Dieu des montagnes. La circoncision.
     
      
     
      \paragraph{Yhwh} le nom est révélé à Moise au buisson ardent (Ex 3, 1-15). une part mystérieuse. Envoi de Moise en mission, il n'abandonne pas son Peuple. un nom qui est mystérieux, pas un objet mais celui qu'on ne peut pas saisir. La Paque.
      
      \paragraph{Découplage du politique et du religieux} les trois termes sont donnés à Israël avant son organisation en pays, et donc il n'y a pas besoin d'un Etat pour vénérer Yhwh. Il y a un étalement de l'économie, et le véritable culte, celui d'Israel, avec la Pâque, n'exclue pas les autres formes de vénérations, que ce soit la circoncision.
      
      \begin{Synthesis}
      La pensée sacerdotale pense une théologie inclusive dont Israël serait le véritable culte
      \end{Synthesis}
      On peut discuter, il y a des tensions dans le texte biblique. Eviter d'unifier car il y a un surplus de sens qui permet de bouger.
      
      
  
   
    
    \subsection{Les quatre alliances scellées par Dieu}
    
    Jadis, les pères de l'Eglise avaient soulignés 4 alliances dans l'AT, avec une progression dans les Alliances : 
    
\begin{quote}
« Quatre alliances furent données à l'humanité : la première avant le
déluge, au temps d'Adam ; la seconde, après le déluge, au temps de Noé :
la troisième, qui est le don de la Loi, au temps de Moïse ; la quatrième
enfin, qui renouvelle l'homme et récapitule tout en elle, celle qui, par
l'Evangile, élève les hommes et leur fait prendre leur envol vers le
royaume céleste » (Irénée, \emph{Contre les Hérésies}, III, 11, 8).
\end{quote}
  

\paragraph{ Adam, Gen 1-5} Le terme Alliance n'est pas mentionné. Néanmoins, dans le Siracide : 
      \begin{quote}
          Si 17,1-2.12 : « Le Seigneur a tiré l’homme de la terre (...) Il a assigné aux hommes un nombre précis de jours et un temps déterminé (...) Il a conclu avec
eux une alliance éternelle et leur a fait connaître ses jugements ».
      \end{quote}
      De même en Jérémie : 
      \begin{quote}
          Jr 33,20-26 : « Ainsi parle Yahvé. Si vous pouvez rompre mon alliance avec le
jour et mon alliance avec la nuit, de sorte que le jour et la nuit n’arrivent plus
au temps fixé, mon alliance sera aussi rompue avec David mon serviteur, de
sorte qu’il n’aura plus de fils régnants sur son trône, ainsi qu’avec les lévites,
les prêtres qui assurent mon service ».
      \end{quote}
     
      
\paragraph{Noé, Gn 5-9} L'arc en ciel. Une alliance universelle. \pageref{Alterite}. Danielou avait souligné les Saints d'Israel \sn{DANIÉLOU, J., Le mystère du salut des nations, Paris 1948}. Les chrétiens sont pour les juifs dans l'Alliance Noashique. 
      \begin{quote}
          « Dieu parla ainsi à Noé et à ses fils : `Voici que j'établis mon
alliance avec vous et avec vos descendants après vous (\ldots) Et Dieu
dit : `Voici le signe de l'alliance que j'institue entre moi et vous et
tous les êtres vivants qui sont avec vous, pour les générations à venir
: je mets mon  arc dans la nuée et il deviendra un signe d'alliance entre moi et la
terre (\ldots) Quand l'arc sera dans la nuée, je le verrai et me
souviendrai de l'alliance éternelle qu'il y a entre Dieu et tous les
êtres vivants, en somme toute chair qui est sur la terre. (Gn
9,8-9.12-13.16).
      \end{quote}
     
      
\paragraph{Abraham, Gn 12 et Moïse}
      
  
\begin{Synthesis}
On part du plus large jusqu'à un peuple d'Israel. Mais une Alliance inclusive.
\end{Synthesis}   
    
    \subsection{Une économie universelle : Parole, Sagesse et Esprit}
    

  

     
      
      La Parole de Dieu.
      
     
      
      La Sagesse de Dieu.
      
     
      
      L'Esprit de Dieu.
      
  
   
    
    \subsection{Un universalisme manifesté par Israël}
    
   
    
    \subsection{Comment les premières communautés chrétiennes ont-elle vécu la
    tension particularisme-universalité ?}
    

  

     
      
      Quelle attitude l'Église apostolique va-t-elle adopter vis-à-vis
      des autres
      
  




« religions » ?


\subsection{
  L'universalisme se lit en filigrane dans certains passages
  évangéliques}
  
 
  Les samaritains
  
 

 %-----------------------------------------
  \section{La « christologisation » de l'universalisme
  biblique} 



   
    
    \subsection{Le problème}
    
   
    
    \subsection{Le Logos : un pont entre la tradition biblique et
    l'hellénisme}
    

  

     
      
      L'idée de Logos dans le NT
      
     
      
      La convergence du christianisme et de la philosophie grecque
      
  
   
    
    \subsection{Universalité du Logos et le thème des « semences du Verbe »}
    

  

     
      
      Le thème du Logos disséminé : Justin et les autres
      
     
      
      Une économie du Logos
      
  

  ~

 %-----------------------------------------
  \section{L'universalisme chrétien à l'épreuve du
  salut} 



   
    
    \subsection{Universalité de l'annonce de l'Evangile et non universalité du
    salut}
    
   
    
    \subsection{Changement de paradigme : vers la possibilité du salut
    universel}
    

  

     
      
      La non culpabilité des indiens et la question d'un salut pour les
      « infidèles »
      
     
      
      L'humanisme et la remise en cause d'une vision de l'enfer
      
  
   
    
    \subsection{La conséquence sur le regard chrétien vis-à-vis des autres
    traditions}
    
   
    
    \subsection{L'universalité du christianisme et les autres religions}
    


 %-----------------------------------------
  \section{L' « absoluité » du christianisme dans la théologie
  protestante} 
 
  
  \section{Les théologies catholiques de l'accomplissement au XXe siècle}
  



   
    
    \subsection{La théologie de Jean Daniélou}
    

  

     
      
      La théologie de l'histoire
      
     
      
      « Christologisation » de l'histoire
      
     
      
      Les différentes phases dans de l'histoire du salut et la place des
      autres religions
      
     
      
      Le christianisme accomplit les autres traditions
      
  
   
    
    \subsection{La théologie de Henri de Lubac}
    
   
    
    \subsection{Le concile Vatican II}
    

 
  \section{Les limites des théologies de
  l'accomplissement}



   
    
    \subsection{Les critiques des théologiens pluralistes.}
    
   
    
    \subsection{Une critique de la théorie de l'accomplissement dans Vatican
    II}
    

 






 
\underline{Textes 6}

« Ainsi se comporte ma Parole du moment qu'elle sort de ma bouche : elle
ne retourne pas vers moi sans résultat, sans avoir exécuté ce qui me
plaît et fait aboutir ce pour quoi je l'avais envoyée » (Is 55,11).

\underline{Texte 7}

Ml 1,11 : « Car du Levant au Couchant, grand est mon nom parmi les
nations. En tout lieu un sacrifice d'encens est présenté à mon nom,
ainsi qu'une offrande pure, car grand est mon nom parmi les nations, dit
le Seigneur, le tout-puissant ? ».

\underline{Texte 8}

« Ceux qui ont vécu selon le Logos sont chrétiens, même s'ils ont été
tenus pour athées, comme par exemple, chez les Grecs, Socrate,
Héraclite, et d'autres pareils, et, chez les Barbares, Abraham, Ananias,
Azarias, Misaël, Elie et quantité d'autres, dont nous renonçons pour
l'instant à énumérer les œuvres et les noms (\ldots). Dès lors aussi,
ceux qui, parmi les hommes des temps passés, ont vécu loin du Logos,
furent mauvais, ennemis du Christ, meurtriers de ceux qui vivaient selon
le Logos, tandis que ceux qui ont vécu et qui vivent encore selon le
Logos sont chrétiens, sans crainte et sans inquiétude » (Justin,
\emph{Apologie}, 46,3- 4).

\underline{Texte 9}

« Avant l'avenue du Seigneur la philosophie était indispensable aux
grecs pour les conduire à la justice ; maintenant elle devient utile
pour les conduire à la vénération de Dieu. Elle sert de formation
préparatoire aux esprits qui veulent gagner leur foi par la
démonstration (\ldots). Dieu est la cause de toutes les bonnes choses,
des unes immédiatement et pour elles-mêmes, comme de l'AT et du NT, des
autres par corollaire, comme de la philosophie. Peut-être même la
philosophie a-t-elle été donnée elle aussi comme un bien direct aux
Grecs, avant que le Seigneur eût élargi son appel jusqu'à eux : car elle
faisait leur éducation, tout comme la Loi celle des juifs, pour aller au
Christ. La philosophie est un travail préparatoire ; elle ouvre la route
à celui que le Christ rend ensuite parfait (\ldots). Il n'y a, certes,
qu'une route de la vérité, mais elle est comme un fleuve intarissable,
vers lequel débouchent les autres cours d'eau venus d'une peu partout »
(Clément d'Alexandrie, \emph{Stromates}, I, 5,1-3).

\underline{Texte 10}

« Si (\ldots) la venue du Roi est annoncée par avance par les serviteurs
que l'on envoie, c'est pour la préparation de ceux qui auront à
accueillir leur Seigneur. Mais lorsque le Roi est arrivé, que ses sujets
ont été remplis de la joie annoncée, qu'ils ont reçu de lui la liberté,
qu'ils ont bénéficié de sa vue, entendu ses paroles et joui de ses dons,
alors du moins pour les gens sensés, ne se pose plus la question de
savoir ce que le Roi a apporté de nouveau par rapport à ceux qui
annoncèrent sa venue » (Irénée, \emph{Contre les hérésies}, VI, 34, 1).
