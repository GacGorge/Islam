\chapter{Les religions comme des cultures }

\mn{le 9/5/22}

\section{Bibliographie}
CHENO, R., Dieu au pluriel. Penser les religions, Cerf, Paris, 2017. 

CDF, Dominus Iesus, 2000. 

CTI, Le christianisme et les religions, 1997 

DUBUISSON, D., L’invention des religions, Paris, CNRS éditions, 2020. 

DUPUIS, J. Vers une théologie chrétienne du pluralisme religieux, Cerf, Paris, 1997. 

DURAND, M-L., « Le rapport Église/peuple juif comme paradigme du rapport aux autres religions et quasi-religions séculières ? » dans E. PISANI (dir.), Maximum illud. Aux sources d’une nouvelle ère missionnaire, Paris, Cerf, 2020, p. 141-152. 

DURKHEIM, E., Les formes élémentaires de la vie religieuse, PUF, Paris, 20086. 

ELIADE, M., Aspects du mythe, Gallimard, Paris, 1963. 

FEDOU, M., Les religions selon la foi chrétienne, Paris, Cerf, 1996. 

GUÉ, X., « Comment la théologie postlibérale pense-t-elle l’universalité des vérités religieuses ? Une relecture de La nature des doctrines de G. Lindbeck » dans E. PISANI (dir.), Les doctrines religieuses sont-elles condamnées à s’opposer ? Actes du colloque de l’ISTR des 6 et 7 février 2020, Paris, Cerf, 2021, 25-51. 

JEAN-PAUL II, Lettre encyclique Redemptoris missio, 1990. 

LINDBECK, G. A., La nature des doctrines. Religion et théologie à l’âge du postlibéralisme, tr. de M. HEBERT, Paris, Van Dieren éditeur, 2002. 

PHILIPS, l’Église et son mystère au IIe concile du Vatican, t. 1, Desclée, Paris 1967 

SALENSON, C., « La Missio Dei » dans E. PISANI (dir.), Maximum illud. Aux sources d’une nouvelle ère missionnaire, Paris, Cerf, 2020, p. 125-140. 

VILLEMIN, L. et CHEVALLIER, G. , « La distinction « incorporé à / ordonné à » dans Lumen Gentium : quelles conséquences pour la compréhension du rapport Eglise / Royaume ? », in Christoph THEOBALD (dir.), Pourquoi l’Eglise ? La dimension ecclésiale de la foi dans l’horizon du salut, Paris : Bayard, 2014, pp. 165-196. 

WILLAIME, J-P., Sociologie des religions, PUF, Paris, 20125. 



\section{Introduction}
\paragraph{la religion comme socialisation} en répondant aux grandes questions

\begin{quote}

À notre époque où le genre humain devient de jour en jour plus étroitement uni et où les relations entre les divers peuples se multiplient, l’Église examine plus attentivement quelles sont ses relations avec les religions non chrétiennes. Dans sa tâche de promouvoir l’unité et la charité entre les hommes, et aussi entre les peuples, elle examine ici d’abord ce que les hommes ont en commun et qui les pousse à vivre ensemble leur destinée.

Tous les peuples forment, en effet, une seule communauté ; ils ont une seule origine, puisque Dieu a fait habiter tout le genre humain sur toute la face de la terre [1] ; ils ont aussi une seule fin dernière, Dieu, dont la providence, les témoignages de bonté et les desseins de salut s’étendent à tous [2], jusqu’à ce que les élus soient réunis dans la Cité sainte, que la gloire de Dieu illuminera et où tous les peuples marcheront à sa lumière [3].

Les hommes attendent des diverses religions la réponse aux énigmes cachées de la condition humaine, qui, hier comme aujourd’hui, agitent profondément le cœur humain : Qu’est-ce que l’homme? Quel est le sens et le but de la vie? Qu’est-ce que le bien et qu’est-ce que le péché? Quels sont l’origine et le but de la souffrance? Quelle est la voie pour parvenir au vrai bonheur? Qu’est-ce que la mort, le jugement et la rétribution après la mort ? Qu’est-ce enfin que le mystère dernier et ineffable qui embrasse notre existence, d’où nous tirons notre origine et vers lequel nous tendons ?
    Nostra aetate 1
\end{quote}

\subsection{Mircea Eliade et la spécificité du phénomène religieux }

\paragraph{Le phénomène religieux }

\begin{quote}
    « Le mythe raconte une histoire sacrée ; il relate un événement qui a eu lieu dans le temps primordial, le temps fabuleux des ‘commencements’. Autrement dit, le mythe raconte comment, grâce aux exploits des Etres Surnaturels, une réalité est venue à l’existence (…). C’est donc toujours le récit d’une ‘création’ : on rapporte comment quelque chose a été produit, a commencé à être. Le mythe ne parle que de ce qui est arrivé réellement, de ce qui s’est pleinement manifesté (…). En somme, les mythes décrivent les diverses, et parfois dramatiques, irruptions du sacré (ou du ‘sur-naturel’) dans le Monde. C’est cette irruption du sacré qui fonde réellement le Monde et qui le fait tel qu’il est aujourd’hui. Plus encore : c’est à la suite des interventions des Etres Surnaturels que l’homme est ce qu’il est aujourd’hui, un être mortel, sexué et culturel » (Eliade, Aspects du mythe, 16-17). 
\end{quote}

\paragraph{Le mythe à l’origine de la culture et de la société }

\begin{quote}
    « Du fait que le mythe relate les gesta des Etres Surnaturels et la manifestation de leurs puissances sacrées, il devient le modèle exemplaire de toutes les activités humaines significatives » (Eliade, Aspects, 17-18).  
\end{quote}

\begin{quote}
    « Pour l’homme des sociétés archaïques (…) ce qui s’est passé ab origine est susceptible de se répéter par la force des rites. L’essentiel est donc, pour lui, de connaître les mythes. Non seulement parce que les mythes lui offrent une explication du Monde et de son propre mode d’exister dans le Monde, mais surtout parce que, en se les remémorant, en les réactualisant, il est capable de répéter ce que les Dieux, les Héros ou les Ancêtres ont fait a l'origine. Connaître les mythes, c’est apprendre le secret de l’origine des choses. En d’autres termes, on apprend non seulement comment les choses sont venues à l’existence, mais aussi où les trouver et comment les faire réapparaître lorsqu’elles disparaissent » (Eliade, Aspects, 26). 
\end{quote}

\subsection{Durkheim et la « sociogenèse » de la religion }

Thèse sur le totem en Australie, \textit{idéalisation de la société}. 
Les formes élémentaires de la religion

\paragraph{Distinction entre profane et le sacré} On se comporte différemment. La religion génère du sacré : 
\begin{quote}
    « Toutes les croyances religieuses connues (….) supposent une classification  des choses (…) que se représentent les hommes, en deux classes, en deux genres opposés (…) le profane et le sacré. (…) Les croyances, les mythes (…) sont ou des représentations ou des systèmes de représentations qui expriment la nature des choses sacrées, les vertus et les pouvoirs qui leur sont attribués, leur histoire, leurs rapports les unes avec les autres et avec les choses profanes » (Durkheim, 50-51).  
\end{quote}

Ce sont les hommes qui ont séparés ces deux mondes. Le monde séparé entre profane et religieux.

\paragraph{Définition de la religion}
\begin{quote}
« Une religion\sn{Durkheim, Les formes élémentaires
de la vie religieuse, 65} est un système solidaire de croyances et de pratiques relatives à des choses sacrées, c’est-à-dire séparées, inter-
dites, croyances et pratiques qui unissent en une même communauté
morale, appelée Église, tous ceux qui y adhèrent »
\end{quote}

\paragraph{Distinction entre Religion et Magie}
La magie ne fait pas \textit{corps}, \textit{Eglise}. Les clients d'un magicien peuvent n'avoir aucun lien entre eux. 
\begin{quote}
    « Les croyances proprement religieuses sont toujours communes à une collectivité déterminée qui fait profession d’y adhérer et de pratiquer les rites qui en sont solidaires. (…) \textbf{(Les croyances) sont la chose du groupe et elles en font l’unité.} Les  individus qui composent (la collectivité) se sentent liés les uns aux autres, par cela seul qu’ils ont une foi commune. Une société dont les membres sont unis parce qu’ils se représentent de la même manière le monde sacré et ses rapports avec le monde profane, et parce qu’ils traduisent cette représentation commune dans des pratiques identiques, c’est ce qu’on appelle une Église » (Durkheim, 60). 
\end{quote}

La religion génère les groupements, elle est par nature collective.


\paragraph{la société est la source de la religion} Pourquoi ont ils fait la distinction entre profane et sacré ? Alors que le sensible ne permet pas de faire cette distinction ?

\begin{quote}
    « Nous avons montré quelles forces morales (la société) développe et comment elle éveille ce sentiment d’appui, de sauvegarde, de dépendance tutélaire qui attache le fidèle à son culte. C’est elle qui l’élève au-dessus de lui-même : c’est même elle qui le fait. Car ce qui fait l’homme, c’est cet ensemble de bien intellectuels qui constitue la civilisation, et la civilisation est l’œuvre de la société.[…] Pour que les principaux aspects de la vie collective aient commencé par n’être que des aspects variés de la vie religieuse, il faut évidemment que la vie religieuse soit la forme éminente et comme une expression raccourcie de la vie collective tout entière. \textsc{Si la religion a engendré tout ce qu’il y a d’essentiel dans la société, c’est que l’idée de la société est l’âme de la religion} » (Durkheim, 598-599). 
\end{quote}

\begin{Synthesis}
la société élève l'homme, qui permet de se dépasser. 
\end{Synthesis}



\begin{quote}
    « Pour que la société puisse prendre conscience de soi et entretenir, au degré d’intensité nécessaire, le sentiment qu’elle a d’elle-même, il faut qu’elle s’assemble et se concentre. Or, cette concentration détermine une exaltation de la vie morale qui se traduit par un ensemble de conceptions idéales où vient se peindre la vie nouvelle qui s’est ainsi éveillé (…) Une société ne peut ni se créer ni se recréer sans, du même coup, créer l’idéal. (…) La société idéale n’est pas en dehors de la société réelle ; elle en fait partie. (…)\textsc{ Car une société n’est pas simplement constituée par la masse des individus qui la composent (…) mais, avant tout, par l’idée qu’elle se fait d’elle-même.} »(Durkheim, 603-604). 
\end{quote}

\begin{Def}[Religion]
On va sacraliser la société, une hypostase qui a une conscience.  Appartenir à cette société hypostasiée. Elle génère la religion. 
\end{Def}

Et comment fait on dans une société pluraliste ?
\begin{Ex}
Lors de la séparation de l'Eglise et de l'Etat, la société a essayé de singer la religion, avec l'instituteur comme \textit{prêtre}.
\end{Ex}

\paragraph{La religion comme un facteur de cohésion et une force sociale } Force sociale pour supporter les épreuves.
\begin{quote}
    « Le fidèle qui a communié avec son dieu n’est pas seulement un homme qui voit des vérités nouvelles que l’incroyant ignore ; c’est un homme qui peut davantage. Il sent en lui plus de force soit pour supporter les difficultés de l’existence, soit pour les vaincre » (Durkheim, 595).  
\end{quote}

\paragraph{Critique de la théorie sociologique de Durkheim } Première critique, Dieu n'est pas pris en charge dans la religion. Dieu serait une projection de la société. Feuerbach : 
\begin{quote}
    projection d'une... divinisé
\end{quote}

Par ailleurs, la religion est une instance critique de la société. On ne peut pas la réduire à une simple projection idéale de la société. Et comment on fait en cas de société pluri-confessionnelle ?


\begin{quote}
    « Les hommes vivent sur la même planète, mais dans des mondes différents. Chaque culture (…) conçoit en effet le monde à sa manière » (Dubuisson, 211). 
\end{quote}

\begin{quote}
    « Toutes les cultures, toutes les sociétés ont conçu le monde, l’univers à leur manière. Et (…) toujours en ont imaginé la genèse dans un mythe ou une série de mythes cosmogoniques. Chacune de ces conceptions est différente des autres, mais toutes se ressemblent en ce sens où chaque culture vit (…) dans son propre monde et que ces mondes remplissent partout les mêmes fonctions » (Dubuisson, 212). 
\end{quote}

La langue est toujours dans une culture, et la religion est aussi insérée.

\section{L’approche théologique post-libérale des religions }

\subsection{La critique de l’idée d’une essence commune à toutes les religions}
\paragraph{La remise en cause question d’un « lieu tiers } jusqu'à présent, on a essayé de trouver un PPDC : expérience spirituelle,...

\begin{quote}
    « Il suffirait, selon les pluralistes, de débarrasser chaque religion de son appareil dogmatique et symbolique propre pour retrouver ce lieu tiers, cet arrière-plan commun. Le théologien pluraliste, c’est celui qui sait dépasser sa propre tradition pour s’enraciner dans ce lieu tiers d’où il observe toutes les religions et contemple leur convergence » (Chéno, 111-112). 
\end{quote}

Il s'agit de changer cette approche et de se remettre au sein de notre identité, dans notre religion.
De plus, cette mise en aplomb, trouvant un terrain commun, est une version occidentale, mais qui ne respecte pas forcément l'autre.


\paragraph{Le tournant opéré par G. Lindbeck  } Un des initiateurs du \textit{post-libéralisme}, contre le libéralisme (protestant). Le courant libéral dans le catholicisme était attenué par la force du magistère.
Lindbeck a travaillé au concile.  En 1984, il publie la \textit{nature des doctrines}. \textit{Comment on conçoit la vérité en Christianisme ?}

 
 \begin{quote}
     « On considère (…) les religions comme des idiomes différents permettant d’interpréter la réalité, d’exprimer l’expérience et d’organiser la vie (…) Ainsi, les questions qui se posent quand on compare les religions concernent tout d’abord l’adéquation de leurs catégories » (Lindbeck p. 55). 
 \end{quote}
 Mon voisin fait une expérience ineffable mais va le dire de façon différente. Quelle est la vérité noumenale ? Efficacité de leur symbole : mesurera la qualité de la répone. 
  \paragraph{Approche culturo linguisitique}
 

  Le religions sont comme des archipels.
 
 \subsection{Les religions sont incommensurables quant à leur contenu}



 
 \paragraph{Dialoguer entre religion est il pertinent ? }
 
  \begin{quote}
     Qu’une religion soit raisonnable [donc universelle] dépend largement de ses pouvoirs d’assimilation, de sa capacité à fournir dans ses propres termes une interprétation intelligible des diverses situations et réalités que rencontrent ses adhérents. Les religions que nous qualifions de primitives échouent régulièrement à ce test quand elles sont confrontées à des changements importants, tandis que les religions mondiales développent de plus grandes ressources pour faire face aux vicissitudes (Lindbeck, 175). 
 \end{quote}
 
 Toute religion est comme une langue et donc universelle.
 
  \subsection{Les religions sont comparables quant à leur fonction}
  
  Lindbeck ne se met dans une confession particulière : ni chrétienne, ni théologique, mais scientifique et sociologique. Décentrement. 
  Il les définit comme : 
  
   \begin{quote}
     « Les religions […] ressemblent à des langues, ce qui les assimile à des cultures (dans la mesure où ces cultures sont comprises de manière sémiotique comme des systèmes de réalités et de valeurs, c’est-à-dire comme des idiomes qui permettent de construire la réalité et de vivre au quotidien). »  (Lindbeck, p. 16). 
 \end{quote}
  
  \begin{Def}[religion pour Lindbeck]
  Les religions […] ressemblent à des langues.
  \end{Def}
  
  Des mythes, des récits (ex : Gn 1 permet de comprendre le monde, moi-même). Par exemple, pour dire une conversion, on va reprendre le schéma de Charles de Foucauld ou Paul. Jamais pure innovation.
  \paragraph{Image des lunettes} Les religions sont des lunettes à travers lesquelles on voit le monde. 
  
  Ce n'est pas l'idée qui précède le langage, c'est le langage qui nous précède et nous utilisons le langage qui nous est disponible. 
  
  \paragraph{Fécondité performatrice des religions}
  En se référant à Luther, la vérité se vit parce qu'elle est reçue qui y adhère. 
  
  

\begin{quote}
    dire la vérité religieuse \ldots c'est s'engager dans un style de vie.
\end{quote}
 Pas une définition théorique de la vérité, mais une vérité pratique.
  
  \subparagraph{Performatrice} \textit{Je promets de te rester fidèle}. 
  
  \subparagraph{Pour découvrir la tradition de l'autre, rencontrer quelqu'un qui essaye de conformer sa vie à sa tradition} Sinon, on risque d'avoir une mauvaise vision de l'autre.
  
  
  
  Si la religion permet de se comprendre, pourquoi il n'y aurait pas des religions plus universelles que d'autres ? Si on voit que certains sont heureux, il y a une dimension universelle ? Un aspect qui dépasse l'individu, qui serait \textsc{partageable}.
  
  \subparagraph{image de la langue} certaines langues nous ouvrent au monde, qui nous permette d'accueillir de nouvelles experiences du monde, et d'autres moins universelles que d'autres. C'est Lindberg qui dit cela, après avoir dit l'individualité.
  
  \begin{quote}
      Qu’une religion soit raisonnable [donc universelle] dépend largement de ses pouvoirs d’assimilation, de sa capacité à fournir dans ses propres termes une interprétation intelligible des diverses situations et réalités que rencontrent ses adhérents. Les religions que nous qualifions de primitives échouent régulièrement à ce test quand elles sont confrontées à des changements importants, tandis que les religions mondiales développent de plus grandes ressources pour faire face aux vicissitudes (Lindbeck, 175)
  \end{quote}
  
La vérité des religions est testée dans l'histoire. Il y a des traditions religieuses qui ont disparu. On verra à la fin de l'histoire, de façon eschatologique, la vérité des religions du jour. Si cela dure, il y a quelque chose de solide. 

  \paragraph{Fécondité sociale et universalité} 
  
  un autre aspect de la fécondité envisagé par Lindbeck, c'est la fécondité sociale.  Il réaffirme que la religion socialise les hommes, 
  
  \subparagraph{contraste avec l'approche libérale,} où des milliers d'hommes sont contraints à se lancer individuellement dans un supermarché religieux \sn{Petit bambou, retraite dans une abbaye,...}. Il ne s'agit pas de vivre dans son coin sa relation à Dieu mais de vivre socialement. 
  
  \subparagraph{Religion : un tout cohérent} Dans le Christianisme prend cette cohérence quand l'Evangile doit socialiser les personnes, du vivre ensemble, leur permettant d'organiser leur vie, de faciliter la vie ensemble, de lire le monde ensemble. 
  
  \subparagraph{une doctrine n'est vraie que si elle permet de vivre de façon cohérente} des adeptes de telle religion. Les \textit{vérités} des autres traditions religieuses n'ont pas a être méprisées à partir du moment où elles permettent de vivre en cohérence au sein de cette religion.
  
  \subparagraph{une vision communautaire ?} pas forcément.
  
  
  
  
  % ------------------------------------------------
  
  \section{Une théologie post-libérale des religions : penser la différence/l’altérité}
    
 Selon Lindbeck, les religions peuvent être une anticipation voulue et approuvée par Dieu du Royaume à venir.
 
 Du point de vue Chrétien, \paragraph{une préfiguration du Royaume dans les différentes communautés (dans sa dimension sociale)} Des socialisations qui se font à différents niveaux. 
    
\begin{quote}
    Dès lors qu’il est clair qu’une théologie catholique des religions peut affirmer le caractère distinct des fins poursuivies par les autres religions sans préjudice pour une affirmation de l’unique valeur de la communauté chrétienne ou de ses doctrines du salut, alors il devient possible d’affirmer que Dieu veut que les autres religions jouent des fonctions dans son plan pour l’humanité qui ne sont perçues aujourd’hui que de façon confuse et qui seront complètement révélées dans la consommation de l’histoire attendue par les chrétiens non pas parce qu’elles seraient des canaux de la grâce ou des moyens de salut pour leurs adeptes, mais parce qu’elles jouent un rôle réel, même si peut-être pas complètement déterminé, dans le plan divin auquel la communauté rend témoignage. (Joseph DINOIA, The Diversity of Religions : A Christian Perspective, Baltimore, The Catholic University of America Press, 1992, p. 91) 
\end{quote}    
  % ------------------------------------------------
  
  \paragraph{Abu Dhabi}
  le pape François a remis en avant la diversité des hommes.
  
  Dans cette vision des choses, l'Eglise doit socialiser ces membres mais doit être aussi au service du dialogue inter-religieux, le dialogue comme dans le plan de Dieu. 
  
\begin{quote}
    « […]Le Pape Jean-Paul II le disait à Assise, à la fin de la Journée de prière, de jeûne et de pèlerinage pour la paix: 
    \begin{quote}
       «Voyons en ceci une \textsc{anticipation} de ce que Dieu voudrait voir se réaliser dans l’histoire de l’humanité: un cheminement fraternel dans lequel nous nous accompagnons mutuellement vers un objectif transcendant qu’il prépare pour nous »  
    \end{quote}
    (Dialogue et annonce\sn{Document intéressant, en 1991, qui a deux parties, le conseil pontifical pour le dialogue inter religieux et l'autre partie, la propagation de la foi}, § 79). 
\end{quote}  
  
\begin{quote}
« L’Eglise encourage et stimule le dialogue interreligieux non seulement entre elle-même et d’autres traditions religieuses mais aussi entre ces traditions religieuses elles-mêmes. C’est une manière pour elle de remplir son rôle de «sacrement», c’est-à-dire «de signe et instrument de l’union intime avec Dieu et de l’unité de tout le genre humain» (Lumen gentium, 1). L’Esprit l’invite à encourager toutes les institutions et tous les mouvements de caractère religieux à se rencontrer, à collaborer et à se purifier afin de promouvoir la vérité et la vie, la sainteté et la justice, l’amour et la paix, dimensions de ce Règne que le Christ, à la fin des temps, remettra à son Père (cf. 1 Co 15, 24). Par là, le dialogue interreligieux fait vraiment partie du dialogue de salut dont Dieu a pris l’initiative » (Dialogue et annonce, § 80). 
\end{quote}
L'Eglise essaye de faire rencontrer des bouddhistes avec les musulmans. C'est pour elle être \textit{sacrement}. 


Une certaine remise en cause de notre propre socialisation pour l'Eglise. 
Un thème théologique très actuel : comment le dialogue avec les autres peut nous aider à penser le dogme du Règne de Dieu, et dans son lien avec l'Eglise.
Dans VII, des documents.
Peut nous permettre de vivre de façon plus féconde le
  
  
  
 % ------------------------------------------------------------------------------------- 
\section{Église et royaume de Dieu réinterprétés par la théologie des religions}
  
  
  \begin{quote}
      Depuis le temps de Jean-Baptiste jusqu’à présent, le royaume des cieux est forcé, et ce sont les violents qui s’en emparent. Car tous les prophètes et la loi ont prophétisé jusqu’à Jean; et, si vous voulez le comprendre, c’est lui qui est l’Élie qui devait venir. Que celui qui a des oreilles pour entendre entende. Mt 11,12
  \end{quote}
  
\subsection{La signification du Règne de Dieu}

\paragraph{La problématique} 


\begin{quote}
    « L’Église n’a pas été reconnue comme elle l’espérait et elle a souvent réagi avec agressivité et violence pour trouver sa place. Ni attendue  ni désirée ni reconnue, l’Église porte en elle cette expérience douloureuse et longtemps impensée en tant que telle.(…) Le  statut de greffon  ne se gère pas en s’imposant mais en ayant pleinement conscience que l’on n’est chrétien seulement par adoption c’est-à-dire par choix. Etre adopté, ce n’est pas s’imposer » (M.-L. Durand, 151). 
\end{quote}

\begin{quote}
    « À partir des textes bibliques et des témoignages patristiques, comme des documents du Magistère de l'Église, on ne déduit une acception univoque ni pour Royaume des Cieux, Royaume de Dieu et Royaume du Christ ni pour leur rapport avec l'Église, elle-même mystère irréductible à un concept humain. Diverses explications théologiques peuvent donc exister sur ces problèmes. Cependant, aucune de ces explications possibles ne doit refuser ou réduire à néant le lien étroit entre le Christ, le Royaume et l'Église. » (CDF, Dominus Iesus, § 18). 
\end{quote}

\begin{quote}
    « Le christianisme et l’Église s’identifient-ils au Règne de Dieu, pour autant qu’il est présent dans le monde et dans l’histoire ? Ou, au contraire, le Règne de Dieu est-il une réalité universelle qui s’étend au-delà des limites de l’Église chrétienne ? Et, s’il en est ainsi, comment l’Église et les religions sont-elles respectivement reliées au Règne de Dieu ? (…) Et que dire encore du Règne de Dieu dans son achèvement eschatologique au-delà de l’histoire et de son rapport à l’Église et aux ‘autres’ ? » (Dupuis 505). 
\end{quote}
\paragraph{La relation du Règne de Dieu avec l’Église dans Lumen Gentium } 
\begin{quote}
    « C’est pourquoi le Christ, pour accomplir la volonté du Père, inaugura le royaume des cieux sur la terre (…) L’Église, qui est le règne de Dieu déjà mystérieusement présent, opère dans le monde (…) sa croissance visible » (LG 3).  
\end{quote}

\begin{quote}
    « Le texte définit le rapport entre le Royaume et l’Église au moyen de deux formules complémentaires. L’Église est l’exorde [début] sur terre du Royaume des cieux, elle est aussi la révélation du mystère du Christ. Autrement dit, elle est le Royaume du Christ présent in mysterio, de façon mystérieuse, car le mystère est à la fois révélé et caché ; en d’autres termes, la révélation n’éclate pas en pleine lumière mais se déroule sous le couvert des ombres. De plus elle est progressive ; non par ses propres forces, mais par la force de Dieu, l’Église développe sans cesse de façon visible son rôle d’annonciatrice du mystère »  (Philips, l’Église et son mystère au IIe concile du Vatican, t. 1, Desclée, Paris 1967, 86). 
\end{quote}

\begin{quote}
    « L’Église (…) reçoit mission d’annoncer le royaume du Christ et de Dieu [on ajoute ici ‘du Christ’] et l’instaurer dans toutes les nations, formant de ce royaume le germe et le commencement sur la terre » (LG 5). 
\end{quote}

\begin{quote}
    «  La Constitution dogmatique Lumen gentium parle d’une ordination graduelle à l’Église du point de vue de l’appel universel au salut qui inclut l’appel à l’Église. En contrepartie, la Constitution pastorale Gaudium et spes ouvre une perspective christologique, pneumatologique et sotériologique plus vaste. Ce qu’on dit des chrétiens vaut également pour tous les hommes de bonne volonté dans les cœurs desquels la grâce agit de manière invisible. Eux aussi peuvent être associés par le Saint-Esprit au mystère pascal, et ils peuvent par conséquent être conformés à la mort du Christ et marcher à la rencontre de la résurrection. » (CTI 1997, § 71). 
\end{quote}
\paragraph{Le rapport du Règne de Dieu à l’humanité dans Gaudium et spes }



\subsection{L’idée de Royaume de Dieu dans le contexte du pluralisme religieux}
    \paragraph{Le Royaume de Dieu et le salut des non-chrétiens } 
    
    
\begin{quote}
    « Le concile Vatican II reprend à son compte la formule extra Ecclesiam nulla salus. Mais avec elle, il s’adresse explicitement aux catholiques et il limite sa validité à ceux qui connaissent la nécessité de l’Église pour le salut. Le concile considère que l’affirmation est fondée sur la nécessité de la foi et du baptême, affirmée par le Christ. (CTI 1997, § 67). 
\end{quote}

\begin{quote}
    « Certes, l'Eglise n'est pas à elle-même sa propre fin, car elle est ordonnée au Royaume de Dieu dont elle est germe, signe et instrument. Mais, alors qu'elle est distincte du Christ et du Royaume, l'Eglise est unie indissolublement à l'un et à l'autre. Le Christ a doté l'Eglise, son corps, de la plénitude des biens et des moyens de salut; l'Esprit Saint demeure en elle, la vivifie de ses dons et de ses charismes, il la sanctifie, la guide et la renouvelle sans cesse. Il en résulte une relation singulière et unique qui, sans exclure l'action du Christ et de l'Esprit Saint hors des limites visibles de l'Eglise, confère à celle-ci un rôle spécifique et nécessaire. D'où aussi le lien spécial de l'Eglise avec le Royaume de Dieu et du Christ qu'elle a «la mission d'annoncer et d'instaurer dans toutes les nations». » (Redemptoris Missio, § 18).  
\end{quote}

\begin{quote}
    « Il est donc vrai que la réalité commencée du Royaume peut se trouver également au-delà des limites de l'Eglise, dans l'humanité entière, dans la mesure où celle-ci vit les «valeurs évangéliques » et s'ouvre à l'action de l'Esprit qui souffle où il veut et comme il veut (cf. Jn 3, 8); mais il faut ajouter aussitôt que cette dimension temporelle du Royaume est incomplète si elle ne s'articule pas avec le Règne du Christ, présent dans l'Eglise et destiné à la plénitude eschatologique. » (Redemptoris Missio, § 20). 
\end{quote}
    
\paragraph{Le Royaume de Dieu dans le contexte du dialogue interreligieux} 
\begin{quote}
    Une partie [de la mission de l’Église] consiste […] à reconnaître que la réalité de ce Royaume peut se trouver à l’état inchoatif aussi au-delà des frontières de l’Eglise, par exemple dans le cœur des membres d’autres traditions religieuses dans la mesure où ils vivent des valeurs évangéliques et sont ouverts à l’action de l’Esprit. Il faut rappeler cependant que cette réalité est en vérité à l’état inchoatif; elle trouvera son achèvement en étant ordonnée au Royaume du Christ qui est déjà présent dans l’Eglise mais qui ne se réalisera pleinement que dans le monde à venir (Dialogue et annonce,  § 35). 
\end{quote}
\paragraph{Le règne de Dieu et les autres religions }

\begin{quote}
    « Quand les non-chrétiens, justifiés par la grâce de Dieu, sont associés au mystère pascal de Jésus-Christ, ils le sont aussi au mystère de son Corps qui est l’Église. Le mystère de l’Église dans le Christ est une réalité dynamique dans le Saint-Esprit. Bien qu’à cette union spirituelle il manque l’expression visible de l’appartenance à l’Église, les non-chrétiens justifiés sont inclus dans l’Église « Corps mystique du Christ » et « communauté spirituelle ». C’est en ce sens que les Pères de l’Église peuvent dire que les non-chrétiens justifiés font partie de l’Ecclesia ab Abel. Tandis que ces derniers sont réunis dans l’Église universelle avec le Père, ceux qui appartiennent certes « de corps » à l’Église mais non pas « de cœur » ne seront pas sauvés, parce qu’ils n’ont pas persévéré dans la charité. » (CTI 1997, § 72). 
\end{quote}

\begin{quote}
    « Ainsi, on peut parler non seulement en général d’une ordination des non-chrétiens justifiés à l’Église, mais aussi d’un lien avec le mystère du Christ et de son Corps, l’Église. On ne devrait cependant pas parler d’appartenance à l’Église, ni d’appartenance graduelle à l’Église, ni d’une communion imparfaite avec l’Église, expression réservée aux chrétiens non catholiques ; car l’Église est par essence une réalité complexe, constituée de l’union visible et de la communion spirituelle. Bien évidemment, par la mise en pratique de l’amour envers Dieu et le prochain, les non-chrétiens qui ne sont pas coupables de ne pas appartenir à l’Église entrent dans la communion de ceux qui sont appelés au Royaume de Dieu ; cette communion se révélera comme Ecclesia universalis lors de la consommation du Royaume de Dieu et du Christ. » (CTI 1997, § 73). 
\end{quote}
\subsection{Questions ecclésiologiques} 




\section{Autres textes}
