\chapter{La religion comme le lien entre les hommes et Dieu }

\mn{ 11 avril 22
Théologie des religions C/4 Théologie chrétienne des religions C-4 La religion comme le lien entre les hommes et Dieu
Absent du cours / notes récupérées de Bodhan Burko}

\section{Bibliographie}
AMALADOSS, M., The Asian Jesus, Marylnoll – New York, Orbis Books, 2006. 

CONGRÉGATION POUR LA DOCTRINE DE LA FOI, Notification sur le livre du P. JACQUES DUPUIS S.J., « Vers une théologie chrétienne du pluralisme religieux » Paris, Cerf 1997, Cité du Vatican 2001. 


DALFERTH, I. U., Der auferweckte Gekreuzigte. Zur Grammatik der Christologie, Tübingen 1994. 

DHAVAMONY, M., « Religion » dans Dictionnaire de théologie fondamentale, Paris – Montréal, Cerf – Ed. Bellarmin, 1992, p. 1032-1064. 

DUBUISSON, D., L’invention des religions, Paris, CNRS éditions, 2020. 

DUPUIS, J., Vers une théologie chrétienne du pluralisme religieux, tr. par O. PARACHINI, Paris 1997.  

GEFFRE, C., « Religion et religions » dans Catholicisme XII, Paris, 1990, col. 783-802. HICK, J. (dir.), The Myth of God Incarnate, 1977. 

KNITTER, P., No Othe rame ? A critical Survey of Christian Attitudes Toward the World Religions, Maryknoll 1985, 171-231. 




\section{Introduction }


 
	On est toujours sur cette idée que pour dialogue il faut avoir toujours quelque chose en commun avec les autres. 
	Comment les différentes traditions accueillent elles l’idée des religions ?
\section{ La religion comme relation (salvifique) à Dieu/au divin }

\subsection{ Religio : un débat autour de deux étymologies}
\mn{voir cours sur Christologie des religions p. \pageref{ChristologieCultureReligio}}

 
	\emph{Religio} est un term romain, il va être adapter par les chrétiens. 2 étymologies possibles :
	\begin{itemize}
	    \item Cicéron - relegere, relire, recueillir, respecter des actes culturels, etc.
	\item relier, roligare, relier l’homme à Dieu.
	\end{itemize}
	
 \begin{quote}
     	« Les Romains, pas plus que les Grecs d’ailleurs, n’avaient eux-mêmes de religion au sens futur et chrétien du terme. (...). A la différence d’une orthodoxie, une orthopraxie (....) n’est ni vraie ni fausse. On l’accomplit aussi exactement, aussi scrupuleusement que possible ; un point c’est tout. Les Romains ou les Grecs ne pouvaient pas considérer que les autres religions étaient fausses, alors que la tradition chrétienne ne peut imaginer une religion qui ne se prétende pas vraie, et tant qu’à faire, être elle-même la seul à l’être » (Dubuisson, 22-23)
 \end{quote}

	Pour les romains, l’importance n’est pas est-ce que Dieu est vrai ou pas, mais bien faire la culte. Il n’y a pas de vérité ici, cela n’est pas ni vrai, ni faut.
	Les premiers chrétiens christianisent le concept de Religion, et présentaient le christianisme comme véritable religion. 
\subsection{Rendre hommage et plaire à Dieu }

	Livre 3 Saint Thomas  d’Aquin :
	\begin{quote}
	    	« Parce qu’il est connaturel à l’homme de recevoir sa connaissance par les sens et qu’il lui est difficile de dépasser le sensible, Dieu a pourvu à ce que, même dans le domaine du sensible, il trouve un rappel des choses divines (...) A cette fin ont été institués des sacrifices sensibles que l’homme offre à Dieu, non parce qu’il en a besoin, mais pour signifier à l’homme qu’il doit rapporter à Dieu, comme à sa fin et à son Créateur, Roi et Seigneur de l’univers, et sa personne et ses biens ».

   « Le culte de Dieu consiste dans ces hommages sensibles (...) D’où le nom de religion donné au culte de Dieu : par ce genre d’actes l’homme se lie d’une certaine manière pour ne pas errer loin de Dieu ; de même instinctivement il se sent obligé de rendre à sa manière ses hommages à Dieu, principe de son être et de tout son bien (...) De là encore ce nom de piété que prend la religion. Par la piété en effet nous rendons à nos parents l’honneur qui leur est dû. Il est donc normal que l’homme rende à Dieu, Père de toutes les créatures, l’hommage de sa piété » (Thomas d’Aquin, Somme contre les Gentils, III, 119). 
\end{quote}
\subsection{Obtenir l’intervention de Dieu }
	Pour se tourner vers Dieu on a besoin un lien, et de passer par des éléments concrets. 
	
\begin{quote}
    « La définition générale de la religion désigne les relations de l’homme avec le sacré, le divin. La religion est la reconnaissance consciente et réelle d’une réalité absolue, du sacré ou du divin, dont l’homme sait qu’il dépend existentiellement, soit en s’y soumettant, soit en s’identifiant entièrement ou en partie avec elle. (…) Comme la religion consiste dans une relation de l’homme à quelque chose qui est perçu par lui comme étant ‘absolument l’autre’, cet ‘autre’ est présenté de bien des manières : comme une puissance, une personne, une réalité absolue, etc. (…) L’homme se sent incapable d’accomplir ce qu’il veut par ses propres forces, et alors il désire que l’Etre surhumain, surnaturel réponde à ses aspirations (…). L’homme religieux instaure lui-même les symboles et les rites qui assurent l’intervention divine, ou il reçoit historiquement de divers types de médiations et d’intermédiaires la réponse divine et l’aide dont il a besoin pour atteindre le but de sa religion »  (DTF, Dhavamony, 1034-1035). 
\end{quote}





\section{Le cadre interprétatif : « Plusieurs voies, un Dieu »} 

\subsection{Présupposés} 

	\paragraph{Frazer}\sn{\href{https://fr.wikipedia.org/wiki/James_George_Frazer}{Frazer}}  la religion a 2 elements, théorique et pratique. 
	Il distingue la théologie avec la religion. Cela ne sert à rien, si on a une bonne théologie, mais qu'on ne croit pas.
\subsection{L’émergence de la « théologie pluraliste » des religions }


	\paragraph{Pourquoi on entre en dialogue avec les autres religions ?} Car les religions ne sont pas les idéologies. Dieu, la	 divinité on ne cherche pas d'abord le définir mais… Toutes les traditions cherchent la même Divinité, mais les chemins sont différents. Il a fait une théorie de Christo-centrisme, le Christ est en centre, toutes les autres religions se tournent vers le christianisme et même le christianisme avec les autres se tourne Dieu.
	\begin{quote}
	    « Ces auteurs veulent abandonner l’opinion (…) pour laquelle c’est le mystère de Jésus-Christ qui est au centre. Dans la perspective théocentrique, Dieu, et Dieu seul, est au centre. ‘Pluralisme’ indique la substitution de l’unique, universelle et constitutive médiation de JésusChrist par un nombre de ‘voies’, ou figures salvatrices, conduisant à Dieu-le-centre  (…) Si le christianisme cherche sincèrement le dialogue avec les autres traditions religieuses – qu’il ne peut entretenir que sur un pied d’égalité – il doit d’abord renoncer à toute prétention d’unicité pour la personne et l’œuvre de Jésus-Christ comme élément constitutif universel de salut » (Dupuis, 281-282). 
	\end{quote}
	
\section{La christologie « pluraliste » : réinterpréter le Christ} 

\subsection{ Le Christ « décentré »} 


	Un modèle théo-centrique. 
 



\subsection{ Le mythe du Dieu incarné}
	\paragraph{Hick}- Mythe - un raport avec l’ontologie - l’idée du fils de Dieu. On a passé de Fils de Dieu à Dieu le Fils. 
	
	
	Comment Jésus est-il unique, et si oui comment ? 
	\begin{quote}
	    « Nous voulons dire de Jésus qu’il était \textit{totus Deus}, 'tout à fait Dieu', dans le sens que son agape était véritablement l’agape de Dieu à l’œuvre sur la terre, mais pas qu’il était \textit{totum Dei}, ‘la totalité de Dieu’, dans le sens que l’agape divine a été exprimée sans réserve dans chacune ou voire même dans une partie de son action. » (Hick, God and the Universe, 159). 
	\end{quote}
	
	Pour lui jusqu’à maintenant, Jésus Christ a été pensé comme inclusif et exclusif. 
	\paragraph{Inclusif} ; sera utilisé par les mondes catholiques. Il faut rencontrer le Christ pour rencontrer Dieu. Dans l’histoire on voit bien qu’il y a des autres religions et les autres traditions; le christianisme ne peut pas entrer en dialogue s'il se pense comme autonome. Il y a une culture autoritaire dans l’Église. Dans les autres contextes on dialogue. Mais dans l’Église il y a une question de la vérité. Si on a la vérité on ne la cherche plus ? Ici est important d’être capable de dialoguer; Idée de relation, Jésus est une manifestation universelle le salut de Dieu. 

 
\begin{quote}
    « Selon lui, les mythes expriment une évaluation et induisent un comportement, c’est-à-dire qu’ils expriment une obligation ou un jugement de valeur et qu’ils ne sont pas littéralement vrais (…) Il définit le mythe comme une histoire racontée, mais pas littéralement vraie, ou une idée ou une image (...)  mais pas littéralement vraie, mais suscitant chez les auditeurs une attitude ou une posture particulière » (Dalferth, 20).
\end{quote}
	



\subsection{La christologie d’en-bas de Paul Knitter }

\paragraph{Knitter et ses questions } Rencontre avec le bouddhisme : \textit{sans Bouddha, je ne serais pas Chrétien} Knitter\sn{\href{https://en.wikipedia.org/wiki/Paul_F._Knitter}{Paul Knitter}}

\paragraph{Jésus était théocentrique} 
	Quand les disciples ont demandé Jésus comment prier ? Il a enseigné le Notre Père. Jésus est théocentrique, on parle de NT. 
	1 Corinthiens 3
	\begin{quote}
	    22 soit Paul, soit Apollos, soit Céphas, soit le monde, soit la vie, soit la mort, soit les choses présentes, soit les choses à venir.

23 Tout est à vous; et vous êtes à Christ, et Christ est à Dieu.
	\end{quote}
\paragraph{Du royaume de Dieu au Fils de Dieu} 
	{Une expérience Big-bang} les disciples au départ étaient des gens normaux, c’est une expérience qu’elle on peut appeler Salut. Il y a quelques choses qui ont bouleversé des disciples.
	Parousie qui est venue. Jésus comme un homme divin qui faut des miracles. Et à la fin de christologie de la sagesse et de logos. Selon les juifs la sagesse à été personnalisé; Pâques - quelle est une des expériences possibles.
	
\paragraph{La christologie est, depuis le début, dialogique, pluriforme et évolutive. }

\begin{quote}
    « Toutes les images de Jésus surgissent dans le contexte de la dialectique entre sa personne et sa vie et la vie des disciples. Elles répondent à la question : que signifie-t-il pour nous aujourd'hui ? Elles ont des racines doubles, l'une dans sa vie telle qu'elle nous est rapportée dans les évangiles et l'autre dans la culture et l'histoire des disciples, même si l'une de ces racines peut être plus forte dans une image particulière. Par exemple, alors que le Jésus crucifié est plus enraciné dans sa vie, le Sacré-Cœur est plus culturel. » (Amaladoss, p. 2). 
\end{quote}
	La christologie est n’est pas uniforme au départ. Elle est un fruit de dialogue. Il y a une thèse «  au départ il avait plusieurs manières d’interpréter Jésus, mais avec le temps on a tous uniformisées. On a perdu cette diversité de christologie. Sa question, comment les premiers disciples ont compris Jésus ? 
	\begin{quote}
	    « La proclamation de l’Église apostolique, centré sur le Christ, a dénaturé le message de Jésus. L’Église apostolique a été la première responsable du changement de paradigme qui s’est produit, du théocentrisme au christocentrisme » (Dupuis, 425). 
	\end{quote}
	Jésus est indéfinissable, on ne peut pas définir le Christ; Aujourd'hui on veut définir Jésus parce que non cela ne va pas parler aux gens.

\paragraph{Pourquoi un processus convergeant vers l’unicité et l’exclusivité du Christ dans le NT ? } {	Pourquoi on a voulu uniformisé la christologie ?}
\begin{quote}
    1 Tm 2; 5. Un seul médiateur. Car il y a un seul Dieu, et aussi un seul médiateur entre Dieu et les hommes, Jésus-Christ homme.
	Ac 4, 12 : Il n'y a de salut en aucun autre; car il n'y a sous le ciel aucun autre nom qui ait été donné parmi les hommes, par lequel nous devions être sauvés.
\end{quote}

	Cela est une culture classique, lorsque il rencontrait Jésus il disaient « unique médiateur ». Une culture apocalyptique. Un langage de de survie «  Dieu comme unique et seul sauveur. 
	La christologie pluraliste. « Jésus Asiatique » - Jésus est montré ici par des symboles. L’idée qui est Jésus pour moi aujourd'hui ? Est-ce qu’il y a un sens pour moi aujourd'hui Jésus le Christ ? Pour que Jésus me rejoint aujourd'hui il faut mettre en valeur un aspect. 

\subsection{Quelques exemples de l’application de la christologie « pluraliste »} 

\paragraph{Christ gourou}
\begin{quote}
    « Jésus en tant que gourou, par exemple, nous montre la façon de vivre en tant que disciples. Dire qu'il est divin ou qu'il est sauveur peut ajouter de la profondeur mais n'ajoute rien à son rôle de gourou dans nos vies. Au contraire, il me sauve précisément en me montrant la manière de vivre en tant que personne sauvée et en me permettant de le faire. Il devient alors un gourou d'un genre particulier. » (Amaladoss, p. 5). 
\end{quote}

\begin{Synthesis}
La théologie pluraliste : on affirme un seul dieu mais plusieurs voies pour l'atteindre. Chez certains théologiens chrétiens (Hick et Knitter) qui ont essayé d'interpréter la foi chrétienne à ce modèle
\end{Synthesis}

\subsection{Les critiques adressées à la christologie « pluraliste » }

\paragraph{Réduction de la christologie en jésuologie}  L'idée des théologiens pluralistes est d'entrer en dialogue avec les autres, sans impérialisme (les autres ont quelque chose à dire). Or, la confession de foi ne concerne pas uniquement le Jésus historique mais le \textit{jésus historique crucifié et ressuscité}. L'identité du Jésus historique n'est pas le sujet, c'est le mystère pascal qui est au centre de la foi : \textsc{Pourquoi Jésus est Christ}. On ne peut pas mettre la résurrection était une interprétation parmi d'autres. 

\begin{quote}
    [pour les théologiens pluralistes] « La proclamation de l’Église apostolique, centré sur le Christ, a dénaturé le message de Jésus. L’Église apostolique a été la première responsable du changement de paradigme qui s’est produit, du théocentrisme au christocentrisme » (Dupuis, 425). 
\end{quote}

\paragraph{un lien entre Knitter et Bultmann}
Un peu la même chose que Bultmann qui disait que la resurrection était une interprétation des premiers disciples (\textit{mythe} pour exprimer la foi en Jésus vivant). Knitter dans la même université que Bultmann.
Pour Knitter, il s'agit de neutraliser les affirmations sur le Verbe \textit{pour dialoguer}. Mais est-ce honnête ? 


\paragraph{Réduction de la sotériologie} Selon Ingoldman DALFERTH, chez les théologiens pluralistes, Jésus n'est pas un sauveur mais par son exemple. Il n'est pas la grâce lui-même mais un exemple : 

\begin{quote}
    Hicks a une christologie pélagienne (son salut soi-même). 
\end{quote}

Si Jésus est un exemple, alors il est possible d'avoir plusieurs saluts et d'autres voies de salut sont possibles.
Mais alors il n'est plus l'auto-communication de Dieu mais simplement un exemple. 
\begin{quote}
    Le Dialogue doit il passer par la suppression de ce qui est l'originalité du christianisme. 
    \ldots Ce n'est plus une théologie chrétienne mais une approche philosophique des religions. 
\end{quote}

\paragraph{Le fondement philosophique de l’approche pluraliste}  
On retrouve la dichotomie introduite par E. Kant\mn{Emmanuel Kant fait la distinction entre les sciences (phenomene), où l'accord est facile, et la théologie, qui crée de fortes tension. Il distingue donc les deux : \begin{itemize}
    \item Le phénomène
L’effet produit par un objet sur la capacité de représentation, dans la mesure où nous sommes affectés par lui, est une \textit{sensation}. L’\textit{intuition} qui se rapporte à l’objet à travers une sensation s’appelle empirique. L’objet indéterminé d’une intuition empirique s’appelle phénomène (Critique de la raison pure, Esthétique transcendantale, §1, AK, III, 50, p. 117). Toute notre intuition n’est rien que la représentation du phénomène (Ibid., §8, AK, III, 64, p. 133)
 \item  Le noumène (en grec ancien νοούμενoν / nooúmenon) est un terme employé à l'origine par Platon pour désigner les « Idées », c'est-à-dire la réalité intelligible (par opposition au monde sensible), accessible à la connaissance rationnelle. Au contraire, chez Emmanuel Kant, auquel le terme de « noumène » renvoie le plus souvent, il s'agit de tout ce qui existe et que la sensibilité ne peut atteindre, restreignant par là les prétentions de la raison quant à la connaissance. « Noumène » est parfois considéré comme synonyme de chose en soi, faisant référence aux faits tels qu'ils sont absolument et en eux-mêmes, par opposition au terme de phénomène, faisant référence à ce qui est connaissable.

\end{itemize}
}, entre le \textit{noumen} (ce qui est transcendant, inaccessible) et le \textit{phenoumen} (ce qui apparaît, compréhensible). La théologie pluraliste emploit cette distinction mais fait de Jésus un phenoumen.

\begin{quote}
    La conception épistémologique sous-jacente à la position pluraliste emploie la distinction de Kant entre noumène et phénomène. Dieu (ou la Réalité ultime) étant transcendant et inaccessible à l’homme, on ne pourra en faire l’expérience que comme phénomène, exprimé par des images et des notions culturellement conditionnées ; cela explique que différentes représentations de la même réalité ne s’excluent pas nécessairement a priori l’une l’autre. […] Il y a ici, […] sous-jacente, une conception qui sépare radicalement le Transcendant, le Mystère, l’Absolu, de ses représentations : toutes choses étant relatives, puisqu’elles sont imparfaites et inadéquates, elles ne peuvent pas revendiquer d’exclusivité dans la question de la vérité (CTI, Christianisme et religions, § 14). 
\end{quote}

Est-ce qu'on a accès à Dieu exclusif ? Est ce que Jésus est la Vérité de Dieu? 
Jésus n'est il pas le \textit{noumen} lui-même ?

\begin{quote}
    « La conséquence la plus importante de cette conception est que Jésus-Christ ne peut pas être considéré comme le médiateur unique et exclusif. C’est pour les chrétiens seulement qu’il est la forme humaine de Dieu, rendant adéquatement possible la rencontre de l’homme avec Dieu, quoique sans exclusivité. Il est totus Deus, parce qu’il est l’amour actif de Dieu sur cette terre, mais non pas totum Dei, puisqu’il n’épuise pas en lui l’amour de Dieu. […] ». (CTI, § 21) 
\end{quote}

On est toujours dans cette idée du dialogue. Pour les pluralistes, le dogme de l'incarnation est relativisé. Or, la foi en Jésus Christ, vrai Dieu mais on ne peut pas relativiser le mystère de l'incarnation. 


\section{Un Dieu, un Christ, des voies convergentes}

\paragraph{Jacques Dupuis}, jésuite Belge en Inde. Dupuis parle de Dieu et son Logos. Dieu veut rejoindre l'homme, et tous les moyens sont bons, tous les chemins utiles.

\subsection{Sauver la théologie chrétienne pluraliste}   
 Dupuis essaye de tenir la foi Chrétienne et la pluralité des chemins salvifique.
 
 \paragraph{Le théocentrisme } Important, car la foi chrétienne n'est pas \textit{christomonisme } (on réduit tout à la foi en Christ). 
 
 \begin{quote}
     « Du côté de Dieu, on devra montrer clairement que Jésus-Christ ne peut jamais être considéré comme se substituant au Père. Tout comme Jésus lui-même qui était entièrement ‘centré sur Dieu’, l’interprétation de foi de Jésus le Christ proposée par le kérygme chrétien doit l’être et le rester en tout temps elle aussi. L’évangile selon Jean appelait Jésus ‘le chemin et la vérité et la vie’ (Jn 14,6) – jamais le but ou la fin ; le même évangile précisait que le but de l’existence humaine – et de l’histoire – est le mystère insondable de Dieu »  (Dupuis 313). 
 \end{quote}
 
 Comment on pense la foi chrétienne que Jésus est celui qui nous conduit au Père. Ce n'est pas parce qu'il y a unité avec le Père qu'il n'y a pas de différence.
 
 Ce regard partant de Dieu permet d'entrevoir une pluralité.
 
 \paragraph{Les religions comme des voies de salut }
 
 \begin{quote}
     « Toutes les religions se présentent à leurs adeptes comme des voies de salut-libération  […] On peut risquer de suggérer un concept universel de salut-libération de la façon suivante : la recherche et l’obtention d’une plénitude de vie, de complétude, d’autoréalisation et d’intégration1 » (Dupuis 465).   
 \end{quote}
 Dupuis prend ici des risques mais il essaye de penser les religions comme voies de salut, un chemin pour que l'homme soit heureux. 
 
 \paragraph{Du salut individuel des « infidèles » au salut par les traditions religieuses }
 
 La grande question à partir du nouveau monde, comment des gens qui n'ont pas pu connaître Jésus Christ peuvent être sauvés. Après une longue maturition, on est arrivé à un compromis.
 

 \paragraph{Les textes ecclésiaux évoquant les autres traditions comme des voies }
 
 
 \begin{quote}
     « A ceux-là mêmes qui, sans faute de leur part, ne sont pas encore parvenus à une connaissance expresse de Dieu, mais travaillent, non sans la grâce divine, à avoir une vie droite, la divine Providence ne refuse pas les secours nécessaires à leur salut » (LG 16). 
 \end{quote}
 
 \begin{quote}
     « Puisque le Christ est mort pour tous et que la vocation dernière de l’homme est réellement unique, à savoir divine, nous devons tenir que l’Esprit-Saint offre à tous, d’une façon que Dieu connaît, la possibilité d’être associé au mystère pascal » (GS 22,5) 
 \end{quote}
Quand on parle de Baptême, il ne s'agit pas uniquement du baptême sacramentel. Beaucoup d'autres personnes sont baptisées.

S'il peut être sauvé, ce n'est pas uniquement une démarche individuelle loin de toute vie religieuse. 

\begin{quote}
    « (la vie religieuse) ne consiste pas, ni ne peut consister en états d’âme purement spirituels. Pour exister, la vie religieuse doit s’exprimer en symboles, rites et pratiques religieux. (…) (Ils) servent à la fois d’expression et de soutien aux aspirations de l’esprit humain. Il n’y a pas de vie religieuse sans pratique religieuse. En ce sens, il n’y a pas non plus de foi sans religion » (Dupuis 481).  
\end{quote}
Quelqu'un qui a une relation forte avec Dieu, cela s'extériorise. Et réciproquement les rites vont nous permettre d'intérioriser la relation à Dieu.
Le problème, c'est quand l'extériorité de la vie sociale ne résonne pas dans l'intériorité, c'est alors l'hypocrisie. Par exemple, les sacrifices religieux, extérieurs.
Nous sommes \textit{personnes} car nous sommes en relation, et nous sommes en relation car nous sommes \textit{personnes}.
\begin{Ex}
Si on ne parle pas à un bébé, il ne parlera jamais.
\end{Ex}

\subsection{Théocentrisme et christocentrisme }
 
 Si V 2 a reconnu l'aspect positif des autres religions, il ne s'est pas positionné sur l'aspect \textit{salvifique} de ces voies de salut. Dans certains textes, on peut néanmoins voir des aspects de cet ordre : 
 \begin{quote}
     « Tout ce qui se trouvait déjà de vérité et de grâce chez les nations comme par une secrète présence de Dieu, [l’activité missionnaire] le délivre des contacts mauvais et le rend au Christ son auteur […]. Aussi tout ce qu’on découvre de bon semé dans le cœur et l’âme des hommes ou dans les rites particulier et les civilisations particulières des peuples, non seulement ne périt pas, mais est purifié, élevé et porté à sa perfection pour la gloire de Dieu, la confusion du démon et le bonheur de l’homme » (AG\sn{Ad Gentes, décrêt sur les missions} 9). 
 \end{quote}
 
 Certes, on ne parle pas de religion mais de rites et civilisation, liées aux religions.
 De même, Jean-Paul II\sn{Redemptoris Missio 5} : 
 \begin{quote}
      « Le concours des médiations de types et d’ordres divers n’est pas exclu, mais celles-ci tirent leur sens et leur valeur uniquement de celle du Christ, et elles ne peuvent être considérées comme parallèles ou complémentaires ».  
 \end{quote}
 Jean-Paul II est prudent mais ouvre la porte pour que le Christ suscite d'autres médiations.
 De façon contemporaine, le secrétariat pour la relation avec les religions : 
 \begin{quote}
     « Concrètement, c’est dans la \textit{pratique sincère de ce qui est bon dans leurs traditions religieuses} et en suivant les directives de leur conscience, que les membres des autres religions répondent positivement à l’appel de Dieu et reçoivent le salut en Jésus-Christ même s’ils ne le reconnaissent et ne le confessent pas comme leur Sauveur [voir AG 3, 9, 11] » (Dialogue et annonce, § 29). 
 \end{quote}
 
 \begin{Ex}
 un musulman qui pratique sa religion peut être sauvé. Il reçoit le salut en Jésus Christ.
 \end{Ex}
 
 la CTI fait la glose de \textit{Redemptoris Missio} :
 \begin{quote}
     « Etant donné cette reconnaissance explicite de la présence de l’Esprit du Christ dans les religions, on ne peut exclure\sn{prudence, on n'est pas Dieu} la possibilité que celles-ci exercent, en tant que telles, une certaine fonction salvifique, c-à-d qu’elles aident les hommes à atteindre leur fin ultime, même malgré leur ambiguïté. Dans les religions, est thématisée explicitement le relation de l’homme avec l’Absolu, sa dimension transcendante. Il serait difficilement pensable que ce que l’Esprit Saint réalise dans le cœur des hommes pris individuellement ait une valeur salvifique, et que l’ait pas ce que ce même Esprit réalise dans les religions et dans les cultures. Le magistère récent ne semble pas autoriser une différenciation si radicale (…) » (CTI §  85). 
 \end{quote}
 L'argumentation est intéressante : si l'ES peut intervenir dans les consciences, on ne peut exclure l'action de l'ES dans les sociétés et les religions du fait de la nature sociale de l'homme. On ne peut pas aller beaucoup plus loin.
 
 \begin{Synthesis}
 Les religions sont donc des moyens de salut
 \end{Synthesis}
 

 
  
 \paragraph{Au-delà de la contradiction} Le défi est d'articuler des affirmations théologiques.
 Comment penser la christologie (le christ unique médiateur) et en même temps les autres traditions religieuses peuvent être des médiations ? Comment on pense l'articulation.
 \subparagraph{un Dieu, un Christ, des voies convergentes} : titre de Dupuis qui insert le Christ à Hicks. Comment affirmer la centralité du Christ et la légitimité des autres religions ? 
 Selon Dupuis,
 \begin{quote}
     « Une théologie du pluralisme religieux trouve sa place au-delà des paradigmes inclusivistes et ‘pluraliste’ conçus comme mutuellement contradictoires et s’excluant l’un l’autre […] Une théologie chrétienne du pluralisme religieux doit chercher à résoudre le dilemme entre inclusivité christocentrique et pluralisme théocentrique, conçus comme paradigmes contradictoires » (Dupuis 309… 311). 
 \end{quote}
 
 
 \begin{Def}[paradigme]
 mot utilisé par Thomas Kuhn, dans l'histoire des sciences, comment on passe de modèles différents : à un moment on passe du modèle ptoléméen au modèle coperniciens 
 \end{Def}
  Hicks l'applique à la théologie mais Dupuis critique cette analogie, la théologie intégrant la pensée précédente.
 
 
 \paragraph{L’articulation entre deux types de médiation, de voie ? 
} 
 Il est plus facile de dire ce que ce n’est pas plus que de dire ce que c’est. On parle plus facilement par négation.
 
 \begin{quote}
     1 Tm 2,5 Car il y a un seul Dieu, et aussi un seul médiateur entre Dieu et les hommes, Jésus-Christ homme,[Jésus est l’unique médiateur]
     
Ga 3, 19-20 Pourquoi donc la loi? Elle a été donnée ensuite à cause des transgressions, jusqu'à ce que vînt la postérité à qui la promesse avait été faite; elle a été promulguée par des anges, au moyen d'un médiateur. 20Or, le médiateur n'est pas médiateur d'un seul, tandis que Dieu est un seul. [ Moise comme médiateur (mais comme intermédiaire)]
 \end{quote}

Le même terme grec peut avoir deux sens différents.
Dupuis : 
\begin{quote}
    « ‘Un Dieu, un Christ, des voies convergentes’ évoque […] le caractère fondateur de l’événement-Christ comme garantie des multiples modes d’automanifestation, d’autorévélation et de don de soi divins au genre humain, en une économie de salut aux nombreux aspects, mais organiquement structurée, à travers laquelle les diverses voies tendent vers une convergence mutuelle dans le mystère divin absolu qui constitue leur terme ultime, commun à toutes » (Dupuis 319).

\end{quote}

1 Tm 2, 5 parle l’élément fondateur. Une seule économie de salut mais diverses voies (comme celle de Moise). 
\paragraph{Ni inclusivisme, ni relativisme} : les médiations sont fondées sur la médiation du Christ, qui n’épuise pas néanmoins toutes les médiations. 
\begin{quote}
    Ni absolue, ni relative
\end{quote}

On va passer à constitutive et relationnelle : 
\begin{quote}
    « Elle est à la fois ‘constitutive’ et ‘relationnelle’ […] . ‘Relationnelle’ […] le terme est conçu pour affirmer la relation réciproque qui existe entre la ‘voie’ qui est en Jésus-Christ et les diverses ‘voies’ de salut proposées par les traditions religieuses à leurs membres » […] La question […] est donc de savoir comment, dans la Providence même de Dieu, l’ ‘unique voie’ se rapporte aux ‘nombreuses voies’ […] Loin d’écarter toute valeur salvatrice des autres voies, la ‘voie’ qui est en Christ l’implique et la postule » (Dupuis 463-464).
\end{quote}


Il essaye de penser que la voie qu’est le Christ suscite et garantie les autres voies. Cette tension va renouveler la vision de la christologie.


 
\subsection{La réinterprétation de la christologie}  
Mettre en valeur l’universlaité du salut par la médiation du Christ.
\paragraph{Une christologie trinitaire selon deux économies}
Un modèle intégral autour d’une théologie trinitaire (yc ES). A partir de là, l’originalité de Dupuis est de discerner une double économie : 
\begin{quote}
    « Les autres religions peuvent-elles contenir et signifier, de quelque manière, la présence de Dieu aux êtres humains en Jésus-Christ ? Dieu leur devient-il présent dans la pratique même de leur religion ? On doit nécessairement l’admettre. En fait, leur propre pratique religieuse est la réalité qui donne expression à leur expérience de Dieu et du mystère du Christ. C’est l’élément visible, le signe, le sacrement de cette expérience. Cette pratique exprime, soutient, supporte, et contient, pour ainsi dire, leur rencontre avec Dieu en Jésus-Christ […] Dans l’Église (…) (le mystère du Christ) est ouvertement et explicitement présent, en la visibilité totale de sa pleine médiation. Dans les autres traditions religieuses, il est présent de manière implicite et cachée, en vertu d’un mode de médiation incomplet constitué par ces traditions » (484-485). 
\end{quote}


Il y a deux modèles : 
\begin{itemize}
    \item	Le modèle sacramentel (la grâce de Dieu nous est donné par des signes visibles). 
\item	Mais Dieu n’est pas limité par les médiations visibles.
\end{itemize}

\paragraph{L’économie selon le modèle sacramentaire}  Dupuis argumente que Dieu se rend manifeste par des signes visibles. Le sacrement originaire, \textit{Ur-Sakrament}, c’est le Christ lui-même. Dans les traditions, on trouve aussi des signes visibles. On est dans quelque chose de connu.  
Une critique qu’on pourrait faire à Dupuis, c’est que si le Christ est caché dans les autres traditions et explicite dans le Christianisme, pourquoi les autres religions ne le reconnaissent pas en Christ.

\begin{quote}
    « Le Verbe de Dieu – dont l’action est décrite d’après le modèle de la Sagesse en Si 24 – est la source de lumière pour tous les êtres humains au cours de l’histoire, y compris la période qui précède sa venue dans la chair ; son pouvoir illuminant et sauveur est universel et s’étend en fait à tous les temps et à toutes les personnes » (Dupuis 485).
\end{quote}



\paragraph{L’économie « illuminative » du  Logos asarkos} L’originalité de Dupuis : le logos éternel. Distinction entre le logos asarkos et le logos ensarkos, incarné. L’évènement Christ n’épuise pas le pouvoir du Logos. Le pouvoir du logos est plus grand que le pouvoir de Jésus. Il y a bien identité entre les deux. Comme St Thomas d’Aquin, les sacrements n’épuisent pas la grâce divine : Dieu peut donner sa grâce en dehors des sacrements. Dupuis s’appuie sur Jn 1, 9 : 
\begin{quote}
    Cette  lumière était la véritable lumière, qui, en venant dans le monde, éclaire tout homme.
\end{quote}

\subparagraph{Semences du verbe} Il va reprendre la théologie des semences du Verbe. 

Il y a donc : 
\begin{quote}
    « Le Logos divin continue, encore aujourd’hui, de répandre ses semences parmi les peuples et dans leurs traditions : la vérité révélée et la grâce salutaire sont présentes en elles par l’entremise du Logos » (Dupuis 486).
\end{quote}


\paragraph{ Manifestations, incarnation et révélation }
Il va montrer que les manifestations dans les autres traditions, on ne peut pas les réduire au Christ incarné.
\begin{quote}
    « On peut à bon droit parler de manifestations distinctes\sn{mais pas séparé} du Verbe dans l’histoire. Le Verbe qui ‘illumine tout homme’ (Jn 1,9) est la source de l’il-lumin-ation de Gautama le Bouddha ; le même s’est fait chair en Jésus-Christ (Jn 1,14). Toutes les manifestations du Verbe n’ont pas la même signification. L’Incarnation, au regard de l’illumination, a une consistance historique qui lui est propre. Ce qui n’empêche pas, toutefois, la complémentarité réciproque des valeurs salutaires de sagesse et d’amour transmises par les deux traditions : compassion aimante et amour compatissant » (Dupuis, 497-498).
\end{quote}

On ne peut pas tout réduire au christianisme car les autres religions viennent du logos.
L’incarnation du Verbe est l’accomplissement du salut mais pas la complétude de la révélation. La révélation est la totalité des manifestations du logos dans l’histoire. La récapitulation de ces manifestations sera lors de la parousie.
\begin{quote}
    « Le salut opère en tout lieu ; mais dans la figure concrète du Christ crucifié, l’œuvre du salut se trouve accomplie. Jésus-Christ est donc l’ ‘unique Sauveur’, non pas comme l’unique manifestation du Verbe de Dieu, qui est Dieu lui-même ; ni même dans le sens qu’en lui la révélation divine est complète et exhaustive – ce qu’elle n’est ni ne peut être ; mais par rapport au processus universel de la révélation divine qui a lieu par des manifestations concrètes et limitées » (Dupuis 499).
\end{quote}

Il s’appuie sur DV 4
\begin{quote}
« Le salut opère en tout lieu ; mais dans la figure concrète du Christ crucifié, l’œuvre du salut se trouve accomplie. Jésus-Christ est donc l’ ‘unique Sauveur’, non pas comme l’unique manifestation du Verbe de Dieu, qui est Dieu lui-même ; ni même dans le sens qu’en lui la révélation divine est complète et exhaustive – ce qu’elle n’est ni ne peut être ; mais par rapport au processus universel de la révélation divine qui a lieu par des manifestations concrètes et limitées » (Dupuis 499).   
\end{quote}

Il a un regard eschatologique à partir de la parousie.

\paragraph{Le christianisme peut aussi apprendre de l’autre} 

\begin{quote}
    « La porte est ouverte à la possibilité de valeurs complémentaires entre les différentes traditions religieuses, y compris le christianisme (…) Il s’agit d’une ‘complémentarité mutuelle’, par laquelle un échange et un partage de valeurs salutaires ont lieu entre le christianisme et les autres traditions, d’où peut découler un enrichissement et une transformation réciproques entre les traditions elles-mêmes » (Dupuis 494).
\end{quote}


 
Ces illuminations peuvent nous apprendre quelque chose du Christ

\subsection{ Evaluation critique }

\paragraph{ Critique sur le plan de l’interprétation des Ecritures  } Jn 1, 9 parle du Verbe. Dupuis dit que c’est le logos Asarkos. X. Leon-Dufour pense que c’est le logos Asarkos mais pour Brown, c’est le logos incarné. Et il y a une certaine confusion entre le logos Asarkos et l’ES.

\paragraph{ La question de l’unicité du Christ} Séparation entre le Logos éternel et le christ incarné ? Une sorte de nestorianisme ? En Jésus Christ, la nature divine et humaine agissent dans l’école d’Antioche qui a du mal à mettre les deux ensembles. Et chez Dupuis, on a cette tendance.
\begin{quote}
    « Il est […] contraire à la foi catholique non seulement d’affirmer une séparation entre le Verbe et Jésus ou une séparation entre l’action salvifique du Verbe et celle de Jésus, mais aussi de soutenir la thèse d’une action salvifique du Verbe comme tel, dans sa divinité, indépendamment de l’humanité du Verbe incarné. » (Notification CDF) « Il est conforme à la doctrine catholique d’affirmer que les grains de vérité et de bonté qui se trouvent dans les autres religions participent d’une certaine manière aux vérités contenues par/en Jésus-Christ. Par contre, considérer que ces éléments de vérité et de bonté, ou certains d’entre eux, ne dérivent pas ultimement de la médiation-source de Jésus-Christ, est une opinion erronée. » (Notification CDF).
\end{quote}


Mais Dupuis ne parle pas de séparation mais de distinction. Mais peut être qu’il n’est pas assez précis là-dessus. Il s’appuie sur un modèle sacramentaire mais il ne pense peut être pas assez l’unité du logos et du logos incarné.
Dans GS 5, on parle bien de l’ES, lié au Christ ressuscité. Mais est ce que l’ES agit indépendamment du Christ ?
\begin{Synthesis}
Des chemins difficiles mais approche intéressante trinitaire.
\end{Synthesis}

