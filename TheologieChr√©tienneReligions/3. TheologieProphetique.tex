\chapter{Une théologie prophétique et critique}
\section{Eléments bibliographiques}

\begin{itemize}
    \item AUGUSTIN, La cité de Dieu v. 1 Livres I à X, tr. L. MOREAU (1846) et introduction J-C. ESLIN, Paris 1994. Voir aussi l’édition de la Bibliothèque Augustinienne et celle de la
Pléiade.
    \item BARTH, K., L’Epître aux Romains, tr. par P. Jundt, Genève 1972.
    \item BARTH, K., Dogmatique, I/2**, Genève 1954, 71-147.
    \item BONHOEFFER, D., Résistance et soumission, Genève 1963
    \item FITZGERALD, A.-D (ed.), Encyclopédie Saint Augustin, Paris 2005.
    \item GEFFRE, C., « Théologie de la religion », Catholicisme XII, 799-802
    \item KÜNG, H., « Vers une éthique universelle des religions du monde. Questions fondamentales
d’éthique sur l’horizon du monde actuel », Concilium 228 (1990), p. 121-139
    \item LUTHER, « Cours sur l’Epître aux Romains [extraits] » dans Œuvres I, Paris 1999, 1-96.
    \item RÖMER, T., L’invention de Dieu, Paris 2014.
Vocabulaire de théologie biblique.

\end{itemize}






%--------------------------------------------
\section{Introduction}



  \begin{Def}[Exclusivisme]
 On parle parfois d'exclusivisme, 
 mais plutot d'approche critique ou prophétique
\end{Def}
 
 On commence par Paul :
 \begin{quote}
     Rm 1,20-25 : « [Les Grecs et les barbares] sont donc inexcusables, puisque connaissant Dieu, ils ne lui
ont rendu ni la gloire ni l’action de grâce qui reviennent à Dieu ; au contraire, ils se sont fourvoyés
dans leurs vains raisonnements et leur cœur insensé est devenu la proie des ténèbres : se prétendant
sages, ils sont devenus fous ; ils ont troqué la gloire du Dieu incorruptible contre des images
représentant l’homme corruptible, des oiseaux, des quadrupèdes, des reptiles. […] Ils ont échangé la
vérité de Dieu contre le mensonge, adoré et servi la créature au lieu du Créateur qui est béni
éternellement ».

 \end{quote}
C'est une dénociation de l'idolatrie.
\begin{Ex}
Mais il peut y avoir de l'idolatrie dans le christianisme, par exemple quand on se met à genou devant une relique.
\end{Ex}

Les religions ne sont pas dénoncés en tant que telles mais du fait du risque d'idolatrie : 

\begin{quote}
    « Certains en sont arrivés, je ne sais comment, à une telle erreur, qu’ils adorent, non pas Dieu, mais
une œuvre divine, le soleil, la lune, tout le chœur des astres ; contre toute raison, ils les considèrent
comme des dieux, quand ils ne sont que les instruments du temps…Qu’aucun d’entre vous n’adore le
soleil, mais qu’il dirige ses désirs vers le fabricateur du soleil ; qu’il ne divinise pas le monde, mais
qu’il recherche le créateur du monde ! » \sn{Clément d'Alexandrie, \textit{Protreptique} IV,63, 1 et 5.}
\end{quote}

Marcel Gauchet \sn{Marcel Gauchet, \textit{le désenchantement du Monde}} a désacralisé le monde : 
`\begin{quote}
    le christianisme comme Religion de la sortie de la Religion. 
\end{quote}



Trois mouvements : 
\begin{itemize}
    \item L'émergence après l'Exil
    \item Cité de Dieu de Saint Augustin
    \item Karl Barth
\end{itemize}

\paragraph{Israel, un Dieu unique et universelle} L'émergence du monotheisme met en cause ces divinités attachés \mn{vision de THomas Romer, retrouver l'histoire d'Israel, nouveau regard sur le \textit{Pentateuque}, collège de France}


    
    
%--------------------------------------------
\section{Israël dans son rapport aux autres divinités et aux autres traditions dans l’AT}

\subsection{Un rapport de compétition politico-religieuse}

\paragraph{Eli} lutte contre le Baal Yahvé. Cet ordalie, cette confrontation au mont Carmen.

\begin{quote}
    « La voix de Yhwh retentit sur les eaux (…) ; Yhwh est sur les grandes eaux. La voix de Yhwh, avec
puissance, la voix de Yhwh, avec magnificence, la voix de Yhwh brise les cèdres ; Yhwh brise les
cèdres du Liban, il les fait bondir comme un torillon, il fait bondir le Liban et le Siriôn comme un
jeune buffle. La voix de Yhwh fait jaillir des feux flambloyants. La voix de Yhwh fait trembler le
désert ; Yhwh fait trembler la sainte steppe. La voix de Yhwh fait accoucher les biches, elle dépouille
les forêts. Et dans son temple tout s’écrie : ‘Gloire’ » (Ps 29, 3-9).
\end{quote}

Yhwh est le vrai Baal, l'eau et le feu. 

\begin{quote}
    « Le Dieu qui répondra par le feu, c’est lui qui est Dieu » (1 R 18,24) et « Le feu du Seigneur tomba et
dévora l’holocauste, le bois, les pierres, la poussière, et il absorba l’eau qui était dans le fossé » (1
R18,38). \mn{Texte avec de l'humour  : "criez plus fort"}
\end{quote}



\subsection{L’émergence d’un Dieu unique et la dépotentialisation des autres dieux}

Les autres Dieux n'ont plus de puissance.
\mn{Tjs la thèse de Römer}
\paragraph{La crise religieuse} C'est l'expérience de la défaite qui va faire le Dieu unique. En 622, la réforme de Josias. 597 : déportation; le temple de Jérusalem est détruit et Babylone. donc une crise réelle surtout pour les élites déportées. Il aurait été plus facile d'expliquer la défaite de Yhwh par les Dieux Mardouk ? 

\paragraph{Surmonter la crise} L'élite déportée a voulu expliquer l'histoire, cette histoire correspond à la tradition Deutéronomiste, qui impose le Dieu unique, relecture de cette crise (autre hypothèse : Deutéronome écrit au moment de josias, un peu avant). Cette histoire va du Dt au 2R, la relation de Dieu avec son peuple. On parle du passé pour expliquer le présent. On réutilise le matériau existant (Moise, les deux royaumes,...).

\paragraph{Le sens de cette histoire : Punition et salut} La défaite n'est pas liée à  Dieu mais à la \textit{desobeissance du peuple de Dieu}. Dt rappelle l'Alliance, traité asymétrique. Punir Juda pour le punir de son infidelité (2R 24,3)

Colère de Yhwh contre son Peuple et ses chefs :
\begin{quote}
    « Alors Yhwh envoya contre lui des troupes de Chaldéens, des troupes d’Araméens, des troupes de
Moabites et des troupes d’Ammonites ; il les envoya contre Juda pour le faire disparaître, selon la
parole que Yhwh avait prononcée par l’intermédiaire de ses serviteurs les prophètes […] C’est à cause
de la colère de Yhwh que ceci arriva à Jérusalem et à Juda, au point qu’il les rejeta loin de sa
présence » (2 R 24,2.20).
\end{quote}

\begin{Synthesis}
Interprétation théologique de l'histoire
\end{Synthesis}


\paragraph{La puissance divine en faveur du peuple} Dieu punit certes son peuple mais le \textit{sauve}. Ce n'est pas le jugement définitif mais il \textit{instruit} son Peuple et va le sauver.

\begin{quote}
    Is 45,3 : « A Cyrus que je tiens par sa main droite »
    
Is 45,1 : Yhwh a choisi Cyrus « pour abaisser devant lui les nations»

Is 44,28 : « Yhwh dit à Cyrus : ‘C’est mon berger’».

Is 45,13 : « Il renverra mes déportés à leurs localités».

Is 45,6 : « Afin qu’on reconnaisse, au levant du soleil comme à son couchant, qu’en dehors de moi :
néant ! C’est moi Yhwh, il n’y a en pas d’autre».

Is 44,15 : les autres divinités ne sont que du « bois à brûler».

Is 44,9-10 : « Ceux qui façonnent des idoles ne sont tous que nullité, les figurines qu’ils recherchent
ne sont d’aucun profit…Qui a jamais façonné un dieu pour une absence de profit ?»
\end{quote}

L'action de Yhwh qui permet à son peuple de revenir. l'affirmation du monothéisme est alors claire et tous les autres Dieu \textit{sont du bois à brûler}, ie du néant.


\paragraph{Pourquoi parvient-on au monothéisme ?} On passe du monothéisme à la monolatrie : 
\begin{quote}
    « Reconnais-le aujourd’hui, et réfléchis : c’est Yhwh qui est Dieu, en haut dans le ciel et en bas sur la
terre ; il n’y en a pas d’autre » (Dt 4,39).
\end{quote}

S'il y a le seul vrai Dieu, pourquoi est-il uniquement celui d'Israel ? Les Deutéronomistes vont développer le concept d'\textit{élection} :

\begin{quote}
Dt 10,14-15 : « Oui, à Yhwh ton Dieu appartient les cieux et les cieux des cieux, la terre et tout ce qui
s’y trouve. Or c’est à tes pères seulement que Yhwh s’est attaché pour les aimer ; et après eux, c’est à
leur descendance, c’est-à-dire à vous, qu’il a choisis entre tous les peuples comme on le constate
aujourd’hui »
\end{quote}

Si on considère que cela s'est passé comme le récit le dit, il n'y a plus de théologie, interprétation d'événements historiques qu'on met dans une lecture théologique.


\subsection{La dénonciation de l’idolâtrie} Elle est d'abord interne car l'idolâtrie touche aussi Israel

\paragraph{Yhwh dénonce l’idolâtrie de son propre peuple} Un double regard sur les pratiques religieuses : 
\begin{itemize}
    \item Sur son propre peuple : cf Ex 20, "Aucune image sculptée", "moi je suis un Dieu un, un Dieu Jaloux".
    \begin{quote}
        Os 8,4-7 : « Israël a repoussé le bien que l’ennemi le poursuive ! Ils ont créé des rois sans moi, sans
moi nommé des chefs. De leur argent et de leur or ils se sont fait des idoles, pour être anéantis euxmêmes. Il est repoussant ton veau, Samarie ! Ma colère s’est enflammée contre eux. Jusqu’à quand
seront-ils incapables de pureté ? Il vient d’Israël, un artisan l’a fait, il n’est pas Dieu ; oui, le veau de
Samarie s’en ira en morceaux. Ils sèment le vent, ils récolteront la tempête. »
    \end{quote}
    de la même façon, au temps des Maccabées, il faudra choisir le culte aux idoles ou le martyr : 
    \begin{quote}
        1 M 1,43 : « Beaucoup d’Israélites firent bon accueil à son culte (du roi), sacrifiant aux idoles et
profanant le sabbat ».
    \end{quote}
    1 Co 10, 14
    
    
\end{itemize}



Le culte du Dieu unique est très proche de la pratique de la justice : le Dieu unique est le Dieu juste et si on quitte le Dieu unique, on quitte la justice. Attention aux pauvres


\paragraph{La critique des dieux des autres peuples} Cette critique est seconde par rapport à l'autocritique. Les autres Dieux, 

\begin{itemize}
    \item incapables de se défendre ou de défendre leurs adeptes. On juge le vrai Dieu à son pouvoir, à sa puissance.
    \item Incapables de guérir et de sauver. 
\begin{quote}
    2 R 5,15.17 : « Maintenant, je sais qu’il n’y a pas de Dieu sur toute la terre si ce n’est en Israël (…)
    
Ton serviteur n’offrira plus d’holocauste ni de sacrifice à d’autres dieux qu’au Seigneur » 
\end{quote}
\item car formé par les hommes; divinités fabriqués par l'homme. 
\item se fier aux autres Dieux conduit au Mal. Dans le livre de la Sg : 
\begin{quote}
    Sg 14,23-28 : « Avec leurs rites infanticides, leurs mystères occultes ou leur processions frénétiques
aux coutumes extravagantes, ils ne respectent plus ni les vies, ni la pureté des mariages, mais l’un
supprime l’autre traîtreusement ou l’afflige par l’adultère. Tout est mêlé : sang et meurtre, vol et
fourberie, corruption et déloyauté, troubles, parjure, confusion des valeurs, oubli des bienfaits,
souillure des âmes, inversion sexuelle, anarchie des mariages, adultère et débauche. Car le culte des
idoles impersonnelles est le commencement, la cause et le comble de tout mal, soit qu’on s’abandonne
à une joie délirante ou qu’on profère de faux oracles, soit qu’on vive dans l’injustice ou qu’on se
parjure immédiatement »
\end{quote}
Paul reprendra l'argument en 1 Co 10, 20 : sacrifier aux idoles, c'est sacrifier au démon. 
On devient aliéné (face au pouvoir, à l'argent...). 
\end{itemize}

\begin{Synthesis}
A la critique du polythéisme, va s'ajouter la critique des fausses médiations (y compris les religions elles mêmes) dans le Christianisme.
\end{Synthesis}




 %-------------------------------------------------
\section{Le christianisme : vraie religion et voie véritable du salut}
 
Finalement, on n'a pas une théologie des religions mais une \textit{critique des autres Dieux}, d'abord interne puis externe.
 
 %-------------------------------------------------
 \subsection{Une théologie critique des religions dans La Cité de Dieu (s. Augustin)}
 
 On va retrouver cette critique chez S. Augustin, la \textit{vraie religion} qui met les autres traditions religieuses de façon inconfortable.
 
\paragraph{Le contexte} Le premier à en parler, c'est Tertullien et aussi Lactance (IV) \sn{Converti, va écrire un traité d'apologétique, \textit{les Institutions Divines}, il y montre que la fausse religion est la philosophie et que la vraie religion, c'est le christianisme}. Cela se passe au début de l'Empire Chrétien.  Augustin va écrire un traité sur la Vraie Religion, capable de restaurer la relation à Dieu. 
Le Christianisme devient \textit{Religion} d'Etat, un terme romain. 

Il meurt en 432 et écrit la Cité de Dieu de 410 à 426. En 410, pour la première fois depuis des siècles, le \textit{sac de Rome} par Alaric (Goths). La Pax Romana se fissure. \textit{un traumatisme}, un évènement qui va toucher les consciences. On va accuser les Chrétiens de ce désastre. Certes, l'empereur est devenu chrétien mais la culture chrétienne n'a pas imprégnée la société. 
\begin{quote}
    On a rejeté les dieux de la cité et ils ne nous ont pas protégé contre les envahisseurs. 
\end{quote}
une critique plus perfide des Chrétiens et de leur morale, par rapport à la citoyenneté romaine. "On appartient au Christ". les chrétiens sont ils loyaux ? \textit{citoyens de deux cités, l'une céleste}. Les Chrétiens, plus aptes à prier pour leur gouvernants mais peut être moins à prendre la défense de l'empire.
D'où l'apologétique : 
\begin{itemize}
    \item Pourquoi Rome est il tombé aux mains des Barbares ?
    \item pourquoi la fin de l'Empire ?
\end{itemize}

S. Augustin répond en disant que le \textit{patriotisme est une obligation religieuse}. Ces Lois sont établis par Dieu et il faut respecter les autorités civiles. 
Mais il va montrer que ce n'est pas l'abandon des cultes paiens qui est à l'origine de la chute de Rome.
L'empereur Théodose (390) va interdire les cultes paiens. Plus de tolérance. Le christianisme devient religion d'Etat et tout le monde doit devenir chrétien.

\paragraph{L’incapacité des religions païennes à préserver la paix et promouvoir la gloire de Rome}
 Dans les 5 premiers livres, il va prendre soin de répondre à cette question. 
 
 Les Dieux romains n'ont jamais protégé Rome : il y a eu des défaites, des guerres civiles. 
 
 \begin{Prop}
 Si la société va mal, n'y a t'il pas un lieu avec notre façon de croire ou de ne pas croire ?
 \end{Prop}
 
 La grandeur de l'empire romain ne tient pas de ses Dieux : désacralisation. 
 

\begin{Synthesis}
La religion n'est pas une question de Salut mais une question de cohésion entre les Peuples.
La Religion n'est pas uniquement la recherche du bonheur terrestre, mais ce qui apporte le bonheur et la vie éternelle : critère d'une vraie religion, celle qui nous relit au Dieu unique (Ratzinger).
\end{Synthesis}
 
 \begin{quote}
     
 \end{quote}
 
 
 \subsection{Les religions et la vie éternelle}
 \paragraph{L’incapacité des divinités mythiques et des cultes civils à procurer}
 

\paragraph{Varron} (116-27 av JC) va 
\begin{itemize}
    \item La théologie mythique ou fabuleuse
    \item la métaphysique, la nature des Dieux (des livres 6 à 7)
    \item la théologie civile ou politique (des livres 8 à 10)
\end{itemize}


 \paragraph{l’immortalité}
 
 \begin{quote}
      « C’est adorer au lieu de [Dieu] ce qui n’est pas lui, et lui offrir un culte qui ne doit être offert ni à lui,
ni à tout autre que lui. Or, quel est le caractère du culte païen, quel mélange d’horreurs et d’infamies,
c’est chose notoire […]. Plus de doute maintenant ; ce sont les esprits de malice et d’impureté que
toute cette théologie civile attire sous ces stupides emblèmes pour s’emparer de ces cœurs abrutis »
(Augustin, Cité de Dieu, VII, 27).
 \end{quote}
 Cela revient même aujourd'hui. 
 
 
 \paragraph{La sagesse philosophique et le Dieu unique} Augustin doit prendre en compte la religion naturelle (la nature même de Dieu d'un point de vue philosophique). Augustin discute avec les néoplatoniciens, qu'il admire. Les néoplatoniciens ont une vue très haute de la divinité, proche des chrétiens.
 
 
 
 \begin{quote}
     « [Les philosophes platoniciens] confessent un Dieu supérieur à tout âme, créateur non seulement de
ce monde visible, souvent appelé le ciel et la terre, mais encore de toutes les âmes raisonnables et
intelligentes, telles que l’âme humaine, âmes qu’il rend heureuses par la participation de sa lumière
incorporelle et immuable » (CD VIII,1).
 \end{quote}
 

 
 On sent la proximité de S. Augustin du néoplatonisme.
 
 \paragraph{Pourquoi les critique-t-il}
 
  \begin{quote}
     « Ces philosophes reconnaissent qu’il est un être où réside cette forme première, immuable, par
conséquent à nulle autre comparable ; et ils croient très légitimement que cet être est le principe
suprême, principe qui a fait toutes choses et n’a point été fait. Ainsi, ‘\textit{ce qui peut se connaître de Dieu
naturellement, ils l’ont connue ; Dieu le leur a dévoilé. Car, depuis la création du monde, l’œil de
l’intelligence voit, par le miroir des réalités visibles, les perfections invisibles de Dieu, son éternelle
puissance et sa divinité}’ (Rm 1,19) » (CD VIII,6).
 \end{quote}
 
 Intéressant la manière dont S. Augustin interprète S. Paul. Dans Rm 1, Paul montre l'idolâtrie mais S. Augustin va reprendre Rm et insiste sur le côté positif, \textit{ils l'ont connu}.
 
 \mn{Thèse du larçin : Moise a inspiré les théories néoplatoniciennes}
 
  \begin{quote}
     « Connaissant Dieu, les platoniciens découvrent à la fois le principe qui a fondé l’univers, la lumière
où l’on jouit de la vérité, la source où l’on s’abreuve de la félicité » (CD VIII, 10).
 \end{quote}
 
 
 \begin{Synthesis}
 Augustin : Les sages ont connu le vrai Dieu. Mais le problème des sages, c'est qu'ils ne vont pas trouvé les médiateurs. la question de la religion comme médiation.
 \end{Synthesis}
 
 
 \subsection{Le rejet des fausses médiations et des faux médiateurs}
 
 Il y a un déplacement. Pour les néoplatoniciens, il y avait une conviction que l' un passerait au multiple par \textit{des intermédiaires}. Or, Augustin va critiquer les intermédiaires
 
 \paragraph{L’intercession des « démons »\sn{Démon au sens d'être intermédiaire}, médiateurs du Dieu unique}. Il y a une telle transcendance dans l'antiquité tardive que le passage de l'UN au multiple, du Mortel à l'Immortalité. Ils instaurent des intermédiaires, des esprits, appelés Démons, avec les passions de l'âme, mais l'immortalité du corps.
 
\begin{quote}
    « Le ciel est la demeure des dieux ; la terre est le séjour des hommes ; l’air celui des démons (…) Les
démons partagent avec les dieux l’immortalité du corps ; et avec les hommes, les passions de l’âme »
(CD VIII, 14).
\end{quote}

 \begin{quote}
 \textbf{  « Quant à l’éternité, est-ce donc un bien sans le bonheur ? Mieux vaut la félicité dans le temps qu’une
éternité de misère »} (CD VIII, 16). 
\end{quote}
 Les passions de l'âme sont des sentiments négatifs.
 
\begin{quote}
    « Quant aux démons, quoique l’impureté de leur esprit ait souvent trahi leur misère et leur malice,
médiateurs faux et perfides, ils profitent des avantages de leur séjour et de l’agile subtilité de leurs
corps […] pour détourner le progrès de nos âmes, et loin de nous ouvrir la voie qui mène à Dieu, ils la
sèment de pièges. » (CD IX, 18).
\end{quote}





 \paragraph{Le sacrifice aux anges ou aux dieux} Il ne faut pas les adorer non plus  :  
 \begin{quote}
    « Les bons anges ne peuvent […] pas tenir le milieu entre les mortels malheureux et les bienheureux
immortels, car ils sont eux-mêmes et bienheureux et immortels ; mais les mauvais anges peuvent le
tenir, car ils sont immortels avec les uns et malheureux avec les autres. Leur adversaire est le bon
médiateur qui, à leur immortalité et à leur misère, a voulu opposer sa mortalité temporelle et la
permanence de son éternelle félicité » (CD IX,14).
\end{quote}
Ils n'ont rien de commun avec nous, ils sont immortels et le malheur.
 
 \paragraph{La véritable médiation} Ces intermédiaires sont insatisfaisants; il faut l'incarnation du Christ comme intermédiaire. Un médiateur doit appartenir à l'un ET l'autre côté pour le salut de l'un.
 
\begin{quote}
« L’immortel malheureux \sn{le démon} n’intervient donc que pour nous fermer le passage à la bienheureuse
immortalité […] mais le mortel bienheureux s’est fait médiateur, il a subi l’épreuve mortelle pour
donner l’immortalité aux morts […] et aux malheureux, la béatitude qui ne s’est jamais retirée de lui »
(CD IX, 14).
\end{quote}
En déduit le mystère de l'incarnation.


\begin{quote}
    « Si d’après l’opinion la plus probable et la plus digne de confiance, tous les hommes sont
nécessairement malheureux tant qu’ils demeurent sujets à la mort, il faut chercher un médiateur qui ne
soit pas seulement homme, mais Dieu et, par l’intervention de sa mortalité bienheureuse, retirant les
hommes de leur misère mortelle, les conduise à la bienheureuse immortalité. Or, ce médiateur ne
devait ni être exempt de la mort, ni demeurer à jamais son esclave. Il s’est fait mortel, sans infirmer la
divinité du Verbe, mais en épousant l’infirmité de la chair […] Il fallait donc que ce médiateur entre
nous et Dieu réunisse une mortalité passagère et une béatitude permanente, afin d’être conforme aux
mortels par ce qui passe, et de les rappeler, du fond de la mort, à ce qui demeure. » (CD IX, 14).
\end{quote}

Dans le contexte de la discussion avec le néoplatonisme, il dit que le sujet n'est pas le vrai dieu mais la véritable médiation. Une médiation spatiale n'est pas suffisante : \textit{le temps plus important que l'espace}


\subsection{Hors de cette voie pas de salut }
 
 

\begin{quote}
« C’est cette voie qui purifie tout l’homme, et, mortel, le prépare en tout lui-même à l’immortalité.
Car, afin que l’homme ne cherche pas une voie de purification pour cette partie de l’âme que Porphyre
appelle intellectuelle, une voie pour la partie spirituelle, et une autre pour le corps, il se charge de tout
l’homme, ce purificateur véritable, ce puissant Rédempteur. Hors de cette voie, qui, soit au temps des
promesses, soit au temps de l’accomplissement, n’a jamais manqué au genre humain, nul n’a été
délivré, nul n’est délivré, nul ne sera délivré » (CD X, 32).
\end{quote}

Nous permet d'être en communion avec Dieu, béatitude, et seul l'unique médiateur nous permet d'y accéder.

 

\begin{quote}
    «  C’est  cette  voie  qui  purifie  tout  l’homme,  et,  mortel,  le  prépare  en  tout  lui-même  à  l’immortalité. Car,  afin  que  l’homme  ne  cherche  pas  une  voie  de  purification  pour  cette  partie  de  l’âme  que  Porphyre appelle  intellectuelle,  une  voie  pour  la  partie  spirituelle,  et  une  autre  pour  le  corps,  il  se  charge  de  tout l’homme,  ce  purificateur  véritable,  ce  puissant  Rédempteur.  Hors  de  cette  voie,  qui,  soit  au  temps  des promesses,  soit  au  temps  de  l’accomplissement,  n’a  jamais  manqué  au  genre  humain,  nul  n’a  été délivré,  nul  n’est  délivré,  nul  ne  sera  délivré  »  (CD  X,  32). 
\end{quote}
 
 Certains pensaient que c'était l'âme qu'il fallait sauver. Ici, on a une vision biblique, c'est la totalité de l'homme qu'il faut sauver. Et donc, besoin d'un homme, sauveur médiateur pour le sauver.
 \begin{quote}
     ce qui n'a pas été assumé par le Seigneur ne peut être sauvé \sn{Pères grecs ?}
 \end{quote}

L'homme peut donc offrir de nouveau des sacrifices, le sien, en Jésus-Christ. Le culte est (r)établi.  
On va donc passer progressivement sur corps du Christ au \textit{Corps de l'Eglise}. On transpose le Christ médiateur à l'Eglise médiatrice. \mn{Au temps d'Augustin, il n'y a pas d'Islam. Quand l'Islam va émerger, on n'a pas de critères. Ce n'est plus comme le judaisme une religion qui prépare le christianisme. D'où l'idée de l'Islam comme une hérésie ou une déviance. Manque d'outils théologiques pour penser l'Islam.}

\begin{Synthesis}

Saint Augustin : Il répond sans difficulté à la religion civique mais pas à l'Islam pour lequel il ne donne pas véritable de cadre de pensée car ce n'est pas sa préoccupation.
importance de la médiation pour accéder à Dieu; Le Christ véritable médiateur. 

\end{Synthesis}

 
 %-------------------------------------------------
 \section{La religion critiquée au nom de la foi}
 \mn{31/1/22}
 
 
\subsection{Luther et la Réforme}

\begin{quote}
«  Mais  que  le  nombre  est  grand  de  ceux  qui,  encore  aujourd’hui,  n’adorent  pas  Dieu  lui-même  mais  la représentation  qu’ils  se  font  de  lui  !  On  assiste  aux  étrangetés  et  aux  rites  superstitieux  les  plus insensés.  N’est-ce  pas  changer  la  gloire  de  Dieu  en  représentations  imaginaires  et  fantaisistes  que  de  le servir  avec  des  œuvres  de  notre  choix,  en  négligeant  celles  qu’il  était  de  notre  devoir  d’accomplir  » (Luther,  Œuvres,  14).
\end{quote}
Souligne en Rm 1, 23 la permanence de l’idolatrie. Il critique les rites d’idolâtrie des chrétiens. C’est blanc ou noir, tout se joue dans cette véritable relation à Dieu. 
\begin{quote}
    «  [L’idolâtrie  spirituelle]  est  fréquente  de  nos  jours.  Elle  consiste  à  ne  pas  adorer  Dieu  tel  qu’il  est mais  tel  qu’on  se  l’imagine  ou  qu’on  se  le  représente.  Car  l’ingratitude  et  l’amour  de  la  vanité  (c’est-àdire  l’attachement  à  son  propre  jugement,  à  la  propre  justice  ou  à  ce  qu’on  appelle  les  bonnes intentions)  aveuglent  les  hommes  à  un  point  tel  qu’ils  en  deviennent  incorrigibles  et  ne  peuvent s’empêcher  de  croire  qu’ils  agissent  pour  le  mieux  et  qu’ils  plaisent  à  Dieu.  Par  là,  ils  se  façonnent  un Dieu  propice  quand  il  ne  l’est  pas.  Ils  rendent  donc  plus  véritablement  un  culte  à  leurs  représentations qu’au  vrai  Dieu,  qu’ils  croient  semblables  à  ces  représentations.  Ce  faisant,  ils  le  changent,  en  lui  prêtant  la  figure  de  leurs  imaginations  pétries  de  sagesse  charnelle  et  d’affections  corrompues.  Voilà donc  le  grand  mal  qu’est  l’ingratitude  ;  elle  entraîne  bientôt  avec  elle  l’amour  de  toute  vanité,  celle-ci l’aveuglement,  et  ce  dernier  l’idolâtrie  qui,  elle,  creuse  le  gouffre  de  tous  les  vices  »  (Luther,  Œuvres, 15-16).
\end{quote}



\subsection{La critique de la religion au nom de la foi au XXe siècle}

La question de la critique de la religion en tant que \textit{idolâtre} a été discuté au XX\textsuperscript{e}. Après la première guerre mondiale, on doit repenser la supériorité du christianisme et le drame du XX. On ne peut plus dire que nous ne sommes pas une religion, au dessus des religions. 


Barth dans le courant de la \textit{théologie dialectique} qui nait après la première guerre mondiale. Réaction à la \textit{théologie libérale} qui pourrait composer avec le monde moderne. Mais ce monde moderne a créé les guerres mondiales et la Shoah.

\paragraph{Barth et le Commentaire de l’Epître aux Romains}

L'épître au Romains a toujours été le texte fondateur de la Réforme pour reinterpréter la théologie chrétienne. 
La distinction religion / foi vient de la distinction paulinienne de des Loi / Grâce.
\begin{quote}
    «  A  quelque  niveau  qu’elle  se  situe,  l’expérience  religieuse,  si  elle  entend  être  plus  qu’un  vide,  si  elle entend  être  contenu,  possession  et  jouissance  de  Dieu,  est  l’anticipation  impudente,  et  vouée  à  l’échec, de  ce  qui  ne  saurait  et  devenir  vrai  qu’en  partant  du  Dieu  inconnu.  Sous  son  aspect  historique,  matériel et  concret,  elle  constitue  toujours  la  trahison  envers  Dieu.  Elle  engendre  le  Non-Dieu,  l’idole.  (…)  Et là,  en  quelque  lieu  que  ce  soit,  où  cette  distance  caractérisée  entre  l’homme  et  l’Ultime  qui  le  fonde, est  perdue  de  vue  et  méprisée,  là  survient  nécessairement  le  fétichisme  qui  expérimente  Dieu  dans  les ‘oiseaux  et  les  quadrupèdes  et  les  reptiles’  et,  en  premier  et  en  dernier  lieu,  dans  la  ‘figure  de  l’homme périssable’  (…)  et  dans  ses  créations,  formations  et  représentations  spirituelles  et  matérielles  (…)  et abandonne  à  son  sort  Dieu  qui  habite  au-delà  de  tout  ceci  et  de  tout  cela.  Ainsi  se  trouve  érigé  le  Non Dieu  ;  ainsi  de  trouvent  érigées  les  idoles  »  (K.  Barth,  L’Epître  aux  Romains,  tr.  par  P.  Jundt,  Genève 1972,  54-55). \mn{voir le \emph{le Seigneur des Anneaux} par rapport à la fantasy actuelle qui remet les idôles pour justifier la diversité des points de vue}
\end{quote}

Religion : auto-justification de l’homme pécheur. Par ses œuvres, il peut dire que je suis un bon chrétien. C’est de la religion mais ne sauve pas. L’homme pense faire sa propre justification alors que la justice ne vient que de la grâce. \mn{cf la critique du Pape du \emph{neo-pélagianisme}}

Il y a des accents prophétiques chez Barth.


\paragraph{La position de Barth dans la Dogmatique}

Une pensée dialectique qui se précise : 
\begin{quote}
    «  La  révélation  de  Dieu  par  l’effusion  du  Saint-Esprit  est  sa  présence  critique  et  réconciliatrice  dans  le monde  des  religions  humaines,  c’est-à-dire  dans  le  domaine  des  tentatives  faites  par  l’homme  pour  se justifier  et  se  sanctifier  lui-même,  devant  l’image  de  la  divinité  qu’il  se  compose  de  son  propre  chef  et arbitrairement.  \textit{L’Église  est  le  lieu  de  la  vraie  religion  uniquement  dans  la  mesure  où,  par  grâce,  elle vit  de  la  grâce}  »  (Barth,  Dogmatique\sn{Années 30},  71)
\end{quote}
Le christianisme est la vraie religion en tant qu'elle a été assumée dans la grâce de Dieu. Il y a une dimension critique et une réconciliatrice. Si elle est vraie, la religion vit de la grâce. Sinon, la religion est celle de l'homme incrédule et elle doit être sauvé par la Christ. 
il applique le schéma : "homme pécheur justifié par la Grâce" à l'Eglise, pécheresse, qui peut être justifiée par la Grâce.

\begin{quote}
    «  La  vraie  religion  est,  comme  l’homme  justifié,  une  œuvre  de  grâce.  Or,  la  grâce  n’est  rien  d’autre que  la  révélation  divine  qui  démasque  le  mensonge  des  religions  et  l’injustice  foncière  de  l’homme justifié  (…)  Et  de  même  que  la  grâce  est  capable  de  ressusciter  des  morts  et  d’amener  des  pécheur  à  la repentance,  elle  est  capable  aussi  de  créer,  dans  l’immense  domaine  des  manifestations  religieuses mensongères,  une  religion  qui  soit  vraie.  Affirmer  que  la  religion  est  ‘assumée’  par  la  révélation  ne veut  pas  dire  qu’elle  soit  purement  niée,  ni  même  qu’elle  doive  être  définitivement  considérée  comme incrédulité  (…).  La  religion  peut  précisément  être  assumée \sn{\emph{Aufheben}; terme hégélien, assumer mais aussi intégrer les termes antérieurs.}  et  supportée,  mieux  encore  :  justifiée  et sanctifiée  par  la  révélation.  La  révélation  possède  le  pouvoir  de  rendre  la  religion  vraie  et  de  la caractériser  comme  telle  (…).  La  religion  chrétienne  est  la  vraie  religion  »  (Barth,  Dogmatique,  115)
\end{quote}
la Révélation divine n'est pas une religion, elle est une dénonciation qui dénonce nos idoles et permet de transformer notre religion en vraie religion.

\subparagraph{Accusation d'incrédulité} Mais  il  faut  tenir  compte  du  fait  que  pour  Barth,  l’accusation  d’incrédulité  concerne  d’abord  la  religion chrétienne  :  
\begin{quote}
   «  Nous  devons  (…)  nous  appliquer  à  comprendre  que  le  jugement  de  la  révélation  sur  la religion  nous  concerne  d’abord  nous-mêmes,  et  cela  de  la  manière  la  plus  rigoureuse  ;  et  qu’il  ne  vaut pour  les  autres  que  dans  la  mesure  où  nous  nous  reconnaissons  en  eux  (…)  Car  la  promesse  de  Dieu vaut  pour  ceux  qui  s’humilient  et  la  foi  est  donnée  à  ceux  qui  se  laissent  convaincre  d’incrédulité  » (Barth,  Dogmatique,  116). 
\end{quote}

Pour les catholiques, on essaye d'articuler oeuvre et grâce. ici, on est dans un contexte protestant où seule la grâce compte.


Ainsi,  Barth  peut  préciser 

\begin{quote}
     que  «  cela  signifie  que  toutes  nos  activités  chrétiennes  –  qu’il  s’agisse  de nos  conceptions  de  Dieu,  de  notre  théologie,  de  nos  cultes,  des  formes  de  notre  vie  communautaire,  de notre  éthique  et  de  notre  esthétique,  de  nos  directives  individuelles  et  sociales,  de  notre  politique d’église  -   ne  correspondent    nullement,  pour  autant  qu’elles  sont  des  entreprises  menées  par  nous-mêmes  et  qui,  de  ce  fait,  se  situent  sur  le  même  plan  que  celle  des  autres  religions,  à  ce  qu’elles devraient  être  en  fait  :  à  savoir  une  expression  adéquate  de  la  foi  et  de  l’obéissance  à  la  révélation.  Au contraire,  tout  cela  relève  (…)  de  la  pure  incrédulité,  c-à-d  de  l’opposition  à  Dieu  et  à  sa  révélation,  de l’idolâtrie  et de  la  propre  justice  »  (Barth,  Dogmatique,  117). 
\end{quote}

Fédou, sj : dès que nous pensons avoir la vérité, nous sommes touchés par l'orgueil. Se laisser saisir par la vérité.


\paragraph{D. Bonhoeffer (1906-1945)} Résistant. 
Dans le monde contemporain, la Religion nous infantilise. \ldots Dieu est le \textit{Deus ex Machina}. La bible renvoie à la souffrance; Seul le Dieu souffrant peut nous aider.



\paragraph{H. Küng (1928-2021 : religion et éthique}

On doit critiquer les religions non seulement par rapport à la Foi mais aussi par rapport à l'éthique et la morale : liberté religieuse, opprimé, \textit{responsabilité par rapport à la planète}.

\begin{quote}
    Une religion qui entraverait les valeurs humaines serait discréditée.
\end{quote}

\begin{quote}
    la vraie religion c'est celle qui aide l'humanité
\end{quote}
La question, c'est quelles sont les valeurs humaines ? 
\begin{quote}
    Une religion qui ne respecterait par les valeurs humaines \ldots
    \sn{Concilium, éthique }
\end{quote}

Les religions ne sont pas enfermées en elles-mêmes, elles sont interrogées par l'éthique et les valeurs du monde. Et elles contribuent à cette éthique.
\subsection{Conclusion}
 
 \begin{Synthesis}
 Est ce qu'il n'y a pas quelque chose de la Foi dans la \textit{pratique} ? Trop rapide de juger que cela n'est que de l'idolatrie. La réalité n'est jamais très claire. La position de Barth manque de nuances.
 Dire que toutes les autres religions ne sont que des religions des oeuvres est un peu rapide car l'Esprit Saint peut parler à travers d'autres temps et culture \mn{le bon grain et l'ivraie}
 \end{Synthesis}
 
 Claude Geffré : 
 \begin{quote}
     il y a une certaine position vis à vis d'Israel. Cela vient d'une fausse interprétation de la critique de Jésus par rapport à la \textit{religion de la lettre}. 
     
 \end{quote}
 
 le Christianisme est une religion paradoxale\sn{COMEAU Geneviève, \emph{Grâce à l'autre, le pluralisme religieux, une
chance pour la foi}, Editions de l'Atelier, Paris, 2004 (159 p.)} :
 \begin{itemize}
    \item une grâce que nous recevons de Dieu
     \item cette initiative de Dieu touche aussi les autres, hors de l'Eglise
 \end{itemize}
 La critique et la dimension prophétique sont importantes, mais ce n'est pas par pure sadisme mais pour nous purifier parce que c'est le Seigneur qui nous appelle. 