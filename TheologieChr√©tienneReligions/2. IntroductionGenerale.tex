\chapter{Introduction générale}

\section{Eléments bibliographiques}
\bi

\item BRILLANT, M. – NEDONCELLE, M., Apologétique. Nos raisons de croire, réponses aux
objections, Paris, Bloud \& Gay, 1937.
\item COMMISSION THEOLOGIQUE INTERNATIONALE, Le Christianisme et les religions,
Rome, 1997.
\item GEFFRE, C., De Babel à la Pentecôte, essais de théologie interreligieuse, Cogitatio fidei n°
247, Paris, Cerf, 2006.
\item JONCHERAY, J., « L’ISTR au tournant de l’an 2000 » dans E. PISANI, Religions et
dialogues, 50 ans d’histoire de l’ISTR de Paris, Paris, Cerf, 2020, p. 37-46.
\item MARIN, C., « La fondation de l’ISTR en 1967 : la rencontre entre une tradition de l’Institut
catholique de Paris et la pensée du Concile » dans PISANI, E., (dir.), Religions et
dialogues. 50 ans d’histoire de l’ISTR de Paris, Paris, Cerf, 2020, p. 15-25.
\item MOULARD, A. – VINCENT, F., Apologétique Chrétienne, Paris, Bloud et Gay, 1918.
\item THILS, G., Propos et problèmes de la théologie des religions non chrétiennes, Tournai,
Casterman, 1966.
\ei

%------------------------------------------------------------------------------------
\section{La nouveauté de la « théologie des religions »}




\paragraph{quelque chose de nouveau} une approche naissante de cette théologie des religions

\paragraph{Comment le concept émerge ?} Elle apparait dans les années 1960 dans deux ouvrages : 
\bi 
\item 1963 : Robert Schlette, \textit{pour une théologie des Religions}, traduit en Français en 1971
\begin{quote}
  « Le problème de l'essence, de la forme et du sens des religions n'est plus
 uniquement l'apanage des spécialistes de la science des religions. De nos
jours, avec une intensité jusqu'alors inconnue, historiens, politiciens, sociolo~
gues... chacun essaye d 'acquerir une connaissance fondée des religions de l'hmanité. Tous prennent conscience de la nécessité d'une étude théologique des religions comme
 telles et non plus seulement, comme on l'a fait trop exclusivement jusqu'ici,
 de la religion en général. 
 
En fait, le problème théologique des religions, s'il n'est traité systématiquement,
ici ou là, que depuis un siècle, s"est posé dès !"origine à la pensée
chrétienne, et même à la pensée biblique de rAncien Testament. Les jugements
sévères des prophètes d'Israël sur toute forme d'idolâtrie ont été généralement suivis dans l'histoire de l'Eglise (p. 23 s.) et ont contribué à façonner
la doctrine traditionnelle entièrement négative à l'égard des religions. Et
pourtant, une attitude plus compréhensive était déjà préconisée par quelques
rabbins, sur la base des alliances adamique et noachique, avec l'humanité entière.
Elle ne pouvait qu'être accentuée, bien que différemment, par l'universalisme
évangélique, avec de nombreux textes du Nouveau Testament. Celui
des apôtres, et de Paul en particulier, n'est pas non plus contestable, malgré
\textbf{exclusivisme apparent de Romains 1 : 20}  qui a joué un rôle important
dans l'attitude hostile de la théologie traditionnelle, et de la mission chrétienne
(p. 10). Il eût valu la peine de faire une exégèse de ce texte dont on a si
souvent usé et mésusé en extrayant arbitrairement du grand contexte paulinien.
On pouvait aussi insister davantage sur les attitudes compréhensives et
positives qui ne sont pas tellement isolées dans l'histoire de l'Eglise, et dont
H. Pinard de la Boullaye, S.J. a fait un remarquable inventaire dans un ouvrage
fondamental : \textit{L' Etude comparée des religions}, en 3 volumes, 3e éd.
(1re 1922-25), Paris, Beauchesne, 1931. Ce précurseur dans la théologie romaine
n'a pas été oublié; mais pourquoi ne pas l'avoir mentionné et utilisé plus
souvent Sa méthode comparative, telle qu'elle est présentée p. 49, devrait,
d'ailleurs, faire l'objet d'un examen critique, et celle de l'auteur en même
temps. 

Pour être scientifique jusqu'au bout, la recherche doit accepter le risque,
le risque de la foi (jides quaerens intellectum), et ne pas s'ancrer sur la
position confortable (suave mari magno) adoptée, selon la formule (nihil
obstat et imprimatur), par un théologien du même bord: \textit{« Forts de ces principes
... , fiers d'appartenir à une Eglise, comme à une religion, qui n'a besoin
que de la vérité»}, les catholiques peuvent aborder sans crainte, surtout dans
des ouvrages non influencés par les préjugés rationalistes décrits plus haut.
l'étude des religions». 
Cela dit, en bonne critique, on n'assumera pas soi-même une position négative
à l'égard de ce livre où l'on trouve des observations et réflexions excellentes,
même quand on les souhaiterait plus nuancées. Il est certain que,
depuis Vatican II, un esprit nouveau (et pourquoi pas l'ancien, le primordial?)
souffle sur l'Eglise romaine, et qu'en s'y référant, l'auteur se sent plus
libre (p. I). Son analyse des possibilités théologiques actuelles (p. 27 ss.) est
intéressante, mais serait à retoucher, notamment en ce qui concerne les positions
protestantes qui sont diverses ; elles ne se groupent pas sous la rubrique
I (p. 27), qui serait tout au plus celle d'un certain Barthisme. Calvin lui-même
entrevoyait dans l'âme païenne quelques « flammettes » ou « estincelles» de la grâce divine, et tel missionnaire, de tradition calvinienne, a rendu
un beau témoignage à l'un des remarquables bénéficiaires de cette grâce dans
une région jusque là fermée à toute influence, à toute présence européenne ou
chrétienne. 
Parmi les nombreux traits à retenir, on pourrait signaler le voeu d'une
critique aujourd'hui indispensable de la théologie de l'histoire du salut »
(p. II). L'auteur, pour son compte, préfèrerait en éviter l'expression courante
(p. 9 s.); mais il s'en sert quand même très souvent. Il se demande si le redoutable
 "Extra Ecclesiam salus non est" doit être interprété au sens exclusif
et romain que Pie IX lui donnait en 1854. Il croit s'en tirer, si on ;e
comprend bien, en faisant entrer "les religions" dans une histoire générale
du salut, comme moyens de salut, tandis qu'Israël figurerait dans une histoire spéciale du salut. Tout, en somme, aboutirait à l'Eglise catholique, apostolique
et romaine (pp. 78 ss., 87 ss.). Pour ce faire, on écartera « la question déplaisante
et déplacée du salut individuel» (p. 71). La théologie, "ainsi libérée"
pourra "\textit{ hardiment réfléchir à ce problème, maintenant plus important,
du sens à donner aux religions en tant que phénomènes objectifs, historiques
et sociaux de l'humanité en tant qu'elle est reliée au mystère transcendant}" ,
et,  "\textit{à partir de là, en tant que moyens de salut" (p. 71). Il semble que le
salut personnel doive aussi trouver place dans cette combinaison théologique
où l'on n'insiste guère sur le fait que les religions ont été souvent des agents
de dépravation et de perdition, aussi souvent, peut-être, que des moyens de
salut} (p. 114).\sn{H. Clavier. \textit{Revue des livres dans Revue d'histoire et de Philosophie religieuses}}
\end{quote}
\item 1966 : Gustave Thils (Be)\sn{Expert Lumen Gentium, va être nommé au secrétariat des relations avec les religions}, \textit{Propos et problème de la théologie non chrétienne} 

\ei 
Avant, on ne parlait pas de théologie des religions mais de relation du \textit{christianisme avec les religions non chrétienne}\sn{Cf texte de Karl Rahner}

\paragraph{Nostra Aetate}, 1965. et par ailleurs, les conseils Oecuméniques travaillent la question (1971).
\paragraph{1964 : secrétariat pour les relations avec les non-chrétiens}. On définit l'autre par rapport à soi-même. Et maintenant, on appelle \textit{conseil pontifical inter-religieux} en 1988 : veut promouvoir le dialogue inter religieux : compréhension mutuelle, respect,... et pour cela, étudier les religions et former les personnes en lien à ce dialogue.

\paragraph{Fondation de l'ISTR} en 1967 par Henri Bouillard et Danielou, sj. ISTR : Théologie des Religions. Dès le début du XX, il y avait une chaire d'histoire des religions. En 1954, Institut d'Ethnologie et de Sociologie religieuse (sciences humaines). On prépare des futurs missionnaires à découvrir des nouvelles religions. 

\mn{Aujourd'hui, l'\textit{exotisme est partout}} dans le sens que les religions sont ici.

Promouvoir un regard chrétien sur les autres, et connaître l'autre en lui même. Danielou (histoire des religions), de Lubac (Théologie des religions), ...
\begin{Def}[la Science des religions]
science au sigulier, une science qui s'intéresse au religieux. 
\end{Def}

\subsection{Kairos de ce moment}
\paragraph{On commence à s'intéresser aux autres parce que malgré les missions, les grandes traditions ne se sont pas convertis}
Ces traditions disent quelque chose du dessein de Dieu. et qu'est ce qu'elles nous disent de Dieu.

\paragraph{Temps de la décolonisation} Levi strauss : la civilisation Européenne n'est pas la dernière; accueillir les autres avec bienveillance.

\paragraph{le sens des autres religions} Quel est leur place dans le plan de Dieu ? accidentel ?
on change de paradigme. Il ne s'agit plus d'\emph{apologétique} (fondamentale).

%------------------------------------------------------------------------------------
\section{Les autres religions dans le cadre de l’apologétique}
\subsection{De Thomas d’Aquin aux traités d’apologétique du XXe s.}
 

Elle était abordée dans les traités d'apologétique comme faire-valoir du christianisme et : 
Quelle est la vérité ? et de cette vérité, on peut juger les autres.
St Thomas d'Aquin, \mn{La somme contre les Gentils, 1260. Il s'adresse à des gens qui ne sont pas chrétiens} va développer une apologétique, dans le cadre de la mission des dominicains. L'Europe commence à intensifier les liens avec le monde musulman. Et il fallait donner des arguments pour la mission. 
\paragraph{Pas facile de les convaincre} car on ne connait pas très bien leurs textes. Avec les juifs, on avait un texte commun. Avec les musulmans, on n'a plus de texte commun, on a juste la \emph{raison}. Saint Thomas introduit cette manière de faire, apologétique sur un discours rationnel.
\paragraph{Erreur dans leur texte}. A la différence des paiens, comme les chrétiens étaient d'anciens paiens, il pouvent vraiment discuter avec eux. 
Certains d'autres eux ("Mahometan"). \sn{Il a fallu du temps pour qu'on prenne le terme d'Islam, utilisé par les musulmans. }
\begin{quote}
    « Réfuter toute les erreurs est difficile, pour deux raisons. La première, c’est que les affirmations
sacrilèges de chacun de ceux qui sont tombés dans l’erreur ne nous sont pas tellement connues que
nous puissions en tirer des arguments pour les confondre. C’était pourtant ainsi que faisaient les
anciens docteurs pour détruire les erreurs des païens, dont ils pouvaient connaître les positions, soit
parce qu’eux-mêmes avaient été païens, soit, du moins, parce qu’ils vivaient au milieu des païens et
qu’ils étaient renseignés sur leurs doctrines. La seconde raison, c’est que certains d’entre eux,
comme les Mahométans et les païens, ne s’accordent pas avec nous pour reconnaître l’autorité de
l’Ecriture, grâce à laquelle on pourrait les convaincre, alors qu’à l’encontre des Juifs, nous pouvons
disputer sur le terrain de l’Ancien Testament, et qu’à l’encontre des Hérétiques, nous pouvons
disputer sur le terrain du Nouveau Testament. Mahométans et Païens n’admettent ni l’un ni l’autre.
Force est alors de recourir à la raison naturelle à laquelle tous sont obligés de donner leur adhésion.
Mais la raison naturelle est faillible dans les choses de Dieu » \sn{Thomas d’Aquin, SCG, Livre I, II,
Paris, Lethielleux, 1961, 135.}
\end{quote}



Dans une apologétique chrétienne (1919), on y déduit une religion révélée, et une seule est véritable. P 135, \textit{le judaisme n'est pas la vraie religion}, p. 136, \textit{le bouddhisme n'est pas la vraie religion}.... il doit montrer la fausseté des autres religions.



Dans une autre livre, (1937), on présente les autres religions, pour ensuite les comparer du Christianime. 

% ------------------------------------------------------------
\subsection{La logique du regard chrétien sur les autres religions}



\paragraph{Place du surnaturel - Concile Vatican I - 1870} On peut connaître Dieu par la Raison mais surtout par la Révélation Surnaturelle, la raison ne suffisant pas. D'où des théologies révélées et des religions naturelles (hors judéo christianisme)

\paragraph{naturel vs surnaturel} L'Eglise reconnait les religions naturelles, qui reconnait l'inclination naturelle de l'homme vers Dieu. Le Christianisme est une religion révélée et va plus loin que la religion naturelle.
\begin{Def}[Surnaturel]
Le christianisme est une religion qui va plus loin que le naturel.
La révélation, grâce, va plus loin et découvrir l'aspect surnaturel. 
\end{Def}
Les autres religions ne sont pas mauvaise mais elle ne peuvent dépasser la raison. 
% ------------------------------------------------------------
\subsection{De l’apologétique à la mission}


la \textit{mission ad gentes} est influencée par cette vision. plus on est convaincu d'avoir raison et que les autres ont tord, une motivation
\textit{Maximum Illud} \mn{\href{https://www.vatican.va/content/benedict-xv/fr/apost_letters/documents/hf_ben-xv_apl_19191130_maximum-illud.html}{Maximum Illud}}

\begin{quote}
    Une fois l’Amérique découverte, une foule d’apôtres (…) se consacre à la protection des pauvres
indigènes, contre l’infâme tyrannie des hommes, afin de les libérer du très dur esclavage des
démons (Maximum illud).

En vérité, c’est un motif de grand étonnement de constater qu’après tant de si graves peines
endurées par les nôtres pour répandre la Foi (…) nombreux sont encore ceux qui gisent dans les
ténèbres et dans l’ombre de la mort, étant donné que le nombre des infidèles, selon des données
récentes, s’élève à un milliard (Maximum illud).

Il serait assez malvenu que les messagers de la vérité soient inférieurs aux ministres de l’erreur.
Ainsi donc, les séminaristes appelés par Dieu seront convenablement préparés pour les Missions
étrangères et devront être instruits dans toutes les disciplines, sacrées et profanes, nécessaires au
Missionnaire.(Maximum illud).

[Le missionnaire] est disposé à tout, à tolérer généreusement les désagréments, les vilénies, la faim,
les privations et même la mort la plus dure, pourvu qu’il puisse arracher une seule âme aux abîmes
de l’enfer.(Maximum illud).

Et quand les Supérieurs apprendront que leurs Missionnaires ont porté avec succès certaines
populations de l’abjecte superstition à la sagesse chrétienne et y ont fondé une Église assez stable,
ils permettront aussi que ces soldats vétérans du Christ se transfèrent ailleurs pour arracher un autre
peuple aux mains du diable et laissent à d’autres, sans regret, la tâche d’agrandir et d’améliorer ce
qu’ils ont eux-mêmes assuré au Christ (Maximum illud). \sn{Benoît XV, Encyclique Maximum illud, 1919}
\end{quote}

Benoit XV utilise des expressions très négative sur les autres religions, "contre la tyrannie des Démons". Nombreux sont encore ceux qui vivent dans les ténèbres et l"ombre de la mort". C'est donc un jugement dur ("infideles"). 

% ------------------------------------------------------------
\subsection{L’intérêt missionnaire de l’étude des autres religions}




\begin{quote}
    
« Pendant la période d’expansion missionnaire intense des XIXe-XXe siècles (…) disons que la
mission était vécue comme une conquête, un combat contre l’erreur et le mal destiné à sauver les
gens vivant dans les ténèbres et le péché, et risquant l’enfer. Cette mission se définissait comme 
l’implantation de l’Église par des apôtres originaires des terres de chrétienté et partant au loin dans
des pays païens. Une immense générosité, beaucoup d’amour pour les âmes et de compassion pour
les corps caractérisaient une entreprise assez sûre d’elle-même et se développant historiquement
dans un contexte de colonisation et d’expansion occidentale » (P. COULON, dans E. PISANI
(dir.), Religions et dialogues, p. 47). \sn{P. Coulon, dans E. Pisani (dir.), Religions et dialogues, Cerf, Paris, 2020.}
\end{quote}

\begin{Synthesis}
On avait une vision très surplombante des autres religions. 
\end{Synthesis}
%------------------------------------------------------------------------------------
\section{La théologie des religions : un regard complexe sur les autres religion}

 
\paragraph{Thils} constate que les autres religions se développement comme si le christianisme n'existait pas. 
\begin{quote}
     faut il s'efforcer de penser

\end{quote}

Eviter une pensée hors sol comme des mathématiques. Or, la théologie a une histoire. On ne peut pas penser les autres religions uniquement sur leur déclin.
Eviter de renier le caractère exclusif que nous attribuons au Christianisme. Tension pas simple à gérer.


\paragraph{le concile Vatican II} va tenir compte de la réalité, de l'histoire. La question n'est pas justifier le christianisme au détriment des autres. Le concile part de son discours sur les autres religions de ce qui est positif, et qui vient ensuite.

\begin{quote}
    
« [L’] activité [de l’Église] n’a qu’un but : tout ce qu’il y a de germes de bien dans le cœur et la
pensée des hommes ou dans leurs rites propres et leur culture, non pas seulement le laisser perdre,
mais le guérir, l’élever, l’achever (…) » (LG 17).\sn{Concile Vatican II, Lumen Gentium.}
\end{quote}

\paragraph{Quel sens ont les religions pour Dieu} On ne va pas se mettre à la perspective.

Commission Théologique des Religions  : ... dans la totalité de l'histoire du salut. ... religions concrète avec un contenu bien défini. 

% ------------------------------------------------------------
\subsection{Un regard critique}


\paragraph{Courant prophétique de l'Ancien Testament} A la dimension idolâtre et mortelle, une tradition critique prophétique.
On considérait le judaisme comme une hérésie, l'Islam comme le réceptacle des hérésies du IV\textsuperscript{ème siècle}. La théologie joue un rôle purificateur et critique : si on regarde l'autre de façon critique, il faut appliquer d'abord ces critères à nous mêmes.

\paragraph{Le christianisme ne serait pas alors une religion mais une révélation}  La critique appliquée d'abord à nous même nous permet d'éviter une approche subjectivisme. Rapport à l'autre; l'injustice, il faut la dénoncer qu'elle soit chez nous ou l'autre.

% ------------------------------------------------------------
\subsection{Un regard universaliste}


\paragraph{Les religions dans le plan du salut} On se souvient que les Pères Apologistes (II-III) ont essayé de montrer la \textit{semence du Verbe} dans d'autres traditions, en particulier la philosophie grecque : Origène, Justin, Athenagore. 

\paragraph{Vatican II}\sn{lumen gentium 17}
\begin{quote}
    17. Le caractère missionnaire de l’Église

En effet tout comme il a été envoyé par le Père, le Fils lui-même a envoyé ses Apôtres (cf. Jn 20, 21) en disant : « Allez donc, enseignez toutes les nations, les baptisant au nom du Père et du Fils et du Saint-Esprit, leur apprenant à observer tout ce que je vous ai prescrit. Et moi, je suis avec vous tous les jours jusqu’à la consommation des temps » (Mt 28, 19-20). Ce solennel commandement du Christ d’annoncer la vérité du salut, l’Église l’a reçu des Apôtres pour en poursuivre l’accomplissement jusqu’aux extrémités de la terre (cf. Ac 1, 8). C’est pourquoi elle fait siennes les paroles de l’Apôtre : « Malheur à moi si je ne prêchais pas l’Évangile » (1 Co 9, 16) : elle continue donc inlassablement à envoyer les hérauts de l’Évangile jusqu’à ce que les jeunes Églises soient pleinement établies et en état de poursuivre elles aussi l’œuvre de l’évangélisation. L’Esprit Saint la pousse à coopérer à la réalisation totale du dessein de Dieu qui a fait du Christ le principe du salut pour le monde tout entier. En prêchant l’Évangile, l’Église dispose ceux qui l’entendent à croire et à confesser la foi, elle les prépare au baptême, les arrache à l’esclavage de l’erreur et les incorpore au Christ pour croître en lui par la charité jusqu’à ce que soit atteinte la plénitude. Son activité a le résultat non seulement de ne pas se laisser perdre tout ce qu’il y a de germe de bien dans le cœur et la pensée des hommes ou de leurs rites propres et leur culture ; mais de le guérir, l’élever, l’achever pour la gloire de Dieu, la confusion du démon et le bonheur de l’homme. À tout disciple du Christ incombe pour sa part la charge de l’expansion de la foi [35]. Mais si le baptême peut être donné aux croyants par n’importe qui, c’est aux prêtres cependant qu’il revient de procurer l’édification du Corps par le sacrifice eucharistique en accomplissant les paroles de Dieu quand il dit par la voix du prophète : « De l’Orient jusqu’au couchant, mon Nom est grand parmi les nations, et en tous lieux est offert à mon Nom un sacrifice et une offrande pure » (Ml 1, 11) [36]. Ainsi, l’Église unit prière et travail pour que le monde entier dans tout son être soit transformé en Peuple de Dieu, en Corps du Seigneur et temple du Saint-Esprit, et que soient rendus dans le Christ, chef de tous, au Créateur et Père de l’univers, tout honneur et toute gloire.
\end{quote}


Si nous croyons au Dieu unique, c'est le dieu de toute l'histoire; aucune religion ne saurait être totalement étranger à Dieu.
\begin{quote}
    Il a fait que tous les hommes, sortis d'un seul sang, habitassent sur toute la surface de la terre, ayant déterminé la durée des temps et les bornes de leur demeure;

 il a voulu qu'ils cherchassent le Seigneur, et qu'ils s'efforçassent de le trouver en tâtonnant, bien qu'il ne soit pas loin de chacun de nous,
\sn{Paul à Athènes (Ac 17, 26-27)}
\end{quote}

C'est difficile de séparer la dimension naturelle et surnaturelle. Est ce qu'il n'y a pas une réponse à la grâce de Dieu dans les religions ? On ne peut pas évacuer le fait que les autres religions soient une réponse à cette \textit{proximité de Dieu qui vient à notre rencontre}. 



Guérir.
\sn{Claude Geffré} On est resté assez prudente entre une démarche critique et une approche universaliste.
\begin{quote}
    
« (La théologie des religions)\sn{C. Geffré, De Babel à la Pentecôte, essais de théologie interreligieuse, Paris, Cerf, 2006} est même beaucoup plus qu’un chapitre supplémentaire à l’intérieur
de l’organisation classique de la dogmatique. Elle tend à devenir l’horizon à partir duquel il
convient de réinterpréter les vérités les plus fondamentales de la foi chrétienne » (Geffré, Babel,
134).

« Il s’agissait en fait de réinterpréter certains énoncés fondamentaux du christianisme à la lumière
d’une expérience historique inédite (…) Une théologie responsable, c’est une théologie qui prend
en compte les dimensions historiques nouvelles de l’intelligence de la foi (…). Aujourd’hui, en ce
début du XXIe s., la théologie doit affronter un nouveau défi, celui du pluralisme religieux »
(Geffré, Babel, p. 27). 
\end{quote}
Il faut néanmoins convertir cette approche universaliste à une approche plus concrète, sous peine d'avoir une approche inclusiviste.

\begin{Ex}
Les droits de l'homme, abstrait, et on essaye d'appliquer ces droits un peu partout. Parfois, c'est tellement abstrait que cela pose un problème
\end{Ex}
\begin{Synthesis}
Or, la foi chrétienne n'est pas abstrait mais concret mais passe par des \textit{personnes}. Penser l'approche universaliste de façon \textit{eschatologique}

\end{Synthesis}

Claude Geffré \sn{Babel et Pentecôte} propose de dépasser \textit{l'impérialisme chrétien} pour entrer en véritable dialogue. Ce n'est pas une vérité abstraite qui surplombe mais qui se fait dans l'histoire. 

Commission Théologique Internationale 1980. Tension entre l'universel et l'aspect critique. Mais il manque ici un troisième regard, dialogale.


% ------------------------------------------------------------
\subsection{Un regard dialogal}
 
\begin{Synthesis}
Les autres traditions ne sont pas réductibles au Christianisme.
\end{Synthesis}


\mn{Roger-Pol Droit}

Si c'était le cas, tous les autres religions se seraient converties.

\paragraph{Accepter le pluralisme religieux, ces différences.} On peut mentionner les rencontres de Jésus avec les paiens, sans qu'il leur demande de le suivre ou d'aller à Jérusalem, à rebours d'un prosélytisme. 

Dans cette approche, la définition des autres religions n'est pas aisée, "voix de salut".
Dans les années post vatican II, les \textit{trentes Glorieuses du dialogue interreligieux}


\begin{Def}[Les différents modèles]
Différents modèles sont possibles : 
\bi
\item  Ecclésiocentrisme : correspond à l'exclusivisme
\item Christocentrisme : inclusivisme
\item Théocentrisme : pluralisme
\ei

\end{Def}

Les modèles sont intéressants car ils permettent d'éclairer mais il faut accepter les \textsc{paradoxes}. Comme Calcédoine, articuler les différents modèles. Si on est dans un modèle, on est dans \textbf{l'idéologie}. Les grands théologiens ont toujours essayé de rester dans le paradoxe, la position équilibrée. Il faut tenir ici les trois.

Henri de la Houille : Sortir de l'exclusivisme sans tomber sur le relativisme (Tout se vaut) ou le réductionisme (l'autre est comme moi).




%------------------------------------------------------------------------------------
\section{La notion de « religion » dans la théologie des religions}


\paragraph{le dialogue suppose un certain décentrement} pour regarder l'autre dans sa différence. mais le dialogue présuppose une partie commune avec l'autre : sinon, pas de dialogue. Ce qui est commun, c'est que ces Traditions sont considérées comme des religions.

\begin{Ex}
Ce que nous partageons avec l'Islam, ce n'est pas uniquement que nous sommes hommes mais que nous partageons une partie sociale
\end{Ex}


\subsection{Déterminer une base commune pour dialoguer}



Il faut définir ce commun.

\begin{enumerate}
    \item on va d'abord définir ce qu'est une religion
    \item comment les différentes religions historiques se présentent par rapport à cette définition de religion. Le Bouddhisme est il une religion ? non, au XIX, car on avait une vision de la religion platonicienne
    \item On y insère le christianisme, on se décentre
    \item Approche des sciences humaines
\end{enumerate}

\subsection{quelques difficultés de cette démarche}

\paragraph{une définition chrétienne de la religion} La plupart des concepts ont été créés en occident, et donc on risque de projeter l'essence du christianisme dans l'essence de la religion ("cheval de Troie" du christianisme) \sn{L'invention des religions, de Daniel Dubuisson, athée}

\paragraph{Une définition qui conduirait au relativisme} On remet en cause la foi chrétienne à partir d'un concept et non de sa Révélation. On glisse vers une \textit{philosophie des religions}

\paragraph{Y a t'il un accord sur la théologie des religions} Une formulation qui peut être développée par des athées. Est ce que les autres religions sont d'accord ?

\begin{Synthesis}
Comment la rencontre / dialogue avec les religions ne nous toucherait pas nous même ?
\end{Synthesis}




\subsection{Quelques difficultés}




%------------------------------------------------------------------------------------
\section{La théologie des religions comme le renouvellement de la théologie chrétienne
par le dialogue}
\subsection{Réinterpréter la théologie chrétienne à partir du dialogue interreligieux}


\paragraph{Se laisser convertir par la rencontre} mieux croire en Jésus Christ, en supposant par le fait qu'il passe par les autres, les événements de l'histoire et pas uniquement par les textes.

Commission internationale théologiques, 7 : la rencontre... nous conduit à nous réinterpréter et approfondir notre foi.

Claude Geffré : pas uniquement un chapitre de la dogmatique, ... reinterpréter les dogmes les plus fondamentaux de la religion.

\begin{Ex}
quand le christianisme a affronté le monde grec, il s'est réinterprété dans un autre langage. Aujourd'hui se joue un peu le même phénomène. 
\end{Ex}

Paul Tillich \sn{mort en 1965} écrivait : 
\begin{quote}
    penser l'unicité du christianisme face à la pluralité religieuse.
\end{quote}
On est impacté par l'autre.



\subsection{Du dialogue avec les religions au dialogue avec telle religion}
 
 
Selon la Tradition que nous rencontrons, nous allons interpréter différemment notre religion. 
\begin{Ex}
Claverie, Foucault, Thibirine : dans chaque contexte religieux, il y a une propre vision.
\end{Ex}
\begin{Ex}
La théorie de la substitution (Eglise replace le peuple d'Israel) n'est plus possible dans le dialogue avec les Juifs
\end{Ex}

\subsection{en conclusion}
\begin{Synthesis}
Dans un monde mondialisé, renouveler sa foi pour habiter ce monde de façon heureuse, pour annoncer la \textit{bonne nouvelle}.
La théologie est là pour nous soutenir pour rendre compte de la parole de Dieu. Relation de Paix tout en étant nous même.
\end{Synthesis}
Sinon, on risque de nous enfermer dans des cénacles, des francs maçonneries.
