\chapter{Introduction}

\mn{P. Xavier Gué, directeur ISTR, 2022, Théologie Chrétienne des Religions}

\bi 
\item Introduction générale
\item A. Une théologie prophétique et critique
\item B. Une théologie universelle et inclusive
\item C. Une théologie en dialogue et en chemin
\bi 
\item C-1 Le renouvellement du regard de l’Église sur le judaïsme
\item C-2 Nostra Aetate et l’approche dialogale
\item C.3 L’universalité de l’expérience religieuse
\item C.4 La religion comme médiation entre l’homme et le divin
\item C.5 Les religions comme des cultures
 \ei 
\ei 

\hypertarget{bibliographie-guxe9nuxe9rale}{%
\section{Bibliographie générale}\label{bibliographie-guxe9nuxe9rale}}

\begin{quote}
\emph{Reprise de la bibliographie d'Henri de la Hougue avec des
compléments. Une bibliographie complémentaire sera donnée avec le plan
de chaque séance.}

.
\end{quote}

\hypertarget{thuxe9ologie-chruxe9tienne-des-religions}{%
\section{Théologie chrétienne des
religions}\label{thuxe9ologie-chruxe9tienne-des-religions}}

\begin{quote}
AEBISCHER-CRETTOL Monique, \emph{Vers un œcuménisme interreligieux},
Cogitatio fidei n° 221, Cerf, Paris, 2001 (777 p.)

AVELINE Jean-Marc, \emph{L'enjeu christologique en théologie des
religions}, Cogitatio fidei n°227, Paris, 2003 (756p.)

BASSET Jean-Claude, \emph{Le dialogue interreligieux, histoire et
avenir,} Coll. Cogitatio Fidei n°197, Cerf, Paris, 1996 (503p.).

de BETHUNE Pierre-François, « Le dialogue des spiritualités », in
\emph{Chemins de dialogue} n°13 Marseille, 1999, pp.67-79.

BOESPFLUG François ET LABBE Yves (DIR) : \emph{Assise, 10 ans après},
Cerf, Paris, 1996 (302p)

BOUSQUET François et LA HOUGUE Henri de, \emph{Le dialogue
interreligieux, le christianisme face aux autres traditions}, DDB, Paris
2009 (224p.)

CAPERAN Louis, \emph{Le problème du salut des infidèles. Essai
historique,} Grand Séminaire, Toulouse, 1934 (616p.).

CAPERAN Louis, \emph{Le problème du salut des infidèles. Essai
théologique,} Grand Séminaire, Toulouse, 1934 (150p.).

COMMISSION THEOLOGIQUE INTERNATIONALE : \emph{Le Christianisme et les
religions}, Centurion/Cerf, Paris 1997 (99 p.)

CONGREGATION POUR LA DOCTRINE DE LA FOI: Déclaration « Dominus Iesus »
sur l'unicité et l'universalité salvifique de Jésus-Christ et de son
Eglise, in Documentation Catholique n° 2233, octobre 2000

COMEAU Geneviève, \emph{Grâce à l'autre, le pluralisme religieux, une
chance pour la foi}, Editions de l'Atelier, Paris, 2004 (159 p.)

COMEAU Geneviève, \emph{Le dialogue interreligieux}, Fidélité, 2008
(87p.)

CONSEIL PONTIFICAL POUR LE DIALOGUE INTERRELIGIEUX : \emph{Le dialogue
interreligieux dans l'enseignement officiel de l'Eglise catholique}
(1963-2005), Editions de Solesmes, Solesmes, 2006 (1700p.)

COURAU Thierry-Marie et VIVIER-MURESAN Anne-Sophie, \emph{Dialogue et
Conversion, mission impossible} ?, DDB, Paris, 2012 (245 p.)

DANIELOU Jean, \emph{Le mystère du salut des nations,} Seuil, Paris,
1946 (147p.)

DORE Joseph, « La présence du Christ dans les religions non chrétiennes
», in \emph{Chemins de dialogue} n°9, Marseille, 1997, pp. 13-50.

DOURNES Jacques, \emph{Dieu aime les païens,} Coll. Théologie n°54,
Aubier, Paris, 1963 (172p.).

DUPUIS Jacques, Jésus-Christ à la rencontre des religions, Coll. Jésus
et Jésus-Christ n°39, Desclée, Paris, 1989 (344 p.)

DUPUIS Jacques, \emph{Vers une théologie chrétienne du pluralisme
religieux}, Cogitatio fidei n°200, Cerf, Paris, 1999 (657p.)

FEDOU, M., \emph{Les religions selon la foi chrétienne}, Cerf, Paris,
1996 (123p.)

FEDOU Michel (dir.) \emph{Le Fils unique et ses frères, unicité du
Christ et pluralisme religieux}, Editions faculté jésuite de Paris, 2002
(164p.), notamment : « Le débat sur l'unicité du Christ : problématiques
actuelles et témoignages de la tradition », pp. 9-48

FITZGERALD Michael L., « L'expérience religieuse dans un contexte
pluraliste », in \emph{Pro Dialoguo} n°89, Rome, 1995, pp. 143-154.

FITZGERALD Michael L., « Toward a Christian Theology of Religious
Pluralism », in \emph{Pro Dialoguo} n°108, Rome, 2001, pp. 334-341.

FITZGERALD Michael L., « The Relevance of \emph{Nostra Aetate} in
Changed Times », in \emph{Islamochristiana} n°32, Rome, 2006, pp. 63-87.

GEFFRE Claude, \emph{De Babel à la Pentecôte, essais de théologie
interreligieuse}, Cogitatio fidei n° 247, cerf, Paris, 2006 (363p.)

GEFFRE Claude, \emph{Le christianisme comme religion de l'Evangile},
Cerf, Paris, 2012 (352p.) GUIBERT Vincent, \emph{A l'ombre de l'Esprit},
Parole et Silence, Paris, 2009 (380p.)

HICK John, \emph{God and the universe of Faiths,} Macmillan Press,
Houndmills, Basingstoke, Hampshire and London 1988 (201p.).

HICK John (Ed.), \emph{The Myth of Christian Uniqueness,} SCM Press LTD,
London, 1988 (227p.). LA HOUGUE Henri de, \emph{L'estime de la foi des
autres}, DDB, Paris, 2011 (361p.)

LA HOUGUE Henri de, "la distinction entre la foi des croyants et la
croyance dans les autres religions dans "Dominus Iesus", RSR
Janvier-mars 2011, Tome 99/1, pp. 105-123.

LA HOUGUE, H. (de), \emph{L'Église et la diversité des religions},
Paris, Salvator, 2020.

KÜNG Hans, « Pour une théologie œcuménique des religions », in
\emph{Concilium} n° 203, Paris, 1986, pp. 151-159

LEVRAT Jacques, \emph{Dynamique de la rencontre: Une approche
anthropologique du dialogue}, L'Harmattan, Paris, 1999 (206p.)

MAURIER Henri, \emph{Essai d'une théologie du paganisme}, Orante, Paris,
1965, (327p.)

O'LEARY Joseph S., \emph{La vérité chrétienne à l'âge du pluralisme
religieux,} Coll. Cogitatio Fidei n°181, Cerf, Paris, 1994 (330p.).

PANIKKAR Raimon, \emph{L'expérience de Dieu}, Albin Michel, Paris, 2002
(218p.) PANIKKAR Raimon, \emph{Le dialogue intrareligieux,}
Aubier-Montaigne, 1992 (175p.)

PISANI, E. (dir.), Maximum illud. \emph{Aux sources d'une nouvelle ère
missionnaire}, Paris, Cerf, 2020. PISANI, E., (dir.), \emph{Religions et
dialogues. 50 ans d'histoire de l'ISTR de Paris}, Paris, Cerf, 2020

PIVOT Maurice, \emph{Au pays de l'autre}, Editions de l'Atelier, paris,
2009, (191 p.)

PLOUX Jean-Marie, \emph{Le dialogue change-t-il la foi ?} Editions de
l'Atelier, Paris 2004 (207 p.)

RIGAL Jean, « La culture d'aujourd'hui invite-t-elle au relativisme
religieux ? » in \emph{Esprit et vie} n° 209, Mars 2009, pp. 19-29

SESBOÜE Bernard, \emph{Hors de l'Eglise pas de salut}, Desclée de
Brouwer, Paris, 2004 (395p.)

SMITH Wilfred C., \emph{Faith and Belief: the difference between Them,}
Oneworld, Oxford, 1998 (337p.).
\end{quote}

\hypertarget{contexte-musulman}{%
\section{Contexte musulman}\label{contexte-musulman}}

\begin{quote}
GAUDEUL Jean-Marie, \emph{Disputes ? Ou Rencontres ? L'islam et le
christianisme au fil des siècles, t.1 Survol historique ; t.2 Textes
témoins,} PISAI, Rome, 1998 (379p et 398p.).

GROUPE DE RECHERCHE ISLAMO-CHRETIEN (GRIC), \emph{Ces Écritures qui nous
questionnent : La Bible et le Coran}, Centurion, Paris 1987 (159p.).

LA HOUGUE Henri de et JAZARI MAMOEI Saeid, \emph{Dieu est-il l'auteur de
la Bible et du Coran}, Salvator, Paris 2016 (219p.)

RÖMER, T., CHABBI, J., \emph{Dieu de la Bible, Dieu du Coran}, Seuil,
Paris, 2020.

SALENSON christian, Christian de Chergé, pour une théologie de
l'espérance, Bayard Centurion, Paris, 2009 (253 p.)

VAN NISPEN TOT SEVENAER Christian, \emph{Chrétiens et musulmans, frères
devant Dieu ?} Editions de l'Atelier, Paris 2004 (189 p.) réédition 2008
\end{quote}

\hypertarget{contexte-des-religions-dasie}{%
\section{Contexte des religions
d'Asie}\label{contexte-des-religions-dasie}}

\begin{quote}
AMALADOSS Michaël, « Qui suis-je ? Un catholique hindou », in
\emph{Christus} n°86, Paris, 1975, pp. 159-171. AMALADOSS Michaël, «
Vivre dans un monde pluraliste », in \emph{Christus} n°150, Paris, 1991,
pp. 159-170.

FEDOU Michel, \emph{Regards asiatiques sur le Christ,} Coll. Jésus et
Jésus-Christ n°77, Desclée, Paris, 1998 (297p.).

GIRA Dennis ET SCHEUER Jacques (dir.), \emph{Vivre de plusieurs
religions, promesse ou illusion ?} Editions de l'Atelier, Paris 2000
(208 p.)

GIRA Dennis (et Fabrice MIDAL), \emph{Jésus Bouddha, quelle rencontre
possible ?,} Bayard, 2006 (191p.).
\end{quote}

\hypertarget{contexte-juif}{%
\section{Contexte juif}\label{contexte-juif}}

\begin{quote}
BUBER Martin, \emph{Deux types de foi, foi juive et foi chrétienne,}
Coll. Patrimoines Judaïsme, Cerf, Paris, 1991 (166p.).

COMEAU Geneviève, \emph{Juifs et chrétiens. le nouveau dialogue},
Editions de l'Atelier, Paris, 2001 (159p.)

\emph{COMMISSION BIBLIQUE PONTIFICALE} Le peuple juif et ses saintes
Écritures dans la Bible chrétienne, \emph{Cerf, Paris, 2001.}

THOMA Clemens, \emph{Théologie chrétienne du judaïsme,} Parole et
Silence, Paris, 2005 (271p.).
\end{quote}


