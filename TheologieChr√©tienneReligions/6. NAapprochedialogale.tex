\chapter{Nostra Aetate et l’approche dialogale}

\mn{14/3 
Théologie des religions C/2 }

\section{Bibliograohie}
 AVELINE, J.-M., « Evolution des problématiques en théologie des religions », RSR 94 (2006) 499-522. 
 
 COMEAU, G., Le dialogue interreligieux, Namur 2008. COMMISSION THEOLOGIQUE INTERNATIONALE, Le christianisme et les religions, Rome 1997.
 
 CONSEIL PONTIFICAL POUR LE DIALOGUE INTERRELIGIEUX ET CONGRÉGATION POUR L’ÉVANGELISATION DES PEUPLES, Dialogue et annonce. Réflexions et orientations concernant le dialogue interreligieux et l’annonce de l’Evangile, Rome 1991. 
 
 COUREAU, T-M., Le salut comme dia-logue. De saint Paul VI à François, Paris 2018. 
 
 DUPUIS, J., Vers une théologie chrétienne du pluralisme religieux, Paris 1997. GEFFRE, C., De Babel à Pentecôte. Essais de théologie interreligieuse, Paris 2006. 
 
 GEFFRE, C., « Où en est la théologie des religions vingt ans après Assise ? » dans BOUSQUET, F. et LA HOUGUE (de), H. (éd), Le dialogue interreligieux. Le christianisme face aux autres traditions, Paris 2009, 173-200. GEFFRE, C., Le christianisme comme religion de l’Evangile, Paris 2012. 
 
 JEAN-PAUL II, Redemptoris Missio, Rome 1991. PAUL VI, Ecclesiam suam, Rome 1964. SCHEUER, J., « A 50 ans de Nostra Aetate. Dialogue interreligieux et théologie des religions », Revue théologie de Louvain 46 (2015)153-177.
 
 
 
 
 \section{Introduction}
 
 NA reconnait de façon positive les autres religions. 
 L'altérité entre le judaisme et le christianisme  s'etend il aux autres religions ?
 Est ce un pluralisme de fait "les autres religions existent" ou est ce que cela fait partie du plan de Dieu (pluralisme de droit).
 
 
 \section{Nostra Aetate : un regard positif sur les autres religions}
 
 \subsection{Commentaire du Nostra Aetate}
 
 \paragraph{Le plan de la déclaration}  Contient 5 paragraphes, celui sur le judaïsme est le plus long.
Le 5ème paragraphe est un appel à la fraternité.
 Après une explication anthrologique du fait religieux (homme « un être religieux »), les religions sont 


\begin{quote}
  \textbf{  Préambule}
À notre époque où le genre humain devient de jour en jour plus étroitement uni et où les relations entre les divers peuples se multiplient, l’Église examine plus attentivement quelles sont ses relations avec les religions non chrétiennes. Dans sa tâche de promouvoir l’unité et la charité entre les hommes, et aussi entre les peuples, elle examine ici d’abord ce que les hommes ont en commun et qui les pousse à vivre ensemble leur destinée.
Tous les peuples forment, en effet, une seule communauté ; ils ont une seule origine, puisque Dieu a fait habiter tout le genre humain sur toute la face de la terre [1] ; ils ont aussi une seule fin dernière, Dieu, dont la providence, les témoignages de bonté et les desseins de salut s’étendent à tous [2], jusqu’à ce que les élus soient réunis dans la Cité sainte, que la gloire de Dieu illuminera et où tous les peuples marcheront à sa lumière [3].
Les hommes attendent des diverses religions la réponse aux énigmes cachées de la condition humaine, qui, hier comme aujourd’hui, agitent profondément le cœur humain : Qu’est-ce que l’homme? Quel est le sens et le but de la vie? Qu’est-ce que le bien et qu’est-ce que le péché? Quels sont l’origine et le but de la souffrance? Quelle est la voie pour parvenir au vrai bonheur? Qu’est-ce que la mort, le jugement et la rétribution après la mort ? Qu’est-ce enfin que le mystère dernier et ineffable qui embrasse notre existence, d’où nous tirons notre origine et vers lequel nous tendons ?


\textbf{2. Les diverses religions non chrétiennes}


Depuis les temps les plus reculés jusqu’à aujourd’hui, on trouve dans les différents peuples une certaine perception de cette force cachée qui est présente au cours des choses et aux événements de la vie humaine, parfois même une reconnaissance de la Divinité suprême, ou même d’un Père. Cette perception et cette reconnaissance pénètrent leur vie d’un profond sens religieux. Quant aux religions liées au progrès de la culture, elles s’efforcent de répondre aux mêmes questions par des notions plus affinées et par un langage plus élaboré. Ainsi, dans l’hindouisme, les hommes scrutent le mystère divin et l’expriment par la fécondité inépuisable des mythes et par les efforts pénétrants de la philosophie ; ils cherchent la libération des angoisses de notre condition, soit par les formes de la vie ascétique, soit par la méditation profonde, soit par le refuge en Dieu avec amour et confiance. Dans le bouddhisme, selon ses formes variées, l’insuffisance radicale de ce monde changeant est reconnue et on enseigne une voie par laquelle les hommes, avec un cœur dévot et confiant, pourront acquérir l’état de libération parfaite, soit atteindre l’illumination suprême par leurs propres efforts ou par un secours venu d’en haut. De même aussi, les autres religions qu’on trouve de par le monde s’efforcent d’aller, de façons diverses, au-devant de l’inquiétude du cœur humain en proposant des voies, c’est-à-dire des doctrines, des règles de vie et des rites sacrés.
L’Église catholique ne rejette rien de ce qui est vrai et saint dans ces religions. Elle considère avec un respect sincère ces manières d’agir et de vivre, ces règles et ces doctrines qui, quoiqu’elles diffèrent sous bien des rapports de ce qu’elle-même tient et propose, cependant reflètent souvent un rayon de la vérité qui illumine tous les hommes. Toutefois, elle annonce, et elle est tenue d’annoncer sans cesse, le Christ qui est « la voie, la vérité et la vie » (Jn 14, 6), dans lequel les hommes doivent trouver la plénitude de la vie religieuse et dans lequel Dieu s’est réconcilié toutes choses [4]. Elle exhorte donc ses fils pour que, avec prudence et charité, par le dialogue et par la collaboration avec les adeptes d’autres religions, et tout en témoignant de la foi et de la vie chrétiennes, ils reconnaissent, préservent et fassent progresser les valeurs spirituelles, morales et socio-culturelles qui se trouvent en eux.


\textbf{3. La religion musulmane}


L’Église regarde aussi avec estime les musulmans, qui adorent le Dieu unique, vivant et subsistant, miséricordieux et tout-puissant, créateur du ciel et de la terre [5], qui a parlé aux hommes. Ils cherchent à se soumettre de toute leur âme aux décrets de Dieu, même s’ils sont cachés, comme s’est soumis à Dieu Abraham, auquel la foi islamique se réfère volontiers. Bien qu’ils ne reconnaissent pas Jésus comme Dieu, ils le vénèrent comme prophète ; ils honorent sa Mère virginale, Marie, et parfois même l’invoquent avec piété. De plus, ils attendent le jour du jugement, où Dieu rétribuera tous les hommes après les avoir ressuscités. Aussi ont-ils en estime la vie morale et rendent-ils un culte à Dieu, surtout par la prière, l’aumône et le jeûne.
Même si, au cours des siècles, de nombreuses dissensions et inimitiés se sont manifestées entre les chrétiens et les musulmans, le saint Concile les exhorte tous à oublier le passé et à s’efforcer sincèrement à la compréhension mutuelle, ainsi qu’à protéger et à promouvoir ensemble, pour tous les hommes, la justice sociale, les valeurs morales, la paix et la liberté.

\textbf{4. La religion juive}


Scrutant le mystère de l’Église, le saint Concile rappelle le lien qui relie spirituellement le peuple du Nouveau Testament à la lignée d’Abraham.
L’Église du Christ, en effet, reconnaît que les prémices de sa foi et de son élection se trouvent, selon le mystère divin du salut, chez les patriarches, Moïse et les prophètes. Elle confesse que tous les fidèles du Christ, fils d’Abraham selon la foi [6], sont inclus dans la vocation de ce patriarche, et que le salut de l’Église est mystérieusement préfiguré dans la sortie du peuple élu hors de la terre de servitude. C’est pourquoi l’Église ne peut oublier qu’elle a reçu la révélation de l’Ancien Testament par ce peuple avec lequel Dieu, dans sa miséricorde indicible, a daigné conclure l’antique Alliance, et qu’elle se nourrit de la racine de l’olivier franc sur lequel ont été greffés les rameaux de l’olivier sauvage que sont les Gentils [7]. L’Église croit, en effet, que le Christ, notre paix, a réconcilié les Juifs et les Gentils par sa croix et en lui-même, des deux, a fait un seul [8].
L’Église a toujours devant les yeux les paroles de l’apôtre Paul sur ceux de sa race « à qui appartiennent l’adoption filiale, la gloire, les alliances, la législation, le culte, les promesses et les patriarches, et de qui est né, selon la chair, le Christ » (Rm 9, 4-5), le Fils de la Vierge Marie. Elle rappelle aussi que les Apôtres, fondements et colonnes de l’Église, sont nés du peuple juif, ainsi qu’un grand nombre des premiers disciples qui annoncèrent au monde l’Évangile du Christ.
Selon le témoignage de l’Écriture Sainte, Jérusalem n’a pas reconnu le temps où elle fut visitée [9] ; les Juifs, en grande partie, n’acceptèrent pas l’Évangile, et même nombreux furent ceux qui s’opposèrent à sa diffusion [10]. Néanmoins, selon l’Apôtre, les Juifs restent encore, à cause de leurs pères, très chers à Dieu, dont les dons et l’appel sont sans repentance [11]. Avec les prophètes et le même Apôtre, l’Église attend le jour, connu de Dieu seul, où tous les peuples invoqueront le Seigneur d’une seule voix et « le serviront sous un même joug » (So 3, 9) [12].
Du fait d’un si grand patrimoine spirituel, commun aux chrétiens et aux Juifs, le saint Concile veut encourager et recommander la connaissance et l’estime mutuelles, qui naîtront surtout d’études bibliques et théologiques, ainsi que d’un dialogue fraternel. Encore que des autorités juives, avec leurs partisans, aient poussé à la mort du Christ [13], ce qui a été commis durant sa Passion ne peut être imputé ni indistinctement à tous les Juifs vivant alors, ni aux Juifs de notre temps. S’il est vrai que l’Église est le nouveau Peuple de Dieu, les Juifs ne doivent pas, pour autant, être présentés comme réprouvés par Dieu ni maudits, comme si cela découlait de la Sainte Écriture. Que tous donc aient soin, dans la catéchèse et la prédication de la Parole de Dieu, de n’enseigner quoi que ce soit qui ne soit conforme à la vérité de l’Évangile et à l’esprit du Christ.
En outre, l’Église, qui réprouve toutes les persécutions contre tous les hommes, quels qu’ils soient, ne pouvant oublier le patrimoine qu’elle a en commun avec les Juifs, et poussée, non pas par des motifs politiques, mais par la charité religieuse de l’Évangile, déplore les haines, les persécutions et les manifestations d’antisémitisme, qui, quels que soient leur époque et leurs auteurs, ont été dirigées contre les Juifs.
D’ailleurs, comme l’Église l’a toujours tenu et comme elle le tient encore, le Christ, en vertu de son immense amour, s’est soumis volontairement à la Passion et à la mort à cause des péchés de tous les hommes et pour que tous les hommes obtiennent le salut. Le devoir de l’Église, dans sa prédication, est donc d’annoncer la croix du Christ comme signe de l’amour universel de Dieu et comme source de toute grâce.


\textbf{5. La fraternité universelle excluant toute discrimination}


Nous ne pouvons invoquer Dieu, Père de tous les hommes, si nous refusons de nous conduire fraternellement envers certains des hommes créés à l’image de Dieu. La relation de l’homme à Dieu le Père et la relation de l’homme à ses frères humains sont tellement liées que l’Écriture dit : « Qui n’aime pas ne connaît pas Dieu » (1 Jn 4, 8). Par là est sapé le fondement de toute théorie ou de toute pratique qui introduit entre homme et homme, entre peuple et peuple, une discrimination en ce qui concerne la dignité humaine et les droits qui en découlent.
L’Église réprouve donc, en tant que contraire à l’esprit du Christ, toute discrimination ou vexation dont sont victimes des hommes en raison de leur race, de leur couleur, de leur condition ou de leur religion. En conséquence, le saint Concile, suivant les traces des saints Apôtres Pierre et Paul, prie ardemment les fidèles du Christ « d’avoir au milieu des nations une belle conduite » (1 P 2, 12), si c’est possible, et de vivre en paix, pour autant qu’il dépend d’eux, avec tous les hommes [14], de manière à être vraiment les fils du Père qui est dans les cieux [15].
Tout l’ensemble et chacun des points qui ont été édictés dans cette déclaration ont plu aux Pères du Concile. Et Nous, en vertu du pouvoir apostolique que Nous tenons du Christ, en union avec les vénérables Pères, Nous les approuvons, arrêtons et décrétons dans le Saint-Esprit, et Nous ordonnons que ce qui a été ainsi établi en Concile soit promulgué pour la gloire de Dieu.
Rome, à Saint-Pierre, le 28 octobre 1965.
Moi, Paul, évêque de l’Église catholique.


\end{quote}
\paragraph{Un parallèle avec Rm 1}

   \begin{quote}
       « Ce que l’on peut connaître de Dieu est pour eux manifeste : Dieu le leur a manifesté. En effet, depuis la création du monde, ses perfections invisibles, éternelle puissance et divinité, sont visibles dans ses œuvres pour l’intelligence ; ils sont donc inexcusables, puisque, connaissant Dieu, ils ne lui ont rendu ni la gloire ni l’action de grâce qui reviennent à Dieu » (Rm 1, 19-21).


   \end{quote}
   
   Tout le monde est bénéficiaire à cette manifestation de Dieu, d’où la culpabilité de ceux qui n’ont pas reconnu les vrais dieux. Or, dans NA, le texte reprend les prémices de Paul mais pour une autre interprétation : 
   \begin{quote}
       Depuis les temps les plus reculés jusqu’à aujourd’hui, on trouve dans les différents peuples une certaine perception de cette force cachée qui est présente au cours des choses et aux événements de la vie humaine, parfois même une reconnaissance de la Divinité suprême, ou même d’un Père. Cette perception et cette reconnaissance pénètrent leur vie d’un profond sens religieux. (NA 2)
   \end{quote}

Ce qui est intéressant, c’est qu’ils ont une \textit{reconnaissance de la divinité}.

   \paragraph{De Lumen Gentium (1964) à Nostra aetate (1965)}
   Dans LG 16, on était sur le lien entre les Traditions religieuses et l'Eglise. Ce qui est intéressant, c'est de voir comment sont cités les autres religions, du plus proches au plus lointains : Eglise, Juifs, Musulmans, les autres qui cherchent...
   \begin{quote}
       16. Les non-chrétiens

Enfin, pour ceux qui n’ont pas encore reçu l’Évangile, sous des formes diverses, eux aussi sont ordonnés au Peuple de Dieu [32] et, en premier lieu, ce peuple qui reçut les alliances et les promesses, et dont le Christ est issu selon la chair (cf. Rm 9, 4-5), peuple très aimé du point de vue de l’élection, à cause des Pères, car Dieu ne regrette rien de ses dons ni de son appel (cf. Rm 11, 28-29). Mais le dessein de salut enveloppe également ceux qui reconnaissent le Créateur, en tout premier lieu les musulmans qui, professant avoir la foi d’Abraham, adorent avec nous le Dieu unique, miséricordieux, futur juge des hommes au dernier jour. Et même des autres, qui cherchent encore dans les ombres et sous des images un Dieu qu’ils ignorent, de ceux-là mêmes Dieu n’est pas loin, puisque c’est lui qui donne à tous vie, souffle et toutes choses (cf. Ac 17, 25-28), et puisqu’il veut, comme Sauveur, amener tous les hommes au salut (cf. 1 Tm 2, 4). En effet, ceux qui, sans qu’il y ait de leur faute, ignorent l’Évangile du Christ et son Église, mais cherchent pourtant Dieu d’un cœur sincère et s’efforcent, sous l’influence de sa grâce, d’agir de façon à accomplir sa volonté telle que leur conscience la leur révèle et la leur dicte, eux aussi peuvent arriver au salut éternel [33]. À ceux-là mêmes qui, sans faute de leur part, ne sont pas encore parvenus à une connaissance expresse de Dieu, mais travaillent, non sans la grâce divine, à avoir une vie droite, la divine Providence ne refuse pas les secours nécessaires à leur salut. En effet, tout ce qui, chez eux, peut se trouver de bon et de vrai, l’Église le considère comme une préparation évangélique [34] et comme un don de Celui qui illumine tout homme pour que, finalement, il ait la vie. Bien souvent, malheureusement, les hommes, trompés par le démon, se sont égarés dans leurs raisonnements, ils ont délaissé le vrai Dieu pour des êtres de mensonge, servi la créature au lieu du Créateur (cf. Rm 1, 21.25) 21.25) ou bien, vivant et mourant sans Dieu dans ce monde, ils sont exposés aux extrémités du désespoir. C’est pourquoi l’Église, soucieuse de la gloire de Dieu et du salut de tous ces hommes, se souvenant du commandement du Seigneur : « Prêchez l’Évangile à toutes créatures» (Mc 16, 16), met tout son soin à encourager et soutenir les missions.
   \end{quote}
       LG cite le verset Rm 1, 25 pour légitimer la mission (LG 17) : aspect missionnaire. Cet aspect disparait dans NA. 
       \begin{quote}
         L’Église catholique ne rejette rien de ce qui est vrai et saint dans ces religions. Elle considère avec un respect sincère ces manières d’agir et de vivre, ces règles et ces doctrines qui, quoiqu’elles diffèrent sous bien des rapports de ce qu’elle-même tient et propose, cependant reflètent souvent un rayon de la vérité qui illumine tous les hommes.  
       \end{quote}

Parfois une économie du langage est préférable : viser les non-dits. Ce n’est pas parce que l’étranger est étrange que c’est une menace.   

   
   \subparagraph{NA part de l'homme} LG part de l'Eglise pour arriver à la mission quand NA part de l'homme, vision universaliste. 
   
   Hospitalité du texte NA par le Cardinal Béa qui compare le texte au grain de Senevé, développement inattendu.
   
   
   

Ch. de \textit{Dialogue et Annonce}\sn{
\href{https://relations-catholiques-musulmans.cef.fr/ressources/textes/textes-de-reference-de-leglise/4399-dialogue-et-annonce/}{Dialogue et annonce}}, texte qui relit NA. Evangélisation veut dire \textit{rendre neuve l'humanité elle-même}. Les convertir là où ils sont. Si un homme vit dans tel culture, dans sa culture, il va se renouveler dans sa culture. Vision large .
Un autre terme d'Evangélisation, l'annonce du Kerygme et on confond parfois les deux.

\begin{quote}
    L’\textbf{évangélisation}\sn{Dialogue et Annonce, 8}

 Le terme mission évangélisatrice, ou plus simplement évangélisation, se réfère à la mission de l’Église dans son ensemble. Dans l’exhortation apostolique Evangelii nuntiandi, le terme «évangélisation» est utilisé de différentes manières. Il signifie «porter la Bonne Nouvelle à toute l’humanité et, par son impact, transformer du dedans, rendre neuve l’humanité elle-même» (Evangelii nuntiandi, 18). Ainsi, par l’évangélisation, l’Eglise «cherche à convertir, par la seule énergie divine du Message qu’elle annonce, les consciences personnelles et collectives, les activités dans lesquelles les hommes sont engagés, leurs manières de vivre, et les milieux concrets dans lesquels ils vivent» (ibid.). L’Eglise accomplit sa mission d’évangélisation dans des activités variées. Il y a donc un sens large du concept d’évangélisation. Cependant, dans le même document, évangélisation est aussi pris dans un sens plus spécifique comme «l’annonce claire et sans ambiguïté du Seigneur Jésus» (Evangelii nuntiandi, 22). L’exhortation dit que «cette annonce, kerygma, prédication et catéchèse, occupe une place tellement importante dans l’évangélisation qu’elle en est souvent synonyme; cependant, ce n’est qu’un aspect de l’évangélisation» (ibid.). Dans ce document, le terme mission évangélisatrice est utilisé pour évangélisation au sens large, tandis que le sens plus spécifique est rendu par le terme annonce.
\end{quote}


Nous ne vendons pas Jésus Christ comme des savonettes, mais nous avons à être des serviteurs du ROyaume de Dieu.

   \paragraph{Le « commun » entre les religions  }
Par exemple, le bouddhisme est présenté comme une libération. \textit{désir de la libération de la souffrance}. 
L'Islam, \textit{Dieu Un et parlant aux hommes, jugement et vie morale}.

\begin{Synthesis}
   On ne met pas les différences en avant mais ce qui peut rapprocher
\end{Synthesis}
   \paragraph{Invitation au dialogue}
   Dans \textit{Dialogue et Annonce}, le document souligne le dialogue (\textit{B) La place du dialogue interreligieux dans la mission évangélisatrice de l’Église}). POurtant, le dialogue n'est mentionné que deux fois : 
   \begin{quote}
       Elle exhorte donc ses fils pour que, avec prudence et charité, par le dialogue et par la collaboration avec les adeptes d’autres religions, et tout en témoignant de la foi et de la vie chrétiennes, ils reconnaissent, préservent et fassent progresser les valeurs spirituelles, morales et socio-culturelles qui se trouvent en eux.
       NA 2
   \end{quote}
   et avec les juifs
   \begin{quote}
   Du fait d’un si grand patrimoine spirituel, commun aux chrétiens et aux Juifs, le saint Concile veut encourager et recommander la connaissance et l’estime mutuelles, qui naîtront surtout d’études bibliques et théologiques, ainsi que d’un dialogue fraternel.
       NA 4
   \end{quote}
   
   \subparagraph{Fraternité au coeur de l'Eglise}
   Michel Dujarier : on parle de \textit{la Fraternité}, Eglise est le signe de cette fraternité. 
   Le Dialogue est intrinsèque à la mission de l'Eglise. \mn{Lire Fraterni Tutti}
   
 \subsection{Pourquoi la relation avec le judaïsme fonde plus largement la théologie des religions ?}
  
  NA est une reflexion avec le judaisme. Comme les pères conciliaires n'étaient pas d'accord et le seul moyen était d'ouvrir aux autres religions.
  
  \begin{quote}
      « À l’origine, Saint Jean XXIII avait proposé que le Concile promulgue un Tractatus de Iudaeis, mais à la fin il fut décidé que Nostra Ætate prendrait en considération toutes les grandes religions mondiales. Cependant le quatrième article de cette Déclaration conciliaire, qui préfigure un nouveau rapport théologique avec le judaïsme, représente en quelque sorte le coeur de ce document où une place est faite également aux rapports de l’Église catholique avec les autres religions. En ce sens, les rapports de l’Église catholique avec le judaïsme peuvent être considérés comme le catalyseur qui a poussé le Concile à déterminer ses rapports avec les autres grandes religions mondiales. » (\textit{Les dons sont irrévocables, 19). } 
  \end{quote}
  
  
  NA est un document \textsc{Accidentel} qui est devenu \textit{théologique}. Altérité durable entre le christianisme et les autres religions. 
  
  
   \subsection{ Pluralisme religieux et regard positif de l’Église }
 
 Ce constat explique pourquoi on peut parler d'une \textit{théologie du pluralisme religieux}. La pluralité appartiendrait peut être au plan de Dieu. 
 
 \begin{quote}
     « Aujourd’hui, dans notre monde caractérisé par la rapidité des communications, la mobilité des peuples, l’interdépendance, il existe une nouvelle prise de conscience du pluralisme religieux. Les religions ne se contentent pas tout simplement d’exister ou même de survivre. En certains cas, elles manifestent un réel renouveau. Elles continuent à inspirer et à influencer la vie de millions de leurs membres. Dans le contexte actuel du pluralisme religieux, on ne peut donc pas oublier le rôle important que jouent les traditions religieuses. » (\textit{Dialogue et annonce} n.4) 
 \end{quote}
 
 Dans la déclaration d'Abu dhabi, une étape nouvelle est franchie par le Pape François, le 4 février 2019\sn{\href{https://www.vatican.va/content/francesco/fr/travels/2019/outside/documents/papa-francesco_20190204_documento-fratellanza-umana.html}{Document d'Abu Dhabi sur la fraternité humaine pour la paix mondiale et la coexistence commune }} : 
 
 \begin{quote}
     La liberté est un droit de toute personne : chacune jouit de la liberté de croyance, de pensée, d’expression et d’action. Le pluralisme et les diversités de religion, de couleur, de sexe, de race et de langue sont une sage volonté divine, par laquelle Dieu a créé les êtres humains. Cette Sagesse divine est l’origine dont découle le droit à la liberté de croyance et à la liberté d’être différents. C’est pourquoi on condamne le fait de contraindre les gens à adhérer à une certaine religion ou à une certaine culture, comme aussi le fait d’imposer un style de civilisation que les autres n’acceptent pas.
 \end{quote}
 % ------------------------------------
 \section{L’Église s’ouvre au dialogue : Ecclesiam suam}
 
 Concomitant à une nouvelle approche dans l'Église, le dialogue.


  \subsection{Une culture autoritaire}
  
  On a été nourri par cette vision autoritaire, pas évident pour des catholiques.
  \paragraph{Mère et enseignante} Ratzinger\sn{(J. RATZINGER, \textit{Mon Concile}, p. 225-226). } analyse finement le problème de cette vision : 
  \begin{quote}
      « Pour un dialogue chrétien vers l’extérieur, c’est-à-dire qui sort de la sphère de la foi pour aller vers le monde profane, on manquait, sur le plan du magistère, d’expérience et de modèles. Le type même du langage magistériel est, depuis le début, d’un côté le symbole, c’est-à-dire la \textsc{confession de foi} reçue par tradition et qui lie celui qui l’accepte, et de l’autre côté l’\textsc{anathème} qui exclut. Dans les deux cas, il s’agit d’une forme d’expression qui n’a de sens qu’à l’intérieur de la sphère de la foi, parce qu’elle repose sur la prise en compte de l’\textbf{autorité de la foi}. » 
  \end{quote}
  
  On en sort difficilement, cf la difficulté de la démarche synodale.
  \paragraph{Une exception} La démarche universitaire et la \textit{disputatio}
  
  \subsection{Des figures prophétiques }
  
  \paragraph{Saint François d'Assise} en 1219, au moment de la croisade. Il rencontre le Sultan qui le reçoit très bien. Cela a changé la vision de Saint François sur la mission et les autres. \textit{une expérience}. 
  
  \paragraph{Nicolas de Cue}
  
  \paragraph{Charles de Foucault} va s'inculturer. Il n'a converti aucun musulman, aucun disciple. Dictionnaire Français - Touareg. 
  
  \paragraph{Louis Massignon} \mn{p. \pageref{Theo:Massignon}} a du influencer le texte NA. 
  
  
  \paragraph{Henri le Saux} Idécouvrir l'autre par l'intérieur. Faire des ponts.
  
  
  \subsection{ Le modèle du dialogue œcuménique}
  XX.
  
  \subsection{Vatican II et Ecclesiam suam }
  
  
   \paragraph{Vatican II} 
 
   \begin{itemize}
       \item Dialogue avec le monde qui s'est émancipé de la \textit{chistianitas}
       \item avec les autres Eglises (Dei Verbum, Unitis Reintegratio)
       \item attitude qui ouvre à un dialogue avec les croyants des autres religions
    \end{itemize}
    
     50 occurrences du moit dialogue.  avec différents lieu, 45\% dialogue avec le monde. Surtout dans Gaudium et Spes (12 fois). 
       
    \subparagraph{Gaudium et Spes 44}
\begin{quote}
  1. De même qu’il importe au monde de reconnaître l’Église comme une réalité sociale de l’histoire et comme son ferment, de même l’Église n’ignore pas tout ce qu’elle a reçu de l’histoire et de l’évolution du genre humain.  
\end{quote}


Un dialogue avec une certaine réciprocité : L’Eglise a quelque chose a donné au monde mais inversement, elle a à recevoir du monde.
   
   \begin{quote}
       
Dei Verbum  2 Par cette révélation, le Dieu invisible (cf. Col 1, 15 ; 1 Tm 1, 17) s’adresse aux hommes en son surabondant amour comme à des amis (cf. Ex 33, 11 ; Jn 15, 14-15), il s’entretient avec eux (cf. Ba 3, 28) pour les inviter et les admettre à partager sa propre vie. 


   \end{quote}
    S’entretenir avec Dieu  / Dialoguer 

\paragraph{ Ecclesiam suam (1964)} 
   Encyclique de Paul VI donne une impulsion forte au dialogue. Alors que les documents du concile ne parle pas du Dialogue, Ecclesiam Suam impulse la volonté de Dialogue.
Rencontre de l’Eglise avec la société humaine.  Trois parties dans ce document : 
\begin{itemize}
    \item l’Eglise doit mieux connaître son mystère (Recollection)
\item par cette prise de conscience, elle doit se renouveler
\item par ce renouvellement, elle puisse aller entrer en dialogue
\end{itemize}

\begin{quote}
    
« Cette distinction d'avec le monde n'est pas séparation. Bien plus, elle n'est pas indifférence, ni crainte, ni mépris. Quand l'Eglise se distingue de l'humanité, elle ne s'oppose pas à elle ; au contraire elle s'y unit. Il en est de l'Eglise comme d'un médecin : connaissant les pièges d'une maladie contagieuse, le médecin cherche à se garder lui-même et les autres de l'infection ; mais en même temps il s'emploie à guérir ceux qui en sont atteints ; de même l'Eglise ne se réserve pas comme un privilège exclusif la miséricorde à elle concédée par la bonté divine ; elle ne tire pas de son propre bonheur une raison de se désintéresser de qui ne l'a pas atteint, mais elle trouve dans son propre salut un motif d'intérêt et d'amour envers tous ceux qui lui sont proches et pour tous ceux que, dans son effort de communion universelle, il lui est possible d'approcher. » (Ecclesiam Suam § 65). 


\end{quote}
pas d'opposition entre le monde et l'Eglise. Dialogue avec le monde. L'Eglise se fait \textit{parole}, \textit{conversation}.


\begin{quote}
    « La Révélation qui est la relation surnaturelle que Dieu lui-même a pris l’initiative d’instaurer avec l’humanité, peut être représentée comme un dialogue dans lequel le Verbe de Dieu s’exprime par l’Incarnation, et ensuite par l’Evangile. (…) L’histoire du salut raconte précisément ce dialogue long et divers qui part de Dieu et noue avec l’homme une conversation variée et étonnante. C’est dans cette conversation du Christ avec les hommes (cf. Ba 3,38) que Dieu laisse comprendre quelque chose de lui-même, le mystère de sa vie (…) » (Ecclesiam Suam § 72). 
\end{quote}
 Il va se rattacher à la révélation et va traduire l'incarnation surnaturelle du logos par le mot \textit{dialogue}. Figure dans le prologue de Saint Jean. 
 
 
 Pour bien comprendre Paul 6, il faut voir qu'il s'oppose à une vision défensive du monde avec des anathèmes ou au contraire de théocratie, mais le dialogue. 
 Ne vise pas à convertir directement l'autre, mais à avancer vers une communion : 
 
 \begin{quote}
     « Cette forme de rapport indique une volonté de courtoisie, d'estime, de sympathie, de bonté de la part de celui qui l'entreprend ; elle exclut la condamnation a priori, la polémique offensante et tournée en habitude, l'inutilité de vaines conversations. Si elle ne vise pas à obtenir immédiatement la conversion de l'interlocuteur parce qu'elle respecte sa dignité et sa liberté, elle vise cependant à procurer son avantage et voudrait le disposer à une communion plus pleine de sentiments et de convictions. » (Ecclesiam Suam § 81). 
 \end{quote}
   
   
   
   \begin{quote}
       « Puis, autour de nous nous voyons se dessiner un autre cercle immense, lui aussi, mais moins éloigné de nous : c'est avant tout celui des hommes qui adorent le Dieu unique et souverain, celui que nous adorons nous aussi ; Nous faisons allusion aux fils, dignes de Notre affectueux respect, du peuple hébreu, fidèles à la religion que Nous nommons de l'Ancien Testament ; puis aux adorateurs de Dieu selon la conception de la religion monothéiste - musulmane en particulier - qui méritent admiration pour ce qu'il y a de vrai et de bon dans leur culte de Dieu ; et puis encore aux fidèles des grandes religions afro-asiatiques. Nous ne pouvons évidemment partager ces différentes expressions religieuses, ni ne pouvons demeurer indifférent, comme si elles s'équivalaient toutes, chacune à sa manière, et comme si elles dispensaient leurs fidèles de chercher si Dieu lui-même n'a pas révélé la forme exempte d'erreur, parfaite et définitive, sous laquelle il veut être connu, aimé et servi ; au contraire, par devoir de loyauté, nous devons manifester notre conviction que la vraie religion est unique et que c'est la religion chrétienne, et nourrir l'espoir de la voir reconnue comme telle par tous ceux qui cherchent et adorent Dieu. » (Ecclesiam Suam § 111). 
   \end{quote}
   
   Encore une vision : il parle de \textit{vraie religion} : Saint Augustin, Lactance. Mais cela ne veut pas mépriser les autres religions. 
   
   \begin{quote}
       « Nous ne voulons pas refuser de reconnaître avec respect les valeurs spirituelles et morales des différentes confessions religieuses non chrétiennes ; nous voulons avec elles promouvoir et défendre les idéaux que nous pouvons avoir en commun dans le domaine de la liberté religieuse, de la fraternité humaine, de la sainte culture, de la bienfaisance sociale et de l'ordre civil. Au sujet de ces idéaux communs, un dialogue de notre part est possible et nous ne manquerons pas de l'offrir là où, dans un respect réciproque et loyal, il sera accepté avec bienveillance. » (Ecclesiam Suam § 112). 
   \end{quote}

\begin{Synthesis}
   Il y a quelque chose de nouveau dans NA, d'un regard positif des religions et une approche dialogale des différents textes de VII, avec surtout l'encyclique \textit{Ecclesiam Suam} dans son troisième chapitre.
\end{Synthesis}   
   
\paragraph{ Réception critique d’Ecclesiam suam }
   
   \subparagraph{Une certaine asymétrie} Image du médecin au malade comme analogie choisie. On n'a pas vraiment dialogue mais une \textsc{invitation au dialogue}. 
   
   \subparagraph{comment l'encyclique parle du dialogue} introduction du dialogue pour que l'Eglise n'impose pas son message et même par prosélytisme. 
   La personne qui accueille la parole doit pouvoir le faire \textbf{librement}. D'où l'importance de la parole par le Pape.
   \textbf{Une réciprocité souhaitée}
   
   \subparagraph{Le dialogue mis dans le but de la mission} Dans \textit{Redemptoris Missio} 55\sn{55. Le dialogue inter-religieux fait partie de la mission évangélisatrice de l'Eglise. Entendu comme méthode et comme moyen en vue d'une connaissance et d'un enrichissement réciproques, il ne s'oppose pas à la mission ad gentes, au contraire il lui est spécialement lié et il en est une expression. Car cette mission a pour destinataires les hommes qui ne connaissent pas le Christ ni son Evangile et qui, en grande majorité, appartiennent à d'autres religions. Dieu appelle à lui toutes les nations dans le Christ, il veut leur communiquer la plénitude de sa révélation et de son amour, il ne manque pas non plus de manifester sa présence de beaucoup de manières, non seulement aux individus mais encore aux peuples, par leurs richesses spirituelles dont les religions sont une expression principale et essentielle, bien qu'elles comportent «des lacunes, des insuffisances et des erreurs»98. Le Concile et les enseignements ultérieurs du magistère ont amplement souligné tout cela, maintenant toujours avec fermeté que le salut vient du Christ et que le dialogue ne dispense pas de l'évangélisation.}, l'encyclique de la mission du Pape Jean-Paul II. Attention au dialogue hameçon, pour capturer le destinataire.  G. Comeau :
   
   On prend un risque dans un dialogue, on ne sait pas ce qui va en sortir. 
   
 % ------------------------------------
   \section{Réflexions sur la dimension théologique du dialogue}
   
   \subsection{Pourquoi le dialogue ?}
   
   \paragraph{Afin de partager des valeurs communes} Dans un dialogue amical, on cherche des valeurs communes : on cherche à créer quelque chose en commun, briser les préjugés. 
   
   \paragraph{pour faire la vérité} Jean-Claude Basset, qui a beaucoup travaillé sur le dialogue : 
   \begin{quote}
     dialogue \ldots  rien de moins que la vérité.  
   \end{quote}
   Ici, c'est une vérité qui fait vivre, existentielle. 
   \begin{quote}
          Si des croyants parlent de leur foi, \ldots vérité qui les fait vivre
   \end{quote}
    Un dialogue qui nous permette d'être meilleur chrétien et que l'autre devienne aussi meilleur dans sa foi.     
   \begin{quote}
       Par le dialogue, les chrétiens et les autres sont invités à approfondir les dimensions religieuses de leur engagement et à répondre, avec une sincérité croissante, à l’appel personnel de Dieu et au don gratuit qu’il fait de lui-même, don qui passe toujours, comme notre foi nous le dit, par la médiation de Jésus Christ et l’œuvre de son Esprit. » (Dialogue et annonce n. 40). « Etant donné cet objectif, à savoir une conversion plus profonde de tous à Dieu, le dialogue interreligieux possède sa propre valeur. » (Dialogue et annonce, n. 41). 
   \end{quote}
   
   
 \paragraph{Dialogue et mission} un texte paru en 1991. Le dialogue nous permet de nous purifier, en sortant de ce qui nous sommes. 
   
   \subsection{Le dialogue n'est pas une posture, il est fondé théologiquement}
   
   \begin{Ex}
   Vatican II serait pastoral et non doctrinal. Erreur, car une théologie qui ne serait pas pastorale et qui ne rentrerait pas dans la vie concrète n'a pas d'intérêt.
   \end{Ex}
   
   \paragraph{le dialogue dans l'économie du Salut} Le dialogue se trouve dans l'intention de Dieu :
   \begin{quote}
       « La Révélation qui est la relation surnaturelle que Dieu lui-même a pris l’initiative d’instaurer avec l’humanité, \textsc{peut être représentée comme un dialogue dans lequel le Verbe de Dieu s’exprime par l’Incarnation, et ensuite par l’Evangile.} (…) L’histoire du salut raconte précisément ce dialogue long et divers qui part de Dieu et noue avec l’homme une conversation variée et étonnante. C’est dans cette conversation du Christ avec les hommes (cf. Ba 3,38) que Dieu laisse comprendre quelque chose de lui-même, le mystère de sa vie (…) » (Ecclesiam Suam § 72). 
   \end{quote}
   
  \paragraph{Dieu entre en dialogue avec l'humanité} Emmanuel : la façon dont Dieu fait alliance, s'approche de l'homme, fait \textit{dialogue.} 
  \emph{Logos} : St Jean. \textit{Tout fut par \emph{dia} lui \emph{logos}}. Paul VI le traduit par le mot dialogue, intuition de Paul VI. 
   \begin{quote}
       Le colloque paternel et saint, interrompu entre Dieu et l'homme à cause du péché originel, est merveilleusement repris dans le cours de l'histoire. ES 72.
   \end{quote}
   
   \paragraph{Dans la parole, on s'implique} Un véritable dialogue, on s'engage soi-même. 
   \begin{quote}
       le dialogue est arrivé à son terme (Ratzinger). \ldots le but de ce dialogue, [pas un échange de vérité], l'amour, \ldots le tu et le tu se touchent. il signifie communion. Dans le Christ,... ce dialogue atteint sa plénitude, vrai homme et vrai Dieu. 
   \end{quote}
   
   \paragraph{De l'économie à la théologie} de la façon dont Dieu se manifeste à nous (\textit{économie)}, on peut dire des affirmations sur Dieu lui-même.
   Il se révèle dans le Dialogue, qui n'est pas étranger à lui-même. Dans une théologie trinitaire, image de la relation (une personne = relation). L'avantage de \textit{dialogue} par rapport à \textit{relation}, c'est qu'on ne peut avoir un dialogue qu'avec une seule personne, alors qu'on peut avoir plusieurs relations.
   
   \paragraph{le courant personaliste}
   Martin Buber : je / tu  et Mühlen, à partir de la théologie trinitaire.
   
   \paragraph{un dernier enracinement théologique} un modèle pneumatique et ascendant : partir du fait que l'ES suscite la diversité tout en suscitant l'unité. \emph{Babel}, pour apprendre à dialoguer peu à peu entre nous. 
   
   Ch. Salenson : 
   \begin{quote}
       Le Père envoie son fils et son Esprit (Gal), L'esprit nous parle dans notre propre langue. Mais il crée aussi les différences. ... sage volonté divine de créer différentes cultures. \sn{On peut travailler ces questions en validation}
       \begin{quote}
           Dialogue : forme même de la mission. 
       \end{quote}
   \end{quote}
   
   \section{La remise en cause de l’Occident et de la modernité }
  Déjà abordé auparavant.
  
   