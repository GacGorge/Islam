%{\Large\textbf{Quel l’enjeu du thème du Royaume de Dieu dans la théologie des religions ? }}


\chapter{Quel l’enjeu du thème du Royaume de Dieu dans la théologie des religions ? }

Guillaume Gorge - Théologie chrétienne des religions - 16 mai 2022

\section{Introduction}

\paragraph{La Théologie des Religions}
Dans son livre \textit{pour une théologie des Religions}, Robert Schlette présente en 1963 le programme que se donne cette nouvelle discipline : 
\begin{quote}
   [La théologie, "ainsi libérée"
pourra] "{ hardiment réfléchir à ce problème, maintenant plus important,
du sens à donner aux religions en tant que phénomènes objectifs, historiques
et sociaux de l'humanité en tant qu'elle est reliée au mystère transcendant}" ,
et,  "{à partir de là, en tant que moyens de salut" (p. 71). } .\sn{souligné par H. Clavier. {Revue des livres dans Revue d'histoire et de Philosophie religieuses}}
\end{quote}

Il y a d'abord un constat objectif : dans un monde mondialisé, d'autres religions continuent à exister et à se développer, apportant sens, chemin vers la transcendance et lien communautaire à une grande partie de l'humanité. Si elles durent, c'est qu'elles portent quelque chose de solide. 
Du point de vue de la théologie chrétienne des religions, cet apport des religions s'enracine dans la signification de l'action salvifique de Jésus-Christ et il s'agit donc de rendre compte de ce lien.


\paragraph{La dimension sociale du Salut : \textit{le Règne de Dieu}} Or, l'une des dimensions essentielles du salut annoncé par le Christ est l'avènement du \textit{Règne de Dieu}. Cette dimension, redécouverte en théologie par les théologiens Wolfhart Pannenberg et Jürgen Moltmann, questionne aussi la théologie des religions, en ce sens que le salut n'est pas seulement pensé de façon individuelle mais dans sa dimension collective. Suivant la pensée de Pannenberg (\cite{Pannenberg:RoyaumeDieu}), 

\begin{quote}
    Toute Église chrétienne qui veut rester fidèle au message de Jésus doit se comprendre comme une communauté en relation avec le royaume de Dieu annoncé par Jésus. Qui dit royaume de Dieu dit futur du monde et de l’humanité tout entière.  \cite[pp. 74]{Pannenberg:RoyaumeDieu}. 
\end{quote}



Comme toute religion a une dimension sociale - on peut mettre les définir à partir de cette dimension 
 \sn{cf Schleiermacher : la Religion comme \textit{passage à l'expression collective d'une expérience individuelle} ou Durkheim : \textit{ Une religion est un système solidaire de croyances et de pratiques relatives à des choses sacrées, c’est-à-dire séparées, interdites, croyances et pratiques qui unissent en une même communauté
morale, appelée Église, tous ceux qui y adhèrent} } - , le Règne de Dieu questionne particulièrement la théologie des religions. Nous proposons de présenter l’enjeu du thème du Royaume de Dieu dans la théologie des religions, en partant de la réflexion de Pannenberg : 

\begin{itemize}
    \item Prendre au sérieux l'annonce du le Règne de Dieu. Après avoir étudié les conséquences pour la Christologie et l'Ecclésiologie selon la pensée de Pannenberg, nous esquisserons les enjeux que nous pouvons en déduire pour la théologie des Religions.
    \item Quelle actualisation de ces enjeux dans le contexte actuel de la post-modernité et du développement de la théologie depuis 50 ans
\end{itemize}



% ------------------------------------------------------------------------------------------------------------------------------------------------------------------------------------
\section{Prendre au sérieux l'annonce du le Règne de Dieu}
% ------------------------------------------------------------------------------------------------------------------------------------------------------------------------------------

Pannenberg souhaite dire en quel sens Jésus-Christ est interprété comme le fondement de l’Église et de son unité.(\cite[p. 82]{Pannenberg:RoyaumeDieu}). Or la mission de Jésus est déterminée par l'annonce du Règne de Dieu (on connaît l'aphorisme de Loisy : \textit{Jésus annonçait le Royaume, et c'est l'Église qui est venue.}). Le titre même de \textit{Christ}, fondement du kerygme chrétien et référence au Roi d'Israël, renvoie au règne de Dieu. 
Il y a donc un triangle  aux liens étroits entre Jésus-Christ, l'Eglise et son unité, et le Règne de Dieu, chacune se nourrissant de notre compréhension plus fine de l'autre.
 
% - ----------------------------------------
\subsection{Lien et conséquence pour l'Ecclésiologie}
\paragraph{le Règne qui vient : penser l'Eglise théologiquement à partir du Royaume}
Pannenberg commence par les conséquences pour l'Ecclésiologie, se comprenant à l'époque comme \textit{Peuple de Dieu} : 
\begin{quote}
   Si l’Église s’est comprise dès le christianisme primitif comme le nouvel Israël, comme le nouveau peuple de Dieu du temps final, cette compréhension de soi ne concerne pas seulement le lien conscient avec le peuple de Dieu de l’ancienne alliance : elle implique aussi l’idée de la royauté de Dieu, en lien avec le règne définitif attendu de Dieu sur le monde. Comme le règne de Dieu est l’avenir auquel fait face l’humanité tout entière, \textit{l’autocompréhension de l’Église comme peuple de Dieu est seulement à justifier par sa relation totale au monde et à l’humanité. }\cite[pp. 75]{Pannenberg:RoyaumeDieu}. C'est nous qui soulignons). 
\end{quote}
 Pannenberg ne s'oppose pas à l'expression d'\textit{Eglise comme Peuple de Dieu} mais ne doit pas être pensé comme le \textit{petit reste d'Israël}, ou tout du moins dans une articulation avec le monde.

 L'Eglise doit se nourrir de l’idée du règne de Dieu qui vient \cite[pp. 76]{Pannenberg:RoyaumeDieu} à travers un double mouvement lié à l'attente du futur propre au Royaume de Dieu. 
 
 \paragraph{Penser l'Eglise dans le monde où elle est}
 Dans un premier mouvement, toute interprétation de l'Eglise doit prendre en compte sa relation au contexte de vie englobant du monde social. En reprenant les terminologies de Lindbeck, nous pourrions dire que la \textit{langue} dans laquelle s'exprime l'Eglise reprend la langue du monde dans laquelle elle vit.
    
\paragraph{L'Eglise comme \textit{Anticipation de la nouvelle humanité}}
Dans un second mouvement, penser le Royaume de Dieu, c'est  penser son \textit{entremêlement au monde comme un moment essentiel de l'Eglise}\cite[pp. 76]{Pannenberg:RoyaumeDieu}). L'Eglise ne doit pas se retirer du monde mais l'aider à se conformer toujours plus au dessein de Dieu et se penser comme anticipation [Vorwegnahme] de la nouvelle humanité, une humanité régie par Dieu et son Esprit \cite[pp. 76-77]{Pannenberg:RoyaumeDieu}.  


% - ----------------------------------------
\subsection{le Règne de Dieu nourri par la christologie}

Quels critères pour juger de la fidélité de l'Eglise au message du Règne de Dieu ? Comment aussi discerner dans tout corps social, et donc dans d'autres religions le Règne qui vient ? 

\begin{quote}
    Attribuer à Jésus le titre et la dignité de Christ, de Messie, c’est exprimer la manière dont le royaume de Dieu qui vient s’est montré comme une force déterminant le présent, à travers la vie et la mort de Jésus pour toute l’humanité, et comme la transformation du présent par l’amour créateur. Toutes les immenses paroles et formules de la christologie disent vrai, dans la mesure où elles expriment la manière dont le futur du règne de Dieu est la force transformatrice de la vie de Jésus, dans le présent, et, à travers lui, de l’histoire de l’humanité. Dans l’engagement de Jésus pour le futur du règne de Dieu, ce dernier devint présent et, par lui, présent à tous les êtres humains. (\cite[pp. 94-95]{Pannenberg:RoyaumeDieu}). 
\end{quote}
Pour saisir le Règne de Dieu, il nous faut donc revenir à Jésus mais dans une approche renouvelée, transformée par l'Esprit Saint. 

\paragraph{Adéquation au Règne de Dieu}
Une réalité sociale ne préfigure le Règne de Dieu que dans la mesure où elle correspond à l'annonce de Jésus. Pratiquement, Pannenberg propose un certain nombre de pistes permettant de mesurer cette adéquation: 
\begin{itemize}
    \item Tout d'abord, un style, celui d'être sensible à l'approche dynamique de l'Esprit, à la mort et la Resurrection. Cela doit se traduire par l'acceptation du dynamisme des organisations, d'une attention aux signes des temps. 
    \item Une aspiration à la plus grande unité de l'humanité, à l'image de l'unité de Dieu, mais qui ne peut se faire par la violence,
    \item Préférer la justice et amour au droit, dans une conception qui ne soit pas individualiste.
    \item L'importance de la guérison \cite[pp. 87]{Pannenberg:RoyaumeDieu}, comme action de l'Esprit vivifiant.
\end{itemize}

\paragraph{Une vocation spécifique de l'Eglise ?} Pannenberg insiste sur l'aspect eschatologique du Royaume et le risque de toute société humaine d'oublier qu'elle est faillible et de devenir totalisante. 
 L'Eglise, parce qu'elle vit déjà maintenant de la certitude du futur de Dieu sans se laisser tromper par l'expérience du provisoire, est \textit{nécessaire} en tant que signe que nous ne sommes pas près du Royaume. Cet \textit{écart} avec l'idéal, c'est aussi laisser l'espace au travail de l'Esprit Saint dans notre monde.

\paragraph{Enjeu pour la théologie des religions}
Il s'agit donc de faire place des religions en respectant le lien étroit entre le Christ, le Royaume et l'Eglise (CDF, Dominus Iesus, § 18). In fine, du fait de ce lien étroit, on ne pourra pas articuler notre rapport aux autres religions uniquement vis à vis du Christ, ou de l'Eglise mais bien dans les trois dimensions de façon simultanée.

% ------------------------------------------------------------------------------------------------------------------------------------------------------------------------------------
\subsection{quels enjeux du thème du Règne de Dieu pour la théologie des Religions}  

L'objet de Pannenberg est d'honorer la place du \textit{Règne de Dieu} dans la mission du Christ et ses conséquences tant en Christologie qu'en Ecclésiologie. Mais par ce travail, il ouvre un certain nombre de questions qui peuvent nourrir une théologie Chrétienne des Religions.

Pannenberg insiste sur la dimension du Règne de Dieu dépassant l'Eglise mais surtout qui oblige l'Eglise à prendre en compte la dimension universelle du salut : 
\begin{quote}
L’opposition au monde ne peut pas être le motif premier et fondamental quand il est question d’exprimer le rapport de l’Église à la société et à l’humanité. Cela n’est pas seulement dû à une contradiction formelle, érigeant l’opposition, comme élément de la relation, en élément constitutif de l’Église. Il est plus important encore de souligner que la simple opposition de l’Église au monde néglige la tendance universaliste inhérente à l’idée du règne de Dieu. Le règne de Dieu est plus grand que l’Église, et celle-ci ne trouve sa fonction et sa signification spécifiques que dans son orientation vers le Règne de Dieu.(\cite[pp. 75-76]{Pannenberg:RoyaumeDieu} )
 \end{quote}

 Cela \textit{oblige} à prendre en compte ce que les autres religions ont d'irréductibles, car cette irréductibilité fait partie de l'humanité.  
\paragraph{Les religions comme \textit{institutions honnêtes}}
Les religions, et au premier titre, l'Église, sont des institutions et à ce titre se doivent être des institutions honnêtes, que Pannenberg définit comme
\begin{quote}
      une institution qui démasque les limites des formes présentes de la vie sociale et politique et renvoie les êtres humains à la réalité ultime englobant leur destination dernière. La relation de l’être humain à sa destination ultime ne peut devenir consciente, dans sa concrétude mondaine, qu’à la condition de prendre honnêtement en compte la vie présente dans ses limites et dans ses échecs, et qu’on n’essaie pas d’échapper à ses aspects négatifs.\cite[pp. 98]{Pannenberg:RoyaumeDieu}
\end{quote}
Parce que les religions proposent un \textit{ sens véritable de l'existence} et \textit{une pratique liturgique, rituelle et une éthique de vie qui permettent d'y accéder} \sn{DiNoia cité par \cite{Cheno:DieuPluriel}}, elles sont des institutions honnêtes particulièrement structurantes pour l'humanité, mais uniquement dans la mesure où elles démasquent les limites des formes présentes de la vie sociale.

\paragraph{Participation au Règne de Dieu }
Le règne du Christ est actif, partout où les êtres humains prennent conscience de la venue du Règne de Dieu.\cite[p. 85]{Pannenberg:RoyaumeDieu} : faire advenir la justice, l'amour, s'éloigner d'un juridisme froid, lire les signes de l'Esprit, guérir, viser à l'unité de l'humanité dans la paix, ce sont des signes que les religions non chrétiennes participent à la venue du Règne de Dieu.



% ------------------------------------------------------------------------------------------------------------------------------------------------------------------------------------    
\section{enjeux contemporains pour la théologie des religions}    
% ------------------------------------------------------------------------------------------------------------------------------------------------------------------------------------

Le livre de Pannenberg a été écrit à partir de conférences données dans les années 1960. Or, la vision du monde a beaucoup changé, avec la vision marxiste remplacée par celle  \textit{post-moderne} . De plus, la compréhension du Christ, de l'Eglise et du Règne de Dieu ont aussi évolué, et nous souhaiterions aborder comment l’enjeu du thème du Royaume de Dieu peut être pensé dans la théologie des religions aujourd'hui.

% ------------------------------------------------------------------------------------------------------------------------------------------------------------------------------------    
\subsection{Dans une société post-moderne}

Avec 9 références au marxisme dans le chapitre 2, la grille de lecture marxiste de Pannenberg est explicite. Or, le marxisme, s'inscrit dans le courant moderne dans le sens où il ne renonce pas à une clé de lecture globalisante. Cette prétention à la totalité explique d'ailleurs la critique que lui fait Pannenberg de penser que le Règne de Dieu puisse se réaliser ici bas. Cette clé de lecture est moins pertinente aujourd'hui, dans une société \textit{post-moderne} qui a renoncé à toute vérité autre que partielle.


\paragraph{ Quel corps social pour un monde post-moderne ?} Dans un monde post-moderne, le risque est en effet moins le discours totalisant qu'une dilution de tout corps social parallèlement à la recherche d'un salut individualiste. Face à cette évolution, l'annonce du Règne de Dieu est probablement moins audible mais d'autant plus pertinente : il est important de rappeler que le salut Chrétien ne peut se penser de façon individuelle, mais doit intégrer la dimension sociale et universelle du \textit{Règne de Dieu}.


\paragraph{Penser le rôle des autres religions dans ce contexte}
Lindbeck propose de regarder la prétention d'une religion à être raisonnable et universelle à sa capacité à fournir dans ses propres termes une interprétation intelligible des diverses situations et réalités que rencontrent ses adhérents (Lindbeck, 175). Dans un monde marqué par le post-modernisme et l'hyper-individualisation, une attention renouvelée sera portée à la capacité des religion à "relier", à créer un corps social et à faire l'unité de l'humanité sans violence, éléments soulignés par Pannenberg comme des critères d'authenticité de l'annonce du \textit{Règne de Dieu}. 
Dans un contexte post-moderne, certains théologiens \sn{DiNoia, cité par \cite[p.133]{Cheno:DieuPluriel}} proposent d'abandonner le principe sotériologique car c'est appliquer à d'autres religions un cadre de pensée pensé à partir du Christianisme : plutot que le \textit{salut}, chrétien, penser une réalité plus large de \textit{but ultime de la vie}.  Sans entrer dans la discussion des implications d'un tel abandon, il nous semble \textit{a minima} important de rappeler l'importance que ce but ultime de la vie intègre une dimension sociale explicite. 



% ------------------------------------------------------------------------------------------------------------------------------------------------------------------------------------    
\paragraph{Le Règne de Dieu d'après Pagola}
La théologie a bien sûr beaucoup évolué en 50 ans. Ainsi, le terme de \textit{Peuple de Dieu} pour penser l'Eglise, très présent chez Pannenberg, est moins central aujourd'hui même s'il n'a pas perdu sa pertinence pour expliciter la dimension collective du Salut. Il n'est pas question de développer ici toutes les pistes explorées par la théologie  mais de remarquer qu'en insistant aujourd'hui sur certaines dimensions du \textit{règne de Dieu}, le théologiens éclairera la dimension sociale du salut de nouveaux éléments.
Ainsi Pagola propose une vision du Royaume de Dieu éclairée par le jugement et la mort du Christ : 
\begin{quote}
    Le Royaume de Dieu défendu par Jésus remet en question à la fois l'édifice romain et le système du Temple. Les autorités juives, fidèles au Dieu du Temple, se sentent tenues de réagir: Jésus est un gêneur. Il invoque Dieu pour défendre la vie des exclus. Caïphe et les siens l'invoquent pour défendre les intérêts du Temple. ils condamnent Jésus au nom de Dieu mais, ce faisant, ils condamnent le Dieu du Royaume, le seul Dieu vivant en qui Jésus croie. Il en va de même avec L'Empire. Jésus ne voit pas dans le système défendu par Pilate un monde organisé selon le coeur de Dieu. Il défend, lui, les intérêts des oubliés de l'Empire. Pilate protège les intérêts de Rome. Le Dieu de Jésus pense aux plus démunis. Les dieux de l'empire protègent la \textit{pax romana}.\cite[p. 402]{Pagola:Jesus} 
\end{quote}

Cet éclairage de Pagola nous incite à chercher les traces du Règne de Dieu dans la défense des plus pauvres et des oubliés, dans le refus de mettre en priorité les intérêts religieux et de l'ordre établi. Par cet éclairage, on pourra répondre à nouveaux frais aux questions suivantes :  comment les religions oeuvrent par rapport au défi écologique face à l'ordre établi ? les migrants ? les personnes fragiles ou handicapées ? Comment les religions acceptent de \textit{gêner}, d'\textit{empêcher le monde de tourner en rond} ? Comment aussi elles \textit{acceptent la critique} de leurs propres institutions ? 


L'Église, dans son rôle d'anticipation du Règne de Dieu, veillera à s'appliquer d'abord cette grille exigeante puis pourra la proposer aux autres religions dans un dialogue sans fard mais fécond.

% ------------------------------------------------------------------------------------------------------------------------------------------------------------------------------------    
\section{Conclusion}    
% ------------------------------------------------------------------------------------------------------------------------------------------------------------------------------------

Le Règne de Dieu permet à l'Eglise de se décentrer : ce qui doit advenir, c'est le Règne de Dieu, qui a vocation à toucher la totalité de l'humanité. Notre vision de l'universel a évolué, probablement plus humble, reconnaissant une partie irréductible dans chaque religion. 
Cette dimension sociale des religions est particulièrement importante à souligner dans un monde \textit{post-moderne} où le salut peut être pensé de façon strictement individuelle. Contre la tention du repli, du confort de l'individu \cite[p.107]{Pannenberg:RoyaumeDieu}, l'Eglise a vocation a reconnaître en elle mais aussi dans tout corps social et toute religion ce qui est de l'annonce du Règne et de la venue du Règne dès ce monde, à partir du message et de la vie de Jésus, toujours relu et actualisé. 


% ------------------------------------------------------------------------------------------------------------------------------------------------------------------------------------
% ------------------------------------------------------------------------------------------------------------------------------------------------------------------------------------
