\chapter{La philosophie en islam}


\section{Introduction
}
\begin{quote}
Un jour, le calife al-Ma'mūn fit un rêve~: «~J'ai vu en rêve un homme
assis dans la posture des sages, je lui ai demandé «~qui es-tu~?~», il
m'a répondu «~Aristote le Sage~». Alors je lui posai la question
«~Dis-moi qu'est-ce que le bien? ~», Aristote de dire : «~ce qui est
conforme à la raison~». Al-Ma'mūn : «~Mais encore~?~», Aristote : «ce
qui est bien selon la révélation~~». Je lui dis : et après ? ~Il me dit
: «~Ce qui est bien aux yeux de tous~». Je lui dis : `et après'. Il me
dit : `il n'y a pas d'après'~Je lui dis : `dis-moi encore autre chose'.
Il me dit : `celui qui te conseille bien au sujet de l'or, qu'il soit
pour toi comme de l'or ; et il te faut reconnaître un Dieu unique'~».
\end{quote}

De ce rêve, supposé reconstruit et visant à justifier le recours à la
philosophie en islam, naquit le projet de créer la Maison de la Sagesse
en 832 qui devint un lieu où de nombreux savants s'attelèrent à traduire
les œuvres de la philosophie grecque\textbf{,} et également \textbf{des}
traités de mathématiques et de médecine. Dimitri Gutas a montré que
cette entreprise de traduction doit être comprise au sein d'un
environnement politique, social et idéologique propre à l'empire
abbasside en quête de légitimité\sn{Dimitri \textsc{Gutas},
  \emph{Pensée grecque, culture arabe : Le mouvement de traduction
  gréco-arabe à Bagdad et la société abbasside primitive
  (IIe-IVe/VIIIe-Xe siècles)}, traduit de l'anglais par Abdessalam
  Cheddadi, Paris, Aubier, 2005.}\textsuperscript{.} Les
descriptions\textbf{,} sous la plume d'auteurs contemporains\textbf{,}
d'une maison de rencontres où se côtoient des juifs, des chrétiens et
des musulmans ressemblent davantage à la projection d'un idéal moderne
qu'à une réalité historique\sn{~À l'exemple de Souleymane Bachir
  \textsc{Diagne}, \emph{Comment philosopher en islam~?}, Paris,
  Panorama, 2008, p. 36. Toutefois, au-delà de représentations
  idéologiques, il est certain que les premiers traducteurs étaient des
  chrétiens issus de milieux jacobites ou nestoriens qui avaient
  préservé dans les bibliothèques de leurs monastères l'essentiel de
  l'héritage aristotélicien. L'assertion d'Alain de Libera, selon
  laquelle la philosophie en terre d'islam a été «~une histoire
  musulmane, cela va sans dire, aussi une histoire chrétienne et une
  histoire juive~»~est bien fondée~: Alain \textsc{de Libera}, \emph{La
  Philosophie médiévale}, Paris, PUF, 1993, p. 54.}. Ce qui prédomine,
ce sont les interactions scientifiques entre philosophes et non des
échanges de théologie. Il reste que la philosophie grecque rencontre
l'univers de l'islam, la \emph{philosophia} devient \emph{falsafa}. Il
s'y affirme l'idée d'une sagesse éternelle (\emph{philosophia perennis})
traversant les âges et les communautés humaines, qu'elles fussent ou non
gratifiées d'une révélation. Cette sagesse {est} transmise grâce
à des hommes comme le «~divin Platon~»\sn{~Al-Ġazālī reprend
  l'expression du «~divin Platon~(Platon \emph{al-ilāhi})~» mais non
  sans ironie~: \textsc{al-Ġazālī}, \emph{Tahāfut al-falāsifa},
  p. 8. {[}Marmuna, p. 4{]}}, sagesse que le croyant ne peut méconnaître
mais qu'il doit au contraire s'approprier. Dans ce foisonnement de
traductions et de rencontres -- Platon, Aristote, Plotin sont ainsi
traduits en arabe dès la fin du IXe siècle, l'élite intellectuelle débat
de questions nouvelles. Les controverses foisonnent entre musulmans et
chrétiens ou juifs, entre sunnites et šī`ites, entre mutazilites et
aš`arites où, derrière un style non dénué d'expressions caustiques,
l'atmosphère y est celle d'une ouverture à l'autre comme en témoigne Abū
Hayyān al-Tawhīdī\sn{Marc \textsc{Berg}É, \emph{Pour un humanisme
  vécu~: Abū Hayyān al-Tawhīdī}, Institut français de Damas, 1979.}.
Munis des outils et des questions de la philosophie grecque, des
penseurs arabes en viennent à élaborer leurs propres systèmes
philosophiques. Trois auteurs occupent une place singulière dans cette
histoire~: al-Kindī (m. 870), al-Fārābī (m. 950) et Ibn Sīnā (m. 1037).
Il s'agit d'Avicenne. Avec al-Fārābī, le \emph{kalām} devient une
science spécifique et les connaissances prophétiques et philosophiques
renvoient à deux modes distincts d'appréhension d'un même objet. Sous
l'influence du néo-platonisme et du milieu šī`ite, al-Fārābī construit
une cosmogonie caractérisée par la procession des intellects au sein
d'un monisme émanatiste\textbf{.} Mais la séparation entre philosophie
et orthodoxie musulmane atteint son paroxysme avec les écrits d'Ibn
Sīnā. La création n'y est plus comprise comme un acte de volonté libre
comme l'affirme le Coran, mais elle relève d'une nécessité propre à
l'être divin. Par ailleurs, Ibn Sīnā déconstruit les dogmes musulmans
par une lecture symbolique qui privilégie le sens ésotérique
(\emph{bāṭin}) et relaie au second plan le sens exotérique
(\emph{ẓāhir}) des versets coraniques. Pour Ibn Sīnā, la résurrection
des corps est de l'ordre de la métaphore et non de la
réalité\sn{~Ibn \textsc{S}Ī\textsc{n}Ā, \emph{Al-Risāla
  al-Aḍḥawiyya}/}.

Les systèmes philosophiques développés par les philosophes arabes ne
manquent pas d'audace et d'originalité pour la pensée musulmane de
l'époque, mais peuvent-ils s'accorder avec l'islam~? Relèvent-ils de
questions périphériques à l'islam ou contredisent-ils l'enseignement de
la révélation coranique~? Relèvent-ils d'une sagesse universelle ou
d'une logique singulière qui ne saurait s'accorder avec la logique
propre à l'arabe, autrement dit, peut-on projeter sur les versets du
Coran une lecture philosophique grecque alors même qu'elle s'oppose à la
logique de la langue arabe qui n'est autre que la grammaire
arabe~?\sn{Allusion ici à la célèbre controverse entre le
  Nestorien Mattā b. Yūnus (m. 328/940) et le musulman al-Sīrāfī.
  Voir~pour l'édition~: Abdelali \textsc{Elamrani-Jamal}, \emph{Logique
  aristotélicienne et grammaire arabe}, Paris, Vrin, 1983. Voir
  aussi~David Samuel \textsc{Margoliouth}, «~The discussion between Abu
  Bishr Matta and Abu Saʿid al-Sirafī on the merits of logic and
  grammar~», dans \emph{Journal of the royal Asiatic society}, London,
  1905, p. 79-129.} Questions éminemment redoutables, questions toujours
modernes. Pour autant, initialement et contrairement à l'Occident, le
débat ne porte pas tant sur l'opposition entre la foi et la raison que
sur celle entre la religion officielle et un système philosophique.


\section{Logique grecque et logique islamique~:
débats autour de la langue
}


\subsection{1.1 le statut de la langue
arabe}

La philosophie se dit en arabe \emph{falsafa} et les philosophes se
nomment \emph{falāsifa}. Par cette désignation se trouve clairement
indiquée l'origine grecque de cette discipline. Elle se situe en dehors
des sciences traditionnelles de l'islam. Si l'interrogation est
indépendante en soi de l'appartenance religieuse, dans quelle mesure y
a-t-il eu une appropriation de la pensée grecque par l'islam~? Dans
quelle mesure y a-t-il eu une islamisation de cette pensée~? Mais aussi,
en recourant au grec, la pensée musulmane ne s'est-elle pas hellénisée~?
Dans ce cas, n'y a-t-il pas risque de voir trahi le message originel de
la révélation donné en arabe~?

En effet, le Coran en ses versets de la sourate 41, 2-3 fait mention
explicite~de l'arabe comme langue de la révélation coranique :
\begin{quote}
    «~C'est
une révélation descendue de la part du Tout Miséricordieux, du Très
Miséricordieux. Un Livre dont les versets sont détaillés, un Coran arabe
pour des gens qui savent~»
\end{quote}.
Cette expression a été comprise de deux
manières~:

\begin{itemize}
\item
  l'arabe est une langue sacrée, la langue choisie pour dire la parole
  de Dieu, langue de Dieu.
\item
  L'arabe est une langue humaine, elle n'est qu'humaine. Chaque messager
  a reçu un message dans la langue de son peuple~: «~Nous n'avons envoyé
  de messager qu'avec la langue de son peuple, afin de les éclairer~»
  (S. 14,4). Ainsi, s'il y a une langue du Coran, il n'y a pas pour
  autant une langue de l'islam.
\end{itemize}

Pour les partisans de la première interprétation, l'arabe comme langue
de Dieu, comme langue sacrée, ne saurait subir l'introduction d'un
vocabulaire étranger sans perdre sa pureté. Les mots étrangers en
altérant la langue altèrent l'islam même. Ici, la \emph{falsafa} est
montrée du doigt et bannie. Ce débat renvoie à une controverse fameuse
entre Mattā Ibn Yūnus et al-Sīrāfī.


\subsection{La controverse entre Mattā Ibn Yūnus et
al-Sīrāfī
}
Mattā est un chrétien nestorien et al-Sīrāfī est un grammairien
musulman. Entre eux deux s'est tenue une controverse très animée et
polémique.

Sīrāfī, en effet, dénonce les prétentions de la logique grecque à
s'immiscer dans la langue arabe. Or, elle ne saurait y trouver accueil
et hospitalité~! L'animosité est aussi confessionnelle puisque Mattā est
chrétien. Il est l'exemple même de la dérive possible due aux religions
étrangères. Sīrāfī entend donc ridiculiser son adversaire. Il lui pose
des colles comme pour montrer son incompétence en arabe et pour
l'empêcher de développer ses thèses. Pourtant, Mattā veut montrer qu'une
vérité universelle peut être traduite dans d'autres idiomes, et que
c'est cette possibilité qui définit le vrai. Les intelligibles sont les
mêmes, quels que soient les langues dans lesquelles ils se disent~: «~ne
vois-tu pas que 4+4 font huit, qu'ils font toujours 8 en grec comme en
arabe~! » lance-t-il à al-Sīrāfī. Mattā défend donc la thèse d'une
vérité universelle contre le relativisme linguistique de Sīrāfī. Pour
Sīrāfī, chaque langue est un monde à part, avec une logique propre,
spécifique. Il n'y a pas de grammaire générale mais seulement des
grammaires particulières. Il n'y a pas de grammaire philosophique, il y
a une logique grecque et il y a une logique arabe laquelle n'est autre
que la grammaire arabe. Traduire la logique grecque en arabe, c'est donc
trahir la logique arabe et donc la Révélation.

Ce débat est d'une étonnante modernité, que l'on songe à toute la
difficulté de la notion d'être qui n'existe pas dans certaines langues,
et qui est une catégorie métaphysique grecque.

Mais derrière des enjeux linguistiques et techniques se joue celui de
l'ouverture à l'autre. Al-Sīrāfī est pour une exégèse de la révélation
qui n'utilise pas les catégories dialectiques de la Grèce. Mattā est
pour le contraire. Il veut que l'arabe devienne aussi une langue de la
philosophie.


\section{La philosophie islamique
}


\subsection{Définition
}
Le titre de ce chapitre «~la philosophie en islam~» pose une certaine
difficulté En effet, ne faut-il pas parler de philosophie arabe~? Dans
ce cas, cela inclurait les chrétiens et les juifs qui s'expriment en
arabe. Faut-il plutôt parler de philosophie islamique de langue arabe~?
Mais que faire d'al-Ġazālī \label{theol:AlGazali26} qui s'est exprimé en arabe et en
persan\ldots~? Pour le professeur Rémi Brague, il n'y a pas de
philosophie islamique. Il n'y a qu'un usage de pensées philosophiques de
la part des musulmans. La philosophie a été pratiquée par des musulmans.

Le philosophe Suleyman Diagne propose cependant de définir la
philosophie islamique comme étant la relecture des récits de la
tradition islamique à la lumière de la raison.


\subsection{Illustration~: le Commentaire d'Avicenne
sur le voyage nocturne de Muḥammad
}

\subsubsection{Récit du voyage
nocturne}

Il en est question dans le Coran et dans la Sira.

Dans le Coran, on en trouve~mention aux sourates 17\textsuperscript{ème}
et 53\textsuperscript{ème}.

S. 17,1~: Gloire et Pureté à Celui qui de nuit, fit voyager Son
serviteur {[}Muhammad{]}, de la Mosquée Al-Haram à la Mosquée Al-Aqsa
dont Nous avons béni les alentours, afin de lui faire voir certaines de
Nos merveilles. C'est Lui, vraiment, qui est l'Audient, le
Clairvoyant\sn{S. 17, 1

  \TArabe{سُبْحَانَ الَّذِي أَسْرَى بِعَبْدِهِ لَيْلاً مِّنَ الْمَسْجِدِ
  الْحَرَامِ إِلَى الْمَسْجِدِ الأَقْصَى الَّذِي بَارَكْنَا حَوْلَهُ
  لِنُرِيَهُ مِنْ آيَاتِنَا إِنَّهُ هُوَ السَّمِيعُ البَصِيرُ}}.

Dans la S. 53, il est question d'une rencontre, d'une ascension depuis
Jérusalem jusqu'au Lotus de la Limite. Il a vu alors les merveilles de
son Seigneur.

Mais ce sont surtout les \emph{ḥadīṯs} qui donnent les détails les plus
nombreux de ce voyage. Ils sont compilés par al-Buḫārī et
Muslim\sn{Traduit de l'arabe du livre de Sheikh Yâsîn Rushdî,
  \emph{Fî Ri\underline{h}âb Al-Mu\underline{st}afâ} (En Compagnie de
  l'Élu), disponible sur le site
  \url{http://www.mouassa.org/Arabic/FiRehab/index.htm}.
  traduction non revue et consultable sur le site
  \url{http://www.islamophile.org/spip/Le-Voyage-Nocturne-et-l-Ascension,1368.html}
}.

\begin{quote}
Au sujet du Voyage Nocturne, le Prophète --- paix et bénédictions sur
lui --- dit~: «~Le plafond de ma maison s'ouvrit alors que j'étais à La
Mecque. Jibrîl descendit, il ouvrit ma poitrine et la lava à l'eau de
Zamzam. Puis il apporta une bassine en or, emplie de sagesse et de foi,
qu'il vida dans ma poitrine, puis il la referma. Ensuite, on m'apporta
le Burâq~; il s'agit d'une bête blanche élancée, entre l'âne et le
mulet, qui pose son sabot à perte de vue. Je l'enfourchai et arrivai à
la Mosquée de Jérusalem. Je l'attachai à l'anneau que les prophètes
emploient à cet effet, puis j'entrai dans la Mosquée.~»~
Al-\underline{H}asan --- que Dieu l'agrée --- de poursuivre~: «~Il y
trouva Abraham, Moïse et Jésus dans une assemblée de prophètes. Alors,
le Messager de Dieu fut leur imam dans une prière qu'ils accomplirent en
congrégation. Puis, on apporta deux récipients, l'un contenant du vin,
l'autre du lait.~» Il poursuivit~: «~Le Messager de Dieu prit celui qui
contenait du lait et en but. Il laissa celui qui contenait du vin.~» Il
poursuivit~: «~Alors Jibrîl lui annonça~: ``Tu as été guidé vers la
nature primordiale (\emph{fi\underline{t}rah}) et ta communauté a été
guidée, ô Mu\underline{h}ammad. Le vin vous a été défendu.'' Puis le
Messager de Dieu retourna à la Mecque. Le lendemain matin, il relata cet
événement aux gens de Quraysh.

Le Prophète dit~: «~Puis Jibrîl vint avec le \emph{Mi`râj} Puis nous
montâmes au ciel. Alors Jibrîl demanda l'accès. On lui dit~: ``Qui
es-tu~?'' Il répondit~: ``Jibrîl''. On s'enquit~: ``Qui est avec toi~?''
Il répondit~: ``Mu\underline{h}ammad''. On demanda~: ``Fut-il investi de
sa mission prophétique~?'' Il dit~: ``Oui, il en fut investi''. On nous
ouvrit et je vis alors Adam qui m'accueillit chaleureusement et fit des
invocations en ma faveur.

Puis nous montâmes au deuxième ciel et Jibrîl demanda l'ouverture du
ciel. On lui dit~: ``Qui es-tu~?'' Il répondit~: ``Jibrîl.'' On
s'enquit~: ``Qui est avec toi~?'' Il répondit~:
``Mu\underline{h}ammad.'' On demanda~: ``Fut-il investi de sa mission
prophétique~?'' Il dit~: ``Oui, il en fut investi.'' On nous ouvrit et
je vis alors les deux cousins maternels --- Jésus (`Îsâ), le fils de
Marie, et Jean-Baptiste (Ya\underline{h}yâ), le fils de Zacharie ---
paix et bénédictions sur eux ---. Ils m'accueillirent chaleureusement et
firent des invocations en ma faveur.

Puis nous montâmes au troisième ciel. Alors Jibrîl demanda l'accès. On
lui dit~: ``Qui es-tu~?'' Il répondit~: ``Jibrîl.'' On s'enquit~: ``Qui
est avec toi~?'' Il répondit~: ``Mu\underline{h}ammad.'' On demanda~:
``Fut-il investi de sa mission prophétique~?'' Il dit~: ``Oui, il en fut
investi.'' On nous ouvrit et je vis alors Joseph --- paix et
bénédictions sur lui --- qui, à lui seul, avait reçu la moitié de la
beauté. Il m'accueillit chaleureusement et fit des invocations en ma
faveur.

Puis nous montâmes au quatrième ciel. Alors Jibrîl demanda l'accès. On
lui dit~: ``Qui es-tu~?'' Il répondit~: ``Jibrîl.'' On s'enquit~: ``Qui
est avec toi~?'' Il répondit~: ``Mu\underline{h}ammad.'' On demanda~:
``Fut-il investi de sa mission prophétique~?'' Il dit~: ``Oui, il en fut
investi.'' On nous ouvrit et je vis alors Idrîs --- paix et bénédictions
sur lui --- qui m'accueillit chaleureusement et fit des invocations en
ma faveur. Dieu dit à son sujet~: ``Nous l'élevâmes à un haut rang.''

Puis nous montâmes au cinquième ciel. Alors Jibrîl demanda l'accès. On
lui dit~: ``Qui es-tu~?'' Il répondit~: ``Jibrîl.'' On s'enquit~: ``Qui
est avec toi~?'' Il répondit~: ``Mu\underline{h}ammad.'' On demanda~:
``Fut-il investi de sa mission prophétique~?'' Il dit~: ``Oui, il en fut
investi.'' On nous ouvrit et je vis alors Aaron --- paix et bénédictions
sur lui --- qui m'accueillit chaleureusement et fit des invocations en
ma faveur.

Puis nous montâmes au sixième ciel. Alors Jibrîl demanda l'accès. On lui
dit~: ``Qui es-tu~?'' Il répondit~: ``Jibrîl.'' On s'enquit~: ``Qui est
avec toi~?'' Il répondit~: ``Mu\underline{h}ammad.'' On demanda~:
``Fut-il investi de sa mission prophétique~?'' Il dit~: ``Oui, il en fut
investi.'' On nous ouvrit et je vis alors Moïse --- paix et bénédictions
sur lui --- qui m'accueillit chaleureusement et fit des invocations en
ma faveur.

Puis nous montâmes au septième ciel. Alors Jibrîl demanda l'accès. On
lui dit~: ``Qui es-tu~?'' Il répondit~: ``Jibrîl.'' On s'enquit~: ``Qui
est avec toi~?'' Il répondit~: ``Mu\underline{h}ammad.'' On demanda~:
``Fut-il investi de sa mission prophétique~?'' Il dit~: ``Oui, il en fut
investi.'' On nous ouvrit et je vis alors Abraham qui était adossé au
Temple Peuplé (\emph{Al-Bayt Al-Ma`mûr}). Chaque jour soixante-dix
milles anges y pénètrent et n'y retournent guère. Puis il m'emmena à
\emph{Sidrat Al-Muntahâ} (Arbre de l'Aboutissement). Ces feuilles
étaient grandes comme des oreilles d'éléphant et ses fruits étaient
comme des amphores. Mais par un ordre divin, il se transforma si bien
que nulle créature ne pouvait le décrire tant il était beau. Alors Dieu
me révéla ce qu'Il me révéla, et me prescrivit cinquante prières
quotidiennes. Puis je descendis vers Moïse qui me demanda~: ``Quelle
prescription Dieu a-t-il imposé à ta communauté~?'' Je lui dis~:
``Cinquante prières.'' Alors il dit~: ``Retourne voir ton Seigneur, et
demande lui un allégement, car ta communauté ne pourra le supporter~;
j'ai éprouvé les Fils d'Israël et j'ai vu ce qu'il en était.'' Je
retournai ainsi vers mon Seigneur et lui dis~: ``Seigneur~! Allège ce
que Tu as demandé à ma communauté.'' Alors, il réduisit de cinq le
nombre de prières. J'allai voir Moïse et lui dis~: Il les a réduites de
cinq prières. Alors il me dit~: ``Ta communauté ne pourra supporter
cela, retourne vers ton Seigneur, et demande lui un allégement
supplémentaire.'' Je ne cessai d'aller voir mon Seigneur --- Exalté
soit-Il --- puis de retourner à Moïse, jusqu'à ce qu'Il me dise~:
«~Mu\underline{h}ammad, cinq prières quotidiennes sont prescrites.
Chacune en vaut dix~; elles valent ainsi cinquante prières. De plus,
quiconque a l'intention de faire une œuvre pie, s'il ne l'accomplit pas,
il recevra une rétribution simple et s'il l'accomplit il en recevra dix.
Et quiconque a l'intention de commettre une mauvaise œuvre et s'en
abstient, elle ne sera point comptée à son détriment~; mais, s'il la
commet, elle comptera pour une seule mauvaise action.~» Quand je vins à
Moïse, il me dit~: ``Retourne voir ton Seigneur, et demande lui un
allégement.'' Je répondis~: ``Je n'ai cessé d'aller voir mon Seigneur à
ce sujet au point d'en être
embarrassé.''~»

Les gens divergèrent également sur la question de savoir si le Prophète
--- paix et bénédictions sur lui --- vit son Seigneur ou non. À ce
sujet, citons les narrations les plus authentiques.
\end{quote}


\subsubsection{Une lecture
philosophique}

C'est un récit qui fait appel à l'imagination, récit chargé d'images, de
merveilleux. Mais au-delà de ces images la tâche du philosophe est de
les interpréter selon la raison.

Or, Avicenne dans son Livre \emph{Miraj Nama}, \emph{Le livre de
l'Ascension}, s'applique à donner une lecture rationnelle de ce récit.


\subsubsection{
Avicenne}

Son nom est Abū Ali al-Hussayn Ibn `Abdallah Ibn Sīnā.

Il est d'origine persane. Né en 980 près de Boukhara (Ouzbékistan
actuel). Il vit dans un contexte d'une grande instalbilité politique. Il
a appris les sciences du Coran. Il est déoué d'une mémoire prodigieuse
au point qu'il a mémorisé l'ensemble du Coran à l'âge de 10 ans. Son
génie se manifeste en médecine. À l'âge de 17 ans, il parvient à guérir
un prince musulman de sa maladie. En guise de remerciement, le prince
ouvre au jeune étudiant sa bibliothèque, ce qui lui permettra d'acquérir
un savoir encyclopédique. Il lit 40 fois la \emph{Métaphysique}
d'Aristote. Un commentaire d'al-Fārābī, autre philosophe musulman, lui
en donne les clefs. On l'appelle le troisième Maître après Aristote et
al-Fārābī.

Ses ouvrages les plus célèbres sont~: \emph{Le livre de la guérison},
\emph{al-Šifā'}, et \emph{Le canon de la médecine}. Il a écrit une
centaine d'autres ouvrages, parmi lesquels une réfutation de
l'astrologie~: s'il juge les outils scientifiques, ses principes ne le
sont pas.

Ibn Sīnā était excellent mathématicien, mais s'il a été novateur en
médecine et philosophie, il n'a pas été novateur en mathématique. En
revanche, il comprenait ce qui se découvrait en mathématique. Pour lui,
l'usage de la raison n'est pas en opposition avec la foi religieuse.

En philosophie, il a une conception de l'universel~(\emph{kull}) ; il
généralise le terme de \emph{šay'}~: les choses peuvent ne pas être
existantes~; il distingue entre l'existence et l'essence. Pour lui,
l'existence s'ajoute à l'essence sauf dans le cas de Dieu où essence et
existence sont indistincts car Dieu est \emph{ṣamād}, impénétrable~; la
plénitude divine fait que l'existence ne peut pas s'y ajouter.

Il connut des épreuves et dut fuir à travers la Transoxiane et la Perse.
Il connut la prison pour être rentré en contact avec un prince rival. Il
meurt en 1037.

Ibn Sīnā est le produit d'une société qui a connut une dynamique
d'assimilation des traductions et de créativité. Son aura est d'abord
médicale~: Le canon de la médecine est une encyclopédie médicale. Il y
traite des instruments de chirurgie, d'anatomie, de la circulation du
sang. Il décrit plus de 700 médicaments qu'il invite à tester sur les
animaux. Il souligne aussi les effets de la musique et de ses incidences
bénéfiques pour le malade. Il démontre que la contagion a lieu par
l'eau. Ce canon est à la fois la synthèse de la médecine antique mais
aussi le fruit de ses observations personnelles. Jusqu'au
17\textsuperscript{ème} siècle, son livre sera la référence de tous les
étudiants en médecine en Europe.

Avicenne se démarque dans les domaines de l'ophtalmologie, de la
gynéco-obstétrique et de la psychologie. Il s'attache beaucoup à la
description des symptômes, décrivant toutes les maladies répertoriées à
l'époque, y compris celles relevant de la psychiatrie.

Mais avant tout, Avicenne s'intéresse aux moyens de conserver la santé.
Il recommande la pratique régulière du sport et préconise une médecine
préventive et curative. Il insiste sur l'importance des relations
humaines dans la conservation d'une bonne santé mentale et somatique. En
s'appuyant sur la logique, il propose de guérir par les contraires, idée
de la dialectique aristotélicienne. C'est ici son innovation médicale
majeure. Il y a une fécondation par des savoirs qui étaient isolés. Pour
lui, 
\begin{quote}
    
\emph{«~}la médecine est l'art de conserver la santé et
éventuellement, de guérir la maladie survenue dans le corps\emph{~».}

\end{quote}
Une des grandes innovations de cet ouvrage est son «~architecture~»,
avec une mise en ordre des savoirs de la médecine, et l'intégration des
apports de Rāzī ou Gallien dans une conception renouvelée de la
médecine.

 
\subsubsection{Une pensée émanatiste
}

Pour Avicenne, la pensée divine produit une succession d'Intelligences
(les anges) qui veillent sur nous. Ce sont ces Intelligences qui
éclairent l'intelligence humaine et nous permettent de comprendre les
réalités intelligibles~comme la lumière nous permet de voir les choses
sensibles. Notre intelligence devient active grâce à l'effusion de
l'Intellect agent dont la source est divine.

Mais en l'homme se trouve une faculté végétative et sensitive. Comme
tout procède de Dieu, tout désire retourner à Dieu. Ce désir en l'homme
est tourné vers Dieu. Et pour Avicenne, c'est ce désir qui arrache le
Prophète à son sommeil. Il est amené à la mosquée sacrée qui est le
centre de ce monde-ci et donc la porte qui ouvre sur l'autre monde.


\subsubsection{La nature humaine dans le Livre
de l'Ascension}

L'Avertissement est clair~: il s'agit de répondre à ceux qui
s'enquièrent d'une explication rationnelle. Pas de conte, pas de
merveilleux. Il ne sert à rien de divulguer les secrets à une personne
qui ne peut les entendre. Pire, c'est une faute. En revanche, à celle
qui peut les entendre, c'est un devoir de le faire.

La première partie parle de la nature humaine~: elle est composée de
deux entités différentes~: le corps et l'esprit. L'union du corps et de
l'esprit est comme celle du cavalier et de sa monture.

Il y a dans le corps trois types d'âme~: la naturelle connectée au foie,
l'animale reliée au cœur, la rationnelle reliée au cerveau. Les deux
premières doivent servir la troisième~: c'est ce degré de service qui
différencie les hommes entre eux. Tout dépend donc de la domination
exercée par telle ou telle âme.

L'âme rationnelle dirige, ordonne, sous la supervision de l'intelligence
appelée aussi esprit~; c'est la faculté la plus haute qui est fixée sur
l'Intelligence agente.

L'humain a le désir de se réaliser dans la plénitude qui le conduit vers
le plaisir. Ce plaisir plein est celui que procure l'intelligence des
choses intelligibles. Sa recherche ne s'arrête pas à une ascension vers
soi-même. L'intellect qui est le centre de notre humanité est aussi une
ouverture sur toutes les intelligences émanées de Dieu. Dès lors que
l'on est en contact avec telle ou telle intelligence, cela engendre en
nous une connaissance, une attitude. Celui qui est uni à la Première
Intelligence émanée, Intelligence universelle, acquiert une
compréhension universelle. Son intellect devient alors prophétique.

Cette description de la nature humaine n'est pas un préalable au voyage
nocturne de Muḥammad. Elle est son explication. L'ascension est
l'accession de l'homme à ce qu'il est et à ce qu'il doit devenir.



\subsubsection{Commentaire du
Voyage}

\begin{itemize}
\item
  Le prophète se trouvait dans un état de veille~: il faut le silence
  des sens, de la faculté irascible et de l'imagination pour commencer à
  s'éveiller spirituellement. Le désir du vrai nécessite donc ce
  silence, ce sommeil.
\item
  L'arrivée de Gabriel~: c'est l'Intelligence émanée, l'Esprit Saint,
  qui éclaire l'intellect humain et lui transmet le commandement de
  Dieu.
\item
  Avicenne fait mention de l'enlacement du prophète par l'ange qui se
  présente comme son frère~et qui l'embrasse : cela traduit la
  compénétration de l'intellect humain et de l'Esprit Saint.
\item
  Al-Bourakh~: le coursier ailé entre l'âne et la mule~: cette hybridité
  traduit la faculté de l'Esprit Saint d'être un intermédiaire entre
  l'intelligence humaine et l'Intelligence universelle. Cet Esprit Saint
  a la faculté de transporter l'intellect humain vers des réalités
  spirituelles pour lesquelles il est fait.
\item
  Le choix entre le lait et le vin~: dans cette ascension, les appétits
  tirent vers le bas
\item
  Tous les prophètes sont autant d'intelligences.
\item
  La rencontre avec Dieu~par laquelle on devient prophète. Mais être
  prophète, ce n'est pas s'anéantir dans l'ineffable, s'unir
  mystiquement à lui. C'est revenir aux autres pour s'adresser aux
  humains. Un prophète doit redescendre de la montagne pour parler aux
  humains. Mais le langage humain se heurte à celui du langage divin. Il
  convient donc de mettre en image, par la faculté imaginative. Les
  images traduisent l'expérience de la vérité. Elles parlent à tous~: à
  ceux qui s'arrêtent à la lettre, comme à ceux qui remontent à son
  esprit.
\end{itemize}

On voit ici ce que philosophie islamique peut signifier~: il ne s'agit
pas seulement de traduire un texte grec en arabe et de le commenter,
mais il s'agit de lire un récit islamique à la lumière du \emph{De
anima} d'Aristote. Pour les orthodoxes, cette démarche est indue.
L'islam n'a que faire de la philosophie païenne, elle crée même une
distorsion, une déviation. Elle contribue même à introduire des idées
pernicieuses et contraires à la foi.

Dans la tradition musulmane, celui qui est perçu comme le plus grand
contradicteur de la philosophie islamique porte le nom d'al-Ġazālī  \label{theol:AlGazali2}. Il
semblerait qu'après sa critique, il ne peut plus y avoir de philosophie
en islam. Voyons de plus prêt.

\hypertarget{la-critique-dal-ux121azux101lux12b-de-la-philosophie}{%
\section{3. La critique d'al-Ġazālī de la
philosophie}\label{la-critique-dal-ux121azux101lux12b-de-la-philosophie}}

Le \emph{Tahāfut al-falāsifa} est indéniablement l'ouvrage où al-Ġazālī \label{theol:AlGazali3}
expose la critique la plus rigoureuse et la plus systématique des thèses
philosophiques qui ont cours à son époque.

Les prolégomènes et l'épilogue à l'ouvrage mettent en lumière le lien
entre la mécréance (\emph{kufr}) et l'adhésion à ces doctrines
philosophiques qui relèvent pour al-Ġazālī de l'ordre du credo
(\emph{`aqīda}). Ainsi, après avoir demandé à Dieu de le préserver des
flambeaux de l'obscurité dans une apostrophe sublime, al-Ġazālī dénonce
l'impiété d'un groupe de personnes (\emph{ṭā'ifa}) parmi les musulmans
dont les sentiments de supériorité et d'acuité intellectuelle les ont
conduit à s'éloigner des hommes pieux, à «~mépriser les cérémonies de la
religion, les divisions de la prière, la crainte du péché, {[}à{]} se
railler des prescriptions de la loi et de ses limitations~»\sn{\textsc{al-Ġazālī},
  \emph{Tahāfut al-falāsifa}, \emph{op. cit}., p. 4, {[}Marmuna, p.
  1{]}. Nous reprenons ici la traduction du B\textsuperscript{on}
  \textsc{Carra de Vaux}, p. 146.}. Citant le Coran, ils sont pour
al-Ġazālī de ceux qui «~détournent de la voie de Dieu, qui cherchent à
la rendre tortueuse et qui au dernier jour feront partie des
incroyants~(\emph{kāfirīna}) (S. 11, 19)~»\sn{\textsc{al-Ġazālī},
  \emph{Tahāfut al-falāsifa}, \emph{op. cit.,} p. 4, {[}Marmuna, p.
  2{]}.}. La source de leur incroyance (\emph{maṣdar kufrihum}) est à la
fois religieuse et morale~: ils ont en effet délaissé la certitude de la
religion de leurs pères d'une manière insouciante en prêtant l'oreille à
la nouveauté des opinions, en suivant «~les gens de l'innovation et les
esclaves des passions (\emph{ahl al-bida`} \emph{wa
al-ahwā'})~»\sn{\textsc{al-Ġazālī}, \label{theol:AlGazali3} \emph{Tahāfut
  al-falāsifa}, \emph{op. cit}., p. 4, {[}Marmuna, p. 2{]}.}. Afin de
donner une caution intellectuelle à leur posture immorale, leurs meneurs
prirent appui sur les noms honorables des philosophes grecs tels
Socrate, Hippocrate, Platon ou Aristote. Forts de cette assise
philosophique, ils sont persuadés de pouvoir scruter la profondeur des
choses cachées (\emph{al-'umūr al-ḫafiyya}). Or, font-ils remarquer, ces
philosophes, éclairés par leur savoir, niaient la loi et les religions.
Il convient donc de les imiter.

D'emblée, remarquons qu'al-Ġazālī \label{theol:AlGazali4} ne nie pas la beauté des principes de
la philosophie antique, la qualité et la précision de leur logique, de
leur géométrie ou de leur physique, mais il voit dans le recours à ces
philosophes une nouvelle forme d'imitation (\emph{taqlīd}) qui vient se
substituer à la transmission traditionnelle de la foi. Or, cet abandon
de la connaissance de Dieu au profit de l'imitation d'esprits impies,
sous couvert d'intelligence et d'habileté, est pour lui pure
transgression et démence~(\emph{ḫarq wa ḫabāl})\sn{\textsc{al-Ġazālī},
  \emph{Tahāfut al-falāsifa}, \emph{op. cit}., p. 5, {[}Marmuna, p.
  2{]}.}. En outre, al-Ġazālī relève que non contents d'épouser ces
idées, ces pseudo-philosophes accusent de sottise et d'illusion la
répugnance qu'éprouvent les serviteurs de Dieu à changer leurs
croyances.

Devant ce constat de crise, de dénigrement condescendant mais aussi de
contagion, al-Ġazālī  \label{theol:AlGazali5} entreprend la réfutation (\emph{radd}) des
philosophes pour montrer «~l'incohérence de leurs croyances et la
contradiction de leur discours (\emph{tanāquḍ kalimatihim}) en ce qui
concerne la métaphysique~(\emph{al-ilāhiyyāt})~»\sn{\textsc{al-Ġazālī},
  \emph{Tahāfut al-falāsifa}, \emph{op. cit}., p. 6, {[}Marmuna, p.
  3{]}.}.

L'intention première d'al-Ġazālī, telle qu'elle transparaît dans ces
prolégomènes, n'est donc pas de combattre les philosophes, mais plutôt
ceux qui se réclament des philosophes et s'appuient sur eux pour
s'écarter de l'islam, tout en voulant justifier et fonder
rationnellement leur conduite. Il est symptomatique à cet égard
qu'al-Ġazālī n'expose pas le système d'un philosophe en particulier,
mais plutôt des thèses qui ont court en philosophie. À l'exception d'Ibn
Sīnā, cité quinze fois, d'al-Fārābī, cité deux fois et de quelques
philosophes grecs, al-Ġazālī \label{theol:AlGazali6} ne nomme pas explicitement les philosophes.
Il en expose seulement les idées. Par suite, il n'accuse pas
nominalement d'infidélité ceux qui suivent ces thèses, mais il soutient
que l'affirmation de telle thèse philosophique relève de l'incroyance~:

\begin{quote}
«~Tous les écrits de ces philosophes s'accordent dans la foi en Dieu et
au dernier jour. Toutes les thèses en discussion se ramènent pour moi,
écrit al-Ġazālī, à des écarts à partir de ces deux pôles de la foi, pour
l'affermissement desquels ont été envoyés les prophètes, armés de
miracles. Personne ne songe à les nier, sauf un petit noyau d'hommes à
l'esprit perverti et à la raison renversée, dont on ne s'occupe pas et
qu'on ne cite pas parmi les spéculatifs, mais qui ne peuvent être
comptés que dans la bande des mauvais démons, dans la tourbe des
grossiers et des sots~»\sn{\textsc{al-Ġazālī},
  \emph{Tahāfut al-falāsifa}, \emph{op. cit}., p. 6-7, {[}Marmuna, p.
  3{]}, traduction du B\textsuperscript{on} \textsc{Carra de Vaux,} p.
  148.}.
\end{quote}

Al-Ġazālī \label{theol:AlGazali7}  n'accuse pas les philosophes en général d'impiété. Au
contraire, il affirme qu'ils croient en Dieu et en ses messagers.
Cependant, «~ils s'égarent sur des détails qui suivent les principes
(\emph{fa innahum} \emph{iḫtabaṭū fī tafāṣīl ba`d haḏihi
al-uṣūl})~entraînant une certaine confusion »\sn{\textsc{al-Ġazālī},
  \emph{Tahāfut al-falāsifa}, \emph{op. cit}., p. 7, {[}Marmuna, p.
  3{]}.} pouvant conduire les esprits hors du droit chemin, celui de
l'islam. Dans l'introduction au \emph{Tahāfut}, al-Ġazālī pose donc la
distinction entre les philosophes qui ont foi en Dieu et en ses
messagers et ceux qui ne croient pas. Les premiers peuvent adhérer à des
idées qui ne sont pas celles des croyants de la tradition, mais ils
restent du même avis sur les fondamentaux de la foi. En revanche, ils
peuvent être récupérés par les seconds dont l'esprit «~pervers~» est en
quête de distinction ou d'autojustification.


\section{Conclusion}\

En réponse à al-Ġazālī \label{theol:AlGazali8}, Averroès a rédigé un \emph{Tahāfut al-tahāfut}
c'est-à-dire l'Incohérence de l'Incohérence. Il répond à al-Ġazālī \label{theol:AlGazali25}~:
dans un premier temps, il reconnaît que les critiques philosophiques
d'al-Ġazālī ne sont pas sans justesse. Mais le maître est lui-même
critiquable et c'est ce qu'il entreprend de démontrer dans cet ouvrage.

Je noterai que, contrairement à une idée reçue, al-Ġazālī n'est pas
contre les philosophes. Il rejette certaines de leurs thèses contraires
à la foi, mais lui-même était philosophe. Revenir aux fondations de la
philosophie en islam, renouer avec les anciens, avec leur méthode, leur
épistémologie est sans aucun doute une piste pour l'islam contemporain
en quête de réforme. Car si bien des musulmans appellent à refonder la
pensée musulmane, le grand problème posé porte sur le comment. Comment
s'y prend-t-on~? Ici, la philosophie ouvre des voies qui permettent une
distance aux textes et leur donnent une intelligence.


\section{Bibliographie}


Rémi \textsc{Brague}, «~En quoi la philosophie islamique est-elle
islamique ?~» dans : Geneviève Gobillot, Marie-Thérèse Urvoy,
\emph{L'Orient chrétien dans l'empire musulman : hommage au professeur
Gérard Troupeau}, Paris, Edition de Paris, 2005, p. 119-141.

Souleymane Bachir \textsc{Diagne}, «~Islam et philosophie~: leçons d'une
rencontre~», dans \emph{Diogène}, 2003/2, n° 202, p. 145-151.

A\textsc{l-Ghaz}Ā\textsc{l}Ī, \emph{The Incoherence of the
Philosophers}, \emph{Tahāfut al-falāsifa}, A parallel English-Arabic
text translated, introduced, and annotated by Michael E. Marmura,
Islamic Translation Series, Brigham Young University Press, Provo, Utah,
2000² (1997).

Léon \textsc{Gauthier}, \emph{Ibn Thofaīl, sa vie, ses œuvres}, Paris,
Vrin, 1983.

Peter \textsc{Heath},\emph{Allegory
and Philosophy in Avicenna: With a Translation of the Book of the
Prophet Muhammad's Ascent to Heaven}, University of Pennsylvania Press,
1992.

Jean-Michel \textsc{Hirt}, \emph{Le voyageur nocturne}, Lire à l'infini
le Coran, Paris, Bayard, 2010.

Ibn \textsc{Tufayl}, \emph{Le Philosophe autodidacte}, Paris, Mille et
une nuit, 1999.
