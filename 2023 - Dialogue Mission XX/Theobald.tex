%\chapter{Urgences pastorales du moment présent Christoph Theobald }





%------------------------------------------------------------
\section{ Le contexte et l’auteur}
%------------------------------------------------------------

\paragraph{Christoph Theobald} Théologien jésuite, d'origine allemande mais vivant en France, probablement le théologien francophone vivant le plus connu dans le monde. Il est réputé d'accès difficile du fait d'une écriture serrée et surtout d'un dialogue toujours renouvelée avec la philosophie et la théologie française et allemande (en particulier, il est connu comme le traducteur des oeuvres de Rahner en français).  Son livre le plus connu, \textit{le Christianisme comme style} \cite{theobald_christianisme_2007} essaye de définir le christianisme à partir de son \textit{style}, dont l'origine est à chercher dans la vie concrète de Jésus, vie qu'il résume par le terme d'\textit{hospitalité} : comme Marie-Madeleine au jardin, on reconnaît le christianisme dans ce style, même s'il est transformé par la culture dans lequel il vit.


\paragraph{Une théologie libérale "d'en bas"} La pensée de Ch. Theobald est marquée par différents points que l'on retrouve dans le texte : 
\begin{itemize}
    \item \textit{On ne peut accéder au Jésus historique} : on ne peut atteindre Jésus historique  on part des récits évangéliques, éclairés par la Résurrection, travaillés par elle. D'où l'importance des textes bibliques et de l'évangile.
    \item  \textit{Une approche éthique, pragmatique partant de Jésus.}    Approche originale et adaptée à notre monde. Theobald développe essentiellement une  théologie au \textit{service} des contemporains. 
    \item \textit{Un contexte hors chrétienté : lien social dans une société pluraliste} Dans une société pluraliste,  Le lien n'est plus assuré par le religion mais par l'Etat, neutre.
\end{itemize}
  
 
 
 

%------------------------------------------------------------
 \subsection{contexte d'urgence pastorale}


\paragraph{Une Eglise minoritaire dans les pays traditionnellement Chrétiens} Ch. Theobald part du constat des nombreuses crises de l'Eglise : chute de la pratique, ... L'Eglise est devenue minoritaire dans les pays Occidentaux. 
 

\paragraph{Evangelii Gaudium : lien intrinsèque entre réforme et mission} Un autre élément de contexte est la publication de \textit{Evangelii Gaudium} du Pape François. François incarne ce que Theobald \cite{theobald_pastoralite_2021} appelle une \textit{herméneutique de la réforme}. Le premier chapitre d'EG fonde cette réforme sur la pastorale missionnaire; dans la lignée du décret \textit{Ad gentes}. Si l'on insiste sur la tâche missionnaire \cite[p.162]{theobald_courage_2021}, alors la réception du \textit{texte lui même} devient essentiel. D'où le choix de François de rester \textit{discret par rapport aux options théologiques, voire doctrinales} dans ses écrits : \textit{lex omnis evangelisationis } (GS 44).
\begin{singlequote}
     la variété [des "diverses lignes de pensées philosophique, théologique et pastorale] aide à manifester et à mieux développer les divers aspects de la richesse inépuisable de l'Évangile » [EG 40]
\end{singlequote}

François reprend aussi que d'autres aspects de la pastoralité conciliaire, peu présents ces dernières décennies : l'insistance sur le peuple des baptisés comme peuple missionnaire, porté par le «sens de la foi»; la réévaluation de la pluralité interne de l'Église et de nos cultures. \cite{theobald_pastoralite_2021}



%------------------------------------------------------------
\subsection{Le livre : Urgences pastorales du moment présent}

\paragraph{Urgences pastorales du moment présent} Le livre \textit{Urgences pastorales du moment présent - Comprendre, partager, réformer} \cite{theobald_urgences_2017}, dont est tiré le chapitre que nous étudions, est un livre de théologie appliquée :   comment inciter l’Église à retrouver un dynamisme missionnaire. Et cela passe par une réforme qui « porte sur la capacité des chrétiens et de leurs Églises à mettre l’Évangile du règne de Dieu à la disposition de toute l’Humanité et de toute la terre comme "ressource" salvatrice » (p. 13).

\paragraph{le constat de l'auteur sur la situation de l'Eglise}  \cite[p 26-29]{theobald_urgences_2017} A CREUSER 

 \paragraph{première partie : le Diagnostic } Theobald analyse quelques-uns des défis actuels : montée de l’islam, crise écologique, évolution du rapport à la mort,… Mais Theobald ne s'arrête pas à ce diagnostic sombre.  Le Christ lui aussi s'est confronté aux réalités les plus dures mais qui ne sont jamais des enfermements. Quels sont les nouvelles manières d’habiter le monde de nos contemporains capable de résonner avec les urgences de l’époque et de resister à la tentation du repli : exigence de respect de l’autonomie personnelle,  sens d’une solidarité fraternelle large,  conscience de l'enjeu pour l’avenir de notre planète. Beaucoup  sont prêts à s’engager et à payer de leur personne. Il y a ici un potentiel de créativité insoupçonné. Or la « proposition de sainteté » que l’Église donne à percevoir rejoint difficilement ces aspirations nouvelles . À tort ou à raison, elle est perçue comme légaliste, abstraite, trop préoccupée de sexualité ;  » (p. 84). 

\paragraph{Quelle réforme face à  l'affaiblissement de l'Eglise} Pour Ch. Theobald, 
l'Eglise ne prend pas vraiment au sérieux le kerygme, sa propre source, l'hospitalité inconditionnelle de Jésus.\cite{etienne_grieu_leglise_nodate}.
 Souvent encombrée par le souci de sa propre perpétuation, elle a bien du mal à laisser l’Esprit donner naissance à des figures nouvelles. Elle parvient difficilement, par exemple, à penser la communauté chrétienne plus largement que selon le mode paroissial, construite autour du ministre ordonné et accaparée avant tout par les services religieux (funérailles, baptêmes, mariages). Theobald, soucieux de l’enracinement local de l’Église, propose de « laisser advenir de véritables communautés sur place » qui soient « sujets collectifs » et « missionnaires pour leur environnement » (p. 324). Cela suppose aussi de passer à une nouvelle figure du pasteur, « passeur » plutôt que « pivot » (pp. 329-332).\cite{etienne_grieu_leglise_nodate}
 
 
 



%------------------------------------------------------------
\section{Chapitre étudié : Etape d'une Ecclésiogénèse}
%------------------------------------------------------------

\paragraph{Ecclésiogènese}
Dans ce dernier chapitre, Ch. Theobald propose un \textit{processus} pour la création et le développement des Eglise. 
Le terme d'\textit{Eccésiogénèse} vient d'Amérique Latine, Leonardo Boff, un des théologiens de la libération les plus influents. Ch. Theobald en donne la définition suivante : 
\begin{singlequote}
    un devenir ou une germination possible quand l'existence ecclésiale se réduit à un \textit{Presque rien}.
\end{singlequote}

Une telle définition a une vocation d'antidote, nous renvoyant aux débuts de l'Eglise, qui ne peut être que missionnaire et non tournée vers elle-même. 

\subsection{Le décret Ad Gentes (AG) : une conception génétique de l'Eglise} Quand l'Eglise occidentale n'est plus réduite qu'à un presque rien, il n'y a plus de distinction à faire entre pays de mission et pays christianisé. L'A. propose donc de reprendre le décrit conciliaire sur l'activité missionnaire de l'Eglise \textit{Ad Gentes} (1965) et de voir sa pertinence par rapport à la crise de l'Eglise en Occident. 
Ce texte est en effet est le premier du magistère qui développe une \textit{ecclésiogènese}. Dès les principes d'\textit{Ad Gentes}, il y a identification entre la nature même de l'Eglise et son activité missionnaire .  C'est une nouveauté par rapport à LG qui séparait les deux activités. 

\paragraph{Approche génétique et historique } Selon AG, \textit{l'activité missionnaire de l'Eglise se différencie selon les "circonstances"} (AG 6) et donc l'Eglise développe différents \textit{moyens}. Mais approche historique ne veut pas dire développement linéaire, les différentes phases n'étant pas dépassées. C'est d'ailleurs ce point qui permet à AG de s'appliquer à toutes les Eglises, aucune n'étant \textit{parfaitement constituée}.

\paragraph{les différentes phases de la Genèse de l'Eglise selon AG} AG distinguent trois types idéaux : 
\begin{itemize}
    \item tout d'abord le témoignage, la présence et la \textit{conversation} avec les contemporains, à l'image de Jésus et ses contemporains.
    \item puis la \textit{prédication}
    \item enfin seulement le \textit{façonnement de la communauté Chrétienne}
    
\end{itemize}

\paragraph{Une référence d'AG à l'Ecriture : les Actes des Apôtres} Ch. theobald prend au sérieux les nombreuses références aux Actes des Apôtres dans le décret ce qui pousse à lire \textit{Ac} comme le récit d'une genèse d' Eglise. A partir de l'analyse de MF Baslez, Theobald souligne l'importance des structures associatives pour la genèse de l'Eglise, mais que le baptême entraine une mixité sociale inédite. De la même manière, quelles sont les canaux contemporains qui permettent le développement des églises dans leur milieux culturels ? \textit{Eglise} dans les \textit{Ac} n'a pas la signification technique qu'il a aujourd'hui mais prend souvent son sens original d'\textit{assemblée} : la désignation \textit{Eglise} intervient progressivement : d'abord \textit{quelques-uns}, \textit{maisons}...
A partir de cette analyse, Theobald retient quelques points d'ancrage: 
\begin{itemize}
    \item Importance du lieu comme enracinement culturel : les Eglises locales sont autonomes car c'est le lieu qui marque la réalité culturelle, avec des différences notables comme celle du concile dit de \textit{Jérusalem}.
    \item lors d'un différent, le récit permet de discerner l'action de Dieu parmi nous. Puis il note l'étape de délibération et de décision.
    
\end{itemize}

\subsection{Que retenir de ce périple ?}
\pararagraph{Une Ecclesiologie "d'en bas"} A la différence du début de \textit{Lumen Gentium} qui part du mystère de l'Eglise dans le dessein trinitaire de Dieu, AG et les \textit{Actes} propose une \textit{ecclesiologie} d'en bas. Et Theoblad souligne que même dans LG, des traces de l'ecclésiogénèse sont visibles.

\paragraph{lien entre Actes et le Royaume de Dieu} La difficulté de rattacher l'ecclésiogénèse aux \textit{Actes}, est de définir le lien entre Eglise et Jésus.  Pour cela, dans l'esprit de LG 5, définit le commencement de l'Eglise dans le message de Jésus sur le Royaume de Dieu. Mais ce message ne se comprend que dans les signes de Jésus, signes messianiques : les pauvrent entendent, les aveugles voient.
Ces signes, comme les charismes pauliniens sont d'abord, souligne Theobald, des \textit{personnes vivantes}.

\paragraph{l'Eglise sacrement universel de Salut} On peut mieux comprendre l'expression \textit{Eglise sacrement universel de Salut} à partir des signes messianiques synoptiques et des charismes pauliniens : les deux, "sacrementum", ou "mysterion", sont marqués par la surabondance et dépassent la sphère classique des 7 sacrements. Pour retrouver la dimensions messianique de ces signes, l'Eglise doit porter l'attention aux personnes et non aux gestes. La prolifération des formes de gouvernance (les Soixante-douze, les Sept) autour de la relation voulue par Jésus avec les Douze, laisse une place à l'adaptation et à l'Esprit Saint \textit{au grès des circonstances.}



 


%------------------------------------------------------------
\subsection{ Et aujourd'hui ?}
Theoblad conclut par une proposition d'un itinéraire d'ecclesiogénèse autour de   sept étapes , découpées, sur la base du décret missionnaire Ad gentes et des Actes des Apôtres.

\begin{singlequote}
   
  La naissance d'une expérience missionnaire suppose dans un premier temps qu'une communauté se laisse surprendre par ce qui se passe en telle personne ou en tel groupe, non seulement en tel catéchumène, mais bien plus largement en «quiconque», qu'elle éprouve donc concrètement d'être sans cesse « précédée » par !'Esprit. Ce n'est pas quelque chose que l'on commande ou que l'on peut produire par soi-même, mais pour percevoir cette « précédence », des dispositifs peuvent être mis en place : des espaces hospitaliers (1ere étape), la lecture commune des Écritures qui ne cessent de nous confronter à la diversité quasi infinie des présences de l'Esprit dans l'histoire (2e étape) et l'attention à ce que l'Esprit donne aux personnes (3è étape) - et d'autres dispositifs qui peuvent encore être inventés. A partir d'un certain moment, ce qui est éprouvé par des individus doit devenir une expérience collective et un bien commun. C'est ce qui peut se passer entre la troisième et la quatrième étape et  
déterminer toute la suite de l'itinéraire. De nouveau, des dispositifs doivent aider à percevoir la communauté tout entière, et non seulement tel individu, comme « sujet missionnaire » : la délibération et la découverte de l'œuvre de l'Esprit dans l'inattendue entente mutuelle (4e étape), l'expérience d'une surprise spirituelle qui devient possible, quand on prend soin de la corporéité de nos célébrations et de l'Église (Se étape) et la découverte du travail de l'Esprit de Pentecôte dans une Église qui ne cesse de nous surprendre par ses dimensions à la fois universelles et polyédriques. C'est la contemplation sous ses multiples formes (7e étape) qui achève l'itinéraire. Elle rend aussi intérieurement évident que cet itinéraire ne s'achève jamais, tant que l'histoire dure, et qu'il faut reparcourir sans cesse ces différentes étapes - ou d'autres - avec toujours plus de profondeur.\cite[p. 460]{theobald_urgences_2017}

 \end{singlequote}
 



\begin{comment}
    \begin{enumerate}
    \item la genèse d'Eglise dans un \textit{espace hospitalier}
    \item la relation à l'écriture sainte
    \item la découverte de personnes nouvelles et de leurs charismes
    \item une communauté qui se met à délibérer
    \item la perception des dimensions corporelles de la Foi
    \item la prise en compte de la dimension toujours plus universelle de l'Eglise

    \item contemplation
\end{enumerate}
regard sur le chemin parcouru
\end{comment}
 

%------------------------------------------------------------
\section{ Elements de discussion}
%------------------------------------------------------------


\subsection{sur le texte}
\paragraph{Apport de l'exégèse} Un premier point marquant, c'est le détour pour parler des réponses à la crise de l'Eglise par les textes conciliaires \textit{Ad Gentes} et \textit{Gaudium et Spes} et surtout à une exégèse des références bibliques sous-jacentes à ces deux textes, d'un côté les Actes des Apôtres, de l'autre, les Evangiles Synoptiques et St Paul. Même s'il est peu cité, l'influence du pape François et en particulier EG peut être repéré dans tout le document (herméneutique de la réforme, structure polyédrique,...)



\paragraph{un processus proposé} Theobald propose donc un processus, un \textit{exercice}, qui reprend beaucoup des méthodes ignatiennes communautaires en particulier pour déceler les priorités \textit{apostoliques} des jésuites. 



%------------------------------------------------------------
\subsection{sur l'approche proposée}


\paragraph{Place des pauvres} Une attention particulière pourrait être donnée aux \textit{pauvres} : l'hospitalité de Jésus s'adresse à tous, le pharisien, les scribes. Mais avec une \textit{partialité bienveillante pour les pauvres} qui ne sont pas particulièrement nommé ici 
\begin{singlequote}
    En somme, la proposition de Theobald est une invitation pour l’Église à se laisser questionner par ce que vivent nos contemporains, avec la confiance que, de cette rencontre, ne peut surgir qu’un surcroît de vitalité évangélique. Bien sûr, les axes ainsi ouverts sont faits pour être débattus (pour ma part, je serais enclin à insister davantage sur une présence prioritaire à ceux qui vivent des situations de grande détresse, ce qui permet de mieux honorer, je crois, le rendez-vous pascal de la vocation chrétienne).\cite{etienne_grieu_leglise_nodate}

\end{singlequote}

 
 


\paragraph{Proximité et différence entre les 6 étapes de la conversion au Christ et le processus proposé }
\begin{itemize}
    \item Apprentissage de Jésus (He 5, 8) dans ses rencontres, il vit la sainteté

\item Jésus ne s'impose pas mais permet à l'autre d'accéder à lui -même. Il va libérer l'autre de ses craintes et se dire. 
\item l'émergence de la Foi : quand on accède à son identité, elle commence à être sauvé. Foi en Dieu. cf la Femme adultère, \textit{Ta Foi t'a sauvé}. de l'hospitalité de Jésus arrive la Foi. Il prend des exemples dans l'Evangile : elle libère les personnes rencontrées et les ouvre à une vie nouvelle
\item refiguration, mue d'identité : conversion des personnes qui croisent Jésus, ces gens qui ont changé de vie en rencontrant Jésus. Chacun à adopter le style de vie messianique. Nous avons à devenir hospitalier et leur permettre d'advenir à leur identité.
\item les personnes qui accèdent à cette vie nouvelle ({le possédé à Gerasa reste et ne devient pas disciple de Jésus}) ne sont pas forcément tous disciples. Mais ces rencontres ne sont pas forcément un changement de vie. Refus. Comment alors maintenir le lien social quand il y a refus de la rencontre ? 
Théobald introduit la règle d'Or.
\end{itemize}
 Par rapport à une discussion que nous avons eue sur la \textit{potentialité d'une société véritablement chrétienne}, on peut noter que le processus individuel invoque \textit{une mue d'identité}, ce qui n'est pas le cas pour l' \textit{Eglise} à moins de considérer la mission comme véritable mue d'identité pour l'Eglise.


\paragraph{Et si on réabilitait les tradi ? } \paragraph{quand un presque rien est centré sur lui-même  : le \textit{reste}}


\paragraph{7 vs 12 : quand le lieu n'est pas l'unique déterminant culturel }

 
\paragraph{quelle est la théologie de la mission dans cette lettre}


\paragraph{il n'y a pas de visibilité sans témoignage : question d'Alexis Fayez} Diognète. 

 

