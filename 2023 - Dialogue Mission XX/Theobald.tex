\chapter{Urgences pastorales du moment présent
Christoph Theobald
 }

\mn{Urgences pastorales du moment présent
- Comprendre, partager, réformer Christoph Theobald}






\section{ETAPES D'UNE ECCLESIOGENÈSE}

Ce dernier chapitre s'encha1ne immédiatement au précédent,.clôturant notre tentative de rassembler et d'articuler, de manière cohérente, les différents paramètres d'une pédagogie de la réforme, tant appelée de ses vœux par le pape François dansson exhortation Evangelii gaudium. Une telle pédagogie est nécessairement anticipatrice : contre tout passéisme ou pragmatisme pastoral-« ona toujours fuit ainsi» (EG, 33)-,  elle doit d'abord avoir le courage d'expliciter la vision d' Iise qui est en train de s'esquisser au sein de nos sociétés européennes, car sans une telle vision d'avenir, au moins embryonnaire, il est impossible d'envisager une« ré-forme» missionnaire de l'Eglise et de former des acteurs capables d'œuvrer en sa faveur (cf chap. 7). Sur la base d'une telle vision, respectueuse des quatre principes du pape François1, en particulier du troisième -  « La réalité est
\mn{l. Rappelons les quatre principes ; l. Le temps est supérieur à l'espace ;
2. L'unité prévaut sur Je conflit; 3. La réalité est plus importante que l'idée;
4. Le tout est supérieur à la partie (EG, 221-237).}
plus importante que l'idée}} (EG, 231-233) -, il faut ensuite développer une manière de procéder (ce que nous venons de faire) et tracer un chemin qui nous rapproche de ce que now, avons perçu; c'est ce que nous allons faire maintenant. L'aspcl-t
« utopique » de la vision se trouve ainsi « contrebalancé» par les moyens concrets mis à notre disposition pour avancer à partir du point exact où se trouve l'Église d'aujourd'hui.
Venu de l'Amérique latine (Léonardo Boff), le conœpt d' « ecdésiogenèse » désigne précisément cet aspect du devenir : un devenir ou une germination possible quand l'existence ecclésiale se réduit à nn « presque rien». Il se situe donc à l'endroit même de la «crise», évoquée dès les premières pages de cet ouvrage1, et au carrefour qui se présente ici : crise fatale 0 1 point de départ d'une transformation et d'une maturation. A ce titre, il exerce la fonction d'un antidote ; ou, pour s'inspirer d'une formule de Dietrich Bonhoeffer (en mai 1944), « il nous renvoie aux débuts de l'Église2 ».

\mn{1.	Cf plus haut, p. 26-29.	.
2.	"Aujourd'hui tu reçois le baptême chrétien. On prononcera sur 101 toutes les grandes paroles anciennes de l'annonce chrétienne et on accom plira sur toi l'ordre de baptiser, donné par le Christ sans que tu n'y com prennes rien. Nous aussi nous sommes renvoyés aux déb ts du compren_dre._Ce. que signifient réconciliation et rédemptiou, nouvelle na1ssauce et Espnt samt. amour des ennemis, croix et résurrection, vie en Christ et imitation de fésus Christ, tout cela est devenu si difficile et si lointain que c'est à peine si nuu osons encore en parler. Nous soupçonnons un souffle uouveau et boulever• sant dans les paroles et les actions qui nous ont été transmises, sans pouvoir encore le saisir et l'exprimer. C'est notre propre faute. Notre Église, 11111 n'a lutté, pendant ces années, que pour se maintenir en vie, comme si elle étt11t}



Cette ecdésiogenèse ne peut cependant être que missionnaire, celle d'une Église « décentrée », non tournée vers elle-même. Vatican Il et l'expérience ecclésiale des cinquante dernières années nous l'auront appris, du moins on peut l'espérer. C'est en tout cas la conviction fondamentale qui anime cet ouvrage. Nous avons déjà montré, dans le chap. 3, que la distinction classique, encore en vigueur au dernier concile, entre <( pays christianisés » et« pays de mission » ne fonctionne plus aujourd'hui, la France, voire l'Europe tout entière, étant devenue un vaste ensemble de pays de mission. Ce qui signifie qu'il faut impérativement sortir le décret conciliaire sur l'activité missionnaire de l'Église Ad gentes (1965) de son statut marginal (ou seulement pertinent sur d'autres continents) et lui donner la fonction de dé d'interprétation des deux grandes constitutions dogmatique et pastorale sur l'Église et l'Église dans le monde de ce temps, diptyque ecclésio-anthropologique qui domine jusqu'à nos jours la réception de Vatican 111.
Or, Ad gentes est le premier document officiel qui développe une « ecclésiogenèse », précisément dans une perspective

son propre but, est incapable d'être la porteuse de la parole réconciliatrice et rédemptrice pour les êtres humains et le monde. C'est pourquoi les paroles antérieures doivent perdre leur force et céder au silence; notre être chrétien ne peut aujourd'hui consister qu'en deux choses: ln prière et faire ce qui est juste parmi les humains. Toute peusée, toute parole et toute organisation, dans le domaine du christianisme, doivent renaitre à partir de cette prière et de
cette action » (D. Bonhoeffer, « Pensées pour le jour du baptême de D.W.R., Mai 1944 », Résistance er.soumission. Lettre et notes de captivité, Genève, Labor et Fides, 2006, p. 353; c'est nous qui mettons en relief certains passages).
\mn{1.	Cf plus haut, p. 104-106, 130 sqq., 204 sqq. et 222.431}

«décentrée» et missionnaire. Nous commencerons donc par expliciter quelque peu cette référence conciliaire trop peu connue, avant de reporter ce parcours à sa matrice scripturaire, essentiellement les Actes des Apôtres qui nous ont accompagnés tout au long de cette troisième partie, et de terminer par le tracé d'un chemin type pour aujourd'hui.


LE DÉCRET AD GENTES :
UNE CONCEPTION GÉNÉTIQ.UE DE L'ÉGLISE·

Avant d'aborder le devenir de l'Église proprement dit, le décret missionnaire Ad gentes traite, dans son premier chapitre, de quelques « principes doctrinaux1 ».

DE SA NATURE L'ÉGLISE EST MISSIONNAIRE

L'exposé de ces principes - soit dit au préalable - est très intéressant ; car il ressaisit, voire développe, en 1965 ce qui, dans la constitution sur l'Église, promulguée l'année précédente, reste encore au stade inchoatif, en particulier l'identification entre la « nature même » de l'Église et son « activité missionnaire », considérées comme différentes dans Lumen gentium (cf. LG, 1). Cette identification est décisive, comme nous l'avons montré au chap. 4 de cet ouvrage, car elle représente le point aveugle de la réception de Vatican II en Europe:


l. Pour ce qui suit, cf C. Theobald, Le concile Vatican TI. Quel avenir?,
Paris, Le Cerf, 2015, p. 205-226.

432
 
ÉTAPES D'UNE ECCLÉSIOGENÈSE

« De sa nature, l'Église, durant son pèlerinage sur terre, est missionnaire, puisqu'elle-même tire son origine de la mission du Fils et de la mission du Saint Esprit, selon le dessein de Dieu le Père» (AG, 2). Les n°' 2, 3 et 4 du décret explicitent alors cette affirmation trinitaire (le dessein du Père, la mission du Fils et la mission de }'Esprit), selon le même ordre que celui qui fut adopté par les n°5 2, 3 et 4 de la constitution Lumen gentium. Mais les différences sont notables et concernent non seulement la perspective missionnaire qui, en 1965, devient dominante, mais aussi l'enracinement biblique
du texte, désormais plus construit et plus nettement appuyé sur l'œuvre lucanienne.
L'approche génétique et historique du décret apparaît dans toute sa clarté au n° 6. L'activité missionnaire de l'Église se différencie, premièrement, selon les « circonstances » (condiciones), c'est-à-dire selon« les peuples, les groupes humains ou les hommes à qui s'adresse la mission» (AG, 6 § 2) et conduit de là, deuxièmement, à un choix d'activités et de (<moyens» (nous en avons développé certains dans le chapitre précédent) ; ce qui fait que telle Église particulière « connaît des commencements et des degrés dans son action [... ] ; de surcroît, elle est parfois contrainte, après des débuts heureux, de déplorer
de nouveau un recul, ou tout au moins de demeurer dans un état de semi-complétude et d'insuffisance» (ibid1.). Dans le
quatrième paragraphe, cette vue est encore plus clairement articulée, après quelques ultimes corrections : les différentes phases de croissance de l'Église ne sont jamais « dépassées »

1.	C'est nous qui soulignons.


433
 
URGENCES PASTORALES DU MOMENT PRÉSENT

(superatis), comme l'affirmait au contraire l'avant-dernière version du texte; elles ne peuvent être qu'« accomplies)> (expletis). Et les Églises particulières ne sont jamais « parfaitement constituées)) (perfecte constitutis) - c'est ainsi que le formulait l'avant-dernière version -, elles sont simplement « déjà constituées» (iam constitutis) ; elles ont dès lors le devoir de continuer l'activité missionnaire.

LA GENÈSE DE L'ÉGLISE SELON AD GENTES

Le but de la mission, celui d'annoncer l'Évangile et d'implanter l'Église, est nettement souligné dès le début du décret missionnaire Ad gentes (AG, 6 § 3), mais sa mise en œuvre effective n'est articulée que dans le deuxième chapitre en trois étapes distinctes d'une genèse <l'Église. Le premier article (I) décrit le point de départ : le « témoignage » des chrétiens dans un environnement non chrétien ; leur << présence » (praesentia, AG, 12 § 1), « afin que les autres, considérant leurs bonnes œuvres, glorifient le Père et - ajout d'importance - perçoivent plus pleinement le sens authentique de la vie humaine et le lien universel de communion entre les hommes» (AG, 11 § 1). On retrouve ici la racine de certaines insistances de notre chap. 6 sur un« nouvel humanisme». Notons en particulier l'intérêt du texte pour la « conversation » des chrétiens avec leurs contemporains, dans l'esprit même des dialogues du Christ Jésus dans les récits évangéliques1•

1.	Cf C. Theobald, « La conversation spirituelle aujourd'hui. Une expérience pastorale)>, Paroles humaines, parole de Dieu, op. cit., p. 159-186.

434
 
ÉTAPES D'UNE ECCLÉSIOGENÈSE

Suit, dans un deuxième article (Il), la prédication, rendue possible par Dieu lui-même. La formule inaugurale du n° 13 est le meilleur antidote contre tout volontarisme missionnaire :
« Partout où Dieu ouvre la porte à la prédication... ». Le texte
analyse alors le processus de conversion et sa structure spirituelle et. sacramentelle, le but étant le rassemblement du peuple de Dieu (AG, 13 et 14). Ce n'est que dans le troisième article (III) qu'est retracé, pas à pas et pour ainsi dire d'en bas, le « façonnement» de la communauté chrétienne, les services qui y sont nécessaires étant mentionnés dans toute leur ampleur (AG, 15
§ 7), avant qu'il ne soit question de la fonction presbytérale et d'autres vocations spécifiques (AG, 16 et 18).
.  es trois étapes sont évidemment à entendre comme des types ideaux; le texte renvoie à plusieurs reprises à leur interconnexion et à leur conditionnement historique. Cela vaut particulièrement aujourd'hui où les conditions de vie postmodernes rendent difficile des trajectoires de vie continues sur le long terme et où la carte du chemin à emprunter s'est quelque peu perdue. Il s'agit dès lors de se rendre attentif à un autre aspect du texte, qui a été trop peu remarqué : sa référence à ['Écriture. Sans pouvoir en présenter ici une analyse détaillée, je voudrais souligner la place des deux volets de l'œuvre lucanienne dans la structure du décret missionnaire: elle apparaît pour la première fois au n° 4 qui, en prolongeant de manière fort originale le n°4 de Lumen gentium, décrit l'envoi de l'Esprit; et, à partir de là, elle est présente tout au long du texte :
« À travers toutes les époques, c'est le Saint-Esprit qui "unifie l'Église tout entière dans la communion et le ministère, qui la munit des divers dons hiérarchiques et charismatiques" (LG, 4),

435
 
URGENCES PASTORALES DU MOMENT PRÉSENT

vivifiant à la façon d'une âme les institutions ecclésiastiques et insinuant dans le cœur des fidèles le 'même esprit missionnaire qui avait poussé le Christ lui-même. Parfois même il prévient visiblement l'action apostolique (Ac 10,44-47; 11,15; 15,8), tout comme il ne cesse de l'accompagner et de la diriger de diverses manières (Ac 4,8; 5,32; 8,26.29.39; 9,31 ; 10 ; 11,24.28;
13,2.4.9; 16,6-7; 20,22-23; 21,11, etc.)» (AG, 4).
On peut négliger ces citations et renvois, les considérant comme de simples appuis. Mais on peut aussi, à l'inverse - et selon les consignes de la constitution sur la révélation Dei verhum (DV, 24) -, comprendre l'œuvre lucanienne, en référence critique à d'autres théologies néotestamentaires, comme matrice d'un processus générateur <l'Église et ensuite, à partir d'Ad gentes, lire la constitution sur l'Église sous ce nouvel angle
de vue.


LIRE LES ACTES DES APÔTRES
COMME RÉCIT D'UNE GENÈSE D'ÉGLISE

Suite à ces quelques indications, il peut être utile, et faire partie des « manières de procéder» dont il a été question dans le chapitre précédent, de lire à plusieurs les Actes des Apôtres, surtout en période synodale1• Ce texte nous propose en effet w1e

1.	On peut se référer ici au commentaire de D. Marguerat, Les Actes des apôtres (1-12) et Les Actes des Apôtres (13-28),Genève, Labor et Fides, 2007 et 2015. Cf aussi C. Theobald, Présences d'Évangile II, op.cit., p. 61-145; on peul lire cet ouvrage comme un petit « manuel» de mission sur nos territoires.

436
 
ÉTAPES D'UNE ECCLÉSIOGENÈSE

véritable vision ecclésiogénétique de l'Église et nous autorise à imaginer aujourd'hui des itinéraires analogues ; ce qui, comme nous venons de le voir, nous est également suggéré par le décret missionnaire Ad gentes.

L'ÉGLISE DU NOUVEAU TESTAMENT

Quand nous parlons de «l'Église» nous nous référons spontanément à une entité parfaitement constituée, sans imaginer l'époque où elle n'existait pas encore ou était en train de naître. Certes, le tout premier écrivain du Nouveau Testament, l'apôtre Paul, dispose déjà d'un vocabulaire ecclésial très précis et d'une véritable « ecclésiologie » comme on dit aujourd'hui. Pensons simplement aux adresses de ses lettres : « Paul, Silvain et Timothée à l'Église des Thessaloniciens qui est en Dieu le Père et dans le Seigneur Jésus Christ» (1 Th 1,1) ou encore: « Paul, appelé apôtre du Christ Jésus, par la volonté de Dieu, et Sosthène le frère, à l'Église de Dieu qui est à Corinthe» (1 Co 1-2). Et l'un des derniers écrits de nos Écritures, !'Apocalypse, débute avec sept lettres, chacune adressée à l'une des sept Églises d'Asie mineure.
Mais a-t-on réalisé que ni l'Évangile de Marc, ni celui de Luc, ni encore celui de Jean ne connaissent le mot « Église » ? Matthieu se contente de deux occurrences, devenues d'ailleurs célèbres : la première dans l'échange de Jésus avec Simon
- « Tu es Pierre et sur cette pierre je bâtirai (au futur) mon Église» (Mt 16,18) - ; la seconde dans la règle de correction fraternelle - « S'il refuse d'écouter (une ou deux autres personnes que toi), dis-le à l'Église, et s'il refuse d'écouter l'Église,

437
 
URGENCES PASTORALES DU MOMENT PRÉSENT

qu'il soit pour toi comme le païen et le collecteur d'impôts »
(Mt 18,17).
Peut-être ce quasi-silence des Évangiles1 signifie-t-il que le
« site » ecclésial où se vit concrètement la vie chrétienne naissante se constitue d'abord sur des chemins, dans des maisons et dans un pays, comme réseau de relations entre personnes (disciples, apôtres, sympathisants, etc.), avant de représenter aussi une réalité sociale, un« corps)> dira Paul, le « corps même du Christ» (1 Co 12). Nos historiens ont montré que cette genèse a grandement bénéficié de structures « associatives », florissantes dans l'Empire romain2, le signe distinctif étant sans aucun doute une mixité sociale, fondée sur le baptême, inexistante par ailleurs :
« Oui, vous tous, lisons-nous chez Paul, qui avez été baptisés en Christ, vous avez revêtu Christ. Il n'y a plus ni Juif, ni Grec; il n'y a plus ni esclave, ni homme libre ; il n'y a plus l'homme et la femme; car tous, vous n'êtes qu'un en Jésus Christ. Et si vous appartenez au Christ, c'est donc que vous êtes la descendance d'Abraham; selon la promesse, vous êtes héritiers» (Ga 3,27-29).




1.	Nous parlons de quasi-silence car l'analyse historique parvient à détecter av une plus ou moins grande probabilité derrière certains épisodes évangehques (par exemple Le 10,38-42) des situations de l':E.glise primitive.
2.	Cf M.-F. Baslez, « La diffusion du christianisme aux r"-m• siècles.
L':E.glise des réseaux>>, RSR 101/4 (2013), p. 549-576.

438
 
ÉTAPES D'UNE ECCLÉSIOGENÈSE


UN EXERCICE DE LECTURE
L'œuvre de Luc confirme cette vision et la pousse jusqu'au bout : absent dans l'Évangile, le mot « Église » intervient avec force dans les Actes des Apôtres, avec pas moins de vingt-trois occurrences; ce qui, nous allons le voir, confirme la perspective génétique qui est en train de se dessiner. Je n'alignerai pas ici toutes les citations où figure le terme « Église » ; les groupes de lecteurs intéressés pourront, pour cet exercice, se référer à un tableau qui se trouve dans un volume antérieur1•
Avant qu'il ne soit question, pour la première fois,
d'« Église)) (Ac 5,11), les Actes mettent les lecteurs en présence de quelques-uns, cent vingt personnes environ avec les Onze, quelques femmes, dont Marie, et les frères de Jésus (Ac 1,13-15), qui se réunissent dans des «maisons» et fréquentent le temple (Ac 2,46). La désignation « Église>) n'apparaît donc que très progressivement et se charge de connotations toujours plus riches, au fur et à mesure que l'expérience ecclésiale se précise et se propage. Par deux fois, Luc rappelle incidemment l'origine biblique et politique du mot: « l'assemblée >> du désert révoltée contre Moïse (Ac 7,38) et « l'assemblée >> d'Ëphèse réunie au théâtre. Il accompagne cette dernière mention d'une note pleine d'humour, tout en soulignant la forme légale de ce genre d' « assemblée » (Ac 19,39.40) : « Chacun bien sûr criait autre chose que son voisin, et la confusion régnait dans l'assemblée où la plupart ignoraient même les motifs de la réunion>> (Ac 19,32) ;

1.	Présences d'Évangile II, op. cit., p. 71-73.

439
 
URGENCES PASTORALES DU MOMENT PRÉSENT

comme s'il voulait nous rappeler que le mot « Église >> est d'abord à prendre en son sens le plus élémentaire et avec ce!> connotations critiques.
Retenons les moments les plus significatifs de la mise en
série du vocabulaire ecclésial. Nommée pour la première fois en lien avec un «événement» -	une affaire d'argent (Ananiai. et Saphira) qui concerne toute la communauté et a des répercussions en dehors de celle-ci (Ac 5,11) -	et en lien avec la persécution lancée par « Saul qui pénètre dans les maisons» (Ac 8,1.3), «l'Église» émerge comme « sujet» qui grandit à l'image d'un enfant : « L'Église, sur toute l'étendue de la Judée, de la Galilée, et de la Samarie, vivait donc en paix, elle s'édifiait et marchait dans )a crainte du Seigneur et, grâce à l'appui du Saint-Esprit, elle s'accroissait» (Ac 9,31).
Or, cette E.glise est liée à un lieu. et à une terre ou une région:
ayant d'abord parlé des maisons et de l'Église de manière générale, Luc introduit rapidement l'expression « Église de Jérusalem >1 (Ac 8,1), désignation qu'il universalise ensuite quand, en parlant d'Antioche, il utilise le terme technique « E.glise du lieu>> (Ac 13,1). Progressivement on voit apparaître l'autonomie de ces :Bglises, pourvues d' « anciens » (presbuteroi) désignés par leur fondateur (Ac 14,23) ou d'autres référents comme des « prophètes » et des « hommes chargés del'enseignement » (Ac 13,l). Le lieu est décisif, parce qu'il marque de ses conditions culturelles et économiques la vie quotidienne des Églises : gestion de la pluralité des langues (Ac 2,5-13) et de la différence financiere entre membres (Ac 2,44 sqq.). On voit réapparaître ici le versant messianique de la tradition chrétienne, dont il a été question

440
 
ÉTAPES D'UNE ECCLÉSIOGENÈSE

à plusieurs reprises'. Des conflits ne manqueront donc pas de surgir, portant sur le partage des biens et surtout sur l'interprétation de certains événements - viennent-ils de Dieu ? - et sur l'interprétation des Écritures. L':Bglise de Jérusalem a déjà développé des oreilles : « La nouvelle de cet événement (il s'agit de la conversion de grecs à Antioche) parvint aux oreilles de l'Église qui était à Jérusalem» (Ac 11,22). Des visites mutuelles et l'envoi de délégués deviennent nécessaires pour régler le différend: « l'Église d'Antioche pourvoit à leur voyage» (Ac 15,3). Le récit est alors le moyen privilégié pour communiquer ce que Dieu fait de neuf parmi nous : « À leur arrivée, ils réunirent l'Église et raèontaient tout ce que Dieu avait réalisé avec eux et surtout comment il avait ouvert aux païens la porte de la foi» (Ac 14,27). Le conflit d'interprétation - faut-il, oui ou non, circoncire les païens convertis ?- se résout dans une assemblée qui passe du récit à la délibération et de la délibération à la décision:« D'accord avec toute l'Église, les apôtres et les anciens
décidèrent alors de choisir dans leurs rangs des délégués qu'ils enverraient à Antioche avec Pau·let Barnabas» (Ac 15,222).
Cette phase de maturation se termine quand Luc rapporte brièvement que Paul parcourt la Syrie et la Cilicie pour « affermir les Églises » : « Les Églises devenaient plus fortes et croissaient en nombre de jour en jour» (Ac 15,41 et 5,15). C'est Paul, dans son testament aux anciens d'Éphèse, qui conduit le devenir de l'Église jusqu'au bout en y introduisant le vocabulaire pastoral et en rapportant à Dieu toute la genèse ecclésiale, et cela en fin
 	
1.	Cf. plus haut, p. 345 sqq., 353 sqq. et 389-392.
2.	Cf pl.us haut, p. 326, 407 sqq. et 412 sqq.

4Al
 
URGENCES PASTORALES DU MOMENT PRÉSENT

de parcours:« Prenez soin de vous-mêmes, dit-il a  ancien , et de toutle troupeau dont l'Esprit Saint vous a é blis_les gard1_ens (episcopoi)>soyez les bergers (poimainein) de 1 f.gltse deDieu, qu'il s'est acquise par son propre sang» (Ac 20,28).

QUE RETENIR DE CE PÉRIPLE ?
1.	D'abord l'expérience d'une Égl:ise en genèse, expé ie ce ,qui peut encore devenir la nôtre : le récit de Luc ab?ut1t 1 ou la constitution sur l'Église Lumen gentium du concile Vat:I an I débute, à savoir au « mystère de l'Église » dans 1  d ssern tri-
m.ta1•red e D'1eu (LG, J... chap1·tre • Le mystère de l Eghse). Il est
donc possible de lire la constitution en s ns inverse-  p ur ainsi dire« d'en bas»-,précisément à partir dela« perspective de fondation», telle qu'on la trouve dans les Actes et dans le
décret missionnaire.	.  . . Notons cependant que l'introduction d'une dimension his-
torique dans Lumen gentium en 1963 - avec le chap.2 sur l.e peuple de Dieu et déjà dans le n°S du premier chapitre-. rédwt la distance entre ces textes. Dans ce no 5, la « perspect ve fo dationnelle » (fundatio) ou génétique est po r- l.a prennè e f 1s explicitée : elle re\ie le «commencement» (mitiu.m) de1 Égli e
dans le message de Jésus sur le Royaume de Dieu (LG, ,s§.  )
au «	commencement >> du	Royaume de Dieu dans l Église (LG, 5 § 2). Cette même pers ective apparaît é a ement d s un paragraphe du n°26 snr les Eglises locales et, sil on poursuit cette piste, aussi dans le n° 19 sur le collège des apôtres et le no24 sur la « diaconie » des apôtres et de lems successeurs. Ce sontsurtout les n""24 et 26 de Lumen gentium qui adoptent la
 
ÉTAPES D'UNE ECCLÉSIOGENÈSE

perspecti.ve narrati.ve des Actes des Apôtres, explicitement documentée dans les renvois scripturaires. Heureusement certains Pères conciliaires n'ont pas oublié la petitesse des débuts et en ont laissé quelques traces dans le texte ; aussi lit-on maintenant dans la constitution cette petite phrase, introduite à la dernière ute:« Dans ces communautés, si petites et pauvres qu'elles puisse-nt être souvent ou dispersées, le Christ est présent par la vert duquel se constitue(consociatur) l'Église une, sainte, catholique et apostolique» (LG, 26). C'est précisément ce que nous pouvons percevoir aujourd'hui dans beaucoup de régions, en France et ei:i Europe, et ce qui représente le point de départ de
notre lecture des Actes avec l'œil exercé par la lecture du décret missionnaire de Vatican II.
2.	L'approche de type lucanien des f:glises locales soulève cependant la question de savoir comment relier la perspective lucanienne et synoptique défendue dans Lumen gentium, 5, qui situe le « commencement» de l'Église dans le message de Jésus sur le Royaume de Dieu, et l'ecdésiologie paulinienne et deutéro-paulinienne, et surtout la doctrine sur les charismes introduite aux n"' 4 et 7 de Lumen gentiunt1. Comme nous l'avons déjà évoqué, le n<>4 d'Ad gentes, influencé par Luc, se réfère simultanément à la doctrine paulinienne des charismes de
Lumen gentium 4 et 7 et établit ainsi une connexion, du moins sous forme d'une esquisse2.
Or, le message de Jésus sur le Royaume de Dieu ne peut être séparé de ses signes et gestes messianiques dont l'effet libérateur

l. Cf plus haut, p. 312 sqq.
2. Cf. plus haut, p. 435 sqq.

443
 
URGENCES PASTORALES DU MOMENT PRÉSENT

est explicitement signalé en Lumen gentium, 5 (Le 11,20). Da_ns la ligne d'Isaïe, ces signes du commencement du emps messianique sont des personnes vivantes : les pauvres qm entendent la Bonne Nouvelle; les prisonniers qui sont libérés ; les aveugles qui voient (Le 4,17-21-7,21-23), etc.; Luc en établit une liste pour la compléter ensuite dans les Actes des Apôtres1. a?s la doctrine des charismes de Paul, plutôt axée sur la vie mtérieure de l'Église, il s'agit également de perso_nnes vi; ntes, e seulement dans un deuxième temps de fonctions prec1ses qm peuvent être comprises et reçues comme des modalités d'apparition de la grâce; cette importante nuance est retenue dans Lumen gentium, 7 à l'aide du concept charismaticus2• Cependant, selon Paul, les « membres du corps que nous tenons pour les plus faibles» et« ceux que nous tenons pour_ les moi?s ?norables » (1 Co 12,22 sqq.) participent de mamère part1cuhere à la construction du Corps du Christ.
Ici apparaît le point de convergence messianique t pneu­
matologique décisif entre la vision synoptique luc menne et l'approche paulinienne. L'idée fondamentale del 1Éghse comme
« sacrement universel du salut », bien présente dans Lumen gentium (LG, 1, 9 et 45) et reprise à nouveau dans Ad e es (AG, 1 et 5), acquiert en cet endroit n n uvell pla s_1 1hté critique3 : les signes messianiques, qm s a;ere t intfrevisibles, et les charismes, donnés hic et mmc par 1Espnt, depassent la

I. Cf- plus haut, p. 345 sqq.
2.	Cf. plus haut, p. 312, et la distinction de P. Congar entre « sacrements- choses » et« sacrements-personnes», p. 352 sqq.
3.	Cf plus haut, p. 343 sqq.

444
 
ÉTAPES D'UNE ECCLÉSIOGENÈSE

sphère classique des sept sacrements et dépassent même en un certai? s ns, l'Église. Signes messianiques et charismes euvent ê!re reums dans le concept biblique de mysterion dont la dirnens10n corporelle et symbolique est ressaisie de manière précise dans sa traduction par sacramentum. Cela nous conduità faire valoir, avec Ad gentes (4), la dimension événementielle et historique du mystère, face à une ritualisation unilatérale, età faire passer notre attention première des gestes significatifs aux personnes elles-mêmes et à leur sollicitude mutuelle (1 Co 12,24 sqq.) comme étant les signes messianiques par excellence.
3. Ce e insistance sur des « personnes significatives» nous recondmt vers le récit lucanien et vers ce qu'il nous apprend, dans une même perspective génétique ou inductive, de ces personnes et de leur relation avec l'Église naissante. Selon l'ordre génétique, elles précèdent l'émergence des groupes ou com, una tés sur place; ce qui est d'ailleurs enregistré par la des1gnat1, n ?u de_uxième livre de Luc qui ne s'intitule pas
« Actes del Eghse naissante » mais, depuis saint Irénée« Actes des_ foAtres ». C mme il a été signalé à plusieurs rep' rises, le
trots1eme Évangile nous offre la phase constitutive de ce jeu
re ationnel entre Jésus et quelques-uns, avec l'appel des premiers disciples, l'institution des Douze et le début d'une étonnante prolifération au moment de l'envoi des Soixante-douze. Cette prolifération continue à œuvrer dans le second livre où s'ajouten_t le  ept don l'un, _l'itinér nt Philippe, jouera un rô e particulier. On v01t enswte paraitre, au sein des Églises naissantes, des« prophètes »  personnes qui, pour faire bref,

l. Sur ce principe dei< prolifération», cf. plus haut, p. 31I-316.

445
 
URGENCES PASTORALES DU MOMENT PRÉSENT

parlent sous l'influence de l'Esprit - et des « hommes chargés de l'enseignement» (Ac 13,1 et 1 Co 12,28) jusqu'à ce qu'émergent « les anciens» (presbuteroi), établis dans chacune des E.glises fondées par Paul et Barnabas. S'il y a, au point de départ, une structure ferme, la relation entre Jésus et les Douze, d'autres figures s'inventent au gré des circonstances et au fur et à mesure que le nombre des croyants augmente.
Ces développements résonnent évidemment avec notre actualité pastorale, surtout aux endroits où le tissu communautaire est très distendu, voire inexistant. Il faut que nous nous en rappelions au moment où nous entreprenons de tracer un chemin type de genèse ecclésiale supposant à la fois la vision esquissée dans le chap. 7 de notre parcours et la manière de procéder explicitée dans le chap. 8.


ET AUJOURD'HUI ?

Derrière notre diagnostic d'une « exculturation >> toujours plus grave de l'Église d'Europe (surtout occidentale), qui a transformé la France et d'autres régions en pays de mission, se tient la conviction théologique selon laquelle l'avenir ne peut être abordé qu'à partir d'un rapport créatif avec les « origines>> du christianisme. D'où notre tentative de souligner la« perspective fondationnelle » et « génétique » dans les textes conciliaires et la fonction à la fois critique et inspiratrice des Écritures. Il nous reste, pour finir, à distinguer quelques «seuils» ou étapes d'une telle genèse dans notre situation actuelle, sans reprendre tout ce qui a été déjà apporté et fondé dans le chap. 7. Nous
 
ÉTAPES D'UNE ECCLÉSIOGENÈSE

visons plutôt la simplicité et l'utilité, car notre but est d'aider telle communauté concrète à discerner le ii point » où elle en est sur le hemin qui pourrait la conduire d'une << pastorale de reproduction » vers une « pastorale missionnaire ».

1.	LA GENÈSE D'ÉGLISE
DANS UN << ESPACE HOSPITALIER »

L'Église naît et renaît là où la fois'engendre. Entendons-nous bien :« foi»  ne désigne pas immédiatement foi en Dieu ou
e? C rist, m s d' b r et vant tout la capacité mystérieuse d un etre à fazre credtt a la vie, à rester debout, même dans les moments les pl s difficiles, en espérant que la vie tient sa promes e, perspective longuement développée dans la deuxième
p rt1e de cet ouvrage. Personne ne peut poser cet acte à la place d	autre. Pourtant cette foi s'engendre; si fragile et cachée so1t.-elle, elle peut être ranimée par ceux qui la perçoivent eyt cr en . ela e pe t se faire que dans un espace hospitalier, qu ils aglsse d un heu ecclésial, d'une maison privée ou d'un
d s multiples espaces de fortune que nous habitons à l'improVIste, par exemple lors d'une conversation avec autrui, etc. Le deuxième chapitre d'Ad gentes parle du «témoignage)> ou de la
« présence» des chrétiens dans un milieu non chrétien comme point de départ d'une genèse <l'Église sur place et utilise ici les co cepts de« conversation » (conversatio),d'«entretien» (colloquium) et de «dialogue» (dialogus) pour décrire ces situations de« présence» où de l'inattendu peut se produire.
Il arrive alors que ceux qui ont bénéficié d'une telle présence mettent ces« croyants-chrétiens >> en position de {{ témoins».
 

446	447
 
URGENCES PASTORALES DU MOMENT PRÉSENT	ÉTAPES D'UNE ECCLÉSIOG ENÈSE

 
Ils les interrogent et leur donnent ainsi l'occasion de révéler comment ils ont été eux-mêmes engendrés à la foi par d'autres et comment ces « relais » les ont mis en relation avec le Christ Jésus. L'Église naît en ces rencontres significatives où l'intérêt gratuit pour la« foi» d'autrui ouvre en même temps un espace où celui-ci peut découvrir le Christ. C'est sur ce «seuil» fondamental qu'est située« l'annonce de l'Évangile»,« partout où Dieu ouvre une porte à la prédication », selon la belle formule du décret missionnaire (AG, 13, 1).
Une communauté chrétienne (ou son conseil pastoral) peut
alors s'interroger sur son esprit d'hospitalité, non seulement sur un plan collectif mais aussi du côté de ses membres, non seulement en termes d'accueil mais aussi sur l'intérêt désintéressé qu'elle porte à son environnement social, au mal-être de telle personne ou de tel groupe, bref aux lieux où la (( foi » en la vie est en jeu. Cette interrogation peut être un tout premier pas vers une « pastorale missionnaire».

2.	LA RELATION À L'ÉCRITURE SAINTE
Un nouveau « seuil» est franchi quand !'Ecriture entre dans le champ visuel de ceux qui se côtoient déjà par ailleurs ; ce qui ne va nullement de soi dans notre société médiatique. Il faut d'ailleurs distinguer ici très clairement entre la Bible comme classique de-la culture européenne et expression d'une certaine humanité et l'Écriture sainte comme livre de l'Église. Cette différenciation se répand de plus en plus dans nos sociétés laïques
 
entrer en jeu le critère distinctif de la pratique religieuse1. La lecture et l'étude communes de ce livre permettent aux chrétiens d'identifier, au contact avec d'autres chrétiens, la genèse de leur propre foi spécifiquement messianique, de faire peut-être l'expérience du lien intime entre l'écoute de l'Évangile et de son annonce et de comprendre comment l'Église a trouvé, pas à pas, sa forme et continue à la trouver aujourd'hui.
Parfois des événements, vécus par les uns ou les autres entrent dansl' hange, ouvrent vers d'autres dimensions psychologiques ou politiques de la lecture en commun et conduisent à tisser des lien plus approfondis _entre les membres d'un groupe. Il se peut aussi quela confrontation aux textes bibliques et à leur humanité crée subitement un climat de recueillement où le silence vécu en commun se fait méditation pour les uns et prière pour les autres.
Là enco e, une commun uté chrétienne peut s'interroger sur la place quelle donne aux Ecritures en son sein, non seulement dans sa liturgie, mais aussi à d'autres occasions, en d'autres lieux ou da s ses groupes et petites communautés plus restreintes. Elle doit se e ander si,,dans les faits, elle croit réellement que c: sont es Ecn u.res (quelle confesse comme saintes ou inspiees) q 1 vont 1 aider à trouver son propre chemin pastoral en mteract1on avec ce qui arrive dans nos sociétés.
 
et permet de réunir autour d'une même table des chrétiens, des		
 
sympathisants et des non-chrétiens, sans immédiatement faire
 
l. Cf. plus haut, p. 380-383.
 

448	449
 
URGENCES PASTORALES DU MOMENT PRÉSENT


3.	LA DÉCOUVERTE DE PERSONNES NOUVELLES ET DE LEURS CHARISMES
Un autre« seuil» est passé quand, dans cet ensemble (< multîtudinariste » de chrétiens, sympathisants et autres, se profilent des personnes, parfois des personnes nouvelles, orteus_es d.e
<<charismes» spécifiques. C'est une double attention qm ,d01t alors émerger ici : une attention aux: personnes, aux. dons qu elles ont reçus et au don qu'elles représent nt pour la c mmunauté et une attention renouvelée aux besoins de celle-ci n ter es defonctions élémentaires à assurer. Ces deux types d attention sont difficiles à articuler, car une focalisation trop forte sur les besoins risque d'occulter la perception de ce qui est eff ct vement donné par Dieu à telle communauté. On p urra1t .etre tenté de se servir immédiatement du schéma des trms fonctions classiques du Christ et de l'Église -	les fonctions sacerdotale, prophétique et royale -	comme critère de discernement • Cela risque d'être prématuré et on passerait sans doute à c,ôté d s charismes les plus importants au cours de cette étape, a savoir celui de la« visite» et celui appelé plus haut« charisme de sourcier ou détecteur de chercheurs de sens2 ». Or, si ces char smes sont déjà en place, on trouvera rapidement leur pend nt mtraecclésial, si je puis m'exprimer ainsi: à sav ir le <, cha:1sme» d: l'accompagnateur de catéchumènes et celm de catéchete c arge de l'initiation chrétienne, comme il est dit dans le demaème article du décret missionnaire Ad gentes (AG, 14).

1.	Cf. plus haut, p. 189-191.
2.	Cf. plus haut, p. 317 sqq.

4S0
 
ÉTAPES D'UNE ECCLÉSIOGENÈSE

On perçoit bien que cette troisième étape représente un véritable carrefour, délicat à passer, sur le chemin d'une pastorale de reproduction vers une pastorale missionnaire. Elle exige en effet une sensibilité pastorale plus affinée par rapport à ce qui se passe en profondeur à la « porte baptismale» de la communauté chrétienne : nous l'avons bien vu quand il a été question des demandes de baptêmes'. Conditionné par le rayonnement hospitalier de l'Église (1re étape), l'intérêt que d'autres lui portent (mouvement centripète) dépend aussi de sa «présence» auprès d'eux (mouvement centrifuge) et de sa capacité de prendre au sérieux celles et ceux qui frappent effectivement à sa porte. Mais cette sensibilité missionnaire ne peut pas devenir réalité si la communauté n'apprend pas en même temps à mettre en lien, sans aucun forcing, les personnes (éventuellement nouvelles) que Dieu lui donne et ses propres besoins en termes missionnaires. Ce n'est pas en effet le prêtre seul qui peut être porteur d'une telle sensibilité pastorale et du discernement que celle-ci exige. D'autres doivent prendre le relais.

4.	UNE COMMUNAUTÉ QUI SE MET À DÉLIBÉRER

C'est précisément à ce moment que peut s'engager une quatrième étape de l'ecclésiogenèse missionnaire traitée dans le troisième article du décret Ad gentes sous le titre de « la formation de la communauté chrétienne (De communitate efformanda) » (AG, 15 et la suite), appelée aussi plus haut « devenir sujet

1. Cf. plus haut, p. 345 et 348 sqq.

4S1
 
URGENCES PASTORALES OU MOMENT PRÉSENT

missionnaire » de l'Église'. La difficulté des trois premières étapes réside en effet dans l'énorme différence des prises de conscience au sein de nos communautés. Parfois seules certaines personnes
-	parmi elles, le prêtre -	ont expérimenté le lien entre l'écoute de l'Évangile et la nécessité intérieure de l'annoncer2 et sont donc conscientes des enjeux missionnaires du christianisme et de leurs conditions de réalisation; parfois c'est le conseil pastoral qui porte cette conscience, plus rarement oute, une communa :é. La perception de ces différences et de bien d autres ncore (he:s à des options culturelles, politiques, anthropologiques,, cclesiales, etc.) conduit vers le désir qu'une communauté chretlenne soit d'abord un lieu où on écoute l'autre, mais aussi où l'on attend sa « participation active» (SC, 11), non seulement dans la liturgie, mais également et plus fondamentalement dans 1 domaine du témoignage et de la simple présence auprès d'autrm, chacun vivant cette exigence missionnaire selon sa mesure. Et puisque rien ne peut être forcé, tout dépendant du travail de l'Esprit saint dans les consciences et entre elles, seuls l'art de la délibération et la synodalité dont il a été longuement question-'\ sont susceptibles de créer un« sentir » commun et une véritable conscience communautaire qui transforme une Église locale en véritable « sujet».
On peut espérer qu'une communauté qui a commencé à compter sur tel ou tel charisme et sur des personn:s ayant pris conscience de ce qui se passe à la « porte baptismale »

1.	Cf plus haut, p. 318-328.
2.	Cf plus haut, chap. 4.
3.	Cf plus haut, p. 325-327 et p. 406-416.

4S2
 
ÉTAPES D'UNE ECCLÉSIOGENÈSE

de l'Église (3e étape) passe aussi ce nouveau seuil qui consiste à faire participer un maximum de ses membres à ses orientations d'avenir, celles-ci ne reposant pas sur la présence du prêtre (toujours accueilli comme nécessaire garant apostolique de l'annonce évangélique), mais sur une véritable conscience partagée de la communauté dans son incarnation locale. Ce
point mérite, comme les précédents et ceux qui suivent, un examen régulier.

5.	LA PERCEPTION
DES DIMENSIONS CORPORELLES DE LA FOI

Un« seuil » d'un autre ordre est passé encore, quand 1a dimension corporelle de la foi est davantage perçue : c'est ici qu'interviennent la sacramentalité de l'Église au sens défini plus haut' et les signes sacramentels, le baptême en premier lieu, le repas du Seigneur, etc., les services pastoraux et le pastorat.
Si l'on a fait l'expérience de la naissance de l'Église dans nos rencontres et relations quotidiennes (1re étape), alors on comprend aussi le caractère relationnel des sacrements2• Ce sont toujours des personnes qui posent des signes et sont ou deviennent ainsi des signes : celui qui est baptisé, mais aussi celui qui est chargé d'un ministère. Le Repas du Seigneur mène à son accomplissement cette « transition » du « poser un signe » à « être un signe» et implique les croyants dans le processus relationnel initié par Jésus au sein de la société galiléenne, processus qui trouve son ultime crédibilité en son don de soi à« quiconque».

1.	Cf. plus haut, p. 343 sqq.
2.	Cf plus haut, p. 352 sqq.


453
 
URGENCES PASTORALES DU MOMENT PRÉSENT

C'est par cette voie aussi que le« souci pastoral» pour l'avenir de l'Évangile et de l'Église peut naître ez  quelques-uns et que l'appel à donner forme à ce « souci» dans sa pro?re existence peut être entendu. La vocation, l'institution collégiale et l'envoi en mission des Douze ainsi que de leurs successeurs peuvent tout à fait être compris dans le cadre_ de la _structure relationnelle et événementielle du Règne de Dieu qm f nde la sacramentalité de l'Église. La question, qui tour ente ?.1en _des chrétiens aujourd'hui à propos du «pourquoi» de 1mstl tion ecclésiale, ne peut plus trouver de réponse p r le b1a s d'une argumentation purement sociologique ou juridique, mais nécessite une justification théologique « simple), que seul pe t offrir le concept de « mission ,, déjà impliqué dans l«' Évangile
de Dieu1 ».
Or, beaucoup de chrétiens ne perçoivent plus que le v,ers t
rituel ou festif du régime liturgique et sacramentel de 1Église dont il a été longuement question au chap. 72 . C' est alors qu'il faut rendre réellement - ce qui veut dire corporellemen-t
perceptibles le lien de nos rituels avec _les.gestes év n?éliqu_es "'.-1 Christ Jésus et leur signification mess1amque et m1sswnmur,e il en va du rayonnement même de l'Évangil: qui f t appelà	os sens et à l'émotion. Certes, la beauté des celébrat1ons et de 1 espace liturgique y est pour beaucoup ; mais l'art et le senti11:ent esthétique qu'il est capable de susciter restent des phénomenes fugitifs tant qu'ils ne parviennent pas à toucher le fond es cœurs humains, cet endroit mystérieux où chacun peut farre
 
ÉTAPES D'UNE ECCLÉSIOGENÈSE

l'heureuse expérience de l'intimité de Dieu et de la proximité du prochain, et peut puiser l'énergie de la sortie de soi.
Il ne suffit pas de se décharger du soin de la corporéité de la foi sur des équipes liturgiques capables d'appliquer quelques bonnes recettes. C'est un signe de maturité que pose une communauté, quand, tout entière, elle prend conscience de cet enjeu et parvient à examiner paisiblement sa situation sur ce point, sans absolutiser des différences légitimes de sensibilité. Cela ne va pas de soi, reconnaissons-le, car même sur un plan plus large, l'fglise de France n'a pas réussi jusqu'à maintenant à engager une délibération collective sur cette pratique hautement sensible1•

6.	LA PRISE EN COMPTE DE LA DIMENSION TOUJOURS PLUS UNIVERSELLE DE L'ÉGLISE
Un autre « seuil» est encore à franchir dès lors qu'une communauté, « si petite et pauvre soit-elle» (LG, 26), perçoit que la « fraternité » chrétienne dépasse toutes les frontières d'espace et de temps et qu'elle éprouve alors le désir d'un échange plus intense avec d'autres groupes et communautés. L'hospitalité prend alors figure, des visites mutuelles ont lieu, les engagements dans la société s'affermissent: la communauté devient sujet de ses actes et signe sacramentel d'une unité toujours plus ample
-		« catholique ». Simultanément,on voit naître un rapport nouveau à la tradition, capable de dépasser des oppositions stériles et sans cesse rejouées entre progressisme et traditionalisme : la
 

1.	Cf. plus haut, p. 327 sqq.
2.	Cf plus haut, p. 351-367,	1. Cf plus haut, p. 189 sqq.

454	455
 
URGENCES PASTORALES OU MOMENT PRÉSENT

gratitude envers les anciens qui nous ont communiqué leur foi à travers des écrits, des monuments et des institutions de toutes sortes va de pair avec la liberté à leur égard et le souci de s'inscrire de manière créatrice dans la trame qu'ils nous ont laissée.

7.	CONTEMPLATION
La genèse de l':Sglise s'achève -	provisoirement-	quand une communauté passe le « seuil » dela contemplation.A vrai dire, elle rejoint alors consciemment l'impulsion initiale du parcours que nousvenons d'accomplir. La << moisson >} est abondante pour ceux qui savent la voir : elle comprend non seulement la fécondité de
la foi des chrétiens, mais surtout le « simple faire crédit à la vie» que perçoivent et ravivent ceux qui sont proches d'autrui. Or, être
« témoin >> de ce qui se passe en quelqu,un ou dans les profondeurs de nos sociétés peut susciter l'action de grâce et la supplication, parfois seulement un gémissement ou l'adoration••• Dans ces actes de prière, l'Église se dessaisit de ce qu'elle reçoit et découvre qu'au sein de l'humanité l'Esprit est en train de construire un
«temple» qui n'est pas fait de mains d'hommes; en admirant ce travail de l'Esprit elle devient « corps du Christ » et reconnaît que Dieu est l'origine abyssale d'un « peuple>>aux dimensions mystérieuses et en attente d'une paix universelle (LG, 17). Notre récit rejoint ici le début du texte de la Constitution.

REGARD SUR LE CHEMIN PARCOURU
On aura compris que le «chemin» retracé ici permet une multiplicité de variantes et ne doit donc pas être figé en un schéma linéaire. Il s'agit plutôt d'une« carte routière» qui peut

456
 
ÊTAPES D'UNE ECCLÉSIOGENÈSE

s'avérer utile quand la pratique pastorale est menacée par l'illisibilité et la discontinuité postmodernes et quand il faut aider les communautés à déchiffrer leurs propres itinéraires. Bien d'autres étapes peuvent devenir importantes et s'inscrire dans les interstices du schéma élémentaire qu'on vient de présenter. N'oublions pas non plus que telle communauté peut connaître des phases de régression, voire disparaître, tandis qu'à d'autres endroits des foyers de vie chrétienne se développent et prennent une figure communautaire.
Le décret missionnaire Ad gentes et les Actes des Apôtres partent évidemment de la situation plus radicale de pays, voire de tout un Empire, qui ne connaissent encore ni l'Évangile de Dieu, ni t>Église. Dans nos régions devenues des pays de mission, ce n'est pas le cas. La conception génétique de l'Église est alors une manière de sortir de la simple reproduction - qui serait devenue inconsciente des enjeux de l'Évangile - pour reparcourir patiemment le chemin du « devenir ecclésial», en y intégrant dès le départ l'élément essentiel qu'est la mission, et donc le devenir « sujet missionnaire » de ]'Église. Le fruit d'un tel parcours est alors une« ré-forme» qui s'enracine dans une expérience évangélique et s'appuie sur des prises de conscience successives, susceptibles de donner progressivement à notre vécu ecclésial une forme véritablement spirituelle ou expérientielle. Le langage du<<seuil» et del'« étape», utilisé dans le parcours
«narratif» qu'on vient d'effectuer, marque à l'évidence que
l'essentiel de l'attitude pastorale consiste à se rendre et à rester sensible aux « événements qui se produisent parmi nous » (cf. Le 1,1), la plupart du temps à l'improviste.

4S7
 
URGENCES PASTORALES DU MOMENT PRÉSENT



CONCLUSION

Rassemblons, une fois encore, quelques acquis.
1.	On peut s'étonner, et m_ême exprimer des doute , devant la perspective génétique de l'Eglise que nous veno s d e;4'lorer dans ce dernier chapitre. Or, il faut se demander s1 cet etonn, ment ne provient pas du simple fait que nous n_ou somme deJa tellement habitués à la situation de crise des Eglises au sem de nos sociétés européennes, qu'on la perçoit comme fatalement irréversible. Dès que la conscience missionnaire émerge en s force évangélique, une conception ecclésio-génétique -	c ll -ci ou une autre -	s'impose : on est inévitablement co_ndu ta la question du chemin à emprunter pour sortir de cet,te s1tua 10_n et de l'horizon vers lequel se mettre en route ; le decret m1ss1onnaire Ad gentes comme les Actes des Apôtres confirment cette
hypothèse.	,
Avancer sur ce chemin implique le respect de ce qu on pour
rait appeler une« normativité ecclésiologique », respect g_arant dans l'ensemble de l'ouvrage et surtout dans ce chap1tre- 1 par un va-et-vient entre les Écritures et les documen:s conci­
liaires, en particulier entre Ad gentes et Lumen fent1 1:7· O_n repérera difficilement des éléments importants de1eccles1ol g1e catholique qui n'ont pas trouvé leur place (ou ne pourraient pas la trouver aisément) dans ce qui vient d'êtr: prop é. _En revanche, il nous faudra bien apprendre, au sem del  Eg_hse, à mieux distinguer tel schématisme doctrinal et normatif et l'itinéraire concret de telle Église ou communauté locale. Ce n'est pas parce que l'accent est mis, dans ce livre, sur ce vernuit

4S8
 
ÉTAPES D'UNE ECCLÉSIOGENÈSE

historique et génétique qu'on aboutit à une vision réductrice de l'Église, loin s'en faut.
Cette insistance sur le processus historique au cours duquel l'Église prend forme en s'adaptant au moment et au lieu où elle s'incarne est le fruit de la vision tripolaire qui nous a accompagnés depuis le début de notre parcours1 et qui relie inséparablement le référent ultime de la foi chrétienne, l'Évangile du Règne de Dieu, la situation historique de la société qui est son espace d'accueil éventuel et l'actuelle figure de l'Église en sa forme multiple et polyédrique (enregistrée par toutes sortes d'enquêtes). Prendre réellement en compte cette historicité de l'Église et, surtout, accepter de voir qu'il y ait des situations en demi-teinte, voire de «crise» - ce que le décret missionnaire du Concile reconnaît comme tout à fait possible (AG, 6 § 2) -, cela nous conduit inévitablement à nous interroger sur les carences et surtout sur les ressorts d'une réforme pour y remédier. Aucun doute ne subsiste sur la réponse : ces ressources se trouvent dans l'expérience missionnaire, telle qu'elle a été présentée au chap. 4 et explicitée ensuite dans les deux chapitres suivants. La question plus précise est alors celle-ci : comment et par quel chemin laisser advenir cette expérience fondamentale au sein et à partir de nos communautés existantes.
2.	A cet endroit et si la question est réellement entendue, le concile Vatican II peut nous donner de précieuses indications, à condition de dégager de son corpus textuel, non pas une vision statique, mais un ensemble processuel complexe, constitué d'une vision attirante, d'une série de méthodes ou de manières de

l. Cf plus haut, p. 54 sqq.

459
 
URGENCES PASTORALES OU MOMENT PRÉSENT	ÉTAPES D'UNE ECCLÉSIOGENÈSE

 
procéder, articulées entre elles, et d'une esquisse des étapes à parcourir en direction de ce qui est perçu à l'horizon. Cet ensemble très souple (que nous venons de solliciter dans les trois derniers chapitres de cet ouvrage) est parfaitement adapté à notre situation spirituelle où nous devons à la fois compter sur les surprises de l'Esprit de Dieu et développer en même temps des dispositifs qui nous permettent de nous disposer à sa manifestation, intérieurement et collectivement.
Les sept étapes d'un itinéraire ecclésiogénétique, que nous venons de parcourir, ont été découpées, sur la base du décret missionnaire Ad gentes et des Actes des Apôtres, de façon à respecter et à rendre même déterminant la double dynamique qui vient d'être énoncée. La naissance d'une expérience missionnaire suppose dans un premier temps qu'une communauté se laisse surprendre par ce qui se passe en telle personne ou en tel groupe, non seulement en tel catéchumène, mais bien plus largement en «quiconque», qu'elle éprouve donc concrètement d'être sans cesse « précédée » par !'Esprit. Ce n'est pas quelque chose que l'on commande ou que l'on peut produire par soi-même, mais pour percevoir cette « précédence », des dispositifs peuvent être mis en place : des espaces hospitaliers (1re étape), la lecture commune des Écritures qui ne cessent de nous confronter à la diversité quasi infinie des présences de l'Esprit dans l'histoire (2eétape) et l'attention à ce que l'Esprit donne aux personnes (3• étape) - et d'autres dispositifs qui peuvent encore être inventés. A partir d'un certa.in moment, ce qui est éprouvé par des individus doit devenir une expérience collective et un bien commun. C'est ce qui peut se passer entre la troisième et la quatrième étape et
 
déterminer toute la suite de l'itinéraire. De nouveau, des dispositifs doivent aider à percevoir la communauté tout entière, et non seulement tel individu, comme « sujet missionnaire » : la délibération et la découverte de l'œuvre de l'Esprit dans l'inattendue entente mutuelle (4c étape), l'expérience d'une surprise spirituelle qui devient possible, quand on prend soin de la corporéité de nos célébrations et de l'Église (Se étape) et la découverte du travail de l'Esprit de Pentecôte dans une Église qui ne cesse de nous surprendre par ses dimensions à la fois universelles et polyédriques. C'est la contemplation sous ses multiples formes (7e étape) qui achève l'itinéraire. Elle rend aussi intérieurement évident que cet itinéraire ne s'achève jamais, tant que l'histoire dure, et qu'il faut reparcourir sans cesse ces différentes étapes - ou d'autres - avec toujours plus de profondeur.
3.	Ce qui vient d'être proposé est certes bien exigeant pour nos communautés chrétiennes. Mais l'expérience montre que celles-ci sont en attente d'une certaine exigence et de rigueur. Ce qui, dans beaucoup de cas -	paroissiaux ou diocésains - pose plutôt problème, c'est l'absence de culture de la concertation, soit par absence de volonté commune soit parce que les changements sont imposés de l'extérieur par un petit groupe. Or la pédagogie de la réforme proposée dans ce livre tente de faire vivre aux communautés une expérience ecclésiale véritable. Cela suppose que les communautés acceptent d'établir avec leur pasteur un « état de santé >> de leur vie et examinent leur vitalité, non pas d'abord au nombre d'actions réalisées ou d'événements produits (ce qui n'est nullement négligeable), mais selon les critères spirituels et missionnaires proposés plus haut, à
 

46D	461
 
URGENCES PASTORALES DU MOMENT PRÉSENT
 
savoir la perception qu'elles ont du travail de l'Esprit saint dans leur environnement et en leur sein. Des exercices de relect :e, adaptés à chaque situation, sont alors à inventer. La der_mere partie de ce chapitre - nos sept é apes - y aura apporte une contribution et, espérons-le, une aide.
 





CONCLUSION GÉNÉRALE



L'ouvrage qu'on vient de lire se veut être un témoignage d'espérance réaliste, ou de réalisme qui donne à espérer. Pendant les vingt-cinq dernières années tous les grands indicateurs statistiques concernant la pratique et les croyances chrétiennes ont été divisés par deux; d'aucuns parlent d'un véritable déclin du christianisme en France et dans de larges régions d'Europe. S'il semble subsister un« archipel» catholique en France (avec 53 % de la population qui se disent catholiques, dont 23 % «engagés» qui se sentent rattachés à l'Église par leurs dons, leur vie familiale, parfois leurs engagements), il ressemble plutôt, au regard de leur pratique, à une pyramide avec, à la base, une immense majorité de très faibles pratiquants ou de non-pratiquants, et, au sommet, une fine pointe de personnes ayant une « pratique régulière» (peut-être 5 %), et par ailleurs assez divisées entre elles quant à leurs options de fond. Il est clair aussi - et les statistiques le montrent également depuis longtemps - que cette mutation gigantesque s'accompagne d'une érosion des représentations de la foi, réduites dans la plupart des cas à un système de valeurs

463
