\chapter{Introduction}

\subsection{Programme du séminaire ISTR 2023-2024}

\mn{Dialogue, mission et inculturation aux XXe-XXIe s. :  
De l’histoire à la théologie}

\begin{enumerate}
    \item	12 septembre : Introduction et présentation du séminaire

    \item 	19 septembre : H. de LUBAC, Le fondement théologique des missions, Paris, Seuil, 1946, p. 11-51.

    \item 	26 septembre : R. VOILLAUME, R., Au cœur des masses. La vie religieuse des Petits Frères du Père Foucauld, Paris, Cerf, 19566, p. 139-146 ; 147-164 et 215-229). \textit{Réflexion de l'expérience missionnaire du père de Foucault et son expérience va féconder des missions. Au coeur des masses : aller là où on vit. }

    \item 	3 octobre : Celso COSTANTINI, Réforme des missions au XXe siècle, Casterman, 1960, p. 237-246 et Jean PIROTTE, « Mise en perspective. Images et vie chrétienne. Quels liens ? Pour quel message », dans J. PIROTTE, Caroline SAPPIA et Olivier SERVAIS (dir.), Images et diffusion du christianisme. Expressions graphiques en contexte missionnaire XVIe-XXe siècles, Paris, Karthala, 2012, p. 19-42. \sn{personnage atypique : pouquoi les nouvelles Eglises n'ont pas leurs propres Eveques ? Eglise dans la culture locale. Pirotte : article qui complète les écrits de C. Constantini sur la place des images}

    \item 	10 octobre : Pedro ARRUPE, Écrits pour évangéliser, présentés par J-Y CALVEZ, Paris, Desclée de Brouwer – Bellarmin, 1985, p. 39-46 et 169-177. \textit{référénce sur l'inculturation}

    \item 	17 octobre : Maurice MAURIN, « De l’acculturation à l’inculturation. L’apport du Père Pédro Arrupe s.j. dans le débat théologique catholique », Cahiers de sociologie économique et culturelle, vol. 15, 1991, p. 61-82 et Paul COULON, « Sur un “gros mot” qui ne figure pas sur la couverture : inculturation », Histoire des missions chrétiennes, vol. 5, 2008, p. 3-8. \textit{commente l'article d'Arrupe sur l'inculturation}

    \item 	24 octobre : SYNODE DES ÉVÊQUES XVIe ASSEMBLÉE GÉNÉRALE ORDINAIRE, Pour une Église synodale : communion, participation, mission. Instrumentum laboris et fiches de travail, Cité du Vatican, 2023. \textit{qu'est ce qui se passe à Rome à ce moment là} \sn{Attention, je ne serai pas là}

    \item 	7 novembre : Une première réception de la session du synode à partir d’articles de presse. \textit{on nous demandera de travailler sur les documents "chauds"} 

    \item 	14 novembre : Pierre CLAVERIE, Lettres et messages d’Algérie, Paris, Karthala, 19964. \textit{on retrouve son message dans ses lettres}
\begin{itemize}
    \item Lettre n° 4 : Plaidoyer pour le respect (p. 43-49)
    \item Lettre n° 5 : Avent : attentes, espoirs et illusions (p. 51-54)
    \item Lettre n° 10 : Allez… (p. 81-87)
    \item Lettre n° 12 : Avent : qui sera le plus fort ? (p. 93-96)
    \item Lettre n° 35 : Rester ? Partir ? (p. 177-181).

\end{itemize}

    \item 	21 novembre : SYNODE DES ÉVÊQUES XIIIème ASSEMBLÉE GÉNÉRALE ORDINAIRE, La nouvelle évangélisation pour la transmission de la foi chrétienne. Instrumentum laboris, Ch II : « le temps d’une nouvelle évangélisation », Cité du Vatican, 2012. \textit{intéressant de voir le document qui a servi à Evangelii Gaudium }

    \item 	28 novembre : C. THEOBALD, « Étapes d’une ecclésiogenèse », dans Urgences pastorales. Comprendre, partager, réformer, Paris, Bayard, 2017, p. 429-462. 

    \item 5 décembre : Conclusion du séminaire

\end{enumerate}

\paragraph{des textes sur l'inculturation} depuis Arrupe. H. de Lubac a fait avancer la réflexion théologique des missions (rencontre christianisme et Bouddhisme). 

\subsection{Ce qu'on peut retenir de l'année dernière}

\paragraph{l'histoire} QU'est ce qu'on entend le mot Histoire.
\begin{Def}[Histoire]
    Faits ... qui se transforment en narration, en récit historique : 
    \begin{itemize}
        \item les sources de l'intrigue
        \item l'intrigue elle-même
        \item la narration de l'intrigue
    \end{itemize}
\end{Def}
Cela demande un esprit de discernement et esprit critique

\begin{Def}[La Mission]
    l'envoi depuis Jésus Christ, que nous devons partir. 
    témoignage et annonce du message
\end{Def}
On parle aujourd'hui de disciples-missionnaires.

\begin{Def}[Les Missions]
    histoire des missions, qui commence au XX siècle. George Goilleau a écrit sur l'histoire des missions.
    différentes périodes : l'espace et le temps. 
    
\end{Def}

L'histoire des missions est liée à l'histoire des transports. 
\begin{Ex}
    Ainsi, pendant la révolution, plus de transport, plus de mission.
\end{Ex}

\paragraph{Contexte politique}
 et refus du politique à toutes les époques

 \paragraph{dynamique missionnaire} On retrouve cela à toutes les périodes : Evangile, Eglise (instruction de 1659 : référence, de ne pas apporter la culture européenne)

\paragraph{des Eglises} jeunes.

 \paragraph{au XIX} Benoit XV : vous n'êtes pas des professionnels de la mission, vous êtes au service de la mission. Charles de Foucault : \textit{présence missionnaire}

 \paragraph{Foi du missionnaire} d'aller porter le salut aux hommes.
 Notion d'urgence. François-Xavier et qui ne cesse de dire :  \textit{il faut se presser}.
\begin{singlequote}
    Aller porter l'Evangile au contrefort de l'Enfer (missionnaire du XVIII)
\end{singlequote}

\subsection{Entre-deux}

\paragraph{l'entre-deux} une zone qui se situe entre le missionnaire qui est porteur d'une culture, et le missionné, qui lui aussi porteur de culture. chacun pose \textit{question à l'autre} : 
\begin{Ex}
    le missionnaire et son livre : pour les habitants de Tahiti, le livre est grigri et les tahitiens veulent s'approprier le livre.
    Le missionnaire découvre la culture, la nourriture,...
\end{Ex}
Et dans cet entre-deux, zone d'accommodement (adaptation au XIX, acculturation viendra post-Vatican II), qui peut être rejet du missionnaire ou de son l'acceptation. Remise en cause pour le missionnaire.
\begin{Ex}
    Ex au Vietnam, discussion sur les 7 enfers bouddhistes. 
\end{Ex}


\paragraph{entrer en conversation}


\subsection{Qu'est ce qu'on missionnaire ?}

\begin{Def}[Missionnaire]
    initialement, celui qui prêche
\end{Def}

\paragraph{Marie de l'Incarnation} parle elle de présence, évangélique. 
d'où l'expression de \textit{disciple missionnaire} chère à François.

La professionalisation de l'activité missionnaire, est remise en cause au XIX. 
La question qui se pose pour certains missionnaires : il va convertir. Est ce que je suis un gendarme de la conversion des nouveaux chrétiens ? 
Et de l'autre côté, jusqu'où doit on aller pour l'adaptation ?
\begin{Ex}[querelle des rites]
en Chine    
\end{Ex}


\subsection{aspect Théologique}

\paragraph{Mission Ad Gentes} mission externe

\paragraph{Quelle culture ?} Certaines plus pénétrables que d'autres : cf FX face à l'Inde ou au Japon (plus compliqué car culture forte).

\paragraph{Quelle vision théologique sous jacente} Ces missionnaires avaient une théologie sous-jacente. 


\section{comment on interprète un texte non théologique de façon théologique ?}

\paragraph{Quelques principes de base} Un auteur chrétien peut faire de la théologie sans le savoir.  Où du moins ne pas justifier ce qu'il affirme. 

\paragraph{quand un auteur chrétien écrit} il est témoin non critique de la théologie de son époque et lieu. 
Cela permet de voir comment une théologie est reçue dans un milieu.

\paragraph{Chaque chrétien a une façon nouvelle de vivre l'Evangile} et crée sa propre théologie.
\textit{Dei Verbum} 8 : \sn{Polyèdre du Pape François. }
\begin{singlequote}
    Cette Tradition qui vient des Apôtres progresse dans l’Église [12], sous l’assistance du Saint-Esprit ; en effet, la \textbf{perception des réalités }aussi bien que des paroles transmises s’accroît, soit par la contemplation et l’étude des croyants qui les méditent en leur cœur (cf. Lc 2, 19.51), soit par l’intelligence intérieure qu’ils éprouvent des réalités spirituelles, soit par la prédication de ceux qui, avec la succession épiscopale, ont reçu un charisme certain de vérité. Ainsi l’Église, tandis que les siècles s’écoulent, tend constamment vers la plénitude de la divine vérité, jusqu’à ce que soient accomplies en elle les paroles de Dieu.
\end{singlequote}

\begin{Ex}
    Thèrèse de Lisieux permettant de sortir d'une certaine vision janséniste à l'époque.
\end{Ex}

\subsection{méthode}
\paragraph{Quelle est la théologie sous-jacente ?} Il faut être curieux et chercher la théologie sous-jacente.  \textit{Qu'est ce qui est en jeu ? }

\paragraph{Quel est le format du texte } Lettre, encyclique, catéchisme,...

\paragraph{Quel est le contexte de sa rédaction} Pourquoi ce texte ?

\paragraph{qui est l'auteur ?} Bien mettre le rapport entre l'auteur et le contexte du texte. \textit{Eviter la mondanité}

\paragraph{A qui cela s'est adressé ?} AU départ, les encycliques adressés aux Evèques, Charles de Foucault à sa cousine,..;

\paragraph{Quelle est sa visée} lettre informative, consigne, consolation.


\subsection{Dimension théologique du texte}

\paragraph{implique une relation à Dieu} ES, Jésus Christ. Comment et quand on parle de Dieu en interaction avec les hommes ? Quel mot pour Dieu ?

\paragraph{Quelles sont les références ? } Références implicites (ex : je cite la samaritaine sans donner les réf). Comment on interpréte les textes ? Relever les images et les analogies religieuses ? 
Thèmes théologiques sous-jacents \sn{thèmes théologiques sous-jacents}

\paragraph{quel présupposé théologique qui affleure ?} L'imaginaire théologique : cela doit transparaitre dans son écriture. 
\paragraph{le monde du texte de Paul Ricoeur}

\paragraph{Nouveauté théologique qui apparait} demande une culture théologique pour sentir les choses. 


\subsection{Quelques exemples}

\paragraph{Première lettre de Pierre Claverie} Evèque d'Oran. Article dans \textit{le Lien}. 1988 : début des violences en Algérie. p. 30 : condition de la paix que nous souhaitons. Fossé entre la théologie et la vie concrete de la communauté chrétienne en Algérie. La Paix, ce n'est pas la défense de l'ordre établi, qui peut être injuste. Le droit, la vérité, la liberté. 
p31 : la paix ne saurait être qu'un don de Dieu.  [\ldots] c'est pourquoi il est juste de prier pour la paix de Dieu.
Rejoint \textit{Gaudium \& Spes}, seul Dieu peut nous fournir la paix. Il utilise la théologie du concile pour répondre à une situation concrète d'une communauté chrégtienne en Algérie.

\paragraph{Extrait du Cantique des Missions} LM Grignon de Montfort, XVII. Mission en France. chant, suscite l'émotion. 
\begin{singlequote}
 1- Malheureuse âme damnée,
Qui t’a mise dans ces feux ?
Qui t’a mise, infortunée,
Dans ces cachots ténébreux ?

LE DAMNÉ :
2- Ah ! c’est ma pure malice
Qui m’a plongée en ce feu,
Où j’éprouve la justice
Et la vengeance de Dieu !

3- Ma perte est universelle :
Dieu perdu, tout est perdu !
Dieu perdu, perte cruelle !
Ce mot n’est point entendu.

4- Ah ! que je suis misérable !
Car je ne puis aimer Dieu.
Oh! malheur insupportable
Qu’on ne comprend qu’en ce lieu !

5- Je n’ai plus Dieu pour mon Père !
Il est mon juge irrité,
Qui dans toute sa colère
Punit mon iniquité.

6- Comme je suis tout contraire
À ce Dieu saint et puissant,
Il me rend guerre pour guerre
Et m’accable à chaque instant.

7- J’ai, pour une bagatelle,
Pour un plaisir d’un moment,
Perdu la vie éternelle :
J’en enrage incessamment !

8- Hélas ! ma vie est passée.
Oh ! souvenir très cruel !
Je sens mon âme rongée
D’un repentir immortel.

9- Je gémis sans pénitence,
Je brûle sans consumer,
Je souffre sans espérance,
Je me repens sans aimer.

10- Je ne respire que flamme
Tant au dehors qu’au dedans,
Le feu pénètre mon âme :
Je suis un charbon ardent.

11- Dans tout ce qui m’environne
Je trouve un nouveau tourment ;
Je souffre sans qu’on me donne
Le moindre soulagement.

12- Tous les démons me tourmentent :
Les démons sont mes bourreaux !
Ces cruels tyrans inventent
Des tourments toujours nouveaux !

13- Le désespoir et la rage
Et les grincements de dents
Sont mon unique langage
Au milieu de mes tourments !

14- Je me déchire et me mange !
Je me dépite et maudis !
Car mon malheur est étrange,
Car mes maux sont infinis !

15- Une peine qui m’accable,
C’est la longue éternité.
Oh ! « Jamais » épouvantable !
Oh ! Terrible vérité !

16- Pour jamais avec les diables,
Les damnés et les serpents,
Dans des feux insupportables
Et dans des cachots puants !

17- Pour jamais cette demeure !
Pour jamais être damné !
Malheureuse et maudite heure
À laquelle je suis né !

18- Rage, désespoir, blasphème,
Puisqu’il faut toujours souffrir,
Puisqu’il faut rester de même
Sans jamais pouvoir mourir !

19- Je t’attends, ô maudit frère,
Qui m’as fait offenser Dieu !
Viens, je te ferai la guerre
À tout jamais en ce lieu !

20- Homme mortel, fais-toi sage,
Et le fais à ses dépens :
Si tu n’entends son langage,
Tu souffriras son tourment !

21- Oh ! quel malheur, quel langage !
J’en frémis, j’en suis touché,
Oui, je veux me rendre sage,
En évitant le péché.

\end{singlequote}
Vision janséniste (Enfer permanent), individualiste (pas de Royaume). Dieu, justice mais "il se fait justice". Les Démons sont des bourreaux qui sont là pour le juste. \textit{La pastorale de la peur.}


\paragraph{Odette Vercruysse} 
\begin{singlequote}
    Allez vous en sur les tables ?
\end{singlequote}
On n'est plus dans le rapport à Dieu car c'est Jésus qui parle. Il s'agit d'annoncer la bonne nouvelle, le péché serait de ne pas \textit{avancer la bonne nouvelle}. \textit{vision personaliste}. 


\paragraph{pratiquement} 15-20 mn. Auteur, contexte, quelle est la visée du texte ? Si c'est un texte théologique, la problématique. 

\begin{Ex}
    Claverie : sa problématique : l'Afrique.
    Et l'hypothèse : comment il va répondre à sa question. 
\end{Ex}

\paragraph{S'appuyer sur les autorités} Bible, dogmes, magistère.
\paragraph{s'appuyer sur l'argument logique, raisonnable} cohérent dans une logique donnée. 
\paragraph{Argument de fécondité, de performativité} quels fruits pour l'Eglise ? 



\mn{28 novembre : C. Theobald}