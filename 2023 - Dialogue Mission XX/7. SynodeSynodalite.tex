\chapter{Synode de la synodalité}


\section{CR du document lobaris instrumentem}


\mn{Instrumentum laboris synode 2023
Badacer Neto.
Compte rendu des échanges de la séance 07 du 24 Octobre 2023 sur l’extrait du l’Instrumentum Laboris du Synode sur la synodalité : « Pour une Église synodale : Communion, Participation, Mission », qui s’est passé à Rome du 4 au 29 octobre 2023. Présenté pour Jean-Christophe NORMAND. }
 





Le compte rendu est présenté dans deux parties : la synthèse de la présentation de l’extrait et la synthèse des discussions du groupe.
Dans une première partie, Jean-Christophe nous a présenté le texte, d’abord dans son contexte. C’est un document de synthèse des travaux des conférences des évêques continentales que le secrétariat général pour le synode a préparé et est donc un instrument de travail.  Le synode fait partie de la tradition de l’église, mais c’était le pape Paul VI qui l’a récupéré dans l’histoire. Il s’inscrit dans le cadre de la XVIème Assemblée générale ordinaire du Synode des évêques et a pour thème : « Pour une Église synodale : communion, participation, mission » prévu d’être finalisé en octobre 2024. La structure générale de l’extrait se compose d’un préambule avec un rappel du chemin parcouru et des précisions de méthodologie et d’articulation des trois thèmes : Communion, Participation et Mission, cependant, à la suite du paragraphe 43 l’ordre d’apparition des trois termes change pour « Communion, Mission, et Participation ». La « Mission » est mise au centre en renforçant le lien avec « Communion » et la « Participation » reste en troisième place comme une conséquence des deux premiers et une posture à être adoptée par tous les acteurs. Puis, la méthode par rapport à la conversion dans l’Esprit est une nouveauté du point de vue du processus synodal qui donne une supériorité au processus par rapport au rendu final des travaux. En plus, le document se développe en articulant deux questions sur les trois thèmes où chacun est explicité à partir d’une problématique générale, suivie d’une question de discernement pour approfondir et pour préciser le problème et enfin, une fiche de travail avec neuf questions guides pour les discussions des participants. 

\paragraph{les thèmes}
Ensuite, selon lui, le premier thème, « communion » renvoie au problème de l’unité dans et de l’Église. Le second thème, « mission » est évalué en termes de co-responsabilité en rapport avec la vocation-nature missionnaire de toute l’Église. La responsabilité de tout baptisé est interpellée par rapport au service de la mission. 

Des questions sont posées : \begin{itemize}
    \item comment favoriser une conscience partagée de la mission ? 
        \item comment renforcer la coresponsabilité missionnaire dans la perspective « d’une Église toute ministérielle » ? 
            \item comment donner aux femmes une plus grande participation dans les missions de l’Église ? 
     \item    comment mieux articuler la coresponsabilité entre laïcs et ministères ordonnés ? toujours dans la perspective synodale missionnaire.         
\end{itemize}Puis, sur les questions de la troisième partie « Participation », gouvernance et autorité n’apparaîtront pas dans l’extrait, en revanche les fiches de travail qui suivent ont le même format. 

\subsection{les échanges}
Dans une seconde partie, sur les échanges du groupe nous remarquons, d’abord que le document pose des questions mais ne formule pas des orientations prédéterminées une fois que le texte accompagne un processus qui n'est pas encore arrivé à son terme.
\paragraph{herméneutique de réforme}
Ensuite, le message clé est celui de la mise en œuvre de l’Église synodale et l’enjeu est de réussir à faire émerger une forme de consensus sur la manière avec laquelle l’Église d’aujourd’hui puisse répondre à sa vocation et le but de l’exercice est de vivre la conversion dans l’Esprit, dans une logique de co-responsabilité sans compromettre le contenu de l’identité chrétienne.  
\paragraph{questions de la mission}
Puis, sur la question de la mission, plusieurs points ont été relevés : Le rappel que l’Église est par nature missionnaire. La recherche d’un nouveau concept de mission où il y avait des remarques sur le paragraphe 52 qui dit « la mission n’est pas la promotion d’un produit religieux, mais la construction d’une communauté… ». Dans le paragraphe 53 il est dit que « la mission, elle-même, a une dimension constitutivement synodale ». 

\paragraph{autres questions}
D’autres remarques, sur des questions comme l’ordination des hommes mariés, la place des femmes etc.
Pour conclure le texte, des questions restent ouvertes étant donné que le synode n’est pas encore fini, mais pour la séance 08 du 07 novembre, aujourd’hui, nous allons continuer la réflexion à partir de « La lettre au peuple de Dieu », « l’article de la Croix » et du « Rapport de synthèse du synode 2023 » qui seront présentés.  



\paragraph{une démarche à vivre} difficile à transmettre.  Nathalie Becquart.


\paragraph{rapport de synthèse} On des points de convergences, des questions, et recommandations. Mais c'est un rapport intermédiaire en 2024.


\subsection{les 20 chapitres du rapport de synthèse}


\paragraph{nouvelle expérience de la synodalité} donnée anthropologique, à l'appui de cette méthode. Elargissement des personnes à inviter. Démarche inter-générationnel.

\paragraph{synodalité et approche trinitaire}


\paragraph{partie 2 : Tous missionnaire} "Affirmons que l'Eglise est mission".

\paragraph{lecteur...} prédication par des laics

\paragraph{les femmes} On reconnait le besoin de les prendre en compte. 



quelques remarques : 
\begin{itemize}
    \item Un retard du droit canon par rapport à la réalité (proces canonique, lecteur,...). Est ce que la dogmatique passe forcément par le droit canon ?
    \item un peu déroutant que l'approche de François d'initier les processus plutot que de contrôler les espaces. Les recommandations sont un peu décevantes, mais peut être n'est ce pas ce qu'il faut chercher, chacun étant plus sensible à l'un ou l'autre des aspects.
    On avait ainsi la même chose pour Amoris Laetitia qui ne parle que de la contraception pour dire l'opposition à la contrainte. Mais Humanae Vitae reste la référence, note de base de page. 

    
\end{itemize}

\section{Comment ce document présente la mission de l'Eglise ?}

Termes de missions, écoute (8 fois), inculturation (et culture). 



\paragraph{Les références} bcp à l'évangile, pas de théologiens, réference peu nombreuses au magistère (Ecclesiam Suam),

\paragraph{on revient à une définition pre-Ignatienne de la mission} définition théologique de la mission et non de l'"homme envoyé". L'individu est envoyé (avec la modernité). 


\paragraph{style} conversation. 

\paragraph{on ne pourra écouter le monde si on ne s'écoute pas} A la manière de Jésus, modèle missionnaire. (à la différence de Paul). cf Benoit XVI \textit{les batisseurs de l'Eglise}, commence par la figure de Jésus comme 


\paragraph{numérique} à relire

\paragraph{savoir répondre aux attentes de ce temps, face à l'individualisme, le populisme et le } \textit{le temps est plus grand que l'espace}

\paragraph{comment gérer l'équilibre entre ceux qui veulent aller vite et ceux qui veulent aller lentement }  Option préférentielle pour les pauvres. S'ils ne sont pas présents, on ne soit pas dans la vérité. Aspect culturel.

\paragraph{theobald} la conversion se fait dans la rencontre qui nous transforme. La mission

\paragraph{Ignace de Loyola} A modifié la vision "le père envoie le fils qui nous envoie". "Allez" : profesionnalisation de la mission : figure du \textit{missionnaire}. "vous êtes mes témoins" : rayonnez de la Foi.


\section{CR}

\mn{Nuria}