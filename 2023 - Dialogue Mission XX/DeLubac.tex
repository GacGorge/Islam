\chapter{Le fondement théologique des missions - Henri de Lubac}


\paragraph{Hypothèse} L'Eglise est missionnaire par essence, avec deux parties : démarche à l'écoute de la révélation (Aude Fidei), la mission, préfigurée dans l'AT : Israel est une figure de l'AT, une dimension missionnaire

\paragraph{une question : fait comme si Israel ne continuait pas avoir des missions} Israel demeure dans le judaisme (cf Romain 11) : 
\begin{singlequote}
    Je dis donc: Dieu a-t-il rejeté son peuple? Loin de là! Car moi aussi je suis Israélite, de la postérité d'Abraham, de la tribu de Benjamin.

2 Dieu n'a point rejeté son peuple, qu'il a connu d'avance. 


5 De même aussi dans le temps présent il y un reste, selon l'élection de la grâce.
 
20 Cela est vrai; elles ont été retranchées pour cause d'incrédulité, et toi, tu subsistes par la foi. Ne t'abandonne pas à l'orgueil, mais crains;

26 Et ainsi tout Israël sera sauvé, selon qu'il est écrit: Le libérateur viendra de Sion, Et il détournera de Jacob les impiétés;

27 Et ce sera mon alliance avec eux, Lorsque j'ôterai leurs péchés.

28 En ce qui concerne l'Évangile, ils sont ennemis à cause de vous; mais en ce qui concerne l'élection, ils sont aimés à cause de leurs pères.

29 Car Dieu ne se repent pas de ses dons et de son appel.

\end{singlequote}

\paragraph{missiologie incluse dans Ecclésiologie} pourquoi la fonder théologiquement

\paragraph{approche historique : d'abord un mandat du Christ} Mt 28,16-20


\begin{singlequote}
    L'oeuvre missionnaire ne sera donc aucunement une oeuvre surérogatoire, elle ne sera pas comme à la périphérie de ses oeuvre; elle constitue une part de son activité essentielle
\end{singlequote}

\paragraph{Les missions, premier moyen par lequel l'Eglise accomplit sa mission}

\paragraph{pas slt un ordre de Jésus, mais la mission de l'Eglise} pourquoi ? c'est la question de l'article à travers le petit nombre ?

\paragraph{Israel} d'abord deux moyens de passer du monolâtrisme (Dieu particulier) au monothéisme (Dieu universel) gardant la place particulier d'Israël : exterminer les autres religions ou que les autres peuples reconnaissent plus ou moins par la force le Dieu d'Israel. Israël est alors la caste des prêtres de Yahvé.

\paragraph{Troisième voie }Reste une troisième voie, celle du second isaie et de Tobie. Deuxième Isaie : Israel est le serviteur de Yahvé. Pluriel et singulier se mélange de façon féconde. Mais cette troisième fois compliquée par le renforcement de la place de la Loi post Exil (Ezéchiel). 
\begin{singlequote}
    son culte toujours plus national et son Dieu de plus en plus universel, deveitn une contradiction vivante.
\end{singlequote}

\paragraph{quelles solutions dès lors}

\begin{itemize}
    \item solution eschatologique :s'en remettre à Dieu seul du soin de les réaliser.  "au jour de sa puissance"
    \item "les prosélytes", les "craignant dieu" : entrebaillé la porte du Royaume.
    \item nomocratie : repousser les gentils méprisés. cf Akiba dans le cantique
\end{itemize}

\begin{Synthesis}
    Enraciner la mission dans le monothéisme
\end{Synthesis}

Impasse sauf par une mue prodigieuse, mort selon la lettre :
\begin{singlequote}
    Je mettrai ma Loi en eu dans dans leurs coeurs, je l'écrirai, et je serai leur Dieu et eux seront mon Peuple ! Jn 31, 33
\end{singlequote}

\paragraph{Christianisme} 29
Cette attente était plus de contradictions nouvelles.
Héritant d'Israel, Jésus le transforme en l'Eglise (St Paul : coup de génie , P. Lagrange : coup de génie certes mais par l'arrivée du Messie).
Quand Jésus confie aux disciples d'évangéliser, il ne fait que résumer la mission du Christ.


\paragraph{Eglise catholique} L'Eglise se sait en droit universelle. Elle vise à l'être en fait 30.

\paragraph{Face aux donatistes, Augustin} leur reproche de se satisfaire de leur Elgise, sans élan missionnaire. Catholique "de la mer à la mer", chantier dans l'univers entier le cantique toujours nouveau de l'universelle charité. 


\paragraph{Va et vient AT et NT} Reprise de ce que faisait les pères de l'Eglise.

\paragraph{va un peu vite sur le fait qu'Israel est missionnaire} Or, la particularisme d'Israel fait que la \textit{peuple d'Israel} a une vocation sacerdotale mais pas forcément de mission. \sn{Et Islam ? missions ? }


\section{Deuxième partie}



\paragraph{pourquoi les missions : triple volonté} du Christ, Ecriture et histoire, volonté enfin inscrite dans la structure même de cette Eglise et dans la conscience qu'elle a tjs eue d'elle même.

\paragraph{Analogie de la Foi} instruction de Vatican (I). 


\paragraph{la question ici est le salut des non chrétiens : rigorisme ou laxisme} Est ce que c'est une facilitation du salut ou une clé du salut ? poids de la société et de la liberté individuelle ? Si le christianisme est religion d'amour, on ne peut pas croire que la damnation condamne le non croyant.
Chanoine Glorieux :
\begin{singlequote}
    les missions ne sont pas tant une affaire de vie ou de mort que de plénitude de vie.
\end{singlequote}
cf Matteo Ricci : "la naturelle."  ouvrir au contraire généreusement le ciel aux infidèles

\paragraph{détour par la charité} c'est une nécessité pour moi d'aimer et d"évangéliser, non ma gloire (St augustin) p 41
Dieu nous choisit pour les autres.


\paragraph{but des missions} n'allons pas trop vite à la conversion. Juste mais si m'on considère que le but est le salut des âmes, catégorie spirituelle et non catégorie métaphysique \sn{pas convertir les musulmans mais Ahmed}


\paragraph{"planter l'Eglise"}
Est ce que le moyen est l'Eglise partout ou le salut à tous. \textit{dimension communautaire de la mission}.
\begin{singlequote}
    Mais ce « Royaume du Christ», qui est ici synonyme de l'Église, peut être lui-même compris en deux sens. L'Eglise est un moyen, le grand moyen du salut, et elle est aussi elle-même une fin, la fin suprême ·de la création. Elle est un corps visible et transitoire, et elle est le Corps du Christ, mystérieux et éternel. Comme le Christ est la voie qui mène à la vie, mais aussi bien la vie même à laquelle aboutit cette voie, ainsi de l'Église, selon l'aspect sous lequel on l'envisage : voie de salut, elle est aussi le terme ; elle est cette unité spirituelle en quoi consiste la réalité du salut. p.47
\end{singlequote}

\paragraph{histoire de la Mission} Henri de Lubac veut sortir de la vision professionnelle de la mission : 
\begin{singlequote}
    On enracine. les Pardons en Bretagne : on plantait la croix à un endroit donné. 
    Notion de planter la croix. Quand les Portugais partaient découvrir le monde, ils plantaient des croix.
    Ce n'est pas forcément une création d'une Eglise mais présence chrétienne.

    Ignace : Mission à partir du XVI. 

    
\end{singlequote}

\paragraph{Arrière fond historique} 1946 : peur de l'athéisme communiste : il faut se battre. 
\begin{Synthesis}
    Charité pour tous et pour moi.
    Dimension eschatologique de l'Eglise qui dépasse la dimension terrestre de l'Eglise sur terre.  
\end{Synthesis}

\textit{planter} au Brésil : fait penser à la colonisation. 
\section{termes à réutiliser}

les considérations qui précèdent, par lesquelles nous avons tâché de...,; nous laisse le sentiment qu'elles ...

Ainsi,


\paragraph{Compagnie de Jésus}
\begin{singlequote}
    
\end{singlequote}




