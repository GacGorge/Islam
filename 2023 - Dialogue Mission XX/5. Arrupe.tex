\chapter{Arrupe, le Problème de la Culture}


\section{Arrupe}

\subsection{exposé}

Né à Bilbao en 1907, Pedro Arrupe entre dans la Compagnie de Jésus en 1927, étant lui-même animé par une véritable passion pour l’évangélisation. Il est envoyé comme missionnaire au Japon à partir de 1938. Il y passe donc la 2nde guerre mondiale et y connait la prison. Il devient ensuite provincial du Japon.
Le 22 mai 1965 –après 23 ans au Japon - il est élu Supérieur Général de la Compagnie de Jésus, fonction qu’il va exercer pendant 16 ans. Il prend ainsi part à la dernière session du concile Vatican II (celle qui aboutit notamment à la promulgation de Gaudium et Spes, sur l’Eglise dans le monde de ce temps, et Ad Gentes, sur l’activité missionnaire de l’Eglise). Pedro Arrupe est profondément marqué par les documents et l’atmosphère du Concile, qui aborde de front la question missionnaire. Le premier texte du corpus correspond ainsi à une conférence qu’il a donnée le 20 octobre 1965 pendant le concile, conférence donnée aux journalistes accrédités. Il y aborde le thème de la culture.
Entre 1965 et 1966 a lieu la 31ème congrégation générale de la Compagnie de Jésus, puis entre 1974 et 1975, la 32ème congrégation générale. Elles abordent de nombreuses questions liées notamment à l’identité et la mission de la Compagnie, évoquant entre autres la promotion de la justice et la manière d’évangéliser. Pedro Arrupe est très vigilant à adapter la mission et la spiritualité ignacienne aux temps présents, tout en voulant rester fidèle à Saint Ignace. La 32ème congrégation générale aboutit à la production d’un document d’orientation sur la mission de la Compagnie, invitant à promouvoir une « inculturation de la foi et de la vie chrétienne sur tous les continents ». Le deuxième document étudié est ainsi une lettre écrite 3 ans après par Arrupe dans ce cadre.
On peut noter qu’en tant que supérieur général, Pedro Arrupe a beaucoup voyagé à la rencontre des jésuites du monde entier, ce qui l’a conduit à enrichir sa réflexion sur l’inculturation1. Il avait de plus passé 23 ans au Japon et jouissait donc d’une expérience de terrain.
Il quitte sa charge de supérieur général en 1983 officiellement.
Plusieurs thèmes sont très chers à Arrupe. Dans le recueil « Ecrits pour évangéliser » d’où sont extraits les documents, trois parties thématiques sont ainsi identifiées2 :
\begin{itemize}
    \item  	Identifier les besoins nouveaux du monde et les urgences dans l’Eglise (lire les
« signes des temps »).
    \item 	Les différentes dimensions de l’évangélisation (les non chrétiens, l’œcuménisme, l’inculturation, la justice…). Lire notamment sa conférence de 1968 sur la foi chrétienne et l’action missionnaire aujourd’hui3.
    \item Les « serviteurs de l’Evangile ».
\end{itemize}


\textit{L'inculturation est un nouveau terme apparu dans l'Eglise catholique durant les années 1970 pour exprimer la relation entre
l'Evangile et les cultures. Les missionnaires, en portant l'Evangile, transforment les cultures locales et se laissent transformer
par elles. Ce terme a été préféré à celui, plus sociologique, d'acculturation. Cet article retrace l'histoire de l'apparition de ce
concept, à travers les écrits de son principal promoteur : le Père Pedro ARRUPE, Supérieur Général de la Compagnie de
Jésus.}

Le Père Pedro ARRUPE (1907-1991) est d'abord un missionnaire \sn{De l'acculturation à l'inculturation. L'apport du Père Pédro Arrupe s.
j. dans le débat théologique catholique
Maurice Maurin
}. Basque
espagnol, médecin de formation, il entre dans la Compagnie de Jésus en 1927. Après sa
formation, en 1938, il débarque au Japon et devient vite curé dans un poste isolé, à plus
de vingt heures de train de Tokyo. Il fait l'expérience de la solitude du missionnaire,
soupçonné en plus d'espionnage du fait de la guerre. 
\begin{singlequote}
    "Considérons ce qu'un
missionnaire en pays païen doit être prêt à endurer : se retrouver seul dans une grande
ville, sans aucun ami ni connaissance, sans ressources d'aucune sorte, qu'il s'agisse
d'équipement physique ou de l'appui et de la sécurité que donnent de simples relations
humaines; être pauvre jusqu'en son langage, incapable de s'exprimer, de pouvoir dire qui
on est, ce que l'on sait; se retrouver toujours en situation inférieure, comme un enfant qui
apprend à parler, que l'on écarte avec dédain de toute discussion", dira-t-il plus tard (1).
\end{singlequote}
En 1942, il devient maître des novices, en poste à Hiroshima. Il sera un des
rescapés du 6 août 1945. Après guerre, il publie huit ouvrages de théologie en japonais,
ainsi que des traductions de S. François-Xavier et S. Jean de la Croix, ce qui suppose
une bonne connaissance de la culture et de la langue japonaise. En 1954, il prend la
responsabilité de ce qui va devenir la Province du Japon. Il se fait remarquer pour son
dynamisme et sa volonté de dialoguer avec la culture du pays qui l'accueille. Sous son
impulsion, l'Université catholique "Sophia" de Tokyo prend une grande extension.
Cet homme d'Eglise est plus pasteur que théologien. Il est une figure du
catholicisme social et intégral. Il allie à la fois un humanisme jésuite, une grande attention
au monde et à ses possibles, et une ferme affirmation de la foi catholique. L'habite
l'intime conviction que le Christianisme est un apport déterminant pour la vie du monde.
Arrupe sera fidèle à cette perspective comme Supérieur Général de l'ordre. En 1968, dans
une intervention sur les missions, il reste très méfiant vis-à-vis de toutes les nouvelles
théories qui discernent dans les religions une fonction salvifique permanente. Dans toute
religion, il y a des éléments de vérité, mais ces éléments sont provisoires, jusqu'à ce que
l'Eglise convertisse au Christ. \paragraph{Quel rôle attribue-t-il au dialogue ?} 
\begin{singlequote}
    "En mettant en contact
avec l'Evangile, il fera apparaître ce qu'elles ont de bon et de vrai, et en même temps ce
qu'elles ont d'insuffisant" (2).
\end{singlequote} 
Son jugement sur ces religions reste sans appel. 
\begin{singlequote}
    "Que sont
en effet les religions non chrétiennes sinon des efforts, au résultat incomplet ou même
tout à fait manqué, pour donner à l'homme d'un espace socio-culturel déterminé la
réponse à ces problèmes universels qu'il éprouve en lui-même et auxquels il cherche une
réponse ?".
\end{singlequote}
 Le missionnaire accepte le dialogue, se refuse de procéder à destruction de la
culture autochtone, mais il est là pour porter la bonne odeur du Christ. Nous sommes
dans le régime du témoignage apologétique.
Le 22 mai 1965 Pedro Arrupe est élu Supérieur Général de la Compagnie de
Jésus. Après 27 ans passés au Japon, à 58 ans, il va donner quelques impulsions
majeures qui vont permettre à la Compagnie d'être un instrument d'évangélisation
efficace en cette fin du XXème siècle. Nous sommes à la fin du Concile Vatican II,
moment où l'Eglise catholique s'interroge sur sa relation au monde profane, et renonce à
des sentiments de forteresse assiégée par le modernisme. Paul VI assigne en 1965 à la
Compagnie de Jésus la mission de combattre l'athéisme. La question est d'abord
occidentale, face aux menaces de marxisme, du nihilisme philosophique, et du
matérialisme des sociétés de consommation. A partir de 1968, le débat se déplace. Il se
cristallise autour du thème de "justice". On peut y voir une prise de conscience en
Occident des injustices régnantes et de la nécessité de luttes sociales et politiques pour
changer les états de fait. Cependant plus encore se font entendre dans l'Eglise catholique
les voix du Tiers-Monde pour dénoncer les situations de pauvretés extrêmes. En se
mondialisant, l'action de l'Eglise n'est plus simplement réfutation de l'athéisme
occidental, mais engagement dans des luttes de libération face aux injustices et aux
déséquilibres mondiaux. On est, de ce fait, amené à s'interroger à nouveaux frais sur le
rapport de toute l'activité humaine au Royaume de Dieu et, en particulier, des activités de
libération, de promotion de l'homme. Cette question de l'engagement pour la justice va
devenir un thème fondamental du Généralat de P. Arrupe. \begin{singlequote}
    "C'est une option prioritaire,
elle doit avoir une influence sur toute notre vie".
\end{singlequote} En affirmant ceci lors d'une de ses
interventions, il évoque avec réalisme les conséquences que cela va entraîner pour la
Compagnie. 
\begin{singlequote}
    "Si nous voulons travailler sérieusement pour la justice, et jusqu'à ses
conséquences extrêmes, la croix se présentera tout de suite à nous, et bien souvent
accompagnée d'une douleur cuisante". 
\end{singlequote}

Et Arrupe se fait concret. 
\begin{singlequote}
"Ce sont d'excellents
chrétiens, nos bienfaiteurs, nos amis, des membres de nos familles qui nous accuseront
de marxisme et de subversion, nous retireront leur confiance et leur aide économique"
(3). 
\end{singlequote}

\paragraph{A LA RECHERCHE D'UN NOUVEAU VOCABULAIRE DANS
L'EGLISE}
\section{Plan-+ 1965 le problème de la culture}
\begin{itemize}
\item désintégration de la culture
\item réunifier l'homme
\item mission de l'église : 
\begin{itemize}
    \item un fait : rapport entre l'Eglise / cluture : fait. Parole de l'Eglise : fait un héritage culturel. pas d'indépendance
    \item une vérité : pas une résignation, mais une vérité : l'homme sauvé n'est pas un individu isolé mais membre d'une communauté fraternelle. 
    Citation : 1659 Chine. Ne pas annoncer la France,...
    \textit{Omnia Omnibus}

    \item ferment des cultures
     \textit{solidarité nécessaire entre le message évangélique et l'équilibre culturel des homme auxquelles elle s'adresse}. Dans les pauvretés culturelles humaines, développer 
     \item l'accueil des cultures
     \item le discernement
     \item purification
     \item deux questions :  quelle est la \textit{culture propre à notre civilisation} b/ rôle missionnaire actuel de l'église.  L'Eglise doit maintenir son unité tout en étant pleinement adaptée aux requêtes légitime de chaque culture.
\end{itemize}
 
\end{itemize}


\subsection{présentation du texte}

Le premier document \sn{Poulain Nicolas – ICP 2023-2024} est une conférence du 20 octobre 1965, pendant le Concile, sur le thème de la culture, conférence donnée devant les journalistes. Le plan du discours est plutôt clair : après un premier temps de diagnostic, d’état des lieux concernant la culture, il propose un remède face à la crise identifiée, avant de décrire la mission de l’Eglise dans ce cadre.
1.	Le diagnostic : une crise actuelle de la culture

Le premier temps du raisonnement proposé par Arrupe consiste donc en un état des lieux. Il commence ainsi par affirmer que la culture ne peut être le développement des facultés séparées les unes des autres. Il y a ici une question de totalité, d’intégralité (voire d’intégrité) de l’homme.
Or, il constate que l’homme s’est séparé de Dieu, au fur et à mesure, en lien sans doute avec une forme d’humanisme érigeant l’homme en valeur suprême (peut-être peut-on parler ici de l’humanisme athée dénoncé par de Lubac4). Cela a conduit à dissocier les divers éléments de la culture et à entrer dans une volonté de spécialisation, sans garder de caractère unificateur, de point de vue global. Chaque thème ou spécialité a été considéré comme un absolu à part entière… ce qui conduit irrémédiablement à un échec.
\paragraph{2.	Un remède}


Fort de ce constat, Arrupe déduit qu’il y a un besoin : celui de « réunifier l’homme en réintégrant son savoir ». Il s’agit de l’amener à se regarder lui-même en tant qu’homme, et non de se concentrer sur une discipline particulière. Pour illustrer son propos, Pedro Arrupe reprend la maxime inscrite sur le fronton du temple de Delphes, « Connais-toi toi-même », utilisée particulièrement par Socrate. Cette invitation à connaître sa condition, sa place dans le monde, est encore valable aujourd’hui : l’homme ne doit pas seulement chercher à comprendre le monde extérieur de manière isolée, il doit aussi, pour unifier ce savoir, comprendre ce qu’il est lui-même. C’est tout l’enjeu de l’anthropologie.
Or, cela ne peut se faire sans appel au Christ. L’homme ne peut se comprendre pleinement qu’à partir du Christ, qui récapitule tout en lui. On peut penser au fameux « Ecce homo »5 de Pilate présentant le Christ. Si la culture crée en l’homme une soif insatiable de totalité… ne peut –on y voir une image de la soif de Dieu? Il s’agit alors de viser une synthèse de la culture dans la foi et par la foi, à l’image de l’universitas médiévale.\sn{4 Cf. de LUBAC, Le drame de l’humanisme athée, 1945
5 Jn 19, 5}


 \paragraph{3.	La mission de l’Eglise dans ce contexte}


Quelle est alors la mission de l’Eglise ? Elle doit offrir son message au monde, notamment pour permettre à l’homme de mieux se comprendre lui-même, et ceci dans un langage accessible à tous. Pour cela, il faut prendre conscience que la proclamation de l’Evangile a toujours lieu dans un temps et un lieu donné, elle ne peut donc pas être séparée de la culture de ses destinataires. Pour Arrupe, si l’on ne fait pas cet effort, cette proclamation ne peut aboutir.
A ce fait doit être ajoutée une vérité : l’homme est membre d’une communauté, et il ne peut être sauvé sans elle. Il s’agit de sauver l’homme dans et avec son milieu. Le jésuite s’appuie ici sur un texte du Saint Siège en 1659 – déjà évoqué la semaine dernière chez Costantini - indiquant de « ne pas introduire en Chine un pays, mais une foi qui respecte la culture », et la convocation de plusieurs figures biblique et historiques, tels Paul, mais aussi Nobili, Ricci…
Il y a un enrichissement mutuel dans ces liens entre Eglise et culture: en annonçant l’Evangile, \textit{l’Eglise permet aux cultures de se développer} ; en parallèle, elle reçoit elle aussi de ces cultures qui la conduisent à s’interroger sur ce qu’elle transmet, et donc à avancer. Elle vient aider à discerner dans les différentes cultures ce qui les épanouit et les renferme sur elles- mêmes ; \textit{en parallèle, les cultures viennent aider l’Eglise à purifier son message, en détachant dans son contenu ce qui essentiel de ce qui n’est que culturel.}
En ouverture, Arrupe pose deux questions : quelle est la culture propre à la civilisation industrielle ? Et comment l’Eglise peut-elle se rendre accessible dans chacune des cultures, quelles adaptations doit-elle effectuer, et comment peut-elle intégrer les richesses des autres cultures ?

\paragraph{4.	Eléments d’analyse de cette argumentation
}

Les liens entre la conférence de Pedro Arrupe et les travaux du concile sont évidents, du fait même du contexte : dans la structure même du discours, on a l’impression de retrouver le texte de Gaudium et Spes, qui ne sera publié qu’en décembre 1965. \sn{6 12.2. Mais qu’est-ce que l’homme ? Sur lui-même, il a proposé et propose encore des opinions multiples, diverses et même opposées, suivant lesquelles, souvent, ou bien il s’exalte lui-même comme une norme absolue, ou bien il se rabaisse jusqu’au désespoir : d’où ses doutes et ses angoisses. Ces difficultés, l’Église les ressent à fond. Instruite par la Révélation divine, elle peut y apporter une réponse, où se trouve dessinée la condition véritable de l’homme, où sont mises au clair ses faiblesses, mais où peuvent en même temps être justement reconnues sa dignité et sa vocation.}
\sn{7 22. 1. En réalité, le mystère de l’homme ne s’éclaire vraiment que dans le mystère du Verbe incarné. Adam, en effet, le premier homme, était la figure de celui qui devait venir, le Christ Seigneur. Nouvel Adam, le Christ, dans la révélation même du mystère du Père et de son amour, manifeste pleinement l’homme à lui-même et lui découvre la sublimité de sa vocation.} Ainsi, les deux raisonnements commencent par analyser la situation actuelle de l’homme et de la société humaine, puis décrivent une crise, avant d’établir que cette crise s’ancre dans un besoin d’une meilleure compréhension de ce qu’est l’homme. Pour y répondre, un passage par la Révélation en Jésus-Christ est nécessaire. Sur ce point, on peut ici penser à Gaudium et Spes 126 et 227 par
exemple. Il s’agit alors d’apporter l’éclairage de l’Eglise elle-même éclairée par la Révélation. Notons que Gaudium et Spes évoque également les relations mutuelles entre Eglise et société, et le fait que chacun doit recevoir de l’autre : le chapitre 4, Le rôle de l’Église dans le monde de ce temps, évoque ainsi successivement les rapports mutuels de l’Eglise du monde (GS 40), puis l’aide que l’Eglise veut offrir à tout homme, à la société, à l’activité humaine (GS 41-43), avant d’aborder l’aide que l’Eglise reçoit du monde aujourd’hui (GS 44). Il y a même toute une partie consacrée à la culture, dans laquelle on retrouve nombre d’idées avancées par Arrupe, comme le fait que la proclamation de l’Evangile vient aider à discerner dans une culture ce qui l’élève ou non (GS 58)8.
Pour appuyer ses dires, Pedro Arrupe convoque différentes sources:
-	L’Ecriture : référence à Saint Paul, par une citation mais aussi par des évocations de sa prédication à Athènes
-	Des éléments de la tradition de l’Eglise (texte du Saint Siège en 1659)
-	Des éléments historiques (personnages historiques : Ricci, Nobili ; rôle culturel de l’Eglise au Moyen-âge)
-	Un élément de « philosophie grecque » (le « connais-toi toi-même »)
-	Des éléments de l’actualité, montrant que l’Eglise est effectivement à l’écoute du monde (par exemple la question du dialogue avec le monde scientifique)
-	Quelques éléments de théologie plus discrets (le Christ récapitulateur).


\begin{Prop}[réciprocité entre Eglise et culture]
    L'Evangile permet de trouver ce qui est la vraie culture.
    Mais en sens inverse, il y a un retour, à recevoir pour développer des aspects peu développés du christianisme. Il ne s'agit pas de vérité qu'avait le christianisme en \textit{stock}. Découverte (et non redécouverte) de ses propres richersses. 
\end{Prop}

\paragraph{Nouveauté de ce message} discussion sur la nouveauté. 

\textit{de Nobili} : jusqu'où peut on aller dans l'inculturation ? Entre en conversation en se laissant aller sans a priori, sans retenue pour aller le plus loin possible dans la culture.
\textit{François Xavier} s'était posé la question des semences du verbe en Chine. p. 172. Xavier pour l'adaptation. 

\paragraph{Purification (partie 4)} le terme purification est employé pour l'Eglise elle-même.  Par rapport au polyèdre, le polyèdre n'est pas transparent : on est tjs dans une culture. Et une culture source, particulière, celle de l'Evangile, élement de discernement.
Discernement a posteriori à faire.




\section{Sur l'inculturation 1978}
\begin{itemize}
    \item notion, actualité et universalité de l'inculturation
    \begin{singlequote}
        l'influence novatrice et transformatrice de l'expérience chrétienne dans une culture, une fois résolue la crise que peut faire naître la confrontation entre l'une et l'autre, contribue à donner une cohésion nouvelle à cette culture. En second lieu, elle aide à assimiler les valeurs universelles qu'aucune culture particulière ne peut épuiser. Elle Invite, enfin , à entrer dans une communion nouvelle et profonde avec d'autres cultures, dans la mesure où toutes sont appelées à former, dans un enrichissement et une complémentarité mutuels, le \textit{Vêtement aux mille couleurs} qu'est la réalité culturelle de l'unique Peuple de Dieu en marche vers son Seingeur.
    \end{singlequote}
    \item Attitudes requises. Attitude fondamentale : vision unitaire, docilité à l'Esprit
    . indifférence spirituelle, accueil et don. puis discernement (éviter les jugements superficiels= Ouverture intérieure. Longue patience.  \textit{semina Verbi}, pierre d'attente. Sensus Ecclesiae très ignatien. 
        Conséquences internes
       \item attitudes requises
       \item conséquences internes
       
    
    
    
\end{itemize}
 \subsection{textes sur l'inculturation}
Ce qui peut être intéressant à mettre à jour ici, c’est la démarche théologique qui
sous-tend cette recherche. "L'Incarnation du Fils est la raison première et le modèle
parfait de l'inculturation" (n.ll). C'est le paradoxe divin que d'exprimer son universalité
dans un particularisme. Tout homme est en contact avec la Parole de Dieu par des
médiations enracinées dans une histoire et un lieu. "Lorsqu'une communauté s'ouvre à la
Bonne Nouvelle, tout en gardant son identité culturelle, son insertion dans l'Eglise est
plus authentique et l'Eglise s'enrichit de nouvelles valeurs" (n. 13).

\paragraph{d) Postérité de cette réflexion
}
Les années 70 ont été le temps d'une première élucidation du rapport
évangélisation / cultures. La Compagnie de Jésus a largement favorisé l'utilisation des
deux concepts d'acculturation et d'inculturation, et le passage de l'un à l'autre. Les
années 80 voient une inflation de ces concepts, en particulier parce que la question de la
culture est chère à Jean-Paul II. Par sa réflexion, il souhaite nous engager sur une voie
exigeante, celle de la réconciliation rédemptrice opérée par le Christ au coeur des cultures.
Depuis son discours à l'UNESCO en 1980, jusqu'aux propos les plus récents, Jean-Paul
II s'attarde longuement à ce rapport entre le christianisme et les cultures. Au point que le
discours de 1980 en laisse plus d'un dans l'étonnement à l'égard du caractère tranchant
des thèses sur le "lien organique et constitutif" entre religion et culture. La réflexion du
Pape sur la culture et l'inculturation est largement relayée par le travail du Conseil
Pontifical de la Culture, présidé par le cardinal Poupard, comme par la Commission
73
Théologique Internationale (16). Nous ne pouvons ici entrer dans toutes les citations.
Nous nous contentons de souligner quelques approches et aspects. 


\subsection{Analyse}

Ce deuxième document est une lettre adressée aux jésuites, en 1978, dans la suite de la 32ème Congrégation Générale et à la demande de celle-ci, pour expliquer le concept d’inculturation de la foi et la vie chrétiennes, voté dans un bref décret de cette congrégation. Cette lettre est envoyée le jour de la Pentecôte, ce qui est déjà un signe en soi : c’est le jour où chacun comprend les apôtres dans sa propre langue ! Il avait déjà défini l’acculturation en 1977, dans une conférence aux religieux de Colombie prônant l’insertion dans le monde, la présentant comme :

\begin{Def}[acculturation]
    « l’assimilation des éléments culturels d’un milieu déterminé dans l’optique d’exprimer la foi de manière intelligible aux personnes de cette culture »9.
\end{Def}
 Mais l’inculturation semble aller un cran plus loin.
 \paragraph{1.	Le concept d’inculturation
}

Arrupe commence par définir le concept d’inculturation : il s’agit de :
\begin{Def}[Inculturation]
    « l’incarnation de la vie et du message chrétiens dans une aire culturelle concrète, en sorte que non seulement l’expérience chrétienne s’exprime avec les éléments propres à la culture en question, mais aussi que cette même expérience devienne un principe d’inspiration, à la fois norme et force d’unification qui transforme et recrée cette culture, étant ainsi à l’origine d’une « nouvelle création ».
\end{Def}



 
 Il ne s’agit pas que d’exprimer le message chrétien dans une autre culture, mais que ceci vienne transformer cette culture. L’action va au-delà de la simple annonce, elle est transformante.
Pour Arrupe, elle est aujourd’hui nécessaire et ce, de manière universelle (y compris dans les pays de tradition chrétienne : il parle ainsi de réinculturation de la foi). Il voit trois apports principaux de cette inculturation : elle vient donner une nouvelle cohésion à la culture à laquelle elle s’adresse, elle vient aider à assimiler les valeurs universelles, elle vient favoriser la communion entre les cultures.

\paragraph{2.	Dans la tradition ignacienne
}

Arrupe s’attache alors à inscrire cette notion d’inculturation dans la lignée de Saint Ignace : si le terme n’existe pas encore à son époque, ce qu’il désigne est cependant contenu dans ses écrits et les Constitutions. Ainsi, les Exercices soulignent les notions de dialogue, de solidarité, et d’universalité ; ils viennent normalement aider à discerner ce qui est essentiel dans la foi chrétienne et ce qui n’est qu’accessoire ou accidentel. Il insiste sur le fait qu’il faut tenir compte des circonstances et de la culture à qui l’on s’adresse. Arrupe souligne que c’est ce principe qu’ont mis en œuvre les grandes figures missionnaires jésuites. Il s’agit alors de continuer dans cette lignée aujourd’hui, pour « évangéliser les cultures ».

\paragraph{3.	Modalités de l’inculturation
}

Comment le faire efficacement ? Plusieurs facteurs peuvent aider : être à l’écoute de l’Esprit Saint dans la prière, sans idée préconçue ; se mettre dans une attitude de discernement (pour être dans un juste don et une juste réception de ce que l’autre peut me transmettre) ; faire preuve d’ouverture intérieure et d’humilité, pour reconnaitre aussi ses erreurs ; faire preuve de patience dans la recherche des semences du Verbe présentes en d’autres cultures (cf. saint Justin et saint Clément d’Alexandrie: à tous les hommes, et donc dans toutes les cultures, ont pu parvenir des fragments de la vérité); témoigner d’une charité discrète, équilibre entre l’audace de l’annonce et la prudence. Enfin, cela ne peut se faire sans l’Eglise.
Cette nécessaire inculturation « vers l’extérieur » est donc aussi le fruit d’une attitude intérieure, personnelle et au sein de la Compagnie, en profondeur. Cela doit passer par une expérience personnelle, par cette rencontre concrète de l’autre, pour aider à prendre conscience de sa différence. Ainsi, on fait l’expérience d’une tension entre le particulier et l’universel : la culture de mon peuple et la disponibilité à toute autre culture ; tension similaire à celle existante entre Eglise particulière, locale, et Eglise universelle. Alors, la Compagnie de Jésus pourra mieux servir l’Evangélisation du monde entier, en permettant une communion qui soit respectueuse de l’identité de chacun.

 \paragraph{4.	Eléments d’analyse
}

\paragraph{A qui Arrupe s'adresse ?}


A travers cette lettre, on perçoit la volonté forte de Pedro Arrupe d’ancrer le concept d’inculturation, qu’il définit, dans la tradition de l’Eglise et dans celle de la Compagnie. Il s’agit de montrer que cela s’inscrit en continuité avec Saint Ignace. Ainsi, il cite les Exercices Spirituels (soit littéralement, soit en renvoyant directement à des numéros des exercices) mais aussi des lettres de Saint Ignace ; il évoque les grands missionnaires de la Compagnie (François Xavier, Ricci, Nobili) pour montrer qu’eux aussi ont vécu ce concept d’inculturation. Puis il cite Paul VI à travers Evangelii Nuntiandi et des propos qu’il a tenus à la Congrégation générale. Ceci lui permet de montrer la conformité à la Tradition de l’Eglise –en filigrane on peut y lire une allusion au vœu spécial d’obéissance au pape que font les jésuites. Il ne manque pas de faire référence au concile Vatican II. Les concepts eux-mêmes, tels celui du \textbf{discernement} et celui de l’\textbf{indifférence} spirituelle, reprennent des notions chères à saint Ignace, tout en étant en quelque sorte réorientés différemment pour asseoir le propos d’Arrupe. Il fait en quelque sorte déjà un acte d’inculturation dans cette lettre.
Quelques citations bibliques sont à l’appui du propos d’Arrupe, dont une particulièrement mise en évidence puisque placée à la fin de la lettre : Jn 11, 52. Notons qu’elle est citée complètement hors contexte : le texte évoque la mort du Christ non pour la seule nation juive, mais pour rassembler dans l’unité « tous les enfants de Dieu qui sont dispersés ». Certes, elle insiste sur l’universalité malgré la dispersion mais le rapport à l’inculturation n’est pas direct. Ce n’est qu’en extrapolant qu’on peut y voir la diversité culturelle. De manière générale, la majorité des citations évoque ou l’universalité de l’Evangile ou la diversité des cultures. Un des apports de Pedro Arrupe est de rapprocher ces deux blocs pour en extraire le concept d’inculturation.
\begin{Prop}[rapport universel Particulier]
  Théologiquement, la définition donnée par Arrupe au terme d’inculturation ici est significative : en employant le terme d’incarnation, il l’associe directement à l’Incarnation du Christ ; qui manifeste particulièrement ce rapport entre l’universel et le particulier…  
\end{Prop}

Enfin, les deux textes étudiés ont de nombreux points communs : insistance sur la réciprocité entre Eglise et culture, sur ce que chacune peut recevoir de l’autre, sur la nécessaire adaptation du message de l’Evangile à la culture à laquelle on veut l’adresser ; le tout à partir de références notamment historiques. On perçoit cependant quelques différences : les destinataires sont différents et donc l’enjeu de chacun des documents diffère ; le premier est un discours destiné à être largement diffusé, y compris à un auditoire non chrétien, tandis que le second s’adresse aux jésuites et vise à montrer que l’inculturation est bien dans la tradition de la compagnie, ainsi qu’à donner quelques éléments concrets pour bien la vivre.


\section{et aujourd'hui}

\paragraph{en 1960, encore des cultures géographiques stables}

\paragraph{aujourd'hui, nivellement par la globalisation} avec la confrontation des cultures, plus forts. Radicalité soit une indifférence.

\paragraph{173 semina verbis} aller chercher les semences du verbe. Quels critères ? catholicité vs plaquage d'une certaine forme de confrontation de la foi. 


\paragraph{Reception du message et de l'Evangile}

\paragraph{Question de l'humaniste athée} pour être vriament homme, il faut s'émanciper. "air de l'époque" : Pour Arrupe, l'idéal de l'homme n'est pas à la hauteur de l'homme. Il définit le mot culture : 
\begin{Def}[culture]
    La culture est pour l'homme l'idéal de perfection humaine à laquelle il aspire dans tout son être individuel et social.
\end{Def}
N'est ce pas la religion, une tension vers le transcendant ?
\begin{Ex}
    L’homme est créé
pour louer, respecter et servir Dieu notre Seigneur
et par là sauver son âme,
et les autres choses sur la face de la terre
sont créées pour l’homme,
et pour l’aider dans la poursuite de la fin
pour laquelle il est créé.
\end{Ex}

A la différence de \textit{Lubac} avec l'histoire, Arrupe remplace par la \textit{culture}. Il sort du projet des lumières, de la raison. Il prend le contrepieds. 

\paragraph{La culture a un aspect d'intégration} Teilhard de Chardin. Intègre toutes les dimensions de la vie de l'homme. Pape François. Dimension salvifique de la culture.
Dans certaines dictatures, on attaque la culture. 


7 déc 1965. Texte de Paul VI. Grand texte à lire.
