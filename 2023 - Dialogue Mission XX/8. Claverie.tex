\chapter{Discours de clôture de Mgr Moulins-Beaufort }

 
\mn{Discours de clôture de l’Assemblée plénière d’automne de la Conférence des
évêques de France
le mercredi 8 novembre 2023}

 je suis heureux de récapituler le chemin parcouru pendant cette Assemblée plénière d’automne des évêques de France, afin d’en garder trace et de rappeler les décisions et les engagements que les évêques ont pu prendre, et de vous rendre compte du labeur des six jours écoulés.
Il me semble pouvoir dire que ces jours nous ont tournés vers l’espérance ; l’espérance comme vertu théologale qui vient de Dieu et qui nous unit à lui. Nous étions arrivés à Lourdes avec la belle mémoire des Journées Mondiales de la Jeunesse de Lisbonne, des Rencontres Méditerranéennes tenues à Marseille et de la visite du pape François qui les a conclues, du rassemblement Kérygma et de quelques autres moments intenses où nous nous étions retrouvés, nous évêques, fortifiés et encouragés par la joie du peuple chrétien. Nous étions aussi graves, marqués par les douleurs et les inquiétudes du moment de l’histoire que nous vivons tous.
Dès le début de notre rencontre, nous avons été conduits par le cardinal Bustillo, dans le temps spirituel de vendredi matin, à passer des gémissements au tressaillement pour aller jusqu’à l’exultation et la bénédiction. Les raisons de gémir sont nombreuses et de tous ordres, et il faut tout un travail pour s’ouvrir au « tressaillement », selon le mot du pape François au stade Vélodrome. 


\paragraph{Tressaillement} devant Dieu qui s’approche, qui nous rejoint dans notre humanité blessée, divisée, souffrante, et qui vient prendre chair en nous avec toute sa bonté. Mais nous sommes appelés plus loin encore, jusqu’à l’exultation qui monte en nous lorsque nous reconnaissons l’action de Dieu ; action libératrice, action réconciliatrice, action vivifiante, celle que nous chantons dans l’Exultet de la Vigile pascale ou en nous unissant au Magnificat de Marie. 
\begin{singlequote}
    « Dans la nuit, ton peuple s’avance, libre et vainqueur ! »
\end{singlequote}, chantons- nous dans la Vigile. La nuit n’est pas dissipée par miracle, les ténèbres demeurent, mais il est permis au peuple de Dieu d’avancer et de s’éprouver libre et vainqueur, non par lui-même, non par ses propres forces, mais par la force de Dieu, déployée dans le Christ Jésus, mort et ressuscité.

C’est cela, l’espérance chrétienne : non pas une prolongation optimiste des tendances déjà repérables, ou le rêve que les obstacles vont se dissiper miraculeusement ; mais la découverte que le Dieu vivant agit en faveur des humains et qu’aucune ténèbre, quoi qu’il en soit des apparences, ne peut l’empêcher de nous rejoindre et de nous tirer vers sa plénitude. Et l’aboutissement, pour le peuple de Dieu, n’est pas qu’il devienne un peuple dominant, captant pour lui toutes les richesses de ce monde, mais qu’il devienne, malgré le péché qui œuvre toujours en ses membres et dans le monde, un peuple porteur de la bénédiction de Dieu pour tous et source de bénédiction en faveur de tous.
Je me permets de suggérer que nous avons été accompagnés dans cette marche vers l’espérance, plus secrètement, par saint Paul, dont la liturgie de la Messe nous a fait entendre chaque jour, sauf dimanche, un morceau de la lettre aux Romains.
 
Dès vendredi, nous avons reçu le cri de l’Apôtre : \begin{singlequote}
    « J’ai dans le cœur une grande tristesse, une douleur incessante. Moi-même, pour les Juifs, mes frères de race, je souhaiterais être anathème, séparé du Christ » (Rm 9, 2-3)
\end{singlequote} cri qui prenait une résonnance toute particulière après les atroces attaques terroristes commises par le Hamas dans le Sud d’Israël et la capture de 240 personnes emmenées comme otages. Saint Paul, bien sûr, n’écrivait pas cela pour commenter la situation d’Israël aujourd’hui. Dans la longue réflexion sur l’œuvre de salut du Dieu vivant qu’il écrit pour les Romains et pour tous ceux et celles qui mettent leur foi dans le Christ Jésus, l’Apôtre se heurte à un fait massif : le refus non pas de tous, car lui-même est juif, comme Marie et les autres Apôtres et de nombreux autres, mais de la plus grande partie d’Israël, de reconnaître en Jésus crucifié et ressuscité le Messie promis ; et dans le don de l’Esprit-Saint aux païens, la réalisation des promesses de l’Alliance. Saint Paul constate ce qui paraît mettre en échec sa prédication, mais il y reconnaît, lui, la logique de l’action de Dieu, qui travaille pour le salut de l’humanité entière et qui, comme lui, Paul, ose l’écrire : « enferme tous les hommes dans le refus de croire, pour faire à tous miséricorde » ; comme nous l’avons entendu lundi matin.


Au lieu de voir dans cette résistance un échec, ou tout au moins, une limite au succès de la prédication de l’Évangile ou du fameux « kérygme » qu’est l’annonce de la victoire de Jésus, le Crucifié ressuscité, saint Paul y reconnaît la logique profonde de l’action du Dieu vivant, et l’espérance d’un salut qui puisse englober tous les êtres humains, même les plus éloignés et les plus endurcis. Il tire son espérance, non pas de l’espoir d’un retournement miraculeux des esprits, ni d’un renversement inespéré des tendances constatées, mais de sa contemplation de l’œuvre de Dieu dans l’histoire de son peuple, qui surmonte toutes les résistances en les dépassant par un surcroît de miséricorde. Dieu n’est pas vainqueur au long de l’histoire parce qu’il userait les réticences des Hommes, mais parce qu’il s’abaisse plus profondément, parce qu’il tire de lui-même des profondeurs de pardon et de vie toujours inattendues et inespérées et qu’il suscite, par-là, chez les êtres humains, dans les profondeurs de la liberté, des désirs de vie et d’amour eux aussi inattendus et inespérés. D’une compréhension plus juste de cet enseignement de l’apôtre saint Paul est venue, nous le savons, la profonde révision que l’Église a menée de sa pensée et de son action concernant Israël en toutes ses dimensions.

\paragraph{une théologie de l'histoire}
Il en résulte que l’histoire de l’humanité ne conduit pas seulement au futur par la prolongation des tendances que l’on peut constater et/ou analyser dans les psychologies humaines ; dans les réalités économiques, sociales, politiques, culturelles ou climatiques. L’histoire conduit vers l’avenir qui est l’avènement de ce que Dieu seul peut produire et qui échappe à nos prises. Dieu seul soulève ce
« couvercle bas et lourd » qui pèse sur nos existences, et perce vers la communion éternelle avec lui et en lui avec tous. Lui seul fait que l’histoire humaine est une destinée spirituelle qui avance vers une plénitude que les humains ne peuvent se donner, mais qu’ils sont faits pour recevoir, dans laquelle ils sont faits pour entrer.
Et c’est ainsi qu’au terme du chapitre 11 de sa lettre, l’apôtre s’exclame, nous avons entendu cela lundi : « Quelle profondeur dans la richesse, la sagesse et la connaissance de Dieu ! Ses décisions sont insondables, ses chemins sont impénétrables ! Qui a connu la pensée du Seigneur ? Qui a été son conseiller ? Qui lui a donné en premier et mériterait de recevoir en retour ? Car tout est de lui, et par lui, et pour lui. À lui la gloire pour l’éternité ! Amen. »


\paragraph{mission de l'Eglise}
L’essentiel de notre travail pendant cette assemblée a consisté à approfondir notre compréhension de la mission de l’Église. Ce travail s’inscrit dans cette perspective de l’espérance. Nous sommes envoyés : mission vient du verbe latin qui signifie envoyer, comme le Fils est envoyé par le Père ; comme l’Esprit- Saint est envoyé dans le monde. Parler de mission, reconnaître que l’Église est et doit être
missionnaire, n’est pas se motiver pour relancer une campagne d’adhésion qui se serait essoufflée. C’est entrer plus avant dans l’œuvre de Dieu, collaborer à l’œuvre de Dieu, qui ne laisse pas l’histoire de l’humanité glisser vers le néant, mais la tire vers sa vie à lui, pour une plénitude qui n’aura pas de fin. La résistance rencontrée, l’indifférence même à laquelle l’annonce de la bonne nouvelle peut être confrontée, ne signifient pas que Dieu aurait cessé de travailler ; que l’Esprit-Saint se serait mis en repos ; que le Ressuscité n’attirerait plus à lui toutes choses. Mais l’histoire connaît des phases ; des phases d’adhésion et des phases de refus ; des phases de diffusion et des phases de retrait apparent ; parce que Dieu veut atteindre l’humanité en toute son extension et en toutes ses profondeurs ; dévoilant le péché, c’est-à-dire la capacité de refus, pour mieux nous en libérer et pour ouvrir en l’humanité et en chacune et chacun, des capacités de don insoupçonnées.
Nous souhaitons continuer dans d’autres assemblées notre travail sur ce thème dans d’autres assemblées ; travail théologique et travail pastoral. Nous voulons mieux comprendre ce qu’est l’ « acte intégral d’évangélisation » ; mieux déterminer comment les différentes dimensions de cet acte global, peuvent se composer aujourd’hui, dans le temps spirituel où nous sommes, et le déterminer ensemble pour mieux le servir. Nous accueillons avec émerveillement dans nos diocèses, les catéchumènes et les recommençants. En chacune ou chacun d’eux, nous reconnaissons l’œuvre totale de Dieu se déployant depuis le plus intime de la Trinité sainte pour atteindre la profondeur d’une liberté humaine et lui ouvrir un chemin inespéré. Nous avons déjà compris qu’il était capital que ces candidats au baptême soit accueillis dans des fraternités qui les initient à la vie chrétienne, plutôt que seulement accompagnés par des équipes qui les préparent aux sacrements de l’initiation ; car il s’agit pour eux et pour elles de se laisser engendrer à une vie nouvelle, et non pas tant d’acquérir quelques idées différentes, ou quelques exigences morales de plus. Rien ne s’oppose en cela, mais une vue plus large et plus juste, nous fera mieux servir l’œuvre de Dieu. Nous rendons grâce pour les initiatives missionnaires nombreuses, variées, foisonnantes, que nous avons pu recenser ; et nous nous émerveillons de la générosité de tant de chrétiens, heureux de donner gratuitement ce qu’ils ont reçu gratuitement. Nous voulons ensemble identifier des critères plus précis pour que le zèle missionnaire soit en tout et partout le zèle pour le Dieu vivant, et non une volonté de conquête ou de reconquête.

\paragraph{dialogue avec les musulmans}
Nous avons travaillé aussi, en deux séquences, sur nos relations avec les Musulmans, l’occasion en étant le cinquantième anniversaire du service national de la CEF chargé de ce thème. Cette réflexion s’articulait bien (et le magistère nous y invite lui-même) à nos réflexions sur la mission. En effet, l’Église catholique situe le dialogue non pas en opposition à l’annonce de l’Évangile, mais comme l’expression habituelle de toute relation engagée par les disciples de Jésus avec leurs contemporains ; une relation mue par un intérêt gratuit pour autrui, dans l’infini respect des consciences ; et qui sait pourtant dire la fierté d’appartenir au Christ.
Le Service National pour les Relations avec les Musulmans avait commencé sous le nom de « SRI » :
« service des relations avec l’islam », pour aider à l’accueil des immigrés musulmans en France. C’était alors un devoir de justice sociale et un devoir religieux. Il s’agissait de veiller au respect pour eux du droit à la liberté religieuse, et d’aider à ce que son exercice effectif soit possible ; de leur manifester aussi, à leur arrivée en France, respect et amitié. L’histoire a avancé, et la présence des musulmans en France s’est transformée. Le service aussi.

Il reste toujours un devoir de justice et un devoir religieux : contribuer à ce que nos concitoyens musulmans puissent exercer tous leurs droits de citoyens ; veiller avec eux au respect du droit à la liberté religieuse ; les aider à faire confiance au cadre républicain et laïc de notre pays ; mais aussi les connaître dans la diversité de leurs traditions culturelles et spirituelles ; recevoir leur témoignage de croyants et leur apporter le nôtre, dans la perspective d’une fraternité toujours plus réelle. Nous espérons contribuer à former une seule humanité, appelée à vivre dans la communion éternelle avec
 Dieu et entre nous ; et ce qui nous distingue et nous sépare en cela nous encourage, nous chrétiens, à chercher à en vivre davantage nous-mêmes, pour mieux servir la destinée spirituelle de l’humanité entière.

 
Nous nous sommes rappelés les uns aux autres ce que cette attitude supposait de respect, d’attention, de travail de connaissance, et aussi de liberté intérieure de notre part ; ainsi que notre disponibilité pour des amitiés réelles, c’est-à-dire aussi gratuites ou désintéressées. Nous portons pour les musulmans comme pour tout être humain, l’espérance qu’ils découvrent la richesse et la profondeur de l’amour de Dieu ; comme nous la portons, nous portons cette espérance, pour nous qui savons nos aveuglements et nos réticences ; et nous recevons d’eux la force de leur expérience de croyants ; eux qui sont si capables de reconnaître la présence et l’action de Dieu dans les autres. Nous savons bien, à l’école de l’apôtre Paul, que l’espérance est tout autre chose qu’une entreprise de promotion ou d’intimidation. 
\paragraph{quelle vision pour l'islam ? } Entre une vision minimaliste qui ne verrait dans l’islam qu’une erreur ou une hérésie, et une vision maximaliste qui y verrait une voie de salut parmi d’autres, notre réflexion nous a poussés à tendre à une relation avec les musulmans plutôt inspirée par ce « chemin supérieur à tous les autres » qu’évoque saint Paul (1ère Lettre aux Corinthiens 12,31) : celui de la charité, cet amour que les chrétiens contemplent sur le visage du Christ et qui est répandu en nos cœurs par le Saint-Esprit. C’est dans le décentrement de soi, c’est dans le don désintéressé de soi, que se trouve la vie de Dieu et la vie en Dieu ; et nous croyons, nous, que Jésus et Jésus seul nous le rend possible en vérité, en toute justice, malgré la force du péché en nous et autour de nous ; et cela parce que Lui, Jésus est l’envoyé du Père, l’engagement ultime de Dieu en notre histoire humaine ; venu non pour s’y perdre, mais pour nous y attirer vers le Père, dans la gloire de sa filiation unique.
L’espérance s’accorde aussi avec la patience, le respect du temps de Dieu et du temps de chaque liberté, et elle passe par l’amour sincère de l’amitié. Nous avons évoqué les personnes venues de l’islam qui découvrent le Christ Seigneur et décident de le suivre, et nous nous réjouissons que de tels chemins soient possibles dans notre pays. Nous avons mieux pris conscience de notre responsabilité de veiller à leur accueil et à leur accompagnement. En ces personnes, nous voyons la puissance de l’appel du Christ à l’œuvre, et nous admirons leur courage spirituel. Nous avons été relancés dans notre devoir d’appeler et de former des acteurs de tous ordres pour le dialogue et la rencontre. En votre nom, je remercie le P. Jean-François Bour, responsable du Service national, M. Younès qui a aidé à notre séquence de travail, professeur de l’Université catholique de Lyon, les collaborateurs et collaboratrices réguliers ou occasionnels du service, et aussi nos délégués aux relations avec les musulmans et toutes les personnes qui y travaillent avec nous et pour nous.

\paragraph{victimes}
En cette session, nous avons entendu le point d’étape du groupe de travail sur l’accompagnement que nous devons et pouvons apporter aux personnes qui auraient été victimes à l’âge adulte de violences ou agressions sexuelles de la part de prêtres, ou dans un cadre ecclésial. Nous avons écouté avec émotion une personne victime venue nous témoigner ce qu’elle avait vécu. Elle s’est faite le porte-voix des autres. En l’entendant, nous avons renouvelé l’expérience qui avait été si décisive pour nous il y a deux ans, lorsque nous avions entendu des personnes ayant été agressées alors qu’elles étaient mineures, qui avaient accepté de venir nous parler et travailler avec nous. Par son témoignage et par ses réflexions, cette femme, à son tour, nous a décrit un chemin de progression pour l’Église. Car s’il est terrifiant de découvrir comment une relation pastorale, qui devrait être une relation de vie, peut être détournée et pervertie pour devenir une relation d’emprise qui prive la personne victime de sa liberté de pensée et de jugement ; la description de ce phénomène montre aussi qu’il peut en être autrement. Au nom de tous les évêques, je remercie Mme Corinne Boiley qui a accepté de conduire ce groupe de travail ; je remercie la personne qui est venue avec elle passer deux jours ici avec nous ; je remercie les autres membres de ce groupe de travail, et ceux et celles qui ont été ou seront
auditionnés. Nous suivons avec attention les réflexions juridiques en cours, en vue de mieux définir l’emprise, et d’en établir les critères. L’aide de la justice pénale de l’État nous paraît essentielle, car elle dispose de moyens d’enquête et d’évaluation que la justice canonique ne peut ni ne pourra réunir, dont il est meilleur même qu’elle ne prétende pas les réunir un jour. Évêques, nous avons à veiller à ce que la tradition chrétienne de l’éducation au renoncement à soi, ne soit pas confondue avec l’emprise, et ne le garantirons d’autant mieux que nous serons conscients que la perversion de l’emprise peut s’y glisser. Un effort a été mené ces derniers mois auprès des associations de fidèles menant la vie commune, pour qu’elles confient à la Commission Reconnaissance et Réparation de la CORREF (Conférence des Religieuses et Religieux de France) le soin d’accueillir les personnes qui auraient été victimes en leur sein, et de les accompagner sur un chemin de réparation ou qu’elles se dotent des moyens de le faire d’une manière respectueuse de ces personnes et de la justice qui leur est due. Nous avons mesuré aussi l’importance du travail commencé en vue d’établir une charte de la relation d’accompagnement spirituel. En mars dernier, notre assemblée a demandé aux conseils presbytéraux dans les diocèses d’étudier le rapport du groupe de travail à ce sujet et d’y réagir. D’autres personnes sont concernées, car l’accompagnement spirituel n’est pas un monopole des prêtres. Il conviendra d’associer ces autres personnes, et de faire aboutir cette recommandation d’ici mars 2025, selon le délai prévu.
Puis-je dire que nous avons abordé ce thème douloureux avec crainte, et cependant dans la même perspective d’espérance que le reste ? Car l’apôtre, dans les chapitres 6 à 8 de sa lettre, montre bien que l’espérance n’oublie pas la gravité du péché et du mal commis ; ni ne consiste à se persuader que le péché peut être surmonté facilement. Au contraire : saint Paul exprime avec force la prégnance du péché dans la liberté humaine et combien le péché peut être porteur de mort. Mais la grâce du Christ et la puissance de l’Esprit-Saint qui nous est donné, font qu’il est possible et qu’il vaut la peine de nommer le mal pour ce qu’il est ; de faire la vérité et la justice ; et de chercher à avancer vers des relations nouvelles.
Nous avons pu ainsi dans cette session confirmer notre engagement dans la reconnaissance et la réparation que nous devons et voulons apporter à ceux et celles qui ont été victimes dans l’Église. Dans ce travail de vérité et de justice, nous reconnaissons l’œuvre de Dieu qui sanctifie l’Église. Il nous conduit à ajuster toujours mieux nos relations pour qu’elles soient dignes de lui, dignes du Seigneur Jésus, vraiment au service de la vie dans l’Esprit-Saint. Plus largement, c’est à cette liberté spirituelle qu’il nous faut nous éduquer nous-mêmes et aider les prêtres, les diacres, toutes les personnes ayant une autorité dans l’Église et tous les membres de l’Église, tous les fidèles, à grandir.
\paragraph{acteurs de la mission}
Car la mission suppose des acteurs, l’envoi suppose des envoyés. Notre réflexion à ce sujet trouve aussi un encouragement dans la lecture de la lettre aux Romains. Hier mardi, le passage lu, tiré du chapitre 12, dans lequel saint Paul exhorte les chrétiens à assumer pleinement leur responsabilité spirituelle en menant leurs vies avec intensité, devant Dieu et devant les hommes, dans un combat spirituel exigeant et joyeux, ce passage, donc chapitre 12, nous a fait entendre un appel étonnant : « Nous qui sommes plusieurs, nous sommes un seul corps dans le Christ, et membres les uns des autres, chacun pour sa part. » L’apôtre reprend la comparaison du corps, dans lequel chaque membre est différent des autres et a une fonction qui le distingue et qui est nécessaire à tous ; mais il la subvertit, il la transforme lorsqu’il ose formuler que « nous sommes membres les uns des autres ». Il désigne ainsi dans la communion de l’Église une appartenance mutuelle qui dépasse la simple appartenance commune ; une hospitalité réciproque dont nous ne mesurons pas toujours les conséquences.

 \paragraph{synode sur la synodalité}
Cette formule de saint Paul contient une forte lumière pour comprendre ce qui est en jeu dans la synodalité de l’Église, et l’effort présent pour la déployer et l’exercer plus pleinement. Nous avons entendu avec gratitude le récit que les membres français de l’assemblée synodale sur la synodalité nous ont fait. Ils étaient quatre évêques élus : Mgr Alexandre Joly, Mgr Matthieu Rougé, Mgr Benoît Bertrand, Mgr Jean-Marc Eychenne ; le cardinal Jean-Marc Aveline et Mme Anne Ferrand, consacrée du diocèse de Rodez, ayant été choisis par le Pape. Chacun d’eux nous a partagé, de manière vivante, ses découvertes, ses surprises, ses questions, ses attentes. Un des défis des mois qui viennent sera sans doute d’impliquer davantage nos Églises particulières dans le processus synodal, et d’intéresser plus résolument les jeunes, les pauvres, les prêtres : trois catégories de fidèles qui ont parfois eu des raisons de ne pas s’y considérer invités. Nous retenons aussi la richesse de la démarche de la
« conversation dans l’Esprit », certainement transposable en beaucoup d’instances de notre vie ecclésiale ; et aussi la nécessité de coordonner justement doctrine, ou plutôt la réflexion théologique, et le partage d’expérience. En tout cas, nous pourrions retenir de saint Paul que la synodalité ne saurait être la simple juxtaposition des différentes catégories de fidèles, mais bien l’expression de l’appartenance réciproque que nous recevons du Christ Jésus, qui nous donne les uns aux autres et nous rend responsables les uns des autres et tous ensemble, d’agir pour que l’unité de l’Église soit une unité de bénédictions échangées.


Dans cette perspective aussi est venu notre travail sur les ministères laïcs. Nous recevons de l’évêque de Rome une forte impulsion à reconnaître les charismes des uns et des autres et à instituer des fidèles pour le bien de tout le corps. Les modalités concrètes nous échappent encore un peu. Plus d’expérience sera nécessaire, sans doute, pour stabiliser des ministères tout en les voyant s’exercer dans des offices diversifiés. Cependant, l’essentiel est que l’Église suscitée par le Christ Jésus grandisse comme un corps qui n’est pas soudé par la force ou la contrainte, mais par la foi, l’espérance et la charité ou, pour le dire autrement, par le combat spirituel et la conscience d’être envoyés dans ce monde pour partager à tous les richesses reçues de l’Alliance avec eux, celle d’Israël et de son Messie élargie par pure grâce à tous êtres les humains.

Dans ce même esprit d’appartenance mutuelle et d’espérance, nous avons réfléchi à l’avenir de nos diocèses. Nous sommes conscients que le Christ a institué le ministère apostolique pour que son Église soit un corps vivant, mais non pas la carte des diocèses, pas plus que celle des paroisses. Quelles forces humaines, spirituelles et aussi matérielles faut-il pouvoir mobiliser, pour rendre témoignage au Christ et à sa Résurrection ? En portant cette question encore, nous avons voté les nouveaux statuts de notre Conférence, revus en fonction de la transformation que nous avons décidée en mars dernier au terme d’un processus de deux ans. Le travail de description de la nouvelle organisation en pôles va pouvoir se poursuivre, et nous espérons la mettre en place l’été prochain. Les évêques actuellement en charge d’un Conseil aideront à envisager les futures manières de travailler et d’agir. Des choix devront être faits. Des thématiques devront être assumées autrement, si possible avec plus de transversalité et avec plus de circulation entre les équipes des pôles et le Conseil permanent de notre conférence. Mais les évêques ne pourront le faire que grâce aux directeurs et directrices des services nationaux, et à tous les collaboratrices et collaborateurs de la Maison de la Conférence. Nous avons profité à nouveau, en cette session, de leurs compétences et de leur disponibilité. Je les remercie en votre nom à tous.
\paragraph{ Le plein accomplissement de la loi, c’est l’amour}

Ce matin même, saint Paul nous a relancés encore, lui qui écrit au chapitre 13 de sa lettre aux Romains :
« Le plein accomplissement de la loi, c’est l’amour. » L’amour n’est certes pas qu’un sentiment ou une émotion. Il est l’engagement de chacun ou de chacune vers le bien, ainsi que l’affirmait Mgr Olivier Leborgne dans son homélie pour les obsèques de Monsieur Dominique Bernard, ce professeur
assassiné à Arras : \begin{singlequote}
    « L’amour est une détermination de la liberté qui s’engage pour le bien de l’autre, jusqu’au bien de tous et de la cité ».
\end{singlequote}
Parce que nous en sommes convaincus, parce que nous manquerions à notre mission si nous n’appelions encore et toujours les hommes et les femmes vers qui nous sommes envoyés à vivre cette beauté, cette exigence, cette vérité, de l’amour ; dans la conscience des graves défis de notre temps ; des épreuves ordinaires que vivent beaucoup d’hommes et de femmes ; des risques considérables qui pèsent sur l’humanité du fait de la quantité d’énergie dont elle a depuis des décennies besoin pour vivre, et de la quantité de déchets qu’elle produit ; en portant dans notre prière ceux et celles qui, d’une manière ou d’une autre, souffrent violence, nous voulons, au terme de notre assemblée, lancer une triple série d’appels :
\begin{itemize}
    \item 	des appels à notre pays d’abord, et à nos sociétés occidentales. D’abord, pour dire notre inquiétude à l’idée que la liberté d’avorter puisse être inscrite dans la Constitution. Alors même que nous recevons de notre foi que l’altérité entre hommes et femmes est une richesse de notre humanité, et un signe de l’altérité de Dieu qui nous appelle tous ; nous appelons de tout cœur à ce que les droits des femmes soient mieux garantis et mieux promus ; à ce qu’une réelle égalité civile et sociale leur soit assurée ; à ce qu’elles soient mieux protégées des violences que les hommes peuvent exercer sur elles. Mais l’avortement, dont la décision est rarement un choix de pleine liberté, ne peut être compris sous le seul prisme des droits des femmes. Nos sociétés peuvent mieux promouvoir le respect mutuel des hommes et des femmes ; l’éducation à la sexualité et l’intégration de la sexualité dans la masculinité et la féminité ; la responsabilité de tous à l’égard de l’enfant à naître.
        \item 
Ensuite, pour redire notre inquiétude non moins grande devant le projet de loi en préparation concernant la fin de vie. Une société humaine doit être fraternelle pour tous et pour toutes jusqu’à la fin de la vie, et promouvoir l’aide active à vivre tout en se gardant d’un « certain goût pour la mort » inhérent à notre humanité marquée par le péché.
            \item 
Enfin, nous lançons un appel pour lancer un sursaut d’humanité face au fait des migrations. Nous voulons relayer la voix du pape François à Marseille qui a su si fortement toucher les cœurs et les intelligences, en partageant son angoisse de sentir nos sociétés s’endurcir et se fermer à la compassion et à la fraternité. Nous pouvons, en France, encore, recevoir comme des frères et sœurs en humanité ceux et celles qui viennent chez nous dans l’espoir d’une vie meilleure pour eux ou leurs enfants, en accueillant leurs talents et leurs énergies ; cette attitude permettant, bien mieux que l’illusion d’empêcher toute migration, de fixer des règles ; d’exiger le respect de nos lois et de notre équilibre social et culturel ; et de travailler avec les pays de départ, pour que puissent y rester et y trouver de quoi vivre dignement le plus possible de leurs citoyens.
\item 
-	autre série d’appels : un appel aux jeunes ensuite, et un encouragement aux prêtres et aux diacres. Aux jeunes, d’abord. Nous les avons accompagnés aux JMJ. Nous avons reçu d’eux un formidable encouragement. Nous avons constaté leur soif de connaître et de suivre le Christ ; de hisser leur vie sur la crête de la suite de Jésus ; nous avons découvert leur sérieux, leur détermination à assurer les grands défis qui se présentent à leurs générations. Ces jeunes sont très divers, mais le futur leur appartient et ils veulent y vivre un avenir pour eux et pour tous les autres. Le message que nous avons publié hier veut les encourager à vivre leur vie comme une grande aventure spirituelle. Une séquence nous a fait revenir sur les émeutes du mois de juin et sur la place de certains jeunes, voire très jeunes, dans ces événements. Nous savons que la vie est rude dans certains quartiers et pour certaines familles ; surtout pour les femmes qui élèvent seules leurs enfants. Mme Véronique Devise, présidente du Secours catholique, nous a montré comme la pauvreté augmente dans notre pays ; tandis que M. Janvier Hongla, a illustré les discriminations que subissent les jeunes originaires même lointainement
de l’immigration. Il nous a aussi mieux fait comprendre la situation particulière des jeunes catholiques issus de l’immigration ou venus des Outre-mers dans nos villes. Nous remercions Mme Devise et M. Hongla de nous avoir rejoints ici. Nous avons salué les efforts pastoraux qui sont faits en certains de ces lieux, et dit notre reconnaissance à celles et ceux, prêtres, religieux, religieuses, laïcs, qui s’y engagent. Nous aimerions pouvoir en envoyer davantage. A côté des travailleurs sociaux, ils entretiennent des relations qui contribuent à l’amitié sociale et peuvent être des points d’appui pour des familles et des jeunes.
Au long de ces jours, nous avons souvent évoqué les prêtres, nos premiers collaborateurs. Sans eux, la mission ne serait pas celle du Christ, celle du Fils envoyé par le Père et qui nous envoie à son tour. Si, dimanche, nous avons entendu dans les lectures, le prophète Malachie apostropher durement les prêtres de Jérusalem ; et le Seigneur Jésus demander qu’on ne se fasse pas donner le titre de « rabbi », ni qu’on appelle quiconque « père » ; c’était pour que même ces prêtres-là le soient davantage, avec plus de justesse ; et pour que ceux qui ont autorité l’exercent vraiment comme lui, le Maître et le Fils, le fait. Car le Seigneur lui-même se rend proche des humains. Par les ministres ordonnés, évêques, prêtres et diacres, il rapproche sa présence et son action, pour que, de celles-ci, tout croyant puisse tirer lumière, force et paix. Je crois traduire ce que chacun des évêques porte en lui en remerciant les prêtres de leur engagement, et en les encourageant à mener leur mission avec toute leur intelligence, toute leur force, tout leur esprit. A l’approche des 60 ans de la restauration du diaconat permanent, nous saluons aussi les diacres de nos diocèses ; nous remercions les épouses et les enfants de nos diacres, de leur permettre d’exercer le ministère au nom du Christ. Nous nous réjouissons aussi du prochain rassemblement de 700 séminaristes qui aura lieu à Paris début décembre, le premier depuis 2009. Nous voulons dire aux jeunes hommes que le ministère de prêtre est beau et remplit une vie ; et aux jeunes hommes et aux jeunes femmes, que la consécration au Christ est un signe d’espérance donné à tous.
\item 	enfin nous voulons lancer d’ici un appel à la paix. La paix souffre violence. La lettre aux Romains nous aide à comprendre que la paix vient de la réconciliation de ceux qui pourraient se haïr ; de ceux et celles qui auraient des raisons de se méfier les uns des autres. Nous le disons avec humilité, et aussi avec force, comme Français qui avons éprouvé la haine puis la réconciliation avec les Allemands, le 11 novembre tout proche nous invite à y penser à nouveau, et nous le demandons, comme chrétiens, avec toute l’intensité de la prière, pour les Ukrainiens et pour les Russes ; pour les Arméniens et les Azéris ; pour les Juifs et pour les Palestiniens ; pour bien des peuples d’Afrique et d’ailleurs, soumis à des faits de guerre ou de terrorisme. Saint Paul nous fait comprendre, à nous chrétiens, que l’amitié avec le peuple juif, peuple mis à part par Dieu, est le secret de la destinée totale de l’humanité. Nous souffrons de voir remise en cause la légitimité de l'existence d'Israël ; nous souffrons des actes terroristes du Hamas ; nous souffrons avec les otages et leurs familles. Nous avons reçu avec émotion les remerciements de certaines d'entre elles, après que nous nous soyons associés à un geste demandé aux familles juives par le Grand Rabbin, pour exprimer l’attente du retour des otages. Nous souffrons de voir une guerre brutale opposer Israël et le Hamas, faisant de nombreuses victimes civiles ; nous souffrons pour les Juifs, nos pères dans la foi, menacés par une inquiétante vague d'antisémitisme. Nous souffrons pour les Palestiniens, nos frères et nos sœurs dans l’humanité et, pour certains aussi, dans la foi ; eux qui sont fiers souvent de se présenter en descendants des premiers chrétiens, des frères et sœurs de Jésus ; nous portons dans notre supplication les morts, les blessés, les mutilés, les familles meurtries, les enfants dont la vie est brisée, traumatisée une fois de plus.
Nous appelons à la justice pour le peuple palestinien qui a droit à un État libre, maître de lui-même, et dont l’humanité entière a besoin. De même, nous demandons une reconnaissance claire, partout, du droit à exister pour l’Etat d’Israël qui est appelé, sans doute, à devenir, pour le Proche-Orient, un acteur
de progrès, de prospérité et de paix, grâce à une coopération stable avec ses voisins. Nous appelons aussi tous nos concitoyens, en France, à ne pas céder à la logique simpliste de l’affrontement entre communautés religieuses ; et nous nous élevons contre les attitudes racistes, antisémites et anti- musulmanes qu’une telle logique induirait. Nous souffrons pour l’humanité divisée, fracturée, par des conflits dont plusieurs sont dus à l’avidité de quelques-uns pour le pouvoir.
Nous condamnons toute prétention à faire la guerre au nom de Dieu, car nous entendons le prophète Isaïe, repris par l’Apôtre Paul : « J’appellerai mon peuple celui qui n’était pas mon peuple, et bien-aimée celle qui n’était pas la bien-aimée ». La destinée de l’humanité doit conduire à la fraternité de tous en Dieu. Aujourd’hui le chemin en passe par le respect du droit international et par la négociation. Au terme de ce discours, dans un bref instant donc, nous allons partir en procession déposer devant notre Dame de Lourdes, fille de Sion, femme de Palestine, trois grands cierges, symboles de notre intercession continue pour l’Ukraine, pour l’Arménie, pour Israël et pour la Palestine.
\end{itemize}




\chapter{Claverie}


\section{présentation}
\mn{INSTITUT CATHOLIQUE DE PARIS ISTR
SEMINAIRE « DIALOGUE, MISSION, INCULTURATION AUX 20ème – 21ème Siècle »
Exposé par Kabuge Albert 
Pierre CLAVERIE, Lettres et messages d’Algérie}

\subsection{La vie de Pierre Claverie}


Pour ce séminaire nous notre exposé se base sur les Lettres et messages d’Algérie. A deux nous présentons cinq lettres tirées de cette 4ème édition revue et augmentée, éditée à Karthala en 1996.  C’est une présentation qui se base sur Pierre Claverie qui est né le 08 mai 1938 à Bab-el-Oued à Alger dans une famille pied-noir présente en Algérie depuis quatre générations et mort assassiné le 1er Août 1996 à Oran. C’est un prêtre dominicain français d’Algérie, il est mort Evêque, il a été proclamé bienheureux le 8 décembre 2018. 

Il est compté parmi les martyrs d’Algérie. En poursuivant ses études, Pierre Claverie il découvre que le monde où il a grandi n’est pas parfait. Ainsi il l’appellera la Bulle coloniale. Cela nous montre dans quel contexte il a vécu, un monde de tension, d’instabilité et dans un monde musulman dans lequel beaucoup de musulmans voulaient entrer dans l’institut d’études arabe et islamique pour connaitre leur culture où il était responsable. Il avait choisi la vie religieuse et il entre au noviciat au noviciat en 1958 à Lille chez les Dominicains il a été ordonné prêtre en 1965. En 1967, il décide de rentrer en Algérie. 
En lisant sa vie, nous découvrons son aspect missionnaire : « Il était un homme de dialogue, et il a participé dans des nombreuses rencontres entre chrétiens et musulmans, il avait une parfaite connaissance de l’islam ainsi on l’appelait Evêque des musulmans, il avait un cœur tout donné pour ses destinataires algériens… » Il est nommé Evêque d’Oran, le 21 Mai 1981.
 
En 1992, la guerre civile éclate et toute la population est menacée, il refuse de quitter l’Algérie, il ne voulait pas laisser cette population qui était en difficulté. Dans cette situation de crise, il refuse de se taire devant le Front Islamique du Salut et le gouvernement algérien. Le 26 mai 1996, les moines de Tibhirine sont assassinés et le 1er Aout 1996 lui aussi est assassiné avec son ami, un jeune algérien musulman de 21 an Mohamed Bouchicki. 

Nous avons découvert beaucoup de publications, nous avons fait choix de celles qui ont attiré notre attention : 
\begin{itemize}
    \item  	Le livre de la Foi publié en 1996
\item 	Donner sa vie. Six jours de retraite sur l’Eucharistie publié en 2003. 
\item 	Humanité plurielle, présenté par sœur Anne-Catherine, 2008. 
\item 	Un amour plus fort que la mort, 2018. 
\end{itemize}

\subsection{Brève compréhension des lettre et messages : Période et contenu synthétique}
 
Ce sont des courts textes écrit en huit ans donc de l'année 1988 au mois de juillet 1996, à la veille de l'assassinat de Pierre Claverie, à Oran.  Ces Lettres et messages parlent de la crise algérienne, de l'islam, du dialogue, de la présence chrétienne en Algérie. Il n'est pas si fréquent aujourd'hui qu'on nous parle de l'évangile et de ses exigences dans une langue que nous entendons.  Pierre Claverie a osé dans ce milieu musulman de se parler pour défendre ses proches.

\subsection{Structure des deux lettres choisies}
 

Deux lettres à analyser, il s’agit de la Lettre n° 12 : Avent : qui sera le plus fort ? (p. 93-96) écrite au mois de Novembre 1990 et la Lettre n° 35 : Rester ? Partir ? (p.177-181) écrite au mois de février 1995, probablement en période de Carême. 
Nos deux lettres sont curieusement écrites avec les titres au style interrogatif et qui présentent d’emblée la faiblesse des chrétiens face à l’islam. Est-ce une manière de se venger contre les chrétiens que présentent les musulmans. La position musulmane qui prédomine. Mais pour arriver à vivre ensemble, la structure de ces lettres veut exprimer l’esprit du dialogue et de la rencontre, et selon notre appréhension cela semble un peu un échec. 

4 . La problématique : Selon notre lecture, nous découvrons dans ces articles que la préoccupation de Pierre Claverie est de voir comment réaliser le dialogue entre les chrétiens et les musulmans malgré les réticences, et le moment de la guerre civile. Quel serait le témoignage de l’Eglise qui est au cœur de ce peuple en souffrance du peuple ?

\subsection{Analyse de deux lettres }
 
 \paragraph{La première lettre sur « Qui sera le plus fort ? » }
 
 C’est une lettre écrite pendant le temps d’Avent et qui se termine avec un appel d’entendre le message dans la nuit de Noël. 
Avent, temps qui marque espérance et dans ce milieu et en ce contexte des conflits, est-ce que cette attente d’un dialogue pacifique ne devient plus illusion ? Parce qu’en temps de crise les attentes se font multiples . Cette lettre nous rapporte cette domination islamique dans l’expression dans le style comparatif :\begin{quote}
    « Autrefois vous étiez les plus forts, alors vous essayiez de nous convertir. Aujourd’hui vous êtes en position de faiblesse et l’Islam se renforce et s’étend. Vous avez peur, alors vous essayez une autre tactique pour nous affaiblir : le dialogue. »
\end{quote} 

Cette lettre adressée aux chrétiens qui veut faire réfléchir les chrétiens à savoir sur ce qui s’est passé dans l’histoire et c’est la situation qui crée une sorte de révolte chez les musulmans ainsi on décèle les condamnations et accusations faites aux chrétiens, aux européens, au monde et en particulier à la métropole. Parler des droits de l’homme, laïcité, pluralisme, la démocratie ce sont des signes ou action contre l’expansion de l’islam. On y voit dans ces expressions, une agression de part et d’autre. On s’interroge qui sera fort dans cette situation d’accusation. 

C’est une lettre d’interpellation et qui permet à comprendre que bien quelquefois on est dans les illusions croyant que le règne de Dieu doit dépendre de nos divisions, de la puissance de nos missiles, de nos démonstrations scientifiques…
Cette lettre exprime une déception du fait que ce qu’on attend et on espère en temps de l’Avent n’arrive pas : « Chaque année, nous nous remettons en attente, comme si un événement nouveau allait changer le cours de nos vies. Malheureusement souvent il n’arrive rien probablement parce que nous n’attendons rien de ce que Dieu avait peut-être à donner ? Nous passons ainsi à côté, sans savoir, parce que notre désir était à mesure et no à celle de Dieu.
Comment croire à un roi faible, condamné par les hommes, ils l’ont mis sur la croix, c’est un roi dérisoire ; il n’est pas le plus fort. 
Selon notre analyse, cette lettre présente deux situations ou position, partir de la faiblesse de la condamnation, de la crucifixion du roi des chrétiens et après pour montrer la force des chrétiens, de Jésus, du Règne de Dieu. 
Cette lettre précise en quoi réside la force, c’est la force intérieure de l’Esprit qui assure la vie et conjure la mort.

\paragraph{La deuxième lettre : « Rester ? Partir ? »}
 
Cette lettre est probablement écrite pendant le temps de Carême du fait qu’elle fait un rappel : \begin{singlequote}
    « Prendre sa croix chaque jour : nous savions cela. Les religieux le professaient depuis longtemps. Le Carême nous le rappelle encore, dans l’attente de Pâques. »
\end{singlequote}
Dans des situations de tensions, d’attaque les multiples questions reviennent sur la présence en territoire de mission surtout en temps des guerres civiles ou dans d’autres situations de trouble. C’est une lettre qui appelle à des réflexions sérieuses face à cette instabilité, faudra-t-il partir et laisser les Algériens dans la galère ? Et faut-t-il rester dans cette situation difficile, que faire malgré les appels de quitter ce milieu !!! 


Cette double question a sa raison d’exister et d’être posée « Rester ou partir ? », voyons les arguments présentés dans cette à cette situation :
\paragraph{Partir ?} : Pierre Claverie exprime son regret de constater des désengagements collectifs organisés pour quitter l’Algérie malgré toutes les bonnes raisons de la santé, mais d’autres partent par peur. 

\paragraph{Rester ? }
 il y a ceux qui ont décidé de rester car ils n’ont pas de choix à cause de lien de famille, l’investissement matériel ou affectif fait dans le pays… Partir pour eux ce n’est pas une solution de stabilité, pour eux partir c’est la mort, ils partagent la vie, le sort des Algériens chrétiens qui sont eux aussi considérés comme étrangers. 
 
La question qui se fait sentir du côté de l’Eglise qui s’obstine à rester en Algérie du fait qu’elle veut rester auprès de la minorité sans moyen de défenses, la soutenir et lui exprimer la fraternité et l’amitié. L’Eglise est appelée à être signe d’Alliance que Dieu propose à un peuple. Il argumente ce motif de rester en ces termes : \begin{quote}
    « Si nous sommes là pour justifier que Dieu est amour, comment envisager de quitter l’Algérie alors qu’elle se débat dans une crise aussi grave ? Les calculs trop humains, aujourd’hui, risquent de pervertir le mouvement profond de la mission chrétienne : « L’Eglise n’est pas au monde pour le conquérir ni pour se sauver, elle, ses biens et son personnel. Elle est là, avec Jésus, liée à l’humanité souffrante. »
\end{quote}

\subsection{Appréciation critique  }
 
                En parcourant ces lettres, nous sommes interpellés par la vie de Pierre Claverie et ces lettres sont à utiliser pour y puiser les méthodes missionnaires, s’y référer comme source d’inspiration pour le dialogue interreligieux et modèle d’inculturation et un modèle courage missionnaire dans une terre dangereuse. 
                Nous apercevons dans ces écrits quelques aspects théologiques, liturgiques et la référence aux écritures dans l’expression sans citer les passages. C’est dans le sens on aperçoit sa référence au temps liturgique : l’Avent, Noël, le Carême, la Pâques, la résurrection, …

\paragraph{parole de Dieu}
Pour ce qui est de la parole de Dieu : nous avons pu déceler certaines paroles tirées de la Bible : \begin{singlequote}
    « Si le grain de blé jeté en terre ne meurt pas il reste seul ; s’il meurt : il porte beaucoup de fruit. »
\end{singlequote}, ce passage exprime la fidélité missionnaire une fidélité qui marque un amour sans condition, aimer jusqu’au bout. Pourquoi quitter ce peuple algérien en situation grave, c’est le moment de le soutenir. Pour Pierre Claverie en ce temps, il sent que quand on aime quelqu’un on ne le laisse pas tomber même si cette personne n’est pas à la hauteur des attentes qu’on avait mis en lui. Comme l’exemple des parents qui souffrent pour leurs enfants . 
                 A la page 179, son expression montre cette attitude du bon pasteur , ne pas laisser ses brebis ainsi il dit \begin{singlequote}
                     « nous sommes Eglise, signe de l’Alliance que Dieu propose à un peuple. 
                 \end{singlequote}
                A la page 95 nous voyons ce passage de la passion de Jésus, il ressort qu’il est un roi dérisoire qui a été écrasé…n’est-il pas le fils du charpentier…La puissance de Dieu soulève bien l’humanité comme le levain la pâte, de l’intérieur, … 
                
Pour ce qui est de l’aspect missionnaire, nous découvrons des moments qui la mission à l’épreuve, et il la crise missionnaire, le doute qui naît surtout dans la deuxième lettre avec cette double question « Rester ? Partir ? » (p.177) et il ressort de toutes ces crises cet aspect de la fidélité missionnaire et ce courage missionnaire. 
                Ces lettres et messages cadrent avec le titre de notre séminaire du fait que nous y avons découvert la théologie de la rencontre, de la proximité « parvenir à exprimer Dieu par la présence dans une culture à majorité musulmane. »
               En faisant cette lecture une difficulté rencontrée était de savoir pour Pierre Claverie n’a pas voulu préciser les références bibliques, est-ce par respect aux musulmans dans l’esprit du dialogue interreligieux.  Et il a inversé le titre dans la deuxième lettre, le titre c’est « Rester ? Partir ? » et pour le développement il commence par « Partir »

\subsection{Apport du texte  }
 
La lecture de ces lettres fait un déplacement dans la manière de comprendre le rôle de l’Eglise et du missionnaire, Pierre Claverie d’abord retourne en Algérie pour soutenir ce peuple et y reste, il y donne sa vie jusqu’à l’assassinat…Je découvre la conception de la mission ad vitam, le missionnaire devrait partir pour ne plus rentrer dans sa terre natale, il avait embrassé une autre nationalité comme Pierre Claverie avait été algérien par alliance. 
Pierre Claverie, est une figure qui montre qu’on ne peut réaliser la mission par ses propres forces, la vraie force est celle qui vient de l’intérieure et elle vient de l’Esprit de Dieu, se confier dans sa volonté malgré les crises, les épreuves qu’on peut rencontrer. 
Pierre Claverie nous exhorte : 
\begin{singlequote}
    « Pour poursuivre notre mission de disciple de Celui qui a donné sa vie pour tous, nous n’avons besoin que d’ancrer en lui notre espérance. Nous n’avons pas d’autre richesse ni d’autre raison de vivre que lui. Ce que nous faisons nous le faisons en son nom car il est Celui en qui nous sommes devenus de Dieu par adoption . 
\end{singlequote}
              Pierre Claverie m’éclaire dans sa vie missionnaire dans des moments des troubles, il a voulu rester, être avec ce peuple en vivant dans le dialogue de vie. 	

\subsection{Conclusion}

             Au terme de notre exposé que pouvons-nous dire de plus !!! Nous avons fait notre voyage en Algérie pour rester à l’exhortation de Mgr Pierre Claverie qui nous a permis de parcourir les deux lettres sures « Qui sera le plus fort » et « Rester ? Partir ? » 
            Cette lecture nous a permis de découvrir que la force réside en Jésus, ce n’est pas celui qui s’impose par la violence qui est fort, et nous avons appréhendé sur la vie de la mission quand on aime une terre, une personne il ne faut la laisser tomber, il faut lui rester fidèle. 
            Ces lettres nous invitent à continuer la méditation et elles sont des références pour tout celui qui se lance dans la vie missionnaire peut y tirer les expériences du dialogue, de l’inculturation, le vivre ensemble pour tendre à l’unité dans la diversité.
            
\paragraph{pas une théologie en surplomb} Claverie noue les evenements et les textes de l'Evangile, pour introduire à une théologie subtile. 

\paragraph{quelle est la théologie de la mission dans cette lettre}
 
\paragraph{Eglise est mission}

\begin{itemize}
    \item fraternité ouverte
    \item quand "pourquoi s'obstiner à rester alors que l'Algérie et l'islam nous rejettent ? p. 179
    \begin{itemize}
        \item On ne reste pas par naiveté : vérité sur la situation; texte ad intra
        \item On ne reste pas pour rendre l'Algérie Chrétienne
        \item on reste : présence. rapport à Jésus. Croix au centre de la mission.  p 180 : \textit{il s'st mis sur les lignes de fracture nées de ce péché}
        L'Eglise accomplit sa vocation et sa mission quand elle est présente aux ruptures qui crucifient l'humanité dans sa chair et dans son unité. 
    \end{itemize}
\end{itemize}

 Compassion. 
 \paragraph{c'est une présence au risque de l'impuissance} René Voillaume. 

 \paragraph{le levain} image revient deux fois. Présence à l'autre. A noter qu'en Algérie, les Eglises restent des témoignages, des questions. 

\paragraph{contexte de pluralisme} quand il y a deux ou trois religions, la possibilité du dialogue. 
\paragraph{Question de la violence} La sécularisation permet de mettre à distance à l'état. Ici, Claverie propose l'\textit{humilité}. Mt 25. Présence. 