\chapter{P. René Voillaume - Père de Foucauld}



\paragraph{Attention au contexte} Faire la différence entre Foucauld et Voillaume.

\paragraph{Vie cachée} vie cachée car vie de prédication pas trop attirée.

\paragraph{Au coeur des masses : 6eme édition en 1951} Projet spirituel.
Principal disciple de Foucauld.

\paragraph{1953}
1953 : entre 1943 (coeur de mission), et Vatican II (1963).

\paragraph{l'expérience de la première guerre mondiale} avec une première 

\paragraph{Petits frères de  Jésus en 1933}

1er partie : cite Foucauld. Vie cachée comme Foucauld à Nazareth, à l’inverse des 3 ans de vie publique.
Choix de la vie de Jésus. Simple ouvrier de Nazareth. Projette l’expérience de la vie ouvrière du XX dans la vie cachée.

Ce qui est premier, c’est partager la vie des chrétiens au quotidien.


\subsection{Topographie spirituelle}
Dans le deuxième extrait, évolution de la spiritualité de fr. Charles.
Concrètement, désir d’imitation : 
\begin{itemize}
    \item -	Concret : terre sainte. 
    \begin{enumerate}
        \item  Il recherche l’abjection. Selon lui, abjection par choix, d’abord humilité, puis austérité et enfin humiliation lors de la passion.
 \item 	Le but est d’imiter Jésus
    \end{enumerate}

    \item Du coup, les petits frères doivent être tournés vers Jésus.
    \item Activité rédemptrice cachée de Jésus à Nazareth (Foucauld sort du monastère). Mt 25 : va vers les autres. Beni Abbes
    \item A Tamanrasset, vie avec les Touaregs, dictionnaire. Incarnation de l’imitation de Jésus.
\end{itemize}

\begin{itemize}
    \item Vie humble

   \item 	Vie d’adoration

   \item 	Vie en contact

   \item Vie de labeur (mais Foucauld intellectuel).
\end{itemize}



\paragraph{ce qu'a vécu Jésus} \textit{par amour}. partager la vie \textit{par amour}, donc on se recentre sur le Christ.

\paragraph{souffrance} pas rédemptrice mais dit quelque chose de ce qu'on l'on peut donner par amour.

\paragraph{Esprit d'immolation} Dans sa définition, Voillaume est de \textit{partager} l'oeuvre de salut de Jésus.

\paragraph{Jean XXIII} dira que la lecture de \textit{au coeur des masses}, l'a touché : influence VII.


\subsection{influence spirituelle}
\paragraph{méditation des deux étendards des ES} recherche de l'humiliation. Réflexion sur \textit{l'opprobre}
\begin{singlequote}
    Le troisième point. Considérer le discours que le Christ notre Seigneur adresse à tous ses serviteurs et à tous ses amis qu'il envoie à cette expédition, leur recommandant de vouloir aider tous les hommes en les amenant premièrement à la plus grande pauvreté spirituelle, et non moins, si sa divine Majesté devait en être servie et voulait bien les y choisir, à la pauvreté effective ; deuxièmement, au désir des opprobres et des mépris, parce que de ces deux choses résulte l'humilité. De sorte qu'il y ait trois échelons : Le premier, la pauvreté à l'opposé de la richesse ; le deuxième, l'opprobre ou le mépris à l'opposé de l'honneur mondain ; le troisième, l'humilité à l'opposé de l'orgueil. Et à partir de ces trois échelons, qu'ils les entraînent à toutes les autres vertus.

Les Colloques :

Un colloque à Notre-Dame afin qu'elle m'obtienne de son Fils et Seigneur la grâce d'être reçu sous son étendard :
1. Premièrement, dans la plus grande pauvreté spirituelle et, si sa divine Majesté devait en être servie et voulait me choisir et recevoir, non moins dans la pauvreté effective ;
2. Secondement, en endurant opprobres et outrages afin de l'imiter par-là davantage, pourvu que je puisse les endurer sans qu'il ait péché de quiconque, ni déplaisir de sa divine Majesté.
Après cela, un Ave Maria.
\end{singlequote}
\paragraph{Thérèse de l'enfant Jésus} Approche spirituelle nouvelle entre une spiritualité du XIX assez soft et cette nouvelle spiritualité : Rôle de Leon XIII : travail sur l'exegese, sur la spiritualité. Au fond, cela a renouvelé la spiritualité.

\paragraph{Notion de Nazaeth} Marie de la Passion \sn{\href{https://fr.wikipedia.org/wiki/Marie_de_la_Passion_de_Chappotin}{Marie de la Passion}}, 10000 religieuses en 1950. Tjs une référence à Nazareth. Il faut retourner au fondamental de la vie chrétienne. \textit{au service des autres}. 

\subsection{et les missions aujourd'hui dans tout cela ? }

\paragraph{expérience fondamentae de Benis Abbes} vie simple. pourrait expliquer son désir de la vie cachée. Désappropriation de lui-même.


\paragraph{tension entre l'abaissement et la tension} Jésus ne se manifeste pas dans l'ouragan mais la brise légère. De plus en plus, de la vie de Nazareth à la Passion. C'est finalement l'humilité, l'abaissement. 

\paragraph{les échecs féconds} Une négativité qui n'est pas enfermée sur lui-même. De son vivant, c'est un échec, personne ne le suit, personne converti.


\paragraph{Humiliation - humus} une terre qui rend du fruit à la terre.
Mission à long terme.


\paragraph{Mission de la présence} par ce que l'on est, cette simplicité, au milieu de toute culture : Montchanin, Thibirine.
Il va y avoir un \textit{langage de la présence}, sans parole. Langage de la gestuelle. Mère Teresa, prenant la main des lépreux.

L'Evangile donne une gestuelle, un langage de la présence, \textit{style chrétien}. Subtilité féminine de la présence de l'art d'être l'évangile. 

\paragraph{pas une stratégie, pas d'idée derrière la tête : la grâce} on ne va pas voir les autres pour convertir. Grâce de Dieu qui s'approche de nous. Même idée ici évangélique. \sn{p. 227}

\paragraph{est ce missionnaire ou pas ?} interroge notre vision de la mission. 

\paragraph{frère universel} Conversion de cette vision par la rencontre des musulmans. En 1901, il a failli mourir du scorbut et est sauvé des femmes touaregs : "sa deuxième conversion". On voit l'autre  et la mission différemment. se laisser convertir par l'autre.


\begin{singlequote}
    La joie (Claudel Soulier de Satin)
\end{singlequote}


\paragraph{Théorie de l'enfouissement}

\paragraph{Thérèse et le père Roulland, MEP} \href{https://missionsetrangeres.com/lunion-apostolique-de-sainte-therese-et-du-pere-adolphe-roulland-missionnaire-en-chine/}{Père Roulland et Thérèse de Lisieux}. Lire la correspondance.


\section{compte rendu}
\begin{Synthesis}
    Abjection : desappropriation de soi, lié à l'amour. Maitre Eckaert.
    Souffrance du Christ. 
    Adéquation de la mission et de l'Evangélisation : Lubac plus intellectuel. 
\end{Synthesis}

\subsection{A. Présentation générale
}
Les documents étudiés sont trois extraits de « Au coeur des masses » de René VOILLAUME. Il s’agit d’un regroupement de plusieurs écrits successifs qui viennent éclairer le projet spirituel de la communauté des Petits Frères de Jésus qu’il a fondée en 1933. Le livre est paru en 1946, dans l’après-guerre, alors que l’on prend conscience que le travail missionnaire est à effectuer aussi en France.
\paragraph{1. Le mystère de Nazareth}

René Voillaume insiste dans un premier temps sur la vie cachée de Jésus à Nazareth, à comprendre non pas dans le sens vie retirée du monde mais plutôt comme une vie ordinaire, de la même manière que Jésus, conscient de sa divinité, a décidé de la masquer à Nazareth, vivant une vie d’anonyme, partageant la vie des hommes (selon Charles de Foucauld).
\paragraph{2. Comment le père de Foucauld a découvert et vécu le mystère de Nazareth}
Dans ce chapitre, l’auteur analyse l’évolution spirituelle de Charles de Foucauld dans sa manière de comprendre et vivre son désir d’imitation du Christ. A Nazareth, il la perçoit dans un désir d’austérité, d’acceptation de l’humiliation et du mépris (désir d’abjection) ; à Béni Abbès, l’activité rédemptrice cachée de Jésus à Nazareth passe au premier plan, avec le souhait de rencontrer le plus pauvre. A Tamanrasset, il vit un épanouissement de sa vocation, notamment par sa proximité avec les Touaregs, dans une vie humble, de prière, d’adoration.
\paragraph{3. Sauveurs avec Jésus
}
René Voillaume cherche dans ce chapitre à dégager le sens de cette imitation pour les Petits Frères : il s’agit de concilier vie de prière et vie de travailleur, à l’imitation de Jésus partageant la vie des hommes, par amour. L’auteur voit aussi dans le partage de la souffrance de ceux qui entourent les Petits frères et la manière de supporter cette souffrance une continuation de la Passion de Jésus, avec une dimension rédemptrice. Il s’agit de l’accueillir dans un esprit d’immolation et de vivre l’eucharistie comme le lieu ultime de cette communion à la souffrance des autres.

\subsection{B. Discussions autour du texte}

\paragraph{1. Une conception de la vie cachée du Christ}
On peut noter comme un tiraillement entre vouloir imiter Jésus dans sa vie cachée, et s’associer à sa Passion par la manière dont on vit les souffrances (les siennes propres, mais aussi celles de ceux dont on partage le mode de vie). Le ministère public de Jésus n’est pas mentionné. Cette conception de la vie cachée a interrogé : dans le Temple, Jésus ne montre-t-il pas déjà sa nature divine ? le Jésus « humain » avait-il vraiment conscience de sa divinité et sa mission rédemptrice ? Mais il faut remettre ces affirmations dans le contexte théologique de l’époque : la vision béatifique du Christ ne faisait pas encore débat. Aujourd’hui, il peut sembler difficile de dire que Jésus a dissimulé ce qu’il était vraiment ; cela pourrait même nous conduire à un certain docétisme (hérésie selon laquelle Jésus a fait semblant d’être un homme).
ISTR – Dialogue, mission, inculturation aux XXème – XXIème
La vie cachée est interprétée ici par la Passion et la mort de Jésus, avec une forte insistance sur l’humilité dans l’Incarnation du Fils1. Dieu semble se révéler en quelque sorte de manière cachée, non dans les miracles ou la gloire (même dans l’eucharistie, Jésus est en quelque sorte caché, tout comme sur la croix sa gloire n’éclate pas vraiment).

\paragraph{2. Abjection et « Souffrance rédemptrice »
}
Le terme d’« abjection » a beaucoup marqué : s’il désigne le « dernier degré de l’abaissement, de la dégradation morale, synonyme d’ignominie », il se conjugue ici avec le désir de désappropriation de soi. Voillaume souligne que ce désir d’abjection doit être lié à l’amour : fixer le regard sur le Christ et vivre cette souffrance en communion avec lui, par amour. Certains ont pu faire le lien avec Maitre Eckhart qui va jusqu’à l’anéantissement de toute forme de volonté.
Cette acceptation de la souffrance tout en conservant une paix intérieure a soulevé la question: cela ne fait-il pas courir le risque de « s’endormir », d’oublier le scandale de la croix qui permet, pourtant, de ne pas rester passif ?
René Voillaume insiste beaucoup sur le fait de partager la souffrance du Christ, s’unir à lui dans sa passion, pour participer mystérieusement à la rédemption dans cette compassion. Le lien a été fait avec Col 1, 24 « Maintenant je trouve la joie dans les souffrances que je supporte pour vous ; ce qui reste à souffrir des épreuves du Christ dans ma propre chair, je l’accomplis pour son corps qui est l’Église», et avec la citation de Pascal : « Jésus sera en agonie jusqu’à la fin du monde ». Le texte ne mentionne ni la Résurrection, ni l’Esprit Saint : cette théologie reste très centrée sur la croix. Ce serait sans doute à rééquilibrer pour éviter tout risque de dolorisme.
\paragraph{3. Une évolution dans la pensée de Charles de Foucauld}
Le texte manifeste une grande perspicacité, distinguant le psychologique du spirituel, alors que la frontière entre les deux peut être poreuse. Ceci se voit par exemple en différenciant la recherche délibérée de l’abjection (psy) et le désir de souffrir l’abjection (spi). En présentant les différentes étapes de la construction spirituelle de Charles de Foucauld, qui passe sa vie semble-t-il à chercher sa place, on a l’impression d’une évolution : d’une spiritualité plutôt individuelle au départ, basée sur l’effort personnel, il s’ouvre vers une spiritualité de l’autre, fondée sur le double commandement de l’amour. Charles lui-même a vécu une sorte de « deuxième conversion », lorsque, souffrant, il a bénéficié des soins de ses « soeurs » voisines : cela l’a conduit à développer sa volonté de se faire le « frère universel ».
La notion de « goût » en islam peut aussi nous éclairer : tant que l’on n’a pas goûté à la souffrance de l’autre, on ne peut se rendre compte de ce qu’il ressent. Charles de Foucauld n’a pas été très apprécié en Algérie pendant longtemps (on voyait en lui une attitude hautaine, méprisante ; à quoi s’ajoutait une certaine tension entre colonisation et évangélisation), jusqu’à ce qu’il « goûte » la vie de ceux qui l’entouraient. Il apprend à leur contact l’humilité.

\paragraph{4. Eléments de contexte}
On a noté le lien entre Charles de Foucauld et les autres auteurs spirituels. D’une part, avec saint Ignace de Loyola, évoquant lui aussi une forme de réponse à Jésus dans l’humiliation, pour certains appels spécifiques, dans les Exercices Spirituels et la méditation des deux étendards. D’autre part, avec les contemporains de Charles de Foucauld en cette période de renouveau spirituel très fort en France: principalement avec Thérèse de Lisieux, mais aussi Marie de la Passion, fondatrice des franciscaines missionnaires de Marie. La notion de Nazareth a ainsi été remise à l’honneur à la fin du XIXème, avec le désir d’un retour aux fondamentaux de la vie chrétienne par la proximité et le service des autres.
\paragraph{5. Une conception de la mission originale et actuelle}

La pensée développée par René Voillaume prend place dans le contexte d’après-guerre et la volonté de rechristianisation de la France. Il développe une conception de la mission à l’opposé d’une mission de l’Eglise conquérante, sans mentionner d’annonce explicite ni les structures de l’Eglise locale (paroisse, etc.) –on peut faire le lien avec la notion d’enfouissement. La « méthode missionnaire » proposée consiste plutôt en un rayonnement par la présence simple au milieu de toute culture ; c’est donc plutôt une mission discrète de proximité. Il y a alors comme un langage de la présence qui n’est pas langage de la parole (pensons au langage gestuel de Mère Térésa prenant la main des lépreux par exemple), comme une façon d’être, disciplinée par l’Evangile. Cette attitude par elle-même suscite un questionnement chez les autres, elle permet de « crier l’Evangile par toute sa vie ».

La question a été soulevée de l’adéquation entre le terme mission et cette conception de la mission. On a remarqué que ce texte semble en opposition avec la conception d’Henri de Lubac, alors qu’ils sont datés de la même période. Voillaume écrit à partir de son expérience enracinée dans le terrain, alors que de Lubac a une approche plus intellectuelle. Le terme même de « mission » est comme caché dans le texte de Voillaume. Le statut du texte a été interrogé : est-ce plutôt à considérer comme un traité spirituel sur Charles de Foucauld ou une réflexion sur la mission ? Comment la spiritualité intérieure développée, plutôt contemplative, peut-elle s’articuler avec une redéfinition de la mission? Une des pistes serait de comprendre que la mission en elle-même est indissociable de l’attitude intérieure.

Cette vision de la mission sans annonce explicite peut donner l’impression d’un certain échec (c’est le sentiment que Charles lui-même a eu). Mais cet échec peut avoir une fécondité (distincte de l’efficacité, qui est dans un temps court, la fécondité peut s’inscrire dans un temps plus long). Ainsi peut-on penser aux communautés des Petits Frères et Petites Soeurs et à leurs actions (exemple du Brésil). il y a donc ici une conception de la mission à long terme. On peut penser à l’étymologie des termes humilité et humiliation, renvoyant à la notion d’humus, c’est-à-dire ce qui rend la terre riche. Cette mission va progressivement porter du fruit.
On a relevé le caractère très actuel du texte, encore aujourd’hui, qui nous invite à considérer que toute évangélisation ne peut passer que par la rencontre avec l’autre, elle-même possible que si je suis capable de m’effacer devant lui. Il s’agit ici d’une spiritualité de la présence auprès des autres dans leurs souffrances, partageant celles-ci sans but ni sans stratégie : être là au milieu
 
d’eux, gratuitement, de la même manière que Dieu s’approche totalement gratuitement de nous, par grâce. On est loin d’un « programme missionnaire » ; le but est d’aimer sans chercher à utiliser l’autre ou le manipuler.
\paragraph{Une remarque de méthode} : attention à bien distinguer les sources (ici, qui parle ? Voillaume ou Charles de Foucauld ?), à bien contextualiser la pensée de l’auteur pour identifier ce qui vient de lui ou non.