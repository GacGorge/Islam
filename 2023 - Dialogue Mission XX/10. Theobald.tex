%\chapter{Urgences pastorales du moment présent Christoph Theobald }





%------------------------------------------------------------
\section{ Le contexte et l’auteur}
%------------------------------------------------------------

\paragraph{Christoph Theobald} Théologien jésuite, d'origine allemande mais vivant en France, probablement le théologien francophone vivant le plus connu dans le monde. Il est réputé d'accès difficile du fait d'une écriture serrée et surtout d'un dialogue toujours renouvelée avec la philosophie et la théologie française et allemande (en particulier, il est connu comme le traducteur des oeuvres de Rahner en français).  Son livre le plus connu, \textit{le Christianisme comme style} \cite{theobald_christianisme_2007} essaye de définir le christianisme à partir de son \textit{style}, dont l'origine est à chercher dans la vie concrète de Jésus, vie qu'il résume par le terme d'\textit{hospitalité} : comme Marie-Madeleine au jardin, on reconnaît le christianisme dans ce style, même s'il est transformé par la culture dans lequel il vit.


\paragraph{Une théologie libérale "d'en bas"} La pensée de Ch. Theobald est marquée par différents points que l'on retrouve dans le texte : 
\begin{itemize}
    \item \textit{On ne peut accéder au Jésus historique} : on ne peut atteindre Jésus historique  on part des récits évangéliques, éclairés par la Résurrection, travaillés par elle. D'où l'importance des textes bibliques et de l'évangile.
    \item  \textit{Une approche éthique, pragmatique partant de Jésus.}    Approche originale et adaptée à notre monde. Theobald développe essentiellement une  théologie au \textit{service} des contemporains. 
    \item \textit{Un contexte hors chrétienté : lien social dans une société pluraliste} Dans une société pluraliste,  le lien n'est plus assuré par le religion mais par l'Etat, neutre.
\end{itemize}
  
 
 
 

%------------------------------------------------------------
 \subsection{contexte d'urgence pastorale}


\paragraph{Une Eglise minoritaire dans les pays traditionnellement Chrétiens} Ch. Theobald part du constat des nombreuses crises de l'Eglise : chute de la pratique, ... L'Eglise est devenue minoritaire dans les pays Occidentaux. \paragraph{perte du leadership de la culture}. Une vraie exculturation, Hervieu-Leger (p 28 et dans notre texte, p. 446, diagnostic), sortie de la culture. 
 

\paragraph{Evangelii Gaudium : lien intrinsèque entre réforme et mission} Un autre élément de contexte est la publication de \textit{Evangelii Gaudium} du Pape François. François incarne ce que Theobald \cite{theobald_pastoralite_2021} appelle une \textit{herméneutique de la réforme} : \begin{singlequote}
C’est précisément dans cet ensemble de continuité et de discontinuité à divers niveaux que
consiste la nature de la véritable réforme. \cite{benoit_xvi_discours_2005}
\end{singlequote} Le premier chapitre d'EG fonde cette réforme sur la pastorale missionnaire; dans la lignée du décret \textit{Ad gentes}. Si l'on insiste sur la tâche missionnaire \cite[p.162]{theobald_urgences_2017}, alors la réception du \textit{texte lui même} devient essentiel. D'où le choix de François de rester \textit{discret par rapport aux options théologiques, voire doctrinales} dans ses écrits : \textit{lex omnis evangelisationis } (GS 44).
\begin{singlequote}
     la variété [des "diverses lignes de pensées philosophique, théologique et pastorale] aide à manifester et à mieux développer les divers aspects de la richesse inépuisable de l'Évangile » [EG 40]
\end{singlequote}

François reprend aussi que d'autres aspects de la pastoralité conciliaire, peu présents ces dernières décennies : l'insistance sur le peuple des baptisés comme peuple missionnaire, porté par le «sens de la foi»; la réévaluation de la pluralité interne de l'Église et de nos cultures. \cite{theobald_pastoralite_2021}



%------------------------------------------------------------
\subsection{Le livre : Urgences pastorales du moment présent}

\paragraph{Urgences pastorales du moment présent} Le livre \textit{Urgences pastorales du moment présent - Comprendre, partager, réformer} \cite{theobald_urgences_2017}, dont est tiré le chapitre que nous étudions, est un livre de théologie appliquée :   comment inciter l’Église à retrouver un dynamisme missionnaire. Et cela passe par une réforme qui « porte sur la capacité des chrétiens et de leurs Églises à mettre l’Évangile du règne de Dieu à la disposition de toute l’Humanité et de toute la terre comme "ressource" salvatrice » (p. 13).


 Le livre se structure en 3 parties : 

\begin{itemize}
    \item une première partie \textbf{"s'asseoir", sur le diagnostic du moment présent},  Theobald analyse quelques-uns des défis actuels : montée de l’islam, crise écologique, évolution du rapport à la mort, \ldots
    Après un chapitre sur la situation du catholicisme Européen en crise, puis sur  la question de la crédibilité de la tradition chrétienne,  un dernier chapitre clôt cette partie, sur l'enjeu de la présence géographique, avec la disparition des paroisses (on passe en Creuse de 260 paroisses à 6 paroisses). Mais Theobald ne s'arrête pas à ce diagnostic sombre : "je ne peux que miser sur l'avenir du catholicisme".   Quels sont les nouvelles manières d’habiter le monde de nos contemporains capable de résonner avec les urgences de l’époque et de resister à la tentation du repli : exigence de respect de l’autonomie personnelle,  sens d’une solidarité fraternelle large,  conscience de l'enjeu pour l’avenir de notre planète. Beaucoup  sont prêts à s’engager et à payer de leur personne. Il y a ici un potentiel de créativité insoupçonné. Or la « proposition de sainteté » que l’Église donne à percevoir rejoint difficilement ces aspirations nouvelles . À tort ou à raison, elle est perçue comme légaliste, abstraite, trop préoccupée de sexualité ;  » (p. 84). 
    
    \item la seconde partie \textbf{la moison est abondante}, {propose d'engager une conversion} missionnaire de l'Eglise. Pour Ch. Theobald, 
l'Eglise ne prend pas vraiment au sérieux le kerygme, sa propre source, l'hospitalité inconditionnelle de Jésus.\cite{etienne_grieu_leglise_nodate}.  Or le  kérygme et donc à la hierarchie des vérités  permet seule une adaptation à chaque contexte local.
    \item la troisième partie propose une pédagogie, \textit{méthode}, une voie : \textbf{"comprends tu vraiment... ?"}. 
 Souvent encombrée par le souci de sa propre perpétuation, L'Eglise a bien du mal à laisser l’Esprit donner naissance à des figures nouvelles. Elle parvient difficilement, par exemple, à penser la communauté chrétienne plus largement que selon le mode paroissial, construite autour du ministre ordonné et accaparée avant tout par les services religieux (funérailles, baptêmes, mariages). Theobald, soucieux de l’enracinement local de l’Église, propose de « laisser advenir de véritables communautés sur place » qui soient « sujets collectifs » et « missionnaires pour leur environnement » (p. 324). Cela suppose aussi de passer à une nouvelle figure du pasteur, « passeur » plutôt que « pivot » (pp. 329-332).\cite{etienne_grieu_leglise_nodate}. Le dernier chapitre de cette partie, que nous étudions, \textit{étapes d'une ecclésiogenèse} propose pratiquement une méthode.
    \item enfin, p. 463, la conclusion générale du livre
\end{itemize}



\begin{comment}
    \begin{itemize}
    \item Une chute statistique des catholiques en France \textbf{}
    \item une Eglise en \textit{crise}, mais crise à qualifier : échec, propnléme de fonctionnement, maturation sans remettre présuppositions ? 
    \begin{itemize}
        \item Optimiste, le rapport Dagens : derrière la crise, un laboratoire
        \item plus pessimiste, Danièle Hervieu-Léger adopte la thèse de l'exculturation du catholicisme (à la différence de l'inculturation vue avec le texte d'Arrupe). \cite[p. 28{theobald_urgences_2017}
    \end{itemize}
\end{itemize}


 

\paragraph{Quelle réforme face à  l'affaiblissement de l'Eglise} 


\end{comment}
 
 


 
 
 




%------------------------------------------------------------
\section{Chapitre étudié : Etape d'une Ecclésiogenèse}
%------------------------------------------------------------

\paragraph{Ecclésiogenèse}
Dans ce dernier chapitre, Ch. Theobald propose un \textit{processus} pour la création et le développement des Eglise. 
Le terme d'\textit{Ecclésiogenèse} vient d'Amérique Latine, Leonardo Boff, un des théologiens de la libération les plus influents. Ch. Theobald en donne la définition suivante : 
\begin{singlequote}
    un devenir ou une germination possible quand l'existence ecclésiale se réduit à un \textit{Presque rien}.
\end{singlequote}

Une telle définition a une vocation d'antidote, nous renvoyant aux débuts de l'Eglise, qui ne peut être que missionnaire et non tournée vers elle-même. 

\subsection{Le décret Ad Gentes (AG) : une conception génétique de l'Eglise} Quand l'Eglise occidentale n'est plus réduite qu'à un presque rien, il n'y a plus de distinction à faire entre pays de mission et pays christianisé. L'A. propose donc de reprendre le décrit conciliaire sur l'activité missionnaire de l'Eglise \textit{Ad Gentes} (1965) et de voir sa pertinence par rapport à la crise de l'Eglise en Occident. 
Ce texte est en effet est le premier du magistère qui développe une \textit{écclésiogenèse}. Dès les principes d'\textit{Ad Gentes}, il y a identification entre la nature même de l'Eglise et son activité missionnaire.  C'est une nouveauté par rapport à LG qui séparait les deux activités. 

\paragraph{Approche génétique et historique } Selon AG, \textit{l'activité missionnaire de l'Eglise se différencie selon les "circonstances"} (AG 6) et donc l'Eglise développe différents \textit{moyens}. Mais approche historique ne veut pas dire développement linéaire, les différentes phases n'étant pas dépassées. C'est d'ailleurs ce point qui permet à AG de s'appliquer à toutes les Eglises, aucune n'étant \textit{parfaitement constituée}.

\paragraph{les différentes phases de la Genèse de l'Eglise selon \textit{Ad gentes}} AG distinguent trois types idéaux : 
\begin{itemize}
    \item tout d'abord le témoignage, la présence et la \textit{conversation} avec les contemporains, à l'image de Jésus et ses contemporains.
    \item puis la \textit{prédication}
    \item enfin seulement le \textit{façonnement de la communauté Chrétienne}
    
\end{itemize}

\paragraph{Une référence d'AG à l'Ecriture : les Actes des Apôtres} Ch. theobald prend au sérieux les nombreuses références aux Actes des Apôtres dans le décret ce qui pousse à lire \textit{Ac} comme le récit d'une genèse d' Eglise. A partir de l'analyse de MF Baslez, Theobald souligne l'importance des structures associatives pour la genèse de l'Eglise, mais que le baptême entraine une mixité sociale inédite. De la même manière, quelles sont les canaux contemporains qui permettent le développement des églises dans leur milieux culturels ? \textit{Eglise} dans les \textit{Ac} n'a pas la signification technique qu'il a aujourd'hui mais prend souvent son sens original d'\textit{assemblée} : la désignation \textit{Eglise} intervient progressivement : d'abord \textit{quelques-uns}, \textit{maisons}...
A partir de cette analyse, Theobald retient quelques points d'ancrage: 
\begin{itemize}
    \item Importance du lieu comme enracinement culturel : les Eglises locales sont autonomes car c'est le lieu qui marque la réalité culturelle, avec des différences notables comme celle du concile dit de \textit{Jérusalem}.
    \item lors d'un différent, le récit permet de discerner l'action de Dieu parmi nous. Puis il note l'étape de délibération et de décision.
    
\end{itemize}

\subsection{Que retenir de ce périple ?}
\paragraph{Une Ecclesiologie "d'en bas"} A la différence du début de \textit{Lumen Gentium} qui part du mystère de l'Eglise dans le dessein trinitaire de Dieu, AG et les \textit{Actes} propose une \textit{écclesiologie} d'en bas. Et Theobald souligne que même dans LG, des traces de l'ecclésiogenèse sont visibles.

\paragraph{lien entre Actes et le Royaume de Dieu} La difficulté de rattacher l'Ecclésiogenèse aux \textit{Actes}, est de définir le lien entre Eglise et Jésus.  Pour cela, dans l'esprit de LG 5, définit le commencement de l'Eglise dans le message de Jésus sur le Royaume de Dieu. Mais ce message ne se comprend que dans les signes de Jésus, signes messianiques : les pauvrent entendent, les aveugles voient.
Ces signes, comme les charismes pauliniens sont d'abord, souligne Theobald, des \textit{personnes vivantes}.

\paragraph{l'Eglise sacrement universel de Salut} On peut mieux comprendre l'expression \textit{Eglise sacrement universel de Salut} à partir des signes messianiques synoptiques et des charismes pauliniens : les deux, "sacrementum", ou "mysterion", sont marqués par la surabondance et dépassent la sphère classique des 7 sacrements. Pour retrouver la dimensions messianique de ces signes, l'Eglise doit porter l'attention aux personnes et non aux gestes. La prolifération des formes de gouvernance (les Soixante-douze, les Sept) autour de la relation voulue par Jésus avec les Douze, laisse une place à l'adaptation et à l'Esprit Saint \textit{au grès des circonstances.}



 


%------------------------------------------------------------
\subsection{ Et aujourd'hui ?}
Theobald conclut par une proposition d'un itinéraire d'écclesiogenèse autour de   sept étapes , découpées, sur la base du décret missionnaire \textit{Ad gentes }et des Actes des Apôtres : 
\begin{itemize}
    \item  1ere étape : la genèse d'Eglise dans un \textit{espace hospitalier}
    \item 2e étape : la relation à l'écriture sainte,  qui ne cessent de nous confronter à la diversité quasi infinie des présences de l'Esprit dans l'histoire 
    \item 3è étape : la découverte de personnes nouvelles et de leurs charismes que l'Esprit donne . La naissance d'une expérience missionnaire suppose qu'une communauté se laisse surprendre par ce qui se passe en telle personne ou en tel groupe, non seulement en tel catéchumène, mais bien plus largement en «quiconque», qu'elle éprouve donc concrètement d'être sans cesse « précédée » par l'Esprit. 
    \item entre la 3è et la 4è étape : A partir d'un certain moment, ce qui est éprouvé par des individus doit devenir une expérience collective et un bien commun.  
 
\item 4e étape : une communauté qui se met à délibérer. la délibération et la découverte de l'œuvre de l'Esprit dans l'inattendue entente mutuelle, va permettre que la communauté se sente missionnaire

    \item 5e étape : la perception des dimensions corporelles de la Foi. l'expérience d'une surprise spirituelle qui devient possible, quand on prend soin de la corporéité de nos célébrations et de l'Église 
    \item 6ème étape : la prise en compte de la dimension toujours plus universelle de l'Eglise. La découverte du travail de l'Esprit de Pentecôte dans une Église qui ne cesse de nous surprendre par ses dimensions à la fois universelles et polyédriques. 
    \item 7è étape : contemplation. C'est la contemplation sous ses multiples formes qui achève l'itinéraire. Elle rend aussi intérieurement évident que cet itinéraire ne s'achève jamais, tant que l'histoire dure, et qu'il faut reparcourir sans cesse ces différentes étapes - ou d'autres - avec toujours plus de profondeur.\cite[p. 460]{theobald_urgences_2017}
\end{itemize}
   
A noter la notation typiquement ignatienne "\textit{- et d'autres dispositifs qui peuvent encore être inventés.} 
 
 
 

%------------------------------------------------------------
\section{ Elements de discussion}
%------------------------------------------------------------


\subsection{sur le texte}
\paragraph{Apport de l'exégèse} Un premier point marquant, c'est le détour pour parler des réponses à la crise de l'Eglise par les textes conciliaires \textit{Ad Gentes} et \textit{Gaudium et Spes} et surtout à une exégèse des références bibliques sous-jacentes à ces deux textes, d'un côté les Actes des Apôtres, de l'autre, les Evangiles Synoptiques et St Paul. Même s'il est peu cité, l'influence du pape François et en particulier EG peut être repéré dans tout le document (herméneutique de la réforme, structure polyédrique,...)



\paragraph{Un processus proposé pour les communautés } Theobald propose donc un processus, un \textit{exercice}, qui reprend beaucoup des méthodes ignatiennes communautaires en particulier pour déceler les priorités \textit{apostoliques} des jésuites.  De la même façon que les exercices spirituels, la méthode ne garantie pas le résultat : "Ce n'est pas quelque chose que l'on commande ou que l'on peut produire par soi-même, mais pour percevoir cette « précédence » [de l'Esprit Saint]" \cite[p. 460]{theobald_urgences_2017}. on peut passer \textit{à travers} les exercices sans être convertis. Mais il donne un cadre fondé théologiquement qui permet de structurer l'approche. En proposant un processus, on met d'emblée la communauté \textit{en dynamique}. Comme le dit le pape François, il convient d'\textit{initier des processus} et non de viser à maîtriser des espaces.(EG 223) 


\paragraph{Proximité et différence entre les 6 étapes de la conversion au Christ et le processus proposé } On peut d'ailleurs noter les ressemblances et les différences avec le processus individuel de conversion proposé par Theobald dans \cite{theobald_christianisme_2007}.
\begin{itemize}
    \item Apprentissage de Jésus (He 5, 8) dans ses rencontres, il vit la sainteté

\item Jésus ne s'impose pas mais permet à l'autre d'accéder à lui -même. Il va libérer l'autre de ses craintes et se dire. 
\item l'émergence de la Foi : quand on accède à son identité, elle commence à être sauvé. Foi en Dieu. cf la Femme adultère, \textit{Ta Foi t'a sauvé}. de l'hospitalité de Jésus arrive la Foi. Il prend des exemples dans l'Evangile : elle libère les personnes rencontrées et les ouvre à une vie nouvelle
\item refiguration, mue d'identité : conversion des personnes qui croisent Jésus, ces gens qui ont changé de vie en rencontrant Jésus. Chacun à adopter le style de vie messianique. Nous avons à devenir hospitalier et leur permettre d'advenir à leur identité.
\item les personnes qui accèdent à cette vie nouvelle ({le possédé à Gerasa reste et ne devient pas disciple de Jésus}) ne sont pas forcément tous disciples. Mais ces rencontres ne sont pas forcément un changement de vie. Refus. Comment alors maintenir le lien social quand il y a refus de la rencontre ? 
Théobald introduit la règle d'Or.
\end{itemize}
 Par rapport à une discussion que nous avons eue sur la \textit{potentialité d'une société véritablement chrétienne}, on peut noter que le processus individuel invoque \textit{une mue d'identité}, ce qui n'est pas le cas pour l' \textit{Eglise} à moins de considérer la mission comme véritable mue d'identité pour l'Eglise.

%------------------------------------------------------------
\subsection{sur l'approche proposée}

\paragraph{Un lien fort entre théologie et conséquence pastorale} Theobald porte un soin à argumenter théologiquement ces choix pastoraux (et inversement) : il ne s'agit pas de faire une liste de "management" mais bien de s'assurer de la cohérence entre entre théologie et pastorale, signe du \textit{style} chrétien. Ainsi
son soucis d'enraciner géographiquement la pastorale et non de développer les mouvements d'Eglise spécialisés : , enjeu théologique, avec une lecture de \textit{Laudato Si'} qui recommande une structure polyédrique soucieuse de la vie réelle des personnes en lien physique et leur façon de vivre l'Evangile de façon originale.

\paragraph{Place des pauvres} Une attention particulière pourrait être donnée aux \textit{pauvres} : l'hospitalité de Jésus s'adresse à tous, le pharisien, les scribes. Mais avec une \textit{partialité bienveillante pour les pauvres}, attitude de Jésus que ne retient pas particulièrement Theobald :  
\begin{singlequote}
    En somme, la proposition de Theobald est une invitation pour l’Église à se laisser questionner par ce que vivent nos contemporains, avec la confiance que, de cette rencontre, ne peut surgir qu’un surcroît de vitalité évangélique. Bien sûr, les axes ainsi ouverts sont faits pour être débattus (pour ma part, je serais enclin à insister davantage sur une présence prioritaire à ceux qui vivent des situations de grande détresse, ce qui permet de mieux honorer, je crois, le rendez-vous pascal de la vocation chrétienne).\cite{etienne_grieu_leglise_nodate}

\end{singlequote}

 
 




\paragraph{Et si on réhabilitait les tradi ? } Au début du livre, Theobald propose une vraie alternative, avec deux stratégies de résistance possible pour l'Eglise en Europe :  
\begin{itemize}
    \item une stratégie d'accommodement : sur des petits restes, une contre-culture de "témoin-prophète". Pour inverser la tendance, s'appuyer sur les prêtres étrangers pour servir nos communautés en manque de pasteurs (comme les Eglises Evangéliques). \cite[p. 31]{theobald_urgences_2017}. On peut rester avec une figure catholique classique "intégraliste", avec la stabilité de la doctrine. Avec le risque de devenir une "secte" au sens de Weber, une forte adhérence à un corpus mais une difficulté à l'ouverture à l'universel et un risque d'être auto-référencé.
    \item une stratégie de dépassement : faire vivre le catholicisme européen dans la polyphonie des Eglises particulières : elle a la mémoire et est capable de séparer la partie culturelle et la partie normative du fait de sa longue histoire. La responsabilité de cette Eglise est de \textit{de-méditerraneiser} la foi catholique. 
    
\end{itemize}

Cette distinction donne à penser, et montre le chemin ambitieux à parcourir par la stratégie de dépassement.



\paragraph{les "sept" : quand le lieu n'est pas l'unique déterminant culturel } Une des questions qui parcourt le livre, c'est l'importance de communauté géographique. l'A. insiste sur l'importance géographique, pour s'ouvrir à la réalité vécue localement : quels sont les pauvres de notre communauté (locale) ? ce qui est difficile si on pense en terme de réseaux (END, JOC, MCC,...). Mais on voit aussi que dès le début de l'Eglise, il a fallu créer une Eglise "grecque" avec les "Sept", au \textit{service des tables et des pauvres} car les "Douze" (Hébreux) servaient mal les veuves "grecques".  A noter que le livre fait mention de l'"utopie" d'un partage entre action catholique, vie chrétienne et aumônerie d'hôpital \cite[p3292]{theobald_urgences_2017} afin d'arriver à déterminer un "jugement" basé sur la diversité des expériences de chacun.  


 
\paragraph{La théologie de la mission dans ce texte} Theobald est théologien et pense d'emblée l'Eglise missionnaire dans un contexte de "petit reste" en occident. la question est plus pour lui les conséquences en terme de processus : comment être en sortie et à comprendre les attentes de ce monde selon l'esprit d'\textit{Ad Gentes} ?  Le charisme qui semble manquer le plus à l'Eglise de notre temps, c'est celui de lire les signes des temps, "le jugement" \cite[p 392]{theobald_urgences_2017}. Et l'approche "géographique" d'un processus par communauté permet vraiment d'être en mission, par la rencontre et l'hospitalité non pas de celui qui me ressemble mais des "pauvres", comme Jésus l'a fait pratiquement.
Ce n'est pas non plus un projet d'une Eglise \textit{enfouie} car elle demande d'aller à la rencontre des autres (étape 3) en particulier quand il frappe à la porte pour un sacrement. Lors d'un échange précédent, Alexis nous partageait sa conviction de la nécessité du témoignage pour la visibilité. Il me semble que la question est ici remplacée par la place que nous faisons à l'autre : est ce que nous proposons une vision "intégraliste" à prendre ou à laisser, ou bien est-ce que nous nous laissons bousculer par la personne que nous rencontrons ?   


 \section{discussions}

\paragraph{fondation de l'Eglise du Canada} référence aux \textit{Actes}, \textit{primitive Eglise}. Planter l'Eglise : est ce qu'il n'y a pas de lien à faire

\paragraph{} Remarque de Jean-Christophe

\begin{singlequote}
    « Aujourd'hui tu reçois le baptême chrétien. On prononcera sur toi
toutes les grandes paroles anciennes de l'annonce chrétienne et on accomplira sur toi l'ordre de baptiser, donné par le Christ sans que tu n'y comprennes rien. Nous aussi nous sommes renvoyésa ux débats du comprendre.c.
que signifient réconciliation et rédemption, nouvelle naissance et Espnt samt
amour des ennemis, croix et résurrection, vie en Christ et imitation de Jésus
Christ, tout cela est devenu si difficile et si lointain que c'est à peine si nous
osons encore en parler. Nous soupçonnons un souffle nouveau et bouleversant dans les paroles et les actions qui nous ont été transmises, sans pouvoir
encore le saisir et l'exprimer. C'est notre propre faute. Notre Église, q111
n'a lutté, pendant ces années, que pour se maintenir en vie, comme si elle était
son propre but, est incapable d'être la porteuse de la parole réconciliatrice et
rédemptrice pour les êtres humains et le monde. C'est pourquoi les paroles
antérieures doivent perdre leur force et céder au silence ; notre être chrétien ne
peut aujourd'hui consister qu'en deux choses : la prière et faire ce qui est juste
parmi les humains. Toute pensée, toute parole et toute organisation, dans
le domaine du christianisme, doivent renaître à partir de cette prière et de
cette action » (D. Bonhoeffer, « Pensées pour le jour du baptême de D.W.R.,
Mai 1944 », Résistance et soumission. Lettre et notes de captivité, Genève, Labor
et Fides, 2006, p. 353 ; c'est nous qui mettons en relief certains passages).
\end{singlequote}


\paragraph{Ecclésiogenèse} Les Actes se fondent sur un terreau juif, qui font déjà synagogue : il y a deja une communauté.  Par ailleurs, Règne de Dieu de Jésus. Rôle du mémoire de la Cène : fonde une certaine sociabilité. 


\paragraph{presque rien} il n'y a pas rien "même 2 ou 3 sont réunis en mon nom". 
si l'Eglise n'est pas propriétaire de l'ES, 

\paragraph{Amerique Latine} Leonardo Boff. Theobald a écrit un livre sur les \textit{communautés de base} en Amérique latine et la réception des textes de Vatican II. Parecida : synode des évèques : disciple et missionnaire. Pendant la mission, c'est la formation [par l'ES ?].

\paragraph{Maturation} revient souvent.
voir annexe sur la pastorale d'engendrement

\paragraph{l'Eglise comme sujet missionnaire} intéressant. Expérience de conversation déliberatrice. conscience commune. On est loin de l'image du \textit{troupeau}. 

\paragraph{lire \textit{ad gentes} à partir de \textit{Lumen Gentium}} et non l'inverse. Aujourd'hui, pour notre propre église, lecture d'en haut et non d'en bas. 

\paragraph{Benoit XV} Vous missionnaire, vous partez dans le pays mais vous partirez quand le clergé local sera constitué. Vous serez serviteur de cette Eglise. Les nouvelles Eglises  autonomie. Et maintenant, c'est le peuple de Dieu qui devient central. 

\paragraph{pour la prochaine fois}
Noter pour chaque texte, l'idée que je retiens pour la mission. Idée de la mission. Insister sur l'approche nouvelle de la mission on a perçu dans chaque texte, entre \textit{de Lubac} et \textit{Theobald}.

