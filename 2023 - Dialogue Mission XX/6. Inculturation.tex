\chapter{Inculturation}


\section{Marcel Maurin - de l'acculturation à l'inculturation}

\paragraph{M. Maurin} Prêtre au Havre

\paragraph{Vie Arrupe}


\paragraph{querelle des rites} la question : pour les dominicains, 

\subsection{de l'acculturation}

\paragraph{Approche historique} et c'est seulement dans sa IVeme partie qu'il ouvre à la théologie. Prendre un peu de distance historique et critique. Le risque étant la confusion. 

\begin{Prop}
    Dans un mémoire, définir le mot. et éventuellement les nuances entre les textes. 
\end{Prop}

\paragraph{Jean Paul II et la culture, liée à la nation et non l'Etat} Pol



\section{Etude d’un texte de Paul COULON intitulé : « Sur un « gros mot » qui ne figure pas sur la couverture : inculturation » }



ICP.ISTR.MC.JOHN 17/10/2023
Séminaire Dialogue, mission, inculturation aux 20ème et 21ème siècle : de l’Histoire à la théologie
 Père Xavier GUE et Mme Catherine MARIN 

\mn{Revue Histoire et Missions Chrétiennes 2008/1 (N°5), pages 3 à 8. Ed. Karthala. Article \href{https://www.cairn.info/revue-histoire-monde-et-cultures-religieuses1-2008-1-page-3.htm}{disponible} }

\subsubsection{« Sur un « gros mot » qui ne figure pas sur la couverture : inculturation »}


Avec ce titre attractif, l’auteur nous invite à aller, immédiatement, sur la revue pour en connaître la couverture et les autres mots. 
Avant d’entrer dans ce texte, commençons par aborder brièvement la revue Histoire et Missions chrétiennes et notre auteur Paul Coulon. Ensuite, terminons par un court paragraphe sur l’intérêt personnel qu’il a suscité. 

\paragraph{1-La revue Histoire, monde et cultures religieuses}

La revue Histoire, monde et cultures religieuses (anciennement Histoire et Missions Chrétiennes) fut créée en 2007 et édité par Karthala avec pour objectif de « mieux comprendre le phénomène de la diffusion et de l’inculturation du christianisme sans cesse renouvelé depuis vingt siècles, les missions ayant été en quelque sorte « un des premiers agents de globalisation », selon les termes d’Alberto Melloni . 
Cette revue diffuse des études historiques sur tous les sujets autour des missions chrétiennes, des origines à nos jours et dans l’interdisciplinarité, afin de mieux appréhender les questions religieuses d’aujourd’hui et de demain. 
L’historien Philippe Delisle   est le directeur de la Revue en 2008 et dirige les études de ce Numéro 5, intitulé « Acculturation syncrétisme métissage créolisation : Amérique, Océanie XVIème - XIXème s. » qui « s’interroge sur la manière dont les historiens  des missions utilisent des concepts anthropologiques  comme acculturation, syncrétisme, métissage …en prêtant attention à la complexité des phénomènes qu’ils recouvrent ainsi qu’au travail de conceptualisation effectué par les anthropologues. » 


\paragraph{2- L’auteur : Père Paul COULON}

Paul Coulon, rédacteur en chef, rédige le liminaire de ce N°5 constitué de 7 études différentes (couvrant le Mexique, la Nouvelle-Espagne, les Antilles, …) par 7 auteurs. Paul Coulon est un prêtre spiritain, d’abord enseignant en sociologie, puis journaliste à « La semaine africaine à Brazzaville ». Docteur en Théologie, il devient Directeur de l’ISTR à l’ICP. Responsable d’une collection aux éditions Karthala, il se spécialise sur le sujet des Missions et du Développement et publie plus d’une centaine d’articles.  Rappelons que la spiritualité des Spiritains est, entre autres, tournée vers le dialogue interreligieux, le service missionnaire pour tous, la dignité de la personne humaine. 


\paragraph{3- Retour commenté sur l’article de Paul COULON }

Dès son introduction Coulon fait une mise en garde : le mot « inculturation » est employé à tort et à travers. Il voit la nécessité d’éclaircir le concept « d’inculturation du christianisme » et de rechercher s’il a un sens théologique, en miroir avec tous les autres mots qui s’en rapprochent et les définitions anthropologique et ethnologique déjà connues.  Les « gros mots » - Acculturation syncrétisme métissage créolisation et inculturation- apparaissent dans les sujets de la revue au travers d’analyses sur l’histoire de la christianisation au Mexique, dans les colonies « esclavagistes » des Antilles françaises, au Canada ou encore en Afrique. 
Coulon commence son article par un argument historique : le sens toujours actuel du mot « mission », plus précisément « mission chrétienne » nous vient des jésuites, car elle est à la source même de leur vocation « se laisser envoyer en mission », tel François-Xavier envoyé en Inde en 1541 juste après la création de la Compagnie de Jésus en 1539. Les jésuites ont contribué à faire connaître à l’occident l’existence de cultures différentes aux quatre coins du globe. Il cite les « Lettres édifiantes et curieuses ». 



Dans le 2ème paragraphe, Le christianisme et les cultures du monde, Coulon se positionne du côté des sciences humaines et relève que, depuis toujours, les hommes se rencontrent et « affrontent des univers culturels autres que le leur » ; d’où l’intérêt des historiens, des anthropologues et des ethnologues pour l’étude de la diffusion du christianisme qui entre dans ce cadre. Il cite l’historien Serge Gruzinski qui y voit « une première mondialisation ». 


P. Coulon portant son regard du point de vue des sciences humaines y voit un phénomène d’acculturation. Il nous propose la définition de Madeleine Grawitz  :
\begin{Def}[Acculturation]
    l’acculturation est le « processus de changement culturel résultant de contacts entre des groupes de cultures différentes ». 
\end{Def}

Alors que le Cardinal Ratzinger avait avancé le terme d’B pour remplacer celui d’inculturation, P.H. Kolvenbach Sj., déclare en 2008 que le message chrétien ne doit pas être imposé et que, pour être reçue, la transmission de la foi devait bénéficier d’une participation active réciproque et du respect mutuel entre les cultures. Ainsi Mateo Ricci n’a pas imposé ses convictions mais les a proposées par « une lecture chinoise du mystère du Christ ». [page 5]



Dans son 3ème paragraphe, L’inculturation ou le point de vue théologique, Coulon met en lumière le glissement sémantique de l’inculturation vers l’incarnation depuis la définition du P. Pedro Arrupe en 1978 : 
\begin{Def}[inculturation]
    « …l’inculturation est l’\textit{incarnation} de la vie et du message chrétien dans une aire culturelle concrète, en sorte que, non seulement, l’expérience chrétienne s’exprime avec les éléments propres à la culture en question (…) mais aussi que cette même expérience devienne un principe d’inspiration, à la fois norme et force d’unification, qui transforme et recréé cette culture étant, ainsi, à l’origine d’une nouvelle création. (…) Expérience donc d’une Eglise locale, discernant le passé et bâtissant l’avenir, dans le moment présent. »  
\end{Def}

Poursuivant la réflexion du rapprochement entre inculturation et incarnation, Coulon cite le théologien indien Michaël Amaladoss lequel distingue 3 types de médiations historiques dans le processus de l’annonce de l’Evangile que l’on peut résumer ainsi : 

\begin{itemize}
    \item 1-	L’apport d’un Evangile vécu et inculturé dans la culture d’origine des missionnaires.
    \item 2-	La tentative de traduction de l’Evangile dans la culture des gens auxquels ils s’adressent.
    \item 3-	La réception de l’Evangile par les gens qui l’entendent, l’interprètent et le vivent. 
\end{itemize}

Pour Amaladoss, l’inculturation se situe au 3ème stade. Et c’est sous l’action de l’Esprit Saint que le dialogue entre l’Evangile et la communauté va se construire. 
\paragraph{Conclusion : inculturation, pas une science humaine}
Coulon conclut que, dès lors que l’on rapproche un principe de foi au concept d’inculturation, ce terme ne relève plus des sciences humaines. En matière théologique on parlera d’inculturation alors qu’en matière anthropologique ou ethnologique il s’agira d’acculturation (soit « un changement culturel » selon l’anthropologue J-F. Barf ) car ces sciences humaines étudient les contextes historiques, les modalités d’acceptation ou de refus de l’annonce de l’Evangile faite d’homme à homme en faisant abstraction d’une transmission par l’action de l’Esprit Saint. 
En dernier paragraphe, « Toute religion est syncrétique » (citation de Marc Augé), Coulon introduit les articles de la revue qui, tous, oscillent entre inculturation et acculturation, soit aussi entre étude anthropologique et réflexion théologique. 


\paragraph{4-	Intérêt et commentaires personnels sur cet article }

Utiliser un mot pour un autre est fréquent notamment pour des sujets tels que ceux qui tournent autour de la culture et de l’inculturation. Mal employé, le mot « inculturation » devient un « gros mot » qu’il faut corriger. Je vois dans ce texte un appel à la vigilance surtout lorsque l’on aborde la christianisation au cours de l’histoire soit pour rendre compte de l’historicité des faits des missions chrétiennes, soit pour évoquer cette historicité afin de comprendre la mission du chrétien d’aujourd’hui voire de demain. Coulon nous prévient : \textit{l’inculturation est donc bien distincte de l’acculturation.} Et cette mise en garde s’adresse probablement aussi aux sept auteurs de la revue tous relevant des sciences humaines, historiens ou anthropologues. 



Pour affirmer son propos, Coulon fait appel à pas moins d’une dizaine d’auteurs, catholiques et protestants, majoritairement issus des sciences humaines, face à six théologiens (Card. Ratzinger, St Jean-Paul II, P-H. Kolvenbach Sj., P. Arrupe Sj., M. Amaladoss Sj., E. de Rosnay Sj.) et il cite St Paul sur un très beau passage (1Cor 12, 4-6,12). Je pense que sa démonstration se situe, essentiellement, dans la phrase d’Amaladoss : 
\begin{singlequote}
    « la foi chrétienne nous fait tenir que c’est sous l’action de l’Esprit Saint que se construit ce dialogue entre l’Evangile et une communauté. » (page 6)  L’Esprit c’est le souffle, le vent ou la respiration, « tant qu’il demeure en l’homme, ce souffle divin lui appartient réellement, il fait de sa chair inerte un être agissant, une âme vivante (Gn 2,7). »  
\end{singlequote}

Ce travail m’a fait découvrir la revue « « Histoire et missions chrétiennes » et des auteurs que je ne connaissais pas.  Il m’a poussé à rechercher les définitions des « gros mots » de ce numéro et à en rechercher d’autres . Par cet article, Coulon témoigne de la complexité du langage que l’on emploie trop souvent sans réfléchir et confirme l’enlacement presque inextricable qui existe entre les sciences humaines et la théologie.   





\paragraph{querelle des rites} la question : pour les dominicains, le sujet était l’adaptation et la question jusqu’où ? Alors que les jésuites étaient déjà dans une autre théologie, celle de l’inculturation, « l’Evangile doit s’incarner dans une culture ». Il y avait aussi que les Jésuites évangélisaient via les mandarins et les dominicains, le peuple. 

\paragraph{secularisation} la sécularisation est une adaptation pour Arrupe au monde moderne. En s’acculturant trop, on perd sa propre culture. Si le christianisme est une culture, il se perd. 

p. 69 : « acculturation ; adaptation à une culture étrangère et réexpression à l’intérieur de celle-ci.
Pour Arrupe, c’est la même définition de l’inculturation qu’il donnera plus tard. 

N°14  


Jean Danielou, 1947 : « il faut assumer les choses bonnes dans les autres traditions, et bruler ce qui doit mourir : la croix ». Spiritualité missionnaire. Maurin ne fait pas le lien mais il est évident entre Arrupe et Danielou.
C’est la parole de Dieu qui s’inculture. Ce n’est pas l’effort du missionnaire qui s’acculture.

Acculturation : on essaye d’implanter l’Eglise que l’on souhaite
Planter l’Eglise
Inculturation : on n’arrive pas avec sa culture d’Eglise que l’on souhaite mais l’on germe. Charles de Foucault,… image de mourir à soi-même.
Place de ceux qui reçoivent et qui vont développer l’Eglise. 1930 : congrégations religieuses locales.
Etre présent dans une culture + des acteurs locaux.
\begin{Ex}[Eglise du Togo en 1914]
Les missionnaires allemands sont partis en 1914 et ont laissé les acteurs locaux développer leur propre Eglise.
\end{Ex}
Cf : Amaladoss

\paragraph{définition de la culture en milieu catholique : un peu abstrait} L’Eglise christianisme la culture. 
Il n’est pas choquant que l’Eglise définisse la culture selon sa propre révélation (de la même façon que l’homme) mais doit être plus humble.


