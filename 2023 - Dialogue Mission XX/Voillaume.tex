\chapter{P. René Voillaume - Père de Foucauld}



\paragraph{Attention au contexte} Faire la différence entre Foucauld et Voillaume.

\paragraph{Vie cachée} vie cachée car vie de prédication pas trop attirée.

\paragraph{Au coeur des masses : 6eme édition en 1951} Projet spirituel.
Principal disciple de Foucauld.

\paragraph{1953}
1953 : entre 1943 (coeur de mission), et Vatican II (1963).

\paragraph{l'expérience de la première guerre mondiale} avec une première 

\paragraph{Petits frères de  Jésus en 1933}

1er partie : cite Foucauld. Vie cachée comme Foucauld à Nazareth, à l’inverse des 3 ans de vie publique.
Choix de la vie de Jésus. Simple ouvrier de Nazareth. Projette l’expérience de la vie ouvrière du XX dans la vie cachée.

Ce qui est premier, c’est partager la vie des chrétiens au quotidien.


\subsection{Topographie spirituelle}
Dans le deuxième extrait, évolution de la spiritualité de fr. Charles.
Concrètement, désir d’imitation : 
\begin{itemize}
    \item -	Concret : terre sainte. 
    \begin{enumerate}
        \item  Il recherche l’abjection. Selon lui, abjection par choix, d’abord humilité, puis austérité et enfin humiliation lors de la passion.
 \item 	Le but est d’imiter Jésus
    \end{enumerate}

    \item Du coup, les petits frères doivent être tournés vers Jésus.
    \item Activité rédemptrice cachée de Jésus à Nazareth (Foucauld sort du monastère). Mt 25 : va vers les autres. Beni Abbes
    \item A Tamanrasset, vie avec les Touaregs, dictionnaire. Incarnation de l’imitation de Jésus.
\end{itemize}

\begin{itemize}
    \item Vie humble

   \item 	Vie d’adoration

   \item 	Vie en contact

   \item Vie de labeur (mais Foucauld intellectuel).
\end{itemize}



\paragraph{ce qu'a vécu Jésus} \textit{par amour}. partager la vie \textit{par amour}, donc on se recentre sur le Christ.

\paragraph{souffrance} pas rédemptrice mais dit quelque chose de ce qu'on l'on peut donner par amour.

\paragraph{Esprit d'immolation} Dans sa définition, Voillaume est de \textit{partager} l'oeuvre de salut de Jésus.

\paragraph{Jean XXIII} dira que la lecture de \textit{au coeur des masses}, l'a touché : influence VII.


\subsection{influence spirituelle}
\paragraph{méditation des deux étendards des ES} recherche de l'humiliation. Réflexion sur \textit{l'opprobre}
\begin{singlequote}
    Le troisième point. Considérer le discours que le Christ notre Seigneur adresse à tous ses serviteurs et à tous ses amis qu'il envoie à cette expédition, leur recommandant de vouloir aider tous les hommes en les amenant premièrement à la plus grande pauvreté spirituelle, et non moins, si sa divine Majesté devait en être servie et voulait bien les y choisir, à la pauvreté effective ; deuxièmement, au désir des opprobres et des mépris, parce que de ces deux choses résulte l'humilité. De sorte qu'il y ait trois échelons : Le premier, la pauvreté à l'opposé de la richesse ; le deuxième, l'opprobre ou le mépris à l'opposé de l'honneur mondain ; le troisième, l'humilité à l'opposé de l'orgueil. Et à partir de ces trois échelons, qu'ils les entraînent à toutes les autres vertus.

Les Colloques :

Un colloque à Notre-Dame afin qu'elle m'obtienne de son Fils et Seigneur la grâce d'être reçu sous son étendard :
1. Premièrement, dans la plus grande pauvreté spirituelle et, si sa divine Majesté devait en être servie et voulait me choisir et recevoir, non moins dans la pauvreté effective ;
2. Secondement, en endurant opprobres et outrages afin de l'imiter par-là davantage, pourvu que je puisse les endurer sans qu'il ait péché de quiconque, ni déplaisir de sa divine Majesté.
Après cela, un Ave Maria.
\end{singlequote}
\paragraph{Thérèse de l'enfant Jésus} Approche spirituelle nouvelle entre une spiritualité du XIX assez soft et cette nouvelle spiritualité : Rôle de Leon XIII : travail sur l'exegese, sur la spiritualité. Au fond, cela a renouvelé la spiritualité.

\paragraph{Notion de Nazaeth} Marie de la Passion \sn{\href{https://fr.wikipedia.org/wiki/Marie_de_la_Passion_de_Chappotin}{Marie de la Passion}}, 10000 religieuses en 1950. Tjs une référence à Nazareth. Il faut retourner au fondamental de la vie chrétienne. \textit{au service des autres}. 

\subsection{et les missions aujourd'hui dans tout cela ? }

\paragraph{expérience fondamentae de Benis Abbes} vie simple. pourrait expliquer son désir de la vie cachée. Désappropriation de lui-même.


\paragraph{tension entre l'abaissement et la tension} Jésus ne se manifeste pas dans l'ouragan mais la brise légère. De plus en plus, de la vie de Nazareth à la Passion. C'est finalement l'humilité, l'abaissement. 

\paragraph{les échecs féconds} Une négativité qui n'est pas enfermée sur lui-même. De son vivant, c'est un échec, personne ne le suit, personne converti.


\paragraph{Humiliation - humus} une terre qui rend du fruit à la terre.
Mission à long terme.


\paragraph{Mission de la présence} par ce que l'on est, cette simplicité, au milieu de toute culture : Montchanin, Thibirine.
Il va y avoir un \textit{langage de la présence}, sans parole. Langage de la gestuelle. Mère Teresa, prenant la main des lépreux.

L'Evangile donne une gestuelle, un langage de la présence, \textit{style chrétien}. Subtilité féminine de la présence de l'art d'être l'évangile. 

\paragraph{pas une stratégie, pas d'idée derrière la tête : la grâce} on ne va pas voir les autres pour convertir. Grâce de Dieu qui s'approche de nous. Même idée ici évangélique. \sn{p. 227}

\paragraph{est ce missionnaire ou pas ?} interroge notre vision de la mission. 

\paragraph{frère universel} Conversion de cette vision par la rencontre des musulmans. En 1901, il a failli mourir du scorbut et est sauvé des femmes touaregs : "sa deuxième conversion". On voit l'autre  et la mission différemment. se laisser convertir par l'autre.


\begin{singlequote}
    La joie (Claudel Soulier de Satin)
\end{singlequote}


\paragraph{Théorie de l'enfouissement}

\paragraph{Thérèse et le père Roulland, MEP} \href{https://missionsetrangeres.com/lunion-apostolique-de-sainte-therese-et-du-pere-adolphe-roulland-missionnaire-en-chine/}{Père Roulland et Thérèse de Lisieux}. Lire la correspondance.
