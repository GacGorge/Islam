 

\chapter{« La nouvelle évangélisation pour la transmission de la foi chrétienne »}
\section{XIIIe Assemblée générale ordinaire du Synode des Évêques – 2012}
\mn{Julien Tellier – séance du 21 novembre 2023}
\section{Introduction}
Le terme « nouvelle évangélisation » est si diffusé qu’il revêt une multitude de sens si bien que nous
ne savons plus très bien ce qu’il signifie. Aussi il est nécessaire de revenir aux sources de ce concept
pour mieux comprendre ce qu’il désigne et en quoi il est pertinent pour la mission de l’Eglise dans
notre monde sécularisé. C’est ce que nous faisons en étudiant un passage de l’\textit{Instrumentum Laboris}
tel qu’il a été travaillé lors de la 13e assemblée générale ordinaire du synode des évêques qui a eu lieu
en 2012.
Dans une première partie nous verrons dans quel contexte le synode s’inscrit. Ensuite nous
examinerons quelle méthode a permis la rédaction de l’Instrumentum Laboris. Enfin nous
parcourrons le texte. Ce faisant nous essayerons de comprendre en quoi le terme de « nouvelle
évangélisation » peut éclairer celui de mission.
\subsection{1. Contexte historique}
Avant d’aborder l’\textit{Instrumentum Laboris}, je fais un rapide excursus contextuel et historique sur le
terme nouvelle évangélisation. Je m’appuie sur l’article « la nouvelle évangélisation selon les
lineamenta » publié par Henri-Jérôme Gagey, prêtre du diocèse de Créteil et professeur de théologie
à l’Institut Catholique de Paris. Cet article est publié dans la revue Lumen Vitae de juin 20121.

\begin{Def}[Nouvelle évangélisation]
    La nouvelle évangélisation parcourt toute l’histoire de l’église d’occident. Elle part de l’idée que
l’évangélisation est toujours à renouveler : bien qu’évangélisé, on ne devient jamais totalement
chrétien. 
\end{Def}
Cette conviction fonde l’importance des missions à partir du concile de Trente. Il y a l’idée
d’une ré-évangélisation, c’est-à-dire refaire venir à la foi les hommes et les femmes qui s’en éloignent.
Le terme prend ensuite un sens différent à cause de la déchristianisation issue des Lumières qui s’est
majorée au XIXème siècle et à la séparation de plus en plus grande entre la culture moderne et la
tradition chrétienne.
Une première tentative de ré-évangélisation est menée par les mouvements d’Action Catholique, dont
l’objectif était de rechristianiser les frères éloignés de la foi : « nous referons chrétiens nos frères ».
Aujourd’hui, nous sommes désormais dans une ère postchrétienne ou postmoderne dans laquelle le
christianisme s’est comme éclipsé de la vie de nombreux individus.
\mn{1 Henri-Jérôme Gagey, « La nouvelle évangélisation selon les Lineamenta », p. 153-162, Revue Lumen Vitae, volume
LXVII, juin 2012, n°2}
 
L’expression « nouvelle évangélisation » apparait pour la première fois en 1979 dans une homélie du
pape Jean-Paul II prononcée à Mogila en Pologne : \begin{singlequote}
« Une nouvelle évangélisation est commencée,
comme s'il s'agissait d'une deuxième annonce, bien qu'en réalité ce soit toujours la même » 

2
\end{singlequote}
L’expression sera reprise tout au long du pontificat de Jean-Paul II, elle apparaissait déjà en germe
dans l’exhortation apostolique du pape Paul VI, Evangelii nuntiandi dans laquelle il évoquait des
« temps nouveaux d’évangélisation »

3. Pour lui « l’action évangélisatrice de l’Église doit chercher
constamment les moyens et le langage adéquats pour proposer ou reproposer la Révélation de Dieu
et la foi en Jésus-Christ »

4
Partant de ce constat une nouvelle évangélisation est nécessaire et c’est précisément l’objet de la 13e
assemblée générale du synode des évêques intitulée « La nouvelle évangélisation pour la transmission
de la foi chrétienne » que nous étudions aujourd’hui et qui s’est tenue du 7 au 28 octobre 2012 à
Rome.
\paragraph{2. Présentation des Lineamenta}
Au cours d’une conférence de presse, Mgr Nikola Eterović, secrétaire général du synode présentait
les lineamenta, document publié en 2011 et particulièrement important pour le processus synodal5.
Ce dernier a été rédigé par le conseil ordinaire du secrétariat général du synode des évêques avec
l’aide d’experts. Il fait suite à la création en 2010 par le pape Benoit XVI du conseil pontifical pour
la promotion de la nouvelle évangélisation. Traduit en huit langues il vise le débat au niveau de
l’Eglise universelle. Pour ce faire il propose un questionnaire se composant de 71 questions
regroupées en trois parties et précédées par des apports théologiques. Il a été envoyé aux diocèses,
paroisses, congrégations, mouvements, associations, groupes de fidèles… qui ont pu travailler ces
questions, y répondre et rédiger une synthèse. L’ensemble a été collecté et synthétisé pour le
secrétariat du synode pour établir l’Instrumentum Laboris texte publié le 27 mai 2012 (en la fête de
la Pentecôte) que nous travaillons aujourd’hui.
Le texte des \textit{Lineamenta} se compose de trois parties : discernement du contexte (ch.1 Le temps d’une
nouvelle évangélisation), définition des modalités (ch.2 Proclamer l’Evangile de Jésus-Christ) et
présentation des instruments (ch.3 Initier à l’expérience chrétienne).
 
\mn{3 Exhortation apostolique Evangelii Nuntiandi du pape Paul VI sur l’évangélisation dans le monde moderne, le 8
décembre 1975, n°2
4 Ibid., n°56}
 
Une des intentions des Lineamenta est de susciter le débat quant à ce concept de « nouvelle
évangélisation », concept renvoyant à un procédé dynamique que le texte présente de la manière
suivante : « l'effort de renouvellement que l'Eglise est appelée à faire pour être à la hauteur des défis
que le contexte social et culturel contemporain pose à la foi chrétienne »6.
Les défis suscités par la nouvelle évangélisation sont travaillés dans les six grands thèmes appelés
scènes : sécularisation, migration, moyens de communication, domaine économique, recherche
scientifique et technologique et enfin la politique. Nous trouvons dans les Lineamenta une
caractéristique de la nouvelle évangélisation comme étant une dynamique de sortie, évitant le repli
sur soi, l’autosuffisance ou le statut quo pour quitter le critère confortable du « on a toujours fait
comme ça »7 (business as usual).
\paragraph{3. Instrumentum Laboris8}
L’Instrumentum Laboris est donc l’étape suivante du processus synodal, à savoir le résultat de la
synthèse des réponses aux Lineamenta, il en porte de nombreuses traces et constitue la feuille de route
du synode. Il se compose des quatre chapitres suivants : 
\begin{itemize}
    \item Jésus-Christ, Evangile de Dieu pour l’homme
    \item Le temps d’une nouvelle Evangélisation
    \item Transmettre la foi
    \item Raviver l’action pastorale.
\end{itemize}
Nous lisons
aujourd’hui le chapitre 2 dont l’objectif est d’opérer un discernement sur la manière de vivre la foi
dans les communautés aujourd’hui. Parcourons ce chapitre.
Partant du fait que la mission de l’Eglise demeure inchangée depuis l’invitation du Christ à « aller
dans le monde entier proclamer l’évangile »9 l’Instrumentum Laboris pose le constat que cette
annonce semble plus difficile aujourd’hui que par le passé. Cet élément été déjà constaté par les pères
conciliaires lors du concile Vatican II et cela doit conduire à une étude approfondie pour en
comprendre les raisons. Les transformations du monde au fil des décennies ont apporté des éléments
positifs comme le développement de la culture et la croissance de l’homme dans bien des domaines
mais elles ont aussi remis en cause un certain nombre de valeurs et de fondements communs altérant
ainsi la foi des personnes et conduisant à un individualisme sans transcendance.
\mn{6 XIIIème assemblée générale ordinaire, la nouvelle évangélisation pour la transmission de la foi chrétienne,
lineamenta, n°5
7 XIIIème assemblée générale ordinaire, la nouvelle évangélisation pour la transmission de la foi chrétienne,
lineamenta, n°10
8 XIIIème assemblée générale ordinaire «  
9 Mc 16, 15}
 
Dans ce contexte de crise, que cela soit à l’intérieur ou à l’extérieur de l’Eglise, celle-ci doit rendre
raison de l’espérance qu’elle porte10 et cela dans un contexte qui est de plus en plus sécularisé. C’est
en cela que consiste la nouvelle évangélisation qui est considérée comme une exigence, une opération
de discernement et un encouragement pour les communautés chrétiennes et l’Eglise.
L'exigence d’une « nouvelle évangélisation »
L’Instrumentum revient sur la définition du concept de « nouvelle évangélisation » tel qu’il est
présenté par Jean-Paul II dans une formule désormais célèbre. Dans un discours adressé en 1983 à la
XIX assemblée du CELAM le pape Jean-Paul II11 indique : 
\begin{singlequote}
    « la commémoration du demi millénaire
d'évangélisation aura sa signification totale si elle est votre engagement comme évêques, unis à vos
prêtres et fidèles; engagement non de ré-évangélisation mais d'une nouvelle évangélisation. Nouvelle
par son ardeur, par ses méthodes, dans son expression ». 
\end{singlequote}Trente ans plus tard, le pape interpelle
l’Europe dans son exhortation apostolique Ecclesia in Europa dans laquelle il rappelle que le
christianisme ne doit pas uniquement se référer à une ancienne tradition mais doit être renouvelé par
la rencontre personnelle du Christ et de son message, aujourd’hui et demain12. Pour cela, les chrétiens
doivent s’enraciner solidement en la présence du Christ ressuscité en étant guidés par l’Esprit-Saint
et en goûtant la communion avec le Père en Jésus.
Les réponses à la question « qu’est-ce que la nouvelle évangélisation ? » formulée dans les
Lineamanta coïncident avec la définition du pape Jean-Paul II. Nous constatons que tous les pays du
monde sont confrontés à des phénomènes sociaux et culturels d’affaiblissement des institutions et des
traditions religieuses. Pour le catholicisme les signes de ce constat sont nombreux : faiblesse de la vie
de foi, non reconnaissance du magistère, rupture dans la transmission de la foi aux générations
futures.
Dans ce contexte culturel nouveau, l’Eglise doit entendre l’appel à s’engager sur la voie de la
proposition évangélique ad intra. Si beaucoup d’églises n’ont pas suffisamment pris conscience de
cette sécularisation et de ses influences négatives, des chrétiens essayent de vivre de leur foi dans ces
contextes difficiles et font rayonner l’Evangile.
\paragraph{Les scènes de la nouvelle évangélisation}
Dans cette partie, l’Instrumentum Laboris aborde les thèmes cités précédemment : sécularisation,
migration, moyens de communication, domaine économique, recherche scientifique et technologique
\mn{
\textit{10 1 P 3, 15
11 Jean-Paul II, Discours à la XIXème Assemblée du CELAM (Port-au-Prince, 09.03.1983), 3 : AAS 75 I (1983) 778.
12 Jean-Paul II, Exhortation Apostolique Post-synodale Ecclesia in Europa (28.06.2003)}}
 
et enfin politique. Il s’agit d’analyser aujourd’hui comment les chrétiens vivent leur foi et en
témoignent dans ces différentes sphères.
\paragraph{La scène culturelle :}
 elle est marquée par la sécularisation visible surtout en occident qui se
caractérise par une image positive de libération d’une humanité sans transcendance. Le militantisme
athée a cédé sa place au développement d’une culture basée sur l’hédonisme et le consumérisme dans
laquelle Dieu est absent. La sécularisation est aussi à l’œuvre à l’intérieur même des communautés
ecclésiales. La question religieuse est reléguée à la sphère privée dans laquelle chacun forge sa propre
religiosité individuelle dans une certaine forme de relativisme.
Les chrétiens sont invités à rencontrer et à dialoguer avec les \textbf{non-croyants sur ce qu’ils ont en
commun : l’humain.} Cette confrontation est féconde car elle permet au croyant de \textbf{purifier} sa foi en
la faisant grandir et au non-croyant à s’engager dans le témoignage.
\paragraph{La scène migratoire :}
 le brassage des cultures fait courir le risque d’une perte de repères au sujet du
sens de la vie ou des valeurs communes. Dans ce contexte mouvant, il y a de moins en moins de place
pour les grandes traditions y compris religieuses. Il s’agit de la mondialisation qui peut être négative
si elle se limite à la seule sphère économique et productive mais qui peut aussi apporter une richesse
lorsqu’elle œuvre en vue du bien commun.
\paragraph{La scène économique :}
 les réponses collectées aux \textit{Lineamenta} montrent des tensions et de la
violence causées par le manque de justice dans la répartition des biens non seulement à l’intérieur des
nations mais aussi entre elles. Les riches sont de plus en plus riches et les pauvres plus pauvres.
L’engagement concret des communautés chrétiennes contribue à lutter contre la pauvreté.
\paragraph{La scène politique }
: la sortie du communisme et la fin de la guerre froide ont permis des possibilités
nouvelles avec l’apparition à l’échelle mondiale de nouveaux acteurs économiques, politiques et
religieux. Ce qui est une richesse peut aussi constituer un risque lorsque la domination ou le pouvoir
guettent. Dans ce contexte, les réponses aux Lineamenta soulignent l’importance de l'engagement
pour la paix […] la recherche de formes possibles d'écoute, de vie en commun, de dialogue et de
collaboration entre les différentes cultures et religions […] ; la sauvegarde de la création et
l'engagement pour l'avenir de notre planète.
\paragraph{La scène scientifique et technique}
 : le document note les bénéfices apportés par le progrès. Il
présente aussi le risque d’absolutiser la science pour en faire « une nouvelle religion » qui est comme
une sorte de gnose s’affirmant comme une nouvelle sagesse et qui instrumentalise les pratiques
religieuses.
 
\paragraph{Les nouvelles frontières de la communication }
: l’Instrumentum Laboris rappelle que la
communication est présente dans le monde entier notamment grâce aux outils numériques. Les
moyens de communication peuvent être perçus de manière positive, avec sagesse et discernement,
comme des lieux où peuvent s’élargir les potentialités humaines et où l’annonce de l’Evangile peut
se faire… mais elle n’est pas sans risque. Une attention trop grande aux besoins et aux pensées
individuelles risque d’aboutir à une culture de l’éphémère et de l’apparence. Dans ce contexte, les
chrétiens sont invités à faire entendre leur voix en faisant la promotion du patrimoine éducatif et de
savoir de la tradition chrétienne.
\paragraph{Les mutations de la scène religieuse }
: si la sécularisation atrophie un certain nombre de personnes,
elle en réveille d’autres dans de nombreuses régions du monde, en particulier chez les plus jeunes.
Ce phénomène favorise l’expérience religieuse de la personne en quête de sens mais qui peut être
confrontée au problème de sa dimension émotionnelle avec le risque d’en rester à une certaine
superficialité. Malgré les aspects positifs de la redécouverte de Dieu et du sacré se sont nichés des
phénomènes dangereux qui peuvent conduire à la violence voire au terrorisme. Dans ce contexte, la
rencontre et le dialogue avec les grandes religions est nécessaire et fructueuse.
\paragraph{En tant que chrétiens dans ces situations}
Le document fait un état des lieux de ce que vivent les communautés chrétiennes dans les différentes
scènes évoquées. Certains évoquent la distance prise par les chrétiens dans une sorte « d’apostasie
silencieuse » du fait que l’Eglise n’aurait pas donné de réponse convaincante aux défis du temps
présent. Ce constat se présente sous la forme de toute une liste d’obstacles qui empêchent le
dynamisme des communautés ecclésiales : bureaucratisation, manque d’élan missionnaire etc…
Les réponses notent aussi des succès dans chacune des thématiques évoquées pour la croissance de
l’expérience chrétienne. Nous avons par exemple noté les bienfaits du processus migratoire pour la
rencontre et l’échange de dons entre églises. La sphère économique est aussi concernée avec l’action
de nombreuses communautés chrétiennes en faveur des pauvres. La dimension œcuménique est elle
aussi évoquée, celle-ci doit s’enraciner dans la prière avant d’être réalisée dans les œuvres. Elle
souligne la nécessité d’unir les chrétiens en se laissant transformer par l’Esprit pour être toujours plus
conforme au Christ. \textit{La scène interreligieuse est un lieu de dialogue et de confrontation dans laquelle
se trouve approfondie notre propre foi. }Enfin les chrétiens des églises orientales ou persécutées
peuvent fournir leur témoignage de persévérance, de ténacité aux églises qui ne connaissent pas ces
problématiques\textit{.}


\subsection{Mission Ad Gentes, charge pastorale, nouvelle évangélisation}
Il n’est plus possible de désigner d’un côté les pays de vieille tradition chrétienne comme évangélisés
et de l’autre les terres de missions qui n’ont jamais reçu l’Evangile. C’est ce qu’affirmait déjà en son
temps Jean-Paul II dans son encyclique Redemptoris Missio : \begin{quote}
    « les frontières de la charge pastorale
des fidèles, de la nouvelle évangélisation et de l'activité missionnaire spécifique ne sont pas nettement
définissables et on ne saurait créer entre elles des barrières ou une compartimentation rigide. [...] Les
Églises de vieille tradition chrétienne, par exemple, aux prises avec la lourde tâche de la nouvelle
évangélisation, comprennent mieux qu'elles ne peuvent être missionnaires à l'égard des non-chrétiens
d'autres pays ou d'autres continents si elles ne se préoccupent pas sérieusement des non-chrétiens de
leurs pays : l'esprit missionnaire ad intra est un signe très sûr et un stimulant pour l'esprit missionnaire
ad extra, et réciproquement »13
.
\end{quote}
La nouvelle évangélisation ne désigne donc pas un nouveau modèle rendant caduc d’autres formes
d’actions mais un processus de relance de la mission fondamentale de l’Eglise. Cela se fait par un
discernement de la vie ecclésiale et pastorale afin que les communautés chrétiennes répondent
davantage au mandat de Jésus d’être envoyé pour porter la Bonne Nouvelle à tous.
\paragraph{Transformations de la paroisse et nouvelle évangélisation}
Beaucoup de chrétiens sont engagés dans la création ou le renouvellement des paroisses sur des
territoires de plus en plus étendus dans des structures désignées comme des « petites communautés
chrétiennes ». Quant aux églises de vieille tradition, elles travaillent à renouveler leurs programmes
pastoraux pour ne pas s’enfermer dans un fonctionnement purement administratif et bureaucratique.
Dans ce contexte la nouvelle évangélisation rappelle aux communautés chrétiennes l’importance de
la mission qui se fait dans la pastorale ordinaire et dans une attention particulière pour l’annonce de
la foi à tous. Dans cette entreprise il est fait mention de l’importance des communautés de laïcs, des
communautés de vie consacrée, des groupes et mouvements et des sanctuaires. Tout cela permet à
l’Eglise d’assurer sa mission en étant proche de la vie quotidienne des personnes et en leur annonçant
le message vivifiant de l’Evangile. Enfin les réponses notent enfin le manque de prêtres et proposent
la part plus active des laïcs comme collaborateurs de ces derniers.
\paragraph{Une définition et sa signification}
Lors de la création du conseil pontifical pour la nouvelle évangélisation en 2010, le pape Benoit XVI
définit ce terme de la manière suivante : 
\begin{Def}[Nouvelle Evangélisation selon Benoit XVI]
    « faisant donc mienne la préoccupation de mes vénérés 
prédécesseurs, je considère opportun d’offrir des réponses adéquates afin que l’Église toute entière,
se laissant régénérer par la force de l’Esprit Saint, se présente au monde contemporain avec un élan
missionnaire en mesure de promouvoir une nouvelle évangélisation »14. 
\end{Def}

 Les textes précisent que son
champ d’action est principalement l’occident chrétien et désignent les chrétiens qui vivent des
situations nouvelles compromettant leur foi. Pour pallier à cette difficulté le texte conseille de ramener
la question sur Dieu dans ce monde marqué par l’indifférence religieuse. Cette attention est également
celle que doivent porter d’autres communautés chrétiennes présentes dans d’autres régions du globe.
La nouvelle évangélisation est avant tout cette conversion pastorale et missionnaire orientée ad intra,
vers les chrétiens qui se sont éloignés de l’Eglise. Pour ce faire, il est nécessaire de se laisser
renouveler par l’annonce toujours nouvelle du Salut opéré en Jésus-Christ : toutes les églises et toutes
les communautés chrétiennes sont concernées par ce programme.
\subsection{Conclusion}
En collectant et en analysant les différentes réponses fournies à partir des Lineamenta, l’Instrumentum
Laboris brosse le tableau d’un monde en crise (ce mot est cité à plus de 10 reprises dans le document).
\textbf{Crise} de la vie chrétienne, crise engendrée par les défis culturels, crises économiques et politiques,
crise éducative. Le document constate que dans notre monde postmoderne la sphère de la
\textbf{sécularisation} est mondiale : elle touche tous les continents.
C’est dans ce milieu qu’apparait le concept de Nouvelle Evangélisation tel qu’il a été élaboré par
Jean-Paul II et repris ensuite par Benoit XVI. Notons que leurs textes sont abondamment cités dans
l’instrument laboris. Plus de 20 ans séparent Redemptoris Missio du pape Jean-Paul II du synode sur
la Nouvelle Evangélisation, ce qui nous amène à penser que les pères du synode ont peut-être profité
de cet anniversaire pour relire le document du pape à la lumière du contexte de 2012.
Nous constatons aussi un glissement du concept de mission qui ne concerne plus seulement la mission
ad gentes mais désormais tous les continents et même toutes les communautés chrétiennes qui
peuvent être confrontées à une sorte d’affadissement de leur foi et de leur zèle missionnaire. Comme
le disait Paul VI dans son exhortation apostolique Evangeli Nuntiandi : « l’Eglise existe pour
évangéliser »15. Nous pouvons aujourd’hui définir le concept de Nouvelle Evangélisation comme un
surcroit de vitalité de l’Eglise en réponse aux cultures marquées par la sécularisation. Bien
qu’omniprésente dans l’Instrumentum Laboris, l’expression « nouvelle évangélisation » n’est plus
citée que 14 fois dans l’exhortation apostolique Evangelii Gaudium du pape François. 
\mn{14 Pape Benoît XVI, motu proprio Ubicumque et Semper, 21/09/2010
15 Pape Paul VI, exhortation apostolique, Evangelii Nuntiandi, n°14}
 Elle est
totalement absente du synode sur les synodalités, que ça soit dans l’Instrumentum Laboris ou dans le
rapport de synthèse. Le mot \textbf{mission} par contre revient en force : il apparait plus de 100 fois dans ce
dernier document.
La nouvelle évangélisation présente ce chantier missionnaire toujours d’actualité. Il constitue une
réponse aux signes des temps qui doivent être lus avec espérance. Lors de l’audience générale du 15 novembre, le pape François disait, concernant la nouvelle évangélisation : \begin{quote}
    « […] les premiers qui
doivent être évangélisés c’est nous, chrétiens : c’est nous. Et c’est très important. Immergés dans le
climat actuel, rapide et confus, même nous en effet nous pouvons nous aussi vivre la foi avec un sens
subtil du renoncement, convaincus que l'Évangile n'est plus audible et qu'il ne vaut plus la peine de
s’engager pour l'annoncer. \textbf{Nous pourrions même être tentés par l'idée de laisser "les autres" suivre
leur propre chemin}. En revanche, c'est précisément le moment de revenir à l'Évangile pour découvrir
que le Christ "est toujours jeune et source constante de nouveauté" (Evangelii gaudium, 11) »16
.
\end{quote}


\section{discussions}

\paragraph{Un texte de synthèse, sans responsable} L'exhortation post synodale sera signée par François.  Et un document de travail pour présenter la situation, pas forcément les solutions. Après les discussions, \textit{message aux peuples de Dieu}. Il revient au terme de mission, et non \textit{et nouvelle Evangélisation}.

Chez Benoit XVI, la nouvelle Evangélisation, c'est surtout la vieille chrétienté. \textit{apostasie silencieuse}. Chez Benoit XVI, la définition c'est ceux qui ne pratiquent \textit{plus} . 
\mn{un contexte politique : théologie de la libération ? }

\paragraph{Ad Intra} Evangélisation dans son propre pays ? Mais au sein même de l'Eglise ?

\paragraph{perte du leadership de la culture} combien de fois apparait le mot inculturation ? en Asie, Afrique. C'est très curieux, car le problème majeur, c'est la perte de la culture en Europe. Au lieu de \textit{continuons notre inculturation} mais beaucoup plus de proposer \textit{une autre culture}. 

Cependant, \textit{parvis des gentils} : \cite{ravasi_parvis_2012}.


\paragraph{le style chrétien}
On parle du style chrétien. A l'opposé des fondamentalistes. 


\paragraph{Quel regard théologique ? Les signes des temps n'apparait plus} C'est repérer comment Dieu nous parle dans les évènements ?
Le risque c'est de rester à un niveau moral : 
\begin{Ex}
    la technique, c'est bien car la santé; mais c'est mal parce que nous sommes aliénés. OK mais on se situe au niveau moral.
\end{Ex}

\begin{Def}[Signes des Temps]
  Dans G\&S, Dieu continue à nous parler  aujourd'hui. Dieu ne nous abandonne pas. 
\end{Def}

\mn{Theobald et les signes des temps : savoir les lire avec les charismes}

 
\paragraph{Culture chez Benoit XVI}
 : Athènes, Rome et Jérusalem. Benoit XVI restait dans cette culture. 