\chapter{Liste des théologiens}

\section{Grands théologiens Musulmans}


\subsection{al-Ġazālī}

le joyau d'al-Ġazālī~: \emph{Le Tabernacle des Lumières}, traduit
par Deladrière, Paris, Seuil, texte très dense et très profond aux
implications nombreuses.
\cpageref{theol:AlGazali1,theol:AlGazali4,theol:AlGazali5,theol:AlGazali6,theol:AlGazali7,theol:AlGazali9,theol:AlGazali10,theol:AlGazali11,theol:AlGazali13,theol:AlGazali14,theol:AlGazali16,theol:AlGazali17,theol:AlGazali18,theol:AlGazali19,theol:AlGazali20,theol:AlGazali21,theol:AlGazali22,theol:AlGazali23,theol:AlGazali24}
\pageref{theol:AlGazali29}
\pageref{theol:AlGazali2}
\pageref{theol:AlGazali3}
\pageref{theol:AlGazali8}
%\pageref{theol:AlGazali31}
\pageref{theol:AlGazali25}
%theol:AlGazali31,theol:AlGazali32,theol:AlGazali33,theol:AlGazali34,theol:AlGazali35,theol:AlGazali36,theol:AlGazali37,theol:AlGazali38} 
%\label{theol:AlGazali1}
\section{Ibn Taymiyya}


 

C'est le très grand penseur (controversé) du 13\textsuperscript{ème}
siècle. Un certain nombre de ses ouvrages ont été traduits (souvent mal,
je sélectionne les meilleures traductions).


\emph{La lettre de Palmyre} traite de deux questions théologiques~: les
attributs divins et la prédestination~!

\includegraphics[width=1.27534in,height=1.63243in]{Images/image26.png}

-~Ibn Taymiyya, \emph{Réponse Raisonnable aux Chretiens ?} édité,
traduit et commenté par Laurent Basanese, sj., Ifpo, 2011.

-~Ibn Taymiyya, \emph{Musique et danse selon Ibn Taymiyya}: Le livre du
\emph{Samâ°} et de la danse (\emph{Kitâb al-Samâ° wa l-Raq.s}), Paris,
Vrin, 2000.

-~Ibn Taymiyya, \emph{Pourquoi les savants divergent,} Al-Hadith
éditions, 2012


Voir p. \pageref{ibn-taymiyya}.

\section{Autres théologiens classiques}
\paragraph{Ibn Hanbal}

\pageref{Theol:IbnHanbal1}

\paragraph{Ibn Salah}
Ibn Salah
\pageref{Ibnsalah1}

\paragraph{Ibn Khaldūn}
Le penseur andalou Ibn Khaldūn \pageref{theol:IbnKhaldun} 

\paragraph{Ibn Qutayba}
Ibn Qutayba -- si ce nom ne vous est pas encore familier, cela devrait
faire `tilt' car nous l'avons rencontré au début de cette leçon. Il a
écrit un Traité sur comment rendre compte et comprendre les divergences
dans le \emph{ḥadīṯ.} A-t-il été traduit en français~? La réponse est en
note 3 --
\pageref{Theol:IbnQutayba1}

\paragraph{Kalābāḏī}
  est un auteur persan, mort aux environs de 990. Cet ouvrage
cherche à réconcilier le soufisme et l'orthodoxie. 
\pageref{theol:Kalabadi}


\paragraph{ʿAlī Ṭanṭāwī} \label{theo:AliAlTawani}
{Ali Al tantawi est originaire d'une
famille de lettrés égyptiens qui a émigré à Damas à la fin du XIXème
siècle.


Il s'est opposé à l'impérialisme occidental dans les pays
arabes et, en particulier, à la présence de la France comme mandataire
en Syrie et celle de l'Angleterre en Irak. Après l'indépendance de la
Syrie, en 1947, ses positions contre le communisme, qu'il considère
incompatible avec l'Islam lui valent d'être menacé dans son propre pays.
En 1963, il quitte la Syrie pour l'Arabie Saoudite et devient
enseignant.
Extrêmement populaire dans son pays d'adoption, il a
présenté des programmes à la radio et à la télévision pendant un quart
de
siècle.}


\subsection{Ibn Toumart}
\label{IbnToumart}
\mn{E.B., « Ibn Toumart », in 23 | Hiempsal – Icosium, Aix-en-Provence, Edisud (« Volumes »,
no 23) , 2000 \href{http://
encyclopedieberbere.revues.org/1629}{revue}}


Ibn Toumart est la personnalité religieuse et politique la plus marquante du Maghreb au
XIIe siècle. Fondateur du mouvement almohade*, il devait préparer la revanche des Sanhadja
montagnards contre l’empire déjà vacillant des Almoravides. Bien que ses disciples aient
manipulé sans vergogne sa généalogie pour le rattacher à la descendance du Prophète et en
faire, donc, un chérif, il est sûr qu’Ibn Toumart était issu d’une tribu du Sous, celle des Hergha,
appartenant au groupe montagnard des Maçmouda.
 L’un de ses premiers disciples, le pieux el Baïdaq, se fit son chroniqueur et grâce à son
récit, souvent dithyrambique, il est possible de saisir l’évolution spirituelle de celui qui
devait mériter le titre de Mahdi Almohade et le qualificatif d’Impeccable. Célèbre dès son
adolescence, pour son zèle religieux et son érudition qui lui avait fait donner le surnom d’asufu
(le tison, le flambeau, en berbère), Ibn Toumart quitta un beau jour son village d’Igliz et
ses montagnes pour compléter, en Orient, sa connaissance de l’islam et jeter les bases d’une
réforme radicale.
 Son séjour en Espagne n’est pas assuré, mais demeurent des concordances étroites entre
les textes d’Ibn Hazm et ses propres propositions. En revanche, sa présence à Bagdad est
pleinement confirmée, alors que son passage à Damas est peut-être légendaire et les entretiens
qu’on lui prête avec Ghazali certainement inventés.
 Dix ans après son départ d’Igliz, Ibn Toumart entreprend un long voyage de retour au Maghreb,
au cours duquel il multiplie les étapes, passant par Alexandrie, Tripoli, Mahdia, Tunis,
Constantine et Béjaia. Sa condamnation des moeurs citadines relâchées provoque souvent des
échauffourées. A Béjaia, ses violences verbales déclenchent la fureur populaire contre lui. Le
sultan hammadite, qui l’avait d’abord bien accueilli, lança ses sicaires à sa poursuite, mais Ibn
Toumart trouva refuge dans la tribu voisine, celle des Beni Urigol, dans le village de Melala.
\paragraph{
La doctrine almohade}
 Ibn Toumart y élabore sa doctrine et réunit ses premiers disciples. Le plus cher à son coeur,
celui qu’il considère comme l’homme providentiel qui doit lui succéder, est Abd el Moumen,
le fils d’un potier de Nédroma (Algérie occidentale). El Baïdaq nous a laissé le récit émouvant
de la désignation du futur calife. Le réformateur proclama un soir en prenant sa main : “La
mission sur laquelle repose la vie de la religion ne triomphera que par Abd el Moumen ben Ali,
le flambeau des Almohades.” Celui-ci, en pleurant, dit avec humilité : “Ô Maître, je n’étais
nullement qualifié pour ce rôle, je ne suis qu’un homme qui cherche ce qui pourra effacer ses
péchés.” – “Ce qui te purifiera de tes péchés, répondit Ibn Toumart, ce sera précisément le rôle
que tu joueras dans la réforme de ce monde.”
 Une conversation avec deux pèlerins de l’Atlas qui passaient par Bougie est l’occasion du
départ des premiers Almohades vers le Maghreb el Aqsa. La petite troupe, d’une dizaine de
personnes, gagne Marrakech non sans avoir semé la bonne parole et causé quelques troubles
dans les villes traversées : Tlemcen, Oujda, Taza, Fès, où Ibn Toumart se fait remarquer par
le saccage des magasins des marchands de musique, contre lesquels il semble avoir eu une
aversion certaine. Il réitère à Marrakech, brisant à coups de bâton instruments de musique
et jarres de vin, pourchassant sous les huées la soeur de l’émir almoravide, qui chevauchait
dévoilée dans les rues de la capitale.
 Après la prise de Tin Mel (1123), il se proclame Mahdi et, de retour dans les tribus Masmouda,
ses frères de race, il organise solidement la communauté almohade avec un soin et une
connaissance des hommes qui font de ce clerc un grand homme d’État. Il crée un véritable
État montagnard, solidement organisé, disposant d’une armée fanatisée chargée de répandre
la doctrine almohade jusqu’en Ifriqiya et en Espagne.

Nous retrouvons dans cette réforme la même tendance innée vers le rigorisme moral et la
simplicité doctrinale que nous ont révélés tous les schismes et hérésies nés en Berbérie à travers
les siècles.
Dans la condamnation absolue des richesses de ce monde et de ses frivolités, c’est la voix
d’Ibn Toumart qui tonne, mais elle fait écho à celle, non moins véhémente, de Tertullien. La
lente marche des Berbères vers le Dieu unique semble ici se parachever dans la proclamation
de l’Unicité absolue de Dieu, dont Ibn Toumart rejette jusqu’aux adjectifs (le Puissant,
le Miséricordieux, le Victorieux) que lui dorment les musulmans, parce qu’ils risquent de
faire apparaître comme divisible la puissance divine. La conséquence inévitable de la toutepuissance
de Dieu ainsi comprise est la prédestination de tous les êtres créés : chacun doit
attendre dans la soumission totale ce qui lui a été assigné de toute éternité.
Cette forme de l’islam ne peut qu’être fanatique, elle ne supporte ni relâchement des moeurs,
ni relativisme dans le dogme, ni présence d’Infidèles.
11 Ces données concordaient trop bien avec l’intransigeance fondamentale des Berbères pour ne
pas aboutir : aussi, sous Abd el Moumen, le raz de marée almohade balaya le Maghreb de
toute impureté. C’est alors, semble-t-il, que disparurent les dernières communautés chrétiennes
autochtones.
\paragraph{L’État almohade}
Respectueux des traditions tribales des Berbères du Haut Atlas, Ibn Toumart organisa son
gouvernement en établissant une hiérarchie entre différents conseils imités des assemblées
tribales. Au sommet siège le Conseil des Dix, qui sont les premiers et les plus fidèles
compagnons (Abd el-Moumen*, Abou Hafs Omar*, El Bachir...). Au-dessous du Conseil
des Dix, le Conseil des Cinquante est composé de contribules d’Ibn Toumart, des Hergha et
d’autres Maçmouda de Tin Mel ou des Hintata. Les différentes tribus de la montagne étaient
ainsi représentées dans ce Conseil dont les pouvoirs étaient restreints.
Toute la société almohade était strictement hiérarchisée. A l’intérieur des groupements
ethniques apparaissait une autre hiérarchie, fondée sur les fonctions exercées, depuis celles
des compagnons les plus proches jusqu’à celles confiées aux abid (serviteurs noirs). Au
sommet de la pyramide, le Mahdi tenait solidement les rênes d’un pouvoir absolu. Cette
domination reposait sur une logique implacable : tout Almohade suspecté de tiédeur risquait
l’élimination : ainsi lors de la “journée du tri” plusieurs milliers d’almohades “infidèles” furent
massacrés. C’est par de telles actions qu’Ibn Toumart réussit à construire l’État almohade et à
assurer la naissance de la nouvelle dynastie moumenide. Seuls le prestige et la volonté d’Ibn
Toumart réussirent à faire admettre Abd el-Moumen comme le successeur désigné du Mahdi.
Encore fut-il nécessaire de cacher la mort de celui-ci pendant plus de deux ans avant de faire
reconnaître le nouveau souverain par les Cheikhs almohades.
\paragraph{références}
voir p. \cpageref{theol:IbnTumart1}

\section{Théologiens modernes}
\subsection{Tarik Ramadan}
 \begin{itemize}
  \item Paradoxe de Tarik Ramadan~: dire que le radicalisme vient de
    l'occident. Et ensuite, valoriser le communitarisme pour refuser les
    coutumes locales et en particulier celles de la laicité.
  \end{itemize}
  
Voir réflexion sur le moratoire.

%-------------------------------------------------
\section{Théologiens pronant un retour à l'Islam pur}
\label{theol:SayyidQutb}
\paragraph{Sayyid Qutb}
Sayyid Qutb (arabe :\TArabe{ سيد قطب,} Sayyid Quṭb), né le 9 octobre 1906, dans le sud de l'Égypte, et exécuté par pendaison le 29 août 1966 au Caire, est un poète, essayiste et critique littéraire égyptien, puis un militant musulman membre des Frères musulmans. Il entrera en rupture avec ces derniers à la suite du développement d'une idéologie islamiste radicale, le \textbf{qutbisme}.


Les idées de Sayyid Qutb se résument schématiquement ainsi :
\begin{itemize}
    \item 
L'islam est en crise. Les millions de gens qui se réclament de l'islam n'en comprennent en réalité pas grand-chose, ils ne sont pas de vrais musulmans. Qutb prononce donc une condamnation très forte de la société égyptienne contemporaine.
  \item 
Un retour aux vraies valeurs de l'islam est nécessaire. Malheureusement les masses populaires manipulées par le nassérisme sont incapables de s’en sortir. Il appartient donc à une élite de guider les masses en jouant le même rôle que celui des compagnons du prophète de l'islam; cette élite qu'il appellera dans plus d'un ouvrage "\textit{annawâte assoulba}" (littéralement "le noyau dur"). Le but est de réislamiser la société.
  \item 
L'islam apporte une solution complète à tous les problèmes, politiques, économiques, sociaux. En revanche, les influences occidentales sont dangereuses et nuisibles. Il dénie le qualificatif de « civilisation » (notamment dans son livre Mushkilât al-hadâra : Problèmes de la civilisation) aux blocs de l'est (socialiste) et de l'ouest (capitaliste), qu'il renvoie dos à dos comme représentant deux faces d'une même entité qu'il qualifie de \textit{Jahiliya} (ignorance). Ce terme, qui renvoie à la période précédant l'islam durant laquelle l'Arabie était polythéiste et ignorante donc du vrai Dieu, a une forte connotation négative dans l'imaginaire musulman.
  \item 
L'idée d'une « lutte contre les Juifs » fut aussi présente dans la pensée de Sayyid Qutb, qui écrivit au début des années 1950 l'opuscule \textit{Notre combat contre les Juifs}. Dans son commentaire de la sourate 5, Sayyid Qutb réaffirmera l’accusation : « Depuis les premiers jours de l’islam, le monde musulman a toujours dû affronter des problèmes issus de complots juifs. (…) Leurs intrigues ont continué jusqu’à aujourd’hui et ils continuent à en ourdir de nouvelles. » 
\end{itemize}

%--------------------------------------------------
\section{Théologiens libéraux}

\paragraph{Mohammed Arkoun}
Savant à la pensée profonde, Mohammed Arkoun (1928-2010) était également un intellectuel engagé. Son analyse serrée des processus à l’œuvre dans l’islam d’hier était indissociable de ses appels répétés à une réforme des sociétés islamiques contemporaines. Il n’a cessé de porter ce message dans les divers colloques où il était convié, y compris là où l’on ne s’attendrait guère à croiser un islamologue : à un congrès de psychanalystes lacaniens, dans des conférences sur la condition féminine…
Il a choisi de consacrer les dernières années de sa vie à retravailler les textes issus de ces rencontres. Traitant de la nécessité de la réforme, voire de la « subversion » de l’islam, de l’ouverture lacanienne à la parole et à la « raison émergente », de la condition féminine en islam ou encore du rapprochement entre sunnites et shî‘ites, ils montrent combien la pensée de Mohammed Arkoun est plus que jamais féconde pour penser notre époque.
Voir  \cpageref{theol:Arkoun1,theol:Arkoun2} 
\label{theol:Arkoun3}

\paragraph{Rachid Benzine}.
Islamologue et historien, Rachid Benzine s’est intéressé à ces questions après sa rencontre avec le prêtre catholique Christian Delorme à Lyon et a beaucoup travaillé avec des théologiens protestants.



\section{Islamologues}

\subsection{Louis Massignon}


L’islamologue Massignon s’est avant tout situé dans la grande lignée des études sur l’Islam orthodoxe, son premier souci étant de démontrer que l’Islam a une dimension mystique et que c’est l’Islam sunnite essentiellement qui se prête à cette dimension. Mais que ce soit à travers son thème de recherche principal, Ḥallāğ, ou dans le reste de son œuvre, dans ses cours au Collège de France et à l’École des Hautes Études, dans ses incessants déplacements en Iran, en Syrie, au Liban, il s’est heurté au šī‘isme à tous les carrefours.

  \paragraph{Mansur al-Ḥallāğ pour Massignon} figure du Christ de l'Islam.

  Dieu demande à toutes les créatures devant l'homme. Une créature
  refuse. Traditionnellement, c'est considéré comme un refus
  d'obeissance de l'ange (c'est de la boue puante~»). Pour al Hallaj
  c'est uniquement devant Dieu et c'est une épreuve de Dieu, c'était
  révélé à l'ange ce qu'est Dieu. Dieu est dans l'homme.
  \url{https://www.youtube.com/watch?v=Let1X-8zsXU\&t=1428s}
  C'est à cause de cette divinisation qu'il sera condamné, hétérodoxe.

\paragraph{Pierre Lory}
\label{Theol:PierreLory}
Un des grands connaisseurs de Qušayrī est
Pierre Lory.

Même en dehors des cercles salafistes, nombreux sont aujourd'hui les
musulmans~

\begin{cite}
« qui voudraient que le Coran soit un discours unique,
et qui se méfient des interprétations différentes les unes des autres
»,~
\end{cite}
\emph{}déplore~{{Pierre
Lory}}. Cet islamologue a contribué au récent site {{Coran 12-21}}
Internet~\url{https://coran12-21.org/fr/} , qui
présente différentes versions du Coran, dans trois langues différentes.

Pour lui, comme pour d'autres spécialistes, considérer le Coran comme un
tout dogmatique et intouchable est non seulement dangereux, mais aussi
erroné.

\paragraph{Abdessamad
Belhaj}
\begin{cite}
« La lecture littérale est en elle-même une
interprétation, puisqu'elle est fondée sur la prémisse que les propos du
Coran sont généralisables et peuvent faire fi des circonstances
particulières »,~
\end{cite}
remarque
l'islamologue~{\underline{Abdessamad
Belhaj}}, chercheur au Centre interdisciplinaire d'études de l'islam
dans le monde contemporain de l'Université catholique de Louvain.

\subsection{Autres Islamologues - IDEO}
\paragraph{Emmanuel Pisani}
Frère dominicain et islamologue, Emmanuel Pisani est directeur de l’Institut de Science et Théologie des religions (ISTR) à l’Institut catholique de Paris.

\paragraph{Adrien Candiard}



\section{Matériel d'Etude}\label{matuxe9riel}

\url{https://www.onelittleangel.com/livres/sacres/le-saint-coran.asp}

\url{https://coran.oumma.com/sourate/20} . Conseillé par Emmanuel
Pisani.

\url{https://www.lexilogos.com/clavier/arabe_latin.htm}

\url{https://referenceworks.brillonline.com/browse/encyclopedie-de-l-islam}

\url{https://www.lexilogos.com/coran.htm} : lexilogos y compris mot à
mot




\section{Notes diverses à repositionner}







Important de connaître un auteur pour avoir un avis objectif.





\begin{itemize}
\
  
\item
  Est-ce que j'utilise la raison, l'analogie, la coutume locale~? c'est
  cela les différences.
\item
  Si la coutume est la laicité, je dois en tenir compte pour mon avis.

  \begin{itemize}
  \item
    Le shavinisme qui intègre le ur, la coutume,
  \item
    le hanafisme, aussi~
  \item
    la où on le ferait moins, c'est le hanbalisme.
  \end{itemize}
\end{itemize}



  Pour quitter l'Islam, la peine est celle de la Loi locale. En France,
  si on définit l'islam comme une loi, on dit son aversion. Tout le
  chapitre sur la loi, «~islam = loi~», est un schématisme redoutable.
  Ce n'est pas qu'un texte législatif. Malheureusement, il y a tout un
  courant dans l'Islam qui encourage cette lecture caricaturale. Dans
  les pays occidentaux, on peut combattre avec les idées l'islam
  radical.
  
  
  \subsection{Islam compliqué}


Rachid Benzime

Islam, compliqué à lire le coran sans clé herméneutique

Passé colonial~en France~:

champ sémantique~: passer d'un champ indigène, à immigré, à ~musulman.
La religion devient le marqueur identitaire.

Pisani~:

Compliqué l'islam~; islam~: complexe

Edgar Morin~: c'est quoi la complexité d'un fait social. Si complexe,
reponse complexe

Ici, les arrières pensées~: on croit connaitre de l'islam alors que ce
n'est qu'une réalité.

Les musulmans comme «~citadelles assiégées~»

Difficile pour les musulmans de voir une certaine réalité car l'islam
quelque chose de beau.

«~un terroriste qui se dit musulman, on n'a pas le droit de lui dire
qu'il n'est pas musulman~»

Derrida~: «~il faut bien séparer l'Islam de l'Islam~».

Accepter que l'islam est pluriel, alors que l'islam est vécu par les
personnes comme unique.

Pisani~:

Macron~: l'islam est en crise.

Ce n'est pas possible pour les musulmans~: «~l'islam ne peut pas être en
crise~» - méta religieux.

Mohammed Arkoun~: le fait islamique. Dieu est absent.

Trop de représentations dans le champ «~Islam~». Mot trop chargé.

What is Islma.

\section{Le Coran peut-il être interprété ?}

 

Considéré par la plupart des musulmans comme un livre « incréé », et
donc parfois intouchable, le Coran a fait l'objet, au Moyen Âge, d'une
tradition de commentaires d'une grande profusion. 

\begin{itemize}
\item
  Mélinée Le Priol,~
\end{itemize}


Plusieurs anecdotes transmises par la tradition islamique montrent un
croyant venant consulter le prophète Mohammed sur le sens de tel ou tel
verset.

\subsection{Pourquoi l'interprétation du Coran est-elle un sujet sensible
?}

Synonyme de « récitation »,~\emph{Al Qur'an}~(en français le Coran)
contient, selon la tradition islamique, la révélation reçue par le
prophète de l'islam Mohammed, entre 610 et 632. L'ange Gabriel lui
aurait dicté les versets tels quels, et ceux-ci auraient été mis par
écrit une vingtaine d'années après la mort du Prophète, qui n'aurait
fait que les réciter à ses compagnons. Malgré son statut bien connu
de~\href{https://www.la-croix.com/Religion/Islam/Comprendre-Coran-2016-06-10-1200767802}{\underline{livre
« incréé »}}, le Coran a été abondamment étudié et commenté.

\emph{« Le courant littéraliste, qui considère que le Coran se suffit à
lui-même, que ses ambiguïtés sont voulues par Dieu et que l'interpréter
est source d'égarements, a toujours existé, mais il a longtemps été très
marginal en islam »,}~rappelle l'historien Mohammad
Ali Amir-Moezzi 
 . C'est depuis une cinquantaine d'années, avec
l'essor du salafisme, que cette conception a gagné du terrain,
valorisant essentiellement l'apprentissage par cœur.
 
 

\subsection{ Quelle est l'histoire du commentaire coranique ?}

Plusieurs anecdotes transmises par la tradition islamique montrent un
croyant venant consulter le prophète Mohammed sur le sens de tel ou tel
verset. Il faut dire que le texte coranique est un corpus
particulièrement ardu, au contenu souvent allusif, parfois
contradictoire. Non content d'entremêler des thèmes et registres
différents, il n'a pas été agencé selon l'ordre chronologique de la
révélation.

 
D'où la nécessité de l'interpréter. Riche de plusieurs milliers de
volumes, le commentaire du Coran a connu son âge d'or du
VIII\textsuperscript{e}~au XII\textsuperscript{e}~siècle.\emph{~« Les
grands courants de pensée islamiques se sont tous développés à partir de
la même interrogation : comment comprendre l'écriture sainte
?,~}explique Mohammad Ali Amir-Moezzi.\emph{~­L'islam a hérité de cette
culture exégétique des milieux bibliques au sein desquels il s'est
développé. »}

Les commentateurs ne pouvaient toutefois pas interpréter le Coran à leur
guise. On dénombre trois méthodes principales : la traditionnelle avait
essentiellement recours aux sources scripturaires (le Coran et les
hadiths) et à des analyses sur la langue arabe et la culture tribale ;
la rationnelle faisait appel à la logique et à la pensée spéculative, la
logique aristotélicienne ; et la mystique reposait sur une~\emph{«
illumination »}.

 

\emph{« Une posture postmoderne veut que le Coran soit dorénavant ouvert
à toute interprétation,~}s'inquiète Abdessamad Belhaj.\emph{~Mais cela
ne doit pas être une excuse pour ne pas explorer l'intelligence interne
du texte lui-même. »}~Le lecteur contemporain du Coran doit tout de même
recouvrer son~\emph{« autonomie »,}~reconnaît-il toutefois, regrettant
que l'abondante littérature produite dans les siècles passés ait pu
être~\emph{« sacralisée »}.

\subsection{► Comment renouveler l'interprétation du Coran au
XXI\textsuperscript{e}~siècle ?}

Si la~\emph{« quasi-totalité »}~des commentaires du Coran se font,
encore aujourd'hui, dans le registre traditionnel,~\emph{« on sent
depuis trois décennies les frémissements d'une nouvelle exégèse, qui
recourt davantage aux méthodes académiques occidentales »,~}observe
Mohammad Ali Amir-Moezzi. Non confessante, cette islamologie née dans le
monde occidental commence à trouver un écho dans des pays musulmans
comme l'Iran, la Tunisie ou la Turquie.

 

Pour ces chercheurs, l'enjeu est de ne plus seulement étudier le Coran à
partir des sources musulmanes datant d'au moins un siècle et demi après
la mort de Mohammed, mais de recourir aussi aux sources non musulmanes
(notamment juives et chrétiennes) du contexte religieux de l'Antiquité
tardive au sein duquel le Coran a émergé. Longtemps restées cloisonnées,
ces deux approches -- confessante et scientifique -- pourraient bientôt
se réconcilier.

Islam, pourquoi cette sévérité avec les autres croyants et les
incroyants ?

\emph{Explication~}

« Mécréants », « infidèles » : les terroristes islamistes s'en sont pris
violemment, ces dernières années, à tous ceux qu'ils jugent hors de
l'islam « authentique ». Une intolérance fondée sur une lecture
littérale du Coran. 

\begin{itemize}
\item
  Mélinée Le Priol,~
\item
  le~28/01/2021 à 13:04~
\item
  Modifié le~28/01/2021 à 13:12
\end{itemize}

Lecture en 3 min.

\includegraphics[width=6.3in,height=4.20208in]{media/image4.jpeg}

Selon la théologie musulmane, l'islam est la religion originelle de
l'humanité.VICTOR MOUSSA - STOCK.ADOBE.COM

\subsection{► Que dit la tradition ?}

Selon la théologie musulmane, l'islam est la religion originelle de
l'humanité.~\emph{« Tout homme est né musulman »,}~dit un hadith
attribué
au~\href{https://www.la-croix.com/sacralite-prophete-lislam-2020-11-06-1101123195}{\underline{prophète
Mohammed}}. L'homme est né pour adorer Dieu : certes, il a reçu une
dignité plus haute que les autres créatures, mais celle-ci est
conditionnée à sa soumission au Dieu unique. Plus un homme applique la
loi divine (\emph{charia}), plus il devient humain. Quant au « mécréant
» (\emph{kâfir}), qui refuse de suivre la charia, il se situe en quelque
sorte à un degré inférieur d'humanité.

Cette sévérité envers les non-musulmans s'appuie sur la lecture du texte
coranique qui s'est imposée à partir du IX\textsuperscript{e}~siècle,
lors de la transformation de l'islam en un empire soucieux de se
légitimer. Confortée par des hadiths rédigés à cette époque, elle
dépeint une vérité unique et non négociable. Elle insiste sur les
versets du Coran particulièrement virulents envers les polythéistes,
païens ou idolâtres, qualifiés d'\emph{« associateurs
»}~(\emph{mouchrikoun}) car ils « associent » à Dieu d'autres divinités.

Quant aux athées,~\emph{« ils appartiennent, selon la théologie
musulmane, à une catégorie de mécréance encore inférieure aux
polythéistes, aux juifs et aux chrétiens »,~}explique l'islamologue
Abdessamad Belhaj, chercheur au Centre interdisciplinaire d'études de
l'islam dans le monde contemporain de l'Université catholique de
Louvain. Même si des institutions comme le Haut Conseil des oulémas du
Maroc ou la Maison de la fatwa en Égypte considèrent que les apostats ne
peuvent plus être condamnés à mort, cette peine reste appliquée dans une
dizaine de pays, comme l'Afghanistan ou
la~\href{https://www.la-croix.com/Monde/Afrique/prisons-Mauritanie-calvaire-dun-apostat-2019-09-30-1201051050}{\underline{Mauritanie}}.

\subsection{ Pourquoi juifs et chrétiens bénéficient-ils d'un statut
spécifique ?}

Selon la tradition musulmane, chrétiens et juifs font l'objet d'un
traitement différent des autres non-musulmans : ils bénéficient dans le
droit islamique d'une protection juridique particulière (\emph{dhimma})
toutefois accompagnée d'injonctions humiliantes, comme l'interdiction de
monter à cheval ou de construire des lieux de culte dépassant ceux des
musulmans.
 

\emph{« Le Coran est très ambivalent au sujet des ``gens du Livre''
»,~}rappelle
l'historien~\href{https://www.la-croix.com/Culture/Livres-et-idees/historiens-decryptent-Coran-avant-lislam-2019-11-27-1201063090}{\underline{Guillaume
Dye}}, professeur à l'Université libre de Bruxelles (1). Selon la
sourate 5, juifs et chrétiens ne doivent pas être pris pour~\emph{«
alliés »~}(5, 51) mais, quelques versets plus loin, on lit qu'ils ne
seront~\emph{« point affligés »~}(5, 69). Les chrétiens se voient
reprocher de nier l'unicité de Dieu mais du respect est exprimé pour les
prêtres et les moines, qui~\emph{« ne s'enflent pas d'orgueil ».}

Selon une théologie dite de la falsification (\emph{tahrif}), les juifs
et les chrétiens ont altéré le message transmis par leurs prophètes
respectifs (Moïse, Jésus), message qui n'était autre que l'islam. Le
Coran, lui, corrige cette déviation en transmettant fidèlement le
message révélé à un ultime prophète, Mohammed. À Médine, celui-ci aurait
signé une~\emph{« Constitution »~}disposant que les juifs, notamment,
pouvaient pratiquer leur religion en sécurité, mais ces relations se
sont rapidement détériorées.

\subsection{► Quelles pistes pour une « théologie du pluralisme » ?}

Les attentats visant des « mécréants » en terrasse à Paris, les
persécutions contre les Yézidis ou les chrétiens en Irak, sont autant de
conséquences d'une lecture littéraliste du Coran encouragée par l'essor
du salafisme saoudien à partir des années 1970. D'autres lectures ont
pourtant existé dès les premiers siècles de l'islam. Contrairement à la
doctrine sunnite traditionnelle, l'exégèse rationaliste a par exemple
conclu très tôt à une~\emph{« égalité entre tous les êtres humains, tous
étant dotés de la même raison les rendant aptes à comprendre la parole
de Dieu »,~}rappelle l'islamologue Pierre Lory, directeur d'études à
l'École pratique des hautes études (EPHE).

Pour Abdessamad Belhaj, tout l'enjeu est aujourd'hui de refonder le
rapport à l'altérité sur la base de l'éthique, et de\emph{~« mettre
l'homme au cœur de la théologie »}. Pour cela, certaines valeurs
présentes dans l'islam gagneraient à être redécouvertes, comme celles du
soin, du don et du service à l'humanité, longtemps éclipsées selon lui
par l'autorité et la loyauté à la communauté musulmane ou à la tribu.

(1) Il a codirigé avec Mohammad Ali Amir-Moezzi, Le Coran des
historiens, 2019, Éd. du Cerf, 3~408~p., 89~€.

Faudra-t-il sauver les salafistes ?

Le gouvernement français a voulu lancer en octobre 2019 une offensive
contre l'islamisme et les courants radicaux, rapidement relayée par un
emballement médiatique qui a échappé à tout contrôle. Or, l'ennemi
désigné n'a nullement été identifié selon des termes juridiques, pas
plus que ses torts. On lui reproche sa piété rigoureuse, son voile, sa
pratique du jeûne de Ramadan, sa barbe fournie, son refus de toucher les
femmes, ce qui le rapproche dangereusement de n'importe quel fidèle
conservateur.

L'offensive vise donc une manière de concevoir la piété musulmane, et
nullement une qualification criminelle ou une atteinte à l'ordre public.
C'est dire que nous sommes confrontés à un « délit de sale gueule »,
lequel échappe à la tradition juridique républicaine, délit qui est
indiscernable, sans limite, extensible, mais politiquement pratique
auprès d'une opinion chauffée à blanc par les attentats et
l'immigration.

\subsection{Un engagement d'abord religieux}

Si l'islamiste ainsi décrit ressemble évidemment
au~\href{https://www.la-croix.com/Religion/Islam/Quest-salafisme-2018-10-14-1200975866}{\underline{salafiste}},
c'est oublier un peu vite que l'écrasante majorité des~\emph{salafi~}--
ceux qui sont attachés au modèle des « anciens » (les~\emph{salaf}),
c'est-à-dire les compagnons du Prophète -- se veulent quiétistes : leur
mode d'action est la prédication et l'action missionnaire
(la~\emph{da`wa}). Le salafiste souhaite d'abord vivre un islam épuré et
intégriste -- au sens d'intégral -- dans le cadre de sa famille et de sa
communauté.

Ce mouvement est distinct d'un engagement politique, de sorte que les
salafistes sont rarement liés aux Frères musulmans, qui eux forment un
mouvement politique. Si la matrice religieuse et idéologique du
salafisme imprègne les mentalités djihadistes, elle ne se confond pas
avec celles-ci, ni dans la pensée, ni dans les faits. La radicalisation
concerne donc à des degrés différents et sous des formes incomparables
les sympathisants du salafisme et les partisans du djihadisme de Daech.
Les premiers ont un engagement d'abord religieux, tandis que les autres
sont mus à la fois par la volonté de puissance, des facteurs politiques,
sociaux et religieux.

\subsection{L'autodidacte de l'islam présente plus de risques que le
salafiste}

L'hostilité des salafistes envers les courants djihadistes a été prouvée
à de nombreuses reprises par des déclarations publiques et surtout en
fournissant du renseignement de qualité auprès des services de police.
Le meilleur ennemi du terroriste est souvent le~\emph{salafi}, et
l'autodidacte de l'islam présente plus de risques que le salafiste.

En outre, le salafisme n'a pas été désavoué par les représentants du
culte musulman pour la simple raison que ce courant n'est pas une
idéologie : il faudrait donc lui enlever son~\emph{isme}~final et
l'appeler, selon la tradition religieuse, la~\emph{salafiya~}; il s'agit
d'un vieux courant légitime de l'islam, qui a fourni des générations
d'imams et de lettrés attachés au sens littéral du Coran et de la Sunna.

\subsection{Un « écosystème » étroit mais rassurant}

Il est évident que le salafisme représente une alternative culturelle et
sociale au modèle français, modèle égalitaire, inclusif, ouvert (au
moins en théorie). Les quelques salafi que j'ai connus -- des convertis
à 25 ou 30 \% d'entre eux -- vivaient dans un étroit triangle
géographique. Parce qu'ils souhaitent faire les cinq prières à leur
heure, sans les décaler, et ce dans une salle de prière, ils sont
contraints de vivre et de travailler non loin d'une mosquée. Ils passent
ainsi de leur habitation au lieu de travail et à la salle de prière,
lesquels se situent nécessairement dans un « écosystème » étroit mais
rassurant. Ils ne peuvent guère être exigeants sur le plan
professionnel.

\includegraphics[width=1.97917in,height=1.40972in]{media/image6.jpeg}

\href{https://www.la-croix.com/Religion/Le-Coran-peut-etre-interprete-2021-01-25-1201136852}{Le
Coran peut-il être interprété ?}

Le salafisme, qui représente au moins 40 000 individus, est socialement
dangereux car il impose l'auto-ségrégation, le refus des contacts avec «
ceux qui n'en sont pas ». C'est la raison pour laquelle les spécialistes
des questions de sécurité se refusent à les impliquer dans la lutte
contre le djihadisme. Salafistes et terroristes participeraient à une
même matrice intellectuelle, celle du bien contre le mal, une sorte de
vision sectaire du monde. La différence vient du rapport à la violence :
assumé chez les djihadistes, rejeté chez les salafistes. Leur
fondamentalisme présente l'avantage d'une certaine forme de morale : à
Sartrouville les quartiers salafisés ont vu s'effondrer la toxicomanie
et la délinquance, avec le soutien de la mairie.

\subsection{Confondre l'approche culturelle avec la lutte contre le
terrorisme}

Ces courants ne peuvent être incriminés sur le plan sécuritaire. On
confond donc l'approche culturelle avec la lutte contre le terrorisme. À
moins de changer tout le droit européen, la première doit être menée par
l'éducation, la philosophie, la raison, le débat ; quant à la seconde
elle doit s'appuyer sur le droit et sur des qualifications pénales, et
non sur de vagues impressions de « radicalisation », notion qui n'a
toujours pas été appréhendée de façon rigoureuse en termes sociologiques
et psychologiques.

Comme la guerre d'Algérie nous l'enseigne, une telle manière de
concevoir l'action politique va aboutir à l'effet inverse de celui
recherché : le renforcement de la méfiance collective, le repli
communautaire du côté musulman, l'action violente du côté des « anti »,
et, finalement, la fragmentation sociale et l'insécurité.

\subsection{Islam : les fumées de la radicalisation}

Olivier Hanne, médiéviste (université de Poitiers), chercheur en
islamologie, estime qu'il est très difficile de définir le parcours type
d'une personne radicalisée. Le dernier de trois articles consacrés à
l'islam en France. 
 

Qui parle d'islam aujourd'hui pense aussitôt à la radicalisation. En
2015, on estimait entre 8 000 et 10 000 le nombre de Français
radicalisés. Leurs profils sont si variés qu'il est difficile de donner
des catégories fixes : les mineurs représentent 25 \% des cas, les
femmes 27 \%, les personnes signalées sont plutôt jeunes (entre 16 et 30
ans), leur niveau scolaire est généralement faible, même si l'on
rencontre des diplômés.

La plupart travaillent. Internet représente pour tous ces individus un
passage obligé, même s'il se concrétise différemment : terrain initial
de la radicalisation, facteur de renforcement ou vecteur unique de
l'expression radicale, le partage des contenus djihadistes sur Internet
n'a pas du tout la même fonction chez une adolescente connectée, un
salafiste convaincu et un combattant expérimenté déjà parti en Syrie.

\subsection{Les autorités font feu de tout bois}

De toute évidence, l'attraction pour la radicalité religieuse n'est pas
nécessairement liée à un phénomène de rupture sociale. Les failles de la
société contemporaine (éclatement des familles, déclin des autorités et
des idéologies, chômage, ghettoïsation) créent un terreau facilitateur,
mais nullement déterminant. La frustration individuelle alimente le
recours à des convictions extrêmes, voire le passage à l'acte
terroriste, mais n'est qu'un facteur parmi tant d'autres.

Les autorités font feu de tout bois pour tenter de faire face à une
radicalisation multiforme. En avril 2015, le premier ministre français,
Manuel Valls, annonçait l'ouverture d'une dizaine de centres de
prévention de la radicalisation, dont la plupart furent un échec. Des
sites Internet officiels sont créés et proposent des fiches techniques
contre la radicalisation et le terrorisme, dont le contenu est souvent
simple, voire binaire. Ainsi sur le site
français~\emph{stop-djihadisme.gouv.fr}, un bandeau intitulé «
Radicalisation djihadiste, les premiers signes qui peuvent alerter »
énonce pêle-mêle : « ils se méfient des anciens amis qu'ils considèrent
maintenant comme des impurs » ; « ils changent brutalement leurs
habitudes alimentaires » ; « ils arrêtent d'écouter de la musique car
elle les détourne de leur mission » ; « ils ne regardent plus la
télévision et ne vont plus au cinéma ». Autant de signes extérieurs qui
se rapprochent de l'adolescente anorexique\ldots{} L'efficacité de ces
dispositifs a d'ailleurs été très contestée dès 2015.

\subsection{L'État, tenté d'être omniprésent}

Toute l'entreprise de déradicalisation définit en creux le modèle
positif occidental : monde de loisirs, de consommation, d'épanouissement
personnel et professionnel. Le vocabulaire de la radicalisation masque
le rejet de ce modèle culturel. Et les pouvoirs publics d'hésiter à
appeler leur objectif par son vrai nom : le reconditionnement mental.

Le danger de la déradicalisation se situe dans l'élargissement des
intrusions de l'État : en voulant réinsérer, l'État pénètre dans
l'intimité des individus afin de redéfinir le religieux et lui redonner
une place acceptable. Or, l'État a-t-il compétence pour définir ce
qu'est l'islam, le « bon » islam ? Ne sachant cerner la menace, l'État
est tenté d'être omniprésent, sans en avoir la capacité légale. La
déradicalisation pourrait relever de la posture intellectuelle.

Le problème vient sans doute des hésitations du vocabulaire. Car,
après-tout, qu'est-ce que la radicalisation ? Au
XIX\textsuperscript{e}~le mot anglais~\emph{radical}~était employé pour
désigner les partis politiques britanniques exigeant une réforme
démocratique libérale. Transféré tel quel en France, on l'appliqua aux
partis de gauche, laïques et libéraux qui voulaient réformer la société.

\subsection{Réactions épidermiques}

Le verbe « radicaliser » fut employé régulièrement dans les années
1960-1970 dans une acception politique avec l'idée de « devenir plus
intransigeant, se durcir » ou « plus extrême ». Le premier sens était
donc politique et pas nécessairement négatif. Se déradicaliser était un
synonyme pour « se compromettre ». Appliqué à l'islamisme, le verbe
impose une redéfinition complète des termes : à partir de quand
juge-t-on l'islam intransigeant ou extrême ? par rapport à quelle norme
? à quelle moyenne ?

Les réactions épidermiques qui ont suivi le meurtre de l'enseignant de
Conflans-Sainte-Honorine en octobre 2020 sont tristement révélatrices :
les imams doivent s'exprimer ! les musulmans doivent désavouer le
terrorisme et faire allégeance à la France ! Mais quand ils le font,
c'est encore insuffisant, déloyal et mensonger. Le gouvernement proposa
même qu'ils prient pour la République au cours de la prière collective
du vendredi. Nos références sur la question religieuse restent
tragiquement celles de la Révolution française : comme il y eut les «
prêtres jureurs », adhérant à la loi, contre les « prêtres réfractaires
», obstinés dans leur obéissance à Rome, de la même façon il nous faut
des « imams jureurs », intimement républicains. L'État se retrouve donc
juge des reins et des cœurs.
