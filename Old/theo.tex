
\usepackage[framemethod=TikZ]{mdframed}

\usepackage{thmtools}
\usepackage{blindtext} % avoid to cut theorem
% avoid to have theorem or definition in the list of theorm
\makeatletter
\patchcmd\thmt@mklistcmd
  {\thmt@thmname}
  {\check@optarg{\thmt@thmname}}
  {}{}
\patchcmd\thmt@mklistcmd
  {\thmt@thmname\ifx}
  {\check@optarg{\thmt@thmname}\ifx}
  {}{}
\protected\def\check@optarg#1{%
  \@ifnextchar\thmtformatoptarg\@secondoftwo{#1}%
}
\makeatother
% format of theorem
\declaretheoremstyle[
    headfont=\scshape, 
    notebraces={\scshape : }{.},
    bodyfont=\normalfont,
    headpunct={},
    postheadspace=\newline,
%    postheadhook={\textcolor{red}{\rule[.6ex]{\linewidth}{0.4pt}}\\},
    spacebelow=\parsep,
    spaceabove=\parsep,
    preheadhook={\begin{mdframed}[backgroundcolor=white!20, 
        splittopskip = \topskip,
            linecolor=blue!30, 
            linewidth = 2pt,
            innertopmargin=6pt,
            roundcorner=1pt, 
            innerbottommargin=6pt, 
            skipabove=\parsep,     
            skipbelow=\parsep]},
            postfoothook=\end{mdframed}]{Definitionstyle}


% example environment - thmtools


\declaretheorem[ style = Definitionstyle, name = {Definition}]{definition}
\declaretheorem[ style = Definitionstyle, name = {Definition}, sibling=definition]{Def}


\declaretheoremstyle[
    headfont=\scshape, 
    notebraces={\scshape : }{.},
    bodyfont=\normalfont,
    headpunct={},
    postheadspace=\newline,
%    postheadhook={\textcolor{red}{\rule[.6ex]{\linewidth}{0.4pt}}\\},
    spacebelow=\parsep,
    spaceabove=\parsep
]{Exercisestyle}
% example environment - thmtools



\declaretheorem[ style = Exercisestyle, name = Exemple ]{Ex}
