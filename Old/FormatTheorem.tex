
%--------------------------------------------------------------
% Frame
%--------------------------------------------------------------

\usepackage[framemethod=TikZ]{mdframed}

\usepackage{thmtools}
%\usepackage{amsthm}

\usepackage{blindtext} % avoid to cut theorem
% avoid to have theorem or definition in the list of theorm
\makeatletter
\newcommand{\theosep}{\parsep}
\renewcommand{\theosep}{20pt}


%--------------------------------------------------------------
% Titre des listes de théorèmes
%--------------------------------------------------------------

\renewcommand{\listtheoremname}{List of Important Theorems}

\makeatletter
\def\ll@theorem{%
  \protect\numberline{\csname the\thmt@envname\endcsname}%
  \ifx\@empty\thmt@shortoptarg
    \thmt@thmname
  \else
    \thmt@shortoptarg
  \fi}
\def\l@thmt@theorem{} 
 \makeatother
 

% avoid to have theorem or definition in the list of theorm
\makeatletter
\patchcmd\thmt@mklistcmd
  {\thmt@thmname}
  {\check@optarg{\thmt@thmname}}
  {}{}
\patchcmd\thmt@mklistcmd
  {\thmt@thmname\ifx}
  {\check@optarg{\thmt@thmname}\ifx}
  {}{}
\protected\def\check@optarg#1{%
  \@ifnextchar\thmtformatoptarg\@secondoftwo{#1}%
}

 
\makeatother

% format of theorem


            
\declaretheoremstyle[
    headfont=\scshape, 
    notebraces={\scshape : }{.},
    bodyfont=\normalfont,
    headpunct={},
    postheadspace=\newline,
%    postheadhook={\textcolor{red}{\rule[.6ex]{\linewidth}{0.4pt}}\\},
    spacebelow=\parsep,
    spaceabove=\parsep,
    mdframed={
            backgroundcolor=white!20, 
            splittopskip = \topskip,
            linecolor=blue!30, 
            linewidth = 2pt,
            innertopmargin=\myinnertopmargin,
            roundcorner=1pt, 
            innerbottommargin=6pt, 
            skipabove=\parsep,     
            skipbelow=\parsep} 
    ]{Definitionstyle}
    
\declaretheoremstyle[
    headfont=\scshape, 
    notebraces={\scshape : }{.},
    bodyfont=\normalfont,
    headpunct={},
    postheadspace=\newline,
%    postheadhook={\textcolor{red}{\rule[.6ex]{\linewidth}{0.4pt}}\\},
    spacebelow=\parsep,
    spaceabove=\parsep,
    mdframed={backgroundcolor=white!20, 
            splittopskip = \topskip,
            linecolor=red!30, 
            linewidth = 2pt,
            innertopmargin=\myinnertopmargin,
            roundcorner=1pt, 
            innerbottommargin=6pt, 
            skipabove=\parsep,     
            skipbelow=\parsep} 
    ]{Propertystyle}

%,    postfoothook=
% example environment - thmtools
\declaretheoremstyle[
    headfont=\scshape, 
    notebraces={\scshape : }{.},
    bodyfont=\normalfont,
    headpunct={},
    postheadspace=\newline, 
%    postheadhook={\textcolor{red}{\rule[.6ex]{\linewidth}{0.4pt}}\\},
    spacebelow=\parsep,
    spaceabove=\parsep
]{Exercisestyle}
% example environment - thmtools








\declaretheorem[ style = Exercisestyle, numbered=no,name = Property]{property}
\declaretheorem[ style = Propertystyle, name = {Property} ]{Prop}
\declaretheorem[ style = Propertystyle, name = Theorem, sibling=Prop]{Theo}
\declaretheorem[ style = Propertystyle, name = Theorem, sibling=Prop]{theorem}
\declaretheorem[ style = Propertystyle, name = Lemma, sibling=Prop]{lemma}
\declaretheorem[ style = Exercisestyle, numbered=no,name = {Remark}]{rem}
\declaretheorem[ style = Definitionstyle, name = {Definition}]{definition}
\declaretheorem[ style = Definitionstyle, name = {Definition}, sibling=definition]{Def}
\declaretheorem[ style = Exercisestyle, name = Exercise]{exercise}
\declaretheorem[ style = Exercisestyle, name = Exercise, sibling=exercise]{Exercise}
\declaretheorem[ style = Exercisestyle, name = Exercise, sibling=exercise]{Exc}
\declaretheorem[ style = Exercisestyle, name = Exercise, sibling=exercise]{Exo}
\declaretheorem[ style = Exercisestyle, name = Problem, sibling=exercise]{problem}
\declaretheorem[ style = Exercisestyle, name = Example]{example}
\declaretheorem[ style = Exercisestyle, name = Example, sibling=example]{Ex}
\makeatother


%--------------------------------------------------------------
% Code
%--------------------------------------------------------------

%% Permet de mettre du code
\usepackage{listings}
\lstdefinestyle{mystyle}{
    basicstyle=\ttfamily\footnotesize,
    breakatwhitespace=false,         
    breaklines=true,                 
    captionpos=b,                    
    keepspaces=true,                 
    numbers=left,                    
    numbersep=5pt,                  
    showspaces=false,                
    showstringspaces=false,
    showtabs=false,                  
    tabsize=2
}
\lstset{%
	aboveskip=\topsep,
	belowskip=\topsep,
	xleftmargin=\parindent}

\lstset{style=mystyle}




