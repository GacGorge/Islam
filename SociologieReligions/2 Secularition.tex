\chapter{Modernité et sécularisation}

\section{le paradigme de la modernité}

\begin{Def}[Paradigme] ensemble de convictions, de questions ou de dogmes acceptés et partagés par une
communauté scientifique à un moment donné.
\end{Def}

\subsection{Un processus intellectuellement construit}
\paragraph{Un outil de compréhension de la réalité sociale}
manière de restituer  une période. 
Artificiellement construit et qu'on ne retrouve pas forcément de façon intégrale. 
Ce n'est pas un processus homogène et unilatéral.
C'est un imaginaire occidental qui structure la façon de se voir avec les autres : environnement, religion,...

Cette modernité s'est diffusée. On est immergé.


\subsection{L’homme au centre de la réflexion}

Cela aurait été créé par les grandes découvertes. Les découvreurs vont arriver sur des territoires peu façonnés par l'homme, peu habités. A la différence de l'Europe, \textit{domestiquée}, l'humain devant l'immensité du territoire, et envie d'extraire cette matière première et comment s'organiser avec les populations locales.

L'alterité, ne vient pas directement de la rencontre de l'autre, mais la découverte qu'on peut vivre de façon différente qu'en Europe, du fait de l'accumulation des matières premières. Accroissement exponentiel du savoir qui donne des ouvertures auquel on n'avait pas pensé avant.

On va même repenser le statut de l'humain. L'\textit{humanisme}, quels sont les compétences de l'être humain. 

\paragraph{En opposition à la notion de tradition}
\paragraph{Importance de la Raison} 
\paragraph{Croire au Progrès} 

\subsection{Conséquences politiques, sociales, économiques}

\paragraph{Une autre manière de concevoir les relations entre les hommes, avec la nature, distinction entre
nature et culture.} 
\paragraph{Processus de rationalisation qui conduit aux sociétés démocratiques.} 
\paragraph{Développement de l’individualisme.} 

\section{le paradigme de la sécularisation}
 \begin{Def}[Sécularisation] perte d’emprise progressive de la religion sur les institutions.
\end{Def}
 

\subsection{Conséquences du processus de sécularisation}
\begin{itemize}
    \item  perte de pouvoir social des institutions religieuses
   \item  pluralisation de l’offre religieuse
   \item  individualisation
   \item  rationalisation
\end{itemize}


\subsection{Origine de la sécularisation}

\paragraph{Une matrice religieuse de la sortie de la religion ?}

\subsection{Remise en cause de ce paradigme}
\paragraph{Sommes-nous encore en Modernité ?}



\section{La modernité}
\mn{Jean BAUDRILLARD : maître assistant de sociologie à l'université Paris-X-Nanterre
Alain BRUNN : ancien élève de l'École normale supérieure, agrégé de lettres modernes, université de Paris- III-Sorbonne nouvelle
Jacinto LAGEIRA : professeur en esthétique à l'université de Paris-I-Panthéon-Sorbonne, critique d'art

Jean BAUDRILLARD, Alain BRUNN, Jacinto LAGEIRA, « MODERNITÉ »,
Encyclopædia Universalis [en ligne], consulté le 26 septembre 2022. URL : http://www.univers alis-edu.com.icp.idm.oclc.org/encyclopedie/modernite/}
 





La modernité n'est ni un concept sociologique, ni un concept politique, ni proprement un concept historique. C'est un mode de civilisation caractéristique, qui s'oppose au mode de la tradition, c'est-à-dire à toutes les autres cultures antérieures ou traditionnelles : face à la diversité géographique et symbolique de celles-ci, la modernité s'impose comme une, homogène, irradiant mondialement à partir de l'Occident. Pourtant elle demeure une notion confuse, qui connote globalement toute une évolution historique et un changement de mentalité.
Inextricablement mythe et réalité, la modernité se spécifie dans tous les domaines : État moderne, technique moderne, musique et peinture modernes, mœurs et idées modernes – comme une sorte de catégorie générale et d'impératif culturel. Née de certains bouleversements profonds de l'organisation économique et sociale, elle s'accomplit au niveau des mœurs, du mode de vie et de la quotidienneté – jusque dans la figure caricaturale du modernisme. Mouvante dans ses formes, dans ses contenus, dans le temps et dans l'espace, elle n'est stable et irréversible que comme système de valeurs, comme mythe – et, dans cette acception, il faudrait l'écrire avec une majuscule : la Modernité. En cela, elle ressemble à la Tradition.
Comme elle n'est pas un concept d'analyse, il n'y a pas de lois de la modernité, il n'y a que des traits de la modernité. Il n'y a pas non plus de théorie, mais une logique de la modernité, et une idéologie. Morale canonique du changement, elle s'oppose à la morale canonique de la tradition, mais elle se garde tout autant du changement radical. C'est la
« tradition du nouveau » (Harold Rosenberg). Liée à une crise historique et de structure, la modernité n'en est pourtant que le symptôme. Elle n'analyse pas cette crise, elle l'exprime de façon ambiguë, dans une fuite en avant continuelle. Elle joue comme idée-force et comme idéologie maîtresse, sublimant les contradictions de l'histoire dans les effets de civilisation. Elle fait de la crise une valeur, une morale contradictoire. Ainsi, en tant qu'idée où toute une civilisation se reconnaît, elle assume une fonction de régulation culturelle et rejoint par là subrepticement la tradition.

\section{Genèse de la modernité}


 
L'histoire de l'adjectif « moderne » est plus longue que celle de la « modernité ». Dans n'importe quel contexte culturel, l'« ancien » et le « moderne » alternent significativement. Mais il n'existe pas pour autant partout une « modernité », c'est-à-dire une structure historique et polémique de changement et de crise. Celle-ci n'est repérable qu'en Europe à partir du XVIe siècle, et ne prend tout son sens qu'à partir du XIXe siècle.
Les manuels scolaires font succéder les Temps modernes au Moyen Âge à la date de la découverte de l'Amérique par Christophe Colomb (1492). L'invention de l'imprimerie, les découvertes de Galilée inaugurent l'humanisme moderne de la Renaissance. Sur le plan des arts, et singulièrement de la littérature, va se développer, pour culminer au XVIIe et au XVIIIe siècle, la querelle des Anciens et des Modernes. Les échos profonds du partage de la modernité se font aussi dans le domaine religieux : par l'événement de la Réforme (Luther affiche à Wittenberg ses quatre-vingt-quinze thèses contre les indulgences le 31 octobre 1517) et la rupture qu'elle inaugure pour les pays protestants, mais aussi par la répercussion sur le monde catholique (Concile de Trente, 1545-1549, 1551-1552, 1562-1563). L'Église catholique opère déjà une mise à jour, se fait, avec la Compagnie de Jésus, moderne, mondaine et missionnaire, ce qui explique peut-être que ce soit dans les pays qui ont gardé la tradition romaine, ses rites et ses mœurs, tout en les rénovant progressivement, que le terme de modernité ait une acception plus courante, plus significative. Le terme en effet ne prend force que dans les pays de longue tradition. Parler de modernité n'a guère de sens quand il s'agit d'un pays sans tradition ni Moyen Âge, comme les États-Unis, et, inversement, la modernisation a un impact très fort dans les pays du Tiers Monde, de forte culture traditionnelle.
Dans les pays touchés par la Renaissance catholique, la conjonction d'un humanisme laïc et séculier avec le ritualisme plus mondain des formes et des mœurs dans le monde catholique se prête mieux à toute la complexité de la vie sociale et artistique qu'implique le développement de la modernité que la stricte alliance du rationalisme et du moralisme dans la culture protestante. Car la modernité n'est pas seulement la réalité des bouleversements techniques, scientifiques et politiques depuis le XVIe siècle, c'est aussi le jeu de signes, de mœurs et de culture qui traduit ces changements de structure au niveau du rituel et de l'habitus social.
Pendant les XVIIe et XVIIIe siècles se mettent en place les fondements philosophiques et politiques de la modernité : la pensée individualiste et rationaliste moderne dont Descartes et la philosophie des Lumières sont représentatifs ; l'État monarchique centralisé, avec ses techniques administratives, succédant au système féodal ; les bases d'une science physique et naturelle, qui entraînent les premiers effets d'une technologie appliquée (l'Encyclopédie).
Culturellement, c'est la période de la sécularisation totale des arts et des sciences. La querelle des Anciens et des Modernes, qui traverse toute cette période, de Perrault (Parallèle des Anciens et des Modernes, 1688) et Fontenelle (Digression sur les Anciens et les Modernes, 1688), dégageant une loi de progrès de l'esprit humain, jusqu'à Rousseau (Dissertation sur la musique
moderne, 1750) et à Stendhal (Racine et Shakespeare, 1823), lequel conçoit le « romanticisme » comme un modernisme radical, prenant pour thème les mœurs du jour et les sujets empruntés à l'histoire nationale, cette querelle définit un mouvement autonome, dégagé de toute « Renaissance » ou imitation. La modernité n'est pas encore un mode de vie (le terme n'existe alors pas). Mais elle est devenue une idée (jointe à celle de progrès). Elle a pris du coup une tonalité bourgeoise libérale qui ne cessera depuis de la marquer idéologiquement.
La Révolution de 1789 met en place l'État bourgeois moderne, centralisé et démocratique, la nation avec son système constitutionnel, son organisation politique et bureaucratique.
Le progrès continuel des sciences et des techniques, la division rationnelle du travail industriel introduisent dans la vie sociale une dimension de changement permanent, de destructuration des mœurs et de la culture traditionnelle. Simultanément, la division sociale du travail introduit des clivages politiques profonds, une dimension de luttes sociales et de conflits qui se répercuteront à travers le XIXe et le XXe siècle.
Ces deux aspects majeurs, auxquels viendront s'ajouter la croissance démographique, la concentration urbaine et le développement gigantesque des moyens de communication et d'information, marqueront de façon décisive la modernité comme pratique sociale et mode de vie articulé sur le changement, l'innovation, mais aussi sur l'inquiétude, l'instabilité, la mobilisation continuelle, la subjectivité mouvante, la tension, la crise, et comme représentation idéale ou mythologie. À ce titre, la date d'apparition du mot lui-même (Théophile Gautier, Baudelaire, 1850 environ) est significative : c'est le moment où la société moderne se réfléchit comme telle, se pense en termes de modernité. Celle-ci devient alors une valeur transcendante, un modèle culturel, une morale – un mythe de référence partout présent, et masquant par là en partie les structures et les contradictions historiques qui lui ont donné naissance.

\section{La logique de la modernité}

\paragraph{Concept techno-scientifique}

L'essor prodigieux, surtout depuis un siècle, des sciences et des techniques, le développement rationnel et systématique des moyens de production, de leur gestion et de leur organisation marquent la modernité comme l'ère de la productivité : intensification du travail humain et de la domination humaine sur la nature, l'un et l'autre réduits au statut de forces productives et aux schémas d'efficacité et de rendement maximal. C'est là le commun dénominateur de toutes les nations modernes. Si cette « révolution » des forces productives, parce qu'elle laisse relativement inchangés les rapports de production et les rapports sociaux, n'a pas changé la vie, elle modifie du moins d'une génération à l'autre les conditions de vie. Elle instaure aujourd'hui une mutation profonde dans la modernité : le passage d'une civilisation du travail et du progrès à une civilisation de la consommation et
du loisir. Mais cette mutation n'est pas radicale : elle ne change pas la finalité productiviste, le découpage chronométrique du temps, les contraintes prévisionnelles et opérationnelles qui restent les coordonnées fondamentales de l'éthique moderne de la société productive.
\paragraph{Concept politique}
\begin{quote}
    « L'abstraction de l'État politique comme tel n'appartient qu'aux Temps modernes, parce que l'abstraction de la vie privée n'appartient qu'aux Temps modernes... Au Moyen Âge, la vie du peuple et la vie de l'État sont identiques : l'homme est le principe réel de l'État... les Temps modernes sont le dualisme abstrait, l'opposition abstraite réfléchie » (Marx, Critique de la philosophie de l'État de Hegel).
\end{quote}

C'est en effet la transcendance abstraite de l'État, sous le signe de la Constitution, et le statut formel de l'individu, sous le signe de la propriété privée, qui définissent la structure politique de la modernité. La rationalité (bureaucratique) de l'État et celle de l'intérêt et de la conscience privés se répondent dans la même abstraction. Cette dualité marque la fin de tous les systèmes antérieurs, où la vie politique se définissait comme une hiérarchie intégrée de relations personnelles. L'hégémonie de l'État bureaucratique n'a fait que croître avec les progrès de la modernité. Liée à l'extension du champ de l'économie politique et des systèmes d'organisation, elle investit tous les secteurs de la vie, les mobilisant à son profit, les rationalisant à son image. Ce qui lui résiste (vie affective, langues et cultures traditionnelles) parfois obstinément, peut être dit résiduel. Toutefois, ce qui fut une des dimensions essentielles (sinon la dimension essentielle) de la modernité, l'État abstrait centralisé, est peut-être en train de vaciller. La contrainte hégémonique de l'État, la saturation bureaucratique de la vie sociale et individuelle préparent sans doute de grandes crises en ce domaine.
\paragraph{Concept psychologique}
Face au consensus magique, religieux, symbolique de la société traditionnelle (communauté), l'ère moderne est marquée par l'émergence de l'individu, avec son statut de conscience autonome, sa psychologie et ses conflits personnels, son intérêt privé, voire son inconscient et, pris de plus en plus dans le réseau des médias, des organisations, des institutions, son aliénation moderne, son abstraction, sa perte d'identité dans le travail et le loisir, l'incommunicabilité, etc., que cherche à compenser tout un système de personnalisation à travers les objets et les signes.
\paragraph{La modernité et le temps}
Sous tous les aspects, la temporalité moderne est spécifique.

\textbf{L'aspect chronométrique }: le temps qui se mesure et auquel on mesure ses activités, celui qui scande la division du travail et la vie sociale, ce temps abstrait qui s'est substitué au rythme des travaux et des fêtes, est celui de la contrainte productive ; la temporalité bureaucratique règne même sur le temps « libre » et les loisirs.
 
\textbf{L'aspect linéaire} : le temps « moderne » n'est plus cyclique, il se développe selon une ligne passé-présent-avenir, selon une origine et une fin supposées. La tradition semble axée sur le passé, la modernité sur l'avenir, mais, dans le fait, seule la modernité projette un passé (le temps du révolu) en même temps qu'un avenir, selon une dialectique qui lui est propre.
\textbf{L'aspect historique} : surtout depuis Hegel, l'histoire est devenue l'instance dominante de la modernité. À la fois comme devenir réel de la société et comme référence transcendante laissant entrevoir son accomplissement final. La modernité se pense historiquement, et non plus mythiquement.
Mesurable, irréversible, succession chronométrique ou devenir dialectique, de toute façon la modernité a sécrété une temporalité tout à fait nouvelle, dimension cruciale, image de ses contradictions. Mais à l'intérieur de ce temps indéfini, et qui ne connaît plus d'éternité, une chose distingue la modernité : elle se veut toujours « contemporaine », c'est-à-dire simultanéité mondiale. Après avoir d'abord privilégié la dimension du progrès et de l'avenir, elle semble se confondre aujourd'hui de plus en plus avec l'actualité, l'immédiateté, la quotidienneté, l'envers pur et simple de la durée historique.

\section{La rhétorique de la modernité}

\paragraph{Innovation et avant-garde}

Dans le domaine de la culture et des mœurs, la modernité se traduit, en opposition formelle mais en relation fondamentale avec la centralisation bureaucratique et politique, avec l'homogénéisation des formes de la vie sociale, par une exaltation de la subjectivité profonde, de la passion, de la singularité, de l'authenticité, de l'éphémère et de l'insaisissable, par l'éclatement des règles et l'irruption de la personnalité, consciente ou non.
« Le peintre de la vie moderne » de Baudelaire, à la charnière du romantisme et de la modernité contemporaine, marque le départ de cette quête du nouveau, de cette dérive du subjectif : « Ainsi il va, il court, il cherche. Que cherche-t-il ? À coup sûr, cet homme tel que je l'ai dépeint, ce solitaire d'une imagination active, voyageant à travers le grand désert d'hommes... cherche ce quelque chose qu'on nous permettra d'appeler la modernité. »
La modernité va susciter à tous les niveaux une esthétique de rupture, de créativité individuelle, d'innovation partout marquée par le phénomène sociologique de l'avant-garde (que ce soit dans le domaine de la culture ou dans celui de la mode) et par la destruction toujours plus poussée des formes traditionnelles (les genres en littérature, les règles de l'harmonie en musique, les lois de la perspective et de la figuration en peinture, l'académisme et, plus généralement, l'autorité et la légitimité des modèles antérieurs en matière de mode, de sexualité et de conduites sociales).

\paragraph{Mass media, mode et culture de masse}
 
Cette tendance fondamentale est suractivée depuis le XXe siècle par la diffusion industrielle des moyens culturels, l'extension d'une culture de masse et l'intervention gigantesque des médias (presse, cinéma, radio, télévision, publicité). Le caractère éphémère des contenus et des formes s'est accentué, les révolutions de style, de mode, d'écriture, de mœurs ne se comptent plus. En se radicalisant ainsi dans un changement à vue, dans un travelling continuel, la modernité change de sens. Elle perd peu à peu toute valeur substantielle de progrès qui la sous-tendait au départ, pour devenir une esthétique du changement pour le changement. Elle s'abstrait et se déploie en une nouvelle rhétorique, elle s'inscrit dans le jeu d'un ou de multiples systèmes de signes. À la limite, elle rejoint ici purement et simplement la mode, qui est en même temps la fin de la modernité.
Car elle rentre alors dans un changement cyclique, où resurgissent d'ailleurs toutes les formes du passé (archaïques, folkloriques, rustiques, traditionnelles), vidées de leur substance, mais exaltées comme signes dans un code où tradition et néo, ancien et moderne s'équivalent et jouent alternativement. La modernité n'a plus du tout alors valeur de
rupture ; elle s'alimente des vestiges de toutes les cultures au même titre que de ses gadgets techniques ou de l'ambiguïté de toutes les valeurs.

\section{Tradition et modernité dans les sociétés du Tiers Monde}


\paragraph{Destructuration et changement}
Les traits distinctifs, les ferments, la problématique et les contradictions de la modernité se révèlent avec le plus de force là où son impact historique et politique est le plus brutal : sur les sociétés tribales ou traditionnelles colonisées. Apter voit dans le colonialisme une « force modernisante », un « modèle par lequel la modernisation a été universalisée ».
Les anciens systèmes d'échange sont destructurés par l'irruption de la monnaie et d'une économie de marché. Les systèmes de pouvoir traditionnels s'effacent sous la pression des administrations coloniales ou des nouvelles bureaucraties indigènes.
Cependant, faute d'une révolution politique et industrielle en profondeur, ce sont souvent les aspects les plus techniques, les plus exportables de la modernité qui touchent les sociétés en voie de développement : les objets de production et de consommation industrielle, les mass media. C'est dans sa matérialité technique et comme spectacle que la modernité les investit d'abord, et non selon le long processus de rationalisation économique et politique qui fut celui de l'Occident. Pourtant, ces retombées de la modernité ont à elles seules un retentissement politique : elles accélèrent la destructuration du mode de vie et précipitent les revendications sociales de changement.
\paragraph{Résistance et amalgame}
 
Si donc, dans un premier temps, la modernité apparaît bien ici aussi comme rupture, l'analyse plus fine inaugurée depuis la Seconde Guerre mondiale par l'anthropologie politique (Balandier, Leach, Apter, Althabe) montre que les choses sont plus complexes. Le système traditionnel (tribal, clanique, lignager) oppose au changement la plus forte résistance, et les structures modernes (administratives, morales, religieuses) y nouent avec la tradition de curieux compromis. La modernité y passe toujours par une résurgence de la tradition, sans que celle-ci ait pour autant un sens conservateur. Favret décrit même comment les paysans des Aurès réactivent des mécanismes politiques traditionnels par exigence de progrès, pour protester contre la trop lente diffusion, dans leur région, des instruments et des signes de la modernité.
Cela est important : le terrain de l'anthropologie montre, plus clairement que l'histoire européenne, la vérité de la modernité, à savoir qu'elle n'est jamais changement radical ou révolution, mais qu'elle entre toujours en implication avec la tradition dans un jeu culturel subtil, dans un débat où les deux ont partie liée, dans un processus d'amalgame et d'adaptation. La dialectique de la rupture y cède largement à une dynamique de l'amalgame.
\paragraph{Les idéologies comme signe de la modernité}

L'analyse des sociétés décolonisées fait apparaître une autre expression spécifique de la modernité : l'idéologie. Les idéologies (nationales, culturelles, politiques) sont contemporaines de la détribalisation et de la modernisation. Importées d'Occident et imprégnées de rituels et de croyances traditionnelles, elles n'en constituent pas moins, plus que l'infrastructure économique, le lieu du changement et du conflit, du bouleversement des valeurs et des mentalités. Il s'agit là encore plutôt d'une rhétorique de la modernité, qui se déploie en pleine ambiguïté dans des sociétés dont elle compense le retard réel et le non- développement.
De telles constatations peuvent aider à définir le paradoxe de la modernité. Destruction et changement, mais aussi ambiguïté, compromis, amalgame : la modernité est paradoxale, elle n'est pas dialectique. Si l'idéologie est un concept typiquement « moderne », si les idéologies sont l'expression de la modernité, sans doute aussi la modernité elle-même n'est-elle qu'un immense processus idéologique.

\section{Idéologie de la modernité}

\paragraph{Un conservatisme par le changement}

La dynamique de la modernité se révèle ainsi, aussi bien en Occident que dans le Tiers Monde, à la fois lieu d'émergence des facteurs de rupture et solution de compromis avec les facteurs d'ordre et de tradition. La mobilité qu'elle implique à tous les niveaux (sociale, professionnelle, géographique, matrimoniale, de mode et de libération sexuelle) ne définit encore que la part de changement tolérable par le système, sans qu'il soit changé pour
l'essentiel. Balandier dit des pays d'Afrique noire : \begin{quote}
    « Les affrontements politiques s'expriment dans une large mesure, mais non exclusivement, par le débat du traditionnel et du moderne : ce dernier apparaît surtout comme leur moyen et non comme leur cause principale. » 
\end{quote}Ainsi l'on peut dire que, dans les pays développés, la modernité n'est pas ce qui retrace la structure ni l'histoire sociale : elle est bien plutôt (dans son jeu avec la tradition), le lieu où elles viennent affleurer pour être masquées, le lieu où la dialectique du sens social vient s'estomper dans le code rhétorique et mythique de la modernité.
Une ambiguïté spectaculaire
Les changements de structure politiques, économiques, technologiques, psychologiques sont les facteurs historiques objectifs de la modernité. Ils ne constituent pas en eux-mêmes la modernité. Celle-ci se définirait plutôt comme la dénégation de ces changements structurels, tout au moins comme leur réinterprétation en termes de style culturel, de mentalité, de mode de vie, de quotidienneté.
La modernité n'est pas la révolution technologique et scientifique, c'est le jeu et l'implication de celle-ci dans le spectacle de la vie privée et sociale, dans la dimension quotidienne des médias, des gadgets, du bien-être domestique ou de la conquête de l'espace. La science ni la technique elles-mêmes ne sont « modernes » : ce sont les effets de la science et de la technique qui le sont. Et la modernité, tout en se fondant sur l'émergence historique de la science, ne vit qu'au niveau du mythe de la science.
La modernité n'est pas la rationalité ni l'autonomie de la conscience individuelle, qui pourtant la fonde. C'est, après la phase d'avènement triomphal des libertés et des droits individuels, l'exaltation réactionnelle d'une subjectivité menacée de partout par l'homogénéisation de la vie sociale. C'est le recyclage de cette subjectivité perdue dans un système de « personnalisation », dans les effets de mode et d'aspiration dirigée.
La modernité n'est pas dialectique de l'histoire : elle est l'événementialité, le jeu permanent de l'actualité, l'universalité du fait divers par le moyen des médias.
La modernité n'est pas la transmutation de toutes les valeurs, c'est la destructuration de toutes les valeurs anciennes sans leur dépassement, c'est l'ambiguïté de toutes les valeurs sous le signe d'une combinatoire généralisée. Il n'y a plus ni bien ni mal, mais nous ne sommes pas pour autant « au-delà du bien et du mal » (cf. la critique de la modernité chez Nietzsche).
La modernité n'est pas la révolution, même si elle s'articule sur des révolutions (industrielle, politique, révolution de l'information, révolution du bien-être, etc). \begin{quote}
    Elle est, comme dit Lefèbvre, « l'ombre de la révolution manquée, sa parodie » (Introduction à la modernité). « À l'intérieur du monde renversé et non remis sur ses pieds, la modernité accomplit les tâches de la révolution : dépassement de l'art, de la morale, des idéologies... »
\end{quote} , on pourrait ajouter :
mobilité, abondance, libérations de toutes sortes. Mais elle les accomplit sur le mode d'une révolution permanente des formes, dans le jeu du changement, finalement dans un cycle où se referme la brèche ouverte dans le monde de la tradition.
\paragraph{Une culture de la quotidienneté}

La tradition vivait de continuité et de transcendance réelle. La modernité, ayant inauguré la rupture et le discontinu, s'est refermée sur un nouveau cycle. Elle a perdu l'impulsion idéologique de la raison et du progrès et se confond de plus en plus avec le jeu formel du changement. Même ses mythes se retournent contre elle (celui de la technique, jadis triomphal, est aujourd'hui lourd de menaces). Les idéaux, les valeurs humaines qu'elle s'était donnés lui échappent : elle se caractérise de plus en plus par la transcendance abstraite de tous les pouvoirs. La liberté y est formelle, le peuple y devient masse, la culture y devient mode. Après avoir été une dynamique du progrès, la modernité devient lentement un activisme du bien-être. Son mythe recouvre l'abstraction grandissante de la vie politique et sociale, sous laquelle elle se réduit peu à peu à n'être qu'une culture de la quotidienneté.
\sn{— Jean BAUDRILLARD}

\section{Esthétique}


Si le terme latin modernus, au sens d'« actuel » et non de « nouveau », apparaît au Ve siècle, il faut attendre la Renaissance italienne pour que l'adjectif « moderne » (et ses équivalents dans les langues européennes) soit utilisé avec les connotations de nouveauté et d'innovation qu'il possède encore aujourd'hui. Puis au XIXe siècle, la « modernité » fait irruption sous la plume d'écrivains tels que François-René de Chateaubriand, Honoré de Balzac ou Théophile Gautier.
C'est à ce dernier que l'on doit l'introduction du mot dans le domaine de la critique d'art, avec un article du Moniteur universel, en 1855, à propos d'un tableau du peintre William Mulready : « Il serait difficile de rattacher cet artiste à aucune école ancienne, car le caractère de la peinture anglaise est, comme nous l'avons dit, la modernité. Le substantif existe-t-il ?
Le sentiment qu'il exprime est si récent que le mot pourrait bien ne pas se trouver dans les dictionnaires. »

\paragraph{Baudelaire et la modernité}


Les critiques d'art de Charles Baudelaire restent cependant la référence principale pour une notion qui marquera son époque, ainsi que pour la plupart des approches ultérieures. Dès le Salon de 1845, Baudelaire écrit de Delacroix, dans « Tableaux d'histoire », qu'il est
« décidément le peintre le plus original des temps anciens et des temps modernes. » Dans le Salon de l'année suivante, outre le texte intitulé « De l'héroïsme de la vie moderne », on
peut lire dans « Qu'est-ce que le romantisme ? » : « Qui dit romantisme dit art moderne – c'est-à-dire intimité, spiritualité, couleur, aspirations vers l'infini, exprimées par tous les moyens que contiennent les arts. [...] Que la couleur joue un rôle très important dans l'art moderne, quoi d'étonnant ? Le romantisme est fils du Nord, et le Nord est coloriste... »
En 1863, les idées de moderne et de modernité se trouvent longuement développées dans
« Le Peintre de la vie moderne », texte fondateur pour les idées d'actualité, d'originalité, pour l'imagination « reine des facultés », et surtout, pour le culte du nouveau. De l'artiste moderne, Baudelaire écrit : « Il cherche ce quelque chose qu'on nous permettra d'appeler la modernité. [...] Il s'agit, pour lui, de dégager de la mode ce qu'elle peut contenir de poétique dans l'historique, de tirer l'éternel du transitoire. [...] La modernité c'est le transitoire, le fugitif, le contingent, la moitié de l'art, dont l'autre moitié est l'éternel et l'immuable ». Cet essai donnera lieu à une longue série de spéculations, d'affirmations, de programmes esthétiques autour du fait d'être moderne en art.
Considérer le texte de Baudelaire comme l'une des principales sources de l'art moderne et de la modernité équivaut déjà à adopter une position moderniste. Or les périodisations ne sont pas toujours les mêmes selon qui les considère, un historien de l'art pouvant faire débuter l'art moderne aux environs de 1863 – année où, lors du Salon des refusés, Manet présente \textit{Le Déjeuner sur l'herbe} –, quand, dans le champ de l'histoire, les Temps modernes commencent en 1492, avec la découverte de l'Amérique par Christophe Colomb.
Il s'agit bien d'une conception prédéterminant ce qui peut ou non entrer dans le cadre de la modernité. Rejeter la naissance de l'art moderne en 1863, c'est rejeter l'idée d'une rupture radicale revendiquée par des critiques tels que Baudelaire, et considérer cette période comme un simple moment de transition. Il est vrai que les diverses querelles des Anciens et des Modernes, au XVIIe siècle, tendent à consolider une telle conception. Mais c'est alors hypothéquer la lecture des pratiques et théories artistiques intervenues après le milieu du XIXe siècle, et risquer de ne plus pouvoir les comprendre, puisque, des avant-gardes historiques du début du XXe siècle à la période actuelle, l'origine de l'art moderne et de la modernité est toujours située autour de 1860.
Chronologiquement, « avant-garde », « moderne » et « modernité » artistiques apparaissent entre 1825 et 1855, moins comme de véritables synonymes que comme les facettes d'un même projet que l'on retrouvera régulièrement tout au long du XXe siècle, à savoir la création d'une œuvre originale. Tous les courants d'importance situés entre 1905 (moment
de la naissance du fauvisme) et 1939 (rupture de la Seconde Guerre mondiale) se comprennent comme modernes, constituant autant d'avant-gardes inscrites dans l'ample mouvement de la modernité. Même les retours au réalisme survenus en Europe à partir de 1919 (notamment la Nouvelle Objectivité allemande), tout en revendiquant un ancrage dans le passé – ce que refusent nettement les avant-gardes « historiques », du futurisme à \textit{De Stijl}
en passant par le constructivisme russe –, continuent de prétendre appartenir à la modernité, y compris lorsqu'il s'agit de critiquer avec virulence une modernité sociale et politique détestée.

\paragraph{Avant-garde, modernité, modernisme}


Dans l'entre-deux-guerres se manifeste une transformation considérable par rapport à la conception baudelairienne de la modernité, plutôt en retrait dans le champ sociopolitique, voire élitiste. Apparaît une critique sociale, politique et morale, due principalement aux chocs physiques et psychiques provoqués par le premier conflit mondial. Mais les avant- gardes des années 1920 et 1930 ont beau tenter de reprendre le flambeau des utopies nées au XIXe siècle, leurs tentatives seront réduites à néant par la Seconde Guerre mondiale, dont la modernité ne saura pas vraiment se relever.
Bien que l'on puisse distinguer plusieurs formes de modernité – par exemple selon les mouvements, depuis l'impressionnisme jusqu'à l'art minimal, en passant par le cubisme, le Bauhaus, l'expressionnisme abstrait ou le pop art –, le projet est à ce point partagé dans ses fondements que les conceptions liées à l'avant-garde, au moderne et à la modernité disparaîtront ensemble, vers le milieu des années 1970. Entre-temps, au début XXe siècle, est apparu un autre vocable, celui de modernisme, et avec lui un autre contenu esthétique.
L'une des premières raisons de l'éclatement de la modernité, lorsqu'elle se transforme en modernisme, est la considérable disparité des acceptions. Elle est augmentée par
l'embrasement intellectuel qui accompagne le cheminement des utopies et des mots d'ordres de la modernité dans les premières décennies du XXe siècle, aussi bien dans les pays d'Amérique du Sud (notamment au Brésil, où naît le mouvement appelé Modernismo) qu'en
Europe ou aux États-Unis. Une fois passée l'étape du creuset historique commun, à la fin du XIXe siècle, le modernisme laisse place à des modernismes aux enjeux très différents.
À la diversité de ces enjeux, qu'ils soient plastiques ou culturels (le modernisme au Brésil mêle ainsi les héritages européen, africain et indien), il faut ajouter les inégalités socioéconomiques et politiques entre les pays. Certains modernismes étaient et demeurent moins connus que d'autres, parce que leurs productions n'ont pas eu, ou n'ont toujours pas les moyens d'être diffusées. Une telle mise à l'écart, ou même oblitération du modernisme extra-occidental déforme la vision de ce qui fait partie ou non de l'histoire du modernisme, dont le modèle reste celui de l'art occidental. En ce sens, le modernisme, tel qu'il est aujourd'hui encore conçu en Occident, court le risque d'être, paradoxalement, antimoderne, alors que l'internationalisation des arts a été l'un des piliers de la modernité au cours des années 1920 et 1930.
 
L'histoire, partielle et partiale, du modernisme à l'occidentale a dominé et domine encore dans la plupart des esprits et, partant, dans les ouvrages d'historiens de l'art. Plus précisément, son modèle anglo-saxon prend ses sources en Angleterre, par deux voies différentes. On considère en effet que le modernisme, en tant que dénomination, y est né à la fin des années 1920 – bien qu'il s'applique alors à des artistes qui travaillaient déjà avant la Première Guerre mondiale, tels Wyndham Lewis, Jacob Epstein et Henri Gaudier-Brezska, mais également des poètes tels que Robert Graves, W. B. Yeats, Ezra Pound et T. S. Eliot (autrement dit, le groupe des imagistes et celui des vorticistes). L'autre voie, bien différente, est liée aux essais des critiques d'art Clive Bell et Roger Fry, qui développent la théorie de la
« forme signifiante », prônant l'importance de la forme au détriment des contenus. Cette idée influencera ensuite, outre-Atlantique, celles du héraut d'un tout autre modernisme : le critique d'art américain Clement Greenberg (1909-1994).

\paragraph{Vision moderniste de Greenberg}


Dès ses deux importants essais parus dans la Partisan Review, « Avant-garde et kitsch » (1939) et « Vers un nouveau Laocoon » (1940), Greenberg définit les lignes de force de ce qui va constituer la réflexion moderniste telle qu'elle est couramment reprise depuis, dans les pratiques comme dans les théories. Sa conception alimentera la majorité des débats esthétiques du demi-siècle suivant. Le second texte défend plusieurs points qui donnent lieu, aujourd'hui encore, à des querelles entre modernistes greenbergiens et antimodernistes – ou postmodernes –, car l'histoire des arts y apparaît soumise à une sorte d'évolution inéluctable, laquelle se manifeste nettement, selon l'auteur, dans les problématiques picturales.
Greenberg défend ainsi le purisme en ces termes : « La pureté, pour les peintres, consiste à être conscient des limites spécifiques du médium de chaque discipline et à les accepter pleinement. [...] C'est par la nature du médium qu'un art est unique, exclusivement lui- même. » Pour le critique, la modernité consisterait en une autocritique des arts, chaque médium tendant à la définition de sa spécificité (la peinture, faite pour l'œil, met en avant le plan, tandis que la sculpture s'adresse aux valeurs tactiles et met en avant le volume), ses qualités plastiques se trouvant définies par ses caractéristiques formelles. Par ailleurs, Greenberg voit dans l'abandon par les avant-gardes historiques des « luttes idéologiques de la société » une avancée. « L'histoire de la peinture d'avant-garde, dit-il, est celle de sa reddition progressive à la résistance à son médium ». Ces éléments font de lui un formaliste.
Or, dans une conférence sur « La Peinture moderniste » en 1960, Greenberg inscrit son travail et celui des artistes qu'il défend (Jackson Pollock, Morris Louis, Kenneth Noland) dans la suite logique de l'œuvre de Manet. En effet, selon l'auteur, « l'essence de l'esprit
moderne se définit par l'utilisation de certaines méthodes propres à une discipline pour critiquer cette discipline elle-même, non pas dans un but subversif, mais afin de délimiter exactement son domaine de compétence ».

\paragraph{Un projet inachevé}


Si l'on ne peut nier que Greenberg a mis l'accent, au-delà du médium, sur un regard direct porté sur les œuvres et leur concrétude (ce qui manquera trop souvent dans les travaux ultérieurs de critiques), en insistant sur l'importance du jugement esthétique, sa tendance formaliste l'a cependant conduit à évacuer l'aspect sociopolitique des œuvres et les questions qu'il pose, dont le moins que l'on puisse dire est qu'elles ne manquaient pas dans le champ de l'art, entre 1940 et 1980.
Même reprises, à partir du milieu des années 1960, par le critique et historien de l'art américain Michael Fried (né en 1939), les problématiques modernistes et formalistes de Greenberg n'auront plus le même impact. Sa vision avait fait l'impasse sur un grand nombre d'œuvres, de théories et de problématiques qui n'entraient pas dans le schéma d'une spécification progressive et critique des différents médias. Aussi nombre d'artistes et de mouvements contemporains de cette conception moderniste ou venant peu après – de Jasper Johns et du pop art aux tenants de l'art minimal et conceptuel, du Nouveau Réalisme à Gerhard Richter ou Daniel Buren – refuseront-ils radicalement ce que l'on nommera le dogmatisme moderniste.
Parallèlement à leur rejet, certains artistes, tels ceux du groupe anglais Art & Language, ou le Canadien Jeff Wall, proposent, à partir des années 1970, une autre lecture de la modernité et du modernisme. Puisant aux mêmes sources de l'art moderne que Greenberg, ils se réfèrent à Courbet et Manet, à Baudelaire ou aux impressionnistes. Par là même, ils démontrent alors, travaux et textes théoriques à l'appui, que le modernisme peut être compris différemment, et qu'il peut encore être riche d'enseignements. Leur réexamen du modernisme a un double avantage. D'une part, il permet de lutter contre l'éclectisme naissant de la postmodernité. D'autre part, il signifie que, dans une perspective critique non formaliste, il est encore possible, selon les termes du philosophe allemand Jürgen Habermas, d'appréhender la modernité comme « \textit{un projet inachevé }».
\sn{— Jacinto LAGEIRA}

\paragraph{La modernité en littérature : chercher du nouveau}


Concept incomplet, la modernité n'a en effet de sens que placée en regard d'un autre terme qui la définit par contraste, et qui lui-même varie selon le moment où le couple est utilisé : moderne et ancien, moderne et classique, moderne et postmoderne. Ce terme a une extension à la fois politique, technique, scientifique ; il possède en histoire et en philosophie
une acception claire et précise, puisqu'il désigne, pour la première, la période qui sépare les grandes découvertes (1492) de la Révolution française (1789), et pour la seconde, le moment philosophique du sujet, de Montaigne et Descartes à Kant et Hegel ; mais il ne cesse, en littérature, de se déplacer, et d'être régulièrement réactivé. Car cela fait plusieurs siècles que l'Occident prétend être moderne (l'adjectif précède de beaucoup, dans la langue, le substantif apparu au XIXe siècle). S'interroger sur la modernité revient à questionner cette permanence qui s'affirme toujours comme rupture : la modernité est en effet le mythe constamment reconduit d'une séparation avec ce qui précède ; elle est le nom porté par cette logique de différence sans cesse réaffirmée.
La modernité littéraire se construit par la revendication de critères esthétiques nouveaux, en réaction contre ceux qu'a légués la tradition ; en ce sens, le concept de modernité va toujours de pair avec l'histoire, c'est-à-dire avec une conscience profonde de soi comme acteur historique. L'humanisme de la Renaissance, qui médiatise la conscience de son présent dans son rapport à une Antiquité qu'il redécouvre, n'est à cet égard pas vraiment moderne. C'est bien plutôt la Querelle des Anciens et des Modernes, à la fin du XVIIe siècle, qui fixe les termes dans lesquels se pense la « modernité » : en effet, elle met en place un système où les premiers, partisans de l'imitation de l'Antiquité, et donc d'une tradition maintenue, s'opposent aux seconds qui invoquent la nécessité d'une esthétique nouvelle pour des temps nouveaux. Pour ces Modernes originels, c'est parce qu'ils appartiennent (au présent) à un grand siècle, celui de Louis XIV, qu'ils participent d'un régime esthétique comparable à ceux de Périclès ou d'Auguste (Charles Perrault, Parallèle des Anciens et des Modernes, 1688-1696). La raison de leur grandeur esthétique est d'ordre historique ; Houdard de la Motte, lorsque éclate la querelle d'Homère au début du XVIIIe siècle, ne dit pas autre chose, qui veut adapter l'Iliade au goût moderne en supprimant les passages qu'il juge ennuyeux ou vieillis. La modernité littéraire rejoint alors la modernité philosophique : pour l'homme moderne, ce qui prime est le rapport qu'il entretient au présent dans lequel il est pris.
Mais la modernité littéraire rencontre aussi la modernité philosophique et juridique parce qu'elle fait de l'individu le lieu de l'expérience artistique, la source de la norme esthétique, comme il est, de Kant à Habermas, celle du jugement normatif. C'est que la logique de la modernité porte ses propres et nouveaux critères de littérarité ; elle promeut comme sa règle la rupture des règles. L'auteur se définit par sa capacité non plus à imiter, mais à inventer, par son originalité ; celle-ci devient garante de la réussite de l'œuvre d'art par sa portée subversive, le rejet qu'elle sait opérer de ce qui précède. La modernité esthétique choisit le mélange des genres, des tons, des registres, contre les classifications héritées et la rhétorique (Jean Paulhan, La Terreur dans les Lettres, 1941) ; c'est Hugo qui met « un bonnet rouge au vieux dictionnaire » et rompt avec les partages et hiérarchies classiques au théâtre, c'est Baudelaire qui, en 1863, célèbre Le Peintre de la vie moderne. Hugo et Baudelaire, donc, et plus encore Rimbaud, Lautréamont, et Sade comme point de départ absolu.
 
 \paragraph{L'œuvre en état de crise}

Tel serait le régime paradoxal propre à la modernité, qui propose le nouveau comme valeur, ce qui la conduit à se nier elle-même : d'avant-garde en avant-garde, de rupture en rupture, la modernité est condamnée à se renouveler sans cesse ; elle prend dès lors la forme d'une crise jamais résolue, mais au contraire nécessairement toujours reconduite. Toujours menacée de se figer, elle ne peut triompher sans aussi, ipso facto, mourir. La modernité, c'est ainsi le romantisme contre le néo-classicisme, le Parnasse et le naturalisme contre le romantisme, le symbolisme contre le Parnasse et le naturalisme, et ainsi de suite, d'avant- garde en avant-garde. Antoine Compagnon a souligné dans Les Cinq Paradoxes de la modernité le caractère équivoque de sa formulation baudelairienne : la « passion du présent » de cette
« modernité esthétique » est aussi « calvaire », réaction contre la modernité sociale et industrielle. L'œuvre même de Baudelaire en témoigne par son ambiguïté et ses contradictions, qui conjoint forme poétique classique et thème urbain contemporain, forme poétique nouvelle et rejet de l'innovation ou de la rupture.
La modernité, en effet, s'attache à dire le monde dans sa contemporanéité, en tant que celle-ci est éminemment singulière ; elle se veut l'expression de la situation historique désenchantée de l'homme moderne, qui ne trouve plus dans la métaphysique ou la religion de quoi donner sens au monde : Dieu y est mort, ou en tout cas en est absent. La modernité esthétique exprime à la fois le constat historique de cette rationalisation du monde, son assomption (le mythe fondateur du progrès interdit, du reste, de penser qu'il puisse en aller autrement), et le regret, la déploration de cet état de fait.
Le romantisme joue à cet égard le rôle d'un modernisme radical (Stendhal, Racine et Shakespeare, 1823) : il bouleverse les sujets proposés à l'intérêt littéraire, choisissant plutôt les sujets nationaux, médiévaux ou même contemporains, que les fables empruntées à l'Antiquité classique ; est romantique ce qui est moderne, est moderne ce qui n'est pas classique, est classique ce qui n'est pas romantique. Le cercle, parfait, domine le XIXe siècle tout entier. Le processus de sécularisation des sciences et des arts peut se déployer, et conférer en retour à la littérature toute la sacralité laissée disponible par cette vacance du religieux qu'elle favorise. Le développement du roman, notamment psychologique, porte l'autonomie croissante de l'individu, et son rapport à un corps social qui lui est de plus en plus étranger : il valorise, de Balzac à Zola, la singularité, l'authenticité appelée à devenir grille de lecture du réel.
L'histoire de la modernité est ainsi celle d'une constante redéfinition, qui est aussi bien dénégation : le moderne n'est tel que de récuser les modernités passées, faute de quoi il risque de devenir passé, tradition, répétition, cela même que par définition il nie. C'est ce qui favorise le développement des avant-gardes, qui fonctionnent en littérature comme la mode dans la sociologie : la métaphore militaire elle-même porte témoignage de cette condamnation portée par le mythe du progrès à sans cesse avancer vers un \textit{terminus ad quem}
supposé exister, mais par définition inatteignable. La remise en cause de ce qui préexiste se poursuit alors indéfiniment. C'est alors la postmodernité qui seule peut venir arrêter ce procès.
\sn{— Alain BRUNN}



\section{Bibliographie}



※ Genèse de la modernité
D. APTER, The Politics of Modernization, Chicago, 1965


R. ARON, Les Désillusions du progrès. Essai sur la dialectique de la modernité, Paris, 1969


G. BALANDIER, Anthropologie politique, Paris, 1967


D. J. BOORSTIN, L'Image (The Image, or What Happened to the American Dream, 1962), trad. J. Claude, Paris, 1963
