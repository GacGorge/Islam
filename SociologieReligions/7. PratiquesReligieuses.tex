\chapter{Des pratiques religieuses revisités}
\mn{PLAN 4. ISTR. C. Valasik
}

\section{des modèles revisités}
 (D. HERVIEU-LEGER)
\subsection{La figure du converti}
\mn{Pélerin et le converti, Hervieu Leger}
Avant ma vie n'avait pas trop de sens. Disqualification de la vie précédente : 
\begin{quote}
    il s’adonne jusqu’à 30 ans «aux vanités du monde, avec un grand et vain désire d’y gagner de l’honneur».Récit du pélerin
\end{quote}
\paragraph{révélation}
On aime à y revenir et le raconter. Cette révélation donne un sens à la vie.

\paragraph{ma vie a changé} je peux vivre des épisodes difficiles mais je me sens porté. Mais je suis porté non seulement par la personne que j'ai rencontré mais aussi la communauté.

\paragraph{Radicalisation} Ils vont devenir très normatifs. \textit{je vais faire ce qui est la norme}. 
\begin{Ex}
    je vais faire les 5 prières de l'Islam
\end{Ex}
Je vis de cette radicalisation car comme je suis dans la joie et donc, si je m'éloigne de la pratique, je risque de quitter cette joie. \textit{visibilité de cette pratique}.

\begin{Ex}
    On assiste actuellement à une radicalisation des jeunes juifs par rapport à la Casheroute.
\end{Ex}

\begin{Ex}
    Conversion dans les professions de santé : comprendre pourquoi cela va mal et permet de gérer l'impuissance. 
\end{Ex}

\paragraph{Je veux témoigner de ma vie} Je vais faire du prosélytisme : \textit{volonté de témoigner}. 

\paragraph{intégrer le converti dans une Eglise} Des processus pour intégrer la personne qui bouge. 

Témoignage et expérience du choix individuel
\paragraph{3 modèles de converti}
\begin{itemize}
    \item Ceux qui n'avaient aucune religion
    \item Ceux qui avaient une religion (bcp catholicisme vs pentecotisme ou Islam)
    \item converti de l'intérieur.
\end{itemize}

\paragraph{Aux USA} la foi qu'on peut avoir une seconde chance. Cette figure du converti est très forte. En France, pas de seconde chance valorisée.


\paragraph{Avec l'individualisme} on est dans des logiques de conversion intérieure. Du fait de l'importance de la liberté.


\subsection{La figure du pèlerin}

Mobilité et sociabilité extra-ordinaires

\paragraph{figure très moderne} car décision

\paragraph{cela coupe du quotidien} On va se centrer sur ce temps

\paragraph{Rencontres} Proximité émotionnelle et affective très forte des personnes qu'on rencontre. 

\textit{autres espaces, autres temps, autres personnes}

\paragraph{A des moments particuliers de la vie} réflexion, chomage, retraite, maladie (Lourdes).

\paragraph{fatigue} les JMJ sont considérés comme des temps de pélerinage, et les jeunes aiment. 

\paragraph{Mais pas de suite sur du temps long dans la paroisse}





\section{les formes modernes de la religiosité }
- Individualisation du croire
- Importance de l’émotion : le protestantisme émotionnel, les charismatiques catholiques
- La guérison
- La conversion

