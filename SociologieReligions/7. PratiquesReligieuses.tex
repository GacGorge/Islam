\chapter{Des pratiques religieuses revisités}
\mn{PLAN 4. ISTR. C. Valasik
}

\section{des modèles revisités}
 (D. HERVIEU-LEGER)
\subsection{La figure du converti}
\mn{Pélerin et le converti, Hervieu Leger}
Avant ma vie n'avait pas trop de sens. Disqualification de la vie précédente : 
\begin{quote}
    il s’adonne jusqu’à 30 ans «aux vanités du monde, avec un grand et vain désire d’y gagner de l’honneur».Récit du pélerin
\end{quote}
\paragraph{révélation}
On aime à y revenir et le raconter. Cette révélation donne un sens à la vie.

\paragraph{ma vie a changé} je peux vivre des épisodes difficiles mais je me sens porté. Mais je suis porté non seulement par la personne que j'ai rencontré mais aussi la communauté.

\paragraph{Radicalisation} Ils vont devenir très normatifs. \textit{je vais faire ce qui est la norme}. 
\begin{Ex}
    je vais faire les 5 prières de l'Islam
\end{Ex}
Je vis de cette radicalisation car comme je suis dans la joie et donc, si je m'éloigne de la pratique, je risque de quitter cette joie. \textit{visibilité de cette pratique}.

\begin{Ex}
    On assiste actuellement à une radicalisation des jeunes juifs par rapport à la Casheroute.
\end{Ex}

\begin{Ex}
    Conversion dans les professions de santé : comprendre pourquoi cela va mal et permet de gérer l'impuissance. 
\end{Ex}

\paragraph{Je veux témoigner de ma vie} Je vais faire du prosélytisme : \textit{volonté de témoigner}. 

\paragraph{intégrer le converti dans une Eglise} Des processus pour intégrer la personne qui bouge. 

Témoignage et expérience du choix individuel
\paragraph{3 modèles de converti}
\begin{itemize}
    \item Ceux qui n'avaient aucune religion
    \item Ceux qui avaient une religion (bcp catholicisme vs pentecotisme ou Islam)
    \item converti de l'intérieur.
\end{itemize}

\paragraph{Aux USA} la foi qu'on peut avoir une seconde chance. Cette figure du converti est très forte. En France, pas de seconde chance valorisée.


\paragraph{Avec l'individualisme} on est dans des logiques de conversion intérieure. Du fait de l'importance de la liberté.

\paragraph{Changement de paradigme} sur le religieux. \mn{Coming Out : très drole}


\begin{Synthesis}
    Une approche de converti très moderne du fait de l'individualisme.
    Les convertis vont prendre certains aspects de la religion plus que d'autres.
    
\end{Synthesis}
\mn{29/11/22}
\paragraph{on va étendre ce concept à d'autres domaines} \textit{conversion politique}. \textit{Aborder la conversion écologique}

\subsection{La figure du pèlerin}

\begin{Def}[pélerinage]
    On ne vit que par et pour sa religion pendant un moment. On va vivre des moments intenses.
\end{Def}

\paragraph{mobilité géographique} temps extraordinaire. 

Mobilité et sociabilité extra-ordinaires

\paragraph{figure très moderne} car décision

\paragraph{cela coupe du quotidien} On va se centrer sur ce temps

\paragraph{Rencontres} Proximité émotionnelle et affective très forte des personnes qu'on rencontre. 

\textit{autres espaces, autres temps, autres personnes}

\paragraph{A des moments particuliers de la vie} réflexion, chomage, retraite, maladie (Lourdes).

\paragraph{fatigue} les JMJ sont considérés comme des temps de pélerinage, et les jeunes aiment. \mn{Taizé pareil.}


\paragraph{Attrait des jeunes} Les modalités d'engagement des jeunes : on donne tout à fond. Avec une difficulté, qui est le \textit{retour}.


\paragraph{Mais pas de suite sur du temps long dans la paroisse} Recherche d'une pratique valorisant la louange, pas dans la paroisse (Emmanuel,...).


\paragraph{Ce qui est important, c'est le chemin, pas le but}

\paragraph{Renforce l'appartenance religieuse} Un \textit{nous} qui va se matérialiser par d'autres pélerinages, d'autant plus 




\paragraph{Ne pas opposer pelerin et converti}




\paragraph{De la religiosité populaire} Ce n'est pas la même population qui va à chaque pélerinage. Impuissance des institutions qui essayent d'orienter mais qui lui échappent.
On peut voir aussi des pratiques perpétuées (flagellation,...). 

\mn{Elisabeth Claverie : anthropologue des pélerinage, décrit les pélerinages : qu'est ce qui se passe pour eux. Pourquoi ils vont à Mejugorge ? Enjeux commerciaux, politiques}


\subsection{les fidèles habituels}
Continuent d'être en lien avec l'institution, ne l'ont pas quitté.
Ceux qui quittent l'Eglise cathologique sont par définition difficiles à identifier, s'ils partent de façon silencieuse. 
Les fidèles habituels continuent à avoir un lien avec l'institution mais ils s'autorisent des accommodements.

\section{les formes modernes de la religiosité }

Il s'agit ici de revoir les différents points déjà vus de façon différente.

\begin{Def}[Religiosité]
    Cela relève du religieux mais pas toujours de façon très bien définie
\end{Def}


  \paragraph{Individualisation du croire} : quelque soit mon modèle, je suis travaillé par la modernité et l'individualisation. Je crois en fonction de l'héritage reçu et du cadre (Gabon, France) mais je prends que ce qui fait sens à un moment dans ma vie. Les gens n'ont pas forcément conscience.

    \begin{itemize}
        \item  on va mettre de l'exigence mais on va faire valider notre pratique par des laics et non des religieux. 
    \end{itemize}
    
      \paragraph{Importance de l’émotion} : le protestantisme émotionnel, les charismatiques catholiques
    \begin{itemize}
        \item Face à la suprématie de la raison, un certain nombre de courants s'opposent à cette suprématie de la raison \mn{Mai 68 était aussi anti raison et pourtant les religions sont souvent anti 68}
        \item on pense que la raison éloigne de la vérité (croyance) et que finalement, à trop vouloir comprendre, on s'éloigne de ce qu'est la vie.  Tous ceux qui ont la connaissance, auraient une position du surplomb, méconnaitraient la \textit{vraie vie} qui elle serait non raison.
        \item Le prêtre, le patron, Policier, président, Père, professeur : les figures d'autorité vont être remis en cause dans les années 60. 
        \item valoriser les sentiments au bénéfice de la raison. Ce qui serait juste dans ma vie, c'est ce que je ressens. \mn{Cela s'auto justifie : \textit{je ne sens pas le prof}, \textit{je ne le sens pas} et donc je ne le fais pas. Ce n'est pas argumenté.} 
        \item si vous critiquez ce que je ressens, vous m'attaquez. Quand on est sur des discussions sur des argumentaires et la raison, on ne se sent pas attaqué et donc le dialogue devient possible.  D'où la communication non violente : \textit{je comprends ton ressenti}. Attention, cela ne veut pas dire que le sentiment serait faux. 
        \item dans les mouvements religieux, important en particulier pour les jeunes.  Des discours de prêtres à partir du témoignage et pas des figures de conviction par la connaissance. \textit{des liturgies fortes qui engagent le corps}. 
        \item Les charismatiques catholiques et évangéliques vont valoriser l'émotion. Certains Groupes évangéliques : \mn{Sebastien Fath}
        \begin{Ex}[Pentecotisme - Immédiaté]
            Je dois changer de travail. je tire une phrase de la bible au hasard.
            Dire ce que la phrase me fait ressentir. 
            Si j'ai le Lévitique, j'ai le droit de retirer. Comment les personnes questionnent la phrase et se l'approprient. \textit{Comment le texte religieux nous aide au quotidien}. Avec le Groupe qui travaille avec vous.
        \end{Ex}
         Il faut que la religion soit efficace. 
         \item \textsc{Ces personnes ont besoin d'un cadre normatif très clair} : s'habiller, s'alimenter. 
         \begin{Ex}[Pentecotisme]
             On va nettoyer sa vaisselle avant de sortir de chez soi. 
         \end{Ex}
         Et finalement, on va rationaliser sa vie par ce cadre. Je le fais \textit{parce que je le sens}. Dans notre société, il y a tellement de possibilités de choix que 
    \end{itemize}
     \paragraph{La guérison}
 Demande que la religion nous guérisse. dés-envoûtement. La religion est le lieu du bien mais aussi le seul lieu où on parle du mal (exorcisme). Cela peut être actionné par des flux migratoires. C'était en sommeil mais cela se réveille.

 Les religions sont un soutien auprès des personnes malades. Comme la maladie découpe la personne (\textit{patient touché par tel maladie}). Ce que propose les confessions religieuses, c'est de proposer un parcours au malade : \textit{tu es certes malade mais tu n'es pas que cette maladie}. 

 Il peut y avoir une injonction moderne de trouver du sens, alors qu'il n'y en a pas forcément. Les personnes en situation de maladie se tournent vers la religion, avec une acceptation de la maladie par les religions (cf S Jean Paul II).

 Différentes demandes : 
 \begin{itemize}
     \item guérison
     \item moins souffrir
     \item homéopathie, ostéopathe : aide à reconstruire une globalité. les gens qui sont intéressés par ce type de pratique vont être attirés par les groupes spirituels.
 \end{itemize}
     
     \paragraph{La conversion}
Valorisation, surtout dans le protestantisme évangélique, avec l'impératif de témoigner de sa conversion. Je vais convertir sur mon témoignage. 
     
 \mn{voir le film sur Bourdieu : \textit{la sociologie est un sport de contrat}}


\paragraph{devoir} Pour les travaux, s'appuyer sur des textes ou des livres. Identifier la question de départ. Et du coup, on peut savoir si on a besoin de lire ou pas. Quels sont les moyens de répondre. 