\chapter{Laicité}



 
\section{Retour sur la laicisation et la laicité}

\begin{Def}[laicisation]
Mouvement historique porté par des groupes particuliers
\end{Def}

\begin{Def}[laicité]
Complexe, on ne le traduit pas
\end{Def}

\paragraph{Historique de la révolution française} se passe dans un pays dominant et peu dominé, pays puissant, régime installé qui parait stable et qui tout à coup, est détruit par son peuple. il est possible qu'un régime politique soit détruit.  C'est en ce sens qu'elle est toujours un modèle puissant.



\subsection{Citoyenneté}
 

\begin{Def}[citoyenneté]
La citoyenneté française et le modèle français : concept juridique. Séparation entre la sphère publique
(en lien avec les institutions étatiques) et le privé, entre le citoyen et l’individu

\end{Def}

\paragraph{Individu} Il y a des individus, sphères privées. A l'individu, des particularismes : âge, milieu social, langue, vegan, couleur,...

\paragraph{citoyen} Dans la sphère publique (école, hôpital,...), je suis un citoyen ou une citoyenne et je dois être détaché de mes particularismes. Il ne peut y avoir de citoyen que si j'abandonne mes particularismes. 


\subsection{Le Modèle républicain}

\paragraph{but de l'école en France : faire des citoyens} Le but de l'école, enlever les particularismes pour en faire des citoyens. On doit donner à l'enfant des valeurs communes pour faire son choix de façon citoyen. Idée politique; 

\begin{Ex}
Les premières influences enlevées, c'est les influences politiques et syndicales. On voit bien la difficulté.
\end{Ex}

\paragraph{On n'a pas besoin de corps intermédiaire} puisque les citoyens votent (sinon la nation est en danger), on choisit un programme qui doit être appliqué. Modèle très centralisé. Fluidité par le vote entre l'individu et l'Etat. 

\paragraph{contre les corps intermédiaires} les corps intermédiaires rassembleraient les particularismes. Les individus prenant les droits spécifiques sur la défense des droits collectifs.

\begin{Synthesis}
Un modèle très centralisé, universel, rationnel, et donc on va pouvoir surmonter les particularismes et s'implanter partout. 

\end{Synthesis}

\paragraph{Qu'une langue : le Français} interdiction du folklore, du patois. avant de s'imposer aux étrangers, d'abord aux français, avec des résistances (Bretagne,...). Toute affirmation du particularisme (religieux) cristallise en France (Gilets jaunes,...). 

\paragraph{une égalité de droit mais pas une égalité de fait} A partir du \textit{Front Populaire}, le terme d'\textit{Etat Providence}, des répartitions conséquentes, pour éviter la reproduction des élites (égalité des chances).
On verse des allocations aux \textit{travailleurs} et non pas à des individus. La place de l'identité au travail est importante car elle fait référence au citoyen et non en individu. 

\paragraph{Genre} Invariant : homme / femme et non particularisme. 

\paragraph{Récit prophétique de l'homme clé} centralisation. Management vertical. Sociologie politique du \textit{conflit}. \sn{Allemagne : compromis, Suède : le consensus}. Comme il y a peu d'intermédiaires, on manifeste dans la rue (on se compte dans l'espace semi publique) : masculin.  Quand cela bloque (essence,;..), on compte\sn{ancien : \textit{les Jacqueries}}. Puis on arrive au dialogue.  On commence par la phrase \textit{oui, mais}. vu comme arrogant.  




Laïcisation : processus par lequel les diverses institutions sociales conquièrent leur autonomie et se
dotent d’idéologies, de références et de règles de fonctionnement propres.

\paragraph{Philippe Portier} Comment la laicité, qui est une norme, est devenu progressivement une valeur (clivante), avec conflit d'interprétation : laicité ouverte,...  \sn{18/10/22 A lire Portieer. Puis on regardera scientologie et temple solaire} Depuis les voiles, comment cette laicité va être reinterpétée. 



\section{L’inclination identitaire de la laïcité française.
Retour sur une controverse (1988-2018)}


\mn{Philippe Portier 
Philippe Portier est directeur d’études à l’École pratique des hautes études (psL).}




Le texte précédent s’arrête au moment où la laïcité se laisse pénétrer par la dynamique de la reconnaissance 1. Nous sommes alors dans les années 1960-1980. Sans remiser la loi de 1905, la République s’ouvre dès lors, par le truchement de nouvelles législations comme la loi Debré de 1959, qui prévoit le financement des écoles privées, ou de nouvelles circulaires comme celle qui, en 1975, admet la possibilité de constituer des
« carrés confessionnels » dans les cimetières municipaux, à une nouvelle donne juridique. La IIIe République avait construit le régime de laïcité autour de la séparation du privé et du public2. Les frontières entre les deux ordres de réalité deviennent là bien plus poreuses : le public se privatise, tandis que le privé se publicise.

Or, ce modèle, qui trouvait son principe de constitution dans l’idéologie de la diversité heureuse, émanée elle-même de la « deuxième révolution française », s’est trouvé bientôt bousculé par toute une série d’évolutions : les enjeux se sont reconfigurés, les discours se sont ré-agencés, les règles se sont transformées. Une « nouvelle laïcité » en est résultée qui, sans remettre en cause les éléments récognitifs de la période immédiatement antécédente, non plus d’ailleurs que certains des principes fondamentaux

\mn{}
hérités de la loi de séparation (tels que la neutralité absolue du service public, le non-financement direct des cultes, ou la liberté de croire et de ne pas croire), leur a adjoint un dispositif panoptique : l’expression publique du religieux s’est trouvée placée, sans qu’on puisse parler cependant d’un retour au régime juridictionnaliste du xixe siècle, sous le contrôle accentué de l’État.

\subsection{épreuves}
\mn{1.	La reconnaissance ne renvoie pas ici au système des cultes reconnus de l’époque concordataire. On veut rappeler simplement que l’État accorde des prérogatives inédites aux cultes et aux croyants, sans les doter cependant d’un statut de droit public.
2.	Il y avait eu toutefois des évolutions en ce sens dès 1908 : la loi accepte alors que les collectivités publiques assument sur leurs budgets l’entretien et les réparations des édifices du culte dont elles sont propriétaires. Ces évolutions existent à la marge, comme des concessions aux exigences de l’ordre public. Sur ce point, Philippe Portier, L’État et les religions en France. Une sociologie historique de la laïcité, Rennes, pur, 2016.
3.	Guillaume Cuchet, Comment notre monde a cessé d’être chrétien. Anatomie d’un effondrement, Paris, Le Seuil, 2018.
4.	Alfred Dittgen, « Évolution des rites religieux dans l’Europe contemporaine. Statistiques et contextes », Annales de démographie historique, n° 106, 2003/2.
}
\begin{Synthesis}
Un apaisement avec le catholicisme du fait de son poids relatif plus faible.
\end{Synthesis}

La transformation de la laïcité a partie liée avec la transformation du religieux. Le conflit des deux France s’était apaisé dans les années 1960- 1970, malgré quelques tensions autour de la question scolaire. Pour deux raisons. Au plan quantitatif, le catholicisme, contre lequel s’était construite la laïcité hexagonale, montrait des signes d’affaiblissement. Près de 35 \% des Français assistaient chaque dimanche à la messe dans les années 1950. Le chanoine Boulard, grand statisticien de l’église catholique, saisit déjà une diminution, en particulier dans les catégories les plus jeunes, dès le milieu des années 19603. Le déclin s’accentuera encore dans les années 1970 : la pratique régulière ne concerne plus que 14 \% de la population en 19754. Au plan qualitatif, le catholicisme révoque la politique de l’intransigeance qu’il avait construite après la Révolution française, pour faire droit dorénavant, à la faveur des évolutions permises par les grands textes du concile Vatican II tels Gaudium et spes et Dignatatis humanae, aux principes de la démocratie constitutionnelle. Cette reconfiguration ne pouvait que faciliter, du côté de l’État, une pratique d’ouverture, d’autant que les autres cultes – juif et protestant – persévéraient alors dans le tropisme modernisateur qu’ils avaient dessiné déjà sous la IIIe République.

\paragraph{Desécularisation dans les années 80 ?}À partir des années 1980, le religieux se recompose. Sans doute le mot de « désécularisation », employé par Peter Berger\sn{Peter Berger (sous la
direction de), The Desecularization of
the World : Resurgent Religion and
World Politics, Washington, D.C.,
Ethics and Public Policy Center,
Grand Rapids, Michigan, W.B.
Eerdmans, 199}, est-il excessif : l’époque n’est nullement celle du retour uniforme au religieux intrusif et englobant. Les phénomènes de détachement s’approfondissent, selon un mouvement accéléré depuis le milieu des années 1970. En France, la part des sans religion dans la population est aujourd’hui de plus de 40 \%, contre 4 \% dans les années 1950. Les catholiques déclarés ne sont plus que 50 \% ; ils étaient au-dessus de 90 \% après la Seconde Guerre mondiale. Il reste que se sont construits, concomitamment, des pôles de réaffirmation confessionnelle, comme s’il s’agissait au fond, de ce côté de la société, de contrebalancer par un sursaut d’identité l’expansion des processus de désaffiliation. Aucun des mondes religieux ne s’est tenu à l’écart de cette
« revanche de Dieu », ni le monde juif qui, au cours de ces dernières années, a vu se consolider ses fractions orthodoxes et ultra-orthodoxes, ni le monde protestant au sein duquel le courant évangélique a connu une expansion significative (en particulier dans les milieux de récente immigration), ni le monde catholique, comme on l’a observé, par exemple, lors de la controverse autour du « mariage pour tous ». La même tendance marque l’univers musulman, et de manière bien plus visible socialement, parce que la population musulmane est numériquement importante (6-7 \% de la population), parce que l’affirmation de son appartenance tranche davantage avec la culture d’une société demeurée attachée, sans même s’en rendre compte, à tout un corpus de gestes chrétiens, parce que reste enfin dans le tréfonds de la société française le vif souvenir de l’antagonisme colonial.
\paragraph{Ce revival musulman, }
dont on voit les indices dès la fin des années 1980, en particulier dans les jeunes générations mal intégrées (ou qui s’estiment discriminées6), s’exprime souvent dans le cadre de la sphère privée, où se sont développés la pratique de la prière, l’observance du jeûne du mois de Ramadan, le respect des obligations alimentaires (autour de la notion, de plus en plus extensive, du halal). Il affecte aussi la sphère publique : le sursaut identitaire s’accompagne fréquemment, en effet, d’une demande de reconnaissance adressée aux pouvoirs publics. Cette demande vaut sur le terrain financier (aide à la construction de mosquées), sur le terrain politique (demande d’insertion dans les dispositifs de participation politique), sur le terrain symbolique (obtention du droit de porter des signes religieux dans l’espace étatique). C’est d’ailleurs, principalement, autour de cette dernière question, celle du vêtement – si prégnante en France depuis la Révolution de 1789 –, que s’est nouée une part essentielle de notre dispute publique au cours des dernières décennies. Sans doute faut-il ajouter que ces attitudes, à la structuration desquelles ont contribué toute une succession de groupements religieux installés dans les « banlieues de la République » (tabligh d’abord, mouvement frériste ensuite, puis groupes salafistes), sont portées généralement par une idéologie « absolutiste7 » : 
\begin{Def}[Idéologie absolutiste]
les populations concernées accordent volontiers à la religion musulmane un privilège absolu de véridiction, auquel ils adjoignent une large défiance à l’égard des valeurs du libéralisme culturel et politique.

\end{Def}

\paragraph{pourquoi cette requête de reconnaissance aussi tôt en France}
Ces requêtes de reconnaissance publique auraient pu ne pas provoquer le débat. En Angleterre par exemple, il a fallu les attentats des années 2000 pour qu’une controverse émerge, sous des formes différentes d’ailleurs de celles rencontrées en France. Ici, la dispute se construit dès la fin des années 1980, à partir de l’affaire des foulards du collège de Creil en septembre-octobre 1989, dans un contexte échauffé par la condamnation
\mn{6.	Hakim Al Karoui, L’islam, une religion française, Paris, Gallimard, 2018.
7.	Olivier Galland et Anne Muxel, La tentation radicale, Paris, Puf, 2018.
} 
à mort par l’ayatollah Khomeini de Salman Rushdie, en raison de la publication des Versets sataniques : elle porte d’emblée sur la signification qu’il convient de donner au concept de laïcité. Les décennies suivantes la verront s’approfondir sous l’effet d’abord de l’importation sur le territoire national de la guerre algérienne, et bientôt des attentats de New York, Londres, Madrid. Le terrorisme lié à Daech dans les années 2010, avec pour points d’orgue les drames de Charlie Hebdo, de l’Hyper Casher et du Bataclan en 2015, renforcera encore – sur le fondement d’un questionnement autour du lien de consécution entre identité, radicalité et terrorisme – la réflexion publique autour des modes de régulation étatique du religieux.

\subsection{doctrines}
Comment répondre donc à l’épreuve que constitue pour la société française la multiplication des revendications identitaires ? Depuis le début des années 1990, les positions discursives dans le champ intellectuel s’agencent autour de deux grandes polarités, se plaçant l’une et l’autre du reste sous l’égide de l’idée de République. 
\paragraph{pluralisme vs universalisme} L’école « multiculturaliste »(ou pluraliste), attachée à une laïcité inclusive, est encore dominante au début de la période ; elle est supplantée par l’école « universaliste » (ou moniste) à partir du tournant du siècle, avec le soutien d’une opinion publique sans cesse plus défavorable aux requêtes musulmanes\sn{Sur la genèse de ce débat, Philippe Portier, L’État et les religions en France…, op. cit.,
chapitre 9.
}.

\paragraph{L’école « multiculturaliste »} Porté dans les années 1990 par des figures intellectuelles comme Alain Touraine, Michel Wieviorka, ou Alain Renaut, inspirés les uns et les autres par les travaux de Charles Taylor ou de Will Kymlicka, le courant « multiculturaliste » se refuse à identifier la neutralité politique avec l’abstraction religieuse : un État est d’autant plus neutre qu’il accepte la pluralité en sa sphère même. Les tenants de cette thèse n’entendent pas promouvoir les droits collectifs ou communautaires. Leur dessein est ici, bien plutôt, d’obtenir une reconnaissance du droit individuel à la différence. Deux raisons viennent justifier cette aspiration à une « laïcité de reconnaissance \sn{9.	Selon l’expression de Jean-Paul Willaime, Le retour du religieux dans l’espace public. Vers une laïcité de dialogue et de reconnaissance, Lyon, Olivetan, 2018.} ». D’une part, au plan philosophique, il est dans l’ordre de la démocratie constitutionnelle que chacun puisse exprimer librement son « authenticité », pourvu que celle-ci, dans son déploiement externe, ne remette pas en cause les droits d’autrui ni l’ordre public. D’autre part, en laissant les individus manifester leur identité, la République abaisse le niveau de frustration des nouveaux arrivants, et favorise par là leur intégration dans la société globale. Cette ligne juridique, qui bouscule la séparation privé-public, n’est rien d’autre d’ailleurs, ajoutent nos auteurs, qu’un prolongement des principes de la loi de 1905, dont l’orientation libérale,
 voulue par Briand, portait en germes déjà cette possibilité \textit{récognitive}\sn{Selon l’une des thèses de Jean Baubérot, Les sept laïcités, Paris, éditions de la Maison des sciences de l’homme, 2015.}. Ce discours est volontiers porté aujourd’hui par une essayiste comme Rokhaya Diallo, l’une des fondatrices du groupe des Indivisibles.

\paragraph{les indigènes de la républiques} Du côté des élaborations multiculturalistes, il y a lieu de faire place aussi aux positions développées par les Indigènes de la République (dont Houria Bouteldja est la porte-parole). Ce groupe créé en 2005 (après le vote de la loi sur le port du voile à l’école publique) s’inspire volontiers de la pensée d’anthropologues comme Talal Asad ou Saba Mahmood. De facture holiste et non individualiste, son raisonnement débouche sur une ossification des identités communautaires. D’une part, les Indigènes récusent notre laïcité présente, dans laquelle ils voient une expression du « pouvoir blanc » : sa singularité est, en effet, de remettre en cause le caractère englobant de la religion que la « personnalité musulmane » place au cœur de son imaginaire. Cette privatisation marquait déjà l’éthos colonial. L’État continue de l’imposer, comme on l’a vu avec la législation sur le voile intégral en 2010. D’autre part, les Indigènes défendent le principe de l’autonomie des groupes subalternes. Pour ce qui a trait à la communauté musulmane, il convient, affirment-ils, de faire droit à la structure de sens qui la définit, sans chercher à lui imposer les catégories restrictives venues du seul Occident. Cette idéologie « décoloniale », qui anime également le Collectif contre l’islamophobie, amène le groupe d’Houria Bouteldja à récuser la liberté de caricaturer des personnages saints, au motif qu’elle nie le système islamique de la figuration. L’opposition de ce courant à Charlie Hebdo, relayée par Edwy Plenel du journal Médiapart, procède de cette analyse, dont on voit qu’\textit{elle réduit le sujet à n’être que l’expression d’une culture qui l’englobe, sans souci d’établir des ponts entre les différents groupes constitutifs de la société.}

\paragraph{Universalisme} L’école de la « laïcité universaliste » – qu’animent, dès la fin des années 1980, Catherine Kintzler ou Élisabeth Badinter – entend bien, quant à elle, fixer dans le marbre la séparation traditionnelle du privé et du public, non cependant sans donner du « public » une définition bien plus extensive que ne le faisait la IIIe République : ici, l’impératif de neutralité suppose \textit{l’exil du religieux en dehors de la sphère d’étaticité}. Les arguments s’opposent trait pour trait à ceux de la ligne précédente. D’abord, l’expression des différences peut mettre en péril la liberté individuelle. On l’a vu avec le port du voile à l’école. Il exprime une abdication de soi. Dès 1989, au moment de l’affaire des foulards de Creil, Élisabeth Badinter et Régis Debray affirment que \begin{quote}
« tolérer le foulard islamique, ce n’est pas accueillir un être libre ».     
\end{quote}
Ensuite, le multiculturalisme porte atteinte à la cohésion de la nation. C’est le grand thème du « communautarisme » : en autorisant la manifestation des identités singulières, l’État contribue à figer les individus dans des solidarités partielles qui les éloignent de
la communauté politique. Dominique Schnapper rappelle ainsi dans La France de l’intégration qu’on ne peut 
\begin{quote}
« être Français que par la pratique d’une langue, l’apprentissage d’une culture, la volonté de participer à la vie politique et économique\sn{Dominique Schnapper, La France de l’intégration, Paris, Gallimard, 1991, p. 63.
12.	En dépit de quelques connivences avec certains militants du Nouveau parti anticapitaliste ou de la France insoumise.
} ».     
\end{quote}
C’est autour de ce modèle rationaliste que s’est constitué en 2015, en réponse aux attentats, le Printemps républicain, animé par Laurent Bouvet : son idée est de reconstruire un État fort, capable d’unifier les conduites et les pensées autour du « commun », étant entendu, comme l’affirme son Manifeste, que 
\begin{quote}
« la République, ce n’est pas la projection des croyances de la société».
    
\end{quote}

\paragraph{identité chrétienne}
Dans ce courant moniste\sn{Système qui considère l'ensemble des choses comme réductible à un seul principe (opposé à dualisme, pluralisme).}, il faut situer aussi les auteurs qui considèrent que la France doit se retrouver de nouveau autour de sa « marque chrétienne ». Ce modèle de la « nation » se retrouve également, selon des harmoniques diverses, chez des auteurs comme Rémi Brague, Jean-Luc Marion ou Pierre Manent. Le dernier Finkielkraut va également dans ce sens dans son apologie du judéo-christianisme. Cette tendance s’exprime, de manière plus abrupte, mais non moins efficace, dans les textes de Philippe de Villiers (\textit{Les cloches sonneront-elles encore demain ?}) ou d’éric Zemmour (\textit{Le suicide français}). Les tenants de cette thèse se rapprochent de l’école « républicaine » évoquée plus haut, en défendant le principe d’une politique restrictive à l’égard des revendications musulmanes. Ils s’en distinguent cependant par le fait que rapportant la France à son identité religieuse, plus qu’à sa provenance rationaliste, ils se montrent davantage favorables à la reconnaissance des prérogatives des religions
« traditionnelles » (catholicisme, protestantisme, judaïsme).

Comment la sphère politique s’est-elle articulée avec la sphère intellectuelle ? Le courant « multiculturaliste holiste » a peu essaimé en son sein12. Au cours de la période récente, la laïcité récognitive a pu inspirer des acteurs comme Lionel Jospin dans les années 1990 ou Emmanuel Macron aujourd’hui, la laïcité rationaliste bien davantage Manuel Valls ou Jacques Chirac, la laïcité identitaire Nicolas Sarkozy ou Marine Le Pen\sn{	François Hollande marque par son indécision : il hésite entre la première et la seconde laïcité. Sur l’opposition des trois dernières politiques présidentielles en matière de laïcité, voir Philippe Portier, entretien avec Bernadette Sauvaget, Libération, 12 mars 2018.
 }.

\subsection{politiques}
\paragraph{Dispute est performative, elle crée l'invention de dispositifs juridiques }La sociologie des disputes politiques montre qu’une controverse ne se résume pas à n’être qu’un échange d’arguments. Elle porte aussi une efficacité performative : à mesure qu’elle se déroule, elle en vient, sous le regard arbitral du public, à redistribuer les positions de pouvoir, et, de là, à permettre l’invention de nouveaux dispositifs juridiques. Comme
l’écrit Christophe Prochasson, 
\begin{quote}
« la controverse a un rôle créateur et ordonnateur […]. Elle est une étape dans la construction d’un espace social et culturel, toujours instable\sn{
Christophe Prochasson, « Les espaces de la controverse. Roland Barthes contre Raymond Picard : un prélude à Mai 68 », Mil neuf cent. Revue d’histoire intellectuelle, n° 25, 2007/1.
 } ».     
\end{quote}
C’est bien ce qui est advenu : sous l’effet de la montée en puissance de l’école moniste (rationaliste et identitaire), la laïcité, sans remettre en cause tous les dispositifs recognitifs hérités des années 1960-1970, en les accentuant même sur certains terrains, s’est réorientée dans le sens d’une multiplication des surveillances, à destination en particulier de la religion musulmane.
\paragraph{Reconnaissance}
Le schéma de la reconnaissance se maintient donc. Lui sont associées d’abord des politiques distributives. En contradiction avec ses règles premières (loi Goblet de 1886), la République s’est engagée, depuis 1959, dans un financement massif des écoles privées, sans préjudice pour leur
« caractère propre », qu’elle n’a cessé de consolider depuis lors ; elle n’a pas hésité même à accorder des garanties d’emprunt et des subventions indirectes aux associations cultuelles. Le Conseil d’État a consacré, du reste, cette ouverture dans une série d’arrêts d’Assemblée du 19 juillet 2011, dès lors qu’un intérêt local est en jeu, et, selon la « théorie de l’objet mixte », que le financement ne vise pas à subventionner une activité exclusivement cultuelle. Des politiques symboliques sont venues également. Certes, les autorités gouvernementales n’ont cessé, depuis les années 1990, de rappeler la neutralité de l’État, de ses espaces et de ses personnels, comme on l’a vu récemment encore dans la loi d’avril 2016 relative à la déontologie et aux droits et obligations des fonctionnaires. Il reste que des « accommodements » ont été introduits au cours de la période récente. On admet, par exemple, que des autorisations d’absence puissent être accordées aux fonctionnaires lors des grandes fêtes de leur religion, que les cimetières puissent accueillir des carrés confessionnels (la circulaire de 2008 est même incitative en ce domaine, et non simplement permissive), que, dans les hôpitaux, les « besoins spirituels » des patients puissent être intégrés dans les dispositifs de soins (comme on le voit dans les divers programmes de développement des soins palliatifs depuis les années 2000).

Si l’État appuie les communautés confessionnelles, il attend d’elles aussi qu’elles le soutiennent. Il leur demande, par exemple, d’assumer des fonctions d’expertise (Emmanuel Macron évoquait récemment leur rôle indispensable dans le débat bioéthique), ou des fonctions de médiation (comme dans les opérations de dialogue interreligieux ou interconvictionnel qu’il encourage). Il entend également les doter d’une fonction de représentation : Lionel Jospin a en 2002, avec l’accord de Jacques Chirac, institué une instance de dialogue avec l’Église catholique, Nicolas Sarkozy en 2003 a intronisé le Conseil français du culte musulman, Emmanuel Macron a promis l’installation d’un Conseil national des cultes. Cette
évolution générale répond à la dynamique même de la démocratie libérale : son souci de faire droit à l’« égale dignité » de ses assujettis l’a amenée à reconnaître leurs droits civils et civiques et, bientôt, sociaux ; elle s’est engagée en outre à faire droit à leurs revendications sur le terrain des droits culturels, d’autant que les institutions internationales vont clairement dans ce sens depuis les années 1960. Mais elle est aussi un effet de la crise du politique : confronté à son impotence matérielle et symbolique, l’État a besoin désormais, dans une société de plus en plus mobile et incertaine, de l’apport des ressources cognitives et matérielles des églises15.

On touche là au second aspect de la « nouvelle laïcité ». Peu évoqué dans les années 1960-1970, l’exigence de cohésion – non seulement sociale mais également morale – s’est manifestée puissamment dans les années 1990-2000, comme en témoignent les rapports officiels (Fragonnard, Delevoye, Debré, Baroin, Stasi, Rossinot…) qui se sont succédé au cours de la période. Les attentats n’ont fait qu’amplifier ce désir de lien. L’idée n’est pas simplement d’amener les citoyens à respecter extérieurement les droits d’autrui et les principes constitutionnels qui les garantissent (ce que thématise le « devoir de civilité » propre au libéralisme traditionnel16), mais de les faire adhérer intimement aux « valeurs républicaines », et au « mode de vie » qu’on leur associe. La « politique transformative17 » à laquelle donne lieu ce programme intégrationniste comporte deux volets. Un volet éducatif, d’abord, qu’illustrent l’affichage des « chartes de la laïcité » dans les administrations d’État, dans la fonction publique hospitalière et, bien sûr, dans les écoles, mais aussi, depuis la fin des années 2000, la réactivation de l’enseignement moral et civique dans les programmes scolaires. Un volet coercitif, ensuite. Il faut faire arrêt ici sur la loi de 2004 sur le port des signes religieux ostensibles à l’école publique et sur la loi de 2010 sur la dissimulation du visage dans l’espace public. Elles introduisent deux modifications essentielles dans le droit de la laïcité, qu’on trouverait de même dans la loi « travail » de juillet 2016 ou dans les propositions de loi visant à proscrire le port des signes religieux à l’université18.

Le législateur a, d’une part, redéfini les espaces d’application de la règle de la neutralité. Dans le modèle de 1905, seuls les espaces d’État étaient concernés par l’abstention religieuse. Encore s’agissait-il de l’imposer aux fonctionnaires seulement, et encore dans l’exercice de leur mission. Les deux lois à l’instant citées déplacent les frontières. Celle de mars 2004 étend aux usagers du service public de l’éducation, les élèves
\mn{15.	Jürgen Habermas, « La religion dans la sphère publique », dans Entre naturalisme et religion : les défis de la démocratie, Paris, Gallimard, 2008, respectivement p. 152-169 et 170-211.
16.	John Rawls, Libéralisme politique, Paris, Puf, 1993.
17.	Selon l’expression de Stephen Macedo, Diversity and Distrust, Civic Education in Multicultural Democracy, Cambridge (Mass.), Harvard University Press, 2000.
18.	Sur ce point, Philippe Portier, « La politique du voile en France. Droits et valeurs dans la fabrique de la laïcité », Revue du droit des religions, n° 2, novembre 2016, p. 61 sq.
 
}
dans le primaire et le secondaire, une proscription que le Conseil d’État n’avait pas retenue dans son avis de novembre 1989 sur le port du voile à l’école publique, non plus que dans sa jurisprudence subséquente. Celle d’octobre 2010, qui ne fait pas référence en tant que telle à la laïcité, va plus loin en prohibant de facto certains vêtements religieux pour les personnes ordinaires dans l’espace même du commun – la voie publique, les commerces ou les salles de spectacle –, ce que la loi appelle, de manière inédite, l’« espace public ». Le législateur a de surcroît réévalué les motifs de restriction de l’autonomie. La notion d’ordre public qu’évoquent les textes de droit lorsqu’ils veulent limiter la liberté religieuse est ici centrale. Les autorités politiques (le législateur et le juge) l’avaient investie originellement d’une signification matérielle, en la renvoyant à des éléments objectifs (la sécurité, la tranquillité et la salubrité). Elles la dotent désormais d’une valence complémentaire, immatérielle celle-là, en la rapportant de plus en plus à un modèle substantiel de comportement, lié, selon l’expression du Conseil constitutionnel dans sa décision du 7 octobre 2010, aux « exigences minimales de la vie en société » ou, selon l’expression de la Cour européenne des droits de l’homme dans sa décision SAS c. France du 1er juillet 2014, aux « conditions du vivre ensemble ».

Dans la pratique, ces restrictions concernent l’islam, et non les autres confessions. Malgré quelques effets de halo sur les Églises chrétiennes, les autorités publiques, comme on l’a déjà signalé, n’hésitent pas, même, à faire référence, mais le plus souvent en l’appariant à l’ordre démo-libéral, aux « racines chrétiennes de la France », et à publiciser, dans l’espace qu’elles régissent, certains symboles de la tradition comme les crèches de la Nativité. Quelques analystes ont pu, de là, repérer l’existence, dans le droit français des cultes (mais la tendance est européenne), d’un « double standard » distinguant les droits pléniers des fidèles de la religion chrétienne et les droits plus limités des fidèles de la religion musulmane19.

Le parcours des trente dernières années nous aura donc confrontés à un changement décisif dans la pratique de la laïcité. S’est opéré un changement de la figure de la régulation publique du religieux. On peut, sur ce terrain, reprendre l’opposition présentée par Jean-Marc Ferry. La laïcité relevait, au début du xxe siècle, du registre de la norme : elle s’employait à permettre à chacun d’être traité, de manière juste, en tant que « personne libre et égale à toute autre20 ». Elle décrivait de la sorte un cadre procédural permettant à chacun de cultiver sa croyance ou son incroyance à son gré, sans que l’État puisse peser directement sur l’organisation et l’activité des \mn{19.	Alessandro Ferrari, « Religious freedom and the public-private divide: A broken promise for Europe ? », dans Silvio Ferrari et Sabrina Pastorelli (sous la direction de), Religion in Public Spaces. A European Perspectives, Ashgate, Farnham, 2012, p. 71-91.
20.	Jean-Marc Ferry, Valeurs et normes. La question de l’éthique, Bruxelles, Éditions de l’université de Bruxelles, 2002.}
institutions religieuses21. Cent-dix ans plus tard, on use autrement de la laïcité. Les gouvernements, de droite mais aussi de gauche (en tout cas, jusqu’à la présidence d’Emmanuel Macron, qui semble s’inscrire dans une ligne plus inclusive), en ont fait un instrument de reconfiguration de l’esprit public : elle s’agence désormais en un dispositif de diffusion de la valeur, en essayant de ramener les citoyens au bien que l’État définit.






































21.	Ce qui n’empêchait pas que l’école pût avoir un rôle recteur dans la formation de la conscience des élèves, dans le cadre cependant d’un régime qui laissait l’expression de la croyance religieuse en dehors de l’emprise de la sphère d’État.
