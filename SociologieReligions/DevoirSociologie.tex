%\chapter{Pourquoi lit-on \LS ?}



%-----------------------------------------------------------------------------------------------------------------------------

\section{Comment les religions répondent aux  enjeux écologiques}

 



\paragraph{L'enjeu climatique interroge la pertinence des religions universelles.} Max Weber définit la religion comme \textit{une espèce particulière de façon d'agir en communauté dont il s'agit d'étudier les conditions et les effets}. Mais que se passe-t-il quand les conditions changent ? Par exemple, en cas de changement climatique ? Comment les religions s'\textit{adaptent} pour proposer une façon d'\textit{agir en communauté } à la hauteur de l'enjeu, question essentielle pour les religions universelles comme le Christianisme ou l'Islam : 

\begin{singlequote}
        Qu’une religion soit raisonnable [donc universelle] dépend largement de ses
pouvoirs d’assimilation, de sa capacité à fournir dans ses propres termes une
interprétation intelligible des diverses situations et réalités que rencontrent
ses adhérents. \cite[ p. 175]{lindbeck_nature_2002}.
\end{singlequote}


\paragraph{Interroger les religions sur leurs effets collectifs}
La spécificité de l'enjeu écologique est en effet de questionner non seulement les pratiques individuelles mais aussi collectives, et ceci à un niveau mondial : si on arrête de produire en France pour moins polluer et déplacer le problème dans un autre pays, cela n'est d'aucune utilité par rapport à la crise climatique.
Notre sujet est donc, plus que la transformation \textit{individuelle}, celui de la transformation de l'action \textit{collective}.






\paragraph{Une limitation de la littérature au terreau chrétien} Vue l'ambition de ce travail, nous nous concentrerons sur le terreau Chrétien et plus spécifiquement à  la réception de l'encyclique \LS dans le contexte européen francophone. 

\paragraph{Qu'est ce que la \textit{réception} d'un texte ?} On peut dire qu'un texte est reçu  s'il transforme dans le temps les communautés chrétiennes, ici dans leur rapport \textit{à la maison commune }\cite{revol_reception_2017}. Une approche {sociologique}, qui s'intéresse aux pratiques, est donc pertinente pour juger de cette réception. 

\subsection{Introduction rapide à \LS}


\paragraph{\LS n'est pas le début de toute forme d'écologie chrétienne} En effet, une franche de militants chrétiens
 existent depuis plusieurs années et essayent de lier spiritualité et la cause climatique, telle la figure du \textit{Militant-Méditant} étudiée par \cite{monnot_figure_2021}. Ces mouvements touchaient peu les institutions ecclésiales.


\paragraph{\LS a été reçu de façon étonnante}
    L'encyclique Laudato si', publiée en juin 2015, marque un seuil dans la prise de conscience par l'Église catholique d'une nécessaire « conversion écologique » \cite{lang_generations_2020}.  La réception de \LS est étonnante : il s’est vendu  150 000  exemplaires de l'encyclique, figurant dans le top 20 des ventes françaises pendant l’été 2015. 
    
    
Accueilli positivement en dehors de l'Eglise (cf les contributions diverses de \cite{revol_reception_2017}), sa réception semble plus lente parmi les catholiques.
\begin{singlequote}
    Quand l'écologie fait son entrée dans les milieux catholiques, elle révèle à la fois les fragilités et les inerties des Églises, et le travail de renouveau qui s'opère dans les périphéries souvent inattendues. Monastères, lieux nouveaux, création d'associations sont autant de signes de la diversité de l'écologie chrétienne, dite intégrale.\cite[4ème de couverture]{lang_generations_2020}
\end{singlequote}

\paragraph{Le vocabulaire de \LS }
L'encyclique introduit un vocabulaire spécifique : conversion écologique et \textit{écologie intégrale}. Elle reprend en particulier 10 fois ce dernier terme: 

\begin{singlequote}
Il s’agit vraiment d’une nouveauté par rapport à ses prédécesseurs qui mettaient plus en avant la notion d’« écologie humaine ». [\ldots] Les thèmes qui entrent dans la structuration de cette écologie intégrale sont les suivants selon le Saint-Père : L’intime relation entre les pauvres et la fragilité de la planète ; la conviction que tout est lié dans le monde ; la critique du nouveau paradigme et des formes de pouvoir qui dérivent de la technologie ; l’invitation à chercher d’autres façons de comprendre l’économie et le progrès ; la valeur propre de chaque créature ; le sens humain de l’écologie ; la nécessité de débats sincères et honnêtes ; la grave responsabilité de la politique internationale et locale ; la culture du déchet et la proposition d’un nouveau style de vie. \cite{revol_lencyclique_2016}
\end{singlequote}

%-----------------------------------
\subsection{Typologie de la réception individuelle de \LS}  

Nous proposons de reprendre la typologie proposée par \cite[\textit{un an et demi après, que peut on dire de la reception de} \LS]{revol_reception_2017} :

\paragraph{Catégorie des Catholiques qui prennent conscience mais ne changent rien} Cette catégorie semble la plus commune. 
Ces catholiques s'informent (\textit{ils participent à la session organisée par leur paroisse (sic)}) mais pour l'instant, rien ne change vraiment.

\paragraph{Quand la lecture devient action}
Parmi eux, certains ont été particulièrement transformés dans leur coeur mais se sentent impuissants dans l'action. C'est la force des rites, de la méditations de textes, de la psalmodie, de la calligraphie ou la lecture attentive de textes \textit{classiques} d'actionner les \textit{émotions}, comme l'écrit Michel de Certeau et de devenir \textit{action} : 
\begin{singlequote}
    La lecture peut se faire « jardin des affects »  : Les « saveurs », « goût », « ferveurs » qui la ponctuent supposent une lecture faite de mouvements : émotions et motions s’y conjuguent ; l’\textit{affectus} implique et stimule un \textit{motus}. Aussi la \textit{lectio} est-elle considérée comme une \textit{actio}.   Michel de Certeau, La Fable mystique, II, op. cit., p. 208. 
 
\end{singlequote}
Ce sous-groupe est donc intéressant car potentiellement un vivier pour une conversion effective dans le temps. 

\paragraph{Surfer sur la vague verte} Un second type, souvent membres du clergé, analyse l'encyclique comme la réponse à une mode : le coeur et l'action ne sont pas transformés.
 
\paragraph{La catégorie des Légitimistes : de l'écoscepticisme à l'obeissance} Profondément attachés à l'autorité du pape, ces chrétiens prennent au sérieux l'encyclique : 
\begin{singlequote}
 
J'ai été témoin de changements dans des communautés religieuses ou de laïcs qui ont vraiment modifié leur enseignement.  [\ldots]

Leur état d'esprit serait à peu près : changeons des choses dans notre vie, si le pape nous le demande, c'est que ce sera sûrement quelque chose de bon pour nous et pour le salut du monde. \cite{revol_reception_2017}
\end{singlequote}
\paragraph{Opposition} La réception n'est bien sûr pas unanime parmi les catholiques, avec deux formes d'oppositions;
Tout d'abord les \textit{meurtris},  marqués à gauche, engagés dans le progrès humain des plus pauvres et qui ne comprennent pas l'extension de \LS à la Création, ni le refus du progrès. 

 De l'autre, \textit{Ceux qui sont forts dans leur foi},  de droite,   confiants dans le système économique libérale et plutôt climato-sceptiques : \LS leur apparaît comme une idéologie à l'opposé de leurs convictions. Leur état d'esprit est de s'expliquer avec le pape, lutter contre lui, ou attendre qu'il y ait un nouveau pape (sic).
 
\paragraph{Rejoindre les précurseurs} Ce sont les chrétiens écologistes de la première heure, se sentant reconnus par l'encyclique et heureux que l'Eglise bouge enfin. Je proposerai plus loin une typologie plus précise de ce groupe. 

\paragraph{Rejoindre les parvis} Ce sont ceux qui ont quitté l'Eglise à cause de leur conviction écologique et qui grâce à \LS, envisagent d'y revenir. \cite[p. 15]{lang_generations_2020} mentionne ainsi l'intervention de Cécile Duflot en janvier 2016 où elle témoigne de sa joie à lire l'encyclique. 

\paragraph{Rejoindre les non-chrétiens ?} \LS apparaît comme un texte pouvant réconcilier les écologistes non-chrétien avec le christianisme, surtout  ceux à la recherche d'une dimension spirituelle. 


% ----------------------------------------------------------------------
\subsection{Typologie au sein des précurseurs}


\paragraph{Les relais nécessaires de cette adaptation} Par rapport à l'ampleur de la transformation à opérer, l'encyclique doit s'appuyer sur des relais au sein des \textit{précurseurs}. 


\paragraph{Pourquoi se concentrer sur les précurseurs ? } Lors de l'écriture au livre collectif de Cécile Renouard(\cite{Renouard_entreprise_2015}), j'ai pu me rendre compte de l'importance des \textit{précurseurs} dans la conversion individuelle ou collective (par exemple les entreprises) : les \textit{précurseurs} mettent en mouvement. \cite{lang_generations_2020} ne dit pas autrement :
\begin{singlequote}
        Au fil de cet ouvrage, [\ldots] j'évoque de manière très subjective des visages et des rencontres qui m'ont aidé moi-même à cheminer. Car le processus vital de la conversion passe toujours, rappelle le pape François, par la rencontre bienveillante avec tous, et notamment avec ceux qui sont déjà plus avancés sur le chemin. \cite[p. 11]{lang_generations_2020}
\end{singlequote}
Notre sujet sera d'étudier qui sont ces précurseurs, comment ils le deviennent, puis comment ils peuvent devenir relais de l'encyclique, comment ils se pensent en tant que communauté, et comment ils interagissent  avec l'institution ecclésiale.


\paragraph{Quelques profils de \textit{précurseurs}} Nous ne retenons pas la proposition de typologie de \cite{carle_contre-revolutions_2017} basée sur une lecture politique ("les décroissants", "les écologistes chrétiens") et proposons une typologie basée sur l'articulation des deux termes "écologie" et "intégrale" (tout est lié) : 
\begin{itemize}
    \item ceux qui sont insérés dans l'Eglise, avec \textit{une sensibilité écologique forte}, souvent complété d'une fibre sociale : on peut citer les personnes travaillant autour du \textit{Campus de la Transition} qui assume ce statut de précurseur (« Comprendre pour agir, former pour transformer »), \textit{ CCFD-Terre Solidaire}, \textit{les Semaines Sociales},  certains ordres religieux (nous avons noté en particulier franciscains, jésuites, assomptionistes et beaucoup d'ordres monastiques) qui sont en pointe dans la réception de \LS    
    \item Ceux qui viennent de l'écologie et moins insérés dans l'Eglise catholique. Ils peuvent être déstabilisés par la dimension \textit{intégrale} de l'encyclique, comme Michel Maxime Egger, figure du \textit{militant-méditant} - \textit{cf infra}. (\cite{alexandre_grandjean_christophe_monnot_irene_becci_spiritualites_2018})
    \item ceux pour qui \textit{la dimension intégrale} ("tout est liée") est essentielle et qui ont pu venir à l'\textit{écologie intégrale} non par la crise climatique mais par des combats éthiques, comme le  \textit{Courant pour une écologie humaine}, fondé par Tugdual Derville et la revue \RLimite. Par la proximité de ces combats (Manif pour tous, ...), ces groupes se rattachent souvent aux \textit{légitimistes} mentionnés \textit{supra} même si leur sensibilité écologiste est souvent plus ancienne. 
\end{itemize}
 On a donc une tension au sein des précurseurs catholiques, entre une sensibilité écologie-fibre sociale ("catho de gauche") et une sensibilité éthique-\textit{écologie intégrale} ("catho conservateur"), qui se reconnaissent étonnement tous les deux dans la même encyclique.
 
\paragraph{L'exemple de Michel Maxime Egger } Dans un article de la \textit{Vie} du 19/05/2015, Michel Maxime Egger - MME, sociologue de formation et figure du \textit{Méditant-militant} écologique en Suisse (\cite{alexandre_grandjean_christophe_monnot_irene_becci_spiritualites_2018}) souligne qu'il est heureux de l'encyclique mais précise son inconfort sur la dimension \textit{intégrale} de l'encyclique et sur le manque de solutions concrètes proposées (le pape ne s'oppose pas au nucléaire par exemple): 
\begin{singlequote}
 [\ldots]
Par contre, je suis davantage gêné par ce besoin qu'ont les hiérarques catholiques, et le pape ne fait pas ici exception, de toujours revenir sur les questions de bioéthique ; la contraception, l'avortement, etc. Certes, je reçois l'argument selon lequel le respect de la nature implique le respect de l'homme et réciproquement. Reste que nous sommes dans des ordres de réalités assez différents. Je reste également sur ma faim quant aux solutions proposées, qui restent assez diffuses. Passons sur le fait qu'il ne parle pas du tout du nucléaire et qu'il est prudent, voire ambigu, par rapport aux OGM.[\ldots]
\end{singlequote}

La critique est intéressante car elle montre en creux l'originalité de l'encyclique par l'absence assumée de solutions concrètes : 
\begin{singlequote}
    [\ldots] la réflexion devrait identifier de possibles scénarios futurs, \textit{parce qu’il n’y a pas une seule issue}. Cela donnerait lieu à divers apports qui pourraient entrer dans un dialogue en vue de réponses intégrales. Sur beaucoup de questions concrètes, en principe, l’Église n’a pas de raison de proposer une parole définitive et elle comprend qu’elle doit écouter puis promouvoir le débat honnête entre scientifiques, en respectant la diversité d’opinions.  (\LS, 60-61)
\end{singlequote}
On est bien loin de la tonalité d'\textit{Humanae Vitae} et sa condamnation des \textit{solutions contraceptives précises}... C'est même une transformation du rapport à la vérité et au dialogue. 


\paragraph{Les \textit{précurseurs} semblent souvent en Groupe ou en réseau.}  Dans les exemples ci-dessus, on peut noter l'articulation entre \textit{individus} et \textit{Groupes} plus ou moins formalisés. L'action \textit{Collective} permet une impulsion beaucoup plus forte : 

\begin{singlequote}
    En certains diocèses, spécialement en France, des évêques ont procédé à la nomination de ministres ordonnés ou de ministres laïques dédiés aux questions environnementales, là où ces fonctions n'existaient pas déjà, tel qu'en Allemagne, en Suisse ou même au Québec. [\ldots] Des ordres religieux ont réorienté leurs investissements monétaires vers des ressources énergétiques renouvelables après avoir désinvesti des ressources fossiles. 
 
\cite{revol_reception_2017}
\end{singlequote}

Il sera intéressant d'étudier comment le rapport entre les individus et le collectif fonctionne et il est ressenti par les les individus aux différentes phases de la \textit{conversion}.
 

\paragraph{quelques conclusions intermédiaires}

 A l'issue de ce premier état des lieux, nous retenons la thèse de \cite{revol_reception_2017} d'un \textit{temps long} nécessaire à la réception de l'encyclique, temps d'\textit{inculturation} , d'\textit{assimilation} de la foi chrétienne à l'enjeu écologique. 
De multiples facteurs vont donc concourir à la réception du texte : A court terme, il faudra  que l'encyclique parle \textit{au coeur} de ses lecteurs, à la fois par sa forme (dialogue avec le monde scientifique, prière,...) et par son fond (réponse à une question, même diffuse déjà portée par les chrétiens).
Sur un temps plus long, l'encyclique a besoin de \textit{relais} qui entretiennent et développent le feu. 

Et c'est sur eux que nous proposons de poser notre question de départ. 


%-----------------------------------------------------------------------------------------------------------------------------
\section{Définition de la  question de départ}

\paragraph{Enjeu des \textit{relais} et de leur lien avec l'institution et la communauté qui les entourent} Ces relais sont des \textit{précurseurs} (au sens de la typologie présentée \textit{supra}) qui se reconnaissent explicitement dans \LS, dans leur discours et actions. 
Cette interaction évolue en fonction du temps, de la  prise de conscience à l'action :
\begin{itemize}
    \item \textsc{phase initiale}, de \textit{prise de conscience} préalable à \LS : \textit{les précurseurs sont-ils sans prédécesseurs ?}  qui donne l'impulsion de départ ? Sont-ils les héritiers ? et alors qui ont été leurs références et ceux qu'ils reconnaissent comme des précurseurs ? ou bien, ils n'ont pas de référence ni de précurseurs (ou bien au contraire de multiples précurseurs), et ils font oeuvre d'une pensée foncièrement originale. Dans ce cas, est-ce un processus individuel ou bien collectif voire institutionnel ? Rôle de la dimension religieuse ? 
    \item  \textsc{phase de cristallisation} : Est-ce que l'institution ou le collectif a confirmé l'impulsion ou au contraire a été une résistance  ? 
    \item \textsc{Lecture de \LS} : coïncide-t-elle avec la phase de cristallisation ? pourquoi ? comment la lecture de \LS a transformé / ajusté leur vision préalable ? En quoi le fait que ce soit un texte de nature religieuse est-elle importante ? Lecture collective ? 
   \item  \textsc{phase de déploiement et d'action}  : soutien de l'institution ? du réseau ou du cercle de \textit{précurseurs}? , ou au contraire départs, scission, en quoi \LS a pu être une ressource ? ...
\end{itemize}

On peut voir dans ces questions, deux types de problématiques qui nous intéressent particulièrement: 
d'abord, la présence de la dimension religieuse et de sa capacité à assimiler l'écologie dans les termes chrétiens et de donner une interprétation intelligible de la crise écologique aux chrétiens. Le but de \LS est de montrer de que manière l'engagement pour la sauvegarde de la maison commune doit naturellement surgir de la foi des chrétiens dans le Christ ressuscité, \cite{revol_reception_2017}. Mais est-il reçu ainsi ?
Ensuite, la problématique de l'articulation entre les différents niveaux de réflexion et d'action : individus, communauté/réseau et enfin institution ecclésiale. 




De ces questions, nous proposons comme question de départ : 
\begin{singlequote}
     Comment devient-t-on \textit{relais} de \LS ? L'enjeu de la réception à long terme
\end{singlequote}

Notre question de départ, souligne un paradoxe, les \textit{relais} de l'encyclique semblent moins les réseaux historiques de l'institution ecclésiale, c'est à dire les paroisses que des précurseurs de la cause écologique, aux formes diverses et rattachés de façon lâche à cette institution. 



 


   
   

%-----------------------------------------------------------------------------------------------------------------------------
\section{Un OVNI dans le paysage médiatique Français  : \textit{Limite, revue d'écologie intégrale.}}


Nous proposons de nous intéresser à un terrain particulier de \textit{relais} de \LS, la revue \RLimite. 

%-----------------------------------------------------------------------------------------------------------------------------
\subsection{Qu'est ce que la revue \RLimite ?}
\paragraph{\textit{Limite, revue d'écologie intégrale.}}
 \LS n'a pas inventé le terme d’\textit{écologie intégrale}, proposé par exemple en France par l'essayiste \textit{Falk Van Gaver}. 
Rappelons en quelques mots ce que ce terme recouvre : 
\begin{singlequote}
   [l'écologie intégrale]  articule dans une perspective unifiée les différents aspects de la vie humaine en rapport avec son environnement, considère que le rapport à Dieu, le rapport à soi, le rapport aux autres, et le rapport à la nature, sont des relations dont il faut prendre soin dans une mesure similaire afin de ne pas introduire de désordre dans le monde (le désordre écologique en est un). Le déséquilibre de ces rapports est à l’origine anthropologique de la crise écologique.\cite{revol_lencyclique_2016}
\end{singlequote}
On voit que dans une telle définition, le "désordre écologique" est l'un de ces désordres mais pas la seule porte d'entrée, comme le montre le mouvement des \textit{Veilleurs}.

\paragraph{A l'origine en France, le mouvement des "veilleurs".} les \textit{Veilleurs} (LV) sont de  jeunes intellectuels qui se sont engagés dans la Cité  au moment de la « Manif pour tous » (LMPT), en réaction à la loi sur le mariage homosexuel. Ils se sont d'abord manifestés par des regroupements, d'abord assis, bougies allumées, le soir, lisant des poèmes ou des philosophes. Lors d'une soirée, les participants ont ainsi récité des passages de Jean-Jacques Rousseau, Émile Zola, Mahatma Gandhi ou encore Max Weber et Antonio Gramsci (\cite{geva_non_2019}) :
  \begin{singlequote}
      Ils refusent d'enfermer leur cause dans une case, cherchant le bien commun au-delà des faux clivages et des querelles partisanes.  Ils n'ont ni drapeau ni slogan, ni chef ni porte-parole.(\cite[p. 8]{bes_nos_2014})
      \end{singlequote}

\paragraph{Un enjeu de reconnaissance culturelle} Cette volonté de dialogue souligne d'une part un regard positif sur le monde, qu'il est possible de transformer (par opposition à une position extra-mondaine de repli sur soi); d'autre part, elle peut être lu comme un enjeu de reconnaissance culturelle, selon la grille de lecture de Bourdieu.  
 
\cite{geva_non_2019} souligne le haut niveau culturel des responsables et activistes LMPT et LV et l'importance du catholicisme pour eux. Le contenu théologique des débats en jeu doit certes être étudié (\textit{Studies of conservatism should thus not only analyse the theological content of religiously-grounded conservative movements}). Mais elle souligne l'importance de l'\textit{engagement pour la reconnaissance} (\textit{struggle for distinction}) des bourgeois catholiques éduqués, reconnaissance à la fois vis-à-vis de l'élite bourgeoise financière mondialisée plus riche et aussi vis-à-vis de l'élite culturelle sécularisée accusée d'avoir perdu toute référence morale : 
\begin{singlequote}
    Highly educated Catholics therefore struggle for recognition in a field of bourgeois distinction where they cannot convert their moral knowledge into cultural capital in the secular field of distinction. They struggle for distinction against the putative empty morality of financial elites who are often the wealthier members of the bourgeoisie, and against the putative moral paucity of secular cultural elites whose secular knowledge hierarchy they perceive as a form of moral epistemics rivaling their own,especially on issues to do with human nature, family, sexuality, and gender relations. 
    \end{singlequote}
Cette opposition vis-à-vis des producteurs de savoir reprend une opposition plus ancienne repérée par Bourdieu entre \textit{Grandes Ecoles} et \textit{Université} : 
    \begin{singlequote}
Yet, they arguably also reflect increasing opposition to university-based production of knowledge in the experimental, medical, and social sciences.
\end{singlequote}
 

\paragraph{Un livre comme héritage du mouvement des Veilleurs}
A la suite de ce phénomène, un groupe de \textit{Veilleurs} fédéré par Gaultier Bès  décide d'écrire un livre 
      \textit{Nos limites : pour une écologie
intégrale} (\cite{bes_nos_2014}). 
      \begin{singlequote}
      Notre intuition est simple : l'être humain ne saurait s'épanouir, ni même subsister, sans reconnaître humblement sa finitude, c'est à dire sans accepter les limites de sa condition.  Aussi lui faut-il consentir à voir ses désirs circonscrits par la nature ou par la société. ( \cite[p. 9]{bes_nos_2014})
  \end{singlequote}

Notons que la référence à l'\textit{écologie intégrale} ne vient pas directement de l'urgence écologique mais de l'action contre le \textit{mariage pour tous}. 





\paragraph{Du livre \textit{Nos Limites} au journal \RLimite} La revue \RLimite   naît de cette filiation, avec une soirée de lancement (5 septembre 2015) et 27 numéros jusqu'à son extinction en 2022. C'est un objet plus complexe que le livre \textit{Nos Limites} dans le positionnement du champ politique droite / gauche (\cite{flipo_limite_2019}), avec Paul Piccarreta comme directeur de la publication.  
\begin{singlequote}
    Quand nous avons fondé \RLimite  avec une petite bande d'amis, nous pensions mettre au monde un fanzine d'étudiants qui ne durerait pas plus que quelques numéros. Les spécialistes ne s'y trompaient pas, qui nous promettaient la fin prochaine de la revue à chaque nouveau numéro. Mais comme nous avancions chaque trimestre avec plus de détermination [\ldots] les spécialistes ont fini par admettre que \RLimite  et l'écologie n'étaient peut être pas un feu de paille. (Edito du dernier numéro (27)). 
\end{singlequote}
 
\paragraph{Ecologie intégrale : une réception de \LS}

 Ce journal s'était donné pour \textit{manifeste} : 
\begin{singlequote}
La revue promeut une écologie intégrale qui se fonde sur le sens des équilibres et le respect des limites propre à chaque chose.
[\ldots]
Dans cette perspective, \RLimite  est orchestrée par différentes sensibilités qui coexistent dans un projet commun : encourager toutes les alternatives à la société de marché. Refusant l’« alternance sans alternative » du clivage droite/gauche, \RLimite  tend la main à tous ceux qui combattent le double empire de la technique sans âme et du marché sans loi. (\href{https://revuelimite.fr/notre-manifeste}{Manifeste de \RLimite})
\end{singlequote}
Au delà d'un manifeste autour de l'\textit{Ecologie intégrale}, nous retenons aussi la volonté explicite d'accueillir différentes sensibilités et non de s'enfermer dans un cercle étroit.

Il est intéressant de comparer ce manifeste avec le dernier édito du journal \RLimite, qui reconnaît une filiation directe avec \LS qui n'était pas présente dans le manifeste :  
\begin{singlequote}
    Nous nous sommes inscrits dans une tradition, celle de l'écologie politique et de la presse militante, parfois satirique. [\ldots] Nous avons cultivé des modèles : [\ldots]. Beaucoup de morts, et un pape bien vivant, François, qui en publiant \LS au moment où nous lancions notre premier numéro nous a confirmés dans notre intuition. Ainsi, nous voulons rester comme la "génération François", la génération d'un pape romain et altermondialiste, venu d'Amérique du Sud pour réveiller l'Europe, plus écolo que les écolo.
\end{singlequote}
Nous faisons l'hypothèse que  les fondateurs \RLimite ont été influencés par l'encyclique, d'une façon qu'il conviendra d'expliciter. 

\begin{comment}
    Eviter le gloubi boulga; se donner un texte normatif par rapport aux crises, départ, critiques
\end{comment}



\paragraph{Choix de la \RLimite comme terreau spécifique de \textit{relais} de \LS} Nous avons classé \RLimite dans la catégorie peu nombreuses des \textit{précurseurs} venus par les enjeux éthiques et non l'écologie au sens premier du terme. Le dernier édito de \RLimite nous permet de ranger la revue dans la catégorie des \textit{relais} puisqu'ils placent \textit{a posteriori} leur action dans l'élan de \LS. C'est aussi le \textit{relais} le plus cité par le quotidien \textit{Le Monde} (entre 2015 et aujourd'hui, 15 articles mentionnaient \RLimite, 6 le \textit{Campus de la Transition}, 5 les \textit{Semaines Sociales}, seul le CCFD récoltant nettement plus de mentions. A titre de comparaison, la théologienne Elena Lasida, chargée de la mission écologie de la conférence des Évêques de France n'a jamais été citée  par \textit{le Monde}). 




%-----------------------------------------------------------------------------------------------------------------------------
\subsection{\textit{Limite}, un \textit{relais} de \LS ? }
\paragraph{Un objet dépassant largement le nombre de ces lecteurs} La revue à son pic se vendait à 3000 exemplaires par numéro. En 2022, le nombre d'abonnements était limité à 1200, un lectorat faible donc. Pourtant, la revue a déclenché un intérêt médiatique certain. Le Monde, Libération, Basta ! (\cite{Basta_2015_Limite}), \textit{la Revue du crieur} (\cite{carle_contre-revolutions_2017}),  et même des revues comme \textit{Débats} ( \cite{de_boissieu_quest_2016}) ou \textit{Esprit} (\cite{Schlegel_2018_Limite}). Rien que pour \textit{Le Monde}, on dénombre 3 articles entièrement consacrés à la revue. La tonalité est initialement assez négative, si on en juge par l'indicateur de tonalité Europresse.   

\begin{singlequote}
    Différents articles de presse ont, par exemple, été très critiques du travail de la jeune équipe éditoriale de la revue \RLimite. Ainsi, dans Le Monde du 13 avril 2018, l'écologie intégrale défendue par cette revue à travers différents positionnements est assimilée à une opération d'enfumage pour recycler de vieilles thématiques conservatrices. L'originalité de l'approche du pape François, à laquelle \RLimite fait indirectement écho, n'est donc pas aisée à faire comprendre dans certains milieux de la société française. \cite[p. 9]{lang_generations_2020}
\end{singlequote}

 Cette attention a forcément rejailli sur la publication, à la fois comme un élément de motivation supplémentaire, mais aussi de prudence sur les thèmes abordés pour éviter d'être taxé d'extrême droite.  

\paragraph{Une évolution de \RLimite } La revue est difficilement classable, d'abord par ce que ces membres ont évolués, mais aussi, et cela une hypothèse à vérifier lors des entretiens, parce que la vision des membres de la rédaction a elle aussi pu évoluer : 
\begin{singlequote}
   La revue a évolué, souligne Piccarreta ; certains fondateurs
tels Jacques de Guillebon sont partis vers L’Incorrect ; Eugénie Bastié ne dirige plus la rubrique « politique ». Le souci est de porter la parole écologiste et sociale du côté des conservateurs, et
avec une certaine réussite, d’après lui. Soit. Mais comment l’objectiver, comment avoir les preuves ? Combien de convertis ? [\ldots] transfuge de quoi à quoi ?  \cite{flipo_limite_2019}
\end{singlequote}
Une lecture politique en est fait par \textit{le Monde} :
\begin{singlequote}
    Eugénie Bastié a quitté la revue en 2019, n’appréciant pas « le déséquilibre » , selon elle, entre conservateurs et « cathos de gauche » qui composent alors la revue. « Le clivage droite-gauche nous a finalement rattrapés » , juge Gaultier Bès. (Le monde 27/10/22 - La revue « écolo catho » « Limite » cesse de paraître)
\end{singlequote}
On peut néanmoins s'interroger si c'est bien le clivage droite-gauche ou bien une clarification des priorités (écolo vs éthique voire d'une conception non présente dans \LS de la \textit{limite} comme frontière) que pouvait cacher le concept un peu flou d' \textit{écologie intégrale}. 

\paragraph{\textit{Ecologie intégrale}, une aporie difficile à tenir face à la réalité} Au coeur de \RLimite, il y a le triptyque  : l’écologie, la critique de la technique et la défense de « la vie » sous toutes ses formes, avec une tonalité décroissante et  anti-capitaliste (\cite{flipo_limite_2019}). 
\cite{Schlegel_2018_Limite} souligne la pertinence de sa critique contre l'écologie politique  : 
\begin{singlequote}
     Répétons-le, l’aporie (plus que la contradiction) objectée par \RLimite à l’écologie politique reste pertinente : comment l’assentiment, sans réserve ni débat, aux interventions techniciennes sur le corps humain, que des lois récentes présupposent ou induisent, peut-il aller de pair avec le refus sans concession opposé aux organismes génétiquement modifiés, à tout ce qui pollue, empoisonne, détruit la nature extérieure, au nucléaire ? Il faudrait au moins s’en expliquer, mais chez les Verts français, seul José Bové a manifesté, à notre connaissance, ses réserves sur des propositions de loi bioéthiques en cours. \cite{Schlegel_2018_Limite}
 \end{singlequote}

Mais la revue est aussi critiquée pour ses positions abstraites, restant aux principes et non à la réalité de la société telle qu'elle se présente \cite{flipo_limite_2019}. Face à cette réalité  (par exemple le soutien des gilets jaunes, antilibéraux mais pas très écolo), la synthèse est moins facile dans la réalité que sur le papier. De même, la position anti-contraceptive de \RLimite : 

 \begin{singlequote}
 Le catastrophisme de \RLimite est-il vraiment toujours justifié ou éclairé ? En tout cas, le «féminisme intégral» de \RLimite, où, par exemple, des femmes jeunes et parfois sans enfants défendent non sans arrogance le refus de toute contraception non naturelle, mériterait une sérieuse discussion contradictoire.  \cite{Schlegel_2018_Limite}
\end{singlequote}

Une dernière hypothèse de cette évolution, c'est la transformation qu'aurait opérée \LS sur les rédacteurs de la revue. Ou en tout cas,  ils semblent accepter l'étiquette "chrétienne" de leur démarche. Ainsi, dans l'avant dernier numéro de la revue, on trouve une section de 15 pages de \textit{témoignages chrétiens} donnant par exemple une visibilité à Isabelle Priaulet et le livre issue de sa thèse à la Catho de Lyon : \textit{penser les fondements philosophiques de la conversion écologique}.


 

\paragraph{Importance du vocabulaire} Le vocabulaire de \RLimite est marquée par l'orientation de la revue (Ecologie intégrale, Conversion écologique). Ils semblent importants, mais on peut se demander si ce n'est qu'un problème de vocabulaire : 

\begin{singlequote}
 Paul Piccarretta concède qu’il y a parfois des désaccords au sein de la revue : « Les points d’achoppement tiennent beaucoup à la définition des mots. Le Comptoir n’aime pas le mot “conservateur”, Eugénie [Bastié] n’aime pas “révolution”. »  [\ldots] Gautier Bès tente une synthèse : « Je suis conservateur au sens où je pense qu’il faut préserver nos conditions d’existence, rien à voir avec conserver les privilèges, un certain ordre du monde qui favorise certains et en exclut d’autres. » « En tout cas, comme l’affirme Paul Piccaretta, on n’est certainement pas progressistes » 
    \cite{carle_contre-revolutions_2017}
\end{singlequote}




%-----------------------------------------------------------------------------------------------------------------------------

\section{Méthodologie proposée pour l'enquête}



\paragraph{Utiliser d'abord le matériau accessible : les journaux \RLimite} Nous avons noté une évolution de la revue. La première phase sera d'objectiver autant que possible cette évolution, en observant l'évolution des thèmes et du vocabulaire dans chaque numéro dans les quatre dimensions structurant selon nous la revue, les dimensions anti-libérale sur le plan économique, anti-technique sur le plan éthique, écologique et religieuse (en suivant les mentions explicites de \LS, sujet de notre travail). 


Ce travail pourra être complété par une analyse statistique à partir de  Twitter. Twitter est en effet l'agora moderne où communiquent la plupart des journalistes de \RLimite. Par une analyse des champs lexicaux et leur évolution dans le temps, des liens entre les différents membres à travers leurs citations (et éventuelle évolution de ces liens), on peut aborder de façon objective notre enquête. Je n'ai pu mener ce travail ici mais il est aujourd'hui assez standardisé (\cite{m_analyse_2020}).

\paragraph{Puis organiser les entretiens avec les acteurs de \RLimite.} La source principale d'information sera recueillie lors d'entretiens semi-directifs avec les membres de la revue soit un potentiel de 20 à 30 personnes. Idéalement, nous viserons un panachage des différents profils : personnes ayant connues la revue du début jusqu'à la fin, personnes ayant quitté \RLimite au bout d'un certain temps, personnalités médiatiquement visibles et rédacteurs plus épisodiques.

\paragraph{Difficultés escomptées} En dehors du problème classique de \textit{rentrer} dans la \textit{communauté} des acteurs de la revue, l'enquête nous paraît raisonnable par sa taille. 


\paragraph{Effectuer des entretiens exploratoires } Vu le profil des interviewés (avec une concentration importante d'agrégés) et leur expérience d'entretiens journalistiques serrés, nous risquons d'obtenir des entretiens très lisses. La préparation de l'entretien semi-dirigé sera essentielle. Une attitude d'\textit{Oie blanche} - \textit{je ne connais rien au sujet} - peut aussi limiter les jeux de pouvoir et faciliter une parole moins normée. Le fait que la revue soit arrêtée devrait néanmoins faciliter le travail car l'enjeu du le positionnement politique de la revue a disparu. 

  



%-----------------------------------------------------------------------------------------------------------------------------
\section{présentation de la grille d’entretien }
 
 
\paragraph{La préparation de l'entretien : de la difficulté d'interroger des intellectuels} L'entretien sociologique vise normalement à interroger les pratiques et non directement les \textit{normes et valeurs}. Or, ici, nous allons interroger des intellectuels et leurs \textit{croyances}. Une attention particulière sera donc portée sur les pratiques, changements (ex : \textit{j'ai quitté }\RLimite), pour rebondir en cas de mentions de ces pratiques ou actions et inviter la personne interrogée à approfondir  (ex : \textit{comment c'est passé votre départ ?}, \textit{Y-a-t-il eu un fait précédent votre départ ? }). 


\paragraph{Grille préparatoire } Nous présentons ici la grille en mettant en parallèle les normes et valeurs que nous cherchons à interroger et les pratiques que nous pouvons interroger lors de l'entretien. Pour nous aider, et créer un lien personnel avec l'interviewé, nous avons repris des questions posées par Gaultier Bès à Isabelle Priaulet (\RLimite, n° 26, 2022) - question en italique et nous pourrions ainsi adapter les questions à chaque interviewé.  
 
\begin{tabular}{p{.3\textwidth}p{.6\textwidth}}
\toprule
 Questions de normes et valeurs & Quelle pratique questionner \\
 \midrule
 Originalité du parcours & \textit{vous avez un parcours plutôt original ? Comment en êtes vous venue à l'engagement intellectuel pour l'écologie ?} \\
 Comment devient on \textsc{précurseur ? Relais ?}  & Pouvez vous me raconter l'histoire de \RLimite ? comment avez-vous connu les autres membres du journal ? \\
 
 
  Ecologie \textit{vs} Intégrale     &     
    La revue \RLimite  s'appelle \textit{revue d'écologie intégrale}. Que recouvre ce terme ?  \textit{pourquoi commencer par aborder la technique ? plutôt que la contemplation par exemple ? }
        
        \\
     Références et précurseurs.  &  \textit{Au fil des pages de \RLimite, vous dialoguez avec de grands noms, ... Qu'est ce qui rapproche toutes ces figures ?   }  \\     
         Impact de \LS        &   Comment \LS vous a touché ? Une phrase ou un passage particulier ? Cela a-t-il changé l'orientation de la revue ? Votre participation ?     \\
    Rapport à la Foi Chrétienne ? & Comment vous décririez vous d'un point de vue religieux ou spirituel ? Est ce un élément important dans votre engagement ?  \\

         Rapport à l'Eglise Institution & Allez vous à la messe ? Quels lieux d'Eglise fréquentez-vous ? \\
Gestion du Conflit & Comment était géré le conflit dans la rédaction ? Y a t'il eu des départs ? \\
 Evolution de \RLimite  & Quelle est votre histoire avec \RLimite ? pourquoi êtes vous restés ? partis ? La revue a-t-elle évoluée dans le temps ? De façon concrète, comment cela s'est il traduit ? Y-a-t-il une date charnière ?  \\

 Tester le modèle du Converti & Peut-on parler de conversion à l'écologie intégrale ? Est-ce que la conversion  vous donne de la joie ? \\
 
 Succès de \RLimite      & Pourquoi les médias ont été intéressés par  \RLimite ? \\
 




  
 \textsc{Devenir relais} : Quelles sont les convictions que vous souhaiter transmettre ?   & pourquoi êtes-vous devenu journaliste ? \LS a-t-il transformé votre engagement ? comment ? \\

 

 \bottomrule
\end{tabular}
 
 

%-----------------------------------------------------------------------------------------------------------------------------

\section{Eclairage du cours de sociologie des Religions}

\subsection{Sécularisation et modernité}

\RLimite propose une critique \textit{intégrale} et cohérente du paradigme de progrès.

\paragraph{Critique de la modernité et du paradigme techno-scientifique de progrès} Ce paradigme, c'est à dire l'ensemble de convictions, de questions ou de dogmes acceptés 
et partagés par une communauté à un moment donné, d'un progrès sans \textit{limite} est  au coeur de la modernité. \cite{universalis_modernite_nodate}. 

 \paragraph{une conviction écologique que le tout est supérieur à ces parties} Emile Durkheim analyse le rite comme ce qui permet de symboliser que le \textit{nous} est supérieur à la somme des parties qui le compose. Or, l'écologie présente une proximité du religieux,  sorte d'« affinités électives » \cite{hervieu-leger_religion_1993} : 
 [\ldots] en réaction à une anthropologie individualiste libérale [\ldots] les écologies radicales ont au coeur que le tout est supérieur à la somme de ses parties, ce qui implique que les désirs individuels s’effacent devant les intérêts de la communauté. \cite[35]{carle_contre-revolutions_2017}
 




\paragraph{Dimension contestataire de l'écologie intégrale} La religion a, dans son mode usuel, un rôle d'attestation de l'ordre social, c'est dire qu'elle soutient que l'individu peut se changer mais pas le monde. Mais, à travers la lecture de \LS par la revue \RLimite, on voit se dessiner un exemple de religion contestataire (nous avons vu en cours l'exemple de la théologie de la libération). La question du mal est ici claire et liée à la structure de la société :  \textit{les racines les plus profondes des dérèglements actuels [\ldots] sont en rapport avec l’orientation, les fins, le sens et le contexte social de la croissance technologique et économique} (LS 91). Cette définition du mal peut expliquer une réception plus difficile de \LS dans les milieux libéraux (les \textit{anywhere}, \cite{atlantico_anywhere_2022}), qui sont les principaux bénéficiaires de cet ordre actuel.
 



\subsection{\textit{Limite}, une figure d'autorité dans un monde sécularisé }

\paragraph{Effacement de la Paroisse, marque traditionnelle de l'institution Eclesiale} \LS est une encyclique qui s'adresse \textit{chaque personne qui habite cette planète}. Il ne s'agit plus de s'adresser aux évèques ou aux prêtres, et par eux, à leurs paroissiens mais directement à tout homme. Et de fait, comme l'a noté \cite{revol_reception_2017}, les prêtres et les paroisses ne sont pas les principaux lieux de réception de l'Encyclique.

\paragraph{Les théologiens remplacés par les communiquants ?} Nous avons vu en cours l'importance de la médiatisation aujourd'hui, avec la valorisation de la figure de Papes comme Jean-Paul II ou François. En revanche, les théologiens sont dévalorisés, au profit de ceux qui savent communiquer. Or, s'il faut bien reconnaître une qualité à la génération de \RLimite, c'est leur grande agilité à communiquer avec les codes d'aujourd'hui sans se faire enfermer dans des étiquettes : art consommé de l'humour et la \textit{provoc'} (cf \textit{le pense-bête du progressisme, \RLimite vous a concocté le lexique du parfait progressiste libéral. Désormais, vous pourrez postuler chez McKinsey.} N° 26). La \textit{provoc'} est aussi un discours contestataire puissant qui montre les incohérences du monde actuel. 

\paragraph{une figure pertinente pour la conversion écologique : la figure du converti} La figure du converti \cite{hervieu-leger_pelerin_2001} est en recherche de normativité : après l'expérience joyeuse et transformante de la conversion, il a peur de s'éloigner de l'orthopraxie (\textit{si je m'éloigne de la pratique, je risque de quitter cette joie du converti}). Par ailleurs, le converti a une volonté prosélyte. Sans même y ajouter la composante \textit{intégrale}; l'\textit{écologie} propose un système cohérent de vie,  qui fait facilement un matériau  pour les \href{https://www.youtube.com/watch?v=RhFp_K8oTgc}{caricaturistes} (le fameux \textit{khmer vert}) : c'est donc qu'il y a une vraie orthopraxie \textit{écologiste}, même si elle n'est pas formulée dans un décalogue ou un livre du \textit{Lévitique}. Par ailleurs, la volonté prosélyte est assurément présente dans cette génération de \RLimite qui n'hésite pas à s'exposer, y compris sur des sujets polémiques. La figure du \textit{converti} paraît donc pertinente.  


\subsection{l'Encyclique : forme moderne de communication ? }

\paragraph{Figure valorisée du pape} Le Pape François utilise divers moyens de communication, d'abord ses actes (la maison Ste Marthe, les appels téléphoniques à de simples chrétiens,...), marque de son \textit{style} théologique. Il a aussi une parole qui s'exprime à travers ses homélies, ses discussions dans l'avion de retour de voyage et bien sûr ses encycliques ou exhortations apostoliques.

\paragraph{Des encycliques d'une forme nouvelle} Les encycliques du Pape François sont d'une lecture accessible. \LS contient même des prières, y compris à la Sainte Famille. Comme l'a souligné \cite{revol_reception_2017}, elle peut générer de l'émotion de son lecteur, émotion qui est l'un des critères du vrai dans un monde sécularisé : "Si cela me fait du bien dans l'immédiat, c'est que c'est vrai". Autre prise en compte d'un monde sécularisé, refusant toute autorité auto-instituée, le pape se veut en dialogue avec les scientifiques et les autres religions tout en proposant une vision propre de l'enjeu écologique : "tout est lié". 

\paragraph{Renforcement du rôle normatif de l'encyclique} L'encyclique devient un élément majeur de ce qui fait \textit{le catholicisme} dans un monde sécularisé, court-circuitant d'une certaine manière l'institution ecclésiale. 



 

% -------------------------------------------------------------------------------------

\section{Hypothèse de réponses}


Nous proposons ici quelques hypothèses de réponses à notre question de départ,  
    \textit{ Comment devient-t-on \textit{relais} de \LS ? L'enjeu de la réception dans le temps long}, qui pourront être validés ou non lors des entretiens.


\paragraph{\LS a été normatif pour les rédacteurs de \RLimite} Les jeunes intellectuels qui ont formé \RLimite citent de nombreuses références, ne voulant pas tomber dans la caricature des cathos conservateurs qu'on peut trop facilement étiqueter. Notre hypothèse est que l'expérience des \textit{Veilleurs} a révélé un \textit{style}, à la fois contestataire  et en dialogue intellectuel avec la société. Finalement \LS a pu unifier leurs multiples références et ainsi faciliter le dialogue. Si cette hypothèse est juste, cela a eu pour conséquence : 
\begin{itemize}
    \item une référence explicite à \LS de plus en plus présente 
    \item un regard extérieur (le Monde,...) moins négatif. 
    \item des départs des membres de la revue qui, sans rejeter \LS, peuvent ne pas se sentir à l'aise avec l'une ou l'autre de ses dimensions, en particulier sur la question des frontières et de l'immigration (cf \textit{Il n’y a pas de frontières ni de barrières politiques ou sociales qui nous permettent de nous isoler, et pour cela même il n’y a pas non plus de place pour la globalisation de l’indifférence. LS 52}. 
    \item d'une certaine façon, si on donne poids au dernier édito de \RLimite, on peut lire l'arrêt de la revue comme l'aboutissement de la transformation d'une revue de \textit{veilleurs} en une revue de \textit{relais} de \LS.
\end{itemize}
 
 




%-----------------------------------------------------------------------------------------------------------------------------

\section{Conclusion}

A la fin de ce travail, le sujet s'est d'une certaine façon simplifié. Je suis parti de l'idée initiale de \RLimite comme d'un cercle restreint, une \textit{secte} au sens de Max Weber, avec des idées et pratiques très affirmées. Cela m'intéressait de voir comment ce groupe interagit avec l'Eglise-institution à travers la réception de \LS (avec l'idée que l'encyclique venant du Pape venait de l'institution). 

On voit dans la proposition d'hypothèse de réponses énoncée plus haut que ma vision a évolué, sensible d'abord à la typologie de réception de \LS et au besoin de \textit{relais-précurseurs} de cette encyclique.
Après avoir travaillé le sujet, je trouve plus pertinente la figure du \textit{convertis} pour \RL : cela peut expliquer certaines incompréhensions comme la discussion sur la pilule  (\cite{Schlegel_2018_Limite}). Le converti n'est pas dans une logique de pure rationalité mais d'orthopraxie qu'il convient de transmettre.

Au cours de ce travail, j'ai été étonné par la sélection qu'opèrent les media et l'intérêt qu'ils ont porté pour cette revue. 
J'ai été aussi touché par la figure proposée par \cite{revol_reception_2017} du chrétien ayant le coeur transformé à la lecture de \LS. C'est la force d'une religion et de ses symboles, rites, textes \textit{classiques} que de faire naître cette émotion qui nous fait sentir \textit{membre} d'une communauté. En guise de touche personnelle, m'est revenu cette phrase étrange d'un vieux chant carolingien entendue il y a trente ans en la cathédrale de Rouen : \textit{A furore Normannorum libera nos, Domine}. Cette phrase, conservée malgré la disparition de la peur viking, nous rappelle que nous ne sommes pas les premiers à subir des tribulations, et que la foi peut nous donner des armes pour lutter sans désespérance mais avec réalisme contre des enjeux qui peuvent légitiment nous effrayer.
 
 
%-----------------------------------------------------------------------------------------------------------------------------
  




 