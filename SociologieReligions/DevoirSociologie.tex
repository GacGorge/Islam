\chapter{La réception de Laudato Si : enquête sur la revue Limite}

Réduire Kung. mettre Charte

%-----------------------------------------------------------------------------------------------------------------------------

\section{quel est votre sujet? expliquez sa pertinence}
  avoir une question la plus courte possible. On peut avoir la question sans avoir tout de suite de sujet
La réception de Laudato
Si : enquête sur la revue
Limite

Réception de Laudato Si : un succès paradoxal Si on en juge par le nombre de publications et d’articles, Laudato Si est un des textes les plus reçus
du magistère récent. Pourtant, dans le milieu catholique, on peut s’interroger sur son effet réel sur les pratiques. 
La question est d’importance : la crise écologique est une crise qui touche l’humanité tout entière et dans son existence
même. Si les religions n’ont rien à dire, ou bien si elles disent mais que cela n’a
pas d’effet, on peut légitimement se poser la question de leur pertinence.


\paragraph{Question de départ la plus simple possible}
Questionner un comportement : "pourquoi cela ?". 
A qui sert Laudato si ? A l'institution ?  à des 
Cela pose le postulat que cela sert.

Pourquoi lire Laudato Si ? des gens ne le lise pas ?  Constat du succès. 

pourquoi ce succès et auprès de qui ? 
Présupposé le moins possible. 


\paragraph{idée } faible réception de Laudato Si mais un petit groupe très chaud. et se coupe.

%-----------------------------------------------------------------------------------------------------------------------------
\section{état des lieux de la littérature scientifique à ce sujet}
J'ai repéré quelques lectures : 
- La réception de l’encyclique Laudato si’ dans la militance écologiste. Les Éditions du Cerf
- Brugidou, Pierre ; Roblot, Jacqueline ; Pian, Christian ;
Institut supérieur de pastorale catéchétique (Paris) : La réception de l’encyclique Laudato si’ : pour une église renouvelée de l’intérieur et un
monde repensé pour demain, Diss.. – Dissertation/Thesis
- Flipo, Fabrice : Limite : une revue conservatrice, mais pas d’extrême- droite. https://go.exlibris.link/j4P0Qzjq. – Publisher : MAUSS
- Lang, Dominique : Générations ”Laudato si’”. Bayard

faire la recherche sur CAIRN. Regarder la bibliographie et trouver les auteurs récurrents, ou alors les sous champs de la question. Pour EuroPresse, il faugt aller sur pdf. 

être en alerte sur ce qui m'étonne. Par exemple, peu d'articles ou bien qu'il a fallu que je cherche via un autre thème.
\section{quelle est votre question de départ: elle doit correspondre aux cinq points vus en cours.}

Bonjour,

J'ai lu votre mail avec intérêt. Le sujet est pertinent et vous posez des questions stimulantes. Essayez de trouver une question de départ simple afin d'avoir un fil directeur. 
Bien à vous,

%-----------------------------------------------------------------------------------------------------------------------------

\section{quel est votre terrain?}
\paragraph{Un terreau d'étude limité : les tenants de l'écologie intégrale} Il semble néanmoins qu'une franche des catholiques aient été fortement touchée par l'appel de l'encyclique, un ou plutot des groupes dans la mouvance de l'écologie intégrale, et qui est apparu à travers la revue \textit{limite}.

\paragraph{Adhésion mais division} Ce journal, qui vient d'éditer son dernier numéro, s'était donné pour \textit{manifeste} : 
\begin{quote}
    Limite est une revue d’écologie fondée en 2015.

La revue promeut une écologie intégrale qui se fonde sur le sens des équilibres et le respect des limites propre à chaque chose.

L’écologie, parce qu’elle est une science des interactions et des conditions d’existence, ne saurait choisir l’humain contre la nature ou la nature contre l’humain.

Promouvoir l’écologie intégrale, c’est donc se soucier aussi bien des plus fragiles et des opprimés que s’opposer à tout ce que nos modes de vie peuvent avoir de dégradant et d’aliénant.

Refusant la toute-puissance de la technique et de l’argent, Limite souhaite œuvrer à la prise de conscience écologique en promouvant la sobriété, la relocalisation de nos existences, la convivialité et la fraternité.

Dans cette perspective, Limite est orchestrée par différentes sensibilités qui coexistent dans un projet commun : encourager toutes les alternatives à la société de marché. Refusant l’« alternance sans alternative » du clivage droite/gauche, Limite tend la main à tous ceux qui combattent le double empire de la technique sans âme et du marché sans loi.
\end{quote}

Or, la courte histoire de Limite, inspirée d'une vision transcendant les divisions propre à l'Eglise Catholique, a plutot vécu une histoire marquée par les divisions et les anathèmes (extrême droite, extrême gauche). 
L'idée serait d'interroger les personnes à l'origine de cette revue, de ce qui a changé à la lecture de \textit{laudato Si},  et de comprendre comment l'articulation \textit{Eglise / Secte} (au sens Weberien) peut jouer dans la réception d'un texte magistérielle.

%-----------------------------------------------------------------------------------------------------------------------------

\section{quelle serait la méthodologie pertinente pour répondre à votre question ?}


%-----------------------------------------------------------------------------------------------------------------------------

\section{présentation d’une grille d’entretien ou d’observation ou questionnaire.}


%-----------------------------------------------------------------------------------------------------------------------------

\section{quelles sont les notions, concepts, vus en cours qui éclairent le sujet ?}


%-----------------------------------------------------------------------------------------------------------------------------

\section{proposez des hypothèses (réponses à la question de départ)\sn{le document final doit garder la trace de ces hypothèses. }}


%-----------------------------------------------------------------------------------------------------------------------------

\section{une partie réflexivité peut constituer la conclusion: quelles étaient mes prénotions sur ce sujet au début ? Qu’ai-je découvert ? Qu’est-ce qui m’a surpris ?}
\sn{C'est rare qu'on arrive à la fin à tous nos présupposés}
 mettre des prénotions dans la conclusion


%-----------------------------------------------------------------------------------------------------------------------------

\section{Méthodologie}
\paragraph{méthodo}
4-10 pages max.
Suivre la démarche. Si on passe d'un point à l'autre, rappeler. 
Oblige à requalifier.
commencer à remplir la grille.
dire les lectures qu'on a faites.
 
L’ensemble de cet écrit sera tapuscrit et comprendra entre 7 et 10 pages. On peut rajouter des annexes. Vos noms et prénoms
doivent être indiqués sur le document déposé
 
Les étudiants déposent sur la plateforme (Moodle) pour le 06 janvier 2023 à 20h au plus tard
leurs écrits. Un espace spécifique dans Devoirs sera créé quelques semaines avant afin de
pouvoir laisser les documents. Les écrits seront inspectés par le logiciel Urkund (anti-plagiat).
 



 
     

\paragraph{Réception de Laudato Si : un succès paradoxal} Si on en juge par le nombre de publications et d'articles, Laudato Si est un des textes les plus \textit{reçus} du magistère récent. Pourtant, dans le milieu catholique, on peut s'interroger sur son effet réel sur les pratiques.
La question est d'importance : la crise écologique est une crise qui touche l'humanité tout entière et dans son existence même. Si les religions n'ont rien à dire, ou bien si elles disent mais que cela n'a pas d'effet, on peut légitimement se poser la question de leur pertinence.





\paragraph{méthode} expliciter tout.
\subparagraph{Grille de questions} au moment de l'entretien, on est uniquement dans la relance. Permets de reprendre ses propres mots de la personne. 
Tester les comportements. ou depuis quand, le Groupe ? AUtour d'une même phrase, jouer : pourquoi, quand, comment ? Eviter une grosse question ? 
\paragraph{Mettre dans le terrain la population}
\paragraph{forme pas importante} Mais on doit répondre à toutes les questions.


\textit{la question de départ ne sera jamais parfaite} on peut avoir plusieurs questions. 

\paragraph{dire quand on a reçu la réponse}


\paragraph{Pour vérifier la pertinence des religions, } il faut qu'elles puisent avoir une parole pertinente (et donc transformante) par rapport à l'écologie. Encore faut il qu'il soit entendu.

\paragraph{Questions possible}
\begin{itemize}
    \item Est ce que Laudato Si a été reçu ? une génération laudato Si ?
    \item par les cathos ? non cathos ?
    \item que dit le fait que Limite arrête
    \item ou sont les cathos écolo ?
    \item pourquoi la réception est compliquée ? la cohérence avec la foi ? 
\end{itemize}



\section{A lire}

\paragraph{« À la messe, pas un mot sur l’écologie ! »}
\href{https://www.la-croix.com/Debats/A-messe-pas-mot-lecologie-2022-11-13-1201241840?utm_source=newsletter&utm_medium=email&utm_campaign=NEWSLETTER__CRX_REL_EDITO&utm_content=20221114}{La croix}
Eric de Kermel se désole de l’absence de prières en direction de la nature et du vivant dans la plupart des messes catholiques. Pour lui, c’est le résultat d’une vision anthropocentrique qui ne correspond ni au message biblique, ni aux encycliques du pape François.
\mn{Eric de Kermel, le 13/11/2022 à 08:10}
 
 
J’ai récemment participé à un rassemblement de catholiques engagées dans la société. Comme il se doit lors d’une telle rencontre, une messe nous a rassemblés, nombreux. Et de là est née ma perplexité à l’origine de cette chronique.

Que ce soit au moment des intentions de prière ou de l’homélie, pas un mot sur l’écologie. Nous avons prié pour les malades, les migrants, les familles, ceux qui nous ont quittés, les entrepreneurs… nous avons prié pour nous…

À lire aussi« Dans la transition écologique, les catholiques ont plus de pouvoir qu’ils l’imaginent »
Une prière anthropocentrique, oubliant l’invitation faite par le pape François à considérer autrement notre terre-mère-sœur, élargissant par là même l’idée de fraternité au reste du monde vivant. Comment oser un éloge du futur qui ne soit pas écologique nous rappellent sans cesse, en particulier, les plus jeunes ?
 
Ces jeunes qui, quelques jours auparavant avaient aspergé de sauce tomate « Les Tournesols » de Van Gogh à la National Gallery de Londres. Action ayant provoqué une réaction offusquée de moins jeunes trouvant très violent de maculer la vitre protectrice du célèbre tableau. Lors de la messe, je pensais à nos fameuses journées du patrimoine où nous célébrons chaque année l’extraordinaire beauté de nos châteaux, de nos églises, de nos palais, de nos musées… Là aussi, dans une approche très anthropocentrique puisque seul le patrimoine issu de la « main de l’homme » est mis à l’honneur.

Il y a, dans le moment que nous vivons, un décalage et une incompréhension grandissants, entre ceux qui crient au scandale au sujet d’une vitre salie dans un musée, et ceux qui, par ce geste, dénoncent les agressions toujours plus démesurées et permanentes qui sont subies en toute légalité par la nature, que ce soit sur terre, dans nos forêts, ou dans la mer. Où est la violence ?
 
Si nous croyons, et nous y croyons n’est-ce pas, à la force de la prière, à la force de ce moment qui rassemble les énergies de tous dans une intention commune qui nous dépasse et nous met également en mouvement, il est indispensable que ces moments soient l’occasion d’exprimer notre empathie pas seulement envers nos semblables. Nous sommes DE la nature, c’est une évidence biologique que conforte le récit de notre genèse biblique. Comme nous y invite le philosophe Baptiste Morizot, il est urgent d’entamer un dialogue diplomatique avec le vivant non-humain.


Quelle meilleure table que celle de nos eucharisties pour inviter à cette communion renouvelée avec tout ce qui vit. L’émerveillement est le premier pas vers la protection. Dotés des plus beaux textes, les chrétiens ont les ressources et les lieux pour faire leur part. Je ne fais pas seulement référence au récit de la Création ou à la prière de Saint François mais aussi à l’encyclique Laudato si qui invite à cette écologie « intégrale » tressant d’un même mouvement le rapport à soi, aux autres et à tout le monde vivant.
\paragraph{L’arbre de l’année}
le grand public est invité à élire « l’arbre de l’année » parmi ceux qui portent les couleurs de chacune des régions françaises. Annuellement, cette opération organisée par le magazine Terre sauvage, connaît un immense succès et atteste de la relation forte que nous entretenons avec les arbres. Ou plutôt, avec UN arbre. Car en effet, si nous y réfléchissons un peu, on a tous un arbre dans notre vie. Celui de la place du village, de la cour d’école, celui au pied duquel nous aimons nous rassembler en famille, qui a été le témoin d’un baptême, d’un mariage ou d’un départ.
 
Relation singulière entre un(e) représentant(e) du peuple des humains et un représentant de celui du végétal… Ces liens sont précieux et participent du tissage de la toile du vivant. Relation gratuite, accessible à tous, même au plus blessé des humains, car un arbre est aussi capable de recueillir nos prières et de les confier aux ailes du vent. On a tous un arbre dans notre vie, pas tous un Van Gogh…

\paragraph{A lire}
\cite{revol_reception_2017}

\cite{lang_generations_2020}

\cite{brugidou_reception_2020}

\cite{flipo_limite_2019}
https://esprit.presse.fr/article/jean-louis-schlegel/les-limites-de-limite-39837 Les limites de Limite
par

Jean-Louis Schlegel
\paragraph{Islam Indonésien}


\paragraph{Mettre en place la grille}
Pas d'entretien mais on peut imaginer l'entretien : quelle question je poserais : double grille : question que je souhaite poser, et les questions que j'aimerais poser.
permet d'accueillir la question. 
Questions les plus simples. Focaliser sur les comportements.  Qu'est ce que j'en fais. 



\paragraph{Il peut y avoir des terrains }
