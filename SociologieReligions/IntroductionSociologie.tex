\chapter{Introduction à la sociologie des Religions}

\mn{Corinne Valadic - ICP sociologie famille, religions EHESS  ISTR 2022-23 - thèmes de recherche  : question de l'identité confessionnels, institutions confessionnelles - prêtres africains}


\section{Intérêt pour la matière}

Abus; religion; identité du Tamil Nadu; 
Cyriaque (M2) : Rwanda.

\paragraph{Salut différé} .
\begin{quote}
    le salut n'est pas dans le futur; il est aujourd'hui. L. Ferry
\end{quote}
\paragraph{pertinence des religions par rapport aux questions actuelles} Ecologie.


\subsection{Syllabus}

Débuter par une présentation de la sociologie. Regard sociologique (on part de la pratique pour arriver à la théorie).
\paragraph{La diagonale du vide} Comment à Moulins on s'est construit une identité en dehors de toute religion.



Albert Piette : site (écouter en Audio). 



\chapter{Démarches sociologiques}

\section{But : comment cela tient ?}

même s'il y a des tensions, comment cela se fait que cela marche (ou ne marche pas) ? Qu'est ce qui fait société ? \textit{Pourquoi cela tient ensemble ?}

La sociologie est née après la Révolution Industrielle, du fait du changement des sociétés Européennes, rurales à des sociétés urbaines. Nouvelle manière de considérer les territoires, avec des identités importées qui se sont mélangées avec les identités urbaines. 

\paragraph{Décalage entre ce que l'on dit et ce que l'on fait} Ainsi Marx fait la distinction entre l'\textit{égalité de droit} et l\textit{égalité de faits}.  

\paragraph{le social parle à travers nous} Une partie de nous nous échappe, l'\textit{inconscient collectif}, la langue. Quelle est la part qui nous échappe ? Claude Levi Strauss pensait quasi 100\% alors que les sociologues actuels sont plus à considérer 40-50\%. Qu'est ce qui est de l'inné et ce qui est de l'acquis. Ce qui nous est donné par la société,  nos cellules,...

\paragraph{Sociologie d’accompagnement} Françoise Singly. accompagner / Soin. parce que la sociologie est un outil mature.




\section{Démarche sociologique}
\paragraph{Démarche : repérer ce qui est évident ; ce qui est naturel}. Ex : Euthanasie. Qu’est ce qui est naturel. On ne va pas dire ce qu’il faut faire. On repère d’où les personnes parlent . 

\paragraph{ne pas avoir peur du ridicule} Oser y aller. 
\subsection{de l'empathie à la prise de distance}
\paragraph{empathie} va et vient. L'empathie, c'est de ne pas être dans le jugement de l'autre, puis chercher si d'autres ont ces mêmes questions.

\paragraph{prise de distance} On ne doit pas avoir de recommandations.


\subsection{Choisir un sujet}

\paragraph{Socioanalyse} les collègues nous interrogent sur notre sujet et être au clair soit même pourquoi on choisit son sujet. Puis ensuite, on va éviter que ces biais nous empèchent pas de travailler.

\paragraph{questionner nos prénotions} On se lâche sur le sujet. Préjugé : c'est ce qui constitue nos relations. Nous avons la \textit{question de départ}
\begin{itemize}
\item forme interrogative
\item question simple, efficace
\item on ne doit pas répondre par oui ou non
\item évolutive
\item elle tente de soulever un paradoxe (pas forcément extraordinaire). Idée : questionner les comportements. 

\begin{quote}
On essaye de sauver la face \sn{Erwin Goefman, }
\end{quote}


\end{itemize}

On apprend plusieurs rôles : les voyages, les langues, nous permettent d'avoir différents rôles. Ce sont des facettes différentes. Ce qui va nous interesser ici, ce n'est pas les facettes, mais le rôle dans la société. Ce que les individus nous disent n'est pas représentative de ce qu'ils font : \textbf{le décalage}.


