\chapter{Introduction à la sociologie des Religions}

\mn{Corinne Valadic - ICP sociologie famille, religions EHESS  ISTR 2022-23 - thèmes de recherche  : question de l'identité confessionnels, institutions confessionnelles - prêtres africains}


\section{Intérêt pour la matière}

Abus; religion; identité du Tamil Nadu; 
Cyriaque (M2) : Rwanda.

\paragraph{Salut différé} .
\begin{quote}
    le salut n'est pas dans le futur; il est aujourd'hui. L. Ferry
\end{quote}
\paragraph{pertinence des religions par rapport aux questions actuelles} Ecologie.


\subsection{Syllabus}

Débuter par une présentation de la sociologie. Regard sociologique (on part de la pratique pour arriver à la théorie).
\paragraph{La diagonale du vide} Comment à Moulins on s'est construit une identité en dehors de toute religion.



Albert Piette : site (écouter en Audio). 



\chapter{Démarches sociologiques}

\section{But : comment cela tient ?}

même s'il y a des tensions, comment cela se fait que cela marche (ou ne marche pas) ? Qu'est ce qui fait société ? \textit{Pourquoi cela tient ensemble ?}

La sociologie est née après la Révolution Industrielle, du fait du changement des sociétés Européennes, rurales à des sociétés urbaines. Nouvelle manière de considérer les territoires, avec des identités importées qui se sont mélangées avec les identités urbaines. 

\paragraph{Décalage entre ce que l'on dit et ce que l'on fait} Ainsi Marx fait la distinction entre l'\textit{égalité de droit} et l\textit{égalité de faits}.  

\paragraph{le social parle à travers nous} Une partie de nous nous échappe, l'\textit{inconscient collectif}, la langue. Quelle est la part qui nous échappe ? Claude Levi Strauss pensait quasi 100\% alors que les sociologues actuels sont plus à considérer 40-50\%. Qu'est ce qui est de l'inné et ce qui est de l'acquis. Ce qui nous est donné par la société, nos 
