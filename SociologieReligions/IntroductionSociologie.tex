\chapter{Introduction à la sociologie des Religions}

\mn{Corinne Valasik - ICP sociologie famille, religions EHESS  ISTR 2022-23 - thèmes de recherche  : question de l'identité confessionnels, institutions confessionnelles - prêtres africains}


Comment la sociologie peut-elle aider à comprendre le croire contemporain ?    

Argument de la brochure :  L’objectif de cours est de découvrir la sociologie en tant que discipline et de saisir comment elle contribue à la compréhension de certaines pratiques et modalités du croire religieux. Puis de tenter de cerner de quelle manière ces nouvelles connaissances peuvent aider dans l’exercice quotidien.   

\paragraph{Compétences à acquérir à l’issue de l’enseignement }
\begin{itemize}
\item Acquérir une posture de réflexivité 
\item Savoir pratiquer certaines étapes de la démarche de recherche des sciences sociales 
\item Construction d’une posture qui tend vers l’objectivité 
\item Capacité à comprendre et à restituer certains concepts sociologiques 
\item Acquisition d’une approche différente de concevoir la société  
\item Interagir avec d’autres étudiants de manière ajustée en prenant en compte les différences de contexte 
\item Exploiter ses connaissances dans des situations pratiques  
\end{itemize}


\paragraph{Plan du cours}
1- A la découverte de la sociologie : démarche, méthodologie, concepts 

2- Comment les sociologues étudient le religieux : présentation de travaux majeurs 3- Quels sont les enjeux actuels du croire ?  


\paragraph{Pédagogie et méthodologie	}
 La pédagogie s’appuiera sur une interactivité forte entre les étudiants et l’enseignant durant le cours, permise par la réalisation de travail personnel des étudiants (lecture de textes, mise en pratique d’une étape de la démarche etc). 

Ceci devrait permettre une appropriation et une restitution plus aisée des connaissances.  

\subsection{Bibliographie }
Ouvrages à lire au cours de l’enseignement (5 au maximum)  
\begin{itemize}
\item 
CHAMPION Françoise et HERVIEU-LEGER Danièle De l’émotion en religion, Renouveaux  et traditions, Paris, Le Centurion, 1990 
\item ESQUERRE, Arnaud, La manipulation mentale. Sociologie des sectes en France. Paris, Fayard, 2009. 
\item LURIE Alison, Des amis imaginaires, Paris,  Rivages Poche / Bibliothèque étrangère,  2006 [1967] (roman) .net), Le Fait religieux. Une théorie de la religion ordinaire , Paris, 
\item PIETTE Albert (http://www.albertpiette Economica, 2003. 
\item WILLAIME JeanPaul et PORTIER Philippe, Armand Colin, 2021 
\end{itemize}

\subsection{Mode d’évaluation }
La Religion dans la France contemporaine 
Réalisation d’un dossier alliant connaissances sociologiques et expérience de terrain. 


\section{Intérêt pour la matière}

Abus; religion; identité du Tamil Nadu; 
Cyriaque (M2) : Rwanda.

\paragraph{Salut différé} .
\begin{quote}
    le salut n'est pas dans le futur; il est aujourd'hui. L. Ferry
\end{quote}
\paragraph{pertinence des religions par rapport aux questions actuelles} Ecologie.


\subsection{Syllabus}

Débuter par une présentation de la sociologie. Regard sociologique (on part de la pratique pour arriver à la théorie).
\paragraph{La diagonale du vide} Comment à Moulins on s'est construit une identité en dehors de toute religion.



Albert Piette : site (écouter en Audio). 



\chapter{Démarches sociologiques}

\section{But : comment cela tient ?}

même s'il y a des tensions, comment cela se fait que cela marche (ou ne marche pas) ? Qu'est ce qui fait société ? \textit{Pourquoi cela tient ensemble ?}

La sociologie est née après la Révolution Industrielle, du fait du changement des sociétés Européennes, rurales à des sociétés urbaines. Nouvelle manière de considérer les territoires, avec des identités importées qui se sont mélangées avec les identités urbaines. 

\paragraph{Décalage entre ce que l'on dit et ce que l'on fait} Ainsi Marx fait la distinction entre l'\textit{égalité de droit} et l\textit{égalité de faits}.  

\paragraph{le social parle à travers nous} Une partie de nous nous échappe, l'\textit{inconscient collectif}, la langue. Quelle est la part qui nous échappe ? Claude Levi Strauss pensait quasi 100\% alors que les sociologues actuels sont plus à considérer 40-50\%. Qu'est ce qui est de l'inné et ce qui est de l'acquis. Ce qui nous est donné par la société,  nos cellules,...

\paragraph{Sociologie d’accompagnement} Françoise Singly. accompagner / Soin. parce que la sociologie est un outil mature.




\section{Démarche sociologique}
\paragraph{Démarche : repérer ce qui est évident ; ce qui est naturel}. Ex : Euthanasie. Qu’est ce qui est naturel. On ne va pas dire ce qu’il faut faire. On repère d’où les personnes parlent . 

\paragraph{ne pas avoir peur du ridicule} Oser y aller. 
\subsection{de l'empathie à la prise de distance}
\paragraph{empathie} va et vient. L'empathie, c'est de ne pas être dans le jugement de l'autre, puis chercher si d'autres ont ces mêmes questions. Il ne s'agit pas de se mettre en aplomb. La démarche sociologique vise à une neutralité même si l'utilisation sociale de la sociologie peut être, elle, récuperer les résultats de la sociologie pour une visée.

\paragraph{prise de distance} On ne doit pas avoir de recommandations.


\subsection{Choisir un sujet}

\paragraph{Socioanalyse} les collègues nous interrogent sur notre sujet et être au clair soit même pourquoi on choisit son sujet. Puis ensuite, on va éviter que ces biais nous empèchent pas de travailler.

\paragraph{questionner nos prénotions} On se lâche sur le sujet. Préjugé : c'est ce qui constitue nos relations. Nous avons la \textit{question de départ}


\begin{Ex}
Je fais une recherche sur le Yoga et j'ai comme prénotion, le fait qu'il y a plus de femmes.
\end{Ex}
\begin{Def}[question de départ]
Questionner quelque chose qui parait évident.
\begin{itemize}
\item forme interrogative
\item question simple, efficace
\item on ne doit pas répondre par oui ou non
\item évolutive
\item elle tente de soulever un paradoxe (pas forcément extraordinaire). Idée : questionner les comportements. 
\end{Def}


\begin{quote}
On essaye de sauver la face \sn{Erwin Goefman, }
\end{quote}


\end{itemize}

On apprend plusieurs rôles : les voyages, les langues, nous permettent d'avoir différents rôles. Ce sont des facettes différentes. Ce qui va nous intéresser ici, ce n'est pas les facettes, mais le rôle dans la société. Ce que les individus nous disent n'est pas représentative de ce qu'ils font : \textbf{le décalage}.

\paragraph{Choisir un terrain}  (ex : fidei Donum, ou prêtre étudiant). Il faut que ce terrain soit accessible. Il ya des terrains plus ou moins ouverts. La question de départ se pose sur ce terrain spécifique. 

\begin{Ex}
Ex : Sujet Fidélité
Question encore ouverte
le Terrain : couples mariés, mais alors hétérosexuels; homosexuels, différentes religions ? si oui, alors protestant (pentecôtiste). enfants ? Dans mon cadre, je ne la mets pas dans la question. 
Avoir la question la plus simple possible, et du coup, le terrain peut être complexe. Ex de questions : "Pourquoi ne pas divorcer" (provocateur). "Pourquoi être fidèle ?". "A qui sert la fidélité ?", "Qu'est ce que la fidélité ? Corps ? ..."


\end{Ex}

\paragraph{Partir du plus petit terrain} Si on veut partir à l'aventure, il faut avoir le plus petit terrain. sinon on reste dans des généralités. On va bien sûr lire tout ce qui a été écrit dessus 
\begin{Ex}
Par exemple sur la fidélité, on va regarder depuis quand la fidélité existe, quand elle n'est plus un motif de divorce.
Aller sur un site internet pornographique / Texto : preuve recevable d'infidelité au UK, même sans passage à l'acte

Aller voir Feydeau.  
\end{Ex}

\paragraph{La question}
\begin{Ex}
Ex : sujet : les sites de rencontre des couples mariés.
Pourquoi on va sur ces sites ?

\end{Ex}

\paragraph{Appareillage d'enquete} quand on a la question, le sujet et le terrain.

Confidentiel, anonyme, jamais enregistrement sans le dire. Est ce que l'enquete est faisable ? C'est là qu'on restreint encore les choses. 

\begin{itemize}
\item pourquoi étudier à l'ISTR ? 
\item quand étudie t on à l'ISTR ? plus de temps ? ou rien à voir ?
\item comment étudie-t-on à l'ISTR ?
\end{itemize}
Trois questions différentes. L'obsession, c'est d'avoir des hypothèses et les tester ? 


\paragraph{Une approche épistémologique commune à bcp de sciences sociales } L'anthropologie et la sociologie, même approche. Et même la théologie (même si l'appareillage d'enquête sera différent) !

\begin{Synthesis}
Démarche sociologie vise à la neutralité : empathie et prise de distance. 
Pour comprendre : on choisit un sujet, simple dans la formulation
\end{Synthesis}

\mn{cours du 20/09/22}

\paragraph{Lire les recherches effectuées sur le sujet}

Lire ce qui existe, par exemple beaucoup de choses sur les intellectuels en France, peu sur les associations confessionnelles. Qu'est ce que cela signifie ?
\begin{itemize}
\item tabou ?
\item question neuve ? Ex : Intelligence Artificielle
\end{itemize}

\begin{Ex}
La France est réputée par la sociologie politique du \textit{conflit} (à la différence du consensus en Suède, négociation en Allemagne).
\end{Ex}


\paragraph{Etudier la faisabilité de l’enquête (méthodes, durée etc.)}
\subparagraph{Regarder le vocabulaire}

\begin{Ex}
Apprendre le vocabulaire pour étudier le nomadisme des chauffeurs routiers.
\end{Ex}

\subparagraph{travailler un exemple}
\begin{Ex}[conversion]
- Pourquoi se convertit-on ?
Question de départ, mais qui a besoin d'un terrain : 
\begin{itemize}
\item j'avais une religion et je change; 
\item je n'avais pas de religion. et je me convertis dans une religion
\item Il peut y avoir aussi une \textit{conversion de l'intérieur}, selon Hervieux Leger, \textit{je me convertis de l'intérieur}, d'une religion culture à la foi

\end{itemize}
Ici, pour moi, la conversion ne fait partie que d'une démarche religieuse. 

On pourrait prendre comme question de départ : \textit{quand se convertit on ? }, \textit{comment se convertit-on ?}
avec des populations différentes. C'est pour cela que la

Le terrain ? 

\begin{itemize}
\item je peux prendre des couples mixtes, 
\item la \textit{carrière du converti}, je peux regarder les convertis il y a 10 ans, 
\item la population des français vivant à l'étranger.
\end{itemize} 
La question de départ, c'est le fil directeur, les autres questions vont apparaitre mais risque de s'éparpiller.
\end{Ex}


\paragraph{Effectuer des entretiens exploratoires }

\begin{Def}[méthode boule de neige]
on fait un premier entretien et on demande des personnes que l'on peut contacter.
\end{Def}


Au moment des entretiens exploratoires, l'imagination est au pouvoir. 

Une colonne avec les questions : 
\begin{itemize}
\item contexte
\item quel âge pour la conversion ?
\item Role réseaux sociaux ?
\item comment la conversion ?
\item Intermédiaires ?
\iitem comment est on après la vonverions ?
\end{itemize}

On veut une réponse à ces questions mais pour accéder aux normes et valeurs, il faut passer par la pratique :
\begin{itemize}
\item Allez vous à la messe ?
\item comment vos parents ont réagi ? (les responsables de la transmission)
\item comment vos enfants ? amis ? conjoint ? ont réagi ?
\mn{la grille d'entretien permet une approche scientifique.  mettre à plat les comportements. }
\item avez vous changé de pratiques alimentaire ? \mn{où achetez vous ? voir les accommodements ?}
\item depuis quand êtes vous bouddhiste ? d'elle même, elle va parler de la révélation et va avoir envie de relater ? La personne devrait nous dire en quoi elle est bouddhiste ? Et on peut rebondir : ex : j'ai mis un petit autel. Rebondir : où avez vous acheter les objets de l'autel ?
\item votre conjoint est il croyant ?

\end{itemize}

\begin{Def}[L'oie blanche]
Approche naive, je ne sais pas et je pose la question.
\end{Def}

je vois un signe religieux (le voile) et je pose la question : \textit{qu'elle est la signification du voile ?}, \textit{comment le portez vous ? }, \textit{qui vous a appris à le porter ?}, \textit{où l'avez vous acheter ?}

\begin{Prop}
la question sur les enfants est souvent de la sphère privée : Ex : "je n'ai rien contre l'homosexualité mais si mon fils est homosexuel, comment je réponds"
Où votre enfant est scolarisé ? pourquoi vous avez choisi cet établissement ?

\end{Prop}

\begin{Exo}
Passer des questions qui nous tiennent à coeur aux pratiques à interroger dans l'entretien.
Le faire sur la question qu'on souhaite aborder.
\end{Exo}


\paragraph{Typologie}
\mn{27/9/22}

Faire une typologie des personnes. Anonymisation. 


\paragraph{Problématique} ou hypothèse coeur. Va permettre d'aller plus loin dasn le travail. 

\begin{Ex}
Si on veut voir les militants écologistes chrétiens, comment on les aborde ? on va voir les militants et on leur demande s'ils sont chrétiens ?

\end{Ex}




\section{Bibliographe}

\paragraph{Général sociologie}  
\begin{itemize}
\item   AMADIEU Jean-François, Le Poids des apparences. Beauté, amour et gloire, Paris, Odile Jacob, 2002. - Les Clés du destin. École, amour, carrière, Paris, Odile Jacob, 2006. - La Société du paraître-Les beaux, les jeunes ...et les autres, Paris, Odile Jacob, 2016 
\item BOLTANSKI Luc et Ève CHIAPELLO, Le nouvel esprit du capitalisme, Paris,  Gallimard, coll. « NRF essais », 1999. DUBET François, À quoi sert vraiment un sociologue ?, Armand Colin, coll. « Dites-  nous », 2011. - Le Déclin de l'institution, Le Seuil, coll. « L'Épreuve des faits », septembre 2002. 
\item Alain EHRENBERG, Le Culte de la performance, Paris, Calmann-Lévy, 1991. - La Fatigue d’être soi – dépression et société, Paris, Odile Jacob, 1998  (rééd. Poches Odile Jacob) 
\item ELIAS Norbert, La Civilisation des mœurs, Paris, Calmann-Lévy, 1973, puis Pocket, 2002 (traduction de Pierre Kamnitzer) - La Dynamique de l’Occident, Paris, Calmann-Lévy, 1975, puis Pocket, 2003 (traduction de Pierre Kamnitzer)
\item La Solitude des mourants, Paris, Christian Bourgois, 1987, puis Pocket, 2002 (traduction de Sybille Muller et Claire Nancy) 
\item GALLAND Olivier, "Optimisme personnel, déprime sociétale : le paradoxe de la jeunesse française", Regards croisés sur l’économie 2017/1 (n° 20), p. 35-44 - Sociologie de la jeunesse, Paris, Armand Colin, 2017. 
\item HALL Edward T., Langage silencieux, Paris, Seuil, 1984 (en) The Silent Language, 1959) - La Dimension cachée, Paris, Seuil, 1971 ((en) The Hidden Dimension, 1966) 
\item SINGLY de François http://www.singly.org/francois/ - avec Christophe GIRAUD et Olivier MARTIN, Nouveau manuel de sociologie, éditions Armand Colin, 2010 
\item TAGUIEFF Pierre-André,  Le Sens du progrès. Une approche historique et   philosophique, Paris, Flammarion, « Champs », 2004 ; 2006   


\end{itemize}

\paragraph{sociologie des religions}  
\begin{itemize}
\item  \href{https://irel.ephe.psl.eu/institut}{Institut d’étude des religions et de la laïcité  } Bibliothèque virtuelle très complète. De nombreux liens internet EUREL 
\item données sociologiques et juridiques sur la religion en Europe \href{http://eurel.u-strasbg.fr/}{lien}  

\item Observatoire international du religieux (CERI/GSRL) :  : https://obsreligion.cnrs.fr/  
\item BOURDIEU Pierre, « Genèse et structure du champ religieux », Revue Française de   sociologie, 1971a, vol. XII, no 2, pp. 295-334. 
\item HALBWACHS Maurice, Les Cadres sociaux de la mémoire, Paris, Félix Alcan, coll.  « Bibliothèque de philosophie contemporaine », 1925.  
\item   La Topographie légendaire des Évangiles en Terre Sainte; étude de mémoire collective, [1941]. Paris, PUF, 2008,  Quadrige. Grands textes. 
\item HERVIEU-LEGER Danièle et WILLAIME Jean-Paul, Sociologies et religion, Paris, PUF,  2001  
\item HERVIEU-LEGER Danièle, La religion pour mémoire, Paris, Cerf, 1993 
\item LURIE Alison, Des amis imaginaires, Paris,  Rivages Poche / Bibliothèque étrangère,  2006 [1967] (roman) 
\item PIETTE Albert (http://www.albertpiette.net), Le Fait religieux. Une théorie de la religion  ordinaire, Paris, Economica, 2003. TRIGANO Shmuel, Qu’est-ce que la religion ? La transcendance des sociologues, Paris,   Flammarion, 2001  BAUBEROT Jean, Les sept laïcités françaises. Le modèle français de laïcité n'existe pas,  Paris, Maison des Sciences de l'Homme, 2015 
\item CHANTIN Jean-Pierre et MOULINET Daniel (dir.), La Séparation de 1905. Les hommes  et les lieux, Paris, Éditions de l’Atelier, « Patrimoine », 2005. 

\item OZOUF Mona, Composition française. Retour sur une enfance bretonne, Paris,  Gallimard, 2009 (roman) 
\item PORTIER Philippe, L'État et les religions en France. Une sociologie historique de la  laïcité, Collection Histoire, Rennes, Presses universitaires de Rennes, 2016. 
\item POULAT Émile, Les Diocésaines. République française, Église catholique : Loi de  1905 et associations cultuelles, le dossier d'un litige et de sa solution (1903-2003), Paris, Documentation française, 2007.  


\end{itemize}
\paragraph{Catholicisme}
\begin{itemize}  
\item BÉRAUD Céline, Prêtres, diacres, laïcs. Révolution silencieuse dans le catholicisme  français, Paris, Puf, « Le Lien social », 2007. 
\item BERAUD Céline, Le métier de prêtre, Les Editions de l’Atelier, Paris, 2006. DONEGANI Jean-Marie, La Liberté de choisir. Pluralisme religieux et pluralisme politique  dans le catholicisme français contemporain, Paris, Presses de la Fondation nationale des sciences politiques, 1993  
\item HERVIEU-LEGER Danièle, Catholicisme, la fin d’un monde, Paris, Bayard, 2003 - Le temps des moines ; Clôture et hospitalité, Paris, PUF, 2017 LAMBERT Yves, Dieu change en Bretagne. La religion à Limerzel, de 1900 à nos jours,  Paris, Éd. du Cerf, 1985. 

\item LAUTMAN Françoise et MAITRE Jacques (sous la dir. de), Ni Eve ni Marie. Luttes et  incertitudes des héritières de la Bible,  Genève, Labor et Fides, 1998. 
\item PELLETIER Denis, les Catholiques en France de 1789 à nos jours, Paris, Albin Michel, 2019. - La crise catholique. Religion, société, politique en France (1965-1978), Paris, Payot, 2002. 
\item  Denis Pelletier, Jean-Louis Schlegel (Dir.), À la gauche du Christ. Les chrétiens de gauche en France de 1945 à nos jours, Paris, Éditions du Seuil, coll.  Histoire », 2012 - Les catholiques en France de 1789 à nos jours, Paris, Albin Michel, 2019,  PORTIER Philippe et Willaime Jean-Paul, La Religion dans la France contemporaine,   Paris, Armand Colin, 2021 POULAT Émile, Église contre bourgeoisie. Introduction au devenir du catholicisme actuel,  Tournai, Casterman, coll. « Religion et Sociétés », 1977, 291 p. 
\item SUAUD Charles et VIET-DEPAULE N., Prêtres et ouvriers. Une double fidélité mise à  l’épreuve (1944-1969), Paris, Khartala, 2004. 
\item Jean-Paul Willaime, « Les reconfigurations de la religion et de sa critique dans l’ultramodernité contemporaine » in Tackling blasphemy, insult and hatred in a democratic society/Blasphème, injure et haine : la réponse de la société démocratique, Strasbourg, Commission de Venise/Conseil de l’Europe, 2008, p. 313–321. 

\end{itemize}
\paragraph{Autres religions}
\begin{itemize}
  \item BONELI Laurent et CARRE Fabien, La fabrique de la radicalité. Une sociologie des jeunes  djihadistes français, Paris, Seuil, 2018. \item 
  CHAMPION Françoise et HERVIEU-LEGER Danièle De l’émotion en religion,  Renouveaux et traditions, Paris, Le Centurion, 1990 
  \item CHAMPION Françoise, COHEN Martine (sous la direction de ), Sectes et Démocratie,  Paris, Seuil, 1999 
\item  CHANTIN Jean-Pierre, Des sectes dans la France contemporaine?, Paris, Privat, 2004 
\item ESQUERRE, Arnaud, La manipulation mentale. Sociologie des sectes en France.  Paris, Fayard, 2009. FATH Sébastien, Dieu XXL, La révolution des megachurches, Paris, Autrement, 2008. - Du ghetto au réseau. Les protestants évangéliques en France, 1800- 2005, Genève, Labor et Fides, 2005. -  (dir.), Le Protestantisme évangélique. Un christianisme de conversion,  Paris, Turnhout, Brépols, 2004 - (dir., avec Cédric MAYRARGUE), Les nouveaux christianismes en Afrique Paris, revue AFRIQUE CONTEMPORAINE, n°252, 2015 KAUFMANN Jean-Claude, Burkini, autopsie d'un fait divers, Paris, Les liens qui libèrent,  2017. 
\item KHOSROKHAVAR Fahrad et Wieviorka Michel, Les Juifs, les Musulmans et la République,  Paris, Robert Laffont, 2017 
 \item LAMINE Anne-Sophie, La cohabitation des Dieux. Pluralité religieuse et laïcité, Paris,  PUF, 2004.  - Quand le religieux fait conflit. Désaccords, négociations ou arrangements, Rennes, PUR, 2014. 
\item ROY Olivier, Le Djihad et la Mort, Paris, Le Seuil, coll. « Débats », 2016.  
\end{itemize}



