\chapter{Modernité}

\section{Le paradigme de la modernité}
\begin{Def}[Paradigme]
ensemble de convictions, de questions ou de dogmes acceptés et partagés par une
communauté scientifique à un moment donné.
\end{Def}

\begin{Ex}
Voir article du Monde sur Bruno Latour. Il a beaucoup travaillé sur le Christianisme et sur Gaia.
Individualisme comme concept, ni bien ni mal.
Injonction à être heureux.
Et a travaillé sur nature.
\end{Ex}
\subsection{Un processus intellectuellement construit}

Un outil de compréhension de la réalité sociale





\section{L’homme au centre de la réflexion}
En opposition à la notion de tradition
Importance de la Raison
Croire au Progrès






\begin{Synthesis}
distancier de la tradition car la tradition n'est pas créative (par exemple de la monarchie de droit divin)
mais il faut définir les raisons de l'autonomie : par exemple, je possède la capacité par la raison d'être autonome.
\end{Synthesis}

\paragraph{Raison fondement de la modernité}Accumuler du savoir (modèle des encyclopédies)

C'est une croyance. Il faut savoir utiliser la raison, d'où l'éducation. 

\paragraph{Progrès fondement de la modernité}Une autre vision de la modernité : La raison plus le progrès \sn{ TAGUIEFF Pierre-André,  Le Sens du progrès. Une approche historique et   philosophique, Paris, Flammarion, « Champs », 2004 ; 2006   }
Le progrès est particulièrement démontré dans la mortalité infantile. Vue comme positif.
Ce n'est que très récemment qu'il y a une nouvelle vision de la \textit{technique} où l'on commence d'interroger les questions éthiques : si on produit, c'est que c'est bien (les gaz dans la première guerre mondiale, les camps de concentration, la bombe atomique dans la guerre mondiale).

\paragraph{par cette question du progrès, sortie de la modernité}


3 âges d'or pour 
\begin{itemize}
\item Age d'or avant : la nature
\item age d'or après : le Paradis
\item age d'or devant mais sur terre. Pourquoi attendre ?
\end{itemize}
\paragraph{ Age d'or derrière nous : dans l'antiquité}. Pour les penseurs, penser cet age d'or et trouver les \textit{traces} de l'âge d'or. C'est dans la \textsc{Nature} qu'on trouve cet âge d'or. \mn{Lire passionnant les textes de Lucrèce, \textit{de la Nature}. "Le climat n'est plus ce qu'il était"}
\paragraph{changement au moyen age}   : des mondes enchantés, où le divin est présent (si tempête, c'est Zeus). La \textbf{magie} est présente. la notion de \textit{destin} est importante, \textit{les dieux ont décidé pour nous}. On passe au Moyen-Age au monothéisme, avec des sociétés en partie un désenchantement du monde. L'age d'or ce n'est plus derrière mais devant, avec la mort et le paradis. \sn{on a d'ailleurs plus de représentations de l'Enfer et du purgatoire que du Paradis}. Même si la vie est difficile, on a une vue d'une vie meilleure. 
    Portée par les institutions religieuses et permet de conserver les institutions, à travers ce système de \textit{rétribution}. 
    \paragraph{Tendance au martyr} car la vie dans le paradis est tellement bien qu'on a envie d'y accéder. Figure très ancienne. Les figures religieuses vont mettre en avant ces figures, avec les saints et martyrs et à la fois, vont donner envie de rester sur terre : se recentrer sur la vie terrestre. Reenchanter le quotidien. \textit{hic \& nunc}
    
    \paragraph{Utopie} François VIllon,... Delumeau montre que le Paradis critique la société existante et remet en cause la société qui existe car elle donne à voir qu'elle peut être autrement.
    
    \paragraph{Peu à peu vont se juxtaposer raison, progrès et justice sociale à venir : la modernité} Pourquoi attendre pour faire le bonheur dès maintenant ? faire advenir la société la plus juste dès maintenant.
    
    Va être disqualifié la religion. Chaque génération va mieux vivre que la génération précédente grâce à la technique et le savoir : maladie, education, paix...
    
    
    \begin{Ex}
    L'âge d'or peut être différent selon les périodes et les endroits. 
    Par exemple, aujourd'hui, le sentiment que l'âge d'or serait les \textit{trente glorieuses}. Avec un conflit générationnel : \textit{vous en avez bien profité}
    \end{Ex}
    
    
\section{Conséquences politiques, sociales, économiques} 


\paragraph{Une autre manière de concevoir les relations entre les hommes, avec la nature, distinction entre
nature et culture.} Baudrillard. Séparation entre corps et esprit.
On va survaloriser la religion et on va mettre de côté, ce qui est du côté de la nature et ne peut être maitrisé par la raison : le corps, le sentiment, les émotions. \textit{Surtout en France, en Italie, en Allemagne}
\begin{Ex}
Aux US, on va demander ce qu'on aime, les gouts,... alors que en France, on va demander quel métier. 
\textit{Dressage des corps}, devenir maître de son corps, sport, corps à table, fourchette à trois dents. \mn{lire  ELIAS Norbert, La Civilisation des mœurs, Paris, Calmann-Lévy, 1973, puis Pocket, 2002 (traduction de Pierre Kamnitzer) - La Dynamique de l’Occident, Paris, Calmann-Lévy, 1975, puis Pocket, 2003 (traduction de Pierre Kamnitzer)  on passe de culture à civilisation} les arts de la table : Comment on s'est compliqué la vie en France à table, plein d'obligations. 

\end{Ex}
\subsection{Sphère Privée / Publique}
\paragraph{Sphère Privée / Publique } On va passer aussi à la pudeur. N. Elias. On va se protéger dans la sphère publique. Au moyen âge, on se baignait homme et femme ensemble. Nous n'étions pas dans une distance sociale.\textsc{ Aujourd'hui un double mouvement de retour de la pudeur et à la fois de l'exposition.}

\paragraph{Le lever du roi} Ce n'était pas impudique de ne pas être seul dans la sphère privée.

\paragraph{Le travail dans le domaine publique} On en sort, avec le télétravail d'un côté mais aussi les réseaux sociaux qui montrent l'intimité.

\paragraph{Du coup, il n'y a plus de sphère privée} avec la disparition de l'intimité. On se met en scène via les réseaux sociaux

\paragraph{ce qui ne change pas,c'est la distance sociale} Hall. La distance sociale, 
dans un aerogare, 
C'est une valeur car on ne s'en rend pas compte. 
\begin{itemize}
    \item 30 cm : distance de l'intime
    \item 50 cm : distance de l'amitié
    \item 2 m
     \item avec la Covid, l'Etat a imposé la distance et a cassé certains codes sociaux.
\end{itemize}

\paragraph{le langage silencieux}

\paragraph{Crise de la masculinité} Caton l'ancien :"les femmes veulent conduire les chars". Elle est structurante de la masculinité.

Processus de rationalisation qui conduit aux sociétés démocratiques.
\subsection{Développement de l’individualisme.}

Au sens neutre, ni bien ni mal.

\begin{Def}[individualisme]
Je me connais du mieux possible; ce qui me définit par héritage et ce que je veux devenir.

\end{Def}
Quelle vie avoir ? M'extraire de ma famille ? 

\begin{Synthesis}
injonction contradictoire : 
nous sommes enjoint d'être heureux.
Hypothèse societale : la société vous donne vos chances.
Si on y arrive pas, on a tout pour y arriver : coaching, divorce,... On a tout pour réussir. On a toutes les recettes pour être heureux.
Et pourtant contradictoire, car on ne peut enjoindre à être heureux.
\end{Synthesis}

Comme tout est à disposition, si on échoue, \textsc{C'est notre faute}.

\mn{Alain Errenberg : culte de la performance, la fatigue d'être soi.1991 Il s'intéresse au cas français, antidépresseur. Sentiment en France, si on ne réussit pas sa vie (salaire), on en en échec.

Daniel Cohen, le tabou français.}

Et on ne voit pas que lorsqu'on échoue, cela peut être parfois parce qu'on ne peut y arriver.


\section{Le paradigme de la sécularisation}

\mn{le 11/2/22}
\begin{Def}
[Sécularisation] : perte d’emprise progressive de la religion sur les institutions.

\end{Def}

\paragraph{Concept de Weber} mouvement de l'histoire


\begin{Def}[Mouvement de l'histoire]
pas voulu par une personne; on ne peut y échapper
\end{Def}
A la différence de la Laicisation est portée par des individus. 

\paragraph{Elle est liée intrinséquement à la modernité} La modernité prone l'autonomie de l'humain. Mais 

\paragraph{moins de pouvoir des institutions} cela ne veut pas dire que les gens sont moins croyants. Sauf système coercitif type Iran.
\subsection{Conséquences du processus de sécularisation}

\paragraph{ perte de pouvoir social des institutions religieuses} Moins de pouvoir sur le politique et le social. Le dimanche, on allait à la messe ou au bistrot mais cela structurait la vie sociale. \mn{la France est l'un des rares pays où l'on se dit Athée qu'agnostique. 10\% de la population se dit athée}


\paragraph{ pluralisation de l’offre religieuse} Elles se retrouvent en concurrence, confrontées les unes aux autres. La plupart des religions vont se concentrer sur les questions de début et de fin de vie. Et tout ce qui est autour de la charité. Les religions vont quitter l'économie, pas de pensée économique, technique ou l'école...
Elles vont se concentrer dans les domaines où elles sont considérées comme légitimes; \textit{qu'est ce qui est vivant et non vivant}.

\subparagraph{logique de concurrence }
\begin{Ex}
On va adapter les heures en fonction des fidèles; le samedi soir pour les jeunes. A Grenoble, après l'heure des heures de ski. Adaptation des institutions aux demandes des fidèles.
\end{Ex}

\paragraph{Déterioralisation} Groupe whatsapp de prières évangéliques


\subparagraph{Nomadisme religieux}


\subparagraph{Thématiques profanes} reprises par des religieux : développement personnel : \textit{comment réussir sa vie ?} Livre magique : les gens achètent le livre mais ne le lisent pas. Ils ont les recettes. \textit{une étape}

\paragraph{Reprise du matériau religieux} Par exemple, le chant orthodoxe.

\paragraph{On ne va pas mettre les sachants mais les meilleurs communiquants} dévalorisation des théologiens, sauf s'ils arrivent à être dans une grande vulgarisation. Le pape François, comme Jean-Paul 2, génère cette médiatisation.

\paragraph{Rapport de force} une grande médiatisation peut changer les rapports de force au sein d'une organisation.


\paragraph{bricolage religieux} ce n'est pas dévalorisation. Claude Levi Strauss a défini le vocabulaire de \textit{bricolage}. A côté des personnes qui suivent la \textit{norme}. et un certain nombre bricolent: je me pose des questions et à partir des questions, je vais chercher du côté des religions. je fais mon \textit{dispositif de sens} en allant chercher dans l'une ou l'autre de religions. Selon l'énergie que je mets, je peux faire un bricolage léger (avec une sourate / bouddhiste) ou de la virtuosité (avec de longues études). Je ne vais pas voir les institutions pour confirmer que je suis musulman / bouddhiste mais des groupes. 

\paragraph{comme il y a de moins en moins de contrôle} risque de dérives. 

\paragraph{individualisation} Le critère c'est l'émotion. \textit{Je sens que cela me fait du bien, donc je le fais. } On coupe du coup toute rationalité. Il faut que le \textit{texte me parle}, \textit{je ne m'interroge pas sur le contexte}.
\begin{Synthesis}
Il faut que je sois dans l'authenticité; il faut que cela soit juste pour moi; Dans cette demande d'individualisation, c'est moi qui sait ce qui est bon pour moi. 
\end{Synthesis}
\begin{Ex}
Des demandes d'exorcisme.
\end{Ex}
Sur un registre émotionnel, il n'y a pas de dialogue, car on ne peut pas opposer un argument d'ordre rationnel.

\subparagraph{registre de la nourriture} champ Halal allant jusqu'à l'eau. \textit{besoin d'un univers où tout est validé extérieurement}

\paragraph{la rationalisation} Nous sommes obligé de \textit{rendre compte de nos manières de faire}, de structurer nos discours. Important des communiquants qui vont travailler le discours. 



\subsection{Origine de la sécularisation}

\paragraph{Max Weber et Marcel Gauchet} \mn{Le désenchantement du monde} 


\paragraph{Une matrice religieuse de la sortie de la religion ?}
Selon Max Weber,  
\begin{Synthesis}
La sécularisation aurait comme origine la religion elle même. 
\end{Synthesis}
A travers deux monothéisme, le monothéisme antique\sn{Max Weber, monothéisme antique}. 

Avant, on est dans des \subparagraph{sociétés enchantées,} \textit{magifiées}, où la magnie est présente au quotidien. S'il y a une tempête, elle vient de la colère d'un Dieu. Et donc on va renouer avec les forces de la nature. Le divin est présent au quotidien. Devin, ...

\paragraph{Radicalité du Monothéisme juif} Emerge la trancendance : Dieu quitte le monde, il est inconnaissable, on ne peut le représenter. \textit{Dieu inconnaissable}. Dieu est loin de nous, et ce n'est pas parce qu'on prie, qu'il intervient. \mn{la Galuth, ? \textit{exil de Dieu}}
Première séparation entre le divin et le profane chez Weber. 

\paragraph{Développement des monastères} autre séparation, sur un même territoire, deux manières de vivre se cotoient. le village et le monastère, dans lequel on va retrouver les moines qui vivent de façon radicalement différente, avec un temps de vie purement religieux : tout est centré autour de la religion. 
\mn{Ethique du protestantisme 19/10/22}

\subsection{Ethique du protestantisme}
\paragraph{Calvinisme} Nous sommes élus à notre naissance mais nous n'avons aucun moyen de le savoir, quelque soit les actes que nous faisons. Les conséquences pour M. Weber : 
\begin{itemize}
    \item Dieu a fait le bon choix : nous n'avons pas à remettre ce choix. Si je suis un croyant calviniste, je dois accepter et ne pas chercher à savoir pourquoi.
    \item pour vivre le moins mal possible, ils vont faire comme s'il était élu. Trop dur de vivre en pensant qu'on n'est pas élu.
    \item cela génère une angoisse forte car grâce individuelle forte (même si on est élu, est ce que mes enfants le sont). Cela augmente encore notre relation à Dieu car la communauté ne peut nous rassurer. Relation encore plus individualiste. Dieu est à la fois très loin et à la fois partout
\end{itemize}


\paragraph{Beruf - Ernst Troelsch} Beruf, c'est le travail mais aussi la \textit{vocation}, et rendre \textit{gloire à Dieu}. Max Weber en conclut : 
\begin{quote}
    plus je travaille, plus je rends gloire à Dieu et les rassure (une indication qu'ils sont élus).
\end{quote}
\paragraph{Affinité élective entre protestantisme et capitalisme}  Ils ne vont pas dépenser les revenus de leur travail. très austère. On achètera le nécessaire et pas le superflu.

\begin{Def}[théodicée]
Justification de la bonté divine par la réfutation des arguments tirés de l'existence du mal.
Toute théodicée est avant tout une sociodicée

\end{Def}

\paragraph{Ainsi, le calvinisme aurait développé le capitalisme} Une réponse différente de Marx permettant de répondre à la création du capitalisme. Weber explique ensuite que le capitalisme sort ensuite du calvinisme.


\begin{Synthesis}
D'un monde magifié et magnifié, on passe avec le monachisme et le capitalisme, selon Weber à un monde d'où Dieu est sorti. Ce sont les sociétés elles-mêmes qui ont créé des récits hors religion.  
Peter Berger : \textit{le christianisme est le fossoyeur de la religion}. Marcel Gauchet : \textit{le christianisme est la religion de la sortie de la religion}. Ce sont les religions elles-mêmes qui se sont exculturées.
\end{Synthesis}


\subsection{Remise en cause de ce paradigme}
\paragraph{Quelle besoin d'enchantement } Marcel Gauchet : c'est le politique qui devait réenchanter le monde. Mais s'est fourvoyé. 

\paragraph{Retour de l'émotion} comme critère du vrai. C'est par ce que cela me fait du bien que c'est vrai. Immédiateté de la religion. Revalorisation du corps. \sn{Technique du Yoga pour rejoindre mon moi profond.}

\paragraph{flou privé / public} sphère privée de moins en moins privée (télétravail).

\paragraph{Sommes-nous encore en Modernité ?}




\paragraph{Tout est religion : \textit{theologico-politique}} Fr. Bayrou, Pierre Manent... s'inspirent de la sociologie pour dire que si la société va mal, c'est lié à la modernité et sa sécularisation. La politique ne peut se penser sans le religieux. Il faut donc revenir à la religion. 

\paragraph{Islam} pas de sécularisation historique en Islam. Mais cela ne veut pas dire que ce mouvement ne se fera pas.  Il ya toujours une volonté de certains croyants de ramener la religion au centre. \mn{\href{https://www.lesechos.fr/idees-debats/editos-analyses/la-jeunesse-des-pays-arabes-ne-reve-plus-de-printemps-1870023}{Sécularisation Islam}}








\section{La religion, mode de Croire, D. Hervieu-Léger}
\mn{\cite{caille_religion_2004}}


\paragraph{Impératif reductionniste} par essence, la sociologie est réduction de l'objet religieux.  Compétition. pour donner un sens total au monde.
La méthode socilogique par essence réduit le religieux. 
\begin{quote}
    mais je ne peux le faire, précisément, que parce que je tiens tout aussi fermement et en même temps le principe selon lequel le "point de vue" sociologique ne saurait prétendre "épuiser" l'objet dont il se saisit.
\end{quote}

\paragraph{l'enjeu de la définition} en définissant les religions comme historiques, la sociologie a implicitement comme \textit{paradigme } la sécularisation

\mn{Eric Maigret, sociologie des BD. La masculinité à travers les super héros}


\section{Bruno Latour}

Nous n'avons jamais été dans la modernité; hypervalorisation des progrès. 
\mn{\href{https://www.arte.tv/fr/videos/RC-022018/entretien-avec-bruno-latour/}{Entretiens avec Bruno Latour}}


\paragraph{Sociologue iconoclaste}

Le sociologue, anthropologue et philosophe Bruno Latour est mort dans la nuit du samedi 8 au dimanche 9 octobre à l’âge de 75 ans, à Paris, a appris Le Monde de sources familiales et des éditions La Découverte. C’est l’un des intellectuels français les plus importants de sa génération qui disparaît, après une longue lutte contre la maladie. « Le plus célèbre et le plus incompris des philosophes français », avait écrit le New York Times, le 25 octobre 2018. Célèbre et célébré à l’étranger, récipiendaire du prix Holberg (2013) et du prix de Kyoto (2021) pour l’ensemble de ses travaux, Bruno Latour fut, en et, un temps incompris en France, tant ses objets de recherche semblaient disparates, ce qui pouvait masquer une grande cohérence. Il faut dire qu’ il a  touché à presque tous les domaines du savoir : l’écologie, le droit, la modernité, la religion et, bien sûr, les sciences et les techniques, avec ses inaugurales et détonantes études sur la vie de laboratoire. D’autant que, à l’exception notable de Michel Serres, avec qui Bruno Latour conçut un livre d’entretiens, Eclaircissements (François Bourin, 1992), la philosophie en France s’est souvent tenue à l’écart de la pensée et de la pratique des sciences. \mn{\href{https://www.lemonde.fr/idees/article/2021/12/10/bruno-latour-l-ecologie-c-est-la-nouvelle-lutte-des-classes_6105547_3232.html}{Ecologie comme nouvelle lutte des classes }}

\section{La jeunesse des pays arabes ne rêve plus de printemps}



Laura-Maï Gaveriaux Correspondante à Dubaï

LAURA-MAI GAVERIAUX
La dernière édition de l'Arab Youth Survey dresse le profil d'une jeunesse arabe entrée de plain-pied dans la technologie, bien mieux éduquée et plus diplômée que ses aînés, mais qui reste parallèlement conservatrice dans ses choix politiques. Elle semble avoir définitivement tourné la page des « printemps arabes » .

page 11
Chaque année, l'agence de relations publiques Asda'a BCW, basée à Dubaï, publie une étude régionale très attendue sur la jeunesse arabe. L'Arab Youth Survey permet de saisir les tendances à la fois démographiques et politiques de la région, en les centrant sur les 18-24 ans, soit près de 200 millions de personnes.

Pour l'édition 2022, 3.400 d'entre elles ont été interrogées lors d'entretiens individuels, menés dans 50 villes à travers 17 pays, classés dans trois grands ensembles : le Golfe, le Levant et le Maghreb. Quel dénominateur commun entre l'ado des faubourgs déshérités du Kef, en Tunisie, où le soufisme irrigue la religiosité quotidienne, et la jeune fille née dans une famille wahabite de Médine, qui n'envisage pas de sortir sans son niqab ? La langue ne saurait être un critère : chaque pays, voire chaque province, a son dialecte, plus spontanément parlé que l'arabe scolaire. D'autant qu'une majorité des jeunes interrogés, 55 \%, très anglophiles, déclarent que l'arabe est moins important pour eux qu'il ne l'était pour leur parents. Sauf dans les pays du Golfe, où ils ne sont que 44 \%.

La stabilité plus importante que la démocratie

Les similitudes sont ailleurs, et elles ont de quoi surprendre pour un lecteur occidental. Ainsi, deux tiers des jeunes interrogés estime que la démocratie ne fonctionnera jamais au Moyen-Orient : ils sont 57 \% dans le Golfe à l'affirmer, 62 \% au Maghreb, et le chiffre grimpe à 72 \% au Levant, épicentre d'une guerre interminable, en Syrie, qui a fait plus de 500.000 morts.

Née dans le sillage des Printemps arabes de 2011, déclenchés en Tunisie par ceux qui sont finalement les grands frères des sondés d'aujourd'hui, cette guerre syrienne signe un échec. De même que le retour d'un autocrate à Tunis, le naufrage libyen, ou encore la révolution avortée au Liban... C'est sans surprise qu'ils déclarent à 82 \% estimer que la stabilité est plus importante que la démocratie, quand ils étaient 92 \% à vouloir le contraire en 2009. Cela ne les empêche pas de reconnaître qu'ils ont plus de droits que leurs aînés, grâce à cette décennie de soulèvements (63 \% le déclarent), mais ils semblent considérer qu'il est temps de fermer la parenthèse, et de ne garder que ce qui fonctionne.

Forte aspiration pour les libertés individuelles

S'ils s'accommodent d'un pouvoir autoritaire, ils conservent une forte aspiration pour les libertés individuelles, tirée de l'expérience démocratique. Ils ne sont plus disposés à laisser le gouvernement régenter leur existence, et l'estiment trop intrusif à 60 \%. Même attente envers la religion, marquant une vraie rupture générationnelle : 73 \% la jugent trop présente dans leur vie.

Cette jeunesse qui souhaite avoir sa place sur le marché du travail (49 \% s'inquiètent de ne pas y avoir accès dans leur pays d'origine), veut donc être plus libre de ses choix de vie. Et pour cela, ils attendent beaucoup de l'éducation (83 \% se disent préoccupés par le niveau de formation dans leur pays), censée leur permettre de créer leur entreprise (28 \% ont cette ambition, un chiffre qui a doublé depuis 2019). Un exemple est évocateur : sur une population de 34,5 millions de personnes, 90 \% de la jeunesse est alphabétisée (selon les chiffres officiels des autorités), alors que les plus de 65 ans ne l'étaient qu'à 60 \%, d'après les chiffres de l'Unesco en 2017.

Cette soif d'évolution les incite-t-elle à tirer un trait sur leurs traditions ? Pas forcément : pour 65 \% d'entre eux, il est plus important de conserver leur identité religieuse et culturelle que de s'adapter à la mondialisation. Un positionnement qui fait écho au message que les leaders de la région font passer depuis quelques temps déjà : les Arabes du Golfe veulent prendre une place plus importante dans le monde, mais en offrant un autre modèle de société, alternatif à celui promu par les démocraties libérales occidentales depuis deux siècles. Et le profil de cette jeunesse en trace les contours : hyperconnectée, individualiste, conservatrice et ambitieuse.

L'attribution de la Coupe du monde de football au Qatar en 2010 (elle se tient du 20 novembre au 18 décembre) signe le premier acte de ce repositionnement. L'émir al Thani à Doha, Mohammed ben Salmane à Riyad et Mohammed ben Zayed à Abu Dhabi : trois dirigeants qui sont arrivés trentenaires et imposent, à leurs sociétés jusqu'ici très fermées, une modernisation à marche forcée. Symbole de cette course au développement : il n'aura suffi à l'Arabie saoudite que de quatre ans pour créer un marché du tourisme qui pèse aujourd'hui 38 \% du PIB et qui n'atteignait même pas 5\% il y a dix ans.

Les économies du Golfe accueillent ces nouveaux marchés du loisir et du tourisme depuis peu, mais les usages technologiques qui y sont liées étaient déjà bien ancrés dans les moeurs et ont préparé le terrain. Ces jeunes sont ainsi de grands consommateurs de réseaux sociaux : en 2021, une étude Forbes avait montré qu'un utilisateur au Moyen-Orient avait en moyenne 8,4 comptes sur différentes plateformes numériques, et même 10,5 aux Emirats, soit le record mondial. Il n'est donc pas étonnant que 89 \% déclarent faire leurs courses en ligne plusieurs fois par mois, un chiffre qui a plus que doublé en cinq ans. Malgré cette ouverture très appuyée sur le reste du monde et ce vent de modernité qui souffle sur les sociétés arabes, un grand malentendu perdure entre les Européens, pour qui les violations des droits humains sont venus entacher la future compétition de foot, et les Arabes, affichant leur fierté de la voir organisée, pour la première fois, au Moyen-Orient. La Coupe du monde, mais aussi les grands prix de Formule 1 au Bahreïn, le Louvre Abu Dhabi, le festival international de cinéma à Jeddah, tout ceci se fait moins à l'attention des pays extérieurs qu'à celle des Arabes de cette grande région qui va du Caire à Oman, de Rabat à Ramallah.

Laura-Maï Gaveriaux