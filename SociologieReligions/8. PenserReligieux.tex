\chapter{Comment penser le religieux et la religion}

\mn{PLAN 5. ISTR. C. Valasik Les approches fonctionnaliste et substantiviste.}



Il y a deux approches possibles pour penser le religieux : 
\begin{itemize}
    \item L'approche fonctionnaliste. La religion joue une fonction. Elle peut être remplacée.
    \item l'approche  substantiviste.
\end{itemize}
 \section{Genese du Religieux}

 \subsection{Emile Durkheim (1858-1917) }
 
 Différenciation entre le profane et le sacré  Le savoir religieux est de nature symbolique. Il exprime des réalités sociales indicibles dans une langue particulière.  La religion fournit un sens du monde, elle structure le sens du non-sens.  
 \mn{Le livre de référence : \textit{les formes élémentaires de la vie religieuse}}

 Ce qui l'intéresse, ce n'est pas d'étudier la vie religieuse au XIX mais \textit{l'origine de la religion}. Pourquoi à un moment donné la religion est apparue ?  est ce que cela provient du social ou des individus ? Il va dire : du social dans lequel vont baigner les individus. Plus un élément est ancien, plus on va avoir du mal à penser vivre en dehors.

 \paragraph{la religion dans la société} Le mieux serait de trouver la forme religieuse la plus simple. Hypothèse :
 \begin{Prop}[complexification avec le temps]
 Du coup, on peut chercher les formes élémentaires (au sens historique)
 \end{Prop}

\begin{Ex}[arborigène australien]
    Totem; on peut l'approcher de temps en temps; le toucher; le décorer.
    Femme avec sang des règles : non.
\end{Ex}
 
 \paragraph{séparation entre sacré et profane} le sacré, c'est ce qui attire mais fait peur.

 \paragraph{matérialisation du sacré} des statuts (île des Pâques), des forêts, rivière, cimetière. 
 
 \paragraph{hierarchie } Il constate que dans chaque religion, il y a du sacré, et une hierarchie vis à vis du sacré. Et du coup, des médiateurs. Façon codifiée.

 \paragraph{la religion, la langue d'avant la religion} Pour les premiers groupes humains, besoin de se repérer des espaces, ce qui était dangereux et pas dangereux. La première séparation aurait été dans cette séparation, \textit{acceptable et non acceptable} : haut, bas; vide, plein. \textsc{Ordonner}. Parce qu'on a besoin d'interdits en temps que Groupe.
 Les langues apparaissent plus tard. Les religions sont toujours transnationales, communes à tous les croyants. Au dessus de cette catégorisation, des langues locales, qui façonnent notre façon d'être \textit{dans un lieu.}

 \paragraph{Religion, symbolique} Toute religion, même si elle se prétend égalitaire, va introduire une hiérarchie. Un besoin car nous sommes en relation mais nous avons besoin de catégories. 

 \paragraph{Un rite  religieux} Des individus regroupés dans une cérémonie. Ils vont vivre un moment qui sort de l'ordinaire. Les individus vont se sentir bien (\textit{nous}, moins préoccupé par leur \textit{je}). Ils oublient leur quotidien. Quand ils sont dans ce \textit{nous}, chaque \textit{je} est rechargé par le \textit{nous}. 

\begin{equation}
    \text{nous} > \sum    \text{je} 
\end{equation}
 
 \begin{Def}[Force collective du Groupe]
      Une force qui échappe à tout le monde, une \textit{force collective du Groupe}.
 \end{Def}
 Ils vont chercher une raison à cette force, et ils vont l'attribuer au \textit{sacré}.
 \begin{quote}
      Lorsque la société pratique la religion, la religion ne fait rien d'autre que s'auto-adorer.
 \end{quote}

 \begin{Ex}[symboles de la république]
     Insistance dans marianne,...
 \end{Ex}

 Le sacré est produit par le Groupe humain, ils n'ont pas conscience qu'ils le construisent, et se soumettent à cette religion.

 Du coup, une religion peut être laïc. La religion est la manière dont la société fait un \textit{nous}, collectif. 

\paragraph{Religion, un mode d'être au monde} qui nous permet d'exister. On a donné du sens à ce qui nous échappait; On peut y retourner pour se ressourcer.

Si le sacré est accessible à tout le monde, il met en danger le sacré : la société ne peut s'auto-adorer.


\begin{Ex}[mise à distance du mort]
On va avoir une distance avec le corps du mort.
Même les proches vont être mis à distance, le deuil. (car la personne endeuillée peut être triste, aggressive).  
Et reintégration après un certain temps. 
    
\end{Ex}


\begin{Ex}[mise à distance du mort avec le Covid]
Avec la disparition du corps, une vraie question. \mn{\textit{Sociologie de la mort}, Gaelle Fladandier, très beau libre. }    
\end{Ex}



 \subsection{Max Weber (1864-1920) }

 Le développement des villes, l’urbanisation et la division du travail engendrent une demande du sens.  Cette attente de sens de la part des laïcs des villes qui engendre une rationalisation, une systématisation et une moralisation. Inégale répartition des biens de salut : monopole ou autogestion

\paragraph{Question de la domination} comment dans une société démocratique, on accepte une domination \sn{il n'aimait pas la relation avec les professeurs}. Il s'intéresse à la domination religion.

\paragraph{En quoi le fait de croire change les comportements quotidiens} 

\paragraph{La raison} Pour calmer la colère du Dieu, on pouvait faire un sacrifice. On cherche aujourd'hui des causes rationnelles. Si c'est un monde magique, tant qu'on est dans un monde rural, même s'il est rationalisé, il reste la figure du sorciers, qui possèdent des dons irrationnels. On n'est pas dans un monde désenchanté, car on voit du non humain : de la nature, le travail de cette nature. 

\paragraph{Dans les villes} presque plus rien ne nous échappe. Plus de hasard. Si on ajoute la division du travail, cela crée une perte de sens du travail. Du coup, dans les villes, les demandes de sens sont beaucoup plus fortes que dans les campagnes.

Dans les campagnes, la religion n'est pas le seul lieu pouvoyeur de sens. Par contre dans les villes, où le manque de sens est criant, que les religions sont les seuls lieux porteurs de sens.
\begin{Prop}
    Des visions du monde propres aux citadins. 
\end{Prop}
Le champ religieux, champ de questions et les religions vont aller dans cet espace et proposer des réponses. 

\paragraph{Professionalisation du religieux} pour répondre à ces questions urbaines, professionalisation / formation pour répondre à ce besoin des urbains. Des spécialistes; et des professeurs de ces spécialistes.

\paragraph{des professionnels} ex : les rabbins, pour interpréter les textes. Les laics ont fini par accepter cette séparation, les spécialistes monopolisant les \textit{biens de salut}. 
\begin{Def}[les biens de salut]
    Ce qui pour M. Weber, nous permet d'obtenir le salut
\end{Def}



% -------------------------------------------------------
 \subsection{Monopole et dépossession }

 Dépossession et méconnaissance de la dépossession


\begin{quote}
    les Laics acceptent d'être totalement dominés par un groupe de professionnels pour tout ce qui concerne le sens de leur vie.  
\end{quote}
Les laics méconnaissent leur dépossession. 

\paragraph{Reprise de possession de ces espaces} quand les laics se rendent compte de la perte d'espace, ils peuvent essayer de la récupérer.
\begin{Ex}
    Laics se formant à la théologie

    
    On peut aussi voir le lien entre arrêt de pratique dominicale et l'impact sur la professionalisation des clercs.
\end{Ex}

\paragraph{Sociologie et pouvoir} Nous sommes souvent en méconnaissance des rapports de pouvoir qui nous traverse.
\begin{Ex}[culture du conflit en France]
    culture de la grève en France. 
\end{Ex}


% -----------------------------------------------
 \section{Le fonctionnement du champ religieux}


\subsection{ Rôles majeurs remplit par la religion}

\paragraph{Rôle attestataire de la religion via la théodicée}
\begin{Prop}[Rôle attestataire de la religion via la théodicée]
         Ajout de la sociologie de la communication pour appréhender la pluralité des discours religieux en fonction des récepteurs Le charisme se définit par le charisme   
\end{Prop}
Plus la religion est numériquement conséquente, comment elle arrive à s'adresser à des groupes spécifiques. 
\begin{Ex}
    comment une personne va se sentir musulmane alors que c'est sur un territoire différent
\end{Ex}
Alors que le politique va opérer dans un cadre local. Un \textit{nous} qui s'incrit dans une histoire, qui est \textit{unique} et \textit{varié}.

\paragraph{Une formation de professionnels qui font un \textit{nous}} Comme les religions vont former à peu près de même façon les professionnels, on va avoir une \textit{homogamie}, même habitus entre les professionnels et les laics. Même catégorie de pensée. Et du coup, chaque groupe va avoir l'impression que c'est ainsi que se vit la religion.

\paragraph{Quand la pluralité s'invite} quand on invite un prêtre africain, un imam arabe, on s'aperçoit de la pluralité de religion.
Du fait, du rôle essentiel social, du groupe social pour \textit{faire religion}; 
\begin{Prop}
    Toute théodicée serait une sociodicée
\end{Prop}

      Intéressant de voir comme les représentants des religions vont \textit{médiatiser } les textes parfois anciens pour le contexte social.

\paragraph{Religion attestataire} Attester l'ordre social, c'est dire que le monde ne peut pas changer, on peut éventuellement se changer. Très souvent, les religions peuvent critiquer les valeurs mais ont un rôle d'attestation sociale.

\begin{Ex}
    Il peut y avoir des religions contestataires comme la \textit{théologie de la libération}, ou bien l'Eglise catholique polonaise dans le régime communiste. 
\end{Ex}

La question du mal va se poser différemment si on est une religion contestataire ou attestataire.
\begin{Ex}
    La manière dont une religion explique le mal va expliquer son rôle vis à vis de la société. 
\end{Ex}

\paragraph{L'eglise en France, plus contestataire ?} Manif pour tous, vaccins, méfiance vis à vis de la société qui ne porte plus les valeurs chrétiennes.


\paragraph{ce qui resiste dans la religion}
      
      
      \subsection{ Les tensions majeures au sein du champ }

      \paragraph{entre l’Eglise et le prophète : processus de désacralisation/sacralisation }
      \paragraph{Entre le prophète et le sorcier : relation à l’économie}
      \paragraph{ Entre l’orthodoxie et l’hérésie : liens entre l’hérésie et le conflit politique.  }  Influence du conflit politique (ex : feminisme, changement climaatique, LGBT) dans les orthodoxies et hérésies (ex : théologie féministe, considérée par certains comme hétérodoxes par certains théologiens).
          
      \begin{Ex}
          Yoga, religieux hors religion ? Patrick Michel.
      \end{Ex}
      
      \section{Conclusion}
      
      Sortir de l’opposition par le croire D. Hervieu Léger même croire. 
      
      \begin{Def}[Le croire]
         « la religion et la politique sont considérées comme deux modalités d’un dispositif idéologique, pratique et symbolique par lequel est constituée, entretenue, développée et contrôlée la conscience (individuelle et collective) de l’appartenance à une lignée croyante particulière » in \textit{La Religion pour mémoire }
      \end{Def}   

Le croire religieux, c'est surtout une référence à une lignée croyante ancienne, alors que dans le politique, appartenance à un Groupe. Rajout d'une temporalité qui dépasse le temps présent et qui permet de renouer avec des origines qui nous dépasse, sacré première. \sn{\textit{La religion pour mémoire}. En opposition, Patrick Michel : essayer d'observer le religieux en dehors de la religion (concert de rock, foot)}

\paragraph{Prêtre, prophète et Roi} Le Roi est une fonction qui manque dans le tableau, fonction du Roi religieux. En sociologie, le Roi est la fonction politique, et une fonction non religieuses en tant que telle et la question de la légitimisation réciproque des religions et politique.
