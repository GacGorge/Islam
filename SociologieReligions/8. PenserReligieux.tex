\chapter{Comment penser le religieux et la religion}

\mn{PLAN 5. ISTR. C. Valasik Les approches fonctionnaliste et substantiviste.}


 \section{Genese du Religieux}

 \subsection{Emile Durkheim (1858-1917) }
 
 Différenciation entre le profane et le sacré  Le savoir religieux est de nature symbolique. Il exprime des réalités sociales indicibles dans une langue particulière.  La religion fournit un sens du monde, elle structure le sens du non-sens.  

 \subsection{Max Weber (1864-1920) }

 Le développement des villes, l’urbanisation et la division du travail engendrent une demande du sens.  Cette attente de sens de la part des laïcs des villes qui engendre une rationalisation, une systématisation et une moralisation. Inégale répartition des biens de salut : monopole ou autogestion

 \subsection{Monopole et dépossession }

 Dépossession et méconnaissance de la dépossession

 \section{Le fonctionnement du champ religieux}


\subsection{ Rôles majeurs remplit par la religion}
      A) Rôle attestataire de la religion via la théodicée Ajout de la sociologie de la communication pour appréhender la pluralité des discours religieux en fonction des récepteurs Le charisme se définit par le charisme   
      
      
      \subsection{ Les tensions majeures au sein du champ }
      B) Entre l’Eglise et le prophète : processus de désacralisation/sacralisation Entre le prophète et le sorcier : relation à l’économie Entre l’orthodoxie et l’hérésie : liens entre l’hérésie et le conflit politique.      
      
      
      \section{Conclusion}
      
      Sortir de l’opposition par le croire D. HervieuLéger même croire. 
      
      \begin{Def}[Le croire]
         « la religion et la politique sont considérées comme deux modalités d’un dispositif idéologique, pratique et symbolique par lequel est constituée, entretenue, développée et contrôlée la conscience (individuelle et collective) de l’appartenance à une lignée croyante particulière » in \textit{La Religion pour mémoire }
      \end{Def}   
