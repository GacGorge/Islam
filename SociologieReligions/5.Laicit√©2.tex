\chapter{Laicité}


 \begin{Synthesis}
     La laïcité française provient d'une histoire spécifique. C'est pour ça qu'elle n'est reproductible  dans d'autres pays.
     
 \end{Synthesis}
 

\paragraph{La laicité française se pense universelle} Même si elle se veut universelle, que cette prétendue universelle. bien qu'elle se pense universelle, elle ne peut pas totalement l'être puisqu'elle provient d'une histoire spécifique. Alors certains principes peuvent être.

\subsection{La laicisation}
\paragraph{Le texte d'origine : une logique conflictuelle} 
La culture politique est une culture politique du conflit en France,  de façon assez logique, on est en situation de conflit entre le religieux et politique.
 
\paragraph{ et les décret d'application}  Mais ce qu'il faut retenir, c'est que ce conflit n'est pas appliquées différemment.

\paragraph{le religieux piloté par le politique}
Église catholique qui est dominante en France depuis des siècles. 
Bien sous la domination de l'Église catholique et même du petit qui le fait que les valeurs à ce moment là sont vues comme peu
compatibles.




\begin{Ex}[mariage]
    c'est à  dire que désormais l'état civil sera géré par l'Etat et non plus par l'Eglise catholique. On ne sera plus catholique, on n'est pas d'abord français. Comme le mariage. Désormais, le mariage est d'abord un mariage à  la mairie. Après, vous pouvez pas vous marier religieusement. Auparavant, tout était fait pour préparer les naissances, tenir le registre paroissial.
\end{Ex}

\paragraph{l'importance du contexte comme le syllabus}
 on va assister à  ce conflit extrêmement fort entre l'Etat et l'Eglise catholique. Mais encore une fois, ce conflit aurait.
pu être différent si les acteurs avaient été différents. Peut être que si ça avait pas été ce pape qui condamne avec les syllabus \sn{condamne les quatre-
Vingts erreurs de notre temps,}  peut être que ça aurait été différent. Peut être qu'en France, du côté des Républicains, ça a été Jules Ferry, puis comme ça aurait été différent.
C'est pour prendre les individus en face en France. 

\subsection{loi de 1905}
\paragraph{une tension qui diminue partiellement avec la loi de 1905}
1905 : la loi de laïcité. Cette particularité, elle définit jamais la laïcité.Cela permet à  chacun de s'y retrouver. 


\paragraph{trois principes de laicité} \mn{ le lien de Légifrance pour avoir accès à  la loi de 1905.}
Jean Baubérot 
montre que  trois principes de laïcité sont interprétés différement, chacun étant convaincu d'avoir la bonne définition de la bonne manière. 
trois principes qui sont clairs, qui sont pour tout le monde. 

\begin{itemize}
    \item  séparation entre l'Eglise et l'Etat, entre les cultes et l'Etat. Désormais, les cultes font partie du domaine privé, ne font plus partie du domaine public et les cultes s'auto gèrent.
\item neutralité :  l'Etat finance aucun culte. Concernant les religions, il n'y a plus de religion d'Etat, il n'y a pas de religion qui soit favorisée par rapport à  une troisième.
\item l'Etat est   garant de la possibilité pour les croyants d'exercer leur culte. l'Etat doit permettre.
à chaque religion d'exister sur le territoire français.
C'est la liberté de culte, c'est la liberté de croire ou de ne pas croire.\mn{cette troisième vision est souvent oublié} 
\end{itemize} 
\paragraph{Premiere guerre mondiale}
 
Puisqu'on va se rendre compte que finalement les catholiques vont soutenir indirectement le régime républicain en place et non pas. comme le premier républicain, s'allier au démon et aux autres puissances pour changer les choses le jour de la Terre, ça vient toujours de l'école et  passer en France qui est considéré comme une menace à  former les futurs citoyens et  à  classe.


\subsection{Cette vision de la laicité a un impact}

\paragraph{Islam}

 Après la loi de 1905 et hormis la Grande Mosquée de Paris, toutes les mosquées vont être financé sur fonds propres et doivent être financé sur fonds propres, ce qui crée une inégalité.

 \paragraph{Alsace Lorraine}
Parce qu'en fait vous avez raison de dire qu'au moment ou la loi va être mise en place, l'Alsace-Moselle n'appartient qu'à  de qu'elle reste au régime précédent concordataire. Et lorsqu'elle redevient de ces régions.
Elles vont rester.
Sous le régime de Vichy. Mais c'est pour ça que.
Cela et ce qui est intéressant pour les gens le actuellement. C'est vrai que de toute façon, le régime concordataire est ce qu'il faudrait un petit peu avant le début 1905 . Après, il y a des exemples comme les fabriques.

\paragraph{pratiquement}
 l'Etat est devenu propriétaire de ces édifices religieux.
Il les prêtent gratuitement essentiellement au culte catholique.
Mais en partie l'obligation d'entretien et cet entretien peut être porté soit au niveau du patrimoine comme à  Notre-Dame de Paris qui est considéré comme monument national avec une prise en charge par les caisses de l'Etat.
D'autres être pris en certains départements et par les régions, d'autres par les municipalités. Tout va dépendre de la valeur totale de cet édifice. 


\paragraph{Lieux interculturels ? }
En faire des lieux interculturels en tant que tel et de permettre notamment à  d'autres confessions de partager cet espace, comme l'atmosphère dans d'autres pays, notamment en Espagne ou en Allemagne.
Et notamment les ouvrir aux musulmans. Il y a quand même une petite population choquée par l'appel à la prière de musulmans dans son domicile,  

\paragraph{Adapter les lieux}
Les pentecôtistes qui ont des lieux très chauds, très bien chauffés, ou les gens peuvent rester.
Les gens arrive à en avance et rester un peu après. Pour les choses là , les gens restent de bonne humeur et après c'est l'enfer.

Par exemple, en Allemagne, il y a des lieux interculturels. 


\paragraph{pas de culture du partage cultuel}

\paragraph{déterritorialisation}

  déterritorialisation du catholicisme en France. Les gens ne pratiquent plus dans leur paroisse.
  
%-----------------------------------------------------------------------
\section{les formes modernes de l'agir religieux}


Selon Max Weber (1864-1920), 

\begin{quote}
    la religion est « une espèce particulière de façon d’agir en
communauté » dont il s’agit d’étudier les conditions et les effets.
\end{quote}
 La religion n’est pas analysée comme
« un système de croyance » mais davantage comme un « système de réglementation de la vie ». Sa
sociologie des religions s’inscrit dans une sociologie de la domination. Mise en évidence avec Ernst
Troeltsch (1865-1923) de deux types de communalisation religieuse (groupement hiérocratique): secte
et Eglise et de trois types d’autorité religieuse.




\paragraph{La question de Max Weber}

\textit{En quoi le fait d'être croyant modifie ou non la vie ? }
Est-ce que je serai différent si je suis chef d'entreprise et musulman ou chrétien ? 
\begin{Ex}
    Au Royaume-Uni, Rishi Sunak leader pour le Parti conservateur. Symbolique puisque en fait, il a été ou est en train d'être connu le jour de la fête des Mères dans l'hindouisme, et il a mis en évidence justement l'importance de sa famille.
\end{Ex}
Il n'y a peut être pas de différence de comportement. Dans la perspective weberienne, quel est le récit de leur croyance et ce que cela change. C'est  un système de croyance mais surtout de réglementation. On ne définit pas a priori s'ils sont croyants ou pas : s'ils disent qu'ils sont croyants, on les croie. 
En Allemagne, il est habitué à un pluralisme des religions en Allemagne.


\begin{Def}[communalisation religieuse]
  Le terme de communalisation religieuse,  c'est la manière de vivre ensemble, de faire groupe autour de la foi ou la pratique. C'est que la forme de hiérarchie au sein de ce groupe sur le plan de la Vie concrète.
\end{Def}

Comment on légitime les différentes figures d'autorité de la foi \mn{Cf cours d'Anthropologie sur l'homme et l'islam}. 

\subsection{Définition sociologique de Secte : neutre}
 
\paragraph{Définition utilisée en sociologie: celle de Weber et reprise par Ernst Troeltsch, 1912.}
\begin{Def}[Eglise - Weber]
“ L’Eglise est une institution qui, ayant reçu à la suite de l’œuvre rédemptrice le pouvoir de dispenser
le salut et la grâce, peut s’ouvrir aux masses et s’adapter au monde; car elle peut, dans une certaine
mesure, faire abstraction de la sainteté subjective dans l’intérêt des biens objectifs que sont la grâce
et la rédemption. ”
\end{Def}

Etymologie: Sequi: suivre et secare: couper

\begin{Def}[Secte - Weber]
“ La secte est une libre association de chrétiens austères et conscients qui, parce que véritablement
régénérés, se réunissent ensemble, se séparent du monde et se restreignent à leurs petits cercles.
Plutôt que sur la grâce, ils mettent l’accent sur la loi, et pratiquent au sein de leur groupe, et d’une
manière plus ou moins radicale, la loi chrétienne de l’amour: tout cela en vue de préparer et d’attendre
la venue du Royaume de Dieu ”
\end{Def}

En Allemagne, le terme de secte est neutre : c'est une façon de gérer le conflit. Quand on a un conflit, on part, alors que dans l'Eglise catholique, on gère (ou pas) le conflit à l'intérieur. 
En France, on pense l'Eglise de façon organisée et hierarchique. Et on a du mal à penser un autre modèle et on projette dans les autres religions cette organisation. 
\paragraph{la France demande une organisation et homogénéité des religions à l'image de l'Eglise catholique} Difficile pour le protestantisme et l'islam où le représentant ne représente que lui-même. 

\paragraph{Traits principaux: }
\begin{itemize}
    \item \textbf{choix} 
  \item \textbf{autorité}
  \item \textbf{intégration/séparation}
  \item \textbf{routinisation du charisme}
\end{itemize}

 \begin{table}[h!]
 \begin{small}
     

     

\begin{tabular}{p{2cm}p{4.5cm}p{4.5cm}}
\toprule
 & Eglise & Secte \\
 \midrule
\textbf{choix}  &   ce sont généralement nos parents qui ont choisi que l'on faisait partie d'une Eglise particulière. Sauf conversion. Ce choix passe par un rite, surtout pour les jeunes garçons (potentiellement visible corporellement). Généralement pas de sanction si pas de pratique.     &  Les individus choisissent de rentrer dans la secte.        \\
  &    L'Eglise accepte tout le monde (à part le judaisme avec le pb de la filiation)    &  Le choix est réciproque  :   franchir certaines étapes mais le Groupe vous choisit (sélection).      Très valorisant pour les personnes fragiles de rentrer dans les sectes (\textit{identité sociale en creux  :} je récupère une identité positive de moi-même parce que je suis différent de la société    \\
  &Arrivé à  l'âge adulte, il n'est pas demandé aux pratiquants de réitérer leur foi. & \\
\textbf{autorité} &        &       \\
\textbf{intégration/séparation} &   il n'y a pas forcément de sanctions.     &       \\
\textbf{routinisation du charisme} &        &      \\
\bottomrule
\end{tabular}
  \end{small}
\end{table}
 
 \FloatBarrier

\paragraph{Identité en Creux }
quand on se {définit par ce qu'on n'a pas et pas ce qu'on a}. La secte nous donne une nouvelle identité. Mais le discours ambiant est de refuser qu'il y ait pu avoir choix. Quand la personne sort de la secte, elle se redéfinit en creux. Risque de retourner dans une secte très important.

 

  

 \paragraph{Secte en Sociologie : oublier la connotation négative}
On a gardé en sociologie le terme secte, même si  vu de façon très négative, 
Et c'est le terme de cette formule, un terme qui provient du contexte protestant et le contexte protestant, lui ne voit pas façon négative.
 \begin{Ex}[Dans le protestantisme,]
      quand vous n'êtes pas d'accord avec la ligne dominante d'un groupe, Vous pouvez tout à  fait, de façon assez légitime, partir et créer un autre groupe qui va s'appeler sectaire, qui suit la parole dite d'origine tout en coupant.
    
 \end{Ex}
 \begin{Ex}[Dans le catholicisme, ]
   la question de la séparation n'est pas du tout la même.
Il n'y a pas du tout cette possibilité là , parce qu'au contraire il y a la volonté que tout le monde fasse toujours partie de la même famille catholique. C'est une autre manière de gérer le conflit. On a beau dire qu'il y a pas de conflits, il n'y a pas d'opposition et qu'il y a des courants un peu différents, qu'on va essayer de tous conserver autrement.
 \end{Ex}
 \paragraph{Eglise en Sociologie : ne pas penser Eglise catholique} 
C'est toute institution religieuse organisée. C'est qu'en France, on a une vision de la religion qui est très marquée par l'Eglise catholique, qui a un modèle hiérarchique, hiérarchisé, pyramidal. On a beaucoup de mal à  comprendre que les autres, ne se sont pas organisés de la même manière et c'est même quasiment la seule religion organisée de cette façon là.

\paragraph{France}
On tend à plaquer le modèle catholique sur les autres confessions religieuses. On pense que la tête de l'islam est une statue dans le bouddhisme, une tête unique que dans le protestantisme, une tête uniforme i.  Et autant on a du mal à  penser le pluralisme religieux sur le sol français, autant on a du mal à  penser le pluralisme au sein de chaque religion. En France, on a demandé à  ce que chaque religion s'organise avec un représentant d'une organisation de la religion, dont la légitimité ne vient que de l'Etat : La Fédération protestante de France existe pour répondre à  la demande de l'Etat. l'Islam a du mal à  organiser justement en une représentation unique parce il y a un pluralisme ethnique et confessionnel en interne.
Et c'est l'Etat français qui exige en fait.
Une reproduction de ce qu'il peut être modèle la personne qui représente les bouddhistes de France, qui ne représente qu'elle même. Elle n'a aucune hiérarchie à l'inverse du catholicisme. 

 \paragraph{Processus valorisant des sectes}
Pour les personnes ont une identité sociale à  ce moment là , fragile ou en crise, très valorisant dans le service, le processus.
D'entrée dans une secte ou si, par exemple, c'est une personne qui dans sa famille, arrive à  dire le rôle de la personne pour qui tout est un peu compliqué, qui se fait dans les fratries. Des fois, il y a ces classements entre les fesses ou qui, avec ses amis ou dans sa vie sentimentale ou professionnelle.
  ces personnes qui ont cette identité sociale en creux, quand elles commencent cette étape d'intégration pour entrer dans la secte, commencent à  avoir ce qu'on appelle un renversement du stigmate. C'est à  dire que d'être vus comme des personnes qui auraient un peu raté, un peu échoué, un peu dans la marque du Stigmate, c'est au contraire de vous que je suis, que je suis en train de devenir membre d'une élite.
Même, je récupère une identité positive de moi même. 

Dans le processus d'adhésion à  la secte, il y a bien cette question fondamentale de choix qui est vraiment structurante.
Les ceux qui ont découvert leur foi dans les sectes, puis dans l'Eglise catholique. Alors c'est peut être plus difficile de quitter l'Eglise catholique, 

\paragraph{la secte ne crée pas la fragilité }c'est que la secte ne crée pas une fragilité.
Mais éventuellement, ça va dépendre des personnes qui peuvent exploiter les fragilités ou c'est exactement la même chose en et surtout, il faut bien comprendre que les gens vont avoir des systèmes d'emprise. C'est compliqué de sortir de la secte. Quand on sort de la secte, on perd cette identité valorisée, rebasculer dans une identité dévalorisée parce que avoir fait partie d'une secte, c'est souvent mal vu pour être accepté de la société quand on sort de la secte.

Souvent, le discours en France à  tenir est le suivant je suis rentré à  la suite moi même, j'étais fragile, j'étais sous emprise, j'étais pas vraiment volontaire et après on essaie d'avoir un témoignage ultime qui n'est pas très capable. Mais ce que la société entend et accepte, c'est uniquement un discours. La société a beaucoup de mal à  accepter qu'une personne, moi, je suis rentré dans la secte pour vénérer les éléphants roses.
J'ai adoré ce groupe. Et maintenant voilà , je suis fatigué de vivre autre chose dans la balance pour être dans cette obligation sociale que si on sort d'une secte, il faut expliquer qu'on a osé rentrer, mais pas volontairement. On a été dominé, ce qui fait que les personnes qui sont rentrées et qui sont restées sans système d'emprise sont peu entendues, que le discours est saturé.

\paragraph{quitter le Groupe ?}
lorsqu'une personne fait partie d'un groupe / secte et qu'elle quitte le groupe :  deux / tiers des adhérents quittent un groupe pour réintégrer un autre Groupe dans les deux, trois ou quatre ans.   les raisons,   c'est le fait de se sentir ensuite de première et pas celle du choix. On choisit de faire partie du groupe. Quand on a été choisi, on est avec d'autres personnes qui eux mêmes ont été choisies.  
 
 \paragraph{Et l'Eglise catholique ?}
 Il y a beaucoup de réflexions du côté de l'Eglise catholique Grande Eglise sur comment institutions, pourquoi les gens partent, peut être oublier ce qu'ils ne trouvent pas dans l'Eglise catholique, ce qu'ils attendent, ce qu'elle voulait, qu'elle voulait dans la prédication des prêtres, par exemple, est une éducation. Il semble que ça manque et ça, c'est du ritualisme.
  Il y a beaucoup de jeunes qui commencent à  prendre de la voie, comme chez la protestants par exemple. Le gospel, la façon d'animer un groupe qui est à  Londres parce l'appelle Frank. Quand vous regardez face à  moi, les prêtres dedans, on voit, c'est le protestant, mais ça montre qu'il est beaucoup.


  \subsection{Principe d'autorité}
  La question d'autorité. Dans une
Institution type l'Eglise, l'autorité a priori est connue. Et on sait aussi type de sanction. certes,  l'Autorité n'appartient qu'à  une personne qui est le leader ou le prophète. Mais peu Importe l'intitulé de la personne que donne.
Le groupe donne l'autorité n'appartient qu'à  une personne. Le chef. Souvent un homme, parce que ce chef est considéré comme ayant des qualités dites extraordinaires, c'est à  dire des qualités que les autres humains n'ont pas.
C'est quelqu'un qui sort de l'ordinaire.
Qui est un humain. si on est sur une secte religieuse, ce qu'on là , c'est que je pense que c'est à  une personne est en lien avec une religion. On va dire c'est une personne qui est en lien direct avec la divinité.

\begin{Def}[Groupe dichotomique]
     des groupes dichotomique, des groupes qui émotionnellement sont chauds émotionnellement. Essentiellement dans l'émotion,  à travers des changements, à  travers éventuellement des modes d'expression particuliers, des thérapies corporelles. 
\end{Def}
 En fait, la plupart de notre vie désormais centrée autour de soi, on vit dans ce qu'on on vit pas, on travaille à  côté, mais on est en lien avec ce groupe de différentes manières. Mais en fait, quand on est avec les autres, avec en première ce système d'autorité centré sur une personne, on est dans une situation d'effervescence émotionnelle.


   

\paragraph{stabilité des institutions}
Ou en tout cas critique les valeurs de la société dans laquelle elle, classiquement les institutions, trouve que.
Les sociétés gagneraient à  mieux s'inspirer de ces valeurs portées par la structure des comportements. Néanmoins, les institutions ne remettent pas totalement en question la société parce qu'il y a lieu de trouver une manière de faire et d'agir qui leur convient en partie.
Ce système ou elles sont condamnées peut être le cas en partie en Chine ou autre, ou en ex-URSS.
Globalement, la plupart des institutions souhaitent conserver le statu quo existant plutôt que de les remettre en question et voir leur place est minorée. C'est le cas en France avec la loi de 1905. Très peu de concessions sur de remettre en route la loi de 1905, parce que finalement, ce sont peut être des adaptations sur la marche. Globalement, ne surtout pas être à  l'inverse, mais c'est tout.


\paragraph{Sectes extra-mondaines}
Par contre, les sectes extra mondaines considèrent que c'est foutu. Il faut se couper, il faut aller vivre totalement en dehors du monde et limiter au maximum le contact avec le monde.
L'objectif alors, c'est d'être plus autosuffisant possible, de créer dans la mesure du possible ses propres aliments, ses propres vêtements, être en autosuffisance. Mais au niveau de l'éducation.
à la porte, c'est notamment ce qu'on retrouve aux Etats-Unis avec un nouveau de une. Dans cette volonté de dire à  un.
Moment donné pour nous, le but, c'est de vivre de cette manière là  coupée du monde.  on va essayer de travailler notre suffisance et de limiter au maximum tout contact avec l'extérieur. Ce que je vous dis, c'est que c'est l'organisation.
Mais ça peut être dans la religion. On retrouve maintenant à  travers des formes d'\textit{écologie intégrale}, et cetera cette volonté en fait de couper pour créer une autre manière de ensemble, une manière de faire communauté, de faire société différente. Car c'est dans ces groupes là , extra mondain que le risque de mort physique qui est plus fort que s'il y a dangerosité dans le groupe après 40 ans, s'il y a dangerosité, pas systématiquement dangerosité, contrairement à vous, c'est plutôt pour votre porte monnaie que vous vous financez.
 

\paragraph{La relève du leader de la secte}
Du leader du groupe tout ca organisé autour de cette personne.
Et  si personne disparaît, que se passe t il ? Et bien soit l'anticiper, ce qui est très très rare. Que des leaders anticipent leur succession, qu'ils se vivent, changent de sentimentalité et  soyez un disciple déclaré qui a été nommé.
Qui va prendre la relève et éventuellement le groupe continue, mais c'est pas si sûr.
Très souvent ce qui se passe, c'est que le groupe s'arrête.  soit il y a une relève de groupe, soit le sage Et les fidèles vont prendre un temps de pause avant souvent de retourner à  nouveau.  
  Pour certains d'entre eux, c'est ce qu'on a fait à  l'utilisation du charisme, le charisme, la particularité, le charisme spécial.
Ces fameuses qualités extraordinaires, la motivation. Charisme, ça peut aussi arriver quand le fondateur fatigue et que des adeptes fatigués de tout ça, ça prend de l'énergie.
C'est très, très épuisant, captivant mais très documenté. le défi, c'est la réutilisation du charisme. C'est quand on commence à  ritualiser l'origine du mouvement, c'est à  dire qu'on va mettre en place des cérémonies particulières qui rappellent.


 \section{les trois types d'autorité religieuse et leurs figures idéales typiques}
La construction sociale de la croyance en la légitimité du pouvoir.
\begin{itemize}
    \item  légitimité rationnelle-légale : le prêtre
    \item légitimité traditionnelle : le sorcier
    \item légitimité charismatique : le prophète
\end{itemize}



\paragraph{Question de Max Weber} La grande question de Max Weber, c'est finalement qu'est ce qui fait que des individus acceptent de se soumettre librement à l'autorité de quelqu'un d'autre permet d'être responsable en acceptant ?
 

\subsection{figure du prêtre}
la forme légale rationnelle. La forme légale rationnelle, c'est celle représentée par le fait.
 le prêtre tire son autorité à  l'écart rationnel, c'est à  dire sous l'autorité d'une institution qui est organisée de façon relativement rationnelle. On croit en l'institution, on croit la justesse de l'organisation, de l'intégration et comme on adhère à  cette institution et qu'on prend la Règlementation de l'institution.
\subparagraph{interchangeable} Ses représentants sont interchangeables.. Ce n'est pas parce que la personne est la meilleure au monde, la plus savante, qu'elle va devenir prêtre si elle n'a pas fait la formation adéquate.
Il faut avoir fait cette formation validée par l'institution validé.  


\subparagraph{Organisation}
Bref, on organise.  Cette domination est appelée rationnel, légal parce que rationnellement ou pas, des textes qui généralement justifient les positionnements des uns et des autres.


Des catégories les uns par rapport aux autres, une hiérarchie,  seconde forme de légitimité.
\subparagraph{le prêtre est totalement dépendant de l'institution.}

 
\subsection{le sorcier}  

\begin{Def}[sorcier]
    c'est celui qui sera en capacité de vous mettre en lien avec des forces magiques. C'est une personne qui possède un don qui lui vient de la tradition. C'est à  dire c'est un bon qui peut être été transmis par un ses parents. 
\end{Def}

Cela peut être un don familial qu'il a obtenu, mais a été reconnu comme après avoir ce don. Et c'est là  le salut distinct des autres.
C'est un premier temps, c'est l'institution qui distingue la personne et c'est dans l'\textbf{aspect Vocationnel} de la personne. 

\subparagraph{Validation par l'institution} Ce qui compte, c'est que l'institution valide et va valider l'inscription de cette personne dans une histoire qui la précède, une histoire qui relève de la tradition des sociétés. 

\paragraph{pas de formation}
Pas besoin de faire de formation, d'avoir pas besoin d'avoir forcément le savoir. Ce qui compte, c'est d'être reconnu pour ses compétences. 

La particularité du sorcier, généralement, c'est qu'il ne vit que par le doute des personnes qui croient. 

si son action n'est pas efficace.
Les attributions, c'est ce qu'on.
Appelle aussi en sociologie le contrôle de la loi. C'est une certitude. 

En général, sorcier ne va pas être un prestataire : C'est la personne qui va décider elle même qui a de l'argent ou quels sont les biens matériels, car elle est donnée en échange de ce que la personne lui donne. Contre don est supérieur au don.
\begin{Ex}
Au Malawi par exemple, on a entendu Il y a un ouvrage très beau de gens, s'appeler Sada, qui s'appelle résumé dans les bouquins, bref sur les sorciers dans les bocages encore en Normandie, sur tout ce qui est invisible.    
\end{Ex}
\paragraph{désenchantement en France}  Alors, avec le système de laïcité en France, du désenchantement, place du sorcier ? Il y a encore des régions très habitées par cet invisible aussi, notamment le centre de la France dans les Cévennes.
Cette vision du rebouteux, de la personne qu'on appelle peut être à  sortir des embouteillages, de la lenteur des nous, c'est à  dire des personnes qui a vécu. la prière du feu.


%-------------------------------------

\paragraph{Chamane}
Cela peut être aussi les chamanes.  
 valorisation de la figure chamanique dont on ne sait pas très bien comment la définir, mais qui serait porteur là  aussi de capacité de se mettre en lien avec des forces magiques.
On a tendance, 
\paragraph{Exorcisme }prêtres de pratiquer l'exorcisme.  question de la magie, des forces occultes, des liens avec les personnes décédées.  

  Que devient le prêtre qui lui est plus d'abord associé à  une institution qui valorise le logique et qui est de c'est bon pour les sorciers ?

  \paragraph{ostéopathes}
Actuellement, on considère que par exemple certains, certains ostéopathes ou coach de vie etc flirtent avec cette figure du sens. 



Parler, de poser la question essentielle que vous avez vous avez pris les sorciers, c'est d'aller dans le positif. Vous l'avez vu parce que chez nous, c'est chez nous, les sorciers. Il est mal vu ce jour. On comprend pourquoi les sorciers c'est pas vu comme quelqu'un qui fait du mal, qui attaque, qui détruit, qui peut entrer dans ta maison même si tu as tout fermé ou qui vient pour faire du mal.
% -------------------------------------------

Est ce qu'il y a aussi des pratiques aujourd'hui que je vois par exemple chelou à  l'époque que les yoga chez nous n'est pas accepté, mal vu comme un danger. Oui yoga c'était comme des pratiques, vraiment on peut dire et je peux dire la arrive que les gens ne pouvaient pas accepter ça, ça devient plus sympa pour moi. C'était  des pratiques qui ne sont pas les la culture, ce que j'ai vu, même s'il y avait utilisé pour la représenter.
 
Ils ont essayé de créer une catégorie dans la réserve de sorciers qui va représenter toutes ces différences là . Vous voyez, c'est tout ce qui finalement doit avoir une légitimité à  travers un talent particulier qui vient finalement de leur capacité à  être en lien.
Avec la de la loi. Comme Vangheluwe ou la mère de l'enfant.
Et qui du coup sont vus comme faisant partie d'une tradition très ancienne, avec un savoir faire accessible à  très peu de gens et.  ils peuvent être.
Craint pour ne pas être mal respectés ou très bons avec.
Et actuellement voir retour.
De la fumée au monde.

\subsection{Figure du Prophète}
au sens de saint.
\begin{Def}[prophète]
    Personne qui tire son autorité de son charisme. Dans les sectes, figure de celui qui dirige la secte dans le regard de l'autre. 
\end{Def}
\paragraph{devenir prophète} prophète devient prophète à  partir pour les personnes qui sont convaincus qu'il possède des qualités qui sont qu'il possède, qui est détenteur de ce fameux charisme. 

\paragraph{charisme mal défini}
On peut dire qu'on est convaincu que cette personne a du charisme pour toucher par son charisme sans forcément pouvoir le décrire. On en fait ou alors on voit que des termes comme lumière, présence.  
\paragraph{Economie}
Pour subvenir à  ses besoins et on retrouve ce qu'on a tout à  l'heure avec le don et le contrôle.
 c'est que les personnes sont des adeptes qui vont croire ou avoir l'impression que ces prophètes disent des choses.   Elle se fatigue dans son chemin. Ils vont vouloir l'aider et l'aider, notamment matériellement. Et  c'est ce qu'on observe, c'est que souvent il y a une circulation financière et économique autour des figures des prophètes qui souvent échappent au Prophète lui même.
 \paragraph{Exemple de charisme lié au hiatus social}
 \begin{Def}[hiatus social]
      élevé par des personnes de deux cultures différentes.
 \end{Def} 
 \begin{Ex}
     vous avez appris jeune, en décalage entre deux manières de faire. Ou alors vos parents provenaient de milieux sociaux particuliers en décalage avec la société, ont vécu une crise en un moment donné.
 \end{Ex}
 La trajectoire de cette personne qui a pu avoir un écart qui fait que cette personne est en capacité de non, de savoir et de ressentir les décalages chez d'autres personnes et combien ces décalages peuvent engendrer des souffrances.
 la personne du fait que c'est une personne à  nommer ce que d'autres ressentent sans être capable même de nommer cette personne par son récit personnel.  
A
\begin{Ex}
  En sport, avec Maradona, on va dire que Maradona va mettre sa propre église. Une coupe avec un lieu de culte, c'est quand même plus convivial. Pas moins de 200 000 Argentins qui affrontent. Regardez regarder les dix commandements. Plus c'est la Main d'or, la Main de Dieu, les.
Adolfo avec Comment.
Monte Dieu et vers la main de Dieu.  
\end{Ex}
 

\subsection{Usage courant du terme secte : définition basée sur la dangerosité}

\mn{15/11/22}
\begin{Synthesis}
    Comment les personnes acceptent différents types d'autorité ? une institution ou une personne 
    hypothèse socioloique : on va accepter cette autorité car on va considérer que cette personnes est légitime, soit par vote (élection), soit par le charisme de la personne, soit par sa formation elle nous est supérieure.
    Par habitude, on est habitué à se soumettre sans volonté (routinisation du comportement) à l'autorité.
    
\end{Synthesis}

\paragraph{Structure familiale reproduisant le modèle de la société}  la société\mn{Martine Segalen,...}

\paragraph{comment on remet en question l'autorité} cf les abus dans l'Eglise. Les personnes vont chercher un autre endroit d'autorité. Rarement égalitaire. On a besoin d'une personne qui décide pour nous. 

\paragraph{En France } En France, on attend la personne sauveur et la décrier très rapidement. 

\chapter{Dangerosité et secte}

\section{Les différents types de sectes : illustration avec des exemples récents}
 
Toutes souhaitent la réalisation d'un monde meilleur, sont issues d'une insatisfaction par rapport à la
société.

\paragraph{Les témoins de Jéhovah} Ils viennent par deux. D'un point de vue ministériel, ils sont maintenant un culte. Comme ils voyaient la fin du monde, ils faisaient peu d'études. Comme la fin du monde n'arrivait pas, ils ont relu leur texte. En cessant d'annoncer l'apocalypse en une date précise, ils ont changé leur approche : études longues, mélanges avec les autres. La France a vu cette évolution : \textit{routinisation weberienne}

\subsection{Les sectes intramondaines: dans la société.}

\begin{Def}[sectes intramondaines]
    Considèrent que l'on peut convertir encore des gens dans la société
\end{Def}

 
\begin{Prop}
    Pourquoi on en est arrivé à associer dangerosité et secte ?
\end{Prop}

\paragraph{Secte venant d'Anglo-saxon} un encouragement à se séparer quand on a un conflit alors que le catholicisme va gérer le conflit en son sein.

\paragraph{En France, va être vu comme négative dans les années 70} Cela vient tout d'abord d'une chronologie des dérives secteurs : 
\begin{itemize}
    \item Secte Moon : se développe au moment de la contre-culture états-unienne\sn{contre le capitalisme, la guerre au vietnam et la ségrégation.}. Lutte contre l'occident via les spiritualité orientale : bouddhisme et hindouisme. \textit{Woodstock}, démocratie égalitaire, bombe atomique ... Cela va déboucher sur "Mai 68"\mn{un épisode contre la suprématie de l'occident qui agrége d'autres mouvements qui émergeaient.}
    \item Mont Carmel, 1993 : vont mourir. 
    \item Octobre 94 : Temple solaire : carbonisé, une balle dans la tête.
    \item Metro de Tokyo. Gaz sarin par la secte Aoun.
\end{itemize}

\paragraph{Logique du non-moi va fasciner les occidentaux} à travers ces spiritualités orientales. Mais ils vont inverser des religions du non choix en spiritualité du choix. 
\begin{quote}
    L'orient est le miroir de l'occident pour se voir.
\end{quote}

\paragraph{la section Moon} se présente dans cette logique de contre-culture, inspiré par l'hindouisme. 
\begin{Def}[Ashram]
    Espace de vie clos (ou on dort en dortoir) et vivant en auto suffisance. Une secte \textit{extramondaine} qui se coupe du monde : forme de décroissance. 
  
\end{Def}
  Relation sexuelle acceptée que dans une logique de procréation. Vie austère.
    Les relations personnelles sont cadrées.
    \paragraph{Contre-société}
    Les premières personnes qui viennent sont un mix : éco-terrorisme, Les gens autour ont des suspiçions. Les jeunes font des cures de desyntoxisation, très dures et vont vivre \textit{une contre-société}. 
\paragraph{Tout va bien ? } Ces jeunes, majeurs, coupent toute relation avec leur famille et leurs familles ne vont pas supporter. Or, les jeunes restent dans la secte du fait du lien chaud qu'ils vivent avec les autres, et se coupent des parents. Les parents vont condamner, à l'aide de psychologie le concept de \textit{lavage de cerveau}. 
\begin{Def}[lavage de cerveaux]
    Si on reste dans la secte, ce serait à cause de l'emprise : nos enfants ne sont plus libres. 
\end{Def}
Les parents vont donc embaucher des mercenaires pour les kidnapper et vont subir la \textit{déprogrammation sensorielle}, dans une pièce, sans lumière, sans bruit,...
Les parents vont être condamnés.
\begin{Prop}
    Quand on a une vision négative sur une secte, il faut voir que c'est le regard des parents. Cela ne veut pas dire que les parents sont responsables ou que les sectes sont sans critique, mais regard pas neutre.
\end{Prop}


\paragraph{Un rapport parlementaire en 1996 } Nous légiférons le plus en France sur les questions religieuses.
Le rapport va parler de \textit{dérive sectaire } : Toute secte dérive en dangeresité. A partir des interpellations des parents. Ce rapport va ainsi interdire 150 (?) sectes dont \textit{l'arbre du milieu}, qui luttaient contre l'inceste. La Mulvilude, toujours très contestée.
Jean-Pierre Raffarin, va quitter sa fonction. Son dernier geste est de discréditer ce rapport. Malheureusement, continue à circuler \textit{secte = danger}. 

\paragraph{Délit en 2021 de manipulation mentale} jamais utilisée. Cela ne veut pas dire que les sectes ne sont pas dangereuses, mais il y a un écart entre la dangeresité reelle et pensée.


\paragraph{Radicalisation} \mn{Voir la série \textit{Khalifa} en Suède, des jeunes filles qui veulent aller en Syrie puis revenir en Suède. Les deux trois premiers épisodes, il faut s'accrocher. La police suédoise et les imams sont démunies. }

\paragraph{Les risques} Les Groupes intra-mondains peuvent être dangereux pour l'argent. Mais les Groupes extra-mondains peuvent 
\begin{itemize}
    \item Argent : est ce qu'on donne tous ces biens en rentrant dans le Groupe. Est ce qu'on peut récupérer une autonomie ? Risque si tout le monde de façon simultanée quitte son travail.
    \item le Pouvoir : non seulement qui existe mais plus stricte. Tout à coup, on va être en emprise totale, méfiante vis à vis des autres. 
    \item achat d'armes : les groupes vont passer à l'acte de façon terroriste.
\end{itemize}


\paragraph{Scientologie} Voir le dessin animé, \textit{South Park}\sn{Très trash}. La particularité de la scientologie, \textit{Lafayette Ron Hubbard}\mn{1891-1988, auteur de science fiction très prolifique. Un jour, il a une révélation suite à la Seconde Guerre mondiale. Il faut repenser la science pour éviter le mal. Il va donc créer une méthode, la \textsc{Dialectique}, la science moderne de la santé mentale (typique des années 50). En 1954, il crée la scientologie, comment la science peut nous aider de sortir du mal } Prendre une posture sociologique d'empathie. Il va créer un récit : les \textit{thêtan}, ce sont des êtres immatériels, éternels, inconnaissables. Les thétans s'ennuient et observent la terre. Ils décident de rentrer dans l'enveloppe des êtres humains, de plusieurs humains. L'humain décède (très vide), ils recherchent un autre humain, et vont d'humain en humain. Au bout d'un moment, ils oublient qu'ils sont thétans, et face à l'explosion démographique, nous avons tous un thétan en nous. 

Quand on ne va pas très bien, c'est le thétan qui ressent des émotions des autres vies. Si vous ne la reconnaissez pas, cette émotion ne vous appartient pas. 
La solution, apprivoiser le thétan. On va expulser le thétan. Pour cela, on va faire des auditions pour voir comment va notre thétan (de 500 à 10 000 Euro la seance). Ces auditions vont permettre d'identifier les différentes vies du thétan, peu à peu le thétan se réveille. Et il peut partir.
Mais ensuite, il faut combler le vide : technique d'insulter un objet. Ce n'est pas anodin (pas accepté socialement, côté obscure),... On va mettre en place des pratiques que seuls les adeptes vont mettre en place. On sent qu'on enfreint quelque chose. Cette critique de la bouteille, on franchit un interdit de la société pendant 5 mn, on sort sa colère, thérapeutique. Dans un cercle qui autorise cette sortie.  

Puis le scientologue va devenir la \textit{bouteille}, va être humilié, Maîtrise d'eux même, avec un récit très fort. 

C'est devenu une Eglise parce que la scientologie n'avait pas payé ses impôts. Négociation avec les impots pour devenir un culte reconnu. Le discours religieux habille l'entreprise. 

En 1983, enquête puis 1989 : \textit{exercice illégale de la médecine}. Cette affaire a donné lieu à un non-lieu pour refus de jugement. 
1998, une tonne et demi de document disparaissent.
En 2009, Alliot Marie amende la loi.
Et pourtant, la Mulvilude ne s'intéresse pas à la scientologie. Ce qui est intéressant, quelle est notre relation et ambivalence par rapport à ces Groupes. 

Cela fonctionne. 







\section{Les sectes extramondaines: se coupent de la société qui est perçue comme dangereuse.}

- issues du mouvement social de contre-culture des années soixante
- le danger de mort: l'Ordre du Temple Solaire

\subsection{L'ordre du temple solaire}
Un mouvement avec beaucoup de professions intellectuelles et de santé. Et pourquoi alors que la secte a déjà tué et montré son danger, les gens sont restés ?

\paragraph{mythe des templiers} Quelque chose de mystique et de secret. Progressivement, l'OTS va mettre en place une compréhension : 
\begin{quote}
    nous comprenons le monde, initiés.
\end{quote}

\paragraph{Il faut se séparer du monde} On continue dans le monde. 

\paragraph{dimension hygieniste} Pas mal de gens avec des TOC liés à la propreté. Ces personnes vont être valorisées : \textit{vous êtes précurseurs car tout ce qui est extérieur est corrompu}. Pour certains, les TOC vont diminuer.

\paragraph{repli progressif} nourriture. Baume et crème qui marchent bien. \textit{conditions qu'ils sont dans le juste}

\paragraph{Luc Royer} fondateur. La justice revient le titiller sur les affaires précédentes. Modification du Groupe : radicalisation du Groupe (à l'inverse des témoins de Jéhovah). Une crypte, sous sol.
\begin{quote}
    {maintenant, vous êtes prets pour la révélation : vous n'êtes pas terrien, vous êtes de Sirius}. C'est pour cela que vous êtes différents.
\end{quote}
On retourne à Sirius en enlevant l'enveloppe terrienne. Probablement l'acceptation par les disciples de cette croyance a renforcé la bascule des dirigeants. \textit{bascule collective}

\paragraph{Double jeu} après le premier suicide, les disciples ont été interrogés et ont dit qu'ils avaient compris. 

\paragraph{En France, discours normatif du bon croire et du mauvais croire} Pour l'ancien croire, pas de problème, mais pour le nouveau croire, besoin de justifier rationnellement.
Or, on a vu que le seul discours des disciples qu'on accueille, c'est celui de victimes. Et on a du mal avec l'aspect positif.

\paragraph{Déradicalisation} ne marche pas trop les approches avec un imam tempéré car on ignore l'aspect positif de l'adhésion à la \textit{secte}. Ce qui semble marcher, c'est travailler sur le \textit{doute} de la personne (ex : Sirius, c'est comment pour un membre de l'OTS)

\section{Comment devient-on adepte d’une secte ?}

- la séduction
- les résultats tangibles dans le quotidien
- les moyens de pression: thérapie, sexualité, les modèles, les raisons familiales et
psychologiques

\begin{Synthesis}
La secte ne crée pas les besoins, elle les exploite.
\end{Synthesis}

